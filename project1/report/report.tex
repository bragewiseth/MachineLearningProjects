\documentclass[twoside,11pt]{report}

% Any additional packages needed should be included after jmlr2e.
% Note that jmlr2e.sty includes epsfig, amssymb, natbib and graphicx,
% and defines many common macros, such as 'proof' and 'example'.
%
% It also sets the bibliographystyle to plainnat; for more information on
% natbib citation styles, see the natbib documentation, a copy of which
% is archived at http://www.jmlr.org/format/natbib.pdf

\usepackage{jmlr2e}
% \usepackage[utf8]{inputenc}%
% \usepackage{tikz}
% \usepackage{cfr-lm}%
\usepackage[T1]{fontenc}%
\usepackage{physics}
\usepackage{amsmath}
% \usepackage{amssymb}
% \usepackage{graphicx}
% \usepackage[margin=3cm]{geometry}
% \usepackage{changepage}
\usepackage{fontspec}
\usepackage{minted}
\usepackage{tcolorbox}
\usepackage{lmodern}
\usepackage{xcolor}
\usepackage{lettrine}
% \usepackage{fontawesome}
\usemintedstyle{perldoc}
\usepackage{hyperref}
\hypersetup{colorlinks=false, pdfborder={0 0 0},  }
\usepackage{fancyhdr}
\usepackage{wrapfig}
\usepackage{adjustbox}



\newtcbox{\codebox}[1][black]{on line, arc=2pt,colback=#1!10!white,colframe=white, before upper={\rule[-3pt]{0pt}{10pt}},boxrule=1pt, boxsep=0pt,left=2pt,right=2pt,top=1pt,bottom=.5pt}
\newtcbox{\deloppg}[1][black]{on line, arc=2pt,colback=#1!10!white,colframe=white, before upper={\rule[-2pt]{0pt}{0pt}},boxrule=0pt, boxsep=0pt,left=.49\linewidth,right=.49\linewidth,top=4pt,bottom=3pt}


\newcommand\blfootnote[1]{ \begingroup \renewcommand\thefootnote{}\footnote{#1} \addtocounter{footnote}{-1} \endgroup }
% \definecolor{antwhite}{HTML}{323333}
\newcommand{\code}[3][]{\codebox{\mintinline[#1]{#2}{#3}}}



% \setmainfont{FreeSans}
% \setmainfont{SF Pro Display}
% \setmainfont{IBM Plex Sans}
% \setmainfont{TeX Gyre Heros}
% \setmainfont{Inter}
% \setmainfont{Iosevka Quasi}

% \setmonofont{Iosevka Custom Extended}
% \setmonofont{JetBrainsMono Nerd Font}
% \setmonofont[Scale=MatchLowercase]{SF-Mono-Medium}





% Definitions of handy macros can go here

\newcommand{\dataset}{{\cal D}}
\newcommand{\fracpartial}[2]{\frac{\partial #1}{\partial  #2}}

% Heading arguments are {volume}{year}{pages}{submitted}{published}{author-full-names}

% \jmlrheading{1}{2000}{1-48}{4/00}{10/00}{https://github.com/bragewiseth/MachineLearningProjects}

% Short headings should be running head and authors last names

\ShortHeadings{\url{https://github.com/bragewiseth/MachineLearningProjects}}{\url{https://github.com/bragewiseth/MachineLearningProjects}}
\firstpageno{1}



\title{{\huge Project 1}}
\author{\name Eirik J. \email eirk@ifi.uio.no \\
       \name Felix H.  \email felixch@ifi.uio.no \\
       \name Brage W. \email bragewi@ifi.uio.no}
\date{\today}											% Date
\makeatletter






% \date{\today}

\begin{document}

%%%%%%%%%%%%%%%%%%%%%%%%%%%%%%%%%%%%%%%%%%%%%%%%%%%%%%%%%%%%%%%%%%%%%%%%%%%%%%%%%%%%%%%%%

\begin{titlepage}
	\centering
    \vspace*{0.5 cm}
    \includegraphics[scale = 0.75]{uio.jpg}\\[1.0 cm]	% University Logo
    \textsc{\LARGE University of Oslo}\\[2.0 cm]	    % University Name
	\textsc{\Large FYS-STK3155}\\[0.5 cm]				% Course Code
	\rule{\linewidth}{0.2 mm} \\[0.4 cm]
	{ \huge \bfseries \@title}\\
	\rule{\linewidth}{0.2 mm} \\[1.5 cm]

	\begin{minipage}{0.4\textwidth}
		\begin{flushleft} \normalsize
			Eirik\\
            Felix\\
            Brage Wiseth\\
			\end{flushleft}
			\end{minipage}~
			\begin{minipage}{0.4\textwidth}
			\begin{flushright} \normalsize
        \textsc{
		  eirik@ifi.uio.no\\
          felix@ifi.uio.no\\
          bragewi@ifi.uio.no\\
        }
		\end{flushright}
        
	\end{minipage}\\[2 cm]
	\@date\\
    \vspace*{25mm}
    \urlstyle{rm}
    \textsc{\url{https://github.com/bragewiseth/MachineLearningProjects}}
	
	
    
    
    
    
	
\end{titlepage}
\nocite{*}
% \maketitle
\newpage
\tableofcontents
\newpage

\begin{abstract}%   <- trailing '%' for backward compatibility of .sty file
\lettrine{I}{}n this paper, we delve into the realm of machine learning model optimization and evaluation. 
Our study encompasses various regression techniques, including Ordinary Least Squares (OLS), Ridge, and 
Lasso regression, to analyze their effectiveness in handling simple and more complex datasets. Additionally, 
we employ bootstrap resampling and cross-validation methodologies to rigorously assess model performance 
and enhance generalization.
A significant portion of our investigation is dedicated to understanding the delicate balance between 
bias and variance. We explore how regularization methods like Ridge and Lasso impact bias-variance trade-offs, 
offering insights into the stability and predictive power of these models. Furthermore, we provide empirical 
evidence of the benefits of cross-validation and bootstrap techniques in mitigating overfitting and improving 
model robustness. We found that \{ ..results.. \}. Additionally we verify and compare our findings with well 
established theory and libraris such as SKLearn.
\end{abstract}
\begin{keywords}
    Linear Regression, Scaling, Bias \& Variance 
\end{keywords}

\section{Introduction}

Machine learning has emerged as a powerful tool in data analysis, providing the ability to uncover complex 
patterns and relationships in diverse datasets. But, at its core, machine learning is all about finding 
functions that capture the underlying structure of the data. The use of machine learning algorithms 
to approximate functions is the essence of this paper.\\

Our motivation for this research lies in the exploration of machine learning techniques to approximate 
the terrain on our planet, which can perhaps be described by such a function. Earth's terrain exhibits 
peaks and troughs, hills and valleys, much like some polynomial functions. Fortunately, we can employ 
standard linear regression techniques to approximate polynomials, but the terrain presents its own set 
of challenges. Firstly, the terrain's true underlying function may not be a polynomial at all, and its complexity may 
vary significantly from one location to another. Secondly, our landscape is teeming with small, intricate 
details. Some regions are characterized by flat and smooth surfaces, while others are marked by rough and 
uneven terrain. Focusing too much on these minute details can lead to model overfitting, making it crucial 
to strike a careful balance between model complexity and generalization.
In this context, regularization and resampling techniques, including Ridge and Lasso regression with bootstrap
and cross validation, have proven indispensable. By introducing regularization and resampling, we aim to find 
the sweet spot between bias and Variance. And getting the best predictions we can with our assumptions.\\
To embark on this exploration, we will begin with a simpler case: "Franke's function." which mimics our real terrain
data. This function 
serves as a foundational starting point, allowing us to assess our model's performance in a controlled 
environment before venturing into the complexity of real-world terrain data. Through this gradual progression, 
we provide ourselves wih a framework that can be applied to more complex and varied real-world terrain datasets.\\

\textbf{Data}: We begin by introducing the dataset used for our analysis, highlighting data collection and preprocessing procedures. 
Understanding the characteristics of the terrain data is fundamental to our modeling endeavor.\\
\textbf{Methods and Scaling}: Next, we delve into the methodology, encompassing the implementation of polynomial regression 
models and the application of regularization techniques such as Ridge and Lasso. Additionally, we will discuss the 
importance of proper scaling for model stability and convergence.\\
\textbf{Bias-Variance Trade-off}: A significant portion of our study will revolve around the critical concept of bias and 
variance. We'll explore how regularization methods influence this trade-off and delve into the fine balance between 
model complexity and generalization.\\
\textbf{Results}: In this section, we will present the outcomes of our experiments, showcasing the performance of different 
models and regularization techniques. Through empirical evidence, we aim to provide insights into the effectiveness of our approach.\\
\textbf{Conclusion}: Finally, we will summarize the key findings and their implications for terrain modeling with machine learning. 
Our conclusion will underscore the importance of regularization in achieving accurate representations of complex terrains and 
provide a perspective on future research directions.




% \section{Data}
% \label{sec:Data}
%
% \begin{wrapfigure}{r}{0.3\textwidth}
% \begin{center}
%     \begin{adjustbox}{clip,trim=4cm 3.4cm 3cm 4.8cm, max width=0.3\textwidth}
%     %% Creator: Matplotlib, PGF backend
%%
%% To include the figure in your LaTeX document, write
%%   \input{<filename>.pgf}
%%
%% Make sure the required packages are loaded in your preamble
%%   \usepackage{pgf}
%%
%% Also ensure that all the required font packages are loaded; for instance,
%% the lmodern package is sometimes necessary when using math font.
%%   \usepackage{lmodern}
%%
%% Figures using additional raster images can only be included by \input if
%% they are in the same directory as the main LaTeX file. For loading figures
%% from other directories you can use the `import` package
%%   \usepackage{import}
%%
%% and then include the figures with
%%   \import{<path to file>}{<filename>.pgf}
%%
%% Matplotlib used the following preamble
%%   
%%   \usepackage{fontspec}
%%   \setmainfont{DejaVuSerif.ttf}[Path=\detokenize{/home/brage/anaconda3/lib/python3.10/site-packages/matplotlib/mpl-data/fonts/ttf/}]
%%   \setsansfont{DejaVuSans.ttf}[Path=\detokenize{/home/brage/anaconda3/lib/python3.10/site-packages/matplotlib/mpl-data/fonts/ttf/}]
%%   \setmonofont{DejaVuSansMono.ttf}[Path=\detokenize{/home/brage/anaconda3/lib/python3.10/site-packages/matplotlib/mpl-data/fonts/ttf/}]
%%   \makeatletter\@ifpackageloaded{underscore}{}{\usepackage[strings]{underscore}}\makeatother
%%
\begingroup%
\makeatletter%
\begin{pgfpicture}%
\pgfpathrectangle{\pgfpointorigin}{\pgfqpoint{8.000000in}{8.000000in}}%
\pgfusepath{use as bounding box, clip}%
\begin{pgfscope}%
\pgfsetbuttcap%
\pgfsetmiterjoin%
\definecolor{currentfill}{rgb}{1.000000,1.000000,1.000000}%
\pgfsetfillcolor{currentfill}%
\pgfsetlinewidth{0.000000pt}%
\definecolor{currentstroke}{rgb}{1.000000,1.000000,1.000000}%
\pgfsetstrokecolor{currentstroke}%
\pgfsetdash{}{0pt}%
\pgfpathmoveto{\pgfqpoint{0.000000in}{0.000000in}}%
\pgfpathlineto{\pgfqpoint{8.000000in}{0.000000in}}%
\pgfpathlineto{\pgfqpoint{8.000000in}{8.000000in}}%
\pgfpathlineto{\pgfqpoint{0.000000in}{8.000000in}}%
\pgfpathlineto{\pgfqpoint{0.000000in}{0.000000in}}%
\pgfpathclose%
\pgfusepath{fill}%
\end{pgfscope}%
\begin{pgfscope}%
\pgfsetbuttcap%
\pgfsetmiterjoin%
\definecolor{currentfill}{rgb}{1.000000,1.000000,1.000000}%
\pgfsetfillcolor{currentfill}%
\pgfsetlinewidth{0.000000pt}%
\definecolor{currentstroke}{rgb}{0.000000,0.000000,0.000000}%
\pgfsetstrokecolor{currentstroke}%
\pgfsetstrokeopacity{0.000000}%
\pgfsetdash{}{0pt}%
\pgfpathmoveto{\pgfqpoint{1.020000in}{0.880000in}}%
\pgfpathlineto{\pgfqpoint{7.180000in}{0.880000in}}%
\pgfpathlineto{\pgfqpoint{7.180000in}{7.040000in}}%
\pgfpathlineto{\pgfqpoint{1.020000in}{7.040000in}}%
\pgfpathlineto{\pgfqpoint{1.020000in}{0.880000in}}%
\pgfpathclose%
\pgfusepath{fill}%
\end{pgfscope}%
\begin{pgfscope}%
\pgfsetbuttcap%
\pgfsetmiterjoin%
\definecolor{currentfill}{rgb}{0.950000,0.950000,0.950000}%
\pgfsetfillcolor{currentfill}%
\pgfsetfillopacity{0.500000}%
\pgfsetlinewidth{1.003750pt}%
\definecolor{currentstroke}{rgb}{0.950000,0.950000,0.950000}%
\pgfsetstrokecolor{currentstroke}%
\pgfsetstrokeopacity{0.500000}%
\pgfsetdash{}{0pt}%
\pgfpathmoveto{\pgfqpoint{1.865152in}{3.114056in}}%
\pgfpathlineto{\pgfqpoint{5.453629in}{3.448859in}}%
\pgfpathlineto{\pgfqpoint{5.487390in}{6.106859in}}%
\pgfpathlineto{\pgfqpoint{1.801724in}{5.841615in}}%
\pgfusepath{stroke,fill}%
\end{pgfscope}%
\begin{pgfscope}%
\pgfsetbuttcap%
\pgfsetmiterjoin%
\definecolor{currentfill}{rgb}{0.900000,0.900000,0.900000}%
\pgfsetfillcolor{currentfill}%
\pgfsetfillopacity{0.500000}%
\pgfsetlinewidth{1.003750pt}%
\definecolor{currentstroke}{rgb}{0.900000,0.900000,0.900000}%
\pgfsetstrokecolor{currentstroke}%
\pgfsetstrokeopacity{0.500000}%
\pgfsetdash{}{0pt}%
\pgfpathmoveto{\pgfqpoint{5.453629in}{3.448859in}}%
\pgfpathlineto{\pgfqpoint{6.688015in}{2.151799in}}%
\pgfpathlineto{\pgfqpoint{6.762234in}{5.077010in}}%
\pgfpathlineto{\pgfqpoint{5.487390in}{6.106859in}}%
\pgfusepath{stroke,fill}%
\end{pgfscope}%
\begin{pgfscope}%
\pgfsetbuttcap%
\pgfsetmiterjoin%
\definecolor{currentfill}{rgb}{0.925000,0.925000,0.925000}%
\pgfsetfillcolor{currentfill}%
\pgfsetfillopacity{0.500000}%
\pgfsetlinewidth{1.003750pt}%
\definecolor{currentstroke}{rgb}{0.925000,0.925000,0.925000}%
\pgfsetstrokecolor{currentstroke}%
\pgfsetstrokeopacity{0.500000}%
\pgfsetdash{}{0pt}%
\pgfpathmoveto{\pgfqpoint{1.865152in}{3.114056in}}%
\pgfpathlineto{\pgfqpoint{2.725574in}{1.736701in}}%
\pgfpathlineto{\pgfqpoint{6.688015in}{2.151799in}}%
\pgfpathlineto{\pgfqpoint{5.453629in}{3.448859in}}%
\pgfusepath{stroke,fill}%
\end{pgfscope}%
\begin{pgfscope}%
\pgfsetrectcap%
\pgfsetroundjoin%
\pgfsetlinewidth{0.803000pt}%
\definecolor{currentstroke}{rgb}{0.000000,0.000000,0.000000}%
\pgfsetstrokecolor{currentstroke}%
\pgfsetdash{}{0pt}%
\pgfpathmoveto{\pgfqpoint{1.865152in}{3.114056in}}%
\pgfpathlineto{\pgfqpoint{2.725574in}{1.736701in}}%
\pgfusepath{stroke}%
\end{pgfscope}%
\begin{pgfscope}%
\pgfsetbuttcap%
\pgfsetroundjoin%
\pgfsetlinewidth{0.803000pt}%
\definecolor{currentstroke}{rgb}{0.690196,0.690196,0.690196}%
\pgfsetstrokecolor{currentstroke}%
\pgfsetdash{}{0pt}%
\pgfpathmoveto{\pgfqpoint{1.914572in}{3.034945in}}%
\pgfpathlineto{\pgfqpoint{5.524744in}{3.374133in}}%
\pgfpathlineto{\pgfqpoint{5.560630in}{6.047693in}}%
\pgfusepath{stroke}%
\end{pgfscope}%
\begin{pgfscope}%
\pgfsetbuttcap%
\pgfsetroundjoin%
\pgfsetlinewidth{0.803000pt}%
\definecolor{currentstroke}{rgb}{0.690196,0.690196,0.690196}%
\pgfsetstrokecolor{currentstroke}%
\pgfsetdash{}{0pt}%
\pgfpathmoveto{\pgfqpoint{2.053545in}{2.812478in}}%
\pgfpathlineto{\pgfqpoint{5.724584in}{3.164146in}}%
\pgfpathlineto{\pgfqpoint{5.766577in}{5.881325in}}%
\pgfusepath{stroke}%
\end{pgfscope}%
\begin{pgfscope}%
\pgfsetbuttcap%
\pgfsetroundjoin%
\pgfsetlinewidth{0.803000pt}%
\definecolor{currentstroke}{rgb}{0.690196,0.690196,0.690196}%
\pgfsetstrokecolor{currentstroke}%
\pgfsetdash{}{0pt}%
\pgfpathmoveto{\pgfqpoint{2.197756in}{2.581627in}}%
\pgfpathlineto{\pgfqpoint{5.931736in}{2.946477in}}%
\pgfpathlineto{\pgfqpoint{5.980268in}{5.708700in}}%
\pgfusepath{stroke}%
\end{pgfscope}%
\begin{pgfscope}%
\pgfsetbuttcap%
\pgfsetroundjoin%
\pgfsetlinewidth{0.803000pt}%
\definecolor{currentstroke}{rgb}{0.690196,0.690196,0.690196}%
\pgfsetstrokecolor{currentstroke}%
\pgfsetdash{}{0pt}%
\pgfpathmoveto{\pgfqpoint{2.347506in}{2.341908in}}%
\pgfpathlineto{\pgfqpoint{6.146608in}{2.720695in}}%
\pgfpathlineto{\pgfqpoint{6.202148in}{5.529460in}}%
\pgfusepath{stroke}%
\end{pgfscope}%
\begin{pgfscope}%
\pgfsetbuttcap%
\pgfsetroundjoin%
\pgfsetlinewidth{0.803000pt}%
\definecolor{currentstroke}{rgb}{0.690196,0.690196,0.690196}%
\pgfsetstrokecolor{currentstroke}%
\pgfsetdash{}{0pt}%
\pgfpathmoveto{\pgfqpoint{2.503121in}{2.092801in}}%
\pgfpathlineto{\pgfqpoint{6.369640in}{2.486338in}}%
\pgfpathlineto{\pgfqpoint{6.432698in}{5.343216in}}%
\pgfusepath{stroke}%
\end{pgfscope}%
\begin{pgfscope}%
\pgfsetbuttcap%
\pgfsetroundjoin%
\pgfsetlinewidth{0.803000pt}%
\definecolor{currentstroke}{rgb}{0.690196,0.690196,0.690196}%
\pgfsetstrokecolor{currentstroke}%
\pgfsetdash{}{0pt}%
\pgfpathmoveto{\pgfqpoint{2.664953in}{1.833743in}}%
\pgfpathlineto{\pgfqpoint{6.601306in}{2.242910in}}%
\pgfpathlineto{\pgfqpoint{6.672435in}{5.149551in}}%
\pgfusepath{stroke}%
\end{pgfscope}%
\begin{pgfscope}%
\pgfsetrectcap%
\pgfsetroundjoin%
\pgfsetlinewidth{0.803000pt}%
\definecolor{currentstroke}{rgb}{0.000000,0.000000,0.000000}%
\pgfsetstrokecolor{currentstroke}%
\pgfsetdash{}{0pt}%
\pgfpathmoveto{\pgfqpoint{1.944285in}{3.037736in}}%
\pgfpathlineto{\pgfqpoint{1.855106in}{3.029358in}}%
\pgfusepath{stroke}%
\end{pgfscope}%
\begin{pgfscope}%
\definecolor{textcolor}{rgb}{0.000000,0.000000,0.000000}%
\pgfsetstrokecolor{textcolor}%
\pgfsetfillcolor{textcolor}%
\pgftext[x=1.732638in,y=2.884612in,,top]{\color{textcolor}\rmfamily\fontsize{12.000000}{14.400000}\selectfont 0.0}%
\end{pgfscope}%
\begin{pgfscope}%
\pgfsetrectcap%
\pgfsetroundjoin%
\pgfsetlinewidth{0.803000pt}%
\definecolor{currentstroke}{rgb}{0.000000,0.000000,0.000000}%
\pgfsetstrokecolor{currentstroke}%
\pgfsetdash{}{0pt}%
\pgfpathmoveto{\pgfqpoint{2.083774in}{2.815374in}}%
\pgfpathlineto{\pgfqpoint{1.993044in}{2.806682in}}%
\pgfusepath{stroke}%
\end{pgfscope}%
\begin{pgfscope}%
\definecolor{textcolor}{rgb}{0.000000,0.000000,0.000000}%
\pgfsetstrokecolor{textcolor}%
\pgfsetfillcolor{textcolor}%
\pgftext[x=1.868261in,y=2.659339in,,top]{\color{textcolor}\rmfamily\fontsize{12.000000}{14.400000}\selectfont 0.2}%
\end{pgfscope}%
\begin{pgfscope}%
\pgfsetrectcap%
\pgfsetroundjoin%
\pgfsetlinewidth{0.803000pt}%
\definecolor{currentstroke}{rgb}{0.000000,0.000000,0.000000}%
\pgfsetstrokecolor{currentstroke}%
\pgfsetdash{}{0pt}%
\pgfpathmoveto{\pgfqpoint{2.228520in}{2.584633in}}%
\pgfpathlineto{\pgfqpoint{2.136183in}{2.575611in}}%
\pgfusepath{stroke}%
\end{pgfscope}%
\begin{pgfscope}%
\definecolor{textcolor}{rgb}{0.000000,0.000000,0.000000}%
\pgfsetstrokecolor{textcolor}%
\pgfsetfillcolor{textcolor}%
\pgftext[x=2.008996in,y=2.425576in,,top]{\color{textcolor}\rmfamily\fontsize{12.000000}{14.400000}\selectfont 0.4}%
\end{pgfscope}%
\begin{pgfscope}%
\pgfsetrectcap%
\pgfsetroundjoin%
\pgfsetlinewidth{0.803000pt}%
\definecolor{currentstroke}{rgb}{0.000000,0.000000,0.000000}%
\pgfsetstrokecolor{currentstroke}%
\pgfsetdash{}{0pt}%
\pgfpathmoveto{\pgfqpoint{2.378824in}{2.345031in}}%
\pgfpathlineto{\pgfqpoint{2.284824in}{2.335659in}}%
\pgfusepath{stroke}%
\end{pgfscope}%
\begin{pgfscope}%
\definecolor{textcolor}{rgb}{0.000000,0.000000,0.000000}%
\pgfsetstrokecolor{textcolor}%
\pgfsetfillcolor{textcolor}%
\pgftext[x=2.155138in,y=2.182832in,,top]{\color{textcolor}\rmfamily\fontsize{12.000000}{14.400000}\selectfont 0.6}%
\end{pgfscope}%
\begin{pgfscope}%
\pgfsetrectcap%
\pgfsetroundjoin%
\pgfsetlinewidth{0.803000pt}%
\definecolor{currentstroke}{rgb}{0.000000,0.000000,0.000000}%
\pgfsetstrokecolor{currentstroke}%
\pgfsetdash{}{0pt}%
\pgfpathmoveto{\pgfqpoint{2.535013in}{2.096047in}}%
\pgfpathlineto{\pgfqpoint{2.439289in}{2.086304in}}%
\pgfusepath{stroke}%
\end{pgfscope}%
\begin{pgfscope}%
\definecolor{textcolor}{rgb}{0.000000,0.000000,0.000000}%
\pgfsetstrokecolor{textcolor}%
\pgfsetfillcolor{textcolor}%
\pgftext[x=2.307004in,y=1.930579in,,top]{\color{textcolor}\rmfamily\fontsize{12.000000}{14.400000}\selectfont 0.8}%
\end{pgfscope}%
\begin{pgfscope}%
\pgfsetrectcap%
\pgfsetroundjoin%
\pgfsetlinewidth{0.803000pt}%
\definecolor{currentstroke}{rgb}{0.000000,0.000000,0.000000}%
\pgfsetstrokecolor{currentstroke}%
\pgfsetdash{}{0pt}%
\pgfpathmoveto{\pgfqpoint{2.697441in}{1.837120in}}%
\pgfpathlineto{\pgfqpoint{2.599928in}{1.826984in}}%
\pgfusepath{stroke}%
\end{pgfscope}%
\begin{pgfscope}%
\definecolor{textcolor}{rgb}{0.000000,0.000000,0.000000}%
\pgfsetstrokecolor{textcolor}%
\pgfsetfillcolor{textcolor}%
\pgftext[x=2.464937in,y=1.668249in,,top]{\color{textcolor}\rmfamily\fontsize{12.000000}{14.400000}\selectfont 1.0}%
\end{pgfscope}%
\begin{pgfscope}%
\pgfsetrectcap%
\pgfsetroundjoin%
\pgfsetlinewidth{0.803000pt}%
\definecolor{currentstroke}{rgb}{0.000000,0.000000,0.000000}%
\pgfsetstrokecolor{currentstroke}%
\pgfsetdash{}{0pt}%
\pgfpathmoveto{\pgfqpoint{6.688015in}{2.151799in}}%
\pgfpathlineto{\pgfqpoint{2.725574in}{1.736701in}}%
\pgfusepath{stroke}%
\end{pgfscope}%
\begin{pgfscope}%
\pgfsetbuttcap%
\pgfsetroundjoin%
\pgfsetlinewidth{0.803000pt}%
\definecolor{currentstroke}{rgb}{0.690196,0.690196,0.690196}%
\pgfsetstrokecolor{currentstroke}%
\pgfsetdash{}{0pt}%
\pgfpathmoveto{\pgfqpoint{2.042759in}{5.858962in}}%
\pgfpathlineto{\pgfqpoint{2.099663in}{3.135935in}}%
\pgfpathlineto{\pgfqpoint{2.985305in}{1.763910in}}%
\pgfusepath{stroke}%
\end{pgfscope}%
\begin{pgfscope}%
\pgfsetbuttcap%
\pgfsetroundjoin%
\pgfsetlinewidth{0.803000pt}%
\definecolor{currentstroke}{rgb}{0.690196,0.690196,0.690196}%
\pgfsetstrokecolor{currentstroke}%
\pgfsetdash{}{0pt}%
\pgfpathmoveto{\pgfqpoint{2.699251in}{5.906207in}}%
\pgfpathlineto{\pgfqpoint{2.738505in}{3.195539in}}%
\pgfpathlineto{\pgfqpoint{3.692296in}{1.837973in}}%
\pgfusepath{stroke}%
\end{pgfscope}%
\begin{pgfscope}%
\pgfsetbuttcap%
\pgfsetroundjoin%
\pgfsetlinewidth{0.803000pt}%
\definecolor{currentstroke}{rgb}{0.690196,0.690196,0.690196}%
\pgfsetstrokecolor{currentstroke}%
\pgfsetdash{}{0pt}%
\pgfpathmoveto{\pgfqpoint{3.349037in}{5.952969in}}%
\pgfpathlineto{\pgfqpoint{3.370994in}{3.254550in}}%
\pgfpathlineto{\pgfqpoint{4.391455in}{1.911216in}}%
\pgfusepath{stroke}%
\end{pgfscope}%
\begin{pgfscope}%
\pgfsetbuttcap%
\pgfsetroundjoin%
\pgfsetlinewidth{0.803000pt}%
\definecolor{currentstroke}{rgb}{0.690196,0.690196,0.690196}%
\pgfsetstrokecolor{currentstroke}%
\pgfsetdash{}{0pt}%
\pgfpathmoveto{\pgfqpoint{3.992220in}{5.999257in}}%
\pgfpathlineto{\pgfqpoint{3.997223in}{3.312977in}}%
\pgfpathlineto{\pgfqpoint{5.082914in}{1.983652in}}%
\pgfusepath{stroke}%
\end{pgfscope}%
\begin{pgfscope}%
\pgfsetbuttcap%
\pgfsetroundjoin%
\pgfsetlinewidth{0.803000pt}%
\definecolor{currentstroke}{rgb}{0.690196,0.690196,0.690196}%
\pgfsetstrokecolor{currentstroke}%
\pgfsetdash{}{0pt}%
\pgfpathmoveto{\pgfqpoint{4.628901in}{6.045076in}}%
\pgfpathlineto{\pgfqpoint{4.617286in}{3.370828in}}%
\pgfpathlineto{\pgfqpoint{5.766797in}{2.055294in}}%
\pgfusepath{stroke}%
\end{pgfscope}%
\begin{pgfscope}%
\pgfsetbuttcap%
\pgfsetroundjoin%
\pgfsetlinewidth{0.803000pt}%
\definecolor{currentstroke}{rgb}{0.690196,0.690196,0.690196}%
\pgfsetstrokecolor{currentstroke}%
\pgfsetdash{}{0pt}%
\pgfpathmoveto{\pgfqpoint{5.259176in}{6.090435in}}%
\pgfpathlineto{\pgfqpoint{5.231273in}{3.428113in}}%
\pgfpathlineto{\pgfqpoint{6.443229in}{2.126155in}}%
\pgfusepath{stroke}%
\end{pgfscope}%
\begin{pgfscope}%
\pgfsetrectcap%
\pgfsetroundjoin%
\pgfsetlinewidth{0.803000pt}%
\definecolor{currentstroke}{rgb}{0.000000,0.000000,0.000000}%
\pgfsetstrokecolor{currentstroke}%
\pgfsetdash{}{0pt}%
\pgfpathmoveto{\pgfqpoint{2.977412in}{1.776138in}}%
\pgfpathlineto{\pgfqpoint{3.001134in}{1.739387in}}%
\pgfusepath{stroke}%
\end{pgfscope}%
\begin{pgfscope}%
\definecolor{textcolor}{rgb}{0.000000,0.000000,0.000000}%
\pgfsetstrokecolor{textcolor}%
\pgfsetfillcolor{textcolor}%
\pgftext[x=3.036453in,y=1.541690in,,top]{\color{textcolor}\rmfamily\fontsize{12.000000}{14.400000}\selectfont 0.0}%
\end{pgfscope}%
\begin{pgfscope}%
\pgfsetrectcap%
\pgfsetroundjoin%
\pgfsetlinewidth{0.803000pt}%
\definecolor{currentstroke}{rgb}{0.000000,0.000000,0.000000}%
\pgfsetstrokecolor{currentstroke}%
\pgfsetdash{}{0pt}%
\pgfpathmoveto{\pgfqpoint{3.683800in}{1.850065in}}%
\pgfpathlineto{\pgfqpoint{3.709333in}{1.813723in}}%
\pgfusepath{stroke}%
\end{pgfscope}%
\begin{pgfscope}%
\definecolor{textcolor}{rgb}{0.000000,0.000000,0.000000}%
\pgfsetstrokecolor{textcolor}%
\pgfsetfillcolor{textcolor}%
\pgftext[x=3.746159in,y=1.617311in,,top]{\color{textcolor}\rmfamily\fontsize{12.000000}{14.400000}\selectfont 0.2}%
\end{pgfscope}%
\begin{pgfscope}%
\pgfsetrectcap%
\pgfsetroundjoin%
\pgfsetlinewidth{0.803000pt}%
\definecolor{currentstroke}{rgb}{0.000000,0.000000,0.000000}%
\pgfsetstrokecolor{currentstroke}%
\pgfsetdash{}{0pt}%
\pgfpathmoveto{\pgfqpoint{4.382371in}{1.923174in}}%
\pgfpathlineto{\pgfqpoint{4.409673in}{1.887234in}}%
\pgfusepath{stroke}%
\end{pgfscope}%
\begin{pgfscope}%
\definecolor{textcolor}{rgb}{0.000000,0.000000,0.000000}%
\pgfsetstrokecolor{textcolor}%
\pgfsetfillcolor{textcolor}%
\pgftext[x=4.447971in,y=1.692090in,,top]{\color{textcolor}\rmfamily\fontsize{12.000000}{14.400000}\selectfont 0.4}%
\end{pgfscope}%
\begin{pgfscope}%
\pgfsetrectcap%
\pgfsetroundjoin%
\pgfsetlinewidth{0.803000pt}%
\definecolor{currentstroke}{rgb}{0.000000,0.000000,0.000000}%
\pgfsetstrokecolor{currentstroke}%
\pgfsetdash{}{0pt}%
\pgfpathmoveto{\pgfqpoint{5.073254in}{1.995479in}}%
\pgfpathlineto{\pgfqpoint{5.102285in}{1.959933in}}%
\pgfusepath{stroke}%
\end{pgfscope}%
\begin{pgfscope}%
\definecolor{textcolor}{rgb}{0.000000,0.000000,0.000000}%
\pgfsetstrokecolor{textcolor}%
\pgfsetfillcolor{textcolor}%
\pgftext[x=5.142019in,y=1.766042in,,top]{\color{textcolor}\rmfamily\fontsize{12.000000}{14.400000}\selectfont 0.6}%
\end{pgfscope}%
\begin{pgfscope}%
\pgfsetrectcap%
\pgfsetroundjoin%
\pgfsetlinewidth{0.803000pt}%
\definecolor{currentstroke}{rgb}{0.000000,0.000000,0.000000}%
\pgfsetstrokecolor{currentstroke}%
\pgfsetdash{}{0pt}%
\pgfpathmoveto{\pgfqpoint{5.756575in}{2.066992in}}%
\pgfpathlineto{\pgfqpoint{5.787295in}{2.031835in}}%
\pgfusepath{stroke}%
\end{pgfscope}%
\begin{pgfscope}%
\definecolor{textcolor}{rgb}{0.000000,0.000000,0.000000}%
\pgfsetstrokecolor{textcolor}%
\pgfsetfillcolor{textcolor}%
\pgftext[x=5.828433in,y=1.839180in,,top]{\color{textcolor}\rmfamily\fontsize{12.000000}{14.400000}\selectfont 0.8}%
\end{pgfscope}%
\begin{pgfscope}%
\pgfsetrectcap%
\pgfsetroundjoin%
\pgfsetlinewidth{0.803000pt}%
\definecolor{currentstroke}{rgb}{0.000000,0.000000,0.000000}%
\pgfsetstrokecolor{currentstroke}%
\pgfsetdash{}{0pt}%
\pgfpathmoveto{\pgfqpoint{6.432458in}{2.137726in}}%
\pgfpathlineto{\pgfqpoint{6.464829in}{2.102952in}}%
\pgfusepath{stroke}%
\end{pgfscope}%
\begin{pgfscope}%
\definecolor{textcolor}{rgb}{0.000000,0.000000,0.000000}%
\pgfsetstrokecolor{textcolor}%
\pgfsetfillcolor{textcolor}%
\pgftext[x=6.507337in,y=1.911518in,,top]{\color{textcolor}\rmfamily\fontsize{12.000000}{14.400000}\selectfont 1.0}%
\end{pgfscope}%
\begin{pgfscope}%
\pgfsetrectcap%
\pgfsetroundjoin%
\pgfsetlinewidth{0.803000pt}%
\definecolor{currentstroke}{rgb}{0.000000,0.000000,0.000000}%
\pgfsetstrokecolor{currentstroke}%
\pgfsetdash{}{0pt}%
\pgfpathmoveto{\pgfqpoint{6.688015in}{2.151799in}}%
\pgfpathlineto{\pgfqpoint{6.762234in}{5.077010in}}%
\pgfusepath{stroke}%
\end{pgfscope}%
\begin{pgfscope}%
\pgfsetbuttcap%
\pgfsetroundjoin%
\pgfsetlinewidth{0.803000pt}%
\definecolor{currentstroke}{rgb}{0.690196,0.690196,0.690196}%
\pgfsetstrokecolor{currentstroke}%
\pgfsetdash{}{0pt}%
\pgfpathmoveto{\pgfqpoint{6.689457in}{2.208652in}}%
\pgfpathlineto{\pgfqpoint{5.454287in}{3.500669in}}%
\pgfpathlineto{\pgfqpoint{1.863917in}{3.167182in}}%
\pgfusepath{stroke}%
\end{pgfscope}%
\begin{pgfscope}%
\pgfsetbuttcap%
\pgfsetroundjoin%
\pgfsetlinewidth{0.803000pt}%
\definecolor{currentstroke}{rgb}{0.690196,0.690196,0.690196}%
\pgfsetstrokecolor{currentstroke}%
\pgfsetdash{}{0pt}%
\pgfpathmoveto{\pgfqpoint{6.697179in}{2.512979in}}%
\pgfpathlineto{\pgfqpoint{5.457808in}{3.777902in}}%
\pgfpathlineto{\pgfqpoint{1.857306in}{3.451483in}}%
\pgfusepath{stroke}%
\end{pgfscope}%
\begin{pgfscope}%
\pgfsetbuttcap%
\pgfsetroundjoin%
\pgfsetlinewidth{0.803000pt}%
\definecolor{currentstroke}{rgb}{0.690196,0.690196,0.690196}%
\pgfsetstrokecolor{currentstroke}%
\pgfsetdash{}{0pt}%
\pgfpathmoveto{\pgfqpoint{6.704948in}{2.819187in}}%
\pgfpathlineto{\pgfqpoint{5.461349in}{4.056675in}}%
\pgfpathlineto{\pgfqpoint{1.850656in}{3.737410in}}%
\pgfusepath{stroke}%
\end{pgfscope}%
\begin{pgfscope}%
\pgfsetbuttcap%
\pgfsetroundjoin%
\pgfsetlinewidth{0.803000pt}%
\definecolor{currentstroke}{rgb}{0.690196,0.690196,0.690196}%
\pgfsetstrokecolor{currentstroke}%
\pgfsetdash{}{0pt}%
\pgfpathmoveto{\pgfqpoint{6.712765in}{3.127293in}}%
\pgfpathlineto{\pgfqpoint{5.464910in}{4.337001in}}%
\pgfpathlineto{\pgfqpoint{1.843969in}{4.024976in}}%
\pgfusepath{stroke}%
\end{pgfscope}%
\begin{pgfscope}%
\pgfsetbuttcap%
\pgfsetroundjoin%
\pgfsetlinewidth{0.803000pt}%
\definecolor{currentstroke}{rgb}{0.690196,0.690196,0.690196}%
\pgfsetstrokecolor{currentstroke}%
\pgfsetdash{}{0pt}%
\pgfpathmoveto{\pgfqpoint{6.720631in}{3.437316in}}%
\pgfpathlineto{\pgfqpoint{5.468490in}{4.618893in}}%
\pgfpathlineto{\pgfqpoint{1.837244in}{4.314195in}}%
\pgfusepath{stroke}%
\end{pgfscope}%
\begin{pgfscope}%
\pgfsetbuttcap%
\pgfsetroundjoin%
\pgfsetlinewidth{0.803000pt}%
\definecolor{currentstroke}{rgb}{0.690196,0.690196,0.690196}%
\pgfsetstrokecolor{currentstroke}%
\pgfsetdash{}{0pt}%
\pgfpathmoveto{\pgfqpoint{6.728546in}{3.749273in}}%
\pgfpathlineto{\pgfqpoint{5.472091in}{4.902365in}}%
\pgfpathlineto{\pgfqpoint{1.830479in}{4.605082in}}%
\pgfusepath{stroke}%
\end{pgfscope}%
\begin{pgfscope}%
\pgfsetbuttcap%
\pgfsetroundjoin%
\pgfsetlinewidth{0.803000pt}%
\definecolor{currentstroke}{rgb}{0.690196,0.690196,0.690196}%
\pgfsetstrokecolor{currentstroke}%
\pgfsetdash{}{0pt}%
\pgfpathmoveto{\pgfqpoint{6.736511in}{4.063182in}}%
\pgfpathlineto{\pgfqpoint{5.475711in}{5.187430in}}%
\pgfpathlineto{\pgfqpoint{1.823676in}{4.897652in}}%
\pgfusepath{stroke}%
\end{pgfscope}%
\begin{pgfscope}%
\pgfsetbuttcap%
\pgfsetroundjoin%
\pgfsetlinewidth{0.803000pt}%
\definecolor{currentstroke}{rgb}{0.690196,0.690196,0.690196}%
\pgfsetstrokecolor{currentstroke}%
\pgfsetdash{}{0pt}%
\pgfpathmoveto{\pgfqpoint{6.744525in}{4.379062in}}%
\pgfpathlineto{\pgfqpoint{5.479353in}{5.474100in}}%
\pgfpathlineto{\pgfqpoint{1.816833in}{5.191918in}}%
\pgfusepath{stroke}%
\end{pgfscope}%
\begin{pgfscope}%
\pgfsetbuttcap%
\pgfsetroundjoin%
\pgfsetlinewidth{0.803000pt}%
\definecolor{currentstroke}{rgb}{0.690196,0.690196,0.690196}%
\pgfsetstrokecolor{currentstroke}%
\pgfsetdash{}{0pt}%
\pgfpathmoveto{\pgfqpoint{6.752590in}{4.696931in}}%
\pgfpathlineto{\pgfqpoint{5.483014in}{5.762391in}}%
\pgfpathlineto{\pgfqpoint{1.809950in}{5.487897in}}%
\pgfusepath{stroke}%
\end{pgfscope}%
\begin{pgfscope}%
\pgfsetbuttcap%
\pgfsetroundjoin%
\pgfsetlinewidth{0.803000pt}%
\definecolor{currentstroke}{rgb}{0.690196,0.690196,0.690196}%
\pgfsetstrokecolor{currentstroke}%
\pgfsetdash{}{0pt}%
\pgfpathmoveto{\pgfqpoint{6.760706in}{5.016808in}}%
\pgfpathlineto{\pgfqpoint{5.486697in}{6.052315in}}%
\pgfpathlineto{\pgfqpoint{1.803027in}{5.785602in}}%
\pgfusepath{stroke}%
\end{pgfscope}%
\begin{pgfscope}%
\pgfsetrectcap%
\pgfsetroundjoin%
\pgfsetlinewidth{0.803000pt}%
\definecolor{currentstroke}{rgb}{0.000000,0.000000,0.000000}%
\pgfsetstrokecolor{currentstroke}%
\pgfsetdash{}{0pt}%
\pgfpathmoveto{\pgfqpoint{6.678481in}{2.220133in}}%
\pgfpathlineto{\pgfqpoint{6.711468in}{2.185629in}}%
\pgfusepath{stroke}%
\end{pgfscope}%
\begin{pgfscope}%
\definecolor{textcolor}{rgb}{0.000000,0.000000,0.000000}%
\pgfsetstrokecolor{textcolor}%
\pgfsetfillcolor{textcolor}%
\pgftext[x=6.950093in,y=2.156880in,,top]{\color{textcolor}\rmfamily\fontsize{12.000000}{14.400000}\selectfont -0.10}%
\end{pgfscope}%
\begin{pgfscope}%
\pgfsetrectcap%
\pgfsetroundjoin%
\pgfsetlinewidth{0.803000pt}%
\definecolor{currentstroke}{rgb}{0.000000,0.000000,0.000000}%
\pgfsetstrokecolor{currentstroke}%
\pgfsetdash{}{0pt}%
\pgfpathmoveto{\pgfqpoint{6.686162in}{2.524223in}}%
\pgfpathlineto{\pgfqpoint{6.719271in}{2.490431in}}%
\pgfusepath{stroke}%
\end{pgfscope}%
\begin{pgfscope}%
\definecolor{textcolor}{rgb}{0.000000,0.000000,0.000000}%
\pgfsetstrokecolor{textcolor}%
\pgfsetfillcolor{textcolor}%
\pgftext[x=6.958652in,y=2.462277in,,top]{\color{textcolor}\rmfamily\fontsize{12.000000}{14.400000}\selectfont 0.07}%
\end{pgfscope}%
\begin{pgfscope}%
\pgfsetrectcap%
\pgfsetroundjoin%
\pgfsetlinewidth{0.803000pt}%
\definecolor{currentstroke}{rgb}{0.000000,0.000000,0.000000}%
\pgfsetstrokecolor{currentstroke}%
\pgfsetdash{}{0pt}%
\pgfpathmoveto{\pgfqpoint{6.693890in}{2.830190in}}%
\pgfpathlineto{\pgfqpoint{6.727122in}{2.797121in}}%
\pgfusepath{stroke}%
\end{pgfscope}%
\begin{pgfscope}%
\definecolor{textcolor}{rgb}{0.000000,0.000000,0.000000}%
\pgfsetstrokecolor{textcolor}%
\pgfsetfillcolor{textcolor}%
\pgftext[x=6.967264in,y=2.769568in,,top]{\color{textcolor}\rmfamily\fontsize{12.000000}{14.400000}\selectfont 0.23}%
\end{pgfscope}%
\begin{pgfscope}%
\pgfsetrectcap%
\pgfsetroundjoin%
\pgfsetlinewidth{0.803000pt}%
\definecolor{currentstroke}{rgb}{0.000000,0.000000,0.000000}%
\pgfsetstrokecolor{currentstroke}%
\pgfsetdash{}{0pt}%
\pgfpathmoveto{\pgfqpoint{6.701666in}{3.138053in}}%
\pgfpathlineto{\pgfqpoint{6.735023in}{3.105716in}}%
\pgfusepath{stroke}%
\end{pgfscope}%
\begin{pgfscope}%
\definecolor{textcolor}{rgb}{0.000000,0.000000,0.000000}%
\pgfsetstrokecolor{textcolor}%
\pgfsetfillcolor{textcolor}%
\pgftext[x=6.975929in,y=3.078773in,,top]{\color{textcolor}\rmfamily\fontsize{12.000000}{14.400000}\selectfont 0.40}%
\end{pgfscope}%
\begin{pgfscope}%
\pgfsetrectcap%
\pgfsetroundjoin%
\pgfsetlinewidth{0.803000pt}%
\definecolor{currentstroke}{rgb}{0.000000,0.000000,0.000000}%
\pgfsetstrokecolor{currentstroke}%
\pgfsetdash{}{0pt}%
\pgfpathmoveto{\pgfqpoint{6.709491in}{3.447829in}}%
\pgfpathlineto{\pgfqpoint{6.742972in}{3.416234in}}%
\pgfusepath{stroke}%
\end{pgfscope}%
\begin{pgfscope}%
\definecolor{textcolor}{rgb}{0.000000,0.000000,0.000000}%
\pgfsetstrokecolor{textcolor}%
\pgfsetfillcolor{textcolor}%
\pgftext[x=6.984649in,y=3.389908in,,top]{\color{textcolor}\rmfamily\fontsize{12.000000}{14.400000}\selectfont 0.57}%
\end{pgfscope}%
\begin{pgfscope}%
\pgfsetrectcap%
\pgfsetroundjoin%
\pgfsetlinewidth{0.803000pt}%
\definecolor{currentstroke}{rgb}{0.000000,0.000000,0.000000}%
\pgfsetstrokecolor{currentstroke}%
\pgfsetdash{}{0pt}%
\pgfpathmoveto{\pgfqpoint{6.717364in}{3.759535in}}%
\pgfpathlineto{\pgfqpoint{6.750972in}{3.728692in}}%
\pgfusepath{stroke}%
\end{pgfscope}%
\begin{pgfscope}%
\definecolor{textcolor}{rgb}{0.000000,0.000000,0.000000}%
\pgfsetstrokecolor{textcolor}%
\pgfsetfillcolor{textcolor}%
\pgftext[x=6.993423in,y=3.702993in,,top]{\color{textcolor}\rmfamily\fontsize{12.000000}{14.400000}\selectfont 0.73}%
\end{pgfscope}%
\begin{pgfscope}%
\pgfsetrectcap%
\pgfsetroundjoin%
\pgfsetlinewidth{0.803000pt}%
\definecolor{currentstroke}{rgb}{0.000000,0.000000,0.000000}%
\pgfsetstrokecolor{currentstroke}%
\pgfsetdash{}{0pt}%
\pgfpathmoveto{\pgfqpoint{6.725286in}{4.073191in}}%
\pgfpathlineto{\pgfqpoint{6.759021in}{4.043110in}}%
\pgfusepath{stroke}%
\end{pgfscope}%
\begin{pgfscope}%
\definecolor{textcolor}{rgb}{0.000000,0.000000,0.000000}%
\pgfsetstrokecolor{textcolor}%
\pgfsetfillcolor{textcolor}%
\pgftext[x=7.002253in,y=4.018044in,,top]{\color{textcolor}\rmfamily\fontsize{12.000000}{14.400000}\selectfont 0.90}%
\end{pgfscope}%
\begin{pgfscope}%
\pgfsetrectcap%
\pgfsetroundjoin%
\pgfsetlinewidth{0.803000pt}%
\definecolor{currentstroke}{rgb}{0.000000,0.000000,0.000000}%
\pgfsetstrokecolor{currentstroke}%
\pgfsetdash{}{0pt}%
\pgfpathmoveto{\pgfqpoint{6.733258in}{4.388814in}}%
\pgfpathlineto{\pgfqpoint{6.767121in}{4.359505in}}%
\pgfusepath{stroke}%
\end{pgfscope}%
\begin{pgfscope}%
\definecolor{textcolor}{rgb}{0.000000,0.000000,0.000000}%
\pgfsetstrokecolor{textcolor}%
\pgfsetfillcolor{textcolor}%
\pgftext[x=7.011138in,y=4.335082in,,top]{\color{textcolor}\rmfamily\fontsize{12.000000}{14.400000}\selectfont 1.07}%
\end{pgfscope}%
\begin{pgfscope}%
\pgfsetrectcap%
\pgfsetroundjoin%
\pgfsetlinewidth{0.803000pt}%
\definecolor{currentstroke}{rgb}{0.000000,0.000000,0.000000}%
\pgfsetstrokecolor{currentstroke}%
\pgfsetdash{}{0pt}%
\pgfpathmoveto{\pgfqpoint{6.741280in}{4.706422in}}%
\pgfpathlineto{\pgfqpoint{6.775272in}{4.677896in}}%
\pgfusepath{stroke}%
\end{pgfscope}%
\begin{pgfscope}%
\definecolor{textcolor}{rgb}{0.000000,0.000000,0.000000}%
\pgfsetstrokecolor{textcolor}%
\pgfsetfillcolor{textcolor}%
\pgftext[x=7.020079in,y=4.654125in,,top]{\color{textcolor}\rmfamily\fontsize{12.000000}{14.400000}\selectfont 1.23}%
\end{pgfscope}%
\begin{pgfscope}%
\pgfsetrectcap%
\pgfsetroundjoin%
\pgfsetlinewidth{0.803000pt}%
\definecolor{currentstroke}{rgb}{0.000000,0.000000,0.000000}%
\pgfsetstrokecolor{currentstroke}%
\pgfsetdash{}{0pt}%
\pgfpathmoveto{\pgfqpoint{6.749353in}{5.026036in}}%
\pgfpathlineto{\pgfqpoint{6.783475in}{4.998302in}}%
\pgfusepath{stroke}%
\end{pgfscope}%
\begin{pgfscope}%
\definecolor{textcolor}{rgb}{0.000000,0.000000,0.000000}%
\pgfsetstrokecolor{textcolor}%
\pgfsetfillcolor{textcolor}%
\pgftext[x=7.029077in,y=4.975191in,,top]{\color{textcolor}\rmfamily\fontsize{12.000000}{14.400000}\selectfont 1.40}%
\end{pgfscope}%
\begin{pgfscope}%
\pgfpathrectangle{\pgfqpoint{1.020000in}{0.880000in}}{\pgfqpoint{6.160000in}{6.160000in}}%
\pgfusepath{clip}%
\pgfsetbuttcap%
\pgfsetroundjoin%
\definecolor{currentfill}{rgb}{0.532568,0.669801,0.990393}%
\pgfsetfillcolor{currentfill}%
\pgfsetlinewidth{0.000000pt}%
\definecolor{currentstroke}{rgb}{0.000000,0.000000,0.000000}%
\pgfsetstrokecolor{currentstroke}%
\pgfsetdash{}{0pt}%
\pgfpathmoveto{\pgfqpoint{5.276765in}{3.968864in}}%
\pgfpathlineto{\pgfqpoint{5.287811in}{4.073960in}}%
\pgfpathlineto{\pgfqpoint{5.298947in}{4.183599in}}%
\pgfpathlineto{\pgfqpoint{5.328200in}{4.001132in}}%
\pgfpathlineto{\pgfqpoint{5.319995in}{4.155243in}}%
\pgfpathlineto{\pgfqpoint{5.307655in}{3.940598in}}%
\pgfpathlineto{\pgfqpoint{5.276765in}{3.968864in}}%
\pgfpathclose%
\pgfusepath{fill}%
\end{pgfscope}%
\begin{pgfscope}%
\pgfpathrectangle{\pgfqpoint{1.020000in}{0.880000in}}{\pgfqpoint{6.160000in}{6.160000in}}%
\pgfusepath{clip}%
\pgfsetbuttcap%
\pgfsetroundjoin%
\definecolor{currentfill}{rgb}{0.510824,0.649397,0.985079}%
\pgfsetfillcolor{currentfill}%
\pgfsetlinewidth{0.000000pt}%
\definecolor{currentstroke}{rgb}{0.000000,0.000000,0.000000}%
\pgfsetstrokecolor{currentstroke}%
\pgfsetdash{}{0pt}%
\pgfpathmoveto{\pgfqpoint{5.213161in}{3.857760in}}%
\pgfpathlineto{\pgfqpoint{5.223760in}{3.932579in}}%
\pgfpathlineto{\pgfqpoint{5.234728in}{4.039387in}}%
\pgfpathlineto{\pgfqpoint{5.266794in}{4.108704in}}%
\pgfpathlineto{\pgfqpoint{5.298947in}{4.183599in}}%
\pgfpathlineto{\pgfqpoint{5.287811in}{4.073960in}}%
\pgfpathlineto{\pgfqpoint{5.276765in}{3.968864in}}%
\pgfpathlineto{\pgfqpoint{5.245527in}{3.967461in}}%
\pgfpathlineto{\pgfqpoint{5.213161in}{3.857760in}}%
\pgfpathclose%
\pgfusepath{fill}%
\end{pgfscope}%
\begin{pgfscope}%
\pgfpathrectangle{\pgfqpoint{1.020000in}{0.880000in}}{\pgfqpoint{6.160000in}{6.160000in}}%
\pgfusepath{clip}%
\pgfsetbuttcap%
\pgfsetroundjoin%
\definecolor{currentfill}{rgb}{0.521696,0.659599,0.987736}%
\pgfsetfillcolor{currentfill}%
\pgfsetlinewidth{0.000000pt}%
\definecolor{currentstroke}{rgb}{0.000000,0.000000,0.000000}%
\pgfsetstrokecolor{currentstroke}%
\pgfsetdash{}{0pt}%
\pgfpathmoveto{\pgfqpoint{5.153080in}{4.105227in}}%
\pgfpathlineto{\pgfqpoint{5.162012in}{4.016848in}}%
\pgfpathlineto{\pgfqpoint{5.171822in}{4.017673in}}%
\pgfpathlineto{\pgfqpoint{5.204379in}{4.136829in}}%
\pgfpathlineto{\pgfqpoint{5.234728in}{4.039387in}}%
\pgfpathlineto{\pgfqpoint{5.223760in}{3.932579in}}%
\pgfpathlineto{\pgfqpoint{5.213161in}{3.857760in}}%
\pgfpathlineto{\pgfqpoint{5.184352in}{4.100589in}}%
\pgfpathlineto{\pgfqpoint{5.153080in}{4.105227in}}%
\pgfpathclose%
\pgfusepath{fill}%
\end{pgfscope}%
\begin{pgfscope}%
\pgfpathrectangle{\pgfqpoint{1.020000in}{0.880000in}}{\pgfqpoint{6.160000in}{6.160000in}}%
\pgfusepath{clip}%
\pgfsetbuttcap%
\pgfsetroundjoin%
\definecolor{currentfill}{rgb}{0.543440,0.680003,0.993051}%
\pgfsetfillcolor{currentfill}%
\pgfsetlinewidth{0.000000pt}%
\definecolor{currentstroke}{rgb}{0.000000,0.000000,0.000000}%
\pgfsetstrokecolor{currentstroke}%
\pgfsetdash{}{0pt}%
\pgfpathmoveto{\pgfqpoint{5.089581in}{4.015281in}}%
\pgfpathlineto{\pgfqpoint{5.101273in}{4.231187in}}%
\pgfpathlineto{\pgfqpoint{5.110301in}{4.151075in}}%
\pgfpathlineto{\pgfqpoint{5.139021in}{3.866847in}}%
\pgfpathlineto{\pgfqpoint{5.171822in}{4.017673in}}%
\pgfpathlineto{\pgfqpoint{5.162012in}{4.016848in}}%
\pgfpathlineto{\pgfqpoint{5.153080in}{4.105227in}}%
\pgfpathlineto{\pgfqpoint{5.120578in}{3.980018in}}%
\pgfpathlineto{\pgfqpoint{5.089581in}{4.015281in}}%
\pgfpathclose%
\pgfusepath{fill}%
\end{pgfscope}%
\begin{pgfscope}%
\pgfpathrectangle{\pgfqpoint{1.020000in}{0.880000in}}{\pgfqpoint{6.160000in}{6.160000in}}%
\pgfusepath{clip}%
\pgfsetbuttcap%
\pgfsetroundjoin%
\definecolor{currentfill}{rgb}{0.500031,0.638508,0.981070}%
\pgfsetfillcolor{currentfill}%
\pgfsetlinewidth{0.000000pt}%
\definecolor{currentstroke}{rgb}{0.000000,0.000000,0.000000}%
\pgfsetstrokecolor{currentstroke}%
\pgfsetdash{}{0pt}%
\pgfpathmoveto{\pgfqpoint{4.963611in}{3.951512in}}%
\pgfpathlineto{\pgfqpoint{4.973724in}{4.016162in}}%
\pgfpathlineto{\pgfqpoint{4.984462in}{4.155476in}}%
\pgfpathlineto{\pgfqpoint{5.013226in}{3.825016in}}%
\pgfpathlineto{\pgfqpoint{5.045873in}{3.973209in}}%
\pgfpathlineto{\pgfqpoint{5.036776in}{4.042286in}}%
\pgfpathlineto{\pgfqpoint{5.026118in}{3.924610in}}%
\pgfpathlineto{\pgfqpoint{4.994659in}{3.910954in}}%
\pgfpathlineto{\pgfqpoint{4.963611in}{3.951512in}}%
\pgfpathclose%
\pgfusepath{fill}%
\end{pgfscope}%
\begin{pgfscope}%
\pgfpathrectangle{\pgfqpoint{1.020000in}{0.880000in}}{\pgfqpoint{6.160000in}{6.160000in}}%
\pgfusepath{clip}%
\pgfsetbuttcap%
\pgfsetroundjoin%
\definecolor{currentfill}{rgb}{0.538004,0.674902,0.991722}%
\pgfsetfillcolor{currentfill}%
\pgfsetlinewidth{0.000000pt}%
\definecolor{currentstroke}{rgb}{0.000000,0.000000,0.000000}%
\pgfsetstrokecolor{currentstroke}%
\pgfsetdash{}{0pt}%
\pgfpathmoveto{\pgfqpoint{5.026118in}{3.924610in}}%
\pgfpathlineto{\pgfqpoint{5.036776in}{4.042286in}}%
\pgfpathlineto{\pgfqpoint{5.045873in}{3.973209in}}%
\pgfpathlineto{\pgfqpoint{5.077521in}{4.000631in}}%
\pgfpathlineto{\pgfqpoint{5.110301in}{4.151075in}}%
\pgfpathlineto{\pgfqpoint{5.101273in}{4.231187in}}%
\pgfpathlineto{\pgfqpoint{5.089581in}{4.015281in}}%
\pgfpathlineto{\pgfqpoint{5.058007in}{3.988628in}}%
\pgfpathlineto{\pgfqpoint{5.026118in}{3.924610in}}%
\pgfpathclose%
\pgfusepath{fill}%
\end{pgfscope}%
\begin{pgfscope}%
\pgfpathrectangle{\pgfqpoint{1.020000in}{0.880000in}}{\pgfqpoint{6.160000in}{6.160000in}}%
\pgfusepath{clip}%
\pgfsetbuttcap%
\pgfsetroundjoin%
\definecolor{currentfill}{rgb}{0.538004,0.674902,0.991722}%
\pgfsetfillcolor{currentfill}%
\pgfsetlinewidth{0.000000pt}%
\definecolor{currentstroke}{rgb}{0.000000,0.000000,0.000000}%
\pgfsetstrokecolor{currentstroke}%
\pgfsetdash{}{0pt}%
\pgfpathmoveto{\pgfqpoint{5.234728in}{4.039387in}}%
\pgfpathlineto{\pgfqpoint{5.242678in}{3.856328in}}%
\pgfpathlineto{\pgfqpoint{5.252821in}{3.880661in}}%
\pgfpathlineto{\pgfqpoint{5.284825in}{3.942413in}}%
\pgfpathlineto{\pgfqpoint{5.318214in}{4.125097in}}%
\pgfpathlineto{\pgfqpoint{5.308432in}{4.141735in}}%
\pgfpathlineto{\pgfqpoint{5.298947in}{4.183599in}}%
\pgfpathlineto{\pgfqpoint{5.266794in}{4.108704in}}%
\pgfpathlineto{\pgfqpoint{5.234728in}{4.039387in}}%
\pgfpathclose%
\pgfusepath{fill}%
\end{pgfscope}%
\begin{pgfscope}%
\pgfpathrectangle{\pgfqpoint{1.020000in}{0.880000in}}{\pgfqpoint{6.160000in}{6.160000in}}%
\pgfusepath{clip}%
\pgfsetbuttcap%
\pgfsetroundjoin%
\definecolor{currentfill}{rgb}{0.565182,0.699438,0.996635}%
\pgfsetfillcolor{currentfill}%
\pgfsetlinewidth{0.000000pt}%
\definecolor{currentstroke}{rgb}{0.000000,0.000000,0.000000}%
\pgfsetstrokecolor{currentstroke}%
\pgfsetdash{}{0pt}%
\pgfpathmoveto{\pgfqpoint{5.298947in}{4.183599in}}%
\pgfpathlineto{\pgfqpoint{5.308432in}{4.141735in}}%
\pgfpathlineto{\pgfqpoint{5.318214in}{4.125097in}}%
\pgfpathlineto{\pgfqpoint{5.347737in}{3.965773in}}%
\pgfpathlineto{\pgfqpoint{5.338861in}{4.061867in}}%
\pgfpathlineto{\pgfqpoint{5.328200in}{4.001132in}}%
\pgfpathlineto{\pgfqpoint{5.298947in}{4.183599in}}%
\pgfpathclose%
\pgfusepath{fill}%
\end{pgfscope}%
\begin{pgfscope}%
\pgfpathrectangle{\pgfqpoint{1.020000in}{0.880000in}}{\pgfqpoint{6.160000in}{6.160000in}}%
\pgfusepath{clip}%
\pgfsetbuttcap%
\pgfsetroundjoin%
\definecolor{currentfill}{rgb}{0.516260,0.654498,0.986407}%
\pgfsetfillcolor{currentfill}%
\pgfsetlinewidth{0.000000pt}%
\definecolor{currentstroke}{rgb}{0.000000,0.000000,0.000000}%
\pgfsetstrokecolor{currentstroke}%
\pgfsetdash{}{0pt}%
\pgfpathmoveto{\pgfqpoint{5.171822in}{4.017673in}}%
\pgfpathlineto{\pgfqpoint{5.181771in}{4.030377in}}%
\pgfpathlineto{\pgfqpoint{5.190728in}{3.942035in}}%
\pgfpathlineto{\pgfqpoint{5.223505in}{4.077029in}}%
\pgfpathlineto{\pgfqpoint{5.252821in}{3.880661in}}%
\pgfpathlineto{\pgfqpoint{5.242678in}{3.856328in}}%
\pgfpathlineto{\pgfqpoint{5.234728in}{4.039387in}}%
\pgfpathlineto{\pgfqpoint{5.204379in}{4.136829in}}%
\pgfpathlineto{\pgfqpoint{5.171822in}{4.017673in}}%
\pgfpathclose%
\pgfusepath{fill}%
\end{pgfscope}%
\begin{pgfscope}%
\pgfpathrectangle{\pgfqpoint{1.020000in}{0.880000in}}{\pgfqpoint{6.160000in}{6.160000in}}%
\pgfusepath{clip}%
\pgfsetbuttcap%
\pgfsetroundjoin%
\definecolor{currentfill}{rgb}{0.532568,0.669801,0.990393}%
\pgfsetfillcolor{currentfill}%
\pgfsetlinewidth{0.000000pt}%
\definecolor{currentstroke}{rgb}{0.000000,0.000000,0.000000}%
\pgfsetstrokecolor{currentstroke}%
\pgfsetdash{}{0pt}%
\pgfpathmoveto{\pgfqpoint{5.110301in}{4.151075in}}%
\pgfpathlineto{\pgfqpoint{5.120228in}{4.167090in}}%
\pgfpathlineto{\pgfqpoint{5.129195in}{4.078589in}}%
\pgfpathlineto{\pgfqpoint{5.158329in}{3.839113in}}%
\pgfpathlineto{\pgfqpoint{5.190728in}{3.942035in}}%
\pgfpathlineto{\pgfqpoint{5.181771in}{4.030377in}}%
\pgfpathlineto{\pgfqpoint{5.171822in}{4.017673in}}%
\pgfpathlineto{\pgfqpoint{5.139021in}{3.866847in}}%
\pgfpathlineto{\pgfqpoint{5.110301in}{4.151075in}}%
\pgfpathclose%
\pgfusepath{fill}%
\end{pgfscope}%
\begin{pgfscope}%
\pgfpathrectangle{\pgfqpoint{1.020000in}{0.880000in}}{\pgfqpoint{6.160000in}{6.160000in}}%
\pgfusepath{clip}%
\pgfsetbuttcap%
\pgfsetroundjoin%
\definecolor{currentfill}{rgb}{0.570616,0.704109,0.997195}%
\pgfsetfillcolor{currentfill}%
\pgfsetlinewidth{0.000000pt}%
\definecolor{currentstroke}{rgb}{0.000000,0.000000,0.000000}%
\pgfsetstrokecolor{currentstroke}%
\pgfsetdash{}{0pt}%
\pgfpathmoveto{\pgfqpoint{4.902085in}{4.130313in}}%
\pgfpathlineto{\pgfqpoint{4.912318in}{4.222876in}}%
\pgfpathlineto{\pgfqpoint{4.920790in}{4.070031in}}%
\pgfpathlineto{\pgfqpoint{4.951472in}{3.962357in}}%
\pgfpathlineto{\pgfqpoint{4.984462in}{4.155476in}}%
\pgfpathlineto{\pgfqpoint{4.973724in}{4.016162in}}%
\pgfpathlineto{\pgfqpoint{4.963611in}{3.951512in}}%
\pgfpathlineto{\pgfqpoint{4.933403in}{4.110452in}}%
\pgfpathlineto{\pgfqpoint{4.902085in}{4.130313in}}%
\pgfpathclose%
\pgfusepath{fill}%
\end{pgfscope}%
\begin{pgfscope}%
\pgfpathrectangle{\pgfqpoint{1.020000in}{0.880000in}}{\pgfqpoint{6.160000in}{6.160000in}}%
\pgfusepath{clip}%
\pgfsetbuttcap%
\pgfsetroundjoin%
\definecolor{currentfill}{rgb}{0.559747,0.694768,0.996075}%
\pgfsetfillcolor{currentfill}%
\pgfsetlinewidth{0.000000pt}%
\definecolor{currentstroke}{rgb}{0.000000,0.000000,0.000000}%
\pgfsetstrokecolor{currentstroke}%
\pgfsetdash{}{0pt}%
\pgfpathmoveto{\pgfqpoint{4.838504in}{4.038838in}}%
\pgfpathlineto{\pgfqpoint{4.848001in}{4.038844in}}%
\pgfpathlineto{\pgfqpoint{4.857658in}{4.059683in}}%
\pgfpathlineto{\pgfqpoint{4.888190in}{3.916442in}}%
\pgfpathlineto{\pgfqpoint{4.920790in}{4.070031in}}%
\pgfpathlineto{\pgfqpoint{4.912318in}{4.222876in}}%
\pgfpathlineto{\pgfqpoint{4.902085in}{4.130313in}}%
\pgfpathlineto{\pgfqpoint{4.869253in}{3.932839in}}%
\pgfpathlineto{\pgfqpoint{4.838504in}{4.038838in}}%
\pgfpathclose%
\pgfusepath{fill}%
\end{pgfscope}%
\begin{pgfscope}%
\pgfpathrectangle{\pgfqpoint{1.020000in}{0.880000in}}{\pgfqpoint{6.160000in}{6.160000in}}%
\pgfusepath{clip}%
\pgfsetbuttcap%
\pgfsetroundjoin%
\definecolor{currentfill}{rgb}{0.559747,0.694768,0.996075}%
\pgfsetfillcolor{currentfill}%
\pgfsetlinewidth{0.000000pt}%
\definecolor{currentstroke}{rgb}{0.000000,0.000000,0.000000}%
\pgfsetstrokecolor{currentstroke}%
\pgfsetdash{}{0pt}%
\pgfpathmoveto{\pgfqpoint{5.318214in}{4.125097in}}%
\pgfpathlineto{\pgfqpoint{5.327367in}{4.051799in}}%
\pgfpathlineto{\pgfqpoint{5.337794in}{4.089016in}}%
\pgfpathlineto{\pgfqpoint{5.368442in}{4.023945in}}%
\pgfpathlineto{\pgfqpoint{5.358546in}{4.035022in}}%
\pgfpathlineto{\pgfqpoint{5.347737in}{3.965773in}}%
\pgfpathlineto{\pgfqpoint{5.318214in}{4.125097in}}%
\pgfpathclose%
\pgfusepath{fill}%
\end{pgfscope}%
\begin{pgfscope}%
\pgfpathrectangle{\pgfqpoint{1.020000in}{0.880000in}}{\pgfqpoint{6.160000in}{6.160000in}}%
\pgfusepath{clip}%
\pgfsetbuttcap%
\pgfsetroundjoin%
\definecolor{currentfill}{rgb}{0.554312,0.690097,0.995516}%
\pgfsetfillcolor{currentfill}%
\pgfsetlinewidth{0.000000pt}%
\definecolor{currentstroke}{rgb}{0.000000,0.000000,0.000000}%
\pgfsetstrokecolor{currentstroke}%
\pgfsetdash{}{0pt}%
\pgfpathmoveto{\pgfqpoint{5.045873in}{3.973209in}}%
\pgfpathlineto{\pgfqpoint{5.056570in}{4.087727in}}%
\pgfpathlineto{\pgfqpoint{5.064932in}{3.931463in}}%
\pgfpathlineto{\pgfqpoint{5.096481in}{3.942423in}}%
\pgfpathlineto{\pgfqpoint{5.129195in}{4.078589in}}%
\pgfpathlineto{\pgfqpoint{5.120228in}{4.167090in}}%
\pgfpathlineto{\pgfqpoint{5.110301in}{4.151075in}}%
\pgfpathlineto{\pgfqpoint{5.077521in}{4.000631in}}%
\pgfpathlineto{\pgfqpoint{5.045873in}{3.973209in}}%
\pgfpathclose%
\pgfusepath{fill}%
\end{pgfscope}%
\begin{pgfscope}%
\pgfpathrectangle{\pgfqpoint{1.020000in}{0.880000in}}{\pgfqpoint{6.160000in}{6.160000in}}%
\pgfusepath{clip}%
\pgfsetbuttcap%
\pgfsetroundjoin%
\definecolor{currentfill}{rgb}{0.521696,0.659599,0.987736}%
\pgfsetfillcolor{currentfill}%
\pgfsetlinewidth{0.000000pt}%
\definecolor{currentstroke}{rgb}{0.000000,0.000000,0.000000}%
\pgfsetstrokecolor{currentstroke}%
\pgfsetdash{}{0pt}%
\pgfpathmoveto{\pgfqpoint{4.984462in}{4.155476in}}%
\pgfpathlineto{\pgfqpoint{4.992334in}{3.933093in}}%
\pgfpathlineto{\pgfqpoint{5.002153in}{3.952554in}}%
\pgfpathlineto{\pgfqpoint{5.034437in}{4.046288in}}%
\pgfpathlineto{\pgfqpoint{5.064932in}{3.931463in}}%
\pgfpathlineto{\pgfqpoint{5.056570in}{4.087727in}}%
\pgfpathlineto{\pgfqpoint{5.045873in}{3.973209in}}%
\pgfpathlineto{\pgfqpoint{5.013226in}{3.825016in}}%
\pgfpathlineto{\pgfqpoint{4.984462in}{4.155476in}}%
\pgfpathclose%
\pgfusepath{fill}%
\end{pgfscope}%
\begin{pgfscope}%
\pgfpathrectangle{\pgfqpoint{1.020000in}{0.880000in}}{\pgfqpoint{6.160000in}{6.160000in}}%
\pgfusepath{clip}%
\pgfsetbuttcap%
\pgfsetroundjoin%
\definecolor{currentfill}{rgb}{0.548876,0.685104,0.994379}%
\pgfsetfillcolor{currentfill}%
\pgfsetlinewidth{0.000000pt}%
\definecolor{currentstroke}{rgb}{0.000000,0.000000,0.000000}%
\pgfsetstrokecolor{currentstroke}%
\pgfsetdash{}{0pt}%
\pgfpathmoveto{\pgfqpoint{5.252821in}{3.880661in}}%
\pgfpathlineto{\pgfqpoint{5.264681in}{4.062545in}}%
\pgfpathlineto{\pgfqpoint{5.274611in}{4.061331in}}%
\pgfpathlineto{\pgfqpoint{5.305475in}{4.009597in}}%
\pgfpathlineto{\pgfqpoint{5.337794in}{4.089016in}}%
\pgfpathlineto{\pgfqpoint{5.327367in}{4.051799in}}%
\pgfpathlineto{\pgfqpoint{5.318214in}{4.125097in}}%
\pgfpathlineto{\pgfqpoint{5.284825in}{3.942413in}}%
\pgfpathlineto{\pgfqpoint{5.252821in}{3.880661in}}%
\pgfpathclose%
\pgfusepath{fill}%
\end{pgfscope}%
\begin{pgfscope}%
\pgfpathrectangle{\pgfqpoint{1.020000in}{0.880000in}}{\pgfqpoint{6.160000in}{6.160000in}}%
\pgfusepath{clip}%
\pgfsetbuttcap%
\pgfsetroundjoin%
\definecolor{currentfill}{rgb}{0.516260,0.654498,0.986407}%
\pgfsetfillcolor{currentfill}%
\pgfsetlinewidth{0.000000pt}%
\definecolor{currentstroke}{rgb}{0.000000,0.000000,0.000000}%
\pgfsetstrokecolor{currentstroke}%
\pgfsetdash{}{0pt}%
\pgfpathmoveto{\pgfqpoint{5.190728in}{3.942035in}}%
\pgfpathlineto{\pgfqpoint{5.200434in}{3.927357in}}%
\pgfpathlineto{\pgfqpoint{5.210279in}{3.924690in}}%
\pgfpathlineto{\pgfqpoint{5.241983in}{3.950329in}}%
\pgfpathlineto{\pgfqpoint{5.274611in}{4.061331in}}%
\pgfpathlineto{\pgfqpoint{5.264681in}{4.062545in}}%
\pgfpathlineto{\pgfqpoint{5.252821in}{3.880661in}}%
\pgfpathlineto{\pgfqpoint{5.223505in}{4.077029in}}%
\pgfpathlineto{\pgfqpoint{5.190728in}{3.942035in}}%
\pgfpathclose%
\pgfusepath{fill}%
\end{pgfscope}%
\begin{pgfscope}%
\pgfpathrectangle{\pgfqpoint{1.020000in}{0.880000in}}{\pgfqpoint{6.160000in}{6.160000in}}%
\pgfusepath{clip}%
\pgfsetbuttcap%
\pgfsetroundjoin%
\definecolor{currentfill}{rgb}{0.565182,0.699438,0.996635}%
\pgfsetfillcolor{currentfill}%
\pgfsetlinewidth{0.000000pt}%
\definecolor{currentstroke}{rgb}{0.000000,0.000000,0.000000}%
\pgfsetstrokecolor{currentstroke}%
\pgfsetdash{}{0pt}%
\pgfpathmoveto{\pgfqpoint{4.775330in}{4.009025in}}%
\pgfpathlineto{\pgfqpoint{4.785029in}{4.052821in}}%
\pgfpathlineto{\pgfqpoint{4.795029in}{4.141655in}}%
\pgfpathlineto{\pgfqpoint{4.826096in}{4.058667in}}%
\pgfpathlineto{\pgfqpoint{4.857658in}{4.059683in}}%
\pgfpathlineto{\pgfqpoint{4.848001in}{4.038844in}}%
\pgfpathlineto{\pgfqpoint{4.838504in}{4.038838in}}%
\pgfpathlineto{\pgfqpoint{4.807020in}{4.039876in}}%
\pgfpathlineto{\pgfqpoint{4.775330in}{4.009025in}}%
\pgfpathclose%
\pgfusepath{fill}%
\end{pgfscope}%
\begin{pgfscope}%
\pgfpathrectangle{\pgfqpoint{1.020000in}{0.880000in}}{\pgfqpoint{6.160000in}{6.160000in}}%
\pgfusepath{clip}%
\pgfsetbuttcap%
\pgfsetroundjoin%
\definecolor{currentfill}{rgb}{0.532568,0.669801,0.990393}%
\pgfsetfillcolor{currentfill}%
\pgfsetlinewidth{0.000000pt}%
\definecolor{currentstroke}{rgb}{0.000000,0.000000,0.000000}%
\pgfsetstrokecolor{currentstroke}%
\pgfsetdash{}{0pt}%
\pgfpathmoveto{\pgfqpoint{4.920790in}{4.070031in}}%
\pgfpathlineto{\pgfqpoint{4.929706in}{3.978396in}}%
\pgfpathlineto{\pgfqpoint{4.938493in}{3.869279in}}%
\pgfpathlineto{\pgfqpoint{4.971587in}{4.072814in}}%
\pgfpathlineto{\pgfqpoint{5.002153in}{3.952554in}}%
\pgfpathlineto{\pgfqpoint{4.992334in}{3.933093in}}%
\pgfpathlineto{\pgfqpoint{4.984462in}{4.155476in}}%
\pgfpathlineto{\pgfqpoint{4.951472in}{3.962357in}}%
\pgfpathlineto{\pgfqpoint{4.920790in}{4.070031in}}%
\pgfpathclose%
\pgfusepath{fill}%
\end{pgfscope}%
\begin{pgfscope}%
\pgfpathrectangle{\pgfqpoint{1.020000in}{0.880000in}}{\pgfqpoint{6.160000in}{6.160000in}}%
\pgfusepath{clip}%
\pgfsetbuttcap%
\pgfsetroundjoin%
\definecolor{currentfill}{rgb}{0.516260,0.654498,0.986407}%
\pgfsetfillcolor{currentfill}%
\pgfsetlinewidth{0.000000pt}%
\definecolor{currentstroke}{rgb}{0.000000,0.000000,0.000000}%
\pgfsetstrokecolor{currentstroke}%
\pgfsetdash{}{0pt}%
\pgfpathmoveto{\pgfqpoint{5.129195in}{4.078589in}}%
\pgfpathlineto{\pgfqpoint{5.137961in}{3.968652in}}%
\pgfpathlineto{\pgfqpoint{5.147805in}{3.971252in}}%
\pgfpathlineto{\pgfqpoint{5.180497in}{4.093186in}}%
\pgfpathlineto{\pgfqpoint{5.210279in}{3.924690in}}%
\pgfpathlineto{\pgfqpoint{5.200434in}{3.927357in}}%
\pgfpathlineto{\pgfqpoint{5.190728in}{3.942035in}}%
\pgfpathlineto{\pgfqpoint{5.158329in}{3.839113in}}%
\pgfpathlineto{\pgfqpoint{5.129195in}{4.078589in}}%
\pgfpathclose%
\pgfusepath{fill}%
\end{pgfscope}%
\begin{pgfscope}%
\pgfpathrectangle{\pgfqpoint{1.020000in}{0.880000in}}{\pgfqpoint{6.160000in}{6.160000in}}%
\pgfusepath{clip}%
\pgfsetbuttcap%
\pgfsetroundjoin%
\definecolor{currentfill}{rgb}{0.586921,0.718121,0.998874}%
\pgfsetfillcolor{currentfill}%
\pgfsetlinewidth{0.000000pt}%
\definecolor{currentstroke}{rgb}{0.000000,0.000000,0.000000}%
\pgfsetstrokecolor{currentstroke}%
\pgfsetdash{}{0pt}%
\pgfpathmoveto{\pgfqpoint{4.712659in}{4.080931in}}%
\pgfpathlineto{\pgfqpoint{4.722542in}{4.173941in}}%
\pgfpathlineto{\pgfqpoint{4.731116in}{4.019834in}}%
\pgfpathlineto{\pgfqpoint{4.763091in}{4.085703in}}%
\pgfpathlineto{\pgfqpoint{4.795029in}{4.141655in}}%
\pgfpathlineto{\pgfqpoint{4.785029in}{4.052821in}}%
\pgfpathlineto{\pgfqpoint{4.775330in}{4.009025in}}%
\pgfpathlineto{\pgfqpoint{4.744207in}{4.079473in}}%
\pgfpathlineto{\pgfqpoint{4.712659in}{4.080931in}}%
\pgfpathclose%
\pgfusepath{fill}%
\end{pgfscope}%
\begin{pgfscope}%
\pgfpathrectangle{\pgfqpoint{1.020000in}{0.880000in}}{\pgfqpoint{6.160000in}{6.160000in}}%
\pgfusepath{clip}%
\pgfsetbuttcap%
\pgfsetroundjoin%
\definecolor{currentfill}{rgb}{0.532568,0.669801,0.990393}%
\pgfsetfillcolor{currentfill}%
\pgfsetlinewidth{0.000000pt}%
\definecolor{currentstroke}{rgb}{0.000000,0.000000,0.000000}%
\pgfsetstrokecolor{currentstroke}%
\pgfsetdash{}{0pt}%
\pgfpathmoveto{\pgfqpoint{4.857658in}{4.059683in}}%
\pgfpathlineto{\pgfqpoint{4.867715in}{4.136195in}}%
\pgfpathlineto{\pgfqpoint{4.876330in}{3.998414in}}%
\pgfpathlineto{\pgfqpoint{4.907433in}{3.932852in}}%
\pgfpathlineto{\pgfqpoint{4.938493in}{3.869279in}}%
\pgfpathlineto{\pgfqpoint{4.929706in}{3.978396in}}%
\pgfpathlineto{\pgfqpoint{4.920790in}{4.070031in}}%
\pgfpathlineto{\pgfqpoint{4.888190in}{3.916442in}}%
\pgfpathlineto{\pgfqpoint{4.857658in}{4.059683in}}%
\pgfpathclose%
\pgfusepath{fill}%
\end{pgfscope}%
\begin{pgfscope}%
\pgfpathrectangle{\pgfqpoint{1.020000in}{0.880000in}}{\pgfqpoint{6.160000in}{6.160000in}}%
\pgfusepath{clip}%
\pgfsetbuttcap%
\pgfsetroundjoin%
\definecolor{currentfill}{rgb}{0.532568,0.669801,0.990393}%
\pgfsetfillcolor{currentfill}%
\pgfsetlinewidth{0.000000pt}%
\definecolor{currentstroke}{rgb}{0.000000,0.000000,0.000000}%
\pgfsetstrokecolor{currentstroke}%
\pgfsetdash{}{0pt}%
\pgfpathmoveto{\pgfqpoint{5.337794in}{4.089016in}}%
\pgfpathlineto{\pgfqpoint{5.345936in}{3.926425in}}%
\pgfpathlineto{\pgfqpoint{5.355940in}{3.924393in}}%
\pgfpathlineto{\pgfqpoint{5.387974in}{3.975543in}}%
\pgfpathlineto{\pgfqpoint{5.378239in}{4.002971in}}%
\pgfpathlineto{\pgfqpoint{5.368442in}{4.023945in}}%
\pgfpathlineto{\pgfqpoint{5.337794in}{4.089016in}}%
\pgfpathclose%
\pgfusepath{fill}%
\end{pgfscope}%
\begin{pgfscope}%
\pgfpathrectangle{\pgfqpoint{1.020000in}{0.880000in}}{\pgfqpoint{6.160000in}{6.160000in}}%
\pgfusepath{clip}%
\pgfsetbuttcap%
\pgfsetroundjoin%
\definecolor{currentfill}{rgb}{0.532568,0.669801,0.990393}%
\pgfsetfillcolor{currentfill}%
\pgfsetlinewidth{0.000000pt}%
\definecolor{currentstroke}{rgb}{0.000000,0.000000,0.000000}%
\pgfsetstrokecolor{currentstroke}%
\pgfsetdash{}{0pt}%
\pgfpathmoveto{\pgfqpoint{5.064932in}{3.931463in}}%
\pgfpathlineto{\pgfqpoint{5.074483in}{3.910414in}}%
\pgfpathlineto{\pgfqpoint{5.085577in}{4.059626in}}%
\pgfpathlineto{\pgfqpoint{5.116929in}{4.038631in}}%
\pgfpathlineto{\pgfqpoint{5.147805in}{3.971252in}}%
\pgfpathlineto{\pgfqpoint{5.137961in}{3.968652in}}%
\pgfpathlineto{\pgfqpoint{5.129195in}{4.078589in}}%
\pgfpathlineto{\pgfqpoint{5.096481in}{3.942423in}}%
\pgfpathlineto{\pgfqpoint{5.064932in}{3.931463in}}%
\pgfpathclose%
\pgfusepath{fill}%
\end{pgfscope}%
\begin{pgfscope}%
\pgfpathrectangle{\pgfqpoint{1.020000in}{0.880000in}}{\pgfqpoint{6.160000in}{6.160000in}}%
\pgfusepath{clip}%
\pgfsetbuttcap%
\pgfsetroundjoin%
\definecolor{currentfill}{rgb}{0.581486,0.713451,0.998314}%
\pgfsetfillcolor{currentfill}%
\pgfsetlinewidth{0.000000pt}%
\definecolor{currentstroke}{rgb}{0.000000,0.000000,0.000000}%
\pgfsetstrokecolor{currentstroke}%
\pgfsetdash{}{0pt}%
\pgfpathmoveto{\pgfqpoint{4.649735in}{4.133102in}}%
\pgfpathlineto{\pgfqpoint{4.658547in}{4.022983in}}%
\pgfpathlineto{\pgfqpoint{4.667597in}{3.962598in}}%
\pgfpathlineto{\pgfqpoint{4.700492in}{4.212646in}}%
\pgfpathlineto{\pgfqpoint{4.731116in}{4.019834in}}%
\pgfpathlineto{\pgfqpoint{4.722542in}{4.173941in}}%
\pgfpathlineto{\pgfqpoint{4.712659in}{4.080931in}}%
\pgfpathlineto{\pgfqpoint{4.680622in}{3.987297in}}%
\pgfpathlineto{\pgfqpoint{4.649735in}{4.133102in}}%
\pgfpathclose%
\pgfusepath{fill}%
\end{pgfscope}%
\begin{pgfscope}%
\pgfpathrectangle{\pgfqpoint{1.020000in}{0.880000in}}{\pgfqpoint{6.160000in}{6.160000in}}%
\pgfusepath{clip}%
\pgfsetbuttcap%
\pgfsetroundjoin%
\definecolor{currentfill}{rgb}{0.532568,0.669801,0.990393}%
\pgfsetfillcolor{currentfill}%
\pgfsetlinewidth{0.000000pt}%
\definecolor{currentstroke}{rgb}{0.000000,0.000000,0.000000}%
\pgfsetstrokecolor{currentstroke}%
\pgfsetdash{}{0pt}%
\pgfpathmoveto{\pgfqpoint{5.274611in}{4.061331in}}%
\pgfpathlineto{\pgfqpoint{5.283896in}{3.999329in}}%
\pgfpathlineto{\pgfqpoint{5.294102in}{4.020108in}}%
\pgfpathlineto{\pgfqpoint{5.323559in}{3.841393in}}%
\pgfpathlineto{\pgfqpoint{5.355940in}{3.924393in}}%
\pgfpathlineto{\pgfqpoint{5.345936in}{3.926425in}}%
\pgfpathlineto{\pgfqpoint{5.337794in}{4.089016in}}%
\pgfpathlineto{\pgfqpoint{5.305475in}{4.009597in}}%
\pgfpathlineto{\pgfqpoint{5.274611in}{4.061331in}}%
\pgfpathclose%
\pgfusepath{fill}%
\end{pgfscope}%
\begin{pgfscope}%
\pgfpathrectangle{\pgfqpoint{1.020000in}{0.880000in}}{\pgfqpoint{6.160000in}{6.160000in}}%
\pgfusepath{clip}%
\pgfsetbuttcap%
\pgfsetroundjoin%
\definecolor{currentfill}{rgb}{0.521696,0.659599,0.987736}%
\pgfsetfillcolor{currentfill}%
\pgfsetlinewidth{0.000000pt}%
\definecolor{currentstroke}{rgb}{0.000000,0.000000,0.000000}%
\pgfsetstrokecolor{currentstroke}%
\pgfsetdash{}{0pt}%
\pgfpathmoveto{\pgfqpoint{5.002153in}{3.952554in}}%
\pgfpathlineto{\pgfqpoint{5.012362in}{4.016469in}}%
\pgfpathlineto{\pgfqpoint{5.020588in}{3.838741in}}%
\pgfpathlineto{\pgfqpoint{5.053190in}{3.964670in}}%
\pgfpathlineto{\pgfqpoint{5.085577in}{4.059626in}}%
\pgfpathlineto{\pgfqpoint{5.074483in}{3.910414in}}%
\pgfpathlineto{\pgfqpoint{5.064932in}{3.931463in}}%
\pgfpathlineto{\pgfqpoint{5.034437in}{4.046288in}}%
\pgfpathlineto{\pgfqpoint{5.002153in}{3.952554in}}%
\pgfpathclose%
\pgfusepath{fill}%
\end{pgfscope}%
\begin{pgfscope}%
\pgfpathrectangle{\pgfqpoint{1.020000in}{0.880000in}}{\pgfqpoint{6.160000in}{6.160000in}}%
\pgfusepath{clip}%
\pgfsetbuttcap%
\pgfsetroundjoin%
\definecolor{currentfill}{rgb}{0.592356,0.722792,0.999434}%
\pgfsetfillcolor{currentfill}%
\pgfsetlinewidth{0.000000pt}%
\definecolor{currentstroke}{rgb}{0.000000,0.000000,0.000000}%
\pgfsetstrokecolor{currentstroke}%
\pgfsetdash{}{0pt}%
\pgfpathmoveto{\pgfqpoint{4.586414in}{4.120179in}}%
\pgfpathlineto{\pgfqpoint{4.595260in}{4.015999in}}%
\pgfpathlineto{\pgfqpoint{4.604534in}{4.014209in}}%
\pgfpathlineto{\pgfqpoint{4.636693in}{4.123908in}}%
\pgfpathlineto{\pgfqpoint{4.667597in}{3.962598in}}%
\pgfpathlineto{\pgfqpoint{4.658547in}{4.022983in}}%
\pgfpathlineto{\pgfqpoint{4.649735in}{4.133102in}}%
\pgfpathlineto{\pgfqpoint{4.618533in}{4.231217in}}%
\pgfpathlineto{\pgfqpoint{4.586414in}{4.120179in}}%
\pgfpathclose%
\pgfusepath{fill}%
\end{pgfscope}%
\begin{pgfscope}%
\pgfpathrectangle{\pgfqpoint{1.020000in}{0.880000in}}{\pgfqpoint{6.160000in}{6.160000in}}%
\pgfusepath{clip}%
\pgfsetbuttcap%
\pgfsetroundjoin%
\definecolor{currentfill}{rgb}{0.527132,0.664700,0.989065}%
\pgfsetfillcolor{currentfill}%
\pgfsetlinewidth{0.000000pt}%
\definecolor{currentstroke}{rgb}{0.000000,0.000000,0.000000}%
\pgfsetstrokecolor{currentstroke}%
\pgfsetdash{}{0pt}%
\pgfpathmoveto{\pgfqpoint{4.938493in}{3.869279in}}%
\pgfpathlineto{\pgfqpoint{4.949106in}{4.000202in}}%
\pgfpathlineto{\pgfqpoint{4.958605in}{3.980887in}}%
\pgfpathlineto{\pgfqpoint{4.990416in}{4.008085in}}%
\pgfpathlineto{\pgfqpoint{5.020588in}{3.838741in}}%
\pgfpathlineto{\pgfqpoint{5.012362in}{4.016469in}}%
\pgfpathlineto{\pgfqpoint{5.002153in}{3.952554in}}%
\pgfpathlineto{\pgfqpoint{4.971587in}{4.072814in}}%
\pgfpathlineto{\pgfqpoint{4.938493in}{3.869279in}}%
\pgfpathclose%
\pgfusepath{fill}%
\end{pgfscope}%
\begin{pgfscope}%
\pgfpathrectangle{\pgfqpoint{1.020000in}{0.880000in}}{\pgfqpoint{6.160000in}{6.160000in}}%
\pgfusepath{clip}%
\pgfsetbuttcap%
\pgfsetroundjoin%
\definecolor{currentfill}{rgb}{0.548876,0.685104,0.994379}%
\pgfsetfillcolor{currentfill}%
\pgfsetlinewidth{0.000000pt}%
\definecolor{currentstroke}{rgb}{0.000000,0.000000,0.000000}%
\pgfsetstrokecolor{currentstroke}%
\pgfsetdash{}{0pt}%
\pgfpathmoveto{\pgfqpoint{5.210279in}{3.924690in}}%
\pgfpathlineto{\pgfqpoint{5.222235in}{4.124689in}}%
\pgfpathlineto{\pgfqpoint{5.229667in}{3.884783in}}%
\pgfpathlineto{\pgfqpoint{5.262854in}{4.043912in}}%
\pgfpathlineto{\pgfqpoint{5.294102in}{4.020108in}}%
\pgfpathlineto{\pgfqpoint{5.283896in}{3.999329in}}%
\pgfpathlineto{\pgfqpoint{5.274611in}{4.061331in}}%
\pgfpathlineto{\pgfqpoint{5.241983in}{3.950329in}}%
\pgfpathlineto{\pgfqpoint{5.210279in}{3.924690in}}%
\pgfpathclose%
\pgfusepath{fill}%
\end{pgfscope}%
\begin{pgfscope}%
\pgfpathrectangle{\pgfqpoint{1.020000in}{0.880000in}}{\pgfqpoint{6.160000in}{6.160000in}}%
\pgfusepath{clip}%
\pgfsetbuttcap%
\pgfsetroundjoin%
\definecolor{currentfill}{rgb}{0.581486,0.713451,0.998314}%
\pgfsetfillcolor{currentfill}%
\pgfsetlinewidth{0.000000pt}%
\definecolor{currentstroke}{rgb}{0.000000,0.000000,0.000000}%
\pgfsetstrokecolor{currentstroke}%
\pgfsetdash{}{0pt}%
\pgfpathmoveto{\pgfqpoint{4.523411in}{4.217638in}}%
\pgfpathlineto{\pgfqpoint{4.531822in}{3.983245in}}%
\pgfpathlineto{\pgfqpoint{4.541553in}{4.126793in}}%
\pgfpathlineto{\pgfqpoint{4.573116in}{4.082391in}}%
\pgfpathlineto{\pgfqpoint{4.604534in}{4.014209in}}%
\pgfpathlineto{\pgfqpoint{4.595260in}{4.015999in}}%
\pgfpathlineto{\pgfqpoint{4.586414in}{4.120179in}}%
\pgfpathlineto{\pgfqpoint{4.554058in}{3.930797in}}%
\pgfpathlineto{\pgfqpoint{4.523411in}{4.217638in}}%
\pgfpathclose%
\pgfusepath{fill}%
\end{pgfscope}%
\begin{pgfscope}%
\pgfpathrectangle{\pgfqpoint{1.020000in}{0.880000in}}{\pgfqpoint{6.160000in}{6.160000in}}%
\pgfusepath{clip}%
\pgfsetbuttcap%
\pgfsetroundjoin%
\definecolor{currentfill}{rgb}{0.500031,0.638508,0.981070}%
\pgfsetfillcolor{currentfill}%
\pgfsetlinewidth{0.000000pt}%
\definecolor{currentstroke}{rgb}{0.000000,0.000000,0.000000}%
\pgfsetstrokecolor{currentstroke}%
\pgfsetdash{}{0pt}%
\pgfpathmoveto{\pgfqpoint{5.355940in}{3.924393in}}%
\pgfpathlineto{\pgfqpoint{5.365828in}{3.910792in}}%
\pgfpathlineto{\pgfqpoint{5.376582in}{3.968775in}}%
\pgfpathlineto{\pgfqpoint{5.407328in}{3.907491in}}%
\pgfpathlineto{\pgfqpoint{5.396460in}{3.843493in}}%
\pgfpathlineto{\pgfqpoint{5.387974in}{3.975543in}}%
\pgfpathlineto{\pgfqpoint{5.355940in}{3.924393in}}%
\pgfpathclose%
\pgfusepath{fill}%
\end{pgfscope}%
\begin{pgfscope}%
\pgfpathrectangle{\pgfqpoint{1.020000in}{0.880000in}}{\pgfqpoint{6.160000in}{6.160000in}}%
\pgfusepath{clip}%
\pgfsetbuttcap%
\pgfsetroundjoin%
\definecolor{currentfill}{rgb}{0.608547,0.735725,0.999354}%
\pgfsetfillcolor{currentfill}%
\pgfsetlinewidth{0.000000pt}%
\definecolor{currentstroke}{rgb}{0.000000,0.000000,0.000000}%
\pgfsetstrokecolor{currentstroke}%
\pgfsetdash{}{0pt}%
\pgfpathmoveto{\pgfqpoint{4.795029in}{4.141655in}}%
\pgfpathlineto{\pgfqpoint{4.804618in}{4.158195in}}%
\pgfpathlineto{\pgfqpoint{4.813686in}{4.087940in}}%
\pgfpathlineto{\pgfqpoint{4.845844in}{4.166758in}}%
\pgfpathlineto{\pgfqpoint{4.876330in}{3.998414in}}%
\pgfpathlineto{\pgfqpoint{4.867715in}{4.136195in}}%
\pgfpathlineto{\pgfqpoint{4.857658in}{4.059683in}}%
\pgfpathlineto{\pgfqpoint{4.826096in}{4.058667in}}%
\pgfpathlineto{\pgfqpoint{4.795029in}{4.141655in}}%
\pgfpathclose%
\pgfusepath{fill}%
\end{pgfscope}%
\begin{pgfscope}%
\pgfpathrectangle{\pgfqpoint{1.020000in}{0.880000in}}{\pgfqpoint{6.160000in}{6.160000in}}%
\pgfusepath{clip}%
\pgfsetbuttcap%
\pgfsetroundjoin%
\definecolor{currentfill}{rgb}{0.489246,0.627536,0.976896}%
\pgfsetfillcolor{currentfill}%
\pgfsetlinewidth{0.000000pt}%
\definecolor{currentstroke}{rgb}{0.000000,0.000000,0.000000}%
\pgfsetstrokecolor{currentstroke}%
\pgfsetdash{}{0pt}%
\pgfpathmoveto{\pgfqpoint{5.294102in}{4.020108in}}%
\pgfpathlineto{\pgfqpoint{5.301893in}{3.821904in}}%
\pgfpathlineto{\pgfqpoint{5.312850in}{3.907037in}}%
\pgfpathlineto{\pgfqpoint{5.343093in}{3.797491in}}%
\pgfpathlineto{\pgfqpoint{5.376582in}{3.968775in}}%
\pgfpathlineto{\pgfqpoint{5.365828in}{3.910792in}}%
\pgfpathlineto{\pgfqpoint{5.355940in}{3.924393in}}%
\pgfpathlineto{\pgfqpoint{5.323559in}{3.841393in}}%
\pgfpathlineto{\pgfqpoint{5.294102in}{4.020108in}}%
\pgfpathclose%
\pgfusepath{fill}%
\end{pgfscope}%
\begin{pgfscope}%
\pgfpathrectangle{\pgfqpoint{1.020000in}{0.880000in}}{\pgfqpoint{6.160000in}{6.160000in}}%
\pgfusepath{clip}%
\pgfsetbuttcap%
\pgfsetroundjoin%
\definecolor{currentfill}{rgb}{0.543440,0.680003,0.993051}%
\pgfsetfillcolor{currentfill}%
\pgfsetlinewidth{0.000000pt}%
\definecolor{currentstroke}{rgb}{0.000000,0.000000,0.000000}%
\pgfsetstrokecolor{currentstroke}%
\pgfsetdash{}{0pt}%
\pgfpathmoveto{\pgfqpoint{5.147805in}{3.971252in}}%
\pgfpathlineto{\pgfqpoint{5.159177in}{4.129468in}}%
\pgfpathlineto{\pgfqpoint{5.166973in}{3.917186in}}%
\pgfpathlineto{\pgfqpoint{5.198058in}{3.873631in}}%
\pgfpathlineto{\pgfqpoint{5.229667in}{3.884783in}}%
\pgfpathlineto{\pgfqpoint{5.222235in}{4.124689in}}%
\pgfpathlineto{\pgfqpoint{5.210279in}{3.924690in}}%
\pgfpathlineto{\pgfqpoint{5.180497in}{4.093186in}}%
\pgfpathlineto{\pgfqpoint{5.147805in}{3.971252in}}%
\pgfpathclose%
\pgfusepath{fill}%
\end{pgfscope}%
\begin{pgfscope}%
\pgfpathrectangle{\pgfqpoint{1.020000in}{0.880000in}}{\pgfqpoint{6.160000in}{6.160000in}}%
\pgfusepath{clip}%
\pgfsetbuttcap%
\pgfsetroundjoin%
\definecolor{currentfill}{rgb}{0.532568,0.669801,0.990393}%
\pgfsetfillcolor{currentfill}%
\pgfsetlinewidth{0.000000pt}%
\definecolor{currentstroke}{rgb}{0.000000,0.000000,0.000000}%
\pgfsetstrokecolor{currentstroke}%
\pgfsetdash{}{0pt}%
\pgfpathmoveto{\pgfqpoint{4.876330in}{3.998414in}}%
\pgfpathlineto{\pgfqpoint{4.886163in}{4.036092in}}%
\pgfpathlineto{\pgfqpoint{4.895220in}{3.960369in}}%
\pgfpathlineto{\pgfqpoint{4.927296in}{4.021714in}}%
\pgfpathlineto{\pgfqpoint{4.958605in}{3.980887in}}%
\pgfpathlineto{\pgfqpoint{4.949106in}{4.000202in}}%
\pgfpathlineto{\pgfqpoint{4.938493in}{3.869279in}}%
\pgfpathlineto{\pgfqpoint{4.907433in}{3.932852in}}%
\pgfpathlineto{\pgfqpoint{4.876330in}{3.998414in}}%
\pgfpathclose%
\pgfusepath{fill}%
\end{pgfscope}%
\begin{pgfscope}%
\pgfpathrectangle{\pgfqpoint{1.020000in}{0.880000in}}{\pgfqpoint{6.160000in}{6.160000in}}%
\pgfusepath{clip}%
\pgfsetbuttcap%
\pgfsetroundjoin%
\definecolor{currentfill}{rgb}{0.597777,0.727330,0.999777}%
\pgfsetfillcolor{currentfill}%
\pgfsetlinewidth{0.000000pt}%
\definecolor{currentstroke}{rgb}{0.000000,0.000000,0.000000}%
\pgfsetstrokecolor{currentstroke}%
\pgfsetdash{}{0pt}%
\pgfpathmoveto{\pgfqpoint{4.731116in}{4.019834in}}%
\pgfpathlineto{\pgfqpoint{4.741151in}{4.131370in}}%
\pgfpathlineto{\pgfqpoint{4.750241in}{4.068003in}}%
\pgfpathlineto{\pgfqpoint{4.781150in}{3.940235in}}%
\pgfpathlineto{\pgfqpoint{4.813686in}{4.087940in}}%
\pgfpathlineto{\pgfqpoint{4.804618in}{4.158195in}}%
\pgfpathlineto{\pgfqpoint{4.795029in}{4.141655in}}%
\pgfpathlineto{\pgfqpoint{4.763091in}{4.085703in}}%
\pgfpathlineto{\pgfqpoint{4.731116in}{4.019834in}}%
\pgfpathclose%
\pgfusepath{fill}%
\end{pgfscope}%
\begin{pgfscope}%
\pgfpathrectangle{\pgfqpoint{1.020000in}{0.880000in}}{\pgfqpoint{6.160000in}{6.160000in}}%
\pgfusepath{clip}%
\pgfsetbuttcap%
\pgfsetroundjoin%
\definecolor{currentfill}{rgb}{0.581486,0.713451,0.998314}%
\pgfsetfillcolor{currentfill}%
\pgfsetlinewidth{0.000000pt}%
\definecolor{currentstroke}{rgb}{0.000000,0.000000,0.000000}%
\pgfsetstrokecolor{currentstroke}%
\pgfsetdash{}{0pt}%
\pgfpathmoveto{\pgfqpoint{4.667597in}{3.962598in}}%
\pgfpathlineto{\pgfqpoint{4.677313in}{4.036341in}}%
\pgfpathlineto{\pgfqpoint{4.686063in}{3.910574in}}%
\pgfpathlineto{\pgfqpoint{4.718584in}{4.073537in}}%
\pgfpathlineto{\pgfqpoint{4.750241in}{4.068003in}}%
\pgfpathlineto{\pgfqpoint{4.741151in}{4.131370in}}%
\pgfpathlineto{\pgfqpoint{4.731116in}{4.019834in}}%
\pgfpathlineto{\pgfqpoint{4.700492in}{4.212646in}}%
\pgfpathlineto{\pgfqpoint{4.667597in}{3.962598in}}%
\pgfpathclose%
\pgfusepath{fill}%
\end{pgfscope}%
\begin{pgfscope}%
\pgfpathrectangle{\pgfqpoint{1.020000in}{0.880000in}}{\pgfqpoint{6.160000in}{6.160000in}}%
\pgfusepath{clip}%
\pgfsetbuttcap%
\pgfsetroundjoin%
\definecolor{currentfill}{rgb}{0.516260,0.654498,0.986407}%
\pgfsetfillcolor{currentfill}%
\pgfsetlinewidth{0.000000pt}%
\definecolor{currentstroke}{rgb}{0.000000,0.000000,0.000000}%
\pgfsetstrokecolor{currentstroke}%
\pgfsetdash{}{0pt}%
\pgfpathmoveto{\pgfqpoint{5.020588in}{3.838741in}}%
\pgfpathlineto{\pgfqpoint{5.031296in}{3.957082in}}%
\pgfpathlineto{\pgfqpoint{5.040782in}{3.927144in}}%
\pgfpathlineto{\pgfqpoint{5.071849in}{3.867088in}}%
\pgfpathlineto{\pgfqpoint{5.103825in}{3.912203in}}%
\pgfpathlineto{\pgfqpoint{5.094803in}{3.997437in}}%
\pgfpathlineto{\pgfqpoint{5.085577in}{4.059626in}}%
\pgfpathlineto{\pgfqpoint{5.053190in}{3.964670in}}%
\pgfpathlineto{\pgfqpoint{5.020588in}{3.838741in}}%
\pgfpathclose%
\pgfusepath{fill}%
\end{pgfscope}%
\begin{pgfscope}%
\pgfpathrectangle{\pgfqpoint{1.020000in}{0.880000in}}{\pgfqpoint{6.160000in}{6.160000in}}%
\pgfusepath{clip}%
\pgfsetbuttcap%
\pgfsetroundjoin%
\definecolor{currentfill}{rgb}{0.613933,0.739923,0.999142}%
\pgfsetfillcolor{currentfill}%
\pgfsetlinewidth{0.000000pt}%
\definecolor{currentstroke}{rgb}{0.000000,0.000000,0.000000}%
\pgfsetstrokecolor{currentstroke}%
\pgfsetdash{}{0pt}%
\pgfpathmoveto{\pgfqpoint{4.459441in}{4.033170in}}%
\pgfpathlineto{\pgfqpoint{4.469135in}{4.226025in}}%
\pgfpathlineto{\pgfqpoint{4.477885in}{4.077697in}}%
\pgfpathlineto{\pgfqpoint{4.509948in}{4.172527in}}%
\pgfpathlineto{\pgfqpoint{4.541553in}{4.126793in}}%
\pgfpathlineto{\pgfqpoint{4.531822in}{3.983245in}}%
\pgfpathlineto{\pgfqpoint{4.523411in}{4.217638in}}%
\pgfpathlineto{\pgfqpoint{4.490926in}{3.968596in}}%
\pgfpathlineto{\pgfqpoint{4.459441in}{4.033170in}}%
\pgfpathclose%
\pgfusepath{fill}%
\end{pgfscope}%
\begin{pgfscope}%
\pgfpathrectangle{\pgfqpoint{1.020000in}{0.880000in}}{\pgfqpoint{6.160000in}{6.160000in}}%
\pgfusepath{clip}%
\pgfsetbuttcap%
\pgfsetroundjoin%
\definecolor{currentfill}{rgb}{0.586921,0.718121,0.998874}%
\pgfsetfillcolor{currentfill}%
\pgfsetlinewidth{0.000000pt}%
\definecolor{currentstroke}{rgb}{0.000000,0.000000,0.000000}%
\pgfsetstrokecolor{currentstroke}%
\pgfsetdash{}{0pt}%
\pgfpathmoveto{\pgfqpoint{4.396307in}{4.168133in}}%
\pgfpathlineto{\pgfqpoint{4.405078in}{4.018333in}}%
\pgfpathlineto{\pgfqpoint{4.414286in}{4.063039in}}%
\pgfpathlineto{\pgfqpoint{4.445538in}{3.858348in}}%
\pgfpathlineto{\pgfqpoint{4.477885in}{4.077697in}}%
\pgfpathlineto{\pgfqpoint{4.469135in}{4.226025in}}%
\pgfpathlineto{\pgfqpoint{4.459441in}{4.033170in}}%
\pgfpathlineto{\pgfqpoint{4.427715in}{4.023127in}}%
\pgfpathlineto{\pgfqpoint{4.396307in}{4.168133in}}%
\pgfpathclose%
\pgfusepath{fill}%
\end{pgfscope}%
\begin{pgfscope}%
\pgfpathrectangle{\pgfqpoint{1.020000in}{0.880000in}}{\pgfqpoint{6.160000in}{6.160000in}}%
\pgfusepath{clip}%
\pgfsetbuttcap%
\pgfsetroundjoin%
\definecolor{currentfill}{rgb}{0.565182,0.699438,0.996635}%
\pgfsetfillcolor{currentfill}%
\pgfsetlinewidth{0.000000pt}%
\definecolor{currentstroke}{rgb}{0.000000,0.000000,0.000000}%
\pgfsetstrokecolor{currentstroke}%
\pgfsetdash{}{0pt}%
\pgfpathmoveto{\pgfqpoint{5.085577in}{4.059626in}}%
\pgfpathlineto{\pgfqpoint{5.094803in}{3.997437in}}%
\pgfpathlineto{\pgfqpoint{5.103825in}{3.912203in}}%
\pgfpathlineto{\pgfqpoint{5.137417in}{4.126765in}}%
\pgfpathlineto{\pgfqpoint{5.166973in}{3.917186in}}%
\pgfpathlineto{\pgfqpoint{5.159177in}{4.129468in}}%
\pgfpathlineto{\pgfqpoint{5.147805in}{3.971252in}}%
\pgfpathlineto{\pgfqpoint{5.116929in}{4.038631in}}%
\pgfpathlineto{\pgfqpoint{5.085577in}{4.059626in}}%
\pgfpathclose%
\pgfusepath{fill}%
\end{pgfscope}%
\begin{pgfscope}%
\pgfpathrectangle{\pgfqpoint{1.020000in}{0.880000in}}{\pgfqpoint{6.160000in}{6.160000in}}%
\pgfusepath{clip}%
\pgfsetbuttcap%
\pgfsetroundjoin%
\definecolor{currentfill}{rgb}{0.576051,0.708780,0.997755}%
\pgfsetfillcolor{currentfill}%
\pgfsetlinewidth{0.000000pt}%
\definecolor{currentstroke}{rgb}{0.000000,0.000000,0.000000}%
\pgfsetstrokecolor{currentstroke}%
\pgfsetdash{}{0pt}%
\pgfpathmoveto{\pgfqpoint{4.604534in}{4.014209in}}%
\pgfpathlineto{\pgfqpoint{4.614039in}{4.062012in}}%
\pgfpathlineto{\pgfqpoint{4.623766in}{4.154320in}}%
\pgfpathlineto{\pgfqpoint{4.654875in}{4.014029in}}%
\pgfpathlineto{\pgfqpoint{4.686063in}{3.910574in}}%
\pgfpathlineto{\pgfqpoint{4.677313in}{4.036341in}}%
\pgfpathlineto{\pgfqpoint{4.667597in}{3.962598in}}%
\pgfpathlineto{\pgfqpoint{4.636693in}{4.123908in}}%
\pgfpathlineto{\pgfqpoint{4.604534in}{4.014209in}}%
\pgfpathclose%
\pgfusepath{fill}%
\end{pgfscope}%
\begin{pgfscope}%
\pgfpathrectangle{\pgfqpoint{1.020000in}{0.880000in}}{\pgfqpoint{6.160000in}{6.160000in}}%
\pgfusepath{clip}%
\pgfsetbuttcap%
\pgfsetroundjoin%
\definecolor{currentfill}{rgb}{0.516260,0.654498,0.986407}%
\pgfsetfillcolor{currentfill}%
\pgfsetlinewidth{0.000000pt}%
\definecolor{currentstroke}{rgb}{0.000000,0.000000,0.000000}%
\pgfsetstrokecolor{currentstroke}%
\pgfsetdash{}{0pt}%
\pgfpathmoveto{\pgfqpoint{5.229667in}{3.884783in}}%
\pgfpathlineto{\pgfqpoint{5.240639in}{3.984052in}}%
\pgfpathlineto{\pgfqpoint{5.249403in}{3.871526in}}%
\pgfpathlineto{\pgfqpoint{5.281915in}{3.961121in}}%
\pgfpathlineto{\pgfqpoint{5.312850in}{3.907037in}}%
\pgfpathlineto{\pgfqpoint{5.301893in}{3.821904in}}%
\pgfpathlineto{\pgfqpoint{5.294102in}{4.020108in}}%
\pgfpathlineto{\pgfqpoint{5.262854in}{4.043912in}}%
\pgfpathlineto{\pgfqpoint{5.229667in}{3.884783in}}%
\pgfpathclose%
\pgfusepath{fill}%
\end{pgfscope}%
\begin{pgfscope}%
\pgfpathrectangle{\pgfqpoint{1.020000in}{0.880000in}}{\pgfqpoint{6.160000in}{6.160000in}}%
\pgfusepath{clip}%
\pgfsetbuttcap%
\pgfsetroundjoin%
\definecolor{currentfill}{rgb}{0.505423,0.643995,0.983157}%
\pgfsetfillcolor{currentfill}%
\pgfsetlinewidth{0.000000pt}%
\definecolor{currentstroke}{rgb}{0.000000,0.000000,0.000000}%
\pgfsetstrokecolor{currentstroke}%
\pgfsetdash{}{0pt}%
\pgfpathmoveto{\pgfqpoint{4.958605in}{3.980887in}}%
\pgfpathlineto{\pgfqpoint{4.967214in}{3.845500in}}%
\pgfpathlineto{\pgfqpoint{4.977163in}{3.880632in}}%
\pgfpathlineto{\pgfqpoint{5.008986in}{3.905310in}}%
\pgfpathlineto{\pgfqpoint{5.040782in}{3.927144in}}%
\pgfpathlineto{\pgfqpoint{5.031296in}{3.957082in}}%
\pgfpathlineto{\pgfqpoint{5.020588in}{3.838741in}}%
\pgfpathlineto{\pgfqpoint{4.990416in}{4.008085in}}%
\pgfpathlineto{\pgfqpoint{4.958605in}{3.980887in}}%
\pgfpathclose%
\pgfusepath{fill}%
\end{pgfscope}%
\begin{pgfscope}%
\pgfpathrectangle{\pgfqpoint{1.020000in}{0.880000in}}{\pgfqpoint{6.160000in}{6.160000in}}%
\pgfusepath{clip}%
\pgfsetbuttcap%
\pgfsetroundjoin%
\definecolor{currentfill}{rgb}{0.576051,0.708780,0.997755}%
\pgfsetfillcolor{currentfill}%
\pgfsetlinewidth{0.000000pt}%
\definecolor{currentstroke}{rgb}{0.000000,0.000000,0.000000}%
\pgfsetstrokecolor{currentstroke}%
\pgfsetdash{}{0pt}%
\pgfpathmoveto{\pgfqpoint{4.332511in}{4.019841in}}%
\pgfpathlineto{\pgfqpoint{4.341558in}{4.031437in}}%
\pgfpathlineto{\pgfqpoint{4.350407in}{3.911762in}}%
\pgfpathlineto{\pgfqpoint{4.382540in}{4.092535in}}%
\pgfpathlineto{\pgfqpoint{4.414286in}{4.063039in}}%
\pgfpathlineto{\pgfqpoint{4.405078in}{4.018333in}}%
\pgfpathlineto{\pgfqpoint{4.396307in}{4.168133in}}%
\pgfpathlineto{\pgfqpoint{4.364221in}{3.997345in}}%
\pgfpathlineto{\pgfqpoint{4.332511in}{4.019841in}}%
\pgfpathclose%
\pgfusepath{fill}%
\end{pgfscope}%
\begin{pgfscope}%
\pgfpathrectangle{\pgfqpoint{1.020000in}{0.880000in}}{\pgfqpoint{6.160000in}{6.160000in}}%
\pgfusepath{clip}%
\pgfsetbuttcap%
\pgfsetroundjoin%
\definecolor{currentfill}{rgb}{0.505423,0.643995,0.983157}%
\pgfsetfillcolor{currentfill}%
\pgfsetlinewidth{0.000000pt}%
\definecolor{currentstroke}{rgb}{0.000000,0.000000,0.000000}%
\pgfsetstrokecolor{currentstroke}%
\pgfsetdash{}{0pt}%
\pgfpathmoveto{\pgfqpoint{5.166973in}{3.917186in}}%
\pgfpathlineto{\pgfqpoint{5.176893in}{3.921914in}}%
\pgfpathlineto{\pgfqpoint{5.186891in}{3.932323in}}%
\pgfpathlineto{\pgfqpoint{5.218662in}{3.950165in}}%
\pgfpathlineto{\pgfqpoint{5.249403in}{3.871526in}}%
\pgfpathlineto{\pgfqpoint{5.240639in}{3.984052in}}%
\pgfpathlineto{\pgfqpoint{5.229667in}{3.884783in}}%
\pgfpathlineto{\pgfqpoint{5.198058in}{3.873631in}}%
\pgfpathlineto{\pgfqpoint{5.166973in}{3.917186in}}%
\pgfpathclose%
\pgfusepath{fill}%
\end{pgfscope}%
\begin{pgfscope}%
\pgfpathrectangle{\pgfqpoint{1.020000in}{0.880000in}}{\pgfqpoint{6.160000in}{6.160000in}}%
\pgfusepath{clip}%
\pgfsetbuttcap%
\pgfsetroundjoin%
\definecolor{currentfill}{rgb}{0.592356,0.722792,0.999434}%
\pgfsetfillcolor{currentfill}%
\pgfsetlinewidth{0.000000pt}%
\definecolor{currentstroke}{rgb}{0.000000,0.000000,0.000000}%
\pgfsetstrokecolor{currentstroke}%
\pgfsetdash{}{0pt}%
\pgfpathmoveto{\pgfqpoint{4.813686in}{4.087940in}}%
\pgfpathlineto{\pgfqpoint{4.822846in}{4.031327in}}%
\pgfpathlineto{\pgfqpoint{4.832926in}{4.115825in}}%
\pgfpathlineto{\pgfqpoint{4.864373in}{4.077456in}}%
\pgfpathlineto{\pgfqpoint{4.895220in}{3.960369in}}%
\pgfpathlineto{\pgfqpoint{4.886163in}{4.036092in}}%
\pgfpathlineto{\pgfqpoint{4.876330in}{3.998414in}}%
\pgfpathlineto{\pgfqpoint{4.845844in}{4.166758in}}%
\pgfpathlineto{\pgfqpoint{4.813686in}{4.087940in}}%
\pgfpathclose%
\pgfusepath{fill}%
\end{pgfscope}%
\begin{pgfscope}%
\pgfpathrectangle{\pgfqpoint{1.020000in}{0.880000in}}{\pgfqpoint{6.160000in}{6.160000in}}%
\pgfusepath{clip}%
\pgfsetbuttcap%
\pgfsetroundjoin%
\definecolor{currentfill}{rgb}{0.478462,0.616564,0.972721}%
\pgfsetfillcolor{currentfill}%
\pgfsetlinewidth{0.000000pt}%
\definecolor{currentstroke}{rgb}{0.000000,0.000000,0.000000}%
\pgfsetstrokecolor{currentstroke}%
\pgfsetdash{}{0pt}%
\pgfpathmoveto{\pgfqpoint{5.312850in}{3.907037in}}%
\pgfpathlineto{\pgfqpoint{5.323344in}{3.948033in}}%
\pgfpathlineto{\pgfqpoint{5.331321in}{3.766618in}}%
\pgfpathlineto{\pgfqpoint{5.363385in}{3.812739in}}%
\pgfpathlineto{\pgfqpoint{5.396173in}{3.917835in}}%
\pgfpathlineto{\pgfqpoint{5.384621in}{3.797187in}}%
\pgfpathlineto{\pgfqpoint{5.376582in}{3.968775in}}%
\pgfpathlineto{\pgfqpoint{5.343093in}{3.797491in}}%
\pgfpathlineto{\pgfqpoint{5.312850in}{3.907037in}}%
\pgfpathclose%
\pgfusepath{fill}%
\end{pgfscope}%
\begin{pgfscope}%
\pgfpathrectangle{\pgfqpoint{1.020000in}{0.880000in}}{\pgfqpoint{6.160000in}{6.160000in}}%
\pgfusepath{clip}%
\pgfsetbuttcap%
\pgfsetroundjoin%
\definecolor{currentfill}{rgb}{0.505423,0.643995,0.983157}%
\pgfsetfillcolor{currentfill}%
\pgfsetlinewidth{0.000000pt}%
\definecolor{currentstroke}{rgb}{0.000000,0.000000,0.000000}%
\pgfsetstrokecolor{currentstroke}%
\pgfsetdash{}{0pt}%
\pgfpathmoveto{\pgfqpoint{5.376582in}{3.968775in}}%
\pgfpathlineto{\pgfqpoint{5.384621in}{3.797187in}}%
\pgfpathlineto{\pgfqpoint{5.396173in}{3.917835in}}%
\pgfpathlineto{\pgfqpoint{5.428106in}{3.950012in}}%
\pgfpathlineto{\pgfqpoint{5.417495in}{3.911679in}}%
\pgfpathlineto{\pgfqpoint{5.407328in}{3.907491in}}%
\pgfpathlineto{\pgfqpoint{5.376582in}{3.968775in}}%
\pgfpathclose%
\pgfusepath{fill}%
\end{pgfscope}%
\begin{pgfscope}%
\pgfpathrectangle{\pgfqpoint{1.020000in}{0.880000in}}{\pgfqpoint{6.160000in}{6.160000in}}%
\pgfusepath{clip}%
\pgfsetbuttcap%
\pgfsetroundjoin%
\definecolor{currentfill}{rgb}{0.576051,0.708780,0.997755}%
\pgfsetfillcolor{currentfill}%
\pgfsetlinewidth{0.000000pt}%
\definecolor{currentstroke}{rgb}{0.000000,0.000000,0.000000}%
\pgfsetstrokecolor{currentstroke}%
\pgfsetdash{}{0pt}%
\pgfpathmoveto{\pgfqpoint{4.750241in}{4.068003in}}%
\pgfpathlineto{\pgfqpoint{4.759752in}{4.076902in}}%
\pgfpathlineto{\pgfqpoint{4.768283in}{3.913870in}}%
\pgfpathlineto{\pgfqpoint{4.800267in}{3.963554in}}%
\pgfpathlineto{\pgfqpoint{4.832926in}{4.115825in}}%
\pgfpathlineto{\pgfqpoint{4.822846in}{4.031327in}}%
\pgfpathlineto{\pgfqpoint{4.813686in}{4.087940in}}%
\pgfpathlineto{\pgfqpoint{4.781150in}{3.940235in}}%
\pgfpathlineto{\pgfqpoint{4.750241in}{4.068003in}}%
\pgfpathclose%
\pgfusepath{fill}%
\end{pgfscope}%
\begin{pgfscope}%
\pgfpathrectangle{\pgfqpoint{1.020000in}{0.880000in}}{\pgfqpoint{6.160000in}{6.160000in}}%
\pgfusepath{clip}%
\pgfsetbuttcap%
\pgfsetroundjoin%
\definecolor{currentfill}{rgb}{0.581486,0.713451,0.998314}%
\pgfsetfillcolor{currentfill}%
\pgfsetlinewidth{0.000000pt}%
\definecolor{currentstroke}{rgb}{0.000000,0.000000,0.000000}%
\pgfsetstrokecolor{currentstroke}%
\pgfsetdash{}{0pt}%
\pgfpathmoveto{\pgfqpoint{4.269046in}{4.123092in}}%
\pgfpathlineto{\pgfqpoint{4.277977in}{4.072055in}}%
\pgfpathlineto{\pgfqpoint{4.286977in}{4.079536in}}%
\pgfpathlineto{\pgfqpoint{4.318803in}{4.051287in}}%
\pgfpathlineto{\pgfqpoint{4.350407in}{3.911762in}}%
\pgfpathlineto{\pgfqpoint{4.341558in}{4.031437in}}%
\pgfpathlineto{\pgfqpoint{4.332511in}{4.019841in}}%
\pgfpathlineto{\pgfqpoint{4.300766in}{4.039235in}}%
\pgfpathlineto{\pgfqpoint{4.269046in}{4.123092in}}%
\pgfpathclose%
\pgfusepath{fill}%
\end{pgfscope}%
\begin{pgfscope}%
\pgfpathrectangle{\pgfqpoint{1.020000in}{0.880000in}}{\pgfqpoint{6.160000in}{6.160000in}}%
\pgfusepath{clip}%
\pgfsetbuttcap%
\pgfsetroundjoin%
\definecolor{currentfill}{rgb}{0.532568,0.669801,0.990393}%
\pgfsetfillcolor{currentfill}%
\pgfsetlinewidth{0.000000pt}%
\definecolor{currentstroke}{rgb}{0.000000,0.000000,0.000000}%
\pgfsetstrokecolor{currentstroke}%
\pgfsetdash{}{0pt}%
\pgfpathmoveto{\pgfqpoint{4.895220in}{3.960369in}}%
\pgfpathlineto{\pgfqpoint{4.904430in}{3.905271in}}%
\pgfpathlineto{\pgfqpoint{4.914702in}{3.995021in}}%
\pgfpathlineto{\pgfqpoint{4.946834in}{4.053028in}}%
\pgfpathlineto{\pgfqpoint{4.977163in}{3.880632in}}%
\pgfpathlineto{\pgfqpoint{4.967214in}{3.845500in}}%
\pgfpathlineto{\pgfqpoint{4.958605in}{3.980887in}}%
\pgfpathlineto{\pgfqpoint{4.927296in}{4.021714in}}%
\pgfpathlineto{\pgfqpoint{4.895220in}{3.960369in}}%
\pgfpathclose%
\pgfusepath{fill}%
\end{pgfscope}%
\begin{pgfscope}%
\pgfpathrectangle{\pgfqpoint{1.020000in}{0.880000in}}{\pgfqpoint{6.160000in}{6.160000in}}%
\pgfusepath{clip}%
\pgfsetbuttcap%
\pgfsetroundjoin%
\definecolor{currentfill}{rgb}{0.613933,0.739923,0.999142}%
\pgfsetfillcolor{currentfill}%
\pgfsetlinewidth{0.000000pt}%
\definecolor{currentstroke}{rgb}{0.000000,0.000000,0.000000}%
\pgfsetstrokecolor{currentstroke}%
\pgfsetdash{}{0pt}%
\pgfpathmoveto{\pgfqpoint{4.541553in}{4.126793in}}%
\pgfpathlineto{\pgfqpoint{4.550369in}{4.007573in}}%
\pgfpathlineto{\pgfqpoint{4.560640in}{4.277697in}}%
\pgfpathlineto{\pgfqpoint{4.591262in}{3.977098in}}%
\pgfpathlineto{\pgfqpoint{4.623766in}{4.154320in}}%
\pgfpathlineto{\pgfqpoint{4.614039in}{4.062012in}}%
\pgfpathlineto{\pgfqpoint{4.604534in}{4.014209in}}%
\pgfpathlineto{\pgfqpoint{4.573116in}{4.082391in}}%
\pgfpathlineto{\pgfqpoint{4.541553in}{4.126793in}}%
\pgfpathclose%
\pgfusepath{fill}%
\end{pgfscope}%
\begin{pgfscope}%
\pgfpathrectangle{\pgfqpoint{1.020000in}{0.880000in}}{\pgfqpoint{6.160000in}{6.160000in}}%
\pgfusepath{clip}%
\pgfsetbuttcap%
\pgfsetroundjoin%
\definecolor{currentfill}{rgb}{0.532568,0.669801,0.990393}%
\pgfsetfillcolor{currentfill}%
\pgfsetlinewidth{0.000000pt}%
\definecolor{currentstroke}{rgb}{0.000000,0.000000,0.000000}%
\pgfsetstrokecolor{currentstroke}%
\pgfsetdash{}{0pt}%
\pgfpathmoveto{\pgfqpoint{5.103825in}{3.912203in}}%
\pgfpathlineto{\pgfqpoint{5.113170in}{3.861473in}}%
\pgfpathlineto{\pgfqpoint{5.123361in}{3.900104in}}%
\pgfpathlineto{\pgfqpoint{5.155805in}{3.986028in}}%
\pgfpathlineto{\pgfqpoint{5.186891in}{3.932323in}}%
\pgfpathlineto{\pgfqpoint{5.176893in}{3.921914in}}%
\pgfpathlineto{\pgfqpoint{5.166973in}{3.917186in}}%
\pgfpathlineto{\pgfqpoint{5.137417in}{4.126765in}}%
\pgfpathlineto{\pgfqpoint{5.103825in}{3.912203in}}%
\pgfpathclose%
\pgfusepath{fill}%
\end{pgfscope}%
\begin{pgfscope}%
\pgfpathrectangle{\pgfqpoint{1.020000in}{0.880000in}}{\pgfqpoint{6.160000in}{6.160000in}}%
\pgfusepath{clip}%
\pgfsetbuttcap%
\pgfsetroundjoin%
\definecolor{currentfill}{rgb}{0.505423,0.643995,0.983157}%
\pgfsetfillcolor{currentfill}%
\pgfsetlinewidth{0.000000pt}%
\definecolor{currentstroke}{rgb}{0.000000,0.000000,0.000000}%
\pgfsetstrokecolor{currentstroke}%
\pgfsetdash{}{0pt}%
\pgfpathmoveto{\pgfqpoint{5.040782in}{3.927144in}}%
\pgfpathlineto{\pgfqpoint{5.050990in}{3.979278in}}%
\pgfpathlineto{\pgfqpoint{5.060367in}{3.932801in}}%
\pgfpathlineto{\pgfqpoint{5.090836in}{3.801234in}}%
\pgfpathlineto{\pgfqpoint{5.123361in}{3.900104in}}%
\pgfpathlineto{\pgfqpoint{5.113170in}{3.861473in}}%
\pgfpathlineto{\pgfqpoint{5.103825in}{3.912203in}}%
\pgfpathlineto{\pgfqpoint{5.071849in}{3.867088in}}%
\pgfpathlineto{\pgfqpoint{5.040782in}{3.927144in}}%
\pgfpathclose%
\pgfusepath{fill}%
\end{pgfscope}%
\begin{pgfscope}%
\pgfpathrectangle{\pgfqpoint{1.020000in}{0.880000in}}{\pgfqpoint{6.160000in}{6.160000in}}%
\pgfusepath{clip}%
\pgfsetbuttcap%
\pgfsetroundjoin%
\definecolor{currentfill}{rgb}{0.559747,0.694768,0.996075}%
\pgfsetfillcolor{currentfill}%
\pgfsetlinewidth{0.000000pt}%
\definecolor{currentstroke}{rgb}{0.000000,0.000000,0.000000}%
\pgfsetstrokecolor{currentstroke}%
\pgfsetdash{}{0pt}%
\pgfpathmoveto{\pgfqpoint{4.414286in}{4.063039in}}%
\pgfpathlineto{\pgfqpoint{4.423309in}{4.018856in}}%
\pgfpathlineto{\pgfqpoint{4.432476in}{4.028528in}}%
\pgfpathlineto{\pgfqpoint{4.464307in}{4.027330in}}%
\pgfpathlineto{\pgfqpoint{4.495981in}{3.981548in}}%
\pgfpathlineto{\pgfqpoint{4.486620in}{3.927615in}}%
\pgfpathlineto{\pgfqpoint{4.477885in}{4.077697in}}%
\pgfpathlineto{\pgfqpoint{4.445538in}{3.858348in}}%
\pgfpathlineto{\pgfqpoint{4.414286in}{4.063039in}}%
\pgfpathclose%
\pgfusepath{fill}%
\end{pgfscope}%
\begin{pgfscope}%
\pgfpathrectangle{\pgfqpoint{1.020000in}{0.880000in}}{\pgfqpoint{6.160000in}{6.160000in}}%
\pgfusepath{clip}%
\pgfsetbuttcap%
\pgfsetroundjoin%
\definecolor{currentfill}{rgb}{0.576051,0.708780,0.997755}%
\pgfsetfillcolor{currentfill}%
\pgfsetlinewidth{0.000000pt}%
\definecolor{currentstroke}{rgb}{0.000000,0.000000,0.000000}%
\pgfsetstrokecolor{currentstroke}%
\pgfsetdash{}{0pt}%
\pgfpathmoveto{\pgfqpoint{4.686063in}{3.910574in}}%
\pgfpathlineto{\pgfqpoint{4.695740in}{3.967642in}}%
\pgfpathlineto{\pgfqpoint{4.706184in}{4.166612in}}%
\pgfpathlineto{\pgfqpoint{4.737180in}{4.021608in}}%
\pgfpathlineto{\pgfqpoint{4.768283in}{3.913870in}}%
\pgfpathlineto{\pgfqpoint{4.759752in}{4.076902in}}%
\pgfpathlineto{\pgfqpoint{4.750241in}{4.068003in}}%
\pgfpathlineto{\pgfqpoint{4.718584in}{4.073537in}}%
\pgfpathlineto{\pgfqpoint{4.686063in}{3.910574in}}%
\pgfpathclose%
\pgfusepath{fill}%
\end{pgfscope}%
\begin{pgfscope}%
\pgfpathrectangle{\pgfqpoint{1.020000in}{0.880000in}}{\pgfqpoint{6.160000in}{6.160000in}}%
\pgfusepath{clip}%
\pgfsetbuttcap%
\pgfsetroundjoin%
\definecolor{currentfill}{rgb}{0.603162,0.731527,0.999565}%
\pgfsetfillcolor{currentfill}%
\pgfsetlinewidth{0.000000pt}%
\definecolor{currentstroke}{rgb}{0.000000,0.000000,0.000000}%
\pgfsetstrokecolor{currentstroke}%
\pgfsetdash{}{0pt}%
\pgfpathmoveto{\pgfqpoint{4.477885in}{4.077697in}}%
\pgfpathlineto{\pgfqpoint{4.486620in}{3.927615in}}%
\pgfpathlineto{\pgfqpoint{4.495981in}{3.981548in}}%
\pgfpathlineto{\pgfqpoint{4.527762in}{3.980487in}}%
\pgfpathlineto{\pgfqpoint{4.560640in}{4.277697in}}%
\pgfpathlineto{\pgfqpoint{4.550369in}{4.007573in}}%
\pgfpathlineto{\pgfqpoint{4.541553in}{4.126793in}}%
\pgfpathlineto{\pgfqpoint{4.509948in}{4.172527in}}%
\pgfpathlineto{\pgfqpoint{4.477885in}{4.077697in}}%
\pgfpathclose%
\pgfusepath{fill}%
\end{pgfscope}%
\begin{pgfscope}%
\pgfpathrectangle{\pgfqpoint{1.020000in}{0.880000in}}{\pgfqpoint{6.160000in}{6.160000in}}%
\pgfusepath{clip}%
\pgfsetbuttcap%
\pgfsetroundjoin%
\definecolor{currentfill}{rgb}{0.543440,0.680003,0.993051}%
\pgfsetfillcolor{currentfill}%
\pgfsetlinewidth{0.000000pt}%
\definecolor{currentstroke}{rgb}{0.000000,0.000000,0.000000}%
\pgfsetstrokecolor{currentstroke}%
\pgfsetdash{}{0pt}%
\pgfpathmoveto{\pgfqpoint{4.832926in}{4.115825in}}%
\pgfpathlineto{\pgfqpoint{4.841537in}{3.970699in}}%
\pgfpathlineto{\pgfqpoint{4.850166in}{3.829474in}}%
\pgfpathlineto{\pgfqpoint{4.882170in}{3.876694in}}%
\pgfpathlineto{\pgfqpoint{4.914702in}{3.995021in}}%
\pgfpathlineto{\pgfqpoint{4.904430in}{3.905271in}}%
\pgfpathlineto{\pgfqpoint{4.895220in}{3.960369in}}%
\pgfpathlineto{\pgfqpoint{4.864373in}{4.077456in}}%
\pgfpathlineto{\pgfqpoint{4.832926in}{4.115825in}}%
\pgfpathclose%
\pgfusepath{fill}%
\end{pgfscope}%
\begin{pgfscope}%
\pgfpathrectangle{\pgfqpoint{1.020000in}{0.880000in}}{\pgfqpoint{6.160000in}{6.160000in}}%
\pgfusepath{clip}%
\pgfsetbuttcap%
\pgfsetroundjoin%
\definecolor{currentfill}{rgb}{0.510824,0.649397,0.985079}%
\pgfsetfillcolor{currentfill}%
\pgfsetlinewidth{0.000000pt}%
\definecolor{currentstroke}{rgb}{0.000000,0.000000,0.000000}%
\pgfsetstrokecolor{currentstroke}%
\pgfsetdash{}{0pt}%
\pgfpathmoveto{\pgfqpoint{5.249403in}{3.871526in}}%
\pgfpathlineto{\pgfqpoint{5.260288in}{3.957207in}}%
\pgfpathlineto{\pgfqpoint{5.270112in}{3.941369in}}%
\pgfpathlineto{\pgfqpoint{5.301222in}{3.896145in}}%
\pgfpathlineto{\pgfqpoint{5.331321in}{3.766618in}}%
\pgfpathlineto{\pgfqpoint{5.323344in}{3.948033in}}%
\pgfpathlineto{\pgfqpoint{5.312850in}{3.907037in}}%
\pgfpathlineto{\pgfqpoint{5.281915in}{3.961121in}}%
\pgfpathlineto{\pgfqpoint{5.249403in}{3.871526in}}%
\pgfpathclose%
\pgfusepath{fill}%
\end{pgfscope}%
\begin{pgfscope}%
\pgfpathrectangle{\pgfqpoint{1.020000in}{0.880000in}}{\pgfqpoint{6.160000in}{6.160000in}}%
\pgfusepath{clip}%
\pgfsetbuttcap%
\pgfsetroundjoin%
\definecolor{currentfill}{rgb}{0.624703,0.748318,0.998719}%
\pgfsetfillcolor{currentfill}%
\pgfsetlinewidth{0.000000pt}%
\definecolor{currentstroke}{rgb}{0.000000,0.000000,0.000000}%
\pgfsetstrokecolor{currentstroke}%
\pgfsetdash{}{0pt}%
\pgfpathmoveto{\pgfqpoint{4.205355in}{4.120181in}}%
\pgfpathlineto{\pgfqpoint{4.214249in}{4.035169in}}%
\pgfpathlineto{\pgfqpoint{4.223274in}{4.273838in}}%
\pgfpathlineto{\pgfqpoint{4.255057in}{4.024725in}}%
\pgfpathlineto{\pgfqpoint{4.286977in}{4.079536in}}%
\pgfpathlineto{\pgfqpoint{4.277977in}{4.072055in}}%
\pgfpathlineto{\pgfqpoint{4.269046in}{4.123092in}}%
\pgfpathlineto{\pgfqpoint{4.237162in}{4.034033in}}%
\pgfpathlineto{\pgfqpoint{4.205355in}{4.120181in}}%
\pgfpathclose%
\pgfusepath{fill}%
\end{pgfscope}%
\begin{pgfscope}%
\pgfpathrectangle{\pgfqpoint{1.020000in}{0.880000in}}{\pgfqpoint{6.160000in}{6.160000in}}%
\pgfusepath{clip}%
\pgfsetbuttcap%
\pgfsetroundjoin%
\definecolor{currentfill}{rgb}{0.592356,0.722792,0.999434}%
\pgfsetfillcolor{currentfill}%
\pgfsetlinewidth{0.000000pt}%
\definecolor{currentstroke}{rgb}{0.000000,0.000000,0.000000}%
\pgfsetstrokecolor{currentstroke}%
\pgfsetdash{}{0pt}%
\pgfpathmoveto{\pgfqpoint{4.623766in}{4.154320in}}%
\pgfpathlineto{\pgfqpoint{4.632108in}{3.928433in}}%
\pgfpathlineto{\pgfqpoint{4.642538in}{4.165690in}}%
\pgfpathlineto{\pgfqpoint{4.673632in}{4.014807in}}%
\pgfpathlineto{\pgfqpoint{4.706184in}{4.166612in}}%
\pgfpathlineto{\pgfqpoint{4.695740in}{3.967642in}}%
\pgfpathlineto{\pgfqpoint{4.686063in}{3.910574in}}%
\pgfpathlineto{\pgfqpoint{4.654875in}{4.014029in}}%
\pgfpathlineto{\pgfqpoint{4.623766in}{4.154320in}}%
\pgfpathclose%
\pgfusepath{fill}%
\end{pgfscope}%
\begin{pgfscope}%
\pgfpathrectangle{\pgfqpoint{1.020000in}{0.880000in}}{\pgfqpoint{6.160000in}{6.160000in}}%
\pgfusepath{clip}%
\pgfsetbuttcap%
\pgfsetroundjoin%
\definecolor{currentfill}{rgb}{0.538004,0.674902,0.991722}%
\pgfsetfillcolor{currentfill}%
\pgfsetlinewidth{0.000000pt}%
\definecolor{currentstroke}{rgb}{0.000000,0.000000,0.000000}%
\pgfsetstrokecolor{currentstroke}%
\pgfsetdash{}{0pt}%
\pgfpathmoveto{\pgfqpoint{4.768283in}{3.913870in}}%
\pgfpathlineto{\pgfqpoint{4.778241in}{3.993969in}}%
\pgfpathlineto{\pgfqpoint{4.786972in}{3.864372in}}%
\pgfpathlineto{\pgfqpoint{4.819340in}{3.966419in}}%
\pgfpathlineto{\pgfqpoint{4.850166in}{3.829474in}}%
\pgfpathlineto{\pgfqpoint{4.841537in}{3.970699in}}%
\pgfpathlineto{\pgfqpoint{4.832926in}{4.115825in}}%
\pgfpathlineto{\pgfqpoint{4.800267in}{3.963554in}}%
\pgfpathlineto{\pgfqpoint{4.768283in}{3.913870in}}%
\pgfpathclose%
\pgfusepath{fill}%
\end{pgfscope}%
\begin{pgfscope}%
\pgfpathrectangle{\pgfqpoint{1.020000in}{0.880000in}}{\pgfqpoint{6.160000in}{6.160000in}}%
\pgfusepath{clip}%
\pgfsetbuttcap%
\pgfsetroundjoin%
\definecolor{currentfill}{rgb}{0.478462,0.616564,0.972721}%
\pgfsetfillcolor{currentfill}%
\pgfsetlinewidth{0.000000pt}%
\definecolor{currentstroke}{rgb}{0.000000,0.000000,0.000000}%
\pgfsetstrokecolor{currentstroke}%
\pgfsetdash{}{0pt}%
\pgfpathmoveto{\pgfqpoint{5.396173in}{3.917835in}}%
\pgfpathlineto{\pgfqpoint{5.403859in}{3.717172in}}%
\pgfpathlineto{\pgfqpoint{5.414632in}{3.769682in}}%
\pgfpathlineto{\pgfqpoint{5.447177in}{3.849095in}}%
\pgfpathlineto{\pgfqpoint{5.437068in}{3.854023in}}%
\pgfpathlineto{\pgfqpoint{5.428106in}{3.950012in}}%
\pgfpathlineto{\pgfqpoint{5.396173in}{3.917835in}}%
\pgfpathclose%
\pgfusepath{fill}%
\end{pgfscope}%
\begin{pgfscope}%
\pgfpathrectangle{\pgfqpoint{1.020000in}{0.880000in}}{\pgfqpoint{6.160000in}{6.160000in}}%
\pgfusepath{clip}%
\pgfsetbuttcap%
\pgfsetroundjoin%
\definecolor{currentfill}{rgb}{0.505423,0.643995,0.983157}%
\pgfsetfillcolor{currentfill}%
\pgfsetlinewidth{0.000000pt}%
\definecolor{currentstroke}{rgb}{0.000000,0.000000,0.000000}%
\pgfsetstrokecolor{currentstroke}%
\pgfsetdash{}{0pt}%
\pgfpathmoveto{\pgfqpoint{5.186891in}{3.932323in}}%
\pgfpathlineto{\pgfqpoint{5.195651in}{3.817799in}}%
\pgfpathlineto{\pgfqpoint{5.207081in}{3.965743in}}%
\pgfpathlineto{\pgfqpoint{5.236220in}{3.726012in}}%
\pgfpathlineto{\pgfqpoint{5.270112in}{3.941369in}}%
\pgfpathlineto{\pgfqpoint{5.260288in}{3.957207in}}%
\pgfpathlineto{\pgfqpoint{5.249403in}{3.871526in}}%
\pgfpathlineto{\pgfqpoint{5.218662in}{3.950165in}}%
\pgfpathlineto{\pgfqpoint{5.186891in}{3.932323in}}%
\pgfpathclose%
\pgfusepath{fill}%
\end{pgfscope}%
\begin{pgfscope}%
\pgfpathrectangle{\pgfqpoint{1.020000in}{0.880000in}}{\pgfqpoint{6.160000in}{6.160000in}}%
\pgfusepath{clip}%
\pgfsetbuttcap%
\pgfsetroundjoin%
\definecolor{currentfill}{rgb}{0.462354,0.599830,0.965857}%
\pgfsetfillcolor{currentfill}%
\pgfsetlinewidth{0.000000pt}%
\definecolor{currentstroke}{rgb}{0.000000,0.000000,0.000000}%
\pgfsetstrokecolor{currentstroke}%
\pgfsetdash{}{0pt}%
\pgfpathmoveto{\pgfqpoint{5.331321in}{3.766618in}}%
\pgfpathlineto{\pgfqpoint{5.342429in}{3.857891in}}%
\pgfpathlineto{\pgfqpoint{5.352157in}{3.827214in}}%
\pgfpathlineto{\pgfqpoint{5.383879in}{3.837507in}}%
\pgfpathlineto{\pgfqpoint{5.414632in}{3.769682in}}%
\pgfpathlineto{\pgfqpoint{5.403859in}{3.717172in}}%
\pgfpathlineto{\pgfqpoint{5.396173in}{3.917835in}}%
\pgfpathlineto{\pgfqpoint{5.363385in}{3.812739in}}%
\pgfpathlineto{\pgfqpoint{5.331321in}{3.766618in}}%
\pgfpathclose%
\pgfusepath{fill}%
\end{pgfscope}%
\begin{pgfscope}%
\pgfpathrectangle{\pgfqpoint{1.020000in}{0.880000in}}{\pgfqpoint{6.160000in}{6.160000in}}%
\pgfusepath{clip}%
\pgfsetbuttcap%
\pgfsetroundjoin%
\definecolor{currentfill}{rgb}{0.592356,0.722792,0.999434}%
\pgfsetfillcolor{currentfill}%
\pgfsetlinewidth{0.000000pt}%
\definecolor{currentstroke}{rgb}{0.000000,0.000000,0.000000}%
\pgfsetstrokecolor{currentstroke}%
\pgfsetdash{}{0pt}%
\pgfpathmoveto{\pgfqpoint{4.350407in}{3.911762in}}%
\pgfpathlineto{\pgfqpoint{4.359863in}{4.139975in}}%
\pgfpathlineto{\pgfqpoint{4.368745in}{4.024374in}}%
\pgfpathlineto{\pgfqpoint{4.400681in}{4.055360in}}%
\pgfpathlineto{\pgfqpoint{4.432476in}{4.028528in}}%
\pgfpathlineto{\pgfqpoint{4.423309in}{4.018856in}}%
\pgfpathlineto{\pgfqpoint{4.414286in}{4.063039in}}%
\pgfpathlineto{\pgfqpoint{4.382540in}{4.092535in}}%
\pgfpathlineto{\pgfqpoint{4.350407in}{3.911762in}}%
\pgfpathclose%
\pgfusepath{fill}%
\end{pgfscope}%
\begin{pgfscope}%
\pgfpathrectangle{\pgfqpoint{1.020000in}{0.880000in}}{\pgfqpoint{6.160000in}{6.160000in}}%
\pgfusepath{clip}%
\pgfsetbuttcap%
\pgfsetroundjoin%
\definecolor{currentfill}{rgb}{0.554312,0.690097,0.995516}%
\pgfsetfillcolor{currentfill}%
\pgfsetlinewidth{0.000000pt}%
\definecolor{currentstroke}{rgb}{0.000000,0.000000,0.000000}%
\pgfsetstrokecolor{currentstroke}%
\pgfsetdash{}{0pt}%
\pgfpathmoveto{\pgfqpoint{4.977163in}{3.880632in}}%
\pgfpathlineto{\pgfqpoint{4.988062in}{4.031074in}}%
\pgfpathlineto{\pgfqpoint{4.997953in}{4.051047in}}%
\pgfpathlineto{\pgfqpoint{5.029562in}{4.036410in}}%
\pgfpathlineto{\pgfqpoint{5.060367in}{3.932801in}}%
\pgfpathlineto{\pgfqpoint{5.050990in}{3.979278in}}%
\pgfpathlineto{\pgfqpoint{5.040782in}{3.927144in}}%
\pgfpathlineto{\pgfqpoint{5.008986in}{3.905310in}}%
\pgfpathlineto{\pgfqpoint{4.977163in}{3.880632in}}%
\pgfpathclose%
\pgfusepath{fill}%
\end{pgfscope}%
\begin{pgfscope}%
\pgfpathrectangle{\pgfqpoint{1.020000in}{0.880000in}}{\pgfqpoint{6.160000in}{6.160000in}}%
\pgfusepath{clip}%
\pgfsetbuttcap%
\pgfsetroundjoin%
\definecolor{currentfill}{rgb}{0.635474,0.756714,0.998297}%
\pgfsetfillcolor{currentfill}%
\pgfsetlinewidth{0.000000pt}%
\definecolor{currentstroke}{rgb}{0.000000,0.000000,0.000000}%
\pgfsetstrokecolor{currentstroke}%
\pgfsetdash{}{0pt}%
\pgfpathmoveto{\pgfqpoint{4.141625in}{4.048291in}}%
\pgfpathlineto{\pgfqpoint{4.150445in}{4.153332in}}%
\pgfpathlineto{\pgfqpoint{4.159337in}{4.107873in}}%
\pgfpathlineto{\pgfqpoint{4.191281in}{4.100703in}}%
\pgfpathlineto{\pgfqpoint{4.223274in}{4.273838in}}%
\pgfpathlineto{\pgfqpoint{4.214249in}{4.035169in}}%
\pgfpathlineto{\pgfqpoint{4.205355in}{4.120181in}}%
\pgfpathlineto{\pgfqpoint{4.173495in}{4.008247in}}%
\pgfpathlineto{\pgfqpoint{4.141625in}{4.048291in}}%
\pgfpathclose%
\pgfusepath{fill}%
\end{pgfscope}%
\begin{pgfscope}%
\pgfpathrectangle{\pgfqpoint{1.020000in}{0.880000in}}{\pgfqpoint{6.160000in}{6.160000in}}%
\pgfusepath{clip}%
\pgfsetbuttcap%
\pgfsetroundjoin%
\definecolor{currentfill}{rgb}{0.603162,0.731527,0.999565}%
\pgfsetfillcolor{currentfill}%
\pgfsetlinewidth{0.000000pt}%
\definecolor{currentstroke}{rgb}{0.000000,0.000000,0.000000}%
\pgfsetstrokecolor{currentstroke}%
\pgfsetdash{}{0pt}%
\pgfpathmoveto{\pgfqpoint{4.560640in}{4.277697in}}%
\pgfpathlineto{\pgfqpoint{4.568988in}{4.028698in}}%
\pgfpathlineto{\pgfqpoint{4.578241in}{4.016838in}}%
\pgfpathlineto{\pgfqpoint{4.609456in}{3.874828in}}%
\pgfpathlineto{\pgfqpoint{4.642538in}{4.165690in}}%
\pgfpathlineto{\pgfqpoint{4.632108in}{3.928433in}}%
\pgfpathlineto{\pgfqpoint{4.623766in}{4.154320in}}%
\pgfpathlineto{\pgfqpoint{4.591262in}{3.977098in}}%
\pgfpathlineto{\pgfqpoint{4.560640in}{4.277697in}}%
\pgfpathclose%
\pgfusepath{fill}%
\end{pgfscope}%
\begin{pgfscope}%
\pgfpathrectangle{\pgfqpoint{1.020000in}{0.880000in}}{\pgfqpoint{6.160000in}{6.160000in}}%
\pgfusepath{clip}%
\pgfsetbuttcap%
\pgfsetroundjoin%
\definecolor{currentfill}{rgb}{0.592356,0.722792,0.999434}%
\pgfsetfillcolor{currentfill}%
\pgfsetlinewidth{0.000000pt}%
\definecolor{currentstroke}{rgb}{0.000000,0.000000,0.000000}%
\pgfsetstrokecolor{currentstroke}%
\pgfsetdash{}{0pt}%
\pgfpathmoveto{\pgfqpoint{4.286977in}{4.079536in}}%
\pgfpathlineto{\pgfqpoint{4.295971in}{4.063204in}}%
\pgfpathlineto{\pgfqpoint{4.304995in}{4.059844in}}%
\pgfpathlineto{\pgfqpoint{4.336680in}{3.908561in}}%
\pgfpathlineto{\pgfqpoint{4.368745in}{4.024374in}}%
\pgfpathlineto{\pgfqpoint{4.359863in}{4.139975in}}%
\pgfpathlineto{\pgfqpoint{4.350407in}{3.911762in}}%
\pgfpathlineto{\pgfqpoint{4.318803in}{4.051287in}}%
\pgfpathlineto{\pgfqpoint{4.286977in}{4.079536in}}%
\pgfpathclose%
\pgfusepath{fill}%
\end{pgfscope}%
\begin{pgfscope}%
\pgfpathrectangle{\pgfqpoint{1.020000in}{0.880000in}}{\pgfqpoint{6.160000in}{6.160000in}}%
\pgfusepath{clip}%
\pgfsetbuttcap%
\pgfsetroundjoin%
\definecolor{currentfill}{rgb}{0.565182,0.699438,0.996635}%
\pgfsetfillcolor{currentfill}%
\pgfsetlinewidth{0.000000pt}%
\definecolor{currentstroke}{rgb}{0.000000,0.000000,0.000000}%
\pgfsetstrokecolor{currentstroke}%
\pgfsetdash{}{0pt}%
\pgfpathmoveto{\pgfqpoint{4.706184in}{4.166612in}}%
\pgfpathlineto{\pgfqpoint{4.714054in}{3.867802in}}%
\pgfpathlineto{\pgfqpoint{4.723847in}{3.933476in}}%
\pgfpathlineto{\pgfqpoint{4.756505in}{4.087831in}}%
\pgfpathlineto{\pgfqpoint{4.786972in}{3.864372in}}%
\pgfpathlineto{\pgfqpoint{4.778241in}{3.993969in}}%
\pgfpathlineto{\pgfqpoint{4.768283in}{3.913870in}}%
\pgfpathlineto{\pgfqpoint{4.737180in}{4.021608in}}%
\pgfpathlineto{\pgfqpoint{4.706184in}{4.166612in}}%
\pgfpathclose%
\pgfusepath{fill}%
\end{pgfscope}%
\begin{pgfscope}%
\pgfpathrectangle{\pgfqpoint{1.020000in}{0.880000in}}{\pgfqpoint{6.160000in}{6.160000in}}%
\pgfusepath{clip}%
\pgfsetbuttcap%
\pgfsetroundjoin%
\definecolor{currentfill}{rgb}{0.527132,0.664700,0.989065}%
\pgfsetfillcolor{currentfill}%
\pgfsetlinewidth{0.000000pt}%
\definecolor{currentstroke}{rgb}{0.000000,0.000000,0.000000}%
\pgfsetstrokecolor{currentstroke}%
\pgfsetdash{}{0pt}%
\pgfpathmoveto{\pgfqpoint{5.123361in}{3.900104in}}%
\pgfpathlineto{\pgfqpoint{5.133016in}{3.879195in}}%
\pgfpathlineto{\pgfqpoint{5.143442in}{3.937251in}}%
\pgfpathlineto{\pgfqpoint{5.175170in}{3.941772in}}%
\pgfpathlineto{\pgfqpoint{5.207081in}{3.965743in}}%
\pgfpathlineto{\pgfqpoint{5.195651in}{3.817799in}}%
\pgfpathlineto{\pgfqpoint{5.186891in}{3.932323in}}%
\pgfpathlineto{\pgfqpoint{5.155805in}{3.986028in}}%
\pgfpathlineto{\pgfqpoint{5.123361in}{3.900104in}}%
\pgfpathclose%
\pgfusepath{fill}%
\end{pgfscope}%
\begin{pgfscope}%
\pgfpathrectangle{\pgfqpoint{1.020000in}{0.880000in}}{\pgfqpoint{6.160000in}{6.160000in}}%
\pgfusepath{clip}%
\pgfsetbuttcap%
\pgfsetroundjoin%
\definecolor{currentfill}{rgb}{0.635474,0.756714,0.998297}%
\pgfsetfillcolor{currentfill}%
\pgfsetlinewidth{0.000000pt}%
\definecolor{currentstroke}{rgb}{0.000000,0.000000,0.000000}%
\pgfsetstrokecolor{currentstroke}%
\pgfsetdash{}{0pt}%
\pgfpathmoveto{\pgfqpoint{4.077621in}{4.257329in}}%
\pgfpathlineto{\pgfqpoint{4.086604in}{4.079468in}}%
\pgfpathlineto{\pgfqpoint{4.095425in}{4.072926in}}%
\pgfpathlineto{\pgfqpoint{4.127359in}{4.136938in}}%
\pgfpathlineto{\pgfqpoint{4.159337in}{4.107873in}}%
\pgfpathlineto{\pgfqpoint{4.150445in}{4.153332in}}%
\pgfpathlineto{\pgfqpoint{4.141625in}{4.048291in}}%
\pgfpathlineto{\pgfqpoint{4.109810in}{3.954437in}}%
\pgfpathlineto{\pgfqpoint{4.077621in}{4.257329in}}%
\pgfpathclose%
\pgfusepath{fill}%
\end{pgfscope}%
\begin{pgfscope}%
\pgfpathrectangle{\pgfqpoint{1.020000in}{0.880000in}}{\pgfqpoint{6.160000in}{6.160000in}}%
\pgfusepath{clip}%
\pgfsetbuttcap%
\pgfsetroundjoin%
\definecolor{currentfill}{rgb}{0.505423,0.643995,0.983157}%
\pgfsetfillcolor{currentfill}%
\pgfsetlinewidth{0.000000pt}%
\definecolor{currentstroke}{rgb}{0.000000,0.000000,0.000000}%
\pgfsetstrokecolor{currentstroke}%
\pgfsetdash{}{0pt}%
\pgfpathmoveto{\pgfqpoint{5.060367in}{3.932801in}}%
\pgfpathlineto{\pgfqpoint{5.070634in}{3.985948in}}%
\pgfpathlineto{\pgfqpoint{5.078543in}{3.771491in}}%
\pgfpathlineto{\pgfqpoint{5.110820in}{3.837495in}}%
\pgfpathlineto{\pgfqpoint{5.143442in}{3.937251in}}%
\pgfpathlineto{\pgfqpoint{5.133016in}{3.879195in}}%
\pgfpathlineto{\pgfqpoint{5.123361in}{3.900104in}}%
\pgfpathlineto{\pgfqpoint{5.090836in}{3.801234in}}%
\pgfpathlineto{\pgfqpoint{5.060367in}{3.932801in}}%
\pgfpathclose%
\pgfusepath{fill}%
\end{pgfscope}%
\begin{pgfscope}%
\pgfpathrectangle{\pgfqpoint{1.020000in}{0.880000in}}{\pgfqpoint{6.160000in}{6.160000in}}%
\pgfusepath{clip}%
\pgfsetbuttcap%
\pgfsetroundjoin%
\definecolor{currentfill}{rgb}{0.559747,0.694768,0.996075}%
\pgfsetfillcolor{currentfill}%
\pgfsetlinewidth{0.000000pt}%
\definecolor{currentstroke}{rgb}{0.000000,0.000000,0.000000}%
\pgfsetstrokecolor{currentstroke}%
\pgfsetdash{}{0pt}%
\pgfpathmoveto{\pgfqpoint{4.914702in}{3.995021in}}%
\pgfpathlineto{\pgfqpoint{4.923297in}{3.852112in}}%
\pgfpathlineto{\pgfqpoint{4.934109in}{4.007417in}}%
\pgfpathlineto{\pgfqpoint{4.965147in}{3.915424in}}%
\pgfpathlineto{\pgfqpoint{4.997953in}{4.051047in}}%
\pgfpathlineto{\pgfqpoint{4.988062in}{4.031074in}}%
\pgfpathlineto{\pgfqpoint{4.977163in}{3.880632in}}%
\pgfpathlineto{\pgfqpoint{4.946834in}{4.053028in}}%
\pgfpathlineto{\pgfqpoint{4.914702in}{3.995021in}}%
\pgfpathclose%
\pgfusepath{fill}%
\end{pgfscope}%
\begin{pgfscope}%
\pgfpathrectangle{\pgfqpoint{1.020000in}{0.880000in}}{\pgfqpoint{6.160000in}{6.160000in}}%
\pgfusepath{clip}%
\pgfsetbuttcap%
\pgfsetroundjoin%
\definecolor{currentfill}{rgb}{0.532568,0.669801,0.990393}%
\pgfsetfillcolor{currentfill}%
\pgfsetlinewidth{0.000000pt}%
\definecolor{currentstroke}{rgb}{0.000000,0.000000,0.000000}%
\pgfsetstrokecolor{currentstroke}%
\pgfsetdash{}{0pt}%
\pgfpathmoveto{\pgfqpoint{4.850166in}{3.829474in}}%
\pgfpathlineto{\pgfqpoint{4.860713in}{3.973947in}}%
\pgfpathlineto{\pgfqpoint{4.870730in}{4.034189in}}%
\pgfpathlineto{\pgfqpoint{4.900994in}{3.819526in}}%
\pgfpathlineto{\pgfqpoint{4.934109in}{4.007417in}}%
\pgfpathlineto{\pgfqpoint{4.923297in}{3.852112in}}%
\pgfpathlineto{\pgfqpoint{4.914702in}{3.995021in}}%
\pgfpathlineto{\pgfqpoint{4.882170in}{3.876694in}}%
\pgfpathlineto{\pgfqpoint{4.850166in}{3.829474in}}%
\pgfpathclose%
\pgfusepath{fill}%
\end{pgfscope}%
\begin{pgfscope}%
\pgfpathrectangle{\pgfqpoint{1.020000in}{0.880000in}}{\pgfqpoint{6.160000in}{6.160000in}}%
\pgfusepath{clip}%
\pgfsetbuttcap%
\pgfsetroundjoin%
\definecolor{currentfill}{rgb}{0.613933,0.739923,0.999142}%
\pgfsetfillcolor{currentfill}%
\pgfsetlinewidth{0.000000pt}%
\definecolor{currentstroke}{rgb}{0.000000,0.000000,0.000000}%
\pgfsetstrokecolor{currentstroke}%
\pgfsetdash{}{0pt}%
\pgfpathmoveto{\pgfqpoint{4.495981in}{3.981548in}}%
\pgfpathlineto{\pgfqpoint{4.505598in}{4.106510in}}%
\pgfpathlineto{\pgfqpoint{4.514763in}{4.081429in}}%
\pgfpathlineto{\pgfqpoint{4.546571in}{4.063049in}}%
\pgfpathlineto{\pgfqpoint{4.578241in}{4.016838in}}%
\pgfpathlineto{\pgfqpoint{4.568988in}{4.028698in}}%
\pgfpathlineto{\pgfqpoint{4.560640in}{4.277697in}}%
\pgfpathlineto{\pgfqpoint{4.527762in}{3.980487in}}%
\pgfpathlineto{\pgfqpoint{4.495981in}{3.981548in}}%
\pgfpathclose%
\pgfusepath{fill}%
\end{pgfscope}%
\begin{pgfscope}%
\pgfpathrectangle{\pgfqpoint{1.020000in}{0.880000in}}{\pgfqpoint{6.160000in}{6.160000in}}%
\pgfusepath{clip}%
\pgfsetbuttcap%
\pgfsetroundjoin%
\definecolor{currentfill}{rgb}{0.505423,0.643995,0.983157}%
\pgfsetfillcolor{currentfill}%
\pgfsetlinewidth{0.000000pt}%
\definecolor{currentstroke}{rgb}{0.000000,0.000000,0.000000}%
\pgfsetstrokecolor{currentstroke}%
\pgfsetdash{}{0pt}%
\pgfpathmoveto{\pgfqpoint{5.270112in}{3.941369in}}%
\pgfpathlineto{\pgfqpoint{5.280108in}{3.939689in}}%
\pgfpathlineto{\pgfqpoint{5.290272in}{3.951445in}}%
\pgfpathlineto{\pgfqpoint{5.320702in}{3.841617in}}%
\pgfpathlineto{\pgfqpoint{5.352157in}{3.827214in}}%
\pgfpathlineto{\pgfqpoint{5.342429in}{3.857891in}}%
\pgfpathlineto{\pgfqpoint{5.331321in}{3.766618in}}%
\pgfpathlineto{\pgfqpoint{5.301222in}{3.896145in}}%
\pgfpathlineto{\pgfqpoint{5.270112in}{3.941369in}}%
\pgfpathclose%
\pgfusepath{fill}%
\end{pgfscope}%
\begin{pgfscope}%
\pgfpathrectangle{\pgfqpoint{1.020000in}{0.880000in}}{\pgfqpoint{6.160000in}{6.160000in}}%
\pgfusepath{clip}%
\pgfsetbuttcap%
\pgfsetroundjoin%
\definecolor{currentfill}{rgb}{0.576051,0.708780,0.997755}%
\pgfsetfillcolor{currentfill}%
\pgfsetlinewidth{0.000000pt}%
\definecolor{currentstroke}{rgb}{0.000000,0.000000,0.000000}%
\pgfsetstrokecolor{currentstroke}%
\pgfsetdash{}{0pt}%
\pgfpathmoveto{\pgfqpoint{4.642538in}{4.165690in}}%
\pgfpathlineto{\pgfqpoint{4.650821in}{3.927851in}}%
\pgfpathlineto{\pgfqpoint{4.660456in}{3.980134in}}%
\pgfpathlineto{\pgfqpoint{4.691939in}{3.911607in}}%
\pgfpathlineto{\pgfqpoint{4.723847in}{3.933476in}}%
\pgfpathlineto{\pgfqpoint{4.714054in}{3.867802in}}%
\pgfpathlineto{\pgfqpoint{4.706184in}{4.166612in}}%
\pgfpathlineto{\pgfqpoint{4.673632in}{4.014807in}}%
\pgfpathlineto{\pgfqpoint{4.642538in}{4.165690in}}%
\pgfpathclose%
\pgfusepath{fill}%
\end{pgfscope}%
\begin{pgfscope}%
\pgfpathrectangle{\pgfqpoint{1.020000in}{0.880000in}}{\pgfqpoint{6.160000in}{6.160000in}}%
\pgfusepath{clip}%
\pgfsetbuttcap%
\pgfsetroundjoin%
\definecolor{currentfill}{rgb}{0.462354,0.599830,0.965857}%
\pgfsetfillcolor{currentfill}%
\pgfsetlinewidth{0.000000pt}%
\definecolor{currentstroke}{rgb}{0.000000,0.000000,0.000000}%
\pgfsetstrokecolor{currentstroke}%
\pgfsetdash{}{0pt}%
\pgfpathmoveto{\pgfqpoint{5.414632in}{3.769682in}}%
\pgfpathlineto{\pgfqpoint{5.425748in}{3.847365in}}%
\pgfpathlineto{\pgfqpoint{5.435339in}{3.799929in}}%
\pgfpathlineto{\pgfqpoint{5.467439in}{3.837965in}}%
\pgfpathlineto{\pgfqpoint{5.455705in}{3.717966in}}%
\pgfpathlineto{\pgfqpoint{5.447177in}{3.849095in}}%
\pgfpathlineto{\pgfqpoint{5.414632in}{3.769682in}}%
\pgfpathclose%
\pgfusepath{fill}%
\end{pgfscope}%
\begin{pgfscope}%
\pgfpathrectangle{\pgfqpoint{1.020000in}{0.880000in}}{\pgfqpoint{6.160000in}{6.160000in}}%
\pgfusepath{clip}%
\pgfsetbuttcap%
\pgfsetroundjoin%
\definecolor{currentfill}{rgb}{0.603162,0.731527,0.999565}%
\pgfsetfillcolor{currentfill}%
\pgfsetlinewidth{0.000000pt}%
\definecolor{currentstroke}{rgb}{0.000000,0.000000,0.000000}%
\pgfsetstrokecolor{currentstroke}%
\pgfsetdash{}{0pt}%
\pgfpathmoveto{\pgfqpoint{4.432476in}{4.028528in}}%
\pgfpathlineto{\pgfqpoint{4.441588in}{4.009176in}}%
\pgfpathlineto{\pgfqpoint{4.450891in}{4.056026in}}%
\pgfpathlineto{\pgfqpoint{4.482774in}{4.049233in}}%
\pgfpathlineto{\pgfqpoint{4.514763in}{4.081429in}}%
\pgfpathlineto{\pgfqpoint{4.505598in}{4.106510in}}%
\pgfpathlineto{\pgfqpoint{4.495981in}{3.981548in}}%
\pgfpathlineto{\pgfqpoint{4.464307in}{4.027330in}}%
\pgfpathlineto{\pgfqpoint{4.432476in}{4.028528in}}%
\pgfpathclose%
\pgfusepath{fill}%
\end{pgfscope}%
\begin{pgfscope}%
\pgfpathrectangle{\pgfqpoint{1.020000in}{0.880000in}}{\pgfqpoint{6.160000in}{6.160000in}}%
\pgfusepath{clip}%
\pgfsetbuttcap%
\pgfsetroundjoin%
\definecolor{currentfill}{rgb}{0.576051,0.708780,0.997755}%
\pgfsetfillcolor{currentfill}%
\pgfsetlinewidth{0.000000pt}%
\definecolor{currentstroke}{rgb}{0.000000,0.000000,0.000000}%
\pgfsetstrokecolor{currentstroke}%
\pgfsetdash{}{0pt}%
\pgfpathmoveto{\pgfqpoint{4.368745in}{4.024374in}}%
\pgfpathlineto{\pgfqpoint{4.377736in}{3.967078in}}%
\pgfpathlineto{\pgfqpoint{4.386867in}{3.974490in}}%
\pgfpathlineto{\pgfqpoint{4.418468in}{3.841496in}}%
\pgfpathlineto{\pgfqpoint{4.450891in}{4.056026in}}%
\pgfpathlineto{\pgfqpoint{4.441588in}{4.009176in}}%
\pgfpathlineto{\pgfqpoint{4.432476in}{4.028528in}}%
\pgfpathlineto{\pgfqpoint{4.400681in}{4.055360in}}%
\pgfpathlineto{\pgfqpoint{4.368745in}{4.024374in}}%
\pgfpathclose%
\pgfusepath{fill}%
\end{pgfscope}%
\begin{pgfscope}%
\pgfpathrectangle{\pgfqpoint{1.020000in}{0.880000in}}{\pgfqpoint{6.160000in}{6.160000in}}%
\pgfusepath{clip}%
\pgfsetbuttcap%
\pgfsetroundjoin%
\definecolor{currentfill}{rgb}{0.651398,0.768121,0.995891}%
\pgfsetfillcolor{currentfill}%
\pgfsetlinewidth{0.000000pt}%
\definecolor{currentstroke}{rgb}{0.000000,0.000000,0.000000}%
\pgfsetstrokecolor{currentstroke}%
\pgfsetdash{}{0pt}%
\pgfpathmoveto{\pgfqpoint{4.013832in}{4.137182in}}%
\pgfpathlineto{\pgfqpoint{4.022530in}{4.165616in}}%
\pgfpathlineto{\pgfqpoint{4.031383in}{4.105686in}}%
\pgfpathlineto{\pgfqpoint{4.063547in}{3.981014in}}%
\pgfpathlineto{\pgfqpoint{4.095425in}{4.072926in}}%
\pgfpathlineto{\pgfqpoint{4.086604in}{4.079468in}}%
\pgfpathlineto{\pgfqpoint{4.077621in}{4.257329in}}%
\pgfpathlineto{\pgfqpoint{4.045711in}{4.200853in}}%
\pgfpathlineto{\pgfqpoint{4.013832in}{4.137182in}}%
\pgfpathclose%
\pgfusepath{fill}%
\end{pgfscope}%
\begin{pgfscope}%
\pgfpathrectangle{\pgfqpoint{1.020000in}{0.880000in}}{\pgfqpoint{6.160000in}{6.160000in}}%
\pgfusepath{clip}%
\pgfsetbuttcap%
\pgfsetroundjoin%
\definecolor{currentfill}{rgb}{0.635474,0.756714,0.998297}%
\pgfsetfillcolor{currentfill}%
\pgfsetlinewidth{0.000000pt}%
\definecolor{currentstroke}{rgb}{0.000000,0.000000,0.000000}%
\pgfsetstrokecolor{currentstroke}%
\pgfsetdash{}{0pt}%
\pgfpathmoveto{\pgfqpoint{4.223274in}{4.273838in}}%
\pgfpathlineto{\pgfqpoint{4.232122in}{4.024765in}}%
\pgfpathlineto{\pgfqpoint{4.241126in}{4.091971in}}%
\pgfpathlineto{\pgfqpoint{4.273137in}{4.146279in}}%
\pgfpathlineto{\pgfqpoint{4.304995in}{4.059844in}}%
\pgfpathlineto{\pgfqpoint{4.295971in}{4.063204in}}%
\pgfpathlineto{\pgfqpoint{4.286977in}{4.079536in}}%
\pgfpathlineto{\pgfqpoint{4.255057in}{4.024725in}}%
\pgfpathlineto{\pgfqpoint{4.223274in}{4.273838in}}%
\pgfpathclose%
\pgfusepath{fill}%
\end{pgfscope}%
\begin{pgfscope}%
\pgfpathrectangle{\pgfqpoint{1.020000in}{0.880000in}}{\pgfqpoint{6.160000in}{6.160000in}}%
\pgfusepath{clip}%
\pgfsetbuttcap%
\pgfsetroundjoin%
\definecolor{currentfill}{rgb}{0.527132,0.664700,0.989065}%
\pgfsetfillcolor{currentfill}%
\pgfsetlinewidth{0.000000pt}%
\definecolor{currentstroke}{rgb}{0.000000,0.000000,0.000000}%
\pgfsetstrokecolor{currentstroke}%
\pgfsetdash{}{0pt}%
\pgfpathmoveto{\pgfqpoint{4.997953in}{4.051047in}}%
\pgfpathlineto{\pgfqpoint{5.005888in}{3.829496in}}%
\pgfpathlineto{\pgfqpoint{5.015046in}{3.758241in}}%
\pgfpathlineto{\pgfqpoint{5.047849in}{3.886603in}}%
\pgfpathlineto{\pgfqpoint{5.078543in}{3.771491in}}%
\pgfpathlineto{\pgfqpoint{5.070634in}{3.985948in}}%
\pgfpathlineto{\pgfqpoint{5.060367in}{3.932801in}}%
\pgfpathlineto{\pgfqpoint{5.029562in}{4.036410in}}%
\pgfpathlineto{\pgfqpoint{4.997953in}{4.051047in}}%
\pgfpathclose%
\pgfusepath{fill}%
\end{pgfscope}%
\begin{pgfscope}%
\pgfpathrectangle{\pgfqpoint{1.020000in}{0.880000in}}{\pgfqpoint{6.160000in}{6.160000in}}%
\pgfusepath{clip}%
\pgfsetbuttcap%
\pgfsetroundjoin%
\definecolor{currentfill}{rgb}{0.548876,0.685104,0.994379}%
\pgfsetfillcolor{currentfill}%
\pgfsetlinewidth{0.000000pt}%
\definecolor{currentstroke}{rgb}{0.000000,0.000000,0.000000}%
\pgfsetstrokecolor{currentstroke}%
\pgfsetdash{}{0pt}%
\pgfpathmoveto{\pgfqpoint{4.786972in}{3.864372in}}%
\pgfpathlineto{\pgfqpoint{4.797242in}{3.987593in}}%
\pgfpathlineto{\pgfqpoint{4.806943in}{4.012106in}}%
\pgfpathlineto{\pgfqpoint{4.837858in}{3.872125in}}%
\pgfpathlineto{\pgfqpoint{4.870730in}{4.034189in}}%
\pgfpathlineto{\pgfqpoint{4.860713in}{3.973947in}}%
\pgfpathlineto{\pgfqpoint{4.850166in}{3.829474in}}%
\pgfpathlineto{\pgfqpoint{4.819340in}{3.966419in}}%
\pgfpathlineto{\pgfqpoint{4.786972in}{3.864372in}}%
\pgfpathclose%
\pgfusepath{fill}%
\end{pgfscope}%
\begin{pgfscope}%
\pgfpathrectangle{\pgfqpoint{1.020000in}{0.880000in}}{\pgfqpoint{6.160000in}{6.160000in}}%
\pgfusepath{clip}%
\pgfsetbuttcap%
\pgfsetroundjoin%
\definecolor{currentfill}{rgb}{0.467678,0.605591,0.968546}%
\pgfsetfillcolor{currentfill}%
\pgfsetlinewidth{0.000000pt}%
\definecolor{currentstroke}{rgb}{0.000000,0.000000,0.000000}%
\pgfsetstrokecolor{currentstroke}%
\pgfsetdash{}{0pt}%
\pgfpathmoveto{\pgfqpoint{5.352157in}{3.827214in}}%
\pgfpathlineto{\pgfqpoint{5.362406in}{3.839820in}}%
\pgfpathlineto{\pgfqpoint{5.371006in}{3.711109in}}%
\pgfpathlineto{\pgfqpoint{5.403933in}{3.818531in}}%
\pgfpathlineto{\pgfqpoint{5.435339in}{3.799929in}}%
\pgfpathlineto{\pgfqpoint{5.425748in}{3.847365in}}%
\pgfpathlineto{\pgfqpoint{5.414632in}{3.769682in}}%
\pgfpathlineto{\pgfqpoint{5.383879in}{3.837507in}}%
\pgfpathlineto{\pgfqpoint{5.352157in}{3.827214in}}%
\pgfpathclose%
\pgfusepath{fill}%
\end{pgfscope}%
\begin{pgfscope}%
\pgfpathrectangle{\pgfqpoint{1.020000in}{0.880000in}}{\pgfqpoint{6.160000in}{6.160000in}}%
\pgfusepath{clip}%
\pgfsetbuttcap%
\pgfsetroundjoin%
\definecolor{currentfill}{rgb}{0.521696,0.659599,0.987736}%
\pgfsetfillcolor{currentfill}%
\pgfsetlinewidth{0.000000pt}%
\definecolor{currentstroke}{rgb}{0.000000,0.000000,0.000000}%
\pgfsetstrokecolor{currentstroke}%
\pgfsetdash{}{0pt}%
\pgfpathmoveto{\pgfqpoint{5.207081in}{3.965743in}}%
\pgfpathlineto{\pgfqpoint{5.215631in}{3.828849in}}%
\pgfpathlineto{\pgfqpoint{5.226183in}{3.885398in}}%
\pgfpathlineto{\pgfqpoint{5.258566in}{3.950218in}}%
\pgfpathlineto{\pgfqpoint{5.290272in}{3.951445in}}%
\pgfpathlineto{\pgfqpoint{5.280108in}{3.939689in}}%
\pgfpathlineto{\pgfqpoint{5.270112in}{3.941369in}}%
\pgfpathlineto{\pgfqpoint{5.236220in}{3.726012in}}%
\pgfpathlineto{\pgfqpoint{5.207081in}{3.965743in}}%
\pgfpathclose%
\pgfusepath{fill}%
\end{pgfscope}%
\begin{pgfscope}%
\pgfpathrectangle{\pgfqpoint{1.020000in}{0.880000in}}{\pgfqpoint{6.160000in}{6.160000in}}%
\pgfusepath{clip}%
\pgfsetbuttcap%
\pgfsetroundjoin%
\definecolor{currentfill}{rgb}{0.581486,0.713451,0.998314}%
\pgfsetfillcolor{currentfill}%
\pgfsetlinewidth{0.000000pt}%
\definecolor{currentstroke}{rgb}{0.000000,0.000000,0.000000}%
\pgfsetstrokecolor{currentstroke}%
\pgfsetdash{}{0pt}%
\pgfpathmoveto{\pgfqpoint{4.578241in}{4.016838in}}%
\pgfpathlineto{\pgfqpoint{4.587184in}{3.924061in}}%
\pgfpathlineto{\pgfqpoint{4.597160in}{4.081732in}}%
\pgfpathlineto{\pgfqpoint{4.628808in}{4.025377in}}%
\pgfpathlineto{\pgfqpoint{4.660456in}{3.980134in}}%
\pgfpathlineto{\pgfqpoint{4.650821in}{3.927851in}}%
\pgfpathlineto{\pgfqpoint{4.642538in}{4.165690in}}%
\pgfpathlineto{\pgfqpoint{4.609456in}{3.874828in}}%
\pgfpathlineto{\pgfqpoint{4.578241in}{4.016838in}}%
\pgfpathclose%
\pgfusepath{fill}%
\end{pgfscope}%
\begin{pgfscope}%
\pgfpathrectangle{\pgfqpoint{1.020000in}{0.880000in}}{\pgfqpoint{6.160000in}{6.160000in}}%
\pgfusepath{clip}%
\pgfsetbuttcap%
\pgfsetroundjoin%
\definecolor{currentfill}{rgb}{0.516260,0.654498,0.986407}%
\pgfsetfillcolor{currentfill}%
\pgfsetlinewidth{0.000000pt}%
\definecolor{currentstroke}{rgb}{0.000000,0.000000,0.000000}%
\pgfsetstrokecolor{currentstroke}%
\pgfsetdash{}{0pt}%
\pgfpathmoveto{\pgfqpoint{4.934109in}{4.007417in}}%
\pgfpathlineto{\pgfqpoint{4.942223in}{3.799218in}}%
\pgfpathlineto{\pgfqpoint{4.953086in}{3.952727in}}%
\pgfpathlineto{\pgfqpoint{4.983377in}{3.764088in}}%
\pgfpathlineto{\pgfqpoint{5.015046in}{3.758241in}}%
\pgfpathlineto{\pgfqpoint{5.005888in}{3.829496in}}%
\pgfpathlineto{\pgfqpoint{4.997953in}{4.051047in}}%
\pgfpathlineto{\pgfqpoint{4.965147in}{3.915424in}}%
\pgfpathlineto{\pgfqpoint{4.934109in}{4.007417in}}%
\pgfpathclose%
\pgfusepath{fill}%
\end{pgfscope}%
\begin{pgfscope}%
\pgfpathrectangle{\pgfqpoint{1.020000in}{0.880000in}}{\pgfqpoint{6.160000in}{6.160000in}}%
\pgfusepath{clip}%
\pgfsetbuttcap%
\pgfsetroundjoin%
\definecolor{currentfill}{rgb}{0.651398,0.768121,0.995891}%
\pgfsetfillcolor{currentfill}%
\pgfsetlinewidth{0.000000pt}%
\definecolor{currentstroke}{rgb}{0.000000,0.000000,0.000000}%
\pgfsetstrokecolor{currentstroke}%
\pgfsetdash{}{0pt}%
\pgfpathmoveto{\pgfqpoint{3.949482in}{4.294322in}}%
\pgfpathlineto{\pgfqpoint{3.958751in}{4.037927in}}%
\pgfpathlineto{\pgfqpoint{3.967560in}{3.983364in}}%
\pgfpathlineto{\pgfqpoint{3.999424in}{4.064823in}}%
\pgfpathlineto{\pgfqpoint{4.031383in}{4.105686in}}%
\pgfpathlineto{\pgfqpoint{4.022530in}{4.165616in}}%
\pgfpathlineto{\pgfqpoint{4.013832in}{4.137182in}}%
\pgfpathlineto{\pgfqpoint{3.981788in}{4.167508in}}%
\pgfpathlineto{\pgfqpoint{3.949482in}{4.294322in}}%
\pgfpathclose%
\pgfusepath{fill}%
\end{pgfscope}%
\begin{pgfscope}%
\pgfpathrectangle{\pgfqpoint{1.020000in}{0.880000in}}{\pgfqpoint{6.160000in}{6.160000in}}%
\pgfusepath{clip}%
\pgfsetbuttcap%
\pgfsetroundjoin%
\definecolor{currentfill}{rgb}{0.581486,0.713451,0.998314}%
\pgfsetfillcolor{currentfill}%
\pgfsetlinewidth{0.000000pt}%
\definecolor{currentstroke}{rgb}{0.000000,0.000000,0.000000}%
\pgfsetstrokecolor{currentstroke}%
\pgfsetdash{}{0pt}%
\pgfpathmoveto{\pgfqpoint{4.304995in}{4.059844in}}%
\pgfpathlineto{\pgfqpoint{4.314079in}{4.088750in}}%
\pgfpathlineto{\pgfqpoint{4.323031in}{4.006382in}}%
\pgfpathlineto{\pgfqpoint{4.354913in}{3.960889in}}%
\pgfpathlineto{\pgfqpoint{4.386867in}{3.974490in}}%
\pgfpathlineto{\pgfqpoint{4.377736in}{3.967078in}}%
\pgfpathlineto{\pgfqpoint{4.368745in}{4.024374in}}%
\pgfpathlineto{\pgfqpoint{4.336680in}{3.908561in}}%
\pgfpathlineto{\pgfqpoint{4.304995in}{4.059844in}}%
\pgfpathclose%
\pgfusepath{fill}%
\end{pgfscope}%
\begin{pgfscope}%
\pgfpathrectangle{\pgfqpoint{1.020000in}{0.880000in}}{\pgfqpoint{6.160000in}{6.160000in}}%
\pgfusepath{clip}%
\pgfsetbuttcap%
\pgfsetroundjoin%
\definecolor{currentfill}{rgb}{0.478462,0.616564,0.972721}%
\pgfsetfillcolor{currentfill}%
\pgfsetlinewidth{0.000000pt}%
\definecolor{currentstroke}{rgb}{0.000000,0.000000,0.000000}%
\pgfsetstrokecolor{currentstroke}%
\pgfsetdash{}{0pt}%
\pgfpathmoveto{\pgfqpoint{5.078543in}{3.771491in}}%
\pgfpathlineto{\pgfqpoint{5.088381in}{3.773227in}}%
\pgfpathlineto{\pgfqpoint{5.099068in}{3.865906in}}%
\pgfpathlineto{\pgfqpoint{5.129626in}{3.739010in}}%
\pgfpathlineto{\pgfqpoint{5.161864in}{3.794505in}}%
\pgfpathlineto{\pgfqpoint{5.152038in}{3.802420in}}%
\pgfpathlineto{\pgfqpoint{5.143442in}{3.937251in}}%
\pgfpathlineto{\pgfqpoint{5.110820in}{3.837495in}}%
\pgfpathlineto{\pgfqpoint{5.078543in}{3.771491in}}%
\pgfpathclose%
\pgfusepath{fill}%
\end{pgfscope}%
\begin{pgfscope}%
\pgfpathrectangle{\pgfqpoint{1.020000in}{0.880000in}}{\pgfqpoint{6.160000in}{6.160000in}}%
\pgfusepath{clip}%
\pgfsetbuttcap%
\pgfsetroundjoin%
\definecolor{currentfill}{rgb}{0.510824,0.649397,0.985079}%
\pgfsetfillcolor{currentfill}%
\pgfsetlinewidth{0.000000pt}%
\definecolor{currentstroke}{rgb}{0.000000,0.000000,0.000000}%
\pgfsetstrokecolor{currentstroke}%
\pgfsetdash{}{0pt}%
\pgfpathmoveto{\pgfqpoint{5.143442in}{3.937251in}}%
\pgfpathlineto{\pgfqpoint{5.152038in}{3.802420in}}%
\pgfpathlineto{\pgfqpoint{5.161864in}{3.794505in}}%
\pgfpathlineto{\pgfqpoint{5.194326in}{3.870525in}}%
\pgfpathlineto{\pgfqpoint{5.226183in}{3.885398in}}%
\pgfpathlineto{\pgfqpoint{5.215631in}{3.828849in}}%
\pgfpathlineto{\pgfqpoint{5.207081in}{3.965743in}}%
\pgfpathlineto{\pgfqpoint{5.175170in}{3.941772in}}%
\pgfpathlineto{\pgfqpoint{5.143442in}{3.937251in}}%
\pgfpathclose%
\pgfusepath{fill}%
\end{pgfscope}%
\begin{pgfscope}%
\pgfpathrectangle{\pgfqpoint{1.020000in}{0.880000in}}{\pgfqpoint{6.160000in}{6.160000in}}%
\pgfusepath{clip}%
\pgfsetbuttcap%
\pgfsetroundjoin%
\definecolor{currentfill}{rgb}{0.554312,0.690097,0.995516}%
\pgfsetfillcolor{currentfill}%
\pgfsetlinewidth{0.000000pt}%
\definecolor{currentstroke}{rgb}{0.000000,0.000000,0.000000}%
\pgfsetstrokecolor{currentstroke}%
\pgfsetdash{}{0pt}%
\pgfpathmoveto{\pgfqpoint{4.723847in}{3.933476in}}%
\pgfpathlineto{\pgfqpoint{4.733161in}{3.905817in}}%
\pgfpathlineto{\pgfqpoint{4.743033in}{3.975920in}}%
\pgfpathlineto{\pgfqpoint{4.774008in}{3.826975in}}%
\pgfpathlineto{\pgfqpoint{4.806943in}{4.012106in}}%
\pgfpathlineto{\pgfqpoint{4.797242in}{3.987593in}}%
\pgfpathlineto{\pgfqpoint{4.786972in}{3.864372in}}%
\pgfpathlineto{\pgfqpoint{4.756505in}{4.087831in}}%
\pgfpathlineto{\pgfqpoint{4.723847in}{3.933476in}}%
\pgfpathclose%
\pgfusepath{fill}%
\end{pgfscope}%
\begin{pgfscope}%
\pgfpathrectangle{\pgfqpoint{1.020000in}{0.880000in}}{\pgfqpoint{6.160000in}{6.160000in}}%
\pgfusepath{clip}%
\pgfsetbuttcap%
\pgfsetroundjoin%
\definecolor{currentfill}{rgb}{0.651398,0.768121,0.995891}%
\pgfsetfillcolor{currentfill}%
\pgfsetlinewidth{0.000000pt}%
\definecolor{currentstroke}{rgb}{0.000000,0.000000,0.000000}%
\pgfsetstrokecolor{currentstroke}%
\pgfsetdash{}{0pt}%
\pgfpathmoveto{\pgfqpoint{4.159337in}{4.107873in}}%
\pgfpathlineto{\pgfqpoint{4.168226in}{4.148780in}}%
\pgfpathlineto{\pgfqpoint{4.177151in}{4.008612in}}%
\pgfpathlineto{\pgfqpoint{4.209156in}{4.133710in}}%
\pgfpathlineto{\pgfqpoint{4.241126in}{4.091971in}}%
\pgfpathlineto{\pgfqpoint{4.232122in}{4.024765in}}%
\pgfpathlineto{\pgfqpoint{4.223274in}{4.273838in}}%
\pgfpathlineto{\pgfqpoint{4.191281in}{4.100703in}}%
\pgfpathlineto{\pgfqpoint{4.159337in}{4.107873in}}%
\pgfpathclose%
\pgfusepath{fill}%
\end{pgfscope}%
\begin{pgfscope}%
\pgfpathrectangle{\pgfqpoint{1.020000in}{0.880000in}}{\pgfqpoint{6.160000in}{6.160000in}}%
\pgfusepath{clip}%
\pgfsetbuttcap%
\pgfsetroundjoin%
\definecolor{currentfill}{rgb}{0.630089,0.752516,0.998508}%
\pgfsetfillcolor{currentfill}%
\pgfsetlinewidth{0.000000pt}%
\definecolor{currentstroke}{rgb}{0.000000,0.000000,0.000000}%
\pgfsetstrokecolor{currentstroke}%
\pgfsetdash{}{0pt}%
\pgfpathmoveto{\pgfqpoint{4.095425in}{4.072926in}}%
\pgfpathlineto{\pgfqpoint{4.104304in}{4.014140in}}%
\pgfpathlineto{\pgfqpoint{4.113185in}{3.961894in}}%
\pgfpathlineto{\pgfqpoint{4.145104in}{4.158415in}}%
\pgfpathlineto{\pgfqpoint{4.177151in}{4.008612in}}%
\pgfpathlineto{\pgfqpoint{4.168226in}{4.148780in}}%
\pgfpathlineto{\pgfqpoint{4.159337in}{4.107873in}}%
\pgfpathlineto{\pgfqpoint{4.127359in}{4.136938in}}%
\pgfpathlineto{\pgfqpoint{4.095425in}{4.072926in}}%
\pgfpathclose%
\pgfusepath{fill}%
\end{pgfscope}%
\begin{pgfscope}%
\pgfpathrectangle{\pgfqpoint{1.020000in}{0.880000in}}{\pgfqpoint{6.160000in}{6.160000in}}%
\pgfusepath{clip}%
\pgfsetbuttcap%
\pgfsetroundjoin%
\definecolor{currentfill}{rgb}{0.597777,0.727330,0.999777}%
\pgfsetfillcolor{currentfill}%
\pgfsetlinewidth{0.000000pt}%
\definecolor{currentstroke}{rgb}{0.000000,0.000000,0.000000}%
\pgfsetstrokecolor{currentstroke}%
\pgfsetdash{}{0pt}%
\pgfpathmoveto{\pgfqpoint{4.514763in}{4.081429in}}%
\pgfpathlineto{\pgfqpoint{4.523635in}{3.965812in}}%
\pgfpathlineto{\pgfqpoint{4.533478in}{4.130237in}}%
\pgfpathlineto{\pgfqpoint{4.564413in}{3.862776in}}%
\pgfpathlineto{\pgfqpoint{4.597160in}{4.081732in}}%
\pgfpathlineto{\pgfqpoint{4.587184in}{3.924061in}}%
\pgfpathlineto{\pgfqpoint{4.578241in}{4.016838in}}%
\pgfpathlineto{\pgfqpoint{4.546571in}{4.063049in}}%
\pgfpathlineto{\pgfqpoint{4.514763in}{4.081429in}}%
\pgfpathclose%
\pgfusepath{fill}%
\end{pgfscope}%
\begin{pgfscope}%
\pgfpathrectangle{\pgfqpoint{1.020000in}{0.880000in}}{\pgfqpoint{6.160000in}{6.160000in}}%
\pgfusepath{clip}%
\pgfsetbuttcap%
\pgfsetroundjoin%
\definecolor{currentfill}{rgb}{0.543440,0.680003,0.993051}%
\pgfsetfillcolor{currentfill}%
\pgfsetlinewidth{0.000000pt}%
\definecolor{currentstroke}{rgb}{0.000000,0.000000,0.000000}%
\pgfsetstrokecolor{currentstroke}%
\pgfsetdash{}{0pt}%
\pgfpathmoveto{\pgfqpoint{4.660456in}{3.980134in}}%
\pgfpathlineto{\pgfqpoint{4.669556in}{3.915919in}}%
\pgfpathlineto{\pgfqpoint{4.679579in}{4.037566in}}%
\pgfpathlineto{\pgfqpoint{4.709984in}{3.751385in}}%
\pgfpathlineto{\pgfqpoint{4.743033in}{3.975920in}}%
\pgfpathlineto{\pgfqpoint{4.733161in}{3.905817in}}%
\pgfpathlineto{\pgfqpoint{4.723847in}{3.933476in}}%
\pgfpathlineto{\pgfqpoint{4.691939in}{3.911607in}}%
\pgfpathlineto{\pgfqpoint{4.660456in}{3.980134in}}%
\pgfpathclose%
\pgfusepath{fill}%
\end{pgfscope}%
\begin{pgfscope}%
\pgfpathrectangle{\pgfqpoint{1.020000in}{0.880000in}}{\pgfqpoint{6.160000in}{6.160000in}}%
\pgfusepath{clip}%
\pgfsetbuttcap%
\pgfsetroundjoin%
\definecolor{currentfill}{rgb}{0.478462,0.616564,0.972721}%
\pgfsetfillcolor{currentfill}%
\pgfsetlinewidth{0.000000pt}%
\definecolor{currentstroke}{rgb}{0.000000,0.000000,0.000000}%
\pgfsetstrokecolor{currentstroke}%
\pgfsetdash{}{0pt}%
\pgfpathmoveto{\pgfqpoint{5.015046in}{3.758241in}}%
\pgfpathlineto{\pgfqpoint{5.025174in}{3.801870in}}%
\pgfpathlineto{\pgfqpoint{5.035069in}{3.814654in}}%
\pgfpathlineto{\pgfqpoint{5.067119in}{3.845904in}}%
\pgfpathlineto{\pgfqpoint{5.099068in}{3.865906in}}%
\pgfpathlineto{\pgfqpoint{5.088381in}{3.773227in}}%
\pgfpathlineto{\pgfqpoint{5.078543in}{3.771491in}}%
\pgfpathlineto{\pgfqpoint{5.047849in}{3.886603in}}%
\pgfpathlineto{\pgfqpoint{5.015046in}{3.758241in}}%
\pgfpathclose%
\pgfusepath{fill}%
\end{pgfscope}%
\begin{pgfscope}%
\pgfpathrectangle{\pgfqpoint{1.020000in}{0.880000in}}{\pgfqpoint{6.160000in}{6.160000in}}%
\pgfusepath{clip}%
\pgfsetbuttcap%
\pgfsetroundjoin%
\definecolor{currentfill}{rgb}{0.500031,0.638508,0.981070}%
\pgfsetfillcolor{currentfill}%
\pgfsetlinewidth{0.000000pt}%
\definecolor{currentstroke}{rgb}{0.000000,0.000000,0.000000}%
\pgfsetstrokecolor{currentstroke}%
\pgfsetdash{}{0pt}%
\pgfpathmoveto{\pgfqpoint{5.290272in}{3.951445in}}%
\pgfpathlineto{\pgfqpoint{5.299914in}{3.914246in}}%
\pgfpathlineto{\pgfqpoint{5.309368in}{3.859098in}}%
\pgfpathlineto{\pgfqpoint{5.340902in}{3.844324in}}%
\pgfpathlineto{\pgfqpoint{5.371006in}{3.711109in}}%
\pgfpathlineto{\pgfqpoint{5.362406in}{3.839820in}}%
\pgfpathlineto{\pgfqpoint{5.352157in}{3.827214in}}%
\pgfpathlineto{\pgfqpoint{5.320702in}{3.841617in}}%
\pgfpathlineto{\pgfqpoint{5.290272in}{3.951445in}}%
\pgfpathclose%
\pgfusepath{fill}%
\end{pgfscope}%
\begin{pgfscope}%
\pgfpathrectangle{\pgfqpoint{1.020000in}{0.880000in}}{\pgfqpoint{6.160000in}{6.160000in}}%
\pgfusepath{clip}%
\pgfsetbuttcap%
\pgfsetroundjoin%
\definecolor{currentfill}{rgb}{0.624703,0.748318,0.998719}%
\pgfsetfillcolor{currentfill}%
\pgfsetlinewidth{0.000000pt}%
\definecolor{currentstroke}{rgb}{0.000000,0.000000,0.000000}%
\pgfsetstrokecolor{currentstroke}%
\pgfsetdash{}{0pt}%
\pgfpathmoveto{\pgfqpoint{4.031383in}{4.105686in}}%
\pgfpathlineto{\pgfqpoint{4.040122in}{4.131545in}}%
\pgfpathlineto{\pgfqpoint{4.048854in}{4.179352in}}%
\pgfpathlineto{\pgfqpoint{4.081104in}{4.029995in}}%
\pgfpathlineto{\pgfqpoint{4.113185in}{3.961894in}}%
\pgfpathlineto{\pgfqpoint{4.104304in}{4.014140in}}%
\pgfpathlineto{\pgfqpoint{4.095425in}{4.072926in}}%
\pgfpathlineto{\pgfqpoint{4.063547in}{3.981014in}}%
\pgfpathlineto{\pgfqpoint{4.031383in}{4.105686in}}%
\pgfpathclose%
\pgfusepath{fill}%
\end{pgfscope}%
\begin{pgfscope}%
\pgfpathrectangle{\pgfqpoint{1.020000in}{0.880000in}}{\pgfqpoint{6.160000in}{6.160000in}}%
\pgfusepath{clip}%
\pgfsetbuttcap%
\pgfsetroundjoin%
\definecolor{currentfill}{rgb}{0.548876,0.685104,0.994379}%
\pgfsetfillcolor{currentfill}%
\pgfsetlinewidth{0.000000pt}%
\definecolor{currentstroke}{rgb}{0.000000,0.000000,0.000000}%
\pgfsetstrokecolor{currentstroke}%
\pgfsetdash{}{0pt}%
\pgfpathmoveto{\pgfqpoint{4.870730in}{4.034189in}}%
\pgfpathlineto{\pgfqpoint{4.879659in}{3.932749in}}%
\pgfpathlineto{\pgfqpoint{4.889037in}{3.895310in}}%
\pgfpathlineto{\pgfqpoint{4.921567in}{3.992884in}}%
\pgfpathlineto{\pgfqpoint{4.953086in}{3.952727in}}%
\pgfpathlineto{\pgfqpoint{4.942223in}{3.799218in}}%
\pgfpathlineto{\pgfqpoint{4.934109in}{4.007417in}}%
\pgfpathlineto{\pgfqpoint{4.900994in}{3.819526in}}%
\pgfpathlineto{\pgfqpoint{4.870730in}{4.034189in}}%
\pgfpathclose%
\pgfusepath{fill}%
\end{pgfscope}%
\begin{pgfscope}%
\pgfpathrectangle{\pgfqpoint{1.020000in}{0.880000in}}{\pgfqpoint{6.160000in}{6.160000in}}%
\pgfusepath{clip}%
\pgfsetbuttcap%
\pgfsetroundjoin%
\definecolor{currentfill}{rgb}{0.677823,0.786546,0.991005}%
\pgfsetfillcolor{currentfill}%
\pgfsetlinewidth{0.000000pt}%
\definecolor{currentstroke}{rgb}{0.000000,0.000000,0.000000}%
\pgfsetstrokecolor{currentstroke}%
\pgfsetdash{}{0pt}%
\pgfpathmoveto{\pgfqpoint{3.885852in}{4.115602in}}%
\pgfpathlineto{\pgfqpoint{3.894219in}{4.204081in}}%
\pgfpathlineto{\pgfqpoint{3.903361in}{4.026169in}}%
\pgfpathlineto{\pgfqpoint{3.934583in}{4.367229in}}%
\pgfpathlineto{\pgfqpoint{3.967560in}{3.983364in}}%
\pgfpathlineto{\pgfqpoint{3.958751in}{4.037927in}}%
\pgfpathlineto{\pgfqpoint{3.949482in}{4.294322in}}%
\pgfpathlineto{\pgfqpoint{3.917661in}{4.199897in}}%
\pgfpathlineto{\pgfqpoint{3.885852in}{4.115602in}}%
\pgfpathclose%
\pgfusepath{fill}%
\end{pgfscope}%
\begin{pgfscope}%
\pgfpathrectangle{\pgfqpoint{1.020000in}{0.880000in}}{\pgfqpoint{6.160000in}{6.160000in}}%
\pgfusepath{clip}%
\pgfsetbuttcap%
\pgfsetroundjoin%
\definecolor{currentfill}{rgb}{0.532568,0.669801,0.990393}%
\pgfsetfillcolor{currentfill}%
\pgfsetlinewidth{0.000000pt}%
\definecolor{currentstroke}{rgb}{0.000000,0.000000,0.000000}%
\pgfsetstrokecolor{currentstroke}%
\pgfsetdash{}{0pt}%
\pgfpathmoveto{\pgfqpoint{4.806943in}{4.012106in}}%
\pgfpathlineto{\pgfqpoint{4.814899in}{3.755687in}}%
\pgfpathlineto{\pgfqpoint{4.825494in}{3.916901in}}%
\pgfpathlineto{\pgfqpoint{4.856661in}{3.814188in}}%
\pgfpathlineto{\pgfqpoint{4.889037in}{3.895310in}}%
\pgfpathlineto{\pgfqpoint{4.879659in}{3.932749in}}%
\pgfpathlineto{\pgfqpoint{4.870730in}{4.034189in}}%
\pgfpathlineto{\pgfqpoint{4.837858in}{3.872125in}}%
\pgfpathlineto{\pgfqpoint{4.806943in}{4.012106in}}%
\pgfpathclose%
\pgfusepath{fill}%
\end{pgfscope}%
\begin{pgfscope}%
\pgfpathrectangle{\pgfqpoint{1.020000in}{0.880000in}}{\pgfqpoint{6.160000in}{6.160000in}}%
\pgfusepath{clip}%
\pgfsetbuttcap%
\pgfsetroundjoin%
\definecolor{currentfill}{rgb}{0.608547,0.735725,0.999354}%
\pgfsetfillcolor{currentfill}%
\pgfsetlinewidth{0.000000pt}%
\definecolor{currentstroke}{rgb}{0.000000,0.000000,0.000000}%
\pgfsetstrokecolor{currentstroke}%
\pgfsetdash{}{0pt}%
\pgfpathmoveto{\pgfqpoint{4.450891in}{4.056026in}}%
\pgfpathlineto{\pgfqpoint{4.460064in}{4.045411in}}%
\pgfpathlineto{\pgfqpoint{4.469158in}{4.001147in}}%
\pgfpathlineto{\pgfqpoint{4.500767in}{3.894949in}}%
\pgfpathlineto{\pgfqpoint{4.533478in}{4.130237in}}%
\pgfpathlineto{\pgfqpoint{4.523635in}{3.965812in}}%
\pgfpathlineto{\pgfqpoint{4.514763in}{4.081429in}}%
\pgfpathlineto{\pgfqpoint{4.482774in}{4.049233in}}%
\pgfpathlineto{\pgfqpoint{4.450891in}{4.056026in}}%
\pgfpathclose%
\pgfusepath{fill}%
\end{pgfscope}%
\begin{pgfscope}%
\pgfpathrectangle{\pgfqpoint{1.020000in}{0.880000in}}{\pgfqpoint{6.160000in}{6.160000in}}%
\pgfusepath{clip}%
\pgfsetbuttcap%
\pgfsetroundjoin%
\definecolor{currentfill}{rgb}{0.640828,0.760752,0.997846}%
\pgfsetfillcolor{currentfill}%
\pgfsetlinewidth{0.000000pt}%
\definecolor{currentstroke}{rgb}{0.000000,0.000000,0.000000}%
\pgfsetstrokecolor{currentstroke}%
\pgfsetdash{}{0pt}%
\pgfpathmoveto{\pgfqpoint{4.241126in}{4.091971in}}%
\pgfpathlineto{\pgfqpoint{4.250092in}{4.056103in}}%
\pgfpathlineto{\pgfqpoint{4.259141in}{4.114739in}}%
\pgfpathlineto{\pgfqpoint{4.291162in}{4.107287in}}%
\pgfpathlineto{\pgfqpoint{4.323031in}{4.006382in}}%
\pgfpathlineto{\pgfqpoint{4.314079in}{4.088750in}}%
\pgfpathlineto{\pgfqpoint{4.304995in}{4.059844in}}%
\pgfpathlineto{\pgfqpoint{4.273137in}{4.146279in}}%
\pgfpathlineto{\pgfqpoint{4.241126in}{4.091971in}}%
\pgfpathclose%
\pgfusepath{fill}%
\end{pgfscope}%
\begin{pgfscope}%
\pgfpathrectangle{\pgfqpoint{1.020000in}{0.880000in}}{\pgfqpoint{6.160000in}{6.160000in}}%
\pgfusepath{clip}%
\pgfsetbuttcap%
\pgfsetroundjoin%
\definecolor{currentfill}{rgb}{0.592356,0.722792,0.999434}%
\pgfsetfillcolor{currentfill}%
\pgfsetlinewidth{0.000000pt}%
\definecolor{currentstroke}{rgb}{0.000000,0.000000,0.000000}%
\pgfsetstrokecolor{currentstroke}%
\pgfsetdash{}{0pt}%
\pgfpathmoveto{\pgfqpoint{4.386867in}{3.974490in}}%
\pgfpathlineto{\pgfqpoint{4.396255in}{4.094030in}}%
\pgfpathlineto{\pgfqpoint{4.405254in}{4.019552in}}%
\pgfpathlineto{\pgfqpoint{4.437135in}{3.977606in}}%
\pgfpathlineto{\pgfqpoint{4.469158in}{4.001147in}}%
\pgfpathlineto{\pgfqpoint{4.460064in}{4.045411in}}%
\pgfpathlineto{\pgfqpoint{4.450891in}{4.056026in}}%
\pgfpathlineto{\pgfqpoint{4.418468in}{3.841496in}}%
\pgfpathlineto{\pgfqpoint{4.386867in}{3.974490in}}%
\pgfpathclose%
\pgfusepath{fill}%
\end{pgfscope}%
\begin{pgfscope}%
\pgfpathrectangle{\pgfqpoint{1.020000in}{0.880000in}}{\pgfqpoint{6.160000in}{6.160000in}}%
\pgfusepath{clip}%
\pgfsetbuttcap%
\pgfsetroundjoin%
\definecolor{currentfill}{rgb}{0.521696,0.659599,0.987736}%
\pgfsetfillcolor{currentfill}%
\pgfsetlinewidth{0.000000pt}%
\definecolor{currentstroke}{rgb}{0.000000,0.000000,0.000000}%
\pgfsetstrokecolor{currentstroke}%
\pgfsetdash{}{0pt}%
\pgfpathmoveto{\pgfqpoint{5.226183in}{3.885398in}}%
\pgfpathlineto{\pgfqpoint{5.235215in}{3.793905in}}%
\pgfpathlineto{\pgfqpoint{5.245401in}{3.811314in}}%
\pgfpathlineto{\pgfqpoint{5.277439in}{3.840106in}}%
\pgfpathlineto{\pgfqpoint{5.309368in}{3.859098in}}%
\pgfpathlineto{\pgfqpoint{5.299914in}{3.914246in}}%
\pgfpathlineto{\pgfqpoint{5.290272in}{3.951445in}}%
\pgfpathlineto{\pgfqpoint{5.258566in}{3.950218in}}%
\pgfpathlineto{\pgfqpoint{5.226183in}{3.885398in}}%
\pgfpathclose%
\pgfusepath{fill}%
\end{pgfscope}%
\begin{pgfscope}%
\pgfpathrectangle{\pgfqpoint{1.020000in}{0.880000in}}{\pgfqpoint{6.160000in}{6.160000in}}%
\pgfusepath{clip}%
\pgfsetbuttcap%
\pgfsetroundjoin%
\definecolor{currentfill}{rgb}{0.505423,0.643995,0.983157}%
\pgfsetfillcolor{currentfill}%
\pgfsetlinewidth{0.000000pt}%
\definecolor{currentstroke}{rgb}{0.000000,0.000000,0.000000}%
\pgfsetstrokecolor{currentstroke}%
\pgfsetdash{}{0pt}%
\pgfpathmoveto{\pgfqpoint{4.953086in}{3.952727in}}%
\pgfpathlineto{\pgfqpoint{4.962536in}{3.917732in}}%
\pgfpathlineto{\pgfqpoint{4.972054in}{3.889729in}}%
\pgfpathlineto{\pgfqpoint{5.004194in}{3.927047in}}%
\pgfpathlineto{\pgfqpoint{5.035069in}{3.814654in}}%
\pgfpathlineto{\pgfqpoint{5.025174in}{3.801870in}}%
\pgfpathlineto{\pgfqpoint{5.015046in}{3.758241in}}%
\pgfpathlineto{\pgfqpoint{4.983377in}{3.764088in}}%
\pgfpathlineto{\pgfqpoint{4.953086in}{3.952727in}}%
\pgfpathclose%
\pgfusepath{fill}%
\end{pgfscope}%
\begin{pgfscope}%
\pgfpathrectangle{\pgfqpoint{1.020000in}{0.880000in}}{\pgfqpoint{6.160000in}{6.160000in}}%
\pgfusepath{clip}%
\pgfsetbuttcap%
\pgfsetroundjoin%
\definecolor{currentfill}{rgb}{0.581486,0.713451,0.998314}%
\pgfsetfillcolor{currentfill}%
\pgfsetlinewidth{0.000000pt}%
\definecolor{currentstroke}{rgb}{0.000000,0.000000,0.000000}%
\pgfsetstrokecolor{currentstroke}%
\pgfsetdash{}{0pt}%
\pgfpathmoveto{\pgfqpoint{4.597160in}{4.081732in}}%
\pgfpathlineto{\pgfqpoint{4.606332in}{4.036943in}}%
\pgfpathlineto{\pgfqpoint{4.615438in}{3.974183in}}%
\pgfpathlineto{\pgfqpoint{4.646449in}{3.776663in}}%
\pgfpathlineto{\pgfqpoint{4.679579in}{4.037566in}}%
\pgfpathlineto{\pgfqpoint{4.669556in}{3.915919in}}%
\pgfpathlineto{\pgfqpoint{4.660456in}{3.980134in}}%
\pgfpathlineto{\pgfqpoint{4.628808in}{4.025377in}}%
\pgfpathlineto{\pgfqpoint{4.597160in}{4.081732in}}%
\pgfpathclose%
\pgfusepath{fill}%
\end{pgfscope}%
\begin{pgfscope}%
\pgfpathrectangle{\pgfqpoint{1.020000in}{0.880000in}}{\pgfqpoint{6.160000in}{6.160000in}}%
\pgfusepath{clip}%
\pgfsetbuttcap%
\pgfsetroundjoin%
\definecolor{currentfill}{rgb}{0.646113,0.764436,0.996868}%
\pgfsetfillcolor{currentfill}%
\pgfsetlinewidth{0.000000pt}%
\definecolor{currentstroke}{rgb}{0.000000,0.000000,0.000000}%
\pgfsetstrokecolor{currentstroke}%
\pgfsetdash{}{0pt}%
\pgfpathmoveto{\pgfqpoint{3.967560in}{3.983364in}}%
\pgfpathlineto{\pgfqpoint{3.975943in}{4.139686in}}%
\pgfpathlineto{\pgfqpoint{3.984846in}{4.054919in}}%
\pgfpathlineto{\pgfqpoint{4.016928in}{4.063467in}}%
\pgfpathlineto{\pgfqpoint{4.048854in}{4.179352in}}%
\pgfpathlineto{\pgfqpoint{4.040122in}{4.131545in}}%
\pgfpathlineto{\pgfqpoint{4.031383in}{4.105686in}}%
\pgfpathlineto{\pgfqpoint{3.999424in}{4.064823in}}%
\pgfpathlineto{\pgfqpoint{3.967560in}{3.983364in}}%
\pgfpathclose%
\pgfusepath{fill}%
\end{pgfscope}%
\begin{pgfscope}%
\pgfpathrectangle{\pgfqpoint{1.020000in}{0.880000in}}{\pgfqpoint{6.160000in}{6.160000in}}%
\pgfusepath{clip}%
\pgfsetbuttcap%
\pgfsetroundjoin%
\definecolor{currentfill}{rgb}{0.543440,0.680003,0.993051}%
\pgfsetfillcolor{currentfill}%
\pgfsetlinewidth{0.000000pt}%
\definecolor{currentstroke}{rgb}{0.000000,0.000000,0.000000}%
\pgfsetstrokecolor{currentstroke}%
\pgfsetdash{}{0pt}%
\pgfpathmoveto{\pgfqpoint{4.743033in}{3.975920in}}%
\pgfpathlineto{\pgfqpoint{4.753011in}{4.059031in}}%
\pgfpathlineto{\pgfqpoint{4.761966in}{3.958413in}}%
\pgfpathlineto{\pgfqpoint{4.792666in}{3.759862in}}%
\pgfpathlineto{\pgfqpoint{4.825494in}{3.916901in}}%
\pgfpathlineto{\pgfqpoint{4.814899in}{3.755687in}}%
\pgfpathlineto{\pgfqpoint{4.806943in}{4.012106in}}%
\pgfpathlineto{\pgfqpoint{4.774008in}{3.826975in}}%
\pgfpathlineto{\pgfqpoint{4.743033in}{3.975920in}}%
\pgfpathclose%
\pgfusepath{fill}%
\end{pgfscope}%
\begin{pgfscope}%
\pgfpathrectangle{\pgfqpoint{1.020000in}{0.880000in}}{\pgfqpoint{6.160000in}{6.160000in}}%
\pgfusepath{clip}%
\pgfsetbuttcap%
\pgfsetroundjoin%
\definecolor{currentfill}{rgb}{0.505423,0.643995,0.983157}%
\pgfsetfillcolor{currentfill}%
\pgfsetlinewidth{0.000000pt}%
\definecolor{currentstroke}{rgb}{0.000000,0.000000,0.000000}%
\pgfsetstrokecolor{currentstroke}%
\pgfsetdash{}{0pt}%
\pgfpathmoveto{\pgfqpoint{5.435339in}{3.799929in}}%
\pgfpathlineto{\pgfqpoint{5.446217in}{3.853952in}}%
\pgfpathlineto{\pgfqpoint{5.455599in}{3.787525in}}%
\pgfpathlineto{\pgfqpoint{5.488417in}{3.875941in}}%
\pgfpathlineto{\pgfqpoint{5.479129in}{3.951103in}}%
\pgfpathlineto{\pgfqpoint{5.467439in}{3.837965in}}%
\pgfpathlineto{\pgfqpoint{5.435339in}{3.799929in}}%
\pgfpathclose%
\pgfusepath{fill}%
\end{pgfscope}%
\begin{pgfscope}%
\pgfpathrectangle{\pgfqpoint{1.020000in}{0.880000in}}{\pgfqpoint{6.160000in}{6.160000in}}%
\pgfusepath{clip}%
\pgfsetbuttcap%
\pgfsetroundjoin%
\definecolor{currentfill}{rgb}{0.505423,0.643995,0.983157}%
\pgfsetfillcolor{currentfill}%
\pgfsetlinewidth{0.000000pt}%
\definecolor{currentstroke}{rgb}{0.000000,0.000000,0.000000}%
\pgfsetstrokecolor{currentstroke}%
\pgfsetdash{}{0pt}%
\pgfpathmoveto{\pgfqpoint{5.161864in}{3.794505in}}%
\pgfpathlineto{\pgfqpoint{5.173008in}{3.918429in}}%
\pgfpathlineto{\pgfqpoint{5.181236in}{3.744793in}}%
\pgfpathlineto{\pgfqpoint{5.215073in}{3.949169in}}%
\pgfpathlineto{\pgfqpoint{5.245401in}{3.811314in}}%
\pgfpathlineto{\pgfqpoint{5.235215in}{3.793905in}}%
\pgfpathlineto{\pgfqpoint{5.226183in}{3.885398in}}%
\pgfpathlineto{\pgfqpoint{5.194326in}{3.870525in}}%
\pgfpathlineto{\pgfqpoint{5.161864in}{3.794505in}}%
\pgfpathclose%
\pgfusepath{fill}%
\end{pgfscope}%
\begin{pgfscope}%
\pgfpathrectangle{\pgfqpoint{1.020000in}{0.880000in}}{\pgfqpoint{6.160000in}{6.160000in}}%
\pgfusepath{clip}%
\pgfsetbuttcap%
\pgfsetroundjoin%
\definecolor{currentfill}{rgb}{0.473070,0.611077,0.970634}%
\pgfsetfillcolor{currentfill}%
\pgfsetlinewidth{0.000000pt}%
\definecolor{currentstroke}{rgb}{0.000000,0.000000,0.000000}%
\pgfsetstrokecolor{currentstroke}%
\pgfsetdash{}{0pt}%
\pgfpathmoveto{\pgfqpoint{5.099068in}{3.865906in}}%
\pgfpathlineto{\pgfqpoint{5.108418in}{3.809388in}}%
\pgfpathlineto{\pgfqpoint{5.118190in}{3.797153in}}%
\pgfpathlineto{\pgfqpoint{5.148627in}{3.656514in}}%
\pgfpathlineto{\pgfqpoint{5.181236in}{3.744793in}}%
\pgfpathlineto{\pgfqpoint{5.173008in}{3.918429in}}%
\pgfpathlineto{\pgfqpoint{5.161864in}{3.794505in}}%
\pgfpathlineto{\pgfqpoint{5.129626in}{3.739010in}}%
\pgfpathlineto{\pgfqpoint{5.099068in}{3.865906in}}%
\pgfpathclose%
\pgfusepath{fill}%
\end{pgfscope}%
\begin{pgfscope}%
\pgfpathrectangle{\pgfqpoint{1.020000in}{0.880000in}}{\pgfqpoint{6.160000in}{6.160000in}}%
\pgfusepath{clip}%
\pgfsetbuttcap%
\pgfsetroundjoin%
\definecolor{currentfill}{rgb}{0.646113,0.764436,0.996868}%
\pgfsetfillcolor{currentfill}%
\pgfsetlinewidth{0.000000pt}%
\definecolor{currentstroke}{rgb}{0.000000,0.000000,0.000000}%
\pgfsetstrokecolor{currentstroke}%
\pgfsetdash{}{0pt}%
\pgfpathmoveto{\pgfqpoint{4.177151in}{4.008612in}}%
\pgfpathlineto{\pgfqpoint{4.186074in}{4.022542in}}%
\pgfpathlineto{\pgfqpoint{4.195030in}{4.144520in}}%
\pgfpathlineto{\pgfqpoint{4.227090in}{4.108691in}}%
\pgfpathlineto{\pgfqpoint{4.259141in}{4.114739in}}%
\pgfpathlineto{\pgfqpoint{4.250092in}{4.056103in}}%
\pgfpathlineto{\pgfqpoint{4.241126in}{4.091971in}}%
\pgfpathlineto{\pgfqpoint{4.209156in}{4.133710in}}%
\pgfpathlineto{\pgfqpoint{4.177151in}{4.008612in}}%
\pgfpathclose%
\pgfusepath{fill}%
\end{pgfscope}%
\begin{pgfscope}%
\pgfpathrectangle{\pgfqpoint{1.020000in}{0.880000in}}{\pgfqpoint{6.160000in}{6.160000in}}%
\pgfusepath{clip}%
\pgfsetbuttcap%
\pgfsetroundjoin%
\definecolor{currentfill}{rgb}{0.500031,0.638508,0.981070}%
\pgfsetfillcolor{currentfill}%
\pgfsetlinewidth{0.000000pt}%
\definecolor{currentstroke}{rgb}{0.000000,0.000000,0.000000}%
\pgfsetstrokecolor{currentstroke}%
\pgfsetdash{}{0pt}%
\pgfpathmoveto{\pgfqpoint{5.371006in}{3.711109in}}%
\pgfpathlineto{\pgfqpoint{5.383668in}{3.923044in}}%
\pgfpathlineto{\pgfqpoint{5.392281in}{3.793526in}}%
\pgfpathlineto{\pgfqpoint{5.426204in}{3.972837in}}%
\pgfpathlineto{\pgfqpoint{5.455599in}{3.787525in}}%
\pgfpathlineto{\pgfqpoint{5.446217in}{3.853952in}}%
\pgfpathlineto{\pgfqpoint{5.435339in}{3.799929in}}%
\pgfpathlineto{\pgfqpoint{5.403933in}{3.818531in}}%
\pgfpathlineto{\pgfqpoint{5.371006in}{3.711109in}}%
\pgfpathclose%
\pgfusepath{fill}%
\end{pgfscope}%
\begin{pgfscope}%
\pgfpathrectangle{\pgfqpoint{1.020000in}{0.880000in}}{\pgfqpoint{6.160000in}{6.160000in}}%
\pgfusepath{clip}%
\pgfsetbuttcap%
\pgfsetroundjoin%
\definecolor{currentfill}{rgb}{0.724041,0.814910,0.975651}%
\pgfsetfillcolor{currentfill}%
\pgfsetlinewidth{0.000000pt}%
\definecolor{currentstroke}{rgb}{0.000000,0.000000,0.000000}%
\pgfsetstrokecolor{currentstroke}%
\pgfsetdash{}{0pt}%
\pgfpathmoveto{\pgfqpoint{3.821117in}{4.287701in}}%
\pgfpathlineto{\pgfqpoint{3.829718in}{4.280624in}}%
\pgfpathlineto{\pgfqpoint{3.838550in}{4.210522in}}%
\pgfpathlineto{\pgfqpoint{3.870463in}{4.289459in}}%
\pgfpathlineto{\pgfqpoint{3.903361in}{4.026169in}}%
\pgfpathlineto{\pgfqpoint{3.894219in}{4.204081in}}%
\pgfpathlineto{\pgfqpoint{3.885852in}{4.115602in}}%
\pgfpathlineto{\pgfqpoint{3.852838in}{4.410450in}}%
\pgfpathlineto{\pgfqpoint{3.821117in}{4.287701in}}%
\pgfpathclose%
\pgfusepath{fill}%
\end{pgfscope}%
\begin{pgfscope}%
\pgfpathrectangle{\pgfqpoint{1.020000in}{0.880000in}}{\pgfqpoint{6.160000in}{6.160000in}}%
\pgfusepath{clip}%
\pgfsetbuttcap%
\pgfsetroundjoin%
\definecolor{currentfill}{rgb}{0.483854,0.622050,0.974808}%
\pgfsetfillcolor{currentfill}%
\pgfsetlinewidth{0.000000pt}%
\definecolor{currentstroke}{rgb}{0.000000,0.000000,0.000000}%
\pgfsetstrokecolor{currentstroke}%
\pgfsetdash{}{0pt}%
\pgfpathmoveto{\pgfqpoint{5.309368in}{3.859098in}}%
\pgfpathlineto{\pgfqpoint{5.318755in}{3.797116in}}%
\pgfpathlineto{\pgfqpoint{5.328953in}{3.805571in}}%
\pgfpathlineto{\pgfqpoint{5.359178in}{3.675894in}}%
\pgfpathlineto{\pgfqpoint{5.392281in}{3.793526in}}%
\pgfpathlineto{\pgfqpoint{5.383668in}{3.923044in}}%
\pgfpathlineto{\pgfqpoint{5.371006in}{3.711109in}}%
\pgfpathlineto{\pgfqpoint{5.340902in}{3.844324in}}%
\pgfpathlineto{\pgfqpoint{5.309368in}{3.859098in}}%
\pgfpathclose%
\pgfusepath{fill}%
\end{pgfscope}%
\begin{pgfscope}%
\pgfpathrectangle{\pgfqpoint{1.020000in}{0.880000in}}{\pgfqpoint{6.160000in}{6.160000in}}%
\pgfusepath{clip}%
\pgfsetbuttcap%
\pgfsetroundjoin%
\definecolor{currentfill}{rgb}{0.592356,0.722792,0.999434}%
\pgfsetfillcolor{currentfill}%
\pgfsetlinewidth{0.000000pt}%
\definecolor{currentstroke}{rgb}{0.000000,0.000000,0.000000}%
\pgfsetstrokecolor{currentstroke}%
\pgfsetdash{}{0pt}%
\pgfpathmoveto{\pgfqpoint{4.533478in}{4.130237in}}%
\pgfpathlineto{\pgfqpoint{4.542488in}{4.047499in}}%
\pgfpathlineto{\pgfqpoint{4.551227in}{3.889835in}}%
\pgfpathlineto{\pgfqpoint{4.583235in}{3.908619in}}%
\pgfpathlineto{\pgfqpoint{4.615438in}{3.974183in}}%
\pgfpathlineto{\pgfqpoint{4.606332in}{4.036943in}}%
\pgfpathlineto{\pgfqpoint{4.597160in}{4.081732in}}%
\pgfpathlineto{\pgfqpoint{4.564413in}{3.862776in}}%
\pgfpathlineto{\pgfqpoint{4.533478in}{4.130237in}}%
\pgfpathclose%
\pgfusepath{fill}%
\end{pgfscope}%
\begin{pgfscope}%
\pgfpathrectangle{\pgfqpoint{1.020000in}{0.880000in}}{\pgfqpoint{6.160000in}{6.160000in}}%
\pgfusepath{clip}%
\pgfsetbuttcap%
\pgfsetroundjoin%
\definecolor{currentfill}{rgb}{0.635474,0.756714,0.998297}%
\pgfsetfillcolor{currentfill}%
\pgfsetlinewidth{0.000000pt}%
\definecolor{currentstroke}{rgb}{0.000000,0.000000,0.000000}%
\pgfsetstrokecolor{currentstroke}%
\pgfsetdash{}{0pt}%
\pgfpathmoveto{\pgfqpoint{4.113185in}{3.961894in}}%
\pgfpathlineto{\pgfqpoint{4.122025in}{3.992145in}}%
\pgfpathlineto{\pgfqpoint{4.130796in}{4.199592in}}%
\pgfpathlineto{\pgfqpoint{4.162960in}{4.030249in}}%
\pgfpathlineto{\pgfqpoint{4.195030in}{4.144520in}}%
\pgfpathlineto{\pgfqpoint{4.186074in}{4.022542in}}%
\pgfpathlineto{\pgfqpoint{4.177151in}{4.008612in}}%
\pgfpathlineto{\pgfqpoint{4.145104in}{4.158415in}}%
\pgfpathlineto{\pgfqpoint{4.113185in}{3.961894in}}%
\pgfpathclose%
\pgfusepath{fill}%
\end{pgfscope}%
\begin{pgfscope}%
\pgfpathrectangle{\pgfqpoint{1.020000in}{0.880000in}}{\pgfqpoint{6.160000in}{6.160000in}}%
\pgfusepath{clip}%
\pgfsetbuttcap%
\pgfsetroundjoin%
\definecolor{currentfill}{rgb}{0.619318,0.744121,0.998931}%
\pgfsetfillcolor{currentfill}%
\pgfsetlinewidth{0.000000pt}%
\definecolor{currentstroke}{rgb}{0.000000,0.000000,0.000000}%
\pgfsetstrokecolor{currentstroke}%
\pgfsetdash{}{0pt}%
\pgfpathmoveto{\pgfqpoint{4.323031in}{4.006382in}}%
\pgfpathlineto{\pgfqpoint{4.332312in}{4.142967in}}%
\pgfpathlineto{\pgfqpoint{4.341189in}{3.995764in}}%
\pgfpathlineto{\pgfqpoint{4.373411in}{4.105242in}}%
\pgfpathlineto{\pgfqpoint{4.405254in}{4.019552in}}%
\pgfpathlineto{\pgfqpoint{4.396255in}{4.094030in}}%
\pgfpathlineto{\pgfqpoint{4.386867in}{3.974490in}}%
\pgfpathlineto{\pgfqpoint{4.354913in}{3.960889in}}%
\pgfpathlineto{\pgfqpoint{4.323031in}{4.006382in}}%
\pgfpathclose%
\pgfusepath{fill}%
\end{pgfscope}%
\begin{pgfscope}%
\pgfpathrectangle{\pgfqpoint{1.020000in}{0.880000in}}{\pgfqpoint{6.160000in}{6.160000in}}%
\pgfusepath{clip}%
\pgfsetbuttcap%
\pgfsetroundjoin%
\definecolor{currentfill}{rgb}{0.581486,0.713451,0.998314}%
\pgfsetfillcolor{currentfill}%
\pgfsetlinewidth{0.000000pt}%
\definecolor{currentstroke}{rgb}{0.000000,0.000000,0.000000}%
\pgfsetstrokecolor{currentstroke}%
\pgfsetdash{}{0pt}%
\pgfpathmoveto{\pgfqpoint{4.469158in}{4.001147in}}%
\pgfpathlineto{\pgfqpoint{4.478595in}{4.070559in}}%
\pgfpathlineto{\pgfqpoint{4.487160in}{3.841093in}}%
\pgfpathlineto{\pgfqpoint{4.519213in}{3.870732in}}%
\pgfpathlineto{\pgfqpoint{4.551227in}{3.889835in}}%
\pgfpathlineto{\pgfqpoint{4.542488in}{4.047499in}}%
\pgfpathlineto{\pgfqpoint{4.533478in}{4.130237in}}%
\pgfpathlineto{\pgfqpoint{4.500767in}{3.894949in}}%
\pgfpathlineto{\pgfqpoint{4.469158in}{4.001147in}}%
\pgfpathclose%
\pgfusepath{fill}%
\end{pgfscope}%
\begin{pgfscope}%
\pgfpathrectangle{\pgfqpoint{1.020000in}{0.880000in}}{\pgfqpoint{6.160000in}{6.160000in}}%
\pgfusepath{clip}%
\pgfsetbuttcap%
\pgfsetroundjoin%
\definecolor{currentfill}{rgb}{0.543440,0.680003,0.993051}%
\pgfsetfillcolor{currentfill}%
\pgfsetlinewidth{0.000000pt}%
\definecolor{currentstroke}{rgb}{0.000000,0.000000,0.000000}%
\pgfsetstrokecolor{currentstroke}%
\pgfsetdash{}{0pt}%
\pgfpathmoveto{\pgfqpoint{4.889037in}{3.895310in}}%
\pgfpathlineto{\pgfqpoint{4.898404in}{3.854513in}}%
\pgfpathlineto{\pgfqpoint{4.907458in}{3.768785in}}%
\pgfpathlineto{\pgfqpoint{4.940629in}{3.946417in}}%
\pgfpathlineto{\pgfqpoint{4.972054in}{3.889729in}}%
\pgfpathlineto{\pgfqpoint{4.962536in}{3.917732in}}%
\pgfpathlineto{\pgfqpoint{4.953086in}{3.952727in}}%
\pgfpathlineto{\pgfqpoint{4.921567in}{3.992884in}}%
\pgfpathlineto{\pgfqpoint{4.889037in}{3.895310in}}%
\pgfpathclose%
\pgfusepath{fill}%
\end{pgfscope}%
\begin{pgfscope}%
\pgfpathrectangle{\pgfqpoint{1.020000in}{0.880000in}}{\pgfqpoint{6.160000in}{6.160000in}}%
\pgfusepath{clip}%
\pgfsetbuttcap%
\pgfsetroundjoin%
\definecolor{currentfill}{rgb}{0.677823,0.786546,0.991005}%
\pgfsetfillcolor{currentfill}%
\pgfsetlinewidth{0.000000pt}%
\definecolor{currentstroke}{rgb}{0.000000,0.000000,0.000000}%
\pgfsetstrokecolor{currentstroke}%
\pgfsetdash{}{0pt}%
\pgfpathmoveto{\pgfqpoint{3.903361in}{4.026169in}}%
\pgfpathlineto{\pgfqpoint{3.911697in}{4.142083in}}%
\pgfpathlineto{\pgfqpoint{3.920277in}{4.178711in}}%
\pgfpathlineto{\pgfqpoint{3.952493in}{4.159581in}}%
\pgfpathlineto{\pgfqpoint{3.984846in}{4.054919in}}%
\pgfpathlineto{\pgfqpoint{3.975943in}{4.139686in}}%
\pgfpathlineto{\pgfqpoint{3.967560in}{3.983364in}}%
\pgfpathlineto{\pgfqpoint{3.934583in}{4.367229in}}%
\pgfpathlineto{\pgfqpoint{3.903361in}{4.026169in}}%
\pgfpathclose%
\pgfusepath{fill}%
\end{pgfscope}%
\begin{pgfscope}%
\pgfpathrectangle{\pgfqpoint{1.020000in}{0.880000in}}{\pgfqpoint{6.160000in}{6.160000in}}%
\pgfusepath{clip}%
\pgfsetbuttcap%
\pgfsetroundjoin%
\definecolor{currentfill}{rgb}{0.576051,0.708780,0.997755}%
\pgfsetfillcolor{currentfill}%
\pgfsetlinewidth{0.000000pt}%
\definecolor{currentstroke}{rgb}{0.000000,0.000000,0.000000}%
\pgfsetstrokecolor{currentstroke}%
\pgfsetdash{}{0pt}%
\pgfpathmoveto{\pgfqpoint{4.679579in}{4.037566in}}%
\pgfpathlineto{\pgfqpoint{4.688975in}{4.025950in}}%
\pgfpathlineto{\pgfqpoint{4.697998in}{3.938336in}}%
\pgfpathlineto{\pgfqpoint{4.729592in}{3.875507in}}%
\pgfpathlineto{\pgfqpoint{4.761966in}{3.958413in}}%
\pgfpathlineto{\pgfqpoint{4.753011in}{4.059031in}}%
\pgfpathlineto{\pgfqpoint{4.743033in}{3.975920in}}%
\pgfpathlineto{\pgfqpoint{4.709984in}{3.751385in}}%
\pgfpathlineto{\pgfqpoint{4.679579in}{4.037566in}}%
\pgfpathclose%
\pgfusepath{fill}%
\end{pgfscope}%
\begin{pgfscope}%
\pgfpathrectangle{\pgfqpoint{1.020000in}{0.880000in}}{\pgfqpoint{6.160000in}{6.160000in}}%
\pgfusepath{clip}%
\pgfsetbuttcap%
\pgfsetroundjoin%
\definecolor{currentfill}{rgb}{0.738826,0.822572,0.968261}%
\pgfsetfillcolor{currentfill}%
\pgfsetlinewidth{0.000000pt}%
\definecolor{currentstroke}{rgb}{0.000000,0.000000,0.000000}%
\pgfsetstrokecolor{currentstroke}%
\pgfsetdash{}{0pt}%
\pgfpathmoveto{\pgfqpoint{3.756660in}{4.325118in}}%
\pgfpathlineto{\pgfqpoint{3.766029in}{4.119860in}}%
\pgfpathlineto{\pgfqpoint{3.774842in}{4.047549in}}%
\pgfpathlineto{\pgfqpoint{3.805957in}{4.322023in}}%
\pgfpathlineto{\pgfqpoint{3.838550in}{4.210522in}}%
\pgfpathlineto{\pgfqpoint{3.829718in}{4.280624in}}%
\pgfpathlineto{\pgfqpoint{3.821117in}{4.287701in}}%
\pgfpathlineto{\pgfqpoint{3.788739in}{4.348571in}}%
\pgfpathlineto{\pgfqpoint{3.756660in}{4.325118in}}%
\pgfpathclose%
\pgfusepath{fill}%
\end{pgfscope}%
\begin{pgfscope}%
\pgfpathrectangle{\pgfqpoint{1.020000in}{0.880000in}}{\pgfqpoint{6.160000in}{6.160000in}}%
\pgfusepath{clip}%
\pgfsetbuttcap%
\pgfsetroundjoin%
\definecolor{currentfill}{rgb}{0.532568,0.669801,0.990393}%
\pgfsetfillcolor{currentfill}%
\pgfsetlinewidth{0.000000pt}%
\definecolor{currentstroke}{rgb}{0.000000,0.000000,0.000000}%
\pgfsetstrokecolor{currentstroke}%
\pgfsetdash{}{0pt}%
\pgfpathmoveto{\pgfqpoint{4.825494in}{3.916901in}}%
\pgfpathlineto{\pgfqpoint{4.835143in}{3.924394in}}%
\pgfpathlineto{\pgfqpoint{4.845346in}{4.013033in}}%
\pgfpathlineto{\pgfqpoint{4.876260in}{3.863345in}}%
\pgfpathlineto{\pgfqpoint{4.907458in}{3.768785in}}%
\pgfpathlineto{\pgfqpoint{4.898404in}{3.854513in}}%
\pgfpathlineto{\pgfqpoint{4.889037in}{3.895310in}}%
\pgfpathlineto{\pgfqpoint{4.856661in}{3.814188in}}%
\pgfpathlineto{\pgfqpoint{4.825494in}{3.916901in}}%
\pgfpathclose%
\pgfusepath{fill}%
\end{pgfscope}%
\begin{pgfscope}%
\pgfpathrectangle{\pgfqpoint{1.020000in}{0.880000in}}{\pgfqpoint{6.160000in}{6.160000in}}%
\pgfusepath{clip}%
\pgfsetbuttcap%
\pgfsetroundjoin%
\definecolor{currentfill}{rgb}{0.494638,0.633022,0.978983}%
\pgfsetfillcolor{currentfill}%
\pgfsetlinewidth{0.000000pt}%
\definecolor{currentstroke}{rgb}{0.000000,0.000000,0.000000}%
\pgfsetstrokecolor{currentstroke}%
\pgfsetdash{}{0pt}%
\pgfpathmoveto{\pgfqpoint{5.245401in}{3.811314in}}%
\pgfpathlineto{\pgfqpoint{5.254788in}{3.751648in}}%
\pgfpathlineto{\pgfqpoint{5.264950in}{3.763137in}}%
\pgfpathlineto{\pgfqpoint{5.298289in}{3.904707in}}%
\pgfpathlineto{\pgfqpoint{5.328953in}{3.805571in}}%
\pgfpathlineto{\pgfqpoint{5.318755in}{3.797116in}}%
\pgfpathlineto{\pgfqpoint{5.309368in}{3.859098in}}%
\pgfpathlineto{\pgfqpoint{5.277439in}{3.840106in}}%
\pgfpathlineto{\pgfqpoint{5.245401in}{3.811314in}}%
\pgfpathclose%
\pgfusepath{fill}%
\end{pgfscope}%
\begin{pgfscope}%
\pgfpathrectangle{\pgfqpoint{1.020000in}{0.880000in}}{\pgfqpoint{6.160000in}{6.160000in}}%
\pgfusepath{clip}%
\pgfsetbuttcap%
\pgfsetroundjoin%
\definecolor{currentfill}{rgb}{0.516260,0.654498,0.986407}%
\pgfsetfillcolor{currentfill}%
\pgfsetlinewidth{0.000000pt}%
\definecolor{currentstroke}{rgb}{0.000000,0.000000,0.000000}%
\pgfsetstrokecolor{currentstroke}%
\pgfsetdash{}{0pt}%
\pgfpathmoveto{\pgfqpoint{5.035069in}{3.814654in}}%
\pgfpathlineto{\pgfqpoint{5.045472in}{3.884135in}}%
\pgfpathlineto{\pgfqpoint{5.055099in}{3.860238in}}%
\pgfpathlineto{\pgfqpoint{5.087568in}{3.929239in}}%
\pgfpathlineto{\pgfqpoint{5.118190in}{3.797153in}}%
\pgfpathlineto{\pgfqpoint{5.108418in}{3.809388in}}%
\pgfpathlineto{\pgfqpoint{5.099068in}{3.865906in}}%
\pgfpathlineto{\pgfqpoint{5.067119in}{3.845904in}}%
\pgfpathlineto{\pgfqpoint{5.035069in}{3.814654in}}%
\pgfpathclose%
\pgfusepath{fill}%
\end{pgfscope}%
\begin{pgfscope}%
\pgfpathrectangle{\pgfqpoint{1.020000in}{0.880000in}}{\pgfqpoint{6.160000in}{6.160000in}}%
\pgfusepath{clip}%
\pgfsetbuttcap%
\pgfsetroundjoin%
\definecolor{currentfill}{rgb}{0.630089,0.752516,0.998508}%
\pgfsetfillcolor{currentfill}%
\pgfsetlinewidth{0.000000pt}%
\definecolor{currentstroke}{rgb}{0.000000,0.000000,0.000000}%
\pgfsetstrokecolor{currentstroke}%
\pgfsetdash{}{0pt}%
\pgfpathmoveto{\pgfqpoint{4.259141in}{4.114739in}}%
\pgfpathlineto{\pgfqpoint{4.268108in}{4.043440in}}%
\pgfpathlineto{\pgfqpoint{4.277047in}{3.935279in}}%
\pgfpathlineto{\pgfqpoint{4.309175in}{4.011247in}}%
\pgfpathlineto{\pgfqpoint{4.341189in}{3.995764in}}%
\pgfpathlineto{\pgfqpoint{4.332312in}{4.142967in}}%
\pgfpathlineto{\pgfqpoint{4.323031in}{4.006382in}}%
\pgfpathlineto{\pgfqpoint{4.291162in}{4.107287in}}%
\pgfpathlineto{\pgfqpoint{4.259141in}{4.114739in}}%
\pgfpathclose%
\pgfusepath{fill}%
\end{pgfscope}%
\begin{pgfscope}%
\pgfpathrectangle{\pgfqpoint{1.020000in}{0.880000in}}{\pgfqpoint{6.160000in}{6.160000in}}%
\pgfusepath{clip}%
\pgfsetbuttcap%
\pgfsetroundjoin%
\definecolor{currentfill}{rgb}{0.473070,0.611077,0.970634}%
\pgfsetfillcolor{currentfill}%
\pgfsetlinewidth{0.000000pt}%
\definecolor{currentstroke}{rgb}{0.000000,0.000000,0.000000}%
\pgfsetstrokecolor{currentstroke}%
\pgfsetdash{}{0pt}%
\pgfpathmoveto{\pgfqpoint{5.455599in}{3.787525in}}%
\pgfpathlineto{\pgfqpoint{5.465202in}{3.737744in}}%
\pgfpathlineto{\pgfqpoint{5.474642in}{3.674485in}}%
\pgfpathlineto{\pgfqpoint{5.507895in}{3.792859in}}%
\pgfpathlineto{\pgfqpoint{5.497684in}{3.798836in}}%
\pgfpathlineto{\pgfqpoint{5.488417in}{3.875941in}}%
\pgfpathlineto{\pgfqpoint{5.455599in}{3.787525in}}%
\pgfpathclose%
\pgfusepath{fill}%
\end{pgfscope}%
\begin{pgfscope}%
\pgfpathrectangle{\pgfqpoint{1.020000in}{0.880000in}}{\pgfqpoint{6.160000in}{6.160000in}}%
\pgfusepath{clip}%
\pgfsetbuttcap%
\pgfsetroundjoin%
\definecolor{currentfill}{rgb}{0.651398,0.768121,0.995891}%
\pgfsetfillcolor{currentfill}%
\pgfsetlinewidth{0.000000pt}%
\definecolor{currentstroke}{rgb}{0.000000,0.000000,0.000000}%
\pgfsetstrokecolor{currentstroke}%
\pgfsetdash{}{0pt}%
\pgfpathmoveto{\pgfqpoint{4.048854in}{4.179352in}}%
\pgfpathlineto{\pgfqpoint{4.057610in}{4.227980in}}%
\pgfpathlineto{\pgfqpoint{4.066504in}{4.181754in}}%
\pgfpathlineto{\pgfqpoint{4.098905in}{3.894320in}}%
\pgfpathlineto{\pgfqpoint{4.130796in}{4.199592in}}%
\pgfpathlineto{\pgfqpoint{4.122025in}{3.992145in}}%
\pgfpathlineto{\pgfqpoint{4.113185in}{3.961894in}}%
\pgfpathlineto{\pgfqpoint{4.081104in}{4.029995in}}%
\pgfpathlineto{\pgfqpoint{4.048854in}{4.179352in}}%
\pgfpathclose%
\pgfusepath{fill}%
\end{pgfscope}%
\begin{pgfscope}%
\pgfpathrectangle{\pgfqpoint{1.020000in}{0.880000in}}{\pgfqpoint{6.160000in}{6.160000in}}%
\pgfusepath{clip}%
\pgfsetbuttcap%
\pgfsetroundjoin%
\definecolor{currentfill}{rgb}{0.581486,0.713451,0.998314}%
\pgfsetfillcolor{currentfill}%
\pgfsetlinewidth{0.000000pt}%
\definecolor{currentstroke}{rgb}{0.000000,0.000000,0.000000}%
\pgfsetstrokecolor{currentstroke}%
\pgfsetdash{}{0pt}%
\pgfpathmoveto{\pgfqpoint{4.405254in}{4.019552in}}%
\pgfpathlineto{\pgfqpoint{4.414055in}{3.856353in}}%
\pgfpathlineto{\pgfqpoint{4.423273in}{3.872300in}}%
\pgfpathlineto{\pgfqpoint{4.455803in}{4.067972in}}%
\pgfpathlineto{\pgfqpoint{4.487160in}{3.841093in}}%
\pgfpathlineto{\pgfqpoint{4.478595in}{4.070559in}}%
\pgfpathlineto{\pgfqpoint{4.469158in}{4.001147in}}%
\pgfpathlineto{\pgfqpoint{4.437135in}{3.977606in}}%
\pgfpathlineto{\pgfqpoint{4.405254in}{4.019552in}}%
\pgfpathclose%
\pgfusepath{fill}%
\end{pgfscope}%
\begin{pgfscope}%
\pgfpathrectangle{\pgfqpoint{1.020000in}{0.880000in}}{\pgfqpoint{6.160000in}{6.160000in}}%
\pgfusepath{clip}%
\pgfsetbuttcap%
\pgfsetroundjoin%
\definecolor{currentfill}{rgb}{0.733898,0.820018,0.970724}%
\pgfsetfillcolor{currentfill}%
\pgfsetlinewidth{0.000000pt}%
\definecolor{currentstroke}{rgb}{0.000000,0.000000,0.000000}%
\pgfsetstrokecolor{currentstroke}%
\pgfsetdash{}{0pt}%
\pgfpathmoveto{\pgfqpoint{3.691820in}{4.416433in}}%
\pgfpathlineto{\pgfqpoint{3.700730in}{4.323658in}}%
\pgfpathlineto{\pgfqpoint{3.709448in}{4.272200in}}%
\pgfpathlineto{\pgfqpoint{3.742459in}{4.098290in}}%
\pgfpathlineto{\pgfqpoint{3.774842in}{4.047549in}}%
\pgfpathlineto{\pgfqpoint{3.766029in}{4.119860in}}%
\pgfpathlineto{\pgfqpoint{3.756660in}{4.325118in}}%
\pgfpathlineto{\pgfqpoint{3.724868in}{4.237937in}}%
\pgfpathlineto{\pgfqpoint{3.691820in}{4.416433in}}%
\pgfpathclose%
\pgfusepath{fill}%
\end{pgfscope}%
\begin{pgfscope}%
\pgfpathrectangle{\pgfqpoint{1.020000in}{0.880000in}}{\pgfqpoint{6.160000in}{6.160000in}}%
\pgfusepath{clip}%
\pgfsetbuttcap%
\pgfsetroundjoin%
\definecolor{currentfill}{rgb}{0.576051,0.708780,0.997755}%
\pgfsetfillcolor{currentfill}%
\pgfsetlinewidth{0.000000pt}%
\definecolor{currentstroke}{rgb}{0.000000,0.000000,0.000000}%
\pgfsetstrokecolor{currentstroke}%
\pgfsetdash{}{0pt}%
\pgfpathmoveto{\pgfqpoint{4.615438in}{3.974183in}}%
\pgfpathlineto{\pgfqpoint{4.625147in}{4.047271in}}%
\pgfpathlineto{\pgfqpoint{4.634113in}{3.947337in}}%
\pgfpathlineto{\pgfqpoint{4.665625in}{3.850943in}}%
\pgfpathlineto{\pgfqpoint{4.697998in}{3.938336in}}%
\pgfpathlineto{\pgfqpoint{4.688975in}{4.025950in}}%
\pgfpathlineto{\pgfqpoint{4.679579in}{4.037566in}}%
\pgfpathlineto{\pgfqpoint{4.646449in}{3.776663in}}%
\pgfpathlineto{\pgfqpoint{4.615438in}{3.974183in}}%
\pgfpathclose%
\pgfusepath{fill}%
\end{pgfscope}%
\begin{pgfscope}%
\pgfpathrectangle{\pgfqpoint{1.020000in}{0.880000in}}{\pgfqpoint{6.160000in}{6.160000in}}%
\pgfusepath{clip}%
\pgfsetbuttcap%
\pgfsetroundjoin%
\definecolor{currentfill}{rgb}{0.467678,0.605591,0.968546}%
\pgfsetfillcolor{currentfill}%
\pgfsetlinewidth{0.000000pt}%
\definecolor{currentstroke}{rgb}{0.000000,0.000000,0.000000}%
\pgfsetstrokecolor{currentstroke}%
\pgfsetdash{}{0pt}%
\pgfpathmoveto{\pgfqpoint{5.392281in}{3.793526in}}%
\pgfpathlineto{\pgfqpoint{5.402341in}{3.783531in}}%
\pgfpathlineto{\pgfqpoint{5.412719in}{3.797999in}}%
\pgfpathlineto{\pgfqpoint{5.441553in}{3.564677in}}%
\pgfpathlineto{\pgfqpoint{5.474642in}{3.674485in}}%
\pgfpathlineto{\pgfqpoint{5.465202in}{3.737744in}}%
\pgfpathlineto{\pgfqpoint{5.455599in}{3.787525in}}%
\pgfpathlineto{\pgfqpoint{5.426204in}{3.972837in}}%
\pgfpathlineto{\pgfqpoint{5.392281in}{3.793526in}}%
\pgfpathclose%
\pgfusepath{fill}%
\end{pgfscope}%
\begin{pgfscope}%
\pgfpathrectangle{\pgfqpoint{1.020000in}{0.880000in}}{\pgfqpoint{6.160000in}{6.160000in}}%
\pgfusepath{clip}%
\pgfsetbuttcap%
\pgfsetroundjoin%
\definecolor{currentfill}{rgb}{0.510824,0.649397,0.985079}%
\pgfsetfillcolor{currentfill}%
\pgfsetlinewidth{0.000000pt}%
\definecolor{currentstroke}{rgb}{0.000000,0.000000,0.000000}%
\pgfsetstrokecolor{currentstroke}%
\pgfsetdash{}{0pt}%
\pgfpathmoveto{\pgfqpoint{4.972054in}{3.889729in}}%
\pgfpathlineto{\pgfqpoint{4.981299in}{3.825457in}}%
\pgfpathlineto{\pgfqpoint{4.989536in}{3.635089in}}%
\pgfpathlineto{\pgfqpoint{5.023061in}{3.839753in}}%
\pgfpathlineto{\pgfqpoint{5.055099in}{3.860238in}}%
\pgfpathlineto{\pgfqpoint{5.045472in}{3.884135in}}%
\pgfpathlineto{\pgfqpoint{5.035069in}{3.814654in}}%
\pgfpathlineto{\pgfqpoint{5.004194in}{3.927047in}}%
\pgfpathlineto{\pgfqpoint{4.972054in}{3.889729in}}%
\pgfpathclose%
\pgfusepath{fill}%
\end{pgfscope}%
\begin{pgfscope}%
\pgfpathrectangle{\pgfqpoint{1.020000in}{0.880000in}}{\pgfqpoint{6.160000in}{6.160000in}}%
\pgfusepath{clip}%
\pgfsetbuttcap%
\pgfsetroundjoin%
\definecolor{currentfill}{rgb}{0.532568,0.669801,0.990393}%
\pgfsetfillcolor{currentfill}%
\pgfsetlinewidth{0.000000pt}%
\definecolor{currentstroke}{rgb}{0.000000,0.000000,0.000000}%
\pgfsetstrokecolor{currentstroke}%
\pgfsetdash{}{0pt}%
\pgfpathmoveto{\pgfqpoint{4.761966in}{3.958413in}}%
\pgfpathlineto{\pgfqpoint{4.770898in}{3.853754in}}%
\pgfpathlineto{\pgfqpoint{4.780074in}{3.790051in}}%
\pgfpathlineto{\pgfqpoint{4.811956in}{3.785398in}}%
\pgfpathlineto{\pgfqpoint{4.845346in}{4.013033in}}%
\pgfpathlineto{\pgfqpoint{4.835143in}{3.924394in}}%
\pgfpathlineto{\pgfqpoint{4.825494in}{3.916901in}}%
\pgfpathlineto{\pgfqpoint{4.792666in}{3.759862in}}%
\pgfpathlineto{\pgfqpoint{4.761966in}{3.958413in}}%
\pgfpathclose%
\pgfusepath{fill}%
\end{pgfscope}%
\begin{pgfscope}%
\pgfpathrectangle{\pgfqpoint{1.020000in}{0.880000in}}{\pgfqpoint{6.160000in}{6.160000in}}%
\pgfusepath{clip}%
\pgfsetbuttcap%
\pgfsetroundjoin%
\definecolor{currentfill}{rgb}{0.500031,0.638508,0.981070}%
\pgfsetfillcolor{currentfill}%
\pgfsetlinewidth{0.000000pt}%
\definecolor{currentstroke}{rgb}{0.000000,0.000000,0.000000}%
\pgfsetstrokecolor{currentstroke}%
\pgfsetdash{}{0pt}%
\pgfpathmoveto{\pgfqpoint{5.181236in}{3.744793in}}%
\pgfpathlineto{\pgfqpoint{5.192528in}{3.877523in}}%
\pgfpathlineto{\pgfqpoint{5.201409in}{3.768043in}}%
\pgfpathlineto{\pgfqpoint{5.234116in}{3.854210in}}%
\pgfpathlineto{\pgfqpoint{5.264950in}{3.763137in}}%
\pgfpathlineto{\pgfqpoint{5.254788in}{3.751648in}}%
\pgfpathlineto{\pgfqpoint{5.245401in}{3.811314in}}%
\pgfpathlineto{\pgfqpoint{5.215073in}{3.949169in}}%
\pgfpathlineto{\pgfqpoint{5.181236in}{3.744793in}}%
\pgfpathclose%
\pgfusepath{fill}%
\end{pgfscope}%
\begin{pgfscope}%
\pgfpathrectangle{\pgfqpoint{1.020000in}{0.880000in}}{\pgfqpoint{6.160000in}{6.160000in}}%
\pgfusepath{clip}%
\pgfsetbuttcap%
\pgfsetroundjoin%
\definecolor{currentfill}{rgb}{0.467678,0.605591,0.968546}%
\pgfsetfillcolor{currentfill}%
\pgfsetlinewidth{0.000000pt}%
\definecolor{currentstroke}{rgb}{0.000000,0.000000,0.000000}%
\pgfsetstrokecolor{currentstroke}%
\pgfsetdash{}{0pt}%
\pgfpathmoveto{\pgfqpoint{5.118190in}{3.797153in}}%
\pgfpathlineto{\pgfqpoint{5.128045in}{3.791914in}}%
\pgfpathlineto{\pgfqpoint{5.138109in}{3.806634in}}%
\pgfpathlineto{\pgfqpoint{5.168683in}{3.676386in}}%
\pgfpathlineto{\pgfqpoint{5.201409in}{3.768043in}}%
\pgfpathlineto{\pgfqpoint{5.192528in}{3.877523in}}%
\pgfpathlineto{\pgfqpoint{5.181236in}{3.744793in}}%
\pgfpathlineto{\pgfqpoint{5.148627in}{3.656514in}}%
\pgfpathlineto{\pgfqpoint{5.118190in}{3.797153in}}%
\pgfpathclose%
\pgfusepath{fill}%
\end{pgfscope}%
\begin{pgfscope}%
\pgfpathrectangle{\pgfqpoint{1.020000in}{0.880000in}}{\pgfqpoint{6.160000in}{6.160000in}}%
\pgfusepath{clip}%
\pgfsetbuttcap%
\pgfsetroundjoin%
\definecolor{currentfill}{rgb}{0.565182,0.699438,0.996635}%
\pgfsetfillcolor{currentfill}%
\pgfsetlinewidth{0.000000pt}%
\definecolor{currentstroke}{rgb}{0.000000,0.000000,0.000000}%
\pgfsetstrokecolor{currentstroke}%
\pgfsetdash{}{0pt}%
\pgfpathmoveto{\pgfqpoint{4.551227in}{3.889835in}}%
\pgfpathlineto{\pgfqpoint{4.559941in}{3.728083in}}%
\pgfpathlineto{\pgfqpoint{4.570382in}{4.018058in}}%
\pgfpathlineto{\pgfqpoint{4.602059in}{3.932600in}}%
\pgfpathlineto{\pgfqpoint{4.634113in}{3.947337in}}%
\pgfpathlineto{\pgfqpoint{4.625147in}{4.047271in}}%
\pgfpathlineto{\pgfqpoint{4.615438in}{3.974183in}}%
\pgfpathlineto{\pgfqpoint{4.583235in}{3.908619in}}%
\pgfpathlineto{\pgfqpoint{4.551227in}{3.889835in}}%
\pgfpathclose%
\pgfusepath{fill}%
\end{pgfscope}%
\begin{pgfscope}%
\pgfpathrectangle{\pgfqpoint{1.020000in}{0.880000in}}{\pgfqpoint{6.160000in}{6.160000in}}%
\pgfusepath{clip}%
\pgfsetbuttcap%
\pgfsetroundjoin%
\definecolor{currentfill}{rgb}{0.586921,0.718121,0.998874}%
\pgfsetfillcolor{currentfill}%
\pgfsetlinewidth{0.000000pt}%
\definecolor{currentstroke}{rgb}{0.000000,0.000000,0.000000}%
\pgfsetstrokecolor{currentstroke}%
\pgfsetdash{}{0pt}%
\pgfpathmoveto{\pgfqpoint{4.341189in}{3.995764in}}%
\pgfpathlineto{\pgfqpoint{4.350263in}{3.971621in}}%
\pgfpathlineto{\pgfqpoint{4.359313in}{3.925809in}}%
\pgfpathlineto{\pgfqpoint{4.391466in}{3.973856in}}%
\pgfpathlineto{\pgfqpoint{4.423273in}{3.872300in}}%
\pgfpathlineto{\pgfqpoint{4.414055in}{3.856353in}}%
\pgfpathlineto{\pgfqpoint{4.405254in}{4.019552in}}%
\pgfpathlineto{\pgfqpoint{4.373411in}{4.105242in}}%
\pgfpathlineto{\pgfqpoint{4.341189in}{3.995764in}}%
\pgfpathclose%
\pgfusepath{fill}%
\end{pgfscope}%
\begin{pgfscope}%
\pgfpathrectangle{\pgfqpoint{1.020000in}{0.880000in}}{\pgfqpoint{6.160000in}{6.160000in}}%
\pgfusepath{clip}%
\pgfsetbuttcap%
\pgfsetroundjoin%
\definecolor{currentfill}{rgb}{0.462354,0.599830,0.965857}%
\pgfsetfillcolor{currentfill}%
\pgfsetlinewidth{0.000000pt}%
\definecolor{currentstroke}{rgb}{0.000000,0.000000,0.000000}%
\pgfsetstrokecolor{currentstroke}%
\pgfsetdash{}{0pt}%
\pgfpathmoveto{\pgfqpoint{5.328953in}{3.805571in}}%
\pgfpathlineto{\pgfqpoint{5.338289in}{3.737170in}}%
\pgfpathlineto{\pgfqpoint{5.347305in}{3.640678in}}%
\pgfpathlineto{\pgfqpoint{5.380576in}{3.767939in}}%
\pgfpathlineto{\pgfqpoint{5.412719in}{3.797999in}}%
\pgfpathlineto{\pgfqpoint{5.402341in}{3.783531in}}%
\pgfpathlineto{\pgfqpoint{5.392281in}{3.793526in}}%
\pgfpathlineto{\pgfqpoint{5.359178in}{3.675894in}}%
\pgfpathlineto{\pgfqpoint{5.328953in}{3.805571in}}%
\pgfpathclose%
\pgfusepath{fill}%
\end{pgfscope}%
\begin{pgfscope}%
\pgfpathrectangle{\pgfqpoint{1.020000in}{0.880000in}}{\pgfqpoint{6.160000in}{6.160000in}}%
\pgfusepath{clip}%
\pgfsetbuttcap%
\pgfsetroundjoin%
\definecolor{currentfill}{rgb}{0.718985,0.811993,0.977656}%
\pgfsetfillcolor{currentfill}%
\pgfsetlinewidth{0.000000pt}%
\definecolor{currentstroke}{rgb}{0.000000,0.000000,0.000000}%
\pgfsetstrokecolor{currentstroke}%
\pgfsetdash{}{0pt}%
\pgfpathmoveto{\pgfqpoint{3.838550in}{4.210522in}}%
\pgfpathlineto{\pgfqpoint{3.847470in}{4.115882in}}%
\pgfpathlineto{\pgfqpoint{3.855942in}{4.159353in}}%
\pgfpathlineto{\pgfqpoint{3.887383in}{4.419141in}}%
\pgfpathlineto{\pgfqpoint{3.920277in}{4.178711in}}%
\pgfpathlineto{\pgfqpoint{3.911697in}{4.142083in}}%
\pgfpathlineto{\pgfqpoint{3.903361in}{4.026169in}}%
\pgfpathlineto{\pgfqpoint{3.870463in}{4.289459in}}%
\pgfpathlineto{\pgfqpoint{3.838550in}{4.210522in}}%
\pgfpathclose%
\pgfusepath{fill}%
\end{pgfscope}%
\begin{pgfscope}%
\pgfpathrectangle{\pgfqpoint{1.020000in}{0.880000in}}{\pgfqpoint{6.160000in}{6.160000in}}%
\pgfusepath{clip}%
\pgfsetbuttcap%
\pgfsetroundjoin%
\definecolor{currentfill}{rgb}{0.505423,0.643995,0.983157}%
\pgfsetfillcolor{currentfill}%
\pgfsetlinewidth{0.000000pt}%
\definecolor{currentstroke}{rgb}{0.000000,0.000000,0.000000}%
\pgfsetstrokecolor{currentstroke}%
\pgfsetdash{}{0pt}%
\pgfpathmoveto{\pgfqpoint{4.907458in}{3.768785in}}%
\pgfpathlineto{\pgfqpoint{4.918522in}{3.957398in}}%
\pgfpathlineto{\pgfqpoint{4.927325in}{3.833384in}}%
\pgfpathlineto{\pgfqpoint{4.958406in}{3.725862in}}%
\pgfpathlineto{\pgfqpoint{4.989536in}{3.635089in}}%
\pgfpathlineto{\pgfqpoint{4.981299in}{3.825457in}}%
\pgfpathlineto{\pgfqpoint{4.972054in}{3.889729in}}%
\pgfpathlineto{\pgfqpoint{4.940629in}{3.946417in}}%
\pgfpathlineto{\pgfqpoint{4.907458in}{3.768785in}}%
\pgfpathclose%
\pgfusepath{fill}%
\end{pgfscope}%
\begin{pgfscope}%
\pgfpathrectangle{\pgfqpoint{1.020000in}{0.880000in}}{\pgfqpoint{6.160000in}{6.160000in}}%
\pgfusepath{clip}%
\pgfsetbuttcap%
\pgfsetroundjoin%
\definecolor{currentfill}{rgb}{0.672538,0.782861,0.991982}%
\pgfsetfillcolor{currentfill}%
\pgfsetlinewidth{0.000000pt}%
\definecolor{currentstroke}{rgb}{0.000000,0.000000,0.000000}%
\pgfsetstrokecolor{currentstroke}%
\pgfsetdash{}{0pt}%
\pgfpathmoveto{\pgfqpoint{3.984846in}{4.054919in}}%
\pgfpathlineto{\pgfqpoint{3.993502in}{4.103281in}}%
\pgfpathlineto{\pgfqpoint{4.002405in}{4.028112in}}%
\pgfpathlineto{\pgfqpoint{4.034483in}{4.075423in}}%
\pgfpathlineto{\pgfqpoint{4.066504in}{4.181754in}}%
\pgfpathlineto{\pgfqpoint{4.057610in}{4.227980in}}%
\pgfpathlineto{\pgfqpoint{4.048854in}{4.179352in}}%
\pgfpathlineto{\pgfqpoint{4.016928in}{4.063467in}}%
\pgfpathlineto{\pgfqpoint{3.984846in}{4.054919in}}%
\pgfpathclose%
\pgfusepath{fill}%
\end{pgfscope}%
\begin{pgfscope}%
\pgfpathrectangle{\pgfqpoint{1.020000in}{0.880000in}}{\pgfqpoint{6.160000in}{6.160000in}}%
\pgfusepath{clip}%
\pgfsetbuttcap%
\pgfsetroundjoin%
\definecolor{currentfill}{rgb}{0.656683,0.771806,0.994914}%
\pgfsetfillcolor{currentfill}%
\pgfsetlinewidth{0.000000pt}%
\definecolor{currentstroke}{rgb}{0.000000,0.000000,0.000000}%
\pgfsetstrokecolor{currentstroke}%
\pgfsetdash{}{0pt}%
\pgfpathmoveto{\pgfqpoint{4.195030in}{4.144520in}}%
\pgfpathlineto{\pgfqpoint{4.203997in}{4.125052in}}%
\pgfpathlineto{\pgfqpoint{4.212963in}{4.054895in}}%
\pgfpathlineto{\pgfqpoint{4.245121in}{4.136546in}}%
\pgfpathlineto{\pgfqpoint{4.277047in}{3.935279in}}%
\pgfpathlineto{\pgfqpoint{4.268108in}{4.043440in}}%
\pgfpathlineto{\pgfqpoint{4.259141in}{4.114739in}}%
\pgfpathlineto{\pgfqpoint{4.227090in}{4.108691in}}%
\pgfpathlineto{\pgfqpoint{4.195030in}{4.144520in}}%
\pgfpathclose%
\pgfusepath{fill}%
\end{pgfscope}%
\begin{pgfscope}%
\pgfpathrectangle{\pgfqpoint{1.020000in}{0.880000in}}{\pgfqpoint{6.160000in}{6.160000in}}%
\pgfusepath{clip}%
\pgfsetbuttcap%
\pgfsetroundjoin%
\definecolor{currentfill}{rgb}{0.565182,0.699438,0.996635}%
\pgfsetfillcolor{currentfill}%
\pgfsetlinewidth{0.000000pt}%
\definecolor{currentstroke}{rgb}{0.000000,0.000000,0.000000}%
\pgfsetstrokecolor{currentstroke}%
\pgfsetdash{}{0pt}%
\pgfpathmoveto{\pgfqpoint{4.487160in}{3.841093in}}%
\pgfpathlineto{\pgfqpoint{4.497060in}{4.047240in}}%
\pgfpathlineto{\pgfqpoint{4.505874in}{3.897640in}}%
\pgfpathlineto{\pgfqpoint{4.538473in}{4.058274in}}%
\pgfpathlineto{\pgfqpoint{4.570382in}{4.018058in}}%
\pgfpathlineto{\pgfqpoint{4.559941in}{3.728083in}}%
\pgfpathlineto{\pgfqpoint{4.551227in}{3.889835in}}%
\pgfpathlineto{\pgfqpoint{4.519213in}{3.870732in}}%
\pgfpathlineto{\pgfqpoint{4.487160in}{3.841093in}}%
\pgfpathclose%
\pgfusepath{fill}%
\end{pgfscope}%
\begin{pgfscope}%
\pgfpathrectangle{\pgfqpoint{1.020000in}{0.880000in}}{\pgfqpoint{6.160000in}{6.160000in}}%
\pgfusepath{clip}%
\pgfsetbuttcap%
\pgfsetroundjoin%
\definecolor{currentfill}{rgb}{0.763363,0.835092,0.955658}%
\pgfsetfillcolor{currentfill}%
\pgfsetlinewidth{0.000000pt}%
\definecolor{currentstroke}{rgb}{0.000000,0.000000,0.000000}%
\pgfsetstrokecolor{currentstroke}%
\pgfsetdash{}{0pt}%
\pgfpathmoveto{\pgfqpoint{3.628003in}{4.287632in}}%
\pgfpathlineto{\pgfqpoint{3.637247in}{4.131122in}}%
\pgfpathlineto{\pgfqpoint{3.645189in}{4.216546in}}%
\pgfpathlineto{\pgfqpoint{3.677489in}{4.210327in}}%
\pgfpathlineto{\pgfqpoint{3.709448in}{4.272200in}}%
\pgfpathlineto{\pgfqpoint{3.700730in}{4.323658in}}%
\pgfpathlineto{\pgfqpoint{3.691820in}{4.416433in}}%
\pgfpathlineto{\pgfqpoint{3.659930in}{4.346312in}}%
\pgfpathlineto{\pgfqpoint{3.628003in}{4.287632in}}%
\pgfpathclose%
\pgfusepath{fill}%
\end{pgfscope}%
\begin{pgfscope}%
\pgfpathrectangle{\pgfqpoint{1.020000in}{0.880000in}}{\pgfqpoint{6.160000in}{6.160000in}}%
\pgfusepath{clip}%
\pgfsetbuttcap%
\pgfsetroundjoin%
\definecolor{currentfill}{rgb}{0.559747,0.694768,0.996075}%
\pgfsetfillcolor{currentfill}%
\pgfsetlinewidth{0.000000pt}%
\definecolor{currentstroke}{rgb}{0.000000,0.000000,0.000000}%
\pgfsetstrokecolor{currentstroke}%
\pgfsetdash{}{0pt}%
\pgfpathmoveto{\pgfqpoint{4.697998in}{3.938336in}}%
\pgfpathlineto{\pgfqpoint{4.707927in}{4.023794in}}%
\pgfpathlineto{\pgfqpoint{4.717286in}{3.995242in}}%
\pgfpathlineto{\pgfqpoint{4.748346in}{3.825932in}}%
\pgfpathlineto{\pgfqpoint{4.780074in}{3.790051in}}%
\pgfpathlineto{\pgfqpoint{4.770898in}{3.853754in}}%
\pgfpathlineto{\pgfqpoint{4.761966in}{3.958413in}}%
\pgfpathlineto{\pgfqpoint{4.729592in}{3.875507in}}%
\pgfpathlineto{\pgfqpoint{4.697998in}{3.938336in}}%
\pgfpathclose%
\pgfusepath{fill}%
\end{pgfscope}%
\begin{pgfscope}%
\pgfpathrectangle{\pgfqpoint{1.020000in}{0.880000in}}{\pgfqpoint{6.160000in}{6.160000in}}%
\pgfusepath{clip}%
\pgfsetbuttcap%
\pgfsetroundjoin%
\definecolor{currentfill}{rgb}{0.718985,0.811993,0.977656}%
\pgfsetfillcolor{currentfill}%
\pgfsetlinewidth{0.000000pt}%
\definecolor{currentstroke}{rgb}{0.000000,0.000000,0.000000}%
\pgfsetstrokecolor{currentstroke}%
\pgfsetdash{}{0pt}%
\pgfpathmoveto{\pgfqpoint{3.774842in}{4.047549in}}%
\pgfpathlineto{\pgfqpoint{3.782356in}{4.303057in}}%
\pgfpathlineto{\pgfqpoint{3.791333in}{4.198291in}}%
\pgfpathlineto{\pgfqpoint{3.823639in}{4.182634in}}%
\pgfpathlineto{\pgfqpoint{3.855942in}{4.159353in}}%
\pgfpathlineto{\pgfqpoint{3.847470in}{4.115882in}}%
\pgfpathlineto{\pgfqpoint{3.838550in}{4.210522in}}%
\pgfpathlineto{\pgfqpoint{3.805957in}{4.322023in}}%
\pgfpathlineto{\pgfqpoint{3.774842in}{4.047549in}}%
\pgfpathclose%
\pgfusepath{fill}%
\end{pgfscope}%
\begin{pgfscope}%
\pgfpathrectangle{\pgfqpoint{1.020000in}{0.880000in}}{\pgfqpoint{6.160000in}{6.160000in}}%
\pgfusepath{clip}%
\pgfsetbuttcap%
\pgfsetroundjoin%
\definecolor{currentfill}{rgb}{0.478462,0.616564,0.972721}%
\pgfsetfillcolor{currentfill}%
\pgfsetlinewidth{0.000000pt}%
\definecolor{currentstroke}{rgb}{0.000000,0.000000,0.000000}%
\pgfsetstrokecolor{currentstroke}%
\pgfsetdash{}{0pt}%
\pgfpathmoveto{\pgfqpoint{5.264950in}{3.763137in}}%
\pgfpathlineto{\pgfqpoint{5.274894in}{3.752509in}}%
\pgfpathlineto{\pgfqpoint{5.284417in}{3.701808in}}%
\pgfpathlineto{\pgfqpoint{5.317651in}{3.828007in}}%
\pgfpathlineto{\pgfqpoint{5.347305in}{3.640678in}}%
\pgfpathlineto{\pgfqpoint{5.338289in}{3.737170in}}%
\pgfpathlineto{\pgfqpoint{5.328953in}{3.805571in}}%
\pgfpathlineto{\pgfqpoint{5.298289in}{3.904707in}}%
\pgfpathlineto{\pgfqpoint{5.264950in}{3.763137in}}%
\pgfpathclose%
\pgfusepath{fill}%
\end{pgfscope}%
\begin{pgfscope}%
\pgfpathrectangle{\pgfqpoint{1.020000in}{0.880000in}}{\pgfqpoint{6.160000in}{6.160000in}}%
\pgfusepath{clip}%
\pgfsetbuttcap%
\pgfsetroundjoin%
\definecolor{currentfill}{rgb}{0.608547,0.735725,0.999354}%
\pgfsetfillcolor{currentfill}%
\pgfsetlinewidth{0.000000pt}%
\definecolor{currentstroke}{rgb}{0.000000,0.000000,0.000000}%
\pgfsetstrokecolor{currentstroke}%
\pgfsetdash{}{0pt}%
\pgfpathmoveto{\pgfqpoint{4.277047in}{3.935279in}}%
\pgfpathlineto{\pgfqpoint{4.286305in}{4.139846in}}%
\pgfpathlineto{\pgfqpoint{4.295164in}{3.936045in}}%
\pgfpathlineto{\pgfqpoint{4.327386in}{4.026577in}}%
\pgfpathlineto{\pgfqpoint{4.359313in}{3.925809in}}%
\pgfpathlineto{\pgfqpoint{4.350263in}{3.971621in}}%
\pgfpathlineto{\pgfqpoint{4.341189in}{3.995764in}}%
\pgfpathlineto{\pgfqpoint{4.309175in}{4.011247in}}%
\pgfpathlineto{\pgfqpoint{4.277047in}{3.935279in}}%
\pgfpathclose%
\pgfusepath{fill}%
\end{pgfscope}%
\begin{pgfscope}%
\pgfpathrectangle{\pgfqpoint{1.020000in}{0.880000in}}{\pgfqpoint{6.160000in}{6.160000in}}%
\pgfusepath{clip}%
\pgfsetbuttcap%
\pgfsetroundjoin%
\definecolor{currentfill}{rgb}{0.451739,0.588181,0.960201}%
\pgfsetfillcolor{currentfill}%
\pgfsetlinewidth{0.000000pt}%
\definecolor{currentstroke}{rgb}{0.000000,0.000000,0.000000}%
\pgfsetstrokecolor{currentstroke}%
\pgfsetdash{}{0pt}%
\pgfpathmoveto{\pgfqpoint{5.474642in}{3.674485in}}%
\pgfpathlineto{\pgfqpoint{5.485549in}{3.723608in}}%
\pgfpathlineto{\pgfqpoint{5.495091in}{3.666076in}}%
\pgfpathlineto{\pgfqpoint{5.529439in}{3.860066in}}%
\pgfpathlineto{\pgfqpoint{5.516003in}{3.627212in}}%
\pgfpathlineto{\pgfqpoint{5.507895in}{3.792859in}}%
\pgfpathlineto{\pgfqpoint{5.474642in}{3.674485in}}%
\pgfpathclose%
\pgfusepath{fill}%
\end{pgfscope}%
\begin{pgfscope}%
\pgfpathrectangle{\pgfqpoint{1.020000in}{0.880000in}}{\pgfqpoint{6.160000in}{6.160000in}}%
\pgfusepath{clip}%
\pgfsetbuttcap%
\pgfsetroundjoin%
\definecolor{currentfill}{rgb}{0.683056,0.790043,0.989768}%
\pgfsetfillcolor{currentfill}%
\pgfsetlinewidth{0.000000pt}%
\definecolor{currentstroke}{rgb}{0.000000,0.000000,0.000000}%
\pgfsetstrokecolor{currentstroke}%
\pgfsetdash{}{0pt}%
\pgfpathmoveto{\pgfqpoint{3.920277in}{4.178711in}}%
\pgfpathlineto{\pgfqpoint{3.929121in}{4.120598in}}%
\pgfpathlineto{\pgfqpoint{3.937795in}{4.135247in}}%
\pgfpathlineto{\pgfqpoint{3.969925in}{4.176614in}}%
\pgfpathlineto{\pgfqpoint{4.002405in}{4.028112in}}%
\pgfpathlineto{\pgfqpoint{3.993502in}{4.103281in}}%
\pgfpathlineto{\pgfqpoint{3.984846in}{4.054919in}}%
\pgfpathlineto{\pgfqpoint{3.952493in}{4.159581in}}%
\pgfpathlineto{\pgfqpoint{3.920277in}{4.178711in}}%
\pgfpathclose%
\pgfusepath{fill}%
\end{pgfscope}%
\begin{pgfscope}%
\pgfpathrectangle{\pgfqpoint{1.020000in}{0.880000in}}{\pgfqpoint{6.160000in}{6.160000in}}%
\pgfusepath{clip}%
\pgfsetbuttcap%
\pgfsetroundjoin%
\definecolor{currentfill}{rgb}{0.430507,0.564883,0.948889}%
\pgfsetfillcolor{currentfill}%
\pgfsetlinewidth{0.000000pt}%
\definecolor{currentstroke}{rgb}{0.000000,0.000000,0.000000}%
\pgfsetstrokecolor{currentstroke}%
\pgfsetdash{}{0pt}%
\pgfpathmoveto{\pgfqpoint{5.412719in}{3.797999in}}%
\pgfpathlineto{\pgfqpoint{5.420241in}{3.579395in}}%
\pgfpathlineto{\pgfqpoint{5.433379in}{3.813554in}}%
\pgfpathlineto{\pgfqpoint{5.463563in}{3.684638in}}%
\pgfpathlineto{\pgfqpoint{5.495091in}{3.666076in}}%
\pgfpathlineto{\pgfqpoint{5.485549in}{3.723608in}}%
\pgfpathlineto{\pgfqpoint{5.474642in}{3.674485in}}%
\pgfpathlineto{\pgfqpoint{5.441553in}{3.564677in}}%
\pgfpathlineto{\pgfqpoint{5.412719in}{3.797999in}}%
\pgfpathclose%
\pgfusepath{fill}%
\end{pgfscope}%
\begin{pgfscope}%
\pgfpathrectangle{\pgfqpoint{1.020000in}{0.880000in}}{\pgfqpoint{6.160000in}{6.160000in}}%
\pgfusepath{clip}%
\pgfsetbuttcap%
\pgfsetroundjoin%
\definecolor{currentfill}{rgb}{0.667253,0.779176,0.992959}%
\pgfsetfillcolor{currentfill}%
\pgfsetlinewidth{0.000000pt}%
\definecolor{currentstroke}{rgb}{0.000000,0.000000,0.000000}%
\pgfsetstrokecolor{currentstroke}%
\pgfsetdash{}{0pt}%
\pgfpathmoveto{\pgfqpoint{4.130796in}{4.199592in}}%
\pgfpathlineto{\pgfqpoint{4.139751in}{4.108488in}}%
\pgfpathlineto{\pgfqpoint{4.148679in}{4.102133in}}%
\pgfpathlineto{\pgfqpoint{4.180840in}{3.982741in}}%
\pgfpathlineto{\pgfqpoint{4.212963in}{4.054895in}}%
\pgfpathlineto{\pgfqpoint{4.203997in}{4.125052in}}%
\pgfpathlineto{\pgfqpoint{4.195030in}{4.144520in}}%
\pgfpathlineto{\pgfqpoint{4.162960in}{4.030249in}}%
\pgfpathlineto{\pgfqpoint{4.130796in}{4.199592in}}%
\pgfpathclose%
\pgfusepath{fill}%
\end{pgfscope}%
\begin{pgfscope}%
\pgfpathrectangle{\pgfqpoint{1.020000in}{0.880000in}}{\pgfqpoint{6.160000in}{6.160000in}}%
\pgfusepath{clip}%
\pgfsetbuttcap%
\pgfsetroundjoin%
\definecolor{currentfill}{rgb}{0.565182,0.699438,0.996635}%
\pgfsetfillcolor{currentfill}%
\pgfsetlinewidth{0.000000pt}%
\definecolor{currentstroke}{rgb}{0.000000,0.000000,0.000000}%
\pgfsetstrokecolor{currentstroke}%
\pgfsetdash{}{0pt}%
\pgfpathmoveto{\pgfqpoint{4.423273in}{3.872300in}}%
\pgfpathlineto{\pgfqpoint{4.432240in}{3.778428in}}%
\pgfpathlineto{\pgfqpoint{4.441837in}{3.928796in}}%
\pgfpathlineto{\pgfqpoint{4.473742in}{3.868970in}}%
\pgfpathlineto{\pgfqpoint{4.505874in}{3.897640in}}%
\pgfpathlineto{\pgfqpoint{4.497060in}{4.047240in}}%
\pgfpathlineto{\pgfqpoint{4.487160in}{3.841093in}}%
\pgfpathlineto{\pgfqpoint{4.455803in}{4.067972in}}%
\pgfpathlineto{\pgfqpoint{4.423273in}{3.872300in}}%
\pgfpathclose%
\pgfusepath{fill}%
\end{pgfscope}%
\begin{pgfscope}%
\pgfpathrectangle{\pgfqpoint{1.020000in}{0.880000in}}{\pgfqpoint{6.160000in}{6.160000in}}%
\pgfusepath{clip}%
\pgfsetbuttcap%
\pgfsetroundjoin%
\definecolor{currentfill}{rgb}{0.510824,0.649397,0.985079}%
\pgfsetfillcolor{currentfill}%
\pgfsetlinewidth{0.000000pt}%
\definecolor{currentstroke}{rgb}{0.000000,0.000000,0.000000}%
\pgfsetstrokecolor{currentstroke}%
\pgfsetdash{}{0pt}%
\pgfpathmoveto{\pgfqpoint{5.055099in}{3.860238in}}%
\pgfpathlineto{\pgfqpoint{5.064451in}{3.803381in}}%
\pgfpathlineto{\pgfqpoint{5.073089in}{3.665065in}}%
\pgfpathlineto{\pgfqpoint{5.107184in}{3.909666in}}%
\pgfpathlineto{\pgfqpoint{5.138109in}{3.806634in}}%
\pgfpathlineto{\pgfqpoint{5.128045in}{3.791914in}}%
\pgfpathlineto{\pgfqpoint{5.118190in}{3.797153in}}%
\pgfpathlineto{\pgfqpoint{5.087568in}{3.929239in}}%
\pgfpathlineto{\pgfqpoint{5.055099in}{3.860238in}}%
\pgfpathclose%
\pgfusepath{fill}%
\end{pgfscope}%
\begin{pgfscope}%
\pgfpathrectangle{\pgfqpoint{1.020000in}{0.880000in}}{\pgfqpoint{6.160000in}{6.160000in}}%
\pgfusepath{clip}%
\pgfsetbuttcap%
\pgfsetroundjoin%
\definecolor{currentfill}{rgb}{0.532568,0.669801,0.990393}%
\pgfsetfillcolor{currentfill}%
\pgfsetlinewidth{0.000000pt}%
\definecolor{currentstroke}{rgb}{0.000000,0.000000,0.000000}%
\pgfsetstrokecolor{currentstroke}%
\pgfsetdash{}{0pt}%
\pgfpathmoveto{\pgfqpoint{4.845346in}{4.013033in}}%
\pgfpathlineto{\pgfqpoint{4.853266in}{3.754215in}}%
\pgfpathlineto{\pgfqpoint{4.863409in}{3.827979in}}%
\pgfpathlineto{\pgfqpoint{4.895549in}{3.855229in}}%
\pgfpathlineto{\pgfqpoint{4.927325in}{3.833384in}}%
\pgfpathlineto{\pgfqpoint{4.918522in}{3.957398in}}%
\pgfpathlineto{\pgfqpoint{4.907458in}{3.768785in}}%
\pgfpathlineto{\pgfqpoint{4.876260in}{3.863345in}}%
\pgfpathlineto{\pgfqpoint{4.845346in}{4.013033in}}%
\pgfpathclose%
\pgfusepath{fill}%
\end{pgfscope}%
\begin{pgfscope}%
\pgfpathrectangle{\pgfqpoint{1.020000in}{0.880000in}}{\pgfqpoint{6.160000in}{6.160000in}}%
\pgfusepath{clip}%
\pgfsetbuttcap%
\pgfsetroundjoin%
\definecolor{currentfill}{rgb}{0.713852,0.808857,0.979386}%
\pgfsetfillcolor{currentfill}%
\pgfsetlinewidth{0.000000pt}%
\definecolor{currentstroke}{rgb}{0.000000,0.000000,0.000000}%
\pgfsetstrokecolor{currentstroke}%
\pgfsetdash{}{0pt}%
\pgfpathmoveto{\pgfqpoint{3.709448in}{4.272200in}}%
\pgfpathlineto{\pgfqpoint{3.717751in}{4.312196in}}%
\pgfpathlineto{\pgfqpoint{3.727459in}{4.047561in}}%
\pgfpathlineto{\pgfqpoint{3.759299in}{4.142433in}}%
\pgfpathlineto{\pgfqpoint{3.791333in}{4.198291in}}%
\pgfpathlineto{\pgfqpoint{3.782356in}{4.303057in}}%
\pgfpathlineto{\pgfqpoint{3.774842in}{4.047549in}}%
\pgfpathlineto{\pgfqpoint{3.742459in}{4.098290in}}%
\pgfpathlineto{\pgfqpoint{3.709448in}{4.272200in}}%
\pgfpathclose%
\pgfusepath{fill}%
\end{pgfscope}%
\begin{pgfscope}%
\pgfpathrectangle{\pgfqpoint{1.020000in}{0.880000in}}{\pgfqpoint{6.160000in}{6.160000in}}%
\pgfusepath{clip}%
\pgfsetbuttcap%
\pgfsetroundjoin%
\definecolor{currentfill}{rgb}{0.489246,0.627536,0.976896}%
\pgfsetfillcolor{currentfill}%
\pgfsetlinewidth{0.000000pt}%
\definecolor{currentstroke}{rgb}{0.000000,0.000000,0.000000}%
\pgfsetstrokecolor{currentstroke}%
\pgfsetdash{}{0pt}%
\pgfpathmoveto{\pgfqpoint{4.989536in}{3.635089in}}%
\pgfpathlineto{\pgfqpoint{5.001441in}{3.897279in}}%
\pgfpathlineto{\pgfqpoint{5.010697in}{3.829367in}}%
\pgfpathlineto{\pgfqpoint{5.041598in}{3.708664in}}%
\pgfpathlineto{\pgfqpoint{5.073089in}{3.665065in}}%
\pgfpathlineto{\pgfqpoint{5.064451in}{3.803381in}}%
\pgfpathlineto{\pgfqpoint{5.055099in}{3.860238in}}%
\pgfpathlineto{\pgfqpoint{5.023061in}{3.839753in}}%
\pgfpathlineto{\pgfqpoint{4.989536in}{3.635089in}}%
\pgfpathclose%
\pgfusepath{fill}%
\end{pgfscope}%
\begin{pgfscope}%
\pgfpathrectangle{\pgfqpoint{1.020000in}{0.880000in}}{\pgfqpoint{6.160000in}{6.160000in}}%
\pgfusepath{clip}%
\pgfsetbuttcap%
\pgfsetroundjoin%
\definecolor{currentfill}{rgb}{0.548876,0.685104,0.994379}%
\pgfsetfillcolor{currentfill}%
\pgfsetlinewidth{0.000000pt}%
\definecolor{currentstroke}{rgb}{0.000000,0.000000,0.000000}%
\pgfsetstrokecolor{currentstroke}%
\pgfsetdash{}{0pt}%
\pgfpathmoveto{\pgfqpoint{4.359313in}{3.925809in}}%
\pgfpathlineto{\pgfqpoint{4.368262in}{3.819535in}}%
\pgfpathlineto{\pgfqpoint{4.377432in}{3.827749in}}%
\pgfpathlineto{\pgfqpoint{4.409789in}{3.949793in}}%
\pgfpathlineto{\pgfqpoint{4.441837in}{3.928796in}}%
\pgfpathlineto{\pgfqpoint{4.432240in}{3.778428in}}%
\pgfpathlineto{\pgfqpoint{4.423273in}{3.872300in}}%
\pgfpathlineto{\pgfqpoint{4.391466in}{3.973856in}}%
\pgfpathlineto{\pgfqpoint{4.359313in}{3.925809in}}%
\pgfpathclose%
\pgfusepath{fill}%
\end{pgfscope}%
\begin{pgfscope}%
\pgfpathrectangle{\pgfqpoint{1.020000in}{0.880000in}}{\pgfqpoint{6.160000in}{6.160000in}}%
\pgfusepath{clip}%
\pgfsetbuttcap%
\pgfsetroundjoin%
\definecolor{currentfill}{rgb}{0.510824,0.649397,0.985079}%
\pgfsetfillcolor{currentfill}%
\pgfsetlinewidth{0.000000pt}%
\definecolor{currentstroke}{rgb}{0.000000,0.000000,0.000000}%
\pgfsetstrokecolor{currentstroke}%
\pgfsetdash{}{0pt}%
\pgfpathmoveto{\pgfqpoint{4.780074in}{3.790051in}}%
\pgfpathlineto{\pgfqpoint{4.789851in}{3.824739in}}%
\pgfpathlineto{\pgfqpoint{4.799777in}{3.879526in}}%
\pgfpathlineto{\pgfqpoint{4.830510in}{3.683370in}}%
\pgfpathlineto{\pgfqpoint{4.863409in}{3.827979in}}%
\pgfpathlineto{\pgfqpoint{4.853266in}{3.754215in}}%
\pgfpathlineto{\pgfqpoint{4.845346in}{4.013033in}}%
\pgfpathlineto{\pgfqpoint{4.811956in}{3.785398in}}%
\pgfpathlineto{\pgfqpoint{4.780074in}{3.790051in}}%
\pgfpathclose%
\pgfusepath{fill}%
\end{pgfscope}%
\begin{pgfscope}%
\pgfpathrectangle{\pgfqpoint{1.020000in}{0.880000in}}{\pgfqpoint{6.160000in}{6.160000in}}%
\pgfusepath{clip}%
\pgfsetbuttcap%
\pgfsetroundjoin%
\definecolor{currentfill}{rgb}{0.667253,0.779176,0.992959}%
\pgfsetfillcolor{currentfill}%
\pgfsetlinewidth{0.000000pt}%
\definecolor{currentstroke}{rgb}{0.000000,0.000000,0.000000}%
\pgfsetstrokecolor{currentstroke}%
\pgfsetdash{}{0pt}%
\pgfpathmoveto{\pgfqpoint{4.066504in}{4.181754in}}%
\pgfpathlineto{\pgfqpoint{4.075358in}{4.181060in}}%
\pgfpathlineto{\pgfqpoint{4.084389in}{4.019109in}}%
\pgfpathlineto{\pgfqpoint{4.116568in}{4.002800in}}%
\pgfpathlineto{\pgfqpoint{4.148679in}{4.102133in}}%
\pgfpathlineto{\pgfqpoint{4.139751in}{4.108488in}}%
\pgfpathlineto{\pgfqpoint{4.130796in}{4.199592in}}%
\pgfpathlineto{\pgfqpoint{4.098905in}{3.894320in}}%
\pgfpathlineto{\pgfqpoint{4.066504in}{4.181754in}}%
\pgfpathclose%
\pgfusepath{fill}%
\end{pgfscope}%
\begin{pgfscope}%
\pgfpathrectangle{\pgfqpoint{1.020000in}{0.880000in}}{\pgfqpoint{6.160000in}{6.160000in}}%
\pgfusepath{clip}%
\pgfsetbuttcap%
\pgfsetroundjoin%
\definecolor{currentfill}{rgb}{0.581486,0.713451,0.998314}%
\pgfsetfillcolor{currentfill}%
\pgfsetlinewidth{0.000000pt}%
\definecolor{currentstroke}{rgb}{0.000000,0.000000,0.000000}%
\pgfsetstrokecolor{currentstroke}%
\pgfsetdash{}{0pt}%
\pgfpathmoveto{\pgfqpoint{4.634113in}{3.947337in}}%
\pgfpathlineto{\pgfqpoint{4.643515in}{3.942495in}}%
\pgfpathlineto{\pgfqpoint{4.653005in}{3.952504in}}%
\pgfpathlineto{\pgfqpoint{4.684668in}{3.877659in}}%
\pgfpathlineto{\pgfqpoint{4.717286in}{3.995242in}}%
\pgfpathlineto{\pgfqpoint{4.707927in}{4.023794in}}%
\pgfpathlineto{\pgfqpoint{4.697998in}{3.938336in}}%
\pgfpathlineto{\pgfqpoint{4.665625in}{3.850943in}}%
\pgfpathlineto{\pgfqpoint{4.634113in}{3.947337in}}%
\pgfpathclose%
\pgfusepath{fill}%
\end{pgfscope}%
\begin{pgfscope}%
\pgfpathrectangle{\pgfqpoint{1.020000in}{0.880000in}}{\pgfqpoint{6.160000in}{6.160000in}}%
\pgfusepath{clip}%
\pgfsetbuttcap%
\pgfsetroundjoin%
\definecolor{currentfill}{rgb}{0.624703,0.748318,0.998719}%
\pgfsetfillcolor{currentfill}%
\pgfsetlinewidth{0.000000pt}%
\definecolor{currentstroke}{rgb}{0.000000,0.000000,0.000000}%
\pgfsetstrokecolor{currentstroke}%
\pgfsetdash{}{0pt}%
\pgfpathmoveto{\pgfqpoint{4.212963in}{4.054895in}}%
\pgfpathlineto{\pgfqpoint{4.221916in}{3.934621in}}%
\pgfpathlineto{\pgfqpoint{4.230953in}{3.997763in}}%
\pgfpathlineto{\pgfqpoint{4.263061in}{3.945854in}}%
\pgfpathlineto{\pgfqpoint{4.295164in}{3.936045in}}%
\pgfpathlineto{\pgfqpoint{4.286305in}{4.139846in}}%
\pgfpathlineto{\pgfqpoint{4.277047in}{3.935279in}}%
\pgfpathlineto{\pgfqpoint{4.245121in}{4.136546in}}%
\pgfpathlineto{\pgfqpoint{4.212963in}{4.054895in}}%
\pgfpathclose%
\pgfusepath{fill}%
\end{pgfscope}%
\begin{pgfscope}%
\pgfpathrectangle{\pgfqpoint{1.020000in}{0.880000in}}{\pgfqpoint{6.160000in}{6.160000in}}%
\pgfusepath{clip}%
\pgfsetbuttcap%
\pgfsetroundjoin%
\definecolor{currentfill}{rgb}{0.500031,0.638508,0.981070}%
\pgfsetfillcolor{currentfill}%
\pgfsetlinewidth{0.000000pt}%
\definecolor{currentstroke}{rgb}{0.000000,0.000000,0.000000}%
\pgfsetstrokecolor{currentstroke}%
\pgfsetdash{}{0pt}%
\pgfpathmoveto{\pgfqpoint{5.201409in}{3.768043in}}%
\pgfpathlineto{\pgfqpoint{5.212482in}{3.872624in}}%
\pgfpathlineto{\pgfqpoint{5.221221in}{3.747663in}}%
\pgfpathlineto{\pgfqpoint{5.255049in}{3.932348in}}%
\pgfpathlineto{\pgfqpoint{5.284417in}{3.701808in}}%
\pgfpathlineto{\pgfqpoint{5.274894in}{3.752509in}}%
\pgfpathlineto{\pgfqpoint{5.264950in}{3.763137in}}%
\pgfpathlineto{\pgfqpoint{5.234116in}{3.854210in}}%
\pgfpathlineto{\pgfqpoint{5.201409in}{3.768043in}}%
\pgfpathclose%
\pgfusepath{fill}%
\end{pgfscope}%
\begin{pgfscope}%
\pgfpathrectangle{\pgfqpoint{1.020000in}{0.880000in}}{\pgfqpoint{6.160000in}{6.160000in}}%
\pgfusepath{clip}%
\pgfsetbuttcap%
\pgfsetroundjoin%
\definecolor{currentfill}{rgb}{0.473070,0.611077,0.970634}%
\pgfsetfillcolor{currentfill}%
\pgfsetlinewidth{0.000000pt}%
\definecolor{currentstroke}{rgb}{0.000000,0.000000,0.000000}%
\pgfsetstrokecolor{currentstroke}%
\pgfsetdash{}{0pt}%
\pgfpathmoveto{\pgfqpoint{5.138109in}{3.806634in}}%
\pgfpathlineto{\pgfqpoint{5.146607in}{3.656016in}}%
\pgfpathlineto{\pgfqpoint{5.157020in}{3.703507in}}%
\pgfpathlineto{\pgfqpoint{5.189517in}{3.764932in}}%
\pgfpathlineto{\pgfqpoint{5.221221in}{3.747663in}}%
\pgfpathlineto{\pgfqpoint{5.212482in}{3.872624in}}%
\pgfpathlineto{\pgfqpoint{5.201409in}{3.768043in}}%
\pgfpathlineto{\pgfqpoint{5.168683in}{3.676386in}}%
\pgfpathlineto{\pgfqpoint{5.138109in}{3.806634in}}%
\pgfpathclose%
\pgfusepath{fill}%
\end{pgfscope}%
\begin{pgfscope}%
\pgfpathrectangle{\pgfqpoint{1.020000in}{0.880000in}}{\pgfqpoint{6.160000in}{6.160000in}}%
\pgfusepath{clip}%
\pgfsetbuttcap%
\pgfsetroundjoin%
\definecolor{currentfill}{rgb}{0.800601,0.850358,0.930008}%
\pgfsetfillcolor{currentfill}%
\pgfsetlinewidth{0.000000pt}%
\definecolor{currentstroke}{rgb}{0.000000,0.000000,0.000000}%
\pgfsetstrokecolor{currentstroke}%
\pgfsetdash{}{0pt}%
\pgfpathmoveto{\pgfqpoint{3.561996in}{4.530158in}}%
\pgfpathlineto{\pgfqpoint{3.571277in}{4.377316in}}%
\pgfpathlineto{\pgfqpoint{3.578949in}{4.492097in}}%
\pgfpathlineto{\pgfqpoint{3.611957in}{4.383635in}}%
\pgfpathlineto{\pgfqpoint{3.645189in}{4.216546in}}%
\pgfpathlineto{\pgfqpoint{3.637247in}{4.131122in}}%
\pgfpathlineto{\pgfqpoint{3.628003in}{4.287632in}}%
\pgfpathlineto{\pgfqpoint{3.595768in}{4.285168in}}%
\pgfpathlineto{\pgfqpoint{3.561996in}{4.530158in}}%
\pgfpathclose%
\pgfusepath{fill}%
\end{pgfscope}%
\begin{pgfscope}%
\pgfpathrectangle{\pgfqpoint{1.020000in}{0.880000in}}{\pgfqpoint{6.160000in}{6.160000in}}%
\pgfusepath{clip}%
\pgfsetbuttcap%
\pgfsetroundjoin%
\definecolor{currentfill}{rgb}{0.494638,0.633022,0.978983}%
\pgfsetfillcolor{currentfill}%
\pgfsetlinewidth{0.000000pt}%
\definecolor{currentstroke}{rgb}{0.000000,0.000000,0.000000}%
\pgfsetstrokecolor{currentstroke}%
\pgfsetdash{}{0pt}%
\pgfpathmoveto{\pgfqpoint{4.927325in}{3.833384in}}%
\pgfpathlineto{\pgfqpoint{4.936533in}{3.763871in}}%
\pgfpathlineto{\pgfqpoint{4.947187in}{3.883665in}}%
\pgfpathlineto{\pgfqpoint{4.977884in}{3.720771in}}%
\pgfpathlineto{\pgfqpoint{5.010697in}{3.829367in}}%
\pgfpathlineto{\pgfqpoint{5.001441in}{3.897279in}}%
\pgfpathlineto{\pgfqpoint{4.989536in}{3.635089in}}%
\pgfpathlineto{\pgfqpoint{4.958406in}{3.725862in}}%
\pgfpathlineto{\pgfqpoint{4.927325in}{3.833384in}}%
\pgfpathclose%
\pgfusepath{fill}%
\end{pgfscope}%
\begin{pgfscope}%
\pgfpathrectangle{\pgfqpoint{1.020000in}{0.880000in}}{\pgfqpoint{6.160000in}{6.160000in}}%
\pgfusepath{clip}%
\pgfsetbuttcap%
\pgfsetroundjoin%
\definecolor{currentfill}{rgb}{0.457046,0.594006,0.963029}%
\pgfsetfillcolor{currentfill}%
\pgfsetlinewidth{0.000000pt}%
\definecolor{currentstroke}{rgb}{0.000000,0.000000,0.000000}%
\pgfsetstrokecolor{currentstroke}%
\pgfsetdash{}{0pt}%
\pgfpathmoveto{\pgfqpoint{5.347305in}{3.640678in}}%
\pgfpathlineto{\pgfqpoint{5.358456in}{3.726367in}}%
\pgfpathlineto{\pgfqpoint{5.368390in}{3.705710in}}%
\pgfpathlineto{\pgfqpoint{5.400872in}{3.759277in}}%
\pgfpathlineto{\pgfqpoint{5.433379in}{3.813554in}}%
\pgfpathlineto{\pgfqpoint{5.420241in}{3.579395in}}%
\pgfpathlineto{\pgfqpoint{5.412719in}{3.797999in}}%
\pgfpathlineto{\pgfqpoint{5.380576in}{3.767939in}}%
\pgfpathlineto{\pgfqpoint{5.347305in}{3.640678in}}%
\pgfpathclose%
\pgfusepath{fill}%
\end{pgfscope}%
\begin{pgfscope}%
\pgfpathrectangle{\pgfqpoint{1.020000in}{0.880000in}}{\pgfqpoint{6.160000in}{6.160000in}}%
\pgfusepath{clip}%
\pgfsetbuttcap%
\pgfsetroundjoin%
\definecolor{currentfill}{rgb}{0.576051,0.708780,0.997755}%
\pgfsetfillcolor{currentfill}%
\pgfsetlinewidth{0.000000pt}%
\definecolor{currentstroke}{rgb}{0.000000,0.000000,0.000000}%
\pgfsetstrokecolor{currentstroke}%
\pgfsetdash{}{0pt}%
\pgfpathmoveto{\pgfqpoint{4.570382in}{4.018058in}}%
\pgfpathlineto{\pgfqpoint{4.579420in}{3.933582in}}%
\pgfpathlineto{\pgfqpoint{4.588189in}{3.781453in}}%
\pgfpathlineto{\pgfqpoint{4.620638in}{3.880657in}}%
\pgfpathlineto{\pgfqpoint{4.653005in}{3.952504in}}%
\pgfpathlineto{\pgfqpoint{4.643515in}{3.942495in}}%
\pgfpathlineto{\pgfqpoint{4.634113in}{3.947337in}}%
\pgfpathlineto{\pgfqpoint{4.602059in}{3.932600in}}%
\pgfpathlineto{\pgfqpoint{4.570382in}{4.018058in}}%
\pgfpathclose%
\pgfusepath{fill}%
\end{pgfscope}%
\begin{pgfscope}%
\pgfpathrectangle{\pgfqpoint{1.020000in}{0.880000in}}{\pgfqpoint{6.160000in}{6.160000in}}%
\pgfusepath{clip}%
\pgfsetbuttcap%
\pgfsetroundjoin%
\definecolor{currentfill}{rgb}{0.538004,0.674902,0.991722}%
\pgfsetfillcolor{currentfill}%
\pgfsetlinewidth{0.000000pt}%
\definecolor{currentstroke}{rgb}{0.000000,0.000000,0.000000}%
\pgfsetstrokecolor{currentstroke}%
\pgfsetdash{}{0pt}%
\pgfpathmoveto{\pgfqpoint{4.717286in}{3.995242in}}%
\pgfpathlineto{\pgfqpoint{4.726246in}{3.890347in}}%
\pgfpathlineto{\pgfqpoint{4.735322in}{3.806519in}}%
\pgfpathlineto{\pgfqpoint{4.767463in}{3.828640in}}%
\pgfpathlineto{\pgfqpoint{4.799777in}{3.879526in}}%
\pgfpathlineto{\pgfqpoint{4.789851in}{3.824739in}}%
\pgfpathlineto{\pgfqpoint{4.780074in}{3.790051in}}%
\pgfpathlineto{\pgfqpoint{4.748346in}{3.825932in}}%
\pgfpathlineto{\pgfqpoint{4.717286in}{3.995242in}}%
\pgfpathclose%
\pgfusepath{fill}%
\end{pgfscope}%
\begin{pgfscope}%
\pgfpathrectangle{\pgfqpoint{1.020000in}{0.880000in}}{\pgfqpoint{6.160000in}{6.160000in}}%
\pgfusepath{clip}%
\pgfsetbuttcap%
\pgfsetroundjoin%
\definecolor{currentfill}{rgb}{0.576051,0.708780,0.997755}%
\pgfsetfillcolor{currentfill}%
\pgfsetlinewidth{0.000000pt}%
\definecolor{currentstroke}{rgb}{0.000000,0.000000,0.000000}%
\pgfsetstrokecolor{currentstroke}%
\pgfsetdash{}{0pt}%
\pgfpathmoveto{\pgfqpoint{4.295164in}{3.936045in}}%
\pgfpathlineto{\pgfqpoint{4.304227in}{3.917350in}}%
\pgfpathlineto{\pgfqpoint{4.313304in}{3.897504in}}%
\pgfpathlineto{\pgfqpoint{4.345701in}{4.056200in}}%
\pgfpathlineto{\pgfqpoint{4.377432in}{3.827749in}}%
\pgfpathlineto{\pgfqpoint{4.368262in}{3.819535in}}%
\pgfpathlineto{\pgfqpoint{4.359313in}{3.925809in}}%
\pgfpathlineto{\pgfqpoint{4.327386in}{4.026577in}}%
\pgfpathlineto{\pgfqpoint{4.295164in}{3.936045in}}%
\pgfpathclose%
\pgfusepath{fill}%
\end{pgfscope}%
\begin{pgfscope}%
\pgfpathrectangle{\pgfqpoint{1.020000in}{0.880000in}}{\pgfqpoint{6.160000in}{6.160000in}}%
\pgfusepath{clip}%
\pgfsetbuttcap%
\pgfsetroundjoin%
\definecolor{currentfill}{rgb}{0.630089,0.752516,0.998508}%
\pgfsetfillcolor{currentfill}%
\pgfsetlinewidth{0.000000pt}%
\definecolor{currentstroke}{rgb}{0.000000,0.000000,0.000000}%
\pgfsetstrokecolor{currentstroke}%
\pgfsetdash{}{0pt}%
\pgfpathmoveto{\pgfqpoint{4.148679in}{4.102133in}}%
\pgfpathlineto{\pgfqpoint{4.157673in}{3.900274in}}%
\pgfpathlineto{\pgfqpoint{4.166579in}{4.125321in}}%
\pgfpathlineto{\pgfqpoint{4.198793in}{4.041813in}}%
\pgfpathlineto{\pgfqpoint{4.230953in}{3.997763in}}%
\pgfpathlineto{\pgfqpoint{4.221916in}{3.934621in}}%
\pgfpathlineto{\pgfqpoint{4.212963in}{4.054895in}}%
\pgfpathlineto{\pgfqpoint{4.180840in}{3.982741in}}%
\pgfpathlineto{\pgfqpoint{4.148679in}{4.102133in}}%
\pgfpathclose%
\pgfusepath{fill}%
\end{pgfscope}%
\begin{pgfscope}%
\pgfpathrectangle{\pgfqpoint{1.020000in}{0.880000in}}{\pgfqpoint{6.160000in}{6.160000in}}%
\pgfusepath{clip}%
\pgfsetbuttcap%
\pgfsetroundjoin%
\definecolor{currentfill}{rgb}{0.733898,0.820018,0.970724}%
\pgfsetfillcolor{currentfill}%
\pgfsetlinewidth{0.000000pt}%
\definecolor{currentstroke}{rgb}{0.000000,0.000000,0.000000}%
\pgfsetstrokecolor{currentstroke}%
\pgfsetdash{}{0pt}%
\pgfpathmoveto{\pgfqpoint{3.855942in}{4.159353in}}%
\pgfpathlineto{\pgfqpoint{3.864248in}{4.263241in}}%
\pgfpathlineto{\pgfqpoint{3.872966in}{4.244932in}}%
\pgfpathlineto{\pgfqpoint{3.905639in}{4.105954in}}%
\pgfpathlineto{\pgfqpoint{3.937795in}{4.135247in}}%
\pgfpathlineto{\pgfqpoint{3.929121in}{4.120598in}}%
\pgfpathlineto{\pgfqpoint{3.920277in}{4.178711in}}%
\pgfpathlineto{\pgfqpoint{3.887383in}{4.419141in}}%
\pgfpathlineto{\pgfqpoint{3.855942in}{4.159353in}}%
\pgfpathclose%
\pgfusepath{fill}%
\end{pgfscope}%
\begin{pgfscope}%
\pgfpathrectangle{\pgfqpoint{1.020000in}{0.880000in}}{\pgfqpoint{6.160000in}{6.160000in}}%
\pgfusepath{clip}%
\pgfsetbuttcap%
\pgfsetroundjoin%
\definecolor{currentfill}{rgb}{0.693321,0.796314,0.986308}%
\pgfsetfillcolor{currentfill}%
\pgfsetlinewidth{0.000000pt}%
\definecolor{currentstroke}{rgb}{0.000000,0.000000,0.000000}%
\pgfsetstrokecolor{currentstroke}%
\pgfsetdash{}{0pt}%
\pgfpathmoveto{\pgfqpoint{4.002405in}{4.028112in}}%
\pgfpathlineto{\pgfqpoint{4.011161in}{4.041867in}}%
\pgfpathlineto{\pgfqpoint{4.019624in}{4.249239in}}%
\pgfpathlineto{\pgfqpoint{4.051961in}{4.204497in}}%
\pgfpathlineto{\pgfqpoint{4.084389in}{4.019109in}}%
\pgfpathlineto{\pgfqpoint{4.075358in}{4.181060in}}%
\pgfpathlineto{\pgfqpoint{4.066504in}{4.181754in}}%
\pgfpathlineto{\pgfqpoint{4.034483in}{4.075423in}}%
\pgfpathlineto{\pgfqpoint{4.002405in}{4.028112in}}%
\pgfpathclose%
\pgfusepath{fill}%
\end{pgfscope}%
\begin{pgfscope}%
\pgfpathrectangle{\pgfqpoint{1.020000in}{0.880000in}}{\pgfqpoint{6.160000in}{6.160000in}}%
\pgfusepath{clip}%
\pgfsetbuttcap%
\pgfsetroundjoin%
\definecolor{currentfill}{rgb}{0.441123,0.576532,0.954545}%
\pgfsetfillcolor{currentfill}%
\pgfsetlinewidth{0.000000pt}%
\definecolor{currentstroke}{rgb}{0.000000,0.000000,0.000000}%
\pgfsetstrokecolor{currentstroke}%
\pgfsetdash{}{0pt}%
\pgfpathmoveto{\pgfqpoint{5.495091in}{3.666076in}}%
\pgfpathlineto{\pgfqpoint{5.505666in}{3.686160in}}%
\pgfpathlineto{\pgfqpoint{5.515443in}{3.644271in}}%
\pgfpathlineto{\pgfqpoint{5.545842in}{3.541432in}}%
\pgfpathlineto{\pgfqpoint{5.537597in}{3.697114in}}%
\pgfpathlineto{\pgfqpoint{5.529439in}{3.860066in}}%
\pgfpathlineto{\pgfqpoint{5.495091in}{3.666076in}}%
\pgfpathclose%
\pgfusepath{fill}%
\end{pgfscope}%
\begin{pgfscope}%
\pgfpathrectangle{\pgfqpoint{1.020000in}{0.880000in}}{\pgfqpoint{6.160000in}{6.160000in}}%
\pgfusepath{clip}%
\pgfsetbuttcap%
\pgfsetroundjoin%
\definecolor{currentfill}{rgb}{0.768034,0.837035,0.952488}%
\pgfsetfillcolor{currentfill}%
\pgfsetlinewidth{0.000000pt}%
\definecolor{currentstroke}{rgb}{0.000000,0.000000,0.000000}%
\pgfsetstrokecolor{currentstroke}%
\pgfsetdash{}{0pt}%
\pgfpathmoveto{\pgfqpoint{3.645189in}{4.216546in}}%
\pgfpathlineto{\pgfqpoint{3.652936in}{4.345222in}}%
\pgfpathlineto{\pgfqpoint{3.661527in}{4.318193in}}%
\pgfpathlineto{\pgfqpoint{3.693583in}{4.381345in}}%
\pgfpathlineto{\pgfqpoint{3.727459in}{4.047561in}}%
\pgfpathlineto{\pgfqpoint{3.717751in}{4.312196in}}%
\pgfpathlineto{\pgfqpoint{3.709448in}{4.272200in}}%
\pgfpathlineto{\pgfqpoint{3.677489in}{4.210327in}}%
\pgfpathlineto{\pgfqpoint{3.645189in}{4.216546in}}%
\pgfpathclose%
\pgfusepath{fill}%
\end{pgfscope}%
\begin{pgfscope}%
\pgfpathrectangle{\pgfqpoint{1.020000in}{0.880000in}}{\pgfqpoint{6.160000in}{6.160000in}}%
\pgfusepath{clip}%
\pgfsetbuttcap%
\pgfsetroundjoin%
\definecolor{currentfill}{rgb}{0.500031,0.638508,0.981070}%
\pgfsetfillcolor{currentfill}%
\pgfsetlinewidth{0.000000pt}%
\definecolor{currentstroke}{rgb}{0.000000,0.000000,0.000000}%
\pgfsetstrokecolor{currentstroke}%
\pgfsetdash{}{0pt}%
\pgfpathmoveto{\pgfqpoint{5.073089in}{3.665065in}}%
\pgfpathlineto{\pgfqpoint{5.084015in}{3.782359in}}%
\pgfpathlineto{\pgfqpoint{5.094520in}{3.848645in}}%
\pgfpathlineto{\pgfqpoint{5.126871in}{3.889829in}}%
\pgfpathlineto{\pgfqpoint{5.157020in}{3.703507in}}%
\pgfpathlineto{\pgfqpoint{5.146607in}{3.656016in}}%
\pgfpathlineto{\pgfqpoint{5.138109in}{3.806634in}}%
\pgfpathlineto{\pgfqpoint{5.107184in}{3.909666in}}%
\pgfpathlineto{\pgfqpoint{5.073089in}{3.665065in}}%
\pgfpathclose%
\pgfusepath{fill}%
\end{pgfscope}%
\begin{pgfscope}%
\pgfpathrectangle{\pgfqpoint{1.020000in}{0.880000in}}{\pgfqpoint{6.160000in}{6.160000in}}%
\pgfusepath{clip}%
\pgfsetbuttcap%
\pgfsetroundjoin%
\definecolor{currentfill}{rgb}{0.478462,0.616564,0.972721}%
\pgfsetfillcolor{currentfill}%
\pgfsetlinewidth{0.000000pt}%
\definecolor{currentstroke}{rgb}{0.000000,0.000000,0.000000}%
\pgfsetstrokecolor{currentstroke}%
\pgfsetdash{}{0pt}%
\pgfpathmoveto{\pgfqpoint{5.284417in}{3.701808in}}%
\pgfpathlineto{\pgfqpoint{5.295129in}{3.757507in}}%
\pgfpathlineto{\pgfqpoint{5.305689in}{3.796789in}}%
\pgfpathlineto{\pgfqpoint{5.338101in}{3.841476in}}%
\pgfpathlineto{\pgfqpoint{5.368390in}{3.705710in}}%
\pgfpathlineto{\pgfqpoint{5.358456in}{3.726367in}}%
\pgfpathlineto{\pgfqpoint{5.347305in}{3.640678in}}%
\pgfpathlineto{\pgfqpoint{5.317651in}{3.828007in}}%
\pgfpathlineto{\pgfqpoint{5.284417in}{3.701808in}}%
\pgfpathclose%
\pgfusepath{fill}%
\end{pgfscope}%
\begin{pgfscope}%
\pgfpathrectangle{\pgfqpoint{1.020000in}{0.880000in}}{\pgfqpoint{6.160000in}{6.160000in}}%
\pgfusepath{clip}%
\pgfsetbuttcap%
\pgfsetroundjoin%
\definecolor{currentfill}{rgb}{0.592356,0.722792,0.999434}%
\pgfsetfillcolor{currentfill}%
\pgfsetlinewidth{0.000000pt}%
\definecolor{currentstroke}{rgb}{0.000000,0.000000,0.000000}%
\pgfsetstrokecolor{currentstroke}%
\pgfsetdash{}{0pt}%
\pgfpathmoveto{\pgfqpoint{4.505874in}{3.897640in}}%
\pgfpathlineto{\pgfqpoint{4.515144in}{3.888527in}}%
\pgfpathlineto{\pgfqpoint{4.524855in}{4.004198in}}%
\pgfpathlineto{\pgfqpoint{4.556848in}{3.968727in}}%
\pgfpathlineto{\pgfqpoint{4.588189in}{3.781453in}}%
\pgfpathlineto{\pgfqpoint{4.579420in}{3.933582in}}%
\pgfpathlineto{\pgfqpoint{4.570382in}{4.018058in}}%
\pgfpathlineto{\pgfqpoint{4.538473in}{4.058274in}}%
\pgfpathlineto{\pgfqpoint{4.505874in}{3.897640in}}%
\pgfpathclose%
\pgfusepath{fill}%
\end{pgfscope}%
\begin{pgfscope}%
\pgfpathrectangle{\pgfqpoint{1.020000in}{0.880000in}}{\pgfqpoint{6.160000in}{6.160000in}}%
\pgfusepath{clip}%
\pgfsetbuttcap%
\pgfsetroundjoin%
\definecolor{currentfill}{rgb}{0.570616,0.704109,0.997195}%
\pgfsetfillcolor{currentfill}%
\pgfsetlinewidth{0.000000pt}%
\definecolor{currentstroke}{rgb}{0.000000,0.000000,0.000000}%
\pgfsetstrokecolor{currentstroke}%
\pgfsetdash{}{0pt}%
\pgfpathmoveto{\pgfqpoint{4.441837in}{3.928796in}}%
\pgfpathlineto{\pgfqpoint{4.451330in}{4.022608in}}%
\pgfpathlineto{\pgfqpoint{4.460215in}{3.882756in}}%
\pgfpathlineto{\pgfqpoint{4.491912in}{3.744487in}}%
\pgfpathlineto{\pgfqpoint{4.524855in}{4.004198in}}%
\pgfpathlineto{\pgfqpoint{4.515144in}{3.888527in}}%
\pgfpathlineto{\pgfqpoint{4.505874in}{3.897640in}}%
\pgfpathlineto{\pgfqpoint{4.473742in}{3.868970in}}%
\pgfpathlineto{\pgfqpoint{4.441837in}{3.928796in}}%
\pgfpathclose%
\pgfusepath{fill}%
\end{pgfscope}%
\begin{pgfscope}%
\pgfpathrectangle{\pgfqpoint{1.020000in}{0.880000in}}{\pgfqpoint{6.160000in}{6.160000in}}%
\pgfusepath{clip}%
\pgfsetbuttcap%
\pgfsetroundjoin%
\definecolor{currentfill}{rgb}{0.743754,0.825125,0.965798}%
\pgfsetfillcolor{currentfill}%
\pgfsetlinewidth{0.000000pt}%
\definecolor{currentstroke}{rgb}{0.000000,0.000000,0.000000}%
\pgfsetstrokecolor{currentstroke}%
\pgfsetdash{}{0pt}%
\pgfpathmoveto{\pgfqpoint{3.791333in}{4.198291in}}%
\pgfpathlineto{\pgfqpoint{3.799500in}{4.306086in}}%
\pgfpathlineto{\pgfqpoint{3.808170in}{4.288336in}}%
\pgfpathlineto{\pgfqpoint{3.841204in}{4.084193in}}%
\pgfpathlineto{\pgfqpoint{3.872966in}{4.244932in}}%
\pgfpathlineto{\pgfqpoint{3.864248in}{4.263241in}}%
\pgfpathlineto{\pgfqpoint{3.855942in}{4.159353in}}%
\pgfpathlineto{\pgfqpoint{3.823639in}{4.182634in}}%
\pgfpathlineto{\pgfqpoint{3.791333in}{4.198291in}}%
\pgfpathclose%
\pgfusepath{fill}%
\end{pgfscope}%
\begin{pgfscope}%
\pgfpathrectangle{\pgfqpoint{1.020000in}{0.880000in}}{\pgfqpoint{6.160000in}{6.160000in}}%
\pgfusepath{clip}%
\pgfsetbuttcap%
\pgfsetroundjoin%
\definecolor{currentfill}{rgb}{0.651398,0.768121,0.995891}%
\pgfsetfillcolor{currentfill}%
\pgfsetlinewidth{0.000000pt}%
\definecolor{currentstroke}{rgb}{0.000000,0.000000,0.000000}%
\pgfsetstrokecolor{currentstroke}%
\pgfsetdash{}{0pt}%
\pgfpathmoveto{\pgfqpoint{4.084389in}{4.019109in}}%
\pgfpathlineto{\pgfqpoint{4.093288in}{3.991924in}}%
\pgfpathlineto{\pgfqpoint{4.102182in}{3.988317in}}%
\pgfpathlineto{\pgfqpoint{4.134306in}{4.185454in}}%
\pgfpathlineto{\pgfqpoint{4.166579in}{4.125321in}}%
\pgfpathlineto{\pgfqpoint{4.157673in}{3.900274in}}%
\pgfpathlineto{\pgfqpoint{4.148679in}{4.102133in}}%
\pgfpathlineto{\pgfqpoint{4.116568in}{4.002800in}}%
\pgfpathlineto{\pgfqpoint{4.084389in}{4.019109in}}%
\pgfpathclose%
\pgfusepath{fill}%
\end{pgfscope}%
\begin{pgfscope}%
\pgfpathrectangle{\pgfqpoint{1.020000in}{0.880000in}}{\pgfqpoint{6.160000in}{6.160000in}}%
\pgfusepath{clip}%
\pgfsetbuttcap%
\pgfsetroundjoin%
\definecolor{currentfill}{rgb}{0.489246,0.627536,0.976896}%
\pgfsetfillcolor{currentfill}%
\pgfsetlinewidth{0.000000pt}%
\definecolor{currentstroke}{rgb}{0.000000,0.000000,0.000000}%
\pgfsetstrokecolor{currentstroke}%
\pgfsetdash{}{0pt}%
\pgfpathmoveto{\pgfqpoint{5.010697in}{3.829367in}}%
\pgfpathlineto{\pgfqpoint{5.019609in}{3.719303in}}%
\pgfpathlineto{\pgfqpoint{5.029501in}{3.725662in}}%
\pgfpathlineto{\pgfqpoint{5.062111in}{3.799817in}}%
\pgfpathlineto{\pgfqpoint{5.094520in}{3.848645in}}%
\pgfpathlineto{\pgfqpoint{5.084015in}{3.782359in}}%
\pgfpathlineto{\pgfqpoint{5.073089in}{3.665065in}}%
\pgfpathlineto{\pgfqpoint{5.041598in}{3.708664in}}%
\pgfpathlineto{\pgfqpoint{5.010697in}{3.829367in}}%
\pgfpathclose%
\pgfusepath{fill}%
\end{pgfscope}%
\begin{pgfscope}%
\pgfpathrectangle{\pgfqpoint{1.020000in}{0.880000in}}{\pgfqpoint{6.160000in}{6.160000in}}%
\pgfusepath{clip}%
\pgfsetbuttcap%
\pgfsetroundjoin%
\definecolor{currentfill}{rgb}{0.451739,0.588181,0.960201}%
\pgfsetfillcolor{currentfill}%
\pgfsetlinewidth{0.000000pt}%
\definecolor{currentstroke}{rgb}{0.000000,0.000000,0.000000}%
\pgfsetstrokecolor{currentstroke}%
\pgfsetdash{}{0pt}%
\pgfpathmoveto{\pgfqpoint{5.433379in}{3.813554in}}%
\pgfpathlineto{\pgfqpoint{5.442123in}{3.692311in}}%
\pgfpathlineto{\pgfqpoint{5.452341in}{3.687853in}}%
\pgfpathlineto{\pgfqpoint{5.484672in}{3.724948in}}%
\pgfpathlineto{\pgfqpoint{5.515443in}{3.644271in}}%
\pgfpathlineto{\pgfqpoint{5.505666in}{3.686160in}}%
\pgfpathlineto{\pgfqpoint{5.495091in}{3.666076in}}%
\pgfpathlineto{\pgfqpoint{5.463563in}{3.684638in}}%
\pgfpathlineto{\pgfqpoint{5.433379in}{3.813554in}}%
\pgfpathclose%
\pgfusepath{fill}%
\end{pgfscope}%
\begin{pgfscope}%
\pgfpathrectangle{\pgfqpoint{1.020000in}{0.880000in}}{\pgfqpoint{6.160000in}{6.160000in}}%
\pgfusepath{clip}%
\pgfsetbuttcap%
\pgfsetroundjoin%
\definecolor{currentfill}{rgb}{0.603162,0.731527,0.999565}%
\pgfsetfillcolor{currentfill}%
\pgfsetlinewidth{0.000000pt}%
\definecolor{currentstroke}{rgb}{0.000000,0.000000,0.000000}%
\pgfsetstrokecolor{currentstroke}%
\pgfsetdash{}{0pt}%
\pgfpathmoveto{\pgfqpoint{4.230953in}{3.997763in}}%
\pgfpathlineto{\pgfqpoint{4.239976in}{3.983246in}}%
\pgfpathlineto{\pgfqpoint{4.248997in}{3.942521in}}%
\pgfpathlineto{\pgfqpoint{4.281296in}{4.052351in}}%
\pgfpathlineto{\pgfqpoint{4.313304in}{3.897504in}}%
\pgfpathlineto{\pgfqpoint{4.304227in}{3.917350in}}%
\pgfpathlineto{\pgfqpoint{4.295164in}{3.936045in}}%
\pgfpathlineto{\pgfqpoint{4.263061in}{3.945854in}}%
\pgfpathlineto{\pgfqpoint{4.230953in}{3.997763in}}%
\pgfpathclose%
\pgfusepath{fill}%
\end{pgfscope}%
\begin{pgfscope}%
\pgfpathrectangle{\pgfqpoint{1.020000in}{0.880000in}}{\pgfqpoint{6.160000in}{6.160000in}}%
\pgfusepath{clip}%
\pgfsetbuttcap%
\pgfsetroundjoin%
\definecolor{currentfill}{rgb}{0.543440,0.680003,0.993051}%
\pgfsetfillcolor{currentfill}%
\pgfsetlinewidth{0.000000pt}%
\definecolor{currentstroke}{rgb}{0.000000,0.000000,0.000000}%
\pgfsetstrokecolor{currentstroke}%
\pgfsetdash{}{0pt}%
\pgfpathmoveto{\pgfqpoint{4.863409in}{3.827979in}}%
\pgfpathlineto{\pgfqpoint{4.873747in}{3.925135in}}%
\pgfpathlineto{\pgfqpoint{4.882928in}{3.850844in}}%
\pgfpathlineto{\pgfqpoint{4.915569in}{3.936846in}}%
\pgfpathlineto{\pgfqpoint{4.947187in}{3.883665in}}%
\pgfpathlineto{\pgfqpoint{4.936533in}{3.763871in}}%
\pgfpathlineto{\pgfqpoint{4.927325in}{3.833384in}}%
\pgfpathlineto{\pgfqpoint{4.895549in}{3.855229in}}%
\pgfpathlineto{\pgfqpoint{4.863409in}{3.827979in}}%
\pgfpathclose%
\pgfusepath{fill}%
\end{pgfscope}%
\begin{pgfscope}%
\pgfpathrectangle{\pgfqpoint{1.020000in}{0.880000in}}{\pgfqpoint{6.160000in}{6.160000in}}%
\pgfusepath{clip}%
\pgfsetbuttcap%
\pgfsetroundjoin%
\definecolor{currentfill}{rgb}{0.843358,0.861820,0.890017}%
\pgfsetfillcolor{currentfill}%
\pgfsetlinewidth{0.000000pt}%
\definecolor{currentstroke}{rgb}{0.000000,0.000000,0.000000}%
\pgfsetstrokecolor{currentstroke}%
\pgfsetdash{}{0pt}%
\pgfpathmoveto{\pgfqpoint{3.497565in}{4.483913in}}%
\pgfpathlineto{\pgfqpoint{3.506472in}{4.391676in}}%
\pgfpathlineto{\pgfqpoint{3.515539in}{4.275709in}}%
\pgfpathlineto{\pgfqpoint{3.547286in}{4.373117in}}%
\pgfpathlineto{\pgfqpoint{3.578949in}{4.492097in}}%
\pgfpathlineto{\pgfqpoint{3.571277in}{4.377316in}}%
\pgfpathlineto{\pgfqpoint{3.561996in}{4.530158in}}%
\pgfpathlineto{\pgfqpoint{3.530450in}{4.403784in}}%
\pgfpathlineto{\pgfqpoint{3.497565in}{4.483913in}}%
\pgfpathclose%
\pgfusepath{fill}%
\end{pgfscope}%
\begin{pgfscope}%
\pgfpathrectangle{\pgfqpoint{1.020000in}{0.880000in}}{\pgfqpoint{6.160000in}{6.160000in}}%
\pgfusepath{clip}%
\pgfsetbuttcap%
\pgfsetroundjoin%
\definecolor{currentfill}{rgb}{0.713852,0.808857,0.979386}%
\pgfsetfillcolor{currentfill}%
\pgfsetlinewidth{0.000000pt}%
\definecolor{currentstroke}{rgb}{0.000000,0.000000,0.000000}%
\pgfsetstrokecolor{currentstroke}%
\pgfsetdash{}{0pt}%
\pgfpathmoveto{\pgfqpoint{3.937795in}{4.135247in}}%
\pgfpathlineto{\pgfqpoint{3.946199in}{4.273272in}}%
\pgfpathlineto{\pgfqpoint{3.955302in}{4.120739in}}%
\pgfpathlineto{\pgfqpoint{3.987483in}{4.168935in}}%
\pgfpathlineto{\pgfqpoint{4.019624in}{4.249239in}}%
\pgfpathlineto{\pgfqpoint{4.011161in}{4.041867in}}%
\pgfpathlineto{\pgfqpoint{4.002405in}{4.028112in}}%
\pgfpathlineto{\pgfqpoint{3.969925in}{4.176614in}}%
\pgfpathlineto{\pgfqpoint{3.937795in}{4.135247in}}%
\pgfpathclose%
\pgfusepath{fill}%
\end{pgfscope}%
\begin{pgfscope}%
\pgfpathrectangle{\pgfqpoint{1.020000in}{0.880000in}}{\pgfqpoint{6.160000in}{6.160000in}}%
\pgfusepath{clip}%
\pgfsetbuttcap%
\pgfsetroundjoin%
\definecolor{currentfill}{rgb}{0.521696,0.659599,0.987736}%
\pgfsetfillcolor{currentfill}%
\pgfsetlinewidth{0.000000pt}%
\definecolor{currentstroke}{rgb}{0.000000,0.000000,0.000000}%
\pgfsetstrokecolor{currentstroke}%
\pgfsetdash{}{0pt}%
\pgfpathmoveto{\pgfqpoint{4.799777in}{3.879526in}}%
\pgfpathlineto{\pgfqpoint{4.809392in}{3.879410in}}%
\pgfpathlineto{\pgfqpoint{4.817943in}{3.708196in}}%
\pgfpathlineto{\pgfqpoint{4.850181in}{3.743381in}}%
\pgfpathlineto{\pgfqpoint{4.882928in}{3.850844in}}%
\pgfpathlineto{\pgfqpoint{4.873747in}{3.925135in}}%
\pgfpathlineto{\pgfqpoint{4.863409in}{3.827979in}}%
\pgfpathlineto{\pgfqpoint{4.830510in}{3.683370in}}%
\pgfpathlineto{\pgfqpoint{4.799777in}{3.879526in}}%
\pgfpathclose%
\pgfusepath{fill}%
\end{pgfscope}%
\begin{pgfscope}%
\pgfpathrectangle{\pgfqpoint{1.020000in}{0.880000in}}{\pgfqpoint{6.160000in}{6.160000in}}%
\pgfusepath{clip}%
\pgfsetbuttcap%
\pgfsetroundjoin%
\definecolor{currentfill}{rgb}{0.586921,0.718121,0.998874}%
\pgfsetfillcolor{currentfill}%
\pgfsetlinewidth{0.000000pt}%
\definecolor{currentstroke}{rgb}{0.000000,0.000000,0.000000}%
\pgfsetstrokecolor{currentstroke}%
\pgfsetdash{}{0pt}%
\pgfpathmoveto{\pgfqpoint{4.653005in}{3.952504in}}%
\pgfpathlineto{\pgfqpoint{4.662621in}{3.984570in}}%
\pgfpathlineto{\pgfqpoint{4.671873in}{3.936408in}}%
\pgfpathlineto{\pgfqpoint{4.704079in}{3.958608in}}%
\pgfpathlineto{\pgfqpoint{4.735322in}{3.806519in}}%
\pgfpathlineto{\pgfqpoint{4.726246in}{3.890347in}}%
\pgfpathlineto{\pgfqpoint{4.717286in}{3.995242in}}%
\pgfpathlineto{\pgfqpoint{4.684668in}{3.877659in}}%
\pgfpathlineto{\pgfqpoint{4.653005in}{3.952504in}}%
\pgfpathclose%
\pgfusepath{fill}%
\end{pgfscope}%
\begin{pgfscope}%
\pgfpathrectangle{\pgfqpoint{1.020000in}{0.880000in}}{\pgfqpoint{6.160000in}{6.160000in}}%
\pgfusepath{clip}%
\pgfsetbuttcap%
\pgfsetroundjoin%
\definecolor{currentfill}{rgb}{0.559747,0.694768,0.996075}%
\pgfsetfillcolor{currentfill}%
\pgfsetlinewidth{0.000000pt}%
\definecolor{currentstroke}{rgb}{0.000000,0.000000,0.000000}%
\pgfsetstrokecolor{currentstroke}%
\pgfsetdash{}{0pt}%
\pgfpathmoveto{\pgfqpoint{4.588189in}{3.781453in}}%
\pgfpathlineto{\pgfqpoint{4.597716in}{3.815408in}}%
\pgfpathlineto{\pgfqpoint{4.607066in}{3.801232in}}%
\pgfpathlineto{\pgfqpoint{4.639433in}{3.863522in}}%
\pgfpathlineto{\pgfqpoint{4.671873in}{3.936408in}}%
\pgfpathlineto{\pgfqpoint{4.662621in}{3.984570in}}%
\pgfpathlineto{\pgfqpoint{4.653005in}{3.952504in}}%
\pgfpathlineto{\pgfqpoint{4.620638in}{3.880657in}}%
\pgfpathlineto{\pgfqpoint{4.588189in}{3.781453in}}%
\pgfpathclose%
\pgfusepath{fill}%
\end{pgfscope}%
\begin{pgfscope}%
\pgfpathrectangle{\pgfqpoint{1.020000in}{0.880000in}}{\pgfqpoint{6.160000in}{6.160000in}}%
\pgfusepath{clip}%
\pgfsetbuttcap%
\pgfsetroundjoin%
\definecolor{currentfill}{rgb}{0.753611,0.830233,0.960871}%
\pgfsetfillcolor{currentfill}%
\pgfsetlinewidth{0.000000pt}%
\definecolor{currentstroke}{rgb}{0.000000,0.000000,0.000000}%
\pgfsetstrokecolor{currentstroke}%
\pgfsetdash{}{0pt}%
\pgfpathmoveto{\pgfqpoint{3.727459in}{4.047561in}}%
\pgfpathlineto{\pgfqpoint{3.734592in}{4.355952in}}%
\pgfpathlineto{\pgfqpoint{3.743754in}{4.214931in}}%
\pgfpathlineto{\pgfqpoint{3.776158in}{4.202417in}}%
\pgfpathlineto{\pgfqpoint{3.808170in}{4.288336in}}%
\pgfpathlineto{\pgfqpoint{3.799500in}{4.306086in}}%
\pgfpathlineto{\pgfqpoint{3.791333in}{4.198291in}}%
\pgfpathlineto{\pgfqpoint{3.759299in}{4.142433in}}%
\pgfpathlineto{\pgfqpoint{3.727459in}{4.047561in}}%
\pgfpathclose%
\pgfusepath{fill}%
\end{pgfscope}%
\begin{pgfscope}%
\pgfpathrectangle{\pgfqpoint{1.020000in}{0.880000in}}{\pgfqpoint{6.160000in}{6.160000in}}%
\pgfusepath{clip}%
\pgfsetbuttcap%
\pgfsetroundjoin%
\definecolor{currentfill}{rgb}{0.839351,0.861167,0.894494}%
\pgfsetfillcolor{currentfill}%
\pgfsetlinewidth{0.000000pt}%
\definecolor{currentstroke}{rgb}{0.000000,0.000000,0.000000}%
\pgfsetstrokecolor{currentstroke}%
\pgfsetdash{}{0pt}%
\pgfpathmoveto{\pgfqpoint{3.433301in}{4.413210in}}%
\pgfpathlineto{\pgfqpoint{3.441637in}{4.397801in}}%
\pgfpathlineto{\pgfqpoint{3.449037in}{4.513089in}}%
\pgfpathlineto{\pgfqpoint{3.482619in}{4.353482in}}%
\pgfpathlineto{\pgfqpoint{3.515539in}{4.275709in}}%
\pgfpathlineto{\pgfqpoint{3.506472in}{4.391676in}}%
\pgfpathlineto{\pgfqpoint{3.497565in}{4.483913in}}%
\pgfpathlineto{\pgfqpoint{3.465717in}{4.408271in}}%
\pgfpathlineto{\pgfqpoint{3.433301in}{4.413210in}}%
\pgfpathclose%
\pgfusepath{fill}%
\end{pgfscope}%
\begin{pgfscope}%
\pgfpathrectangle{\pgfqpoint{1.020000in}{0.880000in}}{\pgfqpoint{6.160000in}{6.160000in}}%
\pgfusepath{clip}%
\pgfsetbuttcap%
\pgfsetroundjoin%
\definecolor{currentfill}{rgb}{0.510824,0.649397,0.985079}%
\pgfsetfillcolor{currentfill}%
\pgfsetlinewidth{0.000000pt}%
\definecolor{currentstroke}{rgb}{0.000000,0.000000,0.000000}%
\pgfsetstrokecolor{currentstroke}%
\pgfsetdash{}{0pt}%
\pgfpathmoveto{\pgfqpoint{5.221221in}{3.747663in}}%
\pgfpathlineto{\pgfqpoint{5.232660in}{3.881592in}}%
\pgfpathlineto{\pgfqpoint{5.241632in}{3.777104in}}%
\pgfpathlineto{\pgfqpoint{5.273273in}{3.750743in}}%
\pgfpathlineto{\pgfqpoint{5.305689in}{3.796789in}}%
\pgfpathlineto{\pgfqpoint{5.295129in}{3.757507in}}%
\pgfpathlineto{\pgfqpoint{5.284417in}{3.701808in}}%
\pgfpathlineto{\pgfqpoint{5.255049in}{3.932348in}}%
\pgfpathlineto{\pgfqpoint{5.221221in}{3.747663in}}%
\pgfpathclose%
\pgfusepath{fill}%
\end{pgfscope}%
\begin{pgfscope}%
\pgfpathrectangle{\pgfqpoint{1.020000in}{0.880000in}}{\pgfqpoint{6.160000in}{6.160000in}}%
\pgfusepath{clip}%
\pgfsetbuttcap%
\pgfsetroundjoin%
\definecolor{currentfill}{rgb}{0.586921,0.718121,0.998874}%
\pgfsetfillcolor{currentfill}%
\pgfsetlinewidth{0.000000pt}%
\definecolor{currentstroke}{rgb}{0.000000,0.000000,0.000000}%
\pgfsetstrokecolor{currentstroke}%
\pgfsetdash{}{0pt}%
\pgfpathmoveto{\pgfqpoint{4.377432in}{3.827749in}}%
\pgfpathlineto{\pgfqpoint{4.386553in}{3.802184in}}%
\pgfpathlineto{\pgfqpoint{4.395912in}{3.882231in}}%
\pgfpathlineto{\pgfqpoint{4.428655in}{4.121396in}}%
\pgfpathlineto{\pgfqpoint{4.460215in}{3.882756in}}%
\pgfpathlineto{\pgfqpoint{4.451330in}{4.022608in}}%
\pgfpathlineto{\pgfqpoint{4.441837in}{3.928796in}}%
\pgfpathlineto{\pgfqpoint{4.409789in}{3.949793in}}%
\pgfpathlineto{\pgfqpoint{4.377432in}{3.827749in}}%
\pgfpathclose%
\pgfusepath{fill}%
\end{pgfscope}%
\begin{pgfscope}%
\pgfpathrectangle{\pgfqpoint{1.020000in}{0.880000in}}{\pgfqpoint{6.160000in}{6.160000in}}%
\pgfusepath{clip}%
\pgfsetbuttcap%
\pgfsetroundjoin%
\definecolor{currentfill}{rgb}{0.804965,0.851666,0.926165}%
\pgfsetfillcolor{currentfill}%
\pgfsetlinewidth{0.000000pt}%
\definecolor{currentstroke}{rgb}{0.000000,0.000000,0.000000}%
\pgfsetstrokecolor{currentstroke}%
\pgfsetdash{}{0pt}%
\pgfpathmoveto{\pgfqpoint{3.578949in}{4.492097in}}%
\pgfpathlineto{\pgfqpoint{3.588761in}{4.250070in}}%
\pgfpathlineto{\pgfqpoint{3.596514in}{4.358261in}}%
\pgfpathlineto{\pgfqpoint{3.629644in}{4.227418in}}%
\pgfpathlineto{\pgfqpoint{3.661527in}{4.318193in}}%
\pgfpathlineto{\pgfqpoint{3.652936in}{4.345222in}}%
\pgfpathlineto{\pgfqpoint{3.645189in}{4.216546in}}%
\pgfpathlineto{\pgfqpoint{3.611957in}{4.383635in}}%
\pgfpathlineto{\pgfqpoint{3.578949in}{4.492097in}}%
\pgfpathclose%
\pgfusepath{fill}%
\end{pgfscope}%
\begin{pgfscope}%
\pgfpathrectangle{\pgfqpoint{1.020000in}{0.880000in}}{\pgfqpoint{6.160000in}{6.160000in}}%
\pgfusepath{clip}%
\pgfsetbuttcap%
\pgfsetroundjoin%
\definecolor{currentfill}{rgb}{0.527132,0.664700,0.989065}%
\pgfsetfillcolor{currentfill}%
\pgfsetlinewidth{0.000000pt}%
\definecolor{currentstroke}{rgb}{0.000000,0.000000,0.000000}%
\pgfsetstrokecolor{currentstroke}%
\pgfsetdash{}{0pt}%
\pgfpathmoveto{\pgfqpoint{4.735322in}{3.806519in}}%
\pgfpathlineto{\pgfqpoint{4.745319in}{3.886728in}}%
\pgfpathlineto{\pgfqpoint{4.754136in}{3.753450in}}%
\pgfpathlineto{\pgfqpoint{4.786359in}{3.781345in}}%
\pgfpathlineto{\pgfqpoint{4.817943in}{3.708196in}}%
\pgfpathlineto{\pgfqpoint{4.809392in}{3.879410in}}%
\pgfpathlineto{\pgfqpoint{4.799777in}{3.879526in}}%
\pgfpathlineto{\pgfqpoint{4.767463in}{3.828640in}}%
\pgfpathlineto{\pgfqpoint{4.735322in}{3.806519in}}%
\pgfpathclose%
\pgfusepath{fill}%
\end{pgfscope}%
\begin{pgfscope}%
\pgfpathrectangle{\pgfqpoint{1.020000in}{0.880000in}}{\pgfqpoint{6.160000in}{6.160000in}}%
\pgfusepath{clip}%
\pgfsetbuttcap%
\pgfsetroundjoin%
\definecolor{currentfill}{rgb}{0.505423,0.643995,0.983157}%
\pgfsetfillcolor{currentfill}%
\pgfsetlinewidth{0.000000pt}%
\definecolor{currentstroke}{rgb}{0.000000,0.000000,0.000000}%
\pgfsetstrokecolor{currentstroke}%
\pgfsetdash{}{0pt}%
\pgfpathmoveto{\pgfqpoint{4.947187in}{3.883665in}}%
\pgfpathlineto{\pgfqpoint{4.956994in}{3.887812in}}%
\pgfpathlineto{\pgfqpoint{4.965460in}{3.717422in}}%
\pgfpathlineto{\pgfqpoint{4.998083in}{3.794850in}}%
\pgfpathlineto{\pgfqpoint{5.029501in}{3.725662in}}%
\pgfpathlineto{\pgfqpoint{5.019609in}{3.719303in}}%
\pgfpathlineto{\pgfqpoint{5.010697in}{3.829367in}}%
\pgfpathlineto{\pgfqpoint{4.977884in}{3.720771in}}%
\pgfpathlineto{\pgfqpoint{4.947187in}{3.883665in}}%
\pgfpathclose%
\pgfusepath{fill}%
\end{pgfscope}%
\begin{pgfscope}%
\pgfpathrectangle{\pgfqpoint{1.020000in}{0.880000in}}{\pgfqpoint{6.160000in}{6.160000in}}%
\pgfusepath{clip}%
\pgfsetbuttcap%
\pgfsetroundjoin%
\definecolor{currentfill}{rgb}{0.473070,0.611077,0.970634}%
\pgfsetfillcolor{currentfill}%
\pgfsetlinewidth{0.000000pt}%
\definecolor{currentstroke}{rgb}{0.000000,0.000000,0.000000}%
\pgfsetstrokecolor{currentstroke}%
\pgfsetdash{}{0pt}%
\pgfpathmoveto{\pgfqpoint{5.368390in}{3.705710in}}%
\pgfpathlineto{\pgfqpoint{5.379098in}{3.748655in}}%
\pgfpathlineto{\pgfqpoint{5.388878in}{3.711957in}}%
\pgfpathlineto{\pgfqpoint{5.420137in}{3.660566in}}%
\pgfpathlineto{\pgfqpoint{5.452341in}{3.687853in}}%
\pgfpathlineto{\pgfqpoint{5.442123in}{3.692311in}}%
\pgfpathlineto{\pgfqpoint{5.433379in}{3.813554in}}%
\pgfpathlineto{\pgfqpoint{5.400872in}{3.759277in}}%
\pgfpathlineto{\pgfqpoint{5.368390in}{3.705710in}}%
\pgfpathclose%
\pgfusepath{fill}%
\end{pgfscope}%
\begin{pgfscope}%
\pgfpathrectangle{\pgfqpoint{1.020000in}{0.880000in}}{\pgfqpoint{6.160000in}{6.160000in}}%
\pgfusepath{clip}%
\pgfsetbuttcap%
\pgfsetroundjoin%
\definecolor{currentfill}{rgb}{0.554312,0.690097,0.995516}%
\pgfsetfillcolor{currentfill}%
\pgfsetlinewidth{0.000000pt}%
\definecolor{currentstroke}{rgb}{0.000000,0.000000,0.000000}%
\pgfsetstrokecolor{currentstroke}%
\pgfsetdash{}{0pt}%
\pgfpathmoveto{\pgfqpoint{4.524855in}{4.004198in}}%
\pgfpathlineto{\pgfqpoint{4.533739in}{3.871243in}}%
\pgfpathlineto{\pgfqpoint{4.543147in}{3.886209in}}%
\pgfpathlineto{\pgfqpoint{4.574895in}{3.783753in}}%
\pgfpathlineto{\pgfqpoint{4.607066in}{3.801232in}}%
\pgfpathlineto{\pgfqpoint{4.597716in}{3.815408in}}%
\pgfpathlineto{\pgfqpoint{4.588189in}{3.781453in}}%
\pgfpathlineto{\pgfqpoint{4.556848in}{3.968727in}}%
\pgfpathlineto{\pgfqpoint{4.524855in}{4.004198in}}%
\pgfpathclose%
\pgfusepath{fill}%
\end{pgfscope}%
\begin{pgfscope}%
\pgfpathrectangle{\pgfqpoint{1.020000in}{0.880000in}}{\pgfqpoint{6.160000in}{6.160000in}}%
\pgfusepath{clip}%
\pgfsetbuttcap%
\pgfsetroundjoin%
\definecolor{currentfill}{rgb}{0.728970,0.817464,0.973188}%
\pgfsetfillcolor{currentfill}%
\pgfsetlinewidth{0.000000pt}%
\definecolor{currentstroke}{rgb}{0.000000,0.000000,0.000000}%
\pgfsetstrokecolor{currentstroke}%
\pgfsetdash{}{0pt}%
\pgfpathmoveto{\pgfqpoint{3.872966in}{4.244932in}}%
\pgfpathlineto{\pgfqpoint{3.881910in}{4.155806in}}%
\pgfpathlineto{\pgfqpoint{3.890608in}{4.152267in}}%
\pgfpathlineto{\pgfqpoint{3.922889in}{4.167677in}}%
\pgfpathlineto{\pgfqpoint{3.955302in}{4.120739in}}%
\pgfpathlineto{\pgfqpoint{3.946199in}{4.273272in}}%
\pgfpathlineto{\pgfqpoint{3.937795in}{4.135247in}}%
\pgfpathlineto{\pgfqpoint{3.905639in}{4.105954in}}%
\pgfpathlineto{\pgfqpoint{3.872966in}{4.244932in}}%
\pgfpathclose%
\pgfusepath{fill}%
\end{pgfscope}%
\begin{pgfscope}%
\pgfpathrectangle{\pgfqpoint{1.020000in}{0.880000in}}{\pgfqpoint{6.160000in}{6.160000in}}%
\pgfusepath{clip}%
\pgfsetbuttcap%
\pgfsetroundjoin%
\definecolor{currentfill}{rgb}{0.576051,0.708780,0.997755}%
\pgfsetfillcolor{currentfill}%
\pgfsetlinewidth{0.000000pt}%
\definecolor{currentstroke}{rgb}{0.000000,0.000000,0.000000}%
\pgfsetstrokecolor{currentstroke}%
\pgfsetdash{}{0pt}%
\pgfpathmoveto{\pgfqpoint{4.313304in}{3.897504in}}%
\pgfpathlineto{\pgfqpoint{4.322548in}{3.987978in}}%
\pgfpathlineto{\pgfqpoint{4.331585in}{3.913732in}}%
\pgfpathlineto{\pgfqpoint{4.363682in}{3.852900in}}%
\pgfpathlineto{\pgfqpoint{4.395912in}{3.882231in}}%
\pgfpathlineto{\pgfqpoint{4.386553in}{3.802184in}}%
\pgfpathlineto{\pgfqpoint{4.377432in}{3.827749in}}%
\pgfpathlineto{\pgfqpoint{4.345701in}{4.056200in}}%
\pgfpathlineto{\pgfqpoint{4.313304in}{3.897504in}}%
\pgfpathclose%
\pgfusepath{fill}%
\end{pgfscope}%
\begin{pgfscope}%
\pgfpathrectangle{\pgfqpoint{1.020000in}{0.880000in}}{\pgfqpoint{6.160000in}{6.160000in}}%
\pgfusepath{clip}%
\pgfsetbuttcap%
\pgfsetroundjoin%
\definecolor{currentfill}{rgb}{0.688188,0.793178,0.988038}%
\pgfsetfillcolor{currentfill}%
\pgfsetlinewidth{0.000000pt}%
\definecolor{currentstroke}{rgb}{0.000000,0.000000,0.000000}%
\pgfsetstrokecolor{currentstroke}%
\pgfsetdash{}{0pt}%
\pgfpathmoveto{\pgfqpoint{4.019624in}{4.249239in}}%
\pgfpathlineto{\pgfqpoint{4.028522in}{4.207808in}}%
\pgfpathlineto{\pgfqpoint{4.037569in}{4.071324in}}%
\pgfpathlineto{\pgfqpoint{4.069921in}{4.009954in}}%
\pgfpathlineto{\pgfqpoint{4.102182in}{3.988317in}}%
\pgfpathlineto{\pgfqpoint{4.093288in}{3.991924in}}%
\pgfpathlineto{\pgfqpoint{4.084389in}{4.019109in}}%
\pgfpathlineto{\pgfqpoint{4.051961in}{4.204497in}}%
\pgfpathlineto{\pgfqpoint{4.019624in}{4.249239in}}%
\pgfpathclose%
\pgfusepath{fill}%
\end{pgfscope}%
\begin{pgfscope}%
\pgfpathrectangle{\pgfqpoint{1.020000in}{0.880000in}}{\pgfqpoint{6.160000in}{6.160000in}}%
\pgfusepath{clip}%
\pgfsetbuttcap%
\pgfsetroundjoin%
\definecolor{currentfill}{rgb}{0.430507,0.564883,0.948889}%
\pgfsetfillcolor{currentfill}%
\pgfsetlinewidth{0.000000pt}%
\definecolor{currentstroke}{rgb}{0.000000,0.000000,0.000000}%
\pgfsetstrokecolor{currentstroke}%
\pgfsetdash{}{0pt}%
\pgfpathmoveto{\pgfqpoint{5.515443in}{3.644271in}}%
\pgfpathlineto{\pgfqpoint{5.525829in}{3.646997in}}%
\pgfpathlineto{\pgfqpoint{5.535421in}{3.589180in}}%
\pgfpathlineto{\pgfqpoint{5.568181in}{3.655722in}}%
\pgfpathlineto{\pgfqpoint{5.559511in}{3.782033in}}%
\pgfpathlineto{\pgfqpoint{5.545842in}{3.541432in}}%
\pgfpathlineto{\pgfqpoint{5.515443in}{3.644271in}}%
\pgfpathclose%
\pgfusepath{fill}%
\end{pgfscope}%
\begin{pgfscope}%
\pgfpathrectangle{\pgfqpoint{1.020000in}{0.880000in}}{\pgfqpoint{6.160000in}{6.160000in}}%
\pgfusepath{clip}%
\pgfsetbuttcap%
\pgfsetroundjoin%
\definecolor{currentfill}{rgb}{0.651398,0.768121,0.995891}%
\pgfsetfillcolor{currentfill}%
\pgfsetlinewidth{0.000000pt}%
\definecolor{currentstroke}{rgb}{0.000000,0.000000,0.000000}%
\pgfsetstrokecolor{currentstroke}%
\pgfsetdash{}{0pt}%
\pgfpathmoveto{\pgfqpoint{4.166579in}{4.125321in}}%
\pgfpathlineto{\pgfqpoint{4.175569in}{4.020881in}}%
\pgfpathlineto{\pgfqpoint{4.184561in}{4.078023in}}%
\pgfpathlineto{\pgfqpoint{4.216815in}{4.027036in}}%
\pgfpathlineto{\pgfqpoint{4.248997in}{3.942521in}}%
\pgfpathlineto{\pgfqpoint{4.239976in}{3.983246in}}%
\pgfpathlineto{\pgfqpoint{4.230953in}{3.997763in}}%
\pgfpathlineto{\pgfqpoint{4.198793in}{4.041813in}}%
\pgfpathlineto{\pgfqpoint{4.166579in}{4.125321in}}%
\pgfpathclose%
\pgfusepath{fill}%
\end{pgfscope}%
\begin{pgfscope}%
\pgfpathrectangle{\pgfqpoint{1.020000in}{0.880000in}}{\pgfqpoint{6.160000in}{6.160000in}}%
\pgfusepath{clip}%
\pgfsetbuttcap%
\pgfsetroundjoin%
\definecolor{currentfill}{rgb}{0.521696,0.659599,0.987736}%
\pgfsetfillcolor{currentfill}%
\pgfsetlinewidth{0.000000pt}%
\definecolor{currentstroke}{rgb}{0.000000,0.000000,0.000000}%
\pgfsetstrokecolor{currentstroke}%
\pgfsetdash{}{0pt}%
\pgfpathmoveto{\pgfqpoint{5.157020in}{3.703507in}}%
\pgfpathlineto{\pgfqpoint{5.167731in}{3.778390in}}%
\pgfpathlineto{\pgfqpoint{5.178357in}{3.841180in}}%
\pgfpathlineto{\pgfqpoint{5.211343in}{3.939028in}}%
\pgfpathlineto{\pgfqpoint{5.241632in}{3.777104in}}%
\pgfpathlineto{\pgfqpoint{5.232660in}{3.881592in}}%
\pgfpathlineto{\pgfqpoint{5.221221in}{3.747663in}}%
\pgfpathlineto{\pgfqpoint{5.189517in}{3.764932in}}%
\pgfpathlineto{\pgfqpoint{5.157020in}{3.703507in}}%
\pgfpathclose%
\pgfusepath{fill}%
\end{pgfscope}%
\begin{pgfscope}%
\pgfpathrectangle{\pgfqpoint{1.020000in}{0.880000in}}{\pgfqpoint{6.160000in}{6.160000in}}%
\pgfusepath{clip}%
\pgfsetbuttcap%
\pgfsetroundjoin%
\definecolor{currentfill}{rgb}{0.510824,0.649397,0.985079}%
\pgfsetfillcolor{currentfill}%
\pgfsetlinewidth{0.000000pt}%
\definecolor{currentstroke}{rgb}{0.000000,0.000000,0.000000}%
\pgfsetstrokecolor{currentstroke}%
\pgfsetdash{}{0pt}%
\pgfpathmoveto{\pgfqpoint{5.094520in}{3.848645in}}%
\pgfpathlineto{\pgfqpoint{5.103632in}{3.759294in}}%
\pgfpathlineto{\pgfqpoint{5.113037in}{3.701296in}}%
\pgfpathlineto{\pgfqpoint{5.145221in}{3.722970in}}%
\pgfpathlineto{\pgfqpoint{5.178357in}{3.841180in}}%
\pgfpathlineto{\pgfqpoint{5.167731in}{3.778390in}}%
\pgfpathlineto{\pgfqpoint{5.157020in}{3.703507in}}%
\pgfpathlineto{\pgfqpoint{5.126871in}{3.889829in}}%
\pgfpathlineto{\pgfqpoint{5.094520in}{3.848645in}}%
\pgfpathclose%
\pgfusepath{fill}%
\end{pgfscope}%
\begin{pgfscope}%
\pgfpathrectangle{\pgfqpoint{1.020000in}{0.880000in}}{\pgfqpoint{6.160000in}{6.160000in}}%
\pgfusepath{clip}%
\pgfsetbuttcap%
\pgfsetroundjoin%
\definecolor{currentfill}{rgb}{0.532568,0.669801,0.990393}%
\pgfsetfillcolor{currentfill}%
\pgfsetlinewidth{0.000000pt}%
\definecolor{currentstroke}{rgb}{0.000000,0.000000,0.000000}%
\pgfsetstrokecolor{currentstroke}%
\pgfsetdash{}{0pt}%
\pgfpathmoveto{\pgfqpoint{4.882928in}{3.850844in}}%
\pgfpathlineto{\pgfqpoint{4.891503in}{3.689935in}}%
\pgfpathlineto{\pgfqpoint{4.901342in}{3.707096in}}%
\pgfpathlineto{\pgfqpoint{4.934588in}{3.869920in}}%
\pgfpathlineto{\pgfqpoint{4.965460in}{3.717422in}}%
\pgfpathlineto{\pgfqpoint{4.956994in}{3.887812in}}%
\pgfpathlineto{\pgfqpoint{4.947187in}{3.883665in}}%
\pgfpathlineto{\pgfqpoint{4.915569in}{3.936846in}}%
\pgfpathlineto{\pgfqpoint{4.882928in}{3.850844in}}%
\pgfpathclose%
\pgfusepath{fill}%
\end{pgfscope}%
\begin{pgfscope}%
\pgfpathrectangle{\pgfqpoint{1.020000in}{0.880000in}}{\pgfqpoint{6.160000in}{6.160000in}}%
\pgfusepath{clip}%
\pgfsetbuttcap%
\pgfsetroundjoin%
\definecolor{currentfill}{rgb}{0.570616,0.704109,0.997195}%
\pgfsetfillcolor{currentfill}%
\pgfsetlinewidth{0.000000pt}%
\definecolor{currentstroke}{rgb}{0.000000,0.000000,0.000000}%
\pgfsetstrokecolor{currentstroke}%
\pgfsetdash{}{0pt}%
\pgfpathmoveto{\pgfqpoint{4.460215in}{3.882756in}}%
\pgfpathlineto{\pgfqpoint{4.469845in}{4.007138in}}%
\pgfpathlineto{\pgfqpoint{4.478386in}{3.746725in}}%
\pgfpathlineto{\pgfqpoint{4.511225in}{3.961257in}}%
\pgfpathlineto{\pgfqpoint{4.543147in}{3.886209in}}%
\pgfpathlineto{\pgfqpoint{4.533739in}{3.871243in}}%
\pgfpathlineto{\pgfqpoint{4.524855in}{4.004198in}}%
\pgfpathlineto{\pgfqpoint{4.491912in}{3.744487in}}%
\pgfpathlineto{\pgfqpoint{4.460215in}{3.882756in}}%
\pgfpathclose%
\pgfusepath{fill}%
\end{pgfscope}%
\begin{pgfscope}%
\pgfpathrectangle{\pgfqpoint{1.020000in}{0.880000in}}{\pgfqpoint{6.160000in}{6.160000in}}%
\pgfusepath{clip}%
\pgfsetbuttcap%
\pgfsetroundjoin%
\definecolor{currentfill}{rgb}{0.791392,0.846750,0.936641}%
\pgfsetfillcolor{currentfill}%
\pgfsetlinewidth{0.000000pt}%
\definecolor{currentstroke}{rgb}{0.000000,0.000000,0.000000}%
\pgfsetstrokecolor{currentstroke}%
\pgfsetdash{}{0pt}%
\pgfpathmoveto{\pgfqpoint{3.661527in}{4.318193in}}%
\pgfpathlineto{\pgfqpoint{3.669462in}{4.422543in}}%
\pgfpathlineto{\pgfqpoint{3.678061in}{4.401020in}}%
\pgfpathlineto{\pgfqpoint{3.711689in}{4.149446in}}%
\pgfpathlineto{\pgfqpoint{3.743754in}{4.214931in}}%
\pgfpathlineto{\pgfqpoint{3.734592in}{4.355952in}}%
\pgfpathlineto{\pgfqpoint{3.727459in}{4.047561in}}%
\pgfpathlineto{\pgfqpoint{3.693583in}{4.381345in}}%
\pgfpathlineto{\pgfqpoint{3.661527in}{4.318193in}}%
\pgfpathclose%
\pgfusepath{fill}%
\end{pgfscope}%
\begin{pgfscope}%
\pgfpathrectangle{\pgfqpoint{1.020000in}{0.880000in}}{\pgfqpoint{6.160000in}{6.160000in}}%
\pgfusepath{clip}%
\pgfsetbuttcap%
\pgfsetroundjoin%
\definecolor{currentfill}{rgb}{0.505423,0.643995,0.983157}%
\pgfsetfillcolor{currentfill}%
\pgfsetlinewidth{0.000000pt}%
\definecolor{currentstroke}{rgb}{0.000000,0.000000,0.000000}%
\pgfsetstrokecolor{currentstroke}%
\pgfsetdash{}{0pt}%
\pgfpathmoveto{\pgfqpoint{4.817943in}{3.708196in}}%
\pgfpathlineto{\pgfqpoint{4.829039in}{3.935232in}}%
\pgfpathlineto{\pgfqpoint{4.837293in}{3.716552in}}%
\pgfpathlineto{\pgfqpoint{4.869848in}{3.787888in}}%
\pgfpathlineto{\pgfqpoint{4.901342in}{3.707096in}}%
\pgfpathlineto{\pgfqpoint{4.891503in}{3.689935in}}%
\pgfpathlineto{\pgfqpoint{4.882928in}{3.850844in}}%
\pgfpathlineto{\pgfqpoint{4.850181in}{3.743381in}}%
\pgfpathlineto{\pgfqpoint{4.817943in}{3.708196in}}%
\pgfpathclose%
\pgfusepath{fill}%
\end{pgfscope}%
\begin{pgfscope}%
\pgfpathrectangle{\pgfqpoint{1.020000in}{0.880000in}}{\pgfqpoint{6.160000in}{6.160000in}}%
\pgfusepath{clip}%
\pgfsetbuttcap%
\pgfsetroundjoin%
\definecolor{currentfill}{rgb}{0.597777,0.727330,0.999777}%
\pgfsetfillcolor{currentfill}%
\pgfsetlinewidth{0.000000pt}%
\definecolor{currentstroke}{rgb}{0.000000,0.000000,0.000000}%
\pgfsetstrokecolor{currentstroke}%
\pgfsetdash{}{0pt}%
\pgfpathmoveto{\pgfqpoint{4.248997in}{3.942521in}}%
\pgfpathlineto{\pgfqpoint{4.258049in}{3.928618in}}%
\pgfpathlineto{\pgfqpoint{4.267104in}{3.898046in}}%
\pgfpathlineto{\pgfqpoint{4.299207in}{3.781506in}}%
\pgfpathlineto{\pgfqpoint{4.331585in}{3.913732in}}%
\pgfpathlineto{\pgfqpoint{4.322548in}{3.987978in}}%
\pgfpathlineto{\pgfqpoint{4.313304in}{3.897504in}}%
\pgfpathlineto{\pgfqpoint{4.281296in}{4.052351in}}%
\pgfpathlineto{\pgfqpoint{4.248997in}{3.942521in}}%
\pgfpathclose%
\pgfusepath{fill}%
\end{pgfscope}%
\begin{pgfscope}%
\pgfpathrectangle{\pgfqpoint{1.020000in}{0.880000in}}{\pgfqpoint{6.160000in}{6.160000in}}%
\pgfusepath{clip}%
\pgfsetbuttcap%
\pgfsetroundjoin%
\definecolor{currentfill}{rgb}{0.559747,0.694768,0.996075}%
\pgfsetfillcolor{currentfill}%
\pgfsetlinewidth{0.000000pt}%
\definecolor{currentstroke}{rgb}{0.000000,0.000000,0.000000}%
\pgfsetstrokecolor{currentstroke}%
\pgfsetdash{}{0pt}%
\pgfpathmoveto{\pgfqpoint{4.671873in}{3.936408in}}%
\pgfpathlineto{\pgfqpoint{4.680880in}{3.836990in}}%
\pgfpathlineto{\pgfqpoint{4.689452in}{3.651104in}}%
\pgfpathlineto{\pgfqpoint{4.723708in}{4.060020in}}%
\pgfpathlineto{\pgfqpoint{4.754136in}{3.753450in}}%
\pgfpathlineto{\pgfqpoint{4.745319in}{3.886728in}}%
\pgfpathlineto{\pgfqpoint{4.735322in}{3.806519in}}%
\pgfpathlineto{\pgfqpoint{4.704079in}{3.958608in}}%
\pgfpathlineto{\pgfqpoint{4.671873in}{3.936408in}}%
\pgfpathclose%
\pgfusepath{fill}%
\end{pgfscope}%
\begin{pgfscope}%
\pgfpathrectangle{\pgfqpoint{1.020000in}{0.880000in}}{\pgfqpoint{6.160000in}{6.160000in}}%
\pgfusepath{clip}%
\pgfsetbuttcap%
\pgfsetroundjoin%
\definecolor{currentfill}{rgb}{0.500031,0.638508,0.981070}%
\pgfsetfillcolor{currentfill}%
\pgfsetlinewidth{0.000000pt}%
\definecolor{currentstroke}{rgb}{0.000000,0.000000,0.000000}%
\pgfsetstrokecolor{currentstroke}%
\pgfsetdash{}{0pt}%
\pgfpathmoveto{\pgfqpoint{5.305689in}{3.796789in}}%
\pgfpathlineto{\pgfqpoint{5.315683in}{3.783342in}}%
\pgfpathlineto{\pgfqpoint{5.326733in}{3.861050in}}%
\pgfpathlineto{\pgfqpoint{5.356276in}{3.653049in}}%
\pgfpathlineto{\pgfqpoint{5.388878in}{3.711957in}}%
\pgfpathlineto{\pgfqpoint{5.379098in}{3.748655in}}%
\pgfpathlineto{\pgfqpoint{5.368390in}{3.705710in}}%
\pgfpathlineto{\pgfqpoint{5.338101in}{3.841476in}}%
\pgfpathlineto{\pgfqpoint{5.305689in}{3.796789in}}%
\pgfpathclose%
\pgfusepath{fill}%
\end{pgfscope}%
\begin{pgfscope}%
\pgfpathrectangle{\pgfqpoint{1.020000in}{0.880000in}}{\pgfqpoint{6.160000in}{6.160000in}}%
\pgfusepath{clip}%
\pgfsetbuttcap%
\pgfsetroundjoin%
\definecolor{currentfill}{rgb}{0.451739,0.588181,0.960201}%
\pgfsetfillcolor{currentfill}%
\pgfsetlinewidth{0.000000pt}%
\definecolor{currentstroke}{rgb}{0.000000,0.000000,0.000000}%
\pgfsetstrokecolor{currentstroke}%
\pgfsetdash{}{0pt}%
\pgfpathmoveto{\pgfqpoint{5.452341in}{3.687853in}}%
\pgfpathlineto{\pgfqpoint{5.463037in}{3.719211in}}%
\pgfpathlineto{\pgfqpoint{5.472290in}{3.636285in}}%
\pgfpathlineto{\pgfqpoint{5.506034in}{3.776746in}}%
\pgfpathlineto{\pgfqpoint{5.535421in}{3.589180in}}%
\pgfpathlineto{\pgfqpoint{5.525829in}{3.646997in}}%
\pgfpathlineto{\pgfqpoint{5.515443in}{3.644271in}}%
\pgfpathlineto{\pgfqpoint{5.484672in}{3.724948in}}%
\pgfpathlineto{\pgfqpoint{5.452341in}{3.687853in}}%
\pgfpathclose%
\pgfusepath{fill}%
\end{pgfscope}%
\begin{pgfscope}%
\pgfpathrectangle{\pgfqpoint{1.020000in}{0.880000in}}{\pgfqpoint{6.160000in}{6.160000in}}%
\pgfusepath{clip}%
\pgfsetbuttcap%
\pgfsetroundjoin%
\definecolor{currentfill}{rgb}{0.532568,0.669801,0.990393}%
\pgfsetfillcolor{currentfill}%
\pgfsetlinewidth{0.000000pt}%
\definecolor{currentstroke}{rgb}{0.000000,0.000000,0.000000}%
\pgfsetstrokecolor{currentstroke}%
\pgfsetdash{}{0pt}%
\pgfpathmoveto{\pgfqpoint{4.607066in}{3.801232in}}%
\pgfpathlineto{\pgfqpoint{4.616675in}{3.843383in}}%
\pgfpathlineto{\pgfqpoint{4.626325in}{3.888454in}}%
\pgfpathlineto{\pgfqpoint{4.657735in}{3.727468in}}%
\pgfpathlineto{\pgfqpoint{4.689452in}{3.651104in}}%
\pgfpathlineto{\pgfqpoint{4.680880in}{3.836990in}}%
\pgfpathlineto{\pgfqpoint{4.671873in}{3.936408in}}%
\pgfpathlineto{\pgfqpoint{4.639433in}{3.863522in}}%
\pgfpathlineto{\pgfqpoint{4.607066in}{3.801232in}}%
\pgfpathclose%
\pgfusepath{fill}%
\end{pgfscope}%
\begin{pgfscope}%
\pgfpathrectangle{\pgfqpoint{1.020000in}{0.880000in}}{\pgfqpoint{6.160000in}{6.160000in}}%
\pgfusepath{clip}%
\pgfsetbuttcap%
\pgfsetroundjoin%
\definecolor{currentfill}{rgb}{0.494638,0.633022,0.978983}%
\pgfsetfillcolor{currentfill}%
\pgfsetlinewidth{0.000000pt}%
\definecolor{currentstroke}{rgb}{0.000000,0.000000,0.000000}%
\pgfsetstrokecolor{currentstroke}%
\pgfsetdash{}{0pt}%
\pgfpathmoveto{\pgfqpoint{5.029501in}{3.725662in}}%
\pgfpathlineto{\pgfqpoint{5.039146in}{3.700648in}}%
\pgfpathlineto{\pgfqpoint{5.048954in}{3.692556in}}%
\pgfpathlineto{\pgfqpoint{5.081813in}{3.787451in}}%
\pgfpathlineto{\pgfqpoint{5.113037in}{3.701296in}}%
\pgfpathlineto{\pgfqpoint{5.103632in}{3.759294in}}%
\pgfpathlineto{\pgfqpoint{5.094520in}{3.848645in}}%
\pgfpathlineto{\pgfqpoint{5.062111in}{3.799817in}}%
\pgfpathlineto{\pgfqpoint{5.029501in}{3.725662in}}%
\pgfpathclose%
\pgfusepath{fill}%
\end{pgfscope}%
\begin{pgfscope}%
\pgfpathrectangle{\pgfqpoint{1.020000in}{0.880000in}}{\pgfqpoint{6.160000in}{6.160000in}}%
\pgfusepath{clip}%
\pgfsetbuttcap%
\pgfsetroundjoin%
\definecolor{currentfill}{rgb}{0.758539,0.832787,0.958408}%
\pgfsetfillcolor{currentfill}%
\pgfsetlinewidth{0.000000pt}%
\definecolor{currentstroke}{rgb}{0.000000,0.000000,0.000000}%
\pgfsetstrokecolor{currentstroke}%
\pgfsetdash{}{0pt}%
\pgfpathmoveto{\pgfqpoint{3.808170in}{4.288336in}}%
\pgfpathlineto{\pgfqpoint{3.816653in}{4.326237in}}%
\pgfpathlineto{\pgfqpoint{3.825447in}{4.282882in}}%
\pgfpathlineto{\pgfqpoint{3.858101in}{4.204083in}}%
\pgfpathlineto{\pgfqpoint{3.890608in}{4.152267in}}%
\pgfpathlineto{\pgfqpoint{3.881910in}{4.155806in}}%
\pgfpathlineto{\pgfqpoint{3.872966in}{4.244932in}}%
\pgfpathlineto{\pgfqpoint{3.841204in}{4.084193in}}%
\pgfpathlineto{\pgfqpoint{3.808170in}{4.288336in}}%
\pgfpathclose%
\pgfusepath{fill}%
\end{pgfscope}%
\begin{pgfscope}%
\pgfpathrectangle{\pgfqpoint{1.020000in}{0.880000in}}{\pgfqpoint{6.160000in}{6.160000in}}%
\pgfusepath{clip}%
\pgfsetbuttcap%
\pgfsetroundjoin%
\definecolor{currentfill}{rgb}{0.887752,0.854040,0.834671}%
\pgfsetfillcolor{currentfill}%
\pgfsetlinewidth{0.000000pt}%
\definecolor{currentstroke}{rgb}{0.000000,0.000000,0.000000}%
\pgfsetstrokecolor{currentstroke}%
\pgfsetdash{}{0pt}%
\pgfpathmoveto{\pgfqpoint{3.366814in}{4.620224in}}%
\pgfpathlineto{\pgfqpoint{3.374551in}{4.675389in}}%
\pgfpathlineto{\pgfqpoint{3.384145in}{4.499052in}}%
\pgfpathlineto{\pgfqpoint{3.417223in}{4.423616in}}%
\pgfpathlineto{\pgfqpoint{3.449037in}{4.513089in}}%
\pgfpathlineto{\pgfqpoint{3.441637in}{4.397801in}}%
\pgfpathlineto{\pgfqpoint{3.433301in}{4.413210in}}%
\pgfpathlineto{\pgfqpoint{3.400892in}{4.414510in}}%
\pgfpathlineto{\pgfqpoint{3.366814in}{4.620224in}}%
\pgfpathclose%
\pgfusepath{fill}%
\end{pgfscope}%
\begin{pgfscope}%
\pgfpathrectangle{\pgfqpoint{1.020000in}{0.880000in}}{\pgfqpoint{6.160000in}{6.160000in}}%
\pgfusepath{clip}%
\pgfsetbuttcap%
\pgfsetroundjoin%
\definecolor{currentfill}{rgb}{0.851372,0.863125,0.881064}%
\pgfsetfillcolor{currentfill}%
\pgfsetlinewidth{0.000000pt}%
\definecolor{currentstroke}{rgb}{0.000000,0.000000,0.000000}%
\pgfsetstrokecolor{currentstroke}%
\pgfsetdash{}{0pt}%
\pgfpathmoveto{\pgfqpoint{3.515539in}{4.275709in}}%
\pgfpathlineto{\pgfqpoint{3.521967in}{4.562881in}}%
\pgfpathlineto{\pgfqpoint{3.530505in}{4.533283in}}%
\pgfpathlineto{\pgfqpoint{3.563562in}{4.443337in}}%
\pgfpathlineto{\pgfqpoint{3.596514in}{4.358261in}}%
\pgfpathlineto{\pgfqpoint{3.588761in}{4.250070in}}%
\pgfpathlineto{\pgfqpoint{3.578949in}{4.492097in}}%
\pgfpathlineto{\pgfqpoint{3.547286in}{4.373117in}}%
\pgfpathlineto{\pgfqpoint{3.515539in}{4.275709in}}%
\pgfpathclose%
\pgfusepath{fill}%
\end{pgfscope}%
\begin{pgfscope}%
\pgfpathrectangle{\pgfqpoint{1.020000in}{0.880000in}}{\pgfqpoint{6.160000in}{6.160000in}}%
\pgfusepath{clip}%
\pgfsetbuttcap%
\pgfsetroundjoin%
\definecolor{currentfill}{rgb}{0.677823,0.786546,0.991005}%
\pgfsetfillcolor{currentfill}%
\pgfsetlinewidth{0.000000pt}%
\definecolor{currentstroke}{rgb}{0.000000,0.000000,0.000000}%
\pgfsetstrokecolor{currentstroke}%
\pgfsetdash{}{0pt}%
\pgfpathmoveto{\pgfqpoint{4.102182in}{3.988317in}}%
\pgfpathlineto{\pgfqpoint{4.111070in}{4.018278in}}%
\pgfpathlineto{\pgfqpoint{4.120006in}{4.007429in}}%
\pgfpathlineto{\pgfqpoint{4.152265in}{4.096674in}}%
\pgfpathlineto{\pgfqpoint{4.184561in}{4.078023in}}%
\pgfpathlineto{\pgfqpoint{4.175569in}{4.020881in}}%
\pgfpathlineto{\pgfqpoint{4.166579in}{4.125321in}}%
\pgfpathlineto{\pgfqpoint{4.134306in}{4.185454in}}%
\pgfpathlineto{\pgfqpoint{4.102182in}{3.988317in}}%
\pgfpathclose%
\pgfusepath{fill}%
\end{pgfscope}%
\begin{pgfscope}%
\pgfpathrectangle{\pgfqpoint{1.020000in}{0.880000in}}{\pgfqpoint{6.160000in}{6.160000in}}%
\pgfusepath{clip}%
\pgfsetbuttcap%
\pgfsetroundjoin%
\definecolor{currentfill}{rgb}{0.698454,0.799450,0.984577}%
\pgfsetfillcolor{currentfill}%
\pgfsetlinewidth{0.000000pt}%
\definecolor{currentstroke}{rgb}{0.000000,0.000000,0.000000}%
\pgfsetstrokecolor{currentstroke}%
\pgfsetdash{}{0pt}%
\pgfpathmoveto{\pgfqpoint{3.890608in}{4.152267in}}%
\pgfpathlineto{\pgfqpoint{3.899223in}{4.183991in}}%
\pgfpathlineto{\pgfqpoint{3.908119in}{4.121301in}}%
\pgfpathlineto{\pgfqpoint{3.941284in}{3.792548in}}%
\pgfpathlineto{\pgfqpoint{3.972696in}{4.192491in}}%
\pgfpathlineto{\pgfqpoint{3.964108in}{4.101303in}}%
\pgfpathlineto{\pgfqpoint{3.955302in}{4.120739in}}%
\pgfpathlineto{\pgfqpoint{3.922889in}{4.167677in}}%
\pgfpathlineto{\pgfqpoint{3.890608in}{4.152267in}}%
\pgfpathclose%
\pgfusepath{fill}%
\end{pgfscope}%
\begin{pgfscope}%
\pgfpathrectangle{\pgfqpoint{1.020000in}{0.880000in}}{\pgfqpoint{6.160000in}{6.160000in}}%
\pgfusepath{clip}%
\pgfsetbuttcap%
\pgfsetroundjoin%
\definecolor{currentfill}{rgb}{0.733898,0.820018,0.970724}%
\pgfsetfillcolor{currentfill}%
\pgfsetlinewidth{0.000000pt}%
\definecolor{currentstroke}{rgb}{0.000000,0.000000,0.000000}%
\pgfsetstrokecolor{currentstroke}%
\pgfsetdash{}{0pt}%
\pgfpathmoveto{\pgfqpoint{3.955302in}{4.120739in}}%
\pgfpathlineto{\pgfqpoint{3.964108in}{4.101303in}}%
\pgfpathlineto{\pgfqpoint{3.972696in}{4.192491in}}%
\pgfpathlineto{\pgfqpoint{4.004968in}{4.241227in}}%
\pgfpathlineto{\pgfqpoint{4.037569in}{4.071324in}}%
\pgfpathlineto{\pgfqpoint{4.028522in}{4.207808in}}%
\pgfpathlineto{\pgfqpoint{4.019624in}{4.249239in}}%
\pgfpathlineto{\pgfqpoint{3.987483in}{4.168935in}}%
\pgfpathlineto{\pgfqpoint{3.955302in}{4.120739in}}%
\pgfpathclose%
\pgfusepath{fill}%
\end{pgfscope}%
\begin{pgfscope}%
\pgfpathrectangle{\pgfqpoint{1.020000in}{0.880000in}}{\pgfqpoint{6.160000in}{6.160000in}}%
\pgfusepath{clip}%
\pgfsetbuttcap%
\pgfsetroundjoin%
\definecolor{currentfill}{rgb}{0.451739,0.588181,0.960201}%
\pgfsetfillcolor{currentfill}%
\pgfsetlinewidth{0.000000pt}%
\definecolor{currentstroke}{rgb}{0.000000,0.000000,0.000000}%
\pgfsetstrokecolor{currentstroke}%
\pgfsetdash{}{0pt}%
\pgfpathmoveto{\pgfqpoint{5.388878in}{3.711957in}}%
\pgfpathlineto{\pgfqpoint{5.398475in}{3.659105in}}%
\pgfpathlineto{\pgfqpoint{5.409280in}{3.704180in}}%
\pgfpathlineto{\pgfqpoint{5.440080in}{3.612467in}}%
\pgfpathlineto{\pgfqpoint{5.472290in}{3.636285in}}%
\pgfpathlineto{\pgfqpoint{5.463037in}{3.719211in}}%
\pgfpathlineto{\pgfqpoint{5.452341in}{3.687853in}}%
\pgfpathlineto{\pgfqpoint{5.420137in}{3.660566in}}%
\pgfpathlineto{\pgfqpoint{5.388878in}{3.711957in}}%
\pgfpathclose%
\pgfusepath{fill}%
\end{pgfscope}%
\begin{pgfscope}%
\pgfpathrectangle{\pgfqpoint{1.020000in}{0.880000in}}{\pgfqpoint{6.160000in}{6.160000in}}%
\pgfusepath{clip}%
\pgfsetbuttcap%
\pgfsetroundjoin%
\definecolor{currentfill}{rgb}{0.510824,0.649397,0.985079}%
\pgfsetfillcolor{currentfill}%
\pgfsetlinewidth{0.000000pt}%
\definecolor{currentstroke}{rgb}{0.000000,0.000000,0.000000}%
\pgfsetstrokecolor{currentstroke}%
\pgfsetdash{}{0pt}%
\pgfpathmoveto{\pgfqpoint{5.241632in}{3.777104in}}%
\pgfpathlineto{\pgfqpoint{5.250082in}{3.623585in}}%
\pgfpathlineto{\pgfqpoint{5.261942in}{3.788352in}}%
\pgfpathlineto{\pgfqpoint{5.293762in}{3.772856in}}%
\pgfpathlineto{\pgfqpoint{5.326733in}{3.861050in}}%
\pgfpathlineto{\pgfqpoint{5.315683in}{3.783342in}}%
\pgfpathlineto{\pgfqpoint{5.305689in}{3.796789in}}%
\pgfpathlineto{\pgfqpoint{5.273273in}{3.750743in}}%
\pgfpathlineto{\pgfqpoint{5.241632in}{3.777104in}}%
\pgfpathclose%
\pgfusepath{fill}%
\end{pgfscope}%
\begin{pgfscope}%
\pgfpathrectangle{\pgfqpoint{1.020000in}{0.880000in}}{\pgfqpoint{6.160000in}{6.160000in}}%
\pgfusepath{clip}%
\pgfsetbuttcap%
\pgfsetroundjoin%
\definecolor{currentfill}{rgb}{0.505423,0.643995,0.983157}%
\pgfsetfillcolor{currentfill}%
\pgfsetlinewidth{0.000000pt}%
\definecolor{currentstroke}{rgb}{0.000000,0.000000,0.000000}%
\pgfsetstrokecolor{currentstroke}%
\pgfsetdash{}{0pt}%
\pgfpathmoveto{\pgfqpoint{4.965460in}{3.717422in}}%
\pgfpathlineto{\pgfqpoint{4.975198in}{3.709700in}}%
\pgfpathlineto{\pgfqpoint{4.986953in}{3.951998in}}%
\pgfpathlineto{\pgfqpoint{5.017951in}{3.815934in}}%
\pgfpathlineto{\pgfqpoint{5.048954in}{3.692556in}}%
\pgfpathlineto{\pgfqpoint{5.039146in}{3.700648in}}%
\pgfpathlineto{\pgfqpoint{5.029501in}{3.725662in}}%
\pgfpathlineto{\pgfqpoint{4.998083in}{3.794850in}}%
\pgfpathlineto{\pgfqpoint{4.965460in}{3.717422in}}%
\pgfpathclose%
\pgfusepath{fill}%
\end{pgfscope}%
\begin{pgfscope}%
\pgfpathrectangle{\pgfqpoint{1.020000in}{0.880000in}}{\pgfqpoint{6.160000in}{6.160000in}}%
\pgfusepath{clip}%
\pgfsetbuttcap%
\pgfsetroundjoin%
\definecolor{currentfill}{rgb}{0.672538,0.782861,0.991982}%
\pgfsetfillcolor{currentfill}%
\pgfsetlinewidth{0.000000pt}%
\definecolor{currentstroke}{rgb}{0.000000,0.000000,0.000000}%
\pgfsetstrokecolor{currentstroke}%
\pgfsetdash{}{0pt}%
\pgfpathmoveto{\pgfqpoint{4.037569in}{4.071324in}}%
\pgfpathlineto{\pgfqpoint{4.046571in}{3.963309in}}%
\pgfpathlineto{\pgfqpoint{4.055153in}{4.185756in}}%
\pgfpathlineto{\pgfqpoint{4.087557in}{4.159525in}}%
\pgfpathlineto{\pgfqpoint{4.120006in}{4.007429in}}%
\pgfpathlineto{\pgfqpoint{4.111070in}{4.018278in}}%
\pgfpathlineto{\pgfqpoint{4.102182in}{3.988317in}}%
\pgfpathlineto{\pgfqpoint{4.069921in}{4.009954in}}%
\pgfpathlineto{\pgfqpoint{4.037569in}{4.071324in}}%
\pgfpathclose%
\pgfusepath{fill}%
\end{pgfscope}%
\begin{pgfscope}%
\pgfpathrectangle{\pgfqpoint{1.020000in}{0.880000in}}{\pgfqpoint{6.160000in}{6.160000in}}%
\pgfusepath{clip}%
\pgfsetbuttcap%
\pgfsetroundjoin%
\definecolor{currentfill}{rgb}{0.603162,0.731527,0.999565}%
\pgfsetfillcolor{currentfill}%
\pgfsetlinewidth{0.000000pt}%
\definecolor{currentstroke}{rgb}{0.000000,0.000000,0.000000}%
\pgfsetstrokecolor{currentstroke}%
\pgfsetdash{}{0pt}%
\pgfpathmoveto{\pgfqpoint{4.395912in}{3.882231in}}%
\pgfpathlineto{\pgfqpoint{4.405184in}{3.907795in}}%
\pgfpathlineto{\pgfqpoint{4.414593in}{3.982355in}}%
\pgfpathlineto{\pgfqpoint{4.446586in}{3.883596in}}%
\pgfpathlineto{\pgfqpoint{4.478386in}{3.746725in}}%
\pgfpathlineto{\pgfqpoint{4.469845in}{4.007138in}}%
\pgfpathlineto{\pgfqpoint{4.460215in}{3.882756in}}%
\pgfpathlineto{\pgfqpoint{4.428655in}{4.121396in}}%
\pgfpathlineto{\pgfqpoint{4.395912in}{3.882231in}}%
\pgfpathclose%
\pgfusepath{fill}%
\end{pgfscope}%
\begin{pgfscope}%
\pgfpathrectangle{\pgfqpoint{1.020000in}{0.880000in}}{\pgfqpoint{6.160000in}{6.160000in}}%
\pgfusepath{clip}%
\pgfsetbuttcap%
\pgfsetroundjoin%
\definecolor{currentfill}{rgb}{0.532568,0.669801,0.990393}%
\pgfsetfillcolor{currentfill}%
\pgfsetlinewidth{0.000000pt}%
\definecolor{currentstroke}{rgb}{0.000000,0.000000,0.000000}%
\pgfsetstrokecolor{currentstroke}%
\pgfsetdash{}{0pt}%
\pgfpathmoveto{\pgfqpoint{4.754136in}{3.753450in}}%
\pgfpathlineto{\pgfqpoint{4.764155in}{3.829094in}}%
\pgfpathlineto{\pgfqpoint{4.773614in}{3.803513in}}%
\pgfpathlineto{\pgfqpoint{4.806326in}{3.897080in}}%
\pgfpathlineto{\pgfqpoint{4.837293in}{3.716552in}}%
\pgfpathlineto{\pgfqpoint{4.829039in}{3.935232in}}%
\pgfpathlineto{\pgfqpoint{4.817943in}{3.708196in}}%
\pgfpathlineto{\pgfqpoint{4.786359in}{3.781345in}}%
\pgfpathlineto{\pgfqpoint{4.754136in}{3.753450in}}%
\pgfpathclose%
\pgfusepath{fill}%
\end{pgfscope}%
\begin{pgfscope}%
\pgfpathrectangle{\pgfqpoint{1.020000in}{0.880000in}}{\pgfqpoint{6.160000in}{6.160000in}}%
\pgfusepath{clip}%
\pgfsetbuttcap%
\pgfsetroundjoin%
\definecolor{currentfill}{rgb}{0.777378,0.840921,0.946149}%
\pgfsetfillcolor{currentfill}%
\pgfsetlinewidth{0.000000pt}%
\definecolor{currentstroke}{rgb}{0.000000,0.000000,0.000000}%
\pgfsetstrokecolor{currentstroke}%
\pgfsetdash{}{0pt}%
\pgfpathmoveto{\pgfqpoint{3.743754in}{4.214931in}}%
\pgfpathlineto{\pgfqpoint{3.752559in}{4.156054in}}%
\pgfpathlineto{\pgfqpoint{3.760469in}{4.312514in}}%
\pgfpathlineto{\pgfqpoint{3.793364in}{4.196356in}}%
\pgfpathlineto{\pgfqpoint{3.825447in}{4.282882in}}%
\pgfpathlineto{\pgfqpoint{3.816653in}{4.326237in}}%
\pgfpathlineto{\pgfqpoint{3.808170in}{4.288336in}}%
\pgfpathlineto{\pgfqpoint{3.776158in}{4.202417in}}%
\pgfpathlineto{\pgfqpoint{3.743754in}{4.214931in}}%
\pgfpathclose%
\pgfusepath{fill}%
\end{pgfscope}%
\begin{pgfscope}%
\pgfpathrectangle{\pgfqpoint{1.020000in}{0.880000in}}{\pgfqpoint{6.160000in}{6.160000in}}%
\pgfusepath{clip}%
\pgfsetbuttcap%
\pgfsetroundjoin%
\definecolor{currentfill}{rgb}{0.826784,0.858205,0.906953}%
\pgfsetfillcolor{currentfill}%
\pgfsetlinewidth{0.000000pt}%
\definecolor{currentstroke}{rgb}{0.000000,0.000000,0.000000}%
\pgfsetstrokecolor{currentstroke}%
\pgfsetdash{}{0pt}%
\pgfpathmoveto{\pgfqpoint{3.596514in}{4.358261in}}%
\pgfpathlineto{\pgfqpoint{3.604988in}{4.346764in}}%
\pgfpathlineto{\pgfqpoint{3.613478in}{4.335090in}}%
\pgfpathlineto{\pgfqpoint{3.645663in}{4.387828in}}%
\pgfpathlineto{\pgfqpoint{3.678061in}{4.401020in}}%
\pgfpathlineto{\pgfqpoint{3.669462in}{4.422543in}}%
\pgfpathlineto{\pgfqpoint{3.661527in}{4.318193in}}%
\pgfpathlineto{\pgfqpoint{3.629644in}{4.227418in}}%
\pgfpathlineto{\pgfqpoint{3.596514in}{4.358261in}}%
\pgfpathclose%
\pgfusepath{fill}%
\end{pgfscope}%
\begin{pgfscope}%
\pgfpathrectangle{\pgfqpoint{1.020000in}{0.880000in}}{\pgfqpoint{6.160000in}{6.160000in}}%
\pgfusepath{clip}%
\pgfsetbuttcap%
\pgfsetroundjoin%
\definecolor{currentfill}{rgb}{0.570616,0.704109,0.997195}%
\pgfsetfillcolor{currentfill}%
\pgfsetlinewidth{0.000000pt}%
\definecolor{currentstroke}{rgb}{0.000000,0.000000,0.000000}%
\pgfsetstrokecolor{currentstroke}%
\pgfsetdash{}{0pt}%
\pgfpathmoveto{\pgfqpoint{4.543147in}{3.886209in}}%
\pgfpathlineto{\pgfqpoint{4.552536in}{3.890071in}}%
\pgfpathlineto{\pgfqpoint{4.562433in}{4.023367in}}%
\pgfpathlineto{\pgfqpoint{4.593803in}{3.807655in}}%
\pgfpathlineto{\pgfqpoint{4.626325in}{3.888454in}}%
\pgfpathlineto{\pgfqpoint{4.616675in}{3.843383in}}%
\pgfpathlineto{\pgfqpoint{4.607066in}{3.801232in}}%
\pgfpathlineto{\pgfqpoint{4.574895in}{3.783753in}}%
\pgfpathlineto{\pgfqpoint{4.543147in}{3.886209in}}%
\pgfpathclose%
\pgfusepath{fill}%
\end{pgfscope}%
\begin{pgfscope}%
\pgfpathrectangle{\pgfqpoint{1.020000in}{0.880000in}}{\pgfqpoint{6.160000in}{6.160000in}}%
\pgfusepath{clip}%
\pgfsetbuttcap%
\pgfsetroundjoin%
\definecolor{currentfill}{rgb}{0.640828,0.760752,0.997846}%
\pgfsetfillcolor{currentfill}%
\pgfsetlinewidth{0.000000pt}%
\definecolor{currentstroke}{rgb}{0.000000,0.000000,0.000000}%
\pgfsetstrokecolor{currentstroke}%
\pgfsetdash{}{0pt}%
\pgfpathmoveto{\pgfqpoint{4.184561in}{4.078023in}}%
\pgfpathlineto{\pgfqpoint{4.193571in}{4.043501in}}%
\pgfpathlineto{\pgfqpoint{4.202595in}{4.014945in}}%
\pgfpathlineto{\pgfqpoint{4.234914in}{4.027165in}}%
\pgfpathlineto{\pgfqpoint{4.267104in}{3.898046in}}%
\pgfpathlineto{\pgfqpoint{4.258049in}{3.928618in}}%
\pgfpathlineto{\pgfqpoint{4.248997in}{3.942521in}}%
\pgfpathlineto{\pgfqpoint{4.216815in}{4.027036in}}%
\pgfpathlineto{\pgfqpoint{4.184561in}{4.078023in}}%
\pgfpathclose%
\pgfusepath{fill}%
\end{pgfscope}%
\begin{pgfscope}%
\pgfpathrectangle{\pgfqpoint{1.020000in}{0.880000in}}{\pgfqpoint{6.160000in}{6.160000in}}%
\pgfusepath{clip}%
\pgfsetbuttcap%
\pgfsetroundjoin%
\definecolor{currentfill}{rgb}{0.909460,0.839386,0.800331}%
\pgfsetfillcolor{currentfill}%
\pgfsetlinewidth{0.000000pt}%
\definecolor{currentstroke}{rgb}{0.000000,0.000000,0.000000}%
\pgfsetstrokecolor{currentstroke}%
\pgfsetdash{}{0pt}%
\pgfpathmoveto{\pgfqpoint{3.303403in}{4.435175in}}%
\pgfpathlineto{\pgfqpoint{3.310239in}{4.581828in}}%
\pgfpathlineto{\pgfqpoint{3.318483in}{4.568531in}}%
\pgfpathlineto{\pgfqpoint{3.351947in}{4.459258in}}%
\pgfpathlineto{\pgfqpoint{3.384145in}{4.499052in}}%
\pgfpathlineto{\pgfqpoint{3.374551in}{4.675389in}}%
\pgfpathlineto{\pgfqpoint{3.366814in}{4.620224in}}%
\pgfpathlineto{\pgfqpoint{3.335463in}{4.482947in}}%
\pgfpathlineto{\pgfqpoint{3.303403in}{4.435175in}}%
\pgfpathclose%
\pgfusepath{fill}%
\end{pgfscope}%
\begin{pgfscope}%
\pgfpathrectangle{\pgfqpoint{1.020000in}{0.880000in}}{\pgfqpoint{6.160000in}{6.160000in}}%
\pgfusepath{clip}%
\pgfsetbuttcap%
\pgfsetroundjoin%
\definecolor{currentfill}{rgb}{0.435815,0.570707,0.951717}%
\pgfsetfillcolor{currentfill}%
\pgfsetlinewidth{0.000000pt}%
\definecolor{currentstroke}{rgb}{0.000000,0.000000,0.000000}%
\pgfsetstrokecolor{currentstroke}%
\pgfsetdash{}{0pt}%
\pgfpathmoveto{\pgfqpoint{5.535421in}{3.589180in}}%
\pgfpathlineto{\pgfqpoint{5.545944in}{3.599239in}}%
\pgfpathlineto{\pgfqpoint{5.558342in}{3.744288in}}%
\pgfpathlineto{\pgfqpoint{5.587900in}{3.573778in}}%
\pgfpathlineto{\pgfqpoint{5.578461in}{3.645378in}}%
\pgfpathlineto{\pgfqpoint{5.568181in}{3.655722in}}%
\pgfpathlineto{\pgfqpoint{5.535421in}{3.589180in}}%
\pgfpathclose%
\pgfusepath{fill}%
\end{pgfscope}%
\begin{pgfscope}%
\pgfpathrectangle{\pgfqpoint{1.020000in}{0.880000in}}{\pgfqpoint{6.160000in}{6.160000in}}%
\pgfusepath{clip}%
\pgfsetbuttcap%
\pgfsetroundjoin%
\definecolor{currentfill}{rgb}{0.875557,0.860242,0.851430}%
\pgfsetfillcolor{currentfill}%
\pgfsetlinewidth{0.000000pt}%
\definecolor{currentstroke}{rgb}{0.000000,0.000000,0.000000}%
\pgfsetstrokecolor{currentstroke}%
\pgfsetdash{}{0pt}%
\pgfpathmoveto{\pgfqpoint{3.449037in}{4.513089in}}%
\pgfpathlineto{\pgfqpoint{3.456519in}{4.622103in}}%
\pgfpathlineto{\pgfqpoint{3.466667in}{4.360324in}}%
\pgfpathlineto{\pgfqpoint{3.498652in}{4.434328in}}%
\pgfpathlineto{\pgfqpoint{3.530505in}{4.533283in}}%
\pgfpathlineto{\pgfqpoint{3.521967in}{4.562881in}}%
\pgfpathlineto{\pgfqpoint{3.515539in}{4.275709in}}%
\pgfpathlineto{\pgfqpoint{3.482619in}{4.353482in}}%
\pgfpathlineto{\pgfqpoint{3.449037in}{4.513089in}}%
\pgfpathclose%
\pgfusepath{fill}%
\end{pgfscope}%
\begin{pgfscope}%
\pgfpathrectangle{\pgfqpoint{1.020000in}{0.880000in}}{\pgfqpoint{6.160000in}{6.160000in}}%
\pgfusepath{clip}%
\pgfsetbuttcap%
\pgfsetroundjoin%
\definecolor{currentfill}{rgb}{0.521696,0.659599,0.987736}%
\pgfsetfillcolor{currentfill}%
\pgfsetlinewidth{0.000000pt}%
\definecolor{currentstroke}{rgb}{0.000000,0.000000,0.000000}%
\pgfsetstrokecolor{currentstroke}%
\pgfsetdash{}{0pt}%
\pgfpathmoveto{\pgfqpoint{5.178357in}{3.841180in}}%
\pgfpathlineto{\pgfqpoint{5.187802in}{3.782616in}}%
\pgfpathlineto{\pgfqpoint{5.198470in}{3.844198in}}%
\pgfpathlineto{\pgfqpoint{5.228356in}{3.637131in}}%
\pgfpathlineto{\pgfqpoint{5.261942in}{3.788352in}}%
\pgfpathlineto{\pgfqpoint{5.250082in}{3.623585in}}%
\pgfpathlineto{\pgfqpoint{5.241632in}{3.777104in}}%
\pgfpathlineto{\pgfqpoint{5.211343in}{3.939028in}}%
\pgfpathlineto{\pgfqpoint{5.178357in}{3.841180in}}%
\pgfpathclose%
\pgfusepath{fill}%
\end{pgfscope}%
\begin{pgfscope}%
\pgfpathrectangle{\pgfqpoint{1.020000in}{0.880000in}}{\pgfqpoint{6.160000in}{6.160000in}}%
\pgfusepath{clip}%
\pgfsetbuttcap%
\pgfsetroundjoin%
\definecolor{currentfill}{rgb}{0.505423,0.643995,0.983157}%
\pgfsetfillcolor{currentfill}%
\pgfsetlinewidth{0.000000pt}%
\definecolor{currentstroke}{rgb}{0.000000,0.000000,0.000000}%
\pgfsetstrokecolor{currentstroke}%
\pgfsetdash{}{0pt}%
\pgfpathmoveto{\pgfqpoint{4.901342in}{3.707096in}}%
\pgfpathlineto{\pgfqpoint{4.910449in}{3.620264in}}%
\pgfpathlineto{\pgfqpoint{4.922011in}{3.866648in}}%
\pgfpathlineto{\pgfqpoint{4.952222in}{3.615516in}}%
\pgfpathlineto{\pgfqpoint{4.986953in}{3.951998in}}%
\pgfpathlineto{\pgfqpoint{4.975198in}{3.709700in}}%
\pgfpathlineto{\pgfqpoint{4.965460in}{3.717422in}}%
\pgfpathlineto{\pgfqpoint{4.934588in}{3.869920in}}%
\pgfpathlineto{\pgfqpoint{4.901342in}{3.707096in}}%
\pgfpathclose%
\pgfusepath{fill}%
\end{pgfscope}%
\begin{pgfscope}%
\pgfpathrectangle{\pgfqpoint{1.020000in}{0.880000in}}{\pgfqpoint{6.160000in}{6.160000in}}%
\pgfusepath{clip}%
\pgfsetbuttcap%
\pgfsetroundjoin%
\definecolor{currentfill}{rgb}{0.532568,0.669801,0.990393}%
\pgfsetfillcolor{currentfill}%
\pgfsetlinewidth{0.000000pt}%
\definecolor{currentstroke}{rgb}{0.000000,0.000000,0.000000}%
\pgfsetstrokecolor{currentstroke}%
\pgfsetdash{}{0pt}%
\pgfpathmoveto{\pgfqpoint{4.689452in}{3.651104in}}%
\pgfpathlineto{\pgfqpoint{4.699417in}{3.739167in}}%
\pgfpathlineto{\pgfqpoint{4.709639in}{3.869123in}}%
\pgfpathlineto{\pgfqpoint{4.741016in}{3.723067in}}%
\pgfpathlineto{\pgfqpoint{4.773614in}{3.803513in}}%
\pgfpathlineto{\pgfqpoint{4.764155in}{3.829094in}}%
\pgfpathlineto{\pgfqpoint{4.754136in}{3.753450in}}%
\pgfpathlineto{\pgfqpoint{4.723708in}{4.060020in}}%
\pgfpathlineto{\pgfqpoint{4.689452in}{3.651104in}}%
\pgfpathclose%
\pgfusepath{fill}%
\end{pgfscope}%
\begin{pgfscope}%
\pgfpathrectangle{\pgfqpoint{1.020000in}{0.880000in}}{\pgfqpoint{6.160000in}{6.160000in}}%
\pgfusepath{clip}%
\pgfsetbuttcap%
\pgfsetroundjoin%
\definecolor{currentfill}{rgb}{0.478462,0.616564,0.972721}%
\pgfsetfillcolor{currentfill}%
\pgfsetlinewidth{0.000000pt}%
\definecolor{currentstroke}{rgb}{0.000000,0.000000,0.000000}%
\pgfsetstrokecolor{currentstroke}%
\pgfsetdash{}{0pt}%
\pgfpathmoveto{\pgfqpoint{5.326733in}{3.861050in}}%
\pgfpathlineto{\pgfqpoint{5.335337in}{3.722614in}}%
\pgfpathlineto{\pgfqpoint{5.344959in}{3.672655in}}%
\pgfpathlineto{\pgfqpoint{5.377054in}{3.682586in}}%
\pgfpathlineto{\pgfqpoint{5.409280in}{3.704180in}}%
\pgfpathlineto{\pgfqpoint{5.398475in}{3.659105in}}%
\pgfpathlineto{\pgfqpoint{5.388878in}{3.711957in}}%
\pgfpathlineto{\pgfqpoint{5.356276in}{3.653049in}}%
\pgfpathlineto{\pgfqpoint{5.326733in}{3.861050in}}%
\pgfpathclose%
\pgfusepath{fill}%
\end{pgfscope}%
\begin{pgfscope}%
\pgfpathrectangle{\pgfqpoint{1.020000in}{0.880000in}}{\pgfqpoint{6.160000in}{6.160000in}}%
\pgfusepath{clip}%
\pgfsetbuttcap%
\pgfsetroundjoin%
\definecolor{currentfill}{rgb}{0.613933,0.739923,0.999142}%
\pgfsetfillcolor{currentfill}%
\pgfsetlinewidth{0.000000pt}%
\definecolor{currentstroke}{rgb}{0.000000,0.000000,0.000000}%
\pgfsetstrokecolor{currentstroke}%
\pgfsetdash{}{0pt}%
\pgfpathmoveto{\pgfqpoint{4.331585in}{3.913732in}}%
\pgfpathlineto{\pgfqpoint{4.340961in}{4.052712in}}%
\pgfpathlineto{\pgfqpoint{4.349974in}{3.946383in}}%
\pgfpathlineto{\pgfqpoint{4.382264in}{3.953100in}}%
\pgfpathlineto{\pgfqpoint{4.414593in}{3.982355in}}%
\pgfpathlineto{\pgfqpoint{4.405184in}{3.907795in}}%
\pgfpathlineto{\pgfqpoint{4.395912in}{3.882231in}}%
\pgfpathlineto{\pgfqpoint{4.363682in}{3.852900in}}%
\pgfpathlineto{\pgfqpoint{4.331585in}{3.913732in}}%
\pgfpathclose%
\pgfusepath{fill}%
\end{pgfscope}%
\begin{pgfscope}%
\pgfpathrectangle{\pgfqpoint{1.020000in}{0.880000in}}{\pgfqpoint{6.160000in}{6.160000in}}%
\pgfusepath{clip}%
\pgfsetbuttcap%
\pgfsetroundjoin%
\definecolor{currentfill}{rgb}{0.500031,0.638508,0.981070}%
\pgfsetfillcolor{currentfill}%
\pgfsetlinewidth{0.000000pt}%
\definecolor{currentstroke}{rgb}{0.000000,0.000000,0.000000}%
\pgfsetstrokecolor{currentstroke}%
\pgfsetdash{}{0pt}%
\pgfpathmoveto{\pgfqpoint{5.113037in}{3.701296in}}%
\pgfpathlineto{\pgfqpoint{5.122273in}{3.624429in}}%
\pgfpathlineto{\pgfqpoint{5.132351in}{3.635747in}}%
\pgfpathlineto{\pgfqpoint{5.166140in}{3.816971in}}%
\pgfpathlineto{\pgfqpoint{5.198470in}{3.844198in}}%
\pgfpathlineto{\pgfqpoint{5.187802in}{3.782616in}}%
\pgfpathlineto{\pgfqpoint{5.178357in}{3.841180in}}%
\pgfpathlineto{\pgfqpoint{5.145221in}{3.722970in}}%
\pgfpathlineto{\pgfqpoint{5.113037in}{3.701296in}}%
\pgfpathclose%
\pgfusepath{fill}%
\end{pgfscope}%
\begin{pgfscope}%
\pgfpathrectangle{\pgfqpoint{1.020000in}{0.880000in}}{\pgfqpoint{6.160000in}{6.160000in}}%
\pgfusepath{clip}%
\pgfsetbuttcap%
\pgfsetroundjoin%
\definecolor{currentfill}{rgb}{0.478462,0.616564,0.972721}%
\pgfsetfillcolor{currentfill}%
\pgfsetlinewidth{0.000000pt}%
\definecolor{currentstroke}{rgb}{0.000000,0.000000,0.000000}%
\pgfsetstrokecolor{currentstroke}%
\pgfsetdash{}{0pt}%
\pgfpathmoveto{\pgfqpoint{5.048954in}{3.692556in}}%
\pgfpathlineto{\pgfqpoint{5.058227in}{3.621195in}}%
\pgfpathlineto{\pgfqpoint{5.069947in}{3.826219in}}%
\pgfpathlineto{\pgfqpoint{5.101218in}{3.734245in}}%
\pgfpathlineto{\pgfqpoint{5.132351in}{3.635747in}}%
\pgfpathlineto{\pgfqpoint{5.122273in}{3.624429in}}%
\pgfpathlineto{\pgfqpoint{5.113037in}{3.701296in}}%
\pgfpathlineto{\pgfqpoint{5.081813in}{3.787451in}}%
\pgfpathlineto{\pgfqpoint{5.048954in}{3.692556in}}%
\pgfpathclose%
\pgfusepath{fill}%
\end{pgfscope}%
\begin{pgfscope}%
\pgfpathrectangle{\pgfqpoint{1.020000in}{0.880000in}}{\pgfqpoint{6.160000in}{6.160000in}}%
\pgfusepath{clip}%
\pgfsetbuttcap%
\pgfsetroundjoin%
\definecolor{currentfill}{rgb}{0.441123,0.576532,0.954545}%
\pgfsetfillcolor{currentfill}%
\pgfsetlinewidth{0.000000pt}%
\definecolor{currentstroke}{rgb}{0.000000,0.000000,0.000000}%
\pgfsetstrokecolor{currentstroke}%
\pgfsetdash{}{0pt}%
\pgfpathmoveto{\pgfqpoint{5.472290in}{3.636285in}}%
\pgfpathlineto{\pgfqpoint{5.483558in}{3.708494in}}%
\pgfpathlineto{\pgfqpoint{5.490357in}{3.435954in}}%
\pgfpathlineto{\pgfqpoint{5.524685in}{3.618146in}}%
\pgfpathlineto{\pgfqpoint{5.558342in}{3.744288in}}%
\pgfpathlineto{\pgfqpoint{5.545944in}{3.599239in}}%
\pgfpathlineto{\pgfqpoint{5.535421in}{3.589180in}}%
\pgfpathlineto{\pgfqpoint{5.506034in}{3.776746in}}%
\pgfpathlineto{\pgfqpoint{5.472290in}{3.636285in}}%
\pgfpathclose%
\pgfusepath{fill}%
\end{pgfscope}%
\begin{pgfscope}%
\pgfpathrectangle{\pgfqpoint{1.020000in}{0.880000in}}{\pgfqpoint{6.160000in}{6.160000in}}%
\pgfusepath{clip}%
\pgfsetbuttcap%
\pgfsetroundjoin%
\definecolor{currentfill}{rgb}{0.500031,0.638508,0.981070}%
\pgfsetfillcolor{currentfill}%
\pgfsetlinewidth{0.000000pt}%
\definecolor{currentstroke}{rgb}{0.000000,0.000000,0.000000}%
\pgfsetstrokecolor{currentstroke}%
\pgfsetdash{}{0pt}%
\pgfpathmoveto{\pgfqpoint{4.837293in}{3.716552in}}%
\pgfpathlineto{\pgfqpoint{4.847435in}{3.786717in}}%
\pgfpathlineto{\pgfqpoint{4.857043in}{3.772261in}}%
\pgfpathlineto{\pgfqpoint{4.888399in}{3.659990in}}%
\pgfpathlineto{\pgfqpoint{4.922011in}{3.866648in}}%
\pgfpathlineto{\pgfqpoint{4.910449in}{3.620264in}}%
\pgfpathlineto{\pgfqpoint{4.901342in}{3.707096in}}%
\pgfpathlineto{\pgfqpoint{4.869848in}{3.787888in}}%
\pgfpathlineto{\pgfqpoint{4.837293in}{3.716552in}}%
\pgfpathclose%
\pgfusepath{fill}%
\end{pgfscope}%
\begin{pgfscope}%
\pgfpathrectangle{\pgfqpoint{1.020000in}{0.880000in}}{\pgfqpoint{6.160000in}{6.160000in}}%
\pgfusepath{clip}%
\pgfsetbuttcap%
\pgfsetroundjoin%
\definecolor{currentfill}{rgb}{0.791392,0.846750,0.936641}%
\pgfsetfillcolor{currentfill}%
\pgfsetlinewidth{0.000000pt}%
\definecolor{currentstroke}{rgb}{0.000000,0.000000,0.000000}%
\pgfsetstrokecolor{currentstroke}%
\pgfsetdash{}{0pt}%
\pgfpathmoveto{\pgfqpoint{3.678061in}{4.401020in}}%
\pgfpathlineto{\pgfqpoint{3.686881in}{4.337429in}}%
\pgfpathlineto{\pgfqpoint{3.696066in}{4.199840in}}%
\pgfpathlineto{\pgfqpoint{3.727839in}{4.349072in}}%
\pgfpathlineto{\pgfqpoint{3.760469in}{4.312514in}}%
\pgfpathlineto{\pgfqpoint{3.752559in}{4.156054in}}%
\pgfpathlineto{\pgfqpoint{3.743754in}{4.214931in}}%
\pgfpathlineto{\pgfqpoint{3.711689in}{4.149446in}}%
\pgfpathlineto{\pgfqpoint{3.678061in}{4.401020in}}%
\pgfpathclose%
\pgfusepath{fill}%
\end{pgfscope}%
\begin{pgfscope}%
\pgfpathrectangle{\pgfqpoint{1.020000in}{0.880000in}}{\pgfqpoint{6.160000in}{6.160000in}}%
\pgfusepath{clip}%
\pgfsetbuttcap%
\pgfsetroundjoin%
\definecolor{currentfill}{rgb}{0.713852,0.808857,0.979386}%
\pgfsetfillcolor{currentfill}%
\pgfsetlinewidth{0.000000pt}%
\definecolor{currentstroke}{rgb}{0.000000,0.000000,0.000000}%
\pgfsetstrokecolor{currentstroke}%
\pgfsetdash{}{0pt}%
\pgfpathmoveto{\pgfqpoint{3.972696in}{4.192491in}}%
\pgfpathlineto{\pgfqpoint{3.981626in}{4.129998in}}%
\pgfpathlineto{\pgfqpoint{3.990825in}{3.931002in}}%
\pgfpathlineto{\pgfqpoint{4.022804in}{4.154286in}}%
\pgfpathlineto{\pgfqpoint{4.055153in}{4.185756in}}%
\pgfpathlineto{\pgfqpoint{4.046571in}{3.963309in}}%
\pgfpathlineto{\pgfqpoint{4.037569in}{4.071324in}}%
\pgfpathlineto{\pgfqpoint{4.004968in}{4.241227in}}%
\pgfpathlineto{\pgfqpoint{3.972696in}{4.192491in}}%
\pgfpathclose%
\pgfusepath{fill}%
\end{pgfscope}%
\begin{pgfscope}%
\pgfpathrectangle{\pgfqpoint{1.020000in}{0.880000in}}{\pgfqpoint{6.160000in}{6.160000in}}%
\pgfusepath{clip}%
\pgfsetbuttcap%
\pgfsetroundjoin%
\definecolor{currentfill}{rgb}{0.613933,0.739923,0.999142}%
\pgfsetfillcolor{currentfill}%
\pgfsetlinewidth{0.000000pt}%
\definecolor{currentstroke}{rgb}{0.000000,0.000000,0.000000}%
\pgfsetstrokecolor{currentstroke}%
\pgfsetdash{}{0pt}%
\pgfpathmoveto{\pgfqpoint{4.267104in}{3.898046in}}%
\pgfpathlineto{\pgfqpoint{4.276273in}{3.977689in}}%
\pgfpathlineto{\pgfqpoint{4.285323in}{3.907129in}}%
\pgfpathlineto{\pgfqpoint{4.317706in}{3.967288in}}%
\pgfpathlineto{\pgfqpoint{4.349974in}{3.946383in}}%
\pgfpathlineto{\pgfqpoint{4.340961in}{4.052712in}}%
\pgfpathlineto{\pgfqpoint{4.331585in}{3.913732in}}%
\pgfpathlineto{\pgfqpoint{4.299207in}{3.781506in}}%
\pgfpathlineto{\pgfqpoint{4.267104in}{3.898046in}}%
\pgfpathclose%
\pgfusepath{fill}%
\end{pgfscope}%
\begin{pgfscope}%
\pgfpathrectangle{\pgfqpoint{1.020000in}{0.880000in}}{\pgfqpoint{6.160000in}{6.160000in}}%
\pgfusepath{clip}%
\pgfsetbuttcap%
\pgfsetroundjoin%
\definecolor{currentfill}{rgb}{0.592356,0.722792,0.999434}%
\pgfsetfillcolor{currentfill}%
\pgfsetlinewidth{0.000000pt}%
\definecolor{currentstroke}{rgb}{0.000000,0.000000,0.000000}%
\pgfsetstrokecolor{currentstroke}%
\pgfsetdash{}{0pt}%
\pgfpathmoveto{\pgfqpoint{4.478386in}{3.746725in}}%
\pgfpathlineto{\pgfqpoint{4.487943in}{3.831027in}}%
\pgfpathlineto{\pgfqpoint{4.497642in}{3.949393in}}%
\pgfpathlineto{\pgfqpoint{4.529795in}{3.916753in}}%
\pgfpathlineto{\pgfqpoint{4.562433in}{4.023367in}}%
\pgfpathlineto{\pgfqpoint{4.552536in}{3.890071in}}%
\pgfpathlineto{\pgfqpoint{4.543147in}{3.886209in}}%
\pgfpathlineto{\pgfqpoint{4.511225in}{3.961257in}}%
\pgfpathlineto{\pgfqpoint{4.478386in}{3.746725in}}%
\pgfpathclose%
\pgfusepath{fill}%
\end{pgfscope}%
\begin{pgfscope}%
\pgfpathrectangle{\pgfqpoint{1.020000in}{0.880000in}}{\pgfqpoint{6.160000in}{6.160000in}}%
\pgfusepath{clip}%
\pgfsetbuttcap%
\pgfsetroundjoin%
\definecolor{currentfill}{rgb}{0.543440,0.680003,0.993051}%
\pgfsetfillcolor{currentfill}%
\pgfsetlinewidth{0.000000pt}%
\definecolor{currentstroke}{rgb}{0.000000,0.000000,0.000000}%
\pgfsetstrokecolor{currentstroke}%
\pgfsetdash{}{0pt}%
\pgfpathmoveto{\pgfqpoint{4.626325in}{3.888454in}}%
\pgfpathlineto{\pgfqpoint{4.635634in}{3.851967in}}%
\pgfpathlineto{\pgfqpoint{4.645193in}{3.866770in}}%
\pgfpathlineto{\pgfqpoint{4.677618in}{3.907194in}}%
\pgfpathlineto{\pgfqpoint{4.709639in}{3.869123in}}%
\pgfpathlineto{\pgfqpoint{4.699417in}{3.739167in}}%
\pgfpathlineto{\pgfqpoint{4.689452in}{3.651104in}}%
\pgfpathlineto{\pgfqpoint{4.657735in}{3.727468in}}%
\pgfpathlineto{\pgfqpoint{4.626325in}{3.888454in}}%
\pgfpathclose%
\pgfusepath{fill}%
\end{pgfscope}%
\begin{pgfscope}%
\pgfpathrectangle{\pgfqpoint{1.020000in}{0.880000in}}{\pgfqpoint{6.160000in}{6.160000in}}%
\pgfusepath{clip}%
\pgfsetbuttcap%
\pgfsetroundjoin%
\definecolor{currentfill}{rgb}{0.922681,0.828568,0.777054}%
\pgfsetfillcolor{currentfill}%
\pgfsetlinewidth{0.000000pt}%
\definecolor{currentstroke}{rgb}{0.000000,0.000000,0.000000}%
\pgfsetstrokecolor{currentstroke}%
\pgfsetdash{}{0pt}%
\pgfpathmoveto{\pgfqpoint{3.236538in}{4.636874in}}%
\pgfpathlineto{\pgfqpoint{3.245154in}{4.576597in}}%
\pgfpathlineto{\pgfqpoint{3.252704in}{4.632866in}}%
\pgfpathlineto{\pgfqpoint{3.285892in}{4.569314in}}%
\pgfpathlineto{\pgfqpoint{3.318483in}{4.568531in}}%
\pgfpathlineto{\pgfqpoint{3.310239in}{4.581828in}}%
\pgfpathlineto{\pgfqpoint{3.303403in}{4.435175in}}%
\pgfpathlineto{\pgfqpoint{3.269507in}{4.591965in}}%
\pgfpathlineto{\pgfqpoint{3.236538in}{4.636874in}}%
\pgfpathclose%
\pgfusepath{fill}%
\end{pgfscope}%
\begin{pgfscope}%
\pgfpathrectangle{\pgfqpoint{1.020000in}{0.880000in}}{\pgfqpoint{6.160000in}{6.160000in}}%
\pgfusepath{clip}%
\pgfsetbuttcap%
\pgfsetroundjoin%
\definecolor{currentfill}{rgb}{0.768034,0.837035,0.952488}%
\pgfsetfillcolor{currentfill}%
\pgfsetlinewidth{0.000000pt}%
\definecolor{currentstroke}{rgb}{0.000000,0.000000,0.000000}%
\pgfsetstrokecolor{currentstroke}%
\pgfsetdash{}{0pt}%
\pgfpathmoveto{\pgfqpoint{3.825447in}{4.282882in}}%
\pgfpathlineto{\pgfqpoint{3.834443in}{4.184286in}}%
\pgfpathlineto{\pgfqpoint{3.842450in}{4.379127in}}%
\pgfpathlineto{\pgfqpoint{3.875403in}{4.228039in}}%
\pgfpathlineto{\pgfqpoint{3.908119in}{4.121301in}}%
\pgfpathlineto{\pgfqpoint{3.899223in}{4.183991in}}%
\pgfpathlineto{\pgfqpoint{3.890608in}{4.152267in}}%
\pgfpathlineto{\pgfqpoint{3.858101in}{4.204083in}}%
\pgfpathlineto{\pgfqpoint{3.825447in}{4.282882in}}%
\pgfpathclose%
\pgfusepath{fill}%
\end{pgfscope}%
\begin{pgfscope}%
\pgfpathrectangle{\pgfqpoint{1.020000in}{0.880000in}}{\pgfqpoint{6.160000in}{6.160000in}}%
\pgfusepath{clip}%
\pgfsetbuttcap%
\pgfsetroundjoin%
\definecolor{currentfill}{rgb}{0.672538,0.782861,0.991982}%
\pgfsetfillcolor{currentfill}%
\pgfsetlinewidth{0.000000pt}%
\definecolor{currentstroke}{rgb}{0.000000,0.000000,0.000000}%
\pgfsetstrokecolor{currentstroke}%
\pgfsetdash{}{0pt}%
\pgfpathmoveto{\pgfqpoint{4.120006in}{4.007429in}}%
\pgfpathlineto{\pgfqpoint{4.128950in}{4.013312in}}%
\pgfpathlineto{\pgfqpoint{4.137929in}{3.983177in}}%
\pgfpathlineto{\pgfqpoint{4.170258in}{4.061057in}}%
\pgfpathlineto{\pgfqpoint{4.202595in}{4.014945in}}%
\pgfpathlineto{\pgfqpoint{4.193571in}{4.043501in}}%
\pgfpathlineto{\pgfqpoint{4.184561in}{4.078023in}}%
\pgfpathlineto{\pgfqpoint{4.152265in}{4.096674in}}%
\pgfpathlineto{\pgfqpoint{4.120006in}{4.007429in}}%
\pgfpathclose%
\pgfusepath{fill}%
\end{pgfscope}%
\begin{pgfscope}%
\pgfpathrectangle{\pgfqpoint{1.020000in}{0.880000in}}{\pgfqpoint{6.160000in}{6.160000in}}%
\pgfusepath{clip}%
\pgfsetbuttcap%
\pgfsetroundjoin%
\definecolor{currentfill}{rgb}{0.441123,0.576532,0.954545}%
\pgfsetfillcolor{currentfill}%
\pgfsetlinewidth{0.000000pt}%
\definecolor{currentstroke}{rgb}{0.000000,0.000000,0.000000}%
\pgfsetstrokecolor{currentstroke}%
\pgfsetdash{}{0pt}%
\pgfpathmoveto{\pgfqpoint{5.409280in}{3.704180in}}%
\pgfpathlineto{\pgfqpoint{5.420015in}{3.741045in}}%
\pgfpathlineto{\pgfqpoint{5.428414in}{3.587819in}}%
\pgfpathlineto{\pgfqpoint{5.461105in}{3.644135in}}%
\pgfpathlineto{\pgfqpoint{5.490357in}{3.435954in}}%
\pgfpathlineto{\pgfqpoint{5.483558in}{3.708494in}}%
\pgfpathlineto{\pgfqpoint{5.472290in}{3.636285in}}%
\pgfpathlineto{\pgfqpoint{5.440080in}{3.612467in}}%
\pgfpathlineto{\pgfqpoint{5.409280in}{3.704180in}}%
\pgfpathclose%
\pgfusepath{fill}%
\end{pgfscope}%
\begin{pgfscope}%
\pgfpathrectangle{\pgfqpoint{1.020000in}{0.880000in}}{\pgfqpoint{6.160000in}{6.160000in}}%
\pgfusepath{clip}%
\pgfsetbuttcap%
\pgfsetroundjoin%
\definecolor{currentfill}{rgb}{0.494638,0.633022,0.978983}%
\pgfsetfillcolor{currentfill}%
\pgfsetlinewidth{0.000000pt}%
\definecolor{currentstroke}{rgb}{0.000000,0.000000,0.000000}%
\pgfsetstrokecolor{currentstroke}%
\pgfsetdash{}{0pt}%
\pgfpathmoveto{\pgfqpoint{5.261942in}{3.788352in}}%
\pgfpathlineto{\pgfqpoint{5.270305in}{3.625767in}}%
\pgfpathlineto{\pgfqpoint{5.280782in}{3.657337in}}%
\pgfpathlineto{\pgfqpoint{5.313523in}{3.722302in}}%
\pgfpathlineto{\pgfqpoint{5.344959in}{3.672655in}}%
\pgfpathlineto{\pgfqpoint{5.335337in}{3.722614in}}%
\pgfpathlineto{\pgfqpoint{5.326733in}{3.861050in}}%
\pgfpathlineto{\pgfqpoint{5.293762in}{3.772856in}}%
\pgfpathlineto{\pgfqpoint{5.261942in}{3.788352in}}%
\pgfpathclose%
\pgfusepath{fill}%
\end{pgfscope}%
\begin{pgfscope}%
\pgfpathrectangle{\pgfqpoint{1.020000in}{0.880000in}}{\pgfqpoint{6.160000in}{6.160000in}}%
\pgfusepath{clip}%
\pgfsetbuttcap%
\pgfsetroundjoin%
\definecolor{currentfill}{rgb}{0.698454,0.799450,0.984577}%
\pgfsetfillcolor{currentfill}%
\pgfsetlinewidth{0.000000pt}%
\definecolor{currentstroke}{rgb}{0.000000,0.000000,0.000000}%
\pgfsetstrokecolor{currentstroke}%
\pgfsetdash{}{0pt}%
\pgfpathmoveto{\pgfqpoint{3.908119in}{4.121301in}}%
\pgfpathlineto{\pgfqpoint{3.916673in}{4.190902in}}%
\pgfpathlineto{\pgfqpoint{3.925388in}{4.209171in}}%
\pgfpathlineto{\pgfqpoint{3.958087in}{4.102653in}}%
\pgfpathlineto{\pgfqpoint{3.990825in}{3.931002in}}%
\pgfpathlineto{\pgfqpoint{3.981626in}{4.129998in}}%
\pgfpathlineto{\pgfqpoint{3.972696in}{4.192491in}}%
\pgfpathlineto{\pgfqpoint{3.941284in}{3.792548in}}%
\pgfpathlineto{\pgfqpoint{3.908119in}{4.121301in}}%
\pgfpathclose%
\pgfusepath{fill}%
\end{pgfscope}%
\begin{pgfscope}%
\pgfpathrectangle{\pgfqpoint{1.020000in}{0.880000in}}{\pgfqpoint{6.160000in}{6.160000in}}%
\pgfusepath{clip}%
\pgfsetbuttcap%
\pgfsetroundjoin%
\definecolor{currentfill}{rgb}{0.902849,0.844796,0.811970}%
\pgfsetfillcolor{currentfill}%
\pgfsetlinewidth{0.000000pt}%
\definecolor{currentstroke}{rgb}{0.000000,0.000000,0.000000}%
\pgfsetstrokecolor{currentstroke}%
\pgfsetdash{}{0pt}%
\pgfpathmoveto{\pgfqpoint{3.384145in}{4.499052in}}%
\pgfpathlineto{\pgfqpoint{3.392034in}{4.539002in}}%
\pgfpathlineto{\pgfqpoint{3.399866in}{4.589332in}}%
\pgfpathlineto{\pgfqpoint{3.432785in}{4.545894in}}%
\pgfpathlineto{\pgfqpoint{3.466667in}{4.360324in}}%
\pgfpathlineto{\pgfqpoint{3.456519in}{4.622103in}}%
\pgfpathlineto{\pgfqpoint{3.449037in}{4.513089in}}%
\pgfpathlineto{\pgfqpoint{3.417223in}{4.423616in}}%
\pgfpathlineto{\pgfqpoint{3.384145in}{4.499052in}}%
\pgfpathclose%
\pgfusepath{fill}%
\end{pgfscope}%
\begin{pgfscope}%
\pgfpathrectangle{\pgfqpoint{1.020000in}{0.880000in}}{\pgfqpoint{6.160000in}{6.160000in}}%
\pgfusepath{clip}%
\pgfsetbuttcap%
\pgfsetroundjoin%
\definecolor{currentfill}{rgb}{0.473070,0.611077,0.970634}%
\pgfsetfillcolor{currentfill}%
\pgfsetlinewidth{0.000000pt}%
\definecolor{currentstroke}{rgb}{0.000000,0.000000,0.000000}%
\pgfsetstrokecolor{currentstroke}%
\pgfsetdash{}{0pt}%
\pgfpathmoveto{\pgfqpoint{5.198470in}{3.844198in}}%
\pgfpathlineto{\pgfqpoint{5.207059in}{3.698828in}}%
\pgfpathlineto{\pgfqpoint{5.216886in}{3.674448in}}%
\pgfpathlineto{\pgfqpoint{5.247742in}{3.562022in}}%
\pgfpathlineto{\pgfqpoint{5.280782in}{3.657337in}}%
\pgfpathlineto{\pgfqpoint{5.270305in}{3.625767in}}%
\pgfpathlineto{\pgfqpoint{5.261942in}{3.788352in}}%
\pgfpathlineto{\pgfqpoint{5.228356in}{3.637131in}}%
\pgfpathlineto{\pgfqpoint{5.198470in}{3.844198in}}%
\pgfpathclose%
\pgfusepath{fill}%
\end{pgfscope}%
\begin{pgfscope}%
\pgfpathrectangle{\pgfqpoint{1.020000in}{0.880000in}}{\pgfqpoint{6.160000in}{6.160000in}}%
\pgfusepath{clip}%
\pgfsetbuttcap%
\pgfsetroundjoin%
\definecolor{currentfill}{rgb}{0.667253,0.779176,0.992959}%
\pgfsetfillcolor{currentfill}%
\pgfsetlinewidth{0.000000pt}%
\definecolor{currentstroke}{rgb}{0.000000,0.000000,0.000000}%
\pgfsetstrokecolor{currentstroke}%
\pgfsetdash{}{0pt}%
\pgfpathmoveto{\pgfqpoint{4.055153in}{4.185756in}}%
\pgfpathlineto{\pgfqpoint{4.064269in}{4.003915in}}%
\pgfpathlineto{\pgfqpoint{4.073279in}{3.907599in}}%
\pgfpathlineto{\pgfqpoint{4.105615in}{3.927112in}}%
\pgfpathlineto{\pgfqpoint{4.137929in}{3.983177in}}%
\pgfpathlineto{\pgfqpoint{4.128950in}{4.013312in}}%
\pgfpathlineto{\pgfqpoint{4.120006in}{4.007429in}}%
\pgfpathlineto{\pgfqpoint{4.087557in}{4.159525in}}%
\pgfpathlineto{\pgfqpoint{4.055153in}{4.185756in}}%
\pgfpathclose%
\pgfusepath{fill}%
\end{pgfscope}%
\begin{pgfscope}%
\pgfpathrectangle{\pgfqpoint{1.020000in}{0.880000in}}{\pgfqpoint{6.160000in}{6.160000in}}%
\pgfusepath{clip}%
\pgfsetbuttcap%
\pgfsetroundjoin%
\definecolor{currentfill}{rgb}{0.527132,0.664700,0.989065}%
\pgfsetfillcolor{currentfill}%
\pgfsetlinewidth{0.000000pt}%
\definecolor{currentstroke}{rgb}{0.000000,0.000000,0.000000}%
\pgfsetstrokecolor{currentstroke}%
\pgfsetdash{}{0pt}%
\pgfpathmoveto{\pgfqpoint{4.773614in}{3.803513in}}%
\pgfpathlineto{\pgfqpoint{4.782692in}{3.711684in}}%
\pgfpathlineto{\pgfqpoint{4.792348in}{3.714794in}}%
\pgfpathlineto{\pgfqpoint{4.825041in}{3.797547in}}%
\pgfpathlineto{\pgfqpoint{4.857043in}{3.772261in}}%
\pgfpathlineto{\pgfqpoint{4.847435in}{3.786717in}}%
\pgfpathlineto{\pgfqpoint{4.837293in}{3.716552in}}%
\pgfpathlineto{\pgfqpoint{4.806326in}{3.897080in}}%
\pgfpathlineto{\pgfqpoint{4.773614in}{3.803513in}}%
\pgfpathclose%
\pgfusepath{fill}%
\end{pgfscope}%
\begin{pgfscope}%
\pgfpathrectangle{\pgfqpoint{1.020000in}{0.880000in}}{\pgfqpoint{6.160000in}{6.160000in}}%
\pgfusepath{clip}%
\pgfsetbuttcap%
\pgfsetroundjoin%
\definecolor{currentfill}{rgb}{0.597777,0.727330,0.999777}%
\pgfsetfillcolor{currentfill}%
\pgfsetlinewidth{0.000000pt}%
\definecolor{currentstroke}{rgb}{0.000000,0.000000,0.000000}%
\pgfsetstrokecolor{currentstroke}%
\pgfsetdash{}{0pt}%
\pgfpathmoveto{\pgfqpoint{4.414593in}{3.982355in}}%
\pgfpathlineto{\pgfqpoint{4.423513in}{3.838962in}}%
\pgfpathlineto{\pgfqpoint{4.433089in}{3.962740in}}%
\pgfpathlineto{\pgfqpoint{4.465561in}{4.021686in}}%
\pgfpathlineto{\pgfqpoint{4.497642in}{3.949393in}}%
\pgfpathlineto{\pgfqpoint{4.487943in}{3.831027in}}%
\pgfpathlineto{\pgfqpoint{4.478386in}{3.746725in}}%
\pgfpathlineto{\pgfqpoint{4.446586in}{3.883596in}}%
\pgfpathlineto{\pgfqpoint{4.414593in}{3.982355in}}%
\pgfpathclose%
\pgfusepath{fill}%
\end{pgfscope}%
\begin{pgfscope}%
\pgfpathrectangle{\pgfqpoint{1.020000in}{0.880000in}}{\pgfqpoint{6.160000in}{6.160000in}}%
\pgfusepath{clip}%
\pgfsetbuttcap%
\pgfsetroundjoin%
\definecolor{currentfill}{rgb}{0.879622,0.858175,0.845844}%
\pgfsetfillcolor{currentfill}%
\pgfsetlinewidth{0.000000pt}%
\definecolor{currentstroke}{rgb}{0.000000,0.000000,0.000000}%
\pgfsetstrokecolor{currentstroke}%
\pgfsetdash{}{0pt}%
\pgfpathmoveto{\pgfqpoint{3.530505in}{4.533283in}}%
\pgfpathlineto{\pgfqpoint{3.537991in}{4.670572in}}%
\pgfpathlineto{\pgfqpoint{3.547800in}{4.444921in}}%
\pgfpathlineto{\pgfqpoint{3.580016in}{4.498818in}}%
\pgfpathlineto{\pgfqpoint{3.613478in}{4.335090in}}%
\pgfpathlineto{\pgfqpoint{3.604988in}{4.346764in}}%
\pgfpathlineto{\pgfqpoint{3.596514in}{4.358261in}}%
\pgfpathlineto{\pgfqpoint{3.563562in}{4.443337in}}%
\pgfpathlineto{\pgfqpoint{3.530505in}{4.533283in}}%
\pgfpathclose%
\pgfusepath{fill}%
\end{pgfscope}%
\begin{pgfscope}%
\pgfpathrectangle{\pgfqpoint{1.020000in}{0.880000in}}{\pgfqpoint{6.160000in}{6.160000in}}%
\pgfusepath{clip}%
\pgfsetbuttcap%
\pgfsetroundjoin%
\definecolor{currentfill}{rgb}{0.624703,0.748318,0.998719}%
\pgfsetfillcolor{currentfill}%
\pgfsetlinewidth{0.000000pt}%
\definecolor{currentstroke}{rgb}{0.000000,0.000000,0.000000}%
\pgfsetstrokecolor{currentstroke}%
\pgfsetdash{}{0pt}%
\pgfpathmoveto{\pgfqpoint{4.202595in}{4.014945in}}%
\pgfpathlineto{\pgfqpoint{4.211577in}{3.790431in}}%
\pgfpathlineto{\pgfqpoint{4.220657in}{3.886093in}}%
\pgfpathlineto{\pgfqpoint{4.253106in}{4.055760in}}%
\pgfpathlineto{\pgfqpoint{4.285323in}{3.907129in}}%
\pgfpathlineto{\pgfqpoint{4.276273in}{3.977689in}}%
\pgfpathlineto{\pgfqpoint{4.267104in}{3.898046in}}%
\pgfpathlineto{\pgfqpoint{4.234914in}{4.027165in}}%
\pgfpathlineto{\pgfqpoint{4.202595in}{4.014945in}}%
\pgfpathclose%
\pgfusepath{fill}%
\end{pgfscope}%
\begin{pgfscope}%
\pgfpathrectangle{\pgfqpoint{1.020000in}{0.880000in}}{\pgfqpoint{6.160000in}{6.160000in}}%
\pgfusepath{clip}%
\pgfsetbuttcap%
\pgfsetroundjoin%
\definecolor{currentfill}{rgb}{0.527132,0.664700,0.989065}%
\pgfsetfillcolor{currentfill}%
\pgfsetlinewidth{0.000000pt}%
\definecolor{currentstroke}{rgb}{0.000000,0.000000,0.000000}%
\pgfsetstrokecolor{currentstroke}%
\pgfsetdash{}{0pt}%
\pgfpathmoveto{\pgfqpoint{4.986953in}{3.951998in}}%
\pgfpathlineto{\pgfqpoint{4.994696in}{3.690228in}}%
\pgfpathlineto{\pgfqpoint{5.005626in}{3.821279in}}%
\pgfpathlineto{\pgfqpoint{5.037519in}{3.791360in}}%
\pgfpathlineto{\pgfqpoint{5.069947in}{3.826219in}}%
\pgfpathlineto{\pgfqpoint{5.058227in}{3.621195in}}%
\pgfpathlineto{\pgfqpoint{5.048954in}{3.692556in}}%
\pgfpathlineto{\pgfqpoint{5.017951in}{3.815934in}}%
\pgfpathlineto{\pgfqpoint{4.986953in}{3.951998in}}%
\pgfpathclose%
\pgfusepath{fill}%
\end{pgfscope}%
\begin{pgfscope}%
\pgfpathrectangle{\pgfqpoint{1.020000in}{0.880000in}}{\pgfqpoint{6.160000in}{6.160000in}}%
\pgfusepath{clip}%
\pgfsetbuttcap%
\pgfsetroundjoin%
\definecolor{currentfill}{rgb}{0.786721,0.844807,0.939810}%
\pgfsetfillcolor{currentfill}%
\pgfsetlinewidth{0.000000pt}%
\definecolor{currentstroke}{rgb}{0.000000,0.000000,0.000000}%
\pgfsetstrokecolor{currentstroke}%
\pgfsetdash{}{0pt}%
\pgfpathmoveto{\pgfqpoint{3.760469in}{4.312514in}}%
\pgfpathlineto{\pgfqpoint{3.769106in}{4.302484in}}%
\pgfpathlineto{\pgfqpoint{3.778084in}{4.211472in}}%
\pgfpathlineto{\pgfqpoint{3.811042in}{4.080846in}}%
\pgfpathlineto{\pgfqpoint{3.842450in}{4.379127in}}%
\pgfpathlineto{\pgfqpoint{3.834443in}{4.184286in}}%
\pgfpathlineto{\pgfqpoint{3.825447in}{4.282882in}}%
\pgfpathlineto{\pgfqpoint{3.793364in}{4.196356in}}%
\pgfpathlineto{\pgfqpoint{3.760469in}{4.312514in}}%
\pgfpathclose%
\pgfusepath{fill}%
\end{pgfscope}%
\begin{pgfscope}%
\pgfpathrectangle{\pgfqpoint{1.020000in}{0.880000in}}{\pgfqpoint{6.160000in}{6.160000in}}%
\pgfusepath{clip}%
\pgfsetbuttcap%
\pgfsetroundjoin%
\definecolor{currentfill}{rgb}{0.909460,0.839386,0.800331}%
\pgfsetfillcolor{currentfill}%
\pgfsetlinewidth{0.000000pt}%
\definecolor{currentstroke}{rgb}{0.000000,0.000000,0.000000}%
\pgfsetstrokecolor{currentstroke}%
\pgfsetdash{}{0pt}%
\pgfpathmoveto{\pgfqpoint{3.318483in}{4.568531in}}%
\pgfpathlineto{\pgfqpoint{3.326542in}{4.578676in}}%
\pgfpathlineto{\pgfqpoint{3.334692in}{4.580252in}}%
\pgfpathlineto{\pgfqpoint{3.368808in}{4.396182in}}%
\pgfpathlineto{\pgfqpoint{3.399866in}{4.589332in}}%
\pgfpathlineto{\pgfqpoint{3.392034in}{4.539002in}}%
\pgfpathlineto{\pgfqpoint{3.384145in}{4.499052in}}%
\pgfpathlineto{\pgfqpoint{3.351947in}{4.459258in}}%
\pgfpathlineto{\pgfqpoint{3.318483in}{4.568531in}}%
\pgfpathclose%
\pgfusepath{fill}%
\end{pgfscope}%
\begin{pgfscope}%
\pgfpathrectangle{\pgfqpoint{1.020000in}{0.880000in}}{\pgfqpoint{6.160000in}{6.160000in}}%
\pgfusepath{clip}%
\pgfsetbuttcap%
\pgfsetroundjoin%
\definecolor{currentfill}{rgb}{0.938326,0.808917,0.741162}%
\pgfsetfillcolor{currentfill}%
\pgfsetlinewidth{0.000000pt}%
\definecolor{currentstroke}{rgb}{0.000000,0.000000,0.000000}%
\pgfsetstrokecolor{currentstroke}%
\pgfsetdash{}{0pt}%
\pgfpathmoveto{\pgfqpoint{3.171496in}{4.622960in}}%
\pgfpathlineto{\pgfqpoint{3.180203in}{4.551050in}}%
\pgfpathlineto{\pgfqpoint{3.187820in}{4.590332in}}%
\pgfpathlineto{\pgfqpoint{3.219456in}{4.696579in}}%
\pgfpathlineto{\pgfqpoint{3.252704in}{4.632866in}}%
\pgfpathlineto{\pgfqpoint{3.245154in}{4.576597in}}%
\pgfpathlineto{\pgfqpoint{3.236538in}{4.636874in}}%
\pgfpathlineto{\pgfqpoint{3.203960in}{4.636591in}}%
\pgfpathlineto{\pgfqpoint{3.171496in}{4.622960in}}%
\pgfpathclose%
\pgfusepath{fill}%
\end{pgfscope}%
\begin{pgfscope}%
\pgfpathrectangle{\pgfqpoint{1.020000in}{0.880000in}}{\pgfqpoint{6.160000in}{6.160000in}}%
\pgfusepath{clip}%
\pgfsetbuttcap%
\pgfsetroundjoin%
\definecolor{currentfill}{rgb}{0.576051,0.708780,0.997755}%
\pgfsetfillcolor{currentfill}%
\pgfsetlinewidth{0.000000pt}%
\definecolor{currentstroke}{rgb}{0.000000,0.000000,0.000000}%
\pgfsetstrokecolor{currentstroke}%
\pgfsetdash{}{0pt}%
\pgfpathmoveto{\pgfqpoint{4.562433in}{4.023367in}}%
\pgfpathlineto{\pgfqpoint{4.571015in}{3.804783in}}%
\pgfpathlineto{\pgfqpoint{4.580073in}{3.711959in}}%
\pgfpathlineto{\pgfqpoint{4.613209in}{3.927621in}}%
\pgfpathlineto{\pgfqpoint{4.645193in}{3.866770in}}%
\pgfpathlineto{\pgfqpoint{4.635634in}{3.851967in}}%
\pgfpathlineto{\pgfqpoint{4.626325in}{3.888454in}}%
\pgfpathlineto{\pgfqpoint{4.593803in}{3.807655in}}%
\pgfpathlineto{\pgfqpoint{4.562433in}{4.023367in}}%
\pgfpathclose%
\pgfusepath{fill}%
\end{pgfscope}%
\begin{pgfscope}%
\pgfpathrectangle{\pgfqpoint{1.020000in}{0.880000in}}{\pgfqpoint{6.160000in}{6.160000in}}%
\pgfusepath{clip}%
\pgfsetbuttcap%
\pgfsetroundjoin%
\definecolor{currentfill}{rgb}{0.527132,0.664700,0.989065}%
\pgfsetfillcolor{currentfill}%
\pgfsetlinewidth{0.000000pt}%
\definecolor{currentstroke}{rgb}{0.000000,0.000000,0.000000}%
\pgfsetstrokecolor{currentstroke}%
\pgfsetdash{}{0pt}%
\pgfpathmoveto{\pgfqpoint{4.709639in}{3.869123in}}%
\pgfpathlineto{\pgfqpoint{4.718620in}{3.758450in}}%
\pgfpathlineto{\pgfqpoint{4.728136in}{3.746454in}}%
\pgfpathlineto{\pgfqpoint{4.761032in}{3.865459in}}%
\pgfpathlineto{\pgfqpoint{4.792348in}{3.714794in}}%
\pgfpathlineto{\pgfqpoint{4.782692in}{3.711684in}}%
\pgfpathlineto{\pgfqpoint{4.773614in}{3.803513in}}%
\pgfpathlineto{\pgfqpoint{4.741016in}{3.723067in}}%
\pgfpathlineto{\pgfqpoint{4.709639in}{3.869123in}}%
\pgfpathclose%
\pgfusepath{fill}%
\end{pgfscope}%
\begin{pgfscope}%
\pgfpathrectangle{\pgfqpoint{1.020000in}{0.880000in}}{\pgfqpoint{6.160000in}{6.160000in}}%
\pgfusepath{clip}%
\pgfsetbuttcap%
\pgfsetroundjoin%
\definecolor{currentfill}{rgb}{0.843358,0.861820,0.890017}%
\pgfsetfillcolor{currentfill}%
\pgfsetlinewidth{0.000000pt}%
\definecolor{currentstroke}{rgb}{0.000000,0.000000,0.000000}%
\pgfsetstrokecolor{currentstroke}%
\pgfsetdash{}{0pt}%
\pgfpathmoveto{\pgfqpoint{3.613478in}{4.335090in}}%
\pgfpathlineto{\pgfqpoint{3.621631in}{4.386666in}}%
\pgfpathlineto{\pgfqpoint{3.629887in}{4.424366in}}%
\pgfpathlineto{\pgfqpoint{3.662420in}{4.428738in}}%
\pgfpathlineto{\pgfqpoint{3.696066in}{4.199840in}}%
\pgfpathlineto{\pgfqpoint{3.686881in}{4.337429in}}%
\pgfpathlineto{\pgfqpoint{3.678061in}{4.401020in}}%
\pgfpathlineto{\pgfqpoint{3.645663in}{4.387828in}}%
\pgfpathlineto{\pgfqpoint{3.613478in}{4.335090in}}%
\pgfpathclose%
\pgfusepath{fill}%
\end{pgfscope}%
\begin{pgfscope}%
\pgfpathrectangle{\pgfqpoint{1.020000in}{0.880000in}}{\pgfqpoint{6.160000in}{6.160000in}}%
\pgfusepath{clip}%
\pgfsetbuttcap%
\pgfsetroundjoin%
\definecolor{currentfill}{rgb}{0.478462,0.616564,0.972721}%
\pgfsetfillcolor{currentfill}%
\pgfsetlinewidth{0.000000pt}%
\definecolor{currentstroke}{rgb}{0.000000,0.000000,0.000000}%
\pgfsetstrokecolor{currentstroke}%
\pgfsetdash{}{0pt}%
\pgfpathmoveto{\pgfqpoint{5.344959in}{3.672655in}}%
\pgfpathlineto{\pgfqpoint{5.356517in}{3.787433in}}%
\pgfpathlineto{\pgfqpoint{5.364925in}{3.631287in}}%
\pgfpathlineto{\pgfqpoint{5.397159in}{3.648641in}}%
\pgfpathlineto{\pgfqpoint{5.428414in}{3.587819in}}%
\pgfpathlineto{\pgfqpoint{5.420015in}{3.741045in}}%
\pgfpathlineto{\pgfqpoint{5.409280in}{3.704180in}}%
\pgfpathlineto{\pgfqpoint{5.377054in}{3.682586in}}%
\pgfpathlineto{\pgfqpoint{5.344959in}{3.672655in}}%
\pgfpathclose%
\pgfusepath{fill}%
\end{pgfscope}%
\begin{pgfscope}%
\pgfpathrectangle{\pgfqpoint{1.020000in}{0.880000in}}{\pgfqpoint{6.160000in}{6.160000in}}%
\pgfusepath{clip}%
\pgfsetbuttcap%
\pgfsetroundjoin%
\definecolor{currentfill}{rgb}{0.435815,0.570707,0.951717}%
\pgfsetfillcolor{currentfill}%
\pgfsetlinewidth{0.000000pt}%
\definecolor{currentstroke}{rgb}{0.000000,0.000000,0.000000}%
\pgfsetstrokecolor{currentstroke}%
\pgfsetdash{}{0pt}%
\pgfpathmoveto{\pgfqpoint{5.490357in}{3.435954in}}%
\pgfpathlineto{\pgfqpoint{5.502679in}{3.585311in}}%
\pgfpathlineto{\pgfqpoint{5.513656in}{3.629409in}}%
\pgfpathlineto{\pgfqpoint{5.544924in}{3.574910in}}%
\pgfpathlineto{\pgfqpoint{5.578042in}{3.657127in}}%
\pgfpathlineto{\pgfqpoint{5.566695in}{3.592795in}}%
\pgfpathlineto{\pgfqpoint{5.558342in}{3.744288in}}%
\pgfpathlineto{\pgfqpoint{5.524685in}{3.618146in}}%
\pgfpathlineto{\pgfqpoint{5.490357in}{3.435954in}}%
\pgfpathclose%
\pgfusepath{fill}%
\end{pgfscope}%
\begin{pgfscope}%
\pgfpathrectangle{\pgfqpoint{1.020000in}{0.880000in}}{\pgfqpoint{6.160000in}{6.160000in}}%
\pgfusepath{clip}%
\pgfsetbuttcap%
\pgfsetroundjoin%
\definecolor{currentfill}{rgb}{0.688188,0.793178,0.988038}%
\pgfsetfillcolor{currentfill}%
\pgfsetlinewidth{0.000000pt}%
\definecolor{currentstroke}{rgb}{0.000000,0.000000,0.000000}%
\pgfsetstrokecolor{currentstroke}%
\pgfsetdash{}{0pt}%
\pgfpathmoveto{\pgfqpoint{3.990825in}{3.931002in}}%
\pgfpathlineto{\pgfqpoint{3.999326in}{4.104618in}}%
\pgfpathlineto{\pgfqpoint{4.008242in}{4.067252in}}%
\pgfpathlineto{\pgfqpoint{4.040711in}{4.047163in}}%
\pgfpathlineto{\pgfqpoint{4.073279in}{3.907599in}}%
\pgfpathlineto{\pgfqpoint{4.064269in}{4.003915in}}%
\pgfpathlineto{\pgfqpoint{4.055153in}{4.185756in}}%
\pgfpathlineto{\pgfqpoint{4.022804in}{4.154286in}}%
\pgfpathlineto{\pgfqpoint{3.990825in}{3.931002in}}%
\pgfpathclose%
\pgfusepath{fill}%
\end{pgfscope}%
\begin{pgfscope}%
\pgfpathrectangle{\pgfqpoint{1.020000in}{0.880000in}}{\pgfqpoint{6.160000in}{6.160000in}}%
\pgfusepath{clip}%
\pgfsetbuttcap%
\pgfsetroundjoin%
\definecolor{currentfill}{rgb}{0.527132,0.664700,0.989065}%
\pgfsetfillcolor{currentfill}%
\pgfsetlinewidth{0.000000pt}%
\definecolor{currentstroke}{rgb}{0.000000,0.000000,0.000000}%
\pgfsetstrokecolor{currentstroke}%
\pgfsetdash{}{0pt}%
\pgfpathmoveto{\pgfqpoint{4.922011in}{3.866648in}}%
\pgfpathlineto{\pgfqpoint{4.930718in}{3.721050in}}%
\pgfpathlineto{\pgfqpoint{4.940880in}{3.769096in}}%
\pgfpathlineto{\pgfqpoint{4.972662in}{3.719960in}}%
\pgfpathlineto{\pgfqpoint{5.005626in}{3.821279in}}%
\pgfpathlineto{\pgfqpoint{4.994696in}{3.690228in}}%
\pgfpathlineto{\pgfqpoint{4.986953in}{3.951998in}}%
\pgfpathlineto{\pgfqpoint{4.952222in}{3.615516in}}%
\pgfpathlineto{\pgfqpoint{4.922011in}{3.866648in}}%
\pgfpathclose%
\pgfusepath{fill}%
\end{pgfscope}%
\begin{pgfscope}%
\pgfpathrectangle{\pgfqpoint{1.020000in}{0.880000in}}{\pgfqpoint{6.160000in}{6.160000in}}%
\pgfusepath{clip}%
\pgfsetbuttcap%
\pgfsetroundjoin%
\definecolor{currentfill}{rgb}{0.510824,0.649397,0.985079}%
\pgfsetfillcolor{currentfill}%
\pgfsetlinewidth{0.000000pt}%
\definecolor{currentstroke}{rgb}{0.000000,0.000000,0.000000}%
\pgfsetstrokecolor{currentstroke}%
\pgfsetdash{}{0pt}%
\pgfpathmoveto{\pgfqpoint{5.132351in}{3.635747in}}%
\pgfpathlineto{\pgfqpoint{5.143230in}{3.728453in}}%
\pgfpathlineto{\pgfqpoint{5.153617in}{3.766294in}}%
\pgfpathlineto{\pgfqpoint{5.185336in}{3.726469in}}%
\pgfpathlineto{\pgfqpoint{5.216886in}{3.674448in}}%
\pgfpathlineto{\pgfqpoint{5.207059in}{3.698828in}}%
\pgfpathlineto{\pgfqpoint{5.198470in}{3.844198in}}%
\pgfpathlineto{\pgfqpoint{5.166140in}{3.816971in}}%
\pgfpathlineto{\pgfqpoint{5.132351in}{3.635747in}}%
\pgfpathclose%
\pgfusepath{fill}%
\end{pgfscope}%
\begin{pgfscope}%
\pgfpathrectangle{\pgfqpoint{1.020000in}{0.880000in}}{\pgfqpoint{6.160000in}{6.160000in}}%
\pgfusepath{clip}%
\pgfsetbuttcap%
\pgfsetroundjoin%
\definecolor{currentfill}{rgb}{0.895882,0.849906,0.823499}%
\pgfsetfillcolor{currentfill}%
\pgfsetlinewidth{0.000000pt}%
\definecolor{currentstroke}{rgb}{0.000000,0.000000,0.000000}%
\pgfsetstrokecolor{currentstroke}%
\pgfsetdash{}{0pt}%
\pgfpathmoveto{\pgfqpoint{3.466667in}{4.360324in}}%
\pgfpathlineto{\pgfqpoint{3.473903in}{4.509869in}}%
\pgfpathlineto{\pgfqpoint{3.482681in}{4.442981in}}%
\pgfpathlineto{\pgfqpoint{3.515356in}{4.427851in}}%
\pgfpathlineto{\pgfqpoint{3.547800in}{4.444921in}}%
\pgfpathlineto{\pgfqpoint{3.537991in}{4.670572in}}%
\pgfpathlineto{\pgfqpoint{3.530505in}{4.533283in}}%
\pgfpathlineto{\pgfqpoint{3.498652in}{4.434328in}}%
\pgfpathlineto{\pgfqpoint{3.466667in}{4.360324in}}%
\pgfpathclose%
\pgfusepath{fill}%
\end{pgfscope}%
\begin{pgfscope}%
\pgfpathrectangle{\pgfqpoint{1.020000in}{0.880000in}}{\pgfqpoint{6.160000in}{6.160000in}}%
\pgfusepath{clip}%
\pgfsetbuttcap%
\pgfsetroundjoin%
\definecolor{currentfill}{rgb}{0.619318,0.744121,0.998931}%
\pgfsetfillcolor{currentfill}%
\pgfsetlinewidth{0.000000pt}%
\definecolor{currentstroke}{rgb}{0.000000,0.000000,0.000000}%
\pgfsetstrokecolor{currentstroke}%
\pgfsetdash{}{0pt}%
\pgfpathmoveto{\pgfqpoint{4.349974in}{3.946383in}}%
\pgfpathlineto{\pgfqpoint{4.359306in}{4.022834in}}%
\pgfpathlineto{\pgfqpoint{4.368136in}{3.808706in}}%
\pgfpathlineto{\pgfqpoint{4.400714in}{3.940035in}}%
\pgfpathlineto{\pgfqpoint{4.433089in}{3.962740in}}%
\pgfpathlineto{\pgfqpoint{4.423513in}{3.838962in}}%
\pgfpathlineto{\pgfqpoint{4.414593in}{3.982355in}}%
\pgfpathlineto{\pgfqpoint{4.382264in}{3.953100in}}%
\pgfpathlineto{\pgfqpoint{4.349974in}{3.946383in}}%
\pgfpathclose%
\pgfusepath{fill}%
\end{pgfscope}%
\begin{pgfscope}%
\pgfpathrectangle{\pgfqpoint{1.020000in}{0.880000in}}{\pgfqpoint{6.160000in}{6.160000in}}%
\pgfusepath{clip}%
\pgfsetbuttcap%
\pgfsetroundjoin%
\definecolor{currentfill}{rgb}{0.467678,0.605591,0.968546}%
\pgfsetfillcolor{currentfill}%
\pgfsetlinewidth{0.000000pt}%
\definecolor{currentstroke}{rgb}{0.000000,0.000000,0.000000}%
\pgfsetstrokecolor{currentstroke}%
\pgfsetdash{}{0pt}%
\pgfpathmoveto{\pgfqpoint{5.558342in}{3.744288in}}%
\pgfpathlineto{\pgfqpoint{5.566695in}{3.592795in}}%
\pgfpathlineto{\pgfqpoint{5.578042in}{3.657127in}}%
\pgfpathlineto{\pgfqpoint{5.611239in}{3.742485in}}%
\pgfpathlineto{\pgfqpoint{5.599509in}{3.655205in}}%
\pgfpathlineto{\pgfqpoint{5.587900in}{3.573778in}}%
\pgfpathlineto{\pgfqpoint{5.558342in}{3.744288in}}%
\pgfpathclose%
\pgfusepath{fill}%
\end{pgfscope}%
\begin{pgfscope}%
\pgfpathrectangle{\pgfqpoint{1.020000in}{0.880000in}}{\pgfqpoint{6.160000in}{6.160000in}}%
\pgfusepath{clip}%
\pgfsetbuttcap%
\pgfsetroundjoin%
\definecolor{currentfill}{rgb}{0.576051,0.708780,0.997755}%
\pgfsetfillcolor{currentfill}%
\pgfsetlinewidth{0.000000pt}%
\definecolor{currentstroke}{rgb}{0.000000,0.000000,0.000000}%
\pgfsetstrokecolor{currentstroke}%
\pgfsetdash{}{0pt}%
\pgfpathmoveto{\pgfqpoint{4.497642in}{3.949393in}}%
\pgfpathlineto{\pgfqpoint{4.506427in}{3.771333in}}%
\pgfpathlineto{\pgfqpoint{4.516131in}{3.874725in}}%
\pgfpathlineto{\pgfqpoint{4.547943in}{3.739978in}}%
\pgfpathlineto{\pgfqpoint{4.580073in}{3.711959in}}%
\pgfpathlineto{\pgfqpoint{4.571015in}{3.804783in}}%
\pgfpathlineto{\pgfqpoint{4.562433in}{4.023367in}}%
\pgfpathlineto{\pgfqpoint{4.529795in}{3.916753in}}%
\pgfpathlineto{\pgfqpoint{4.497642in}{3.949393in}}%
\pgfpathclose%
\pgfusepath{fill}%
\end{pgfscope}%
\begin{pgfscope}%
\pgfpathrectangle{\pgfqpoint{1.020000in}{0.880000in}}{\pgfqpoint{6.160000in}{6.160000in}}%
\pgfusepath{clip}%
\pgfsetbuttcap%
\pgfsetroundjoin%
\definecolor{currentfill}{rgb}{0.419991,0.552989,0.942630}%
\pgfsetfillcolor{currentfill}%
\pgfsetlinewidth{0.000000pt}%
\definecolor{currentstroke}{rgb}{0.000000,0.000000,0.000000}%
\pgfsetstrokecolor{currentstroke}%
\pgfsetdash{}{0pt}%
\pgfpathmoveto{\pgfqpoint{5.428414in}{3.587819in}}%
\pgfpathlineto{\pgfqpoint{5.438882in}{3.600406in}}%
\pgfpathlineto{\pgfqpoint{5.448063in}{3.509260in}}%
\pgfpathlineto{\pgfqpoint{5.481486in}{3.618453in}}%
\pgfpathlineto{\pgfqpoint{5.513656in}{3.629409in}}%
\pgfpathlineto{\pgfqpoint{5.502679in}{3.585311in}}%
\pgfpathlineto{\pgfqpoint{5.490357in}{3.435954in}}%
\pgfpathlineto{\pgfqpoint{5.461105in}{3.644135in}}%
\pgfpathlineto{\pgfqpoint{5.428414in}{3.587819in}}%
\pgfpathclose%
\pgfusepath{fill}%
\end{pgfscope}%
\begin{pgfscope}%
\pgfpathrectangle{\pgfqpoint{1.020000in}{0.880000in}}{\pgfqpoint{6.160000in}{6.160000in}}%
\pgfusepath{clip}%
\pgfsetbuttcap%
\pgfsetroundjoin%
\definecolor{currentfill}{rgb}{0.656683,0.771806,0.994914}%
\pgfsetfillcolor{currentfill}%
\pgfsetlinewidth{0.000000pt}%
\definecolor{currentstroke}{rgb}{0.000000,0.000000,0.000000}%
\pgfsetstrokecolor{currentstroke}%
\pgfsetdash{}{0pt}%
\pgfpathmoveto{\pgfqpoint{4.137929in}{3.983177in}}%
\pgfpathlineto{\pgfqpoint{4.146909in}{3.988154in}}%
\pgfpathlineto{\pgfqpoint{4.155857in}{4.181449in}}%
\pgfpathlineto{\pgfqpoint{4.188312in}{4.001417in}}%
\pgfpathlineto{\pgfqpoint{4.220657in}{3.886093in}}%
\pgfpathlineto{\pgfqpoint{4.211577in}{3.790431in}}%
\pgfpathlineto{\pgfqpoint{4.202595in}{4.014945in}}%
\pgfpathlineto{\pgfqpoint{4.170258in}{4.061057in}}%
\pgfpathlineto{\pgfqpoint{4.137929in}{3.983177in}}%
\pgfpathclose%
\pgfusepath{fill}%
\end{pgfscope}%
\begin{pgfscope}%
\pgfpathrectangle{\pgfqpoint{1.020000in}{0.880000in}}{\pgfqpoint{6.160000in}{6.160000in}}%
\pgfusepath{clip}%
\pgfsetbuttcap%
\pgfsetroundjoin%
\definecolor{currentfill}{rgb}{0.521696,0.659599,0.987736}%
\pgfsetfillcolor{currentfill}%
\pgfsetlinewidth{0.000000pt}%
\definecolor{currentstroke}{rgb}{0.000000,0.000000,0.000000}%
\pgfsetstrokecolor{currentstroke}%
\pgfsetdash{}{0pt}%
\pgfpathmoveto{\pgfqpoint{4.857043in}{3.772261in}}%
\pgfpathlineto{\pgfqpoint{4.866567in}{3.742942in}}%
\pgfpathlineto{\pgfqpoint{4.876962in}{3.837572in}}%
\pgfpathlineto{\pgfqpoint{4.907865in}{3.654449in}}%
\pgfpathlineto{\pgfqpoint{4.940880in}{3.769096in}}%
\pgfpathlineto{\pgfqpoint{4.930718in}{3.721050in}}%
\pgfpathlineto{\pgfqpoint{4.922011in}{3.866648in}}%
\pgfpathlineto{\pgfqpoint{4.888399in}{3.659990in}}%
\pgfpathlineto{\pgfqpoint{4.857043in}{3.772261in}}%
\pgfpathclose%
\pgfusepath{fill}%
\end{pgfscope}%
\begin{pgfscope}%
\pgfpathrectangle{\pgfqpoint{1.020000in}{0.880000in}}{\pgfqpoint{6.160000in}{6.160000in}}%
\pgfusepath{clip}%
\pgfsetbuttcap%
\pgfsetroundjoin%
\definecolor{currentfill}{rgb}{0.489246,0.627536,0.976896}%
\pgfsetfillcolor{currentfill}%
\pgfsetlinewidth{0.000000pt}%
\definecolor{currentstroke}{rgb}{0.000000,0.000000,0.000000}%
\pgfsetstrokecolor{currentstroke}%
\pgfsetdash{}{0pt}%
\pgfpathmoveto{\pgfqpoint{5.280782in}{3.657337in}}%
\pgfpathlineto{\pgfqpoint{5.291679in}{3.724480in}}%
\pgfpathlineto{\pgfqpoint{5.302022in}{3.738781in}}%
\pgfpathlineto{\pgfqpoint{5.332678in}{3.613368in}}%
\pgfpathlineto{\pgfqpoint{5.364925in}{3.631287in}}%
\pgfpathlineto{\pgfqpoint{5.356517in}{3.787433in}}%
\pgfpathlineto{\pgfqpoint{5.344959in}{3.672655in}}%
\pgfpathlineto{\pgfqpoint{5.313523in}{3.722302in}}%
\pgfpathlineto{\pgfqpoint{5.280782in}{3.657337in}}%
\pgfpathclose%
\pgfusepath{fill}%
\end{pgfscope}%
\begin{pgfscope}%
\pgfpathrectangle{\pgfqpoint{1.020000in}{0.880000in}}{\pgfqpoint{6.160000in}{6.160000in}}%
\pgfusepath{clip}%
\pgfsetbuttcap%
\pgfsetroundjoin%
\definecolor{currentfill}{rgb}{0.510824,0.649397,0.985079}%
\pgfsetfillcolor{currentfill}%
\pgfsetlinewidth{0.000000pt}%
\definecolor{currentstroke}{rgb}{0.000000,0.000000,0.000000}%
\pgfsetstrokecolor{currentstroke}%
\pgfsetdash{}{0pt}%
\pgfpathmoveto{\pgfqpoint{5.069947in}{3.826219in}}%
\pgfpathlineto{\pgfqpoint{5.078515in}{3.671411in}}%
\pgfpathlineto{\pgfqpoint{5.088968in}{3.726453in}}%
\pgfpathlineto{\pgfqpoint{5.121396in}{3.757142in}}%
\pgfpathlineto{\pgfqpoint{5.153617in}{3.766294in}}%
\pgfpathlineto{\pgfqpoint{5.143230in}{3.728453in}}%
\pgfpathlineto{\pgfqpoint{5.132351in}{3.635747in}}%
\pgfpathlineto{\pgfqpoint{5.101218in}{3.734245in}}%
\pgfpathlineto{\pgfqpoint{5.069947in}{3.826219in}}%
\pgfpathclose%
\pgfusepath{fill}%
\end{pgfscope}%
\begin{pgfscope}%
\pgfpathrectangle{\pgfqpoint{1.020000in}{0.880000in}}{\pgfqpoint{6.160000in}{6.160000in}}%
\pgfusepath{clip}%
\pgfsetbuttcap%
\pgfsetroundjoin%
\definecolor{currentfill}{rgb}{0.457046,0.594006,0.963029}%
\pgfsetfillcolor{currentfill}%
\pgfsetlinewidth{0.000000pt}%
\definecolor{currentstroke}{rgb}{0.000000,0.000000,0.000000}%
\pgfsetstrokecolor{currentstroke}%
\pgfsetdash{}{0pt}%
\pgfpathmoveto{\pgfqpoint{5.216886in}{3.674448in}}%
\pgfpathlineto{\pgfqpoint{5.224492in}{3.435216in}}%
\pgfpathlineto{\pgfqpoint{5.237463in}{3.708963in}}%
\pgfpathlineto{\pgfqpoint{5.268579in}{3.616035in}}%
\pgfpathlineto{\pgfqpoint{5.302022in}{3.738781in}}%
\pgfpathlineto{\pgfqpoint{5.291679in}{3.724480in}}%
\pgfpathlineto{\pgfqpoint{5.280782in}{3.657337in}}%
\pgfpathlineto{\pgfqpoint{5.247742in}{3.562022in}}%
\pgfpathlineto{\pgfqpoint{5.216886in}{3.674448in}}%
\pgfpathclose%
\pgfusepath{fill}%
\end{pgfscope}%
\begin{pgfscope}%
\pgfpathrectangle{\pgfqpoint{1.020000in}{0.880000in}}{\pgfqpoint{6.160000in}{6.160000in}}%
\pgfusepath{clip}%
\pgfsetbuttcap%
\pgfsetroundjoin%
\definecolor{currentfill}{rgb}{0.949151,0.790785,0.710876}%
\pgfsetfillcolor{currentfill}%
\pgfsetlinewidth{0.000000pt}%
\definecolor{currentstroke}{rgb}{0.000000,0.000000,0.000000}%
\pgfsetstrokecolor{currentstroke}%
\pgfsetdash{}{0pt}%
\pgfpathmoveto{\pgfqpoint{3.105809in}{4.664342in}}%
\pgfpathlineto{\pgfqpoint{3.114815in}{4.562976in}}%
\pgfpathlineto{\pgfqpoint{3.121134in}{4.717868in}}%
\pgfpathlineto{\pgfqpoint{3.153326in}{4.770333in}}%
\pgfpathlineto{\pgfqpoint{3.187820in}{4.590332in}}%
\pgfpathlineto{\pgfqpoint{3.180203in}{4.551050in}}%
\pgfpathlineto{\pgfqpoint{3.171496in}{4.622960in}}%
\pgfpathlineto{\pgfqpoint{3.137676in}{4.739777in}}%
\pgfpathlineto{\pgfqpoint{3.105809in}{4.664342in}}%
\pgfpathclose%
\pgfusepath{fill}%
\end{pgfscope}%
\begin{pgfscope}%
\pgfpathrectangle{\pgfqpoint{1.020000in}{0.880000in}}{\pgfqpoint{6.160000in}{6.160000in}}%
\pgfusepath{clip}%
\pgfsetbuttcap%
\pgfsetroundjoin%
\definecolor{currentfill}{rgb}{0.772706,0.838978,0.949319}%
\pgfsetfillcolor{currentfill}%
\pgfsetlinewidth{0.000000pt}%
\definecolor{currentstroke}{rgb}{0.000000,0.000000,0.000000}%
\pgfsetstrokecolor{currentstroke}%
\pgfsetdash{}{0pt}%
\pgfpathmoveto{\pgfqpoint{3.842450in}{4.379127in}}%
\pgfpathlineto{\pgfqpoint{3.851635in}{4.231225in}}%
\pgfpathlineto{\pgfqpoint{3.860673in}{4.126690in}}%
\pgfpathlineto{\pgfqpoint{3.893197in}{4.108848in}}%
\pgfpathlineto{\pgfqpoint{3.925388in}{4.209171in}}%
\pgfpathlineto{\pgfqpoint{3.916673in}{4.190902in}}%
\pgfpathlineto{\pgfqpoint{3.908119in}{4.121301in}}%
\pgfpathlineto{\pgfqpoint{3.875403in}{4.228039in}}%
\pgfpathlineto{\pgfqpoint{3.842450in}{4.379127in}}%
\pgfpathclose%
\pgfusepath{fill}%
\end{pgfscope}%
\begin{pgfscope}%
\pgfpathrectangle{\pgfqpoint{1.020000in}{0.880000in}}{\pgfqpoint{6.160000in}{6.160000in}}%
\pgfusepath{clip}%
\pgfsetbuttcap%
\pgfsetroundjoin%
\definecolor{currentfill}{rgb}{0.548876,0.685104,0.994379}%
\pgfsetfillcolor{currentfill}%
\pgfsetlinewidth{0.000000pt}%
\definecolor{currentstroke}{rgb}{0.000000,0.000000,0.000000}%
\pgfsetstrokecolor{currentstroke}%
\pgfsetdash{}{0pt}%
\pgfpathmoveto{\pgfqpoint{4.645193in}{3.866770in}}%
\pgfpathlineto{\pgfqpoint{4.653720in}{3.657038in}}%
\pgfpathlineto{\pgfqpoint{4.663831in}{3.782533in}}%
\pgfpathlineto{\pgfqpoint{4.696340in}{3.831307in}}%
\pgfpathlineto{\pgfqpoint{4.728136in}{3.746454in}}%
\pgfpathlineto{\pgfqpoint{4.718620in}{3.758450in}}%
\pgfpathlineto{\pgfqpoint{4.709639in}{3.869123in}}%
\pgfpathlineto{\pgfqpoint{4.677618in}{3.907194in}}%
\pgfpathlineto{\pgfqpoint{4.645193in}{3.866770in}}%
\pgfpathclose%
\pgfusepath{fill}%
\end{pgfscope}%
\begin{pgfscope}%
\pgfpathrectangle{\pgfqpoint{1.020000in}{0.880000in}}{\pgfqpoint{6.160000in}{6.160000in}}%
\pgfusepath{clip}%
\pgfsetbuttcap%
\pgfsetroundjoin%
\definecolor{currentfill}{rgb}{0.624703,0.748318,0.998719}%
\pgfsetfillcolor{currentfill}%
\pgfsetlinewidth{0.000000pt}%
\definecolor{currentstroke}{rgb}{0.000000,0.000000,0.000000}%
\pgfsetstrokecolor{currentstroke}%
\pgfsetdash{}{0pt}%
\pgfpathmoveto{\pgfqpoint{4.285323in}{3.907129in}}%
\pgfpathlineto{\pgfqpoint{4.294594in}{4.030445in}}%
\pgfpathlineto{\pgfqpoint{4.303474in}{3.793302in}}%
\pgfpathlineto{\pgfqpoint{4.336101in}{3.991204in}}%
\pgfpathlineto{\pgfqpoint{4.368136in}{3.808706in}}%
\pgfpathlineto{\pgfqpoint{4.359306in}{4.022834in}}%
\pgfpathlineto{\pgfqpoint{4.349974in}{3.946383in}}%
\pgfpathlineto{\pgfqpoint{4.317706in}{3.967288in}}%
\pgfpathlineto{\pgfqpoint{4.285323in}{3.907129in}}%
\pgfpathclose%
\pgfusepath{fill}%
\end{pgfscope}%
\begin{pgfscope}%
\pgfpathrectangle{\pgfqpoint{1.020000in}{0.880000in}}{\pgfqpoint{6.160000in}{6.160000in}}%
\pgfusepath{clip}%
\pgfsetbuttcap%
\pgfsetroundjoin%
\definecolor{currentfill}{rgb}{0.703587,0.802586,0.982847}%
\pgfsetfillcolor{currentfill}%
\pgfsetlinewidth{0.000000pt}%
\definecolor{currentstroke}{rgb}{0.000000,0.000000,0.000000}%
\pgfsetstrokecolor{currentstroke}%
\pgfsetdash{}{0pt}%
\pgfpathmoveto{\pgfqpoint{3.925388in}{4.209171in}}%
\pgfpathlineto{\pgfqpoint{3.934364in}{4.130301in}}%
\pgfpathlineto{\pgfqpoint{3.943336in}{4.054706in}}%
\pgfpathlineto{\pgfqpoint{3.976072in}{3.927126in}}%
\pgfpathlineto{\pgfqpoint{4.008242in}{4.067252in}}%
\pgfpathlineto{\pgfqpoint{3.999326in}{4.104618in}}%
\pgfpathlineto{\pgfqpoint{3.990825in}{3.931002in}}%
\pgfpathlineto{\pgfqpoint{3.958087in}{4.102653in}}%
\pgfpathlineto{\pgfqpoint{3.925388in}{4.209171in}}%
\pgfpathclose%
\pgfusepath{fill}%
\end{pgfscope}%
\begin{pgfscope}%
\pgfpathrectangle{\pgfqpoint{1.020000in}{0.880000in}}{\pgfqpoint{6.160000in}{6.160000in}}%
\pgfusepath{clip}%
\pgfsetbuttcap%
\pgfsetroundjoin%
\definecolor{currentfill}{rgb}{0.822420,0.856898,0.910795}%
\pgfsetfillcolor{currentfill}%
\pgfsetlinewidth{0.000000pt}%
\definecolor{currentstroke}{rgb}{0.000000,0.000000,0.000000}%
\pgfsetstrokecolor{currentstroke}%
\pgfsetdash{}{0pt}%
\pgfpathmoveto{\pgfqpoint{3.696066in}{4.199840in}}%
\pgfpathlineto{\pgfqpoint{3.703630in}{4.402161in}}%
\pgfpathlineto{\pgfqpoint{3.712207in}{4.398448in}}%
\pgfpathlineto{\pgfqpoint{3.745240in}{4.292360in}}%
\pgfpathlineto{\pgfqpoint{3.778084in}{4.211472in}}%
\pgfpathlineto{\pgfqpoint{3.769106in}{4.302484in}}%
\pgfpathlineto{\pgfqpoint{3.760469in}{4.312514in}}%
\pgfpathlineto{\pgfqpoint{3.727839in}{4.349072in}}%
\pgfpathlineto{\pgfqpoint{3.696066in}{4.199840in}}%
\pgfpathclose%
\pgfusepath{fill}%
\end{pgfscope}%
\begin{pgfscope}%
\pgfpathrectangle{\pgfqpoint{1.020000in}{0.880000in}}{\pgfqpoint{6.160000in}{6.160000in}}%
\pgfusepath{clip}%
\pgfsetbuttcap%
\pgfsetroundjoin%
\definecolor{currentfill}{rgb}{0.933221,0.815557,0.753151}%
\pgfsetfillcolor{currentfill}%
\pgfsetlinewidth{0.000000pt}%
\definecolor{currentstroke}{rgb}{0.000000,0.000000,0.000000}%
\pgfsetstrokecolor{currentstroke}%
\pgfsetdash{}{0pt}%
\pgfpathmoveto{\pgfqpoint{3.252704in}{4.632866in}}%
\pgfpathlineto{\pgfqpoint{3.262208in}{4.477602in}}%
\pgfpathlineto{\pgfqpoint{3.267849in}{4.748999in}}%
\pgfpathlineto{\pgfqpoint{3.302478in}{4.530252in}}%
\pgfpathlineto{\pgfqpoint{3.334692in}{4.580252in}}%
\pgfpathlineto{\pgfqpoint{3.326542in}{4.578676in}}%
\pgfpathlineto{\pgfqpoint{3.318483in}{4.568531in}}%
\pgfpathlineto{\pgfqpoint{3.285892in}{4.569314in}}%
\pgfpathlineto{\pgfqpoint{3.252704in}{4.632866in}}%
\pgfpathclose%
\pgfusepath{fill}%
\end{pgfscope}%
\begin{pgfscope}%
\pgfpathrectangle{\pgfqpoint{1.020000in}{0.880000in}}{\pgfqpoint{6.160000in}{6.160000in}}%
\pgfusepath{clip}%
\pgfsetbuttcap%
\pgfsetroundjoin%
\definecolor{currentfill}{rgb}{0.906154,0.842091,0.806151}%
\pgfsetfillcolor{currentfill}%
\pgfsetlinewidth{0.000000pt}%
\definecolor{currentstroke}{rgb}{0.000000,0.000000,0.000000}%
\pgfsetstrokecolor{currentstroke}%
\pgfsetdash{}{0pt}%
\pgfpathmoveto{\pgfqpoint{3.399866in}{4.589332in}}%
\pgfpathlineto{\pgfqpoint{3.408568in}{4.529253in}}%
\pgfpathlineto{\pgfqpoint{3.417102in}{4.492349in}}%
\pgfpathlineto{\pgfqpoint{3.449450in}{4.531003in}}%
\pgfpathlineto{\pgfqpoint{3.482681in}{4.442981in}}%
\pgfpathlineto{\pgfqpoint{3.473903in}{4.509869in}}%
\pgfpathlineto{\pgfqpoint{3.466667in}{4.360324in}}%
\pgfpathlineto{\pgfqpoint{3.432785in}{4.545894in}}%
\pgfpathlineto{\pgfqpoint{3.399866in}{4.589332in}}%
\pgfpathclose%
\pgfusepath{fill}%
\end{pgfscope}%
\begin{pgfscope}%
\pgfpathrectangle{\pgfqpoint{1.020000in}{0.880000in}}{\pgfqpoint{6.160000in}{6.160000in}}%
\pgfusepath{clip}%
\pgfsetbuttcap%
\pgfsetroundjoin%
\definecolor{currentfill}{rgb}{0.875557,0.860242,0.851430}%
\pgfsetfillcolor{currentfill}%
\pgfsetlinewidth{0.000000pt}%
\definecolor{currentstroke}{rgb}{0.000000,0.000000,0.000000}%
\pgfsetstrokecolor{currentstroke}%
\pgfsetdash{}{0pt}%
\pgfpathmoveto{\pgfqpoint{3.547800in}{4.444921in}}%
\pgfpathlineto{\pgfqpoint{3.557706in}{4.200293in}}%
\pgfpathlineto{\pgfqpoint{3.564275in}{4.496882in}}%
\pgfpathlineto{\pgfqpoint{3.596382in}{4.584688in}}%
\pgfpathlineto{\pgfqpoint{3.629887in}{4.424366in}}%
\pgfpathlineto{\pgfqpoint{3.621631in}{4.386666in}}%
\pgfpathlineto{\pgfqpoint{3.613478in}{4.335090in}}%
\pgfpathlineto{\pgfqpoint{3.580016in}{4.498818in}}%
\pgfpathlineto{\pgfqpoint{3.547800in}{4.444921in}}%
\pgfpathclose%
\pgfusepath{fill}%
\end{pgfscope}%
\begin{pgfscope}%
\pgfpathrectangle{\pgfqpoint{1.020000in}{0.880000in}}{\pgfqpoint{6.160000in}{6.160000in}}%
\pgfusepath{clip}%
\pgfsetbuttcap%
\pgfsetroundjoin%
\definecolor{currentfill}{rgb}{0.548876,0.685104,0.994379}%
\pgfsetfillcolor{currentfill}%
\pgfsetlinewidth{0.000000pt}%
\definecolor{currentstroke}{rgb}{0.000000,0.000000,0.000000}%
\pgfsetstrokecolor{currentstroke}%
\pgfsetdash{}{0pt}%
\pgfpathmoveto{\pgfqpoint{4.580073in}{3.711959in}}%
\pgfpathlineto{\pgfqpoint{4.589731in}{3.766670in}}%
\pgfpathlineto{\pgfqpoint{4.599026in}{3.726931in}}%
\pgfpathlineto{\pgfqpoint{4.632084in}{3.901240in}}%
\pgfpathlineto{\pgfqpoint{4.663831in}{3.782533in}}%
\pgfpathlineto{\pgfqpoint{4.653720in}{3.657038in}}%
\pgfpathlineto{\pgfqpoint{4.645193in}{3.866770in}}%
\pgfpathlineto{\pgfqpoint{4.613209in}{3.927621in}}%
\pgfpathlineto{\pgfqpoint{4.580073in}{3.711959in}}%
\pgfpathclose%
\pgfusepath{fill}%
\end{pgfscope}%
\begin{pgfscope}%
\pgfpathrectangle{\pgfqpoint{1.020000in}{0.880000in}}{\pgfqpoint{6.160000in}{6.160000in}}%
\pgfusepath{clip}%
\pgfsetbuttcap%
\pgfsetroundjoin%
\definecolor{currentfill}{rgb}{0.441123,0.576532,0.954545}%
\pgfsetfillcolor{currentfill}%
\pgfsetlinewidth{0.000000pt}%
\definecolor{currentstroke}{rgb}{0.000000,0.000000,0.000000}%
\pgfsetstrokecolor{currentstroke}%
\pgfsetdash{}{0pt}%
\pgfpathmoveto{\pgfqpoint{5.364925in}{3.631287in}}%
\pgfpathlineto{\pgfqpoint{5.376589in}{3.749742in}}%
\pgfpathlineto{\pgfqpoint{5.385012in}{3.594299in}}%
\pgfpathlineto{\pgfqpoint{5.416265in}{3.527858in}}%
\pgfpathlineto{\pgfqpoint{5.448063in}{3.509260in}}%
\pgfpathlineto{\pgfqpoint{5.438882in}{3.600406in}}%
\pgfpathlineto{\pgfqpoint{5.428414in}{3.587819in}}%
\pgfpathlineto{\pgfqpoint{5.397159in}{3.648641in}}%
\pgfpathlineto{\pgfqpoint{5.364925in}{3.631287in}}%
\pgfpathclose%
\pgfusepath{fill}%
\end{pgfscope}%
\begin{pgfscope}%
\pgfpathrectangle{\pgfqpoint{1.020000in}{0.880000in}}{\pgfqpoint{6.160000in}{6.160000in}}%
\pgfusepath{clip}%
\pgfsetbuttcap%
\pgfsetroundjoin%
\definecolor{currentfill}{rgb}{0.624703,0.748318,0.998719}%
\pgfsetfillcolor{currentfill}%
\pgfsetlinewidth{0.000000pt}%
\definecolor{currentstroke}{rgb}{0.000000,0.000000,0.000000}%
\pgfsetstrokecolor{currentstroke}%
\pgfsetdash{}{0pt}%
\pgfpathmoveto{\pgfqpoint{4.220657in}{3.886093in}}%
\pgfpathlineto{\pgfqpoint{4.229762in}{3.960832in}}%
\pgfpathlineto{\pgfqpoint{4.238876in}{3.992651in}}%
\pgfpathlineto{\pgfqpoint{4.271099in}{3.759117in}}%
\pgfpathlineto{\pgfqpoint{4.303474in}{3.793302in}}%
\pgfpathlineto{\pgfqpoint{4.294594in}{4.030445in}}%
\pgfpathlineto{\pgfqpoint{4.285323in}{3.907129in}}%
\pgfpathlineto{\pgfqpoint{4.253106in}{4.055760in}}%
\pgfpathlineto{\pgfqpoint{4.220657in}{3.886093in}}%
\pgfpathclose%
\pgfusepath{fill}%
\end{pgfscope}%
\begin{pgfscope}%
\pgfpathrectangle{\pgfqpoint{1.020000in}{0.880000in}}{\pgfqpoint{6.160000in}{6.160000in}}%
\pgfusepath{clip}%
\pgfsetbuttcap%
\pgfsetroundjoin%
\definecolor{currentfill}{rgb}{0.603162,0.731527,0.999565}%
\pgfsetfillcolor{currentfill}%
\pgfsetlinewidth{0.000000pt}%
\definecolor{currentstroke}{rgb}{0.000000,0.000000,0.000000}%
\pgfsetstrokecolor{currentstroke}%
\pgfsetdash{}{0pt}%
\pgfpathmoveto{\pgfqpoint{4.433089in}{3.962740in}}%
\pgfpathlineto{\pgfqpoint{4.441690in}{3.691983in}}%
\pgfpathlineto{\pgfqpoint{4.451717in}{3.966953in}}%
\pgfpathlineto{\pgfqpoint{4.483869in}{3.894290in}}%
\pgfpathlineto{\pgfqpoint{4.516131in}{3.874725in}}%
\pgfpathlineto{\pgfqpoint{4.506427in}{3.771333in}}%
\pgfpathlineto{\pgfqpoint{4.497642in}{3.949393in}}%
\pgfpathlineto{\pgfqpoint{4.465561in}{4.021686in}}%
\pgfpathlineto{\pgfqpoint{4.433089in}{3.962740in}}%
\pgfpathclose%
\pgfusepath{fill}%
\end{pgfscope}%
\begin{pgfscope}%
\pgfpathrectangle{\pgfqpoint{1.020000in}{0.880000in}}{\pgfqpoint{6.160000in}{6.160000in}}%
\pgfusepath{clip}%
\pgfsetbuttcap%
\pgfsetroundjoin%
\definecolor{currentfill}{rgb}{0.677823,0.786546,0.991005}%
\pgfsetfillcolor{currentfill}%
\pgfsetlinewidth{0.000000pt}%
\definecolor{currentstroke}{rgb}{0.000000,0.000000,0.000000}%
\pgfsetstrokecolor{currentstroke}%
\pgfsetdash{}{0pt}%
\pgfpathmoveto{\pgfqpoint{4.073279in}{3.907599in}}%
\pgfpathlineto{\pgfqpoint{4.081947in}{4.151185in}}%
\pgfpathlineto{\pgfqpoint{4.091066in}{3.970578in}}%
\pgfpathlineto{\pgfqpoint{4.123465in}{4.028243in}}%
\pgfpathlineto{\pgfqpoint{4.155857in}{4.181449in}}%
\pgfpathlineto{\pgfqpoint{4.146909in}{3.988154in}}%
\pgfpathlineto{\pgfqpoint{4.137929in}{3.983177in}}%
\pgfpathlineto{\pgfqpoint{4.105615in}{3.927112in}}%
\pgfpathlineto{\pgfqpoint{4.073279in}{3.907599in}}%
\pgfpathclose%
\pgfusepath{fill}%
\end{pgfscope}%
\begin{pgfscope}%
\pgfpathrectangle{\pgfqpoint{1.020000in}{0.880000in}}{\pgfqpoint{6.160000in}{6.160000in}}%
\pgfusepath{clip}%
\pgfsetbuttcap%
\pgfsetroundjoin%
\definecolor{currentfill}{rgb}{0.527132,0.664700,0.989065}%
\pgfsetfillcolor{currentfill}%
\pgfsetlinewidth{0.000000pt}%
\definecolor{currentstroke}{rgb}{0.000000,0.000000,0.000000}%
\pgfsetstrokecolor{currentstroke}%
\pgfsetdash{}{0pt}%
\pgfpathmoveto{\pgfqpoint{4.516131in}{3.874725in}}%
\pgfpathlineto{\pgfqpoint{4.525192in}{3.778036in}}%
\pgfpathlineto{\pgfqpoint{4.534677in}{3.802391in}}%
\pgfpathlineto{\pgfqpoint{4.566303in}{3.615659in}}%
\pgfpathlineto{\pgfqpoint{4.599026in}{3.726931in}}%
\pgfpathlineto{\pgfqpoint{4.589731in}{3.766670in}}%
\pgfpathlineto{\pgfqpoint{4.580073in}{3.711959in}}%
\pgfpathlineto{\pgfqpoint{4.547943in}{3.739978in}}%
\pgfpathlineto{\pgfqpoint{4.516131in}{3.874725in}}%
\pgfpathclose%
\pgfusepath{fill}%
\end{pgfscope}%
\begin{pgfscope}%
\pgfpathrectangle{\pgfqpoint{1.020000in}{0.880000in}}{\pgfqpoint{6.160000in}{6.160000in}}%
\pgfusepath{clip}%
\pgfsetbuttcap%
\pgfsetroundjoin%
\definecolor{currentfill}{rgb}{0.543440,0.680003,0.993051}%
\pgfsetfillcolor{currentfill}%
\pgfsetlinewidth{0.000000pt}%
\definecolor{currentstroke}{rgb}{0.000000,0.000000,0.000000}%
\pgfsetstrokecolor{currentstroke}%
\pgfsetdash{}{0pt}%
\pgfpathmoveto{\pgfqpoint{4.792348in}{3.714794in}}%
\pgfpathlineto{\pgfqpoint{4.802419in}{3.781866in}}%
\pgfpathlineto{\pgfqpoint{4.811632in}{3.706782in}}%
\pgfpathlineto{\pgfqpoint{4.844995in}{3.879935in}}%
\pgfpathlineto{\pgfqpoint{4.876962in}{3.837572in}}%
\pgfpathlineto{\pgfqpoint{4.866567in}{3.742942in}}%
\pgfpathlineto{\pgfqpoint{4.857043in}{3.772261in}}%
\pgfpathlineto{\pgfqpoint{4.825041in}{3.797547in}}%
\pgfpathlineto{\pgfqpoint{4.792348in}{3.714794in}}%
\pgfpathclose%
\pgfusepath{fill}%
\end{pgfscope}%
\begin{pgfscope}%
\pgfpathrectangle{\pgfqpoint{1.020000in}{0.880000in}}{\pgfqpoint{6.160000in}{6.160000in}}%
\pgfusepath{clip}%
\pgfsetbuttcap%
\pgfsetroundjoin%
\definecolor{currentfill}{rgb}{0.478462,0.616564,0.972721}%
\pgfsetfillcolor{currentfill}%
\pgfsetlinewidth{0.000000pt}%
\definecolor{currentstroke}{rgb}{0.000000,0.000000,0.000000}%
\pgfsetstrokecolor{currentstroke}%
\pgfsetdash{}{0pt}%
\pgfpathmoveto{\pgfqpoint{5.153617in}{3.766294in}}%
\pgfpathlineto{\pgfqpoint{5.162462in}{3.643439in}}%
\pgfpathlineto{\pgfqpoint{5.172094in}{3.600638in}}%
\pgfpathlineto{\pgfqpoint{5.205900in}{3.765659in}}%
\pgfpathlineto{\pgfqpoint{5.237463in}{3.708963in}}%
\pgfpathlineto{\pgfqpoint{5.224492in}{3.435216in}}%
\pgfpathlineto{\pgfqpoint{5.216886in}{3.674448in}}%
\pgfpathlineto{\pgfqpoint{5.185336in}{3.726469in}}%
\pgfpathlineto{\pgfqpoint{5.153617in}{3.766294in}}%
\pgfpathclose%
\pgfusepath{fill}%
\end{pgfscope}%
\begin{pgfscope}%
\pgfpathrectangle{\pgfqpoint{1.020000in}{0.880000in}}{\pgfqpoint{6.160000in}{6.160000in}}%
\pgfusepath{clip}%
\pgfsetbuttcap%
\pgfsetroundjoin%
\definecolor{currentfill}{rgb}{0.538004,0.674902,0.991722}%
\pgfsetfillcolor{currentfill}%
\pgfsetlinewidth{0.000000pt}%
\definecolor{currentstroke}{rgb}{0.000000,0.000000,0.000000}%
\pgfsetstrokecolor{currentstroke}%
\pgfsetdash{}{0pt}%
\pgfpathmoveto{\pgfqpoint{5.005626in}{3.821279in}}%
\pgfpathlineto{\pgfqpoint{5.014876in}{3.744211in}}%
\pgfpathlineto{\pgfqpoint{5.024849in}{3.752571in}}%
\pgfpathlineto{\pgfqpoint{5.057778in}{3.837894in}}%
\pgfpathlineto{\pgfqpoint{5.088968in}{3.726453in}}%
\pgfpathlineto{\pgfqpoint{5.078515in}{3.671411in}}%
\pgfpathlineto{\pgfqpoint{5.069947in}{3.826219in}}%
\pgfpathlineto{\pgfqpoint{5.037519in}{3.791360in}}%
\pgfpathlineto{\pgfqpoint{5.005626in}{3.821279in}}%
\pgfpathclose%
\pgfusepath{fill}%
\end{pgfscope}%
\begin{pgfscope}%
\pgfpathrectangle{\pgfqpoint{1.020000in}{0.880000in}}{\pgfqpoint{6.160000in}{6.160000in}}%
\pgfusepath{clip}%
\pgfsetbuttcap%
\pgfsetroundjoin%
\definecolor{currentfill}{rgb}{0.796064,0.848693,0.933471}%
\pgfsetfillcolor{currentfill}%
\pgfsetlinewidth{0.000000pt}%
\definecolor{currentstroke}{rgb}{0.000000,0.000000,0.000000}%
\pgfsetstrokecolor{currentstroke}%
\pgfsetdash{}{0pt}%
\pgfpathmoveto{\pgfqpoint{3.778084in}{4.211472in}}%
\pgfpathlineto{\pgfqpoint{3.786157in}{4.351167in}}%
\pgfpathlineto{\pgfqpoint{3.795273in}{4.230684in}}%
\pgfpathlineto{\pgfqpoint{3.827477in}{4.326968in}}%
\pgfpathlineto{\pgfqpoint{3.860673in}{4.126690in}}%
\pgfpathlineto{\pgfqpoint{3.851635in}{4.231225in}}%
\pgfpathlineto{\pgfqpoint{3.842450in}{4.379127in}}%
\pgfpathlineto{\pgfqpoint{3.811042in}{4.080846in}}%
\pgfpathlineto{\pgfqpoint{3.778084in}{4.211472in}}%
\pgfpathclose%
\pgfusepath{fill}%
\end{pgfscope}%
\begin{pgfscope}%
\pgfpathrectangle{\pgfqpoint{1.020000in}{0.880000in}}{\pgfqpoint{6.160000in}{6.160000in}}%
\pgfusepath{clip}%
\pgfsetbuttcap%
\pgfsetroundjoin%
\definecolor{currentfill}{rgb}{0.516260,0.654498,0.986407}%
\pgfsetfillcolor{currentfill}%
\pgfsetlinewidth{0.000000pt}%
\definecolor{currentstroke}{rgb}{0.000000,0.000000,0.000000}%
\pgfsetstrokecolor{currentstroke}%
\pgfsetdash{}{0pt}%
\pgfpathmoveto{\pgfqpoint{4.940880in}{3.769096in}}%
\pgfpathlineto{\pgfqpoint{4.950635in}{3.760065in}}%
\pgfpathlineto{\pgfqpoint{4.960120in}{3.713782in}}%
\pgfpathlineto{\pgfqpoint{4.991130in}{3.564974in}}%
\pgfpathlineto{\pgfqpoint{5.024849in}{3.752571in}}%
\pgfpathlineto{\pgfqpoint{5.014876in}{3.744211in}}%
\pgfpathlineto{\pgfqpoint{5.005626in}{3.821279in}}%
\pgfpathlineto{\pgfqpoint{4.972662in}{3.719960in}}%
\pgfpathlineto{\pgfqpoint{4.940880in}{3.769096in}}%
\pgfpathclose%
\pgfusepath{fill}%
\end{pgfscope}%
\begin{pgfscope}%
\pgfpathrectangle{\pgfqpoint{1.020000in}{0.880000in}}{\pgfqpoint{6.160000in}{6.160000in}}%
\pgfusepath{clip}%
\pgfsetbuttcap%
\pgfsetroundjoin%
\definecolor{currentfill}{rgb}{0.538004,0.674902,0.991722}%
\pgfsetfillcolor{currentfill}%
\pgfsetlinewidth{0.000000pt}%
\definecolor{currentstroke}{rgb}{0.000000,0.000000,0.000000}%
\pgfsetstrokecolor{currentstroke}%
\pgfsetdash{}{0pt}%
\pgfpathmoveto{\pgfqpoint{4.728136in}{3.746454in}}%
\pgfpathlineto{\pgfqpoint{4.737948in}{3.784756in}}%
\pgfpathlineto{\pgfqpoint{4.748027in}{3.865832in}}%
\pgfpathlineto{\pgfqpoint{4.779196in}{3.673884in}}%
\pgfpathlineto{\pgfqpoint{4.811632in}{3.706782in}}%
\pgfpathlineto{\pgfqpoint{4.802419in}{3.781866in}}%
\pgfpathlineto{\pgfqpoint{4.792348in}{3.714794in}}%
\pgfpathlineto{\pgfqpoint{4.761032in}{3.865459in}}%
\pgfpathlineto{\pgfqpoint{4.728136in}{3.746454in}}%
\pgfpathclose%
\pgfusepath{fill}%
\end{pgfscope}%
\begin{pgfscope}%
\pgfpathrectangle{\pgfqpoint{1.020000in}{0.880000in}}{\pgfqpoint{6.160000in}{6.160000in}}%
\pgfusepath{clip}%
\pgfsetbuttcap%
\pgfsetroundjoin%
\definecolor{currentfill}{rgb}{0.847365,0.862472,0.885540}%
\pgfsetfillcolor{currentfill}%
\pgfsetlinewidth{0.000000pt}%
\definecolor{currentstroke}{rgb}{0.000000,0.000000,0.000000}%
\pgfsetstrokecolor{currentstroke}%
\pgfsetdash{}{0pt}%
\pgfpathmoveto{\pgfqpoint{3.629887in}{4.424366in}}%
\pgfpathlineto{\pgfqpoint{3.638244in}{4.447462in}}%
\pgfpathlineto{\pgfqpoint{3.648322in}{4.150680in}}%
\pgfpathlineto{\pgfqpoint{3.679921in}{4.337200in}}%
\pgfpathlineto{\pgfqpoint{3.712207in}{4.398448in}}%
\pgfpathlineto{\pgfqpoint{3.703630in}{4.402161in}}%
\pgfpathlineto{\pgfqpoint{3.696066in}{4.199840in}}%
\pgfpathlineto{\pgfqpoint{3.662420in}{4.428738in}}%
\pgfpathlineto{\pgfqpoint{3.629887in}{4.424366in}}%
\pgfpathclose%
\pgfusepath{fill}%
\end{pgfscope}%
\begin{pgfscope}%
\pgfpathrectangle{\pgfqpoint{1.020000in}{0.880000in}}{\pgfqpoint{6.160000in}{6.160000in}}%
\pgfusepath{clip}%
\pgfsetbuttcap%
\pgfsetroundjoin%
\definecolor{currentfill}{rgb}{0.457046,0.594006,0.963029}%
\pgfsetfillcolor{currentfill}%
\pgfsetlinewidth{0.000000pt}%
\definecolor{currentstroke}{rgb}{0.000000,0.000000,0.000000}%
\pgfsetstrokecolor{currentstroke}%
\pgfsetdash{}{0pt}%
\pgfpathmoveto{\pgfqpoint{5.578042in}{3.657127in}}%
\pgfpathlineto{\pgfqpoint{5.588391in}{3.647765in}}%
\pgfpathlineto{\pgfqpoint{5.597366in}{3.539817in}}%
\pgfpathlineto{\pgfqpoint{5.630672in}{3.629925in}}%
\pgfpathlineto{\pgfqpoint{5.618888in}{3.542399in}}%
\pgfpathlineto{\pgfqpoint{5.611239in}{3.742485in}}%
\pgfpathlineto{\pgfqpoint{5.578042in}{3.657127in}}%
\pgfpathclose%
\pgfusepath{fill}%
\end{pgfscope}%
\begin{pgfscope}%
\pgfpathrectangle{\pgfqpoint{1.020000in}{0.880000in}}{\pgfqpoint{6.160000in}{6.160000in}}%
\pgfusepath{clip}%
\pgfsetbuttcap%
\pgfsetroundjoin%
\definecolor{currentfill}{rgb}{0.962708,0.753557,0.655601}%
\pgfsetfillcolor{currentfill}%
\pgfsetlinewidth{0.000000pt}%
\definecolor{currentstroke}{rgb}{0.000000,0.000000,0.000000}%
\pgfsetstrokecolor{currentstroke}%
\pgfsetdash{}{0pt}%
\pgfpathmoveto{\pgfqpoint{3.038852in}{4.806985in}}%
\pgfpathlineto{\pgfqpoint{3.047158in}{4.769289in}}%
\pgfpathlineto{\pgfqpoint{3.055377in}{4.740374in}}%
\pgfpathlineto{\pgfqpoint{3.088670in}{4.691947in}}%
\pgfpathlineto{\pgfqpoint{3.121134in}{4.717868in}}%
\pgfpathlineto{\pgfqpoint{3.114815in}{4.562976in}}%
\pgfpathlineto{\pgfqpoint{3.105809in}{4.664342in}}%
\pgfpathlineto{\pgfqpoint{3.071410in}{4.822567in}}%
\pgfpathlineto{\pgfqpoint{3.038852in}{4.806985in}}%
\pgfpathclose%
\pgfusepath{fill}%
\end{pgfscope}%
\begin{pgfscope}%
\pgfpathrectangle{\pgfqpoint{1.020000in}{0.880000in}}{\pgfqpoint{6.160000in}{6.160000in}}%
\pgfusepath{clip}%
\pgfsetbuttcap%
\pgfsetroundjoin%
\definecolor{currentfill}{rgb}{0.891817,0.851973,0.829085}%
\pgfsetfillcolor{currentfill}%
\pgfsetlinewidth{0.000000pt}%
\definecolor{currentstroke}{rgb}{0.000000,0.000000,0.000000}%
\pgfsetstrokecolor{currentstroke}%
\pgfsetdash{}{0pt}%
\pgfpathmoveto{\pgfqpoint{3.482681in}{4.442981in}}%
\pgfpathlineto{\pgfqpoint{3.490221in}{4.557784in}}%
\pgfpathlineto{\pgfqpoint{3.499821in}{4.373549in}}%
\pgfpathlineto{\pgfqpoint{3.530633in}{4.652963in}}%
\pgfpathlineto{\pgfqpoint{3.564275in}{4.496882in}}%
\pgfpathlineto{\pgfqpoint{3.557706in}{4.200293in}}%
\pgfpathlineto{\pgfqpoint{3.547800in}{4.444921in}}%
\pgfpathlineto{\pgfqpoint{3.515356in}{4.427851in}}%
\pgfpathlineto{\pgfqpoint{3.482681in}{4.442981in}}%
\pgfpathclose%
\pgfusepath{fill}%
\end{pgfscope}%
\begin{pgfscope}%
\pgfpathrectangle{\pgfqpoint{1.020000in}{0.880000in}}{\pgfqpoint{6.160000in}{6.160000in}}%
\pgfusepath{clip}%
\pgfsetbuttcap%
\pgfsetroundjoin%
\definecolor{currentfill}{rgb}{0.613933,0.739923,0.999142}%
\pgfsetfillcolor{currentfill}%
\pgfsetlinewidth{0.000000pt}%
\definecolor{currentstroke}{rgb}{0.000000,0.000000,0.000000}%
\pgfsetstrokecolor{currentstroke}%
\pgfsetdash{}{0pt}%
\pgfpathmoveto{\pgfqpoint{4.368136in}{3.808706in}}%
\pgfpathlineto{\pgfqpoint{4.377727in}{3.998683in}}%
\pgfpathlineto{\pgfqpoint{4.386762in}{3.887707in}}%
\pgfpathlineto{\pgfqpoint{4.419321in}{3.963427in}}%
\pgfpathlineto{\pgfqpoint{4.451717in}{3.966953in}}%
\pgfpathlineto{\pgfqpoint{4.441690in}{3.691983in}}%
\pgfpathlineto{\pgfqpoint{4.433089in}{3.962740in}}%
\pgfpathlineto{\pgfqpoint{4.400714in}{3.940035in}}%
\pgfpathlineto{\pgfqpoint{4.368136in}{3.808706in}}%
\pgfpathclose%
\pgfusepath{fill}%
\end{pgfscope}%
\begin{pgfscope}%
\pgfpathrectangle{\pgfqpoint{1.020000in}{0.880000in}}{\pgfqpoint{6.160000in}{6.160000in}}%
\pgfusepath{clip}%
\pgfsetbuttcap%
\pgfsetroundjoin%
\definecolor{currentfill}{rgb}{0.925563,0.825517,0.771136}%
\pgfsetfillcolor{currentfill}%
\pgfsetlinewidth{0.000000pt}%
\definecolor{currentstroke}{rgb}{0.000000,0.000000,0.000000}%
\pgfsetstrokecolor{currentstroke}%
\pgfsetdash{}{0pt}%
\pgfpathmoveto{\pgfqpoint{3.334692in}{4.580252in}}%
\pgfpathlineto{\pgfqpoint{3.343293in}{4.529568in}}%
\pgfpathlineto{\pgfqpoint{3.349725in}{4.743575in}}%
\pgfpathlineto{\pgfqpoint{3.384437in}{4.494767in}}%
\pgfpathlineto{\pgfqpoint{3.417102in}{4.492349in}}%
\pgfpathlineto{\pgfqpoint{3.408568in}{4.529253in}}%
\pgfpathlineto{\pgfqpoint{3.399866in}{4.589332in}}%
\pgfpathlineto{\pgfqpoint{3.368808in}{4.396182in}}%
\pgfpathlineto{\pgfqpoint{3.334692in}{4.580252in}}%
\pgfpathclose%
\pgfusepath{fill}%
\end{pgfscope}%
\begin{pgfscope}%
\pgfpathrectangle{\pgfqpoint{1.020000in}{0.880000in}}{\pgfqpoint{6.160000in}{6.160000in}}%
\pgfusepath{clip}%
\pgfsetbuttcap%
\pgfsetroundjoin%
\definecolor{currentfill}{rgb}{0.462354,0.599830,0.965857}%
\pgfsetfillcolor{currentfill}%
\pgfsetlinewidth{0.000000pt}%
\definecolor{currentstroke}{rgb}{0.000000,0.000000,0.000000}%
\pgfsetstrokecolor{currentstroke}%
\pgfsetdash{}{0pt}%
\pgfpathmoveto{\pgfqpoint{5.302022in}{3.738781in}}%
\pgfpathlineto{\pgfqpoint{5.311137in}{3.641951in}}%
\pgfpathlineto{\pgfqpoint{5.320228in}{3.542863in}}%
\pgfpathlineto{\pgfqpoint{5.352729in}{3.577859in}}%
\pgfpathlineto{\pgfqpoint{5.385012in}{3.594299in}}%
\pgfpathlineto{\pgfqpoint{5.376589in}{3.749742in}}%
\pgfpathlineto{\pgfqpoint{5.364925in}{3.631287in}}%
\pgfpathlineto{\pgfqpoint{5.332678in}{3.613368in}}%
\pgfpathlineto{\pgfqpoint{5.302022in}{3.738781in}}%
\pgfpathclose%
\pgfusepath{fill}%
\end{pgfscope}%
\begin{pgfscope}%
\pgfpathrectangle{\pgfqpoint{1.020000in}{0.880000in}}{\pgfqpoint{6.160000in}{6.160000in}}%
\pgfusepath{clip}%
\pgfsetbuttcap%
\pgfsetroundjoin%
\definecolor{currentfill}{rgb}{0.748682,0.827679,0.963334}%
\pgfsetfillcolor{currentfill}%
\pgfsetlinewidth{0.000000pt}%
\definecolor{currentstroke}{rgb}{0.000000,0.000000,0.000000}%
\pgfsetstrokecolor{currentstroke}%
\pgfsetdash{}{0pt}%
\pgfpathmoveto{\pgfqpoint{3.860673in}{4.126690in}}%
\pgfpathlineto{\pgfqpoint{3.868823in}{4.307245in}}%
\pgfpathlineto{\pgfqpoint{3.878000in}{4.164933in}}%
\pgfpathlineto{\pgfqpoint{3.910960in}{4.012000in}}%
\pgfpathlineto{\pgfqpoint{3.943336in}{4.054706in}}%
\pgfpathlineto{\pgfqpoint{3.934364in}{4.130301in}}%
\pgfpathlineto{\pgfqpoint{3.925388in}{4.209171in}}%
\pgfpathlineto{\pgfqpoint{3.893197in}{4.108848in}}%
\pgfpathlineto{\pgfqpoint{3.860673in}{4.126690in}}%
\pgfpathclose%
\pgfusepath{fill}%
\end{pgfscope}%
\begin{pgfscope}%
\pgfpathrectangle{\pgfqpoint{1.020000in}{0.880000in}}{\pgfqpoint{6.160000in}{6.160000in}}%
\pgfusepath{clip}%
\pgfsetbuttcap%
\pgfsetroundjoin%
\definecolor{currentfill}{rgb}{0.698454,0.799450,0.984577}%
\pgfsetfillcolor{currentfill}%
\pgfsetlinewidth{0.000000pt}%
\definecolor{currentstroke}{rgb}{0.000000,0.000000,0.000000}%
\pgfsetstrokecolor{currentstroke}%
\pgfsetdash{}{0pt}%
\pgfpathmoveto{\pgfqpoint{4.008242in}{4.067252in}}%
\pgfpathlineto{\pgfqpoint{4.017129in}{4.054581in}}%
\pgfpathlineto{\pgfqpoint{4.025912in}{4.118978in}}%
\pgfpathlineto{\pgfqpoint{4.058479in}{4.079968in}}%
\pgfpathlineto{\pgfqpoint{4.091066in}{3.970578in}}%
\pgfpathlineto{\pgfqpoint{4.081947in}{4.151185in}}%
\pgfpathlineto{\pgfqpoint{4.073279in}{3.907599in}}%
\pgfpathlineto{\pgfqpoint{4.040711in}{4.047163in}}%
\pgfpathlineto{\pgfqpoint{4.008242in}{4.067252in}}%
\pgfpathclose%
\pgfusepath{fill}%
\end{pgfscope}%
\begin{pgfscope}%
\pgfpathrectangle{\pgfqpoint{1.020000in}{0.880000in}}{\pgfqpoint{6.160000in}{6.160000in}}%
\pgfusepath{clip}%
\pgfsetbuttcap%
\pgfsetroundjoin%
\definecolor{currentfill}{rgb}{0.446431,0.582356,0.957373}%
\pgfsetfillcolor{currentfill}%
\pgfsetlinewidth{0.000000pt}%
\definecolor{currentstroke}{rgb}{0.000000,0.000000,0.000000}%
\pgfsetstrokecolor{currentstroke}%
\pgfsetdash{}{0pt}%
\pgfpathmoveto{\pgfqpoint{5.513656in}{3.629409in}}%
\pgfpathlineto{\pgfqpoint{5.524198in}{3.638791in}}%
\pgfpathlineto{\pgfqpoint{5.536377in}{3.767801in}}%
\pgfpathlineto{\pgfqpoint{5.563498in}{3.406112in}}%
\pgfpathlineto{\pgfqpoint{5.597366in}{3.539817in}}%
\pgfpathlineto{\pgfqpoint{5.588391in}{3.647765in}}%
\pgfpathlineto{\pgfqpoint{5.578042in}{3.657127in}}%
\pgfpathlineto{\pgfqpoint{5.544924in}{3.574910in}}%
\pgfpathlineto{\pgfqpoint{5.513656in}{3.629409in}}%
\pgfpathclose%
\pgfusepath{fill}%
\end{pgfscope}%
\begin{pgfscope}%
\pgfpathrectangle{\pgfqpoint{1.020000in}{0.880000in}}{\pgfqpoint{6.160000in}{6.160000in}}%
\pgfusepath{clip}%
\pgfsetbuttcap%
\pgfsetroundjoin%
\definecolor{currentfill}{rgb}{0.651398,0.768121,0.995891}%
\pgfsetfillcolor{currentfill}%
\pgfsetlinewidth{0.000000pt}%
\definecolor{currentstroke}{rgb}{0.000000,0.000000,0.000000}%
\pgfsetstrokecolor{currentstroke}%
\pgfsetdash{}{0pt}%
\pgfpathmoveto{\pgfqpoint{4.155857in}{4.181449in}}%
\pgfpathlineto{\pgfqpoint{4.164926in}{4.004152in}}%
\pgfpathlineto{\pgfqpoint{4.173989in}{3.736565in}}%
\pgfpathlineto{\pgfqpoint{4.206416in}{3.931442in}}%
\pgfpathlineto{\pgfqpoint{4.238876in}{3.992651in}}%
\pgfpathlineto{\pgfqpoint{4.229762in}{3.960832in}}%
\pgfpathlineto{\pgfqpoint{4.220657in}{3.886093in}}%
\pgfpathlineto{\pgfqpoint{4.188312in}{4.001417in}}%
\pgfpathlineto{\pgfqpoint{4.155857in}{4.181449in}}%
\pgfpathclose%
\pgfusepath{fill}%
\end{pgfscope}%
\begin{pgfscope}%
\pgfpathrectangle{\pgfqpoint{1.020000in}{0.880000in}}{\pgfqpoint{6.160000in}{6.160000in}}%
\pgfusepath{clip}%
\pgfsetbuttcap%
\pgfsetroundjoin%
\definecolor{currentfill}{rgb}{0.494638,0.633022,0.978983}%
\pgfsetfillcolor{currentfill}%
\pgfsetlinewidth{0.000000pt}%
\definecolor{currentstroke}{rgb}{0.000000,0.000000,0.000000}%
\pgfsetstrokecolor{currentstroke}%
\pgfsetdash{}{0pt}%
\pgfpathmoveto{\pgfqpoint{5.088968in}{3.726453in}}%
\pgfpathlineto{\pgfqpoint{5.098013in}{3.623948in}}%
\pgfpathlineto{\pgfqpoint{5.108058in}{3.629726in}}%
\pgfpathlineto{\pgfqpoint{5.141252in}{3.736817in}}%
\pgfpathlineto{\pgfqpoint{5.172094in}{3.600638in}}%
\pgfpathlineto{\pgfqpoint{5.162462in}{3.643439in}}%
\pgfpathlineto{\pgfqpoint{5.153617in}{3.766294in}}%
\pgfpathlineto{\pgfqpoint{5.121396in}{3.757142in}}%
\pgfpathlineto{\pgfqpoint{5.088968in}{3.726453in}}%
\pgfpathclose%
\pgfusepath{fill}%
\end{pgfscope}%
\begin{pgfscope}%
\pgfpathrectangle{\pgfqpoint{1.020000in}{0.880000in}}{\pgfqpoint{6.160000in}{6.160000in}}%
\pgfusepath{clip}%
\pgfsetbuttcap%
\pgfsetroundjoin%
\definecolor{currentfill}{rgb}{0.543440,0.680003,0.993051}%
\pgfsetfillcolor{currentfill}%
\pgfsetlinewidth{0.000000pt}%
\definecolor{currentstroke}{rgb}{0.000000,0.000000,0.000000}%
\pgfsetstrokecolor{currentstroke}%
\pgfsetdash{}{0pt}%
\pgfpathmoveto{\pgfqpoint{4.663831in}{3.782533in}}%
\pgfpathlineto{\pgfqpoint{4.672877in}{3.680824in}}%
\pgfpathlineto{\pgfqpoint{4.682494in}{3.693757in}}%
\pgfpathlineto{\pgfqpoint{4.715083in}{3.749742in}}%
\pgfpathlineto{\pgfqpoint{4.748027in}{3.865832in}}%
\pgfpathlineto{\pgfqpoint{4.737948in}{3.784756in}}%
\pgfpathlineto{\pgfqpoint{4.728136in}{3.746454in}}%
\pgfpathlineto{\pgfqpoint{4.696340in}{3.831307in}}%
\pgfpathlineto{\pgfqpoint{4.663831in}{3.782533in}}%
\pgfpathclose%
\pgfusepath{fill}%
\end{pgfscope}%
\begin{pgfscope}%
\pgfpathrectangle{\pgfqpoint{1.020000in}{0.880000in}}{\pgfqpoint{6.160000in}{6.160000in}}%
\pgfusepath{clip}%
\pgfsetbuttcap%
\pgfsetroundjoin%
\definecolor{currentfill}{rgb}{0.958176,0.771234,0.680301}%
\pgfsetfillcolor{currentfill}%
\pgfsetlinewidth{0.000000pt}%
\definecolor{currentstroke}{rgb}{0.000000,0.000000,0.000000}%
\pgfsetstrokecolor{currentstroke}%
\pgfsetdash{}{0pt}%
\pgfpathmoveto{\pgfqpoint{3.187820in}{4.590332in}}%
\pgfpathlineto{\pgfqpoint{3.193939in}{4.785442in}}%
\pgfpathlineto{\pgfqpoint{3.201585in}{4.827743in}}%
\pgfpathlineto{\pgfqpoint{3.235717in}{4.683866in}}%
\pgfpathlineto{\pgfqpoint{3.267849in}{4.748999in}}%
\pgfpathlineto{\pgfqpoint{3.262208in}{4.477602in}}%
\pgfpathlineto{\pgfqpoint{3.252704in}{4.632866in}}%
\pgfpathlineto{\pgfqpoint{3.219456in}{4.696579in}}%
\pgfpathlineto{\pgfqpoint{3.187820in}{4.590332in}}%
\pgfpathclose%
\pgfusepath{fill}%
\end{pgfscope}%
\begin{pgfscope}%
\pgfpathrectangle{\pgfqpoint{1.020000in}{0.880000in}}{\pgfqpoint{6.160000in}{6.160000in}}%
\pgfusepath{clip}%
\pgfsetbuttcap%
\pgfsetroundjoin%
\definecolor{currentfill}{rgb}{0.527132,0.664700,0.989065}%
\pgfsetfillcolor{currentfill}%
\pgfsetlinewidth{0.000000pt}%
\definecolor{currentstroke}{rgb}{0.000000,0.000000,0.000000}%
\pgfsetstrokecolor{currentstroke}%
\pgfsetdash{}{0pt}%
\pgfpathmoveto{\pgfqpoint{4.876962in}{3.837572in}}%
\pgfpathlineto{\pgfqpoint{4.886697in}{3.832593in}}%
\pgfpathlineto{\pgfqpoint{4.895890in}{3.748679in}}%
\pgfpathlineto{\pgfqpoint{4.926936in}{3.585256in}}%
\pgfpathlineto{\pgfqpoint{4.960120in}{3.713782in}}%
\pgfpathlineto{\pgfqpoint{4.950635in}{3.760065in}}%
\pgfpathlineto{\pgfqpoint{4.940880in}{3.769096in}}%
\pgfpathlineto{\pgfqpoint{4.907865in}{3.654449in}}%
\pgfpathlineto{\pgfqpoint{4.876962in}{3.837572in}}%
\pgfpathclose%
\pgfusepath{fill}%
\end{pgfscope}%
\begin{pgfscope}%
\pgfpathrectangle{\pgfqpoint{1.020000in}{0.880000in}}{\pgfqpoint{6.160000in}{6.160000in}}%
\pgfusepath{clip}%
\pgfsetbuttcap%
\pgfsetroundjoin%
\definecolor{currentfill}{rgb}{0.510824,0.649397,0.985079}%
\pgfsetfillcolor{currentfill}%
\pgfsetlinewidth{0.000000pt}%
\definecolor{currentstroke}{rgb}{0.000000,0.000000,0.000000}%
\pgfsetstrokecolor{currentstroke}%
\pgfsetdash{}{0pt}%
\pgfpathmoveto{\pgfqpoint{4.599026in}{3.726931in}}%
\pgfpathlineto{\pgfqpoint{4.607772in}{3.554782in}}%
\pgfpathlineto{\pgfqpoint{4.617615in}{3.638645in}}%
\pgfpathlineto{\pgfqpoint{4.650425in}{3.745672in}}%
\pgfpathlineto{\pgfqpoint{4.682494in}{3.693757in}}%
\pgfpathlineto{\pgfqpoint{4.672877in}{3.680824in}}%
\pgfpathlineto{\pgfqpoint{4.663831in}{3.782533in}}%
\pgfpathlineto{\pgfqpoint{4.632084in}{3.901240in}}%
\pgfpathlineto{\pgfqpoint{4.599026in}{3.726931in}}%
\pgfpathclose%
\pgfusepath{fill}%
\end{pgfscope}%
\begin{pgfscope}%
\pgfpathrectangle{\pgfqpoint{1.020000in}{0.880000in}}{\pgfqpoint{6.160000in}{6.160000in}}%
\pgfusepath{clip}%
\pgfsetbuttcap%
\pgfsetroundjoin%
\definecolor{currentfill}{rgb}{0.619318,0.744121,0.998931}%
\pgfsetfillcolor{currentfill}%
\pgfsetlinewidth{0.000000pt}%
\definecolor{currentstroke}{rgb}{0.000000,0.000000,0.000000}%
\pgfsetstrokecolor{currentstroke}%
\pgfsetdash{}{0pt}%
\pgfpathmoveto{\pgfqpoint{4.303474in}{3.793302in}}%
\pgfpathlineto{\pgfqpoint{4.312784in}{3.909257in}}%
\pgfpathlineto{\pgfqpoint{4.321895in}{3.851319in}}%
\pgfpathlineto{\pgfqpoint{4.354463in}{3.946299in}}%
\pgfpathlineto{\pgfqpoint{4.386762in}{3.887707in}}%
\pgfpathlineto{\pgfqpoint{4.377727in}{3.998683in}}%
\pgfpathlineto{\pgfqpoint{4.368136in}{3.808706in}}%
\pgfpathlineto{\pgfqpoint{4.336101in}{3.991204in}}%
\pgfpathlineto{\pgfqpoint{4.303474in}{3.793302in}}%
\pgfpathclose%
\pgfusepath{fill}%
\end{pgfscope}%
\begin{pgfscope}%
\pgfpathrectangle{\pgfqpoint{1.020000in}{0.880000in}}{\pgfqpoint{6.160000in}{6.160000in}}%
\pgfusepath{clip}%
\pgfsetbuttcap%
\pgfsetroundjoin%
\definecolor{currentfill}{rgb}{0.467678,0.605591,0.968546}%
\pgfsetfillcolor{currentfill}%
\pgfsetlinewidth{0.000000pt}%
\definecolor{currentstroke}{rgb}{0.000000,0.000000,0.000000}%
\pgfsetstrokecolor{currentstroke}%
\pgfsetdash{}{0pt}%
\pgfpathmoveto{\pgfqpoint{5.237463in}{3.708963in}}%
\pgfpathlineto{\pgfqpoint{5.245913in}{3.549563in}}%
\pgfpathlineto{\pgfqpoint{5.256181in}{3.560902in}}%
\pgfpathlineto{\pgfqpoint{5.290201in}{3.731677in}}%
\pgfpathlineto{\pgfqpoint{5.320228in}{3.542863in}}%
\pgfpathlineto{\pgfqpoint{5.311137in}{3.641951in}}%
\pgfpathlineto{\pgfqpoint{5.302022in}{3.738781in}}%
\pgfpathlineto{\pgfqpoint{5.268579in}{3.616035in}}%
\pgfpathlineto{\pgfqpoint{5.237463in}{3.708963in}}%
\pgfpathclose%
\pgfusepath{fill}%
\end{pgfscope}%
\begin{pgfscope}%
\pgfpathrectangle{\pgfqpoint{1.020000in}{0.880000in}}{\pgfqpoint{6.160000in}{6.160000in}}%
\pgfusepath{clip}%
\pgfsetbuttcap%
\pgfsetroundjoin%
\definecolor{currentfill}{rgb}{0.462354,0.599830,0.965857}%
\pgfsetfillcolor{currentfill}%
\pgfsetlinewidth{0.000000pt}%
\definecolor{currentstroke}{rgb}{0.000000,0.000000,0.000000}%
\pgfsetstrokecolor{currentstroke}%
\pgfsetdash{}{0pt}%
\pgfpathmoveto{\pgfqpoint{4.960120in}{3.713782in}}%
\pgfpathlineto{\pgfqpoint{4.968594in}{3.537266in}}%
\pgfpathlineto{\pgfqpoint{4.978059in}{3.487350in}}%
\pgfpathlineto{\pgfqpoint{5.011435in}{3.629671in}}%
\pgfpathlineto{\pgfqpoint{5.043743in}{3.639211in}}%
\pgfpathlineto{\pgfqpoint{5.033913in}{3.651326in}}%
\pgfpathlineto{\pgfqpoint{5.024849in}{3.752571in}}%
\pgfpathlineto{\pgfqpoint{4.991130in}{3.564974in}}%
\pgfpathlineto{\pgfqpoint{4.960120in}{3.713782in}}%
\pgfpathclose%
\pgfusepath{fill}%
\end{pgfscope}%
\begin{pgfscope}%
\pgfpathrectangle{\pgfqpoint{1.020000in}{0.880000in}}{\pgfqpoint{6.160000in}{6.160000in}}%
\pgfusepath{clip}%
\pgfsetbuttcap%
\pgfsetroundjoin%
\definecolor{currentfill}{rgb}{0.966962,0.735670,0.630877}%
\pgfsetfillcolor{currentfill}%
\pgfsetlinewidth{0.000000pt}%
\definecolor{currentstroke}{rgb}{0.000000,0.000000,0.000000}%
\pgfsetstrokecolor{currentstroke}%
\pgfsetdash{}{0pt}%
\pgfpathmoveto{\pgfqpoint{2.975038in}{4.664533in}}%
\pgfpathlineto{\pgfqpoint{2.982212in}{4.718259in}}%
\pgfpathlineto{\pgfqpoint{2.990294in}{4.696879in}}%
\pgfpathlineto{\pgfqpoint{3.021863in}{4.803703in}}%
\pgfpathlineto{\pgfqpoint{3.055377in}{4.740374in}}%
\pgfpathlineto{\pgfqpoint{3.047158in}{4.769289in}}%
\pgfpathlineto{\pgfqpoint{3.038852in}{4.806985in}}%
\pgfpathlineto{\pgfqpoint{3.005648in}{4.846261in}}%
\pgfpathlineto{\pgfqpoint{2.975038in}{4.664533in}}%
\pgfpathclose%
\pgfusepath{fill}%
\end{pgfscope}%
\begin{pgfscope}%
\pgfpathrectangle{\pgfqpoint{1.020000in}{0.880000in}}{\pgfqpoint{6.160000in}{6.160000in}}%
\pgfusepath{clip}%
\pgfsetbuttcap%
\pgfsetroundjoin%
\definecolor{currentfill}{rgb}{0.586921,0.718121,0.998874}%
\pgfsetfillcolor{currentfill}%
\pgfsetlinewidth{0.000000pt}%
\definecolor{currentstroke}{rgb}{0.000000,0.000000,0.000000}%
\pgfsetstrokecolor{currentstroke}%
\pgfsetdash{}{0pt}%
\pgfpathmoveto{\pgfqpoint{4.451717in}{3.966953in}}%
\pgfpathlineto{\pgfqpoint{4.460639in}{3.816630in}}%
\pgfpathlineto{\pgfqpoint{4.470243in}{3.908150in}}%
\pgfpathlineto{\pgfqpoint{4.502061in}{3.721494in}}%
\pgfpathlineto{\pgfqpoint{4.534677in}{3.802391in}}%
\pgfpathlineto{\pgfqpoint{4.525192in}{3.778036in}}%
\pgfpathlineto{\pgfqpoint{4.516131in}{3.874725in}}%
\pgfpathlineto{\pgfqpoint{4.483869in}{3.894290in}}%
\pgfpathlineto{\pgfqpoint{4.451717in}{3.966953in}}%
\pgfpathclose%
\pgfusepath{fill}%
\end{pgfscope}%
\begin{pgfscope}%
\pgfpathrectangle{\pgfqpoint{1.020000in}{0.880000in}}{\pgfqpoint{6.160000in}{6.160000in}}%
\pgfusepath{clip}%
\pgfsetbuttcap%
\pgfsetroundjoin%
\definecolor{currentfill}{rgb}{0.708720,0.805721,0.981117}%
\pgfsetfillcolor{currentfill}%
\pgfsetlinewidth{0.000000pt}%
\definecolor{currentstroke}{rgb}{0.000000,0.000000,0.000000}%
\pgfsetstrokecolor{currentstroke}%
\pgfsetdash{}{0pt}%
\pgfpathmoveto{\pgfqpoint{3.943336in}{4.054706in}}%
\pgfpathlineto{\pgfqpoint{3.952055in}{4.091685in}}%
\pgfpathlineto{\pgfqpoint{3.960867in}{4.097449in}}%
\pgfpathlineto{\pgfqpoint{3.993446in}{4.081370in}}%
\pgfpathlineto{\pgfqpoint{4.025912in}{4.118978in}}%
\pgfpathlineto{\pgfqpoint{4.017129in}{4.054581in}}%
\pgfpathlineto{\pgfqpoint{4.008242in}{4.067252in}}%
\pgfpathlineto{\pgfqpoint{3.976072in}{3.927126in}}%
\pgfpathlineto{\pgfqpoint{3.943336in}{4.054706in}}%
\pgfpathclose%
\pgfusepath{fill}%
\end{pgfscope}%
\begin{pgfscope}%
\pgfpathrectangle{\pgfqpoint{1.020000in}{0.880000in}}{\pgfqpoint{6.160000in}{6.160000in}}%
\pgfusepath{clip}%
\pgfsetbuttcap%
\pgfsetroundjoin%
\definecolor{currentfill}{rgb}{0.608547,0.735725,0.999354}%
\pgfsetfillcolor{currentfill}%
\pgfsetlinewidth{0.000000pt}%
\definecolor{currentstroke}{rgb}{0.000000,0.000000,0.000000}%
\pgfsetstrokecolor{currentstroke}%
\pgfsetdash{}{0pt}%
\pgfpathmoveto{\pgfqpoint{4.238876in}{3.992651in}}%
\pgfpathlineto{\pgfqpoint{4.247961in}{3.940987in}}%
\pgfpathlineto{\pgfqpoint{4.257060in}{3.896568in}}%
\pgfpathlineto{\pgfqpoint{4.289523in}{3.900656in}}%
\pgfpathlineto{\pgfqpoint{4.321895in}{3.851319in}}%
\pgfpathlineto{\pgfqpoint{4.312784in}{3.909257in}}%
\pgfpathlineto{\pgfqpoint{4.303474in}{3.793302in}}%
\pgfpathlineto{\pgfqpoint{4.271099in}{3.759117in}}%
\pgfpathlineto{\pgfqpoint{4.238876in}{3.992651in}}%
\pgfpathclose%
\pgfusepath{fill}%
\end{pgfscope}%
\begin{pgfscope}%
\pgfpathrectangle{\pgfqpoint{1.020000in}{0.880000in}}{\pgfqpoint{6.160000in}{6.160000in}}%
\pgfusepath{clip}%
\pgfsetbuttcap%
\pgfsetroundjoin%
\definecolor{currentfill}{rgb}{0.462354,0.599830,0.965857}%
\pgfsetfillcolor{currentfill}%
\pgfsetlinewidth{0.000000pt}%
\definecolor{currentstroke}{rgb}{0.000000,0.000000,0.000000}%
\pgfsetstrokecolor{currentstroke}%
\pgfsetdash{}{0pt}%
\pgfpathmoveto{\pgfqpoint{5.448063in}{3.509260in}}%
\pgfpathlineto{\pgfqpoint{5.459678in}{3.609028in}}%
\pgfpathlineto{\pgfqpoint{5.469194in}{3.542005in}}%
\pgfpathlineto{\pgfqpoint{5.502870in}{3.663345in}}%
\pgfpathlineto{\pgfqpoint{5.536377in}{3.767801in}}%
\pgfpathlineto{\pgfqpoint{5.524198in}{3.638791in}}%
\pgfpathlineto{\pgfqpoint{5.513656in}{3.629409in}}%
\pgfpathlineto{\pgfqpoint{5.481486in}{3.618453in}}%
\pgfpathlineto{\pgfqpoint{5.448063in}{3.509260in}}%
\pgfpathclose%
\pgfusepath{fill}%
\end{pgfscope}%
\begin{pgfscope}%
\pgfpathrectangle{\pgfqpoint{1.020000in}{0.880000in}}{\pgfqpoint{6.160000in}{6.160000in}}%
\pgfusepath{clip}%
\pgfsetbuttcap%
\pgfsetroundjoin%
\definecolor{currentfill}{rgb}{0.510824,0.649397,0.985079}%
\pgfsetfillcolor{currentfill}%
\pgfsetlinewidth{0.000000pt}%
\definecolor{currentstroke}{rgb}{0.000000,0.000000,0.000000}%
\pgfsetstrokecolor{currentstroke}%
\pgfsetdash{}{0pt}%
\pgfpathmoveto{\pgfqpoint{4.534677in}{3.802391in}}%
\pgfpathlineto{\pgfqpoint{4.544064in}{3.792552in}}%
\pgfpathlineto{\pgfqpoint{4.553798in}{3.872197in}}%
\pgfpathlineto{\pgfqpoint{4.585442in}{3.677629in}}%
\pgfpathlineto{\pgfqpoint{4.617615in}{3.638645in}}%
\pgfpathlineto{\pgfqpoint{4.607772in}{3.554782in}}%
\pgfpathlineto{\pgfqpoint{4.599026in}{3.726931in}}%
\pgfpathlineto{\pgfqpoint{4.566303in}{3.615659in}}%
\pgfpathlineto{\pgfqpoint{4.534677in}{3.802391in}}%
\pgfpathclose%
\pgfusepath{fill}%
\end{pgfscope}%
\begin{pgfscope}%
\pgfpathrectangle{\pgfqpoint{1.020000in}{0.880000in}}{\pgfqpoint{6.160000in}{6.160000in}}%
\pgfusepath{clip}%
\pgfsetbuttcap%
\pgfsetroundjoin%
\definecolor{currentfill}{rgb}{0.839351,0.861167,0.894494}%
\pgfsetfillcolor{currentfill}%
\pgfsetlinewidth{0.000000pt}%
\definecolor{currentstroke}{rgb}{0.000000,0.000000,0.000000}%
\pgfsetstrokecolor{currentstroke}%
\pgfsetdash{}{0pt}%
\pgfpathmoveto{\pgfqpoint{3.712207in}{4.398448in}}%
\pgfpathlineto{\pgfqpoint{3.719928in}{4.585359in}}%
\pgfpathlineto{\pgfqpoint{3.729746in}{4.316706in}}%
\pgfpathlineto{\pgfqpoint{3.762996in}{4.162596in}}%
\pgfpathlineto{\pgfqpoint{3.795273in}{4.230684in}}%
\pgfpathlineto{\pgfqpoint{3.786157in}{4.351167in}}%
\pgfpathlineto{\pgfqpoint{3.778084in}{4.211472in}}%
\pgfpathlineto{\pgfqpoint{3.745240in}{4.292360in}}%
\pgfpathlineto{\pgfqpoint{3.712207in}{4.398448in}}%
\pgfpathclose%
\pgfusepath{fill}%
\end{pgfscope}%
\begin{pgfscope}%
\pgfpathrectangle{\pgfqpoint{1.020000in}{0.880000in}}{\pgfqpoint{6.160000in}{6.160000in}}%
\pgfusepath{clip}%
\pgfsetbuttcap%
\pgfsetroundjoin%
\definecolor{currentfill}{rgb}{0.404421,0.534643,0.932002}%
\pgfsetfillcolor{currentfill}%
\pgfsetlinewidth{0.000000pt}%
\definecolor{currentstroke}{rgb}{0.000000,0.000000,0.000000}%
\pgfsetstrokecolor{currentstroke}%
\pgfsetdash{}{0pt}%
\pgfpathmoveto{\pgfqpoint{5.320228in}{3.542863in}}%
\pgfpathlineto{\pgfqpoint{5.328701in}{3.389702in}}%
\pgfpathlineto{\pgfqpoint{5.339342in}{3.425048in}}%
\pgfpathlineto{\pgfqpoint{5.371670in}{3.442988in}}%
\pgfpathlineto{\pgfqpoint{5.405440in}{3.579613in}}%
\pgfpathlineto{\pgfqpoint{5.395577in}{3.616822in}}%
\pgfpathlineto{\pgfqpoint{5.385012in}{3.594299in}}%
\pgfpathlineto{\pgfqpoint{5.352729in}{3.577859in}}%
\pgfpathlineto{\pgfqpoint{5.320228in}{3.542863in}}%
\pgfpathclose%
\pgfusepath{fill}%
\end{pgfscope}%
\begin{pgfscope}%
\pgfpathrectangle{\pgfqpoint{1.020000in}{0.880000in}}{\pgfqpoint{6.160000in}{6.160000in}}%
\pgfusepath{clip}%
\pgfsetbuttcap%
\pgfsetroundjoin%
\definecolor{currentfill}{rgb}{0.672538,0.782861,0.991982}%
\pgfsetfillcolor{currentfill}%
\pgfsetlinewidth{0.000000pt}%
\definecolor{currentstroke}{rgb}{0.000000,0.000000,0.000000}%
\pgfsetstrokecolor{currentstroke}%
\pgfsetdash{}{0pt}%
\pgfpathmoveto{\pgfqpoint{4.091066in}{3.970578in}}%
\pgfpathlineto{\pgfqpoint{4.099947in}{4.061911in}}%
\pgfpathlineto{\pgfqpoint{4.109065in}{3.878058in}}%
\pgfpathlineto{\pgfqpoint{4.141453in}{4.063574in}}%
\pgfpathlineto{\pgfqpoint{4.173989in}{3.736565in}}%
\pgfpathlineto{\pgfqpoint{4.164926in}{4.004152in}}%
\pgfpathlineto{\pgfqpoint{4.155857in}{4.181449in}}%
\pgfpathlineto{\pgfqpoint{4.123465in}{4.028243in}}%
\pgfpathlineto{\pgfqpoint{4.091066in}{3.970578in}}%
\pgfpathclose%
\pgfusepath{fill}%
\end{pgfscope}%
\begin{pgfscope}%
\pgfpathrectangle{\pgfqpoint{1.020000in}{0.880000in}}{\pgfqpoint{6.160000in}{6.160000in}}%
\pgfusepath{clip}%
\pgfsetbuttcap%
\pgfsetroundjoin%
\definecolor{currentfill}{rgb}{0.510824,0.649397,0.985079}%
\pgfsetfillcolor{currentfill}%
\pgfsetlinewidth{0.000000pt}%
\definecolor{currentstroke}{rgb}{0.000000,0.000000,0.000000}%
\pgfsetstrokecolor{currentstroke}%
\pgfsetdash{}{0pt}%
\pgfpathmoveto{\pgfqpoint{5.024849in}{3.752571in}}%
\pgfpathlineto{\pgfqpoint{5.033913in}{3.651326in}}%
\pgfpathlineto{\pgfqpoint{5.043743in}{3.639211in}}%
\pgfpathlineto{\pgfqpoint{5.077322in}{3.792814in}}%
\pgfpathlineto{\pgfqpoint{5.108058in}{3.629726in}}%
\pgfpathlineto{\pgfqpoint{5.098013in}{3.623948in}}%
\pgfpathlineto{\pgfqpoint{5.088968in}{3.726453in}}%
\pgfpathlineto{\pgfqpoint{5.057778in}{3.837894in}}%
\pgfpathlineto{\pgfqpoint{5.024849in}{3.752571in}}%
\pgfpathclose%
\pgfusepath{fill}%
\end{pgfscope}%
\begin{pgfscope}%
\pgfpathrectangle{\pgfqpoint{1.020000in}{0.880000in}}{\pgfqpoint{6.160000in}{6.160000in}}%
\pgfusepath{clip}%
\pgfsetbuttcap%
\pgfsetroundjoin%
\definecolor{currentfill}{rgb}{0.441123,0.576532,0.954545}%
\pgfsetfillcolor{currentfill}%
\pgfsetlinewidth{0.000000pt}%
\definecolor{currentstroke}{rgb}{0.000000,0.000000,0.000000}%
\pgfsetstrokecolor{currentstroke}%
\pgfsetdash{}{0pt}%
\pgfpathmoveto{\pgfqpoint{5.385012in}{3.594299in}}%
\pgfpathlineto{\pgfqpoint{5.395577in}{3.616822in}}%
\pgfpathlineto{\pgfqpoint{5.405440in}{3.579613in}}%
\pgfpathlineto{\pgfqpoint{5.438805in}{3.678398in}}%
\pgfpathlineto{\pgfqpoint{5.469194in}{3.542005in}}%
\pgfpathlineto{\pgfqpoint{5.459678in}{3.609028in}}%
\pgfpathlineto{\pgfqpoint{5.448063in}{3.509260in}}%
\pgfpathlineto{\pgfqpoint{5.416265in}{3.527858in}}%
\pgfpathlineto{\pgfqpoint{5.385012in}{3.594299in}}%
\pgfpathclose%
\pgfusepath{fill}%
\end{pgfscope}%
\begin{pgfscope}%
\pgfpathrectangle{\pgfqpoint{1.020000in}{0.880000in}}{\pgfqpoint{6.160000in}{6.160000in}}%
\pgfusepath{clip}%
\pgfsetbuttcap%
\pgfsetroundjoin%
\definecolor{currentfill}{rgb}{0.419991,0.552989,0.942630}%
\pgfsetfillcolor{currentfill}%
\pgfsetlinewidth{0.000000pt}%
\definecolor{currentstroke}{rgb}{0.000000,0.000000,0.000000}%
\pgfsetstrokecolor{currentstroke}%
\pgfsetdash{}{0pt}%
\pgfpathmoveto{\pgfqpoint{5.597366in}{3.539817in}}%
\pgfpathlineto{\pgfqpoint{5.607987in}{3.547538in}}%
\pgfpathlineto{\pgfqpoint{5.619287in}{3.601018in}}%
\pgfpathlineto{\pgfqpoint{5.649974in}{3.505726in}}%
\pgfpathlineto{\pgfqpoint{5.638390in}{3.435282in}}%
\pgfpathlineto{\pgfqpoint{5.630672in}{3.629925in}}%
\pgfpathlineto{\pgfqpoint{5.597366in}{3.539817in}}%
\pgfpathclose%
\pgfusepath{fill}%
\end{pgfscope}%
\begin{pgfscope}%
\pgfpathrectangle{\pgfqpoint{1.020000in}{0.880000in}}{\pgfqpoint{6.160000in}{6.160000in}}%
\pgfusepath{clip}%
\pgfsetbuttcap%
\pgfsetroundjoin%
\definecolor{currentfill}{rgb}{0.919376,0.831273,0.782874}%
\pgfsetfillcolor{currentfill}%
\pgfsetlinewidth{0.000000pt}%
\definecolor{currentstroke}{rgb}{0.000000,0.000000,0.000000}%
\pgfsetstrokecolor{currentstroke}%
\pgfsetdash{}{0pt}%
\pgfpathmoveto{\pgfqpoint{3.417102in}{4.492349in}}%
\pgfpathlineto{\pgfqpoint{3.424303in}{4.634725in}}%
\pgfpathlineto{\pgfqpoint{3.433962in}{4.451032in}}%
\pgfpathlineto{\pgfqpoint{3.465195in}{4.654533in}}%
\pgfpathlineto{\pgfqpoint{3.499821in}{4.373549in}}%
\pgfpathlineto{\pgfqpoint{3.490221in}{4.557784in}}%
\pgfpathlineto{\pgfqpoint{3.482681in}{4.442981in}}%
\pgfpathlineto{\pgfqpoint{3.449450in}{4.531003in}}%
\pgfpathlineto{\pgfqpoint{3.417102in}{4.492349in}}%
\pgfpathclose%
\pgfusepath{fill}%
\end{pgfscope}%
\begin{pgfscope}%
\pgfpathrectangle{\pgfqpoint{1.020000in}{0.880000in}}{\pgfqpoint{6.160000in}{6.160000in}}%
\pgfusepath{clip}%
\pgfsetbuttcap%
\pgfsetroundjoin%
\definecolor{currentfill}{rgb}{0.467678,0.605591,0.968546}%
\pgfsetfillcolor{currentfill}%
\pgfsetlinewidth{0.000000pt}%
\definecolor{currentstroke}{rgb}{0.000000,0.000000,0.000000}%
\pgfsetstrokecolor{currentstroke}%
\pgfsetdash{}{0pt}%
\pgfpathmoveto{\pgfqpoint{5.172094in}{3.600638in}}%
\pgfpathlineto{\pgfqpoint{5.182952in}{3.679599in}}%
\pgfpathlineto{\pgfqpoint{5.192126in}{3.587746in}}%
\pgfpathlineto{\pgfqpoint{5.224269in}{3.584084in}}%
\pgfpathlineto{\pgfqpoint{5.256181in}{3.560902in}}%
\pgfpathlineto{\pgfqpoint{5.245913in}{3.549563in}}%
\pgfpathlineto{\pgfqpoint{5.237463in}{3.708963in}}%
\pgfpathlineto{\pgfqpoint{5.205900in}{3.765659in}}%
\pgfpathlineto{\pgfqpoint{5.172094in}{3.600638in}}%
\pgfpathclose%
\pgfusepath{fill}%
\end{pgfscope}%
\begin{pgfscope}%
\pgfpathrectangle{\pgfqpoint{1.020000in}{0.880000in}}{\pgfqpoint{6.160000in}{6.160000in}}%
\pgfusepath{clip}%
\pgfsetbuttcap%
\pgfsetroundjoin%
\definecolor{currentfill}{rgb}{0.451739,0.588181,0.960201}%
\pgfsetfillcolor{currentfill}%
\pgfsetlinewidth{0.000000pt}%
\definecolor{currentstroke}{rgb}{0.000000,0.000000,0.000000}%
\pgfsetstrokecolor{currentstroke}%
\pgfsetdash{}{0pt}%
\pgfpathmoveto{\pgfqpoint{4.895890in}{3.748679in}}%
\pgfpathlineto{\pgfqpoint{4.903510in}{3.445698in}}%
\pgfpathlineto{\pgfqpoint{4.915056in}{3.686098in}}%
\pgfpathlineto{\pgfqpoint{4.947054in}{3.646441in}}%
\pgfpathlineto{\pgfqpoint{4.978059in}{3.487350in}}%
\pgfpathlineto{\pgfqpoint{4.968594in}{3.537266in}}%
\pgfpathlineto{\pgfqpoint{4.960120in}{3.713782in}}%
\pgfpathlineto{\pgfqpoint{4.926936in}{3.585256in}}%
\pgfpathlineto{\pgfqpoint{4.895890in}{3.748679in}}%
\pgfpathclose%
\pgfusepath{fill}%
\end{pgfscope}%
\begin{pgfscope}%
\pgfpathrectangle{\pgfqpoint{1.020000in}{0.880000in}}{\pgfqpoint{6.160000in}{6.160000in}}%
\pgfusepath{clip}%
\pgfsetbuttcap%
\pgfsetroundjoin%
\definecolor{currentfill}{rgb}{0.548876,0.685104,0.994379}%
\pgfsetfillcolor{currentfill}%
\pgfsetlinewidth{0.000000pt}%
\definecolor{currentstroke}{rgb}{0.000000,0.000000,0.000000}%
\pgfsetstrokecolor{currentstroke}%
\pgfsetdash{}{0pt}%
\pgfpathmoveto{\pgfqpoint{4.811632in}{3.706782in}}%
\pgfpathlineto{\pgfqpoint{4.822096in}{3.827439in}}%
\pgfpathlineto{\pgfqpoint{4.831089in}{3.713655in}}%
\pgfpathlineto{\pgfqpoint{4.862542in}{3.590879in}}%
\pgfpathlineto{\pgfqpoint{4.895890in}{3.748679in}}%
\pgfpathlineto{\pgfqpoint{4.886697in}{3.832593in}}%
\pgfpathlineto{\pgfqpoint{4.876962in}{3.837572in}}%
\pgfpathlineto{\pgfqpoint{4.844995in}{3.879935in}}%
\pgfpathlineto{\pgfqpoint{4.811632in}{3.706782in}}%
\pgfpathclose%
\pgfusepath{fill}%
\end{pgfscope}%
\begin{pgfscope}%
\pgfpathrectangle{\pgfqpoint{1.020000in}{0.880000in}}{\pgfqpoint{6.160000in}{6.160000in}}%
\pgfusepath{clip}%
\pgfsetbuttcap%
\pgfsetroundjoin%
\definecolor{currentfill}{rgb}{0.895882,0.849906,0.823499}%
\pgfsetfillcolor{currentfill}%
\pgfsetlinewidth{0.000000pt}%
\definecolor{currentstroke}{rgb}{0.000000,0.000000,0.000000}%
\pgfsetstrokecolor{currentstroke}%
\pgfsetdash{}{0pt}%
\pgfpathmoveto{\pgfqpoint{3.564275in}{4.496882in}}%
\pgfpathlineto{\pgfqpoint{3.571792in}{4.647354in}}%
\pgfpathlineto{\pgfqpoint{3.581614in}{4.418158in}}%
\pgfpathlineto{\pgfqpoint{3.614374in}{4.399229in}}%
\pgfpathlineto{\pgfqpoint{3.648322in}{4.150680in}}%
\pgfpathlineto{\pgfqpoint{3.638244in}{4.447462in}}%
\pgfpathlineto{\pgfqpoint{3.629887in}{4.424366in}}%
\pgfpathlineto{\pgfqpoint{3.596382in}{4.584688in}}%
\pgfpathlineto{\pgfqpoint{3.564275in}{4.496882in}}%
\pgfpathclose%
\pgfusepath{fill}%
\end{pgfscope}%
\begin{pgfscope}%
\pgfpathrectangle{\pgfqpoint{1.020000in}{0.880000in}}{\pgfqpoint{6.160000in}{6.160000in}}%
\pgfusepath{clip}%
\pgfsetbuttcap%
\pgfsetroundjoin%
\definecolor{currentfill}{rgb}{0.967874,0.725847,0.618489}%
\pgfsetfillcolor{currentfill}%
\pgfsetlinewidth{0.000000pt}%
\definecolor{currentstroke}{rgb}{0.000000,0.000000,0.000000}%
\pgfsetstrokecolor{currentstroke}%
\pgfsetdash{}{0pt}%
\pgfpathmoveto{\pgfqpoint{2.907071in}{4.860940in}}%
\pgfpathlineto{\pgfqpoint{2.913823in}{4.941826in}}%
\pgfpathlineto{\pgfqpoint{2.923765in}{4.768319in}}%
\pgfpathlineto{\pgfqpoint{2.957013in}{4.735736in}}%
\pgfpathlineto{\pgfqpoint{2.990294in}{4.696879in}}%
\pgfpathlineto{\pgfqpoint{2.982212in}{4.718259in}}%
\pgfpathlineto{\pgfqpoint{2.975038in}{4.664533in}}%
\pgfpathlineto{\pgfqpoint{2.941542in}{4.726327in}}%
\pgfpathlineto{\pgfqpoint{2.907071in}{4.860940in}}%
\pgfpathclose%
\pgfusepath{fill}%
\end{pgfscope}%
\begin{pgfscope}%
\pgfpathrectangle{\pgfqpoint{1.020000in}{0.880000in}}{\pgfqpoint{6.160000in}{6.160000in}}%
\pgfusepath{clip}%
\pgfsetbuttcap%
\pgfsetroundjoin%
\definecolor{currentfill}{rgb}{0.608547,0.735725,0.999354}%
\pgfsetfillcolor{currentfill}%
\pgfsetlinewidth{0.000000pt}%
\definecolor{currentstroke}{rgb}{0.000000,0.000000,0.000000}%
\pgfsetstrokecolor{currentstroke}%
\pgfsetdash{}{0pt}%
\pgfpathmoveto{\pgfqpoint{4.386762in}{3.887707in}}%
\pgfpathlineto{\pgfqpoint{4.395741in}{3.750613in}}%
\pgfpathlineto{\pgfqpoint{4.405201in}{3.834226in}}%
\pgfpathlineto{\pgfqpoint{4.437577in}{3.814909in}}%
\pgfpathlineto{\pgfqpoint{4.470243in}{3.908150in}}%
\pgfpathlineto{\pgfqpoint{4.460639in}{3.816630in}}%
\pgfpathlineto{\pgfqpoint{4.451717in}{3.966953in}}%
\pgfpathlineto{\pgfqpoint{4.419321in}{3.963427in}}%
\pgfpathlineto{\pgfqpoint{4.386762in}{3.887707in}}%
\pgfpathclose%
\pgfusepath{fill}%
\end{pgfscope}%
\begin{pgfscope}%
\pgfpathrectangle{\pgfqpoint{1.020000in}{0.880000in}}{\pgfqpoint{6.160000in}{6.160000in}}%
\pgfusepath{clip}%
\pgfsetbuttcap%
\pgfsetroundjoin%
\definecolor{currentfill}{rgb}{0.966962,0.735670,0.630877}%
\pgfsetfillcolor{currentfill}%
\pgfsetlinewidth{0.000000pt}%
\definecolor{currentstroke}{rgb}{0.000000,0.000000,0.000000}%
\pgfsetstrokecolor{currentstroke}%
\pgfsetdash{}{0pt}%
\pgfpathmoveto{\pgfqpoint{3.121134in}{4.717868in}}%
\pgfpathlineto{\pgfqpoint{3.129566in}{4.673629in}}%
\pgfpathlineto{\pgfqpoint{3.135789in}{4.844077in}}%
\pgfpathlineto{\pgfqpoint{3.169271in}{4.778798in}}%
\pgfpathlineto{\pgfqpoint{3.201585in}{4.827743in}}%
\pgfpathlineto{\pgfqpoint{3.193939in}{4.785442in}}%
\pgfpathlineto{\pgfqpoint{3.187820in}{4.590332in}}%
\pgfpathlineto{\pgfqpoint{3.153326in}{4.770333in}}%
\pgfpathlineto{\pgfqpoint{3.121134in}{4.717868in}}%
\pgfpathclose%
\pgfusepath{fill}%
\end{pgfscope}%
\begin{pgfscope}%
\pgfpathrectangle{\pgfqpoint{1.020000in}{0.880000in}}{\pgfqpoint{6.160000in}{6.160000in}}%
\pgfusepath{clip}%
\pgfsetbuttcap%
\pgfsetroundjoin%
\definecolor{currentfill}{rgb}{0.538004,0.674902,0.991722}%
\pgfsetfillcolor{currentfill}%
\pgfsetlinewidth{0.000000pt}%
\definecolor{currentstroke}{rgb}{0.000000,0.000000,0.000000}%
\pgfsetstrokecolor{currentstroke}%
\pgfsetdash{}{0pt}%
\pgfpathmoveto{\pgfqpoint{4.748027in}{3.865832in}}%
\pgfpathlineto{\pgfqpoint{4.756702in}{3.695280in}}%
\pgfpathlineto{\pgfqpoint{4.766677in}{3.750303in}}%
\pgfpathlineto{\pgfqpoint{4.798979in}{3.745216in}}%
\pgfpathlineto{\pgfqpoint{4.831089in}{3.713655in}}%
\pgfpathlineto{\pgfqpoint{4.822096in}{3.827439in}}%
\pgfpathlineto{\pgfqpoint{4.811632in}{3.706782in}}%
\pgfpathlineto{\pgfqpoint{4.779196in}{3.673884in}}%
\pgfpathlineto{\pgfqpoint{4.748027in}{3.865832in}}%
\pgfpathclose%
\pgfusepath{fill}%
\end{pgfscope}%
\begin{pgfscope}%
\pgfpathrectangle{\pgfqpoint{1.020000in}{0.880000in}}{\pgfqpoint{6.160000in}{6.160000in}}%
\pgfusepath{clip}%
\pgfsetbuttcap%
\pgfsetroundjoin%
\definecolor{currentfill}{rgb}{0.804965,0.851666,0.926165}%
\pgfsetfillcolor{currentfill}%
\pgfsetlinewidth{0.000000pt}%
\definecolor{currentstroke}{rgb}{0.000000,0.000000,0.000000}%
\pgfsetstrokecolor{currentstroke}%
\pgfsetdash{}{0pt}%
\pgfpathmoveto{\pgfqpoint{3.795273in}{4.230684in}}%
\pgfpathlineto{\pgfqpoint{3.803803in}{4.265325in}}%
\pgfpathlineto{\pgfqpoint{3.812296in}{4.316255in}}%
\pgfpathlineto{\pgfqpoint{3.845367in}{4.185250in}}%
\pgfpathlineto{\pgfqpoint{3.878000in}{4.164933in}}%
\pgfpathlineto{\pgfqpoint{3.868823in}{4.307245in}}%
\pgfpathlineto{\pgfqpoint{3.860673in}{4.126690in}}%
\pgfpathlineto{\pgfqpoint{3.827477in}{4.326968in}}%
\pgfpathlineto{\pgfqpoint{3.795273in}{4.230684in}}%
\pgfpathclose%
\pgfusepath{fill}%
\end{pgfscope}%
\begin{pgfscope}%
\pgfpathrectangle{\pgfqpoint{1.020000in}{0.880000in}}{\pgfqpoint{6.160000in}{6.160000in}}%
\pgfusepath{clip}%
\pgfsetbuttcap%
\pgfsetroundjoin%
\definecolor{currentfill}{rgb}{0.430507,0.564883,0.948889}%
\pgfsetfillcolor{currentfill}%
\pgfsetlinewidth{0.000000pt}%
\definecolor{currentstroke}{rgb}{0.000000,0.000000,0.000000}%
\pgfsetstrokecolor{currentstroke}%
\pgfsetdash{}{0pt}%
\pgfpathmoveto{\pgfqpoint{5.536377in}{3.767801in}}%
\pgfpathlineto{\pgfqpoint{5.543148in}{3.495266in}}%
\pgfpathlineto{\pgfqpoint{5.554582in}{3.565850in}}%
\pgfpathlineto{\pgfqpoint{5.586642in}{3.562351in}}%
\pgfpathlineto{\pgfqpoint{5.619287in}{3.601018in}}%
\pgfpathlineto{\pgfqpoint{5.607987in}{3.547538in}}%
\pgfpathlineto{\pgfqpoint{5.597366in}{3.539817in}}%
\pgfpathlineto{\pgfqpoint{5.563498in}{3.406112in}}%
\pgfpathlineto{\pgfqpoint{5.536377in}{3.767801in}}%
\pgfpathclose%
\pgfusepath{fill}%
\end{pgfscope}%
\begin{pgfscope}%
\pgfpathrectangle{\pgfqpoint{1.020000in}{0.880000in}}{\pgfqpoint{6.160000in}{6.160000in}}%
\pgfusepath{clip}%
\pgfsetbuttcap%
\pgfsetroundjoin%
\definecolor{currentfill}{rgb}{0.473070,0.611077,0.970634}%
\pgfsetfillcolor{currentfill}%
\pgfsetlinewidth{0.000000pt}%
\definecolor{currentstroke}{rgb}{0.000000,0.000000,0.000000}%
\pgfsetstrokecolor{currentstroke}%
\pgfsetdash{}{0pt}%
\pgfpathmoveto{\pgfqpoint{5.108058in}{3.629726in}}%
\pgfpathlineto{\pgfqpoint{5.117381in}{3.555793in}}%
\pgfpathlineto{\pgfqpoint{5.127492in}{3.564728in}}%
\pgfpathlineto{\pgfqpoint{5.161308in}{3.729368in}}%
\pgfpathlineto{\pgfqpoint{5.192126in}{3.587746in}}%
\pgfpathlineto{\pgfqpoint{5.182952in}{3.679599in}}%
\pgfpathlineto{\pgfqpoint{5.172094in}{3.600638in}}%
\pgfpathlineto{\pgfqpoint{5.141252in}{3.736817in}}%
\pgfpathlineto{\pgfqpoint{5.108058in}{3.629726in}}%
\pgfpathclose%
\pgfusepath{fill}%
\end{pgfscope}%
\begin{pgfscope}%
\pgfpathrectangle{\pgfqpoint{1.020000in}{0.880000in}}{\pgfqpoint{6.160000in}{6.160000in}}%
\pgfusepath{clip}%
\pgfsetbuttcap%
\pgfsetroundjoin%
\definecolor{currentfill}{rgb}{0.640828,0.760752,0.997846}%
\pgfsetfillcolor{currentfill}%
\pgfsetlinewidth{0.000000pt}%
\definecolor{currentstroke}{rgb}{0.000000,0.000000,0.000000}%
\pgfsetstrokecolor{currentstroke}%
\pgfsetdash{}{0pt}%
\pgfpathmoveto{\pgfqpoint{4.173989in}{3.736565in}}%
\pgfpathlineto{\pgfqpoint{4.183022in}{3.979044in}}%
\pgfpathlineto{\pgfqpoint{4.192087in}{3.881515in}}%
\pgfpathlineto{\pgfqpoint{4.224632in}{4.014810in}}%
\pgfpathlineto{\pgfqpoint{4.257060in}{3.896568in}}%
\pgfpathlineto{\pgfqpoint{4.247961in}{3.940987in}}%
\pgfpathlineto{\pgfqpoint{4.238876in}{3.992651in}}%
\pgfpathlineto{\pgfqpoint{4.206416in}{3.931442in}}%
\pgfpathlineto{\pgfqpoint{4.173989in}{3.736565in}}%
\pgfpathclose%
\pgfusepath{fill}%
\end{pgfscope}%
\begin{pgfscope}%
\pgfpathrectangle{\pgfqpoint{1.020000in}{0.880000in}}{\pgfqpoint{6.160000in}{6.160000in}}%
\pgfusepath{clip}%
\pgfsetbuttcap%
\pgfsetroundjoin%
\definecolor{currentfill}{rgb}{0.425199,0.559058,0.946061}%
\pgfsetfillcolor{currentfill}%
\pgfsetlinewidth{0.000000pt}%
\definecolor{currentstroke}{rgb}{0.000000,0.000000,0.000000}%
\pgfsetstrokecolor{currentstroke}%
\pgfsetdash{}{0pt}%
\pgfpathmoveto{\pgfqpoint{5.256181in}{3.560902in}}%
\pgfpathlineto{\pgfqpoint{5.266377in}{3.563403in}}%
\pgfpathlineto{\pgfqpoint{5.276607in}{3.566982in}}%
\pgfpathlineto{\pgfqpoint{5.309206in}{3.602699in}}%
\pgfpathlineto{\pgfqpoint{5.339342in}{3.425048in}}%
\pgfpathlineto{\pgfqpoint{5.328701in}{3.389702in}}%
\pgfpathlineto{\pgfqpoint{5.320228in}{3.542863in}}%
\pgfpathlineto{\pgfqpoint{5.290201in}{3.731677in}}%
\pgfpathlineto{\pgfqpoint{5.256181in}{3.560902in}}%
\pgfpathclose%
\pgfusepath{fill}%
\end{pgfscope}%
\begin{pgfscope}%
\pgfpathrectangle{\pgfqpoint{1.020000in}{0.880000in}}{\pgfqpoint{6.160000in}{6.160000in}}%
\pgfusepath{clip}%
\pgfsetbuttcap%
\pgfsetroundjoin%
\definecolor{currentfill}{rgb}{0.592356,0.722792,0.999434}%
\pgfsetfillcolor{currentfill}%
\pgfsetlinewidth{0.000000pt}%
\definecolor{currentstroke}{rgb}{0.000000,0.000000,0.000000}%
\pgfsetstrokecolor{currentstroke}%
\pgfsetdash{}{0pt}%
\pgfpathmoveto{\pgfqpoint{4.321895in}{3.851319in}}%
\pgfpathlineto{\pgfqpoint{4.331156in}{3.890094in}}%
\pgfpathlineto{\pgfqpoint{4.340174in}{3.756377in}}%
\pgfpathlineto{\pgfqpoint{4.372699in}{3.802932in}}%
\pgfpathlineto{\pgfqpoint{4.405201in}{3.834226in}}%
\pgfpathlineto{\pgfqpoint{4.395741in}{3.750613in}}%
\pgfpathlineto{\pgfqpoint{4.386762in}{3.887707in}}%
\pgfpathlineto{\pgfqpoint{4.354463in}{3.946299in}}%
\pgfpathlineto{\pgfqpoint{4.321895in}{3.851319in}}%
\pgfpathclose%
\pgfusepath{fill}%
\end{pgfscope}%
\begin{pgfscope}%
\pgfpathrectangle{\pgfqpoint{1.020000in}{0.880000in}}{\pgfqpoint{6.160000in}{6.160000in}}%
\pgfusepath{clip}%
\pgfsetbuttcap%
\pgfsetroundjoin%
\definecolor{currentfill}{rgb}{0.960581,0.762501,0.667964}%
\pgfsetfillcolor{currentfill}%
\pgfsetlinewidth{0.000000pt}%
\definecolor{currentstroke}{rgb}{0.000000,0.000000,0.000000}%
\pgfsetstrokecolor{currentstroke}%
\pgfsetdash{}{0pt}%
\pgfpathmoveto{\pgfqpoint{3.267849in}{4.748999in}}%
\pgfpathlineto{\pgfqpoint{3.275983in}{4.748777in}}%
\pgfpathlineto{\pgfqpoint{3.284658in}{4.689365in}}%
\pgfpathlineto{\pgfqpoint{3.315170in}{4.952868in}}%
\pgfpathlineto{\pgfqpoint{3.349725in}{4.743575in}}%
\pgfpathlineto{\pgfqpoint{3.343293in}{4.529568in}}%
\pgfpathlineto{\pgfqpoint{3.334692in}{4.580252in}}%
\pgfpathlineto{\pgfqpoint{3.302478in}{4.530252in}}%
\pgfpathlineto{\pgfqpoint{3.267849in}{4.748999in}}%
\pgfpathclose%
\pgfusepath{fill}%
\end{pgfscope}%
\begin{pgfscope}%
\pgfpathrectangle{\pgfqpoint{1.020000in}{0.880000in}}{\pgfqpoint{6.160000in}{6.160000in}}%
\pgfusepath{clip}%
\pgfsetbuttcap%
\pgfsetroundjoin%
\definecolor{currentfill}{rgb}{0.863392,0.865084,0.867634}%
\pgfsetfillcolor{currentfill}%
\pgfsetlinewidth{0.000000pt}%
\definecolor{currentstroke}{rgb}{0.000000,0.000000,0.000000}%
\pgfsetstrokecolor{currentstroke}%
\pgfsetdash{}{0pt}%
\pgfpathmoveto{\pgfqpoint{3.648322in}{4.150680in}}%
\pgfpathlineto{\pgfqpoint{3.655934in}{4.319532in}}%
\pgfpathlineto{\pgfqpoint{3.663892in}{4.429909in}}%
\pgfpathlineto{\pgfqpoint{3.696872in}{4.367962in}}%
\pgfpathlineto{\pgfqpoint{3.729746in}{4.316706in}}%
\pgfpathlineto{\pgfqpoint{3.719928in}{4.585359in}}%
\pgfpathlineto{\pgfqpoint{3.712207in}{4.398448in}}%
\pgfpathlineto{\pgfqpoint{3.679921in}{4.337200in}}%
\pgfpathlineto{\pgfqpoint{3.648322in}{4.150680in}}%
\pgfpathclose%
\pgfusepath{fill}%
\end{pgfscope}%
\begin{pgfscope}%
\pgfpathrectangle{\pgfqpoint{1.020000in}{0.880000in}}{\pgfqpoint{6.160000in}{6.160000in}}%
\pgfusepath{clip}%
\pgfsetbuttcap%
\pgfsetroundjoin%
\definecolor{currentfill}{rgb}{0.748682,0.827679,0.963334}%
\pgfsetfillcolor{currentfill}%
\pgfsetlinewidth{0.000000pt}%
\definecolor{currentstroke}{rgb}{0.000000,0.000000,0.000000}%
\pgfsetstrokecolor{currentstroke}%
\pgfsetdash{}{0pt}%
\pgfpathmoveto{\pgfqpoint{3.878000in}{4.164933in}}%
\pgfpathlineto{\pgfqpoint{3.886592in}{4.219322in}}%
\pgfpathlineto{\pgfqpoint{3.895173in}{4.286740in}}%
\pgfpathlineto{\pgfqpoint{3.928400in}{4.057945in}}%
\pgfpathlineto{\pgfqpoint{3.960867in}{4.097449in}}%
\pgfpathlineto{\pgfqpoint{3.952055in}{4.091685in}}%
\pgfpathlineto{\pgfqpoint{3.943336in}{4.054706in}}%
\pgfpathlineto{\pgfqpoint{3.910960in}{4.012000in}}%
\pgfpathlineto{\pgfqpoint{3.878000in}{4.164933in}}%
\pgfpathclose%
\pgfusepath{fill}%
\end{pgfscope}%
\begin{pgfscope}%
\pgfpathrectangle{\pgfqpoint{1.020000in}{0.880000in}}{\pgfqpoint{6.160000in}{6.160000in}}%
\pgfusepath{clip}%
\pgfsetbuttcap%
\pgfsetroundjoin%
\definecolor{currentfill}{rgb}{0.703587,0.802586,0.982847}%
\pgfsetfillcolor{currentfill}%
\pgfsetlinewidth{0.000000pt}%
\definecolor{currentstroke}{rgb}{0.000000,0.000000,0.000000}%
\pgfsetstrokecolor{currentstroke}%
\pgfsetdash{}{0pt}%
\pgfpathmoveto{\pgfqpoint{4.025912in}{4.118978in}}%
\pgfpathlineto{\pgfqpoint{4.034937in}{4.039087in}}%
\pgfpathlineto{\pgfqpoint{4.043680in}{4.166392in}}%
\pgfpathlineto{\pgfqpoint{4.076490in}{3.964005in}}%
\pgfpathlineto{\pgfqpoint{4.109065in}{3.878058in}}%
\pgfpathlineto{\pgfqpoint{4.099947in}{4.061911in}}%
\pgfpathlineto{\pgfqpoint{4.091066in}{3.970578in}}%
\pgfpathlineto{\pgfqpoint{4.058479in}{4.079968in}}%
\pgfpathlineto{\pgfqpoint{4.025912in}{4.118978in}}%
\pgfpathclose%
\pgfusepath{fill}%
\end{pgfscope}%
\begin{pgfscope}%
\pgfpathrectangle{\pgfqpoint{1.020000in}{0.880000in}}{\pgfqpoint{6.160000in}{6.160000in}}%
\pgfusepath{clip}%
\pgfsetbuttcap%
\pgfsetroundjoin%
\definecolor{currentfill}{rgb}{0.462354,0.599830,0.965857}%
\pgfsetfillcolor{currentfill}%
\pgfsetlinewidth{0.000000pt}%
\definecolor{currentstroke}{rgb}{0.000000,0.000000,0.000000}%
\pgfsetstrokecolor{currentstroke}%
\pgfsetdash{}{0pt}%
\pgfpathmoveto{\pgfqpoint{5.043743in}{3.639211in}}%
\pgfpathlineto{\pgfqpoint{5.053429in}{3.608508in}}%
\pgfpathlineto{\pgfqpoint{5.062806in}{3.540917in}}%
\pgfpathlineto{\pgfqpoint{5.094840in}{3.518332in}}%
\pgfpathlineto{\pgfqpoint{5.127492in}{3.564728in}}%
\pgfpathlineto{\pgfqpoint{5.117381in}{3.555793in}}%
\pgfpathlineto{\pgfqpoint{5.108058in}{3.629726in}}%
\pgfpathlineto{\pgfqpoint{5.077322in}{3.792814in}}%
\pgfpathlineto{\pgfqpoint{5.043743in}{3.639211in}}%
\pgfpathclose%
\pgfusepath{fill}%
\end{pgfscope}%
\begin{pgfscope}%
\pgfpathrectangle{\pgfqpoint{1.020000in}{0.880000in}}{\pgfqpoint{6.160000in}{6.160000in}}%
\pgfusepath{clip}%
\pgfsetbuttcap%
\pgfsetroundjoin%
\definecolor{currentfill}{rgb}{0.478462,0.616564,0.972721}%
\pgfsetfillcolor{currentfill}%
\pgfsetlinewidth{0.000000pt}%
\definecolor{currentstroke}{rgb}{0.000000,0.000000,0.000000}%
\pgfsetstrokecolor{currentstroke}%
\pgfsetdash{}{0pt}%
\pgfpathmoveto{\pgfqpoint{4.831089in}{3.713655in}}%
\pgfpathlineto{\pgfqpoint{4.840300in}{3.633406in}}%
\pgfpathlineto{\pgfqpoint{4.849769in}{3.591270in}}%
\pgfpathlineto{\pgfqpoint{4.882946in}{3.715938in}}%
\pgfpathlineto{\pgfqpoint{4.915056in}{3.686098in}}%
\pgfpathlineto{\pgfqpoint{4.903510in}{3.445698in}}%
\pgfpathlineto{\pgfqpoint{4.895890in}{3.748679in}}%
\pgfpathlineto{\pgfqpoint{4.862542in}{3.590879in}}%
\pgfpathlineto{\pgfqpoint{4.831089in}{3.713655in}}%
\pgfpathclose%
\pgfusepath{fill}%
\end{pgfscope}%
\begin{pgfscope}%
\pgfpathrectangle{\pgfqpoint{1.020000in}{0.880000in}}{\pgfqpoint{6.160000in}{6.160000in}}%
\pgfusepath{clip}%
\pgfsetbuttcap%
\pgfsetroundjoin%
\definecolor{currentfill}{rgb}{0.624703,0.748318,0.998719}%
\pgfsetfillcolor{currentfill}%
\pgfsetlinewidth{0.000000pt}%
\definecolor{currentstroke}{rgb}{0.000000,0.000000,0.000000}%
\pgfsetstrokecolor{currentstroke}%
\pgfsetdash{}{0pt}%
\pgfpathmoveto{\pgfqpoint{4.109065in}{3.878058in}}%
\pgfpathlineto{\pgfqpoint{4.118111in}{3.783298in}}%
\pgfpathlineto{\pgfqpoint{4.126934in}{4.088962in}}%
\pgfpathlineto{\pgfqpoint{4.159617in}{3.717655in}}%
\pgfpathlineto{\pgfqpoint{4.192087in}{3.881515in}}%
\pgfpathlineto{\pgfqpoint{4.183022in}{3.979044in}}%
\pgfpathlineto{\pgfqpoint{4.173989in}{3.736565in}}%
\pgfpathlineto{\pgfqpoint{4.141453in}{4.063574in}}%
\pgfpathlineto{\pgfqpoint{4.109065in}{3.878058in}}%
\pgfpathclose%
\pgfusepath{fill}%
\end{pgfscope}%
\begin{pgfscope}%
\pgfpathrectangle{\pgfqpoint{1.020000in}{0.880000in}}{\pgfqpoint{6.160000in}{6.160000in}}%
\pgfusepath{clip}%
\pgfsetbuttcap%
\pgfsetroundjoin%
\definecolor{currentfill}{rgb}{0.441123,0.576532,0.954545}%
\pgfsetfillcolor{currentfill}%
\pgfsetlinewidth{0.000000pt}%
\definecolor{currentstroke}{rgb}{0.000000,0.000000,0.000000}%
\pgfsetstrokecolor{currentstroke}%
\pgfsetdash{}{0pt}%
\pgfpathmoveto{\pgfqpoint{4.978059in}{3.487350in}}%
\pgfpathlineto{\pgfqpoint{4.987829in}{3.473920in}}%
\pgfpathlineto{\pgfqpoint{4.998131in}{3.523816in}}%
\pgfpathlineto{\pgfqpoint{5.031590in}{3.664123in}}%
\pgfpathlineto{\pgfqpoint{5.062806in}{3.540917in}}%
\pgfpathlineto{\pgfqpoint{5.053429in}{3.608508in}}%
\pgfpathlineto{\pgfqpoint{5.043743in}{3.639211in}}%
\pgfpathlineto{\pgfqpoint{5.011435in}{3.629671in}}%
\pgfpathlineto{\pgfqpoint{4.978059in}{3.487350in}}%
\pgfpathclose%
\pgfusepath{fill}%
\end{pgfscope}%
\begin{pgfscope}%
\pgfpathrectangle{\pgfqpoint{1.020000in}{0.880000in}}{\pgfqpoint{6.160000in}{6.160000in}}%
\pgfusepath{clip}%
\pgfsetbuttcap%
\pgfsetroundjoin%
\definecolor{currentfill}{rgb}{0.548876,0.685104,0.994379}%
\pgfsetfillcolor{currentfill}%
\pgfsetlinewidth{0.000000pt}%
\definecolor{currentstroke}{rgb}{0.000000,0.000000,0.000000}%
\pgfsetstrokecolor{currentstroke}%
\pgfsetdash{}{0pt}%
\pgfpathmoveto{\pgfqpoint{4.682494in}{3.693757in}}%
\pgfpathlineto{\pgfqpoint{4.692635in}{3.805696in}}%
\pgfpathlineto{\pgfqpoint{4.701455in}{3.654164in}}%
\pgfpathlineto{\pgfqpoint{4.734833in}{3.842916in}}%
\pgfpathlineto{\pgfqpoint{4.766677in}{3.750303in}}%
\pgfpathlineto{\pgfqpoint{4.756702in}{3.695280in}}%
\pgfpathlineto{\pgfqpoint{4.748027in}{3.865832in}}%
\pgfpathlineto{\pgfqpoint{4.715083in}{3.749742in}}%
\pgfpathlineto{\pgfqpoint{4.682494in}{3.693757in}}%
\pgfpathclose%
\pgfusepath{fill}%
\end{pgfscope}%
\begin{pgfscope}%
\pgfpathrectangle{\pgfqpoint{1.020000in}{0.880000in}}{\pgfqpoint{6.160000in}{6.160000in}}%
\pgfusepath{clip}%
\pgfsetbuttcap%
\pgfsetroundjoin%
\definecolor{currentfill}{rgb}{0.968203,0.720844,0.612293}%
\pgfsetfillcolor{currentfill}%
\pgfsetlinewidth{0.000000pt}%
\definecolor{currentstroke}{rgb}{0.000000,0.000000,0.000000}%
\pgfsetstrokecolor{currentstroke}%
\pgfsetdash{}{0pt}%
\pgfpathmoveto{\pgfqpoint{3.055377in}{4.740374in}}%
\pgfpathlineto{\pgfqpoint{3.062181in}{4.840556in}}%
\pgfpathlineto{\pgfqpoint{3.071390in}{4.724455in}}%
\pgfpathlineto{\pgfqpoint{3.102554in}{4.880516in}}%
\pgfpathlineto{\pgfqpoint{3.135789in}{4.844077in}}%
\pgfpathlineto{\pgfqpoint{3.129566in}{4.673629in}}%
\pgfpathlineto{\pgfqpoint{3.121134in}{4.717868in}}%
\pgfpathlineto{\pgfqpoint{3.088670in}{4.691947in}}%
\pgfpathlineto{\pgfqpoint{3.055377in}{4.740374in}}%
\pgfpathclose%
\pgfusepath{fill}%
\end{pgfscope}%
\begin{pgfscope}%
\pgfpathrectangle{\pgfqpoint{1.020000in}{0.880000in}}{\pgfqpoint{6.160000in}{6.160000in}}%
\pgfusepath{clip}%
\pgfsetbuttcap%
\pgfsetroundjoin%
\definecolor{currentfill}{rgb}{0.521696,0.659599,0.987736}%
\pgfsetfillcolor{currentfill}%
\pgfsetlinewidth{0.000000pt}%
\definecolor{currentstroke}{rgb}{0.000000,0.000000,0.000000}%
\pgfsetstrokecolor{currentstroke}%
\pgfsetdash{}{0pt}%
\pgfpathmoveto{\pgfqpoint{4.617615in}{3.638645in}}%
\pgfpathlineto{\pgfqpoint{4.627549in}{3.735101in}}%
\pgfpathlineto{\pgfqpoint{4.636912in}{3.696787in}}%
\pgfpathlineto{\pgfqpoint{4.669261in}{3.688183in}}%
\pgfpathlineto{\pgfqpoint{4.701455in}{3.654164in}}%
\pgfpathlineto{\pgfqpoint{4.692635in}{3.805696in}}%
\pgfpathlineto{\pgfqpoint{4.682494in}{3.693757in}}%
\pgfpathlineto{\pgfqpoint{4.650425in}{3.745672in}}%
\pgfpathlineto{\pgfqpoint{4.617615in}{3.638645in}}%
\pgfpathclose%
\pgfusepath{fill}%
\end{pgfscope}%
\begin{pgfscope}%
\pgfpathrectangle{\pgfqpoint{1.020000in}{0.880000in}}{\pgfqpoint{6.160000in}{6.160000in}}%
\pgfusepath{clip}%
\pgfsetbuttcap%
\pgfsetroundjoin%
\definecolor{currentfill}{rgb}{0.576051,0.708780,0.997755}%
\pgfsetfillcolor{currentfill}%
\pgfsetlinewidth{0.000000pt}%
\definecolor{currentstroke}{rgb}{0.000000,0.000000,0.000000}%
\pgfsetstrokecolor{currentstroke}%
\pgfsetdash{}{0pt}%
\pgfpathmoveto{\pgfqpoint{4.470243in}{3.908150in}}%
\pgfpathlineto{\pgfqpoint{4.479329in}{3.811601in}}%
\pgfpathlineto{\pgfqpoint{4.488283in}{3.671172in}}%
\pgfpathlineto{\pgfqpoint{4.521210in}{3.828923in}}%
\pgfpathlineto{\pgfqpoint{4.553798in}{3.872197in}}%
\pgfpathlineto{\pgfqpoint{4.544064in}{3.792552in}}%
\pgfpathlineto{\pgfqpoint{4.534677in}{3.802391in}}%
\pgfpathlineto{\pgfqpoint{4.502061in}{3.721494in}}%
\pgfpathlineto{\pgfqpoint{4.470243in}{3.908150in}}%
\pgfpathclose%
\pgfusepath{fill}%
\end{pgfscope}%
\begin{pgfscope}%
\pgfpathrectangle{\pgfqpoint{1.020000in}{0.880000in}}{\pgfqpoint{6.160000in}{6.160000in}}%
\pgfusepath{clip}%
\pgfsetbuttcap%
\pgfsetroundjoin%
\definecolor{currentfill}{rgb}{0.435815,0.570707,0.951717}%
\pgfsetfillcolor{currentfill}%
\pgfsetlinewidth{0.000000pt}%
\definecolor{currentstroke}{rgb}{0.000000,0.000000,0.000000}%
\pgfsetstrokecolor{currentstroke}%
\pgfsetdash{}{0pt}%
\pgfpathmoveto{\pgfqpoint{5.192126in}{3.587746in}}%
\pgfpathlineto{\pgfqpoint{5.201695in}{3.534442in}}%
\pgfpathlineto{\pgfqpoint{5.211520in}{3.505114in}}%
\pgfpathlineto{\pgfqpoint{5.244658in}{3.592244in}}%
\pgfpathlineto{\pgfqpoint{5.276607in}{3.566982in}}%
\pgfpathlineto{\pgfqpoint{5.266377in}{3.563403in}}%
\pgfpathlineto{\pgfqpoint{5.256181in}{3.560902in}}%
\pgfpathlineto{\pgfqpoint{5.224269in}{3.584084in}}%
\pgfpathlineto{\pgfqpoint{5.192126in}{3.587746in}}%
\pgfpathclose%
\pgfusepath{fill}%
\end{pgfscope}%
\begin{pgfscope}%
\pgfpathrectangle{\pgfqpoint{1.020000in}{0.880000in}}{\pgfqpoint{6.160000in}{6.160000in}}%
\pgfusepath{clip}%
\pgfsetbuttcap%
\pgfsetroundjoin%
\definecolor{currentfill}{rgb}{0.831148,0.859513,0.903110}%
\pgfsetfillcolor{currentfill}%
\pgfsetlinewidth{0.000000pt}%
\definecolor{currentstroke}{rgb}{0.000000,0.000000,0.000000}%
\pgfsetstrokecolor{currentstroke}%
\pgfsetdash{}{0pt}%
\pgfpathmoveto{\pgfqpoint{3.729746in}{4.316706in}}%
\pgfpathlineto{\pgfqpoint{3.738447in}{4.294935in}}%
\pgfpathlineto{\pgfqpoint{3.746260in}{4.481760in}}%
\pgfpathlineto{\pgfqpoint{3.780208in}{4.175649in}}%
\pgfpathlineto{\pgfqpoint{3.812296in}{4.316255in}}%
\pgfpathlineto{\pgfqpoint{3.803803in}{4.265325in}}%
\pgfpathlineto{\pgfqpoint{3.795273in}{4.230684in}}%
\pgfpathlineto{\pgfqpoint{3.762996in}{4.162596in}}%
\pgfpathlineto{\pgfqpoint{3.729746in}{4.316706in}}%
\pgfpathclose%
\pgfusepath{fill}%
\end{pgfscope}%
\begin{pgfscope}%
\pgfpathrectangle{\pgfqpoint{1.020000in}{0.880000in}}{\pgfqpoint{6.160000in}{6.160000in}}%
\pgfusepath{clip}%
\pgfsetbuttcap%
\pgfsetroundjoin%
\definecolor{currentfill}{rgb}{0.949151,0.790785,0.710876}%
\pgfsetfillcolor{currentfill}%
\pgfsetlinewidth{0.000000pt}%
\definecolor{currentstroke}{rgb}{0.000000,0.000000,0.000000}%
\pgfsetstrokecolor{currentstroke}%
\pgfsetdash{}{0pt}%
\pgfpathmoveto{\pgfqpoint{3.349725in}{4.743575in}}%
\pgfpathlineto{\pgfqpoint{3.358211in}{4.711634in}}%
\pgfpathlineto{\pgfqpoint{3.366637in}{4.688686in}}%
\pgfpathlineto{\pgfqpoint{3.399367in}{4.696545in}}%
\pgfpathlineto{\pgfqpoint{3.433962in}{4.451032in}}%
\pgfpathlineto{\pgfqpoint{3.424303in}{4.634725in}}%
\pgfpathlineto{\pgfqpoint{3.417102in}{4.492349in}}%
\pgfpathlineto{\pgfqpoint{3.384437in}{4.494767in}}%
\pgfpathlineto{\pgfqpoint{3.349725in}{4.743575in}}%
\pgfpathclose%
\pgfusepath{fill}%
\end{pgfscope}%
\begin{pgfscope}%
\pgfpathrectangle{\pgfqpoint{1.020000in}{0.880000in}}{\pgfqpoint{6.160000in}{6.160000in}}%
\pgfusepath{clip}%
\pgfsetbuttcap%
\pgfsetroundjoin%
\definecolor{currentfill}{rgb}{0.399231,0.528528,0.928459}%
\pgfsetfillcolor{currentfill}%
\pgfsetlinewidth{0.000000pt}%
\definecolor{currentstroke}{rgb}{0.000000,0.000000,0.000000}%
\pgfsetstrokecolor{currentstroke}%
\pgfsetdash{}{0pt}%
\pgfpathmoveto{\pgfqpoint{5.619287in}{3.601018in}}%
\pgfpathlineto{\pgfqpoint{5.626699in}{3.383042in}}%
\pgfpathlineto{\pgfqpoint{5.639296in}{3.523111in}}%
\pgfpathlineto{\pgfqpoint{5.670585in}{3.467275in}}%
\pgfpathlineto{\pgfqpoint{5.660130in}{3.477006in}}%
\pgfpathlineto{\pgfqpoint{5.649974in}{3.505726in}}%
\pgfpathlineto{\pgfqpoint{5.619287in}{3.601018in}}%
\pgfpathclose%
\pgfusepath{fill}%
\end{pgfscope}%
\begin{pgfscope}%
\pgfpathrectangle{\pgfqpoint{1.020000in}{0.880000in}}{\pgfqpoint{6.160000in}{6.160000in}}%
\pgfusepath{clip}%
\pgfsetbuttcap%
\pgfsetroundjoin%
\definecolor{currentfill}{rgb}{0.462354,0.599830,0.965857}%
\pgfsetfillcolor{currentfill}%
\pgfsetlinewidth{0.000000pt}%
\definecolor{currentstroke}{rgb}{0.000000,0.000000,0.000000}%
\pgfsetstrokecolor{currentstroke}%
\pgfsetdash{}{0pt}%
\pgfpathmoveto{\pgfqpoint{5.469194in}{3.542005in}}%
\pgfpathlineto{\pgfqpoint{5.482456in}{3.763809in}}%
\pgfpathlineto{\pgfqpoint{5.489156in}{3.477619in}}%
\pgfpathlineto{\pgfqpoint{5.521798in}{3.516809in}}%
\pgfpathlineto{\pgfqpoint{5.554582in}{3.565850in}}%
\pgfpathlineto{\pgfqpoint{5.543148in}{3.495266in}}%
\pgfpathlineto{\pgfqpoint{5.536377in}{3.767801in}}%
\pgfpathlineto{\pgfqpoint{5.502870in}{3.663345in}}%
\pgfpathlineto{\pgfqpoint{5.469194in}{3.542005in}}%
\pgfpathclose%
\pgfusepath{fill}%
\end{pgfscope}%
\begin{pgfscope}%
\pgfpathrectangle{\pgfqpoint{1.020000in}{0.880000in}}{\pgfqpoint{6.160000in}{6.160000in}}%
\pgfusepath{clip}%
\pgfsetbuttcap%
\pgfsetroundjoin%
\definecolor{currentfill}{rgb}{0.875557,0.860242,0.851430}%
\pgfsetfillcolor{currentfill}%
\pgfsetlinewidth{0.000000pt}%
\definecolor{currentstroke}{rgb}{0.000000,0.000000,0.000000}%
\pgfsetstrokecolor{currentstroke}%
\pgfsetdash{}{0pt}%
\pgfpathmoveto{\pgfqpoint{3.581614in}{4.418158in}}%
\pgfpathlineto{\pgfqpoint{3.589588in}{4.500074in}}%
\pgfpathlineto{\pgfqpoint{3.597664in}{4.569579in}}%
\pgfpathlineto{\pgfqpoint{3.632040in}{4.274666in}}%
\pgfpathlineto{\pgfqpoint{3.663892in}{4.429909in}}%
\pgfpathlineto{\pgfqpoint{3.655934in}{4.319532in}}%
\pgfpathlineto{\pgfqpoint{3.648322in}{4.150680in}}%
\pgfpathlineto{\pgfqpoint{3.614374in}{4.399229in}}%
\pgfpathlineto{\pgfqpoint{3.581614in}{4.418158in}}%
\pgfpathclose%
\pgfusepath{fill}%
\end{pgfscope}%
\begin{pgfscope}%
\pgfpathrectangle{\pgfqpoint{1.020000in}{0.880000in}}{\pgfqpoint{6.160000in}{6.160000in}}%
\pgfusepath{clip}%
\pgfsetbuttcap%
\pgfsetroundjoin%
\definecolor{currentfill}{rgb}{0.930669,0.818877,0.759146}%
\pgfsetfillcolor{currentfill}%
\pgfsetlinewidth{0.000000pt}%
\definecolor{currentstroke}{rgb}{0.000000,0.000000,0.000000}%
\pgfsetstrokecolor{currentstroke}%
\pgfsetdash{}{0pt}%
\pgfpathmoveto{\pgfqpoint{3.499821in}{4.373549in}}%
\pgfpathlineto{\pgfqpoint{3.507218in}{4.516704in}}%
\pgfpathlineto{\pgfqpoint{3.514994in}{4.608614in}}%
\pgfpathlineto{\pgfqpoint{3.548032in}{4.563083in}}%
\pgfpathlineto{\pgfqpoint{3.581614in}{4.418158in}}%
\pgfpathlineto{\pgfqpoint{3.571792in}{4.647354in}}%
\pgfpathlineto{\pgfqpoint{3.564275in}{4.496882in}}%
\pgfpathlineto{\pgfqpoint{3.530633in}{4.652963in}}%
\pgfpathlineto{\pgfqpoint{3.499821in}{4.373549in}}%
\pgfpathclose%
\pgfusepath{fill}%
\end{pgfscope}%
\begin{pgfscope}%
\pgfpathrectangle{\pgfqpoint{1.020000in}{0.880000in}}{\pgfqpoint{6.160000in}{6.160000in}}%
\pgfusepath{clip}%
\pgfsetbuttcap%
\pgfsetroundjoin%
\definecolor{currentfill}{rgb}{0.494638,0.633022,0.978983}%
\pgfsetfillcolor{currentfill}%
\pgfsetlinewidth{0.000000pt}%
\definecolor{currentstroke}{rgb}{0.000000,0.000000,0.000000}%
\pgfsetstrokecolor{currentstroke}%
\pgfsetdash{}{0pt}%
\pgfpathmoveto{\pgfqpoint{4.766677in}{3.750303in}}%
\pgfpathlineto{\pgfqpoint{4.775849in}{3.664580in}}%
\pgfpathlineto{\pgfqpoint{4.785285in}{3.622006in}}%
\pgfpathlineto{\pgfqpoint{4.816986in}{3.518884in}}%
\pgfpathlineto{\pgfqpoint{4.849769in}{3.591270in}}%
\pgfpathlineto{\pgfqpoint{4.840300in}{3.633406in}}%
\pgfpathlineto{\pgfqpoint{4.831089in}{3.713655in}}%
\pgfpathlineto{\pgfqpoint{4.798979in}{3.745216in}}%
\pgfpathlineto{\pgfqpoint{4.766677in}{3.750303in}}%
\pgfpathclose%
\pgfusepath{fill}%
\end{pgfscope}%
\begin{pgfscope}%
\pgfpathrectangle{\pgfqpoint{1.020000in}{0.880000in}}{\pgfqpoint{6.160000in}{6.160000in}}%
\pgfusepath{clip}%
\pgfsetbuttcap%
\pgfsetroundjoin%
\definecolor{currentfill}{rgb}{0.968500,0.673977,0.556649}%
\pgfsetfillcolor{currentfill}%
\pgfsetlinewidth{0.000000pt}%
\definecolor{currentstroke}{rgb}{0.000000,0.000000,0.000000}%
\pgfsetstrokecolor{currentstroke}%
\pgfsetdash{}{0pt}%
\pgfpathmoveto{\pgfqpoint{2.842004in}{4.813315in}}%
\pgfpathlineto{\pgfqpoint{2.850945in}{4.718728in}}%
\pgfpathlineto{\pgfqpoint{2.856179in}{4.907749in}}%
\pgfpathlineto{\pgfqpoint{2.889781in}{4.855453in}}%
\pgfpathlineto{\pgfqpoint{2.923765in}{4.768319in}}%
\pgfpathlineto{\pgfqpoint{2.913823in}{4.941826in}}%
\pgfpathlineto{\pgfqpoint{2.907071in}{4.860940in}}%
\pgfpathlineto{\pgfqpoint{2.873831in}{4.891980in}}%
\pgfpathlineto{\pgfqpoint{2.842004in}{4.813315in}}%
\pgfpathclose%
\pgfusepath{fill}%
\end{pgfscope}%
\begin{pgfscope}%
\pgfpathrectangle{\pgfqpoint{1.020000in}{0.880000in}}{\pgfqpoint{6.160000in}{6.160000in}}%
\pgfusepath{clip}%
\pgfsetbuttcap%
\pgfsetroundjoin%
\definecolor{currentfill}{rgb}{0.624703,0.748318,0.998719}%
\pgfsetfillcolor{currentfill}%
\pgfsetlinewidth{0.000000pt}%
\definecolor{currentstroke}{rgb}{0.000000,0.000000,0.000000}%
\pgfsetstrokecolor{currentstroke}%
\pgfsetdash{}{0pt}%
\pgfpathmoveto{\pgfqpoint{4.257060in}{3.896568in}}%
\pgfpathlineto{\pgfqpoint{4.266200in}{3.888135in}}%
\pgfpathlineto{\pgfqpoint{4.275365in}{3.887587in}}%
\pgfpathlineto{\pgfqpoint{4.307995in}{3.979503in}}%
\pgfpathlineto{\pgfqpoint{4.340174in}{3.756377in}}%
\pgfpathlineto{\pgfqpoint{4.331156in}{3.890094in}}%
\pgfpathlineto{\pgfqpoint{4.321895in}{3.851319in}}%
\pgfpathlineto{\pgfqpoint{4.289523in}{3.900656in}}%
\pgfpathlineto{\pgfqpoint{4.257060in}{3.896568in}}%
\pgfpathclose%
\pgfusepath{fill}%
\end{pgfscope}%
\begin{pgfscope}%
\pgfpathrectangle{\pgfqpoint{1.020000in}{0.880000in}}{\pgfqpoint{6.160000in}{6.160000in}}%
\pgfusepath{clip}%
\pgfsetbuttcap%
\pgfsetroundjoin%
\definecolor{currentfill}{rgb}{0.969522,0.700833,0.587508}%
\pgfsetfillcolor{currentfill}%
\pgfsetlinewidth{0.000000pt}%
\definecolor{currentstroke}{rgb}{0.000000,0.000000,0.000000}%
\pgfsetstrokecolor{currentstroke}%
\pgfsetdash{}{0pt}%
\pgfpathmoveto{\pgfqpoint{2.776117in}{4.823759in}}%
\pgfpathlineto{\pgfqpoint{2.784077in}{4.800208in}}%
\pgfpathlineto{\pgfqpoint{2.792291in}{4.759077in}}%
\pgfpathlineto{\pgfqpoint{2.825060in}{4.770724in}}%
\pgfpathlineto{\pgfqpoint{2.856179in}{4.907749in}}%
\pgfpathlineto{\pgfqpoint{2.850945in}{4.718728in}}%
\pgfpathlineto{\pgfqpoint{2.842004in}{4.813315in}}%
\pgfpathlineto{\pgfqpoint{2.809634in}{4.777001in}}%
\pgfpathlineto{\pgfqpoint{2.776117in}{4.823759in}}%
\pgfpathclose%
\pgfusepath{fill}%
\end{pgfscope}%
\begin{pgfscope}%
\pgfpathrectangle{\pgfqpoint{1.020000in}{0.880000in}}{\pgfqpoint{6.160000in}{6.160000in}}%
\pgfusepath{clip}%
\pgfsetbuttcap%
\pgfsetroundjoin%
\definecolor{currentfill}{rgb}{0.967544,0.730850,0.624685}%
\pgfsetfillcolor{currentfill}%
\pgfsetlinewidth{0.000000pt}%
\definecolor{currentstroke}{rgb}{0.000000,0.000000,0.000000}%
\pgfsetstrokecolor{currentstroke}%
\pgfsetdash{}{0pt}%
\pgfpathmoveto{\pgfqpoint{2.712220in}{4.691152in}}%
\pgfpathlineto{\pgfqpoint{2.721051in}{4.603376in}}%
\pgfpathlineto{\pgfqpoint{2.726140in}{4.776019in}}%
\pgfpathlineto{\pgfqpoint{2.759975in}{4.714399in}}%
\pgfpathlineto{\pgfqpoint{2.792291in}{4.759077in}}%
\pgfpathlineto{\pgfqpoint{2.784077in}{4.800208in}}%
\pgfpathlineto{\pgfqpoint{2.776117in}{4.823759in}}%
\pgfpathlineto{\pgfqpoint{2.743752in}{4.785919in}}%
\pgfpathlineto{\pgfqpoint{2.712220in}{4.691152in}}%
\pgfpathclose%
\pgfusepath{fill}%
\end{pgfscope}%
\begin{pgfscope}%
\pgfpathrectangle{\pgfqpoint{1.020000in}{0.880000in}}{\pgfqpoint{6.160000in}{6.160000in}}%
\pgfusepath{clip}%
\pgfsetbuttcap%
\pgfsetroundjoin%
\definecolor{currentfill}{rgb}{0.451739,0.588181,0.960201}%
\pgfsetfillcolor{currentfill}%
\pgfsetlinewidth{0.000000pt}%
\definecolor{currentstroke}{rgb}{0.000000,0.000000,0.000000}%
\pgfsetstrokecolor{currentstroke}%
\pgfsetdash{}{0pt}%
\pgfpathmoveto{\pgfqpoint{4.915056in}{3.686098in}}%
\pgfpathlineto{\pgfqpoint{4.924473in}{3.629385in}}%
\pgfpathlineto{\pgfqpoint{4.933831in}{3.563615in}}%
\pgfpathlineto{\pgfqpoint{4.966607in}{3.621765in}}%
\pgfpathlineto{\pgfqpoint{4.998131in}{3.523816in}}%
\pgfpathlineto{\pgfqpoint{4.987829in}{3.473920in}}%
\pgfpathlineto{\pgfqpoint{4.978059in}{3.487350in}}%
\pgfpathlineto{\pgfqpoint{4.947054in}{3.646441in}}%
\pgfpathlineto{\pgfqpoint{4.915056in}{3.686098in}}%
\pgfpathclose%
\pgfusepath{fill}%
\end{pgfscope}%
\begin{pgfscope}%
\pgfpathrectangle{\pgfqpoint{1.020000in}{0.880000in}}{\pgfqpoint{6.160000in}{6.160000in}}%
\pgfusepath{clip}%
\pgfsetbuttcap%
\pgfsetroundjoin%
\definecolor{currentfill}{rgb}{0.576051,0.708780,0.997755}%
\pgfsetfillcolor{currentfill}%
\pgfsetlinewidth{0.000000pt}%
\definecolor{currentstroke}{rgb}{0.000000,0.000000,0.000000}%
\pgfsetstrokecolor{currentstroke}%
\pgfsetdash{}{0pt}%
\pgfpathmoveto{\pgfqpoint{4.405201in}{3.834226in}}%
\pgfpathlineto{\pgfqpoint{4.414543in}{3.851370in}}%
\pgfpathlineto{\pgfqpoint{4.423625in}{3.750805in}}%
\pgfpathlineto{\pgfqpoint{4.456056in}{3.740158in}}%
\pgfpathlineto{\pgfqpoint{4.488283in}{3.671172in}}%
\pgfpathlineto{\pgfqpoint{4.479329in}{3.811601in}}%
\pgfpathlineto{\pgfqpoint{4.470243in}{3.908150in}}%
\pgfpathlineto{\pgfqpoint{4.437577in}{3.814909in}}%
\pgfpathlineto{\pgfqpoint{4.405201in}{3.834226in}}%
\pgfpathclose%
\pgfusepath{fill}%
\end{pgfscope}%
\begin{pgfscope}%
\pgfpathrectangle{\pgfqpoint{1.020000in}{0.880000in}}{\pgfqpoint{6.160000in}{6.160000in}}%
\pgfusepath{clip}%
\pgfsetbuttcap%
\pgfsetroundjoin%
\definecolor{currentfill}{rgb}{0.414801,0.546874,0.939088}%
\pgfsetfillcolor{currentfill}%
\pgfsetlinewidth{0.000000pt}%
\definecolor{currentstroke}{rgb}{0.000000,0.000000,0.000000}%
\pgfsetstrokecolor{currentstroke}%
\pgfsetdash{}{0pt}%
\pgfpathmoveto{\pgfqpoint{5.339342in}{3.425048in}}%
\pgfpathlineto{\pgfqpoint{5.350965in}{3.542046in}}%
\pgfpathlineto{\pgfqpoint{5.361116in}{3.530215in}}%
\pgfpathlineto{\pgfqpoint{5.392207in}{3.437680in}}%
\pgfpathlineto{\pgfqpoint{5.425705in}{3.545395in}}%
\pgfpathlineto{\pgfqpoint{5.416275in}{3.620319in}}%
\pgfpathlineto{\pgfqpoint{5.405440in}{3.579613in}}%
\pgfpathlineto{\pgfqpoint{5.371670in}{3.442988in}}%
\pgfpathlineto{\pgfqpoint{5.339342in}{3.425048in}}%
\pgfpathclose%
\pgfusepath{fill}%
\end{pgfscope}%
\begin{pgfscope}%
\pgfpathrectangle{\pgfqpoint{1.020000in}{0.880000in}}{\pgfqpoint{6.160000in}{6.160000in}}%
\pgfusepath{clip}%
\pgfsetbuttcap%
\pgfsetroundjoin%
\definecolor{currentfill}{rgb}{0.527132,0.664700,0.989065}%
\pgfsetfillcolor{currentfill}%
\pgfsetlinewidth{0.000000pt}%
\definecolor{currentstroke}{rgb}{0.000000,0.000000,0.000000}%
\pgfsetstrokecolor{currentstroke}%
\pgfsetdash{}{0pt}%
\pgfpathmoveto{\pgfqpoint{4.553798in}{3.872197in}}%
\pgfpathlineto{\pgfqpoint{4.562560in}{3.685608in}}%
\pgfpathlineto{\pgfqpoint{4.572485in}{3.803090in}}%
\pgfpathlineto{\pgfqpoint{4.604089in}{3.598516in}}%
\pgfpathlineto{\pgfqpoint{4.636912in}{3.696787in}}%
\pgfpathlineto{\pgfqpoint{4.627549in}{3.735101in}}%
\pgfpathlineto{\pgfqpoint{4.617615in}{3.638645in}}%
\pgfpathlineto{\pgfqpoint{4.585442in}{3.677629in}}%
\pgfpathlineto{\pgfqpoint{4.553798in}{3.872197in}}%
\pgfpathclose%
\pgfusepath{fill}%
\end{pgfscope}%
\begin{pgfscope}%
\pgfpathrectangle{\pgfqpoint{1.020000in}{0.880000in}}{\pgfqpoint{6.160000in}{6.160000in}}%
\pgfusepath{clip}%
\pgfsetbuttcap%
\pgfsetroundjoin%
\definecolor{currentfill}{rgb}{0.457046,0.594006,0.963029}%
\pgfsetfillcolor{currentfill}%
\pgfsetlinewidth{0.000000pt}%
\definecolor{currentstroke}{rgb}{0.000000,0.000000,0.000000}%
\pgfsetstrokecolor{currentstroke}%
\pgfsetdash{}{0pt}%
\pgfpathmoveto{\pgfqpoint{5.405440in}{3.579613in}}%
\pgfpathlineto{\pgfqpoint{5.416275in}{3.620319in}}%
\pgfpathlineto{\pgfqpoint{5.425705in}{3.545395in}}%
\pgfpathlineto{\pgfqpoint{5.457274in}{3.497615in}}%
\pgfpathlineto{\pgfqpoint{5.489156in}{3.477619in}}%
\pgfpathlineto{\pgfqpoint{5.482456in}{3.763809in}}%
\pgfpathlineto{\pgfqpoint{5.469194in}{3.542005in}}%
\pgfpathlineto{\pgfqpoint{5.438805in}{3.678398in}}%
\pgfpathlineto{\pgfqpoint{5.405440in}{3.579613in}}%
\pgfpathclose%
\pgfusepath{fill}%
\end{pgfscope}%
\begin{pgfscope}%
\pgfpathrectangle{\pgfqpoint{1.020000in}{0.880000in}}{\pgfqpoint{6.160000in}{6.160000in}}%
\pgfusepath{clip}%
\pgfsetbuttcap%
\pgfsetroundjoin%
\definecolor{currentfill}{rgb}{0.451739,0.588181,0.960201}%
\pgfsetfillcolor{currentfill}%
\pgfsetlinewidth{0.000000pt}%
\definecolor{currentstroke}{rgb}{0.000000,0.000000,0.000000}%
\pgfsetstrokecolor{currentstroke}%
\pgfsetdash{}{0pt}%
\pgfpathmoveto{\pgfqpoint{5.127492in}{3.564728in}}%
\pgfpathlineto{\pgfqpoint{5.137042in}{3.512491in}}%
\pgfpathlineto{\pgfqpoint{5.146922in}{3.493449in}}%
\pgfpathlineto{\pgfqpoint{5.180972in}{3.674437in}}%
\pgfpathlineto{\pgfqpoint{5.211520in}{3.505114in}}%
\pgfpathlineto{\pgfqpoint{5.201695in}{3.534442in}}%
\pgfpathlineto{\pgfqpoint{5.192126in}{3.587746in}}%
\pgfpathlineto{\pgfqpoint{5.161308in}{3.729368in}}%
\pgfpathlineto{\pgfqpoint{5.127492in}{3.564728in}}%
\pgfpathclose%
\pgfusepath{fill}%
\end{pgfscope}%
\begin{pgfscope}%
\pgfpathrectangle{\pgfqpoint{1.020000in}{0.880000in}}{\pgfqpoint{6.160000in}{6.160000in}}%
\pgfusepath{clip}%
\pgfsetbuttcap%
\pgfsetroundjoin%
\definecolor{currentfill}{rgb}{0.968533,0.715841,0.606097}%
\pgfsetfillcolor{currentfill}%
\pgfsetlinewidth{0.000000pt}%
\definecolor{currentstroke}{rgb}{0.000000,0.000000,0.000000}%
\pgfsetstrokecolor{currentstroke}%
\pgfsetdash{}{0pt}%
\pgfpathmoveto{\pgfqpoint{3.201585in}{4.827743in}}%
\pgfpathlineto{\pgfqpoint{3.210689in}{4.720665in}}%
\pgfpathlineto{\pgfqpoint{3.217571in}{4.846836in}}%
\pgfpathlineto{\pgfqpoint{3.250194in}{4.871887in}}%
\pgfpathlineto{\pgfqpoint{3.284658in}{4.689365in}}%
\pgfpathlineto{\pgfqpoint{3.275983in}{4.748777in}}%
\pgfpathlineto{\pgfqpoint{3.267849in}{4.748999in}}%
\pgfpathlineto{\pgfqpoint{3.235717in}{4.683866in}}%
\pgfpathlineto{\pgfqpoint{3.201585in}{4.827743in}}%
\pgfpathclose%
\pgfusepath{fill}%
\end{pgfscope}%
\begin{pgfscope}%
\pgfpathrectangle{\pgfqpoint{1.020000in}{0.880000in}}{\pgfqpoint{6.160000in}{6.160000in}}%
\pgfusepath{clip}%
\pgfsetbuttcap%
\pgfsetroundjoin%
\definecolor{currentfill}{rgb}{0.933221,0.815557,0.753151}%
\pgfsetfillcolor{currentfill}%
\pgfsetlinewidth{0.000000pt}%
\definecolor{currentstroke}{rgb}{0.000000,0.000000,0.000000}%
\pgfsetstrokecolor{currentstroke}%
\pgfsetdash{}{0pt}%
\pgfpathmoveto{\pgfqpoint{3.433962in}{4.451032in}}%
\pgfpathlineto{\pgfqpoint{3.441262in}{4.587222in}}%
\pgfpathlineto{\pgfqpoint{3.450100in}{4.516808in}}%
\pgfpathlineto{\pgfqpoint{3.482134in}{4.621559in}}%
\pgfpathlineto{\pgfqpoint{3.514994in}{4.608614in}}%
\pgfpathlineto{\pgfqpoint{3.507218in}{4.516704in}}%
\pgfpathlineto{\pgfqpoint{3.499821in}{4.373549in}}%
\pgfpathlineto{\pgfqpoint{3.465195in}{4.654533in}}%
\pgfpathlineto{\pgfqpoint{3.433962in}{4.451032in}}%
\pgfpathclose%
\pgfusepath{fill}%
\end{pgfscope}%
\begin{pgfscope}%
\pgfpathrectangle{\pgfqpoint{1.020000in}{0.880000in}}{\pgfqpoint{6.160000in}{6.160000in}}%
\pgfusepath{clip}%
\pgfsetbuttcap%
\pgfsetroundjoin%
\definecolor{currentfill}{rgb}{0.758539,0.832787,0.958408}%
\pgfsetfillcolor{currentfill}%
\pgfsetlinewidth{0.000000pt}%
\definecolor{currentstroke}{rgb}{0.000000,0.000000,0.000000}%
\pgfsetstrokecolor{currentstroke}%
\pgfsetdash{}{0pt}%
\pgfpathmoveto{\pgfqpoint{3.960867in}{4.097449in}}%
\pgfpathlineto{\pgfqpoint{3.969302in}{4.290090in}}%
\pgfpathlineto{\pgfqpoint{3.978667in}{4.051148in}}%
\pgfpathlineto{\pgfqpoint{4.011153in}{4.114967in}}%
\pgfpathlineto{\pgfqpoint{4.043680in}{4.166392in}}%
\pgfpathlineto{\pgfqpoint{4.034937in}{4.039087in}}%
\pgfpathlineto{\pgfqpoint{4.025912in}{4.118978in}}%
\pgfpathlineto{\pgfqpoint{3.993446in}{4.081370in}}%
\pgfpathlineto{\pgfqpoint{3.960867in}{4.097449in}}%
\pgfpathclose%
\pgfusepath{fill}%
\end{pgfscope}%
\begin{pgfscope}%
\pgfpathrectangle{\pgfqpoint{1.020000in}{0.880000in}}{\pgfqpoint{6.160000in}{6.160000in}}%
\pgfusepath{clip}%
\pgfsetbuttcap%
\pgfsetroundjoin%
\definecolor{currentfill}{rgb}{0.969289,0.684982,0.568975}%
\pgfsetfillcolor{currentfill}%
\pgfsetlinewidth{0.000000pt}%
\definecolor{currentstroke}{rgb}{0.000000,0.000000,0.000000}%
\pgfsetstrokecolor{currentstroke}%
\pgfsetdash{}{0pt}%
\pgfpathmoveto{\pgfqpoint{2.990294in}{4.696879in}}%
\pgfpathlineto{\pgfqpoint{2.995036in}{4.962047in}}%
\pgfpathlineto{\pgfqpoint{3.003648in}{4.899276in}}%
\pgfpathlineto{\pgfqpoint{3.036657in}{4.891462in}}%
\pgfpathlineto{\pgfqpoint{3.071390in}{4.724455in}}%
\pgfpathlineto{\pgfqpoint{3.062181in}{4.840556in}}%
\pgfpathlineto{\pgfqpoint{3.055377in}{4.740374in}}%
\pgfpathlineto{\pgfqpoint{3.021863in}{4.803703in}}%
\pgfpathlineto{\pgfqpoint{2.990294in}{4.696879in}}%
\pgfpathclose%
\pgfusepath{fill}%
\end{pgfscope}%
\begin{pgfscope}%
\pgfpathrectangle{\pgfqpoint{1.020000in}{0.880000in}}{\pgfqpoint{6.160000in}{6.160000in}}%
\pgfusepath{clip}%
\pgfsetbuttcap%
\pgfsetroundjoin%
\definecolor{currentfill}{rgb}{0.677823,0.786546,0.991005}%
\pgfsetfillcolor{currentfill}%
\pgfsetlinewidth{0.000000pt}%
\definecolor{currentstroke}{rgb}{0.000000,0.000000,0.000000}%
\pgfsetstrokecolor{currentstroke}%
\pgfsetdash{}{0pt}%
\pgfpathmoveto{\pgfqpoint{4.043680in}{4.166392in}}%
\pgfpathlineto{\pgfqpoint{4.052650in}{4.149365in}}%
\pgfpathlineto{\pgfqpoint{4.061831in}{3.970646in}}%
\pgfpathlineto{\pgfqpoint{4.094584in}{3.790729in}}%
\pgfpathlineto{\pgfqpoint{4.126934in}{4.088962in}}%
\pgfpathlineto{\pgfqpoint{4.118111in}{3.783298in}}%
\pgfpathlineto{\pgfqpoint{4.109065in}{3.878058in}}%
\pgfpathlineto{\pgfqpoint{4.076490in}{3.964005in}}%
\pgfpathlineto{\pgfqpoint{4.043680in}{4.166392in}}%
\pgfpathclose%
\pgfusepath{fill}%
\end{pgfscope}%
\begin{pgfscope}%
\pgfpathrectangle{\pgfqpoint{1.020000in}{0.880000in}}{\pgfqpoint{6.160000in}{6.160000in}}%
\pgfusepath{clip}%
\pgfsetbuttcap%
\pgfsetroundjoin%
\definecolor{currentfill}{rgb}{0.435815,0.570707,0.951717}%
\pgfsetfillcolor{currentfill}%
\pgfsetlinewidth{0.000000pt}%
\definecolor{currentstroke}{rgb}{0.000000,0.000000,0.000000}%
\pgfsetstrokecolor{currentstroke}%
\pgfsetdash{}{0pt}%
\pgfpathmoveto{\pgfqpoint{5.062806in}{3.540917in}}%
\pgfpathlineto{\pgfqpoint{5.073299in}{3.598088in}}%
\pgfpathlineto{\pgfqpoint{5.082297in}{3.485347in}}%
\pgfpathlineto{\pgfqpoint{5.115869in}{3.623872in}}%
\pgfpathlineto{\pgfqpoint{5.146922in}{3.493449in}}%
\pgfpathlineto{\pgfqpoint{5.137042in}{3.512491in}}%
\pgfpathlineto{\pgfqpoint{5.127492in}{3.564728in}}%
\pgfpathlineto{\pgfqpoint{5.094840in}{3.518332in}}%
\pgfpathlineto{\pgfqpoint{5.062806in}{3.540917in}}%
\pgfpathclose%
\pgfusepath{fill}%
\end{pgfscope}%
\begin{pgfscope}%
\pgfpathrectangle{\pgfqpoint{1.020000in}{0.880000in}}{\pgfqpoint{6.160000in}{6.160000in}}%
\pgfusepath{clip}%
\pgfsetbuttcap%
\pgfsetroundjoin%
\definecolor{currentfill}{rgb}{0.581486,0.713451,0.998314}%
\pgfsetfillcolor{currentfill}%
\pgfsetlinewidth{0.000000pt}%
\definecolor{currentstroke}{rgb}{0.000000,0.000000,0.000000}%
\pgfsetstrokecolor{currentstroke}%
\pgfsetdash{}{0pt}%
\pgfpathmoveto{\pgfqpoint{4.340174in}{3.756377in}}%
\pgfpathlineto{\pgfqpoint{4.349727in}{3.949322in}}%
\pgfpathlineto{\pgfqpoint{4.358585in}{3.713159in}}%
\pgfpathlineto{\pgfqpoint{4.391106in}{3.731090in}}%
\pgfpathlineto{\pgfqpoint{4.423625in}{3.750805in}}%
\pgfpathlineto{\pgfqpoint{4.414543in}{3.851370in}}%
\pgfpathlineto{\pgfqpoint{4.405201in}{3.834226in}}%
\pgfpathlineto{\pgfqpoint{4.372699in}{3.802932in}}%
\pgfpathlineto{\pgfqpoint{4.340174in}{3.756377in}}%
\pgfpathclose%
\pgfusepath{fill}%
\end{pgfscope}%
\begin{pgfscope}%
\pgfpathrectangle{\pgfqpoint{1.020000in}{0.880000in}}{\pgfqpoint{6.160000in}{6.160000in}}%
\pgfusepath{clip}%
\pgfsetbuttcap%
\pgfsetroundjoin%
\definecolor{currentfill}{rgb}{0.430507,0.564883,0.948889}%
\pgfsetfillcolor{currentfill}%
\pgfsetlinewidth{0.000000pt}%
\definecolor{currentstroke}{rgb}{0.000000,0.000000,0.000000}%
\pgfsetstrokecolor{currentstroke}%
\pgfsetdash{}{0pt}%
\pgfpathmoveto{\pgfqpoint{5.554582in}{3.565850in}}%
\pgfpathlineto{\pgfqpoint{5.565687in}{3.609947in}}%
\pgfpathlineto{\pgfqpoint{5.574596in}{3.493842in}}%
\pgfpathlineto{\pgfqpoint{5.607771in}{3.566120in}}%
\pgfpathlineto{\pgfqpoint{5.639296in}{3.523111in}}%
\pgfpathlineto{\pgfqpoint{5.626699in}{3.383042in}}%
\pgfpathlineto{\pgfqpoint{5.619287in}{3.601018in}}%
\pgfpathlineto{\pgfqpoint{5.586642in}{3.562351in}}%
\pgfpathlineto{\pgfqpoint{5.554582in}{3.565850in}}%
\pgfpathclose%
\pgfusepath{fill}%
\end{pgfscope}%
\begin{pgfscope}%
\pgfpathrectangle{\pgfqpoint{1.020000in}{0.880000in}}{\pgfqpoint{6.160000in}{6.160000in}}%
\pgfusepath{clip}%
\pgfsetbuttcap%
\pgfsetroundjoin%
\definecolor{currentfill}{rgb}{0.516260,0.654498,0.986407}%
\pgfsetfillcolor{currentfill}%
\pgfsetlinewidth{0.000000pt}%
\definecolor{currentstroke}{rgb}{0.000000,0.000000,0.000000}%
\pgfsetstrokecolor{currentstroke}%
\pgfsetdash{}{0pt}%
\pgfpathmoveto{\pgfqpoint{4.701455in}{3.654164in}}%
\pgfpathlineto{\pgfqpoint{4.710614in}{3.568678in}}%
\pgfpathlineto{\pgfqpoint{4.721390in}{3.784083in}}%
\pgfpathlineto{\pgfqpoint{4.752635in}{3.573264in}}%
\pgfpathlineto{\pgfqpoint{4.785285in}{3.622006in}}%
\pgfpathlineto{\pgfqpoint{4.775849in}{3.664580in}}%
\pgfpathlineto{\pgfqpoint{4.766677in}{3.750303in}}%
\pgfpathlineto{\pgfqpoint{4.734833in}{3.842916in}}%
\pgfpathlineto{\pgfqpoint{4.701455in}{3.654164in}}%
\pgfpathclose%
\pgfusepath{fill}%
\end{pgfscope}%
\begin{pgfscope}%
\pgfpathrectangle{\pgfqpoint{1.020000in}{0.880000in}}{\pgfqpoint{6.160000in}{6.160000in}}%
\pgfusepath{clip}%
\pgfsetbuttcap%
\pgfsetroundjoin%
\definecolor{currentfill}{rgb}{0.425199,0.559058,0.946061}%
\pgfsetfillcolor{currentfill}%
\pgfsetlinewidth{0.000000pt}%
\definecolor{currentstroke}{rgb}{0.000000,0.000000,0.000000}%
\pgfsetstrokecolor{currentstroke}%
\pgfsetdash{}{0pt}%
\pgfpathmoveto{\pgfqpoint{5.276607in}{3.566982in}}%
\pgfpathlineto{\pgfqpoint{5.287582in}{3.636246in}}%
\pgfpathlineto{\pgfqpoint{5.295131in}{3.394343in}}%
\pgfpathlineto{\pgfqpoint{5.328593in}{3.504434in}}%
\pgfpathlineto{\pgfqpoint{5.361116in}{3.530215in}}%
\pgfpathlineto{\pgfqpoint{5.350965in}{3.542046in}}%
\pgfpathlineto{\pgfqpoint{5.339342in}{3.425048in}}%
\pgfpathlineto{\pgfqpoint{5.309206in}{3.602699in}}%
\pgfpathlineto{\pgfqpoint{5.276607in}{3.566982in}}%
\pgfpathclose%
\pgfusepath{fill}%
\end{pgfscope}%
\begin{pgfscope}%
\pgfpathrectangle{\pgfqpoint{1.020000in}{0.880000in}}{\pgfqpoint{6.160000in}{6.160000in}}%
\pgfusepath{clip}%
\pgfsetbuttcap%
\pgfsetroundjoin%
\definecolor{currentfill}{rgb}{0.818056,0.855590,0.914638}%
\pgfsetfillcolor{currentfill}%
\pgfsetlinewidth{0.000000pt}%
\definecolor{currentstroke}{rgb}{0.000000,0.000000,0.000000}%
\pgfsetstrokecolor{currentstroke}%
\pgfsetdash{}{0pt}%
\pgfpathmoveto{\pgfqpoint{3.812296in}{4.316255in}}%
\pgfpathlineto{\pgfqpoint{3.820908in}{4.341486in}}%
\pgfpathlineto{\pgfqpoint{3.829990in}{4.239059in}}%
\pgfpathlineto{\pgfqpoint{3.862837in}{4.182929in}}%
\pgfpathlineto{\pgfqpoint{3.895173in}{4.286740in}}%
\pgfpathlineto{\pgfqpoint{3.886592in}{4.219322in}}%
\pgfpathlineto{\pgfqpoint{3.878000in}{4.164933in}}%
\pgfpathlineto{\pgfqpoint{3.845367in}{4.185250in}}%
\pgfpathlineto{\pgfqpoint{3.812296in}{4.316255in}}%
\pgfpathclose%
\pgfusepath{fill}%
\end{pgfscope}%
\begin{pgfscope}%
\pgfpathrectangle{\pgfqpoint{1.020000in}{0.880000in}}{\pgfqpoint{6.160000in}{6.160000in}}%
\pgfusepath{clip}%
\pgfsetbuttcap%
\pgfsetroundjoin%
\definecolor{currentfill}{rgb}{0.968863,0.710838,0.599901}%
\pgfsetfillcolor{currentfill}%
\pgfsetlinewidth{0.000000pt}%
\definecolor{currentstroke}{rgb}{0.000000,0.000000,0.000000}%
\pgfsetstrokecolor{currentstroke}%
\pgfsetdash{}{0pt}%
\pgfpathmoveto{\pgfqpoint{2.645375in}{4.759284in}}%
\pgfpathlineto{\pgfqpoint{2.653801in}{4.698227in}}%
\pgfpathlineto{\pgfqpoint{2.658505in}{4.885270in}}%
\pgfpathlineto{\pgfqpoint{2.693195in}{4.773205in}}%
\pgfpathlineto{\pgfqpoint{2.726140in}{4.776019in}}%
\pgfpathlineto{\pgfqpoint{2.721051in}{4.603376in}}%
\pgfpathlineto{\pgfqpoint{2.712220in}{4.691152in}}%
\pgfpathlineto{\pgfqpoint{2.674989in}{4.983046in}}%
\pgfpathlineto{\pgfqpoint{2.645375in}{4.759284in}}%
\pgfpathclose%
\pgfusepath{fill}%
\end{pgfscope}%
\begin{pgfscope}%
\pgfpathrectangle{\pgfqpoint{1.020000in}{0.880000in}}{\pgfqpoint{6.160000in}{6.160000in}}%
\pgfusepath{clip}%
\pgfsetbuttcap%
\pgfsetroundjoin%
\definecolor{currentfill}{rgb}{0.646113,0.764436,0.996868}%
\pgfsetfillcolor{currentfill}%
\pgfsetlinewidth{0.000000pt}%
\definecolor{currentstroke}{rgb}{0.000000,0.000000,0.000000}%
\pgfsetstrokecolor{currentstroke}%
\pgfsetdash{}{0pt}%
\pgfpathmoveto{\pgfqpoint{4.192087in}{3.881515in}}%
\pgfpathlineto{\pgfqpoint{4.201184in}{3.952918in}}%
\pgfpathlineto{\pgfqpoint{4.210245in}{3.777902in}}%
\pgfpathlineto{\pgfqpoint{4.242909in}{4.022953in}}%
\pgfpathlineto{\pgfqpoint{4.275365in}{3.887587in}}%
\pgfpathlineto{\pgfqpoint{4.266200in}{3.888135in}}%
\pgfpathlineto{\pgfqpoint{4.257060in}{3.896568in}}%
\pgfpathlineto{\pgfqpoint{4.224632in}{4.014810in}}%
\pgfpathlineto{\pgfqpoint{4.192087in}{3.881515in}}%
\pgfpathclose%
\pgfusepath{fill}%
\end{pgfscope}%
\begin{pgfscope}%
\pgfpathrectangle{\pgfqpoint{1.020000in}{0.880000in}}{\pgfqpoint{6.160000in}{6.160000in}}%
\pgfusepath{clip}%
\pgfsetbuttcap%
\pgfsetroundjoin%
\definecolor{currentfill}{rgb}{0.473070,0.611077,0.970634}%
\pgfsetfillcolor{currentfill}%
\pgfsetlinewidth{0.000000pt}%
\definecolor{currentstroke}{rgb}{0.000000,0.000000,0.000000}%
\pgfsetstrokecolor{currentstroke}%
\pgfsetdash{}{0pt}%
\pgfpathmoveto{\pgfqpoint{4.849769in}{3.591270in}}%
\pgfpathlineto{\pgfqpoint{4.859809in}{3.631879in}}%
\pgfpathlineto{\pgfqpoint{4.869240in}{3.579505in}}%
\pgfpathlineto{\pgfqpoint{4.901015in}{3.497354in}}%
\pgfpathlineto{\pgfqpoint{4.933831in}{3.563615in}}%
\pgfpathlineto{\pgfqpoint{4.924473in}{3.629385in}}%
\pgfpathlineto{\pgfqpoint{4.915056in}{3.686098in}}%
\pgfpathlineto{\pgfqpoint{4.882946in}{3.715938in}}%
\pgfpathlineto{\pgfqpoint{4.849769in}{3.591270in}}%
\pgfpathclose%
\pgfusepath{fill}%
\end{pgfscope}%
\begin{pgfscope}%
\pgfpathrectangle{\pgfqpoint{1.020000in}{0.880000in}}{\pgfqpoint{6.160000in}{6.160000in}}%
\pgfusepath{clip}%
\pgfsetbuttcap%
\pgfsetroundjoin%
\definecolor{currentfill}{rgb}{0.554312,0.690097,0.995516}%
\pgfsetfillcolor{currentfill}%
\pgfsetlinewidth{0.000000pt}%
\definecolor{currentstroke}{rgb}{0.000000,0.000000,0.000000}%
\pgfsetstrokecolor{currentstroke}%
\pgfsetdash{}{0pt}%
\pgfpathmoveto{\pgfqpoint{4.488283in}{3.671172in}}%
\pgfpathlineto{\pgfqpoint{4.497823in}{3.720067in}}%
\pgfpathlineto{\pgfqpoint{4.507199in}{3.709017in}}%
\pgfpathlineto{\pgfqpoint{4.539639in}{3.700450in}}%
\pgfpathlineto{\pgfqpoint{4.572485in}{3.803090in}}%
\pgfpathlineto{\pgfqpoint{4.562560in}{3.685608in}}%
\pgfpathlineto{\pgfqpoint{4.553798in}{3.872197in}}%
\pgfpathlineto{\pgfqpoint{4.521210in}{3.828923in}}%
\pgfpathlineto{\pgfqpoint{4.488283in}{3.671172in}}%
\pgfpathclose%
\pgfusepath{fill}%
\end{pgfscope}%
\begin{pgfscope}%
\pgfpathrectangle{\pgfqpoint{1.020000in}{0.880000in}}{\pgfqpoint{6.160000in}{6.160000in}}%
\pgfusepath{clip}%
\pgfsetbuttcap%
\pgfsetroundjoin%
\definecolor{currentfill}{rgb}{0.430507,0.564883,0.948889}%
\pgfsetfillcolor{currentfill}%
\pgfsetlinewidth{0.000000pt}%
\definecolor{currentstroke}{rgb}{0.000000,0.000000,0.000000}%
\pgfsetstrokecolor{currentstroke}%
\pgfsetdash{}{0pt}%
\pgfpathmoveto{\pgfqpoint{4.998131in}{3.523816in}}%
\pgfpathlineto{\pgfqpoint{5.007301in}{3.432496in}}%
\pgfpathlineto{\pgfqpoint{5.017624in}{3.479264in}}%
\pgfpathlineto{\pgfqpoint{5.050434in}{3.535980in}}%
\pgfpathlineto{\pgfqpoint{5.082297in}{3.485347in}}%
\pgfpathlineto{\pgfqpoint{5.073299in}{3.598088in}}%
\pgfpathlineto{\pgfqpoint{5.062806in}{3.540917in}}%
\pgfpathlineto{\pgfqpoint{5.031590in}{3.664123in}}%
\pgfpathlineto{\pgfqpoint{4.998131in}{3.523816in}}%
\pgfpathclose%
\pgfusepath{fill}%
\end{pgfscope}%
\begin{pgfscope}%
\pgfpathrectangle{\pgfqpoint{1.020000in}{0.880000in}}{\pgfqpoint{6.160000in}{6.160000in}}%
\pgfusepath{clip}%
\pgfsetbuttcap%
\pgfsetroundjoin%
\definecolor{currentfill}{rgb}{0.394042,0.522413,0.924916}%
\pgfsetfillcolor{currentfill}%
\pgfsetlinewidth{0.000000pt}%
\definecolor{currentstroke}{rgb}{0.000000,0.000000,0.000000}%
\pgfsetstrokecolor{currentstroke}%
\pgfsetdash{}{0pt}%
\pgfpathmoveto{\pgfqpoint{5.425705in}{3.545395in}}%
\pgfpathlineto{\pgfqpoint{5.434553in}{3.423520in}}%
\pgfpathlineto{\pgfqpoint{5.445093in}{3.436002in}}%
\pgfpathlineto{\pgfqpoint{5.477048in}{3.416910in}}%
\pgfpathlineto{\pgfqpoint{5.509414in}{3.431088in}}%
\pgfpathlineto{\pgfqpoint{5.499866in}{3.499122in}}%
\pgfpathlineto{\pgfqpoint{5.489156in}{3.477619in}}%
\pgfpathlineto{\pgfqpoint{5.457274in}{3.497615in}}%
\pgfpathlineto{\pgfqpoint{5.425705in}{3.545395in}}%
\pgfpathclose%
\pgfusepath{fill}%
\end{pgfscope}%
\begin{pgfscope}%
\pgfpathrectangle{\pgfqpoint{1.020000in}{0.880000in}}{\pgfqpoint{6.160000in}{6.160000in}}%
\pgfusepath{clip}%
\pgfsetbuttcap%
\pgfsetroundjoin%
\definecolor{currentfill}{rgb}{0.871493,0.862309,0.857016}%
\pgfsetfillcolor{currentfill}%
\pgfsetlinewidth{0.000000pt}%
\definecolor{currentstroke}{rgb}{0.000000,0.000000,0.000000}%
\pgfsetstrokecolor{currentstroke}%
\pgfsetdash{}{0pt}%
\pgfpathmoveto{\pgfqpoint{3.663892in}{4.429909in}}%
\pgfpathlineto{\pgfqpoint{3.672730in}{4.373452in}}%
\pgfpathlineto{\pgfqpoint{3.681477in}{4.337037in}}%
\pgfpathlineto{\pgfqpoint{3.714439in}{4.283506in}}%
\pgfpathlineto{\pgfqpoint{3.746260in}{4.481760in}}%
\pgfpathlineto{\pgfqpoint{3.738447in}{4.294935in}}%
\pgfpathlineto{\pgfqpoint{3.729746in}{4.316706in}}%
\pgfpathlineto{\pgfqpoint{3.696872in}{4.367962in}}%
\pgfpathlineto{\pgfqpoint{3.663892in}{4.429909in}}%
\pgfpathclose%
\pgfusepath{fill}%
\end{pgfscope}%
\begin{pgfscope}%
\pgfpathrectangle{\pgfqpoint{1.020000in}{0.880000in}}{\pgfqpoint{6.160000in}{6.160000in}}%
\pgfusepath{clip}%
\pgfsetbuttcap%
\pgfsetroundjoin%
\definecolor{currentfill}{rgb}{0.430507,0.564883,0.948889}%
\pgfsetfillcolor{currentfill}%
\pgfsetlinewidth{0.000000pt}%
\definecolor{currentstroke}{rgb}{0.000000,0.000000,0.000000}%
\pgfsetstrokecolor{currentstroke}%
\pgfsetdash{}{0pt}%
\pgfpathmoveto{\pgfqpoint{5.489156in}{3.477619in}}%
\pgfpathlineto{\pgfqpoint{5.499866in}{3.499122in}}%
\pgfpathlineto{\pgfqpoint{5.509414in}{3.431088in}}%
\pgfpathlineto{\pgfqpoint{5.544521in}{3.647743in}}%
\pgfpathlineto{\pgfqpoint{5.574596in}{3.493842in}}%
\pgfpathlineto{\pgfqpoint{5.565687in}{3.609947in}}%
\pgfpathlineto{\pgfqpoint{5.554582in}{3.565850in}}%
\pgfpathlineto{\pgfqpoint{5.521798in}{3.516809in}}%
\pgfpathlineto{\pgfqpoint{5.489156in}{3.477619in}}%
\pgfpathclose%
\pgfusepath{fill}%
\end{pgfscope}%
\begin{pgfscope}%
\pgfpathrectangle{\pgfqpoint{1.020000in}{0.880000in}}{\pgfqpoint{6.160000in}{6.160000in}}%
\pgfusepath{clip}%
\pgfsetbuttcap%
\pgfsetroundjoin%
\definecolor{currentfill}{rgb}{0.505423,0.643995,0.983157}%
\pgfsetfillcolor{currentfill}%
\pgfsetlinewidth{0.000000pt}%
\definecolor{currentstroke}{rgb}{0.000000,0.000000,0.000000}%
\pgfsetstrokecolor{currentstroke}%
\pgfsetdash{}{0pt}%
\pgfpathmoveto{\pgfqpoint{4.636912in}{3.696787in}}%
\pgfpathlineto{\pgfqpoint{4.646434in}{3.690490in}}%
\pgfpathlineto{\pgfqpoint{4.655381in}{3.558078in}}%
\pgfpathlineto{\pgfqpoint{4.688481in}{3.695350in}}%
\pgfpathlineto{\pgfqpoint{4.721390in}{3.784083in}}%
\pgfpathlineto{\pgfqpoint{4.710614in}{3.568678in}}%
\pgfpathlineto{\pgfqpoint{4.701455in}{3.654164in}}%
\pgfpathlineto{\pgfqpoint{4.669261in}{3.688183in}}%
\pgfpathlineto{\pgfqpoint{4.636912in}{3.696787in}}%
\pgfpathclose%
\pgfusepath{fill}%
\end{pgfscope}%
\begin{pgfscope}%
\pgfpathrectangle{\pgfqpoint{1.020000in}{0.880000in}}{\pgfqpoint{6.160000in}{6.160000in}}%
\pgfusepath{clip}%
\pgfsetbuttcap%
\pgfsetroundjoin%
\definecolor{currentfill}{rgb}{0.969683,0.690484,0.575138}%
\pgfsetfillcolor{currentfill}%
\pgfsetlinewidth{0.000000pt}%
\definecolor{currentstroke}{rgb}{0.000000,0.000000,0.000000}%
\pgfsetstrokecolor{currentstroke}%
\pgfsetdash{}{0pt}%
\pgfpathmoveto{\pgfqpoint{3.135789in}{4.844077in}}%
\pgfpathlineto{\pgfqpoint{3.144715in}{4.755112in}}%
\pgfpathlineto{\pgfqpoint{3.152845in}{4.744694in}}%
\pgfpathlineto{\pgfqpoint{3.184252in}{4.892062in}}%
\pgfpathlineto{\pgfqpoint{3.217571in}{4.846836in}}%
\pgfpathlineto{\pgfqpoint{3.210689in}{4.720665in}}%
\pgfpathlineto{\pgfqpoint{3.201585in}{4.827743in}}%
\pgfpathlineto{\pgfqpoint{3.169271in}{4.778798in}}%
\pgfpathlineto{\pgfqpoint{3.135789in}{4.844077in}}%
\pgfpathclose%
\pgfusepath{fill}%
\end{pgfscope}%
\begin{pgfscope}%
\pgfpathrectangle{\pgfqpoint{1.020000in}{0.880000in}}{\pgfqpoint{6.160000in}{6.160000in}}%
\pgfusepath{clip}%
\pgfsetbuttcap%
\pgfsetroundjoin%
\definecolor{currentfill}{rgb}{0.968500,0.673977,0.556649}%
\pgfsetfillcolor{currentfill}%
\pgfsetlinewidth{0.000000pt}%
\definecolor{currentstroke}{rgb}{0.000000,0.000000,0.000000}%
\pgfsetstrokecolor{currentstroke}%
\pgfsetdash{}{0pt}%
\pgfpathmoveto{\pgfqpoint{2.923765in}{4.768319in}}%
\pgfpathlineto{\pgfqpoint{2.931308in}{4.788123in}}%
\pgfpathlineto{\pgfqpoint{2.937638in}{4.908320in}}%
\pgfpathlineto{\pgfqpoint{2.970382in}{4.926579in}}%
\pgfpathlineto{\pgfqpoint{3.003648in}{4.899276in}}%
\pgfpathlineto{\pgfqpoint{2.995036in}{4.962047in}}%
\pgfpathlineto{\pgfqpoint{2.990294in}{4.696879in}}%
\pgfpathlineto{\pgfqpoint{2.957013in}{4.735736in}}%
\pgfpathlineto{\pgfqpoint{2.923765in}{4.768319in}}%
\pgfpathclose%
\pgfusepath{fill}%
\end{pgfscope}%
\begin{pgfscope}%
\pgfpathrectangle{\pgfqpoint{1.020000in}{0.880000in}}{\pgfqpoint{6.160000in}{6.160000in}}%
\pgfusepath{clip}%
\pgfsetbuttcap%
\pgfsetroundjoin%
\definecolor{currentfill}{rgb}{0.640828,0.760752,0.997846}%
\pgfsetfillcolor{currentfill}%
\pgfsetlinewidth{0.000000pt}%
\definecolor{currentstroke}{rgb}{0.000000,0.000000,0.000000}%
\pgfsetstrokecolor{currentstroke}%
\pgfsetdash{}{0pt}%
\pgfpathmoveto{\pgfqpoint{4.126934in}{4.088962in}}%
\pgfpathlineto{\pgfqpoint{4.136061in}{3.905365in}}%
\pgfpathlineto{\pgfqpoint{4.145099in}{3.917455in}}%
\pgfpathlineto{\pgfqpoint{4.177700in}{3.932884in}}%
\pgfpathlineto{\pgfqpoint{4.210245in}{3.777902in}}%
\pgfpathlineto{\pgfqpoint{4.201184in}{3.952918in}}%
\pgfpathlineto{\pgfqpoint{4.192087in}{3.881515in}}%
\pgfpathlineto{\pgfqpoint{4.159617in}{3.717655in}}%
\pgfpathlineto{\pgfqpoint{4.126934in}{4.088962in}}%
\pgfpathclose%
\pgfusepath{fill}%
\end{pgfscope}%
\begin{pgfscope}%
\pgfpathrectangle{\pgfqpoint{1.020000in}{0.880000in}}{\pgfqpoint{6.160000in}{6.160000in}}%
\pgfusepath{clip}%
\pgfsetbuttcap%
\pgfsetroundjoin%
\definecolor{currentfill}{rgb}{0.538004,0.674902,0.991722}%
\pgfsetfillcolor{currentfill}%
\pgfsetlinewidth{0.000000pt}%
\definecolor{currentstroke}{rgb}{0.000000,0.000000,0.000000}%
\pgfsetstrokecolor{currentstroke}%
\pgfsetdash{}{0pt}%
\pgfpathmoveto{\pgfqpoint{4.423625in}{3.750805in}}%
\pgfpathlineto{\pgfqpoint{4.433037in}{3.782679in}}%
\pgfpathlineto{\pgfqpoint{4.442236in}{3.722048in}}%
\pgfpathlineto{\pgfqpoint{4.474440in}{3.616018in}}%
\pgfpathlineto{\pgfqpoint{4.507199in}{3.709017in}}%
\pgfpathlineto{\pgfqpoint{4.497823in}{3.720067in}}%
\pgfpathlineto{\pgfqpoint{4.488283in}{3.671172in}}%
\pgfpathlineto{\pgfqpoint{4.456056in}{3.740158in}}%
\pgfpathlineto{\pgfqpoint{4.423625in}{3.750805in}}%
\pgfpathclose%
\pgfusepath{fill}%
\end{pgfscope}%
\begin{pgfscope}%
\pgfpathrectangle{\pgfqpoint{1.020000in}{0.880000in}}{\pgfqpoint{6.160000in}{6.160000in}}%
\pgfusepath{clip}%
\pgfsetbuttcap%
\pgfsetroundjoin%
\definecolor{currentfill}{rgb}{0.383662,0.510183,0.917831}%
\pgfsetfillcolor{currentfill}%
\pgfsetlinewidth{0.000000pt}%
\definecolor{currentstroke}{rgb}{0.000000,0.000000,0.000000}%
\pgfsetstrokecolor{currentstroke}%
\pgfsetdash{}{0pt}%
\pgfpathmoveto{\pgfqpoint{5.639296in}{3.523111in}}%
\pgfpathlineto{\pgfqpoint{5.649888in}{3.522818in}}%
\pgfpathlineto{\pgfqpoint{5.659028in}{3.422627in}}%
\pgfpathlineto{\pgfqpoint{5.691637in}{3.453279in}}%
\pgfpathlineto{\pgfqpoint{5.678226in}{3.268258in}}%
\pgfpathlineto{\pgfqpoint{5.670585in}{3.467275in}}%
\pgfpathlineto{\pgfqpoint{5.639296in}{3.523111in}}%
\pgfpathclose%
\pgfusepath{fill}%
\end{pgfscope}%
\begin{pgfscope}%
\pgfpathrectangle{\pgfqpoint{1.020000in}{0.880000in}}{\pgfqpoint{6.160000in}{6.160000in}}%
\pgfusepath{clip}%
\pgfsetbuttcap%
\pgfsetroundjoin%
\definecolor{currentfill}{rgb}{0.968533,0.715841,0.606097}%
\pgfsetfillcolor{currentfill}%
\pgfsetlinewidth{0.000000pt}%
\definecolor{currentstroke}{rgb}{0.000000,0.000000,0.000000}%
\pgfsetstrokecolor{currentstroke}%
\pgfsetdash{}{0pt}%
\pgfpathmoveto{\pgfqpoint{2.578740in}{4.803716in}}%
\pgfpathlineto{\pgfqpoint{2.586828in}{4.762883in}}%
\pgfpathlineto{\pgfqpoint{2.592450in}{4.879900in}}%
\pgfpathlineto{\pgfqpoint{2.629654in}{4.611124in}}%
\pgfpathlineto{\pgfqpoint{2.658505in}{4.885270in}}%
\pgfpathlineto{\pgfqpoint{2.653801in}{4.698227in}}%
\pgfpathlineto{\pgfqpoint{2.645375in}{4.759284in}}%
\pgfpathlineto{\pgfqpoint{2.613654in}{4.679554in}}%
\pgfpathlineto{\pgfqpoint{2.578740in}{4.803716in}}%
\pgfpathclose%
\pgfusepath{fill}%
\end{pgfscope}%
\begin{pgfscope}%
\pgfpathrectangle{\pgfqpoint{1.020000in}{0.880000in}}{\pgfqpoint{6.160000in}{6.160000in}}%
\pgfusepath{clip}%
\pgfsetbuttcap%
\pgfsetroundjoin%
\definecolor{currentfill}{rgb}{0.430507,0.564883,0.948889}%
\pgfsetfillcolor{currentfill}%
\pgfsetlinewidth{0.000000pt}%
\definecolor{currentstroke}{rgb}{0.000000,0.000000,0.000000}%
\pgfsetstrokecolor{currentstroke}%
\pgfsetdash{}{0pt}%
\pgfpathmoveto{\pgfqpoint{5.211520in}{3.505114in}}%
\pgfpathlineto{\pgfqpoint{5.221912in}{3.528970in}}%
\pgfpathlineto{\pgfqpoint{5.232243in}{3.544511in}}%
\pgfpathlineto{\pgfqpoint{5.263719in}{3.469288in}}%
\pgfpathlineto{\pgfqpoint{5.295131in}{3.394343in}}%
\pgfpathlineto{\pgfqpoint{5.287582in}{3.636246in}}%
\pgfpathlineto{\pgfqpoint{5.276607in}{3.566982in}}%
\pgfpathlineto{\pgfqpoint{5.244658in}{3.592244in}}%
\pgfpathlineto{\pgfqpoint{5.211520in}{3.505114in}}%
\pgfpathclose%
\pgfusepath{fill}%
\end{pgfscope}%
\begin{pgfscope}%
\pgfpathrectangle{\pgfqpoint{1.020000in}{0.880000in}}{\pgfqpoint{6.160000in}{6.160000in}}%
\pgfusepath{clip}%
\pgfsetbuttcap%
\pgfsetroundjoin%
\definecolor{currentfill}{rgb}{0.966962,0.735670,0.630877}%
\pgfsetfillcolor{currentfill}%
\pgfsetlinewidth{0.000000pt}%
\definecolor{currentstroke}{rgb}{0.000000,0.000000,0.000000}%
\pgfsetstrokecolor{currentstroke}%
\pgfsetdash{}{0pt}%
\pgfpathmoveto{\pgfqpoint{3.284658in}{4.689365in}}%
\pgfpathlineto{\pgfqpoint{3.293682in}{4.591070in}}%
\pgfpathlineto{\pgfqpoint{3.301518in}{4.629268in}}%
\pgfpathlineto{\pgfqpoint{3.333158in}{4.768540in}}%
\pgfpathlineto{\pgfqpoint{3.366637in}{4.688686in}}%
\pgfpathlineto{\pgfqpoint{3.358211in}{4.711634in}}%
\pgfpathlineto{\pgfqpoint{3.349725in}{4.743575in}}%
\pgfpathlineto{\pgfqpoint{3.315170in}{4.952868in}}%
\pgfpathlineto{\pgfqpoint{3.284658in}{4.689365in}}%
\pgfpathclose%
\pgfusepath{fill}%
\end{pgfscope}%
\begin{pgfscope}%
\pgfpathrectangle{\pgfqpoint{1.020000in}{0.880000in}}{\pgfqpoint{6.160000in}{6.160000in}}%
\pgfusepath{clip}%
\pgfsetbuttcap%
\pgfsetroundjoin%
\definecolor{currentfill}{rgb}{0.782049,0.842864,0.942980}%
\pgfsetfillcolor{currentfill}%
\pgfsetlinewidth{0.000000pt}%
\definecolor{currentstroke}{rgb}{0.000000,0.000000,0.000000}%
\pgfsetstrokecolor{currentstroke}%
\pgfsetdash{}{0pt}%
\pgfpathmoveto{\pgfqpoint{3.895173in}{4.286740in}}%
\pgfpathlineto{\pgfqpoint{3.904311in}{4.164278in}}%
\pgfpathlineto{\pgfqpoint{3.913150in}{4.151502in}}%
\pgfpathlineto{\pgfqpoint{3.945766in}{4.172679in}}%
\pgfpathlineto{\pgfqpoint{3.978667in}{4.051148in}}%
\pgfpathlineto{\pgfqpoint{3.969302in}{4.290090in}}%
\pgfpathlineto{\pgfqpoint{3.960867in}{4.097449in}}%
\pgfpathlineto{\pgfqpoint{3.928400in}{4.057945in}}%
\pgfpathlineto{\pgfqpoint{3.895173in}{4.286740in}}%
\pgfpathclose%
\pgfusepath{fill}%
\end{pgfscope}%
\begin{pgfscope}%
\pgfpathrectangle{\pgfqpoint{1.020000in}{0.880000in}}{\pgfqpoint{6.160000in}{6.160000in}}%
\pgfusepath{clip}%
\pgfsetbuttcap%
\pgfsetroundjoin%
\definecolor{currentfill}{rgb}{0.478462,0.616564,0.972721}%
\pgfsetfillcolor{currentfill}%
\pgfsetlinewidth{0.000000pt}%
\definecolor{currentstroke}{rgb}{0.000000,0.000000,0.000000}%
\pgfsetstrokecolor{currentstroke}%
\pgfsetdash{}{0pt}%
\pgfpathmoveto{\pgfqpoint{4.785285in}{3.622006in}}%
\pgfpathlineto{\pgfqpoint{4.795490in}{3.703552in}}%
\pgfpathlineto{\pgfqpoint{4.804563in}{3.596757in}}%
\pgfpathlineto{\pgfqpoint{4.837425in}{3.666431in}}%
\pgfpathlineto{\pgfqpoint{4.869240in}{3.579505in}}%
\pgfpathlineto{\pgfqpoint{4.859809in}{3.631879in}}%
\pgfpathlineto{\pgfqpoint{4.849769in}{3.591270in}}%
\pgfpathlineto{\pgfqpoint{4.816986in}{3.518884in}}%
\pgfpathlineto{\pgfqpoint{4.785285in}{3.622006in}}%
\pgfpathclose%
\pgfusepath{fill}%
\end{pgfscope}%
\begin{pgfscope}%
\pgfpathrectangle{\pgfqpoint{1.020000in}{0.880000in}}{\pgfqpoint{6.160000in}{6.160000in}}%
\pgfusepath{clip}%
\pgfsetbuttcap%
\pgfsetroundjoin%
\definecolor{currentfill}{rgb}{0.435815,0.570707,0.951717}%
\pgfsetfillcolor{currentfill}%
\pgfsetlinewidth{0.000000pt}%
\definecolor{currentstroke}{rgb}{0.000000,0.000000,0.000000}%
\pgfsetstrokecolor{currentstroke}%
\pgfsetdash{}{0pt}%
\pgfpathmoveto{\pgfqpoint{4.933831in}{3.563615in}}%
\pgfpathlineto{\pgfqpoint{4.942760in}{3.440465in}}%
\pgfpathlineto{\pgfqpoint{4.953816in}{3.594304in}}%
\pgfpathlineto{\pgfqpoint{4.986223in}{3.596068in}}%
\pgfpathlineto{\pgfqpoint{5.017624in}{3.479264in}}%
\pgfpathlineto{\pgfqpoint{5.007301in}{3.432496in}}%
\pgfpathlineto{\pgfqpoint{4.998131in}{3.523816in}}%
\pgfpathlineto{\pgfqpoint{4.966607in}{3.621765in}}%
\pgfpathlineto{\pgfqpoint{4.933831in}{3.563615in}}%
\pgfpathclose%
\pgfusepath{fill}%
\end{pgfscope}%
\begin{pgfscope}%
\pgfpathrectangle{\pgfqpoint{1.020000in}{0.880000in}}{\pgfqpoint{6.160000in}{6.160000in}}%
\pgfusepath{clip}%
\pgfsetbuttcap%
\pgfsetroundjoin%
\definecolor{currentfill}{rgb}{0.399231,0.528528,0.928459}%
\pgfsetfillcolor{currentfill}%
\pgfsetlinewidth{0.000000pt}%
\definecolor{currentstroke}{rgb}{0.000000,0.000000,0.000000}%
\pgfsetstrokecolor{currentstroke}%
\pgfsetdash{}{0pt}%
\pgfpathmoveto{\pgfqpoint{5.361116in}{3.530215in}}%
\pgfpathlineto{\pgfqpoint{5.370914in}{3.486956in}}%
\pgfpathlineto{\pgfqpoint{5.380049in}{3.387281in}}%
\pgfpathlineto{\pgfqpoint{5.413211in}{3.463577in}}%
\pgfpathlineto{\pgfqpoint{5.445093in}{3.436002in}}%
\pgfpathlineto{\pgfqpoint{5.434553in}{3.423520in}}%
\pgfpathlineto{\pgfqpoint{5.425705in}{3.545395in}}%
\pgfpathlineto{\pgfqpoint{5.392207in}{3.437680in}}%
\pgfpathlineto{\pgfqpoint{5.361116in}{3.530215in}}%
\pgfpathclose%
\pgfusepath{fill}%
\end{pgfscope}%
\begin{pgfscope}%
\pgfpathrectangle{\pgfqpoint{1.020000in}{0.880000in}}{\pgfqpoint{6.160000in}{6.160000in}}%
\pgfusepath{clip}%
\pgfsetbuttcap%
\pgfsetroundjoin%
\definecolor{currentfill}{rgb}{0.613933,0.739923,0.999142}%
\pgfsetfillcolor{currentfill}%
\pgfsetlinewidth{0.000000pt}%
\definecolor{currentstroke}{rgb}{0.000000,0.000000,0.000000}%
\pgfsetstrokecolor{currentstroke}%
\pgfsetdash{}{0pt}%
\pgfpathmoveto{\pgfqpoint{4.275365in}{3.887587in}}%
\pgfpathlineto{\pgfqpoint{4.284526in}{3.865457in}}%
\pgfpathlineto{\pgfqpoint{4.293565in}{3.718682in}}%
\pgfpathlineto{\pgfqpoint{4.326357in}{3.906675in}}%
\pgfpathlineto{\pgfqpoint{4.358585in}{3.713159in}}%
\pgfpathlineto{\pgfqpoint{4.349727in}{3.949322in}}%
\pgfpathlineto{\pgfqpoint{4.340174in}{3.756377in}}%
\pgfpathlineto{\pgfqpoint{4.307995in}{3.979503in}}%
\pgfpathlineto{\pgfqpoint{4.275365in}{3.887587in}}%
\pgfpathclose%
\pgfusepath{fill}%
\end{pgfscope}%
\begin{pgfscope}%
\pgfpathrectangle{\pgfqpoint{1.020000in}{0.880000in}}{\pgfqpoint{6.160000in}{6.160000in}}%
\pgfusepath{clip}%
\pgfsetbuttcap%
\pgfsetroundjoin%
\definecolor{currentfill}{rgb}{0.952761,0.782965,0.698646}%
\pgfsetfillcolor{currentfill}%
\pgfsetlinewidth{0.000000pt}%
\definecolor{currentstroke}{rgb}{0.000000,0.000000,0.000000}%
\pgfsetstrokecolor{currentstroke}%
\pgfsetdash{}{0pt}%
\pgfpathmoveto{\pgfqpoint{3.366637in}{4.688686in}}%
\pgfpathlineto{\pgfqpoint{3.375043in}{4.669851in}}%
\pgfpathlineto{\pgfqpoint{3.383415in}{4.657107in}}%
\pgfpathlineto{\pgfqpoint{3.416618in}{4.609616in}}%
\pgfpathlineto{\pgfqpoint{3.450100in}{4.516808in}}%
\pgfpathlineto{\pgfqpoint{3.441262in}{4.587222in}}%
\pgfpathlineto{\pgfqpoint{3.433962in}{4.451032in}}%
\pgfpathlineto{\pgfqpoint{3.399367in}{4.696545in}}%
\pgfpathlineto{\pgfqpoint{3.366637in}{4.688686in}}%
\pgfpathclose%
\pgfusepath{fill}%
\end{pgfscope}%
\begin{pgfscope}%
\pgfpathrectangle{\pgfqpoint{1.020000in}{0.880000in}}{\pgfqpoint{6.160000in}{6.160000in}}%
\pgfusepath{clip}%
\pgfsetbuttcap%
\pgfsetroundjoin%
\definecolor{currentfill}{rgb}{0.425199,0.559058,0.946061}%
\pgfsetfillcolor{currentfill}%
\pgfsetlinewidth{0.000000pt}%
\definecolor{currentstroke}{rgb}{0.000000,0.000000,0.000000}%
\pgfsetstrokecolor{currentstroke}%
\pgfsetdash{}{0pt}%
\pgfpathmoveto{\pgfqpoint{5.146922in}{3.493449in}}%
\pgfpathlineto{\pgfqpoint{5.158544in}{3.651624in}}%
\pgfpathlineto{\pgfqpoint{5.165717in}{3.352036in}}%
\pgfpathlineto{\pgfqpoint{5.198032in}{3.357038in}}%
\pgfpathlineto{\pgfqpoint{5.232243in}{3.544511in}}%
\pgfpathlineto{\pgfqpoint{5.221912in}{3.528970in}}%
\pgfpathlineto{\pgfqpoint{5.211520in}{3.505114in}}%
\pgfpathlineto{\pgfqpoint{5.180972in}{3.674437in}}%
\pgfpathlineto{\pgfqpoint{5.146922in}{3.493449in}}%
\pgfpathclose%
\pgfusepath{fill}%
\end{pgfscope}%
\begin{pgfscope}%
\pgfpathrectangle{\pgfqpoint{1.020000in}{0.880000in}}{\pgfqpoint{6.160000in}{6.160000in}}%
\pgfusepath{clip}%
\pgfsetbuttcap%
\pgfsetroundjoin%
\definecolor{currentfill}{rgb}{0.521696,0.659599,0.987736}%
\pgfsetfillcolor{currentfill}%
\pgfsetlinewidth{0.000000pt}%
\definecolor{currentstroke}{rgb}{0.000000,0.000000,0.000000}%
\pgfsetstrokecolor{currentstroke}%
\pgfsetdash{}{0pt}%
\pgfpathmoveto{\pgfqpoint{4.572485in}{3.803090in}}%
\pgfpathlineto{\pgfqpoint{4.581421in}{3.660987in}}%
\pgfpathlineto{\pgfqpoint{4.591149in}{3.715512in}}%
\pgfpathlineto{\pgfqpoint{4.623618in}{3.709134in}}%
\pgfpathlineto{\pgfqpoint{4.655381in}{3.558078in}}%
\pgfpathlineto{\pgfqpoint{4.646434in}{3.690490in}}%
\pgfpathlineto{\pgfqpoint{4.636912in}{3.696787in}}%
\pgfpathlineto{\pgfqpoint{4.604089in}{3.598516in}}%
\pgfpathlineto{\pgfqpoint{4.572485in}{3.803090in}}%
\pgfpathclose%
\pgfusepath{fill}%
\end{pgfscope}%
\begin{pgfscope}%
\pgfpathrectangle{\pgfqpoint{1.020000in}{0.880000in}}{\pgfqpoint{6.160000in}{6.160000in}}%
\pgfusepath{clip}%
\pgfsetbuttcap%
\pgfsetroundjoin%
\definecolor{currentfill}{rgb}{0.969289,0.684982,0.568975}%
\pgfsetfillcolor{currentfill}%
\pgfsetlinewidth{0.000000pt}%
\definecolor{currentstroke}{rgb}{0.000000,0.000000,0.000000}%
\pgfsetstrokecolor{currentstroke}%
\pgfsetdash{}{0pt}%
\pgfpathmoveto{\pgfqpoint{3.071390in}{4.724455in}}%
\pgfpathlineto{\pgfqpoint{3.077821in}{4.863115in}}%
\pgfpathlineto{\pgfqpoint{3.085901in}{4.852833in}}%
\pgfpathlineto{\pgfqpoint{3.119505in}{4.788721in}}%
\pgfpathlineto{\pgfqpoint{3.152845in}{4.744694in}}%
\pgfpathlineto{\pgfqpoint{3.144715in}{4.755112in}}%
\pgfpathlineto{\pgfqpoint{3.135789in}{4.844077in}}%
\pgfpathlineto{\pgfqpoint{3.102554in}{4.880516in}}%
\pgfpathlineto{\pgfqpoint{3.071390in}{4.724455in}}%
\pgfpathclose%
\pgfusepath{fill}%
\end{pgfscope}%
\begin{pgfscope}%
\pgfpathrectangle{\pgfqpoint{1.020000in}{0.880000in}}{\pgfqpoint{6.160000in}{6.160000in}}%
\pgfusepath{clip}%
\pgfsetbuttcap%
\pgfsetroundjoin%
\definecolor{currentfill}{rgb}{0.968863,0.710838,0.599901}%
\pgfsetfillcolor{currentfill}%
\pgfsetlinewidth{0.000000pt}%
\definecolor{currentstroke}{rgb}{0.000000,0.000000,0.000000}%
\pgfsetstrokecolor{currentstroke}%
\pgfsetdash{}{0pt}%
\pgfpathmoveto{\pgfqpoint{2.513662in}{4.745614in}}%
\pgfpathlineto{\pgfqpoint{2.521185in}{4.736691in}}%
\pgfpathlineto{\pgfqpoint{2.526926in}{4.837734in}}%
\pgfpathlineto{\pgfqpoint{2.564644in}{4.549360in}}%
\pgfpathlineto{\pgfqpoint{2.592450in}{4.879900in}}%
\pgfpathlineto{\pgfqpoint{2.586828in}{4.762883in}}%
\pgfpathlineto{\pgfqpoint{2.578740in}{4.803716in}}%
\pgfpathlineto{\pgfqpoint{2.546432in}{4.760363in}}%
\pgfpathlineto{\pgfqpoint{2.513662in}{4.745614in}}%
\pgfpathclose%
\pgfusepath{fill}%
\end{pgfscope}%
\begin{pgfscope}%
\pgfpathrectangle{\pgfqpoint{1.020000in}{0.880000in}}{\pgfqpoint{6.160000in}{6.160000in}}%
\pgfusepath{clip}%
\pgfsetbuttcap%
\pgfsetroundjoin%
\definecolor{currentfill}{rgb}{0.847365,0.862472,0.885540}%
\pgfsetfillcolor{currentfill}%
\pgfsetlinewidth{0.000000pt}%
\definecolor{currentstroke}{rgb}{0.000000,0.000000,0.000000}%
\pgfsetstrokecolor{currentstroke}%
\pgfsetdash{}{0pt}%
\pgfpathmoveto{\pgfqpoint{3.746260in}{4.481760in}}%
\pgfpathlineto{\pgfqpoint{3.755757in}{4.282911in}}%
\pgfpathlineto{\pgfqpoint{3.764475in}{4.267906in}}%
\pgfpathlineto{\pgfqpoint{3.797145in}{4.279979in}}%
\pgfpathlineto{\pgfqpoint{3.829990in}{4.239059in}}%
\pgfpathlineto{\pgfqpoint{3.820908in}{4.341486in}}%
\pgfpathlineto{\pgfqpoint{3.812296in}{4.316255in}}%
\pgfpathlineto{\pgfqpoint{3.780208in}{4.175649in}}%
\pgfpathlineto{\pgfqpoint{3.746260in}{4.481760in}}%
\pgfpathclose%
\pgfusepath{fill}%
\end{pgfscope}%
\begin{pgfscope}%
\pgfpathrectangle{\pgfqpoint{1.020000in}{0.880000in}}{\pgfqpoint{6.160000in}{6.160000in}}%
\pgfusepath{clip}%
\pgfsetbuttcap%
\pgfsetroundjoin%
\definecolor{currentfill}{rgb}{0.388852,0.516298,0.921373}%
\pgfsetfillcolor{currentfill}%
\pgfsetlinewidth{0.000000pt}%
\definecolor{currentstroke}{rgb}{0.000000,0.000000,0.000000}%
\pgfsetstrokecolor{currentstroke}%
\pgfsetdash{}{0pt}%
\pgfpathmoveto{\pgfqpoint{5.295131in}{3.394343in}}%
\pgfpathlineto{\pgfqpoint{5.306149in}{3.464035in}}%
\pgfpathlineto{\pgfqpoint{5.315782in}{3.408210in}}%
\pgfpathlineto{\pgfqpoint{5.347449in}{3.356625in}}%
\pgfpathlineto{\pgfqpoint{5.380049in}{3.387281in}}%
\pgfpathlineto{\pgfqpoint{5.370914in}{3.486956in}}%
\pgfpathlineto{\pgfqpoint{5.361116in}{3.530215in}}%
\pgfpathlineto{\pgfqpoint{5.328593in}{3.504434in}}%
\pgfpathlineto{\pgfqpoint{5.295131in}{3.394343in}}%
\pgfpathclose%
\pgfusepath{fill}%
\end{pgfscope}%
\begin{pgfscope}%
\pgfpathrectangle{\pgfqpoint{1.020000in}{0.880000in}}{\pgfqpoint{6.160000in}{6.160000in}}%
\pgfusepath{clip}%
\pgfsetbuttcap%
\pgfsetroundjoin%
\definecolor{currentfill}{rgb}{0.409611,0.540759,0.935545}%
\pgfsetfillcolor{currentfill}%
\pgfsetlinewidth{0.000000pt}%
\definecolor{currentstroke}{rgb}{0.000000,0.000000,0.000000}%
\pgfsetstrokecolor{currentstroke}%
\pgfsetdash{}{0pt}%
\pgfpathmoveto{\pgfqpoint{5.082297in}{3.485347in}}%
\pgfpathlineto{\pgfqpoint{5.093078in}{3.569036in}}%
\pgfpathlineto{\pgfqpoint{5.100522in}{3.285431in}}%
\pgfpathlineto{\pgfqpoint{5.133795in}{3.389949in}}%
\pgfpathlineto{\pgfqpoint{5.165717in}{3.352036in}}%
\pgfpathlineto{\pgfqpoint{5.158544in}{3.651624in}}%
\pgfpathlineto{\pgfqpoint{5.146922in}{3.493449in}}%
\pgfpathlineto{\pgfqpoint{5.115869in}{3.623872in}}%
\pgfpathlineto{\pgfqpoint{5.082297in}{3.485347in}}%
\pgfpathclose%
\pgfusepath{fill}%
\end{pgfscope}%
\begin{pgfscope}%
\pgfpathrectangle{\pgfqpoint{1.020000in}{0.880000in}}{\pgfqpoint{6.160000in}{6.160000in}}%
\pgfusepath{clip}%
\pgfsetbuttcap%
\pgfsetroundjoin%
\definecolor{currentfill}{rgb}{0.554312,0.690097,0.995516}%
\pgfsetfillcolor{currentfill}%
\pgfsetlinewidth{0.000000pt}%
\definecolor{currentstroke}{rgb}{0.000000,0.000000,0.000000}%
\pgfsetstrokecolor{currentstroke}%
\pgfsetdash{}{0pt}%
\pgfpathmoveto{\pgfqpoint{4.358585in}{3.713159in}}%
\pgfpathlineto{\pgfqpoint{4.367993in}{3.789545in}}%
\pgfpathlineto{\pgfqpoint{4.377157in}{3.722369in}}%
\pgfpathlineto{\pgfqpoint{4.409596in}{3.673453in}}%
\pgfpathlineto{\pgfqpoint{4.442236in}{3.722048in}}%
\pgfpathlineto{\pgfqpoint{4.433037in}{3.782679in}}%
\pgfpathlineto{\pgfqpoint{4.423625in}{3.750805in}}%
\pgfpathlineto{\pgfqpoint{4.391106in}{3.731090in}}%
\pgfpathlineto{\pgfqpoint{4.358585in}{3.713159in}}%
\pgfpathclose%
\pgfusepath{fill}%
\end{pgfscope}%
\begin{pgfscope}%
\pgfpathrectangle{\pgfqpoint{1.020000in}{0.880000in}}{\pgfqpoint{6.160000in}{6.160000in}}%
\pgfusepath{clip}%
\pgfsetbuttcap%
\pgfsetroundjoin%
\definecolor{currentfill}{rgb}{0.940879,0.805596,0.735167}%
\pgfsetfillcolor{currentfill}%
\pgfsetlinewidth{0.000000pt}%
\definecolor{currentstroke}{rgb}{0.000000,0.000000,0.000000}%
\pgfsetstrokecolor{currentstroke}%
\pgfsetdash{}{0pt}%
\pgfpathmoveto{\pgfqpoint{3.514994in}{4.608614in}}%
\pgfpathlineto{\pgfqpoint{3.523670in}{4.567574in}}%
\pgfpathlineto{\pgfqpoint{3.531145in}{4.714599in}}%
\pgfpathlineto{\pgfqpoint{3.565322in}{4.497369in}}%
\pgfpathlineto{\pgfqpoint{3.597664in}{4.569579in}}%
\pgfpathlineto{\pgfqpoint{3.589588in}{4.500074in}}%
\pgfpathlineto{\pgfqpoint{3.581614in}{4.418158in}}%
\pgfpathlineto{\pgfqpoint{3.548032in}{4.563083in}}%
\pgfpathlineto{\pgfqpoint{3.514994in}{4.608614in}}%
\pgfpathclose%
\pgfusepath{fill}%
\end{pgfscope}%
\begin{pgfscope}%
\pgfpathrectangle{\pgfqpoint{1.020000in}{0.880000in}}{\pgfqpoint{6.160000in}{6.160000in}}%
\pgfusepath{clip}%
\pgfsetbuttcap%
\pgfsetroundjoin%
\definecolor{currentfill}{rgb}{0.430507,0.564883,0.948889}%
\pgfsetfillcolor{currentfill}%
\pgfsetlinewidth{0.000000pt}%
\definecolor{currentstroke}{rgb}{0.000000,0.000000,0.000000}%
\pgfsetstrokecolor{currentstroke}%
\pgfsetdash{}{0pt}%
\pgfpathmoveto{\pgfqpoint{4.869240in}{3.579505in}}%
\pgfpathlineto{\pgfqpoint{4.878273in}{3.468407in}}%
\pgfpathlineto{\pgfqpoint{4.888441in}{3.518663in}}%
\pgfpathlineto{\pgfqpoint{4.920906in}{3.526592in}}%
\pgfpathlineto{\pgfqpoint{4.953816in}{3.594304in}}%
\pgfpathlineto{\pgfqpoint{4.942760in}{3.440465in}}%
\pgfpathlineto{\pgfqpoint{4.933831in}{3.563615in}}%
\pgfpathlineto{\pgfqpoint{4.901015in}{3.497354in}}%
\pgfpathlineto{\pgfqpoint{4.869240in}{3.579505in}}%
\pgfpathclose%
\pgfusepath{fill}%
\end{pgfscope}%
\begin{pgfscope}%
\pgfpathrectangle{\pgfqpoint{1.020000in}{0.880000in}}{\pgfqpoint{6.160000in}{6.160000in}}%
\pgfusepath{clip}%
\pgfsetbuttcap%
\pgfsetroundjoin%
\definecolor{currentfill}{rgb}{0.489246,0.627536,0.976896}%
\pgfsetfillcolor{currentfill}%
\pgfsetlinewidth{0.000000pt}%
\definecolor{currentstroke}{rgb}{0.000000,0.000000,0.000000}%
\pgfsetstrokecolor{currentstroke}%
\pgfsetdash{}{0pt}%
\pgfpathmoveto{\pgfqpoint{4.721390in}{3.784083in}}%
\pgfpathlineto{\pgfqpoint{4.729379in}{3.477668in}}%
\pgfpathlineto{\pgfqpoint{4.739500in}{3.562200in}}%
\pgfpathlineto{\pgfqpoint{4.772334in}{3.630386in}}%
\pgfpathlineto{\pgfqpoint{4.804563in}{3.596757in}}%
\pgfpathlineto{\pgfqpoint{4.795490in}{3.703552in}}%
\pgfpathlineto{\pgfqpoint{4.785285in}{3.622006in}}%
\pgfpathlineto{\pgfqpoint{4.752635in}{3.573264in}}%
\pgfpathlineto{\pgfqpoint{4.721390in}{3.784083in}}%
\pgfpathclose%
\pgfusepath{fill}%
\end{pgfscope}%
\begin{pgfscope}%
\pgfpathrectangle{\pgfqpoint{1.020000in}{0.880000in}}{\pgfqpoint{6.160000in}{6.160000in}}%
\pgfusepath{clip}%
\pgfsetbuttcap%
\pgfsetroundjoin%
\definecolor{currentfill}{rgb}{0.748682,0.827679,0.963334}%
\pgfsetfillcolor{currentfill}%
\pgfsetlinewidth{0.000000pt}%
\definecolor{currentstroke}{rgb}{0.000000,0.000000,0.000000}%
\pgfsetstrokecolor{currentstroke}%
\pgfsetdash{}{0pt}%
\pgfpathmoveto{\pgfqpoint{3.978667in}{4.051148in}}%
\pgfpathlineto{\pgfqpoint{3.987673in}{3.984102in}}%
\pgfpathlineto{\pgfqpoint{3.996411in}{4.065053in}}%
\pgfpathlineto{\pgfqpoint{4.028884in}{4.189060in}}%
\pgfpathlineto{\pgfqpoint{4.061831in}{3.970646in}}%
\pgfpathlineto{\pgfqpoint{4.052650in}{4.149365in}}%
\pgfpathlineto{\pgfqpoint{4.043680in}{4.166392in}}%
\pgfpathlineto{\pgfqpoint{4.011153in}{4.114967in}}%
\pgfpathlineto{\pgfqpoint{3.978667in}{4.051148in}}%
\pgfpathclose%
\pgfusepath{fill}%
\end{pgfscope}%
\begin{pgfscope}%
\pgfpathrectangle{\pgfqpoint{1.020000in}{0.880000in}}{\pgfqpoint{6.160000in}{6.160000in}}%
\pgfusepath{clip}%
\pgfsetbuttcap%
\pgfsetroundjoin%
\definecolor{currentfill}{rgb}{0.414801,0.546874,0.939088}%
\pgfsetfillcolor{currentfill}%
\pgfsetlinewidth{0.000000pt}%
\definecolor{currentstroke}{rgb}{0.000000,0.000000,0.000000}%
\pgfsetstrokecolor{currentstroke}%
\pgfsetdash{}{0pt}%
\pgfpathmoveto{\pgfqpoint{5.574596in}{3.493842in}}%
\pgfpathlineto{\pgfqpoint{5.584591in}{3.455499in}}%
\pgfpathlineto{\pgfqpoint{5.596161in}{3.527700in}}%
\pgfpathlineto{\pgfqpoint{5.626946in}{3.428465in}}%
\pgfpathlineto{\pgfqpoint{5.659028in}{3.422627in}}%
\pgfpathlineto{\pgfqpoint{5.649888in}{3.522818in}}%
\pgfpathlineto{\pgfqpoint{5.639296in}{3.523111in}}%
\pgfpathlineto{\pgfqpoint{5.607771in}{3.566120in}}%
\pgfpathlineto{\pgfqpoint{5.574596in}{3.493842in}}%
\pgfpathclose%
\pgfusepath{fill}%
\end{pgfscope}%
\begin{pgfscope}%
\pgfpathrectangle{\pgfqpoint{1.020000in}{0.880000in}}{\pgfqpoint{6.160000in}{6.160000in}}%
\pgfusepath{clip}%
\pgfsetbuttcap%
\pgfsetroundjoin%
\definecolor{currentfill}{rgb}{0.968863,0.710838,0.599901}%
\pgfsetfillcolor{currentfill}%
\pgfsetlinewidth{0.000000pt}%
\definecolor{currentstroke}{rgb}{0.000000,0.000000,0.000000}%
\pgfsetstrokecolor{currentstroke}%
\pgfsetdash{}{0pt}%
\pgfpathmoveto{\pgfqpoint{3.217571in}{4.846836in}}%
\pgfpathlineto{\pgfqpoint{3.227241in}{4.681598in}}%
\pgfpathlineto{\pgfqpoint{3.233714in}{4.856795in}}%
\pgfpathlineto{\pgfqpoint{3.266611in}{4.859328in}}%
\pgfpathlineto{\pgfqpoint{3.301518in}{4.629268in}}%
\pgfpathlineto{\pgfqpoint{3.293682in}{4.591070in}}%
\pgfpathlineto{\pgfqpoint{3.284658in}{4.689365in}}%
\pgfpathlineto{\pgfqpoint{3.250194in}{4.871887in}}%
\pgfpathlineto{\pgfqpoint{3.217571in}{4.846836in}}%
\pgfpathclose%
\pgfusepath{fill}%
\end{pgfscope}%
\begin{pgfscope}%
\pgfpathrectangle{\pgfqpoint{1.020000in}{0.880000in}}{\pgfqpoint{6.160000in}{6.160000in}}%
\pgfusepath{clip}%
\pgfsetbuttcap%
\pgfsetroundjoin%
\definecolor{currentfill}{rgb}{0.964911,0.640159,0.519806}%
\pgfsetfillcolor{currentfill}%
\pgfsetlinewidth{0.000000pt}%
\definecolor{currentstroke}{rgb}{0.000000,0.000000,0.000000}%
\pgfsetstrokecolor{currentstroke}%
\pgfsetdash{}{0pt}%
\pgfpathmoveto{\pgfqpoint{2.856179in}{4.907749in}}%
\pgfpathlineto{\pgfqpoint{2.863980in}{4.902705in}}%
\pgfpathlineto{\pgfqpoint{2.872194in}{4.866856in}}%
\pgfpathlineto{\pgfqpoint{2.903026in}{5.037828in}}%
\pgfpathlineto{\pgfqpoint{2.937638in}{4.908320in}}%
\pgfpathlineto{\pgfqpoint{2.931308in}{4.788123in}}%
\pgfpathlineto{\pgfqpoint{2.923765in}{4.768319in}}%
\pgfpathlineto{\pgfqpoint{2.889781in}{4.855453in}}%
\pgfpathlineto{\pgfqpoint{2.856179in}{4.907749in}}%
\pgfpathclose%
\pgfusepath{fill}%
\end{pgfscope}%
\begin{pgfscope}%
\pgfpathrectangle{\pgfqpoint{1.020000in}{0.880000in}}{\pgfqpoint{6.160000in}{6.160000in}}%
\pgfusepath{clip}%
\pgfsetbuttcap%
\pgfsetroundjoin%
\definecolor{currentfill}{rgb}{0.677823,0.786546,0.991005}%
\pgfsetfillcolor{currentfill}%
\pgfsetlinewidth{0.000000pt}%
\definecolor{currentstroke}{rgb}{0.000000,0.000000,0.000000}%
\pgfsetstrokecolor{currentstroke}%
\pgfsetdash{}{0pt}%
\pgfpathmoveto{\pgfqpoint{4.061831in}{3.970646in}}%
\pgfpathlineto{\pgfqpoint{4.070810in}{3.960782in}}%
\pgfpathlineto{\pgfqpoint{4.079800in}{3.956769in}}%
\pgfpathlineto{\pgfqpoint{4.112423in}{3.997511in}}%
\pgfpathlineto{\pgfqpoint{4.145099in}{3.917455in}}%
\pgfpathlineto{\pgfqpoint{4.136061in}{3.905365in}}%
\pgfpathlineto{\pgfqpoint{4.126934in}{4.088962in}}%
\pgfpathlineto{\pgfqpoint{4.094584in}{3.790729in}}%
\pgfpathlineto{\pgfqpoint{4.061831in}{3.970646in}}%
\pgfpathclose%
\pgfusepath{fill}%
\end{pgfscope}%
\begin{pgfscope}%
\pgfpathrectangle{\pgfqpoint{1.020000in}{0.880000in}}{\pgfqpoint{6.160000in}{6.160000in}}%
\pgfusepath{clip}%
\pgfsetbuttcap%
\pgfsetroundjoin%
\definecolor{currentfill}{rgb}{0.800601,0.850358,0.930008}%
\pgfsetfillcolor{currentfill}%
\pgfsetlinewidth{0.000000pt}%
\definecolor{currentstroke}{rgb}{0.000000,0.000000,0.000000}%
\pgfsetstrokecolor{currentstroke}%
\pgfsetdash{}{0pt}%
\pgfpathmoveto{\pgfqpoint{3.829990in}{4.239059in}}%
\pgfpathlineto{\pgfqpoint{3.839153in}{4.112706in}}%
\pgfpathlineto{\pgfqpoint{3.847760in}{4.152466in}}%
\pgfpathlineto{\pgfqpoint{3.880229in}{4.230221in}}%
\pgfpathlineto{\pgfqpoint{3.913150in}{4.151502in}}%
\pgfpathlineto{\pgfqpoint{3.904311in}{4.164278in}}%
\pgfpathlineto{\pgfqpoint{3.895173in}{4.286740in}}%
\pgfpathlineto{\pgfqpoint{3.862837in}{4.182929in}}%
\pgfpathlineto{\pgfqpoint{3.829990in}{4.239059in}}%
\pgfpathclose%
\pgfusepath{fill}%
\end{pgfscope}%
\begin{pgfscope}%
\pgfpathrectangle{\pgfqpoint{1.020000in}{0.880000in}}{\pgfqpoint{6.160000in}{6.160000in}}%
\pgfusepath{clip}%
\pgfsetbuttcap%
\pgfsetroundjoin%
\definecolor{currentfill}{rgb}{0.383662,0.510183,0.917831}%
\pgfsetfillcolor{currentfill}%
\pgfsetlinewidth{0.000000pt}%
\definecolor{currentstroke}{rgb}{0.000000,0.000000,0.000000}%
\pgfsetstrokecolor{currentstroke}%
\pgfsetdash{}{0pt}%
\pgfpathmoveto{\pgfqpoint{5.232243in}{3.544511in}}%
\pgfpathlineto{\pgfqpoint{5.242700in}{3.569760in}}%
\pgfpathlineto{\pgfqpoint{5.248845in}{3.188768in}}%
\pgfpathlineto{\pgfqpoint{5.283491in}{3.407853in}}%
\pgfpathlineto{\pgfqpoint{5.315782in}{3.408210in}}%
\pgfpathlineto{\pgfqpoint{5.306149in}{3.464035in}}%
\pgfpathlineto{\pgfqpoint{5.295131in}{3.394343in}}%
\pgfpathlineto{\pgfqpoint{5.263719in}{3.469288in}}%
\pgfpathlineto{\pgfqpoint{5.232243in}{3.544511in}}%
\pgfpathclose%
\pgfusepath{fill}%
\end{pgfscope}%
\begin{pgfscope}%
\pgfpathrectangle{\pgfqpoint{1.020000in}{0.880000in}}{\pgfqpoint{6.160000in}{6.160000in}}%
\pgfusepath{clip}%
\pgfsetbuttcap%
\pgfsetroundjoin%
\definecolor{currentfill}{rgb}{0.909460,0.839386,0.800331}%
\pgfsetfillcolor{currentfill}%
\pgfsetlinewidth{0.000000pt}%
\definecolor{currentstroke}{rgb}{0.000000,0.000000,0.000000}%
\pgfsetstrokecolor{currentstroke}%
\pgfsetdash{}{0pt}%
\pgfpathmoveto{\pgfqpoint{3.597664in}{4.569579in}}%
\pgfpathlineto{\pgfqpoint{3.606292in}{4.547572in}}%
\pgfpathlineto{\pgfqpoint{3.614791in}{4.550764in}}%
\pgfpathlineto{\pgfqpoint{3.648204in}{4.438949in}}%
\pgfpathlineto{\pgfqpoint{3.681477in}{4.337037in}}%
\pgfpathlineto{\pgfqpoint{3.672730in}{4.373452in}}%
\pgfpathlineto{\pgfqpoint{3.663892in}{4.429909in}}%
\pgfpathlineto{\pgfqpoint{3.632040in}{4.274666in}}%
\pgfpathlineto{\pgfqpoint{3.597664in}{4.569579in}}%
\pgfpathclose%
\pgfusepath{fill}%
\end{pgfscope}%
\begin{pgfscope}%
\pgfpathrectangle{\pgfqpoint{1.020000in}{0.880000in}}{\pgfqpoint{6.160000in}{6.160000in}}%
\pgfusepath{clip}%
\pgfsetbuttcap%
\pgfsetroundjoin%
\definecolor{currentfill}{rgb}{0.640828,0.760752,0.997846}%
\pgfsetfillcolor{currentfill}%
\pgfsetlinewidth{0.000000pt}%
\definecolor{currentstroke}{rgb}{0.000000,0.000000,0.000000}%
\pgfsetstrokecolor{currentstroke}%
\pgfsetdash{}{0pt}%
\pgfpathmoveto{\pgfqpoint{4.210245in}{3.777902in}}%
\pgfpathlineto{\pgfqpoint{4.219366in}{3.800636in}}%
\pgfpathlineto{\pgfqpoint{4.228624in}{4.080143in}}%
\pgfpathlineto{\pgfqpoint{4.261179in}{3.920638in}}%
\pgfpathlineto{\pgfqpoint{4.293565in}{3.718682in}}%
\pgfpathlineto{\pgfqpoint{4.284526in}{3.865457in}}%
\pgfpathlineto{\pgfqpoint{4.275365in}{3.887587in}}%
\pgfpathlineto{\pgfqpoint{4.242909in}{4.022953in}}%
\pgfpathlineto{\pgfqpoint{4.210245in}{3.777902in}}%
\pgfpathclose%
\pgfusepath{fill}%
\end{pgfscope}%
\begin{pgfscope}%
\pgfpathrectangle{\pgfqpoint{1.020000in}{0.880000in}}{\pgfqpoint{6.160000in}{6.160000in}}%
\pgfusepath{clip}%
\pgfsetbuttcap%
\pgfsetroundjoin%
\definecolor{currentfill}{rgb}{0.966017,0.646130,0.525890}%
\pgfsetfillcolor{currentfill}%
\pgfsetlinewidth{0.000000pt}%
\definecolor{currentstroke}{rgb}{0.000000,0.000000,0.000000}%
\pgfsetstrokecolor{currentstroke}%
\pgfsetdash{}{0pt}%
\pgfpathmoveto{\pgfqpoint{2.792291in}{4.759077in}}%
\pgfpathlineto{\pgfqpoint{2.800645in}{4.708375in}}%
\pgfpathlineto{\pgfqpoint{2.804690in}{4.975622in}}%
\pgfpathlineto{\pgfqpoint{2.836931in}{5.037370in}}%
\pgfpathlineto{\pgfqpoint{2.872194in}{4.866856in}}%
\pgfpathlineto{\pgfqpoint{2.863980in}{4.902705in}}%
\pgfpathlineto{\pgfqpoint{2.856179in}{4.907749in}}%
\pgfpathlineto{\pgfqpoint{2.825060in}{4.770724in}}%
\pgfpathlineto{\pgfqpoint{2.792291in}{4.759077in}}%
\pgfpathclose%
\pgfusepath{fill}%
\end{pgfscope}%
\begin{pgfscope}%
\pgfpathrectangle{\pgfqpoint{1.020000in}{0.880000in}}{\pgfqpoint{6.160000in}{6.160000in}}%
\pgfusepath{clip}%
\pgfsetbuttcap%
\pgfsetroundjoin%
\definecolor{currentfill}{rgb}{0.414801,0.546874,0.939088}%
\pgfsetfillcolor{currentfill}%
\pgfsetlinewidth{0.000000pt}%
\definecolor{currentstroke}{rgb}{0.000000,0.000000,0.000000}%
\pgfsetstrokecolor{currentstroke}%
\pgfsetdash{}{0pt}%
\pgfpathmoveto{\pgfqpoint{5.509414in}{3.431088in}}%
\pgfpathlineto{\pgfqpoint{5.520662in}{3.489643in}}%
\pgfpathlineto{\pgfqpoint{5.530097in}{3.411054in}}%
\pgfpathlineto{\pgfqpoint{5.561956in}{3.385014in}}%
\pgfpathlineto{\pgfqpoint{5.596161in}{3.527700in}}%
\pgfpathlineto{\pgfqpoint{5.584591in}{3.455499in}}%
\pgfpathlineto{\pgfqpoint{5.574596in}{3.493842in}}%
\pgfpathlineto{\pgfqpoint{5.544521in}{3.647743in}}%
\pgfpathlineto{\pgfqpoint{5.509414in}{3.431088in}}%
\pgfpathclose%
\pgfusepath{fill}%
\end{pgfscope}%
\begin{pgfscope}%
\pgfpathrectangle{\pgfqpoint{1.020000in}{0.880000in}}{\pgfqpoint{6.160000in}{6.160000in}}%
\pgfusepath{clip}%
\pgfsetbuttcap%
\pgfsetroundjoin%
\definecolor{currentfill}{rgb}{0.969683,0.690484,0.575138}%
\pgfsetfillcolor{currentfill}%
\pgfsetlinewidth{0.000000pt}%
\definecolor{currentstroke}{rgb}{0.000000,0.000000,0.000000}%
\pgfsetstrokecolor{currentstroke}%
\pgfsetdash{}{0pt}%
\pgfpathmoveto{\pgfqpoint{2.447293in}{4.762884in}}%
\pgfpathlineto{\pgfqpoint{2.454868in}{4.747676in}}%
\pgfpathlineto{\pgfqpoint{2.461256in}{4.803049in}}%
\pgfpathlineto{\pgfqpoint{2.493262in}{4.870363in}}%
\pgfpathlineto{\pgfqpoint{2.526926in}{4.837734in}}%
\pgfpathlineto{\pgfqpoint{2.521185in}{4.736691in}}%
\pgfpathlineto{\pgfqpoint{2.513662in}{4.745614in}}%
\pgfpathlineto{\pgfqpoint{2.479929in}{4.787626in}}%
\pgfpathlineto{\pgfqpoint{2.447293in}{4.762884in}}%
\pgfpathclose%
\pgfusepath{fill}%
\end{pgfscope}%
\begin{pgfscope}%
\pgfpathrectangle{\pgfqpoint{1.020000in}{0.880000in}}{\pgfqpoint{6.160000in}{6.160000in}}%
\pgfusepath{clip}%
\pgfsetbuttcap%
\pgfsetroundjoin%
\definecolor{currentfill}{rgb}{0.489246,0.627536,0.976896}%
\pgfsetfillcolor{currentfill}%
\pgfsetlinewidth{0.000000pt}%
\definecolor{currentstroke}{rgb}{0.000000,0.000000,0.000000}%
\pgfsetstrokecolor{currentstroke}%
\pgfsetdash{}{0pt}%
\pgfpathmoveto{\pgfqpoint{4.655381in}{3.558078in}}%
\pgfpathlineto{\pgfqpoint{4.665521in}{3.674918in}}%
\pgfpathlineto{\pgfqpoint{4.675046in}{3.657770in}}%
\pgfpathlineto{\pgfqpoint{4.706930in}{3.539630in}}%
\pgfpathlineto{\pgfqpoint{4.739500in}{3.562200in}}%
\pgfpathlineto{\pgfqpoint{4.729379in}{3.477668in}}%
\pgfpathlineto{\pgfqpoint{4.721390in}{3.784083in}}%
\pgfpathlineto{\pgfqpoint{4.688481in}{3.695350in}}%
\pgfpathlineto{\pgfqpoint{4.655381in}{3.558078in}}%
\pgfpathclose%
\pgfusepath{fill}%
\end{pgfscope}%
\begin{pgfscope}%
\pgfpathrectangle{\pgfqpoint{1.020000in}{0.880000in}}{\pgfqpoint{6.160000in}{6.160000in}}%
\pgfusepath{clip}%
\pgfsetbuttcap%
\pgfsetroundjoin%
\definecolor{currentfill}{rgb}{0.419991,0.552989,0.942630}%
\pgfsetfillcolor{currentfill}%
\pgfsetlinewidth{0.000000pt}%
\definecolor{currentstroke}{rgb}{0.000000,0.000000,0.000000}%
\pgfsetstrokecolor{currentstroke}%
\pgfsetdash{}{0pt}%
\pgfpathmoveto{\pgfqpoint{5.017624in}{3.479264in}}%
\pgfpathlineto{\pgfqpoint{5.027489in}{3.468143in}}%
\pgfpathlineto{\pgfqpoint{5.038584in}{3.598957in}}%
\pgfpathlineto{\pgfqpoint{5.070467in}{3.538514in}}%
\pgfpathlineto{\pgfqpoint{5.100522in}{3.285431in}}%
\pgfpathlineto{\pgfqpoint{5.093078in}{3.569036in}}%
\pgfpathlineto{\pgfqpoint{5.082297in}{3.485347in}}%
\pgfpathlineto{\pgfqpoint{5.050434in}{3.535980in}}%
\pgfpathlineto{\pgfqpoint{5.017624in}{3.479264in}}%
\pgfpathclose%
\pgfusepath{fill}%
\end{pgfscope}%
\begin{pgfscope}%
\pgfpathrectangle{\pgfqpoint{1.020000in}{0.880000in}}{\pgfqpoint{6.160000in}{6.160000in}}%
\pgfusepath{clip}%
\pgfsetbuttcap%
\pgfsetroundjoin%
\definecolor{currentfill}{rgb}{0.373552,0.497499,0.909467}%
\pgfsetfillcolor{currentfill}%
\pgfsetlinewidth{0.000000pt}%
\definecolor{currentstroke}{rgb}{0.000000,0.000000,0.000000}%
\pgfsetstrokecolor{currentstroke}%
\pgfsetdash{}{0pt}%
\pgfpathmoveto{\pgfqpoint{5.165717in}{3.352036in}}%
\pgfpathlineto{\pgfqpoint{5.176503in}{3.420096in}}%
\pgfpathlineto{\pgfqpoint{5.187095in}{3.465310in}}%
\pgfpathlineto{\pgfqpoint{5.218074in}{3.331850in}}%
\pgfpathlineto{\pgfqpoint{5.248845in}{3.188768in}}%
\pgfpathlineto{\pgfqpoint{5.242700in}{3.569760in}}%
\pgfpathlineto{\pgfqpoint{5.232243in}{3.544511in}}%
\pgfpathlineto{\pgfqpoint{5.198032in}{3.357038in}}%
\pgfpathlineto{\pgfqpoint{5.165717in}{3.352036in}}%
\pgfpathclose%
\pgfusepath{fill}%
\end{pgfscope}%
\begin{pgfscope}%
\pgfpathrectangle{\pgfqpoint{1.020000in}{0.880000in}}{\pgfqpoint{6.160000in}{6.160000in}}%
\pgfusepath{clip}%
\pgfsetbuttcap%
\pgfsetroundjoin%
\definecolor{currentfill}{rgb}{0.388852,0.516298,0.921373}%
\pgfsetfillcolor{currentfill}%
\pgfsetlinewidth{0.000000pt}%
\definecolor{currentstroke}{rgb}{0.000000,0.000000,0.000000}%
\pgfsetstrokecolor{currentstroke}%
\pgfsetdash{}{0pt}%
\pgfpathmoveto{\pgfqpoint{5.445093in}{3.436002in}}%
\pgfpathlineto{\pgfqpoint{5.456962in}{3.551104in}}%
\pgfpathlineto{\pgfqpoint{5.465380in}{3.393545in}}%
\pgfpathlineto{\pgfqpoint{5.497231in}{3.363188in}}%
\pgfpathlineto{\pgfqpoint{5.530097in}{3.411054in}}%
\pgfpathlineto{\pgfqpoint{5.520662in}{3.489643in}}%
\pgfpathlineto{\pgfqpoint{5.509414in}{3.431088in}}%
\pgfpathlineto{\pgfqpoint{5.477048in}{3.416910in}}%
\pgfpathlineto{\pgfqpoint{5.445093in}{3.436002in}}%
\pgfpathclose%
\pgfusepath{fill}%
\end{pgfscope}%
\begin{pgfscope}%
\pgfpathrectangle{\pgfqpoint{1.020000in}{0.880000in}}{\pgfqpoint{6.160000in}{6.160000in}}%
\pgfusepath{clip}%
\pgfsetbuttcap%
\pgfsetroundjoin%
\definecolor{currentfill}{rgb}{0.548876,0.685104,0.994379}%
\pgfsetfillcolor{currentfill}%
\pgfsetlinewidth{0.000000pt}%
\definecolor{currentstroke}{rgb}{0.000000,0.000000,0.000000}%
\pgfsetstrokecolor{currentstroke}%
\pgfsetdash{}{0pt}%
\pgfpathmoveto{\pgfqpoint{4.507199in}{3.709017in}}%
\pgfpathlineto{\pgfqpoint{4.516735in}{3.741175in}}%
\pgfpathlineto{\pgfqpoint{4.526383in}{3.798334in}}%
\pgfpathlineto{\pgfqpoint{4.558379in}{3.647446in}}%
\pgfpathlineto{\pgfqpoint{4.591149in}{3.715512in}}%
\pgfpathlineto{\pgfqpoint{4.581421in}{3.660987in}}%
\pgfpathlineto{\pgfqpoint{4.572485in}{3.803090in}}%
\pgfpathlineto{\pgfqpoint{4.539639in}{3.700450in}}%
\pgfpathlineto{\pgfqpoint{4.507199in}{3.709017in}}%
\pgfpathclose%
\pgfusepath{fill}%
\end{pgfscope}%
\begin{pgfscope}%
\pgfpathrectangle{\pgfqpoint{1.020000in}{0.880000in}}{\pgfqpoint{6.160000in}{6.160000in}}%
\pgfusepath{clip}%
\pgfsetbuttcap%
\pgfsetroundjoin%
\definecolor{currentfill}{rgb}{0.968105,0.668475,0.550486}%
\pgfsetfillcolor{currentfill}%
\pgfsetlinewidth{0.000000pt}%
\definecolor{currentstroke}{rgb}{0.000000,0.000000,0.000000}%
\pgfsetstrokecolor{currentstroke}%
\pgfsetdash{}{0pt}%
\pgfpathmoveto{\pgfqpoint{2.726140in}{4.776019in}}%
\pgfpathlineto{\pgfqpoint{2.732096in}{4.890937in}}%
\pgfpathlineto{\pgfqpoint{2.741628in}{4.756816in}}%
\pgfpathlineto{\pgfqpoint{2.770765in}{5.036562in}}%
\pgfpathlineto{\pgfqpoint{2.804690in}{4.975622in}}%
\pgfpathlineto{\pgfqpoint{2.800645in}{4.708375in}}%
\pgfpathlineto{\pgfqpoint{2.792291in}{4.759077in}}%
\pgfpathlineto{\pgfqpoint{2.759975in}{4.714399in}}%
\pgfpathlineto{\pgfqpoint{2.726140in}{4.776019in}}%
\pgfpathclose%
\pgfusepath{fill}%
\end{pgfscope}%
\begin{pgfscope}%
\pgfpathrectangle{\pgfqpoint{1.020000in}{0.880000in}}{\pgfqpoint{6.160000in}{6.160000in}}%
\pgfusepath{clip}%
\pgfsetbuttcap%
\pgfsetroundjoin%
\definecolor{currentfill}{rgb}{0.758539,0.832787,0.958408}%
\pgfsetfillcolor{currentfill}%
\pgfsetlinewidth{0.000000pt}%
\definecolor{currentstroke}{rgb}{0.000000,0.000000,0.000000}%
\pgfsetstrokecolor{currentstroke}%
\pgfsetdash{}{0pt}%
\pgfpathmoveto{\pgfqpoint{3.913150in}{4.151502in}}%
\pgfpathlineto{\pgfqpoint{3.921951in}{4.160084in}}%
\pgfpathlineto{\pgfqpoint{3.931107in}{4.034644in}}%
\pgfpathlineto{\pgfqpoint{3.963597in}{4.126262in}}%
\pgfpathlineto{\pgfqpoint{3.996411in}{4.065053in}}%
\pgfpathlineto{\pgfqpoint{3.987673in}{3.984102in}}%
\pgfpathlineto{\pgfqpoint{3.978667in}{4.051148in}}%
\pgfpathlineto{\pgfqpoint{3.945766in}{4.172679in}}%
\pgfpathlineto{\pgfqpoint{3.913150in}{4.151502in}}%
\pgfpathclose%
\pgfusepath{fill}%
\end{pgfscope}%
\begin{pgfscope}%
\pgfpathrectangle{\pgfqpoint{1.020000in}{0.880000in}}{\pgfqpoint{6.160000in}{6.160000in}}%
\pgfusepath{clip}%
\pgfsetbuttcap%
\pgfsetroundjoin%
\definecolor{currentfill}{rgb}{0.576051,0.708780,0.997755}%
\pgfsetfillcolor{currentfill}%
\pgfsetlinewidth{0.000000pt}%
\definecolor{currentstroke}{rgb}{0.000000,0.000000,0.000000}%
\pgfsetstrokecolor{currentstroke}%
\pgfsetdash{}{0pt}%
\pgfpathmoveto{\pgfqpoint{4.293565in}{3.718682in}}%
\pgfpathlineto{\pgfqpoint{4.302934in}{3.854693in}}%
\pgfpathlineto{\pgfqpoint{4.311783in}{3.552036in}}%
\pgfpathlineto{\pgfqpoint{4.344797in}{3.851737in}}%
\pgfpathlineto{\pgfqpoint{4.377157in}{3.722369in}}%
\pgfpathlineto{\pgfqpoint{4.367993in}{3.789545in}}%
\pgfpathlineto{\pgfqpoint{4.358585in}{3.713159in}}%
\pgfpathlineto{\pgfqpoint{4.326357in}{3.906675in}}%
\pgfpathlineto{\pgfqpoint{4.293565in}{3.718682in}}%
\pgfpathclose%
\pgfusepath{fill}%
\end{pgfscope}%
\begin{pgfscope}%
\pgfpathrectangle{\pgfqpoint{1.020000in}{0.880000in}}{\pgfqpoint{6.160000in}{6.160000in}}%
\pgfusepath{clip}%
\pgfsetbuttcap%
\pgfsetroundjoin%
\definecolor{currentfill}{rgb}{0.358415,0.478426,0.896795}%
\pgfsetfillcolor{currentfill}%
\pgfsetlinewidth{0.000000pt}%
\definecolor{currentstroke}{rgb}{0.000000,0.000000,0.000000}%
\pgfsetstrokecolor{currentstroke}%
\pgfsetdash{}{0pt}%
\pgfpathmoveto{\pgfqpoint{5.100522in}{3.285431in}}%
\pgfpathlineto{\pgfqpoint{5.111262in}{3.360536in}}%
\pgfpathlineto{\pgfqpoint{5.121884in}{3.419371in}}%
\pgfpathlineto{\pgfqpoint{5.152972in}{3.285881in}}%
\pgfpathlineto{\pgfqpoint{5.187095in}{3.465310in}}%
\pgfpathlineto{\pgfqpoint{5.176503in}{3.420096in}}%
\pgfpathlineto{\pgfqpoint{5.165717in}{3.352036in}}%
\pgfpathlineto{\pgfqpoint{5.133795in}{3.389949in}}%
\pgfpathlineto{\pgfqpoint{5.100522in}{3.285431in}}%
\pgfpathclose%
\pgfusepath{fill}%
\end{pgfscope}%
\begin{pgfscope}%
\pgfpathrectangle{\pgfqpoint{1.020000in}{0.880000in}}{\pgfqpoint{6.160000in}{6.160000in}}%
\pgfusepath{clip}%
\pgfsetbuttcap%
\pgfsetroundjoin%
\definecolor{currentfill}{rgb}{0.952761,0.782965,0.698646}%
\pgfsetfillcolor{currentfill}%
\pgfsetlinewidth{0.000000pt}%
\definecolor{currentstroke}{rgb}{0.000000,0.000000,0.000000}%
\pgfsetstrokecolor{currentstroke}%
\pgfsetdash{}{0pt}%
\pgfpathmoveto{\pgfqpoint{3.450100in}{4.516808in}}%
\pgfpathlineto{\pgfqpoint{3.457980in}{4.580648in}}%
\pgfpathlineto{\pgfqpoint{3.466082in}{4.617073in}}%
\pgfpathlineto{\pgfqpoint{3.499477in}{4.537285in}}%
\pgfpathlineto{\pgfqpoint{3.531145in}{4.714599in}}%
\pgfpathlineto{\pgfqpoint{3.523670in}{4.567574in}}%
\pgfpathlineto{\pgfqpoint{3.514994in}{4.608614in}}%
\pgfpathlineto{\pgfqpoint{3.482134in}{4.621559in}}%
\pgfpathlineto{\pgfqpoint{3.450100in}{4.516808in}}%
\pgfpathclose%
\pgfusepath{fill}%
\end{pgfscope}%
\begin{pgfscope}%
\pgfpathrectangle{\pgfqpoint{1.020000in}{0.880000in}}{\pgfqpoint{6.160000in}{6.160000in}}%
\pgfusepath{clip}%
\pgfsetbuttcap%
\pgfsetroundjoin%
\definecolor{currentfill}{rgb}{0.963806,0.634188,0.513721}%
\pgfsetfillcolor{currentfill}%
\pgfsetlinewidth{0.000000pt}%
\definecolor{currentstroke}{rgb}{0.000000,0.000000,0.000000}%
\pgfsetstrokecolor{currentstroke}%
\pgfsetdash{}{0pt}%
\pgfpathmoveto{\pgfqpoint{3.003648in}{4.899276in}}%
\pgfpathlineto{\pgfqpoint{3.011511in}{4.901851in}}%
\pgfpathlineto{\pgfqpoint{3.018196in}{5.008574in}}%
\pgfpathlineto{\pgfqpoint{3.052744in}{4.871416in}}%
\pgfpathlineto{\pgfqpoint{3.085901in}{4.852833in}}%
\pgfpathlineto{\pgfqpoint{3.077821in}{4.863115in}}%
\pgfpathlineto{\pgfqpoint{3.071390in}{4.724455in}}%
\pgfpathlineto{\pgfqpoint{3.036657in}{4.891462in}}%
\pgfpathlineto{\pgfqpoint{3.003648in}{4.899276in}}%
\pgfpathclose%
\pgfusepath{fill}%
\end{pgfscope}%
\begin{pgfscope}%
\pgfpathrectangle{\pgfqpoint{1.020000in}{0.880000in}}{\pgfqpoint{6.160000in}{6.160000in}}%
\pgfusepath{clip}%
\pgfsetbuttcap%
\pgfsetroundjoin%
\definecolor{currentfill}{rgb}{0.879622,0.858175,0.845844}%
\pgfsetfillcolor{currentfill}%
\pgfsetlinewidth{0.000000pt}%
\definecolor{currentstroke}{rgb}{0.000000,0.000000,0.000000}%
\pgfsetstrokecolor{currentstroke}%
\pgfsetdash{}{0pt}%
\pgfpathmoveto{\pgfqpoint{3.681477in}{4.337037in}}%
\pgfpathlineto{\pgfqpoint{3.690288in}{4.289668in}}%
\pgfpathlineto{\pgfqpoint{3.697615in}{4.553397in}}%
\pgfpathlineto{\pgfqpoint{3.731222in}{4.383653in}}%
\pgfpathlineto{\pgfqpoint{3.764475in}{4.267906in}}%
\pgfpathlineto{\pgfqpoint{3.755757in}{4.282911in}}%
\pgfpathlineto{\pgfqpoint{3.746260in}{4.481760in}}%
\pgfpathlineto{\pgfqpoint{3.714439in}{4.283506in}}%
\pgfpathlineto{\pgfqpoint{3.681477in}{4.337037in}}%
\pgfpathclose%
\pgfusepath{fill}%
\end{pgfscope}%
\begin{pgfscope}%
\pgfpathrectangle{\pgfqpoint{1.020000in}{0.880000in}}{\pgfqpoint{6.160000in}{6.160000in}}%
\pgfusepath{clip}%
\pgfsetbuttcap%
\pgfsetroundjoin%
\definecolor{currentfill}{rgb}{0.378598,0.503856,0.913692}%
\pgfsetfillcolor{currentfill}%
\pgfsetlinewidth{0.000000pt}%
\definecolor{currentstroke}{rgb}{0.000000,0.000000,0.000000}%
\pgfsetstrokecolor{currentstroke}%
\pgfsetdash{}{0pt}%
\pgfpathmoveto{\pgfqpoint{5.380049in}{3.387281in}}%
\pgfpathlineto{\pgfqpoint{5.391599in}{3.487592in}}%
\pgfpathlineto{\pgfqpoint{5.400068in}{3.331437in}}%
\pgfpathlineto{\pgfqpoint{5.431108in}{3.233246in}}%
\pgfpathlineto{\pgfqpoint{5.465380in}{3.393545in}}%
\pgfpathlineto{\pgfqpoint{5.456962in}{3.551104in}}%
\pgfpathlineto{\pgfqpoint{5.445093in}{3.436002in}}%
\pgfpathlineto{\pgfqpoint{5.413211in}{3.463577in}}%
\pgfpathlineto{\pgfqpoint{5.380049in}{3.387281in}}%
\pgfpathclose%
\pgfusepath{fill}%
\end{pgfscope}%
\begin{pgfscope}%
\pgfpathrectangle{\pgfqpoint{1.020000in}{0.880000in}}{\pgfqpoint{6.160000in}{6.160000in}}%
\pgfusepath{clip}%
\pgfsetbuttcap%
\pgfsetroundjoin%
\definecolor{currentfill}{rgb}{0.467678,0.605591,0.968546}%
\pgfsetfillcolor{currentfill}%
\pgfsetlinewidth{0.000000pt}%
\definecolor{currentstroke}{rgb}{0.000000,0.000000,0.000000}%
\pgfsetstrokecolor{currentstroke}%
\pgfsetdash{}{0pt}%
\pgfpathmoveto{\pgfqpoint{4.804563in}{3.596757in}}%
\pgfpathlineto{\pgfqpoint{4.814199in}{3.579517in}}%
\pgfpathlineto{\pgfqpoint{4.823469in}{3.502336in}}%
\pgfpathlineto{\pgfqpoint{4.857057in}{3.673404in}}%
\pgfpathlineto{\pgfqpoint{4.888441in}{3.518663in}}%
\pgfpathlineto{\pgfqpoint{4.878273in}{3.468407in}}%
\pgfpathlineto{\pgfqpoint{4.869240in}{3.579505in}}%
\pgfpathlineto{\pgfqpoint{4.837425in}{3.666431in}}%
\pgfpathlineto{\pgfqpoint{4.804563in}{3.596757in}}%
\pgfpathclose%
\pgfusepath{fill}%
\end{pgfscope}%
\begin{pgfscope}%
\pgfpathrectangle{\pgfqpoint{1.020000in}{0.880000in}}{\pgfqpoint{6.160000in}{6.160000in}}%
\pgfusepath{clip}%
\pgfsetbuttcap%
\pgfsetroundjoin%
\definecolor{currentfill}{rgb}{0.967544,0.730850,0.624685}%
\pgfsetfillcolor{currentfill}%
\pgfsetlinewidth{0.000000pt}%
\definecolor{currentstroke}{rgb}{0.000000,0.000000,0.000000}%
\pgfsetstrokecolor{currentstroke}%
\pgfsetdash{}{0pt}%
\pgfpathmoveto{\pgfqpoint{3.301518in}{4.629268in}}%
\pgfpathlineto{\pgfqpoint{3.308275in}{4.795004in}}%
\pgfpathlineto{\pgfqpoint{3.317159in}{4.717267in}}%
\pgfpathlineto{\pgfqpoint{3.349950in}{4.730445in}}%
\pgfpathlineto{\pgfqpoint{3.383415in}{4.657107in}}%
\pgfpathlineto{\pgfqpoint{3.375043in}{4.669851in}}%
\pgfpathlineto{\pgfqpoint{3.366637in}{4.688686in}}%
\pgfpathlineto{\pgfqpoint{3.333158in}{4.768540in}}%
\pgfpathlineto{\pgfqpoint{3.301518in}{4.629268in}}%
\pgfpathclose%
\pgfusepath{fill}%
\end{pgfscope}%
\begin{pgfscope}%
\pgfpathrectangle{\pgfqpoint{1.020000in}{0.880000in}}{\pgfqpoint{6.160000in}{6.160000in}}%
\pgfusepath{clip}%
\pgfsetbuttcap%
\pgfsetroundjoin%
\definecolor{currentfill}{rgb}{0.969289,0.684982,0.568975}%
\pgfsetfillcolor{currentfill}%
\pgfsetlinewidth{0.000000pt}%
\definecolor{currentstroke}{rgb}{0.000000,0.000000,0.000000}%
\pgfsetstrokecolor{currentstroke}%
\pgfsetdash{}{0pt}%
\pgfpathmoveto{\pgfqpoint{2.381265in}{4.755487in}}%
\pgfpathlineto{\pgfqpoint{2.387285in}{4.824991in}}%
\pgfpathlineto{\pgfqpoint{2.396660in}{4.705707in}}%
\pgfpathlineto{\pgfqpoint{2.426261in}{4.909780in}}%
\pgfpathlineto{\pgfqpoint{2.461256in}{4.803049in}}%
\pgfpathlineto{\pgfqpoint{2.454868in}{4.747676in}}%
\pgfpathlineto{\pgfqpoint{2.447293in}{4.762884in}}%
\pgfpathlineto{\pgfqpoint{2.413128in}{4.825578in}}%
\pgfpathlineto{\pgfqpoint{2.381265in}{4.755487in}}%
\pgfpathclose%
\pgfusepath{fill}%
\end{pgfscope}%
\begin{pgfscope}%
\pgfpathrectangle{\pgfqpoint{1.020000in}{0.880000in}}{\pgfqpoint{6.160000in}{6.160000in}}%
\pgfusepath{clip}%
\pgfsetbuttcap%
\pgfsetroundjoin%
\definecolor{currentfill}{rgb}{0.548876,0.685104,0.994379}%
\pgfsetfillcolor{currentfill}%
\pgfsetlinewidth{0.000000pt}%
\definecolor{currentstroke}{rgb}{0.000000,0.000000,0.000000}%
\pgfsetstrokecolor{currentstroke}%
\pgfsetdash{}{0pt}%
\pgfpathmoveto{\pgfqpoint{4.442236in}{3.722048in}}%
\pgfpathlineto{\pgfqpoint{4.451730in}{3.768835in}}%
\pgfpathlineto{\pgfqpoint{4.461203in}{3.797658in}}%
\pgfpathlineto{\pgfqpoint{4.493103in}{3.571954in}}%
\pgfpathlineto{\pgfqpoint{4.526383in}{3.798334in}}%
\pgfpathlineto{\pgfqpoint{4.516735in}{3.741175in}}%
\pgfpathlineto{\pgfqpoint{4.507199in}{3.709017in}}%
\pgfpathlineto{\pgfqpoint{4.474440in}{3.616018in}}%
\pgfpathlineto{\pgfqpoint{4.442236in}{3.722048in}}%
\pgfpathclose%
\pgfusepath{fill}%
\end{pgfscope}%
\begin{pgfscope}%
\pgfpathrectangle{\pgfqpoint{1.020000in}{0.880000in}}{\pgfqpoint{6.160000in}{6.160000in}}%
\pgfusepath{clip}%
\pgfsetbuttcap%
\pgfsetroundjoin%
\definecolor{currentfill}{rgb}{0.968533,0.715841,0.606097}%
\pgfsetfillcolor{currentfill}%
\pgfsetlinewidth{0.000000pt}%
\definecolor{currentstroke}{rgb}{0.000000,0.000000,0.000000}%
\pgfsetstrokecolor{currentstroke}%
\pgfsetdash{}{0pt}%
\pgfpathmoveto{\pgfqpoint{2.316754in}{4.662243in}}%
\pgfpathlineto{\pgfqpoint{2.321496in}{4.795002in}}%
\pgfpathlineto{\pgfqpoint{2.330421in}{4.700942in}}%
\pgfpathlineto{\pgfqpoint{2.361794in}{4.801033in}}%
\pgfpathlineto{\pgfqpoint{2.396660in}{4.705707in}}%
\pgfpathlineto{\pgfqpoint{2.387285in}{4.824991in}}%
\pgfpathlineto{\pgfqpoint{2.381265in}{4.755487in}}%
\pgfpathlineto{\pgfqpoint{2.350246in}{4.640257in}}%
\pgfpathlineto{\pgfqpoint{2.316754in}{4.662243in}}%
\pgfpathclose%
\pgfusepath{fill}%
\end{pgfscope}%
\begin{pgfscope}%
\pgfpathrectangle{\pgfqpoint{1.020000in}{0.880000in}}{\pgfqpoint{6.160000in}{6.160000in}}%
\pgfusepath{clip}%
\pgfsetbuttcap%
\pgfsetroundjoin%
\definecolor{currentfill}{rgb}{0.661968,0.775491,0.993937}%
\pgfsetfillcolor{currentfill}%
\pgfsetlinewidth{0.000000pt}%
\definecolor{currentstroke}{rgb}{0.000000,0.000000,0.000000}%
\pgfsetstrokecolor{currentstroke}%
\pgfsetdash{}{0pt}%
\pgfpathmoveto{\pgfqpoint{4.145099in}{3.917455in}}%
\pgfpathlineto{\pgfqpoint{4.154120in}{4.060000in}}%
\pgfpathlineto{\pgfqpoint{4.163248in}{3.892026in}}%
\pgfpathlineto{\pgfqpoint{4.195890in}{3.792265in}}%
\pgfpathlineto{\pgfqpoint{4.228624in}{4.080143in}}%
\pgfpathlineto{\pgfqpoint{4.219366in}{3.800636in}}%
\pgfpathlineto{\pgfqpoint{4.210245in}{3.777902in}}%
\pgfpathlineto{\pgfqpoint{4.177700in}{3.932884in}}%
\pgfpathlineto{\pgfqpoint{4.145099in}{3.917455in}}%
\pgfpathclose%
\pgfusepath{fill}%
\end{pgfscope}%
\begin{pgfscope}%
\pgfpathrectangle{\pgfqpoint{1.020000in}{0.880000in}}{\pgfqpoint{6.160000in}{6.160000in}}%
\pgfusepath{clip}%
\pgfsetbuttcap%
\pgfsetroundjoin%
\definecolor{currentfill}{rgb}{0.399231,0.528528,0.928459}%
\pgfsetfillcolor{currentfill}%
\pgfsetlinewidth{0.000000pt}%
\definecolor{currentstroke}{rgb}{0.000000,0.000000,0.000000}%
\pgfsetstrokecolor{currentstroke}%
\pgfsetdash{}{0pt}%
\pgfpathmoveto{\pgfqpoint{5.659028in}{3.422627in}}%
\pgfpathlineto{\pgfqpoint{5.669896in}{3.438684in}}%
\pgfpathlineto{\pgfqpoint{5.680996in}{3.468511in}}%
\pgfpathlineto{\pgfqpoint{5.712031in}{3.390849in}}%
\pgfpathlineto{\pgfqpoint{5.703195in}{3.512214in}}%
\pgfpathlineto{\pgfqpoint{5.691637in}{3.453279in}}%
\pgfpathlineto{\pgfqpoint{5.659028in}{3.422627in}}%
\pgfpathclose%
\pgfusepath{fill}%
\end{pgfscope}%
\begin{pgfscope}%
\pgfpathrectangle{\pgfqpoint{1.020000in}{0.880000in}}{\pgfqpoint{6.160000in}{6.160000in}}%
\pgfusepath{clip}%
\pgfsetbuttcap%
\pgfsetroundjoin%
\definecolor{currentfill}{rgb}{0.363461,0.484784,0.901019}%
\pgfsetfillcolor{currentfill}%
\pgfsetlinewidth{0.000000pt}%
\definecolor{currentstroke}{rgb}{0.000000,0.000000,0.000000}%
\pgfsetstrokecolor{currentstroke}%
\pgfsetdash{}{0pt}%
\pgfpathmoveto{\pgfqpoint{5.315782in}{3.408210in}}%
\pgfpathlineto{\pgfqpoint{5.327445in}{3.529248in}}%
\pgfpathlineto{\pgfqpoint{5.335163in}{3.304097in}}%
\pgfpathlineto{\pgfqpoint{5.366875in}{3.254897in}}%
\pgfpathlineto{\pgfqpoint{5.400068in}{3.331437in}}%
\pgfpathlineto{\pgfqpoint{5.391599in}{3.487592in}}%
\pgfpathlineto{\pgfqpoint{5.380049in}{3.387281in}}%
\pgfpathlineto{\pgfqpoint{5.347449in}{3.356625in}}%
\pgfpathlineto{\pgfqpoint{5.315782in}{3.408210in}}%
\pgfpathclose%
\pgfusepath{fill}%
\end{pgfscope}%
\begin{pgfscope}%
\pgfpathrectangle{\pgfqpoint{1.020000in}{0.880000in}}{\pgfqpoint{6.160000in}{6.160000in}}%
\pgfusepath{clip}%
\pgfsetbuttcap%
\pgfsetroundjoin%
\definecolor{currentfill}{rgb}{0.968105,0.668475,0.550486}%
\pgfsetfillcolor{currentfill}%
\pgfsetlinewidth{0.000000pt}%
\definecolor{currentstroke}{rgb}{0.000000,0.000000,0.000000}%
\pgfsetstrokecolor{currentstroke}%
\pgfsetdash{}{0pt}%
\pgfpathmoveto{\pgfqpoint{3.152845in}{4.744694in}}%
\pgfpathlineto{\pgfqpoint{3.160965in}{4.736670in}}%
\pgfpathlineto{\pgfqpoint{3.167993in}{4.839231in}}%
\pgfpathlineto{\pgfqpoint{3.199736in}{4.964001in}}%
\pgfpathlineto{\pgfqpoint{3.233714in}{4.856795in}}%
\pgfpathlineto{\pgfqpoint{3.227241in}{4.681598in}}%
\pgfpathlineto{\pgfqpoint{3.217571in}{4.846836in}}%
\pgfpathlineto{\pgfqpoint{3.184252in}{4.892062in}}%
\pgfpathlineto{\pgfqpoint{3.152845in}{4.744694in}}%
\pgfpathclose%
\pgfusepath{fill}%
\end{pgfscope}%
\begin{pgfscope}%
\pgfpathrectangle{\pgfqpoint{1.020000in}{0.880000in}}{\pgfqpoint{6.160000in}{6.160000in}}%
\pgfusepath{clip}%
\pgfsetbuttcap%
\pgfsetroundjoin%
\definecolor{currentfill}{rgb}{0.435815,0.570707,0.951717}%
\pgfsetfillcolor{currentfill}%
\pgfsetlinewidth{0.000000pt}%
\definecolor{currentstroke}{rgb}{0.000000,0.000000,0.000000}%
\pgfsetstrokecolor{currentstroke}%
\pgfsetdash{}{0pt}%
\pgfpathmoveto{\pgfqpoint{4.953816in}{3.594304in}}%
\pgfpathlineto{\pgfqpoint{4.962684in}{3.460636in}}%
\pgfpathlineto{\pgfqpoint{4.972551in}{3.454508in}}%
\pgfpathlineto{\pgfqpoint{5.004954in}{3.453751in}}%
\pgfpathlineto{\pgfqpoint{5.038584in}{3.598957in}}%
\pgfpathlineto{\pgfqpoint{5.027489in}{3.468143in}}%
\pgfpathlineto{\pgfqpoint{5.017624in}{3.479264in}}%
\pgfpathlineto{\pgfqpoint{4.986223in}{3.596068in}}%
\pgfpathlineto{\pgfqpoint{4.953816in}{3.594304in}}%
\pgfpathclose%
\pgfusepath{fill}%
\end{pgfscope}%
\begin{pgfscope}%
\pgfpathrectangle{\pgfqpoint{1.020000in}{0.880000in}}{\pgfqpoint{6.160000in}{6.160000in}}%
\pgfusepath{clip}%
\pgfsetbuttcap%
\pgfsetroundjoin%
\definecolor{currentfill}{rgb}{0.510824,0.649397,0.985079}%
\pgfsetfillcolor{currentfill}%
\pgfsetlinewidth{0.000000pt}%
\definecolor{currentstroke}{rgb}{0.000000,0.000000,0.000000}%
\pgfsetstrokecolor{currentstroke}%
\pgfsetdash{}{0pt}%
\pgfpathmoveto{\pgfqpoint{4.591149in}{3.715512in}}%
\pgfpathlineto{\pgfqpoint{4.600510in}{3.674652in}}%
\pgfpathlineto{\pgfqpoint{4.609129in}{3.456667in}}%
\pgfpathlineto{\pgfqpoint{4.642628in}{3.680123in}}%
\pgfpathlineto{\pgfqpoint{4.675046in}{3.657770in}}%
\pgfpathlineto{\pgfqpoint{4.665521in}{3.674918in}}%
\pgfpathlineto{\pgfqpoint{4.655381in}{3.558078in}}%
\pgfpathlineto{\pgfqpoint{4.623618in}{3.709134in}}%
\pgfpathlineto{\pgfqpoint{4.591149in}{3.715512in}}%
\pgfpathclose%
\pgfusepath{fill}%
\end{pgfscope}%
\begin{pgfscope}%
\pgfpathrectangle{\pgfqpoint{1.020000in}{0.880000in}}{\pgfqpoint{6.160000in}{6.160000in}}%
\pgfusepath{clip}%
\pgfsetbuttcap%
\pgfsetroundjoin%
\definecolor{currentfill}{rgb}{0.457046,0.594006,0.963029}%
\pgfsetfillcolor{currentfill}%
\pgfsetlinewidth{0.000000pt}%
\definecolor{currentstroke}{rgb}{0.000000,0.000000,0.000000}%
\pgfsetstrokecolor{currentstroke}%
\pgfsetdash{}{0pt}%
\pgfpathmoveto{\pgfqpoint{4.739500in}{3.562200in}}%
\pgfpathlineto{\pgfqpoint{4.748861in}{3.505768in}}%
\pgfpathlineto{\pgfqpoint{4.758884in}{3.562879in}}%
\pgfpathlineto{\pgfqpoint{4.790865in}{3.478111in}}%
\pgfpathlineto{\pgfqpoint{4.823469in}{3.502336in}}%
\pgfpathlineto{\pgfqpoint{4.814199in}{3.579517in}}%
\pgfpathlineto{\pgfqpoint{4.804563in}{3.596757in}}%
\pgfpathlineto{\pgfqpoint{4.772334in}{3.630386in}}%
\pgfpathlineto{\pgfqpoint{4.739500in}{3.562200in}}%
\pgfpathclose%
\pgfusepath{fill}%
\end{pgfscope}%
\begin{pgfscope}%
\pgfpathrectangle{\pgfqpoint{1.020000in}{0.880000in}}{\pgfqpoint{6.160000in}{6.160000in}}%
\pgfusepath{clip}%
\pgfsetbuttcap%
\pgfsetroundjoin%
\definecolor{currentfill}{rgb}{0.409611,0.540759,0.935545}%
\pgfsetfillcolor{currentfill}%
\pgfsetlinewidth{0.000000pt}%
\definecolor{currentstroke}{rgb}{0.000000,0.000000,0.000000}%
\pgfsetstrokecolor{currentstroke}%
\pgfsetdash{}{0pt}%
\pgfpathmoveto{\pgfqpoint{4.888441in}{3.518663in}}%
\pgfpathlineto{\pgfqpoint{4.897708in}{3.438493in}}%
\pgfpathlineto{\pgfqpoint{4.907048in}{3.367733in}}%
\pgfpathlineto{\pgfqpoint{4.939254in}{3.339624in}}%
\pgfpathlineto{\pgfqpoint{4.972551in}{3.454508in}}%
\pgfpathlineto{\pgfqpoint{4.962684in}{3.460636in}}%
\pgfpathlineto{\pgfqpoint{4.953816in}{3.594304in}}%
\pgfpathlineto{\pgfqpoint{4.920906in}{3.526592in}}%
\pgfpathlineto{\pgfqpoint{4.888441in}{3.518663in}}%
\pgfpathclose%
\pgfusepath{fill}%
\end{pgfscope}%
\begin{pgfscope}%
\pgfpathrectangle{\pgfqpoint{1.020000in}{0.880000in}}{\pgfqpoint{6.160000in}{6.160000in}}%
\pgfusepath{clip}%
\pgfsetbuttcap%
\pgfsetroundjoin%
\definecolor{currentfill}{rgb}{0.839351,0.861167,0.894494}%
\pgfsetfillcolor{currentfill}%
\pgfsetlinewidth{0.000000pt}%
\definecolor{currentstroke}{rgb}{0.000000,0.000000,0.000000}%
\pgfsetstrokecolor{currentstroke}%
\pgfsetdash{}{0pt}%
\pgfpathmoveto{\pgfqpoint{3.764475in}{4.267906in}}%
\pgfpathlineto{\pgfqpoint{3.773442in}{4.195467in}}%
\pgfpathlineto{\pgfqpoint{3.781690in}{4.304895in}}%
\pgfpathlineto{\pgfqpoint{3.813790in}{4.493682in}}%
\pgfpathlineto{\pgfqpoint{3.847760in}{4.152466in}}%
\pgfpathlineto{\pgfqpoint{3.839153in}{4.112706in}}%
\pgfpathlineto{\pgfqpoint{3.829990in}{4.239059in}}%
\pgfpathlineto{\pgfqpoint{3.797145in}{4.279979in}}%
\pgfpathlineto{\pgfqpoint{3.764475in}{4.267906in}}%
\pgfpathclose%
\pgfusepath{fill}%
\end{pgfscope}%
\begin{pgfscope}%
\pgfpathrectangle{\pgfqpoint{1.020000in}{0.880000in}}{\pgfqpoint{6.160000in}{6.160000in}}%
\pgfusepath{clip}%
\pgfsetbuttcap%
\pgfsetroundjoin%
\definecolor{currentfill}{rgb}{0.967544,0.730850,0.624685}%
\pgfsetfillcolor{currentfill}%
\pgfsetlinewidth{0.000000pt}%
\definecolor{currentstroke}{rgb}{0.000000,0.000000,0.000000}%
\pgfsetstrokecolor{currentstroke}%
\pgfsetdash{}{0pt}%
\pgfpathmoveto{\pgfqpoint{2.250318in}{4.672254in}}%
\pgfpathlineto{\pgfqpoint{2.256262in}{4.733942in}}%
\pgfpathlineto{\pgfqpoint{2.264506in}{4.675460in}}%
\pgfpathlineto{\pgfqpoint{2.297817in}{4.669532in}}%
\pgfpathlineto{\pgfqpoint{2.330421in}{4.700942in}}%
\pgfpathlineto{\pgfqpoint{2.321496in}{4.795002in}}%
\pgfpathlineto{\pgfqpoint{2.316754in}{4.662243in}}%
\pgfpathlineto{\pgfqpoint{2.281984in}{4.750541in}}%
\pgfpathlineto{\pgfqpoint{2.250318in}{4.672254in}}%
\pgfpathclose%
\pgfusepath{fill}%
\end{pgfscope}%
\begin{pgfscope}%
\pgfpathrectangle{\pgfqpoint{1.020000in}{0.880000in}}{\pgfqpoint{6.160000in}{6.160000in}}%
\pgfusepath{clip}%
\pgfsetbuttcap%
\pgfsetroundjoin%
\definecolor{currentfill}{rgb}{0.728970,0.817464,0.973188}%
\pgfsetfillcolor{currentfill}%
\pgfsetlinewidth{0.000000pt}%
\definecolor{currentstroke}{rgb}{0.000000,0.000000,0.000000}%
\pgfsetstrokecolor{currentstroke}%
\pgfsetdash{}{0pt}%
\pgfpathmoveto{\pgfqpoint{3.996411in}{4.065053in}}%
\pgfpathlineto{\pgfqpoint{4.005566in}{3.929049in}}%
\pgfpathlineto{\pgfqpoint{4.014342in}{4.014640in}}%
\pgfpathlineto{\pgfqpoint{4.046856in}{4.158697in}}%
\pgfpathlineto{\pgfqpoint{4.079800in}{3.956769in}}%
\pgfpathlineto{\pgfqpoint{4.070810in}{3.960782in}}%
\pgfpathlineto{\pgfqpoint{4.061831in}{3.970646in}}%
\pgfpathlineto{\pgfqpoint{4.028884in}{4.189060in}}%
\pgfpathlineto{\pgfqpoint{3.996411in}{4.065053in}}%
\pgfpathclose%
\pgfusepath{fill}%
\end{pgfscope}%
\begin{pgfscope}%
\pgfpathrectangle{\pgfqpoint{1.020000in}{0.880000in}}{\pgfqpoint{6.160000in}{6.160000in}}%
\pgfusepath{clip}%
\pgfsetbuttcap%
\pgfsetroundjoin%
\definecolor{currentfill}{rgb}{0.964911,0.640159,0.519806}%
\pgfsetfillcolor{currentfill}%
\pgfsetlinewidth{0.000000pt}%
\definecolor{currentstroke}{rgb}{0.000000,0.000000,0.000000}%
\pgfsetstrokecolor{currentstroke}%
\pgfsetdash{}{0pt}%
\pgfpathmoveto{\pgfqpoint{2.658505in}{4.885270in}}%
\pgfpathlineto{\pgfqpoint{2.666185in}{4.876219in}}%
\pgfpathlineto{\pgfqpoint{2.671662in}{5.016152in}}%
\pgfpathlineto{\pgfqpoint{2.705646in}{4.958707in}}%
\pgfpathlineto{\pgfqpoint{2.741628in}{4.756816in}}%
\pgfpathlineto{\pgfqpoint{2.732096in}{4.890937in}}%
\pgfpathlineto{\pgfqpoint{2.726140in}{4.776019in}}%
\pgfpathlineto{\pgfqpoint{2.693195in}{4.773205in}}%
\pgfpathlineto{\pgfqpoint{2.658505in}{4.885270in}}%
\pgfpathclose%
\pgfusepath{fill}%
\end{pgfscope}%
\begin{pgfscope}%
\pgfpathrectangle{\pgfqpoint{1.020000in}{0.880000in}}{\pgfqpoint{6.160000in}{6.160000in}}%
\pgfusepath{clip}%
\pgfsetbuttcap%
\pgfsetroundjoin%
\definecolor{currentfill}{rgb}{0.446431,0.582356,0.957373}%
\pgfsetfillcolor{currentfill}%
\pgfsetlinewidth{0.000000pt}%
\definecolor{currentstroke}{rgb}{0.000000,0.000000,0.000000}%
\pgfsetstrokecolor{currentstroke}%
\pgfsetdash{}{0pt}%
\pgfpathmoveto{\pgfqpoint{4.675046in}{3.657770in}}%
\pgfpathlineto{\pgfqpoint{4.684281in}{3.579269in}}%
\pgfpathlineto{\pgfqpoint{4.693088in}{3.415974in}}%
\pgfpathlineto{\pgfqpoint{4.725628in}{3.426135in}}%
\pgfpathlineto{\pgfqpoint{4.758884in}{3.562879in}}%
\pgfpathlineto{\pgfqpoint{4.748861in}{3.505768in}}%
\pgfpathlineto{\pgfqpoint{4.739500in}{3.562200in}}%
\pgfpathlineto{\pgfqpoint{4.706930in}{3.539630in}}%
\pgfpathlineto{\pgfqpoint{4.675046in}{3.657770in}}%
\pgfpathclose%
\pgfusepath{fill}%
\end{pgfscope}%
\begin{pgfscope}%
\pgfpathrectangle{\pgfqpoint{1.020000in}{0.880000in}}{\pgfqpoint{6.160000in}{6.160000in}}%
\pgfusepath{clip}%
\pgfsetbuttcap%
\pgfsetroundjoin%
\definecolor{currentfill}{rgb}{0.378598,0.503856,0.913692}%
\pgfsetfillcolor{currentfill}%
\pgfsetlinewidth{0.000000pt}%
\definecolor{currentstroke}{rgb}{0.000000,0.000000,0.000000}%
\pgfsetstrokecolor{currentstroke}%
\pgfsetdash{}{0pt}%
\pgfpathmoveto{\pgfqpoint{5.038584in}{3.598957in}}%
\pgfpathlineto{\pgfqpoint{5.047336in}{3.453119in}}%
\pgfpathlineto{\pgfqpoint{5.055969in}{3.294584in}}%
\pgfpathlineto{\pgfqpoint{5.088424in}{3.302739in}}%
\pgfpathlineto{\pgfqpoint{5.121884in}{3.419371in}}%
\pgfpathlineto{\pgfqpoint{5.111262in}{3.360536in}}%
\pgfpathlineto{\pgfqpoint{5.100522in}{3.285431in}}%
\pgfpathlineto{\pgfqpoint{5.070467in}{3.538514in}}%
\pgfpathlineto{\pgfqpoint{5.038584in}{3.598957in}}%
\pgfpathclose%
\pgfusepath{fill}%
\end{pgfscope}%
\begin{pgfscope}%
\pgfpathrectangle{\pgfqpoint{1.020000in}{0.880000in}}{\pgfqpoint{6.160000in}{6.160000in}}%
\pgfusepath{clip}%
\pgfsetbuttcap%
\pgfsetroundjoin%
\definecolor{currentfill}{rgb}{0.786721,0.844807,0.939810}%
\pgfsetfillcolor{currentfill}%
\pgfsetlinewidth{0.000000pt}%
\definecolor{currentstroke}{rgb}{0.000000,0.000000,0.000000}%
\pgfsetstrokecolor{currentstroke}%
\pgfsetdash{}{0pt}%
\pgfpathmoveto{\pgfqpoint{3.847760in}{4.152466in}}%
\pgfpathlineto{\pgfqpoint{3.856573in}{4.136113in}}%
\pgfpathlineto{\pgfqpoint{3.865413in}{4.116331in}}%
\pgfpathlineto{\pgfqpoint{3.898047in}{4.158646in}}%
\pgfpathlineto{\pgfqpoint{3.931107in}{4.034644in}}%
\pgfpathlineto{\pgfqpoint{3.921951in}{4.160084in}}%
\pgfpathlineto{\pgfqpoint{3.913150in}{4.151502in}}%
\pgfpathlineto{\pgfqpoint{3.880229in}{4.230221in}}%
\pgfpathlineto{\pgfqpoint{3.847760in}{4.152466in}}%
\pgfpathclose%
\pgfusepath{fill}%
\end{pgfscope}%
\begin{pgfscope}%
\pgfpathrectangle{\pgfqpoint{1.020000in}{0.880000in}}{\pgfqpoint{6.160000in}{6.160000in}}%
\pgfusepath{clip}%
\pgfsetbuttcap%
\pgfsetroundjoin%
\definecolor{currentfill}{rgb}{0.409611,0.540759,0.935545}%
\pgfsetfillcolor{currentfill}%
\pgfsetlinewidth{0.000000pt}%
\definecolor{currentstroke}{rgb}{0.000000,0.000000,0.000000}%
\pgfsetstrokecolor{currentstroke}%
\pgfsetdash{}{0pt}%
\pgfpathmoveto{\pgfqpoint{5.596161in}{3.527700in}}%
\pgfpathlineto{\pgfqpoint{5.604876in}{3.396599in}}%
\pgfpathlineto{\pgfqpoint{5.617915in}{3.567633in}}%
\pgfpathlineto{\pgfqpoint{5.648494in}{3.450694in}}%
\pgfpathlineto{\pgfqpoint{5.680996in}{3.468511in}}%
\pgfpathlineto{\pgfqpoint{5.669896in}{3.438684in}}%
\pgfpathlineto{\pgfqpoint{5.659028in}{3.422627in}}%
\pgfpathlineto{\pgfqpoint{5.626946in}{3.428465in}}%
\pgfpathlineto{\pgfqpoint{5.596161in}{3.527700in}}%
\pgfpathclose%
\pgfusepath{fill}%
\end{pgfscope}%
\begin{pgfscope}%
\pgfpathrectangle{\pgfqpoint{1.020000in}{0.880000in}}{\pgfqpoint{6.160000in}{6.160000in}}%
\pgfusepath{clip}%
\pgfsetbuttcap%
\pgfsetroundjoin%
\definecolor{currentfill}{rgb}{0.966962,0.735670,0.630877}%
\pgfsetfillcolor{currentfill}%
\pgfsetlinewidth{0.000000pt}%
\definecolor{currentstroke}{rgb}{0.000000,0.000000,0.000000}%
\pgfsetstrokecolor{currentstroke}%
\pgfsetdash{}{0pt}%
\pgfpathmoveto{\pgfqpoint{2.186046in}{4.568146in}}%
\pgfpathlineto{\pgfqpoint{2.190748in}{4.687296in}}%
\pgfpathlineto{\pgfqpoint{2.196599in}{4.749374in}}%
\pgfpathlineto{\pgfqpoint{2.229421in}{4.772244in}}%
\pgfpathlineto{\pgfqpoint{2.264506in}{4.675460in}}%
\pgfpathlineto{\pgfqpoint{2.256262in}{4.733942in}}%
\pgfpathlineto{\pgfqpoint{2.250318in}{4.672254in}}%
\pgfpathlineto{\pgfqpoint{2.217368in}{4.661758in}}%
\pgfpathlineto{\pgfqpoint{2.186046in}{4.568146in}}%
\pgfpathclose%
\pgfusepath{fill}%
\end{pgfscope}%
\begin{pgfscope}%
\pgfpathrectangle{\pgfqpoint{1.020000in}{0.880000in}}{\pgfqpoint{6.160000in}{6.160000in}}%
\pgfusepath{clip}%
\pgfsetbuttcap%
\pgfsetroundjoin%
\definecolor{currentfill}{rgb}{0.967711,0.662973,0.544323}%
\pgfsetfillcolor{currentfill}%
\pgfsetlinewidth{0.000000pt}%
\definecolor{currentstroke}{rgb}{0.000000,0.000000,0.000000}%
\pgfsetstrokecolor{currentstroke}%
\pgfsetdash{}{0pt}%
\pgfpathmoveto{\pgfqpoint{3.085901in}{4.852833in}}%
\pgfpathlineto{\pgfqpoint{3.092447in}{4.986850in}}%
\pgfpathlineto{\pgfqpoint{3.101800in}{4.861007in}}%
\pgfpathlineto{\pgfqpoint{3.135785in}{4.765426in}}%
\pgfpathlineto{\pgfqpoint{3.167993in}{4.839231in}}%
\pgfpathlineto{\pgfqpoint{3.160965in}{4.736670in}}%
\pgfpathlineto{\pgfqpoint{3.152845in}{4.744694in}}%
\pgfpathlineto{\pgfqpoint{3.119505in}{4.788721in}}%
\pgfpathlineto{\pgfqpoint{3.085901in}{4.852833in}}%
\pgfpathclose%
\pgfusepath{fill}%
\end{pgfscope}%
\begin{pgfscope}%
\pgfpathrectangle{\pgfqpoint{1.020000in}{0.880000in}}{\pgfqpoint{6.160000in}{6.160000in}}%
\pgfusepath{clip}%
\pgfsetbuttcap%
\pgfsetroundjoin%
\definecolor{currentfill}{rgb}{0.667253,0.779176,0.992959}%
\pgfsetfillcolor{currentfill}%
\pgfsetlinewidth{0.000000pt}%
\definecolor{currentstroke}{rgb}{0.000000,0.000000,0.000000}%
\pgfsetstrokecolor{currentstroke}%
\pgfsetdash{}{0pt}%
\pgfpathmoveto{\pgfqpoint{4.079800in}{3.956769in}}%
\pgfpathlineto{\pgfqpoint{4.088943in}{3.807609in}}%
\pgfpathlineto{\pgfqpoint{4.097825in}{3.957237in}}%
\pgfpathlineto{\pgfqpoint{4.130660in}{3.726890in}}%
\pgfpathlineto{\pgfqpoint{4.163248in}{3.892026in}}%
\pgfpathlineto{\pgfqpoint{4.154120in}{4.060000in}}%
\pgfpathlineto{\pgfqpoint{4.145099in}{3.917455in}}%
\pgfpathlineto{\pgfqpoint{4.112423in}{3.997511in}}%
\pgfpathlineto{\pgfqpoint{4.079800in}{3.956769in}}%
\pgfpathclose%
\pgfusepath{fill}%
\end{pgfscope}%
\begin{pgfscope}%
\pgfpathrectangle{\pgfqpoint{1.020000in}{0.880000in}}{\pgfqpoint{6.160000in}{6.160000in}}%
\pgfusepath{clip}%
\pgfsetbuttcap%
\pgfsetroundjoin%
\definecolor{currentfill}{rgb}{0.373552,0.497499,0.909467}%
\pgfsetfillcolor{currentfill}%
\pgfsetlinewidth{0.000000pt}%
\definecolor{currentstroke}{rgb}{0.000000,0.000000,0.000000}%
\pgfsetstrokecolor{currentstroke}%
\pgfsetdash{}{0pt}%
\pgfpathmoveto{\pgfqpoint{5.248845in}{3.188768in}}%
\pgfpathlineto{\pgfqpoint{5.261291in}{3.397323in}}%
\pgfpathlineto{\pgfqpoint{5.271305in}{3.376630in}}%
\pgfpathlineto{\pgfqpoint{5.304950in}{3.491538in}}%
\pgfpathlineto{\pgfqpoint{5.335163in}{3.304097in}}%
\pgfpathlineto{\pgfqpoint{5.327445in}{3.529248in}}%
\pgfpathlineto{\pgfqpoint{5.315782in}{3.408210in}}%
\pgfpathlineto{\pgfqpoint{5.283491in}{3.407853in}}%
\pgfpathlineto{\pgfqpoint{5.248845in}{3.188768in}}%
\pgfpathclose%
\pgfusepath{fill}%
\end{pgfscope}%
\begin{pgfscope}%
\pgfpathrectangle{\pgfqpoint{1.020000in}{0.880000in}}{\pgfqpoint{6.160000in}{6.160000in}}%
\pgfusepath{clip}%
\pgfsetbuttcap%
\pgfsetroundjoin%
\definecolor{currentfill}{rgb}{0.961645,0.758029,0.661782}%
\pgfsetfillcolor{currentfill}%
\pgfsetlinewidth{0.000000pt}%
\definecolor{currentstroke}{rgb}{0.000000,0.000000,0.000000}%
\pgfsetstrokecolor{currentstroke}%
\pgfsetdash{}{0pt}%
\pgfpathmoveto{\pgfqpoint{3.383415in}{4.657107in}}%
\pgfpathlineto{\pgfqpoint{3.390610in}{4.796795in}}%
\pgfpathlineto{\pgfqpoint{3.399463in}{4.727299in}}%
\pgfpathlineto{\pgfqpoint{3.432866in}{4.663365in}}%
\pgfpathlineto{\pgfqpoint{3.466082in}{4.617073in}}%
\pgfpathlineto{\pgfqpoint{3.457980in}{4.580648in}}%
\pgfpathlineto{\pgfqpoint{3.450100in}{4.516808in}}%
\pgfpathlineto{\pgfqpoint{3.416618in}{4.609616in}}%
\pgfpathlineto{\pgfqpoint{3.383415in}{4.657107in}}%
\pgfpathclose%
\pgfusepath{fill}%
\end{pgfscope}%
\begin{pgfscope}%
\pgfpathrectangle{\pgfqpoint{1.020000in}{0.880000in}}{\pgfqpoint{6.160000in}{6.160000in}}%
\pgfusepath{clip}%
\pgfsetbuttcap%
\pgfsetroundjoin%
\definecolor{currentfill}{rgb}{0.956653,0.598034,0.477302}%
\pgfsetfillcolor{currentfill}%
\pgfsetlinewidth{0.000000pt}%
\definecolor{currentstroke}{rgb}{0.000000,0.000000,0.000000}%
\pgfsetstrokecolor{currentstroke}%
\pgfsetdash{}{0pt}%
\pgfpathmoveto{\pgfqpoint{2.937638in}{4.908320in}}%
\pgfpathlineto{\pgfqpoint{2.945372in}{4.916229in}}%
\pgfpathlineto{\pgfqpoint{2.953170in}{4.920339in}}%
\pgfpathlineto{\pgfqpoint{2.986031in}{4.934529in}}%
\pgfpathlineto{\pgfqpoint{3.018196in}{5.008574in}}%
\pgfpathlineto{\pgfqpoint{3.011511in}{4.901851in}}%
\pgfpathlineto{\pgfqpoint{3.003648in}{4.899276in}}%
\pgfpathlineto{\pgfqpoint{2.970382in}{4.926579in}}%
\pgfpathlineto{\pgfqpoint{2.937638in}{4.908320in}}%
\pgfpathclose%
\pgfusepath{fill}%
\end{pgfscope}%
\begin{pgfscope}%
\pgfpathrectangle{\pgfqpoint{1.020000in}{0.880000in}}{\pgfqpoint{6.160000in}{6.160000in}}%
\pgfusepath{clip}%
\pgfsetbuttcap%
\pgfsetroundjoin%
\definecolor{currentfill}{rgb}{0.348323,0.465711,0.888346}%
\pgfsetfillcolor{currentfill}%
\pgfsetlinewidth{0.000000pt}%
\definecolor{currentstroke}{rgb}{0.000000,0.000000,0.000000}%
\pgfsetstrokecolor{currentstroke}%
\pgfsetdash{}{0pt}%
\pgfpathmoveto{\pgfqpoint{5.187095in}{3.465310in}}%
\pgfpathlineto{\pgfqpoint{5.197601in}{3.499099in}}%
\pgfpathlineto{\pgfqpoint{5.205209in}{3.247091in}}%
\pgfpathlineto{\pgfqpoint{5.237462in}{3.237674in}}%
\pgfpathlineto{\pgfqpoint{5.271305in}{3.376630in}}%
\pgfpathlineto{\pgfqpoint{5.261291in}{3.397323in}}%
\pgfpathlineto{\pgfqpoint{5.248845in}{3.188768in}}%
\pgfpathlineto{\pgfqpoint{5.218074in}{3.331850in}}%
\pgfpathlineto{\pgfqpoint{5.187095in}{3.465310in}}%
\pgfpathclose%
\pgfusepath{fill}%
\end{pgfscope}%
\begin{pgfscope}%
\pgfpathrectangle{\pgfqpoint{1.020000in}{0.880000in}}{\pgfqpoint{6.160000in}{6.160000in}}%
\pgfusepath{clip}%
\pgfsetbuttcap%
\pgfsetroundjoin%
\definecolor{currentfill}{rgb}{0.521696,0.659599,0.987736}%
\pgfsetfillcolor{currentfill}%
\pgfsetlinewidth{0.000000pt}%
\definecolor{currentstroke}{rgb}{0.000000,0.000000,0.000000}%
\pgfsetstrokecolor{currentstroke}%
\pgfsetdash{}{0pt}%
\pgfpathmoveto{\pgfqpoint{4.526383in}{3.798334in}}%
\pgfpathlineto{\pgfqpoint{4.535652in}{3.738987in}}%
\pgfpathlineto{\pgfqpoint{4.544693in}{3.614220in}}%
\pgfpathlineto{\pgfqpoint{4.577133in}{3.582824in}}%
\pgfpathlineto{\pgfqpoint{4.609129in}{3.456667in}}%
\pgfpathlineto{\pgfqpoint{4.600510in}{3.674652in}}%
\pgfpathlineto{\pgfqpoint{4.591149in}{3.715512in}}%
\pgfpathlineto{\pgfqpoint{4.558379in}{3.647446in}}%
\pgfpathlineto{\pgfqpoint{4.526383in}{3.798334in}}%
\pgfpathclose%
\pgfusepath{fill}%
\end{pgfscope}%
\begin{pgfscope}%
\pgfpathrectangle{\pgfqpoint{1.020000in}{0.880000in}}{\pgfqpoint{6.160000in}{6.160000in}}%
\pgfusepath{clip}%
\pgfsetbuttcap%
\pgfsetroundjoin%
\definecolor{currentfill}{rgb}{0.414801,0.546874,0.939088}%
\pgfsetfillcolor{currentfill}%
\pgfsetlinewidth{0.000000pt}%
\definecolor{currentstroke}{rgb}{0.000000,0.000000,0.000000}%
\pgfsetstrokecolor{currentstroke}%
\pgfsetdash{}{0pt}%
\pgfpathmoveto{\pgfqpoint{4.823469in}{3.502336in}}%
\pgfpathlineto{\pgfqpoint{4.832323in}{3.360324in}}%
\pgfpathlineto{\pgfqpoint{4.843197in}{3.525831in}}%
\pgfpathlineto{\pgfqpoint{4.874666in}{3.375718in}}%
\pgfpathlineto{\pgfqpoint{4.907048in}{3.367733in}}%
\pgfpathlineto{\pgfqpoint{4.897708in}{3.438493in}}%
\pgfpathlineto{\pgfqpoint{4.888441in}{3.518663in}}%
\pgfpathlineto{\pgfqpoint{4.857057in}{3.673404in}}%
\pgfpathlineto{\pgfqpoint{4.823469in}{3.502336in}}%
\pgfpathclose%
\pgfusepath{fill}%
\end{pgfscope}%
\begin{pgfscope}%
\pgfpathrectangle{\pgfqpoint{1.020000in}{0.880000in}}{\pgfqpoint{6.160000in}{6.160000in}}%
\pgfusepath{clip}%
\pgfsetbuttcap%
\pgfsetroundjoin%
\definecolor{currentfill}{rgb}{0.963772,0.749086,0.649420}%
\pgfsetfillcolor{currentfill}%
\pgfsetlinewidth{0.000000pt}%
\definecolor{currentstroke}{rgb}{0.000000,0.000000,0.000000}%
\pgfsetstrokecolor{currentstroke}%
\pgfsetdash{}{0pt}%
\pgfpathmoveto{\pgfqpoint{2.118585in}{4.623360in}}%
\pgfpathlineto{\pgfqpoint{2.124050in}{4.698132in}}%
\pgfpathlineto{\pgfqpoint{2.131635in}{4.669536in}}%
\pgfpathlineto{\pgfqpoint{2.164841in}{4.672954in}}%
\pgfpathlineto{\pgfqpoint{2.196599in}{4.749374in}}%
\pgfpathlineto{\pgfqpoint{2.190748in}{4.687296in}}%
\pgfpathlineto{\pgfqpoint{2.186046in}{4.568146in}}%
\pgfpathlineto{\pgfqpoint{2.152354in}{4.594765in}}%
\pgfpathlineto{\pgfqpoint{2.118585in}{4.623360in}}%
\pgfpathclose%
\pgfusepath{fill}%
\end{pgfscope}%
\begin{pgfscope}%
\pgfpathrectangle{\pgfqpoint{1.020000in}{0.880000in}}{\pgfqpoint{6.160000in}{6.160000in}}%
\pgfusepath{clip}%
\pgfsetbuttcap%
\pgfsetroundjoin%
\definecolor{currentfill}{rgb}{0.399231,0.528528,0.928459}%
\pgfsetfillcolor{currentfill}%
\pgfsetlinewidth{0.000000pt}%
\definecolor{currentstroke}{rgb}{0.000000,0.000000,0.000000}%
\pgfsetstrokecolor{currentstroke}%
\pgfsetdash{}{0pt}%
\pgfpathmoveto{\pgfqpoint{5.530097in}{3.411054in}}%
\pgfpathlineto{\pgfqpoint{5.540921in}{3.434423in}}%
\pgfpathlineto{\pgfqpoint{5.550513in}{3.365810in}}%
\pgfpathlineto{\pgfqpoint{5.583286in}{3.402146in}}%
\pgfpathlineto{\pgfqpoint{5.617915in}{3.567633in}}%
\pgfpathlineto{\pgfqpoint{5.604876in}{3.396599in}}%
\pgfpathlineto{\pgfqpoint{5.596161in}{3.527700in}}%
\pgfpathlineto{\pgfqpoint{5.561956in}{3.385014in}}%
\pgfpathlineto{\pgfqpoint{5.530097in}{3.411054in}}%
\pgfpathclose%
\pgfusepath{fill}%
\end{pgfscope}%
\begin{pgfscope}%
\pgfpathrectangle{\pgfqpoint{1.020000in}{0.880000in}}{\pgfqpoint{6.160000in}{6.160000in}}%
\pgfusepath{clip}%
\pgfsetbuttcap%
\pgfsetroundjoin%
\definecolor{currentfill}{rgb}{0.635474,0.756714,0.998297}%
\pgfsetfillcolor{currentfill}%
\pgfsetlinewidth{0.000000pt}%
\definecolor{currentstroke}{rgb}{0.000000,0.000000,0.000000}%
\pgfsetstrokecolor{currentstroke}%
\pgfsetdash{}{0pt}%
\pgfpathmoveto{\pgfqpoint{4.228624in}{4.080143in}}%
\pgfpathlineto{\pgfqpoint{4.237773in}{4.028423in}}%
\pgfpathlineto{\pgfqpoint{4.246828in}{3.812672in}}%
\pgfpathlineto{\pgfqpoint{4.279503in}{3.833971in}}%
\pgfpathlineto{\pgfqpoint{4.311783in}{3.552036in}}%
\pgfpathlineto{\pgfqpoint{4.302934in}{3.854693in}}%
\pgfpathlineto{\pgfqpoint{4.293565in}{3.718682in}}%
\pgfpathlineto{\pgfqpoint{4.261179in}{3.920638in}}%
\pgfpathlineto{\pgfqpoint{4.228624in}{4.080143in}}%
\pgfpathclose%
\pgfusepath{fill}%
\end{pgfscope}%
\begin{pgfscope}%
\pgfpathrectangle{\pgfqpoint{1.020000in}{0.880000in}}{\pgfqpoint{6.160000in}{6.160000in}}%
\pgfusepath{clip}%
\pgfsetbuttcap%
\pgfsetroundjoin%
\definecolor{currentfill}{rgb}{0.738826,0.822572,0.968261}%
\pgfsetfillcolor{currentfill}%
\pgfsetlinewidth{0.000000pt}%
\definecolor{currentstroke}{rgb}{0.000000,0.000000,0.000000}%
\pgfsetstrokecolor{currentstroke}%
\pgfsetdash{}{0pt}%
\pgfpathmoveto{\pgfqpoint{3.931107in}{4.034644in}}%
\pgfpathlineto{\pgfqpoint{3.939555in}{4.200192in}}%
\pgfpathlineto{\pgfqpoint{3.948902in}{4.000686in}}%
\pgfpathlineto{\pgfqpoint{3.981745in}{3.949903in}}%
\pgfpathlineto{\pgfqpoint{4.014342in}{4.014640in}}%
\pgfpathlineto{\pgfqpoint{4.005566in}{3.929049in}}%
\pgfpathlineto{\pgfqpoint{3.996411in}{4.065053in}}%
\pgfpathlineto{\pgfqpoint{3.963597in}{4.126262in}}%
\pgfpathlineto{\pgfqpoint{3.931107in}{4.034644in}}%
\pgfpathclose%
\pgfusepath{fill}%
\end{pgfscope}%
\begin{pgfscope}%
\pgfpathrectangle{\pgfqpoint{1.020000in}{0.880000in}}{\pgfqpoint{6.160000in}{6.160000in}}%
\pgfusepath{clip}%
\pgfsetbuttcap%
\pgfsetroundjoin%
\definecolor{currentfill}{rgb}{0.963806,0.634188,0.513721}%
\pgfsetfillcolor{currentfill}%
\pgfsetlinewidth{0.000000pt}%
\definecolor{currentstroke}{rgb}{0.000000,0.000000,0.000000}%
\pgfsetstrokecolor{currentstroke}%
\pgfsetdash{}{0pt}%
\pgfpathmoveto{\pgfqpoint{2.592450in}{4.879900in}}%
\pgfpathlineto{\pgfqpoint{2.599598in}{4.901191in}}%
\pgfpathlineto{\pgfqpoint{2.607700in}{4.862265in}}%
\pgfpathlineto{\pgfqpoint{2.639743in}{4.934116in}}%
\pgfpathlineto{\pgfqpoint{2.671662in}{5.016152in}}%
\pgfpathlineto{\pgfqpoint{2.666185in}{4.876219in}}%
\pgfpathlineto{\pgfqpoint{2.658505in}{4.885270in}}%
\pgfpathlineto{\pgfqpoint{2.629654in}{4.611124in}}%
\pgfpathlineto{\pgfqpoint{2.592450in}{4.879900in}}%
\pgfpathclose%
\pgfusepath{fill}%
\end{pgfscope}%
\begin{pgfscope}%
\pgfpathrectangle{\pgfqpoint{1.020000in}{0.880000in}}{\pgfqpoint{6.160000in}{6.160000in}}%
\pgfusepath{clip}%
\pgfsetbuttcap%
\pgfsetroundjoin%
\definecolor{currentfill}{rgb}{0.954566,0.779055,0.692531}%
\pgfsetfillcolor{currentfill}%
\pgfsetlinewidth{0.000000pt}%
\definecolor{currentstroke}{rgb}{0.000000,0.000000,0.000000}%
\pgfsetstrokecolor{currentstroke}%
\pgfsetdash{}{0pt}%
\pgfpathmoveto{\pgfqpoint{3.531145in}{4.714599in}}%
\pgfpathlineto{\pgfqpoint{3.539851in}{4.674568in}}%
\pgfpathlineto{\pgfqpoint{3.548943in}{4.574730in}}%
\pgfpathlineto{\pgfqpoint{3.581731in}{4.587970in}}%
\pgfpathlineto{\pgfqpoint{3.614791in}{4.550764in}}%
\pgfpathlineto{\pgfqpoint{3.606292in}{4.547572in}}%
\pgfpathlineto{\pgfqpoint{3.597664in}{4.569579in}}%
\pgfpathlineto{\pgfqpoint{3.565322in}{4.497369in}}%
\pgfpathlineto{\pgfqpoint{3.531145in}{4.714599in}}%
\pgfpathclose%
\pgfusepath{fill}%
\end{pgfscope}%
\begin{pgfscope}%
\pgfpathrectangle{\pgfqpoint{1.020000in}{0.880000in}}{\pgfqpoint{6.160000in}{6.160000in}}%
\pgfusepath{clip}%
\pgfsetbuttcap%
\pgfsetroundjoin%
\definecolor{currentfill}{rgb}{0.967711,0.662973,0.544323}%
\pgfsetfillcolor{currentfill}%
\pgfsetlinewidth{0.000000pt}%
\definecolor{currentstroke}{rgb}{0.000000,0.000000,0.000000}%
\pgfsetstrokecolor{currentstroke}%
\pgfsetdash{}{0pt}%
\pgfpathmoveto{\pgfqpoint{2.526926in}{4.837734in}}%
\pgfpathlineto{\pgfqpoint{2.536176in}{4.724850in}}%
\pgfpathlineto{\pgfqpoint{2.542402in}{4.798644in}}%
\pgfpathlineto{\pgfqpoint{2.572443in}{4.994832in}}%
\pgfpathlineto{\pgfqpoint{2.607700in}{4.862265in}}%
\pgfpathlineto{\pgfqpoint{2.599598in}{4.901191in}}%
\pgfpathlineto{\pgfqpoint{2.592450in}{4.879900in}}%
\pgfpathlineto{\pgfqpoint{2.564644in}{4.549360in}}%
\pgfpathlineto{\pgfqpoint{2.526926in}{4.837734in}}%
\pgfpathclose%
\pgfusepath{fill}%
\end{pgfscope}%
\begin{pgfscope}%
\pgfpathrectangle{\pgfqpoint{1.020000in}{0.880000in}}{\pgfqpoint{6.160000in}{6.160000in}}%
\pgfusepath{clip}%
\pgfsetbuttcap%
\pgfsetroundjoin%
\definecolor{currentfill}{rgb}{0.378598,0.503856,0.913692}%
\pgfsetfillcolor{currentfill}%
\pgfsetlinewidth{0.000000pt}%
\definecolor{currentstroke}{rgb}{0.000000,0.000000,0.000000}%
\pgfsetstrokecolor{currentstroke}%
\pgfsetdash{}{0pt}%
\pgfpathmoveto{\pgfqpoint{5.465380in}{3.393545in}}%
\pgfpathlineto{\pgfqpoint{5.476951in}{3.480653in}}%
\pgfpathlineto{\pgfqpoint{5.486612in}{3.418257in}}%
\pgfpathlineto{\pgfqpoint{5.517830in}{3.335861in}}%
\pgfpathlineto{\pgfqpoint{5.550513in}{3.365810in}}%
\pgfpathlineto{\pgfqpoint{5.540921in}{3.434423in}}%
\pgfpathlineto{\pgfqpoint{5.530097in}{3.411054in}}%
\pgfpathlineto{\pgfqpoint{5.497231in}{3.363188in}}%
\pgfpathlineto{\pgfqpoint{5.465380in}{3.393545in}}%
\pgfpathclose%
\pgfusepath{fill}%
\end{pgfscope}%
\begin{pgfscope}%
\pgfpathrectangle{\pgfqpoint{1.020000in}{0.880000in}}{\pgfqpoint{6.160000in}{6.160000in}}%
\pgfusepath{clip}%
\pgfsetbuttcap%
\pgfsetroundjoin%
\definecolor{currentfill}{rgb}{0.592356,0.722792,0.999434}%
\pgfsetfillcolor{currentfill}%
\pgfsetlinewidth{0.000000pt}%
\definecolor{currentstroke}{rgb}{0.000000,0.000000,0.000000}%
\pgfsetstrokecolor{currentstroke}%
\pgfsetdash{}{0pt}%
\pgfpathmoveto{\pgfqpoint{4.377157in}{3.722369in}}%
\pgfpathlineto{\pgfqpoint{4.386698in}{3.837745in}}%
\pgfpathlineto{\pgfqpoint{4.396108in}{3.872319in}}%
\pgfpathlineto{\pgfqpoint{4.428673in}{3.833403in}}%
\pgfpathlineto{\pgfqpoint{4.461203in}{3.797658in}}%
\pgfpathlineto{\pgfqpoint{4.451730in}{3.768835in}}%
\pgfpathlineto{\pgfqpoint{4.442236in}{3.722048in}}%
\pgfpathlineto{\pgfqpoint{4.409596in}{3.673453in}}%
\pgfpathlineto{\pgfqpoint{4.377157in}{3.722369in}}%
\pgfpathclose%
\pgfusepath{fill}%
\end{pgfscope}%
\begin{pgfscope}%
\pgfpathrectangle{\pgfqpoint{1.020000in}{0.880000in}}{\pgfqpoint{6.160000in}{6.160000in}}%
\pgfusepath{clip}%
\pgfsetbuttcap%
\pgfsetroundjoin%
\definecolor{currentfill}{rgb}{0.919376,0.831273,0.782874}%
\pgfsetfillcolor{currentfill}%
\pgfsetlinewidth{0.000000pt}%
\definecolor{currentstroke}{rgb}{0.000000,0.000000,0.000000}%
\pgfsetstrokecolor{currentstroke}%
\pgfsetdash{}{0pt}%
\pgfpathmoveto{\pgfqpoint{3.614791in}{4.550764in}}%
\pgfpathlineto{\pgfqpoint{3.623625in}{4.496904in}}%
\pgfpathlineto{\pgfqpoint{3.632912in}{4.361586in}}%
\pgfpathlineto{\pgfqpoint{3.664708in}{4.561540in}}%
\pgfpathlineto{\pgfqpoint{3.697615in}{4.553397in}}%
\pgfpathlineto{\pgfqpoint{3.690288in}{4.289668in}}%
\pgfpathlineto{\pgfqpoint{3.681477in}{4.337037in}}%
\pgfpathlineto{\pgfqpoint{3.648204in}{4.438949in}}%
\pgfpathlineto{\pgfqpoint{3.614791in}{4.550764in}}%
\pgfpathclose%
\pgfusepath{fill}%
\end{pgfscope}%
\begin{pgfscope}%
\pgfpathrectangle{\pgfqpoint{1.020000in}{0.880000in}}{\pgfqpoint{6.160000in}{6.160000in}}%
\pgfusepath{clip}%
\pgfsetbuttcap%
\pgfsetroundjoin%
\definecolor{currentfill}{rgb}{0.969289,0.684982,0.568975}%
\pgfsetfillcolor{currentfill}%
\pgfsetlinewidth{0.000000pt}%
\definecolor{currentstroke}{rgb}{0.000000,0.000000,0.000000}%
\pgfsetstrokecolor{currentstroke}%
\pgfsetdash{}{0pt}%
\pgfpathmoveto{\pgfqpoint{2.461256in}{4.803049in}}%
\pgfpathlineto{\pgfqpoint{2.471150in}{4.652466in}}%
\pgfpathlineto{\pgfqpoint{2.475609in}{4.824261in}}%
\pgfpathlineto{\pgfqpoint{2.510617in}{4.714697in}}%
\pgfpathlineto{\pgfqpoint{2.542402in}{4.798644in}}%
\pgfpathlineto{\pgfqpoint{2.536176in}{4.724850in}}%
\pgfpathlineto{\pgfqpoint{2.526926in}{4.837734in}}%
\pgfpathlineto{\pgfqpoint{2.493262in}{4.870363in}}%
\pgfpathlineto{\pgfqpoint{2.461256in}{4.803049in}}%
\pgfpathclose%
\pgfusepath{fill}%
\end{pgfscope}%
\begin{pgfscope}%
\pgfpathrectangle{\pgfqpoint{1.020000in}{0.880000in}}{\pgfqpoint{6.160000in}{6.160000in}}%
\pgfusepath{clip}%
\pgfsetbuttcap%
\pgfsetroundjoin%
\definecolor{currentfill}{rgb}{0.394042,0.522413,0.924916}%
\pgfsetfillcolor{currentfill}%
\pgfsetlinewidth{0.000000pt}%
\definecolor{currentstroke}{rgb}{0.000000,0.000000,0.000000}%
\pgfsetstrokecolor{currentstroke}%
\pgfsetdash{}{0pt}%
\pgfpathmoveto{\pgfqpoint{4.972551in}{3.454508in}}%
\pgfpathlineto{\pgfqpoint{4.982506in}{3.456807in}}%
\pgfpathlineto{\pgfqpoint{4.992050in}{3.405782in}}%
\pgfpathlineto{\pgfqpoint{5.023579in}{3.295791in}}%
\pgfpathlineto{\pgfqpoint{5.055969in}{3.294584in}}%
\pgfpathlineto{\pgfqpoint{5.047336in}{3.453119in}}%
\pgfpathlineto{\pgfqpoint{5.038584in}{3.598957in}}%
\pgfpathlineto{\pgfqpoint{5.004954in}{3.453751in}}%
\pgfpathlineto{\pgfqpoint{4.972551in}{3.454508in}}%
\pgfpathclose%
\pgfusepath{fill}%
\end{pgfscope}%
\begin{pgfscope}%
\pgfpathrectangle{\pgfqpoint{1.020000in}{0.880000in}}{\pgfqpoint{6.160000in}{6.160000in}}%
\pgfusepath{clip}%
\pgfsetbuttcap%
\pgfsetroundjoin%
\definecolor{currentfill}{rgb}{0.576051,0.708780,0.997755}%
\pgfsetfillcolor{currentfill}%
\pgfsetlinewidth{0.000000pt}%
\definecolor{currentstroke}{rgb}{0.000000,0.000000,0.000000}%
\pgfsetstrokecolor{currentstroke}%
\pgfsetdash{}{0pt}%
\pgfpathmoveto{\pgfqpoint{4.311783in}{3.552036in}}%
\pgfpathlineto{\pgfqpoint{4.320968in}{3.519979in}}%
\pgfpathlineto{\pgfqpoint{4.330609in}{3.789136in}}%
\pgfpathlineto{\pgfqpoint{4.363375in}{3.842269in}}%
\pgfpathlineto{\pgfqpoint{4.396108in}{3.872319in}}%
\pgfpathlineto{\pgfqpoint{4.386698in}{3.837745in}}%
\pgfpathlineto{\pgfqpoint{4.377157in}{3.722369in}}%
\pgfpathlineto{\pgfqpoint{4.344797in}{3.851737in}}%
\pgfpathlineto{\pgfqpoint{4.311783in}{3.552036in}}%
\pgfpathclose%
\pgfusepath{fill}%
\end{pgfscope}%
\begin{pgfscope}%
\pgfpathrectangle{\pgfqpoint{1.020000in}{0.880000in}}{\pgfqpoint{6.160000in}{6.160000in}}%
\pgfusepath{clip}%
\pgfsetbuttcap%
\pgfsetroundjoin%
\definecolor{currentfill}{rgb}{0.419991,0.552989,0.942630}%
\pgfsetfillcolor{currentfill}%
\pgfsetlinewidth{0.000000pt}%
\definecolor{currentstroke}{rgb}{0.000000,0.000000,0.000000}%
\pgfsetstrokecolor{currentstroke}%
\pgfsetdash{}{0pt}%
\pgfpathmoveto{\pgfqpoint{4.758884in}{3.562879in}}%
\pgfpathlineto{\pgfqpoint{4.768070in}{3.471373in}}%
\pgfpathlineto{\pgfqpoint{4.777925in}{3.492133in}}%
\pgfpathlineto{\pgfqpoint{4.809837in}{3.392920in}}%
\pgfpathlineto{\pgfqpoint{4.843197in}{3.525831in}}%
\pgfpathlineto{\pgfqpoint{4.832323in}{3.360324in}}%
\pgfpathlineto{\pgfqpoint{4.823469in}{3.502336in}}%
\pgfpathlineto{\pgfqpoint{4.790865in}{3.478111in}}%
\pgfpathlineto{\pgfqpoint{4.758884in}{3.562879in}}%
\pgfpathclose%
\pgfusepath{fill}%
\end{pgfscope}%
\begin{pgfscope}%
\pgfpathrectangle{\pgfqpoint{1.020000in}{0.880000in}}{\pgfqpoint{6.160000in}{6.160000in}}%
\pgfusepath{clip}%
\pgfsetbuttcap%
\pgfsetroundjoin%
\definecolor{currentfill}{rgb}{0.967711,0.662973,0.544323}%
\pgfsetfillcolor{currentfill}%
\pgfsetlinewidth{0.000000pt}%
\definecolor{currentstroke}{rgb}{0.000000,0.000000,0.000000}%
\pgfsetstrokecolor{currentstroke}%
\pgfsetdash{}{0pt}%
\pgfpathmoveto{\pgfqpoint{3.233714in}{4.856795in}}%
\pgfpathlineto{\pgfqpoint{3.241764in}{4.866849in}}%
\pgfpathlineto{\pgfqpoint{3.249751in}{4.885875in}}%
\pgfpathlineto{\pgfqpoint{3.282843in}{4.874627in}}%
\pgfpathlineto{\pgfqpoint{3.317159in}{4.717267in}}%
\pgfpathlineto{\pgfqpoint{3.308275in}{4.795004in}}%
\pgfpathlineto{\pgfqpoint{3.301518in}{4.629268in}}%
\pgfpathlineto{\pgfqpoint{3.266611in}{4.859328in}}%
\pgfpathlineto{\pgfqpoint{3.233714in}{4.856795in}}%
\pgfpathclose%
\pgfusepath{fill}%
\end{pgfscope}%
\begin{pgfscope}%
\pgfpathrectangle{\pgfqpoint{1.020000in}{0.880000in}}{\pgfqpoint{6.160000in}{6.160000in}}%
\pgfusepath{clip}%
\pgfsetbuttcap%
\pgfsetroundjoin%
\definecolor{currentfill}{rgb}{0.373552,0.497499,0.909467}%
\pgfsetfillcolor{currentfill}%
\pgfsetlinewidth{0.000000pt}%
\definecolor{currentstroke}{rgb}{0.000000,0.000000,0.000000}%
\pgfsetstrokecolor{currentstroke}%
\pgfsetdash{}{0pt}%
\pgfpathmoveto{\pgfqpoint{4.907048in}{3.367733in}}%
\pgfpathlineto{\pgfqpoint{4.916864in}{3.361003in}}%
\pgfpathlineto{\pgfqpoint{4.926325in}{3.303913in}}%
\pgfpathlineto{\pgfqpoint{4.959364in}{3.378526in}}%
\pgfpathlineto{\pgfqpoint{4.992050in}{3.405782in}}%
\pgfpathlineto{\pgfqpoint{4.982506in}{3.456807in}}%
\pgfpathlineto{\pgfqpoint{4.972551in}{3.454508in}}%
\pgfpathlineto{\pgfqpoint{4.939254in}{3.339624in}}%
\pgfpathlineto{\pgfqpoint{4.907048in}{3.367733in}}%
\pgfpathclose%
\pgfusepath{fill}%
\end{pgfscope}%
\begin{pgfscope}%
\pgfpathrectangle{\pgfqpoint{1.020000in}{0.880000in}}{\pgfqpoint{6.160000in}{6.160000in}}%
\pgfusepath{clip}%
\pgfsetbuttcap%
\pgfsetroundjoin%
\definecolor{currentfill}{rgb}{0.373552,0.497499,0.909467}%
\pgfsetfillcolor{currentfill}%
\pgfsetlinewidth{0.000000pt}%
\definecolor{currentstroke}{rgb}{0.000000,0.000000,0.000000}%
\pgfsetstrokecolor{currentstroke}%
\pgfsetdash{}{0pt}%
\pgfpathmoveto{\pgfqpoint{5.400068in}{3.331437in}}%
\pgfpathlineto{\pgfqpoint{5.411178in}{3.391125in}}%
\pgfpathlineto{\pgfqpoint{5.421736in}{3.403512in}}%
\pgfpathlineto{\pgfqpoint{5.453689in}{3.372160in}}%
\pgfpathlineto{\pgfqpoint{5.486612in}{3.418257in}}%
\pgfpathlineto{\pgfqpoint{5.476951in}{3.480653in}}%
\pgfpathlineto{\pgfqpoint{5.465380in}{3.393545in}}%
\pgfpathlineto{\pgfqpoint{5.431108in}{3.233246in}}%
\pgfpathlineto{\pgfqpoint{5.400068in}{3.331437in}}%
\pgfpathclose%
\pgfusepath{fill}%
\end{pgfscope}%
\begin{pgfscope}%
\pgfpathrectangle{\pgfqpoint{1.020000in}{0.880000in}}{\pgfqpoint{6.160000in}{6.160000in}}%
\pgfusepath{clip}%
\pgfsetbuttcap%
\pgfsetroundjoin%
\definecolor{currentfill}{rgb}{0.478462,0.616564,0.972721}%
\pgfsetfillcolor{currentfill}%
\pgfsetlinewidth{0.000000pt}%
\definecolor{currentstroke}{rgb}{0.000000,0.000000,0.000000}%
\pgfsetstrokecolor{currentstroke}%
\pgfsetdash{}{0pt}%
\pgfpathmoveto{\pgfqpoint{4.609129in}{3.456667in}}%
\pgfpathlineto{\pgfqpoint{4.619512in}{3.648893in}}%
\pgfpathlineto{\pgfqpoint{4.628570in}{3.530375in}}%
\pgfpathlineto{\pgfqpoint{4.661541in}{3.616450in}}%
\pgfpathlineto{\pgfqpoint{4.693088in}{3.415974in}}%
\pgfpathlineto{\pgfqpoint{4.684281in}{3.579269in}}%
\pgfpathlineto{\pgfqpoint{4.675046in}{3.657770in}}%
\pgfpathlineto{\pgfqpoint{4.642628in}{3.680123in}}%
\pgfpathlineto{\pgfqpoint{4.609129in}{3.456667in}}%
\pgfpathclose%
\pgfusepath{fill}%
\end{pgfscope}%
\begin{pgfscope}%
\pgfpathrectangle{\pgfqpoint{1.020000in}{0.880000in}}{\pgfqpoint{6.160000in}{6.160000in}}%
\pgfusepath{clip}%
\pgfsetbuttcap%
\pgfsetroundjoin%
\definecolor{currentfill}{rgb}{0.532568,0.669801,0.990393}%
\pgfsetfillcolor{currentfill}%
\pgfsetlinewidth{0.000000pt}%
\definecolor{currentstroke}{rgb}{0.000000,0.000000,0.000000}%
\pgfsetstrokecolor{currentstroke}%
\pgfsetdash{}{0pt}%
\pgfpathmoveto{\pgfqpoint{4.461203in}{3.797658in}}%
\pgfpathlineto{\pgfqpoint{4.470089in}{3.612998in}}%
\pgfpathlineto{\pgfqpoint{4.479105in}{3.477149in}}%
\pgfpathlineto{\pgfqpoint{4.512436in}{3.713186in}}%
\pgfpathlineto{\pgfqpoint{4.544693in}{3.614220in}}%
\pgfpathlineto{\pgfqpoint{4.535652in}{3.738987in}}%
\pgfpathlineto{\pgfqpoint{4.526383in}{3.798334in}}%
\pgfpathlineto{\pgfqpoint{4.493103in}{3.571954in}}%
\pgfpathlineto{\pgfqpoint{4.461203in}{3.797658in}}%
\pgfpathclose%
\pgfusepath{fill}%
\end{pgfscope}%
\begin{pgfscope}%
\pgfpathrectangle{\pgfqpoint{1.020000in}{0.880000in}}{\pgfqpoint{6.160000in}{6.160000in}}%
\pgfusepath{clip}%
\pgfsetbuttcap%
\pgfsetroundjoin%
\definecolor{currentfill}{rgb}{0.883687,0.856108,0.840258}%
\pgfsetfillcolor{currentfill}%
\pgfsetlinewidth{0.000000pt}%
\definecolor{currentstroke}{rgb}{0.000000,0.000000,0.000000}%
\pgfsetstrokecolor{currentstroke}%
\pgfsetdash{}{0pt}%
\pgfpathmoveto{\pgfqpoint{3.697615in}{4.553397in}}%
\pgfpathlineto{\pgfqpoint{3.707382in}{4.313214in}}%
\pgfpathlineto{\pgfqpoint{3.715732in}{4.374021in}}%
\pgfpathlineto{\pgfqpoint{3.748624in}{4.364344in}}%
\pgfpathlineto{\pgfqpoint{3.781690in}{4.304895in}}%
\pgfpathlineto{\pgfqpoint{3.773442in}{4.195467in}}%
\pgfpathlineto{\pgfqpoint{3.764475in}{4.267906in}}%
\pgfpathlineto{\pgfqpoint{3.731222in}{4.383653in}}%
\pgfpathlineto{\pgfqpoint{3.697615in}{4.553397in}}%
\pgfpathclose%
\pgfusepath{fill}%
\end{pgfscope}%
\begin{pgfscope}%
\pgfpathrectangle{\pgfqpoint{1.020000in}{0.880000in}}{\pgfqpoint{6.160000in}{6.160000in}}%
\pgfusepath{clip}%
\pgfsetbuttcap%
\pgfsetroundjoin%
\definecolor{currentfill}{rgb}{0.388852,0.516298,0.921373}%
\pgfsetfillcolor{currentfill}%
\pgfsetlinewidth{0.000000pt}%
\definecolor{currentstroke}{rgb}{0.000000,0.000000,0.000000}%
\pgfsetstrokecolor{currentstroke}%
\pgfsetdash{}{0pt}%
\pgfpathmoveto{\pgfqpoint{5.680996in}{3.468511in}}%
\pgfpathlineto{\pgfqpoint{5.690316in}{3.378401in}}%
\pgfpathlineto{\pgfqpoint{5.699752in}{3.295765in}}%
\pgfpathlineto{\pgfqpoint{5.732747in}{3.345195in}}%
\pgfpathlineto{\pgfqpoint{5.725379in}{3.562728in}}%
\pgfpathlineto{\pgfqpoint{5.712031in}{3.390849in}}%
\pgfpathlineto{\pgfqpoint{5.680996in}{3.468511in}}%
\pgfpathclose%
\pgfusepath{fill}%
\end{pgfscope}%
\begin{pgfscope}%
\pgfpathrectangle{\pgfqpoint{1.020000in}{0.880000in}}{\pgfqpoint{6.160000in}{6.160000in}}%
\pgfusepath{clip}%
\pgfsetbuttcap%
\pgfsetroundjoin%
\definecolor{currentfill}{rgb}{0.646113,0.764436,0.996868}%
\pgfsetfillcolor{currentfill}%
\pgfsetlinewidth{0.000000pt}%
\definecolor{currentstroke}{rgb}{0.000000,0.000000,0.000000}%
\pgfsetstrokecolor{currentstroke}%
\pgfsetdash{}{0pt}%
\pgfpathmoveto{\pgfqpoint{4.163248in}{3.892026in}}%
\pgfpathlineto{\pgfqpoint{4.172347in}{3.873266in}}%
\pgfpathlineto{\pgfqpoint{4.181465in}{3.711587in}}%
\pgfpathlineto{\pgfqpoint{4.214135in}{3.747950in}}%
\pgfpathlineto{\pgfqpoint{4.246828in}{3.812672in}}%
\pgfpathlineto{\pgfqpoint{4.237773in}{4.028423in}}%
\pgfpathlineto{\pgfqpoint{4.228624in}{4.080143in}}%
\pgfpathlineto{\pgfqpoint{4.195890in}{3.792265in}}%
\pgfpathlineto{\pgfqpoint{4.163248in}{3.892026in}}%
\pgfpathclose%
\pgfusepath{fill}%
\end{pgfscope}%
\begin{pgfscope}%
\pgfpathrectangle{\pgfqpoint{1.020000in}{0.880000in}}{\pgfqpoint{6.160000in}{6.160000in}}%
\pgfusepath{clip}%
\pgfsetbuttcap%
\pgfsetroundjoin%
\definecolor{currentfill}{rgb}{0.947654,0.565976,0.447478}%
\pgfsetfillcolor{currentfill}%
\pgfsetlinewidth{0.000000pt}%
\definecolor{currentstroke}{rgb}{0.000000,0.000000,0.000000}%
\pgfsetstrokecolor{currentstroke}%
\pgfsetdash{}{0pt}%
\pgfpathmoveto{\pgfqpoint{2.872194in}{4.866856in}}%
\pgfpathlineto{\pgfqpoint{2.877142in}{5.086078in}}%
\pgfpathlineto{\pgfqpoint{2.886134in}{4.992558in}}%
\pgfpathlineto{\pgfqpoint{2.919467in}{4.973036in}}%
\pgfpathlineto{\pgfqpoint{2.953170in}{4.920339in}}%
\pgfpathlineto{\pgfqpoint{2.945372in}{4.916229in}}%
\pgfpathlineto{\pgfqpoint{2.937638in}{4.908320in}}%
\pgfpathlineto{\pgfqpoint{2.903026in}{5.037828in}}%
\pgfpathlineto{\pgfqpoint{2.872194in}{4.866856in}}%
\pgfpathclose%
\pgfusepath{fill}%
\end{pgfscope}%
\begin{pgfscope}%
\pgfpathrectangle{\pgfqpoint{1.020000in}{0.880000in}}{\pgfqpoint{6.160000in}{6.160000in}}%
\pgfusepath{clip}%
\pgfsetbuttcap%
\pgfsetroundjoin%
\definecolor{currentfill}{rgb}{0.969683,0.690484,0.575138}%
\pgfsetfillcolor{currentfill}%
\pgfsetlinewidth{0.000000pt}%
\definecolor{currentstroke}{rgb}{0.000000,0.000000,0.000000}%
\pgfsetstrokecolor{currentstroke}%
\pgfsetdash{}{0pt}%
\pgfpathmoveto{\pgfqpoint{2.396660in}{4.705707in}}%
\pgfpathlineto{\pgfqpoint{2.401464in}{4.846435in}}%
\pgfpathlineto{\pgfqpoint{2.412504in}{4.632768in}}%
\pgfpathlineto{\pgfqpoint{2.443161in}{4.779410in}}%
\pgfpathlineto{\pgfqpoint{2.475609in}{4.824261in}}%
\pgfpathlineto{\pgfqpoint{2.471150in}{4.652466in}}%
\pgfpathlineto{\pgfqpoint{2.461256in}{4.803049in}}%
\pgfpathlineto{\pgfqpoint{2.426261in}{4.909780in}}%
\pgfpathlineto{\pgfqpoint{2.396660in}{4.705707in}}%
\pgfpathclose%
\pgfusepath{fill}%
\end{pgfscope}%
\begin{pgfscope}%
\pgfpathrectangle{\pgfqpoint{1.020000in}{0.880000in}}{\pgfqpoint{6.160000in}{6.160000in}}%
\pgfusepath{clip}%
\pgfsetbuttcap%
\pgfsetroundjoin%
\definecolor{currentfill}{rgb}{0.373552,0.497499,0.909467}%
\pgfsetfillcolor{currentfill}%
\pgfsetlinewidth{0.000000pt}%
\definecolor{currentstroke}{rgb}{0.000000,0.000000,0.000000}%
\pgfsetstrokecolor{currentstroke}%
\pgfsetdash{}{0pt}%
\pgfpathmoveto{\pgfqpoint{5.121884in}{3.419371in}}%
\pgfpathlineto{\pgfqpoint{5.131775in}{3.397576in}}%
\pgfpathlineto{\pgfqpoint{5.141764in}{3.384312in}}%
\pgfpathlineto{\pgfqpoint{5.173940in}{3.358252in}}%
\pgfpathlineto{\pgfqpoint{5.205209in}{3.247091in}}%
\pgfpathlineto{\pgfqpoint{5.197601in}{3.499099in}}%
\pgfpathlineto{\pgfqpoint{5.187095in}{3.465310in}}%
\pgfpathlineto{\pgfqpoint{5.152972in}{3.285881in}}%
\pgfpathlineto{\pgfqpoint{5.121884in}{3.419371in}}%
\pgfpathclose%
\pgfusepath{fill}%
\end{pgfscope}%
\begin{pgfscope}%
\pgfpathrectangle{\pgfqpoint{1.020000in}{0.880000in}}{\pgfqpoint{6.160000in}{6.160000in}}%
\pgfusepath{clip}%
\pgfsetbuttcap%
\pgfsetroundjoin%
\definecolor{currentfill}{rgb}{0.358415,0.478426,0.896795}%
\pgfsetfillcolor{currentfill}%
\pgfsetlinewidth{0.000000pt}%
\definecolor{currentstroke}{rgb}{0.000000,0.000000,0.000000}%
\pgfsetstrokecolor{currentstroke}%
\pgfsetdash{}{0pt}%
\pgfpathmoveto{\pgfqpoint{5.335163in}{3.304097in}}%
\pgfpathlineto{\pgfqpoint{5.347056in}{3.439988in}}%
\pgfpathlineto{\pgfqpoint{5.355934in}{3.314725in}}%
\pgfpathlineto{\pgfqpoint{5.389064in}{3.378506in}}%
\pgfpathlineto{\pgfqpoint{5.421736in}{3.403512in}}%
\pgfpathlineto{\pgfqpoint{5.411178in}{3.391125in}}%
\pgfpathlineto{\pgfqpoint{5.400068in}{3.331437in}}%
\pgfpathlineto{\pgfqpoint{5.366875in}{3.254897in}}%
\pgfpathlineto{\pgfqpoint{5.335163in}{3.304097in}}%
\pgfpathclose%
\pgfusepath{fill}%
\end{pgfscope}%
\begin{pgfscope}%
\pgfpathrectangle{\pgfqpoint{1.020000in}{0.880000in}}{\pgfqpoint{6.160000in}{6.160000in}}%
\pgfusepath{clip}%
\pgfsetbuttcap%
\pgfsetroundjoin%
\definecolor{currentfill}{rgb}{0.576051,0.708780,0.997755}%
\pgfsetfillcolor{currentfill}%
\pgfsetlinewidth{0.000000pt}%
\definecolor{currentstroke}{rgb}{0.000000,0.000000,0.000000}%
\pgfsetstrokecolor{currentstroke}%
\pgfsetdash{}{0pt}%
\pgfpathmoveto{\pgfqpoint{4.246828in}{3.812672in}}%
\pgfpathlineto{\pgfqpoint{4.256004in}{3.791848in}}%
\pgfpathlineto{\pgfqpoint{4.265237in}{3.822841in}}%
\pgfpathlineto{\pgfqpoint{4.297877in}{3.752649in}}%
\pgfpathlineto{\pgfqpoint{4.330609in}{3.789136in}}%
\pgfpathlineto{\pgfqpoint{4.320968in}{3.519979in}}%
\pgfpathlineto{\pgfqpoint{4.311783in}{3.552036in}}%
\pgfpathlineto{\pgfqpoint{4.279503in}{3.833971in}}%
\pgfpathlineto{\pgfqpoint{4.246828in}{3.812672in}}%
\pgfpathclose%
\pgfusepath{fill}%
\end{pgfscope}%
\begin{pgfscope}%
\pgfpathrectangle{\pgfqpoint{1.020000in}{0.880000in}}{\pgfqpoint{6.160000in}{6.160000in}}%
\pgfusepath{clip}%
\pgfsetbuttcap%
\pgfsetroundjoin%
\definecolor{currentfill}{rgb}{0.831148,0.859513,0.903110}%
\pgfsetfillcolor{currentfill}%
\pgfsetlinewidth{0.000000pt}%
\definecolor{currentstroke}{rgb}{0.000000,0.000000,0.000000}%
\pgfsetstrokecolor{currentstroke}%
\pgfsetdash{}{0pt}%
\pgfpathmoveto{\pgfqpoint{3.781690in}{4.304895in}}%
\pgfpathlineto{\pgfqpoint{3.790763in}{4.211756in}}%
\pgfpathlineto{\pgfqpoint{3.799562in}{4.191284in}}%
\pgfpathlineto{\pgfqpoint{3.832515in}{4.152111in}}%
\pgfpathlineto{\pgfqpoint{3.865413in}{4.116331in}}%
\pgfpathlineto{\pgfqpoint{3.856573in}{4.136113in}}%
\pgfpathlineto{\pgfqpoint{3.847760in}{4.152466in}}%
\pgfpathlineto{\pgfqpoint{3.813790in}{4.493682in}}%
\pgfpathlineto{\pgfqpoint{3.781690in}{4.304895in}}%
\pgfpathclose%
\pgfusepath{fill}%
\end{pgfscope}%
\begin{pgfscope}%
\pgfpathrectangle{\pgfqpoint{1.020000in}{0.880000in}}{\pgfqpoint{6.160000in}{6.160000in}}%
\pgfusepath{clip}%
\pgfsetbuttcap%
\pgfsetroundjoin%
\definecolor{currentfill}{rgb}{0.419991,0.552989,0.942630}%
\pgfsetfillcolor{currentfill}%
\pgfsetlinewidth{0.000000pt}%
\definecolor{currentstroke}{rgb}{0.000000,0.000000,0.000000}%
\pgfsetstrokecolor{currentstroke}%
\pgfsetdash{}{0pt}%
\pgfpathmoveto{\pgfqpoint{4.693088in}{3.415974in}}%
\pgfpathlineto{\pgfqpoint{4.703704in}{3.602969in}}%
\pgfpathlineto{\pgfqpoint{4.711659in}{3.276658in}}%
\pgfpathlineto{\pgfqpoint{4.745097in}{3.443229in}}%
\pgfpathlineto{\pgfqpoint{4.777925in}{3.492133in}}%
\pgfpathlineto{\pgfqpoint{4.768070in}{3.471373in}}%
\pgfpathlineto{\pgfqpoint{4.758884in}{3.562879in}}%
\pgfpathlineto{\pgfqpoint{4.725628in}{3.426135in}}%
\pgfpathlineto{\pgfqpoint{4.693088in}{3.415974in}}%
\pgfpathclose%
\pgfusepath{fill}%
\end{pgfscope}%
\begin{pgfscope}%
\pgfpathrectangle{\pgfqpoint{1.020000in}{0.880000in}}{\pgfqpoint{6.160000in}{6.160000in}}%
\pgfusepath{clip}%
\pgfsetbuttcap%
\pgfsetroundjoin%
\definecolor{currentfill}{rgb}{0.718985,0.811993,0.977656}%
\pgfsetfillcolor{currentfill}%
\pgfsetlinewidth{0.000000pt}%
\definecolor{currentstroke}{rgb}{0.000000,0.000000,0.000000}%
\pgfsetstrokecolor{currentstroke}%
\pgfsetdash{}{0pt}%
\pgfpathmoveto{\pgfqpoint{4.014342in}{4.014640in}}%
\pgfpathlineto{\pgfqpoint{4.023120in}{4.120778in}}%
\pgfpathlineto{\pgfqpoint{4.032223in}{4.035498in}}%
\pgfpathlineto{\pgfqpoint{4.065192in}{3.872388in}}%
\pgfpathlineto{\pgfqpoint{4.097825in}{3.957237in}}%
\pgfpathlineto{\pgfqpoint{4.088943in}{3.807609in}}%
\pgfpathlineto{\pgfqpoint{4.079800in}{3.956769in}}%
\pgfpathlineto{\pgfqpoint{4.046856in}{4.158697in}}%
\pgfpathlineto{\pgfqpoint{4.014342in}{4.014640in}}%
\pgfpathclose%
\pgfusepath{fill}%
\end{pgfscope}%
\begin{pgfscope}%
\pgfpathrectangle{\pgfqpoint{1.020000in}{0.880000in}}{\pgfqpoint{6.160000in}{6.160000in}}%
\pgfusepath{clip}%
\pgfsetbuttcap%
\pgfsetroundjoin%
\definecolor{currentfill}{rgb}{0.378598,0.503856,0.913692}%
\pgfsetfillcolor{currentfill}%
\pgfsetlinewidth{0.000000pt}%
\definecolor{currentstroke}{rgb}{0.000000,0.000000,0.000000}%
\pgfsetstrokecolor{currentstroke}%
\pgfsetdash{}{0pt}%
\pgfpathmoveto{\pgfqpoint{4.843197in}{3.525831in}}%
\pgfpathlineto{\pgfqpoint{4.852205in}{3.404554in}}%
\pgfpathlineto{\pgfqpoint{4.862230in}{3.433618in}}%
\pgfpathlineto{\pgfqpoint{4.893954in}{3.319017in}}%
\pgfpathlineto{\pgfqpoint{4.926325in}{3.303913in}}%
\pgfpathlineto{\pgfqpoint{4.916864in}{3.361003in}}%
\pgfpathlineto{\pgfqpoint{4.907048in}{3.367733in}}%
\pgfpathlineto{\pgfqpoint{4.874666in}{3.375718in}}%
\pgfpathlineto{\pgfqpoint{4.843197in}{3.525831in}}%
\pgfpathclose%
\pgfusepath{fill}%
\end{pgfscope}%
\begin{pgfscope}%
\pgfpathrectangle{\pgfqpoint{1.020000in}{0.880000in}}{\pgfqpoint{6.160000in}{6.160000in}}%
\pgfusepath{clip}%
\pgfsetbuttcap%
\pgfsetroundjoin%
\definecolor{currentfill}{rgb}{0.313946,0.420052,0.854993}%
\pgfsetfillcolor{currentfill}%
\pgfsetlinewidth{0.000000pt}%
\definecolor{currentstroke}{rgb}{0.000000,0.000000,0.000000}%
\pgfsetstrokecolor{currentstroke}%
\pgfsetdash{}{0pt}%
\pgfpathmoveto{\pgfqpoint{5.205209in}{3.247091in}}%
\pgfpathlineto{\pgfqpoint{5.215455in}{3.253113in}}%
\pgfpathlineto{\pgfqpoint{5.224766in}{3.167498in}}%
\pgfpathlineto{\pgfqpoint{5.257619in}{3.210621in}}%
\pgfpathlineto{\pgfqpoint{5.291042in}{3.304426in}}%
\pgfpathlineto{\pgfqpoint{5.280816in}{3.308584in}}%
\pgfpathlineto{\pgfqpoint{5.271305in}{3.376630in}}%
\pgfpathlineto{\pgfqpoint{5.237462in}{3.237674in}}%
\pgfpathlineto{\pgfqpoint{5.205209in}{3.247091in}}%
\pgfpathclose%
\pgfusepath{fill}%
\end{pgfscope}%
\begin{pgfscope}%
\pgfpathrectangle{\pgfqpoint{1.020000in}{0.880000in}}{\pgfqpoint{6.160000in}{6.160000in}}%
\pgfusepath{clip}%
\pgfsetbuttcap%
\pgfsetroundjoin%
\definecolor{currentfill}{rgb}{0.969851,0.695830,0.581312}%
\pgfsetfillcolor{currentfill}%
\pgfsetlinewidth{0.000000pt}%
\definecolor{currentstroke}{rgb}{0.000000,0.000000,0.000000}%
\pgfsetstrokecolor{currentstroke}%
\pgfsetdash{}{0pt}%
\pgfpathmoveto{\pgfqpoint{2.330421in}{4.700942in}}%
\pgfpathlineto{\pgfqpoint{2.340030in}{4.569483in}}%
\pgfpathlineto{\pgfqpoint{2.342374in}{4.837970in}}%
\pgfpathlineto{\pgfqpoint{2.374896in}{4.880484in}}%
\pgfpathlineto{\pgfqpoint{2.412504in}{4.632768in}}%
\pgfpathlineto{\pgfqpoint{2.401464in}{4.846435in}}%
\pgfpathlineto{\pgfqpoint{2.396660in}{4.705707in}}%
\pgfpathlineto{\pgfqpoint{2.361794in}{4.801033in}}%
\pgfpathlineto{\pgfqpoint{2.330421in}{4.700942in}}%
\pgfpathclose%
\pgfusepath{fill}%
\end{pgfscope}%
\begin{pgfscope}%
\pgfpathrectangle{\pgfqpoint{1.020000in}{0.880000in}}{\pgfqpoint{6.160000in}{6.160000in}}%
\pgfusepath{clip}%
\pgfsetbuttcap%
\pgfsetroundjoin%
\definecolor{currentfill}{rgb}{0.363461,0.484784,0.901019}%
\pgfsetfillcolor{currentfill}%
\pgfsetlinewidth{0.000000pt}%
\definecolor{currentstroke}{rgb}{0.000000,0.000000,0.000000}%
\pgfsetstrokecolor{currentstroke}%
\pgfsetdash{}{0pt}%
\pgfpathmoveto{\pgfqpoint{5.055969in}{3.294584in}}%
\pgfpathlineto{\pgfqpoint{5.066692in}{3.373716in}}%
\pgfpathlineto{\pgfqpoint{5.077014in}{3.403907in}}%
\pgfpathlineto{\pgfqpoint{5.108352in}{3.280846in}}%
\pgfpathlineto{\pgfqpoint{5.141764in}{3.384312in}}%
\pgfpathlineto{\pgfqpoint{5.131775in}{3.397576in}}%
\pgfpathlineto{\pgfqpoint{5.121884in}{3.419371in}}%
\pgfpathlineto{\pgfqpoint{5.088424in}{3.302739in}}%
\pgfpathlineto{\pgfqpoint{5.055969in}{3.294584in}}%
\pgfpathclose%
\pgfusepath{fill}%
\end{pgfscope}%
\begin{pgfscope}%
\pgfpathrectangle{\pgfqpoint{1.020000in}{0.880000in}}{\pgfqpoint{6.160000in}{6.160000in}}%
\pgfusepath{clip}%
\pgfsetbuttcap%
\pgfsetroundjoin%
\definecolor{currentfill}{rgb}{0.962708,0.753557,0.655601}%
\pgfsetfillcolor{currentfill}%
\pgfsetlinewidth{0.000000pt}%
\definecolor{currentstroke}{rgb}{0.000000,0.000000,0.000000}%
\pgfsetstrokecolor{currentstroke}%
\pgfsetdash{}{0pt}%
\pgfpathmoveto{\pgfqpoint{3.466082in}{4.617073in}}%
\pgfpathlineto{\pgfqpoint{3.473966in}{4.688079in}}%
\pgfpathlineto{\pgfqpoint{3.482392in}{4.684919in}}%
\pgfpathlineto{\pgfqpoint{3.515759in}{4.620065in}}%
\pgfpathlineto{\pgfqpoint{3.548943in}{4.574730in}}%
\pgfpathlineto{\pgfqpoint{3.539851in}{4.674568in}}%
\pgfpathlineto{\pgfqpoint{3.531145in}{4.714599in}}%
\pgfpathlineto{\pgfqpoint{3.499477in}{4.537285in}}%
\pgfpathlineto{\pgfqpoint{3.466082in}{4.617073in}}%
\pgfpathclose%
\pgfusepath{fill}%
\end{pgfscope}%
\begin{pgfscope}%
\pgfpathrectangle{\pgfqpoint{1.020000in}{0.880000in}}{\pgfqpoint{6.160000in}{6.160000in}}%
\pgfusepath{clip}%
\pgfsetbuttcap%
\pgfsetroundjoin%
\definecolor{currentfill}{rgb}{0.338377,0.452819,0.879317}%
\pgfsetfillcolor{currentfill}%
\pgfsetlinewidth{0.000000pt}%
\definecolor{currentstroke}{rgb}{0.000000,0.000000,0.000000}%
\pgfsetstrokecolor{currentstroke}%
\pgfsetdash{}{0pt}%
\pgfpathmoveto{\pgfqpoint{4.992050in}{3.405782in}}%
\pgfpathlineto{\pgfqpoint{5.001388in}{3.328298in}}%
\pgfpathlineto{\pgfqpoint{5.009460in}{3.097496in}}%
\pgfpathlineto{\pgfqpoint{5.043145in}{3.244760in}}%
\pgfpathlineto{\pgfqpoint{5.077014in}{3.403907in}}%
\pgfpathlineto{\pgfqpoint{5.066692in}{3.373716in}}%
\pgfpathlineto{\pgfqpoint{5.055969in}{3.294584in}}%
\pgfpathlineto{\pgfqpoint{5.023579in}{3.295791in}}%
\pgfpathlineto{\pgfqpoint{4.992050in}{3.405782in}}%
\pgfpathclose%
\pgfusepath{fill}%
\end{pgfscope}%
\begin{pgfscope}%
\pgfpathrectangle{\pgfqpoint{1.020000in}{0.880000in}}{\pgfqpoint{6.160000in}{6.160000in}}%
\pgfusepath{clip}%
\pgfsetbuttcap%
\pgfsetroundjoin%
\definecolor{currentfill}{rgb}{0.651398,0.768121,0.995891}%
\pgfsetfillcolor{currentfill}%
\pgfsetlinewidth{0.000000pt}%
\definecolor{currentstroke}{rgb}{0.000000,0.000000,0.000000}%
\pgfsetstrokecolor{currentstroke}%
\pgfsetdash{}{0pt}%
\pgfpathmoveto{\pgfqpoint{4.097825in}{3.957237in}}%
\pgfpathlineto{\pgfqpoint{4.106879in}{3.941070in}}%
\pgfpathlineto{\pgfqpoint{4.115895in}{4.003131in}}%
\pgfpathlineto{\pgfqpoint{4.148751in}{3.816753in}}%
\pgfpathlineto{\pgfqpoint{4.181465in}{3.711587in}}%
\pgfpathlineto{\pgfqpoint{4.172347in}{3.873266in}}%
\pgfpathlineto{\pgfqpoint{4.163248in}{3.892026in}}%
\pgfpathlineto{\pgfqpoint{4.130660in}{3.726890in}}%
\pgfpathlineto{\pgfqpoint{4.097825in}{3.957237in}}%
\pgfpathclose%
\pgfusepath{fill}%
\end{pgfscope}%
\begin{pgfscope}%
\pgfpathrectangle{\pgfqpoint{1.020000in}{0.880000in}}{\pgfqpoint{6.160000in}{6.160000in}}%
\pgfusepath{clip}%
\pgfsetbuttcap%
\pgfsetroundjoin%
\definecolor{currentfill}{rgb}{0.399231,0.528528,0.928459}%
\pgfsetfillcolor{currentfill}%
\pgfsetlinewidth{0.000000pt}%
\definecolor{currentstroke}{rgb}{0.000000,0.000000,0.000000}%
\pgfsetstrokecolor{currentstroke}%
\pgfsetdash{}{0pt}%
\pgfpathmoveto{\pgfqpoint{5.617915in}{3.567633in}}%
\pgfpathlineto{\pgfqpoint{5.626958in}{3.457826in}}%
\pgfpathlineto{\pgfqpoint{5.634555in}{3.248646in}}%
\pgfpathlineto{\pgfqpoint{5.670509in}{3.497842in}}%
\pgfpathlineto{\pgfqpoint{5.699752in}{3.295765in}}%
\pgfpathlineto{\pgfqpoint{5.690316in}{3.378401in}}%
\pgfpathlineto{\pgfqpoint{5.680996in}{3.468511in}}%
\pgfpathlineto{\pgfqpoint{5.648494in}{3.450694in}}%
\pgfpathlineto{\pgfqpoint{5.617915in}{3.567633in}}%
\pgfpathclose%
\pgfusepath{fill}%
\end{pgfscope}%
\begin{pgfscope}%
\pgfpathrectangle{\pgfqpoint{1.020000in}{0.880000in}}{\pgfqpoint{6.160000in}{6.160000in}}%
\pgfusepath{clip}%
\pgfsetbuttcap%
\pgfsetroundjoin%
\definecolor{currentfill}{rgb}{0.494638,0.633022,0.978983}%
\pgfsetfillcolor{currentfill}%
\pgfsetlinewidth{0.000000pt}%
\definecolor{currentstroke}{rgb}{0.000000,0.000000,0.000000}%
\pgfsetstrokecolor{currentstroke}%
\pgfsetdash{}{0pt}%
\pgfpathmoveto{\pgfqpoint{4.544693in}{3.614220in}}%
\pgfpathlineto{\pgfqpoint{4.554560in}{3.713487in}}%
\pgfpathlineto{\pgfqpoint{4.563499in}{3.558497in}}%
\pgfpathlineto{\pgfqpoint{4.596369in}{3.622136in}}%
\pgfpathlineto{\pgfqpoint{4.628570in}{3.530375in}}%
\pgfpathlineto{\pgfqpoint{4.619512in}{3.648893in}}%
\pgfpathlineto{\pgfqpoint{4.609129in}{3.456667in}}%
\pgfpathlineto{\pgfqpoint{4.577133in}{3.582824in}}%
\pgfpathlineto{\pgfqpoint{4.544693in}{3.614220in}}%
\pgfpathclose%
\pgfusepath{fill}%
\end{pgfscope}%
\begin{pgfscope}%
\pgfpathrectangle{\pgfqpoint{1.020000in}{0.880000in}}{\pgfqpoint{6.160000in}{6.160000in}}%
\pgfusepath{clip}%
\pgfsetbuttcap%
\pgfsetroundjoin%
\definecolor{currentfill}{rgb}{0.953054,0.585211,0.465373}%
\pgfsetfillcolor{currentfill}%
\pgfsetlinewidth{0.000000pt}%
\definecolor{currentstroke}{rgb}{0.000000,0.000000,0.000000}%
\pgfsetstrokecolor{currentstroke}%
\pgfsetdash{}{0pt}%
\pgfpathmoveto{\pgfqpoint{3.018196in}{5.008574in}}%
\pgfpathlineto{\pgfqpoint{3.027167in}{4.917565in}}%
\pgfpathlineto{\pgfqpoint{3.034801in}{4.944862in}}%
\pgfpathlineto{\pgfqpoint{3.067057in}{5.018227in}}%
\pgfpathlineto{\pgfqpoint{3.101800in}{4.861007in}}%
\pgfpathlineto{\pgfqpoint{3.092447in}{4.986850in}}%
\pgfpathlineto{\pgfqpoint{3.085901in}{4.852833in}}%
\pgfpathlineto{\pgfqpoint{3.052744in}{4.871416in}}%
\pgfpathlineto{\pgfqpoint{3.018196in}{5.008574in}}%
\pgfpathclose%
\pgfusepath{fill}%
\end{pgfscope}%
\begin{pgfscope}%
\pgfpathrectangle{\pgfqpoint{1.020000in}{0.880000in}}{\pgfqpoint{6.160000in}{6.160000in}}%
\pgfusepath{clip}%
\pgfsetbuttcap%
\pgfsetroundjoin%
\definecolor{currentfill}{rgb}{0.554312,0.690097,0.995516}%
\pgfsetfillcolor{currentfill}%
\pgfsetlinewidth{0.000000pt}%
\definecolor{currentstroke}{rgb}{0.000000,0.000000,0.000000}%
\pgfsetstrokecolor{currentstroke}%
\pgfsetdash{}{0pt}%
\pgfpathmoveto{\pgfqpoint{4.396108in}{3.872319in}}%
\pgfpathlineto{\pgfqpoint{4.405037in}{3.677716in}}%
\pgfpathlineto{\pgfqpoint{4.414221in}{3.601033in}}%
\pgfpathlineto{\pgfqpoint{4.447085in}{3.688195in}}%
\pgfpathlineto{\pgfqpoint{4.479105in}{3.477149in}}%
\pgfpathlineto{\pgfqpoint{4.470089in}{3.612998in}}%
\pgfpathlineto{\pgfqpoint{4.461203in}{3.797658in}}%
\pgfpathlineto{\pgfqpoint{4.428673in}{3.833403in}}%
\pgfpathlineto{\pgfqpoint{4.396108in}{3.872319in}}%
\pgfpathclose%
\pgfusepath{fill}%
\end{pgfscope}%
\begin{pgfscope}%
\pgfpathrectangle{\pgfqpoint{1.020000in}{0.880000in}}{\pgfqpoint{6.160000in}{6.160000in}}%
\pgfusepath{clip}%
\pgfsetbuttcap%
\pgfsetroundjoin%
\definecolor{currentfill}{rgb}{0.419991,0.552989,0.942630}%
\pgfsetfillcolor{currentfill}%
\pgfsetlinewidth{0.000000pt}%
\definecolor{currentstroke}{rgb}{0.000000,0.000000,0.000000}%
\pgfsetstrokecolor{currentstroke}%
\pgfsetdash{}{0pt}%
\pgfpathmoveto{\pgfqpoint{4.628570in}{3.530375in}}%
\pgfpathlineto{\pgfqpoint{4.637296in}{3.339744in}}%
\pgfpathlineto{\pgfqpoint{4.647331in}{3.434553in}}%
\pgfpathlineto{\pgfqpoint{4.679926in}{3.435373in}}%
\pgfpathlineto{\pgfqpoint{4.711659in}{3.276658in}}%
\pgfpathlineto{\pgfqpoint{4.703704in}{3.602969in}}%
\pgfpathlineto{\pgfqpoint{4.693088in}{3.415974in}}%
\pgfpathlineto{\pgfqpoint{4.661541in}{3.616450in}}%
\pgfpathlineto{\pgfqpoint{4.628570in}{3.530375in}}%
\pgfpathclose%
\pgfusepath{fill}%
\end{pgfscope}%
\begin{pgfscope}%
\pgfpathrectangle{\pgfqpoint{1.020000in}{0.880000in}}{\pgfqpoint{6.160000in}{6.160000in}}%
\pgfusepath{clip}%
\pgfsetbuttcap%
\pgfsetroundjoin%
\definecolor{currentfill}{rgb}{0.368507,0.491141,0.905243}%
\pgfsetfillcolor{currentfill}%
\pgfsetlinewidth{0.000000pt}%
\definecolor{currentstroke}{rgb}{0.000000,0.000000,0.000000}%
\pgfsetstrokecolor{currentstroke}%
\pgfsetdash{}{0pt}%
\pgfpathmoveto{\pgfqpoint{5.271305in}{3.376630in}}%
\pgfpathlineto{\pgfqpoint{5.280816in}{3.308584in}}%
\pgfpathlineto{\pgfqpoint{5.291042in}{3.304426in}}%
\pgfpathlineto{\pgfqpoint{5.323604in}{3.319093in}}%
\pgfpathlineto{\pgfqpoint{5.355934in}{3.314725in}}%
\pgfpathlineto{\pgfqpoint{5.347056in}{3.439988in}}%
\pgfpathlineto{\pgfqpoint{5.335163in}{3.304097in}}%
\pgfpathlineto{\pgfqpoint{5.304950in}{3.491538in}}%
\pgfpathlineto{\pgfqpoint{5.271305in}{3.376630in}}%
\pgfpathclose%
\pgfusepath{fill}%
\end{pgfscope}%
\begin{pgfscope}%
\pgfpathrectangle{\pgfqpoint{1.020000in}{0.880000in}}{\pgfqpoint{6.160000in}{6.160000in}}%
\pgfusepath{clip}%
\pgfsetbuttcap%
\pgfsetroundjoin%
\definecolor{currentfill}{rgb}{0.323718,0.433158,0.864722}%
\pgfsetfillcolor{currentfill}%
\pgfsetlinewidth{0.000000pt}%
\definecolor{currentstroke}{rgb}{0.000000,0.000000,0.000000}%
\pgfsetstrokecolor{currentstroke}%
\pgfsetdash{}{0pt}%
\pgfpathmoveto{\pgfqpoint{4.926325in}{3.303913in}}%
\pgfpathlineto{\pgfqpoint{4.936484in}{3.338236in}}%
\pgfpathlineto{\pgfqpoint{4.945287in}{3.191362in}}%
\pgfpathlineto{\pgfqpoint{4.977527in}{3.160555in}}%
\pgfpathlineto{\pgfqpoint{5.009460in}{3.097496in}}%
\pgfpathlineto{\pgfqpoint{5.001388in}{3.328298in}}%
\pgfpathlineto{\pgfqpoint{4.992050in}{3.405782in}}%
\pgfpathlineto{\pgfqpoint{4.959364in}{3.378526in}}%
\pgfpathlineto{\pgfqpoint{4.926325in}{3.303913in}}%
\pgfpathclose%
\pgfusepath{fill}%
\end{pgfscope}%
\begin{pgfscope}%
\pgfpathrectangle{\pgfqpoint{1.020000in}{0.880000in}}{\pgfqpoint{6.160000in}{6.160000in}}%
\pgfusepath{clip}%
\pgfsetbuttcap%
\pgfsetroundjoin%
\definecolor{currentfill}{rgb}{0.373552,0.497499,0.909467}%
\pgfsetfillcolor{currentfill}%
\pgfsetlinewidth{0.000000pt}%
\definecolor{currentstroke}{rgb}{0.000000,0.000000,0.000000}%
\pgfsetstrokecolor{currentstroke}%
\pgfsetdash{}{0pt}%
\pgfpathmoveto{\pgfqpoint{4.777925in}{3.492133in}}%
\pgfpathlineto{\pgfqpoint{4.786562in}{3.307298in}}%
\pgfpathlineto{\pgfqpoint{4.795622in}{3.193746in}}%
\pgfpathlineto{\pgfqpoint{4.828242in}{3.212460in}}%
\pgfpathlineto{\pgfqpoint{4.862230in}{3.433618in}}%
\pgfpathlineto{\pgfqpoint{4.852205in}{3.404554in}}%
\pgfpathlineto{\pgfqpoint{4.843197in}{3.525831in}}%
\pgfpathlineto{\pgfqpoint{4.809837in}{3.392920in}}%
\pgfpathlineto{\pgfqpoint{4.777925in}{3.492133in}}%
\pgfpathclose%
\pgfusepath{fill}%
\end{pgfscope}%
\begin{pgfscope}%
\pgfpathrectangle{\pgfqpoint{1.020000in}{0.880000in}}{\pgfqpoint{6.160000in}{6.160000in}}%
\pgfusepath{clip}%
\pgfsetbuttcap%
\pgfsetroundjoin%
\definecolor{currentfill}{rgb}{0.968894,0.679480,0.562812}%
\pgfsetfillcolor{currentfill}%
\pgfsetlinewidth{0.000000pt}%
\definecolor{currentstroke}{rgb}{0.000000,0.000000,0.000000}%
\pgfsetstrokecolor{currentstroke}%
\pgfsetdash{}{0pt}%
\pgfpathmoveto{\pgfqpoint{3.317159in}{4.717267in}}%
\pgfpathlineto{\pgfqpoint{3.323683in}{4.918572in}}%
\pgfpathlineto{\pgfqpoint{3.332238in}{4.883757in}}%
\pgfpathlineto{\pgfqpoint{3.366418in}{4.739410in}}%
\pgfpathlineto{\pgfqpoint{3.399463in}{4.727299in}}%
\pgfpathlineto{\pgfqpoint{3.390610in}{4.796795in}}%
\pgfpathlineto{\pgfqpoint{3.383415in}{4.657107in}}%
\pgfpathlineto{\pgfqpoint{3.349950in}{4.730445in}}%
\pgfpathlineto{\pgfqpoint{3.317159in}{4.717267in}}%
\pgfpathclose%
\pgfusepath{fill}%
\end{pgfscope}%
\begin{pgfscope}%
\pgfpathrectangle{\pgfqpoint{1.020000in}{0.880000in}}{\pgfqpoint{6.160000in}{6.160000in}}%
\pgfusepath{clip}%
\pgfsetbuttcap%
\pgfsetroundjoin%
\definecolor{currentfill}{rgb}{0.939254,0.539581,0.423900}%
\pgfsetfillcolor{currentfill}%
\pgfsetlinewidth{0.000000pt}%
\definecolor{currentstroke}{rgb}{0.000000,0.000000,0.000000}%
\pgfsetstrokecolor{currentstroke}%
\pgfsetdash{}{0pt}%
\pgfpathmoveto{\pgfqpoint{2.804690in}{4.975622in}}%
\pgfpathlineto{\pgfqpoint{2.811946in}{5.009203in}}%
\pgfpathlineto{\pgfqpoint{2.821381in}{4.882020in}}%
\pgfpathlineto{\pgfqpoint{2.851572in}{5.103492in}}%
\pgfpathlineto{\pgfqpoint{2.886134in}{4.992558in}}%
\pgfpathlineto{\pgfqpoint{2.877142in}{5.086078in}}%
\pgfpathlineto{\pgfqpoint{2.872194in}{4.866856in}}%
\pgfpathlineto{\pgfqpoint{2.836931in}{5.037370in}}%
\pgfpathlineto{\pgfqpoint{2.804690in}{4.975622in}}%
\pgfpathclose%
\pgfusepath{fill}%
\end{pgfscope}%
\begin{pgfscope}%
\pgfpathrectangle{\pgfqpoint{1.020000in}{0.880000in}}{\pgfqpoint{6.160000in}{6.160000in}}%
\pgfusepath{clip}%
\pgfsetbuttcap%
\pgfsetroundjoin%
\definecolor{currentfill}{rgb}{0.969522,0.700833,0.587508}%
\pgfsetfillcolor{currentfill}%
\pgfsetlinewidth{0.000000pt}%
\definecolor{currentstroke}{rgb}{0.000000,0.000000,0.000000}%
\pgfsetstrokecolor{currentstroke}%
\pgfsetdash{}{0pt}%
\pgfpathmoveto{\pgfqpoint{2.264506in}{4.675460in}}%
\pgfpathlineto{\pgfqpoint{2.270844in}{4.718284in}}%
\pgfpathlineto{\pgfqpoint{2.276093in}{4.820096in}}%
\pgfpathlineto{\pgfqpoint{2.307513in}{4.922483in}}%
\pgfpathlineto{\pgfqpoint{2.342374in}{4.837970in}}%
\pgfpathlineto{\pgfqpoint{2.340030in}{4.569483in}}%
\pgfpathlineto{\pgfqpoint{2.330421in}{4.700942in}}%
\pgfpathlineto{\pgfqpoint{2.297817in}{4.669532in}}%
\pgfpathlineto{\pgfqpoint{2.264506in}{4.675460in}}%
\pgfpathclose%
\pgfusepath{fill}%
\end{pgfscope}%
\begin{pgfscope}%
\pgfpathrectangle{\pgfqpoint{1.020000in}{0.880000in}}{\pgfqpoint{6.160000in}{6.160000in}}%
\pgfusepath{clip}%
\pgfsetbuttcap%
\pgfsetroundjoin%
\definecolor{currentfill}{rgb}{0.947345,0.794696,0.716991}%
\pgfsetfillcolor{currentfill}%
\pgfsetlinewidth{0.000000pt}%
\definecolor{currentstroke}{rgb}{0.000000,0.000000,0.000000}%
\pgfsetstrokecolor{currentstroke}%
\pgfsetdash{}{0pt}%
\pgfpathmoveto{\pgfqpoint{3.548943in}{4.574730in}}%
\pgfpathlineto{\pgfqpoint{3.557151in}{4.617956in}}%
\pgfpathlineto{\pgfqpoint{3.566038in}{4.553756in}}%
\pgfpathlineto{\pgfqpoint{3.599016in}{4.544208in}}%
\pgfpathlineto{\pgfqpoint{3.632912in}{4.361586in}}%
\pgfpathlineto{\pgfqpoint{3.623625in}{4.496904in}}%
\pgfpathlineto{\pgfqpoint{3.614791in}{4.550764in}}%
\pgfpathlineto{\pgfqpoint{3.581731in}{4.587970in}}%
\pgfpathlineto{\pgfqpoint{3.548943in}{4.574730in}}%
\pgfpathclose%
\pgfusepath{fill}%
\end{pgfscope}%
\begin{pgfscope}%
\pgfpathrectangle{\pgfqpoint{1.020000in}{0.880000in}}{\pgfqpoint{6.160000in}{6.160000in}}%
\pgfusepath{clip}%
\pgfsetbuttcap%
\pgfsetroundjoin%
\definecolor{currentfill}{rgb}{0.949454,0.572388,0.453443}%
\pgfsetfillcolor{currentfill}%
\pgfsetlinewidth{0.000000pt}%
\definecolor{currentstroke}{rgb}{0.000000,0.000000,0.000000}%
\pgfsetstrokecolor{currentstroke}%
\pgfsetdash{}{0pt}%
\pgfpathmoveto{\pgfqpoint{2.741628in}{4.756816in}}%
\pgfpathlineto{\pgfqpoint{2.746315in}{4.964768in}}%
\pgfpathlineto{\pgfqpoint{2.753794in}{4.977667in}}%
\pgfpathlineto{\pgfqpoint{2.787313in}{4.951628in}}%
\pgfpathlineto{\pgfqpoint{2.821381in}{4.882020in}}%
\pgfpathlineto{\pgfqpoint{2.811946in}{5.009203in}}%
\pgfpathlineto{\pgfqpoint{2.804690in}{4.975622in}}%
\pgfpathlineto{\pgfqpoint{2.770765in}{5.036562in}}%
\pgfpathlineto{\pgfqpoint{2.741628in}{4.756816in}}%
\pgfpathclose%
\pgfusepath{fill}%
\end{pgfscope}%
\begin{pgfscope}%
\pgfpathrectangle{\pgfqpoint{1.020000in}{0.880000in}}{\pgfqpoint{6.160000in}{6.160000in}}%
\pgfusepath{clip}%
\pgfsetbuttcap%
\pgfsetroundjoin%
\definecolor{currentfill}{rgb}{0.800601,0.850358,0.930008}%
\pgfsetfillcolor{currentfill}%
\pgfsetlinewidth{0.000000pt}%
\definecolor{currentstroke}{rgb}{0.000000,0.000000,0.000000}%
\pgfsetstrokecolor{currentstroke}%
\pgfsetdash{}{0pt}%
\pgfpathmoveto{\pgfqpoint{3.865413in}{4.116331in}}%
\pgfpathlineto{\pgfqpoint{3.873788in}{4.252484in}}%
\pgfpathlineto{\pgfqpoint{3.882568in}{4.267118in}}%
\pgfpathlineto{\pgfqpoint{3.915659in}{4.182380in}}%
\pgfpathlineto{\pgfqpoint{3.948902in}{4.000686in}}%
\pgfpathlineto{\pgfqpoint{3.939555in}{4.200192in}}%
\pgfpathlineto{\pgfqpoint{3.931107in}{4.034644in}}%
\pgfpathlineto{\pgfqpoint{3.898047in}{4.158646in}}%
\pgfpathlineto{\pgfqpoint{3.865413in}{4.116331in}}%
\pgfpathclose%
\pgfusepath{fill}%
\end{pgfscope}%
\begin{pgfscope}%
\pgfpathrectangle{\pgfqpoint{1.020000in}{0.880000in}}{\pgfqpoint{6.160000in}{6.160000in}}%
\pgfusepath{clip}%
\pgfsetbuttcap%
\pgfsetroundjoin%
\definecolor{currentfill}{rgb}{0.318832,0.426605,0.859857}%
\pgfsetfillcolor{currentfill}%
\pgfsetlinewidth{0.000000pt}%
\definecolor{currentstroke}{rgb}{0.000000,0.000000,0.000000}%
\pgfsetstrokecolor{currentstroke}%
\pgfsetdash{}{0pt}%
\pgfpathmoveto{\pgfqpoint{5.141764in}{3.384312in}}%
\pgfpathlineto{\pgfqpoint{5.149265in}{3.112395in}}%
\pgfpathlineto{\pgfqpoint{5.161153in}{3.292024in}}%
\pgfpathlineto{\pgfqpoint{5.193904in}{3.320165in}}%
\pgfpathlineto{\pgfqpoint{5.224766in}{3.167498in}}%
\pgfpathlineto{\pgfqpoint{5.215455in}{3.253113in}}%
\pgfpathlineto{\pgfqpoint{5.205209in}{3.247091in}}%
\pgfpathlineto{\pgfqpoint{5.173940in}{3.358252in}}%
\pgfpathlineto{\pgfqpoint{5.141764in}{3.384312in}}%
\pgfpathclose%
\pgfusepath{fill}%
\end{pgfscope}%
\begin{pgfscope}%
\pgfpathrectangle{\pgfqpoint{1.020000in}{0.880000in}}{\pgfqpoint{6.160000in}{6.160000in}}%
\pgfusepath{clip}%
\pgfsetbuttcap%
\pgfsetroundjoin%
\definecolor{currentfill}{rgb}{0.960490,0.616276,0.495467}%
\pgfsetfillcolor{currentfill}%
\pgfsetlinewidth{0.000000pt}%
\definecolor{currentstroke}{rgb}{0.000000,0.000000,0.000000}%
\pgfsetstrokecolor{currentstroke}%
\pgfsetdash{}{0pt}%
\pgfpathmoveto{\pgfqpoint{3.167993in}{4.839231in}}%
\pgfpathlineto{\pgfqpoint{3.176544in}{4.791719in}}%
\pgfpathlineto{\pgfqpoint{3.183202in}{4.937327in}}%
\pgfpathlineto{\pgfqpoint{3.216600in}{4.900556in}}%
\pgfpathlineto{\pgfqpoint{3.249751in}{4.885875in}}%
\pgfpathlineto{\pgfqpoint{3.241764in}{4.866849in}}%
\pgfpathlineto{\pgfqpoint{3.233714in}{4.856795in}}%
\pgfpathlineto{\pgfqpoint{3.199736in}{4.964001in}}%
\pgfpathlineto{\pgfqpoint{3.167993in}{4.839231in}}%
\pgfpathclose%
\pgfusepath{fill}%
\end{pgfscope}%
\begin{pgfscope}%
\pgfpathrectangle{\pgfqpoint{1.020000in}{0.880000in}}{\pgfqpoint{6.160000in}{6.160000in}}%
\pgfusepath{clip}%
\pgfsetbuttcap%
\pgfsetroundjoin%
\definecolor{currentfill}{rgb}{0.266381,0.353304,0.801637}%
\pgfsetfillcolor{currentfill}%
\pgfsetlinewidth{0.000000pt}%
\definecolor{currentstroke}{rgb}{0.000000,0.000000,0.000000}%
\pgfsetstrokecolor{currentstroke}%
\pgfsetdash{}{0pt}%
\pgfpathmoveto{\pgfqpoint{5.009460in}{3.097496in}}%
\pgfpathlineto{\pgfqpoint{5.018938in}{3.037219in}}%
\pgfpathlineto{\pgfqpoint{5.028211in}{2.951861in}}%
\pgfpathlineto{\pgfqpoint{5.062245in}{3.133701in}}%
\pgfpathlineto{\pgfqpoint{5.093847in}{3.038512in}}%
\pgfpathlineto{\pgfqpoint{5.086556in}{3.344902in}}%
\pgfpathlineto{\pgfqpoint{5.077014in}{3.403907in}}%
\pgfpathlineto{\pgfqpoint{5.043145in}{3.244760in}}%
\pgfpathlineto{\pgfqpoint{5.009460in}{3.097496in}}%
\pgfpathclose%
\pgfusepath{fill}%
\end{pgfscope}%
\begin{pgfscope}%
\pgfpathrectangle{\pgfqpoint{1.020000in}{0.880000in}}{\pgfqpoint{6.160000in}{6.160000in}}%
\pgfusepath{clip}%
\pgfsetbuttcap%
\pgfsetroundjoin%
\definecolor{currentfill}{rgb}{0.399231,0.528528,0.928459}%
\pgfsetfillcolor{currentfill}%
\pgfsetlinewidth{0.000000pt}%
\definecolor{currentstroke}{rgb}{0.000000,0.000000,0.000000}%
\pgfsetstrokecolor{currentstroke}%
\pgfsetdash{}{0pt}%
\pgfpathmoveto{\pgfqpoint{5.550513in}{3.365810in}}%
\pgfpathlineto{\pgfqpoint{5.561120in}{3.370197in}}%
\pgfpathlineto{\pgfqpoint{5.572265in}{3.411738in}}%
\pgfpathlineto{\pgfqpoint{5.606066in}{3.514476in}}%
\pgfpathlineto{\pgfqpoint{5.634555in}{3.248646in}}%
\pgfpathlineto{\pgfqpoint{5.626958in}{3.457826in}}%
\pgfpathlineto{\pgfqpoint{5.617915in}{3.567633in}}%
\pgfpathlineto{\pgfqpoint{5.583286in}{3.402146in}}%
\pgfpathlineto{\pgfqpoint{5.550513in}{3.365810in}}%
\pgfpathclose%
\pgfusepath{fill}%
\end{pgfscope}%
\begin{pgfscope}%
\pgfpathrectangle{\pgfqpoint{1.020000in}{0.880000in}}{\pgfqpoint{6.160000in}{6.160000in}}%
\pgfusepath{clip}%
\pgfsetbuttcap%
\pgfsetroundjoin%
\definecolor{currentfill}{rgb}{0.738826,0.822572,0.968261}%
\pgfsetfillcolor{currentfill}%
\pgfsetlinewidth{0.000000pt}%
\definecolor{currentstroke}{rgb}{0.000000,0.000000,0.000000}%
\pgfsetstrokecolor{currentstroke}%
\pgfsetdash{}{0pt}%
\pgfpathmoveto{\pgfqpoint{3.948902in}{4.000686in}}%
\pgfpathlineto{\pgfqpoint{3.957754in}{4.014532in}}%
\pgfpathlineto{\pgfqpoint{3.966734in}{3.978153in}}%
\pgfpathlineto{\pgfqpoint{3.999439in}{4.028631in}}%
\pgfpathlineto{\pgfqpoint{4.032223in}{4.035498in}}%
\pgfpathlineto{\pgfqpoint{4.023120in}{4.120778in}}%
\pgfpathlineto{\pgfqpoint{4.014342in}{4.014640in}}%
\pgfpathlineto{\pgfqpoint{3.981745in}{3.949903in}}%
\pgfpathlineto{\pgfqpoint{3.948902in}{4.000686in}}%
\pgfpathclose%
\pgfusepath{fill}%
\end{pgfscope}%
\begin{pgfscope}%
\pgfpathrectangle{\pgfqpoint{1.020000in}{0.880000in}}{\pgfqpoint{6.160000in}{6.160000in}}%
\pgfusepath{clip}%
\pgfsetbuttcap%
\pgfsetroundjoin%
\definecolor{currentfill}{rgb}{0.243520,0.319189,0.771672}%
\pgfsetfillcolor{currentfill}%
\pgfsetlinewidth{0.000000pt}%
\definecolor{currentstroke}{rgb}{0.000000,0.000000,0.000000}%
\pgfsetstrokecolor{currentstroke}%
\pgfsetdash{}{0pt}%
\pgfpathmoveto{\pgfqpoint{4.945287in}{3.191362in}}%
\pgfpathlineto{\pgfqpoint{4.955472in}{3.224242in}}%
\pgfpathlineto{\pgfqpoint{4.964109in}{3.056136in}}%
\pgfpathlineto{\pgfqpoint{4.997425in}{3.155641in}}%
\pgfpathlineto{\pgfqpoint{5.028211in}{2.951861in}}%
\pgfpathlineto{\pgfqpoint{5.018938in}{3.037219in}}%
\pgfpathlineto{\pgfqpoint{5.009460in}{3.097496in}}%
\pgfpathlineto{\pgfqpoint{4.977527in}{3.160555in}}%
\pgfpathlineto{\pgfqpoint{4.945287in}{3.191362in}}%
\pgfpathclose%
\pgfusepath{fill}%
\end{pgfscope}%
\begin{pgfscope}%
\pgfpathrectangle{\pgfqpoint{1.020000in}{0.880000in}}{\pgfqpoint{6.160000in}{6.160000in}}%
\pgfusepath{clip}%
\pgfsetbuttcap%
\pgfsetroundjoin%
\definecolor{currentfill}{rgb}{0.964835,0.744614,0.643239}%
\pgfsetfillcolor{currentfill}%
\pgfsetlinewidth{0.000000pt}%
\definecolor{currentstroke}{rgb}{0.000000,0.000000,0.000000}%
\pgfsetstrokecolor{currentstroke}%
\pgfsetdash{}{0pt}%
\pgfpathmoveto{\pgfqpoint{2.131635in}{4.669536in}}%
\pgfpathlineto{\pgfqpoint{2.143193in}{4.444711in}}%
\pgfpathlineto{\pgfqpoint{2.146685in}{4.619818in}}%
\pgfpathlineto{\pgfqpoint{2.178884in}{4.676593in}}%
\pgfpathlineto{\pgfqpoint{2.212637in}{4.654464in}}%
\pgfpathlineto{\pgfqpoint{2.204201in}{4.723050in}}%
\pgfpathlineto{\pgfqpoint{2.196599in}{4.749374in}}%
\pgfpathlineto{\pgfqpoint{2.164841in}{4.672954in}}%
\pgfpathlineto{\pgfqpoint{2.131635in}{4.669536in}}%
\pgfpathclose%
\pgfusepath{fill}%
\end{pgfscope}%
\begin{pgfscope}%
\pgfpathrectangle{\pgfqpoint{1.020000in}{0.880000in}}{\pgfqpoint{6.160000in}{6.160000in}}%
\pgfusepath{clip}%
\pgfsetbuttcap%
\pgfsetroundjoin%
\definecolor{currentfill}{rgb}{0.968863,0.710838,0.599901}%
\pgfsetfillcolor{currentfill}%
\pgfsetlinewidth{0.000000pt}%
\definecolor{currentstroke}{rgb}{0.000000,0.000000,0.000000}%
\pgfsetstrokecolor{currentstroke}%
\pgfsetdash{}{0pt}%
\pgfpathmoveto{\pgfqpoint{2.196599in}{4.749374in}}%
\pgfpathlineto{\pgfqpoint{2.204201in}{4.723050in}}%
\pgfpathlineto{\pgfqpoint{2.212637in}{4.654464in}}%
\pgfpathlineto{\pgfqpoint{2.247487in}{4.573223in}}%
\pgfpathlineto{\pgfqpoint{2.276093in}{4.820096in}}%
\pgfpathlineto{\pgfqpoint{2.270844in}{4.718284in}}%
\pgfpathlineto{\pgfqpoint{2.264506in}{4.675460in}}%
\pgfpathlineto{\pgfqpoint{2.229421in}{4.772244in}}%
\pgfpathlineto{\pgfqpoint{2.196599in}{4.749374in}}%
\pgfpathclose%
\pgfusepath{fill}%
\end{pgfscope}%
\begin{pgfscope}%
\pgfpathrectangle{\pgfqpoint{1.020000in}{0.880000in}}{\pgfqpoint{6.160000in}{6.160000in}}%
\pgfusepath{clip}%
\pgfsetbuttcap%
\pgfsetroundjoin%
\definecolor{currentfill}{rgb}{0.373552,0.497499,0.909467}%
\pgfsetfillcolor{currentfill}%
\pgfsetlinewidth{0.000000pt}%
\definecolor{currentstroke}{rgb}{0.000000,0.000000,0.000000}%
\pgfsetstrokecolor{currentstroke}%
\pgfsetdash{}{0pt}%
\pgfpathmoveto{\pgfqpoint{5.486612in}{3.418257in}}%
\pgfpathlineto{\pgfqpoint{5.496494in}{3.371930in}}%
\pgfpathlineto{\pgfqpoint{5.506953in}{3.368307in}}%
\pgfpathlineto{\pgfqpoint{5.538518in}{3.309480in}}%
\pgfpathlineto{\pgfqpoint{5.572265in}{3.411738in}}%
\pgfpathlineto{\pgfqpoint{5.561120in}{3.370197in}}%
\pgfpathlineto{\pgfqpoint{5.550513in}{3.365810in}}%
\pgfpathlineto{\pgfqpoint{5.517830in}{3.335861in}}%
\pgfpathlineto{\pgfqpoint{5.486612in}{3.418257in}}%
\pgfpathclose%
\pgfusepath{fill}%
\end{pgfscope}%
\begin{pgfscope}%
\pgfpathrectangle{\pgfqpoint{1.020000in}{0.880000in}}{\pgfqpoint{6.160000in}{6.160000in}}%
\pgfusepath{clip}%
\pgfsetbuttcap%
\pgfsetroundjoin%
\definecolor{currentfill}{rgb}{0.368507,0.491141,0.905243}%
\pgfsetfillcolor{currentfill}%
\pgfsetlinewidth{0.000000pt}%
\definecolor{currentstroke}{rgb}{0.000000,0.000000,0.000000}%
\pgfsetstrokecolor{currentstroke}%
\pgfsetdash{}{0pt}%
\pgfpathmoveto{\pgfqpoint{4.711659in}{3.276658in}}%
\pgfpathlineto{\pgfqpoint{4.722048in}{3.409962in}}%
\pgfpathlineto{\pgfqpoint{4.731683in}{3.397301in}}%
\pgfpathlineto{\pgfqpoint{4.763938in}{3.337414in}}%
\pgfpathlineto{\pgfqpoint{4.795622in}{3.193746in}}%
\pgfpathlineto{\pgfqpoint{4.786562in}{3.307298in}}%
\pgfpathlineto{\pgfqpoint{4.777925in}{3.492133in}}%
\pgfpathlineto{\pgfqpoint{4.745097in}{3.443229in}}%
\pgfpathlineto{\pgfqpoint{4.711659in}{3.276658in}}%
\pgfpathclose%
\pgfusepath{fill}%
\end{pgfscope}%
\begin{pgfscope}%
\pgfpathrectangle{\pgfqpoint{1.020000in}{0.880000in}}{\pgfqpoint{6.160000in}{6.160000in}}%
\pgfusepath{clip}%
\pgfsetbuttcap%
\pgfsetroundjoin%
\definecolor{currentfill}{rgb}{0.510824,0.649397,0.985079}%
\pgfsetfillcolor{currentfill}%
\pgfsetlinewidth{0.000000pt}%
\definecolor{currentstroke}{rgb}{0.000000,0.000000,0.000000}%
\pgfsetstrokecolor{currentstroke}%
\pgfsetdash{}{0pt}%
\pgfpathmoveto{\pgfqpoint{4.479105in}{3.477149in}}%
\pgfpathlineto{\pgfqpoint{4.489052in}{3.648768in}}%
\pgfpathlineto{\pgfqpoint{4.498387in}{3.608944in}}%
\pgfpathlineto{\pgfqpoint{4.530807in}{3.539316in}}%
\pgfpathlineto{\pgfqpoint{4.563499in}{3.558497in}}%
\pgfpathlineto{\pgfqpoint{4.554560in}{3.713487in}}%
\pgfpathlineto{\pgfqpoint{4.544693in}{3.614220in}}%
\pgfpathlineto{\pgfqpoint{4.512436in}{3.713186in}}%
\pgfpathlineto{\pgfqpoint{4.479105in}{3.477149in}}%
\pgfpathclose%
\pgfusepath{fill}%
\end{pgfscope}%
\begin{pgfscope}%
\pgfpathrectangle{\pgfqpoint{1.020000in}{0.880000in}}{\pgfqpoint{6.160000in}{6.160000in}}%
\pgfusepath{clip}%
\pgfsetbuttcap%
\pgfsetroundjoin%
\definecolor{currentfill}{rgb}{0.619318,0.744121,0.998931}%
\pgfsetfillcolor{currentfill}%
\pgfsetlinewidth{0.000000pt}%
\definecolor{currentstroke}{rgb}{0.000000,0.000000,0.000000}%
\pgfsetstrokecolor{currentstroke}%
\pgfsetdash{}{0pt}%
\pgfpathmoveto{\pgfqpoint{4.181465in}{3.711587in}}%
\pgfpathlineto{\pgfqpoint{4.190598in}{3.897661in}}%
\pgfpathlineto{\pgfqpoint{4.199733in}{3.779023in}}%
\pgfpathlineto{\pgfqpoint{4.232492in}{3.809295in}}%
\pgfpathlineto{\pgfqpoint{4.265237in}{3.822841in}}%
\pgfpathlineto{\pgfqpoint{4.256004in}{3.791848in}}%
\pgfpathlineto{\pgfqpoint{4.246828in}{3.812672in}}%
\pgfpathlineto{\pgfqpoint{4.214135in}{3.747950in}}%
\pgfpathlineto{\pgfqpoint{4.181465in}{3.711587in}}%
\pgfpathclose%
\pgfusepath{fill}%
\end{pgfscope}%
\begin{pgfscope}%
\pgfpathrectangle{\pgfqpoint{1.020000in}{0.880000in}}{\pgfqpoint{6.160000in}{6.160000in}}%
\pgfusepath{clip}%
\pgfsetbuttcap%
\pgfsetroundjoin%
\definecolor{currentfill}{rgb}{0.358415,0.478426,0.896795}%
\pgfsetfillcolor{currentfill}%
\pgfsetlinewidth{0.000000pt}%
\definecolor{currentstroke}{rgb}{0.000000,0.000000,0.000000}%
\pgfsetstrokecolor{currentstroke}%
\pgfsetdash{}{0pt}%
\pgfpathmoveto{\pgfqpoint{5.699752in}{3.295765in}}%
\pgfpathlineto{\pgfqpoint{5.712896in}{3.455269in}}%
\pgfpathlineto{\pgfqpoint{5.721876in}{3.340706in}}%
\pgfpathlineto{\pgfqpoint{5.752843in}{3.255722in}}%
\pgfpathlineto{\pgfqpoint{5.743286in}{3.332330in}}%
\pgfpathlineto{\pgfqpoint{5.732747in}{3.345195in}}%
\pgfpathlineto{\pgfqpoint{5.699752in}{3.295765in}}%
\pgfpathclose%
\pgfusepath{fill}%
\end{pgfscope}%
\begin{pgfscope}%
\pgfpathrectangle{\pgfqpoint{1.020000in}{0.880000in}}{\pgfqpoint{6.160000in}{6.160000in}}%
\pgfusepath{clip}%
\pgfsetbuttcap%
\pgfsetroundjoin%
\definecolor{currentfill}{rgb}{0.683056,0.790043,0.989768}%
\pgfsetfillcolor{currentfill}%
\pgfsetlinewidth{0.000000pt}%
\definecolor{currentstroke}{rgb}{0.000000,0.000000,0.000000}%
\pgfsetstrokecolor{currentstroke}%
\pgfsetdash{}{0pt}%
\pgfpathmoveto{\pgfqpoint{4.032223in}{4.035498in}}%
\pgfpathlineto{\pgfqpoint{4.041381in}{3.913043in}}%
\pgfpathlineto{\pgfqpoint{4.050520in}{3.802340in}}%
\pgfpathlineto{\pgfqpoint{4.083333in}{3.753250in}}%
\pgfpathlineto{\pgfqpoint{4.115895in}{4.003131in}}%
\pgfpathlineto{\pgfqpoint{4.106879in}{3.941070in}}%
\pgfpathlineto{\pgfqpoint{4.097825in}{3.957237in}}%
\pgfpathlineto{\pgfqpoint{4.065192in}{3.872388in}}%
\pgfpathlineto{\pgfqpoint{4.032223in}{4.035498in}}%
\pgfpathclose%
\pgfusepath{fill}%
\end{pgfscope}%
\begin{pgfscope}%
\pgfpathrectangle{\pgfqpoint{1.020000in}{0.880000in}}{\pgfqpoint{6.160000in}{6.160000in}}%
\pgfusepath{clip}%
\pgfsetbuttcap%
\pgfsetroundjoin%
\definecolor{currentfill}{rgb}{0.576051,0.708780,0.997755}%
\pgfsetfillcolor{currentfill}%
\pgfsetlinewidth{0.000000pt}%
\definecolor{currentstroke}{rgb}{0.000000,0.000000,0.000000}%
\pgfsetstrokecolor{currentstroke}%
\pgfsetdash{}{0pt}%
\pgfpathmoveto{\pgfqpoint{4.330609in}{3.789136in}}%
\pgfpathlineto{\pgfqpoint{4.339438in}{3.494178in}}%
\pgfpathlineto{\pgfqpoint{4.349185in}{3.776055in}}%
\pgfpathlineto{\pgfqpoint{4.381876in}{3.756467in}}%
\pgfpathlineto{\pgfqpoint{4.414221in}{3.601033in}}%
\pgfpathlineto{\pgfqpoint{4.405037in}{3.677716in}}%
\pgfpathlineto{\pgfqpoint{4.396108in}{3.872319in}}%
\pgfpathlineto{\pgfqpoint{4.363375in}{3.842269in}}%
\pgfpathlineto{\pgfqpoint{4.330609in}{3.789136in}}%
\pgfpathclose%
\pgfusepath{fill}%
\end{pgfscope}%
\begin{pgfscope}%
\pgfpathrectangle{\pgfqpoint{1.020000in}{0.880000in}}{\pgfqpoint{6.160000in}{6.160000in}}%
\pgfusepath{clip}%
\pgfsetbuttcap%
\pgfsetroundjoin%
\definecolor{currentfill}{rgb}{0.947654,0.565976,0.447478}%
\pgfsetfillcolor{currentfill}%
\pgfsetlinewidth{0.000000pt}%
\definecolor{currentstroke}{rgb}{0.000000,0.000000,0.000000}%
\pgfsetstrokecolor{currentstroke}%
\pgfsetdash{}{0pt}%
\pgfpathmoveto{\pgfqpoint{2.671662in}{5.016152in}}%
\pgfpathlineto{\pgfqpoint{2.680281in}{4.946351in}}%
\pgfpathlineto{\pgfqpoint{2.687376in}{4.980307in}}%
\pgfpathlineto{\pgfqpoint{2.720261in}{5.002218in}}%
\pgfpathlineto{\pgfqpoint{2.753794in}{4.977667in}}%
\pgfpathlineto{\pgfqpoint{2.746315in}{4.964768in}}%
\pgfpathlineto{\pgfqpoint{2.741628in}{4.756816in}}%
\pgfpathlineto{\pgfqpoint{2.705646in}{4.958707in}}%
\pgfpathlineto{\pgfqpoint{2.671662in}{5.016152in}}%
\pgfpathclose%
\pgfusepath{fill}%
\end{pgfscope}%
\begin{pgfscope}%
\pgfpathrectangle{\pgfqpoint{1.020000in}{0.880000in}}{\pgfqpoint{6.160000in}{6.160000in}}%
\pgfusepath{clip}%
\pgfsetbuttcap%
\pgfsetroundjoin%
\definecolor{currentfill}{rgb}{0.871493,0.862309,0.857016}%
\pgfsetfillcolor{currentfill}%
\pgfsetlinewidth{0.000000pt}%
\definecolor{currentstroke}{rgb}{0.000000,0.000000,0.000000}%
\pgfsetstrokecolor{currentstroke}%
\pgfsetdash{}{0pt}%
\pgfpathmoveto{\pgfqpoint{3.715732in}{4.374021in}}%
\pgfpathlineto{\pgfqpoint{3.725152in}{4.205380in}}%
\pgfpathlineto{\pgfqpoint{3.732892in}{4.411180in}}%
\pgfpathlineto{\pgfqpoint{3.765927in}{4.384778in}}%
\pgfpathlineto{\pgfqpoint{3.799562in}{4.191284in}}%
\pgfpathlineto{\pgfqpoint{3.790763in}{4.211756in}}%
\pgfpathlineto{\pgfqpoint{3.781690in}{4.304895in}}%
\pgfpathlineto{\pgfqpoint{3.748624in}{4.364344in}}%
\pgfpathlineto{\pgfqpoint{3.715732in}{4.374021in}}%
\pgfpathclose%
\pgfusepath{fill}%
\end{pgfscope}%
\begin{pgfscope}%
\pgfpathrectangle{\pgfqpoint{1.020000in}{0.880000in}}{\pgfqpoint{6.160000in}{6.160000in}}%
\pgfusepath{clip}%
\pgfsetbuttcap%
\pgfsetroundjoin%
\definecolor{currentfill}{rgb}{0.343278,0.459354,0.884122}%
\pgfsetfillcolor{currentfill}%
\pgfsetlinewidth{0.000000pt}%
\definecolor{currentstroke}{rgb}{0.000000,0.000000,0.000000}%
\pgfsetstrokecolor{currentstroke}%
\pgfsetdash{}{0pt}%
\pgfpathmoveto{\pgfqpoint{4.862230in}{3.433618in}}%
\pgfpathlineto{\pgfqpoint{4.869865in}{3.111881in}}%
\pgfpathlineto{\pgfqpoint{4.882252in}{3.476403in}}%
\pgfpathlineto{\pgfqpoint{4.913263in}{3.256969in}}%
\pgfpathlineto{\pgfqpoint{4.945287in}{3.191362in}}%
\pgfpathlineto{\pgfqpoint{4.936484in}{3.338236in}}%
\pgfpathlineto{\pgfqpoint{4.926325in}{3.303913in}}%
\pgfpathlineto{\pgfqpoint{4.893954in}{3.319017in}}%
\pgfpathlineto{\pgfqpoint{4.862230in}{3.433618in}}%
\pgfpathclose%
\pgfusepath{fill}%
\end{pgfscope}%
\begin{pgfscope}%
\pgfpathrectangle{\pgfqpoint{1.020000in}{0.880000in}}{\pgfqpoint{6.160000in}{6.160000in}}%
\pgfusepath{clip}%
\pgfsetbuttcap%
\pgfsetroundjoin%
\definecolor{currentfill}{rgb}{0.323718,0.433158,0.864722}%
\pgfsetfillcolor{currentfill}%
\pgfsetlinewidth{0.000000pt}%
\definecolor{currentstroke}{rgb}{0.000000,0.000000,0.000000}%
\pgfsetstrokecolor{currentstroke}%
\pgfsetdash{}{0pt}%
\pgfpathmoveto{\pgfqpoint{5.077014in}{3.403907in}}%
\pgfpathlineto{\pgfqpoint{5.086556in}{3.344902in}}%
\pgfpathlineto{\pgfqpoint{5.093847in}{3.038512in}}%
\pgfpathlineto{\pgfqpoint{5.128778in}{3.303637in}}%
\pgfpathlineto{\pgfqpoint{5.161153in}{3.292024in}}%
\pgfpathlineto{\pgfqpoint{5.149265in}{3.112395in}}%
\pgfpathlineto{\pgfqpoint{5.141764in}{3.384312in}}%
\pgfpathlineto{\pgfqpoint{5.108352in}{3.280846in}}%
\pgfpathlineto{\pgfqpoint{5.077014in}{3.403907in}}%
\pgfpathclose%
\pgfusepath{fill}%
\end{pgfscope}%
\begin{pgfscope}%
\pgfpathrectangle{\pgfqpoint{1.020000in}{0.880000in}}{\pgfqpoint{6.160000in}{6.160000in}}%
\pgfusepath{clip}%
\pgfsetbuttcap%
\pgfsetroundjoin%
\definecolor{currentfill}{rgb}{0.930669,0.818877,0.759146}%
\pgfsetfillcolor{currentfill}%
\pgfsetlinewidth{0.000000pt}%
\definecolor{currentstroke}{rgb}{0.000000,0.000000,0.000000}%
\pgfsetstrokecolor{currentstroke}%
\pgfsetdash{}{0pt}%
\pgfpathmoveto{\pgfqpoint{3.632912in}{4.361586in}}%
\pgfpathlineto{\pgfqpoint{3.640482in}{4.544942in}}%
\pgfpathlineto{\pgfqpoint{3.649294in}{4.502554in}}%
\pgfpathlineto{\pgfqpoint{3.682402in}{4.466719in}}%
\pgfpathlineto{\pgfqpoint{3.715732in}{4.374021in}}%
\pgfpathlineto{\pgfqpoint{3.707382in}{4.313214in}}%
\pgfpathlineto{\pgfqpoint{3.697615in}{4.553397in}}%
\pgfpathlineto{\pgfqpoint{3.664708in}{4.561540in}}%
\pgfpathlineto{\pgfqpoint{3.632912in}{4.361586in}}%
\pgfpathclose%
\pgfusepath{fill}%
\end{pgfscope}%
\begin{pgfscope}%
\pgfpathrectangle{\pgfqpoint{1.020000in}{0.880000in}}{\pgfqpoint{6.160000in}{6.160000in}}%
\pgfusepath{clip}%
\pgfsetbuttcap%
\pgfsetroundjoin%
\definecolor{currentfill}{rgb}{0.368507,0.491141,0.905243}%
\pgfsetfillcolor{currentfill}%
\pgfsetlinewidth{0.000000pt}%
\definecolor{currentstroke}{rgb}{0.000000,0.000000,0.000000}%
\pgfsetstrokecolor{currentstroke}%
\pgfsetdash{}{0pt}%
\pgfpathmoveto{\pgfqpoint{5.421736in}{3.403512in}}%
\pgfpathlineto{\pgfqpoint{5.431709in}{3.367188in}}%
\pgfpathlineto{\pgfqpoint{5.440938in}{3.270656in}}%
\pgfpathlineto{\pgfqpoint{5.473229in}{3.264503in}}%
\pgfpathlineto{\pgfqpoint{5.506953in}{3.368307in}}%
\pgfpathlineto{\pgfqpoint{5.496494in}{3.371930in}}%
\pgfpathlineto{\pgfqpoint{5.486612in}{3.418257in}}%
\pgfpathlineto{\pgfqpoint{5.453689in}{3.372160in}}%
\pgfpathlineto{\pgfqpoint{5.421736in}{3.403512in}}%
\pgfpathclose%
\pgfusepath{fill}%
\end{pgfscope}%
\begin{pgfscope}%
\pgfpathrectangle{\pgfqpoint{1.020000in}{0.880000in}}{\pgfqpoint{6.160000in}{6.160000in}}%
\pgfusepath{clip}%
\pgfsetbuttcap%
\pgfsetroundjoin%
\definecolor{currentfill}{rgb}{0.958279,0.604335,0.483297}%
\pgfsetfillcolor{currentfill}%
\pgfsetlinewidth{0.000000pt}%
\definecolor{currentstroke}{rgb}{0.000000,0.000000,0.000000}%
\pgfsetstrokecolor{currentstroke}%
\pgfsetdash{}{0pt}%
\pgfpathmoveto{\pgfqpoint{3.101800in}{4.861007in}}%
\pgfpathlineto{\pgfqpoint{3.107924in}{5.039979in}}%
\pgfpathlineto{\pgfqpoint{3.115514in}{5.083716in}}%
\pgfpathlineto{\pgfqpoint{3.151576in}{4.795827in}}%
\pgfpathlineto{\pgfqpoint{3.183202in}{4.937327in}}%
\pgfpathlineto{\pgfqpoint{3.176544in}{4.791719in}}%
\pgfpathlineto{\pgfqpoint{3.167993in}{4.839231in}}%
\pgfpathlineto{\pgfqpoint{3.135785in}{4.765426in}}%
\pgfpathlineto{\pgfqpoint{3.101800in}{4.861007in}}%
\pgfpathclose%
\pgfusepath{fill}%
\end{pgfscope}%
\begin{pgfscope}%
\pgfpathrectangle{\pgfqpoint{1.020000in}{0.880000in}}{\pgfqpoint{6.160000in}{6.160000in}}%
\pgfusepath{clip}%
\pgfsetbuttcap%
\pgfsetroundjoin%
\definecolor{currentfill}{rgb}{0.939254,0.539581,0.423900}%
\pgfsetfillcolor{currentfill}%
\pgfsetlinewidth{0.000000pt}%
\definecolor{currentstroke}{rgb}{0.000000,0.000000,0.000000}%
\pgfsetstrokecolor{currentstroke}%
\pgfsetdash{}{0pt}%
\pgfpathmoveto{\pgfqpoint{2.953170in}{4.920339in}}%
\pgfpathlineto{\pgfqpoint{2.958172in}{5.157992in}}%
\pgfpathlineto{\pgfqpoint{2.969120in}{4.903103in}}%
\pgfpathlineto{\pgfqpoint{2.999782in}{5.111324in}}%
\pgfpathlineto{\pgfqpoint{3.034801in}{4.944862in}}%
\pgfpathlineto{\pgfqpoint{3.027167in}{4.917565in}}%
\pgfpathlineto{\pgfqpoint{3.018196in}{5.008574in}}%
\pgfpathlineto{\pgfqpoint{2.986031in}{4.934529in}}%
\pgfpathlineto{\pgfqpoint{2.953170in}{4.920339in}}%
\pgfpathclose%
\pgfusepath{fill}%
\end{pgfscope}%
\begin{pgfscope}%
\pgfpathrectangle{\pgfqpoint{1.020000in}{0.880000in}}{\pgfqpoint{6.160000in}{6.160000in}}%
\pgfusepath{clip}%
\pgfsetbuttcap%
\pgfsetroundjoin%
\definecolor{currentfill}{rgb}{0.967711,0.662973,0.544323}%
\pgfsetfillcolor{currentfill}%
\pgfsetlinewidth{0.000000pt}%
\definecolor{currentstroke}{rgb}{0.000000,0.000000,0.000000}%
\pgfsetstrokecolor{currentstroke}%
\pgfsetdash{}{0pt}%
\pgfpathmoveto{\pgfqpoint{2.475609in}{4.824261in}}%
\pgfpathlineto{\pgfqpoint{2.483193in}{4.812078in}}%
\pgfpathlineto{\pgfqpoint{2.490200in}{4.835217in}}%
\pgfpathlineto{\pgfqpoint{2.522847in}{4.873091in}}%
\pgfpathlineto{\pgfqpoint{2.557707in}{4.773304in}}%
\pgfpathlineto{\pgfqpoint{2.550800in}{4.739283in}}%
\pgfpathlineto{\pgfqpoint{2.542402in}{4.798644in}}%
\pgfpathlineto{\pgfqpoint{2.510617in}{4.714697in}}%
\pgfpathlineto{\pgfqpoint{2.475609in}{4.824261in}}%
\pgfpathclose%
\pgfusepath{fill}%
\end{pgfscope}%
\begin{pgfscope}%
\pgfpathrectangle{\pgfqpoint{1.020000in}{0.880000in}}{\pgfqpoint{6.160000in}{6.160000in}}%
\pgfusepath{clip}%
\pgfsetbuttcap%
\pgfsetroundjoin%
\definecolor{currentfill}{rgb}{0.457046,0.594006,0.963029}%
\pgfsetfillcolor{currentfill}%
\pgfsetlinewidth{0.000000pt}%
\definecolor{currentstroke}{rgb}{0.000000,0.000000,0.000000}%
\pgfsetstrokecolor{currentstroke}%
\pgfsetdash{}{0pt}%
\pgfpathmoveto{\pgfqpoint{4.563499in}{3.558497in}}%
\pgfpathlineto{\pgfqpoint{4.572918in}{3.528895in}}%
\pgfpathlineto{\pgfqpoint{4.582707in}{3.588716in}}%
\pgfpathlineto{\pgfqpoint{4.614804in}{3.453872in}}%
\pgfpathlineto{\pgfqpoint{4.647331in}{3.434553in}}%
\pgfpathlineto{\pgfqpoint{4.637296in}{3.339744in}}%
\pgfpathlineto{\pgfqpoint{4.628570in}{3.530375in}}%
\pgfpathlineto{\pgfqpoint{4.596369in}{3.622136in}}%
\pgfpathlineto{\pgfqpoint{4.563499in}{3.558497in}}%
\pgfpathclose%
\pgfusepath{fill}%
\end{pgfscope}%
\begin{pgfscope}%
\pgfpathrectangle{\pgfqpoint{1.020000in}{0.880000in}}{\pgfqpoint{6.160000in}{6.160000in}}%
\pgfusepath{clip}%
\pgfsetbuttcap%
\pgfsetroundjoin%
\definecolor{currentfill}{rgb}{0.962701,0.628218,0.507636}%
\pgfsetfillcolor{currentfill}%
\pgfsetlinewidth{0.000000pt}%
\definecolor{currentstroke}{rgb}{0.000000,0.000000,0.000000}%
\pgfsetstrokecolor{currentstroke}%
\pgfsetdash{}{0pt}%
\pgfpathmoveto{\pgfqpoint{2.542402in}{4.798644in}}%
\pgfpathlineto{\pgfqpoint{2.550800in}{4.739283in}}%
\pgfpathlineto{\pgfqpoint{2.557707in}{4.773304in}}%
\pgfpathlineto{\pgfqpoint{2.589068in}{4.890868in}}%
\pgfpathlineto{\pgfqpoint{2.621085in}{4.969548in}}%
\pgfpathlineto{\pgfqpoint{2.615975in}{4.812822in}}%
\pgfpathlineto{\pgfqpoint{2.607700in}{4.862265in}}%
\pgfpathlineto{\pgfqpoint{2.572443in}{4.994832in}}%
\pgfpathlineto{\pgfqpoint{2.542402in}{4.798644in}}%
\pgfpathclose%
\pgfusepath{fill}%
\end{pgfscope}%
\begin{pgfscope}%
\pgfpathrectangle{\pgfqpoint{1.020000in}{0.880000in}}{\pgfqpoint{6.160000in}{6.160000in}}%
\pgfusepath{clip}%
\pgfsetbuttcap%
\pgfsetroundjoin%
\definecolor{currentfill}{rgb}{0.969522,0.700833,0.587508}%
\pgfsetfillcolor{currentfill}%
\pgfsetlinewidth{0.000000pt}%
\definecolor{currentstroke}{rgb}{0.000000,0.000000,0.000000}%
\pgfsetstrokecolor{currentstroke}%
\pgfsetdash{}{0pt}%
\pgfpathmoveto{\pgfqpoint{3.399463in}{4.727299in}}%
\pgfpathlineto{\pgfqpoint{3.407408in}{4.777183in}}%
\pgfpathlineto{\pgfqpoint{3.415392in}{4.825324in}}%
\pgfpathlineto{\pgfqpoint{3.448582in}{4.802406in}}%
\pgfpathlineto{\pgfqpoint{3.482392in}{4.684919in}}%
\pgfpathlineto{\pgfqpoint{3.473966in}{4.688079in}}%
\pgfpathlineto{\pgfqpoint{3.466082in}{4.617073in}}%
\pgfpathlineto{\pgfqpoint{3.432866in}{4.663365in}}%
\pgfpathlineto{\pgfqpoint{3.399463in}{4.727299in}}%
\pgfpathclose%
\pgfusepath{fill}%
\end{pgfscope}%
\begin{pgfscope}%
\pgfpathrectangle{\pgfqpoint{1.020000in}{0.880000in}}{\pgfqpoint{6.160000in}{6.160000in}}%
\pgfusepath{clip}%
\pgfsetbuttcap%
\pgfsetroundjoin%
\definecolor{currentfill}{rgb}{0.951254,0.578799,0.459408}%
\pgfsetfillcolor{currentfill}%
\pgfsetlinewidth{0.000000pt}%
\definecolor{currentstroke}{rgb}{0.000000,0.000000,0.000000}%
\pgfsetstrokecolor{currentstroke}%
\pgfsetdash{}{0pt}%
\pgfpathmoveto{\pgfqpoint{2.607700in}{4.862265in}}%
\pgfpathlineto{\pgfqpoint{2.615975in}{4.812822in}}%
\pgfpathlineto{\pgfqpoint{2.621085in}{4.969548in}}%
\pgfpathlineto{\pgfqpoint{2.655352in}{4.900962in}}%
\pgfpathlineto{\pgfqpoint{2.687376in}{4.980307in}}%
\pgfpathlineto{\pgfqpoint{2.680281in}{4.946351in}}%
\pgfpathlineto{\pgfqpoint{2.671662in}{5.016152in}}%
\pgfpathlineto{\pgfqpoint{2.639743in}{4.934116in}}%
\pgfpathlineto{\pgfqpoint{2.607700in}{4.862265in}}%
\pgfpathclose%
\pgfusepath{fill}%
\end{pgfscope}%
\begin{pgfscope}%
\pgfpathrectangle{\pgfqpoint{1.020000in}{0.880000in}}{\pgfqpoint{6.160000in}{6.160000in}}%
\pgfusepath{clip}%
\pgfsetbuttcap%
\pgfsetroundjoin%
\definecolor{currentfill}{rgb}{0.576051,0.708780,0.997755}%
\pgfsetfillcolor{currentfill}%
\pgfsetlinewidth{0.000000pt}%
\definecolor{currentstroke}{rgb}{0.000000,0.000000,0.000000}%
\pgfsetstrokecolor{currentstroke}%
\pgfsetdash{}{0pt}%
\pgfpathmoveto{\pgfqpoint{4.265237in}{3.822841in}}%
\pgfpathlineto{\pgfqpoint{4.274412in}{3.762393in}}%
\pgfpathlineto{\pgfqpoint{4.283520in}{3.627141in}}%
\pgfpathlineto{\pgfqpoint{4.316324in}{3.691324in}}%
\pgfpathlineto{\pgfqpoint{4.349185in}{3.776055in}}%
\pgfpathlineto{\pgfqpoint{4.339438in}{3.494178in}}%
\pgfpathlineto{\pgfqpoint{4.330609in}{3.789136in}}%
\pgfpathlineto{\pgfqpoint{4.297877in}{3.752649in}}%
\pgfpathlineto{\pgfqpoint{4.265237in}{3.822841in}}%
\pgfpathclose%
\pgfusepath{fill}%
\end{pgfscope}%
\begin{pgfscope}%
\pgfpathrectangle{\pgfqpoint{1.020000in}{0.880000in}}{\pgfqpoint{6.160000in}{6.160000in}}%
\pgfusepath{clip}%
\pgfsetbuttcap%
\pgfsetroundjoin%
\definecolor{currentfill}{rgb}{0.363461,0.484784,0.901019}%
\pgfsetfillcolor{currentfill}%
\pgfsetlinewidth{0.000000pt}%
\definecolor{currentstroke}{rgb}{0.000000,0.000000,0.000000}%
\pgfsetstrokecolor{currentstroke}%
\pgfsetdash{}{0pt}%
\pgfpathmoveto{\pgfqpoint{5.355934in}{3.314725in}}%
\pgfpathlineto{\pgfqpoint{5.366906in}{3.366736in}}%
\pgfpathlineto{\pgfqpoint{5.376504in}{3.300964in}}%
\pgfpathlineto{\pgfqpoint{5.408832in}{3.293658in}}%
\pgfpathlineto{\pgfqpoint{5.440938in}{3.270656in}}%
\pgfpathlineto{\pgfqpoint{5.431709in}{3.367188in}}%
\pgfpathlineto{\pgfqpoint{5.421736in}{3.403512in}}%
\pgfpathlineto{\pgfqpoint{5.389064in}{3.378506in}}%
\pgfpathlineto{\pgfqpoint{5.355934in}{3.314725in}}%
\pgfpathclose%
\pgfusepath{fill}%
\end{pgfscope}%
\begin{pgfscope}%
\pgfpathrectangle{\pgfqpoint{1.020000in}{0.880000in}}{\pgfqpoint{6.160000in}{6.160000in}}%
\pgfusepath{clip}%
\pgfsetbuttcap%
\pgfsetroundjoin%
\definecolor{currentfill}{rgb}{0.378598,0.503856,0.913692}%
\pgfsetfillcolor{currentfill}%
\pgfsetlinewidth{0.000000pt}%
\definecolor{currentstroke}{rgb}{0.000000,0.000000,0.000000}%
\pgfsetstrokecolor{currentstroke}%
\pgfsetdash{}{0pt}%
\pgfpathmoveto{\pgfqpoint{5.634555in}{3.248646in}}%
\pgfpathlineto{\pgfqpoint{5.646153in}{3.314132in}}%
\pgfpathlineto{\pgfqpoint{5.657632in}{3.369127in}}%
\pgfpathlineto{\pgfqpoint{5.690041in}{3.372979in}}%
\pgfpathlineto{\pgfqpoint{5.721876in}{3.340706in}}%
\pgfpathlineto{\pgfqpoint{5.712896in}{3.455269in}}%
\pgfpathlineto{\pgfqpoint{5.699752in}{3.295765in}}%
\pgfpathlineto{\pgfqpoint{5.670509in}{3.497842in}}%
\pgfpathlineto{\pgfqpoint{5.634555in}{3.248646in}}%
\pgfpathclose%
\pgfusepath{fill}%
\end{pgfscope}%
\begin{pgfscope}%
\pgfpathrectangle{\pgfqpoint{1.020000in}{0.880000in}}{\pgfqpoint{6.160000in}{6.160000in}}%
\pgfusepath{clip}%
\pgfsetbuttcap%
\pgfsetroundjoin%
\definecolor{currentfill}{rgb}{0.521696,0.659599,0.987736}%
\pgfsetfillcolor{currentfill}%
\pgfsetlinewidth{0.000000pt}%
\definecolor{currentstroke}{rgb}{0.000000,0.000000,0.000000}%
\pgfsetstrokecolor{currentstroke}%
\pgfsetdash{}{0pt}%
\pgfpathmoveto{\pgfqpoint{4.414221in}{3.601033in}}%
\pgfpathlineto{\pgfqpoint{4.423503in}{3.562973in}}%
\pgfpathlineto{\pgfqpoint{4.433414in}{3.772296in}}%
\pgfpathlineto{\pgfqpoint{4.465789in}{3.638225in}}%
\pgfpathlineto{\pgfqpoint{4.498387in}{3.608944in}}%
\pgfpathlineto{\pgfqpoint{4.489052in}{3.648768in}}%
\pgfpathlineto{\pgfqpoint{4.479105in}{3.477149in}}%
\pgfpathlineto{\pgfqpoint{4.447085in}{3.688195in}}%
\pgfpathlineto{\pgfqpoint{4.414221in}{3.601033in}}%
\pgfpathclose%
\pgfusepath{fill}%
\end{pgfscope}%
\begin{pgfscope}%
\pgfpathrectangle{\pgfqpoint{1.020000in}{0.880000in}}{\pgfqpoint{6.160000in}{6.160000in}}%
\pgfusepath{clip}%
\pgfsetbuttcap%
\pgfsetroundjoin%
\definecolor{currentfill}{rgb}{0.338377,0.452819,0.879317}%
\pgfsetfillcolor{currentfill}%
\pgfsetlinewidth{0.000000pt}%
\definecolor{currentstroke}{rgb}{0.000000,0.000000,0.000000}%
\pgfsetstrokecolor{currentstroke}%
\pgfsetdash{}{0pt}%
\pgfpathmoveto{\pgfqpoint{4.795622in}{3.193746in}}%
\pgfpathlineto{\pgfqpoint{4.805940in}{3.282412in}}%
\pgfpathlineto{\pgfqpoint{4.816103in}{3.340633in}}%
\pgfpathlineto{\pgfqpoint{4.848292in}{3.277435in}}%
\pgfpathlineto{\pgfqpoint{4.882252in}{3.476403in}}%
\pgfpathlineto{\pgfqpoint{4.869865in}{3.111881in}}%
\pgfpathlineto{\pgfqpoint{4.862230in}{3.433618in}}%
\pgfpathlineto{\pgfqpoint{4.828242in}{3.212460in}}%
\pgfpathlineto{\pgfqpoint{4.795622in}{3.193746in}}%
\pgfpathclose%
\pgfusepath{fill}%
\end{pgfscope}%
\begin{pgfscope}%
\pgfpathrectangle{\pgfqpoint{1.020000in}{0.880000in}}{\pgfqpoint{6.160000in}{6.160000in}}%
\pgfusepath{clip}%
\pgfsetbuttcap%
\pgfsetroundjoin%
\definecolor{currentfill}{rgb}{0.388852,0.516298,0.921373}%
\pgfsetfillcolor{currentfill}%
\pgfsetlinewidth{0.000000pt}%
\definecolor{currentstroke}{rgb}{0.000000,0.000000,0.000000}%
\pgfsetstrokecolor{currentstroke}%
\pgfsetdash{}{0pt}%
\pgfpathmoveto{\pgfqpoint{4.647331in}{3.434553in}}%
\pgfpathlineto{\pgfqpoint{4.656932in}{3.429691in}}%
\pgfpathlineto{\pgfqpoint{4.666057in}{3.322380in}}%
\pgfpathlineto{\pgfqpoint{4.698949in}{3.375745in}}%
\pgfpathlineto{\pgfqpoint{4.731683in}{3.397301in}}%
\pgfpathlineto{\pgfqpoint{4.722048in}{3.409962in}}%
\pgfpathlineto{\pgfqpoint{4.711659in}{3.276658in}}%
\pgfpathlineto{\pgfqpoint{4.679926in}{3.435373in}}%
\pgfpathlineto{\pgfqpoint{4.647331in}{3.434553in}}%
\pgfpathclose%
\pgfusepath{fill}%
\end{pgfscope}%
\begin{pgfscope}%
\pgfpathrectangle{\pgfqpoint{1.020000in}{0.880000in}}{\pgfqpoint{6.160000in}{6.160000in}}%
\pgfusepath{clip}%
\pgfsetbuttcap%
\pgfsetroundjoin%
\definecolor{currentfill}{rgb}{0.338377,0.452819,0.879317}%
\pgfsetfillcolor{currentfill}%
\pgfsetlinewidth{0.000000pt}%
\definecolor{currentstroke}{rgb}{0.000000,0.000000,0.000000}%
\pgfsetstrokecolor{currentstroke}%
\pgfsetdash{}{0pt}%
\pgfpathmoveto{\pgfqpoint{5.291042in}{3.304426in}}%
\pgfpathlineto{\pgfqpoint{5.299457in}{3.136565in}}%
\pgfpathlineto{\pgfqpoint{5.311848in}{3.321820in}}%
\pgfpathlineto{\pgfqpoint{5.343260in}{3.231441in}}%
\pgfpathlineto{\pgfqpoint{5.376504in}{3.300964in}}%
\pgfpathlineto{\pgfqpoint{5.366906in}{3.366736in}}%
\pgfpathlineto{\pgfqpoint{5.355934in}{3.314725in}}%
\pgfpathlineto{\pgfqpoint{5.323604in}{3.319093in}}%
\pgfpathlineto{\pgfqpoint{5.291042in}{3.304426in}}%
\pgfpathclose%
\pgfusepath{fill}%
\end{pgfscope}%
\begin{pgfscope}%
\pgfpathrectangle{\pgfqpoint{1.020000in}{0.880000in}}{\pgfqpoint{6.160000in}{6.160000in}}%
\pgfusepath{clip}%
\pgfsetbuttcap%
\pgfsetroundjoin%
\definecolor{currentfill}{rgb}{0.839351,0.861167,0.894494}%
\pgfsetfillcolor{currentfill}%
\pgfsetlinewidth{0.000000pt}%
\definecolor{currentstroke}{rgb}{0.000000,0.000000,0.000000}%
\pgfsetstrokecolor{currentstroke}%
\pgfsetdash{}{0pt}%
\pgfpathmoveto{\pgfqpoint{3.799562in}{4.191284in}}%
\pgfpathlineto{\pgfqpoint{3.808187in}{4.221616in}}%
\pgfpathlineto{\pgfqpoint{3.816401in}{4.371256in}}%
\pgfpathlineto{\pgfqpoint{3.850070in}{4.150054in}}%
\pgfpathlineto{\pgfqpoint{3.882568in}{4.267118in}}%
\pgfpathlineto{\pgfqpoint{3.873788in}{4.252484in}}%
\pgfpathlineto{\pgfqpoint{3.865413in}{4.116331in}}%
\pgfpathlineto{\pgfqpoint{3.832515in}{4.152111in}}%
\pgfpathlineto{\pgfqpoint{3.799562in}{4.191284in}}%
\pgfpathclose%
\pgfusepath{fill}%
\end{pgfscope}%
\begin{pgfscope}%
\pgfpathrectangle{\pgfqpoint{1.020000in}{0.880000in}}{\pgfqpoint{6.160000in}{6.160000in}}%
\pgfusepath{clip}%
\pgfsetbuttcap%
\pgfsetroundjoin%
\definecolor{currentfill}{rgb}{0.698454,0.799450,0.984577}%
\pgfsetfillcolor{currentfill}%
\pgfsetlinewidth{0.000000pt}%
\definecolor{currentstroke}{rgb}{0.000000,0.000000,0.000000}%
\pgfsetstrokecolor{currentstroke}%
\pgfsetdash{}{0pt}%
\pgfpathmoveto{\pgfqpoint{3.966734in}{3.978153in}}%
\pgfpathlineto{\pgfqpoint{3.975932in}{3.842145in}}%
\pgfpathlineto{\pgfqpoint{3.984626in}{3.958536in}}%
\pgfpathlineto{\pgfqpoint{4.017555in}{3.912777in}}%
\pgfpathlineto{\pgfqpoint{4.050520in}{3.802340in}}%
\pgfpathlineto{\pgfqpoint{4.041381in}{3.913043in}}%
\pgfpathlineto{\pgfqpoint{4.032223in}{4.035498in}}%
\pgfpathlineto{\pgfqpoint{3.999439in}{4.028631in}}%
\pgfpathlineto{\pgfqpoint{3.966734in}{3.978153in}}%
\pgfpathclose%
\pgfusepath{fill}%
\end{pgfscope}%
\begin{pgfscope}%
\pgfpathrectangle{\pgfqpoint{1.020000in}{0.880000in}}{\pgfqpoint{6.160000in}{6.160000in}}%
\pgfusepath{clip}%
\pgfsetbuttcap%
\pgfsetroundjoin%
\definecolor{currentfill}{rgb}{0.958279,0.604335,0.483297}%
\pgfsetfillcolor{currentfill}%
\pgfsetlinewidth{0.000000pt}%
\definecolor{currentstroke}{rgb}{0.000000,0.000000,0.000000}%
\pgfsetstrokecolor{currentstroke}%
\pgfsetdash{}{0pt}%
\pgfpathmoveto{\pgfqpoint{3.249751in}{4.885875in}}%
\pgfpathlineto{\pgfqpoint{3.257464in}{4.937006in}}%
\pgfpathlineto{\pgfqpoint{3.266410in}{4.854631in}}%
\pgfpathlineto{\pgfqpoint{3.298457in}{4.969162in}}%
\pgfpathlineto{\pgfqpoint{3.332238in}{4.883757in}}%
\pgfpathlineto{\pgfqpoint{3.323683in}{4.918572in}}%
\pgfpathlineto{\pgfqpoint{3.317159in}{4.717267in}}%
\pgfpathlineto{\pgfqpoint{3.282843in}{4.874627in}}%
\pgfpathlineto{\pgfqpoint{3.249751in}{4.885875in}}%
\pgfpathclose%
\pgfusepath{fill}%
\end{pgfscope}%
\begin{pgfscope}%
\pgfpathrectangle{\pgfqpoint{1.020000in}{0.880000in}}{\pgfqpoint{6.160000in}{6.160000in}}%
\pgfusepath{clip}%
\pgfsetbuttcap%
\pgfsetroundjoin%
\definecolor{currentfill}{rgb}{0.966017,0.646130,0.525890}%
\pgfsetfillcolor{currentfill}%
\pgfsetlinewidth{0.000000pt}%
\definecolor{currentstroke}{rgb}{0.000000,0.000000,0.000000}%
\pgfsetstrokecolor{currentstroke}%
\pgfsetdash{}{0pt}%
\pgfpathmoveto{\pgfqpoint{2.412504in}{4.632768in}}%
\pgfpathlineto{\pgfqpoint{2.414958in}{4.910993in}}%
\pgfpathlineto{\pgfqpoint{2.423721in}{4.828822in}}%
\pgfpathlineto{\pgfqpoint{2.456357in}{4.867936in}}%
\pgfpathlineto{\pgfqpoint{2.490200in}{4.835217in}}%
\pgfpathlineto{\pgfqpoint{2.483193in}{4.812078in}}%
\pgfpathlineto{\pgfqpoint{2.475609in}{4.824261in}}%
\pgfpathlineto{\pgfqpoint{2.443161in}{4.779410in}}%
\pgfpathlineto{\pgfqpoint{2.412504in}{4.632768in}}%
\pgfpathclose%
\pgfusepath{fill}%
\end{pgfscope}%
\begin{pgfscope}%
\pgfpathrectangle{\pgfqpoint{1.020000in}{0.880000in}}{\pgfqpoint{6.160000in}{6.160000in}}%
\pgfusepath{clip}%
\pgfsetbuttcap%
\pgfsetroundjoin%
\definecolor{currentfill}{rgb}{0.373552,0.497499,0.909467}%
\pgfsetfillcolor{currentfill}%
\pgfsetlinewidth{0.000000pt}%
\definecolor{currentstroke}{rgb}{0.000000,0.000000,0.000000}%
\pgfsetstrokecolor{currentstroke}%
\pgfsetdash{}{0pt}%
\pgfpathmoveto{\pgfqpoint{5.572265in}{3.411738in}}%
\pgfpathlineto{\pgfqpoint{5.581379in}{3.305984in}}%
\pgfpathlineto{\pgfqpoint{5.591768in}{3.290637in}}%
\pgfpathlineto{\pgfqpoint{5.625250in}{3.368184in}}%
\pgfpathlineto{\pgfqpoint{5.657632in}{3.369127in}}%
\pgfpathlineto{\pgfqpoint{5.646153in}{3.314132in}}%
\pgfpathlineto{\pgfqpoint{5.634555in}{3.248646in}}%
\pgfpathlineto{\pgfqpoint{5.606066in}{3.514476in}}%
\pgfpathlineto{\pgfqpoint{5.572265in}{3.411738in}}%
\pgfpathclose%
\pgfusepath{fill}%
\end{pgfscope}%
\begin{pgfscope}%
\pgfpathrectangle{\pgfqpoint{1.020000in}{0.880000in}}{\pgfqpoint{6.160000in}{6.160000in}}%
\pgfusepath{clip}%
\pgfsetbuttcap%
\pgfsetroundjoin%
\definecolor{currentfill}{rgb}{0.656683,0.771806,0.994914}%
\pgfsetfillcolor{currentfill}%
\pgfsetlinewidth{0.000000pt}%
\definecolor{currentstroke}{rgb}{0.000000,0.000000,0.000000}%
\pgfsetstrokecolor{currentstroke}%
\pgfsetdash{}{0pt}%
\pgfpathmoveto{\pgfqpoint{4.115895in}{4.003131in}}%
\pgfpathlineto{\pgfqpoint{4.125004in}{3.958727in}}%
\pgfpathlineto{\pgfqpoint{4.134165in}{3.829685in}}%
\pgfpathlineto{\pgfqpoint{4.166956in}{3.869696in}}%
\pgfpathlineto{\pgfqpoint{4.199733in}{3.779023in}}%
\pgfpathlineto{\pgfqpoint{4.190598in}{3.897661in}}%
\pgfpathlineto{\pgfqpoint{4.181465in}{3.711587in}}%
\pgfpathlineto{\pgfqpoint{4.148751in}{3.816753in}}%
\pgfpathlineto{\pgfqpoint{4.115895in}{4.003131in}}%
\pgfpathclose%
\pgfusepath{fill}%
\end{pgfscope}%
\begin{pgfscope}%
\pgfpathrectangle{\pgfqpoint{1.020000in}{0.880000in}}{\pgfqpoint{6.160000in}{6.160000in}}%
\pgfusepath{clip}%
\pgfsetbuttcap%
\pgfsetroundjoin%
\definecolor{currentfill}{rgb}{0.333490,0.446265,0.874452}%
\pgfsetfillcolor{currentfill}%
\pgfsetlinewidth{0.000000pt}%
\definecolor{currentstroke}{rgb}{0.000000,0.000000,0.000000}%
\pgfsetstrokecolor{currentstroke}%
\pgfsetdash{}{0pt}%
\pgfpathmoveto{\pgfqpoint{5.224766in}{3.167498in}}%
\pgfpathlineto{\pgfqpoint{5.235366in}{3.203690in}}%
\pgfpathlineto{\pgfqpoint{5.248592in}{3.483840in}}%
\pgfpathlineto{\pgfqpoint{5.279558in}{3.339203in}}%
\pgfpathlineto{\pgfqpoint{5.311848in}{3.321820in}}%
\pgfpathlineto{\pgfqpoint{5.299457in}{3.136565in}}%
\pgfpathlineto{\pgfqpoint{5.291042in}{3.304426in}}%
\pgfpathlineto{\pgfqpoint{5.257619in}{3.210621in}}%
\pgfpathlineto{\pgfqpoint{5.224766in}{3.167498in}}%
\pgfpathclose%
\pgfusepath{fill}%
\end{pgfscope}%
\begin{pgfscope}%
\pgfpathrectangle{\pgfqpoint{1.020000in}{0.880000in}}{\pgfqpoint{6.160000in}{6.160000in}}%
\pgfusepath{clip}%
\pgfsetbuttcap%
\pgfsetroundjoin%
\definecolor{currentfill}{rgb}{0.929357,0.512254,0.400673}%
\pgfsetfillcolor{currentfill}%
\pgfsetlinewidth{0.000000pt}%
\definecolor{currentstroke}{rgb}{0.000000,0.000000,0.000000}%
\pgfsetstrokecolor{currentstroke}%
\pgfsetdash{}{0pt}%
\pgfpathmoveto{\pgfqpoint{2.886134in}{4.992558in}}%
\pgfpathlineto{\pgfqpoint{2.894282in}{4.965552in}}%
\pgfpathlineto{\pgfqpoint{2.900326in}{5.106334in}}%
\pgfpathlineto{\pgfqpoint{2.933710in}{5.090619in}}%
\pgfpathlineto{\pgfqpoint{2.969120in}{4.903103in}}%
\pgfpathlineto{\pgfqpoint{2.958172in}{5.157992in}}%
\pgfpathlineto{\pgfqpoint{2.953170in}{4.920339in}}%
\pgfpathlineto{\pgfqpoint{2.919467in}{4.973036in}}%
\pgfpathlineto{\pgfqpoint{2.886134in}{4.992558in}}%
\pgfpathclose%
\pgfusepath{fill}%
\end{pgfscope}%
\begin{pgfscope}%
\pgfpathrectangle{\pgfqpoint{1.020000in}{0.880000in}}{\pgfqpoint{6.160000in}{6.160000in}}%
\pgfusepath{clip}%
\pgfsetbuttcap%
\pgfsetroundjoin%
\definecolor{currentfill}{rgb}{0.966962,0.735670,0.630877}%
\pgfsetfillcolor{currentfill}%
\pgfsetlinewidth{0.000000pt}%
\definecolor{currentstroke}{rgb}{0.000000,0.000000,0.000000}%
\pgfsetstrokecolor{currentstroke}%
\pgfsetdash{}{0pt}%
\pgfpathmoveto{\pgfqpoint{3.482392in}{4.684919in}}%
\pgfpathlineto{\pgfqpoint{3.490721in}{4.698296in}}%
\pgfpathlineto{\pgfqpoint{3.498783in}{4.754152in}}%
\pgfpathlineto{\pgfqpoint{3.532267in}{4.682910in}}%
\pgfpathlineto{\pgfqpoint{3.566038in}{4.553756in}}%
\pgfpathlineto{\pgfqpoint{3.557151in}{4.617956in}}%
\pgfpathlineto{\pgfqpoint{3.548943in}{4.574730in}}%
\pgfpathlineto{\pgfqpoint{3.515759in}{4.620065in}}%
\pgfpathlineto{\pgfqpoint{3.482392in}{4.684919in}}%
\pgfpathclose%
\pgfusepath{fill}%
\end{pgfscope}%
\begin{pgfscope}%
\pgfpathrectangle{\pgfqpoint{1.020000in}{0.880000in}}{\pgfqpoint{6.160000in}{6.160000in}}%
\pgfusepath{clip}%
\pgfsetbuttcap%
\pgfsetroundjoin%
\definecolor{currentfill}{rgb}{0.791392,0.846750,0.936641}%
\pgfsetfillcolor{currentfill}%
\pgfsetlinewidth{0.000000pt}%
\definecolor{currentstroke}{rgb}{0.000000,0.000000,0.000000}%
\pgfsetstrokecolor{currentstroke}%
\pgfsetdash{}{0pt}%
\pgfpathmoveto{\pgfqpoint{3.882568in}{4.267118in}}%
\pgfpathlineto{\pgfqpoint{3.891600in}{4.201995in}}%
\pgfpathlineto{\pgfqpoint{3.900653in}{4.131527in}}%
\pgfpathlineto{\pgfqpoint{3.933669in}{4.078870in}}%
\pgfpathlineto{\pgfqpoint{3.966734in}{3.978153in}}%
\pgfpathlineto{\pgfqpoint{3.957754in}{4.014532in}}%
\pgfpathlineto{\pgfqpoint{3.948902in}{4.000686in}}%
\pgfpathlineto{\pgfqpoint{3.915659in}{4.182380in}}%
\pgfpathlineto{\pgfqpoint{3.882568in}{4.267118in}}%
\pgfpathclose%
\pgfusepath{fill}%
\end{pgfscope}%
\begin{pgfscope}%
\pgfpathrectangle{\pgfqpoint{1.020000in}{0.880000in}}{\pgfqpoint{6.160000in}{6.160000in}}%
\pgfusepath{clip}%
\pgfsetbuttcap%
\pgfsetroundjoin%
\definecolor{currentfill}{rgb}{0.248091,0.326013,0.777669}%
\pgfsetfillcolor{currentfill}%
\pgfsetlinewidth{0.000000pt}%
\definecolor{currentstroke}{rgb}{0.000000,0.000000,0.000000}%
\pgfsetstrokecolor{currentstroke}%
\pgfsetdash{}{0pt}%
\pgfpathmoveto{\pgfqpoint{5.028211in}{2.951861in}}%
\pgfpathlineto{\pgfqpoint{5.040220in}{3.184472in}}%
\pgfpathlineto{\pgfqpoint{5.049802in}{3.130096in}}%
\pgfpathlineto{\pgfqpoint{5.082164in}{3.109013in}}%
\pgfpathlineto{\pgfqpoint{5.114904in}{3.132056in}}%
\pgfpathlineto{\pgfqpoint{5.104923in}{3.146151in}}%
\pgfpathlineto{\pgfqpoint{5.093847in}{3.038512in}}%
\pgfpathlineto{\pgfqpoint{5.062245in}{3.133701in}}%
\pgfpathlineto{\pgfqpoint{5.028211in}{2.951861in}}%
\pgfpathclose%
\pgfusepath{fill}%
\end{pgfscope}%
\begin{pgfscope}%
\pgfpathrectangle{\pgfqpoint{1.020000in}{0.880000in}}{\pgfqpoint{6.160000in}{6.160000in}}%
\pgfusepath{clip}%
\pgfsetbuttcap%
\pgfsetroundjoin%
\definecolor{currentfill}{rgb}{0.280550,0.373423,0.818011}%
\pgfsetfillcolor{currentfill}%
\pgfsetlinewidth{0.000000pt}%
\definecolor{currentstroke}{rgb}{0.000000,0.000000,0.000000}%
\pgfsetstrokecolor{currentstroke}%
\pgfsetdash{}{0pt}%
\pgfpathmoveto{\pgfqpoint{5.093847in}{3.038512in}}%
\pgfpathlineto{\pgfqpoint{5.104923in}{3.146151in}}%
\pgfpathlineto{\pgfqpoint{5.114904in}{3.132056in}}%
\pgfpathlineto{\pgfqpoint{5.147854in}{3.177152in}}%
\pgfpathlineto{\pgfqpoint{5.180546in}{3.195382in}}%
\pgfpathlineto{\pgfqpoint{5.169265in}{3.084139in}}%
\pgfpathlineto{\pgfqpoint{5.161153in}{3.292024in}}%
\pgfpathlineto{\pgfqpoint{5.128778in}{3.303637in}}%
\pgfpathlineto{\pgfqpoint{5.093847in}{3.038512in}}%
\pgfpathclose%
\pgfusepath{fill}%
\end{pgfscope}%
\begin{pgfscope}%
\pgfpathrectangle{\pgfqpoint{1.020000in}{0.880000in}}{\pgfqpoint{6.160000in}{6.160000in}}%
\pgfusepath{clip}%
\pgfsetbuttcap%
\pgfsetroundjoin%
\definecolor{currentfill}{rgb}{0.967874,0.725847,0.618489}%
\pgfsetfillcolor{currentfill}%
\pgfsetlinewidth{0.000000pt}%
\definecolor{currentstroke}{rgb}{0.000000,0.000000,0.000000}%
\pgfsetstrokecolor{currentstroke}%
\pgfsetdash{}{0pt}%
\pgfpathmoveto{\pgfqpoint{2.212637in}{4.654464in}}%
\pgfpathlineto{\pgfqpoint{2.218319in}{4.728050in}}%
\pgfpathlineto{\pgfqpoint{2.229414in}{4.522941in}}%
\pgfpathlineto{\pgfqpoint{2.260953in}{4.616507in}}%
\pgfpathlineto{\pgfqpoint{2.291377in}{4.772117in}}%
\pgfpathlineto{\pgfqpoint{2.286789in}{4.632834in}}%
\pgfpathlineto{\pgfqpoint{2.276093in}{4.820096in}}%
\pgfpathlineto{\pgfqpoint{2.247487in}{4.573223in}}%
\pgfpathlineto{\pgfqpoint{2.212637in}{4.654464in}}%
\pgfpathclose%
\pgfusepath{fill}%
\end{pgfscope}%
\begin{pgfscope}%
\pgfpathrectangle{\pgfqpoint{1.020000in}{0.880000in}}{\pgfqpoint{6.160000in}{6.160000in}}%
\pgfusepath{clip}%
\pgfsetbuttcap%
\pgfsetroundjoin%
\definecolor{currentfill}{rgb}{0.353369,0.472069,0.892570}%
\pgfsetfillcolor{currentfill}%
\pgfsetlinewidth{0.000000pt}%
\definecolor{currentstroke}{rgb}{0.000000,0.000000,0.000000}%
\pgfsetstrokecolor{currentstroke}%
\pgfsetdash{}{0pt}%
\pgfpathmoveto{\pgfqpoint{4.731683in}{3.397301in}}%
\pgfpathlineto{\pgfqpoint{4.741215in}{3.363144in}}%
\pgfpathlineto{\pgfqpoint{4.750368in}{3.259728in}}%
\pgfpathlineto{\pgfqpoint{4.783139in}{3.284732in}}%
\pgfpathlineto{\pgfqpoint{4.816103in}{3.340633in}}%
\pgfpathlineto{\pgfqpoint{4.805940in}{3.282412in}}%
\pgfpathlineto{\pgfqpoint{4.795622in}{3.193746in}}%
\pgfpathlineto{\pgfqpoint{4.763938in}{3.337414in}}%
\pgfpathlineto{\pgfqpoint{4.731683in}{3.397301in}}%
\pgfpathclose%
\pgfusepath{fill}%
\end{pgfscope}%
\begin{pgfscope}%
\pgfpathrectangle{\pgfqpoint{1.020000in}{0.880000in}}{\pgfqpoint{6.160000in}{6.160000in}}%
\pgfusepath{clip}%
\pgfsetbuttcap%
\pgfsetroundjoin%
\definecolor{currentfill}{rgb}{0.967711,0.662973,0.544323}%
\pgfsetfillcolor{currentfill}%
\pgfsetlinewidth{0.000000pt}%
\definecolor{currentstroke}{rgb}{0.000000,0.000000,0.000000}%
\pgfsetstrokecolor{currentstroke}%
\pgfsetdash{}{0pt}%
\pgfpathmoveto{\pgfqpoint{2.342374in}{4.837970in}}%
\pgfpathlineto{\pgfqpoint{2.351656in}{4.725689in}}%
\pgfpathlineto{\pgfqpoint{2.358938in}{4.724396in}}%
\pgfpathlineto{\pgfqpoint{2.392188in}{4.727644in}}%
\pgfpathlineto{\pgfqpoint{2.423721in}{4.828822in}}%
\pgfpathlineto{\pgfqpoint{2.414958in}{4.910993in}}%
\pgfpathlineto{\pgfqpoint{2.412504in}{4.632768in}}%
\pgfpathlineto{\pgfqpoint{2.374896in}{4.880484in}}%
\pgfpathlineto{\pgfqpoint{2.342374in}{4.837970in}}%
\pgfpathclose%
\pgfusepath{fill}%
\end{pgfscope}%
\begin{pgfscope}%
\pgfpathrectangle{\pgfqpoint{1.020000in}{0.880000in}}{\pgfqpoint{6.160000in}{6.160000in}}%
\pgfusepath{clip}%
\pgfsetbuttcap%
\pgfsetroundjoin%
\definecolor{currentfill}{rgb}{0.483854,0.622050,0.974808}%
\pgfsetfillcolor{currentfill}%
\pgfsetlinewidth{0.000000pt}%
\definecolor{currentstroke}{rgb}{0.000000,0.000000,0.000000}%
\pgfsetstrokecolor{currentstroke}%
\pgfsetdash{}{0pt}%
\pgfpathmoveto{\pgfqpoint{4.498387in}{3.608944in}}%
\pgfpathlineto{\pgfqpoint{4.507727in}{3.567209in}}%
\pgfpathlineto{\pgfqpoint{4.517133in}{3.541502in}}%
\pgfpathlineto{\pgfqpoint{4.549589in}{3.474305in}}%
\pgfpathlineto{\pgfqpoint{4.582707in}{3.588716in}}%
\pgfpathlineto{\pgfqpoint{4.572918in}{3.528895in}}%
\pgfpathlineto{\pgfqpoint{4.563499in}{3.558497in}}%
\pgfpathlineto{\pgfqpoint{4.530807in}{3.539316in}}%
\pgfpathlineto{\pgfqpoint{4.498387in}{3.608944in}}%
\pgfpathclose%
\pgfusepath{fill}%
\end{pgfscope}%
\begin{pgfscope}%
\pgfpathrectangle{\pgfqpoint{1.020000in}{0.880000in}}{\pgfqpoint{6.160000in}{6.160000in}}%
\pgfusepath{clip}%
\pgfsetbuttcap%
\pgfsetroundjoin%
\definecolor{currentfill}{rgb}{0.318832,0.426605,0.859857}%
\pgfsetfillcolor{currentfill}%
\pgfsetlinewidth{0.000000pt}%
\definecolor{currentstroke}{rgb}{0.000000,0.000000,0.000000}%
\pgfsetstrokecolor{currentstroke}%
\pgfsetdash{}{0pt}%
\pgfpathmoveto{\pgfqpoint{5.161153in}{3.292024in}}%
\pgfpathlineto{\pgfqpoint{5.169265in}{3.084139in}}%
\pgfpathlineto{\pgfqpoint{5.180546in}{3.195382in}}%
\pgfpathlineto{\pgfqpoint{5.212837in}{3.175443in}}%
\pgfpathlineto{\pgfqpoint{5.248592in}{3.483840in}}%
\pgfpathlineto{\pgfqpoint{5.235366in}{3.203690in}}%
\pgfpathlineto{\pgfqpoint{5.224766in}{3.167498in}}%
\pgfpathlineto{\pgfqpoint{5.193904in}{3.320165in}}%
\pgfpathlineto{\pgfqpoint{5.161153in}{3.292024in}}%
\pgfpathclose%
\pgfusepath{fill}%
\end{pgfscope}%
\begin{pgfscope}%
\pgfpathrectangle{\pgfqpoint{1.020000in}{0.880000in}}{\pgfqpoint{6.160000in}{6.160000in}}%
\pgfusepath{clip}%
\pgfsetbuttcap%
\pgfsetroundjoin%
\definecolor{currentfill}{rgb}{0.947345,0.794696,0.716991}%
\pgfsetfillcolor{currentfill}%
\pgfsetlinewidth{0.000000pt}%
\definecolor{currentstroke}{rgb}{0.000000,0.000000,0.000000}%
\pgfsetstrokecolor{currentstroke}%
\pgfsetdash{}{0pt}%
\pgfpathmoveto{\pgfqpoint{3.566038in}{4.553756in}}%
\pgfpathlineto{\pgfqpoint{3.574648in}{4.536796in}}%
\pgfpathlineto{\pgfqpoint{3.583295in}{4.516241in}}%
\pgfpathlineto{\pgfqpoint{3.615721in}{4.613740in}}%
\pgfpathlineto{\pgfqpoint{3.649294in}{4.502554in}}%
\pgfpathlineto{\pgfqpoint{3.640482in}{4.544942in}}%
\pgfpathlineto{\pgfqpoint{3.632912in}{4.361586in}}%
\pgfpathlineto{\pgfqpoint{3.599016in}{4.544208in}}%
\pgfpathlineto{\pgfqpoint{3.566038in}{4.553756in}}%
\pgfpathclose%
\pgfusepath{fill}%
\end{pgfscope}%
\begin{pgfscope}%
\pgfpathrectangle{\pgfqpoint{1.020000in}{0.880000in}}{\pgfqpoint{6.160000in}{6.160000in}}%
\pgfusepath{clip}%
\pgfsetbuttcap%
\pgfsetroundjoin%
\definecolor{currentfill}{rgb}{0.962708,0.753557,0.655601}%
\pgfsetfillcolor{currentfill}%
\pgfsetlinewidth{0.000000pt}%
\definecolor{currentstroke}{rgb}{0.000000,0.000000,0.000000}%
\pgfsetstrokecolor{currentstroke}%
\pgfsetdash{}{0pt}%
\pgfpathmoveto{\pgfqpoint{2.146685in}{4.619818in}}%
\pgfpathlineto{\pgfqpoint{2.155222in}{4.545327in}}%
\pgfpathlineto{\pgfqpoint{2.161679in}{4.575148in}}%
\pgfpathlineto{\pgfqpoint{2.194732in}{4.591400in}}%
\pgfpathlineto{\pgfqpoint{2.229414in}{4.522941in}}%
\pgfpathlineto{\pgfqpoint{2.218319in}{4.728050in}}%
\pgfpathlineto{\pgfqpoint{2.212637in}{4.654464in}}%
\pgfpathlineto{\pgfqpoint{2.178884in}{4.676593in}}%
\pgfpathlineto{\pgfqpoint{2.146685in}{4.619818in}}%
\pgfpathclose%
\pgfusepath{fill}%
\end{pgfscope}%
\begin{pgfscope}%
\pgfpathrectangle{\pgfqpoint{1.020000in}{0.880000in}}{\pgfqpoint{6.160000in}{6.160000in}}%
\pgfusepath{clip}%
\pgfsetbuttcap%
\pgfsetroundjoin%
\definecolor{currentfill}{rgb}{0.909460,0.839386,0.800331}%
\pgfsetfillcolor{currentfill}%
\pgfsetlinewidth{0.000000pt}%
\definecolor{currentstroke}{rgb}{0.000000,0.000000,0.000000}%
\pgfsetstrokecolor{currentstroke}%
\pgfsetdash{}{0pt}%
\pgfpathmoveto{\pgfqpoint{3.649294in}{4.502554in}}%
\pgfpathlineto{\pgfqpoint{3.658000in}{4.482482in}}%
\pgfpathlineto{\pgfqpoint{3.667371in}{4.335397in}}%
\pgfpathlineto{\pgfqpoint{3.700394in}{4.317713in}}%
\pgfpathlineto{\pgfqpoint{3.732892in}{4.411180in}}%
\pgfpathlineto{\pgfqpoint{3.725152in}{4.205380in}}%
\pgfpathlineto{\pgfqpoint{3.715732in}{4.374021in}}%
\pgfpathlineto{\pgfqpoint{3.682402in}{4.466719in}}%
\pgfpathlineto{\pgfqpoint{3.649294in}{4.502554in}}%
\pgfpathclose%
\pgfusepath{fill}%
\end{pgfscope}%
\begin{pgfscope}%
\pgfpathrectangle{\pgfqpoint{1.020000in}{0.880000in}}{\pgfqpoint{6.160000in}{6.160000in}}%
\pgfusepath{clip}%
\pgfsetbuttcap%
\pgfsetroundjoin%
\definecolor{currentfill}{rgb}{0.318832,0.426605,0.859857}%
\pgfsetfillcolor{currentfill}%
\pgfsetlinewidth{0.000000pt}%
\definecolor{currentstroke}{rgb}{0.000000,0.000000,0.000000}%
\pgfsetstrokecolor{currentstroke}%
\pgfsetdash{}{0pt}%
\pgfpathmoveto{\pgfqpoint{4.882252in}{3.476403in}}%
\pgfpathlineto{\pgfqpoint{4.890458in}{3.237809in}}%
\pgfpathlineto{\pgfqpoint{4.899949in}{3.181756in}}%
\pgfpathlineto{\pgfqpoint{4.933125in}{3.260813in}}%
\pgfpathlineto{\pgfqpoint{4.964109in}{3.056136in}}%
\pgfpathlineto{\pgfqpoint{4.955472in}{3.224242in}}%
\pgfpathlineto{\pgfqpoint{4.945287in}{3.191362in}}%
\pgfpathlineto{\pgfqpoint{4.913263in}{3.256969in}}%
\pgfpathlineto{\pgfqpoint{4.882252in}{3.476403in}}%
\pgfpathclose%
\pgfusepath{fill}%
\end{pgfscope}%
\begin{pgfscope}%
\pgfpathrectangle{\pgfqpoint{1.020000in}{0.880000in}}{\pgfqpoint{6.160000in}{6.160000in}}%
\pgfusepath{clip}%
\pgfsetbuttcap%
\pgfsetroundjoin%
\definecolor{currentfill}{rgb}{0.603162,0.731527,0.999565}%
\pgfsetfillcolor{currentfill}%
\pgfsetlinewidth{0.000000pt}%
\definecolor{currentstroke}{rgb}{0.000000,0.000000,0.000000}%
\pgfsetstrokecolor{currentstroke}%
\pgfsetdash{}{0pt}%
\pgfpathmoveto{\pgfqpoint{4.199733in}{3.779023in}}%
\pgfpathlineto{\pgfqpoint{4.208885in}{3.733551in}}%
\pgfpathlineto{\pgfqpoint{4.218007in}{3.572621in}}%
\pgfpathlineto{\pgfqpoint{4.251001in}{3.951286in}}%
\pgfpathlineto{\pgfqpoint{4.283520in}{3.627141in}}%
\pgfpathlineto{\pgfqpoint{4.274412in}{3.762393in}}%
\pgfpathlineto{\pgfqpoint{4.265237in}{3.822841in}}%
\pgfpathlineto{\pgfqpoint{4.232492in}{3.809295in}}%
\pgfpathlineto{\pgfqpoint{4.199733in}{3.779023in}}%
\pgfpathclose%
\pgfusepath{fill}%
\end{pgfscope}%
\begin{pgfscope}%
\pgfpathrectangle{\pgfqpoint{1.020000in}{0.880000in}}{\pgfqpoint{6.160000in}{6.160000in}}%
\pgfusepath{clip}%
\pgfsetbuttcap%
\pgfsetroundjoin%
\definecolor{currentfill}{rgb}{0.248091,0.326013,0.777669}%
\pgfsetfillcolor{currentfill}%
\pgfsetlinewidth{0.000000pt}%
\definecolor{currentstroke}{rgb}{0.000000,0.000000,0.000000}%
\pgfsetstrokecolor{currentstroke}%
\pgfsetdash{}{0pt}%
\pgfpathmoveto{\pgfqpoint{4.964109in}{3.056136in}}%
\pgfpathlineto{\pgfqpoint{4.975029in}{3.177148in}}%
\pgfpathlineto{\pgfqpoint{4.983652in}{3.006828in}}%
\pgfpathlineto{\pgfqpoint{5.017011in}{3.103718in}}%
\pgfpathlineto{\pgfqpoint{5.049802in}{3.130096in}}%
\pgfpathlineto{\pgfqpoint{5.040220in}{3.184472in}}%
\pgfpathlineto{\pgfqpoint{5.028211in}{2.951861in}}%
\pgfpathlineto{\pgfqpoint{4.997425in}{3.155641in}}%
\pgfpathlineto{\pgfqpoint{4.964109in}{3.056136in}}%
\pgfpathclose%
\pgfusepath{fill}%
\end{pgfscope}%
\begin{pgfscope}%
\pgfpathrectangle{\pgfqpoint{1.020000in}{0.880000in}}{\pgfqpoint{6.160000in}{6.160000in}}%
\pgfusepath{clip}%
\pgfsetbuttcap%
\pgfsetroundjoin%
\definecolor{currentfill}{rgb}{0.667253,0.779176,0.992959}%
\pgfsetfillcolor{currentfill}%
\pgfsetlinewidth{0.000000pt}%
\definecolor{currentstroke}{rgb}{0.000000,0.000000,0.000000}%
\pgfsetstrokecolor{currentstroke}%
\pgfsetdash{}{0pt}%
\pgfpathmoveto{\pgfqpoint{4.050520in}{3.802340in}}%
\pgfpathlineto{\pgfqpoint{4.059450in}{3.861053in}}%
\pgfpathlineto{\pgfqpoint{4.068546in}{3.799095in}}%
\pgfpathlineto{\pgfqpoint{4.101295in}{3.894367in}}%
\pgfpathlineto{\pgfqpoint{4.134165in}{3.829685in}}%
\pgfpathlineto{\pgfqpoint{4.125004in}{3.958727in}}%
\pgfpathlineto{\pgfqpoint{4.115895in}{4.003131in}}%
\pgfpathlineto{\pgfqpoint{4.083333in}{3.753250in}}%
\pgfpathlineto{\pgfqpoint{4.050520in}{3.802340in}}%
\pgfpathclose%
\pgfusepath{fill}%
\end{pgfscope}%
\begin{pgfscope}%
\pgfpathrectangle{\pgfqpoint{1.020000in}{0.880000in}}{\pgfqpoint{6.160000in}{6.160000in}}%
\pgfusepath{clip}%
\pgfsetbuttcap%
\pgfsetroundjoin%
\definecolor{currentfill}{rgb}{0.926883,0.505422,0.394866}%
\pgfsetfillcolor{currentfill}%
\pgfsetlinewidth{0.000000pt}%
\definecolor{currentstroke}{rgb}{0.000000,0.000000,0.000000}%
\pgfsetstrokecolor{currentstroke}%
\pgfsetdash{}{0pt}%
\pgfpathmoveto{\pgfqpoint{2.821381in}{4.882020in}}%
\pgfpathlineto{\pgfqpoint{2.827713in}{4.987048in}}%
\pgfpathlineto{\pgfqpoint{2.834552in}{5.056250in}}%
\pgfpathlineto{\pgfqpoint{2.867422in}{5.082690in}}%
\pgfpathlineto{\pgfqpoint{2.900326in}{5.106334in}}%
\pgfpathlineto{\pgfqpoint{2.894282in}{4.965552in}}%
\pgfpathlineto{\pgfqpoint{2.886134in}{4.992558in}}%
\pgfpathlineto{\pgfqpoint{2.851572in}{5.103492in}}%
\pgfpathlineto{\pgfqpoint{2.821381in}{4.882020in}}%
\pgfpathclose%
\pgfusepath{fill}%
\end{pgfscope}%
\begin{pgfscope}%
\pgfpathrectangle{\pgfqpoint{1.020000in}{0.880000in}}{\pgfqpoint{6.160000in}{6.160000in}}%
\pgfusepath{clip}%
\pgfsetbuttcap%
\pgfsetroundjoin%
\definecolor{currentfill}{rgb}{0.968105,0.668475,0.550486}%
\pgfsetfillcolor{currentfill}%
\pgfsetlinewidth{0.000000pt}%
\definecolor{currentstroke}{rgb}{0.000000,0.000000,0.000000}%
\pgfsetstrokecolor{currentstroke}%
\pgfsetdash{}{0pt}%
\pgfpathmoveto{\pgfqpoint{2.276093in}{4.820096in}}%
\pgfpathlineto{\pgfqpoint{2.286789in}{4.632834in}}%
\pgfpathlineto{\pgfqpoint{2.291377in}{4.772117in}}%
\pgfpathlineto{\pgfqpoint{2.326051in}{4.700512in}}%
\pgfpathlineto{\pgfqpoint{2.358938in}{4.724396in}}%
\pgfpathlineto{\pgfqpoint{2.351656in}{4.725689in}}%
\pgfpathlineto{\pgfqpoint{2.342374in}{4.837970in}}%
\pgfpathlineto{\pgfqpoint{2.307513in}{4.922483in}}%
\pgfpathlineto{\pgfqpoint{2.276093in}{4.820096in}}%
\pgfpathclose%
\pgfusepath{fill}%
\end{pgfscope}%
\begin{pgfscope}%
\pgfpathrectangle{\pgfqpoint{1.020000in}{0.880000in}}{\pgfqpoint{6.160000in}{6.160000in}}%
\pgfusepath{clip}%
\pgfsetbuttcap%
\pgfsetroundjoin%
\definecolor{currentfill}{rgb}{0.388852,0.516298,0.921373}%
\pgfsetfillcolor{currentfill}%
\pgfsetlinewidth{0.000000pt}%
\definecolor{currentstroke}{rgb}{0.000000,0.000000,0.000000}%
\pgfsetstrokecolor{currentstroke}%
\pgfsetdash{}{0pt}%
\pgfpathmoveto{\pgfqpoint{5.506953in}{3.368307in}}%
\pgfpathlineto{\pgfqpoint{5.518676in}{3.457781in}}%
\pgfpathlineto{\pgfqpoint{5.527484in}{3.327552in}}%
\pgfpathlineto{\pgfqpoint{5.561693in}{3.457993in}}%
\pgfpathlineto{\pgfqpoint{5.591768in}{3.290637in}}%
\pgfpathlineto{\pgfqpoint{5.581379in}{3.305984in}}%
\pgfpathlineto{\pgfqpoint{5.572265in}{3.411738in}}%
\pgfpathlineto{\pgfqpoint{5.538518in}{3.309480in}}%
\pgfpathlineto{\pgfqpoint{5.506953in}{3.368307in}}%
\pgfpathclose%
\pgfusepath{fill}%
\end{pgfscope}%
\begin{pgfscope}%
\pgfpathrectangle{\pgfqpoint{1.020000in}{0.880000in}}{\pgfqpoint{6.160000in}{6.160000in}}%
\pgfusepath{clip}%
\pgfsetbuttcap%
\pgfsetroundjoin%
\definecolor{currentfill}{rgb}{0.565182,0.699438,0.996635}%
\pgfsetfillcolor{currentfill}%
\pgfsetlinewidth{0.000000pt}%
\definecolor{currentstroke}{rgb}{0.000000,0.000000,0.000000}%
\pgfsetstrokecolor{currentstroke}%
\pgfsetdash{}{0pt}%
\pgfpathmoveto{\pgfqpoint{4.349185in}{3.776055in}}%
\pgfpathlineto{\pgfqpoint{4.358375in}{3.697557in}}%
\pgfpathlineto{\pgfqpoint{4.367652in}{3.663864in}}%
\pgfpathlineto{\pgfqpoint{4.400424in}{3.671604in}}%
\pgfpathlineto{\pgfqpoint{4.433414in}{3.772296in}}%
\pgfpathlineto{\pgfqpoint{4.423503in}{3.562973in}}%
\pgfpathlineto{\pgfqpoint{4.414221in}{3.601033in}}%
\pgfpathlineto{\pgfqpoint{4.381876in}{3.756467in}}%
\pgfpathlineto{\pgfqpoint{4.349185in}{3.776055in}}%
\pgfpathclose%
\pgfusepath{fill}%
\end{pgfscope}%
\begin{pgfscope}%
\pgfpathrectangle{\pgfqpoint{1.020000in}{0.880000in}}{\pgfqpoint{6.160000in}{6.160000in}}%
\pgfusepath{clip}%
\pgfsetbuttcap%
\pgfsetroundjoin%
\definecolor{currentfill}{rgb}{0.966017,0.646130,0.525890}%
\pgfsetfillcolor{currentfill}%
\pgfsetlinewidth{0.000000pt}%
\definecolor{currentstroke}{rgb}{0.000000,0.000000,0.000000}%
\pgfsetstrokecolor{currentstroke}%
\pgfsetdash{}{0pt}%
\pgfpathmoveto{\pgfqpoint{3.332238in}{4.883757in}}%
\pgfpathlineto{\pgfqpoint{3.340482in}{4.887677in}}%
\pgfpathlineto{\pgfqpoint{3.348969in}{4.864198in}}%
\pgfpathlineto{\pgfqpoint{3.382742in}{4.775680in}}%
\pgfpathlineto{\pgfqpoint{3.415392in}{4.825324in}}%
\pgfpathlineto{\pgfqpoint{3.407408in}{4.777183in}}%
\pgfpathlineto{\pgfqpoint{3.399463in}{4.727299in}}%
\pgfpathlineto{\pgfqpoint{3.366418in}{4.739410in}}%
\pgfpathlineto{\pgfqpoint{3.332238in}{4.883757in}}%
\pgfpathclose%
\pgfusepath{fill}%
\end{pgfscope}%
\begin{pgfscope}%
\pgfpathrectangle{\pgfqpoint{1.020000in}{0.880000in}}{\pgfqpoint{6.160000in}{6.160000in}}%
\pgfusepath{clip}%
\pgfsetbuttcap%
\pgfsetroundjoin%
\definecolor{currentfill}{rgb}{0.363461,0.484784,0.901019}%
\pgfsetfillcolor{currentfill}%
\pgfsetlinewidth{0.000000pt}%
\definecolor{currentstroke}{rgb}{0.000000,0.000000,0.000000}%
\pgfsetstrokecolor{currentstroke}%
\pgfsetdash{}{0pt}%
\pgfpathmoveto{\pgfqpoint{5.721876in}{3.340706in}}%
\pgfpathlineto{\pgfqpoint{5.732605in}{3.339068in}}%
\pgfpathlineto{\pgfqpoint{5.742159in}{3.260737in}}%
\pgfpathlineto{\pgfqpoint{5.776889in}{3.410756in}}%
\pgfpathlineto{\pgfqpoint{5.764699in}{3.323934in}}%
\pgfpathlineto{\pgfqpoint{5.752843in}{3.255722in}}%
\pgfpathlineto{\pgfqpoint{5.721876in}{3.340706in}}%
\pgfpathclose%
\pgfusepath{fill}%
\end{pgfscope}%
\begin{pgfscope}%
\pgfpathrectangle{\pgfqpoint{1.020000in}{0.880000in}}{\pgfqpoint{6.160000in}{6.160000in}}%
\pgfusepath{clip}%
\pgfsetbuttcap%
\pgfsetroundjoin%
\definecolor{currentfill}{rgb}{0.435815,0.570707,0.951717}%
\pgfsetfillcolor{currentfill}%
\pgfsetlinewidth{0.000000pt}%
\definecolor{currentstroke}{rgb}{0.000000,0.000000,0.000000}%
\pgfsetstrokecolor{currentstroke}%
\pgfsetdash{}{0pt}%
\pgfpathmoveto{\pgfqpoint{4.582707in}{3.588716in}}%
\pgfpathlineto{\pgfqpoint{4.591939in}{3.504566in}}%
\pgfpathlineto{\pgfqpoint{4.601292in}{3.448194in}}%
\pgfpathlineto{\pgfqpoint{4.633979in}{3.446136in}}%
\pgfpathlineto{\pgfqpoint{4.666057in}{3.322380in}}%
\pgfpathlineto{\pgfqpoint{4.656932in}{3.429691in}}%
\pgfpathlineto{\pgfqpoint{4.647331in}{3.434553in}}%
\pgfpathlineto{\pgfqpoint{4.614804in}{3.453872in}}%
\pgfpathlineto{\pgfqpoint{4.582707in}{3.588716in}}%
\pgfpathclose%
\pgfusepath{fill}%
\end{pgfscope}%
\begin{pgfscope}%
\pgfpathrectangle{\pgfqpoint{1.020000in}{0.880000in}}{\pgfqpoint{6.160000in}{6.160000in}}%
\pgfusepath{clip}%
\pgfsetbuttcap%
\pgfsetroundjoin%
\definecolor{currentfill}{rgb}{0.936780,0.532750,0.418093}%
\pgfsetfillcolor{currentfill}%
\pgfsetlinewidth{0.000000pt}%
\definecolor{currentstroke}{rgb}{0.000000,0.000000,0.000000}%
\pgfsetstrokecolor{currentstroke}%
\pgfsetdash{}{0pt}%
\pgfpathmoveto{\pgfqpoint{2.753794in}{4.977667in}}%
\pgfpathlineto{\pgfqpoint{2.760852in}{5.021955in}}%
\pgfpathlineto{\pgfqpoint{2.770002in}{4.917540in}}%
\pgfpathlineto{\pgfqpoint{2.801421in}{5.048882in}}%
\pgfpathlineto{\pgfqpoint{2.834552in}{5.056250in}}%
\pgfpathlineto{\pgfqpoint{2.827713in}{4.987048in}}%
\pgfpathlineto{\pgfqpoint{2.821381in}{4.882020in}}%
\pgfpathlineto{\pgfqpoint{2.787313in}{4.951628in}}%
\pgfpathlineto{\pgfqpoint{2.753794in}{4.977667in}}%
\pgfpathclose%
\pgfusepath{fill}%
\end{pgfscope}%
\begin{pgfscope}%
\pgfpathrectangle{\pgfqpoint{1.020000in}{0.880000in}}{\pgfqpoint{6.160000in}{6.160000in}}%
\pgfusepath{clip}%
\pgfsetbuttcap%
\pgfsetroundjoin%
\definecolor{currentfill}{rgb}{0.929357,0.512254,0.400673}%
\pgfsetfillcolor{currentfill}%
\pgfsetlinewidth{0.000000pt}%
\definecolor{currentstroke}{rgb}{0.000000,0.000000,0.000000}%
\pgfsetstrokecolor{currentstroke}%
\pgfsetdash{}{0pt}%
\pgfpathmoveto{\pgfqpoint{3.034801in}{4.944862in}}%
\pgfpathlineto{\pgfqpoint{3.041116in}{5.091383in}}%
\pgfpathlineto{\pgfqpoint{3.049480in}{5.057789in}}%
\pgfpathlineto{\pgfqpoint{3.083890in}{4.942640in}}%
\pgfpathlineto{\pgfqpoint{3.115514in}{5.083716in}}%
\pgfpathlineto{\pgfqpoint{3.107924in}{5.039979in}}%
\pgfpathlineto{\pgfqpoint{3.101800in}{4.861007in}}%
\pgfpathlineto{\pgfqpoint{3.067057in}{5.018227in}}%
\pgfpathlineto{\pgfqpoint{3.034801in}{4.944862in}}%
\pgfpathclose%
\pgfusepath{fill}%
\end{pgfscope}%
\begin{pgfscope}%
\pgfpathrectangle{\pgfqpoint{1.020000in}{0.880000in}}{\pgfqpoint{6.160000in}{6.160000in}}%
\pgfusepath{clip}%
\pgfsetbuttcap%
\pgfsetroundjoin%
\definecolor{currentfill}{rgb}{0.378598,0.503856,0.913692}%
\pgfsetfillcolor{currentfill}%
\pgfsetlinewidth{0.000000pt}%
\definecolor{currentstroke}{rgb}{0.000000,0.000000,0.000000}%
\pgfsetstrokecolor{currentstroke}%
\pgfsetdash{}{0pt}%
\pgfpathmoveto{\pgfqpoint{5.440938in}{3.270656in}}%
\pgfpathlineto{\pgfqpoint{5.451238in}{3.258251in}}%
\pgfpathlineto{\pgfqpoint{5.463987in}{3.435635in}}%
\pgfpathlineto{\pgfqpoint{5.495986in}{3.398610in}}%
\pgfpathlineto{\pgfqpoint{5.527484in}{3.327552in}}%
\pgfpathlineto{\pgfqpoint{5.518676in}{3.457781in}}%
\pgfpathlineto{\pgfqpoint{5.506953in}{3.368307in}}%
\pgfpathlineto{\pgfqpoint{5.473229in}{3.264503in}}%
\pgfpathlineto{\pgfqpoint{5.440938in}{3.270656in}}%
\pgfpathclose%
\pgfusepath{fill}%
\end{pgfscope}%
\begin{pgfscope}%
\pgfpathrectangle{\pgfqpoint{1.020000in}{0.880000in}}{\pgfqpoint{6.160000in}{6.160000in}}%
\pgfusepath{clip}%
\pgfsetbuttcap%
\pgfsetroundjoin%
\definecolor{currentfill}{rgb}{0.883687,0.856108,0.840258}%
\pgfsetfillcolor{currentfill}%
\pgfsetlinewidth{0.000000pt}%
\definecolor{currentstroke}{rgb}{0.000000,0.000000,0.000000}%
\pgfsetstrokecolor{currentstroke}%
\pgfsetdash{}{0pt}%
\pgfpathmoveto{\pgfqpoint{3.732892in}{4.411180in}}%
\pgfpathlineto{\pgfqpoint{3.741952in}{4.327327in}}%
\pgfpathlineto{\pgfqpoint{3.750619in}{4.336012in}}%
\pgfpathlineto{\pgfqpoint{3.784088in}{4.208681in}}%
\pgfpathlineto{\pgfqpoint{3.816401in}{4.371256in}}%
\pgfpathlineto{\pgfqpoint{3.808187in}{4.221616in}}%
\pgfpathlineto{\pgfqpoint{3.799562in}{4.191284in}}%
\pgfpathlineto{\pgfqpoint{3.765927in}{4.384778in}}%
\pgfpathlineto{\pgfqpoint{3.732892in}{4.411180in}}%
\pgfpathclose%
\pgfusepath{fill}%
\end{pgfscope}%
\begin{pgfscope}%
\pgfpathrectangle{\pgfqpoint{1.020000in}{0.880000in}}{\pgfqpoint{6.160000in}{6.160000in}}%
\pgfusepath{clip}%
\pgfsetbuttcap%
\pgfsetroundjoin%
\definecolor{currentfill}{rgb}{0.944055,0.553153,0.435548}%
\pgfsetfillcolor{currentfill}%
\pgfsetlinewidth{0.000000pt}%
\definecolor{currentstroke}{rgb}{0.000000,0.000000,0.000000}%
\pgfsetstrokecolor{currentstroke}%
\pgfsetdash{}{0pt}%
\pgfpathmoveto{\pgfqpoint{3.183202in}{4.937327in}}%
\pgfpathlineto{\pgfqpoint{3.190968in}{4.973649in}}%
\pgfpathlineto{\pgfqpoint{3.198544in}{5.031909in}}%
\pgfpathlineto{\pgfqpoint{3.231775in}{5.022292in}}%
\pgfpathlineto{\pgfqpoint{3.266410in}{4.854631in}}%
\pgfpathlineto{\pgfqpoint{3.257464in}{4.937006in}}%
\pgfpathlineto{\pgfqpoint{3.249751in}{4.885875in}}%
\pgfpathlineto{\pgfqpoint{3.216600in}{4.900556in}}%
\pgfpathlineto{\pgfqpoint{3.183202in}{4.937327in}}%
\pgfpathclose%
\pgfusepath{fill}%
\end{pgfscope}%
\begin{pgfscope}%
\pgfpathrectangle{\pgfqpoint{1.020000in}{0.880000in}}{\pgfqpoint{6.160000in}{6.160000in}}%
\pgfusepath{clip}%
\pgfsetbuttcap%
\pgfsetroundjoin%
\definecolor{currentfill}{rgb}{0.373552,0.497499,0.909467}%
\pgfsetfillcolor{currentfill}%
\pgfsetlinewidth{0.000000pt}%
\definecolor{currentstroke}{rgb}{0.000000,0.000000,0.000000}%
\pgfsetstrokecolor{currentstroke}%
\pgfsetdash{}{0pt}%
\pgfpathmoveto{\pgfqpoint{4.666057in}{3.322380in}}%
\pgfpathlineto{\pgfqpoint{4.676152in}{3.412013in}}%
\pgfpathlineto{\pgfqpoint{4.684919in}{3.229976in}}%
\pgfpathlineto{\pgfqpoint{4.718153in}{3.339193in}}%
\pgfpathlineto{\pgfqpoint{4.750368in}{3.259728in}}%
\pgfpathlineto{\pgfqpoint{4.741215in}{3.363144in}}%
\pgfpathlineto{\pgfqpoint{4.731683in}{3.397301in}}%
\pgfpathlineto{\pgfqpoint{4.698949in}{3.375745in}}%
\pgfpathlineto{\pgfqpoint{4.666057in}{3.322380in}}%
\pgfpathclose%
\pgfusepath{fill}%
\end{pgfscope}%
\begin{pgfscope}%
\pgfpathrectangle{\pgfqpoint{1.020000in}{0.880000in}}{\pgfqpoint{6.160000in}{6.160000in}}%
\pgfusepath{clip}%
\pgfsetbuttcap%
\pgfsetroundjoin%
\definecolor{currentfill}{rgb}{0.323718,0.433158,0.864722}%
\pgfsetfillcolor{currentfill}%
\pgfsetlinewidth{0.000000pt}%
\definecolor{currentstroke}{rgb}{0.000000,0.000000,0.000000}%
\pgfsetstrokecolor{currentstroke}%
\pgfsetdash{}{0pt}%
\pgfpathmoveto{\pgfqpoint{5.311848in}{3.321820in}}%
\pgfpathlineto{\pgfqpoint{5.321760in}{3.284366in}}%
\pgfpathlineto{\pgfqpoint{5.329816in}{3.084438in}}%
\pgfpathlineto{\pgfqpoint{5.362958in}{3.143193in}}%
\pgfpathlineto{\pgfqpoint{5.397259in}{3.295691in}}%
\pgfpathlineto{\pgfqpoint{5.386439in}{3.262471in}}%
\pgfpathlineto{\pgfqpoint{5.376504in}{3.300964in}}%
\pgfpathlineto{\pgfqpoint{5.343260in}{3.231441in}}%
\pgfpathlineto{\pgfqpoint{5.311848in}{3.321820in}}%
\pgfpathclose%
\pgfusepath{fill}%
\end{pgfscope}%
\begin{pgfscope}%
\pgfpathrectangle{\pgfqpoint{1.020000in}{0.880000in}}{\pgfqpoint{6.160000in}{6.160000in}}%
\pgfusepath{clip}%
\pgfsetbuttcap%
\pgfsetroundjoin%
\definecolor{currentfill}{rgb}{0.358415,0.478426,0.896795}%
\pgfsetfillcolor{currentfill}%
\pgfsetlinewidth{0.000000pt}%
\definecolor{currentstroke}{rgb}{0.000000,0.000000,0.000000}%
\pgfsetstrokecolor{currentstroke}%
\pgfsetdash{}{0pt}%
\pgfpathmoveto{\pgfqpoint{5.376504in}{3.300964in}}%
\pgfpathlineto{\pgfqpoint{5.386439in}{3.262471in}}%
\pgfpathlineto{\pgfqpoint{5.397259in}{3.295691in}}%
\pgfpathlineto{\pgfqpoint{5.429928in}{3.311040in}}%
\pgfpathlineto{\pgfqpoint{5.463987in}{3.435635in}}%
\pgfpathlineto{\pgfqpoint{5.451238in}{3.258251in}}%
\pgfpathlineto{\pgfqpoint{5.440938in}{3.270656in}}%
\pgfpathlineto{\pgfqpoint{5.408832in}{3.293658in}}%
\pgfpathlineto{\pgfqpoint{5.376504in}{3.300964in}}%
\pgfpathclose%
\pgfusepath{fill}%
\end{pgfscope}%
\begin{pgfscope}%
\pgfpathrectangle{\pgfqpoint{1.020000in}{0.880000in}}{\pgfqpoint{6.160000in}{6.160000in}}%
\pgfusepath{clip}%
\pgfsetbuttcap%
\pgfsetroundjoin%
\definecolor{currentfill}{rgb}{0.516260,0.654498,0.986407}%
\pgfsetfillcolor{currentfill}%
\pgfsetlinewidth{0.000000pt}%
\definecolor{currentstroke}{rgb}{0.000000,0.000000,0.000000}%
\pgfsetstrokecolor{currentstroke}%
\pgfsetdash{}{0pt}%
\pgfpathmoveto{\pgfqpoint{4.433414in}{3.772296in}}%
\pgfpathlineto{\pgfqpoint{4.442476in}{3.630654in}}%
\pgfpathlineto{\pgfqpoint{4.451474in}{3.467768in}}%
\pgfpathlineto{\pgfqpoint{4.484289in}{3.500712in}}%
\pgfpathlineto{\pgfqpoint{4.517133in}{3.541502in}}%
\pgfpathlineto{\pgfqpoint{4.507727in}{3.567209in}}%
\pgfpathlineto{\pgfqpoint{4.498387in}{3.608944in}}%
\pgfpathlineto{\pgfqpoint{4.465789in}{3.638225in}}%
\pgfpathlineto{\pgfqpoint{4.433414in}{3.772296in}}%
\pgfpathclose%
\pgfusepath{fill}%
\end{pgfscope}%
\begin{pgfscope}%
\pgfpathrectangle{\pgfqpoint{1.020000in}{0.880000in}}{\pgfqpoint{6.160000in}{6.160000in}}%
\pgfusepath{clip}%
\pgfsetbuttcap%
\pgfsetroundjoin%
\definecolor{currentfill}{rgb}{0.368507,0.491141,0.905243}%
\pgfsetfillcolor{currentfill}%
\pgfsetlinewidth{0.000000pt}%
\definecolor{currentstroke}{rgb}{0.000000,0.000000,0.000000}%
\pgfsetstrokecolor{currentstroke}%
\pgfsetdash{}{0pt}%
\pgfpathmoveto{\pgfqpoint{5.657632in}{3.369127in}}%
\pgfpathlineto{\pgfqpoint{5.668695in}{3.393817in}}%
\pgfpathlineto{\pgfqpoint{5.678036in}{3.301583in}}%
\pgfpathlineto{\pgfqpoint{5.709853in}{3.264148in}}%
\pgfpathlineto{\pgfqpoint{5.742159in}{3.260737in}}%
\pgfpathlineto{\pgfqpoint{5.732605in}{3.339068in}}%
\pgfpathlineto{\pgfqpoint{5.721876in}{3.340706in}}%
\pgfpathlineto{\pgfqpoint{5.690041in}{3.372979in}}%
\pgfpathlineto{\pgfqpoint{5.657632in}{3.369127in}}%
\pgfpathclose%
\pgfusepath{fill}%
\end{pgfscope}%
\begin{pgfscope}%
\pgfpathrectangle{\pgfqpoint{1.020000in}{0.880000in}}{\pgfqpoint{6.160000in}{6.160000in}}%
\pgfusepath{clip}%
\pgfsetbuttcap%
\pgfsetroundjoin%
\definecolor{currentfill}{rgb}{0.570616,0.704109,0.997195}%
\pgfsetfillcolor{currentfill}%
\pgfsetlinewidth{0.000000pt}%
\definecolor{currentstroke}{rgb}{0.000000,0.000000,0.000000}%
\pgfsetstrokecolor{currentstroke}%
\pgfsetdash{}{0pt}%
\pgfpathmoveto{\pgfqpoint{4.283520in}{3.627141in}}%
\pgfpathlineto{\pgfqpoint{4.292853in}{3.701976in}}%
\pgfpathlineto{\pgfqpoint{4.302151in}{3.720627in}}%
\pgfpathlineto{\pgfqpoint{4.334853in}{3.647487in}}%
\pgfpathlineto{\pgfqpoint{4.367652in}{3.663864in}}%
\pgfpathlineto{\pgfqpoint{4.358375in}{3.697557in}}%
\pgfpathlineto{\pgfqpoint{4.349185in}{3.776055in}}%
\pgfpathlineto{\pgfqpoint{4.316324in}{3.691324in}}%
\pgfpathlineto{\pgfqpoint{4.283520in}{3.627141in}}%
\pgfpathclose%
\pgfusepath{fill}%
\end{pgfscope}%
\begin{pgfscope}%
\pgfpathrectangle{\pgfqpoint{1.020000in}{0.880000in}}{\pgfqpoint{6.160000in}{6.160000in}}%
\pgfusepath{clip}%
\pgfsetbuttcap%
\pgfsetroundjoin%
\definecolor{currentfill}{rgb}{0.266381,0.353304,0.801637}%
\pgfsetfillcolor{currentfill}%
\pgfsetlinewidth{0.000000pt}%
\definecolor{currentstroke}{rgb}{0.000000,0.000000,0.000000}%
\pgfsetstrokecolor{currentstroke}%
\pgfsetdash{}{0pt}%
\pgfpathmoveto{\pgfqpoint{4.899949in}{3.181756in}}%
\pgfpathlineto{\pgfqpoint{4.909191in}{3.090236in}}%
\pgfpathlineto{\pgfqpoint{4.919679in}{3.167071in}}%
\pgfpathlineto{\pgfqpoint{4.951169in}{3.017891in}}%
\pgfpathlineto{\pgfqpoint{4.983652in}{3.006828in}}%
\pgfpathlineto{\pgfqpoint{4.975029in}{3.177148in}}%
\pgfpathlineto{\pgfqpoint{4.964109in}{3.056136in}}%
\pgfpathlineto{\pgfqpoint{4.933125in}{3.260813in}}%
\pgfpathlineto{\pgfqpoint{4.899949in}{3.181756in}}%
\pgfpathclose%
\pgfusepath{fill}%
\end{pgfscope}%
\begin{pgfscope}%
\pgfpathrectangle{\pgfqpoint{1.020000in}{0.880000in}}{\pgfqpoint{6.160000in}{6.160000in}}%
\pgfusepath{clip}%
\pgfsetbuttcap%
\pgfsetroundjoin%
\definecolor{currentfill}{rgb}{0.683056,0.790043,0.989768}%
\pgfsetfillcolor{currentfill}%
\pgfsetlinewidth{0.000000pt}%
\definecolor{currentstroke}{rgb}{0.000000,0.000000,0.000000}%
\pgfsetstrokecolor{currentstroke}%
\pgfsetdash{}{0pt}%
\pgfpathmoveto{\pgfqpoint{3.984626in}{3.958536in}}%
\pgfpathlineto{\pgfqpoint{3.993617in}{3.938463in}}%
\pgfpathlineto{\pgfqpoint{4.002653in}{3.901530in}}%
\pgfpathlineto{\pgfqpoint{4.035541in}{3.908244in}}%
\pgfpathlineto{\pgfqpoint{4.068546in}{3.799095in}}%
\pgfpathlineto{\pgfqpoint{4.059450in}{3.861053in}}%
\pgfpathlineto{\pgfqpoint{4.050520in}{3.802340in}}%
\pgfpathlineto{\pgfqpoint{4.017555in}{3.912777in}}%
\pgfpathlineto{\pgfqpoint{3.984626in}{3.958536in}}%
\pgfpathclose%
\pgfusepath{fill}%
\end{pgfscope}%
\begin{pgfscope}%
\pgfpathrectangle{\pgfqpoint{1.020000in}{0.880000in}}{\pgfqpoint{6.160000in}{6.160000in}}%
\pgfusepath{clip}%
\pgfsetbuttcap%
\pgfsetroundjoin%
\definecolor{currentfill}{rgb}{0.343278,0.459354,0.884122}%
\pgfsetfillcolor{currentfill}%
\pgfsetlinewidth{0.000000pt}%
\definecolor{currentstroke}{rgb}{0.000000,0.000000,0.000000}%
\pgfsetstrokecolor{currentstroke}%
\pgfsetdash{}{0pt}%
\pgfpathmoveto{\pgfqpoint{4.816103in}{3.340633in}}%
\pgfpathlineto{\pgfqpoint{4.825986in}{3.350180in}}%
\pgfpathlineto{\pgfqpoint{4.834194in}{3.099960in}}%
\pgfpathlineto{\pgfqpoint{4.867672in}{3.229003in}}%
\pgfpathlineto{\pgfqpoint{4.899949in}{3.181756in}}%
\pgfpathlineto{\pgfqpoint{4.890458in}{3.237809in}}%
\pgfpathlineto{\pgfqpoint{4.882252in}{3.476403in}}%
\pgfpathlineto{\pgfqpoint{4.848292in}{3.277435in}}%
\pgfpathlineto{\pgfqpoint{4.816103in}{3.340633in}}%
\pgfpathclose%
\pgfusepath{fill}%
\end{pgfscope}%
\begin{pgfscope}%
\pgfpathrectangle{\pgfqpoint{1.020000in}{0.880000in}}{\pgfqpoint{6.160000in}{6.160000in}}%
\pgfusepath{clip}%
\pgfsetbuttcap%
\pgfsetroundjoin%
\definecolor{currentfill}{rgb}{0.763363,0.835092,0.955658}%
\pgfsetfillcolor{currentfill}%
\pgfsetlinewidth{0.000000pt}%
\definecolor{currentstroke}{rgb}{0.000000,0.000000,0.000000}%
\pgfsetstrokecolor{currentstroke}%
\pgfsetdash{}{0pt}%
\pgfpathmoveto{\pgfqpoint{3.900653in}{4.131527in}}%
\pgfpathlineto{\pgfqpoint{3.909516in}{4.133375in}}%
\pgfpathlineto{\pgfqpoint{3.918464in}{4.110225in}}%
\pgfpathlineto{\pgfqpoint{3.951482in}{4.076448in}}%
\pgfpathlineto{\pgfqpoint{3.984626in}{3.958536in}}%
\pgfpathlineto{\pgfqpoint{3.975932in}{3.842145in}}%
\pgfpathlineto{\pgfqpoint{3.966734in}{3.978153in}}%
\pgfpathlineto{\pgfqpoint{3.933669in}{4.078870in}}%
\pgfpathlineto{\pgfqpoint{3.900653in}{4.131527in}}%
\pgfpathclose%
\pgfusepath{fill}%
\end{pgfscope}%
\begin{pgfscope}%
\pgfpathrectangle{\pgfqpoint{1.020000in}{0.880000in}}{\pgfqpoint{6.160000in}{6.160000in}}%
\pgfusepath{clip}%
\pgfsetbuttcap%
\pgfsetroundjoin%
\definecolor{currentfill}{rgb}{0.635474,0.756714,0.998297}%
\pgfsetfillcolor{currentfill}%
\pgfsetlinewidth{0.000000pt}%
\definecolor{currentstroke}{rgb}{0.000000,0.000000,0.000000}%
\pgfsetstrokecolor{currentstroke}%
\pgfsetdash{}{0pt}%
\pgfpathmoveto{\pgfqpoint{4.134165in}{3.829685in}}%
\pgfpathlineto{\pgfqpoint{4.143258in}{3.858554in}}%
\pgfpathlineto{\pgfqpoint{4.152376in}{3.886624in}}%
\pgfpathlineto{\pgfqpoint{4.185255in}{3.851885in}}%
\pgfpathlineto{\pgfqpoint{4.218007in}{3.572621in}}%
\pgfpathlineto{\pgfqpoint{4.208885in}{3.733551in}}%
\pgfpathlineto{\pgfqpoint{4.199733in}{3.779023in}}%
\pgfpathlineto{\pgfqpoint{4.166956in}{3.869696in}}%
\pgfpathlineto{\pgfqpoint{4.134165in}{3.829685in}}%
\pgfpathclose%
\pgfusepath{fill}%
\end{pgfscope}%
\begin{pgfscope}%
\pgfpathrectangle{\pgfqpoint{1.020000in}{0.880000in}}{\pgfqpoint{6.160000in}{6.160000in}}%
\pgfusepath{clip}%
\pgfsetbuttcap%
\pgfsetroundjoin%
\definecolor{currentfill}{rgb}{0.945854,0.559565,0.441513}%
\pgfsetfillcolor{currentfill}%
\pgfsetlinewidth{0.000000pt}%
\definecolor{currentstroke}{rgb}{0.000000,0.000000,0.000000}%
\pgfsetstrokecolor{currentstroke}%
\pgfsetdash{}{0pt}%
\pgfpathmoveto{\pgfqpoint{2.621085in}{4.969548in}}%
\pgfpathlineto{\pgfqpoint{2.632460in}{4.719657in}}%
\pgfpathlineto{\pgfqpoint{2.637761in}{4.867014in}}%
\pgfpathlineto{\pgfqpoint{2.670576in}{4.897177in}}%
\pgfpathlineto{\pgfqpoint{2.699988in}{5.160812in}}%
\pgfpathlineto{\pgfqpoint{2.694812in}{4.992471in}}%
\pgfpathlineto{\pgfqpoint{2.687376in}{4.980307in}}%
\pgfpathlineto{\pgfqpoint{2.655352in}{4.900962in}}%
\pgfpathlineto{\pgfqpoint{2.621085in}{4.969548in}}%
\pgfpathclose%
\pgfusepath{fill}%
\end{pgfscope}%
\begin{pgfscope}%
\pgfpathrectangle{\pgfqpoint{1.020000in}{0.880000in}}{\pgfqpoint{6.160000in}{6.160000in}}%
\pgfusepath{clip}%
\pgfsetbuttcap%
\pgfsetroundjoin%
\definecolor{currentfill}{rgb}{0.968894,0.679480,0.562812}%
\pgfsetfillcolor{currentfill}%
\pgfsetlinewidth{0.000000pt}%
\definecolor{currentstroke}{rgb}{0.000000,0.000000,0.000000}%
\pgfsetstrokecolor{currentstroke}%
\pgfsetdash{}{0pt}%
\pgfpathmoveto{\pgfqpoint{3.415392in}{4.825324in}}%
\pgfpathlineto{\pgfqpoint{3.425744in}{4.560688in}}%
\pgfpathlineto{\pgfqpoint{3.430905in}{4.993206in}}%
\pgfpathlineto{\pgfqpoint{3.466521in}{4.643714in}}%
\pgfpathlineto{\pgfqpoint{3.498783in}{4.754152in}}%
\pgfpathlineto{\pgfqpoint{3.490721in}{4.698296in}}%
\pgfpathlineto{\pgfqpoint{3.482392in}{4.684919in}}%
\pgfpathlineto{\pgfqpoint{3.448582in}{4.802406in}}%
\pgfpathlineto{\pgfqpoint{3.415392in}{4.825324in}}%
\pgfpathclose%
\pgfusepath{fill}%
\end{pgfscope}%
\begin{pgfscope}%
\pgfpathrectangle{\pgfqpoint{1.020000in}{0.880000in}}{\pgfqpoint{6.160000in}{6.160000in}}%
\pgfusepath{clip}%
\pgfsetbuttcap%
\pgfsetroundjoin%
\definecolor{currentfill}{rgb}{0.843358,0.861820,0.890017}%
\pgfsetfillcolor{currentfill}%
\pgfsetlinewidth{0.000000pt}%
\definecolor{currentstroke}{rgb}{0.000000,0.000000,0.000000}%
\pgfsetstrokecolor{currentstroke}%
\pgfsetdash{}{0pt}%
\pgfpathmoveto{\pgfqpoint{3.816401in}{4.371256in}}%
\pgfpathlineto{\pgfqpoint{3.826023in}{4.136310in}}%
\pgfpathlineto{\pgfqpoint{3.834735in}{4.158045in}}%
\pgfpathlineto{\pgfqpoint{3.867481in}{4.217291in}}%
\pgfpathlineto{\pgfqpoint{3.900653in}{4.131527in}}%
\pgfpathlineto{\pgfqpoint{3.891600in}{4.201995in}}%
\pgfpathlineto{\pgfqpoint{3.882568in}{4.267118in}}%
\pgfpathlineto{\pgfqpoint{3.850070in}{4.150054in}}%
\pgfpathlineto{\pgfqpoint{3.816401in}{4.371256in}}%
\pgfpathclose%
\pgfusepath{fill}%
\end{pgfscope}%
\begin{pgfscope}%
\pgfpathrectangle{\pgfqpoint{1.020000in}{0.880000in}}{\pgfqpoint{6.160000in}{6.160000in}}%
\pgfusepath{clip}%
\pgfsetbuttcap%
\pgfsetroundjoin%
\definecolor{currentfill}{rgb}{0.954853,0.591622,0.471337}%
\pgfsetfillcolor{currentfill}%
\pgfsetlinewidth{0.000000pt}%
\definecolor{currentstroke}{rgb}{0.000000,0.000000,0.000000}%
\pgfsetstrokecolor{currentstroke}%
\pgfsetdash{}{0pt}%
\pgfpathmoveto{\pgfqpoint{2.557707in}{4.773304in}}%
\pgfpathlineto{\pgfqpoint{2.563657in}{4.868577in}}%
\pgfpathlineto{\pgfqpoint{2.569228in}{4.989473in}}%
\pgfpathlineto{\pgfqpoint{2.602377in}{5.001940in}}%
\pgfpathlineto{\pgfqpoint{2.637761in}{4.867014in}}%
\pgfpathlineto{\pgfqpoint{2.632460in}{4.719657in}}%
\pgfpathlineto{\pgfqpoint{2.621085in}{4.969548in}}%
\pgfpathlineto{\pgfqpoint{2.589068in}{4.890868in}}%
\pgfpathlineto{\pgfqpoint{2.557707in}{4.773304in}}%
\pgfpathclose%
\pgfusepath{fill}%
\end{pgfscope}%
\begin{pgfscope}%
\pgfpathrectangle{\pgfqpoint{1.020000in}{0.880000in}}{\pgfqpoint{6.160000in}{6.160000in}}%
\pgfusepath{clip}%
\pgfsetbuttcap%
\pgfsetroundjoin%
\definecolor{currentfill}{rgb}{0.358415,0.478426,0.896795}%
\pgfsetfillcolor{currentfill}%
\pgfsetlinewidth{0.000000pt}%
\definecolor{currentstroke}{rgb}{0.000000,0.000000,0.000000}%
\pgfsetstrokecolor{currentstroke}%
\pgfsetdash{}{0pt}%
\pgfpathmoveto{\pgfqpoint{5.591768in}{3.290637in}}%
\pgfpathlineto{\pgfqpoint{5.602444in}{3.294090in}}%
\pgfpathlineto{\pgfqpoint{5.610364in}{3.103516in}}%
\pgfpathlineto{\pgfqpoint{5.645988in}{3.326234in}}%
\pgfpathlineto{\pgfqpoint{5.678036in}{3.301583in}}%
\pgfpathlineto{\pgfqpoint{5.668695in}{3.393817in}}%
\pgfpathlineto{\pgfqpoint{5.657632in}{3.369127in}}%
\pgfpathlineto{\pgfqpoint{5.625250in}{3.368184in}}%
\pgfpathlineto{\pgfqpoint{5.591768in}{3.290637in}}%
\pgfpathclose%
\pgfusepath{fill}%
\end{pgfscope}%
\begin{pgfscope}%
\pgfpathrectangle{\pgfqpoint{1.020000in}{0.880000in}}{\pgfqpoint{6.160000in}{6.160000in}}%
\pgfusepath{clip}%
\pgfsetbuttcap%
\pgfsetroundjoin%
\definecolor{currentfill}{rgb}{0.926883,0.505422,0.394866}%
\pgfsetfillcolor{currentfill}%
\pgfsetlinewidth{0.000000pt}%
\definecolor{currentstroke}{rgb}{0.000000,0.000000,0.000000}%
\pgfsetstrokecolor{currentstroke}%
\pgfsetdash{}{0pt}%
\pgfpathmoveto{\pgfqpoint{2.687376in}{4.980307in}}%
\pgfpathlineto{\pgfqpoint{2.694812in}{4.992471in}}%
\pgfpathlineto{\pgfqpoint{2.699988in}{5.160812in}}%
\pgfpathlineto{\pgfqpoint{2.735816in}{4.985062in}}%
\pgfpathlineto{\pgfqpoint{2.770002in}{4.917540in}}%
\pgfpathlineto{\pgfqpoint{2.760852in}{5.021955in}}%
\pgfpathlineto{\pgfqpoint{2.753794in}{4.977667in}}%
\pgfpathlineto{\pgfqpoint{2.720261in}{5.002218in}}%
\pgfpathlineto{\pgfqpoint{2.687376in}{4.980307in}}%
\pgfpathclose%
\pgfusepath{fill}%
\end{pgfscope}%
\begin{pgfscope}%
\pgfpathrectangle{\pgfqpoint{1.020000in}{0.880000in}}{\pgfqpoint{6.160000in}{6.160000in}}%
\pgfusepath{clip}%
\pgfsetbuttcap%
\pgfsetroundjoin%
\definecolor{currentfill}{rgb}{0.261805,0.346484,0.795658}%
\pgfsetfillcolor{currentfill}%
\pgfsetlinewidth{0.000000pt}%
\definecolor{currentstroke}{rgb}{0.000000,0.000000,0.000000}%
\pgfsetstrokecolor{currentstroke}%
\pgfsetdash{}{0pt}%
\pgfpathmoveto{\pgfqpoint{5.114904in}{3.132056in}}%
\pgfpathlineto{\pgfqpoint{5.125573in}{3.188741in}}%
\pgfpathlineto{\pgfqpoint{5.134572in}{3.067094in}}%
\pgfpathlineto{\pgfqpoint{5.166711in}{3.024966in}}%
\pgfpathlineto{\pgfqpoint{5.199300in}{3.031565in}}%
\pgfpathlineto{\pgfqpoint{5.189492in}{3.070966in}}%
\pgfpathlineto{\pgfqpoint{5.180546in}{3.195382in}}%
\pgfpathlineto{\pgfqpoint{5.147854in}{3.177152in}}%
\pgfpathlineto{\pgfqpoint{5.114904in}{3.132056in}}%
\pgfpathclose%
\pgfusepath{fill}%
\end{pgfscope}%
\begin{pgfscope}%
\pgfpathrectangle{\pgfqpoint{1.020000in}{0.880000in}}{\pgfqpoint{6.160000in}{6.160000in}}%
\pgfusepath{clip}%
\pgfsetbuttcap%
\pgfsetroundjoin%
\definecolor{currentfill}{rgb}{0.964835,0.744614,0.643239}%
\pgfsetfillcolor{currentfill}%
\pgfsetlinewidth{0.000000pt}%
\definecolor{currentstroke}{rgb}{0.000000,0.000000,0.000000}%
\pgfsetstrokecolor{currentstroke}%
\pgfsetdash{}{0pt}%
\pgfpathmoveto{\pgfqpoint{3.498783in}{4.754152in}}%
\pgfpathlineto{\pgfqpoint{3.507939in}{4.650141in}}%
\pgfpathlineto{\pgfqpoint{3.517045in}{4.553583in}}%
\pgfpathlineto{\pgfqpoint{3.549428in}{4.655922in}}%
\pgfpathlineto{\pgfqpoint{3.583295in}{4.516241in}}%
\pgfpathlineto{\pgfqpoint{3.574648in}{4.536796in}}%
\pgfpathlineto{\pgfqpoint{3.566038in}{4.553756in}}%
\pgfpathlineto{\pgfqpoint{3.532267in}{4.682910in}}%
\pgfpathlineto{\pgfqpoint{3.498783in}{4.754152in}}%
\pgfpathclose%
\pgfusepath{fill}%
\end{pgfscope}%
\begin{pgfscope}%
\pgfpathrectangle{\pgfqpoint{1.020000in}{0.880000in}}{\pgfqpoint{6.160000in}{6.160000in}}%
\pgfusepath{clip}%
\pgfsetbuttcap%
\pgfsetroundjoin%
\definecolor{currentfill}{rgb}{0.338377,0.452819,0.879317}%
\pgfsetfillcolor{currentfill}%
\pgfsetlinewidth{0.000000pt}%
\definecolor{currentstroke}{rgb}{0.000000,0.000000,0.000000}%
\pgfsetstrokecolor{currentstroke}%
\pgfsetdash{}{0pt}%
\pgfpathmoveto{\pgfqpoint{4.750368in}{3.259728in}}%
\pgfpathlineto{\pgfqpoint{4.760216in}{3.276581in}}%
\pgfpathlineto{\pgfqpoint{4.769550in}{3.202071in}}%
\pgfpathlineto{\pgfqpoint{4.802482in}{3.245014in}}%
\pgfpathlineto{\pgfqpoint{4.834194in}{3.099960in}}%
\pgfpathlineto{\pgfqpoint{4.825986in}{3.350180in}}%
\pgfpathlineto{\pgfqpoint{4.816103in}{3.340633in}}%
\pgfpathlineto{\pgfqpoint{4.783139in}{3.284732in}}%
\pgfpathlineto{\pgfqpoint{4.750368in}{3.259728in}}%
\pgfpathclose%
\pgfusepath{fill}%
\end{pgfscope}%
\begin{pgfscope}%
\pgfpathrectangle{\pgfqpoint{1.020000in}{0.880000in}}{\pgfqpoint{6.160000in}{6.160000in}}%
\pgfusepath{clip}%
\pgfsetbuttcap%
\pgfsetroundjoin%
\definecolor{currentfill}{rgb}{0.473070,0.611077,0.970634}%
\pgfsetfillcolor{currentfill}%
\pgfsetlinewidth{0.000000pt}%
\definecolor{currentstroke}{rgb}{0.000000,0.000000,0.000000}%
\pgfsetstrokecolor{currentstroke}%
\pgfsetdash{}{0pt}%
\pgfpathmoveto{\pgfqpoint{4.517133in}{3.541502in}}%
\pgfpathlineto{\pgfqpoint{4.526552in}{3.515214in}}%
\pgfpathlineto{\pgfqpoint{4.535600in}{3.379542in}}%
\pgfpathlineto{\pgfqpoint{4.569322in}{3.641681in}}%
\pgfpathlineto{\pgfqpoint{4.601292in}{3.448194in}}%
\pgfpathlineto{\pgfqpoint{4.591939in}{3.504566in}}%
\pgfpathlineto{\pgfqpoint{4.582707in}{3.588716in}}%
\pgfpathlineto{\pgfqpoint{4.549589in}{3.474305in}}%
\pgfpathlineto{\pgfqpoint{4.517133in}{3.541502in}}%
\pgfpathclose%
\pgfusepath{fill}%
\end{pgfscope}%
\begin{pgfscope}%
\pgfpathrectangle{\pgfqpoint{1.020000in}{0.880000in}}{\pgfqpoint{6.160000in}{6.160000in}}%
\pgfusepath{clip}%
\pgfsetbuttcap%
\pgfsetroundjoin%
\definecolor{currentfill}{rgb}{0.954853,0.591622,0.471337}%
\pgfsetfillcolor{currentfill}%
\pgfsetlinewidth{0.000000pt}%
\definecolor{currentstroke}{rgb}{0.000000,0.000000,0.000000}%
\pgfsetstrokecolor{currentstroke}%
\pgfsetdash{}{0pt}%
\pgfpathmoveto{\pgfqpoint{3.266410in}{4.854631in}}%
\pgfpathlineto{\pgfqpoint{3.275170in}{4.793355in}}%
\pgfpathlineto{\pgfqpoint{3.282976in}{4.840303in}}%
\pgfpathlineto{\pgfqpoint{3.315260in}{4.936252in}}%
\pgfpathlineto{\pgfqpoint{3.348969in}{4.864198in}}%
\pgfpathlineto{\pgfqpoint{3.340482in}{4.887677in}}%
\pgfpathlineto{\pgfqpoint{3.332238in}{4.883757in}}%
\pgfpathlineto{\pgfqpoint{3.298457in}{4.969162in}}%
\pgfpathlineto{\pgfqpoint{3.266410in}{4.854631in}}%
\pgfpathclose%
\pgfusepath{fill}%
\end{pgfscope}%
\begin{pgfscope}%
\pgfpathrectangle{\pgfqpoint{1.020000in}{0.880000in}}{\pgfqpoint{6.160000in}{6.160000in}}%
\pgfusepath{clip}%
\pgfsetbuttcap%
\pgfsetroundjoin%
\definecolor{currentfill}{rgb}{0.958279,0.604335,0.483297}%
\pgfsetfillcolor{currentfill}%
\pgfsetlinewidth{0.000000pt}%
\definecolor{currentstroke}{rgb}{0.000000,0.000000,0.000000}%
\pgfsetstrokecolor{currentstroke}%
\pgfsetdash{}{0pt}%
\pgfpathmoveto{\pgfqpoint{2.490200in}{4.835217in}}%
\pgfpathlineto{\pgfqpoint{2.499053in}{4.748445in}}%
\pgfpathlineto{\pgfqpoint{2.505088in}{4.832116in}}%
\pgfpathlineto{\pgfqpoint{2.535728in}{4.997938in}}%
\pgfpathlineto{\pgfqpoint{2.569228in}{4.989473in}}%
\pgfpathlineto{\pgfqpoint{2.563657in}{4.868577in}}%
\pgfpathlineto{\pgfqpoint{2.557707in}{4.773304in}}%
\pgfpathlineto{\pgfqpoint{2.522847in}{4.873091in}}%
\pgfpathlineto{\pgfqpoint{2.490200in}{4.835217in}}%
\pgfpathclose%
\pgfusepath{fill}%
\end{pgfscope}%
\begin{pgfscope}%
\pgfpathrectangle{\pgfqpoint{1.020000in}{0.880000in}}{\pgfqpoint{6.160000in}{6.160000in}}%
\pgfusepath{clip}%
\pgfsetbuttcap%
\pgfsetroundjoin%
\definecolor{currentfill}{rgb}{0.934305,0.525918,0.412286}%
\pgfsetfillcolor{currentfill}%
\pgfsetlinewidth{0.000000pt}%
\definecolor{currentstroke}{rgb}{0.000000,0.000000,0.000000}%
\pgfsetstrokecolor{currentstroke}%
\pgfsetdash{}{0pt}%
\pgfpathmoveto{\pgfqpoint{3.115514in}{5.083716in}}%
\pgfpathlineto{\pgfqpoint{3.125435in}{4.905971in}}%
\pgfpathlineto{\pgfqpoint{3.132260in}{5.026506in}}%
\pgfpathlineto{\pgfqpoint{3.165512in}{5.019055in}}%
\pgfpathlineto{\pgfqpoint{3.198544in}{5.031909in}}%
\pgfpathlineto{\pgfqpoint{3.190968in}{4.973649in}}%
\pgfpathlineto{\pgfqpoint{3.183202in}{4.937327in}}%
\pgfpathlineto{\pgfqpoint{3.151576in}{4.795827in}}%
\pgfpathlineto{\pgfqpoint{3.115514in}{5.083716in}}%
\pgfpathclose%
\pgfusepath{fill}%
\end{pgfscope}%
\begin{pgfscope}%
\pgfpathrectangle{\pgfqpoint{1.020000in}{0.880000in}}{\pgfqpoint{6.160000in}{6.160000in}}%
\pgfusepath{clip}%
\pgfsetbuttcap%
\pgfsetroundjoin%
\definecolor{currentfill}{rgb}{0.252663,0.332837,0.783665}%
\pgfsetfillcolor{currentfill}%
\pgfsetlinewidth{0.000000pt}%
\definecolor{currentstroke}{rgb}{0.000000,0.000000,0.000000}%
\pgfsetstrokecolor{currentstroke}%
\pgfsetdash{}{0pt}%
\pgfpathmoveto{\pgfqpoint{5.049802in}{3.130096in}}%
\pgfpathlineto{\pgfqpoint{5.059264in}{3.060794in}}%
\pgfpathlineto{\pgfqpoint{5.069093in}{3.032150in}}%
\pgfpathlineto{\pgfqpoint{5.101121in}{2.972096in}}%
\pgfpathlineto{\pgfqpoint{5.134572in}{3.067094in}}%
\pgfpathlineto{\pgfqpoint{5.125573in}{3.188741in}}%
\pgfpathlineto{\pgfqpoint{5.114904in}{3.132056in}}%
\pgfpathlineto{\pgfqpoint{5.082164in}{3.109013in}}%
\pgfpathlineto{\pgfqpoint{5.049802in}{3.130096in}}%
\pgfpathclose%
\pgfusepath{fill}%
\end{pgfscope}%
\begin{pgfscope}%
\pgfpathrectangle{\pgfqpoint{1.020000in}{0.880000in}}{\pgfqpoint{6.160000in}{6.160000in}}%
\pgfusepath{clip}%
\pgfsetbuttcap%
\pgfsetroundjoin%
\definecolor{currentfill}{rgb}{0.313946,0.420052,0.854993}%
\pgfsetfillcolor{currentfill}%
\pgfsetlinewidth{0.000000pt}%
\definecolor{currentstroke}{rgb}{0.000000,0.000000,0.000000}%
\pgfsetstrokecolor{currentstroke}%
\pgfsetdash{}{0pt}%
\pgfpathmoveto{\pgfqpoint{5.180546in}{3.195382in}}%
\pgfpathlineto{\pgfqpoint{5.189492in}{3.070966in}}%
\pgfpathlineto{\pgfqpoint{5.199300in}{3.031565in}}%
\pgfpathlineto{\pgfqpoint{5.233372in}{3.180394in}}%
\pgfpathlineto{\pgfqpoint{5.265896in}{3.179320in}}%
\pgfpathlineto{\pgfqpoint{5.257423in}{3.347762in}}%
\pgfpathlineto{\pgfqpoint{5.248592in}{3.483840in}}%
\pgfpathlineto{\pgfqpoint{5.212837in}{3.175443in}}%
\pgfpathlineto{\pgfqpoint{5.180546in}{3.195382in}}%
\pgfpathclose%
\pgfusepath{fill}%
\end{pgfscope}%
\begin{pgfscope}%
\pgfpathrectangle{\pgfqpoint{1.020000in}{0.880000in}}{\pgfqpoint{6.160000in}{6.160000in}}%
\pgfusepath{clip}%
\pgfsetbuttcap%
\pgfsetroundjoin%
\definecolor{currentfill}{rgb}{0.586921,0.718121,0.998874}%
\pgfsetfillcolor{currentfill}%
\pgfsetlinewidth{0.000000pt}%
\definecolor{currentstroke}{rgb}{0.000000,0.000000,0.000000}%
\pgfsetstrokecolor{currentstroke}%
\pgfsetdash{}{0pt}%
\pgfpathmoveto{\pgfqpoint{4.218007in}{3.572621in}}%
\pgfpathlineto{\pgfqpoint{4.227247in}{3.704385in}}%
\pgfpathlineto{\pgfqpoint{4.236516in}{3.808060in}}%
\pgfpathlineto{\pgfqpoint{4.269181in}{3.558937in}}%
\pgfpathlineto{\pgfqpoint{4.302151in}{3.720627in}}%
\pgfpathlineto{\pgfqpoint{4.292853in}{3.701976in}}%
\pgfpathlineto{\pgfqpoint{4.283520in}{3.627141in}}%
\pgfpathlineto{\pgfqpoint{4.251001in}{3.951286in}}%
\pgfpathlineto{\pgfqpoint{4.218007in}{3.572621in}}%
\pgfpathclose%
\pgfusepath{fill}%
\end{pgfscope}%
\begin{pgfscope}%
\pgfpathrectangle{\pgfqpoint{1.020000in}{0.880000in}}{\pgfqpoint{6.160000in}{6.160000in}}%
\pgfusepath{clip}%
\pgfsetbuttcap%
\pgfsetroundjoin%
\definecolor{currentfill}{rgb}{0.905783,0.455186,0.355336}%
\pgfsetfillcolor{currentfill}%
\pgfsetlinewidth{0.000000pt}%
\definecolor{currentstroke}{rgb}{0.000000,0.000000,0.000000}%
\pgfsetstrokecolor{currentstroke}%
\pgfsetdash{}{0pt}%
\pgfpathmoveto{\pgfqpoint{2.969120in}{4.903103in}}%
\pgfpathlineto{\pgfqpoint{2.974122in}{5.146419in}}%
\pgfpathlineto{\pgfqpoint{2.982125in}{5.140274in}}%
\pgfpathlineto{\pgfqpoint{3.014808in}{5.187659in}}%
\pgfpathlineto{\pgfqpoint{3.049480in}{5.057789in}}%
\pgfpathlineto{\pgfqpoint{3.041116in}{5.091383in}}%
\pgfpathlineto{\pgfqpoint{3.034801in}{4.944862in}}%
\pgfpathlineto{\pgfqpoint{2.999782in}{5.111324in}}%
\pgfpathlineto{\pgfqpoint{2.969120in}{4.903103in}}%
\pgfpathclose%
\pgfusepath{fill}%
\end{pgfscope}%
\begin{pgfscope}%
\pgfpathrectangle{\pgfqpoint{1.020000in}{0.880000in}}{\pgfqpoint{6.160000in}{6.160000in}}%
\pgfusepath{clip}%
\pgfsetbuttcap%
\pgfsetroundjoin%
\definecolor{currentfill}{rgb}{0.358415,0.478426,0.896795}%
\pgfsetfillcolor{currentfill}%
\pgfsetlinewidth{0.000000pt}%
\definecolor{currentstroke}{rgb}{0.000000,0.000000,0.000000}%
\pgfsetstrokecolor{currentstroke}%
\pgfsetdash{}{0pt}%
\pgfpathmoveto{\pgfqpoint{5.527484in}{3.327552in}}%
\pgfpathlineto{\pgfqpoint{5.538129in}{3.333189in}}%
\pgfpathlineto{\pgfqpoint{5.548144in}{3.291058in}}%
\pgfpathlineto{\pgfqpoint{5.580799in}{3.304862in}}%
\pgfpathlineto{\pgfqpoint{5.610364in}{3.103516in}}%
\pgfpathlineto{\pgfqpoint{5.602444in}{3.294090in}}%
\pgfpathlineto{\pgfqpoint{5.591768in}{3.290637in}}%
\pgfpathlineto{\pgfqpoint{5.561693in}{3.457993in}}%
\pgfpathlineto{\pgfqpoint{5.527484in}{3.327552in}}%
\pgfpathclose%
\pgfusepath{fill}%
\end{pgfscope}%
\begin{pgfscope}%
\pgfpathrectangle{\pgfqpoint{1.020000in}{0.880000in}}{\pgfqpoint{6.160000in}{6.160000in}}%
\pgfusepath{clip}%
\pgfsetbuttcap%
\pgfsetroundjoin%
\definecolor{currentfill}{rgb}{0.348323,0.465711,0.888346}%
\pgfsetfillcolor{currentfill}%
\pgfsetlinewidth{0.000000pt}%
\definecolor{currentstroke}{rgb}{0.000000,0.000000,0.000000}%
\pgfsetstrokecolor{currentstroke}%
\pgfsetdash{}{0pt}%
\pgfpathmoveto{\pgfqpoint{5.248592in}{3.483840in}}%
\pgfpathlineto{\pgfqpoint{5.257423in}{3.347762in}}%
\pgfpathlineto{\pgfqpoint{5.265896in}{3.179320in}}%
\pgfpathlineto{\pgfqpoint{5.298538in}{3.190651in}}%
\pgfpathlineto{\pgfqpoint{5.329816in}{3.084438in}}%
\pgfpathlineto{\pgfqpoint{5.321760in}{3.284366in}}%
\pgfpathlineto{\pgfqpoint{5.311848in}{3.321820in}}%
\pgfpathlineto{\pgfqpoint{5.279558in}{3.339203in}}%
\pgfpathlineto{\pgfqpoint{5.248592in}{3.483840in}}%
\pgfpathclose%
\pgfusepath{fill}%
\end{pgfscope}%
\begin{pgfscope}%
\pgfpathrectangle{\pgfqpoint{1.020000in}{0.880000in}}{\pgfqpoint{6.160000in}{6.160000in}}%
\pgfusepath{clip}%
\pgfsetbuttcap%
\pgfsetroundjoin%
\definecolor{currentfill}{rgb}{0.404421,0.534643,0.932002}%
\pgfsetfillcolor{currentfill}%
\pgfsetlinewidth{0.000000pt}%
\definecolor{currentstroke}{rgb}{0.000000,0.000000,0.000000}%
\pgfsetstrokecolor{currentstroke}%
\pgfsetdash{}{0pt}%
\pgfpathmoveto{\pgfqpoint{4.601292in}{3.448194in}}%
\pgfpathlineto{\pgfqpoint{4.610957in}{3.462957in}}%
\pgfpathlineto{\pgfqpoint{4.620220in}{3.380167in}}%
\pgfpathlineto{\pgfqpoint{4.652849in}{3.357385in}}%
\pgfpathlineto{\pgfqpoint{4.684919in}{3.229976in}}%
\pgfpathlineto{\pgfqpoint{4.676152in}{3.412013in}}%
\pgfpathlineto{\pgfqpoint{4.666057in}{3.322380in}}%
\pgfpathlineto{\pgfqpoint{4.633979in}{3.446136in}}%
\pgfpathlineto{\pgfqpoint{4.601292in}{3.448194in}}%
\pgfpathclose%
\pgfusepath{fill}%
\end{pgfscope}%
\begin{pgfscope}%
\pgfpathrectangle{\pgfqpoint{1.020000in}{0.880000in}}{\pgfqpoint{6.160000in}{6.160000in}}%
\pgfusepath{clip}%
\pgfsetbuttcap%
\pgfsetroundjoin%
\definecolor{currentfill}{rgb}{0.950956,0.786875,0.704761}%
\pgfsetfillcolor{currentfill}%
\pgfsetlinewidth{0.000000pt}%
\definecolor{currentstroke}{rgb}{0.000000,0.000000,0.000000}%
\pgfsetstrokecolor{currentstroke}%
\pgfsetdash{}{0pt}%
\pgfpathmoveto{\pgfqpoint{3.583295in}{4.516241in}}%
\pgfpathlineto{\pgfqpoint{3.590587in}{4.729411in}}%
\pgfpathlineto{\pgfqpoint{3.600309in}{4.530643in}}%
\pgfpathlineto{\pgfqpoint{3.634145in}{4.384524in}}%
\pgfpathlineto{\pgfqpoint{3.667371in}{4.335397in}}%
\pgfpathlineto{\pgfqpoint{3.658000in}{4.482482in}}%
\pgfpathlineto{\pgfqpoint{3.649294in}{4.502554in}}%
\pgfpathlineto{\pgfqpoint{3.615721in}{4.613740in}}%
\pgfpathlineto{\pgfqpoint{3.583295in}{4.516241in}}%
\pgfpathclose%
\pgfusepath{fill}%
\end{pgfscope}%
\begin{pgfscope}%
\pgfpathrectangle{\pgfqpoint{1.020000in}{0.880000in}}{\pgfqpoint{6.160000in}{6.160000in}}%
\pgfusepath{clip}%
\pgfsetbuttcap%
\pgfsetroundjoin%
\definecolor{currentfill}{rgb}{0.968533,0.715841,0.606097}%
\pgfsetfillcolor{currentfill}%
\pgfsetlinewidth{0.000000pt}%
\definecolor{currentstroke}{rgb}{0.000000,0.000000,0.000000}%
\pgfsetstrokecolor{currentstroke}%
\pgfsetdash{}{0pt}%
\pgfpathmoveto{\pgfqpoint{2.229414in}{4.522941in}}%
\pgfpathlineto{\pgfqpoint{2.233132in}{4.700126in}}%
\pgfpathlineto{\pgfqpoint{2.241471in}{4.638518in}}%
\pgfpathlineto{\pgfqpoint{2.273902in}{4.690980in}}%
\pgfpathlineto{\pgfqpoint{2.308025in}{4.652932in}}%
\pgfpathlineto{\pgfqpoint{2.299418in}{4.727579in}}%
\pgfpathlineto{\pgfqpoint{2.291377in}{4.772117in}}%
\pgfpathlineto{\pgfqpoint{2.260953in}{4.616507in}}%
\pgfpathlineto{\pgfqpoint{2.229414in}{4.522941in}}%
\pgfpathclose%
\pgfusepath{fill}%
\end{pgfscope}%
\begin{pgfscope}%
\pgfpathrectangle{\pgfqpoint{1.020000in}{0.880000in}}{\pgfqpoint{6.160000in}{6.160000in}}%
\pgfusepath{clip}%
\pgfsetbuttcap%
\pgfsetroundjoin%
\definecolor{currentfill}{rgb}{0.960490,0.616276,0.495467}%
\pgfsetfillcolor{currentfill}%
\pgfsetlinewidth{0.000000pt}%
\definecolor{currentstroke}{rgb}{0.000000,0.000000,0.000000}%
\pgfsetstrokecolor{currentstroke}%
\pgfsetdash{}{0pt}%
\pgfpathmoveto{\pgfqpoint{2.423721in}{4.828822in}}%
\pgfpathlineto{\pgfqpoint{2.432026in}{4.773312in}}%
\pgfpathlineto{\pgfqpoint{2.437064in}{4.908080in}}%
\pgfpathlineto{\pgfqpoint{2.470056in}{4.931841in}}%
\pgfpathlineto{\pgfqpoint{2.505088in}{4.832116in}}%
\pgfpathlineto{\pgfqpoint{2.499053in}{4.748445in}}%
\pgfpathlineto{\pgfqpoint{2.490200in}{4.835217in}}%
\pgfpathlineto{\pgfqpoint{2.456357in}{4.867936in}}%
\pgfpathlineto{\pgfqpoint{2.423721in}{4.828822in}}%
\pgfpathclose%
\pgfusepath{fill}%
\end{pgfscope}%
\begin{pgfscope}%
\pgfpathrectangle{\pgfqpoint{1.020000in}{0.880000in}}{\pgfqpoint{6.160000in}{6.160000in}}%
\pgfusepath{clip}%
\pgfsetbuttcap%
\pgfsetroundjoin%
\definecolor{currentfill}{rgb}{0.243520,0.319189,0.771672}%
\pgfsetfillcolor{currentfill}%
\pgfsetlinewidth{0.000000pt}%
\definecolor{currentstroke}{rgb}{0.000000,0.000000,0.000000}%
\pgfsetstrokecolor{currentstroke}%
\pgfsetdash{}{0pt}%
\pgfpathmoveto{\pgfqpoint{4.983652in}{3.006828in}}%
\pgfpathlineto{\pgfqpoint{4.994265in}{3.082692in}}%
\pgfpathlineto{\pgfqpoint{5.004683in}{3.130490in}}%
\pgfpathlineto{\pgfqpoint{5.035840in}{2.955997in}}%
\pgfpathlineto{\pgfqpoint{5.069093in}{3.032150in}}%
\pgfpathlineto{\pgfqpoint{5.059264in}{3.060794in}}%
\pgfpathlineto{\pgfqpoint{5.049802in}{3.130096in}}%
\pgfpathlineto{\pgfqpoint{5.017011in}{3.103718in}}%
\pgfpathlineto{\pgfqpoint{4.983652in}{3.006828in}}%
\pgfpathclose%
\pgfusepath{fill}%
\end{pgfscope}%
\begin{pgfscope}%
\pgfpathrectangle{\pgfqpoint{1.020000in}{0.880000in}}{\pgfqpoint{6.160000in}{6.160000in}}%
\pgfusepath{clip}%
\pgfsetbuttcap%
\pgfsetroundjoin%
\definecolor{currentfill}{rgb}{0.916071,0.833977,0.788693}%
\pgfsetfillcolor{currentfill}%
\pgfsetlinewidth{0.000000pt}%
\definecolor{currentstroke}{rgb}{0.000000,0.000000,0.000000}%
\pgfsetstrokecolor{currentstroke}%
\pgfsetdash{}{0pt}%
\pgfpathmoveto{\pgfqpoint{3.667371in}{4.335397in}}%
\pgfpathlineto{\pgfqpoint{3.675554in}{4.422857in}}%
\pgfpathlineto{\pgfqpoint{3.684132in}{4.437337in}}%
\pgfpathlineto{\pgfqpoint{3.717255in}{4.418460in}}%
\pgfpathlineto{\pgfqpoint{3.750619in}{4.336012in}}%
\pgfpathlineto{\pgfqpoint{3.741952in}{4.327327in}}%
\pgfpathlineto{\pgfqpoint{3.732892in}{4.411180in}}%
\pgfpathlineto{\pgfqpoint{3.700394in}{4.317713in}}%
\pgfpathlineto{\pgfqpoint{3.667371in}{4.335397in}}%
\pgfpathclose%
\pgfusepath{fill}%
\end{pgfscope}%
\begin{pgfscope}%
\pgfpathrectangle{\pgfqpoint{1.020000in}{0.880000in}}{\pgfqpoint{6.160000in}{6.160000in}}%
\pgfusepath{clip}%
\pgfsetbuttcap%
\pgfsetroundjoin%
\definecolor{currentfill}{rgb}{0.467678,0.605591,0.968546}%
\pgfsetfillcolor{currentfill}%
\pgfsetlinewidth{0.000000pt}%
\definecolor{currentstroke}{rgb}{0.000000,0.000000,0.000000}%
\pgfsetstrokecolor{currentstroke}%
\pgfsetdash{}{0pt}%
\pgfpathmoveto{\pgfqpoint{4.451474in}{3.467768in}}%
\pgfpathlineto{\pgfqpoint{4.460695in}{3.388954in}}%
\pgfpathlineto{\pgfqpoint{4.470853in}{3.635122in}}%
\pgfpathlineto{\pgfqpoint{4.503401in}{3.545926in}}%
\pgfpathlineto{\pgfqpoint{4.535600in}{3.379542in}}%
\pgfpathlineto{\pgfqpoint{4.526552in}{3.515214in}}%
\pgfpathlineto{\pgfqpoint{4.517133in}{3.541502in}}%
\pgfpathlineto{\pgfqpoint{4.484289in}{3.500712in}}%
\pgfpathlineto{\pgfqpoint{4.451474in}{3.467768in}}%
\pgfpathclose%
\pgfusepath{fill}%
\end{pgfscope}%
\begin{pgfscope}%
\pgfpathrectangle{\pgfqpoint{1.020000in}{0.880000in}}{\pgfqpoint{6.160000in}{6.160000in}}%
\pgfusepath{clip}%
\pgfsetbuttcap%
\pgfsetroundjoin%
\definecolor{currentfill}{rgb}{0.967317,0.657471,0.538160}%
\pgfsetfillcolor{currentfill}%
\pgfsetlinewidth{0.000000pt}%
\definecolor{currentstroke}{rgb}{0.000000,0.000000,0.000000}%
\pgfsetstrokecolor{currentstroke}%
\pgfsetdash{}{0pt}%
\pgfpathmoveto{\pgfqpoint{3.348969in}{4.864198in}}%
\pgfpathlineto{\pgfqpoint{3.358033in}{4.771457in}}%
\pgfpathlineto{\pgfqpoint{3.366045in}{4.809559in}}%
\pgfpathlineto{\pgfqpoint{3.400894in}{4.585961in}}%
\pgfpathlineto{\pgfqpoint{3.430905in}{4.993206in}}%
\pgfpathlineto{\pgfqpoint{3.425744in}{4.560688in}}%
\pgfpathlineto{\pgfqpoint{3.415392in}{4.825324in}}%
\pgfpathlineto{\pgfqpoint{3.382742in}{4.775680in}}%
\pgfpathlineto{\pgfqpoint{3.348969in}{4.864198in}}%
\pgfpathclose%
\pgfusepath{fill}%
\end{pgfscope}%
\begin{pgfscope}%
\pgfpathrectangle{\pgfqpoint{1.020000in}{0.880000in}}{\pgfqpoint{6.160000in}{6.160000in}}%
\pgfusepath{clip}%
\pgfsetbuttcap%
\pgfsetroundjoin%
\definecolor{currentfill}{rgb}{0.661968,0.775491,0.993937}%
\pgfsetfillcolor{currentfill}%
\pgfsetlinewidth{0.000000pt}%
\definecolor{currentstroke}{rgb}{0.000000,0.000000,0.000000}%
\pgfsetstrokecolor{currentstroke}%
\pgfsetdash{}{0pt}%
\pgfpathmoveto{\pgfqpoint{4.068546in}{3.799095in}}%
\pgfpathlineto{\pgfqpoint{4.077554in}{3.829288in}}%
\pgfpathlineto{\pgfqpoint{4.086597in}{3.846870in}}%
\pgfpathlineto{\pgfqpoint{4.119554in}{3.765612in}}%
\pgfpathlineto{\pgfqpoint{4.152376in}{3.886624in}}%
\pgfpathlineto{\pgfqpoint{4.143258in}{3.858554in}}%
\pgfpathlineto{\pgfqpoint{4.134165in}{3.829685in}}%
\pgfpathlineto{\pgfqpoint{4.101295in}{3.894367in}}%
\pgfpathlineto{\pgfqpoint{4.068546in}{3.799095in}}%
\pgfpathclose%
\pgfusepath{fill}%
\end{pgfscope}%
\begin{pgfscope}%
\pgfpathrectangle{\pgfqpoint{1.020000in}{0.880000in}}{\pgfqpoint{6.160000in}{6.160000in}}%
\pgfusepath{clip}%
\pgfsetbuttcap%
\pgfsetroundjoin%
\definecolor{currentfill}{rgb}{0.559747,0.694768,0.996075}%
\pgfsetfillcolor{currentfill}%
\pgfsetlinewidth{0.000000pt}%
\definecolor{currentstroke}{rgb}{0.000000,0.000000,0.000000}%
\pgfsetstrokecolor{currentstroke}%
\pgfsetdash{}{0pt}%
\pgfpathmoveto{\pgfqpoint{4.367652in}{3.663864in}}%
\pgfpathlineto{\pgfqpoint{4.377120in}{3.721980in}}%
\pgfpathlineto{\pgfqpoint{4.386546in}{3.744821in}}%
\pgfpathlineto{\pgfqpoint{4.418980in}{3.570277in}}%
\pgfpathlineto{\pgfqpoint{4.451474in}{3.467768in}}%
\pgfpathlineto{\pgfqpoint{4.442476in}{3.630654in}}%
\pgfpathlineto{\pgfqpoint{4.433414in}{3.772296in}}%
\pgfpathlineto{\pgfqpoint{4.400424in}{3.671604in}}%
\pgfpathlineto{\pgfqpoint{4.367652in}{3.663864in}}%
\pgfpathclose%
\pgfusepath{fill}%
\end{pgfscope}%
\begin{pgfscope}%
\pgfpathrectangle{\pgfqpoint{1.020000in}{0.880000in}}{\pgfqpoint{6.160000in}{6.160000in}}%
\pgfusepath{clip}%
\pgfsetbuttcap%
\pgfsetroundjoin%
\definecolor{currentfill}{rgb}{0.343278,0.459354,0.884122}%
\pgfsetfillcolor{currentfill}%
\pgfsetlinewidth{0.000000pt}%
\definecolor{currentstroke}{rgb}{0.000000,0.000000,0.000000}%
\pgfsetstrokecolor{currentstroke}%
\pgfsetdash{}{0pt}%
\pgfpathmoveto{\pgfqpoint{4.684919in}{3.229976in}}%
\pgfpathlineto{\pgfqpoint{4.694903in}{3.288433in}}%
\pgfpathlineto{\pgfqpoint{4.704359in}{3.239553in}}%
\pgfpathlineto{\pgfqpoint{4.737193in}{3.260913in}}%
\pgfpathlineto{\pgfqpoint{4.769550in}{3.202071in}}%
\pgfpathlineto{\pgfqpoint{4.760216in}{3.276581in}}%
\pgfpathlineto{\pgfqpoint{4.750368in}{3.259728in}}%
\pgfpathlineto{\pgfqpoint{4.718153in}{3.339193in}}%
\pgfpathlineto{\pgfqpoint{4.684919in}{3.229976in}}%
\pgfpathclose%
\pgfusepath{fill}%
\end{pgfscope}%
\begin{pgfscope}%
\pgfpathrectangle{\pgfqpoint{1.020000in}{0.880000in}}{\pgfqpoint{6.160000in}{6.160000in}}%
\pgfusepath{clip}%
\pgfsetbuttcap%
\pgfsetroundjoin%
\definecolor{currentfill}{rgb}{0.358415,0.478426,0.896795}%
\pgfsetfillcolor{currentfill}%
\pgfsetlinewidth{0.000000pt}%
\definecolor{currentstroke}{rgb}{0.000000,0.000000,0.000000}%
\pgfsetstrokecolor{currentstroke}%
\pgfsetdash{}{0pt}%
\pgfpathmoveto{\pgfqpoint{5.742159in}{3.260737in}}%
\pgfpathlineto{\pgfqpoint{5.752716in}{3.245874in}}%
\pgfpathlineto{\pgfqpoint{5.762827in}{3.201607in}}%
\pgfpathlineto{\pgfqpoint{5.796361in}{3.272996in}}%
\pgfpathlineto{\pgfqpoint{5.786597in}{3.340392in}}%
\pgfpathlineto{\pgfqpoint{5.776889in}{3.410756in}}%
\pgfpathlineto{\pgfqpoint{5.742159in}{3.260737in}}%
\pgfpathclose%
\pgfusepath{fill}%
\end{pgfscope}%
\begin{pgfscope}%
\pgfpathrectangle{\pgfqpoint{1.020000in}{0.880000in}}{\pgfqpoint{6.160000in}{6.160000in}}%
\pgfusepath{clip}%
\pgfsetbuttcap%
\pgfsetroundjoin%
\definecolor{currentfill}{rgb}{0.304174,0.406945,0.845263}%
\pgfsetfillcolor{currentfill}%
\pgfsetlinewidth{0.000000pt}%
\definecolor{currentstroke}{rgb}{0.000000,0.000000,0.000000}%
\pgfsetstrokecolor{currentstroke}%
\pgfsetdash{}{0pt}%
\pgfpathmoveto{\pgfqpoint{4.834194in}{3.099960in}}%
\pgfpathlineto{\pgfqpoint{4.845222in}{3.279457in}}%
\pgfpathlineto{\pgfqpoint{4.854786in}{3.233011in}}%
\pgfpathlineto{\pgfqpoint{4.887188in}{3.190806in}}%
\pgfpathlineto{\pgfqpoint{4.919679in}{3.167071in}}%
\pgfpathlineto{\pgfqpoint{4.909191in}{3.090236in}}%
\pgfpathlineto{\pgfqpoint{4.899949in}{3.181756in}}%
\pgfpathlineto{\pgfqpoint{4.867672in}{3.229003in}}%
\pgfpathlineto{\pgfqpoint{4.834194in}{3.099960in}}%
\pgfpathclose%
\pgfusepath{fill}%
\end{pgfscope}%
\begin{pgfscope}%
\pgfpathrectangle{\pgfqpoint{1.020000in}{0.880000in}}{\pgfqpoint{6.160000in}{6.160000in}}%
\pgfusepath{clip}%
\pgfsetbuttcap%
\pgfsetroundjoin%
\definecolor{currentfill}{rgb}{0.902659,0.447939,0.349721}%
\pgfsetfillcolor{currentfill}%
\pgfsetlinewidth{0.000000pt}%
\definecolor{currentstroke}{rgb}{0.000000,0.000000,0.000000}%
\pgfsetstrokecolor{currentstroke}%
\pgfsetdash{}{0pt}%
\pgfpathmoveto{\pgfqpoint{2.900326in}{5.106334in}}%
\pgfpathlineto{\pgfqpoint{2.909106in}{5.031927in}}%
\pgfpathlineto{\pgfqpoint{2.916373in}{5.079277in}}%
\pgfpathlineto{\pgfqpoint{2.949490in}{5.089896in}}%
\pgfpathlineto{\pgfqpoint{2.982125in}{5.140274in}}%
\pgfpathlineto{\pgfqpoint{2.974122in}{5.146419in}}%
\pgfpathlineto{\pgfqpoint{2.969120in}{4.903103in}}%
\pgfpathlineto{\pgfqpoint{2.933710in}{5.090619in}}%
\pgfpathlineto{\pgfqpoint{2.900326in}{5.106334in}}%
\pgfpathclose%
\pgfusepath{fill}%
\end{pgfscope}%
\begin{pgfscope}%
\pgfpathrectangle{\pgfqpoint{1.020000in}{0.880000in}}{\pgfqpoint{6.160000in}{6.160000in}}%
\pgfusepath{clip}%
\pgfsetbuttcap%
\pgfsetroundjoin%
\definecolor{currentfill}{rgb}{0.967874,0.725847,0.618489}%
\pgfsetfillcolor{currentfill}%
\pgfsetlinewidth{0.000000pt}%
\definecolor{currentstroke}{rgb}{0.000000,0.000000,0.000000}%
\pgfsetstrokecolor{currentstroke}%
\pgfsetdash{}{0pt}%
\pgfpathmoveto{\pgfqpoint{2.161679in}{4.575148in}}%
\pgfpathlineto{\pgfqpoint{2.168061in}{4.609631in}}%
\pgfpathlineto{\pgfqpoint{2.175069in}{4.613563in}}%
\pgfpathlineto{\pgfqpoint{2.203340in}{4.878951in}}%
\pgfpathlineto{\pgfqpoint{2.241471in}{4.638518in}}%
\pgfpathlineto{\pgfqpoint{2.233132in}{4.700126in}}%
\pgfpathlineto{\pgfqpoint{2.229414in}{4.522941in}}%
\pgfpathlineto{\pgfqpoint{2.194732in}{4.591400in}}%
\pgfpathlineto{\pgfqpoint{2.161679in}{4.575148in}}%
\pgfpathclose%
\pgfusepath{fill}%
\end{pgfscope}%
\begin{pgfscope}%
\pgfpathrectangle{\pgfqpoint{1.020000in}{0.880000in}}{\pgfqpoint{6.160000in}{6.160000in}}%
\pgfusepath{clip}%
\pgfsetbuttcap%
\pgfsetroundjoin%
\definecolor{currentfill}{rgb}{0.962701,0.628218,0.507636}%
\pgfsetfillcolor{currentfill}%
\pgfsetlinewidth{0.000000pt}%
\definecolor{currentstroke}{rgb}{0.000000,0.000000,0.000000}%
\pgfsetstrokecolor{currentstroke}%
\pgfsetdash{}{0pt}%
\pgfpathmoveto{\pgfqpoint{2.358938in}{4.724396in}}%
\pgfpathlineto{\pgfqpoint{2.363533in}{4.873736in}}%
\pgfpathlineto{\pgfqpoint{2.370339in}{4.901375in}}%
\pgfpathlineto{\pgfqpoint{2.405076in}{4.826832in}}%
\pgfpathlineto{\pgfqpoint{2.437064in}{4.908080in}}%
\pgfpathlineto{\pgfqpoint{2.432026in}{4.773312in}}%
\pgfpathlineto{\pgfqpoint{2.423721in}{4.828822in}}%
\pgfpathlineto{\pgfqpoint{2.392188in}{4.727644in}}%
\pgfpathlineto{\pgfqpoint{2.358938in}{4.724396in}}%
\pgfpathclose%
\pgfusepath{fill}%
\end{pgfscope}%
\begin{pgfscope}%
\pgfpathrectangle{\pgfqpoint{1.020000in}{0.880000in}}{\pgfqpoint{6.160000in}{6.160000in}}%
\pgfusepath{clip}%
\pgfsetbuttcap%
\pgfsetroundjoin%
\definecolor{currentfill}{rgb}{0.871493,0.862309,0.857016}%
\pgfsetfillcolor{currentfill}%
\pgfsetlinewidth{0.000000pt}%
\definecolor{currentstroke}{rgb}{0.000000,0.000000,0.000000}%
\pgfsetstrokecolor{currentstroke}%
\pgfsetdash{}{0pt}%
\pgfpathmoveto{\pgfqpoint{3.750619in}{4.336012in}}%
\pgfpathlineto{\pgfqpoint{3.759678in}{4.255872in}}%
\pgfpathlineto{\pgfqpoint{3.768116in}{4.327763in}}%
\pgfpathlineto{\pgfqpoint{3.801356in}{4.271004in}}%
\pgfpathlineto{\pgfqpoint{3.834735in}{4.158045in}}%
\pgfpathlineto{\pgfqpoint{3.826023in}{4.136310in}}%
\pgfpathlineto{\pgfqpoint{3.816401in}{4.371256in}}%
\pgfpathlineto{\pgfqpoint{3.784088in}{4.208681in}}%
\pgfpathlineto{\pgfqpoint{3.750619in}{4.336012in}}%
\pgfpathclose%
\pgfusepath{fill}%
\end{pgfscope}%
\begin{pgfscope}%
\pgfpathrectangle{\pgfqpoint{1.020000in}{0.880000in}}{\pgfqpoint{6.160000in}{6.160000in}}%
\pgfusepath{clip}%
\pgfsetbuttcap%
\pgfsetroundjoin%
\definecolor{currentfill}{rgb}{0.967317,0.657471,0.538160}%
\pgfsetfillcolor{currentfill}%
\pgfsetlinewidth{0.000000pt}%
\definecolor{currentstroke}{rgb}{0.000000,0.000000,0.000000}%
\pgfsetstrokecolor{currentstroke}%
\pgfsetdash{}{0pt}%
\pgfpathmoveto{\pgfqpoint{2.291377in}{4.772117in}}%
\pgfpathlineto{\pgfqpoint{2.299418in}{4.727579in}}%
\pgfpathlineto{\pgfqpoint{2.308025in}{4.652932in}}%
\pgfpathlineto{\pgfqpoint{2.338650in}{4.804668in}}%
\pgfpathlineto{\pgfqpoint{2.370339in}{4.901375in}}%
\pgfpathlineto{\pgfqpoint{2.363533in}{4.873736in}}%
\pgfpathlineto{\pgfqpoint{2.358938in}{4.724396in}}%
\pgfpathlineto{\pgfqpoint{2.326051in}{4.700512in}}%
\pgfpathlineto{\pgfqpoint{2.291377in}{4.772117in}}%
\pgfpathclose%
\pgfusepath{fill}%
\end{pgfscope}%
\begin{pgfscope}%
\pgfpathrectangle{\pgfqpoint{1.020000in}{0.880000in}}{\pgfqpoint{6.160000in}{6.160000in}}%
\pgfusepath{clip}%
\pgfsetbuttcap%
\pgfsetroundjoin%
\definecolor{currentfill}{rgb}{0.309060,0.413498,0.850128}%
\pgfsetfillcolor{currentfill}%
\pgfsetlinewidth{0.000000pt}%
\definecolor{currentstroke}{rgb}{0.000000,0.000000,0.000000}%
\pgfsetstrokecolor{currentstroke}%
\pgfsetdash{}{0pt}%
\pgfpathmoveto{\pgfqpoint{5.329816in}{3.084438in}}%
\pgfpathlineto{\pgfqpoint{5.340700in}{3.129826in}}%
\pgfpathlineto{\pgfqpoint{5.352291in}{3.232827in}}%
\pgfpathlineto{\pgfqpoint{5.385834in}{3.314733in}}%
\pgfpathlineto{\pgfqpoint{5.416736in}{3.180272in}}%
\pgfpathlineto{\pgfqpoint{5.406262in}{3.178434in}}%
\pgfpathlineto{\pgfqpoint{5.397259in}{3.295691in}}%
\pgfpathlineto{\pgfqpoint{5.362958in}{3.143193in}}%
\pgfpathlineto{\pgfqpoint{5.329816in}{3.084438in}}%
\pgfpathclose%
\pgfusepath{fill}%
\end{pgfscope}%
\begin{pgfscope}%
\pgfpathrectangle{\pgfqpoint{1.020000in}{0.880000in}}{\pgfqpoint{6.160000in}{6.160000in}}%
\pgfusepath{clip}%
\pgfsetbuttcap%
\pgfsetroundjoin%
\definecolor{currentfill}{rgb}{0.285273,0.380129,0.823469}%
\pgfsetfillcolor{currentfill}%
\pgfsetlinewidth{0.000000pt}%
\definecolor{currentstroke}{rgb}{0.000000,0.000000,0.000000}%
\pgfsetstrokecolor{currentstroke}%
\pgfsetdash{}{0pt}%
\pgfpathmoveto{\pgfqpoint{5.265896in}{3.179320in}}%
\pgfpathlineto{\pgfqpoint{5.275322in}{3.099101in}}%
\pgfpathlineto{\pgfqpoint{5.285847in}{3.117816in}}%
\pgfpathlineto{\pgfqpoint{5.318473in}{3.123775in}}%
\pgfpathlineto{\pgfqpoint{5.352291in}{3.232827in}}%
\pgfpathlineto{\pgfqpoint{5.340700in}{3.129826in}}%
\pgfpathlineto{\pgfqpoint{5.329816in}{3.084438in}}%
\pgfpathlineto{\pgfqpoint{5.298538in}{3.190651in}}%
\pgfpathlineto{\pgfqpoint{5.265896in}{3.179320in}}%
\pgfpathclose%
\pgfusepath{fill}%
\end{pgfscope}%
\begin{pgfscope}%
\pgfpathrectangle{\pgfqpoint{1.020000in}{0.880000in}}{\pgfqpoint{6.160000in}{6.160000in}}%
\pgfusepath{clip}%
\pgfsetbuttcap%
\pgfsetroundjoin%
\definecolor{currentfill}{rgb}{0.373552,0.497499,0.909467}%
\pgfsetfillcolor{currentfill}%
\pgfsetlinewidth{0.000000pt}%
\definecolor{currentstroke}{rgb}{0.000000,0.000000,0.000000}%
\pgfsetstrokecolor{currentstroke}%
\pgfsetdash{}{0pt}%
\pgfpathmoveto{\pgfqpoint{5.463987in}{3.435635in}}%
\pgfpathlineto{\pgfqpoint{5.472822in}{3.305279in}}%
\pgfpathlineto{\pgfqpoint{5.482188in}{3.216162in}}%
\pgfpathlineto{\pgfqpoint{5.515200in}{3.256381in}}%
\pgfpathlineto{\pgfqpoint{5.548144in}{3.291058in}}%
\pgfpathlineto{\pgfqpoint{5.538129in}{3.333189in}}%
\pgfpathlineto{\pgfqpoint{5.527484in}{3.327552in}}%
\pgfpathlineto{\pgfqpoint{5.495986in}{3.398610in}}%
\pgfpathlineto{\pgfqpoint{5.463987in}{3.435635in}}%
\pgfpathclose%
\pgfusepath{fill}%
\end{pgfscope}%
\begin{pgfscope}%
\pgfpathrectangle{\pgfqpoint{1.020000in}{0.880000in}}{\pgfqpoint{6.160000in}{6.160000in}}%
\pgfusepath{clip}%
\pgfsetbuttcap%
\pgfsetroundjoin%
\definecolor{currentfill}{rgb}{0.257234,0.339661,0.789661}%
\pgfsetfillcolor{currentfill}%
\pgfsetlinewidth{0.000000pt}%
\definecolor{currentstroke}{rgb}{0.000000,0.000000,0.000000}%
\pgfsetstrokecolor{currentstroke}%
\pgfsetdash{}{0pt}%
\pgfpathmoveto{\pgfqpoint{5.199300in}{3.031565in}}%
\pgfpathlineto{\pgfqpoint{5.209831in}{3.060978in}}%
\pgfpathlineto{\pgfqpoint{5.220016in}{3.054531in}}%
\pgfpathlineto{\pgfqpoint{5.251978in}{2.997403in}}%
\pgfpathlineto{\pgfqpoint{5.285847in}{3.117816in}}%
\pgfpathlineto{\pgfqpoint{5.275322in}{3.099101in}}%
\pgfpathlineto{\pgfqpoint{5.265896in}{3.179320in}}%
\pgfpathlineto{\pgfqpoint{5.233372in}{3.180394in}}%
\pgfpathlineto{\pgfqpoint{5.199300in}{3.031565in}}%
\pgfpathclose%
\pgfusepath{fill}%
\end{pgfscope}%
\begin{pgfscope}%
\pgfpathrectangle{\pgfqpoint{1.020000in}{0.880000in}}{\pgfqpoint{6.160000in}{6.160000in}}%
\pgfusepath{clip}%
\pgfsetbuttcap%
\pgfsetroundjoin%
\definecolor{currentfill}{rgb}{0.234377,0.305542,0.759680}%
\pgfsetfillcolor{currentfill}%
\pgfsetlinewidth{0.000000pt}%
\definecolor{currentstroke}{rgb}{0.000000,0.000000,0.000000}%
\pgfsetstrokecolor{currentstroke}%
\pgfsetdash{}{0pt}%
\pgfpathmoveto{\pgfqpoint{5.134572in}{3.067094in}}%
\pgfpathlineto{\pgfqpoint{5.144431in}{3.034997in}}%
\pgfpathlineto{\pgfqpoint{5.155031in}{3.077376in}}%
\pgfpathlineto{\pgfqpoint{5.186425in}{2.955593in}}%
\pgfpathlineto{\pgfqpoint{5.220016in}{3.054531in}}%
\pgfpathlineto{\pgfqpoint{5.209831in}{3.060978in}}%
\pgfpathlineto{\pgfqpoint{5.199300in}{3.031565in}}%
\pgfpathlineto{\pgfqpoint{5.166711in}{3.024966in}}%
\pgfpathlineto{\pgfqpoint{5.134572in}{3.067094in}}%
\pgfpathclose%
\pgfusepath{fill}%
\end{pgfscope}%
\begin{pgfscope}%
\pgfpathrectangle{\pgfqpoint{1.020000in}{0.880000in}}{\pgfqpoint{6.160000in}{6.160000in}}%
\pgfusepath{clip}%
\pgfsetbuttcap%
\pgfsetroundjoin%
\definecolor{currentfill}{rgb}{0.363461,0.484784,0.901019}%
\pgfsetfillcolor{currentfill}%
\pgfsetlinewidth{0.000000pt}%
\definecolor{currentstroke}{rgb}{0.000000,0.000000,0.000000}%
\pgfsetstrokecolor{currentstroke}%
\pgfsetdash{}{0pt}%
\pgfpathmoveto{\pgfqpoint{5.678036in}{3.301583in}}%
\pgfpathlineto{\pgfqpoint{5.689604in}{3.356927in}}%
\pgfpathlineto{\pgfqpoint{5.700631in}{3.374372in}}%
\pgfpathlineto{\pgfqpoint{5.732919in}{3.362326in}}%
\pgfpathlineto{\pgfqpoint{5.762827in}{3.201607in}}%
\pgfpathlineto{\pgfqpoint{5.752716in}{3.245874in}}%
\pgfpathlineto{\pgfqpoint{5.742159in}{3.260737in}}%
\pgfpathlineto{\pgfqpoint{5.709853in}{3.264148in}}%
\pgfpathlineto{\pgfqpoint{5.678036in}{3.301583in}}%
\pgfpathclose%
\pgfusepath{fill}%
\end{pgfscope}%
\begin{pgfscope}%
\pgfpathrectangle{\pgfqpoint{1.020000in}{0.880000in}}{\pgfqpoint{6.160000in}{6.160000in}}%
\pgfusepath{clip}%
\pgfsetbuttcap%
\pgfsetroundjoin%
\definecolor{currentfill}{rgb}{0.353369,0.472069,0.892570}%
\pgfsetfillcolor{currentfill}%
\pgfsetlinewidth{0.000000pt}%
\definecolor{currentstroke}{rgb}{0.000000,0.000000,0.000000}%
\pgfsetstrokecolor{currentstroke}%
\pgfsetdash{}{0pt}%
\pgfpathmoveto{\pgfqpoint{5.397259in}{3.295691in}}%
\pgfpathlineto{\pgfqpoint{5.406262in}{3.178434in}}%
\pgfpathlineto{\pgfqpoint{5.416736in}{3.180272in}}%
\pgfpathlineto{\pgfqpoint{5.450720in}{3.296991in}}%
\pgfpathlineto{\pgfqpoint{5.482188in}{3.216162in}}%
\pgfpathlineto{\pgfqpoint{5.472822in}{3.305279in}}%
\pgfpathlineto{\pgfqpoint{5.463987in}{3.435635in}}%
\pgfpathlineto{\pgfqpoint{5.429928in}{3.311040in}}%
\pgfpathlineto{\pgfqpoint{5.397259in}{3.295691in}}%
\pgfpathclose%
\pgfusepath{fill}%
\end{pgfscope}%
\begin{pgfscope}%
\pgfpathrectangle{\pgfqpoint{1.020000in}{0.880000in}}{\pgfqpoint{6.160000in}{6.160000in}}%
\pgfusepath{clip}%
\pgfsetbuttcap%
\pgfsetroundjoin%
\definecolor{currentfill}{rgb}{0.309060,0.413498,0.850128}%
\pgfsetfillcolor{currentfill}%
\pgfsetlinewidth{0.000000pt}%
\definecolor{currentstroke}{rgb}{0.000000,0.000000,0.000000}%
\pgfsetstrokecolor{currentstroke}%
\pgfsetdash{}{0pt}%
\pgfpathmoveto{\pgfqpoint{4.769550in}{3.202071in}}%
\pgfpathlineto{\pgfqpoint{4.778610in}{3.080489in}}%
\pgfpathlineto{\pgfqpoint{4.788768in}{3.140530in}}%
\pgfpathlineto{\pgfqpoint{4.821900in}{3.206809in}}%
\pgfpathlineto{\pgfqpoint{4.854786in}{3.233011in}}%
\pgfpathlineto{\pgfqpoint{4.845222in}{3.279457in}}%
\pgfpathlineto{\pgfqpoint{4.834194in}{3.099960in}}%
\pgfpathlineto{\pgfqpoint{4.802482in}{3.245014in}}%
\pgfpathlineto{\pgfqpoint{4.769550in}{3.202071in}}%
\pgfpathclose%
\pgfusepath{fill}%
\end{pgfscope}%
\begin{pgfscope}%
\pgfpathrectangle{\pgfqpoint{1.020000in}{0.880000in}}{\pgfqpoint{6.160000in}{6.160000in}}%
\pgfusepath{clip}%
\pgfsetbuttcap%
\pgfsetroundjoin%
\definecolor{currentfill}{rgb}{0.441123,0.576532,0.954545}%
\pgfsetfillcolor{currentfill}%
\pgfsetlinewidth{0.000000pt}%
\definecolor{currentstroke}{rgb}{0.000000,0.000000,0.000000}%
\pgfsetstrokecolor{currentstroke}%
\pgfsetdash{}{0pt}%
\pgfpathmoveto{\pgfqpoint{4.535600in}{3.379542in}}%
\pgfpathlineto{\pgfqpoint{4.545366in}{3.443909in}}%
\pgfpathlineto{\pgfqpoint{4.554540in}{3.340094in}}%
\pgfpathlineto{\pgfqpoint{4.587440in}{3.374165in}}%
\pgfpathlineto{\pgfqpoint{4.620220in}{3.380167in}}%
\pgfpathlineto{\pgfqpoint{4.610957in}{3.462957in}}%
\pgfpathlineto{\pgfqpoint{4.601292in}{3.448194in}}%
\pgfpathlineto{\pgfqpoint{4.569322in}{3.641681in}}%
\pgfpathlineto{\pgfqpoint{4.535600in}{3.379542in}}%
\pgfpathclose%
\pgfusepath{fill}%
\end{pgfscope}%
\begin{pgfscope}%
\pgfpathrectangle{\pgfqpoint{1.020000in}{0.880000in}}{\pgfqpoint{6.160000in}{6.160000in}}%
\pgfusepath{clip}%
\pgfsetbuttcap%
\pgfsetroundjoin%
\definecolor{currentfill}{rgb}{0.763363,0.835092,0.955658}%
\pgfsetfillcolor{currentfill}%
\pgfsetlinewidth{0.000000pt}%
\definecolor{currentstroke}{rgb}{0.000000,0.000000,0.000000}%
\pgfsetstrokecolor{currentstroke}%
\pgfsetdash{}{0pt}%
\pgfpathmoveto{\pgfqpoint{3.918464in}{4.110225in}}%
\pgfpathlineto{\pgfqpoint{3.927087in}{4.220039in}}%
\pgfpathlineto{\pgfqpoint{3.936530in}{4.011507in}}%
\pgfpathlineto{\pgfqpoint{3.969723in}{3.907574in}}%
\pgfpathlineto{\pgfqpoint{4.002653in}{3.901530in}}%
\pgfpathlineto{\pgfqpoint{3.993617in}{3.938463in}}%
\pgfpathlineto{\pgfqpoint{3.984626in}{3.958536in}}%
\pgfpathlineto{\pgfqpoint{3.951482in}{4.076448in}}%
\pgfpathlineto{\pgfqpoint{3.918464in}{4.110225in}}%
\pgfpathclose%
\pgfusepath{fill}%
\end{pgfscope}%
\begin{pgfscope}%
\pgfpathrectangle{\pgfqpoint{1.020000in}{0.880000in}}{\pgfqpoint{6.160000in}{6.160000in}}%
\pgfusepath{clip}%
\pgfsetbuttcap%
\pgfsetroundjoin%
\definecolor{currentfill}{rgb}{0.936780,0.532750,0.418093}%
\pgfsetfillcolor{currentfill}%
\pgfsetlinewidth{0.000000pt}%
\definecolor{currentstroke}{rgb}{0.000000,0.000000,0.000000}%
\pgfsetstrokecolor{currentstroke}%
\pgfsetdash{}{0pt}%
\pgfpathmoveto{\pgfqpoint{3.198544in}{5.031909in}}%
\pgfpathlineto{\pgfqpoint{3.207847in}{4.913102in}}%
\pgfpathlineto{\pgfqpoint{3.213732in}{5.152760in}}%
\pgfpathlineto{\pgfqpoint{3.248008in}{5.040677in}}%
\pgfpathlineto{\pgfqpoint{3.282976in}{4.840303in}}%
\pgfpathlineto{\pgfqpoint{3.275170in}{4.793355in}}%
\pgfpathlineto{\pgfqpoint{3.266410in}{4.854631in}}%
\pgfpathlineto{\pgfqpoint{3.231775in}{5.022292in}}%
\pgfpathlineto{\pgfqpoint{3.198544in}{5.031909in}}%
\pgfpathclose%
\pgfusepath{fill}%
\end{pgfscope}%
\begin{pgfscope}%
\pgfpathrectangle{\pgfqpoint{1.020000in}{0.880000in}}{\pgfqpoint{6.160000in}{6.160000in}}%
\pgfusepath{clip}%
\pgfsetbuttcap%
\pgfsetroundjoin%
\definecolor{currentfill}{rgb}{0.908908,0.462433,0.360950}%
\pgfsetfillcolor{currentfill}%
\pgfsetlinewidth{0.000000pt}%
\definecolor{currentstroke}{rgb}{0.000000,0.000000,0.000000}%
\pgfsetstrokecolor{currentstroke}%
\pgfsetdash{}{0pt}%
\pgfpathmoveto{\pgfqpoint{3.049480in}{5.057789in}}%
\pgfpathlineto{\pgfqpoint{3.056287in}{5.165803in}}%
\pgfpathlineto{\pgfqpoint{3.064968in}{5.106487in}}%
\pgfpathlineto{\pgfqpoint{3.097970in}{5.128980in}}%
\pgfpathlineto{\pgfqpoint{3.132260in}{5.026506in}}%
\pgfpathlineto{\pgfqpoint{3.125435in}{4.905971in}}%
\pgfpathlineto{\pgfqpoint{3.115514in}{5.083716in}}%
\pgfpathlineto{\pgfqpoint{3.083890in}{4.942640in}}%
\pgfpathlineto{\pgfqpoint{3.049480in}{5.057789in}}%
\pgfpathclose%
\pgfusepath{fill}%
\end{pgfscope}%
\begin{pgfscope}%
\pgfpathrectangle{\pgfqpoint{1.020000in}{0.880000in}}{\pgfqpoint{6.160000in}{6.160000in}}%
\pgfusepath{clip}%
\pgfsetbuttcap%
\pgfsetroundjoin%
\definecolor{currentfill}{rgb}{0.581486,0.713451,0.998314}%
\pgfsetfillcolor{currentfill}%
\pgfsetlinewidth{0.000000pt}%
\definecolor{currentstroke}{rgb}{0.000000,0.000000,0.000000}%
\pgfsetstrokecolor{currentstroke}%
\pgfsetdash{}{0pt}%
\pgfpathmoveto{\pgfqpoint{4.302151in}{3.720627in}}%
\pgfpathlineto{\pgfqpoint{4.311270in}{3.580343in}}%
\pgfpathlineto{\pgfqpoint{4.320842in}{3.776251in}}%
\pgfpathlineto{\pgfqpoint{4.353482in}{3.626809in}}%
\pgfpathlineto{\pgfqpoint{4.386546in}{3.744821in}}%
\pgfpathlineto{\pgfqpoint{4.377120in}{3.721980in}}%
\pgfpathlineto{\pgfqpoint{4.367652in}{3.663864in}}%
\pgfpathlineto{\pgfqpoint{4.334853in}{3.647487in}}%
\pgfpathlineto{\pgfqpoint{4.302151in}{3.720627in}}%
\pgfpathclose%
\pgfusepath{fill}%
\end{pgfscope}%
\begin{pgfscope}%
\pgfpathrectangle{\pgfqpoint{1.020000in}{0.880000in}}{\pgfqpoint{6.160000in}{6.160000in}}%
\pgfusepath{clip}%
\pgfsetbuttcap%
\pgfsetroundjoin%
\definecolor{currentfill}{rgb}{0.835345,0.860514,0.898970}%
\pgfsetfillcolor{currentfill}%
\pgfsetlinewidth{0.000000pt}%
\definecolor{currentstroke}{rgb}{0.000000,0.000000,0.000000}%
\pgfsetstrokecolor{currentstroke}%
\pgfsetdash{}{0pt}%
\pgfpathmoveto{\pgfqpoint{3.834735in}{4.158045in}}%
\pgfpathlineto{\pgfqpoint{3.842943in}{4.335685in}}%
\pgfpathlineto{\pgfqpoint{3.852209in}{4.206081in}}%
\pgfpathlineto{\pgfqpoint{3.885555in}{4.092638in}}%
\pgfpathlineto{\pgfqpoint{3.918464in}{4.110225in}}%
\pgfpathlineto{\pgfqpoint{3.909516in}{4.133375in}}%
\pgfpathlineto{\pgfqpoint{3.900653in}{4.131527in}}%
\pgfpathlineto{\pgfqpoint{3.867481in}{4.217291in}}%
\pgfpathlineto{\pgfqpoint{3.834735in}{4.158045in}}%
\pgfpathclose%
\pgfusepath{fill}%
\end{pgfscope}%
\begin{pgfscope}%
\pgfpathrectangle{\pgfqpoint{1.020000in}{0.880000in}}{\pgfqpoint{6.160000in}{6.160000in}}%
\pgfusepath{clip}%
\pgfsetbuttcap%
\pgfsetroundjoin%
\definecolor{currentfill}{rgb}{0.688188,0.793178,0.988038}%
\pgfsetfillcolor{currentfill}%
\pgfsetlinewidth{0.000000pt}%
\definecolor{currentstroke}{rgb}{0.000000,0.000000,0.000000}%
\pgfsetstrokecolor{currentstroke}%
\pgfsetdash{}{0pt}%
\pgfpathmoveto{\pgfqpoint{4.002653in}{3.901530in}}%
\pgfpathlineto{\pgfqpoint{4.011457in}{4.007699in}}%
\pgfpathlineto{\pgfqpoint{4.020503in}{3.987034in}}%
\pgfpathlineto{\pgfqpoint{4.053831in}{3.724430in}}%
\pgfpathlineto{\pgfqpoint{4.086597in}{3.846870in}}%
\pgfpathlineto{\pgfqpoint{4.077554in}{3.829288in}}%
\pgfpathlineto{\pgfqpoint{4.068546in}{3.799095in}}%
\pgfpathlineto{\pgfqpoint{4.035541in}{3.908244in}}%
\pgfpathlineto{\pgfqpoint{4.002653in}{3.901530in}}%
\pgfpathclose%
\pgfusepath{fill}%
\end{pgfscope}%
\begin{pgfscope}%
\pgfpathrectangle{\pgfqpoint{1.020000in}{0.880000in}}{\pgfqpoint{6.160000in}{6.160000in}}%
\pgfusepath{clip}%
\pgfsetbuttcap%
\pgfsetroundjoin%
\definecolor{currentfill}{rgb}{0.271104,0.360011,0.807095}%
\pgfsetfillcolor{currentfill}%
\pgfsetlinewidth{0.000000pt}%
\definecolor{currentstroke}{rgb}{0.000000,0.000000,0.000000}%
\pgfsetstrokecolor{currentstroke}%
\pgfsetdash{}{0pt}%
\pgfpathmoveto{\pgfqpoint{4.919679in}{3.167071in}}%
\pgfpathlineto{\pgfqpoint{4.929092in}{3.095338in}}%
\pgfpathlineto{\pgfqpoint{4.939466in}{3.149644in}}%
\pgfpathlineto{\pgfqpoint{4.973007in}{3.256250in}}%
\pgfpathlineto{\pgfqpoint{5.004683in}{3.130490in}}%
\pgfpathlineto{\pgfqpoint{4.994265in}{3.082692in}}%
\pgfpathlineto{\pgfqpoint{4.983652in}{3.006828in}}%
\pgfpathlineto{\pgfqpoint{4.951169in}{3.017891in}}%
\pgfpathlineto{\pgfqpoint{4.919679in}{3.167071in}}%
\pgfpathclose%
\pgfusepath{fill}%
\end{pgfscope}%
\begin{pgfscope}%
\pgfpathrectangle{\pgfqpoint{1.020000in}{0.880000in}}{\pgfqpoint{6.160000in}{6.160000in}}%
\pgfusepath{clip}%
\pgfsetbuttcap%
\pgfsetroundjoin%
\definecolor{currentfill}{rgb}{0.640828,0.760752,0.997846}%
\pgfsetfillcolor{currentfill}%
\pgfsetlinewidth{0.000000pt}%
\definecolor{currentstroke}{rgb}{0.000000,0.000000,0.000000}%
\pgfsetstrokecolor{currentstroke}%
\pgfsetdash{}{0pt}%
\pgfpathmoveto{\pgfqpoint{4.152376in}{3.886624in}}%
\pgfpathlineto{\pgfqpoint{4.161564in}{3.699024in}}%
\pgfpathlineto{\pgfqpoint{4.170688in}{3.908625in}}%
\pgfpathlineto{\pgfqpoint{4.203629in}{3.860257in}}%
\pgfpathlineto{\pgfqpoint{4.236516in}{3.808060in}}%
\pgfpathlineto{\pgfqpoint{4.227247in}{3.704385in}}%
\pgfpathlineto{\pgfqpoint{4.218007in}{3.572621in}}%
\pgfpathlineto{\pgfqpoint{4.185255in}{3.851885in}}%
\pgfpathlineto{\pgfqpoint{4.152376in}{3.886624in}}%
\pgfpathclose%
\pgfusepath{fill}%
\end{pgfscope}%
\begin{pgfscope}%
\pgfpathrectangle{\pgfqpoint{1.020000in}{0.880000in}}{\pgfqpoint{6.160000in}{6.160000in}}%
\pgfusepath{clip}%
\pgfsetbuttcap%
\pgfsetroundjoin%
\definecolor{currentfill}{rgb}{0.968500,0.673977,0.556649}%
\pgfsetfillcolor{currentfill}%
\pgfsetlinewidth{0.000000pt}%
\definecolor{currentstroke}{rgb}{0.000000,0.000000,0.000000}%
\pgfsetstrokecolor{currentstroke}%
\pgfsetdash{}{0pt}%
\pgfpathmoveto{\pgfqpoint{3.430905in}{4.993206in}}%
\pgfpathlineto{\pgfqpoint{3.441438in}{4.704456in}}%
\pgfpathlineto{\pgfqpoint{3.449524in}{4.749314in}}%
\pgfpathlineto{\pgfqpoint{3.482153in}{4.821189in}}%
\pgfpathlineto{\pgfqpoint{3.517045in}{4.553583in}}%
\pgfpathlineto{\pgfqpoint{3.507939in}{4.650141in}}%
\pgfpathlineto{\pgfqpoint{3.498783in}{4.754152in}}%
\pgfpathlineto{\pgfqpoint{3.466521in}{4.643714in}}%
\pgfpathlineto{\pgfqpoint{3.430905in}{4.993206in}}%
\pgfpathclose%
\pgfusepath{fill}%
\end{pgfscope}%
\begin{pgfscope}%
\pgfpathrectangle{\pgfqpoint{1.020000in}{0.880000in}}{\pgfqpoint{6.160000in}{6.160000in}}%
\pgfusepath{clip}%
\pgfsetbuttcap%
\pgfsetroundjoin%
\definecolor{currentfill}{rgb}{0.888390,0.417703,0.327898}%
\pgfsetfillcolor{currentfill}%
\pgfsetlinewidth{0.000000pt}%
\definecolor{currentstroke}{rgb}{0.000000,0.000000,0.000000}%
\pgfsetstrokecolor{currentstroke}%
\pgfsetdash{}{0pt}%
\pgfpathmoveto{\pgfqpoint{2.834552in}{5.056250in}}%
\pgfpathlineto{\pgfqpoint{2.840973in}{5.159241in}}%
\pgfpathlineto{\pgfqpoint{2.848402in}{5.187704in}}%
\pgfpathlineto{\pgfqpoint{2.882358in}{5.137961in}}%
\pgfpathlineto{\pgfqpoint{2.916373in}{5.079277in}}%
\pgfpathlineto{\pgfqpoint{2.909106in}{5.031927in}}%
\pgfpathlineto{\pgfqpoint{2.900326in}{5.106334in}}%
\pgfpathlineto{\pgfqpoint{2.867422in}{5.082690in}}%
\pgfpathlineto{\pgfqpoint{2.834552in}{5.056250in}}%
\pgfpathclose%
\pgfusepath{fill}%
\end{pgfscope}%
\begin{pgfscope}%
\pgfpathrectangle{\pgfqpoint{1.020000in}{0.880000in}}{\pgfqpoint{6.160000in}{6.160000in}}%
\pgfusepath{clip}%
\pgfsetbuttcap%
\pgfsetroundjoin%
\definecolor{currentfill}{rgb}{0.963772,0.749086,0.649420}%
\pgfsetfillcolor{currentfill}%
\pgfsetlinewidth{0.000000pt}%
\definecolor{currentstroke}{rgb}{0.000000,0.000000,0.000000}%
\pgfsetstrokecolor{currentstroke}%
\pgfsetdash{}{0pt}%
\pgfpathmoveto{\pgfqpoint{3.517045in}{4.553583in}}%
\pgfpathlineto{\pgfqpoint{3.524925in}{4.645926in}}%
\pgfpathlineto{\pgfqpoint{3.533694in}{4.604818in}}%
\pgfpathlineto{\pgfqpoint{3.567706in}{4.455602in}}%
\pgfpathlineto{\pgfqpoint{3.600309in}{4.530643in}}%
\pgfpathlineto{\pgfqpoint{3.590587in}{4.729411in}}%
\pgfpathlineto{\pgfqpoint{3.583295in}{4.516241in}}%
\pgfpathlineto{\pgfqpoint{3.549428in}{4.655922in}}%
\pgfpathlineto{\pgfqpoint{3.517045in}{4.553583in}}%
\pgfpathclose%
\pgfusepath{fill}%
\end{pgfscope}%
\begin{pgfscope}%
\pgfpathrectangle{\pgfqpoint{1.020000in}{0.880000in}}{\pgfqpoint{6.160000in}{6.160000in}}%
\pgfusepath{clip}%
\pgfsetbuttcap%
\pgfsetroundjoin%
\definecolor{currentfill}{rgb}{0.373552,0.497499,0.909467}%
\pgfsetfillcolor{currentfill}%
\pgfsetlinewidth{0.000000pt}%
\definecolor{currentstroke}{rgb}{0.000000,0.000000,0.000000}%
\pgfsetstrokecolor{currentstroke}%
\pgfsetdash{}{0pt}%
\pgfpathmoveto{\pgfqpoint{5.610364in}{3.103516in}}%
\pgfpathlineto{\pgfqpoint{5.624637in}{3.354615in}}%
\pgfpathlineto{\pgfqpoint{5.634661in}{3.308124in}}%
\pgfpathlineto{\pgfqpoint{5.667045in}{3.300887in}}%
\pgfpathlineto{\pgfqpoint{5.700631in}{3.374372in}}%
\pgfpathlineto{\pgfqpoint{5.689604in}{3.356927in}}%
\pgfpathlineto{\pgfqpoint{5.678036in}{3.301583in}}%
\pgfpathlineto{\pgfqpoint{5.645988in}{3.326234in}}%
\pgfpathlineto{\pgfqpoint{5.610364in}{3.103516in}}%
\pgfpathclose%
\pgfusepath{fill}%
\end{pgfscope}%
\begin{pgfscope}%
\pgfpathrectangle{\pgfqpoint{1.020000in}{0.880000in}}{\pgfqpoint{6.160000in}{6.160000in}}%
\pgfusepath{clip}%
\pgfsetbuttcap%
\pgfsetroundjoin%
\definecolor{currentfill}{rgb}{0.383662,0.510183,0.917831}%
\pgfsetfillcolor{currentfill}%
\pgfsetlinewidth{0.000000pt}%
\definecolor{currentstroke}{rgb}{0.000000,0.000000,0.000000}%
\pgfsetstrokecolor{currentstroke}%
\pgfsetdash{}{0pt}%
\pgfpathmoveto{\pgfqpoint{4.620220in}{3.380167in}}%
\pgfpathlineto{\pgfqpoint{4.630043in}{3.422022in}}%
\pgfpathlineto{\pgfqpoint{4.639471in}{3.370926in}}%
\pgfpathlineto{\pgfqpoint{4.672111in}{3.339049in}}%
\pgfpathlineto{\pgfqpoint{4.704359in}{3.239553in}}%
\pgfpathlineto{\pgfqpoint{4.694903in}{3.288433in}}%
\pgfpathlineto{\pgfqpoint{4.684919in}{3.229976in}}%
\pgfpathlineto{\pgfqpoint{4.652849in}{3.357385in}}%
\pgfpathlineto{\pgfqpoint{4.620220in}{3.380167in}}%
\pgfpathclose%
\pgfusepath{fill}%
\end{pgfscope}%
\begin{pgfscope}%
\pgfpathrectangle{\pgfqpoint{1.020000in}{0.880000in}}{\pgfqpoint{6.160000in}{6.160000in}}%
\pgfusepath{clip}%
\pgfsetbuttcap%
\pgfsetroundjoin%
\definecolor{currentfill}{rgb}{0.943432,0.802276,0.729172}%
\pgfsetfillcolor{currentfill}%
\pgfsetlinewidth{0.000000pt}%
\definecolor{currentstroke}{rgb}{0.000000,0.000000,0.000000}%
\pgfsetstrokecolor{currentstroke}%
\pgfsetdash{}{0pt}%
\pgfpathmoveto{\pgfqpoint{3.600309in}{4.530643in}}%
\pgfpathlineto{\pgfqpoint{3.608860in}{4.535215in}}%
\pgfpathlineto{\pgfqpoint{3.616635in}{4.681718in}}%
\pgfpathlineto{\pgfqpoint{3.651595in}{4.339049in}}%
\pgfpathlineto{\pgfqpoint{3.684132in}{4.437337in}}%
\pgfpathlineto{\pgfqpoint{3.675554in}{4.422857in}}%
\pgfpathlineto{\pgfqpoint{3.667371in}{4.335397in}}%
\pgfpathlineto{\pgfqpoint{3.634145in}{4.384524in}}%
\pgfpathlineto{\pgfqpoint{3.600309in}{4.530643in}}%
\pgfpathclose%
\pgfusepath{fill}%
\end{pgfscope}%
\begin{pgfscope}%
\pgfpathrectangle{\pgfqpoint{1.020000in}{0.880000in}}{\pgfqpoint{6.160000in}{6.160000in}}%
\pgfusepath{clip}%
\pgfsetbuttcap%
\pgfsetroundjoin%
\definecolor{currentfill}{rgb}{0.527132,0.664700,0.989065}%
\pgfsetfillcolor{currentfill}%
\pgfsetlinewidth{0.000000pt}%
\definecolor{currentstroke}{rgb}{0.000000,0.000000,0.000000}%
\pgfsetstrokecolor{currentstroke}%
\pgfsetdash{}{0pt}%
\pgfpathmoveto{\pgfqpoint{4.386546in}{3.744821in}}%
\pgfpathlineto{\pgfqpoint{4.395964in}{3.751854in}}%
\pgfpathlineto{\pgfqpoint{4.405019in}{3.586105in}}%
\pgfpathlineto{\pgfqpoint{4.437651in}{3.498048in}}%
\pgfpathlineto{\pgfqpoint{4.470853in}{3.635122in}}%
\pgfpathlineto{\pgfqpoint{4.460695in}{3.388954in}}%
\pgfpathlineto{\pgfqpoint{4.451474in}{3.467768in}}%
\pgfpathlineto{\pgfqpoint{4.418980in}{3.570277in}}%
\pgfpathlineto{\pgfqpoint{4.386546in}{3.744821in}}%
\pgfpathclose%
\pgfusepath{fill}%
\end{pgfscope}%
\begin{pgfscope}%
\pgfpathrectangle{\pgfqpoint{1.020000in}{0.880000in}}{\pgfqpoint{6.160000in}{6.160000in}}%
\pgfusepath{clip}%
\pgfsetbuttcap%
\pgfsetroundjoin%
\definecolor{currentfill}{rgb}{0.248091,0.326013,0.777669}%
\pgfsetfillcolor{currentfill}%
\pgfsetlinewidth{0.000000pt}%
\definecolor{currentstroke}{rgb}{0.000000,0.000000,0.000000}%
\pgfsetstrokecolor{currentstroke}%
\pgfsetdash{}{0pt}%
\pgfpathmoveto{\pgfqpoint{5.069093in}{3.032150in}}%
\pgfpathlineto{\pgfqpoint{5.079301in}{3.043775in}}%
\pgfpathlineto{\pgfqpoint{5.089806in}{3.085473in}}%
\pgfpathlineto{\pgfqpoint{5.123207in}{3.163921in}}%
\pgfpathlineto{\pgfqpoint{5.155031in}{3.077376in}}%
\pgfpathlineto{\pgfqpoint{5.144431in}{3.034997in}}%
\pgfpathlineto{\pgfqpoint{5.134572in}{3.067094in}}%
\pgfpathlineto{\pgfqpoint{5.101121in}{2.972096in}}%
\pgfpathlineto{\pgfqpoint{5.069093in}{3.032150in}}%
\pgfpathclose%
\pgfusepath{fill}%
\end{pgfscope}%
\begin{pgfscope}%
\pgfpathrectangle{\pgfqpoint{1.020000in}{0.880000in}}{\pgfqpoint{6.160000in}{6.160000in}}%
\pgfusepath{clip}%
\pgfsetbuttcap%
\pgfsetroundjoin%
\definecolor{currentfill}{rgb}{0.348323,0.465711,0.888346}%
\pgfsetfillcolor{currentfill}%
\pgfsetlinewidth{0.000000pt}%
\definecolor{currentstroke}{rgb}{0.000000,0.000000,0.000000}%
\pgfsetstrokecolor{currentstroke}%
\pgfsetdash{}{0pt}%
\pgfpathmoveto{\pgfqpoint{5.548144in}{3.291058in}}%
\pgfpathlineto{\pgfqpoint{5.558162in}{3.248162in}}%
\pgfpathlineto{\pgfqpoint{5.568323in}{3.214603in}}%
\pgfpathlineto{\pgfqpoint{5.601439in}{3.258032in}}%
\pgfpathlineto{\pgfqpoint{5.634661in}{3.308124in}}%
\pgfpathlineto{\pgfqpoint{5.624637in}{3.354615in}}%
\pgfpathlineto{\pgfqpoint{5.610364in}{3.103516in}}%
\pgfpathlineto{\pgfqpoint{5.580799in}{3.304862in}}%
\pgfpathlineto{\pgfqpoint{5.548144in}{3.291058in}}%
\pgfpathclose%
\pgfusepath{fill}%
\end{pgfscope}%
\begin{pgfscope}%
\pgfpathrectangle{\pgfqpoint{1.020000in}{0.880000in}}{\pgfqpoint{6.160000in}{6.160000in}}%
\pgfusepath{clip}%
\pgfsetbuttcap%
\pgfsetroundjoin%
\definecolor{currentfill}{rgb}{0.586921,0.718121,0.998874}%
\pgfsetfillcolor{currentfill}%
\pgfsetlinewidth{0.000000pt}%
\definecolor{currentstroke}{rgb}{0.000000,0.000000,0.000000}%
\pgfsetstrokecolor{currentstroke}%
\pgfsetdash{}{0pt}%
\pgfpathmoveto{\pgfqpoint{4.236516in}{3.808060in}}%
\pgfpathlineto{\pgfqpoint{4.245678in}{3.678691in}}%
\pgfpathlineto{\pgfqpoint{4.254927in}{3.677549in}}%
\pgfpathlineto{\pgfqpoint{4.287837in}{3.687487in}}%
\pgfpathlineto{\pgfqpoint{4.320842in}{3.776251in}}%
\pgfpathlineto{\pgfqpoint{4.311270in}{3.580343in}}%
\pgfpathlineto{\pgfqpoint{4.302151in}{3.720627in}}%
\pgfpathlineto{\pgfqpoint{4.269181in}{3.558937in}}%
\pgfpathlineto{\pgfqpoint{4.236516in}{3.808060in}}%
\pgfpathclose%
\pgfusepath{fill}%
\end{pgfscope}%
\begin{pgfscope}%
\pgfpathrectangle{\pgfqpoint{1.020000in}{0.880000in}}{\pgfqpoint{6.160000in}{6.160000in}}%
\pgfusepath{clip}%
\pgfsetbuttcap%
\pgfsetroundjoin%
\definecolor{currentfill}{rgb}{0.951254,0.578799,0.459408}%
\pgfsetfillcolor{currentfill}%
\pgfsetlinewidth{0.000000pt}%
\definecolor{currentstroke}{rgb}{0.000000,0.000000,0.000000}%
\pgfsetstrokecolor{currentstroke}%
\pgfsetdash{}{0pt}%
\pgfpathmoveto{\pgfqpoint{3.282976in}{4.840303in}}%
\pgfpathlineto{\pgfqpoint{3.290571in}{4.914015in}}%
\pgfpathlineto{\pgfqpoint{3.299002in}{4.895185in}}%
\pgfpathlineto{\pgfqpoint{3.331525in}{4.974280in}}%
\pgfpathlineto{\pgfqpoint{3.366045in}{4.809559in}}%
\pgfpathlineto{\pgfqpoint{3.358033in}{4.771457in}}%
\pgfpathlineto{\pgfqpoint{3.348969in}{4.864198in}}%
\pgfpathlineto{\pgfqpoint{3.315260in}{4.936252in}}%
\pgfpathlineto{\pgfqpoint{3.282976in}{4.840303in}}%
\pgfpathclose%
\pgfusepath{fill}%
\end{pgfscope}%
\begin{pgfscope}%
\pgfpathrectangle{\pgfqpoint{1.020000in}{0.880000in}}{\pgfqpoint{6.160000in}{6.160000in}}%
\pgfusepath{clip}%
\pgfsetbuttcap%
\pgfsetroundjoin%
\definecolor{currentfill}{rgb}{0.888390,0.417703,0.327898}%
\pgfsetfillcolor{currentfill}%
\pgfsetlinewidth{0.000000pt}%
\definecolor{currentstroke}{rgb}{0.000000,0.000000,0.000000}%
\pgfsetstrokecolor{currentstroke}%
\pgfsetdash{}{0pt}%
\pgfpathmoveto{\pgfqpoint{2.770002in}{4.917540in}}%
\pgfpathlineto{\pgfqpoint{2.775700in}{5.062455in}}%
\pgfpathlineto{\pgfqpoint{2.781798in}{5.181050in}}%
\pgfpathlineto{\pgfqpoint{2.815572in}{5.149935in}}%
\pgfpathlineto{\pgfqpoint{2.848402in}{5.187704in}}%
\pgfpathlineto{\pgfqpoint{2.840973in}{5.159241in}}%
\pgfpathlineto{\pgfqpoint{2.834552in}{5.056250in}}%
\pgfpathlineto{\pgfqpoint{2.801421in}{5.048882in}}%
\pgfpathlineto{\pgfqpoint{2.770002in}{4.917540in}}%
\pgfpathclose%
\pgfusepath{fill}%
\end{pgfscope}%
\begin{pgfscope}%
\pgfpathrectangle{\pgfqpoint{1.020000in}{0.880000in}}{\pgfqpoint{6.160000in}{6.160000in}}%
\pgfusepath{clip}%
\pgfsetbuttcap%
\pgfsetroundjoin%
\definecolor{currentfill}{rgb}{0.905783,0.455186,0.355336}%
\pgfsetfillcolor{currentfill}%
\pgfsetlinewidth{0.000000pt}%
\definecolor{currentstroke}{rgb}{0.000000,0.000000,0.000000}%
\pgfsetstrokecolor{currentstroke}%
\pgfsetdash{}{0pt}%
\pgfpathmoveto{\pgfqpoint{2.699988in}{5.160812in}}%
\pgfpathlineto{\pgfqpoint{2.708179in}{5.124116in}}%
\pgfpathlineto{\pgfqpoint{2.719157in}{4.895019in}}%
\pgfpathlineto{\pgfqpoint{2.750378in}{5.042426in}}%
\pgfpathlineto{\pgfqpoint{2.781798in}{5.181050in}}%
\pgfpathlineto{\pgfqpoint{2.775700in}{5.062455in}}%
\pgfpathlineto{\pgfqpoint{2.770002in}{4.917540in}}%
\pgfpathlineto{\pgfqpoint{2.735816in}{4.985062in}}%
\pgfpathlineto{\pgfqpoint{2.699988in}{5.160812in}}%
\pgfpathclose%
\pgfusepath{fill}%
\end{pgfscope}%
\begin{pgfscope}%
\pgfpathrectangle{\pgfqpoint{1.020000in}{0.880000in}}{\pgfqpoint{6.160000in}{6.160000in}}%
\pgfusepath{clip}%
\pgfsetbuttcap%
\pgfsetroundjoin%
\definecolor{currentfill}{rgb}{0.343278,0.459354,0.884122}%
\pgfsetfillcolor{currentfill}%
\pgfsetlinewidth{0.000000pt}%
\definecolor{currentstroke}{rgb}{0.000000,0.000000,0.000000}%
\pgfsetstrokecolor{currentstroke}%
\pgfsetdash{}{0pt}%
\pgfpathmoveto{\pgfqpoint{5.482188in}{3.216162in}}%
\pgfpathlineto{\pgfqpoint{5.493008in}{3.237562in}}%
\pgfpathlineto{\pgfqpoint{5.505240in}{3.363702in}}%
\pgfpathlineto{\pgfqpoint{5.534379in}{3.109375in}}%
\pgfpathlineto{\pgfqpoint{5.568323in}{3.214603in}}%
\pgfpathlineto{\pgfqpoint{5.558162in}{3.248162in}}%
\pgfpathlineto{\pgfqpoint{5.548144in}{3.291058in}}%
\pgfpathlineto{\pgfqpoint{5.515200in}{3.256381in}}%
\pgfpathlineto{\pgfqpoint{5.482188in}{3.216162in}}%
\pgfpathclose%
\pgfusepath{fill}%
\end{pgfscope}%
\begin{pgfscope}%
\pgfpathrectangle{\pgfqpoint{1.020000in}{0.880000in}}{\pgfqpoint{6.160000in}{6.160000in}}%
\pgfusepath{clip}%
\pgfsetbuttcap%
\pgfsetroundjoin%
\definecolor{currentfill}{rgb}{0.338377,0.452819,0.879317}%
\pgfsetfillcolor{currentfill}%
\pgfsetlinewidth{0.000000pt}%
\definecolor{currentstroke}{rgb}{0.000000,0.000000,0.000000}%
\pgfsetstrokecolor{currentstroke}%
\pgfsetdash{}{0pt}%
\pgfpathmoveto{\pgfqpoint{4.704359in}{3.239553in}}%
\pgfpathlineto{\pgfqpoint{4.714464in}{3.310704in}}%
\pgfpathlineto{\pgfqpoint{4.724748in}{3.408643in}}%
\pgfpathlineto{\pgfqpoint{4.756469in}{3.214978in}}%
\pgfpathlineto{\pgfqpoint{4.788768in}{3.140530in}}%
\pgfpathlineto{\pgfqpoint{4.778610in}{3.080489in}}%
\pgfpathlineto{\pgfqpoint{4.769550in}{3.202071in}}%
\pgfpathlineto{\pgfqpoint{4.737193in}{3.260913in}}%
\pgfpathlineto{\pgfqpoint{4.704359in}{3.239553in}}%
\pgfpathclose%
\pgfusepath{fill}%
\end{pgfscope}%
\begin{pgfscope}%
\pgfpathrectangle{\pgfqpoint{1.020000in}{0.880000in}}{\pgfqpoint{6.160000in}{6.160000in}}%
\pgfusepath{clip}%
\pgfsetbuttcap%
\pgfsetroundjoin%
\definecolor{currentfill}{rgb}{0.916071,0.833977,0.788693}%
\pgfsetfillcolor{currentfill}%
\pgfsetlinewidth{0.000000pt}%
\definecolor{currentstroke}{rgb}{0.000000,0.000000,0.000000}%
\pgfsetstrokecolor{currentstroke}%
\pgfsetdash{}{0pt}%
\pgfpathmoveto{\pgfqpoint{3.684132in}{4.437337in}}%
\pgfpathlineto{\pgfqpoint{3.692721in}{4.453549in}}%
\pgfpathlineto{\pgfqpoint{3.701629in}{4.407215in}}%
\pgfpathlineto{\pgfqpoint{3.735371in}{4.260397in}}%
\pgfpathlineto{\pgfqpoint{3.768116in}{4.327763in}}%
\pgfpathlineto{\pgfqpoint{3.759678in}{4.255872in}}%
\pgfpathlineto{\pgfqpoint{3.750619in}{4.336012in}}%
\pgfpathlineto{\pgfqpoint{3.717255in}{4.418460in}}%
\pgfpathlineto{\pgfqpoint{3.684132in}{4.437337in}}%
\pgfpathclose%
\pgfusepath{fill}%
\end{pgfscope}%
\begin{pgfscope}%
\pgfpathrectangle{\pgfqpoint{1.020000in}{0.880000in}}{\pgfqpoint{6.160000in}{6.160000in}}%
\pgfusepath{clip}%
\pgfsetbuttcap%
\pgfsetroundjoin%
\definecolor{currentfill}{rgb}{0.918282,0.484173,0.377794}%
\pgfsetfillcolor{currentfill}%
\pgfsetlinewidth{0.000000pt}%
\definecolor{currentstroke}{rgb}{0.000000,0.000000,0.000000}%
\pgfsetstrokecolor{currentstroke}%
\pgfsetdash{}{0pt}%
\pgfpathmoveto{\pgfqpoint{2.637761in}{4.867014in}}%
\pgfpathlineto{\pgfqpoint{2.645302in}{4.869077in}}%
\pgfpathlineto{\pgfqpoint{2.648787in}{5.141101in}}%
\pgfpathlineto{\pgfqpoint{2.682023in}{5.153493in}}%
\pgfpathlineto{\pgfqpoint{2.719157in}{4.895019in}}%
\pgfpathlineto{\pgfqpoint{2.708179in}{5.124116in}}%
\pgfpathlineto{\pgfqpoint{2.699988in}{5.160812in}}%
\pgfpathlineto{\pgfqpoint{2.670576in}{4.897177in}}%
\pgfpathlineto{\pgfqpoint{2.637761in}{4.867014in}}%
\pgfpathclose%
\pgfusepath{fill}%
\end{pgfscope}%
\begin{pgfscope}%
\pgfpathrectangle{\pgfqpoint{1.020000in}{0.880000in}}{\pgfqpoint{6.160000in}{6.160000in}}%
\pgfusepath{clip}%
\pgfsetbuttcap%
\pgfsetroundjoin%
\definecolor{currentfill}{rgb}{0.667253,0.779176,0.992959}%
\pgfsetfillcolor{currentfill}%
\pgfsetlinewidth{0.000000pt}%
\definecolor{currentstroke}{rgb}{0.000000,0.000000,0.000000}%
\pgfsetstrokecolor{currentstroke}%
\pgfsetdash{}{0pt}%
\pgfpathmoveto{\pgfqpoint{4.086597in}{3.846870in}}%
\pgfpathlineto{\pgfqpoint{4.095499in}{4.052612in}}%
\pgfpathlineto{\pgfqpoint{4.104790in}{3.832478in}}%
\pgfpathlineto{\pgfqpoint{4.137862in}{3.592083in}}%
\pgfpathlineto{\pgfqpoint{4.170688in}{3.908625in}}%
\pgfpathlineto{\pgfqpoint{4.161564in}{3.699024in}}%
\pgfpathlineto{\pgfqpoint{4.152376in}{3.886624in}}%
\pgfpathlineto{\pgfqpoint{4.119554in}{3.765612in}}%
\pgfpathlineto{\pgfqpoint{4.086597in}{3.846870in}}%
\pgfpathclose%
\pgfusepath{fill}%
\end{pgfscope}%
\begin{pgfscope}%
\pgfpathrectangle{\pgfqpoint{1.020000in}{0.880000in}}{\pgfqpoint{6.160000in}{6.160000in}}%
\pgfusepath{clip}%
\pgfsetbuttcap%
\pgfsetroundjoin%
\definecolor{currentfill}{rgb}{0.478462,0.616564,0.972721}%
\pgfsetfillcolor{currentfill}%
\pgfsetlinewidth{0.000000pt}%
\definecolor{currentstroke}{rgb}{0.000000,0.000000,0.000000}%
\pgfsetstrokecolor{currentstroke}%
\pgfsetdash{}{0pt}%
\pgfpathmoveto{\pgfqpoint{4.470853in}{3.635122in}}%
\pgfpathlineto{\pgfqpoint{4.480183in}{3.579805in}}%
\pgfpathlineto{\pgfqpoint{4.489400in}{3.484377in}}%
\pgfpathlineto{\pgfqpoint{4.522322in}{3.506132in}}%
\pgfpathlineto{\pgfqpoint{4.554540in}{3.340094in}}%
\pgfpathlineto{\pgfqpoint{4.545366in}{3.443909in}}%
\pgfpathlineto{\pgfqpoint{4.535600in}{3.379542in}}%
\pgfpathlineto{\pgfqpoint{4.503401in}{3.545926in}}%
\pgfpathlineto{\pgfqpoint{4.470853in}{3.635122in}}%
\pgfpathclose%
\pgfusepath{fill}%
\end{pgfscope}%
\begin{pgfscope}%
\pgfpathrectangle{\pgfqpoint{1.020000in}{0.880000in}}{\pgfqpoint{6.160000in}{6.160000in}}%
\pgfusepath{clip}%
\pgfsetbuttcap%
\pgfsetroundjoin%
\definecolor{currentfill}{rgb}{0.257234,0.339661,0.789661}%
\pgfsetfillcolor{currentfill}%
\pgfsetlinewidth{0.000000pt}%
\definecolor{currentstroke}{rgb}{0.000000,0.000000,0.000000}%
\pgfsetstrokecolor{currentstroke}%
\pgfsetdash{}{0pt}%
\pgfpathmoveto{\pgfqpoint{5.004683in}{3.130490in}}%
\pgfpathlineto{\pgfqpoint{5.014232in}{3.070340in}}%
\pgfpathlineto{\pgfqpoint{5.024332in}{3.074518in}}%
\pgfpathlineto{\pgfqpoint{5.057537in}{3.132514in}}%
\pgfpathlineto{\pgfqpoint{5.089806in}{3.085473in}}%
\pgfpathlineto{\pgfqpoint{5.079301in}{3.043775in}}%
\pgfpathlineto{\pgfqpoint{5.069093in}{3.032150in}}%
\pgfpathlineto{\pgfqpoint{5.035840in}{2.955997in}}%
\pgfpathlineto{\pgfqpoint{5.004683in}{3.130490in}}%
\pgfpathclose%
\pgfusepath{fill}%
\end{pgfscope}%
\begin{pgfscope}%
\pgfpathrectangle{\pgfqpoint{1.020000in}{0.880000in}}{\pgfqpoint{6.160000in}{6.160000in}}%
\pgfusepath{clip}%
\pgfsetbuttcap%
\pgfsetroundjoin%
\definecolor{currentfill}{rgb}{0.964911,0.640159,0.519806}%
\pgfsetfillcolor{currentfill}%
\pgfsetlinewidth{0.000000pt}%
\definecolor{currentstroke}{rgb}{0.000000,0.000000,0.000000}%
\pgfsetstrokecolor{currentstroke}%
\pgfsetdash{}{0pt}%
\pgfpathmoveto{\pgfqpoint{3.366045in}{4.809559in}}%
\pgfpathlineto{\pgfqpoint{3.373913in}{4.868656in}}%
\pgfpathlineto{\pgfqpoint{3.382832in}{4.797951in}}%
\pgfpathlineto{\pgfqpoint{3.416893in}{4.681483in}}%
\pgfpathlineto{\pgfqpoint{3.449524in}{4.749314in}}%
\pgfpathlineto{\pgfqpoint{3.441438in}{4.704456in}}%
\pgfpathlineto{\pgfqpoint{3.430905in}{4.993206in}}%
\pgfpathlineto{\pgfqpoint{3.400894in}{4.585961in}}%
\pgfpathlineto{\pgfqpoint{3.366045in}{4.809559in}}%
\pgfpathclose%
\pgfusepath{fill}%
\end{pgfscope}%
\begin{pgfscope}%
\pgfpathrectangle{\pgfqpoint{1.020000in}{0.880000in}}{\pgfqpoint{6.160000in}{6.160000in}}%
\pgfusepath{clip}%
\pgfsetbuttcap%
\pgfsetroundjoin%
\definecolor{currentfill}{rgb}{0.969851,0.695830,0.581312}%
\pgfsetfillcolor{currentfill}%
\pgfsetlinewidth{0.000000pt}%
\definecolor{currentstroke}{rgb}{0.000000,0.000000,0.000000}%
\pgfsetstrokecolor{currentstroke}%
\pgfsetdash{}{0pt}%
\pgfpathmoveto{\pgfqpoint{2.241471in}{4.638518in}}%
\pgfpathlineto{\pgfqpoint{2.249286in}{4.604686in}}%
\pgfpathlineto{\pgfqpoint{2.254793in}{4.692560in}}%
\pgfpathlineto{\pgfqpoint{2.287161in}{4.753177in}}%
\pgfpathlineto{\pgfqpoint{2.321982in}{4.681504in}}%
\pgfpathlineto{\pgfqpoint{2.313880in}{4.727660in}}%
\pgfpathlineto{\pgfqpoint{2.308025in}{4.652932in}}%
\pgfpathlineto{\pgfqpoint{2.273902in}{4.690980in}}%
\pgfpathlineto{\pgfqpoint{2.241471in}{4.638518in}}%
\pgfpathclose%
\pgfusepath{fill}%
\end{pgfscope}%
\begin{pgfscope}%
\pgfpathrectangle{\pgfqpoint{1.020000in}{0.880000in}}{\pgfqpoint{6.160000in}{6.160000in}}%
\pgfusepath{clip}%
\pgfsetbuttcap%
\pgfsetroundjoin%
\definecolor{currentfill}{rgb}{0.877149,0.394645,0.311724}%
\pgfsetfillcolor{currentfill}%
\pgfsetlinewidth{0.000000pt}%
\definecolor{currentstroke}{rgb}{0.000000,0.000000,0.000000}%
\pgfsetstrokecolor{currentstroke}%
\pgfsetdash{}{0pt}%
\pgfpathmoveto{\pgfqpoint{2.982125in}{5.140274in}}%
\pgfpathlineto{\pgfqpoint{2.990362in}{5.115427in}}%
\pgfpathlineto{\pgfqpoint{2.998025in}{5.140944in}}%
\pgfpathlineto{\pgfqpoint{3.031955in}{5.084581in}}%
\pgfpathlineto{\pgfqpoint{3.064968in}{5.106487in}}%
\pgfpathlineto{\pgfqpoint{3.056287in}{5.165803in}}%
\pgfpathlineto{\pgfqpoint{3.049480in}{5.057789in}}%
\pgfpathlineto{\pgfqpoint{3.014808in}{5.187659in}}%
\pgfpathlineto{\pgfqpoint{2.982125in}{5.140274in}}%
\pgfpathclose%
\pgfusepath{fill}%
\end{pgfscope}%
\begin{pgfscope}%
\pgfpathrectangle{\pgfqpoint{1.020000in}{0.880000in}}{\pgfqpoint{6.160000in}{6.160000in}}%
\pgfusepath{clip}%
\pgfsetbuttcap%
\pgfsetroundjoin%
\definecolor{currentfill}{rgb}{0.333490,0.446265,0.874452}%
\pgfsetfillcolor{currentfill}%
\pgfsetlinewidth{0.000000pt}%
\definecolor{currentstroke}{rgb}{0.000000,0.000000,0.000000}%
\pgfsetstrokecolor{currentstroke}%
\pgfsetdash{}{0pt}%
\pgfpathmoveto{\pgfqpoint{5.416736in}{3.180272in}}%
\pgfpathlineto{\pgfqpoint{5.426626in}{3.133474in}}%
\pgfpathlineto{\pgfqpoint{5.438113in}{3.212767in}}%
\pgfpathlineto{\pgfqpoint{5.469875in}{3.149735in}}%
\pgfpathlineto{\pgfqpoint{5.505240in}{3.363702in}}%
\pgfpathlineto{\pgfqpoint{5.493008in}{3.237562in}}%
\pgfpathlineto{\pgfqpoint{5.482188in}{3.216162in}}%
\pgfpathlineto{\pgfqpoint{5.450720in}{3.296991in}}%
\pgfpathlineto{\pgfqpoint{5.416736in}{3.180272in}}%
\pgfpathclose%
\pgfusepath{fill}%
\end{pgfscope}%
\begin{pgfscope}%
\pgfpathrectangle{\pgfqpoint{1.020000in}{0.880000in}}{\pgfqpoint{6.160000in}{6.160000in}}%
\pgfusepath{clip}%
\pgfsetbuttcap%
\pgfsetroundjoin%
\definecolor{currentfill}{rgb}{0.309060,0.413498,0.850128}%
\pgfsetfillcolor{currentfill}%
\pgfsetlinewidth{0.000000pt}%
\definecolor{currentstroke}{rgb}{0.000000,0.000000,0.000000}%
\pgfsetstrokecolor{currentstroke}%
\pgfsetdash{}{0pt}%
\pgfpathmoveto{\pgfqpoint{5.352291in}{3.232827in}}%
\pgfpathlineto{\pgfqpoint{5.359525in}{2.963560in}}%
\pgfpathlineto{\pgfqpoint{5.372427in}{3.172563in}}%
\pgfpathlineto{\pgfqpoint{5.404850in}{3.158136in}}%
\pgfpathlineto{\pgfqpoint{5.438113in}{3.212767in}}%
\pgfpathlineto{\pgfqpoint{5.426626in}{3.133474in}}%
\pgfpathlineto{\pgfqpoint{5.416736in}{3.180272in}}%
\pgfpathlineto{\pgfqpoint{5.385834in}{3.314733in}}%
\pgfpathlineto{\pgfqpoint{5.352291in}{3.232827in}}%
\pgfpathclose%
\pgfusepath{fill}%
\end{pgfscope}%
\begin{pgfscope}%
\pgfpathrectangle{\pgfqpoint{1.020000in}{0.880000in}}{\pgfqpoint{6.160000in}{6.160000in}}%
\pgfusepath{clip}%
\pgfsetbuttcap%
\pgfsetroundjoin%
\definecolor{currentfill}{rgb}{0.313946,0.420052,0.854993}%
\pgfsetfillcolor{currentfill}%
\pgfsetlinewidth{0.000000pt}%
\definecolor{currentstroke}{rgb}{0.000000,0.000000,0.000000}%
\pgfsetstrokecolor{currentstroke}%
\pgfsetdash{}{0pt}%
\pgfpathmoveto{\pgfqpoint{4.854786in}{3.233011in}}%
\pgfpathlineto{\pgfqpoint{4.863706in}{3.090442in}}%
\pgfpathlineto{\pgfqpoint{4.875397in}{3.349129in}}%
\pgfpathlineto{\pgfqpoint{4.906919in}{3.172817in}}%
\pgfpathlineto{\pgfqpoint{4.939466in}{3.149644in}}%
\pgfpathlineto{\pgfqpoint{4.929092in}{3.095338in}}%
\pgfpathlineto{\pgfqpoint{4.919679in}{3.167071in}}%
\pgfpathlineto{\pgfqpoint{4.887188in}{3.190806in}}%
\pgfpathlineto{\pgfqpoint{4.854786in}{3.233011in}}%
\pgfpathclose%
\pgfusepath{fill}%
\end{pgfscope}%
\begin{pgfscope}%
\pgfpathrectangle{\pgfqpoint{1.020000in}{0.880000in}}{\pgfqpoint{6.160000in}{6.160000in}}%
\pgfusepath{clip}%
\pgfsetbuttcap%
\pgfsetroundjoin%
\definecolor{currentfill}{rgb}{0.285273,0.380129,0.823469}%
\pgfsetfillcolor{currentfill}%
\pgfsetlinewidth{0.000000pt}%
\definecolor{currentstroke}{rgb}{0.000000,0.000000,0.000000}%
\pgfsetstrokecolor{currentstroke}%
\pgfsetdash{}{0pt}%
\pgfpathmoveto{\pgfqpoint{5.285847in}{3.117816in}}%
\pgfpathlineto{\pgfqpoint{5.295440in}{3.050719in}}%
\pgfpathlineto{\pgfqpoint{5.307527in}{3.204195in}}%
\pgfpathlineto{\pgfqpoint{5.338801in}{3.085643in}}%
\pgfpathlineto{\pgfqpoint{5.372427in}{3.172563in}}%
\pgfpathlineto{\pgfqpoint{5.359525in}{2.963560in}}%
\pgfpathlineto{\pgfqpoint{5.352291in}{3.232827in}}%
\pgfpathlineto{\pgfqpoint{5.318473in}{3.123775in}}%
\pgfpathlineto{\pgfqpoint{5.285847in}{3.117816in}}%
\pgfpathclose%
\pgfusepath{fill}%
\end{pgfscope}%
\begin{pgfscope}%
\pgfpathrectangle{\pgfqpoint{1.020000in}{0.880000in}}{\pgfqpoint{6.160000in}{6.160000in}}%
\pgfusepath{clip}%
\pgfsetbuttcap%
\pgfsetroundjoin%
\definecolor{currentfill}{rgb}{0.378598,0.503856,0.913692}%
\pgfsetfillcolor{currentfill}%
\pgfsetlinewidth{0.000000pt}%
\definecolor{currentstroke}{rgb}{0.000000,0.000000,0.000000}%
\pgfsetstrokecolor{currentstroke}%
\pgfsetdash{}{0pt}%
\pgfpathmoveto{\pgfqpoint{5.762827in}{3.201607in}}%
\pgfpathlineto{\pgfqpoint{5.774171in}{3.233923in}}%
\pgfpathlineto{\pgfqpoint{5.788004in}{3.419370in}}%
\pgfpathlineto{\pgfqpoint{5.820397in}{3.412035in}}%
\pgfpathlineto{\pgfqpoint{5.807390in}{3.282881in}}%
\pgfpathlineto{\pgfqpoint{5.796361in}{3.272996in}}%
\pgfpathlineto{\pgfqpoint{5.762827in}{3.201607in}}%
\pgfpathclose%
\pgfusepath{fill}%
\end{pgfscope}%
\begin{pgfscope}%
\pgfpathrectangle{\pgfqpoint{1.020000in}{0.880000in}}{\pgfqpoint{6.160000in}{6.160000in}}%
\pgfusepath{clip}%
\pgfsetbuttcap%
\pgfsetroundjoin%
\definecolor{currentfill}{rgb}{0.921406,0.491420,0.383408}%
\pgfsetfillcolor{currentfill}%
\pgfsetlinewidth{0.000000pt}%
\definecolor{currentstroke}{rgb}{0.000000,0.000000,0.000000}%
\pgfsetstrokecolor{currentstroke}%
\pgfsetdash{}{0pt}%
\pgfpathmoveto{\pgfqpoint{2.569228in}{4.989473in}}%
\pgfpathlineto{\pgfqpoint{2.576619in}{4.997428in}}%
\pgfpathlineto{\pgfqpoint{2.584171in}{4.996248in}}%
\pgfpathlineto{\pgfqpoint{2.615813in}{5.110984in}}%
\pgfpathlineto{\pgfqpoint{2.648787in}{5.141101in}}%
\pgfpathlineto{\pgfqpoint{2.645302in}{4.869077in}}%
\pgfpathlineto{\pgfqpoint{2.637761in}{4.867014in}}%
\pgfpathlineto{\pgfqpoint{2.602377in}{5.001940in}}%
\pgfpathlineto{\pgfqpoint{2.569228in}{4.989473in}}%
\pgfpathclose%
\pgfusepath{fill}%
\end{pgfscope}%
\begin{pgfscope}%
\pgfpathrectangle{\pgfqpoint{1.020000in}{0.880000in}}{\pgfqpoint{6.160000in}{6.160000in}}%
\pgfusepath{clip}%
\pgfsetbuttcap%
\pgfsetroundjoin%
\definecolor{currentfill}{rgb}{0.818056,0.855590,0.914638}%
\pgfsetfillcolor{currentfill}%
\pgfsetlinewidth{0.000000pt}%
\definecolor{currentstroke}{rgb}{0.000000,0.000000,0.000000}%
\pgfsetstrokecolor{currentstroke}%
\pgfsetdash{}{0pt}%
\pgfpathmoveto{\pgfqpoint{3.852209in}{4.206081in}}%
\pgfpathlineto{\pgfqpoint{3.861334in}{4.119428in}}%
\pgfpathlineto{\pgfqpoint{3.870022in}{4.174477in}}%
\pgfpathlineto{\pgfqpoint{3.903668in}{3.964991in}}%
\pgfpathlineto{\pgfqpoint{3.936530in}{4.011507in}}%
\pgfpathlineto{\pgfqpoint{3.927087in}{4.220039in}}%
\pgfpathlineto{\pgfqpoint{3.918464in}{4.110225in}}%
\pgfpathlineto{\pgfqpoint{3.885555in}{4.092638in}}%
\pgfpathlineto{\pgfqpoint{3.852209in}{4.206081in}}%
\pgfpathclose%
\pgfusepath{fill}%
\end{pgfscope}%
\begin{pgfscope}%
\pgfpathrectangle{\pgfqpoint{1.020000in}{0.880000in}}{\pgfqpoint{6.160000in}{6.160000in}}%
\pgfusepath{clip}%
\pgfsetbuttcap%
\pgfsetroundjoin%
\definecolor{currentfill}{rgb}{0.743754,0.825125,0.965798}%
\pgfsetfillcolor{currentfill}%
\pgfsetlinewidth{0.000000pt}%
\definecolor{currentstroke}{rgb}{0.000000,0.000000,0.000000}%
\pgfsetstrokecolor{currentstroke}%
\pgfsetdash{}{0pt}%
\pgfpathmoveto{\pgfqpoint{3.936530in}{4.011507in}}%
\pgfpathlineto{\pgfqpoint{3.945662in}{3.926611in}}%
\pgfpathlineto{\pgfqpoint{3.954441in}{3.998749in}}%
\pgfpathlineto{\pgfqpoint{3.987543in}{3.962190in}}%
\pgfpathlineto{\pgfqpoint{4.020503in}{3.987034in}}%
\pgfpathlineto{\pgfqpoint{4.011457in}{4.007699in}}%
\pgfpathlineto{\pgfqpoint{4.002653in}{3.901530in}}%
\pgfpathlineto{\pgfqpoint{3.969723in}{3.907574in}}%
\pgfpathlineto{\pgfqpoint{3.936530in}{4.011507in}}%
\pgfpathclose%
\pgfusepath{fill}%
\end{pgfscope}%
\begin{pgfscope}%
\pgfpathrectangle{\pgfqpoint{1.020000in}{0.880000in}}{\pgfqpoint{6.160000in}{6.160000in}}%
\pgfusepath{clip}%
\pgfsetbuttcap%
\pgfsetroundjoin%
\definecolor{currentfill}{rgb}{0.964911,0.640159,0.519806}%
\pgfsetfillcolor{currentfill}%
\pgfsetlinewidth{0.000000pt}%
\definecolor{currentstroke}{rgb}{0.000000,0.000000,0.000000}%
\pgfsetstrokecolor{currentstroke}%
\pgfsetdash{}{0pt}%
\pgfpathmoveto{\pgfqpoint{2.308025in}{4.652932in}}%
\pgfpathlineto{\pgfqpoint{2.313880in}{4.727660in}}%
\pgfpathlineto{\pgfqpoint{2.321982in}{4.681504in}}%
\pgfpathlineto{\pgfqpoint{2.353655in}{4.780537in}}%
\pgfpathlineto{\pgfqpoint{2.386280in}{4.828437in}}%
\pgfpathlineto{\pgfqpoint{2.379055in}{4.822746in}}%
\pgfpathlineto{\pgfqpoint{2.370339in}{4.901375in}}%
\pgfpathlineto{\pgfqpoint{2.338650in}{4.804668in}}%
\pgfpathlineto{\pgfqpoint{2.308025in}{4.652932in}}%
\pgfpathclose%
\pgfusepath{fill}%
\end{pgfscope}%
\begin{pgfscope}%
\pgfpathrectangle{\pgfqpoint{1.020000in}{0.880000in}}{\pgfqpoint{6.160000in}{6.160000in}}%
\pgfusepath{clip}%
\pgfsetbuttcap%
\pgfsetroundjoin%
\definecolor{currentfill}{rgb}{0.931831,0.519086,0.406480}%
\pgfsetfillcolor{currentfill}%
\pgfsetlinewidth{0.000000pt}%
\definecolor{currentstroke}{rgb}{0.000000,0.000000,0.000000}%
\pgfsetstrokecolor{currentstroke}%
\pgfsetdash{}{0pt}%
\pgfpathmoveto{\pgfqpoint{2.505088in}{4.832116in}}%
\pgfpathlineto{\pgfqpoint{2.510546in}{4.952489in}}%
\pgfpathlineto{\pgfqpoint{2.518750in}{4.907459in}}%
\pgfpathlineto{\pgfqpoint{2.551039in}{4.977776in}}%
\pgfpathlineto{\pgfqpoint{2.584171in}{4.996248in}}%
\pgfpathlineto{\pgfqpoint{2.576619in}{4.997428in}}%
\pgfpathlineto{\pgfqpoint{2.569228in}{4.989473in}}%
\pgfpathlineto{\pgfqpoint{2.535728in}{4.997938in}}%
\pgfpathlineto{\pgfqpoint{2.505088in}{4.832116in}}%
\pgfpathclose%
\pgfusepath{fill}%
\end{pgfscope}%
\begin{pgfscope}%
\pgfpathrectangle{\pgfqpoint{1.020000in}{0.880000in}}{\pgfqpoint{6.160000in}{6.160000in}}%
\pgfusepath{clip}%
\pgfsetbuttcap%
\pgfsetroundjoin%
\definecolor{currentfill}{rgb}{0.895885,0.433075,0.338681}%
\pgfsetfillcolor{currentfill}%
\pgfsetlinewidth{0.000000pt}%
\definecolor{currentstroke}{rgb}{0.000000,0.000000,0.000000}%
\pgfsetstrokecolor{currentstroke}%
\pgfsetdash{}{0pt}%
\pgfpathmoveto{\pgfqpoint{3.132260in}{5.026506in}}%
\pgfpathlineto{\pgfqpoint{3.139725in}{5.088139in}}%
\pgfpathlineto{\pgfqpoint{3.146882in}{5.182591in}}%
\pgfpathlineto{\pgfqpoint{3.180891in}{5.109878in}}%
\pgfpathlineto{\pgfqpoint{3.213732in}{5.152760in}}%
\pgfpathlineto{\pgfqpoint{3.207847in}{4.913102in}}%
\pgfpathlineto{\pgfqpoint{3.198544in}{5.031909in}}%
\pgfpathlineto{\pgfqpoint{3.165512in}{5.019055in}}%
\pgfpathlineto{\pgfqpoint{3.132260in}{5.026506in}}%
\pgfpathclose%
\pgfusepath{fill}%
\end{pgfscope}%
\begin{pgfscope}%
\pgfpathrectangle{\pgfqpoint{1.020000in}{0.880000in}}{\pgfqpoint{6.160000in}{6.160000in}}%
\pgfusepath{clip}%
\pgfsetbuttcap%
\pgfsetroundjoin%
\definecolor{currentfill}{rgb}{0.891817,0.851973,0.829085}%
\pgfsetfillcolor{currentfill}%
\pgfsetlinewidth{0.000000pt}%
\definecolor{currentstroke}{rgb}{0.000000,0.000000,0.000000}%
\pgfsetstrokecolor{currentstroke}%
\pgfsetdash{}{0pt}%
\pgfpathmoveto{\pgfqpoint{3.768116in}{4.327763in}}%
\pgfpathlineto{\pgfqpoint{3.776429in}{4.438445in}}%
\pgfpathlineto{\pgfqpoint{3.785959in}{4.249632in}}%
\pgfpathlineto{\pgfqpoint{3.818873in}{4.290911in}}%
\pgfpathlineto{\pgfqpoint{3.852209in}{4.206081in}}%
\pgfpathlineto{\pgfqpoint{3.842943in}{4.335685in}}%
\pgfpathlineto{\pgfqpoint{3.834735in}{4.158045in}}%
\pgfpathlineto{\pgfqpoint{3.801356in}{4.271004in}}%
\pgfpathlineto{\pgfqpoint{3.768116in}{4.327763in}}%
\pgfpathclose%
\pgfusepath{fill}%
\end{pgfscope}%
\begin{pgfscope}%
\pgfpathrectangle{\pgfqpoint{1.020000in}{0.880000in}}{\pgfqpoint{6.160000in}{6.160000in}}%
\pgfusepath{clip}%
\pgfsetbuttcap%
\pgfsetroundjoin%
\definecolor{currentfill}{rgb}{0.576051,0.708780,0.997755}%
\pgfsetfillcolor{currentfill}%
\pgfsetlinewidth{0.000000pt}%
\definecolor{currentstroke}{rgb}{0.000000,0.000000,0.000000}%
\pgfsetstrokecolor{currentstroke}%
\pgfsetdash{}{0pt}%
\pgfpathmoveto{\pgfqpoint{4.320842in}{3.776251in}}%
\pgfpathlineto{\pgfqpoint{4.329968in}{3.628350in}}%
\pgfpathlineto{\pgfqpoint{4.338965in}{3.403768in}}%
\pgfpathlineto{\pgfqpoint{4.372371in}{3.706837in}}%
\pgfpathlineto{\pgfqpoint{4.405019in}{3.586105in}}%
\pgfpathlineto{\pgfqpoint{4.395964in}{3.751854in}}%
\pgfpathlineto{\pgfqpoint{4.386546in}{3.744821in}}%
\pgfpathlineto{\pgfqpoint{4.353482in}{3.626809in}}%
\pgfpathlineto{\pgfqpoint{4.320842in}{3.776251in}}%
\pgfpathclose%
\pgfusepath{fill}%
\end{pgfscope}%
\begin{pgfscope}%
\pgfpathrectangle{\pgfqpoint{1.020000in}{0.880000in}}{\pgfqpoint{6.160000in}{6.160000in}}%
\pgfusepath{clip}%
\pgfsetbuttcap%
\pgfsetroundjoin%
\definecolor{currentfill}{rgb}{0.441123,0.576532,0.954545}%
\pgfsetfillcolor{currentfill}%
\pgfsetlinewidth{0.000000pt}%
\definecolor{currentstroke}{rgb}{0.000000,0.000000,0.000000}%
\pgfsetstrokecolor{currentstroke}%
\pgfsetdash{}{0pt}%
\pgfpathmoveto{\pgfqpoint{4.554540in}{3.340094in}}%
\pgfpathlineto{\pgfqpoint{4.564695in}{3.493663in}}%
\pgfpathlineto{\pgfqpoint{4.574519in}{3.549904in}}%
\pgfpathlineto{\pgfqpoint{4.606579in}{3.353084in}}%
\pgfpathlineto{\pgfqpoint{4.639471in}{3.370926in}}%
\pgfpathlineto{\pgfqpoint{4.630043in}{3.422022in}}%
\pgfpathlineto{\pgfqpoint{4.620220in}{3.380167in}}%
\pgfpathlineto{\pgfqpoint{4.587440in}{3.374165in}}%
\pgfpathlineto{\pgfqpoint{4.554540in}{3.340094in}}%
\pgfpathclose%
\pgfusepath{fill}%
\end{pgfscope}%
\begin{pgfscope}%
\pgfpathrectangle{\pgfqpoint{1.020000in}{0.880000in}}{\pgfqpoint{6.160000in}{6.160000in}}%
\pgfusepath{clip}%
\pgfsetbuttcap%
\pgfsetroundjoin%
\definecolor{currentfill}{rgb}{0.683056,0.790043,0.989768}%
\pgfsetfillcolor{currentfill}%
\pgfsetlinewidth{0.000000pt}%
\definecolor{currentstroke}{rgb}{0.000000,0.000000,0.000000}%
\pgfsetstrokecolor{currentstroke}%
\pgfsetdash{}{0pt}%
\pgfpathmoveto{\pgfqpoint{4.020503in}{3.987034in}}%
\pgfpathlineto{\pgfqpoint{4.029833in}{3.791273in}}%
\pgfpathlineto{\pgfqpoint{4.038923in}{3.750601in}}%
\pgfpathlineto{\pgfqpoint{4.071884in}{3.763105in}}%
\pgfpathlineto{\pgfqpoint{4.104790in}{3.832478in}}%
\pgfpathlineto{\pgfqpoint{4.095499in}{4.052612in}}%
\pgfpathlineto{\pgfqpoint{4.086597in}{3.846870in}}%
\pgfpathlineto{\pgfqpoint{4.053831in}{3.724430in}}%
\pgfpathlineto{\pgfqpoint{4.020503in}{3.987034in}}%
\pgfpathclose%
\pgfusepath{fill}%
\end{pgfscope}%
\begin{pgfscope}%
\pgfpathrectangle{\pgfqpoint{1.020000in}{0.880000in}}{\pgfqpoint{6.160000in}{6.160000in}}%
\pgfusepath{clip}%
\pgfsetbuttcap%
\pgfsetroundjoin%
\definecolor{currentfill}{rgb}{0.266381,0.353304,0.801637}%
\pgfsetfillcolor{currentfill}%
\pgfsetlinewidth{0.000000pt}%
\definecolor{currentstroke}{rgb}{0.000000,0.000000,0.000000}%
\pgfsetstrokecolor{currentstroke}%
\pgfsetdash{}{0pt}%
\pgfpathmoveto{\pgfqpoint{4.939466in}{3.149644in}}%
\pgfpathlineto{\pgfqpoint{4.948357in}{3.006662in}}%
\pgfpathlineto{\pgfqpoint{4.958092in}{2.973146in}}%
\pgfpathlineto{\pgfqpoint{4.990845in}{2.979425in}}%
\pgfpathlineto{\pgfqpoint{5.024332in}{3.074518in}}%
\pgfpathlineto{\pgfqpoint{5.014232in}{3.070340in}}%
\pgfpathlineto{\pgfqpoint{5.004683in}{3.130490in}}%
\pgfpathlineto{\pgfqpoint{4.973007in}{3.256250in}}%
\pgfpathlineto{\pgfqpoint{4.939466in}{3.149644in}}%
\pgfpathclose%
\pgfusepath{fill}%
\end{pgfscope}%
\begin{pgfscope}%
\pgfpathrectangle{\pgfqpoint{1.020000in}{0.880000in}}{\pgfqpoint{6.160000in}{6.160000in}}%
\pgfusepath{clip}%
\pgfsetbuttcap%
\pgfsetroundjoin%
\definecolor{currentfill}{rgb}{0.939254,0.539581,0.423900}%
\pgfsetfillcolor{currentfill}%
\pgfsetlinewidth{0.000000pt}%
\definecolor{currentstroke}{rgb}{0.000000,0.000000,0.000000}%
\pgfsetstrokecolor{currentstroke}%
\pgfsetdash{}{0pt}%
\pgfpathmoveto{\pgfqpoint{2.437064in}{4.908080in}}%
\pgfpathlineto{\pgfqpoint{2.442577in}{5.017119in}}%
\pgfpathlineto{\pgfqpoint{2.452205in}{4.886877in}}%
\pgfpathlineto{\pgfqpoint{2.484113in}{4.979050in}}%
\pgfpathlineto{\pgfqpoint{2.518750in}{4.907459in}}%
\pgfpathlineto{\pgfqpoint{2.510546in}{4.952489in}}%
\pgfpathlineto{\pgfqpoint{2.505088in}{4.832116in}}%
\pgfpathlineto{\pgfqpoint{2.470056in}{4.931841in}}%
\pgfpathlineto{\pgfqpoint{2.437064in}{4.908080in}}%
\pgfpathclose%
\pgfusepath{fill}%
\end{pgfscope}%
\begin{pgfscope}%
\pgfpathrectangle{\pgfqpoint{1.020000in}{0.880000in}}{\pgfqpoint{6.160000in}{6.160000in}}%
\pgfusepath{clip}%
\pgfsetbuttcap%
\pgfsetroundjoin%
\definecolor{currentfill}{rgb}{0.280550,0.373423,0.818011}%
\pgfsetfillcolor{currentfill}%
\pgfsetlinewidth{0.000000pt}%
\definecolor{currentstroke}{rgb}{0.000000,0.000000,0.000000}%
\pgfsetstrokecolor{currentstroke}%
\pgfsetdash{}{0pt}%
\pgfpathmoveto{\pgfqpoint{5.220016in}{3.054531in}}%
\pgfpathlineto{\pgfqpoint{5.229610in}{2.989906in}}%
\pgfpathlineto{\pgfqpoint{5.243626in}{3.341168in}}%
\pgfpathlineto{\pgfqpoint{5.273781in}{3.105607in}}%
\pgfpathlineto{\pgfqpoint{5.307527in}{3.204195in}}%
\pgfpathlineto{\pgfqpoint{5.295440in}{3.050719in}}%
\pgfpathlineto{\pgfqpoint{5.285847in}{3.117816in}}%
\pgfpathlineto{\pgfqpoint{5.251978in}{2.997403in}}%
\pgfpathlineto{\pgfqpoint{5.220016in}{3.054531in}}%
\pgfpathclose%
\pgfusepath{fill}%
\end{pgfscope}%
\begin{pgfscope}%
\pgfpathrectangle{\pgfqpoint{1.020000in}{0.880000in}}{\pgfqpoint{6.160000in}{6.160000in}}%
\pgfusepath{clip}%
\pgfsetbuttcap%
\pgfsetroundjoin%
\definecolor{currentfill}{rgb}{0.969289,0.684982,0.568975}%
\pgfsetfillcolor{currentfill}%
\pgfsetlinewidth{0.000000pt}%
\definecolor{currentstroke}{rgb}{0.000000,0.000000,0.000000}%
\pgfsetstrokecolor{currentstroke}%
\pgfsetdash{}{0pt}%
\pgfpathmoveto{\pgfqpoint{2.175069in}{4.613563in}}%
\pgfpathlineto{\pgfqpoint{2.180191in}{4.713645in}}%
\pgfpathlineto{\pgfqpoint{2.188137in}{4.671878in}}%
\pgfpathlineto{\pgfqpoint{2.219823in}{4.767253in}}%
\pgfpathlineto{\pgfqpoint{2.254793in}{4.692560in}}%
\pgfpathlineto{\pgfqpoint{2.249286in}{4.604686in}}%
\pgfpathlineto{\pgfqpoint{2.241471in}{4.638518in}}%
\pgfpathlineto{\pgfqpoint{2.203340in}{4.878951in}}%
\pgfpathlineto{\pgfqpoint{2.175069in}{4.613563in}}%
\pgfpathclose%
\pgfusepath{fill}%
\end{pgfscope}%
\begin{pgfscope}%
\pgfpathrectangle{\pgfqpoint{1.020000in}{0.880000in}}{\pgfqpoint{6.160000in}{6.160000in}}%
\pgfusepath{clip}%
\pgfsetbuttcap%
\pgfsetroundjoin%
\definecolor{currentfill}{rgb}{0.949454,0.572388,0.453443}%
\pgfsetfillcolor{currentfill}%
\pgfsetlinewidth{0.000000pt}%
\definecolor{currentstroke}{rgb}{0.000000,0.000000,0.000000}%
\pgfsetstrokecolor{currentstroke}%
\pgfsetdash{}{0pt}%
\pgfpathmoveto{\pgfqpoint{2.370339in}{4.901375in}}%
\pgfpathlineto{\pgfqpoint{2.379055in}{4.822746in}}%
\pgfpathlineto{\pgfqpoint{2.386280in}{4.828437in}}%
\pgfpathlineto{\pgfqpoint{2.419907in}{4.819454in}}%
\pgfpathlineto{\pgfqpoint{2.452205in}{4.886877in}}%
\pgfpathlineto{\pgfqpoint{2.442577in}{5.017119in}}%
\pgfpathlineto{\pgfqpoint{2.437064in}{4.908080in}}%
\pgfpathlineto{\pgfqpoint{2.405076in}{4.826832in}}%
\pgfpathlineto{\pgfqpoint{2.370339in}{4.901375in}}%
\pgfpathclose%
\pgfusepath{fill}%
\end{pgfscope}%
\begin{pgfscope}%
\pgfpathrectangle{\pgfqpoint{1.020000in}{0.880000in}}{\pgfqpoint{6.160000in}{6.160000in}}%
\pgfusepath{clip}%
\pgfsetbuttcap%
\pgfsetroundjoin%
\definecolor{currentfill}{rgb}{0.323718,0.433158,0.864722}%
\pgfsetfillcolor{currentfill}%
\pgfsetlinewidth{0.000000pt}%
\definecolor{currentstroke}{rgb}{0.000000,0.000000,0.000000}%
\pgfsetstrokecolor{currentstroke}%
\pgfsetdash{}{0pt}%
\pgfpathmoveto{\pgfqpoint{4.788768in}{3.140530in}}%
\pgfpathlineto{\pgfqpoint{4.799088in}{3.221300in}}%
\pgfpathlineto{\pgfqpoint{4.808450in}{3.143555in}}%
\pgfpathlineto{\pgfqpoint{4.841013in}{3.112067in}}%
\pgfpathlineto{\pgfqpoint{4.875397in}{3.349129in}}%
\pgfpathlineto{\pgfqpoint{4.863706in}{3.090442in}}%
\pgfpathlineto{\pgfqpoint{4.854786in}{3.233011in}}%
\pgfpathlineto{\pgfqpoint{4.821900in}{3.206809in}}%
\pgfpathlineto{\pgfqpoint{4.788768in}{3.140530in}}%
\pgfpathclose%
\pgfusepath{fill}%
\end{pgfscope}%
\begin{pgfscope}%
\pgfpathrectangle{\pgfqpoint{1.020000in}{0.880000in}}{\pgfqpoint{6.160000in}{6.160000in}}%
\pgfusepath{clip}%
\pgfsetbuttcap%
\pgfsetroundjoin%
\definecolor{currentfill}{rgb}{0.865391,0.371128,0.295769}%
\pgfsetfillcolor{currentfill}%
\pgfsetlinewidth{0.000000pt}%
\definecolor{currentstroke}{rgb}{0.000000,0.000000,0.000000}%
\pgfsetstrokecolor{currentstroke}%
\pgfsetdash{}{0pt}%
\pgfpathmoveto{\pgfqpoint{2.916373in}{5.079277in}}%
\pgfpathlineto{\pgfqpoint{2.921885in}{5.270247in}}%
\pgfpathlineto{\pgfqpoint{2.932040in}{5.087707in}}%
\pgfpathlineto{\pgfqpoint{2.963768in}{5.219920in}}%
\pgfpathlineto{\pgfqpoint{2.998025in}{5.140944in}}%
\pgfpathlineto{\pgfqpoint{2.990362in}{5.115427in}}%
\pgfpathlineto{\pgfqpoint{2.982125in}{5.140274in}}%
\pgfpathlineto{\pgfqpoint{2.949490in}{5.089896in}}%
\pgfpathlineto{\pgfqpoint{2.916373in}{5.079277in}}%
\pgfpathclose%
\pgfusepath{fill}%
\end{pgfscope}%
\begin{pgfscope}%
\pgfpathrectangle{\pgfqpoint{1.020000in}{0.880000in}}{\pgfqpoint{6.160000in}{6.160000in}}%
\pgfusepath{clip}%
\pgfsetbuttcap%
\pgfsetroundjoin%
\definecolor{currentfill}{rgb}{0.968105,0.668475,0.550486}%
\pgfsetfillcolor{currentfill}%
\pgfsetlinewidth{0.000000pt}%
\definecolor{currentstroke}{rgb}{0.000000,0.000000,0.000000}%
\pgfsetstrokecolor{currentstroke}%
\pgfsetdash{}{0pt}%
\pgfpathmoveto{\pgfqpoint{3.449524in}{4.749314in}}%
\pgfpathlineto{\pgfqpoint{3.457414in}{4.825149in}}%
\pgfpathlineto{\pgfqpoint{3.466448in}{4.743477in}}%
\pgfpathlineto{\pgfqpoint{3.499245in}{4.801348in}}%
\pgfpathlineto{\pgfqpoint{3.533694in}{4.604818in}}%
\pgfpathlineto{\pgfqpoint{3.524925in}{4.645926in}}%
\pgfpathlineto{\pgfqpoint{3.517045in}{4.553583in}}%
\pgfpathlineto{\pgfqpoint{3.482153in}{4.821189in}}%
\pgfpathlineto{\pgfqpoint{3.449524in}{4.749314in}}%
\pgfpathclose%
\pgfusepath{fill}%
\end{pgfscope}%
\begin{pgfscope}%
\pgfpathrectangle{\pgfqpoint{1.020000in}{0.880000in}}{\pgfqpoint{6.160000in}{6.160000in}}%
\pgfusepath{clip}%
\pgfsetbuttcap%
\pgfsetroundjoin%
\definecolor{currentfill}{rgb}{0.651398,0.768121,0.995891}%
\pgfsetfillcolor{currentfill}%
\pgfsetlinewidth{0.000000pt}%
\definecolor{currentstroke}{rgb}{0.000000,0.000000,0.000000}%
\pgfsetstrokecolor{currentstroke}%
\pgfsetdash{}{0pt}%
\pgfpathmoveto{\pgfqpoint{4.170688in}{3.908625in}}%
\pgfpathlineto{\pgfqpoint{4.179884in}{3.684591in}}%
\pgfpathlineto{\pgfqpoint{4.189079in}{3.815710in}}%
\pgfpathlineto{\pgfqpoint{4.222055in}{3.797771in}}%
\pgfpathlineto{\pgfqpoint{4.254927in}{3.677549in}}%
\pgfpathlineto{\pgfqpoint{4.245678in}{3.678691in}}%
\pgfpathlineto{\pgfqpoint{4.236516in}{3.808060in}}%
\pgfpathlineto{\pgfqpoint{4.203629in}{3.860257in}}%
\pgfpathlineto{\pgfqpoint{4.170688in}{3.908625in}}%
\pgfpathclose%
\pgfusepath{fill}%
\end{pgfscope}%
\begin{pgfscope}%
\pgfpathrectangle{\pgfqpoint{1.020000in}{0.880000in}}{\pgfqpoint{6.160000in}{6.160000in}}%
\pgfusepath{clip}%
\pgfsetbuttcap%
\pgfsetroundjoin%
\definecolor{currentfill}{rgb}{0.964835,0.744614,0.643239}%
\pgfsetfillcolor{currentfill}%
\pgfsetlinewidth{0.000000pt}%
\definecolor{currentstroke}{rgb}{0.000000,0.000000,0.000000}%
\pgfsetstrokecolor{currentstroke}%
\pgfsetdash{}{0pt}%
\pgfpathmoveto{\pgfqpoint{3.533694in}{4.604818in}}%
\pgfpathlineto{\pgfqpoint{3.542710in}{4.526352in}}%
\pgfpathlineto{\pgfqpoint{3.549799in}{4.756978in}}%
\pgfpathlineto{\pgfqpoint{3.584602in}{4.489175in}}%
\pgfpathlineto{\pgfqpoint{3.616635in}{4.681718in}}%
\pgfpathlineto{\pgfqpoint{3.608860in}{4.535215in}}%
\pgfpathlineto{\pgfqpoint{3.600309in}{4.530643in}}%
\pgfpathlineto{\pgfqpoint{3.567706in}{4.455602in}}%
\pgfpathlineto{\pgfqpoint{3.533694in}{4.604818in}}%
\pgfpathclose%
\pgfusepath{fill}%
\end{pgfscope}%
\begin{pgfscope}%
\pgfpathrectangle{\pgfqpoint{1.020000in}{0.880000in}}{\pgfqpoint{6.160000in}{6.160000in}}%
\pgfusepath{clip}%
\pgfsetbuttcap%
\pgfsetroundjoin%
\definecolor{currentfill}{rgb}{0.404421,0.534643,0.932002}%
\pgfsetfillcolor{currentfill}%
\pgfsetlinewidth{0.000000pt}%
\definecolor{currentstroke}{rgb}{0.000000,0.000000,0.000000}%
\pgfsetstrokecolor{currentstroke}%
\pgfsetdash{}{0pt}%
\pgfpathmoveto{\pgfqpoint{5.700631in}{3.374372in}}%
\pgfpathlineto{\pgfqpoint{5.712552in}{3.448552in}}%
\pgfpathlineto{\pgfqpoint{5.721578in}{3.332219in}}%
\pgfpathlineto{\pgfqpoint{5.754655in}{3.367438in}}%
\pgfpathlineto{\pgfqpoint{5.788004in}{3.419370in}}%
\pgfpathlineto{\pgfqpoint{5.774171in}{3.233923in}}%
\pgfpathlineto{\pgfqpoint{5.762827in}{3.201607in}}%
\pgfpathlineto{\pgfqpoint{5.732919in}{3.362326in}}%
\pgfpathlineto{\pgfqpoint{5.700631in}{3.374372in}}%
\pgfpathclose%
\pgfusepath{fill}%
\end{pgfscope}%
\begin{pgfscope}%
\pgfpathrectangle{\pgfqpoint{1.020000in}{0.880000in}}{\pgfqpoint{6.160000in}{6.160000in}}%
\pgfusepath{clip}%
\pgfsetbuttcap%
\pgfsetroundjoin%
\definecolor{currentfill}{rgb}{0.527132,0.664700,0.989065}%
\pgfsetfillcolor{currentfill}%
\pgfsetlinewidth{0.000000pt}%
\definecolor{currentstroke}{rgb}{0.000000,0.000000,0.000000}%
\pgfsetstrokecolor{currentstroke}%
\pgfsetdash{}{0pt}%
\pgfpathmoveto{\pgfqpoint{4.405019in}{3.586105in}}%
\pgfpathlineto{\pgfqpoint{4.414518in}{3.617421in}}%
\pgfpathlineto{\pgfqpoint{4.423589in}{3.459887in}}%
\pgfpathlineto{\pgfqpoint{4.456884in}{3.612684in}}%
\pgfpathlineto{\pgfqpoint{4.489400in}{3.484377in}}%
\pgfpathlineto{\pgfqpoint{4.480183in}{3.579805in}}%
\pgfpathlineto{\pgfqpoint{4.470853in}{3.635122in}}%
\pgfpathlineto{\pgfqpoint{4.437651in}{3.498048in}}%
\pgfpathlineto{\pgfqpoint{4.405019in}{3.586105in}}%
\pgfpathclose%
\pgfusepath{fill}%
\end{pgfscope}%
\begin{pgfscope}%
\pgfpathrectangle{\pgfqpoint{1.020000in}{0.880000in}}{\pgfqpoint{6.160000in}{6.160000in}}%
\pgfusepath{clip}%
\pgfsetbuttcap%
\pgfsetroundjoin%
\definecolor{currentfill}{rgb}{0.280550,0.373423,0.818011}%
\pgfsetfillcolor{currentfill}%
\pgfsetlinewidth{0.000000pt}%
\definecolor{currentstroke}{rgb}{0.000000,0.000000,0.000000}%
\pgfsetstrokecolor{currentstroke}%
\pgfsetdash{}{0pt}%
\pgfpathmoveto{\pgfqpoint{5.155031in}{3.077376in}}%
\pgfpathlineto{\pgfqpoint{5.165657in}{3.119257in}}%
\pgfpathlineto{\pgfqpoint{5.175624in}{3.091953in}}%
\pgfpathlineto{\pgfqpoint{5.209029in}{3.161627in}}%
\pgfpathlineto{\pgfqpoint{5.243626in}{3.341168in}}%
\pgfpathlineto{\pgfqpoint{5.229610in}{2.989906in}}%
\pgfpathlineto{\pgfqpoint{5.220016in}{3.054531in}}%
\pgfpathlineto{\pgfqpoint{5.186425in}{2.955593in}}%
\pgfpathlineto{\pgfqpoint{5.155031in}{3.077376in}}%
\pgfpathclose%
\pgfusepath{fill}%
\end{pgfscope}%
\begin{pgfscope}%
\pgfpathrectangle{\pgfqpoint{1.020000in}{0.880000in}}{\pgfqpoint{6.160000in}{6.160000in}}%
\pgfusepath{clip}%
\pgfsetbuttcap%
\pgfsetroundjoin%
\definecolor{currentfill}{rgb}{0.947345,0.794696,0.716991}%
\pgfsetfillcolor{currentfill}%
\pgfsetlinewidth{0.000000pt}%
\definecolor{currentstroke}{rgb}{0.000000,0.000000,0.000000}%
\pgfsetstrokecolor{currentstroke}%
\pgfsetdash{}{0pt}%
\pgfpathmoveto{\pgfqpoint{3.616635in}{4.681718in}}%
\pgfpathlineto{\pgfqpoint{3.625536in}{4.631598in}}%
\pgfpathlineto{\pgfqpoint{3.635448in}{4.396954in}}%
\pgfpathlineto{\pgfqpoint{3.669103in}{4.292995in}}%
\pgfpathlineto{\pgfqpoint{3.701629in}{4.407215in}}%
\pgfpathlineto{\pgfqpoint{3.692721in}{4.453549in}}%
\pgfpathlineto{\pgfqpoint{3.684132in}{4.437337in}}%
\pgfpathlineto{\pgfqpoint{3.651595in}{4.339049in}}%
\pgfpathlineto{\pgfqpoint{3.616635in}{4.681718in}}%
\pgfpathclose%
\pgfusepath{fill}%
\end{pgfscope}%
\begin{pgfscope}%
\pgfpathrectangle{\pgfqpoint{1.020000in}{0.880000in}}{\pgfqpoint{6.160000in}{6.160000in}}%
\pgfusepath{clip}%
\pgfsetbuttcap%
\pgfsetroundjoin%
\definecolor{currentfill}{rgb}{0.399231,0.528528,0.928459}%
\pgfsetfillcolor{currentfill}%
\pgfsetlinewidth{0.000000pt}%
\definecolor{currentstroke}{rgb}{0.000000,0.000000,0.000000}%
\pgfsetstrokecolor{currentstroke}%
\pgfsetdash{}{0pt}%
\pgfpathmoveto{\pgfqpoint{4.639471in}{3.370926in}}%
\pgfpathlineto{\pgfqpoint{4.648752in}{3.286356in}}%
\pgfpathlineto{\pgfqpoint{4.658818in}{3.365705in}}%
\pgfpathlineto{\pgfqpoint{4.691395in}{3.310300in}}%
\pgfpathlineto{\pgfqpoint{4.724748in}{3.408643in}}%
\pgfpathlineto{\pgfqpoint{4.714464in}{3.310704in}}%
\pgfpathlineto{\pgfqpoint{4.704359in}{3.239553in}}%
\pgfpathlineto{\pgfqpoint{4.672111in}{3.339049in}}%
\pgfpathlineto{\pgfqpoint{4.639471in}{3.370926in}}%
\pgfpathclose%
\pgfusepath{fill}%
\end{pgfscope}%
\begin{pgfscope}%
\pgfpathrectangle{\pgfqpoint{1.020000in}{0.880000in}}{\pgfqpoint{6.160000in}{6.160000in}}%
\pgfusepath{clip}%
\pgfsetbuttcap%
\pgfsetroundjoin%
\definecolor{currentfill}{rgb}{0.902849,0.844796,0.811970}%
\pgfsetfillcolor{currentfill}%
\pgfsetlinewidth{0.000000pt}%
\definecolor{currentstroke}{rgb}{0.000000,0.000000,0.000000}%
\pgfsetstrokecolor{currentstroke}%
\pgfsetdash{}{0pt}%
\pgfpathmoveto{\pgfqpoint{3.701629in}{4.407215in}}%
\pgfpathlineto{\pgfqpoint{3.710565in}{4.357216in}}%
\pgfpathlineto{\pgfqpoint{3.719949in}{4.211675in}}%
\pgfpathlineto{\pgfqpoint{3.753067in}{4.205017in}}%
\pgfpathlineto{\pgfqpoint{3.785959in}{4.249632in}}%
\pgfpathlineto{\pgfqpoint{3.776429in}{4.438445in}}%
\pgfpathlineto{\pgfqpoint{3.768116in}{4.327763in}}%
\pgfpathlineto{\pgfqpoint{3.735371in}{4.260397in}}%
\pgfpathlineto{\pgfqpoint{3.701629in}{4.407215in}}%
\pgfpathclose%
\pgfusepath{fill}%
\end{pgfscope}%
\begin{pgfscope}%
\pgfpathrectangle{\pgfqpoint{1.020000in}{0.880000in}}{\pgfqpoint{6.160000in}{6.160000in}}%
\pgfusepath{clip}%
\pgfsetbuttcap%
\pgfsetroundjoin%
\definecolor{currentfill}{rgb}{0.880896,0.402331,0.317115}%
\pgfsetfillcolor{currentfill}%
\pgfsetlinewidth{0.000000pt}%
\definecolor{currentstroke}{rgb}{0.000000,0.000000,0.000000}%
\pgfsetstrokecolor{currentstroke}%
\pgfsetdash{}{0pt}%
\pgfpathmoveto{\pgfqpoint{3.064968in}{5.106487in}}%
\pgfpathlineto{\pgfqpoint{3.073342in}{5.075917in}}%
\pgfpathlineto{\pgfqpoint{3.080763in}{5.134300in}}%
\pgfpathlineto{\pgfqpoint{3.114756in}{5.069845in}}%
\pgfpathlineto{\pgfqpoint{3.146882in}{5.182591in}}%
\pgfpathlineto{\pgfqpoint{3.139725in}{5.088139in}}%
\pgfpathlineto{\pgfqpoint{3.132260in}{5.026506in}}%
\pgfpathlineto{\pgfqpoint{3.097970in}{5.128980in}}%
\pgfpathlineto{\pgfqpoint{3.064968in}{5.106487in}}%
\pgfpathclose%
\pgfusepath{fill}%
\end{pgfscope}%
\begin{pgfscope}%
\pgfpathrectangle{\pgfqpoint{1.020000in}{0.880000in}}{\pgfqpoint{6.160000in}{6.160000in}}%
\pgfusepath{clip}%
\pgfsetbuttcap%
\pgfsetroundjoin%
\definecolor{currentfill}{rgb}{0.592356,0.722792,0.999434}%
\pgfsetfillcolor{currentfill}%
\pgfsetlinewidth{0.000000pt}%
\definecolor{currentstroke}{rgb}{0.000000,0.000000,0.000000}%
\pgfsetstrokecolor{currentstroke}%
\pgfsetdash{}{0pt}%
\pgfpathmoveto{\pgfqpoint{4.254927in}{3.677549in}}%
\pgfpathlineto{\pgfqpoint{4.264250in}{3.744236in}}%
\pgfpathlineto{\pgfqpoint{4.273622in}{3.829255in}}%
\pgfpathlineto{\pgfqpoint{4.306431in}{3.663586in}}%
\pgfpathlineto{\pgfqpoint{4.338965in}{3.403768in}}%
\pgfpathlineto{\pgfqpoint{4.329968in}{3.628350in}}%
\pgfpathlineto{\pgfqpoint{4.320842in}{3.776251in}}%
\pgfpathlineto{\pgfqpoint{4.287837in}{3.687487in}}%
\pgfpathlineto{\pgfqpoint{4.254927in}{3.677549in}}%
\pgfpathclose%
\pgfusepath{fill}%
\end{pgfscope}%
\begin{pgfscope}%
\pgfpathrectangle{\pgfqpoint{1.020000in}{0.880000in}}{\pgfqpoint{6.160000in}{6.160000in}}%
\pgfusepath{clip}%
\pgfsetbuttcap%
\pgfsetroundjoin%
\definecolor{currentfill}{rgb}{0.646113,0.764436,0.996868}%
\pgfsetfillcolor{currentfill}%
\pgfsetlinewidth{0.000000pt}%
\definecolor{currentstroke}{rgb}{0.000000,0.000000,0.000000}%
\pgfsetstrokecolor{currentstroke}%
\pgfsetdash{}{0pt}%
\pgfpathmoveto{\pgfqpoint{4.104790in}{3.832478in}}%
\pgfpathlineto{\pgfqpoint{4.113841in}{3.928476in}}%
\pgfpathlineto{\pgfqpoint{4.123145in}{3.665195in}}%
\pgfpathlineto{\pgfqpoint{4.156099in}{3.733300in}}%
\pgfpathlineto{\pgfqpoint{4.189079in}{3.815710in}}%
\pgfpathlineto{\pgfqpoint{4.179884in}{3.684591in}}%
\pgfpathlineto{\pgfqpoint{4.170688in}{3.908625in}}%
\pgfpathlineto{\pgfqpoint{4.137862in}{3.592083in}}%
\pgfpathlineto{\pgfqpoint{4.104790in}{3.832478in}}%
\pgfpathclose%
\pgfusepath{fill}%
\end{pgfscope}%
\begin{pgfscope}%
\pgfpathrectangle{\pgfqpoint{1.020000in}{0.880000in}}{\pgfqpoint{6.160000in}{6.160000in}}%
\pgfusepath{clip}%
\pgfsetbuttcap%
\pgfsetroundjoin%
\definecolor{currentfill}{rgb}{0.399231,0.528528,0.928459}%
\pgfsetfillcolor{currentfill}%
\pgfsetlinewidth{0.000000pt}%
\definecolor{currentstroke}{rgb}{0.000000,0.000000,0.000000}%
\pgfsetstrokecolor{currentstroke}%
\pgfsetdash{}{0pt}%
\pgfpathmoveto{\pgfqpoint{5.634661in}{3.308124in}}%
\pgfpathlineto{\pgfqpoint{5.644064in}{3.218344in}}%
\pgfpathlineto{\pgfqpoint{5.656097in}{3.306630in}}%
\pgfpathlineto{\pgfqpoint{5.688923in}{3.324759in}}%
\pgfpathlineto{\pgfqpoint{5.721578in}{3.332219in}}%
\pgfpathlineto{\pgfqpoint{5.712552in}{3.448552in}}%
\pgfpathlineto{\pgfqpoint{5.700631in}{3.374372in}}%
\pgfpathlineto{\pgfqpoint{5.667045in}{3.300887in}}%
\pgfpathlineto{\pgfqpoint{5.634661in}{3.308124in}}%
\pgfpathclose%
\pgfusepath{fill}%
\end{pgfscope}%
\begin{pgfscope}%
\pgfpathrectangle{\pgfqpoint{1.020000in}{0.880000in}}{\pgfqpoint{6.160000in}{6.160000in}}%
\pgfusepath{clip}%
\pgfsetbuttcap%
\pgfsetroundjoin%
\definecolor{currentfill}{rgb}{0.851372,0.863125,0.881064}%
\pgfsetfillcolor{currentfill}%
\pgfsetlinewidth{0.000000pt}%
\definecolor{currentstroke}{rgb}{0.000000,0.000000,0.000000}%
\pgfsetstrokecolor{currentstroke}%
\pgfsetdash{}{0pt}%
\pgfpathmoveto{\pgfqpoint{3.785959in}{4.249632in}}%
\pgfpathlineto{\pgfqpoint{3.794837in}{4.226386in}}%
\pgfpathlineto{\pgfqpoint{3.804246in}{4.065749in}}%
\pgfpathlineto{\pgfqpoint{3.837332in}{4.059941in}}%
\pgfpathlineto{\pgfqpoint{3.870022in}{4.174477in}}%
\pgfpathlineto{\pgfqpoint{3.861334in}{4.119428in}}%
\pgfpathlineto{\pgfqpoint{3.852209in}{4.206081in}}%
\pgfpathlineto{\pgfqpoint{3.818873in}{4.290911in}}%
\pgfpathlineto{\pgfqpoint{3.785959in}{4.249632in}}%
\pgfpathclose%
\pgfusepath{fill}%
\end{pgfscope}%
\begin{pgfscope}%
\pgfpathrectangle{\pgfqpoint{1.020000in}{0.880000in}}{\pgfqpoint{6.160000in}{6.160000in}}%
\pgfusepath{clip}%
\pgfsetbuttcap%
\pgfsetroundjoin%
\definecolor{currentfill}{rgb}{0.266381,0.353304,0.801637}%
\pgfsetfillcolor{currentfill}%
\pgfsetlinewidth{0.000000pt}%
\definecolor{currentstroke}{rgb}{0.000000,0.000000,0.000000}%
\pgfsetstrokecolor{currentstroke}%
\pgfsetdash{}{0pt}%
\pgfpathmoveto{\pgfqpoint{5.089806in}{3.085473in}}%
\pgfpathlineto{\pgfqpoint{5.099600in}{3.046749in}}%
\pgfpathlineto{\pgfqpoint{5.108731in}{2.935165in}}%
\pgfpathlineto{\pgfqpoint{5.142533in}{3.052134in}}%
\pgfpathlineto{\pgfqpoint{5.175624in}{3.091953in}}%
\pgfpathlineto{\pgfqpoint{5.165657in}{3.119257in}}%
\pgfpathlineto{\pgfqpoint{5.155031in}{3.077376in}}%
\pgfpathlineto{\pgfqpoint{5.123207in}{3.163921in}}%
\pgfpathlineto{\pgfqpoint{5.089806in}{3.085473in}}%
\pgfpathclose%
\pgfusepath{fill}%
\end{pgfscope}%
\begin{pgfscope}%
\pgfpathrectangle{\pgfqpoint{1.020000in}{0.880000in}}{\pgfqpoint{6.160000in}{6.160000in}}%
\pgfusepath{clip}%
\pgfsetbuttcap%
\pgfsetroundjoin%
\definecolor{currentfill}{rgb}{0.368507,0.491141,0.905243}%
\pgfsetfillcolor{currentfill}%
\pgfsetlinewidth{0.000000pt}%
\definecolor{currentstroke}{rgb}{0.000000,0.000000,0.000000}%
\pgfsetstrokecolor{currentstroke}%
\pgfsetdash{}{0pt}%
\pgfpathmoveto{\pgfqpoint{5.568323in}{3.214603in}}%
\pgfpathlineto{\pgfqpoint{5.580623in}{3.332867in}}%
\pgfpathlineto{\pgfqpoint{5.591189in}{3.324721in}}%
\pgfpathlineto{\pgfqpoint{5.622198in}{3.214650in}}%
\pgfpathlineto{\pgfqpoint{5.656097in}{3.306630in}}%
\pgfpathlineto{\pgfqpoint{5.644064in}{3.218344in}}%
\pgfpathlineto{\pgfqpoint{5.634661in}{3.308124in}}%
\pgfpathlineto{\pgfqpoint{5.601439in}{3.258032in}}%
\pgfpathlineto{\pgfqpoint{5.568323in}{3.214603in}}%
\pgfpathclose%
\pgfusepath{fill}%
\end{pgfscope}%
\begin{pgfscope}%
\pgfpathrectangle{\pgfqpoint{1.020000in}{0.880000in}}{\pgfqpoint{6.160000in}{6.160000in}}%
\pgfusepath{clip}%
\pgfsetbuttcap%
\pgfsetroundjoin%
\definecolor{currentfill}{rgb}{0.905783,0.455186,0.355336}%
\pgfsetfillcolor{currentfill}%
\pgfsetlinewidth{0.000000pt}%
\definecolor{currentstroke}{rgb}{0.000000,0.000000,0.000000}%
\pgfsetstrokecolor{currentstroke}%
\pgfsetdash{}{0pt}%
\pgfpathmoveto{\pgfqpoint{3.213732in}{5.152760in}}%
\pgfpathlineto{\pgfqpoint{3.221969in}{5.149443in}}%
\pgfpathlineto{\pgfqpoint{3.230351in}{5.132404in}}%
\pgfpathlineto{\pgfqpoint{3.263941in}{5.100446in}}%
\pgfpathlineto{\pgfqpoint{3.299002in}{4.895185in}}%
\pgfpathlineto{\pgfqpoint{3.290571in}{4.914015in}}%
\pgfpathlineto{\pgfqpoint{3.282976in}{4.840303in}}%
\pgfpathlineto{\pgfqpoint{3.248008in}{5.040677in}}%
\pgfpathlineto{\pgfqpoint{3.213732in}{5.152760in}}%
\pgfpathclose%
\pgfusepath{fill}%
\end{pgfscope}%
\begin{pgfscope}%
\pgfpathrectangle{\pgfqpoint{1.020000in}{0.880000in}}{\pgfqpoint{6.160000in}{6.160000in}}%
\pgfusepath{clip}%
\pgfsetbuttcap%
\pgfsetroundjoin%
\definecolor{currentfill}{rgb}{0.521696,0.659599,0.987736}%
\pgfsetfillcolor{currentfill}%
\pgfsetlinewidth{0.000000pt}%
\definecolor{currentstroke}{rgb}{0.000000,0.000000,0.000000}%
\pgfsetstrokecolor{currentstroke}%
\pgfsetdash{}{0pt}%
\pgfpathmoveto{\pgfqpoint{4.338965in}{3.403768in}}%
\pgfpathlineto{\pgfqpoint{4.348576in}{3.563939in}}%
\pgfpathlineto{\pgfqpoint{4.358120in}{3.657244in}}%
\pgfpathlineto{\pgfqpoint{4.390558in}{3.395516in}}%
\pgfpathlineto{\pgfqpoint{4.423589in}{3.459887in}}%
\pgfpathlineto{\pgfqpoint{4.414518in}{3.617421in}}%
\pgfpathlineto{\pgfqpoint{4.405019in}{3.586105in}}%
\pgfpathlineto{\pgfqpoint{4.372371in}{3.706837in}}%
\pgfpathlineto{\pgfqpoint{4.338965in}{3.403768in}}%
\pgfpathclose%
\pgfusepath{fill}%
\end{pgfscope}%
\begin{pgfscope}%
\pgfpathrectangle{\pgfqpoint{1.020000in}{0.880000in}}{\pgfqpoint{6.160000in}{6.160000in}}%
\pgfusepath{clip}%
\pgfsetbuttcap%
\pgfsetroundjoin%
\definecolor{currentfill}{rgb}{0.294718,0.393542,0.834384}%
\pgfsetfillcolor{currentfill}%
\pgfsetlinewidth{0.000000pt}%
\definecolor{currentstroke}{rgb}{0.000000,0.000000,0.000000}%
\pgfsetstrokecolor{currentstroke}%
\pgfsetdash{}{0pt}%
\pgfpathmoveto{\pgfqpoint{4.875397in}{3.349129in}}%
\pgfpathlineto{\pgfqpoint{4.883964in}{3.152761in}}%
\pgfpathlineto{\pgfqpoint{4.893583in}{3.106698in}}%
\pgfpathlineto{\pgfqpoint{4.925929in}{3.047987in}}%
\pgfpathlineto{\pgfqpoint{4.958092in}{2.973146in}}%
\pgfpathlineto{\pgfqpoint{4.948357in}{3.006662in}}%
\pgfpathlineto{\pgfqpoint{4.939466in}{3.149644in}}%
\pgfpathlineto{\pgfqpoint{4.906919in}{3.172817in}}%
\pgfpathlineto{\pgfqpoint{4.875397in}{3.349129in}}%
\pgfpathclose%
\pgfusepath{fill}%
\end{pgfscope}%
\begin{pgfscope}%
\pgfpathrectangle{\pgfqpoint{1.020000in}{0.880000in}}{\pgfqpoint{6.160000in}{6.160000in}}%
\pgfusepath{clip}%
\pgfsetbuttcap%
\pgfsetroundjoin%
\definecolor{currentfill}{rgb}{0.713852,0.808857,0.979386}%
\pgfsetfillcolor{currentfill}%
\pgfsetlinewidth{0.000000pt}%
\definecolor{currentstroke}{rgb}{0.000000,0.000000,0.000000}%
\pgfsetstrokecolor{currentstroke}%
\pgfsetdash{}{0pt}%
\pgfpathmoveto{\pgfqpoint{3.954441in}{3.998749in}}%
\pgfpathlineto{\pgfqpoint{3.963478in}{3.966939in}}%
\pgfpathlineto{\pgfqpoint{3.972833in}{3.789685in}}%
\pgfpathlineto{\pgfqpoint{4.005683in}{3.889451in}}%
\pgfpathlineto{\pgfqpoint{4.038923in}{3.750601in}}%
\pgfpathlineto{\pgfqpoint{4.029833in}{3.791273in}}%
\pgfpathlineto{\pgfqpoint{4.020503in}{3.987034in}}%
\pgfpathlineto{\pgfqpoint{3.987543in}{3.962190in}}%
\pgfpathlineto{\pgfqpoint{3.954441in}{3.998749in}}%
\pgfpathclose%
\pgfusepath{fill}%
\end{pgfscope}%
\begin{pgfscope}%
\pgfpathrectangle{\pgfqpoint{1.020000in}{0.880000in}}{\pgfqpoint{6.160000in}{6.160000in}}%
\pgfusepath{clip}%
\pgfsetbuttcap%
\pgfsetroundjoin%
\definecolor{currentfill}{rgb}{0.343278,0.459354,0.884122}%
\pgfsetfillcolor{currentfill}%
\pgfsetlinewidth{0.000000pt}%
\definecolor{currentstroke}{rgb}{0.000000,0.000000,0.000000}%
\pgfsetstrokecolor{currentstroke}%
\pgfsetdash{}{0pt}%
\pgfpathmoveto{\pgfqpoint{4.724748in}{3.408643in}}%
\pgfpathlineto{\pgfqpoint{4.732850in}{3.103606in}}%
\pgfpathlineto{\pgfqpoint{4.744177in}{3.380181in}}%
\pgfpathlineto{\pgfqpoint{4.775772in}{3.162557in}}%
\pgfpathlineto{\pgfqpoint{4.808450in}{3.143555in}}%
\pgfpathlineto{\pgfqpoint{4.799088in}{3.221300in}}%
\pgfpathlineto{\pgfqpoint{4.788768in}{3.140530in}}%
\pgfpathlineto{\pgfqpoint{4.756469in}{3.214978in}}%
\pgfpathlineto{\pgfqpoint{4.724748in}{3.408643in}}%
\pgfpathclose%
\pgfusepath{fill}%
\end{pgfscope}%
\begin{pgfscope}%
\pgfpathrectangle{\pgfqpoint{1.020000in}{0.880000in}}{\pgfqpoint{6.160000in}{6.160000in}}%
\pgfusepath{clip}%
\pgfsetbuttcap%
\pgfsetroundjoin%
\definecolor{currentfill}{rgb}{0.843703,0.330068,0.270065}%
\pgfsetfillcolor{currentfill}%
\pgfsetlinewidth{0.000000pt}%
\definecolor{currentstroke}{rgb}{0.000000,0.000000,0.000000}%
\pgfsetstrokecolor{currentstroke}%
\pgfsetdash{}{0pt}%
\pgfpathmoveto{\pgfqpoint{2.848402in}{5.187704in}}%
\pgfpathlineto{\pgfqpoint{2.854838in}{5.293945in}}%
\pgfpathlineto{\pgfqpoint{2.864643in}{5.142830in}}%
\pgfpathlineto{\pgfqpoint{2.896996in}{5.223183in}}%
\pgfpathlineto{\pgfqpoint{2.932040in}{5.087707in}}%
\pgfpathlineto{\pgfqpoint{2.921885in}{5.270247in}}%
\pgfpathlineto{\pgfqpoint{2.916373in}{5.079277in}}%
\pgfpathlineto{\pgfqpoint{2.882358in}{5.137961in}}%
\pgfpathlineto{\pgfqpoint{2.848402in}{5.187704in}}%
\pgfpathclose%
\pgfusepath{fill}%
\end{pgfscope}%
\begin{pgfscope}%
\pgfpathrectangle{\pgfqpoint{1.020000in}{0.880000in}}{\pgfqpoint{6.160000in}{6.160000in}}%
\pgfusepath{clip}%
\pgfsetbuttcap%
\pgfsetroundjoin%
\definecolor{currentfill}{rgb}{0.796064,0.848693,0.933471}%
\pgfsetfillcolor{currentfill}%
\pgfsetlinewidth{0.000000pt}%
\definecolor{currentstroke}{rgb}{0.000000,0.000000,0.000000}%
\pgfsetstrokecolor{currentstroke}%
\pgfsetdash{}{0pt}%
\pgfpathmoveto{\pgfqpoint{3.870022in}{4.174477in}}%
\pgfpathlineto{\pgfqpoint{3.878874in}{4.184544in}}%
\pgfpathlineto{\pgfqpoint{3.887888in}{4.146008in}}%
\pgfpathlineto{\pgfqpoint{3.921362in}{4.009416in}}%
\pgfpathlineto{\pgfqpoint{3.954441in}{3.998749in}}%
\pgfpathlineto{\pgfqpoint{3.945662in}{3.926611in}}%
\pgfpathlineto{\pgfqpoint{3.936530in}{4.011507in}}%
\pgfpathlineto{\pgfqpoint{3.903668in}{3.964991in}}%
\pgfpathlineto{\pgfqpoint{3.870022in}{4.174477in}}%
\pgfpathclose%
\pgfusepath{fill}%
\end{pgfscope}%
\begin{pgfscope}%
\pgfpathrectangle{\pgfqpoint{1.020000in}{0.880000in}}{\pgfqpoint{6.160000in}{6.160000in}}%
\pgfusepath{clip}%
\pgfsetbuttcap%
\pgfsetroundjoin%
\definecolor{currentfill}{rgb}{0.229806,0.298718,0.753683}%
\pgfsetfillcolor{currentfill}%
\pgfsetlinewidth{0.000000pt}%
\definecolor{currentstroke}{rgb}{0.000000,0.000000,0.000000}%
\pgfsetstrokecolor{currentstroke}%
\pgfsetdash{}{0pt}%
\pgfpathmoveto{\pgfqpoint{4.958092in}{2.973146in}}%
\pgfpathlineto{\pgfqpoint{4.967369in}{2.879743in}}%
\pgfpathlineto{\pgfqpoint{4.978544in}{3.023659in}}%
\pgfpathlineto{\pgfqpoint{5.010045in}{2.869628in}}%
\pgfpathlineto{\pgfqpoint{5.044415in}{3.061577in}}%
\pgfpathlineto{\pgfqpoint{5.033859in}{3.008917in}}%
\pgfpathlineto{\pgfqpoint{5.024332in}{3.074518in}}%
\pgfpathlineto{\pgfqpoint{4.990845in}{2.979425in}}%
\pgfpathlineto{\pgfqpoint{4.958092in}{2.973146in}}%
\pgfpathclose%
\pgfusepath{fill}%
\end{pgfscope}%
\begin{pgfscope}%
\pgfpathrectangle{\pgfqpoint{1.020000in}{0.880000in}}{\pgfqpoint{6.160000in}{6.160000in}}%
\pgfusepath{clip}%
\pgfsetbuttcap%
\pgfsetroundjoin%
\definecolor{currentfill}{rgb}{0.358415,0.478426,0.896795}%
\pgfsetfillcolor{currentfill}%
\pgfsetlinewidth{0.000000pt}%
\definecolor{currentstroke}{rgb}{0.000000,0.000000,0.000000}%
\pgfsetstrokecolor{currentstroke}%
\pgfsetdash{}{0pt}%
\pgfpathmoveto{\pgfqpoint{5.505240in}{3.363702in}}%
\pgfpathlineto{\pgfqpoint{5.512518in}{3.114636in}}%
\pgfpathlineto{\pgfqpoint{5.525557in}{3.296440in}}%
\pgfpathlineto{\pgfqpoint{5.556609in}{3.182127in}}%
\pgfpathlineto{\pgfqpoint{5.591189in}{3.324721in}}%
\pgfpathlineto{\pgfqpoint{5.580623in}{3.332867in}}%
\pgfpathlineto{\pgfqpoint{5.568323in}{3.214603in}}%
\pgfpathlineto{\pgfqpoint{5.534379in}{3.109375in}}%
\pgfpathlineto{\pgfqpoint{5.505240in}{3.363702in}}%
\pgfpathclose%
\pgfusepath{fill}%
\end{pgfscope}%
\begin{pgfscope}%
\pgfpathrectangle{\pgfqpoint{1.020000in}{0.880000in}}{\pgfqpoint{6.160000in}{6.160000in}}%
\pgfusepath{clip}%
\pgfsetbuttcap%
\pgfsetroundjoin%
\definecolor{currentfill}{rgb}{0.941728,0.546413,0.429707}%
\pgfsetfillcolor{currentfill}%
\pgfsetlinewidth{0.000000pt}%
\definecolor{currentstroke}{rgb}{0.000000,0.000000,0.000000}%
\pgfsetstrokecolor{currentstroke}%
\pgfsetdash{}{0pt}%
\pgfpathmoveto{\pgfqpoint{3.299002in}{4.895185in}}%
\pgfpathlineto{\pgfqpoint{3.306742in}{4.958083in}}%
\pgfpathlineto{\pgfqpoint{3.315043in}{4.958510in}}%
\pgfpathlineto{\pgfqpoint{3.348308in}{4.959146in}}%
\pgfpathlineto{\pgfqpoint{3.382832in}{4.797951in}}%
\pgfpathlineto{\pgfqpoint{3.373913in}{4.868656in}}%
\pgfpathlineto{\pgfqpoint{3.366045in}{4.809559in}}%
\pgfpathlineto{\pgfqpoint{3.331525in}{4.974280in}}%
\pgfpathlineto{\pgfqpoint{3.299002in}{4.895185in}}%
\pgfpathclose%
\pgfusepath{fill}%
\end{pgfscope}%
\begin{pgfscope}%
\pgfpathrectangle{\pgfqpoint{1.020000in}{0.880000in}}{\pgfqpoint{6.160000in}{6.160000in}}%
\pgfusepath{clip}%
\pgfsetbuttcap%
\pgfsetroundjoin%
\definecolor{currentfill}{rgb}{0.261805,0.346484,0.795658}%
\pgfsetfillcolor{currentfill}%
\pgfsetlinewidth{0.000000pt}%
\definecolor{currentstroke}{rgb}{0.000000,0.000000,0.000000}%
\pgfsetstrokecolor{currentstroke}%
\pgfsetdash{}{0pt}%
\pgfpathmoveto{\pgfqpoint{5.024332in}{3.074518in}}%
\pgfpathlineto{\pgfqpoint{5.033859in}{3.008917in}}%
\pgfpathlineto{\pgfqpoint{5.044415in}{3.061577in}}%
\pgfpathlineto{\pgfqpoint{5.077212in}{3.066167in}}%
\pgfpathlineto{\pgfqpoint{5.108731in}{2.935165in}}%
\pgfpathlineto{\pgfqpoint{5.099600in}{3.046749in}}%
\pgfpathlineto{\pgfqpoint{5.089806in}{3.085473in}}%
\pgfpathlineto{\pgfqpoint{5.057537in}{3.132514in}}%
\pgfpathlineto{\pgfqpoint{5.024332in}{3.074518in}}%
\pgfpathclose%
\pgfusepath{fill}%
\end{pgfscope}%
\begin{pgfscope}%
\pgfpathrectangle{\pgfqpoint{1.020000in}{0.880000in}}{\pgfqpoint{6.160000in}{6.160000in}}%
\pgfusepath{clip}%
\pgfsetbuttcap%
\pgfsetroundjoin%
\definecolor{currentfill}{rgb}{0.494638,0.633022,0.978983}%
\pgfsetfillcolor{currentfill}%
\pgfsetlinewidth{0.000000pt}%
\definecolor{currentstroke}{rgb}{0.000000,0.000000,0.000000}%
\pgfsetstrokecolor{currentstroke}%
\pgfsetdash{}{0pt}%
\pgfpathmoveto{\pgfqpoint{4.489400in}{3.484377in}}%
\pgfpathlineto{\pgfqpoint{4.499031in}{3.519510in}}%
\pgfpathlineto{\pgfqpoint{4.508574in}{3.519347in}}%
\pgfpathlineto{\pgfqpoint{4.541454in}{3.507502in}}%
\pgfpathlineto{\pgfqpoint{4.574519in}{3.549904in}}%
\pgfpathlineto{\pgfqpoint{4.564695in}{3.493663in}}%
\pgfpathlineto{\pgfqpoint{4.554540in}{3.340094in}}%
\pgfpathlineto{\pgfqpoint{4.522322in}{3.506132in}}%
\pgfpathlineto{\pgfqpoint{4.489400in}{3.484377in}}%
\pgfpathclose%
\pgfusepath{fill}%
\end{pgfscope}%
\begin{pgfscope}%
\pgfpathrectangle{\pgfqpoint{1.020000in}{0.880000in}}{\pgfqpoint{6.160000in}{6.160000in}}%
\pgfusepath{clip}%
\pgfsetbuttcap%
\pgfsetroundjoin%
\definecolor{currentfill}{rgb}{0.884643,0.410017,0.322507}%
\pgfsetfillcolor{currentfill}%
\pgfsetlinewidth{0.000000pt}%
\definecolor{currentstroke}{rgb}{0.000000,0.000000,0.000000}%
\pgfsetstrokecolor{currentstroke}%
\pgfsetdash{}{0pt}%
\pgfpathmoveto{\pgfqpoint{2.719157in}{4.895019in}}%
\pgfpathlineto{\pgfqpoint{2.722945in}{5.165470in}}%
\pgfpathlineto{\pgfqpoint{2.732007in}{5.070528in}}%
\pgfpathlineto{\pgfqpoint{2.767186in}{4.944595in}}%
\pgfpathlineto{\pgfqpoint{2.797771in}{5.148431in}}%
\pgfpathlineto{\pgfqpoint{2.788685in}{5.244498in}}%
\pgfpathlineto{\pgfqpoint{2.781798in}{5.181050in}}%
\pgfpathlineto{\pgfqpoint{2.750378in}{5.042426in}}%
\pgfpathlineto{\pgfqpoint{2.719157in}{4.895019in}}%
\pgfpathclose%
\pgfusepath{fill}%
\end{pgfscope}%
\begin{pgfscope}%
\pgfpathrectangle{\pgfqpoint{1.020000in}{0.880000in}}{\pgfqpoint{6.160000in}{6.160000in}}%
\pgfusepath{clip}%
\pgfsetbuttcap%
\pgfsetroundjoin%
\definecolor{currentfill}{rgb}{0.338377,0.452819,0.879317}%
\pgfsetfillcolor{currentfill}%
\pgfsetlinewidth{0.000000pt}%
\definecolor{currentstroke}{rgb}{0.000000,0.000000,0.000000}%
\pgfsetstrokecolor{currentstroke}%
\pgfsetdash{}{0pt}%
\pgfpathmoveto{\pgfqpoint{5.438113in}{3.212767in}}%
\pgfpathlineto{\pgfqpoint{5.447113in}{3.092780in}}%
\pgfpathlineto{\pgfqpoint{5.459040in}{3.202061in}}%
\pgfpathlineto{\pgfqpoint{5.492275in}{3.247881in}}%
\pgfpathlineto{\pgfqpoint{5.525557in}{3.296440in}}%
\pgfpathlineto{\pgfqpoint{5.512518in}{3.114636in}}%
\pgfpathlineto{\pgfqpoint{5.505240in}{3.363702in}}%
\pgfpathlineto{\pgfqpoint{5.469875in}{3.149735in}}%
\pgfpathlineto{\pgfqpoint{5.438113in}{3.212767in}}%
\pgfpathclose%
\pgfusepath{fill}%
\end{pgfscope}%
\begin{pgfscope}%
\pgfpathrectangle{\pgfqpoint{1.020000in}{0.880000in}}{\pgfqpoint{6.160000in}{6.160000in}}%
\pgfusepath{clip}%
\pgfsetbuttcap%
\pgfsetroundjoin%
\definecolor{currentfill}{rgb}{0.916071,0.833977,0.788693}%
\pgfsetfillcolor{currentfill}%
\pgfsetlinewidth{0.000000pt}%
\definecolor{currentstroke}{rgb}{0.000000,0.000000,0.000000}%
\pgfsetstrokecolor{currentstroke}%
\pgfsetdash{}{0pt}%
\pgfpathmoveto{\pgfqpoint{3.635448in}{4.396954in}}%
\pgfpathlineto{\pgfqpoint{3.643922in}{4.427546in}}%
\pgfpathlineto{\pgfqpoint{3.653202in}{4.309049in}}%
\pgfpathlineto{\pgfqpoint{3.686582in}{4.264306in}}%
\pgfpathlineto{\pgfqpoint{3.719949in}{4.211675in}}%
\pgfpathlineto{\pgfqpoint{3.710565in}{4.357216in}}%
\pgfpathlineto{\pgfqpoint{3.701629in}{4.407215in}}%
\pgfpathlineto{\pgfqpoint{3.669103in}{4.292995in}}%
\pgfpathlineto{\pgfqpoint{3.635448in}{4.396954in}}%
\pgfpathclose%
\pgfusepath{fill}%
\end{pgfscope}%
\begin{pgfscope}%
\pgfpathrectangle{\pgfqpoint{1.020000in}{0.880000in}}{\pgfqpoint{6.160000in}{6.160000in}}%
\pgfusepath{clip}%
\pgfsetbuttcap%
\pgfsetroundjoin%
\definecolor{currentfill}{rgb}{0.661968,0.775491,0.993937}%
\pgfsetfillcolor{currentfill}%
\pgfsetlinewidth{0.000000pt}%
\definecolor{currentstroke}{rgb}{0.000000,0.000000,0.000000}%
\pgfsetstrokecolor{currentstroke}%
\pgfsetdash{}{0pt}%
\pgfpathmoveto{\pgfqpoint{4.038923in}{3.750601in}}%
\pgfpathlineto{\pgfqpoint{4.047936in}{3.775211in}}%
\pgfpathlineto{\pgfqpoint{4.056987in}{3.788991in}}%
\pgfpathlineto{\pgfqpoint{4.090021in}{3.807186in}}%
\pgfpathlineto{\pgfqpoint{4.123145in}{3.665195in}}%
\pgfpathlineto{\pgfqpoint{4.113841in}{3.928476in}}%
\pgfpathlineto{\pgfqpoint{4.104790in}{3.832478in}}%
\pgfpathlineto{\pgfqpoint{4.071884in}{3.763105in}}%
\pgfpathlineto{\pgfqpoint{4.038923in}{3.750601in}}%
\pgfpathclose%
\pgfusepath{fill}%
\end{pgfscope}%
\begin{pgfscope}%
\pgfpathrectangle{\pgfqpoint{1.020000in}{0.880000in}}{\pgfqpoint{6.160000in}{6.160000in}}%
\pgfusepath{clip}%
\pgfsetbuttcap%
\pgfsetroundjoin%
\definecolor{currentfill}{rgb}{0.964911,0.640159,0.519806}%
\pgfsetfillcolor{currentfill}%
\pgfsetlinewidth{0.000000pt}%
\definecolor{currentstroke}{rgb}{0.000000,0.000000,0.000000}%
\pgfsetstrokecolor{currentstroke}%
\pgfsetdash{}{0pt}%
\pgfpathmoveto{\pgfqpoint{2.321982in}{4.681504in}}%
\pgfpathlineto{\pgfqpoint{2.326874in}{4.811179in}}%
\pgfpathlineto{\pgfqpoint{2.336875in}{4.662581in}}%
\pgfpathlineto{\pgfqpoint{2.369304in}{4.723436in}}%
\pgfpathlineto{\pgfqpoint{2.402393in}{4.747987in}}%
\pgfpathlineto{\pgfqpoint{2.394433in}{4.782471in}}%
\pgfpathlineto{\pgfqpoint{2.386280in}{4.828437in}}%
\pgfpathlineto{\pgfqpoint{2.353655in}{4.780537in}}%
\pgfpathlineto{\pgfqpoint{2.321982in}{4.681504in}}%
\pgfpathclose%
\pgfusepath{fill}%
\end{pgfscope}%
\begin{pgfscope}%
\pgfpathrectangle{\pgfqpoint{1.020000in}{0.880000in}}{\pgfqpoint{6.160000in}{6.160000in}}%
\pgfusepath{clip}%
\pgfsetbuttcap%
\pgfsetroundjoin%
\definecolor{currentfill}{rgb}{0.865391,0.371128,0.295769}%
\pgfsetfillcolor{currentfill}%
\pgfsetlinewidth{0.000000pt}%
\definecolor{currentstroke}{rgb}{0.000000,0.000000,0.000000}%
\pgfsetstrokecolor{currentstroke}%
\pgfsetdash{}{0pt}%
\pgfpathmoveto{\pgfqpoint{2.998025in}{5.140944in}}%
\pgfpathlineto{\pgfqpoint{3.005530in}{5.181854in}}%
\pgfpathlineto{\pgfqpoint{3.013789in}{5.159179in}}%
\pgfpathlineto{\pgfqpoint{3.046959in}{5.176346in}}%
\pgfpathlineto{\pgfqpoint{3.080763in}{5.134300in}}%
\pgfpathlineto{\pgfqpoint{3.073342in}{5.075917in}}%
\pgfpathlineto{\pgfqpoint{3.064968in}{5.106487in}}%
\pgfpathlineto{\pgfqpoint{3.031955in}{5.084581in}}%
\pgfpathlineto{\pgfqpoint{2.998025in}{5.140944in}}%
\pgfpathclose%
\pgfusepath{fill}%
\end{pgfscope}%
\begin{pgfscope}%
\pgfpathrectangle{\pgfqpoint{1.020000in}{0.880000in}}{\pgfqpoint{6.160000in}{6.160000in}}%
\pgfusepath{clip}%
\pgfsetbuttcap%
\pgfsetroundjoin%
\definecolor{currentfill}{rgb}{0.960490,0.616276,0.495467}%
\pgfsetfillcolor{currentfill}%
\pgfsetlinewidth{0.000000pt}%
\definecolor{currentstroke}{rgb}{0.000000,0.000000,0.000000}%
\pgfsetstrokecolor{currentstroke}%
\pgfsetdash{}{0pt}%
\pgfpathmoveto{\pgfqpoint{3.382832in}{4.797951in}}%
\pgfpathlineto{\pgfqpoint{3.390025in}{4.948901in}}%
\pgfpathlineto{\pgfqpoint{3.398579in}{4.929369in}}%
\pgfpathlineto{\pgfqpoint{3.433633in}{4.690685in}}%
\pgfpathlineto{\pgfqpoint{3.466448in}{4.743477in}}%
\pgfpathlineto{\pgfqpoint{3.457414in}{4.825149in}}%
\pgfpathlineto{\pgfqpoint{3.449524in}{4.749314in}}%
\pgfpathlineto{\pgfqpoint{3.416893in}{4.681483in}}%
\pgfpathlineto{\pgfqpoint{3.382832in}{4.797951in}}%
\pgfpathclose%
\pgfusepath{fill}%
\end{pgfscope}%
\begin{pgfscope}%
\pgfpathrectangle{\pgfqpoint{1.020000in}{0.880000in}}{\pgfqpoint{6.160000in}{6.160000in}}%
\pgfusepath{clip}%
\pgfsetbuttcap%
\pgfsetroundjoin%
\definecolor{currentfill}{rgb}{0.333490,0.446265,0.874452}%
\pgfsetfillcolor{currentfill}%
\pgfsetlinewidth{0.000000pt}%
\definecolor{currentstroke}{rgb}{0.000000,0.000000,0.000000}%
\pgfsetstrokecolor{currentstroke}%
\pgfsetdash{}{0pt}%
\pgfpathmoveto{\pgfqpoint{5.372427in}{3.172563in}}%
\pgfpathlineto{\pgfqpoint{5.384267in}{3.287763in}}%
\pgfpathlineto{\pgfqpoint{5.393693in}{3.200233in}}%
\pgfpathlineto{\pgfqpoint{5.426213in}{3.188089in}}%
\pgfpathlineto{\pgfqpoint{5.459040in}{3.202061in}}%
\pgfpathlineto{\pgfqpoint{5.447113in}{3.092780in}}%
\pgfpathlineto{\pgfqpoint{5.438113in}{3.212767in}}%
\pgfpathlineto{\pgfqpoint{5.404850in}{3.158136in}}%
\pgfpathlineto{\pgfqpoint{5.372427in}{3.172563in}}%
\pgfpathclose%
\pgfusepath{fill}%
\end{pgfscope}%
\begin{pgfscope}%
\pgfpathrectangle{\pgfqpoint{1.020000in}{0.880000in}}{\pgfqpoint{6.160000in}{6.160000in}}%
\pgfusepath{clip}%
\pgfsetbuttcap%
\pgfsetroundjoin%
\definecolor{currentfill}{rgb}{0.378598,0.503856,0.913692}%
\pgfsetfillcolor{currentfill}%
\pgfsetlinewidth{0.000000pt}%
\definecolor{currentstroke}{rgb}{0.000000,0.000000,0.000000}%
\pgfsetstrokecolor{currentstroke}%
\pgfsetdash{}{0pt}%
\pgfpathmoveto{\pgfqpoint{5.721578in}{3.332219in}}%
\pgfpathlineto{\pgfqpoint{5.731395in}{3.267101in}}%
\pgfpathlineto{\pgfqpoint{5.740377in}{3.147878in}}%
\pgfpathlineto{\pgfqpoint{5.774802in}{3.267060in}}%
\pgfpathlineto{\pgfqpoint{5.806304in}{3.202944in}}%
\pgfpathlineto{\pgfqpoint{5.795155in}{3.187451in}}%
\pgfpathlineto{\pgfqpoint{5.788004in}{3.419370in}}%
\pgfpathlineto{\pgfqpoint{5.754655in}{3.367438in}}%
\pgfpathlineto{\pgfqpoint{5.721578in}{3.332219in}}%
\pgfpathclose%
\pgfusepath{fill}%
\end{pgfscope}%
\begin{pgfscope}%
\pgfpathrectangle{\pgfqpoint{1.020000in}{0.880000in}}{\pgfqpoint{6.160000in}{6.160000in}}%
\pgfusepath{clip}%
\pgfsetbuttcap%
\pgfsetroundjoin%
\definecolor{currentfill}{rgb}{0.399231,0.528528,0.928459}%
\pgfsetfillcolor{currentfill}%
\pgfsetlinewidth{0.000000pt}%
\definecolor{currentstroke}{rgb}{0.000000,0.000000,0.000000}%
\pgfsetstrokecolor{currentstroke}%
\pgfsetdash{}{0pt}%
\pgfpathmoveto{\pgfqpoint{5.788004in}{3.419370in}}%
\pgfpathlineto{\pgfqpoint{5.795155in}{3.187451in}}%
\pgfpathlineto{\pgfqpoint{5.806304in}{3.202944in}}%
\pgfpathlineto{\pgfqpoint{5.841940in}{3.391653in}}%
\pgfpathlineto{\pgfqpoint{5.829386in}{3.294319in}}%
\pgfpathlineto{\pgfqpoint{5.820397in}{3.412035in}}%
\pgfpathlineto{\pgfqpoint{5.788004in}{3.419370in}}%
\pgfpathclose%
\pgfusepath{fill}%
\end{pgfscope}%
\begin{pgfscope}%
\pgfpathrectangle{\pgfqpoint{1.020000in}{0.880000in}}{\pgfqpoint{6.160000in}{6.160000in}}%
\pgfusepath{clip}%
\pgfsetbuttcap%
\pgfsetroundjoin%
\definecolor{currentfill}{rgb}{0.967317,0.657471,0.538160}%
\pgfsetfillcolor{currentfill}%
\pgfsetlinewidth{0.000000pt}%
\definecolor{currentstroke}{rgb}{0.000000,0.000000,0.000000}%
\pgfsetstrokecolor{currentstroke}%
\pgfsetdash{}{0pt}%
\pgfpathmoveto{\pgfqpoint{2.254793in}{4.692560in}}%
\pgfpathlineto{\pgfqpoint{2.261425in}{4.722557in}}%
\pgfpathlineto{\pgfqpoint{2.267641in}{4.775598in}}%
\pgfpathlineto{\pgfqpoint{2.302772in}{4.692941in}}%
\pgfpathlineto{\pgfqpoint{2.336875in}{4.662581in}}%
\pgfpathlineto{\pgfqpoint{2.326874in}{4.811179in}}%
\pgfpathlineto{\pgfqpoint{2.321982in}{4.681504in}}%
\pgfpathlineto{\pgfqpoint{2.287161in}{4.753177in}}%
\pgfpathlineto{\pgfqpoint{2.254793in}{4.692560in}}%
\pgfpathclose%
\pgfusepath{fill}%
\end{pgfscope}%
\begin{pgfscope}%
\pgfpathrectangle{\pgfqpoint{1.020000in}{0.880000in}}{\pgfqpoint{6.160000in}{6.160000in}}%
\pgfusepath{clip}%
\pgfsetbuttcap%
\pgfsetroundjoin%
\definecolor{currentfill}{rgb}{0.309060,0.413498,0.850128}%
\pgfsetfillcolor{currentfill}%
\pgfsetlinewidth{0.000000pt}%
\definecolor{currentstroke}{rgb}{0.000000,0.000000,0.000000}%
\pgfsetstrokecolor{currentstroke}%
\pgfsetdash{}{0pt}%
\pgfpathmoveto{\pgfqpoint{4.808450in}{3.143555in}}%
\pgfpathlineto{\pgfqpoint{4.817840in}{3.069378in}}%
\pgfpathlineto{\pgfqpoint{4.828278in}{3.156381in}}%
\pgfpathlineto{\pgfqpoint{4.860576in}{3.076768in}}%
\pgfpathlineto{\pgfqpoint{4.893583in}{3.106698in}}%
\pgfpathlineto{\pgfqpoint{4.883964in}{3.152761in}}%
\pgfpathlineto{\pgfqpoint{4.875397in}{3.349129in}}%
\pgfpathlineto{\pgfqpoint{4.841013in}{3.112067in}}%
\pgfpathlineto{\pgfqpoint{4.808450in}{3.143555in}}%
\pgfpathclose%
\pgfusepath{fill}%
\end{pgfscope}%
\begin{pgfscope}%
\pgfpathrectangle{\pgfqpoint{1.020000in}{0.880000in}}{\pgfqpoint{6.160000in}{6.160000in}}%
\pgfusepath{clip}%
\pgfsetbuttcap%
\pgfsetroundjoin%
\definecolor{currentfill}{rgb}{0.835027,0.313644,0.259783}%
\pgfsetfillcolor{currentfill}%
\pgfsetlinewidth{0.000000pt}%
\definecolor{currentstroke}{rgb}{0.000000,0.000000,0.000000}%
\pgfsetstrokecolor{currentstroke}%
\pgfsetdash{}{0pt}%
\pgfpathmoveto{\pgfqpoint{2.781798in}{5.181050in}}%
\pgfpathlineto{\pgfqpoint{2.788685in}{5.244498in}}%
\pgfpathlineto{\pgfqpoint{2.797771in}{5.148431in}}%
\pgfpathlineto{\pgfqpoint{2.830543in}{5.196350in}}%
\pgfpathlineto{\pgfqpoint{2.864643in}{5.142830in}}%
\pgfpathlineto{\pgfqpoint{2.854838in}{5.293945in}}%
\pgfpathlineto{\pgfqpoint{2.848402in}{5.187704in}}%
\pgfpathlineto{\pgfqpoint{2.815572in}{5.149935in}}%
\pgfpathlineto{\pgfqpoint{2.781798in}{5.181050in}}%
\pgfpathclose%
\pgfusepath{fill}%
\end{pgfscope}%
\begin{pgfscope}%
\pgfpathrectangle{\pgfqpoint{1.020000in}{0.880000in}}{\pgfqpoint{6.160000in}{6.160000in}}%
\pgfusepath{clip}%
\pgfsetbuttcap%
\pgfsetroundjoin%
\definecolor{currentfill}{rgb}{0.953054,0.585211,0.465373}%
\pgfsetfillcolor{currentfill}%
\pgfsetlinewidth{0.000000pt}%
\definecolor{currentstroke}{rgb}{0.000000,0.000000,0.000000}%
\pgfsetstrokecolor{currentstroke}%
\pgfsetdash{}{0pt}%
\pgfpathmoveto{\pgfqpoint{2.386280in}{4.828437in}}%
\pgfpathlineto{\pgfqpoint{2.394433in}{4.782471in}}%
\pgfpathlineto{\pgfqpoint{2.402393in}{4.747987in}}%
\pgfpathlineto{\pgfqpoint{2.433848in}{4.867056in}}%
\pgfpathlineto{\pgfqpoint{2.465061in}{5.004014in}}%
\pgfpathlineto{\pgfqpoint{2.461142in}{4.797007in}}%
\pgfpathlineto{\pgfqpoint{2.452205in}{4.886877in}}%
\pgfpathlineto{\pgfqpoint{2.419907in}{4.819454in}}%
\pgfpathlineto{\pgfqpoint{2.386280in}{4.828437in}}%
\pgfpathclose%
\pgfusepath{fill}%
\end{pgfscope}%
\begin{pgfscope}%
\pgfpathrectangle{\pgfqpoint{1.020000in}{0.880000in}}{\pgfqpoint{6.160000in}{6.160000in}}%
\pgfusepath{clip}%
\pgfsetbuttcap%
\pgfsetroundjoin%
\definecolor{currentfill}{rgb}{0.939254,0.539581,0.423900}%
\pgfsetfillcolor{currentfill}%
\pgfsetlinewidth{0.000000pt}%
\definecolor{currentstroke}{rgb}{0.000000,0.000000,0.000000}%
\pgfsetstrokecolor{currentstroke}%
\pgfsetdash{}{0pt}%
\pgfpathmoveto{\pgfqpoint{2.452205in}{4.886877in}}%
\pgfpathlineto{\pgfqpoint{2.461142in}{4.797007in}}%
\pgfpathlineto{\pgfqpoint{2.465061in}{5.004014in}}%
\pgfpathlineto{\pgfqpoint{2.500940in}{4.863837in}}%
\pgfpathlineto{\pgfqpoint{2.535006in}{4.828370in}}%
\pgfpathlineto{\pgfqpoint{2.525540in}{4.949554in}}%
\pgfpathlineto{\pgfqpoint{2.518750in}{4.907459in}}%
\pgfpathlineto{\pgfqpoint{2.484113in}{4.979050in}}%
\pgfpathlineto{\pgfqpoint{2.452205in}{4.886877in}}%
\pgfpathclose%
\pgfusepath{fill}%
\end{pgfscope}%
\begin{pgfscope}%
\pgfpathrectangle{\pgfqpoint{1.020000in}{0.880000in}}{\pgfqpoint{6.160000in}{6.160000in}}%
\pgfusepath{clip}%
\pgfsetbuttcap%
\pgfsetroundjoin%
\definecolor{currentfill}{rgb}{0.875557,0.860242,0.851430}%
\pgfsetfillcolor{currentfill}%
\pgfsetlinewidth{0.000000pt}%
\definecolor{currentstroke}{rgb}{0.000000,0.000000,0.000000}%
\pgfsetstrokecolor{currentstroke}%
\pgfsetdash{}{0pt}%
\pgfpathmoveto{\pgfqpoint{3.719949in}{4.211675in}}%
\pgfpathlineto{\pgfqpoint{3.727647in}{4.438511in}}%
\pgfpathlineto{\pgfqpoint{3.737693in}{4.147627in}}%
\pgfpathlineto{\pgfqpoint{3.770321in}{4.270633in}}%
\pgfpathlineto{\pgfqpoint{3.804246in}{4.065749in}}%
\pgfpathlineto{\pgfqpoint{3.794837in}{4.226386in}}%
\pgfpathlineto{\pgfqpoint{3.785959in}{4.249632in}}%
\pgfpathlineto{\pgfqpoint{3.753067in}{4.205017in}}%
\pgfpathlineto{\pgfqpoint{3.719949in}{4.211675in}}%
\pgfpathclose%
\pgfusepath{fill}%
\end{pgfscope}%
\begin{pgfscope}%
\pgfpathrectangle{\pgfqpoint{1.020000in}{0.880000in}}{\pgfqpoint{6.160000in}{6.160000in}}%
\pgfusepath{clip}%
\pgfsetbuttcap%
\pgfsetroundjoin%
\definecolor{currentfill}{rgb}{0.969289,0.684982,0.568975}%
\pgfsetfillcolor{currentfill}%
\pgfsetlinewidth{0.000000pt}%
\definecolor{currentstroke}{rgb}{0.000000,0.000000,0.000000}%
\pgfsetstrokecolor{currentstroke}%
\pgfsetdash{}{0pt}%
\pgfpathmoveto{\pgfqpoint{2.188137in}{4.671878in}}%
\pgfpathlineto{\pgfqpoint{2.198867in}{4.488774in}}%
\pgfpathlineto{\pgfqpoint{2.203462in}{4.618649in}}%
\pgfpathlineto{\pgfqpoint{2.235731in}{4.686938in}}%
\pgfpathlineto{\pgfqpoint{2.267641in}{4.775598in}}%
\pgfpathlineto{\pgfqpoint{2.261425in}{4.722557in}}%
\pgfpathlineto{\pgfqpoint{2.254793in}{4.692560in}}%
\pgfpathlineto{\pgfqpoint{2.219823in}{4.767253in}}%
\pgfpathlineto{\pgfqpoint{2.188137in}{4.671878in}}%
\pgfpathclose%
\pgfusepath{fill}%
\end{pgfscope}%
\begin{pgfscope}%
\pgfpathrectangle{\pgfqpoint{1.020000in}{0.880000in}}{\pgfqpoint{6.160000in}{6.160000in}}%
\pgfusepath{clip}%
\pgfsetbuttcap%
\pgfsetroundjoin%
\definecolor{currentfill}{rgb}{0.968894,0.679480,0.562812}%
\pgfsetfillcolor{currentfill}%
\pgfsetlinewidth{0.000000pt}%
\definecolor{currentstroke}{rgb}{0.000000,0.000000,0.000000}%
\pgfsetstrokecolor{currentstroke}%
\pgfsetdash{}{0pt}%
\pgfpathmoveto{\pgfqpoint{3.466448in}{4.743477in}}%
\pgfpathlineto{\pgfqpoint{3.475970in}{4.592628in}}%
\pgfpathlineto{\pgfqpoint{3.483150in}{4.779664in}}%
\pgfpathlineto{\pgfqpoint{3.517003in}{4.690050in}}%
\pgfpathlineto{\pgfqpoint{3.549799in}{4.756978in}}%
\pgfpathlineto{\pgfqpoint{3.542710in}{4.526352in}}%
\pgfpathlineto{\pgfqpoint{3.533694in}{4.604818in}}%
\pgfpathlineto{\pgfqpoint{3.499245in}{4.801348in}}%
\pgfpathlineto{\pgfqpoint{3.466448in}{4.743477in}}%
\pgfpathclose%
\pgfusepath{fill}%
\end{pgfscope}%
\begin{pgfscope}%
\pgfpathrectangle{\pgfqpoint{1.020000in}{0.880000in}}{\pgfqpoint{6.160000in}{6.160000in}}%
\pgfusepath{clip}%
\pgfsetbuttcap%
\pgfsetroundjoin%
\definecolor{currentfill}{rgb}{0.248091,0.326013,0.777669}%
\pgfsetfillcolor{currentfill}%
\pgfsetlinewidth{0.000000pt}%
\definecolor{currentstroke}{rgb}{0.000000,0.000000,0.000000}%
\pgfsetstrokecolor{currentstroke}%
\pgfsetdash{}{0pt}%
\pgfpathmoveto{\pgfqpoint{4.893583in}{3.106698in}}%
\pgfpathlineto{\pgfqpoint{4.904198in}{3.196833in}}%
\pgfpathlineto{\pgfqpoint{4.913294in}{3.074277in}}%
\pgfpathlineto{\pgfqpoint{4.944026in}{2.799399in}}%
\pgfpathlineto{\pgfqpoint{4.978544in}{3.023659in}}%
\pgfpathlineto{\pgfqpoint{4.967369in}{2.879743in}}%
\pgfpathlineto{\pgfqpoint{4.958092in}{2.973146in}}%
\pgfpathlineto{\pgfqpoint{4.925929in}{3.047987in}}%
\pgfpathlineto{\pgfqpoint{4.893583in}{3.106698in}}%
\pgfpathclose%
\pgfusepath{fill}%
\end{pgfscope}%
\begin{pgfscope}%
\pgfpathrectangle{\pgfqpoint{1.020000in}{0.880000in}}{\pgfqpoint{6.160000in}{6.160000in}}%
\pgfusepath{clip}%
\pgfsetbuttcap%
\pgfsetroundjoin%
\definecolor{currentfill}{rgb}{0.918282,0.484173,0.377794}%
\pgfsetfillcolor{currentfill}%
\pgfsetlinewidth{0.000000pt}%
\definecolor{currentstroke}{rgb}{0.000000,0.000000,0.000000}%
\pgfsetstrokecolor{currentstroke}%
\pgfsetdash{}{0pt}%
\pgfpathmoveto{\pgfqpoint{2.518750in}{4.907459in}}%
\pgfpathlineto{\pgfqpoint{2.525540in}{4.949554in}}%
\pgfpathlineto{\pgfqpoint{2.535006in}{4.828370in}}%
\pgfpathlineto{\pgfqpoint{2.563688in}{5.128303in}}%
\pgfpathlineto{\pgfqpoint{2.599597in}{4.976246in}}%
\pgfpathlineto{\pgfqpoint{2.590118in}{5.098514in}}%
\pgfpathlineto{\pgfqpoint{2.584171in}{4.996248in}}%
\pgfpathlineto{\pgfqpoint{2.551039in}{4.977776in}}%
\pgfpathlineto{\pgfqpoint{2.518750in}{4.907459in}}%
\pgfpathclose%
\pgfusepath{fill}%
\end{pgfscope}%
\begin{pgfscope}%
\pgfpathrectangle{\pgfqpoint{1.020000in}{0.880000in}}{\pgfqpoint{6.160000in}{6.160000in}}%
\pgfusepath{clip}%
\pgfsetbuttcap%
\pgfsetroundjoin%
\definecolor{currentfill}{rgb}{0.559747,0.694768,0.996075}%
\pgfsetfillcolor{currentfill}%
\pgfsetlinewidth{0.000000pt}%
\definecolor{currentstroke}{rgb}{0.000000,0.000000,0.000000}%
\pgfsetstrokecolor{currentstroke}%
\pgfsetdash{}{0pt}%
\pgfpathmoveto{\pgfqpoint{4.273622in}{3.829255in}}%
\pgfpathlineto{\pgfqpoint{4.282647in}{3.530748in}}%
\pgfpathlineto{\pgfqpoint{4.292124in}{3.680674in}}%
\pgfpathlineto{\pgfqpoint{4.324894in}{3.498060in}}%
\pgfpathlineto{\pgfqpoint{4.358120in}{3.657244in}}%
\pgfpathlineto{\pgfqpoint{4.348576in}{3.563939in}}%
\pgfpathlineto{\pgfqpoint{4.338965in}{3.403768in}}%
\pgfpathlineto{\pgfqpoint{4.306431in}{3.663586in}}%
\pgfpathlineto{\pgfqpoint{4.273622in}{3.829255in}}%
\pgfpathclose%
\pgfusepath{fill}%
\end{pgfscope}%
\begin{pgfscope}%
\pgfpathrectangle{\pgfqpoint{1.020000in}{0.880000in}}{\pgfqpoint{6.160000in}{6.160000in}}%
\pgfusepath{clip}%
\pgfsetbuttcap%
\pgfsetroundjoin%
\definecolor{currentfill}{rgb}{0.457046,0.594006,0.963029}%
\pgfsetfillcolor{currentfill}%
\pgfsetlinewidth{0.000000pt}%
\definecolor{currentstroke}{rgb}{0.000000,0.000000,0.000000}%
\pgfsetstrokecolor{currentstroke}%
\pgfsetdash{}{0pt}%
\pgfpathmoveto{\pgfqpoint{4.574519in}{3.549904in}}%
\pgfpathlineto{\pgfqpoint{4.583708in}{3.439689in}}%
\pgfpathlineto{\pgfqpoint{4.593539in}{3.486455in}}%
\pgfpathlineto{\pgfqpoint{4.626492in}{3.490144in}}%
\pgfpathlineto{\pgfqpoint{4.658818in}{3.365705in}}%
\pgfpathlineto{\pgfqpoint{4.648752in}{3.286356in}}%
\pgfpathlineto{\pgfqpoint{4.639471in}{3.370926in}}%
\pgfpathlineto{\pgfqpoint{4.606579in}{3.353084in}}%
\pgfpathlineto{\pgfqpoint{4.574519in}{3.549904in}}%
\pgfpathclose%
\pgfusepath{fill}%
\end{pgfscope}%
\begin{pgfscope}%
\pgfpathrectangle{\pgfqpoint{1.020000in}{0.880000in}}{\pgfqpoint{6.160000in}{6.160000in}}%
\pgfusepath{clip}%
\pgfsetbuttcap%
\pgfsetroundjoin%
\definecolor{currentfill}{rgb}{0.619318,0.744121,0.998931}%
\pgfsetfillcolor{currentfill}%
\pgfsetlinewidth{0.000000pt}%
\definecolor{currentstroke}{rgb}{0.000000,0.000000,0.000000}%
\pgfsetstrokecolor{currentstroke}%
\pgfsetdash{}{0pt}%
\pgfpathmoveto{\pgfqpoint{4.123145in}{3.665195in}}%
\pgfpathlineto{\pgfqpoint{4.132283in}{3.671207in}}%
\pgfpathlineto{\pgfqpoint{4.141464in}{3.621683in}}%
\pgfpathlineto{\pgfqpoint{4.174482in}{3.664452in}}%
\pgfpathlineto{\pgfqpoint{4.207505in}{3.692053in}}%
\pgfpathlineto{\pgfqpoint{4.198291in}{3.773237in}}%
\pgfpathlineto{\pgfqpoint{4.189079in}{3.815710in}}%
\pgfpathlineto{\pgfqpoint{4.156099in}{3.733300in}}%
\pgfpathlineto{\pgfqpoint{4.123145in}{3.665195in}}%
\pgfpathclose%
\pgfusepath{fill}%
\end{pgfscope}%
\begin{pgfscope}%
\pgfpathrectangle{\pgfqpoint{1.020000in}{0.880000in}}{\pgfqpoint{6.160000in}{6.160000in}}%
\pgfusepath{clip}%
\pgfsetbuttcap%
\pgfsetroundjoin%
\definecolor{currentfill}{rgb}{0.651398,0.768121,0.995891}%
\pgfsetfillcolor{currentfill}%
\pgfsetlinewidth{0.000000pt}%
\definecolor{currentstroke}{rgb}{0.000000,0.000000,0.000000}%
\pgfsetstrokecolor{currentstroke}%
\pgfsetdash{}{0pt}%
\pgfpathmoveto{\pgfqpoint{4.189079in}{3.815710in}}%
\pgfpathlineto{\pgfqpoint{4.198291in}{3.773237in}}%
\pgfpathlineto{\pgfqpoint{4.207505in}{3.692053in}}%
\pgfpathlineto{\pgfqpoint{4.240561in}{3.778459in}}%
\pgfpathlineto{\pgfqpoint{4.273622in}{3.829255in}}%
\pgfpathlineto{\pgfqpoint{4.264250in}{3.744236in}}%
\pgfpathlineto{\pgfqpoint{4.254927in}{3.677549in}}%
\pgfpathlineto{\pgfqpoint{4.222055in}{3.797771in}}%
\pgfpathlineto{\pgfqpoint{4.189079in}{3.815710in}}%
\pgfpathclose%
\pgfusepath{fill}%
\end{pgfscope}%
\begin{pgfscope}%
\pgfpathrectangle{\pgfqpoint{1.020000in}{0.880000in}}{\pgfqpoint{6.160000in}{6.160000in}}%
\pgfusepath{clip}%
\pgfsetbuttcap%
\pgfsetroundjoin%
\definecolor{currentfill}{rgb}{0.869655,0.379274,0.300941}%
\pgfsetfillcolor{currentfill}%
\pgfsetlinewidth{0.000000pt}%
\definecolor{currentstroke}{rgb}{0.000000,0.000000,0.000000}%
\pgfsetstrokecolor{currentstroke}%
\pgfsetdash{}{0pt}%
\pgfpathmoveto{\pgfqpoint{2.648787in}{5.141101in}}%
\pgfpathlineto{\pgfqpoint{2.655137in}{5.226012in}}%
\pgfpathlineto{\pgfqpoint{2.664373in}{5.119599in}}%
\pgfpathlineto{\pgfqpoint{2.697570in}{5.138700in}}%
\pgfpathlineto{\pgfqpoint{2.732007in}{5.070528in}}%
\pgfpathlineto{\pgfqpoint{2.722945in}{5.165470in}}%
\pgfpathlineto{\pgfqpoint{2.719157in}{4.895019in}}%
\pgfpathlineto{\pgfqpoint{2.682023in}{5.153493in}}%
\pgfpathlineto{\pgfqpoint{2.648787in}{5.141101in}}%
\pgfpathclose%
\pgfusepath{fill}%
\end{pgfscope}%
\begin{pgfscope}%
\pgfpathrectangle{\pgfqpoint{1.020000in}{0.880000in}}{\pgfqpoint{6.160000in}{6.160000in}}%
\pgfusepath{clip}%
\pgfsetbuttcap%
\pgfsetroundjoin%
\definecolor{currentfill}{rgb}{0.378598,0.503856,0.913692}%
\pgfsetfillcolor{currentfill}%
\pgfsetlinewidth{0.000000pt}%
\definecolor{currentstroke}{rgb}{0.000000,0.000000,0.000000}%
\pgfsetstrokecolor{currentstroke}%
\pgfsetdash{}{0pt}%
\pgfpathmoveto{\pgfqpoint{4.658818in}{3.365705in}}%
\pgfpathlineto{\pgfqpoint{4.667905in}{3.236121in}}%
\pgfpathlineto{\pgfqpoint{4.677293in}{3.168298in}}%
\pgfpathlineto{\pgfqpoint{4.710535in}{3.241221in}}%
\pgfpathlineto{\pgfqpoint{4.744177in}{3.380181in}}%
\pgfpathlineto{\pgfqpoint{4.732850in}{3.103606in}}%
\pgfpathlineto{\pgfqpoint{4.724748in}{3.408643in}}%
\pgfpathlineto{\pgfqpoint{4.691395in}{3.310300in}}%
\pgfpathlineto{\pgfqpoint{4.658818in}{3.365705in}}%
\pgfpathclose%
\pgfusepath{fill}%
\end{pgfscope}%
\begin{pgfscope}%
\pgfpathrectangle{\pgfqpoint{1.020000in}{0.880000in}}{\pgfqpoint{6.160000in}{6.160000in}}%
\pgfusepath{clip}%
\pgfsetbuttcap%
\pgfsetroundjoin%
\definecolor{currentfill}{rgb}{0.328604,0.439712,0.869587}%
\pgfsetfillcolor{currentfill}%
\pgfsetlinewidth{0.000000pt}%
\definecolor{currentstroke}{rgb}{0.000000,0.000000,0.000000}%
\pgfsetstrokecolor{currentstroke}%
\pgfsetdash{}{0pt}%
\pgfpathmoveto{\pgfqpoint{5.307527in}{3.204195in}}%
\pgfpathlineto{\pgfqpoint{5.317668in}{3.181933in}}%
\pgfpathlineto{\pgfqpoint{5.327507in}{3.131752in}}%
\pgfpathlineto{\pgfqpoint{5.360168in}{3.129480in}}%
\pgfpathlineto{\pgfqpoint{5.393693in}{3.200233in}}%
\pgfpathlineto{\pgfqpoint{5.384267in}{3.287763in}}%
\pgfpathlineto{\pgfqpoint{5.372427in}{3.172563in}}%
\pgfpathlineto{\pgfqpoint{5.338801in}{3.085643in}}%
\pgfpathlineto{\pgfqpoint{5.307527in}{3.204195in}}%
\pgfpathclose%
\pgfusepath{fill}%
\end{pgfscope}%
\begin{pgfscope}%
\pgfpathrectangle{\pgfqpoint{1.020000in}{0.880000in}}{\pgfqpoint{6.160000in}{6.160000in}}%
\pgfusepath{clip}%
\pgfsetbuttcap%
\pgfsetroundjoin%
\definecolor{currentfill}{rgb}{0.516260,0.654498,0.986407}%
\pgfsetfillcolor{currentfill}%
\pgfsetlinewidth{0.000000pt}%
\definecolor{currentstroke}{rgb}{0.000000,0.000000,0.000000}%
\pgfsetstrokecolor{currentstroke}%
\pgfsetdash{}{0pt}%
\pgfpathmoveto{\pgfqpoint{4.423589in}{3.459887in}}%
\pgfpathlineto{\pgfqpoint{4.433299in}{3.562360in}}%
\pgfpathlineto{\pgfqpoint{4.442978in}{3.637561in}}%
\pgfpathlineto{\pgfqpoint{4.475282in}{3.399574in}}%
\pgfpathlineto{\pgfqpoint{4.508574in}{3.519347in}}%
\pgfpathlineto{\pgfqpoint{4.499031in}{3.519510in}}%
\pgfpathlineto{\pgfqpoint{4.489400in}{3.484377in}}%
\pgfpathlineto{\pgfqpoint{4.456884in}{3.612684in}}%
\pgfpathlineto{\pgfqpoint{4.423589in}{3.459887in}}%
\pgfpathclose%
\pgfusepath{fill}%
\end{pgfscope}%
\begin{pgfscope}%
\pgfpathrectangle{\pgfqpoint{1.020000in}{0.880000in}}{\pgfqpoint{6.160000in}{6.160000in}}%
\pgfusepath{clip}%
\pgfsetbuttcap%
\pgfsetroundjoin%
\definecolor{currentfill}{rgb}{0.835345,0.860514,0.898970}%
\pgfsetfillcolor{currentfill}%
\pgfsetlinewidth{0.000000pt}%
\definecolor{currentstroke}{rgb}{0.000000,0.000000,0.000000}%
\pgfsetstrokecolor{currentstroke}%
\pgfsetdash{}{0pt}%
\pgfpathmoveto{\pgfqpoint{3.804246in}{4.065749in}}%
\pgfpathlineto{\pgfqpoint{3.812444in}{4.231618in}}%
\pgfpathlineto{\pgfqpoint{3.822166in}{3.986514in}}%
\pgfpathlineto{\pgfqpoint{3.854880in}{4.106372in}}%
\pgfpathlineto{\pgfqpoint{3.887888in}{4.146008in}}%
\pgfpathlineto{\pgfqpoint{3.878874in}{4.184544in}}%
\pgfpathlineto{\pgfqpoint{3.870022in}{4.174477in}}%
\pgfpathlineto{\pgfqpoint{3.837332in}{4.059941in}}%
\pgfpathlineto{\pgfqpoint{3.804246in}{4.065749in}}%
\pgfpathclose%
\pgfusepath{fill}%
\end{pgfscope}%
\begin{pgfscope}%
\pgfpathrectangle{\pgfqpoint{1.020000in}{0.880000in}}{\pgfqpoint{6.160000in}{6.160000in}}%
\pgfusepath{clip}%
\pgfsetbuttcap%
\pgfsetroundjoin%
\definecolor{currentfill}{rgb}{0.378598,0.503856,0.913692}%
\pgfsetfillcolor{currentfill}%
\pgfsetlinewidth{0.000000pt}%
\definecolor{currentstroke}{rgb}{0.000000,0.000000,0.000000}%
\pgfsetstrokecolor{currentstroke}%
\pgfsetdash{}{0pt}%
\pgfpathmoveto{\pgfqpoint{5.656097in}{3.306630in}}%
\pgfpathlineto{\pgfqpoint{5.666378in}{3.274265in}}%
\pgfpathlineto{\pgfqpoint{5.676553in}{3.233718in}}%
\pgfpathlineto{\pgfqpoint{5.710369in}{3.313782in}}%
\pgfpathlineto{\pgfqpoint{5.740377in}{3.147878in}}%
\pgfpathlineto{\pgfqpoint{5.731395in}{3.267101in}}%
\pgfpathlineto{\pgfqpoint{5.721578in}{3.332219in}}%
\pgfpathlineto{\pgfqpoint{5.688923in}{3.324759in}}%
\pgfpathlineto{\pgfqpoint{5.656097in}{3.306630in}}%
\pgfpathclose%
\pgfusepath{fill}%
\end{pgfscope}%
\begin{pgfscope}%
\pgfpathrectangle{\pgfqpoint{1.020000in}{0.880000in}}{\pgfqpoint{6.160000in}{6.160000in}}%
\pgfusepath{clip}%
\pgfsetbuttcap%
\pgfsetroundjoin%
\definecolor{currentfill}{rgb}{0.309060,0.413498,0.850128}%
\pgfsetfillcolor{currentfill}%
\pgfsetlinewidth{0.000000pt}%
\definecolor{currentstroke}{rgb}{0.000000,0.000000,0.000000}%
\pgfsetstrokecolor{currentstroke}%
\pgfsetdash{}{0pt}%
\pgfpathmoveto{\pgfqpoint{5.243626in}{3.341168in}}%
\pgfpathlineto{\pgfqpoint{5.250367in}{3.004745in}}%
\pgfpathlineto{\pgfqpoint{5.261620in}{3.089726in}}%
\pgfpathlineto{\pgfqpoint{5.293875in}{3.048747in}}%
\pgfpathlineto{\pgfqpoint{5.327507in}{3.131752in}}%
\pgfpathlineto{\pgfqpoint{5.317668in}{3.181933in}}%
\pgfpathlineto{\pgfqpoint{5.307527in}{3.204195in}}%
\pgfpathlineto{\pgfqpoint{5.273781in}{3.105607in}}%
\pgfpathlineto{\pgfqpoint{5.243626in}{3.341168in}}%
\pgfpathclose%
\pgfusepath{fill}%
\end{pgfscope}%
\begin{pgfscope}%
\pgfpathrectangle{\pgfqpoint{1.020000in}{0.880000in}}{\pgfqpoint{6.160000in}{6.160000in}}%
\pgfusepath{clip}%
\pgfsetbuttcap%
\pgfsetroundjoin%
\definecolor{currentfill}{rgb}{0.688188,0.793178,0.988038}%
\pgfsetfillcolor{currentfill}%
\pgfsetlinewidth{0.000000pt}%
\definecolor{currentstroke}{rgb}{0.000000,0.000000,0.000000}%
\pgfsetstrokecolor{currentstroke}%
\pgfsetdash{}{0pt}%
\pgfpathmoveto{\pgfqpoint{3.972833in}{3.789685in}}%
\pgfpathlineto{\pgfqpoint{3.981571in}{3.913388in}}%
\pgfpathlineto{\pgfqpoint{3.990784in}{3.810571in}}%
\pgfpathlineto{\pgfqpoint{4.023780in}{3.874523in}}%
\pgfpathlineto{\pgfqpoint{4.056987in}{3.788991in}}%
\pgfpathlineto{\pgfqpoint{4.047936in}{3.775211in}}%
\pgfpathlineto{\pgfqpoint{4.038923in}{3.750601in}}%
\pgfpathlineto{\pgfqpoint{4.005683in}{3.889451in}}%
\pgfpathlineto{\pgfqpoint{3.972833in}{3.789685in}}%
\pgfpathclose%
\pgfusepath{fill}%
\end{pgfscope}%
\begin{pgfscope}%
\pgfpathrectangle{\pgfqpoint{1.020000in}{0.880000in}}{\pgfqpoint{6.160000in}{6.160000in}}%
\pgfusepath{clip}%
\pgfsetbuttcap%
\pgfsetroundjoin%
\definecolor{currentfill}{rgb}{0.967544,0.730850,0.624685}%
\pgfsetfillcolor{currentfill}%
\pgfsetlinewidth{0.000000pt}%
\definecolor{currentstroke}{rgb}{0.000000,0.000000,0.000000}%
\pgfsetstrokecolor{currentstroke}%
\pgfsetdash{}{0pt}%
\pgfpathmoveto{\pgfqpoint{3.549799in}{4.756978in}}%
\pgfpathlineto{\pgfqpoint{3.558739in}{4.696481in}}%
\pgfpathlineto{\pgfqpoint{3.568879in}{4.440165in}}%
\pgfpathlineto{\pgfqpoint{3.601352in}{4.562506in}}%
\pgfpathlineto{\pgfqpoint{3.635448in}{4.396954in}}%
\pgfpathlineto{\pgfqpoint{3.625536in}{4.631598in}}%
\pgfpathlineto{\pgfqpoint{3.616635in}{4.681718in}}%
\pgfpathlineto{\pgfqpoint{3.584602in}{4.489175in}}%
\pgfpathlineto{\pgfqpoint{3.549799in}{4.756978in}}%
\pgfpathclose%
\pgfusepath{fill}%
\end{pgfscope}%
\begin{pgfscope}%
\pgfpathrectangle{\pgfqpoint{1.020000in}{0.880000in}}{\pgfqpoint{6.160000in}{6.160000in}}%
\pgfusepath{clip}%
\pgfsetbuttcap%
\pgfsetroundjoin%
\definecolor{currentfill}{rgb}{0.877149,0.394645,0.311724}%
\pgfsetfillcolor{currentfill}%
\pgfsetlinewidth{0.000000pt}%
\definecolor{currentstroke}{rgb}{0.000000,0.000000,0.000000}%
\pgfsetstrokecolor{currentstroke}%
\pgfsetdash{}{0pt}%
\pgfpathmoveto{\pgfqpoint{2.584171in}{4.996248in}}%
\pgfpathlineto{\pgfqpoint{2.590118in}{5.098514in}}%
\pgfpathlineto{\pgfqpoint{2.599597in}{4.976246in}}%
\pgfpathlineto{\pgfqpoint{2.632484in}{5.014309in}}%
\pgfpathlineto{\pgfqpoint{2.664373in}{5.119599in}}%
\pgfpathlineto{\pgfqpoint{2.655137in}{5.226012in}}%
\pgfpathlineto{\pgfqpoint{2.648787in}{5.141101in}}%
\pgfpathlineto{\pgfqpoint{2.615813in}{5.110984in}}%
\pgfpathlineto{\pgfqpoint{2.584171in}{4.996248in}}%
\pgfpathclose%
\pgfusepath{fill}%
\end{pgfscope}%
\begin{pgfscope}%
\pgfpathrectangle{\pgfqpoint{1.020000in}{0.880000in}}{\pgfqpoint{6.160000in}{6.160000in}}%
\pgfusepath{clip}%
\pgfsetbuttcap%
\pgfsetroundjoin%
\definecolor{currentfill}{rgb}{0.358415,0.478426,0.896795}%
\pgfsetfillcolor{currentfill}%
\pgfsetlinewidth{0.000000pt}%
\definecolor{currentstroke}{rgb}{0.000000,0.000000,0.000000}%
\pgfsetstrokecolor{currentstroke}%
\pgfsetdash{}{0pt}%
\pgfpathmoveto{\pgfqpoint{5.591189in}{3.324721in}}%
\pgfpathlineto{\pgfqpoint{5.598885in}{3.113096in}}%
\pgfpathlineto{\pgfqpoint{5.612596in}{3.323403in}}%
\pgfpathlineto{\pgfqpoint{5.641719in}{3.081735in}}%
\pgfpathlineto{\pgfqpoint{5.676553in}{3.233718in}}%
\pgfpathlineto{\pgfqpoint{5.666378in}{3.274265in}}%
\pgfpathlineto{\pgfqpoint{5.656097in}{3.306630in}}%
\pgfpathlineto{\pgfqpoint{5.622198in}{3.214650in}}%
\pgfpathlineto{\pgfqpoint{5.591189in}{3.324721in}}%
\pgfpathclose%
\pgfusepath{fill}%
\end{pgfscope}%
\begin{pgfscope}%
\pgfpathrectangle{\pgfqpoint{1.020000in}{0.880000in}}{\pgfqpoint{6.160000in}{6.160000in}}%
\pgfusepath{clip}%
\pgfsetbuttcap%
\pgfsetroundjoin%
\definecolor{currentfill}{rgb}{0.839365,0.321856,0.264924}%
\pgfsetfillcolor{currentfill}%
\pgfsetlinewidth{0.000000pt}%
\definecolor{currentstroke}{rgb}{0.000000,0.000000,0.000000}%
\pgfsetstrokecolor{currentstroke}%
\pgfsetdash{}{0pt}%
\pgfpathmoveto{\pgfqpoint{2.932040in}{5.087707in}}%
\pgfpathlineto{\pgfqpoint{2.938451in}{5.210256in}}%
\pgfpathlineto{\pgfqpoint{2.946724in}{5.182606in}}%
\pgfpathlineto{\pgfqpoint{2.979324in}{5.250881in}}%
\pgfpathlineto{\pgfqpoint{3.013789in}{5.159179in}}%
\pgfpathlineto{\pgfqpoint{3.005530in}{5.181854in}}%
\pgfpathlineto{\pgfqpoint{2.998025in}{5.140944in}}%
\pgfpathlineto{\pgfqpoint{2.963768in}{5.219920in}}%
\pgfpathlineto{\pgfqpoint{2.932040in}{5.087707in}}%
\pgfpathclose%
\pgfusepath{fill}%
\end{pgfscope}%
\begin{pgfscope}%
\pgfpathrectangle{\pgfqpoint{1.020000in}{0.880000in}}{\pgfqpoint{6.160000in}{6.160000in}}%
\pgfusepath{clip}%
\pgfsetbuttcap%
\pgfsetroundjoin%
\definecolor{currentfill}{rgb}{0.852378,0.346492,0.280346}%
\pgfsetfillcolor{currentfill}%
\pgfsetlinewidth{0.000000pt}%
\definecolor{currentstroke}{rgb}{0.000000,0.000000,0.000000}%
\pgfsetstrokecolor{currentstroke}%
\pgfsetdash{}{0pt}%
\pgfpathmoveto{\pgfqpoint{3.146882in}{5.182591in}}%
\pgfpathlineto{\pgfqpoint{3.154664in}{5.218433in}}%
\pgfpathlineto{\pgfqpoint{3.162672in}{5.233881in}}%
\pgfpathlineto{\pgfqpoint{3.198209in}{5.010795in}}%
\pgfpathlineto{\pgfqpoint{3.230351in}{5.132404in}}%
\pgfpathlineto{\pgfqpoint{3.221969in}{5.149443in}}%
\pgfpathlineto{\pgfqpoint{3.213732in}{5.152760in}}%
\pgfpathlineto{\pgfqpoint{3.180891in}{5.109878in}}%
\pgfpathlineto{\pgfqpoint{3.146882in}{5.182591in}}%
\pgfpathclose%
\pgfusepath{fill}%
\end{pgfscope}%
\begin{pgfscope}%
\pgfpathrectangle{\pgfqpoint{1.020000in}{0.880000in}}{\pgfqpoint{6.160000in}{6.160000in}}%
\pgfusepath{clip}%
\pgfsetbuttcap%
\pgfsetroundjoin%
\definecolor{currentfill}{rgb}{0.777378,0.840921,0.946149}%
\pgfsetfillcolor{currentfill}%
\pgfsetlinewidth{0.000000pt}%
\definecolor{currentstroke}{rgb}{0.000000,0.000000,0.000000}%
\pgfsetstrokecolor{currentstroke}%
\pgfsetdash{}{0pt}%
\pgfpathmoveto{\pgfqpoint{3.887888in}{4.146008in}}%
\pgfpathlineto{\pgfqpoint{3.896972in}{4.087297in}}%
\pgfpathlineto{\pgfqpoint{3.906166in}{3.991462in}}%
\pgfpathlineto{\pgfqpoint{3.939311in}{3.985937in}}%
\pgfpathlineto{\pgfqpoint{3.972833in}{3.789685in}}%
\pgfpathlineto{\pgfqpoint{3.963478in}{3.966939in}}%
\pgfpathlineto{\pgfqpoint{3.954441in}{3.998749in}}%
\pgfpathlineto{\pgfqpoint{3.921362in}{4.009416in}}%
\pgfpathlineto{\pgfqpoint{3.887888in}{4.146008in}}%
\pgfpathclose%
\pgfusepath{fill}%
\end{pgfscope}%
\begin{pgfscope}%
\pgfpathrectangle{\pgfqpoint{1.020000in}{0.880000in}}{\pgfqpoint{6.160000in}{6.160000in}}%
\pgfusepath{clip}%
\pgfsetbuttcap%
\pgfsetroundjoin%
\definecolor{currentfill}{rgb}{0.516260,0.654498,0.986407}%
\pgfsetfillcolor{currentfill}%
\pgfsetlinewidth{0.000000pt}%
\definecolor{currentstroke}{rgb}{0.000000,0.000000,0.000000}%
\pgfsetstrokecolor{currentstroke}%
\pgfsetdash{}{0pt}%
\pgfpathmoveto{\pgfqpoint{4.358120in}{3.657244in}}%
\pgfpathlineto{\pgfqpoint{4.367273in}{3.518425in}}%
\pgfpathlineto{\pgfqpoint{4.376538in}{3.440219in}}%
\pgfpathlineto{\pgfqpoint{4.409575in}{3.467974in}}%
\pgfpathlineto{\pgfqpoint{4.442978in}{3.637561in}}%
\pgfpathlineto{\pgfqpoint{4.433299in}{3.562360in}}%
\pgfpathlineto{\pgfqpoint{4.423589in}{3.459887in}}%
\pgfpathlineto{\pgfqpoint{4.390558in}{3.395516in}}%
\pgfpathlineto{\pgfqpoint{4.358120in}{3.657244in}}%
\pgfpathclose%
\pgfusepath{fill}%
\end{pgfscope}%
\begin{pgfscope}%
\pgfpathrectangle{\pgfqpoint{1.020000in}{0.880000in}}{\pgfqpoint{6.160000in}{6.160000in}}%
\pgfusepath{clip}%
\pgfsetbuttcap%
\pgfsetroundjoin%
\definecolor{currentfill}{rgb}{0.624703,0.748318,0.998719}%
\pgfsetfillcolor{currentfill}%
\pgfsetlinewidth{0.000000pt}%
\definecolor{currentstroke}{rgb}{0.000000,0.000000,0.000000}%
\pgfsetstrokecolor{currentstroke}%
\pgfsetdash{}{0pt}%
\pgfpathmoveto{\pgfqpoint{4.056987in}{3.788991in}}%
\pgfpathlineto{\pgfqpoint{4.066249in}{3.640054in}}%
\pgfpathlineto{\pgfqpoint{4.075107in}{3.858675in}}%
\pgfpathlineto{\pgfqpoint{4.108433in}{3.608256in}}%
\pgfpathlineto{\pgfqpoint{4.141464in}{3.621683in}}%
\pgfpathlineto{\pgfqpoint{4.132283in}{3.671207in}}%
\pgfpathlineto{\pgfqpoint{4.123145in}{3.665195in}}%
\pgfpathlineto{\pgfqpoint{4.090021in}{3.807186in}}%
\pgfpathlineto{\pgfqpoint{4.056987in}{3.788991in}}%
\pgfpathclose%
\pgfusepath{fill}%
\end{pgfscope}%
\begin{pgfscope}%
\pgfpathrectangle{\pgfqpoint{1.020000in}{0.880000in}}{\pgfqpoint{6.160000in}{6.160000in}}%
\pgfusepath{clip}%
\pgfsetbuttcap%
\pgfsetroundjoin%
\definecolor{currentfill}{rgb}{0.299441,0.400248,0.839842}%
\pgfsetfillcolor{currentfill}%
\pgfsetlinewidth{0.000000pt}%
\definecolor{currentstroke}{rgb}{0.000000,0.000000,0.000000}%
\pgfsetstrokecolor{currentstroke}%
\pgfsetdash{}{0pt}%
\pgfpathmoveto{\pgfqpoint{5.175624in}{3.091953in}}%
\pgfpathlineto{\pgfqpoint{5.185454in}{3.049480in}}%
\pgfpathlineto{\pgfqpoint{5.197518in}{3.225472in}}%
\pgfpathlineto{\pgfqpoint{5.227583in}{2.965262in}}%
\pgfpathlineto{\pgfqpoint{5.261620in}{3.089726in}}%
\pgfpathlineto{\pgfqpoint{5.250367in}{3.004745in}}%
\pgfpathlineto{\pgfqpoint{5.243626in}{3.341168in}}%
\pgfpathlineto{\pgfqpoint{5.209029in}{3.161627in}}%
\pgfpathlineto{\pgfqpoint{5.175624in}{3.091953in}}%
\pgfpathclose%
\pgfusepath{fill}%
\end{pgfscope}%
\begin{pgfscope}%
\pgfpathrectangle{\pgfqpoint{1.020000in}{0.880000in}}{\pgfqpoint{6.160000in}{6.160000in}}%
\pgfusepath{clip}%
\pgfsetbuttcap%
\pgfsetroundjoin%
\definecolor{currentfill}{rgb}{0.248091,0.326013,0.777669}%
\pgfsetfillcolor{currentfill}%
\pgfsetlinewidth{0.000000pt}%
\definecolor{currentstroke}{rgb}{0.000000,0.000000,0.000000}%
\pgfsetstrokecolor{currentstroke}%
\pgfsetdash{}{0pt}%
\pgfpathmoveto{\pgfqpoint{5.044415in}{3.061577in}}%
\pgfpathlineto{\pgfqpoint{5.053157in}{2.903034in}}%
\pgfpathlineto{\pgfqpoint{5.063886in}{2.970334in}}%
\pgfpathlineto{\pgfqpoint{5.097053in}{3.011530in}}%
\pgfpathlineto{\pgfqpoint{5.130305in}{3.060878in}}%
\pgfpathlineto{\pgfqpoint{5.119689in}{3.018050in}}%
\pgfpathlineto{\pgfqpoint{5.108731in}{2.935165in}}%
\pgfpathlineto{\pgfqpoint{5.077212in}{3.066167in}}%
\pgfpathlineto{\pgfqpoint{5.044415in}{3.061577in}}%
\pgfpathclose%
\pgfusepath{fill}%
\end{pgfscope}%
\begin{pgfscope}%
\pgfpathrectangle{\pgfqpoint{1.020000in}{0.880000in}}{\pgfqpoint{6.160000in}{6.160000in}}%
\pgfusepath{clip}%
\pgfsetbuttcap%
\pgfsetroundjoin%
\definecolor{currentfill}{rgb}{0.275827,0.366717,0.812553}%
\pgfsetfillcolor{currentfill}%
\pgfsetlinewidth{0.000000pt}%
\definecolor{currentstroke}{rgb}{0.000000,0.000000,0.000000}%
\pgfsetstrokecolor{currentstroke}%
\pgfsetdash{}{0pt}%
\pgfpathmoveto{\pgfqpoint{5.108731in}{2.935165in}}%
\pgfpathlineto{\pgfqpoint{5.119689in}{3.018050in}}%
\pgfpathlineto{\pgfqpoint{5.130305in}{3.060878in}}%
\pgfpathlineto{\pgfqpoint{5.163186in}{3.071247in}}%
\pgfpathlineto{\pgfqpoint{5.197518in}{3.225472in}}%
\pgfpathlineto{\pgfqpoint{5.185454in}{3.049480in}}%
\pgfpathlineto{\pgfqpoint{5.175624in}{3.091953in}}%
\pgfpathlineto{\pgfqpoint{5.142533in}{3.052134in}}%
\pgfpathlineto{\pgfqpoint{5.108731in}{2.935165in}}%
\pgfpathclose%
\pgfusepath{fill}%
\end{pgfscope}%
\begin{pgfscope}%
\pgfpathrectangle{\pgfqpoint{1.020000in}{0.880000in}}{\pgfqpoint{6.160000in}{6.160000in}}%
\pgfusepath{clip}%
\pgfsetbuttcap%
\pgfsetroundjoin%
\definecolor{currentfill}{rgb}{0.343278,0.459354,0.884122}%
\pgfsetfillcolor{currentfill}%
\pgfsetlinewidth{0.000000pt}%
\definecolor{currentstroke}{rgb}{0.000000,0.000000,0.000000}%
\pgfsetstrokecolor{currentstroke}%
\pgfsetdash{}{0pt}%
\pgfpathmoveto{\pgfqpoint{4.744177in}{3.380181in}}%
\pgfpathlineto{\pgfqpoint{4.752728in}{3.157635in}}%
\pgfpathlineto{\pgfqpoint{4.762718in}{3.186676in}}%
\pgfpathlineto{\pgfqpoint{4.796272in}{3.295201in}}%
\pgfpathlineto{\pgfqpoint{4.828278in}{3.156381in}}%
\pgfpathlineto{\pgfqpoint{4.817840in}{3.069378in}}%
\pgfpathlineto{\pgfqpoint{4.808450in}{3.143555in}}%
\pgfpathlineto{\pgfqpoint{4.775772in}{3.162557in}}%
\pgfpathlineto{\pgfqpoint{4.744177in}{3.380181in}}%
\pgfpathclose%
\pgfusepath{fill}%
\end{pgfscope}%
\begin{pgfscope}%
\pgfpathrectangle{\pgfqpoint{1.020000in}{0.880000in}}{\pgfqpoint{6.160000in}{6.160000in}}%
\pgfusepath{clip}%
\pgfsetbuttcap%
\pgfsetroundjoin%
\definecolor{currentfill}{rgb}{0.478462,0.616564,0.972721}%
\pgfsetfillcolor{currentfill}%
\pgfsetlinewidth{0.000000pt}%
\definecolor{currentstroke}{rgb}{0.000000,0.000000,0.000000}%
\pgfsetstrokecolor{currentstroke}%
\pgfsetdash{}{0pt}%
\pgfpathmoveto{\pgfqpoint{4.508574in}{3.519347in}}%
\pgfpathlineto{\pgfqpoint{4.517533in}{3.338671in}}%
\pgfpathlineto{\pgfqpoint{4.527291in}{3.394236in}}%
\pgfpathlineto{\pgfqpoint{4.560049in}{3.344873in}}%
\pgfpathlineto{\pgfqpoint{4.593539in}{3.486455in}}%
\pgfpathlineto{\pgfqpoint{4.583708in}{3.439689in}}%
\pgfpathlineto{\pgfqpoint{4.574519in}{3.549904in}}%
\pgfpathlineto{\pgfqpoint{4.541454in}{3.507502in}}%
\pgfpathlineto{\pgfqpoint{4.508574in}{3.519347in}}%
\pgfpathclose%
\pgfusepath{fill}%
\end{pgfscope}%
\begin{pgfscope}%
\pgfpathrectangle{\pgfqpoint{1.020000in}{0.880000in}}{\pgfqpoint{6.160000in}{6.160000in}}%
\pgfusepath{clip}%
\pgfsetbuttcap%
\pgfsetroundjoin%
\definecolor{currentfill}{rgb}{0.895885,0.433075,0.338681}%
\pgfsetfillcolor{currentfill}%
\pgfsetlinewidth{0.000000pt}%
\definecolor{currentstroke}{rgb}{0.000000,0.000000,0.000000}%
\pgfsetstrokecolor{currentstroke}%
\pgfsetdash{}{0pt}%
\pgfpathmoveto{\pgfqpoint{3.230351in}{5.132404in}}%
\pgfpathlineto{\pgfqpoint{3.238691in}{5.121355in}}%
\pgfpathlineto{\pgfqpoint{3.247616in}{5.048387in}}%
\pgfpathlineto{\pgfqpoint{3.280760in}{5.070289in}}%
\pgfpathlineto{\pgfqpoint{3.315043in}{4.958510in}}%
\pgfpathlineto{\pgfqpoint{3.306742in}{4.958083in}}%
\pgfpathlineto{\pgfqpoint{3.299002in}{4.895185in}}%
\pgfpathlineto{\pgfqpoint{3.263941in}{5.100446in}}%
\pgfpathlineto{\pgfqpoint{3.230351in}{5.132404in}}%
\pgfpathclose%
\pgfusepath{fill}%
\end{pgfscope}%
\begin{pgfscope}%
\pgfpathrectangle{\pgfqpoint{1.020000in}{0.880000in}}{\pgfqpoint{6.160000in}{6.160000in}}%
\pgfusepath{clip}%
\pgfsetbuttcap%
\pgfsetroundjoin%
\definecolor{currentfill}{rgb}{0.368507,0.491141,0.905243}%
\pgfsetfillcolor{currentfill}%
\pgfsetlinewidth{0.000000pt}%
\definecolor{currentstroke}{rgb}{0.000000,0.000000,0.000000}%
\pgfsetstrokecolor{currentstroke}%
\pgfsetdash{}{0pt}%
\pgfpathmoveto{\pgfqpoint{5.806304in}{3.202944in}}%
\pgfpathlineto{\pgfqpoint{5.817898in}{3.244033in}}%
\pgfpathlineto{\pgfqpoint{5.829452in}{3.280736in}}%
\pgfpathlineto{\pgfqpoint{5.859340in}{3.119883in}}%
\pgfpathlineto{\pgfqpoint{5.850122in}{3.224579in}}%
\pgfpathlineto{\pgfqpoint{5.841940in}{3.391653in}}%
\pgfpathlineto{\pgfqpoint{5.806304in}{3.202944in}}%
\pgfpathclose%
\pgfusepath{fill}%
\end{pgfscope}%
\begin{pgfscope}%
\pgfpathrectangle{\pgfqpoint{1.020000in}{0.880000in}}{\pgfqpoint{6.160000in}{6.160000in}}%
\pgfusepath{clip}%
\pgfsetbuttcap%
\pgfsetroundjoin%
\definecolor{currentfill}{rgb}{0.363461,0.484784,0.901019}%
\pgfsetfillcolor{currentfill}%
\pgfsetlinewidth{0.000000pt}%
\definecolor{currentstroke}{rgb}{0.000000,0.000000,0.000000}%
\pgfsetstrokecolor{currentstroke}%
\pgfsetdash{}{0pt}%
\pgfpathmoveto{\pgfqpoint{5.525557in}{3.296440in}}%
\pgfpathlineto{\pgfqpoint{5.534569in}{3.176501in}}%
\pgfpathlineto{\pgfqpoint{5.545639in}{3.207573in}}%
\pgfpathlineto{\pgfqpoint{5.579033in}{3.260113in}}%
\pgfpathlineto{\pgfqpoint{5.612596in}{3.323403in}}%
\pgfpathlineto{\pgfqpoint{5.598885in}{3.113096in}}%
\pgfpathlineto{\pgfqpoint{5.591189in}{3.324721in}}%
\pgfpathlineto{\pgfqpoint{5.556609in}{3.182127in}}%
\pgfpathlineto{\pgfqpoint{5.525557in}{3.296440in}}%
\pgfpathclose%
\pgfusepath{fill}%
\end{pgfscope}%
\begin{pgfscope}%
\pgfpathrectangle{\pgfqpoint{1.020000in}{0.880000in}}{\pgfqpoint{6.160000in}{6.160000in}}%
\pgfusepath{clip}%
\pgfsetbuttcap%
\pgfsetroundjoin%
\definecolor{currentfill}{rgb}{0.597777,0.727330,0.999777}%
\pgfsetfillcolor{currentfill}%
\pgfsetlinewidth{0.000000pt}%
\definecolor{currentstroke}{rgb}{0.000000,0.000000,0.000000}%
\pgfsetstrokecolor{currentstroke}%
\pgfsetdash{}{0pt}%
\pgfpathmoveto{\pgfqpoint{4.207505in}{3.692053in}}%
\pgfpathlineto{\pgfqpoint{4.216677in}{3.471554in}}%
\pgfpathlineto{\pgfqpoint{4.226037in}{3.741552in}}%
\pgfpathlineto{\pgfqpoint{4.258948in}{3.510298in}}%
\pgfpathlineto{\pgfqpoint{4.292124in}{3.680674in}}%
\pgfpathlineto{\pgfqpoint{4.282647in}{3.530748in}}%
\pgfpathlineto{\pgfqpoint{4.273622in}{3.829255in}}%
\pgfpathlineto{\pgfqpoint{4.240561in}{3.778459in}}%
\pgfpathlineto{\pgfqpoint{4.207505in}{3.692053in}}%
\pgfpathclose%
\pgfusepath{fill}%
\end{pgfscope}%
\begin{pgfscope}%
\pgfpathrectangle{\pgfqpoint{1.020000in}{0.880000in}}{\pgfqpoint{6.160000in}{6.160000in}}%
\pgfusepath{clip}%
\pgfsetbuttcap%
\pgfsetroundjoin%
\definecolor{currentfill}{rgb}{0.353369,0.472069,0.892570}%
\pgfsetfillcolor{currentfill}%
\pgfsetlinewidth{0.000000pt}%
\definecolor{currentstroke}{rgb}{0.000000,0.000000,0.000000}%
\pgfsetstrokecolor{currentstroke}%
\pgfsetdash{}{0pt}%
\pgfpathmoveto{\pgfqpoint{5.740377in}{3.147878in}}%
\pgfpathlineto{\pgfqpoint{5.751445in}{3.161588in}}%
\pgfpathlineto{\pgfqpoint{5.761682in}{3.121111in}}%
\pgfpathlineto{\pgfqpoint{5.796910in}{3.285048in}}%
\pgfpathlineto{\pgfqpoint{5.829452in}{3.280736in}}%
\pgfpathlineto{\pgfqpoint{5.817898in}{3.244033in}}%
\pgfpathlineto{\pgfqpoint{5.806304in}{3.202944in}}%
\pgfpathlineto{\pgfqpoint{5.774802in}{3.267060in}}%
\pgfpathlineto{\pgfqpoint{5.740377in}{3.147878in}}%
\pgfpathclose%
\pgfusepath{fill}%
\end{pgfscope}%
\begin{pgfscope}%
\pgfpathrectangle{\pgfqpoint{1.020000in}{0.880000in}}{\pgfqpoint{6.160000in}{6.160000in}}%
\pgfusepath{clip}%
\pgfsetbuttcap%
\pgfsetroundjoin%
\definecolor{currentfill}{rgb}{0.851372,0.863125,0.881064}%
\pgfsetfillcolor{currentfill}%
\pgfsetlinewidth{0.000000pt}%
\definecolor{currentstroke}{rgb}{0.000000,0.000000,0.000000}%
\pgfsetstrokecolor{currentstroke}%
\pgfsetdash{}{0pt}%
\pgfpathmoveto{\pgfqpoint{3.737693in}{4.147627in}}%
\pgfpathlineto{\pgfqpoint{3.746544in}{4.124587in}}%
\pgfpathlineto{\pgfqpoint{3.755267in}{4.135226in}}%
\pgfpathlineto{\pgfqpoint{3.787912in}{4.274973in}}%
\pgfpathlineto{\pgfqpoint{3.822166in}{3.986514in}}%
\pgfpathlineto{\pgfqpoint{3.812444in}{4.231618in}}%
\pgfpathlineto{\pgfqpoint{3.804246in}{4.065749in}}%
\pgfpathlineto{\pgfqpoint{3.770321in}{4.270633in}}%
\pgfpathlineto{\pgfqpoint{3.737693in}{4.147627in}}%
\pgfpathclose%
\pgfusepath{fill}%
\end{pgfscope}%
\begin{pgfscope}%
\pgfpathrectangle{\pgfqpoint{1.020000in}{0.880000in}}{\pgfqpoint{6.160000in}{6.160000in}}%
\pgfusepath{clip}%
\pgfsetbuttcap%
\pgfsetroundjoin%
\definecolor{currentfill}{rgb}{0.248091,0.326013,0.777669}%
\pgfsetfillcolor{currentfill}%
\pgfsetlinewidth{0.000000pt}%
\definecolor{currentstroke}{rgb}{0.000000,0.000000,0.000000}%
\pgfsetstrokecolor{currentstroke}%
\pgfsetdash{}{0pt}%
\pgfpathmoveto{\pgfqpoint{4.978544in}{3.023659in}}%
\pgfpathlineto{\pgfqpoint{4.988914in}{3.061899in}}%
\pgfpathlineto{\pgfqpoint{4.998508in}{3.001726in}}%
\pgfpathlineto{\pgfqpoint{5.031970in}{3.074994in}}%
\pgfpathlineto{\pgfqpoint{5.063886in}{2.970334in}}%
\pgfpathlineto{\pgfqpoint{5.053157in}{2.903034in}}%
\pgfpathlineto{\pgfqpoint{5.044415in}{3.061577in}}%
\pgfpathlineto{\pgfqpoint{5.010045in}{2.869628in}}%
\pgfpathlineto{\pgfqpoint{4.978544in}{3.023659in}}%
\pgfpathclose%
\pgfusepath{fill}%
\end{pgfscope}%
\begin{pgfscope}%
\pgfpathrectangle{\pgfqpoint{1.020000in}{0.880000in}}{\pgfqpoint{6.160000in}{6.160000in}}%
\pgfusepath{clip}%
\pgfsetbuttcap%
\pgfsetroundjoin%
\definecolor{currentfill}{rgb}{0.419991,0.552989,0.942630}%
\pgfsetfillcolor{currentfill}%
\pgfsetlinewidth{0.000000pt}%
\definecolor{currentstroke}{rgb}{0.000000,0.000000,0.000000}%
\pgfsetstrokecolor{currentstroke}%
\pgfsetdash{}{0pt}%
\pgfpathmoveto{\pgfqpoint{4.593539in}{3.486455in}}%
\pgfpathlineto{\pgfqpoint{4.601776in}{3.146301in}}%
\pgfpathlineto{\pgfqpoint{4.612403in}{3.373450in}}%
\pgfpathlineto{\pgfqpoint{4.645509in}{3.404390in}}%
\pgfpathlineto{\pgfqpoint{4.677293in}{3.168298in}}%
\pgfpathlineto{\pgfqpoint{4.667905in}{3.236121in}}%
\pgfpathlineto{\pgfqpoint{4.658818in}{3.365705in}}%
\pgfpathlineto{\pgfqpoint{4.626492in}{3.490144in}}%
\pgfpathlineto{\pgfqpoint{4.593539in}{3.486455in}}%
\pgfpathclose%
\pgfusepath{fill}%
\end{pgfscope}%
\begin{pgfscope}%
\pgfpathrectangle{\pgfqpoint{1.020000in}{0.880000in}}{\pgfqpoint{6.160000in}{6.160000in}}%
\pgfusepath{clip}%
\pgfsetbuttcap%
\pgfsetroundjoin%
\definecolor{currentfill}{rgb}{0.960490,0.616276,0.495467}%
\pgfsetfillcolor{currentfill}%
\pgfsetlinewidth{0.000000pt}%
\definecolor{currentstroke}{rgb}{0.000000,0.000000,0.000000}%
\pgfsetstrokecolor{currentstroke}%
\pgfsetdash{}{0pt}%
\pgfpathmoveto{\pgfqpoint{3.398579in}{4.929369in}}%
\pgfpathlineto{\pgfqpoint{3.407538in}{4.858752in}}%
\pgfpathlineto{\pgfqpoint{3.416550in}{4.781789in}}%
\pgfpathlineto{\pgfqpoint{3.449731in}{4.798538in}}%
\pgfpathlineto{\pgfqpoint{3.483150in}{4.779664in}}%
\pgfpathlineto{\pgfqpoint{3.475970in}{4.592628in}}%
\pgfpathlineto{\pgfqpoint{3.466448in}{4.743477in}}%
\pgfpathlineto{\pgfqpoint{3.433633in}{4.690685in}}%
\pgfpathlineto{\pgfqpoint{3.398579in}{4.929369in}}%
\pgfpathclose%
\pgfusepath{fill}%
\end{pgfscope}%
\begin{pgfscope}%
\pgfpathrectangle{\pgfqpoint{1.020000in}{0.880000in}}{\pgfqpoint{6.160000in}{6.160000in}}%
\pgfusepath{clip}%
\pgfsetbuttcap%
\pgfsetroundjoin%
\definecolor{currentfill}{rgb}{0.820401,0.286765,0.245160}%
\pgfsetfillcolor{currentfill}%
\pgfsetlinewidth{0.000000pt}%
\definecolor{currentstroke}{rgb}{0.000000,0.000000,0.000000}%
\pgfsetstrokecolor{currentstroke}%
\pgfsetdash{}{0pt}%
\pgfpathmoveto{\pgfqpoint{2.864643in}{5.142830in}}%
\pgfpathlineto{\pgfqpoint{2.872360in}{5.153118in}}%
\pgfpathlineto{\pgfqpoint{2.877310in}{5.380874in}}%
\pgfpathlineto{\pgfqpoint{2.912471in}{5.248749in}}%
\pgfpathlineto{\pgfqpoint{2.946724in}{5.182606in}}%
\pgfpathlineto{\pgfqpoint{2.938451in}{5.210256in}}%
\pgfpathlineto{\pgfqpoint{2.932040in}{5.087707in}}%
\pgfpathlineto{\pgfqpoint{2.896996in}{5.223183in}}%
\pgfpathlineto{\pgfqpoint{2.864643in}{5.142830in}}%
\pgfpathclose%
\pgfusepath{fill}%
\end{pgfscope}%
\begin{pgfscope}%
\pgfpathrectangle{\pgfqpoint{1.020000in}{0.880000in}}{\pgfqpoint{6.160000in}{6.160000in}}%
\pgfusepath{clip}%
\pgfsetbuttcap%
\pgfsetroundjoin%
\definecolor{currentfill}{rgb}{0.954566,0.779055,0.692531}%
\pgfsetfillcolor{currentfill}%
\pgfsetlinewidth{0.000000pt}%
\definecolor{currentstroke}{rgb}{0.000000,0.000000,0.000000}%
\pgfsetstrokecolor{currentstroke}%
\pgfsetdash{}{0pt}%
\pgfpathmoveto{\pgfqpoint{3.568879in}{4.440165in}}%
\pgfpathlineto{\pgfqpoint{3.576601in}{4.582667in}}%
\pgfpathlineto{\pgfqpoint{3.585613in}{4.513928in}}%
\pgfpathlineto{\pgfqpoint{3.618761in}{4.535043in}}%
\pgfpathlineto{\pgfqpoint{3.653202in}{4.309049in}}%
\pgfpathlineto{\pgfqpoint{3.643922in}{4.427546in}}%
\pgfpathlineto{\pgfqpoint{3.635448in}{4.396954in}}%
\pgfpathlineto{\pgfqpoint{3.601352in}{4.562506in}}%
\pgfpathlineto{\pgfqpoint{3.568879in}{4.440165in}}%
\pgfpathclose%
\pgfusepath{fill}%
\end{pgfscope}%
\begin{pgfscope}%
\pgfpathrectangle{\pgfqpoint{1.020000in}{0.880000in}}{\pgfqpoint{6.160000in}{6.160000in}}%
\pgfusepath{clip}%
\pgfsetbuttcap%
\pgfsetroundjoin%
\definecolor{currentfill}{rgb}{0.929357,0.512254,0.400673}%
\pgfsetfillcolor{currentfill}%
\pgfsetlinewidth{0.000000pt}%
\definecolor{currentstroke}{rgb}{0.000000,0.000000,0.000000}%
\pgfsetstrokecolor{currentstroke}%
\pgfsetdash{}{0pt}%
\pgfpathmoveto{\pgfqpoint{3.315043in}{4.958510in}}%
\pgfpathlineto{\pgfqpoint{3.323476in}{4.945340in}}%
\pgfpathlineto{\pgfqpoint{3.331651in}{4.964842in}}%
\pgfpathlineto{\pgfqpoint{3.365246in}{4.932801in}}%
\pgfpathlineto{\pgfqpoint{3.398579in}{4.929369in}}%
\pgfpathlineto{\pgfqpoint{3.390025in}{4.948901in}}%
\pgfpathlineto{\pgfqpoint{3.382832in}{4.797951in}}%
\pgfpathlineto{\pgfqpoint{3.348308in}{4.959146in}}%
\pgfpathlineto{\pgfqpoint{3.315043in}{4.958510in}}%
\pgfpathclose%
\pgfusepath{fill}%
\end{pgfscope}%
\begin{pgfscope}%
\pgfpathrectangle{\pgfqpoint{1.020000in}{0.880000in}}{\pgfqpoint{6.160000in}{6.160000in}}%
\pgfusepath{clip}%
\pgfsetbuttcap%
\pgfsetroundjoin%
\definecolor{currentfill}{rgb}{0.478462,0.616564,0.972721}%
\pgfsetfillcolor{currentfill}%
\pgfsetlinewidth{0.000000pt}%
\definecolor{currentstroke}{rgb}{0.000000,0.000000,0.000000}%
\pgfsetstrokecolor{currentstroke}%
\pgfsetdash{}{0pt}%
\pgfpathmoveto{\pgfqpoint{4.442978in}{3.637561in}}%
\pgfpathlineto{\pgfqpoint{4.451780in}{3.373850in}}%
\pgfpathlineto{\pgfqpoint{4.461357in}{3.399434in}}%
\pgfpathlineto{\pgfqpoint{4.494454in}{3.435237in}}%
\pgfpathlineto{\pgfqpoint{4.527291in}{3.394236in}}%
\pgfpathlineto{\pgfqpoint{4.517533in}{3.338671in}}%
\pgfpathlineto{\pgfqpoint{4.508574in}{3.519347in}}%
\pgfpathlineto{\pgfqpoint{4.475282in}{3.399574in}}%
\pgfpathlineto{\pgfqpoint{4.442978in}{3.637561in}}%
\pgfpathclose%
\pgfusepath{fill}%
\end{pgfscope}%
\begin{pgfscope}%
\pgfpathrectangle{\pgfqpoint{1.020000in}{0.880000in}}{\pgfqpoint{6.160000in}{6.160000in}}%
\pgfusepath{clip}%
\pgfsetbuttcap%
\pgfsetroundjoin%
\definecolor{currentfill}{rgb}{0.922681,0.828568,0.777054}%
\pgfsetfillcolor{currentfill}%
\pgfsetlinewidth{0.000000pt}%
\definecolor{currentstroke}{rgb}{0.000000,0.000000,0.000000}%
\pgfsetstrokecolor{currentstroke}%
\pgfsetdash{}{0pt}%
\pgfpathmoveto{\pgfqpoint{3.653202in}{4.309049in}}%
\pgfpathlineto{\pgfqpoint{3.661145in}{4.448669in}}%
\pgfpathlineto{\pgfqpoint{3.669750in}{4.466614in}}%
\pgfpathlineto{\pgfqpoint{3.703551in}{4.355986in}}%
\pgfpathlineto{\pgfqpoint{3.737693in}{4.147627in}}%
\pgfpathlineto{\pgfqpoint{3.727647in}{4.438511in}}%
\pgfpathlineto{\pgfqpoint{3.719949in}{4.211675in}}%
\pgfpathlineto{\pgfqpoint{3.686582in}{4.264306in}}%
\pgfpathlineto{\pgfqpoint{3.653202in}{4.309049in}}%
\pgfpathclose%
\pgfusepath{fill}%
\end{pgfscope}%
\begin{pgfscope}%
\pgfpathrectangle{\pgfqpoint{1.020000in}{0.880000in}}{\pgfqpoint{6.160000in}{6.160000in}}%
\pgfusepath{clip}%
\pgfsetbuttcap%
\pgfsetroundjoin%
\definecolor{currentfill}{rgb}{0.962701,0.628218,0.507636}%
\pgfsetfillcolor{currentfill}%
\pgfsetlinewidth{0.000000pt}%
\definecolor{currentstroke}{rgb}{0.000000,0.000000,0.000000}%
\pgfsetstrokecolor{currentstroke}%
\pgfsetdash{}{0pt}%
\pgfpathmoveto{\pgfqpoint{2.336875in}{4.662581in}}%
\pgfpathlineto{\pgfqpoint{2.343061in}{4.723573in}}%
\pgfpathlineto{\pgfqpoint{2.349988in}{4.745030in}}%
\pgfpathlineto{\pgfqpoint{2.383153in}{4.770186in}}%
\pgfpathlineto{\pgfqpoint{2.416447in}{4.787791in}}%
\pgfpathlineto{\pgfqpoint{2.408178in}{4.838303in}}%
\pgfpathlineto{\pgfqpoint{2.402393in}{4.747987in}}%
\pgfpathlineto{\pgfqpoint{2.369304in}{4.723436in}}%
\pgfpathlineto{\pgfqpoint{2.336875in}{4.662581in}}%
\pgfpathclose%
\pgfusepath{fill}%
\end{pgfscope}%
\begin{pgfscope}%
\pgfpathrectangle{\pgfqpoint{1.020000in}{0.880000in}}{\pgfqpoint{6.160000in}{6.160000in}}%
\pgfusepath{clip}%
\pgfsetbuttcap%
\pgfsetroundjoin%
\definecolor{currentfill}{rgb}{0.830187,0.304733,0.254891}%
\pgfsetfillcolor{currentfill}%
\pgfsetlinewidth{0.000000pt}%
\definecolor{currentstroke}{rgb}{0.000000,0.000000,0.000000}%
\pgfsetstrokecolor{currentstroke}%
\pgfsetdash{}{0pt}%
\pgfpathmoveto{\pgfqpoint{3.080763in}{5.134300in}}%
\pgfpathlineto{\pgfqpoint{3.087671in}{5.242810in}}%
\pgfpathlineto{\pgfqpoint{3.094357in}{5.375237in}}%
\pgfpathlineto{\pgfqpoint{3.131193in}{5.049009in}}%
\pgfpathlineto{\pgfqpoint{3.162672in}{5.233881in}}%
\pgfpathlineto{\pgfqpoint{3.154664in}{5.218433in}}%
\pgfpathlineto{\pgfqpoint{3.146882in}{5.182591in}}%
\pgfpathlineto{\pgfqpoint{3.114756in}{5.069845in}}%
\pgfpathlineto{\pgfqpoint{3.080763in}{5.134300in}}%
\pgfpathclose%
\pgfusepath{fill}%
\end{pgfscope}%
\begin{pgfscope}%
\pgfpathrectangle{\pgfqpoint{1.020000in}{0.880000in}}{\pgfqpoint{6.160000in}{6.160000in}}%
\pgfusepath{clip}%
\pgfsetbuttcap%
\pgfsetroundjoin%
\definecolor{currentfill}{rgb}{0.358415,0.478426,0.896795}%
\pgfsetfillcolor{currentfill}%
\pgfsetlinewidth{0.000000pt}%
\definecolor{currentstroke}{rgb}{0.000000,0.000000,0.000000}%
\pgfsetstrokecolor{currentstroke}%
\pgfsetdash{}{0pt}%
\pgfpathmoveto{\pgfqpoint{5.459040in}{3.202061in}}%
\pgfpathlineto{\pgfqpoint{5.468153in}{3.089502in}}%
\pgfpathlineto{\pgfqpoint{5.481775in}{3.323956in}}%
\pgfpathlineto{\pgfqpoint{5.513322in}{3.234695in}}%
\pgfpathlineto{\pgfqpoint{5.545639in}{3.207573in}}%
\pgfpathlineto{\pgfqpoint{5.534569in}{3.176501in}}%
\pgfpathlineto{\pgfqpoint{5.525557in}{3.296440in}}%
\pgfpathlineto{\pgfqpoint{5.492275in}{3.247881in}}%
\pgfpathlineto{\pgfqpoint{5.459040in}{3.202061in}}%
\pgfpathclose%
\pgfusepath{fill}%
\end{pgfscope}%
\begin{pgfscope}%
\pgfpathrectangle{\pgfqpoint{1.020000in}{0.880000in}}{\pgfqpoint{6.160000in}{6.160000in}}%
\pgfusepath{clip}%
\pgfsetbuttcap%
\pgfsetroundjoin%
\definecolor{currentfill}{rgb}{0.967317,0.657471,0.538160}%
\pgfsetfillcolor{currentfill}%
\pgfsetlinewidth{0.000000pt}%
\definecolor{currentstroke}{rgb}{0.000000,0.000000,0.000000}%
\pgfsetstrokecolor{currentstroke}%
\pgfsetdash{}{0pt}%
\pgfpathmoveto{\pgfqpoint{2.267641in}{4.775598in}}%
\pgfpathlineto{\pgfqpoint{2.276473in}{4.690917in}}%
\pgfpathlineto{\pgfqpoint{2.286919in}{4.520370in}}%
\pgfpathlineto{\pgfqpoint{2.315561in}{4.788154in}}%
\pgfpathlineto{\pgfqpoint{2.349988in}{4.745030in}}%
\pgfpathlineto{\pgfqpoint{2.343061in}{4.723573in}}%
\pgfpathlineto{\pgfqpoint{2.336875in}{4.662581in}}%
\pgfpathlineto{\pgfqpoint{2.302772in}{4.692941in}}%
\pgfpathlineto{\pgfqpoint{2.267641in}{4.775598in}}%
\pgfpathclose%
\pgfusepath{fill}%
\end{pgfscope}%
\begin{pgfscope}%
\pgfpathrectangle{\pgfqpoint{1.020000in}{0.880000in}}{\pgfqpoint{6.160000in}{6.160000in}}%
\pgfusepath{clip}%
\pgfsetbuttcap%
\pgfsetroundjoin%
\definecolor{currentfill}{rgb}{0.543440,0.680003,0.993051}%
\pgfsetfillcolor{currentfill}%
\pgfsetlinewidth{0.000000pt}%
\definecolor{currentstroke}{rgb}{0.000000,0.000000,0.000000}%
\pgfsetstrokecolor{currentstroke}%
\pgfsetdash{}{0pt}%
\pgfpathmoveto{\pgfqpoint{4.292124in}{3.680674in}}%
\pgfpathlineto{\pgfqpoint{4.301173in}{3.427414in}}%
\pgfpathlineto{\pgfqpoint{4.310713in}{3.581659in}}%
\pgfpathlineto{\pgfqpoint{4.343793in}{3.594531in}}%
\pgfpathlineto{\pgfqpoint{4.376538in}{3.440219in}}%
\pgfpathlineto{\pgfqpoint{4.367273in}{3.518425in}}%
\pgfpathlineto{\pgfqpoint{4.358120in}{3.657244in}}%
\pgfpathlineto{\pgfqpoint{4.324894in}{3.498060in}}%
\pgfpathlineto{\pgfqpoint{4.292124in}{3.680674in}}%
\pgfpathclose%
\pgfusepath{fill}%
\end{pgfscope}%
\begin{pgfscope}%
\pgfpathrectangle{\pgfqpoint{1.020000in}{0.880000in}}{\pgfqpoint{6.160000in}{6.160000in}}%
\pgfusepath{clip}%
\pgfsetbuttcap%
\pgfsetroundjoin%
\definecolor{currentfill}{rgb}{0.597777,0.727330,0.999777}%
\pgfsetfillcolor{currentfill}%
\pgfsetlinewidth{0.000000pt}%
\definecolor{currentstroke}{rgb}{0.000000,0.000000,0.000000}%
\pgfsetstrokecolor{currentstroke}%
\pgfsetdash{}{0pt}%
\pgfpathmoveto{\pgfqpoint{4.141464in}{3.621683in}}%
\pgfpathlineto{\pgfqpoint{4.150601in}{3.739340in}}%
\pgfpathlineto{\pgfqpoint{4.159824in}{3.664255in}}%
\pgfpathlineto{\pgfqpoint{4.192918in}{3.603864in}}%
\pgfpathlineto{\pgfqpoint{4.226037in}{3.741552in}}%
\pgfpathlineto{\pgfqpoint{4.216677in}{3.471554in}}%
\pgfpathlineto{\pgfqpoint{4.207505in}{3.692053in}}%
\pgfpathlineto{\pgfqpoint{4.174482in}{3.664452in}}%
\pgfpathlineto{\pgfqpoint{4.141464in}{3.621683in}}%
\pgfpathclose%
\pgfusepath{fill}%
\end{pgfscope}%
\begin{pgfscope}%
\pgfpathrectangle{\pgfqpoint{1.020000in}{0.880000in}}{\pgfqpoint{6.160000in}{6.160000in}}%
\pgfusepath{clip}%
\pgfsetbuttcap%
\pgfsetroundjoin%
\definecolor{currentfill}{rgb}{0.968894,0.679480,0.562812}%
\pgfsetfillcolor{currentfill}%
\pgfsetlinewidth{0.000000pt}%
\definecolor{currentstroke}{rgb}{0.000000,0.000000,0.000000}%
\pgfsetstrokecolor{currentstroke}%
\pgfsetdash{}{0pt}%
\pgfpathmoveto{\pgfqpoint{3.483150in}{4.779664in}}%
\pgfpathlineto{\pgfqpoint{3.492930in}{4.592828in}}%
\pgfpathlineto{\pgfqpoint{3.500575in}{4.720796in}}%
\pgfpathlineto{\pgfqpoint{3.534551in}{4.616598in}}%
\pgfpathlineto{\pgfqpoint{3.568879in}{4.440165in}}%
\pgfpathlineto{\pgfqpoint{3.558739in}{4.696481in}}%
\pgfpathlineto{\pgfqpoint{3.549799in}{4.756978in}}%
\pgfpathlineto{\pgfqpoint{3.517003in}{4.690050in}}%
\pgfpathlineto{\pgfqpoint{3.483150in}{4.779664in}}%
\pgfpathclose%
\pgfusepath{fill}%
\end{pgfscope}%
\begin{pgfscope}%
\pgfpathrectangle{\pgfqpoint{1.020000in}{0.880000in}}{\pgfqpoint{6.160000in}{6.160000in}}%
\pgfusepath{clip}%
\pgfsetbuttcap%
\pgfsetroundjoin%
\definecolor{currentfill}{rgb}{0.343278,0.459354,0.884122}%
\pgfsetfillcolor{currentfill}%
\pgfsetlinewidth{0.000000pt}%
\definecolor{currentstroke}{rgb}{0.000000,0.000000,0.000000}%
\pgfsetstrokecolor{currentstroke}%
\pgfsetdash{}{0pt}%
\pgfpathmoveto{\pgfqpoint{5.393693in}{3.200233in}}%
\pgfpathlineto{\pgfqpoint{5.402650in}{3.074049in}}%
\pgfpathlineto{\pgfqpoint{5.415259in}{3.244062in}}%
\pgfpathlineto{\pgfqpoint{5.447066in}{3.169802in}}%
\pgfpathlineto{\pgfqpoint{5.481775in}{3.323956in}}%
\pgfpathlineto{\pgfqpoint{5.468153in}{3.089502in}}%
\pgfpathlineto{\pgfqpoint{5.459040in}{3.202061in}}%
\pgfpathlineto{\pgfqpoint{5.426213in}{3.188089in}}%
\pgfpathlineto{\pgfqpoint{5.393693in}{3.200233in}}%
\pgfpathclose%
\pgfusepath{fill}%
\end{pgfscope}%
\begin{pgfscope}%
\pgfpathrectangle{\pgfqpoint{1.020000in}{0.880000in}}{\pgfqpoint{6.160000in}{6.160000in}}%
\pgfusepath{clip}%
\pgfsetbuttcap%
\pgfsetroundjoin%
\definecolor{currentfill}{rgb}{0.852378,0.346492,0.280346}%
\pgfsetfillcolor{currentfill}%
\pgfsetlinewidth{0.000000pt}%
\definecolor{currentstroke}{rgb}{0.000000,0.000000,0.000000}%
\pgfsetstrokecolor{currentstroke}%
\pgfsetdash{}{0pt}%
\pgfpathmoveto{\pgfqpoint{2.732007in}{5.070528in}}%
\pgfpathlineto{\pgfqpoint{2.738887in}{5.129385in}}%
\pgfpathlineto{\pgfqpoint{2.744772in}{5.260401in}}%
\pgfpathlineto{\pgfqpoint{2.780074in}{5.133379in}}%
\pgfpathlineto{\pgfqpoint{2.811211in}{5.307976in}}%
\pgfpathlineto{\pgfqpoint{2.806905in}{5.048918in}}%
\pgfpathlineto{\pgfqpoint{2.797771in}{5.148431in}}%
\pgfpathlineto{\pgfqpoint{2.767186in}{4.944595in}}%
\pgfpathlineto{\pgfqpoint{2.732007in}{5.070528in}}%
\pgfpathclose%
\pgfusepath{fill}%
\end{pgfscope}%
\begin{pgfscope}%
\pgfpathrectangle{\pgfqpoint{1.020000in}{0.880000in}}{\pgfqpoint{6.160000in}{6.160000in}}%
\pgfusepath{clip}%
\pgfsetbuttcap%
\pgfsetroundjoin%
\definecolor{currentfill}{rgb}{0.368507,0.491141,0.905243}%
\pgfsetfillcolor{currentfill}%
\pgfsetlinewidth{0.000000pt}%
\definecolor{currentstroke}{rgb}{0.000000,0.000000,0.000000}%
\pgfsetstrokecolor{currentstroke}%
\pgfsetdash{}{0pt}%
\pgfpathmoveto{\pgfqpoint{4.677293in}{3.168298in}}%
\pgfpathlineto{\pgfqpoint{4.687076in}{3.177078in}}%
\pgfpathlineto{\pgfqpoint{4.697611in}{3.327642in}}%
\pgfpathlineto{\pgfqpoint{4.729948in}{3.211728in}}%
\pgfpathlineto{\pgfqpoint{4.762718in}{3.186676in}}%
\pgfpathlineto{\pgfqpoint{4.752728in}{3.157635in}}%
\pgfpathlineto{\pgfqpoint{4.744177in}{3.380181in}}%
\pgfpathlineto{\pgfqpoint{4.710535in}{3.241221in}}%
\pgfpathlineto{\pgfqpoint{4.677293in}{3.168298in}}%
\pgfpathclose%
\pgfusepath{fill}%
\end{pgfscope}%
\begin{pgfscope}%
\pgfpathrectangle{\pgfqpoint{1.020000in}{0.880000in}}{\pgfqpoint{6.160000in}{6.160000in}}%
\pgfusepath{clip}%
\pgfsetbuttcap%
\pgfsetroundjoin%
\definecolor{currentfill}{rgb}{0.825294,0.295749,0.250025}%
\pgfsetfillcolor{currentfill}%
\pgfsetlinewidth{0.000000pt}%
\definecolor{currentstroke}{rgb}{0.000000,0.000000,0.000000}%
\pgfsetstrokecolor{currentstroke}%
\pgfsetdash{}{0pt}%
\pgfpathmoveto{\pgfqpoint{2.797771in}{5.148431in}}%
\pgfpathlineto{\pgfqpoint{2.806905in}{5.048918in}}%
\pgfpathlineto{\pgfqpoint{2.811211in}{5.307976in}}%
\pgfpathlineto{\pgfqpoint{2.846586in}{5.167273in}}%
\pgfpathlineto{\pgfqpoint{2.877310in}{5.380874in}}%
\pgfpathlineto{\pgfqpoint{2.872360in}{5.153118in}}%
\pgfpathlineto{\pgfqpoint{2.864643in}{5.142830in}}%
\pgfpathlineto{\pgfqpoint{2.830543in}{5.196350in}}%
\pgfpathlineto{\pgfqpoint{2.797771in}{5.148431in}}%
\pgfpathclose%
\pgfusepath{fill}%
\end{pgfscope}%
\begin{pgfscope}%
\pgfpathrectangle{\pgfqpoint{1.020000in}{0.880000in}}{\pgfqpoint{6.160000in}{6.160000in}}%
\pgfusepath{clip}%
\pgfsetbuttcap%
\pgfsetroundjoin%
\definecolor{currentfill}{rgb}{0.743754,0.825125,0.965798}%
\pgfsetfillcolor{currentfill}%
\pgfsetlinewidth{0.000000pt}%
\definecolor{currentstroke}{rgb}{0.000000,0.000000,0.000000}%
\pgfsetstrokecolor{currentstroke}%
\pgfsetdash{}{0pt}%
\pgfpathmoveto{\pgfqpoint{3.906166in}{3.991462in}}%
\pgfpathlineto{\pgfqpoint{3.914815in}{4.100774in}}%
\pgfpathlineto{\pgfqpoint{3.924495in}{3.824847in}}%
\pgfpathlineto{\pgfqpoint{3.957524in}{3.874401in}}%
\pgfpathlineto{\pgfqpoint{3.990784in}{3.810571in}}%
\pgfpathlineto{\pgfqpoint{3.981571in}{3.913388in}}%
\pgfpathlineto{\pgfqpoint{3.972833in}{3.789685in}}%
\pgfpathlineto{\pgfqpoint{3.939311in}{3.985937in}}%
\pgfpathlineto{\pgfqpoint{3.906166in}{3.991462in}}%
\pgfpathclose%
\pgfusepath{fill}%
\end{pgfscope}%
\begin{pgfscope}%
\pgfpathrectangle{\pgfqpoint{1.020000in}{0.880000in}}{\pgfqpoint{6.160000in}{6.160000in}}%
\pgfusepath{clip}%
\pgfsetbuttcap%
\pgfsetroundjoin%
\definecolor{currentfill}{rgb}{0.323718,0.433158,0.864722}%
\pgfsetfillcolor{currentfill}%
\pgfsetlinewidth{0.000000pt}%
\definecolor{currentstroke}{rgb}{0.000000,0.000000,0.000000}%
\pgfsetstrokecolor{currentstroke}%
\pgfsetdash{}{0pt}%
\pgfpathmoveto{\pgfqpoint{4.828278in}{3.156381in}}%
\pgfpathlineto{\pgfqpoint{4.838905in}{3.266177in}}%
\pgfpathlineto{\pgfqpoint{4.848308in}{3.186078in}}%
\pgfpathlineto{\pgfqpoint{4.880501in}{3.083291in}}%
\pgfpathlineto{\pgfqpoint{4.913294in}{3.074277in}}%
\pgfpathlineto{\pgfqpoint{4.904198in}{3.196833in}}%
\pgfpathlineto{\pgfqpoint{4.893583in}{3.106698in}}%
\pgfpathlineto{\pgfqpoint{4.860576in}{3.076768in}}%
\pgfpathlineto{\pgfqpoint{4.828278in}{3.156381in}}%
\pgfpathclose%
\pgfusepath{fill}%
\end{pgfscope}%
\begin{pgfscope}%
\pgfpathrectangle{\pgfqpoint{1.020000in}{0.880000in}}{\pgfqpoint{6.160000in}{6.160000in}}%
\pgfusepath{clip}%
\pgfsetbuttcap%
\pgfsetroundjoin%
\definecolor{currentfill}{rgb}{0.363461,0.484784,0.901019}%
\pgfsetfillcolor{currentfill}%
\pgfsetlinewidth{0.000000pt}%
\definecolor{currentstroke}{rgb}{0.000000,0.000000,0.000000}%
\pgfsetstrokecolor{currentstroke}%
\pgfsetdash{}{0pt}%
\pgfpathmoveto{\pgfqpoint{5.676553in}{3.233718in}}%
\pgfpathlineto{\pgfqpoint{5.689449in}{3.372730in}}%
\pgfpathlineto{\pgfqpoint{5.696906in}{3.150050in}}%
\pgfpathlineto{\pgfqpoint{5.731536in}{3.279239in}}%
\pgfpathlineto{\pgfqpoint{5.761682in}{3.121111in}}%
\pgfpathlineto{\pgfqpoint{5.751445in}{3.161588in}}%
\pgfpathlineto{\pgfqpoint{5.740377in}{3.147878in}}%
\pgfpathlineto{\pgfqpoint{5.710369in}{3.313782in}}%
\pgfpathlineto{\pgfqpoint{5.676553in}{3.233718in}}%
\pgfpathclose%
\pgfusepath{fill}%
\end{pgfscope}%
\begin{pgfscope}%
\pgfpathrectangle{\pgfqpoint{1.020000in}{0.880000in}}{\pgfqpoint{6.160000in}{6.160000in}}%
\pgfusepath{clip}%
\pgfsetbuttcap%
\pgfsetroundjoin%
\definecolor{currentfill}{rgb}{0.252663,0.332837,0.783665}%
\pgfsetfillcolor{currentfill}%
\pgfsetlinewidth{0.000000pt}%
\definecolor{currentstroke}{rgb}{0.000000,0.000000,0.000000}%
\pgfsetstrokecolor{currentstroke}%
\pgfsetdash{}{0pt}%
\pgfpathmoveto{\pgfqpoint{4.913294in}{3.074277in}}%
\pgfpathlineto{\pgfqpoint{4.923257in}{3.068912in}}%
\pgfpathlineto{\pgfqpoint{4.932870in}{3.014400in}}%
\pgfpathlineto{\pgfqpoint{4.965570in}{2.991454in}}%
\pgfpathlineto{\pgfqpoint{4.998508in}{3.001726in}}%
\pgfpathlineto{\pgfqpoint{4.988914in}{3.061899in}}%
\pgfpathlineto{\pgfqpoint{4.978544in}{3.023659in}}%
\pgfpathlineto{\pgfqpoint{4.944026in}{2.799399in}}%
\pgfpathlineto{\pgfqpoint{4.913294in}{3.074277in}}%
\pgfpathclose%
\pgfusepath{fill}%
\end{pgfscope}%
\begin{pgfscope}%
\pgfpathrectangle{\pgfqpoint{1.020000in}{0.880000in}}{\pgfqpoint{6.160000in}{6.160000in}}%
\pgfusepath{clip}%
\pgfsetbuttcap%
\pgfsetroundjoin%
\definecolor{currentfill}{rgb}{0.822420,0.856898,0.910795}%
\pgfsetfillcolor{currentfill}%
\pgfsetlinewidth{0.000000pt}%
\definecolor{currentstroke}{rgb}{0.000000,0.000000,0.000000}%
\pgfsetstrokecolor{currentstroke}%
\pgfsetdash{}{0pt}%
\pgfpathmoveto{\pgfqpoint{3.822166in}{3.986514in}}%
\pgfpathlineto{\pgfqpoint{3.830407in}{4.157113in}}%
\pgfpathlineto{\pgfqpoint{3.839310in}{4.148191in}}%
\pgfpathlineto{\pgfqpoint{3.872957in}{4.010225in}}%
\pgfpathlineto{\pgfqpoint{3.906166in}{3.991462in}}%
\pgfpathlineto{\pgfqpoint{3.896972in}{4.087297in}}%
\pgfpathlineto{\pgfqpoint{3.887888in}{4.146008in}}%
\pgfpathlineto{\pgfqpoint{3.854880in}{4.106372in}}%
\pgfpathlineto{\pgfqpoint{3.822166in}{3.986514in}}%
\pgfpathclose%
\pgfusepath{fill}%
\end{pgfscope}%
\begin{pgfscope}%
\pgfpathrectangle{\pgfqpoint{1.020000in}{0.880000in}}{\pgfqpoint{6.160000in}{6.160000in}}%
\pgfusepath{clip}%
\pgfsetbuttcap%
\pgfsetroundjoin%
\definecolor{currentfill}{rgb}{0.299441,0.400248,0.839842}%
\pgfsetfillcolor{currentfill}%
\pgfsetlinewidth{0.000000pt}%
\definecolor{currentstroke}{rgb}{0.000000,0.000000,0.000000}%
\pgfsetstrokecolor{currentstroke}%
\pgfsetdash{}{0pt}%
\pgfpathmoveto{\pgfqpoint{5.261620in}{3.089726in}}%
\pgfpathlineto{\pgfqpoint{5.271718in}{3.065398in}}%
\pgfpathlineto{\pgfqpoint{5.281937in}{3.050537in}}%
\pgfpathlineto{\pgfqpoint{5.314893in}{3.067639in}}%
\pgfpathlineto{\pgfqpoint{5.348825in}{3.168826in}}%
\pgfpathlineto{\pgfqpoint{5.338102in}{3.145882in}}%
\pgfpathlineto{\pgfqpoint{5.327507in}{3.131752in}}%
\pgfpathlineto{\pgfqpoint{5.293875in}{3.048747in}}%
\pgfpathlineto{\pgfqpoint{5.261620in}{3.089726in}}%
\pgfpathclose%
\pgfusepath{fill}%
\end{pgfscope}%
\begin{pgfscope}%
\pgfpathrectangle{\pgfqpoint{1.020000in}{0.880000in}}{\pgfqpoint{6.160000in}{6.160000in}}%
\pgfusepath{clip}%
\pgfsetbuttcap%
\pgfsetroundjoin%
\definecolor{currentfill}{rgb}{0.967711,0.662973,0.544323}%
\pgfsetfillcolor{currentfill}%
\pgfsetlinewidth{0.000000pt}%
\definecolor{currentstroke}{rgb}{0.000000,0.000000,0.000000}%
\pgfsetstrokecolor{currentstroke}%
\pgfsetdash{}{0pt}%
\pgfpathmoveto{\pgfqpoint{2.203462in}{4.618649in}}%
\pgfpathlineto{\pgfqpoint{2.207969in}{4.754791in}}%
\pgfpathlineto{\pgfqpoint{2.215196in}{4.752608in}}%
\pgfpathlineto{\pgfqpoint{2.249550in}{4.717990in}}%
\pgfpathlineto{\pgfqpoint{2.286919in}{4.520370in}}%
\pgfpathlineto{\pgfqpoint{2.276473in}{4.690917in}}%
\pgfpathlineto{\pgfqpoint{2.267641in}{4.775598in}}%
\pgfpathlineto{\pgfqpoint{2.235731in}{4.686938in}}%
\pgfpathlineto{\pgfqpoint{2.203462in}{4.618649in}}%
\pgfpathclose%
\pgfusepath{fill}%
\end{pgfscope}%
\begin{pgfscope}%
\pgfpathrectangle{\pgfqpoint{1.020000in}{0.880000in}}{\pgfqpoint{6.160000in}{6.160000in}}%
\pgfusepath{clip}%
\pgfsetbuttcap%
\pgfsetroundjoin%
\definecolor{currentfill}{rgb}{0.825294,0.295749,0.250025}%
\pgfsetfillcolor{currentfill}%
\pgfsetlinewidth{0.000000pt}%
\definecolor{currentstroke}{rgb}{0.000000,0.000000,0.000000}%
\pgfsetstrokecolor{currentstroke}%
\pgfsetdash{}{0pt}%
\pgfpathmoveto{\pgfqpoint{3.013789in}{5.159179in}}%
\pgfpathlineto{\pgfqpoint{3.021611in}{5.175955in}}%
\pgfpathlineto{\pgfqpoint{3.030114in}{5.134315in}}%
\pgfpathlineto{\pgfqpoint{3.063589in}{5.129172in}}%
\pgfpathlineto{\pgfqpoint{3.094357in}{5.375237in}}%
\pgfpathlineto{\pgfqpoint{3.087671in}{5.242810in}}%
\pgfpathlineto{\pgfqpoint{3.080763in}{5.134300in}}%
\pgfpathlineto{\pgfqpoint{3.046959in}{5.176346in}}%
\pgfpathlineto{\pgfqpoint{3.013789in}{5.159179in}}%
\pgfpathclose%
\pgfusepath{fill}%
\end{pgfscope}%
\begin{pgfscope}%
\pgfpathrectangle{\pgfqpoint{1.020000in}{0.880000in}}{\pgfqpoint{6.160000in}{6.160000in}}%
\pgfusepath{clip}%
\pgfsetbuttcap%
\pgfsetroundjoin%
\definecolor{currentfill}{rgb}{0.489246,0.627536,0.976896}%
\pgfsetfillcolor{currentfill}%
\pgfsetlinewidth{0.000000pt}%
\definecolor{currentstroke}{rgb}{0.000000,0.000000,0.000000}%
\pgfsetstrokecolor{currentstroke}%
\pgfsetdash{}{0pt}%
\pgfpathmoveto{\pgfqpoint{4.376538in}{3.440219in}}%
\pgfpathlineto{\pgfqpoint{4.386098in}{3.505089in}}%
\pgfpathlineto{\pgfqpoint{4.395559in}{3.508247in}}%
\pgfpathlineto{\pgfqpoint{4.428120in}{3.304708in}}%
\pgfpathlineto{\pgfqpoint{4.461357in}{3.399434in}}%
\pgfpathlineto{\pgfqpoint{4.451780in}{3.373850in}}%
\pgfpathlineto{\pgfqpoint{4.442978in}{3.637561in}}%
\pgfpathlineto{\pgfqpoint{4.409575in}{3.467974in}}%
\pgfpathlineto{\pgfqpoint{4.376538in}{3.440219in}}%
\pgfpathclose%
\pgfusepath{fill}%
\end{pgfscope}%
\begin{pgfscope}%
\pgfpathrectangle{\pgfqpoint{1.020000in}{0.880000in}}{\pgfqpoint{6.160000in}{6.160000in}}%
\pgfusepath{clip}%
\pgfsetbuttcap%
\pgfsetroundjoin%
\definecolor{currentfill}{rgb}{0.924409,0.498590,0.389059}%
\pgfsetfillcolor{currentfill}%
\pgfsetlinewidth{0.000000pt}%
\definecolor{currentstroke}{rgb}{0.000000,0.000000,0.000000}%
\pgfsetstrokecolor{currentstroke}%
\pgfsetdash{}{0pt}%
\pgfpathmoveto{\pgfqpoint{2.465061in}{5.004014in}}%
\pgfpathlineto{\pgfqpoint{2.474488in}{4.886636in}}%
\pgfpathlineto{\pgfqpoint{2.481388in}{4.919867in}}%
\pgfpathlineto{\pgfqpoint{2.514508in}{4.948611in}}%
\pgfpathlineto{\pgfqpoint{2.546404in}{5.053276in}}%
\pgfpathlineto{\pgfqpoint{2.539984in}{4.984359in}}%
\pgfpathlineto{\pgfqpoint{2.535006in}{4.828370in}}%
\pgfpathlineto{\pgfqpoint{2.500940in}{4.863837in}}%
\pgfpathlineto{\pgfqpoint{2.465061in}{5.004014in}}%
\pgfpathclose%
\pgfusepath{fill}%
\end{pgfscope}%
\begin{pgfscope}%
\pgfpathrectangle{\pgfqpoint{1.020000in}{0.880000in}}{\pgfqpoint{6.160000in}{6.160000in}}%
\pgfusepath{clip}%
\pgfsetbuttcap%
\pgfsetroundjoin%
\definecolor{currentfill}{rgb}{0.939254,0.539581,0.423900}%
\pgfsetfillcolor{currentfill}%
\pgfsetlinewidth{0.000000pt}%
\definecolor{currentstroke}{rgb}{0.000000,0.000000,0.000000}%
\pgfsetstrokecolor{currentstroke}%
\pgfsetdash{}{0pt}%
\pgfpathmoveto{\pgfqpoint{2.402393in}{4.747987in}}%
\pgfpathlineto{\pgfqpoint{2.408178in}{4.838303in}}%
\pgfpathlineto{\pgfqpoint{2.416447in}{4.787791in}}%
\pgfpathlineto{\pgfqpoint{2.446436in}{4.998250in}}%
\pgfpathlineto{\pgfqpoint{2.481388in}{4.919867in}}%
\pgfpathlineto{\pgfqpoint{2.474488in}{4.886636in}}%
\pgfpathlineto{\pgfqpoint{2.465061in}{5.004014in}}%
\pgfpathlineto{\pgfqpoint{2.433848in}{4.867056in}}%
\pgfpathlineto{\pgfqpoint{2.402393in}{4.747987in}}%
\pgfpathclose%
\pgfusepath{fill}%
\end{pgfscope}%
\begin{pgfscope}%
\pgfpathrectangle{\pgfqpoint{1.020000in}{0.880000in}}{\pgfqpoint{6.160000in}{6.160000in}}%
\pgfusepath{clip}%
\pgfsetbuttcap%
\pgfsetroundjoin%
\definecolor{currentfill}{rgb}{0.683056,0.790043,0.989768}%
\pgfsetfillcolor{currentfill}%
\pgfsetlinewidth{0.000000pt}%
\definecolor{currentstroke}{rgb}{0.000000,0.000000,0.000000}%
\pgfsetstrokecolor{currentstroke}%
\pgfsetdash{}{0pt}%
\pgfpathmoveto{\pgfqpoint{3.990784in}{3.810571in}}%
\pgfpathlineto{\pgfqpoint{3.999908in}{3.756880in}}%
\pgfpathlineto{\pgfqpoint{4.008588in}{3.964255in}}%
\pgfpathlineto{\pgfqpoint{4.042178in}{3.696371in}}%
\pgfpathlineto{\pgfqpoint{4.075107in}{3.858675in}}%
\pgfpathlineto{\pgfqpoint{4.066249in}{3.640054in}}%
\pgfpathlineto{\pgfqpoint{4.056987in}{3.788991in}}%
\pgfpathlineto{\pgfqpoint{4.023780in}{3.874523in}}%
\pgfpathlineto{\pgfqpoint{3.990784in}{3.810571in}}%
\pgfpathclose%
\pgfusepath{fill}%
\end{pgfscope}%
\begin{pgfscope}%
\pgfpathrectangle{\pgfqpoint{1.020000in}{0.880000in}}{\pgfqpoint{6.160000in}{6.160000in}}%
\pgfusepath{clip}%
\pgfsetbuttcap%
\pgfsetroundjoin%
\definecolor{currentfill}{rgb}{0.338377,0.452819,0.879317}%
\pgfsetfillcolor{currentfill}%
\pgfsetlinewidth{0.000000pt}%
\definecolor{currentstroke}{rgb}{0.000000,0.000000,0.000000}%
\pgfsetstrokecolor{currentstroke}%
\pgfsetdash{}{0pt}%
\pgfpathmoveto{\pgfqpoint{5.327507in}{3.131752in}}%
\pgfpathlineto{\pgfqpoint{5.338102in}{3.145882in}}%
\pgfpathlineto{\pgfqpoint{5.348825in}{3.168826in}}%
\pgfpathlineto{\pgfqpoint{5.383123in}{3.296694in}}%
\pgfpathlineto{\pgfqpoint{5.415259in}{3.244062in}}%
\pgfpathlineto{\pgfqpoint{5.402650in}{3.074049in}}%
\pgfpathlineto{\pgfqpoint{5.393693in}{3.200233in}}%
\pgfpathlineto{\pgfqpoint{5.360168in}{3.129480in}}%
\pgfpathlineto{\pgfqpoint{5.327507in}{3.131752in}}%
\pgfpathclose%
\pgfusepath{fill}%
\end{pgfscope}%
\begin{pgfscope}%
\pgfpathrectangle{\pgfqpoint{1.020000in}{0.880000in}}{\pgfqpoint{6.160000in}{6.160000in}}%
\pgfusepath{clip}%
\pgfsetbuttcap%
\pgfsetroundjoin%
\definecolor{currentfill}{rgb}{0.446431,0.582356,0.957373}%
\pgfsetfillcolor{currentfill}%
\pgfsetlinewidth{0.000000pt}%
\definecolor{currentstroke}{rgb}{0.000000,0.000000,0.000000}%
\pgfsetstrokecolor{currentstroke}%
\pgfsetdash{}{0pt}%
\pgfpathmoveto{\pgfqpoint{4.527291in}{3.394236in}}%
\pgfpathlineto{\pgfqpoint{4.537210in}{3.486802in}}%
\pgfpathlineto{\pgfqpoint{4.545984in}{3.253808in}}%
\pgfpathlineto{\pgfqpoint{4.579929in}{3.501594in}}%
\pgfpathlineto{\pgfqpoint{4.612403in}{3.373450in}}%
\pgfpathlineto{\pgfqpoint{4.601776in}{3.146301in}}%
\pgfpathlineto{\pgfqpoint{4.593539in}{3.486455in}}%
\pgfpathlineto{\pgfqpoint{4.560049in}{3.344873in}}%
\pgfpathlineto{\pgfqpoint{4.527291in}{3.394236in}}%
\pgfpathclose%
\pgfusepath{fill}%
\end{pgfscope}%
\begin{pgfscope}%
\pgfpathrectangle{\pgfqpoint{1.020000in}{0.880000in}}{\pgfqpoint{6.160000in}{6.160000in}}%
\pgfusepath{clip}%
\pgfsetbuttcap%
\pgfsetroundjoin%
\definecolor{currentfill}{rgb}{0.856716,0.354704,0.285487}%
\pgfsetfillcolor{currentfill}%
\pgfsetlinewidth{0.000000pt}%
\definecolor{currentstroke}{rgb}{0.000000,0.000000,0.000000}%
\pgfsetstrokecolor{currentstroke}%
\pgfsetdash{}{0pt}%
\pgfpathmoveto{\pgfqpoint{3.162672in}{5.233881in}}%
\pgfpathlineto{\pgfqpoint{3.170459in}{5.273517in}}%
\pgfpathlineto{\pgfqpoint{3.179821in}{5.155954in}}%
\pgfpathlineto{\pgfqpoint{3.215028in}{4.967690in}}%
\pgfpathlineto{\pgfqpoint{3.247616in}{5.048387in}}%
\pgfpathlineto{\pgfqpoint{3.238691in}{5.121355in}}%
\pgfpathlineto{\pgfqpoint{3.230351in}{5.132404in}}%
\pgfpathlineto{\pgfqpoint{3.198209in}{5.010795in}}%
\pgfpathlineto{\pgfqpoint{3.162672in}{5.233881in}}%
\pgfpathclose%
\pgfusepath{fill}%
\end{pgfscope}%
\begin{pgfscope}%
\pgfpathrectangle{\pgfqpoint{1.020000in}{0.880000in}}{\pgfqpoint{6.160000in}{6.160000in}}%
\pgfusepath{clip}%
\pgfsetbuttcap%
\pgfsetroundjoin%
\definecolor{currentfill}{rgb}{0.899534,0.440692,0.344107}%
\pgfsetfillcolor{currentfill}%
\pgfsetlinewidth{0.000000pt}%
\definecolor{currentstroke}{rgb}{0.000000,0.000000,0.000000}%
\pgfsetstrokecolor{currentstroke}%
\pgfsetdash{}{0pt}%
\pgfpathmoveto{\pgfqpoint{2.535006in}{4.828370in}}%
\pgfpathlineto{\pgfqpoint{2.539984in}{4.984359in}}%
\pgfpathlineto{\pgfqpoint{2.546404in}{5.053276in}}%
\pgfpathlineto{\pgfqpoint{2.580204in}{5.040465in}}%
\pgfpathlineto{\pgfqpoint{2.612932in}{5.095231in}}%
\pgfpathlineto{\pgfqpoint{2.607005in}{4.987212in}}%
\pgfpathlineto{\pgfqpoint{2.599597in}{4.976246in}}%
\pgfpathlineto{\pgfqpoint{2.563688in}{5.128303in}}%
\pgfpathlineto{\pgfqpoint{2.535006in}{4.828370in}}%
\pgfpathclose%
\pgfusepath{fill}%
\end{pgfscope}%
\begin{pgfscope}%
\pgfpathrectangle{\pgfqpoint{1.020000in}{0.880000in}}{\pgfqpoint{6.160000in}{6.160000in}}%
\pgfusepath{clip}%
\pgfsetbuttcap%
\pgfsetroundjoin%
\definecolor{currentfill}{rgb}{0.877149,0.394645,0.311724}%
\pgfsetfillcolor{currentfill}%
\pgfsetlinewidth{0.000000pt}%
\definecolor{currentstroke}{rgb}{0.000000,0.000000,0.000000}%
\pgfsetstrokecolor{currentstroke}%
\pgfsetdash{}{0pt}%
\pgfpathmoveto{\pgfqpoint{2.599597in}{4.976246in}}%
\pgfpathlineto{\pgfqpoint{2.607005in}{4.987212in}}%
\pgfpathlineto{\pgfqpoint{2.612932in}{5.095231in}}%
\pgfpathlineto{\pgfqpoint{2.646571in}{5.090838in}}%
\pgfpathlineto{\pgfqpoint{2.678853in}{5.176709in}}%
\pgfpathlineto{\pgfqpoint{2.672151in}{5.111274in}}%
\pgfpathlineto{\pgfqpoint{2.664373in}{5.119599in}}%
\pgfpathlineto{\pgfqpoint{2.632484in}{5.014309in}}%
\pgfpathlineto{\pgfqpoint{2.599597in}{4.976246in}}%
\pgfpathclose%
\pgfusepath{fill}%
\end{pgfscope}%
\begin{pgfscope}%
\pgfpathrectangle{\pgfqpoint{1.020000in}{0.880000in}}{\pgfqpoint{6.160000in}{6.160000in}}%
\pgfusepath{clip}%
\pgfsetbuttcap%
\pgfsetroundjoin%
\definecolor{currentfill}{rgb}{0.289996,0.386836,0.828926}%
\pgfsetfillcolor{currentfill}%
\pgfsetlinewidth{0.000000pt}%
\definecolor{currentstroke}{rgb}{0.000000,0.000000,0.000000}%
\pgfsetstrokecolor{currentstroke}%
\pgfsetdash{}{0pt}%
\pgfpathmoveto{\pgfqpoint{5.197518in}{3.225472in}}%
\pgfpathlineto{\pgfqpoint{5.206936in}{3.138158in}}%
\pgfpathlineto{\pgfqpoint{5.215178in}{2.937059in}}%
\pgfpathlineto{\pgfqpoint{5.250092in}{3.138085in}}%
\pgfpathlineto{\pgfqpoint{5.281937in}{3.050537in}}%
\pgfpathlineto{\pgfqpoint{5.271718in}{3.065398in}}%
\pgfpathlineto{\pgfqpoint{5.261620in}{3.089726in}}%
\pgfpathlineto{\pgfqpoint{5.227583in}{2.965262in}}%
\pgfpathlineto{\pgfqpoint{5.197518in}{3.225472in}}%
\pgfpathclose%
\pgfusepath{fill}%
\end{pgfscope}%
\begin{pgfscope}%
\pgfpathrectangle{\pgfqpoint{1.020000in}{0.880000in}}{\pgfqpoint{6.160000in}{6.160000in}}%
\pgfusepath{clip}%
\pgfsetbuttcap%
\pgfsetroundjoin%
\definecolor{currentfill}{rgb}{0.839365,0.321856,0.264924}%
\pgfsetfillcolor{currentfill}%
\pgfsetlinewidth{0.000000pt}%
\definecolor{currentstroke}{rgb}{0.000000,0.000000,0.000000}%
\pgfsetstrokecolor{currentstroke}%
\pgfsetdash{}{0pt}%
\pgfpathmoveto{\pgfqpoint{2.664373in}{5.119599in}}%
\pgfpathlineto{\pgfqpoint{2.672151in}{5.111274in}}%
\pgfpathlineto{\pgfqpoint{2.678853in}{5.176709in}}%
\pgfpathlineto{\pgfqpoint{2.711537in}{5.237301in}}%
\pgfpathlineto{\pgfqpoint{2.744772in}{5.260401in}}%
\pgfpathlineto{\pgfqpoint{2.738887in}{5.129385in}}%
\pgfpathlineto{\pgfqpoint{2.732007in}{5.070528in}}%
\pgfpathlineto{\pgfqpoint{2.697570in}{5.138700in}}%
\pgfpathlineto{\pgfqpoint{2.664373in}{5.119599in}}%
\pgfpathclose%
\pgfusepath{fill}%
\end{pgfscope}%
\begin{pgfscope}%
\pgfpathrectangle{\pgfqpoint{1.020000in}{0.880000in}}{\pgfqpoint{6.160000in}{6.160000in}}%
\pgfusepath{clip}%
\pgfsetbuttcap%
\pgfsetroundjoin%
\definecolor{currentfill}{rgb}{0.373552,0.497499,0.909467}%
\pgfsetfillcolor{currentfill}%
\pgfsetlinewidth{0.000000pt}%
\definecolor{currentstroke}{rgb}{0.000000,0.000000,0.000000}%
\pgfsetstrokecolor{currentstroke}%
\pgfsetdash{}{0pt}%
\pgfpathmoveto{\pgfqpoint{5.612596in}{3.323403in}}%
\pgfpathlineto{\pgfqpoint{5.621408in}{3.189139in}}%
\pgfpathlineto{\pgfqpoint{5.633653in}{3.291724in}}%
\pgfpathlineto{\pgfqpoint{5.665782in}{3.252502in}}%
\pgfpathlineto{\pgfqpoint{5.696906in}{3.150050in}}%
\pgfpathlineto{\pgfqpoint{5.689449in}{3.372730in}}%
\pgfpathlineto{\pgfqpoint{5.676553in}{3.233718in}}%
\pgfpathlineto{\pgfqpoint{5.641719in}{3.081735in}}%
\pgfpathlineto{\pgfqpoint{5.612596in}{3.323403in}}%
\pgfpathclose%
\pgfusepath{fill}%
\end{pgfscope}%
\begin{pgfscope}%
\pgfpathrectangle{\pgfqpoint{1.020000in}{0.880000in}}{\pgfqpoint{6.160000in}{6.160000in}}%
\pgfusepath{clip}%
\pgfsetbuttcap%
\pgfsetroundjoin%
\definecolor{currentfill}{rgb}{0.902659,0.447939,0.349721}%
\pgfsetfillcolor{currentfill}%
\pgfsetlinewidth{0.000000pt}%
\definecolor{currentstroke}{rgb}{0.000000,0.000000,0.000000}%
\pgfsetstrokecolor{currentstroke}%
\pgfsetdash{}{0pt}%
\pgfpathmoveto{\pgfqpoint{3.247616in}{5.048387in}}%
\pgfpathlineto{\pgfqpoint{3.257426in}{4.879175in}}%
\pgfpathlineto{\pgfqpoint{3.263750in}{5.094087in}}%
\pgfpathlineto{\pgfqpoint{3.297739in}{5.028624in}}%
\pgfpathlineto{\pgfqpoint{3.331651in}{4.964842in}}%
\pgfpathlineto{\pgfqpoint{3.323476in}{4.945340in}}%
\pgfpathlineto{\pgfqpoint{3.315043in}{4.958510in}}%
\pgfpathlineto{\pgfqpoint{3.280760in}{5.070289in}}%
\pgfpathlineto{\pgfqpoint{3.247616in}{5.048387in}}%
\pgfpathclose%
\pgfusepath{fill}%
\end{pgfscope}%
\begin{pgfscope}%
\pgfpathrectangle{\pgfqpoint{1.020000in}{0.880000in}}{\pgfqpoint{6.160000in}{6.160000in}}%
\pgfusepath{clip}%
\pgfsetbuttcap%
\pgfsetroundjoin%
\definecolor{currentfill}{rgb}{0.353369,0.472069,0.892570}%
\pgfsetfillcolor{currentfill}%
\pgfsetlinewidth{0.000000pt}%
\definecolor{currentstroke}{rgb}{0.000000,0.000000,0.000000}%
\pgfsetstrokecolor{currentstroke}%
\pgfsetdash{}{0pt}%
\pgfpathmoveto{\pgfqpoint{5.829452in}{3.280736in}}%
\pgfpathlineto{\pgfqpoint{5.839018in}{3.195801in}}%
\pgfpathlineto{\pgfqpoint{5.850584in}{3.230324in}}%
\pgfpathlineto{\pgfqpoint{5.881964in}{3.156433in}}%
\pgfpathlineto{\pgfqpoint{5.871599in}{3.194948in}}%
\pgfpathlineto{\pgfqpoint{5.859340in}{3.119883in}}%
\pgfpathlineto{\pgfqpoint{5.829452in}{3.280736in}}%
\pgfpathclose%
\pgfusepath{fill}%
\end{pgfscope}%
\begin{pgfscope}%
\pgfpathrectangle{\pgfqpoint{1.020000in}{0.880000in}}{\pgfqpoint{6.160000in}{6.160000in}}%
\pgfusepath{clip}%
\pgfsetbuttcap%
\pgfsetroundjoin%
\definecolor{currentfill}{rgb}{0.586921,0.718121,0.998874}%
\pgfsetfillcolor{currentfill}%
\pgfsetlinewidth{0.000000pt}%
\definecolor{currentstroke}{rgb}{0.000000,0.000000,0.000000}%
\pgfsetstrokecolor{currentstroke}%
\pgfsetdash{}{0pt}%
\pgfpathmoveto{\pgfqpoint{4.226037in}{3.741552in}}%
\pgfpathlineto{\pgfqpoint{4.235223in}{3.544223in}}%
\pgfpathlineto{\pgfqpoint{4.244587in}{3.663528in}}%
\pgfpathlineto{\pgfqpoint{4.277794in}{3.750427in}}%
\pgfpathlineto{\pgfqpoint{4.310713in}{3.581659in}}%
\pgfpathlineto{\pgfqpoint{4.301173in}{3.427414in}}%
\pgfpathlineto{\pgfqpoint{4.292124in}{3.680674in}}%
\pgfpathlineto{\pgfqpoint{4.258948in}{3.510298in}}%
\pgfpathlineto{\pgfqpoint{4.226037in}{3.741552in}}%
\pgfpathclose%
\pgfusepath{fill}%
\end{pgfscope}%
\begin{pgfscope}%
\pgfpathrectangle{\pgfqpoint{1.020000in}{0.880000in}}{\pgfqpoint{6.160000in}{6.160000in}}%
\pgfusepath{clip}%
\pgfsetbuttcap%
\pgfsetroundjoin%
\definecolor{currentfill}{rgb}{0.815508,0.277781,0.240294}%
\pgfsetfillcolor{currentfill}%
\pgfsetlinewidth{0.000000pt}%
\definecolor{currentstroke}{rgb}{0.000000,0.000000,0.000000}%
\pgfsetstrokecolor{currentstroke}%
\pgfsetdash{}{0pt}%
\pgfpathmoveto{\pgfqpoint{2.946724in}{5.182606in}}%
\pgfpathlineto{\pgfqpoint{2.953650in}{5.267381in}}%
\pgfpathlineto{\pgfqpoint{2.960609in}{5.352035in}}%
\pgfpathlineto{\pgfqpoint{2.996967in}{5.109421in}}%
\pgfpathlineto{\pgfqpoint{3.030114in}{5.134315in}}%
\pgfpathlineto{\pgfqpoint{3.021611in}{5.175955in}}%
\pgfpathlineto{\pgfqpoint{3.013789in}{5.159179in}}%
\pgfpathlineto{\pgfqpoint{2.979324in}{5.250881in}}%
\pgfpathlineto{\pgfqpoint{2.946724in}{5.182606in}}%
\pgfpathclose%
\pgfusepath{fill}%
\end{pgfscope}%
\begin{pgfscope}%
\pgfpathrectangle{\pgfqpoint{1.020000in}{0.880000in}}{\pgfqpoint{6.160000in}{6.160000in}}%
\pgfusepath{clip}%
\pgfsetbuttcap%
\pgfsetroundjoin%
\definecolor{currentfill}{rgb}{0.843358,0.861820,0.890017}%
\pgfsetfillcolor{currentfill}%
\pgfsetlinewidth{0.000000pt}%
\definecolor{currentstroke}{rgb}{0.000000,0.000000,0.000000}%
\pgfsetstrokecolor{currentstroke}%
\pgfsetdash{}{0pt}%
\pgfpathmoveto{\pgfqpoint{3.755267in}{4.135226in}}%
\pgfpathlineto{\pgfqpoint{3.764719in}{3.975707in}}%
\pgfpathlineto{\pgfqpoint{3.772673in}{4.181143in}}%
\pgfpathlineto{\pgfqpoint{3.806506in}{4.031203in}}%
\pgfpathlineto{\pgfqpoint{3.839310in}{4.148191in}}%
\pgfpathlineto{\pgfqpoint{3.830407in}{4.157113in}}%
\pgfpathlineto{\pgfqpoint{3.822166in}{3.986514in}}%
\pgfpathlineto{\pgfqpoint{3.787912in}{4.274973in}}%
\pgfpathlineto{\pgfqpoint{3.755267in}{4.135226in}}%
\pgfpathclose%
\pgfusepath{fill}%
\end{pgfscope}%
\begin{pgfscope}%
\pgfpathrectangle{\pgfqpoint{1.020000in}{0.880000in}}{\pgfqpoint{6.160000in}{6.160000in}}%
\pgfusepath{clip}%
\pgfsetbuttcap%
\pgfsetroundjoin%
\definecolor{currentfill}{rgb}{0.640828,0.760752,0.997846}%
\pgfsetfillcolor{currentfill}%
\pgfsetlinewidth{0.000000pt}%
\definecolor{currentstroke}{rgb}{0.000000,0.000000,0.000000}%
\pgfsetstrokecolor{currentstroke}%
\pgfsetdash{}{0pt}%
\pgfpathmoveto{\pgfqpoint{4.075107in}{3.858675in}}%
\pgfpathlineto{\pgfqpoint{4.084347in}{3.749929in}}%
\pgfpathlineto{\pgfqpoint{4.093482in}{3.757332in}}%
\pgfpathlineto{\pgfqpoint{4.126670in}{3.724908in}}%
\pgfpathlineto{\pgfqpoint{4.159824in}{3.664255in}}%
\pgfpathlineto{\pgfqpoint{4.150601in}{3.739340in}}%
\pgfpathlineto{\pgfqpoint{4.141464in}{3.621683in}}%
\pgfpathlineto{\pgfqpoint{4.108433in}{3.608256in}}%
\pgfpathlineto{\pgfqpoint{4.075107in}{3.858675in}}%
\pgfpathclose%
\pgfusepath{fill}%
\end{pgfscope}%
\begin{pgfscope}%
\pgfpathrectangle{\pgfqpoint{1.020000in}{0.880000in}}{\pgfqpoint{6.160000in}{6.160000in}}%
\pgfusepath{clip}%
\pgfsetbuttcap%
\pgfsetroundjoin%
\definecolor{currentfill}{rgb}{0.451739,0.588181,0.960201}%
\pgfsetfillcolor{currentfill}%
\pgfsetlinewidth{0.000000pt}%
\definecolor{currentstroke}{rgb}{0.000000,0.000000,0.000000}%
\pgfsetstrokecolor{currentstroke}%
\pgfsetdash{}{0pt}%
\pgfpathmoveto{\pgfqpoint{4.461357in}{3.399434in}}%
\pgfpathlineto{\pgfqpoint{4.470753in}{3.353259in}}%
\pgfpathlineto{\pgfqpoint{4.480249in}{3.337336in}}%
\pgfpathlineto{\pgfqpoint{4.513358in}{3.361454in}}%
\pgfpathlineto{\pgfqpoint{4.545984in}{3.253808in}}%
\pgfpathlineto{\pgfqpoint{4.537210in}{3.486802in}}%
\pgfpathlineto{\pgfqpoint{4.527291in}{3.394236in}}%
\pgfpathlineto{\pgfqpoint{4.494454in}{3.435237in}}%
\pgfpathlineto{\pgfqpoint{4.461357in}{3.399434in}}%
\pgfpathclose%
\pgfusepath{fill}%
\end{pgfscope}%
\begin{pgfscope}%
\pgfpathrectangle{\pgfqpoint{1.020000in}{0.880000in}}{\pgfqpoint{6.160000in}{6.160000in}}%
\pgfusepath{clip}%
\pgfsetbuttcap%
\pgfsetroundjoin%
\definecolor{currentfill}{rgb}{0.963806,0.634188,0.513721}%
\pgfsetfillcolor{currentfill}%
\pgfsetlinewidth{0.000000pt}%
\definecolor{currentstroke}{rgb}{0.000000,0.000000,0.000000}%
\pgfsetstrokecolor{currentstroke}%
\pgfsetdash{}{0pt}%
\pgfpathmoveto{\pgfqpoint{3.416550in}{4.781789in}}%
\pgfpathlineto{\pgfqpoint{3.424977in}{4.783685in}}%
\pgfpathlineto{\pgfqpoint{3.434291in}{4.668149in}}%
\pgfpathlineto{\pgfqpoint{3.467389in}{4.700788in}}%
\pgfpathlineto{\pgfqpoint{3.500575in}{4.720796in}}%
\pgfpathlineto{\pgfqpoint{3.492930in}{4.592828in}}%
\pgfpathlineto{\pgfqpoint{3.483150in}{4.779664in}}%
\pgfpathlineto{\pgfqpoint{3.449731in}{4.798538in}}%
\pgfpathlineto{\pgfqpoint{3.416550in}{4.781789in}}%
\pgfpathclose%
\pgfusepath{fill}%
\end{pgfscope}%
\begin{pgfscope}%
\pgfpathrectangle{\pgfqpoint{1.020000in}{0.880000in}}{\pgfqpoint{6.160000in}{6.160000in}}%
\pgfusepath{clip}%
\pgfsetbuttcap%
\pgfsetroundjoin%
\definecolor{currentfill}{rgb}{0.358415,0.478426,0.896795}%
\pgfsetfillcolor{currentfill}%
\pgfsetlinewidth{0.000000pt}%
\definecolor{currentstroke}{rgb}{0.000000,0.000000,0.000000}%
\pgfsetstrokecolor{currentstroke}%
\pgfsetdash{}{0pt}%
\pgfpathmoveto{\pgfqpoint{4.762718in}{3.186676in}}%
\pgfpathlineto{\pgfqpoint{4.772437in}{3.165341in}}%
\pgfpathlineto{\pgfqpoint{4.782081in}{3.128786in}}%
\pgfpathlineto{\pgfqpoint{4.815531in}{3.210552in}}%
\pgfpathlineto{\pgfqpoint{4.848308in}{3.186078in}}%
\pgfpathlineto{\pgfqpoint{4.838905in}{3.266177in}}%
\pgfpathlineto{\pgfqpoint{4.828278in}{3.156381in}}%
\pgfpathlineto{\pgfqpoint{4.796272in}{3.295201in}}%
\pgfpathlineto{\pgfqpoint{4.762718in}{3.186676in}}%
\pgfpathclose%
\pgfusepath{fill}%
\end{pgfscope}%
\begin{pgfscope}%
\pgfpathrectangle{\pgfqpoint{1.020000in}{0.880000in}}{\pgfqpoint{6.160000in}{6.160000in}}%
\pgfusepath{clip}%
\pgfsetbuttcap%
\pgfsetroundjoin%
\definecolor{currentfill}{rgb}{0.909460,0.839386,0.800331}%
\pgfsetfillcolor{currentfill}%
\pgfsetlinewidth{0.000000pt}%
\definecolor{currentstroke}{rgb}{0.000000,0.000000,0.000000}%
\pgfsetstrokecolor{currentstroke}%
\pgfsetdash{}{0pt}%
\pgfpathmoveto{\pgfqpoint{3.669750in}{4.466614in}}%
\pgfpathlineto{\pgfqpoint{3.679104in}{4.338670in}}%
\pgfpathlineto{\pgfqpoint{3.687610in}{4.382684in}}%
\pgfpathlineto{\pgfqpoint{3.721601in}{4.234584in}}%
\pgfpathlineto{\pgfqpoint{3.755267in}{4.135226in}}%
\pgfpathlineto{\pgfqpoint{3.746544in}{4.124587in}}%
\pgfpathlineto{\pgfqpoint{3.737693in}{4.147627in}}%
\pgfpathlineto{\pgfqpoint{3.703551in}{4.355986in}}%
\pgfpathlineto{\pgfqpoint{3.669750in}{4.466614in}}%
\pgfpathclose%
\pgfusepath{fill}%
\end{pgfscope}%
\begin{pgfscope}%
\pgfpathrectangle{\pgfqpoint{1.020000in}{0.880000in}}{\pgfqpoint{6.160000in}{6.160000in}}%
\pgfusepath{clip}%
\pgfsetbuttcap%
\pgfsetroundjoin%
\definecolor{currentfill}{rgb}{0.968533,0.715841,0.606097}%
\pgfsetfillcolor{currentfill}%
\pgfsetlinewidth{0.000000pt}%
\definecolor{currentstroke}{rgb}{0.000000,0.000000,0.000000}%
\pgfsetstrokecolor{currentstroke}%
\pgfsetdash{}{0pt}%
\pgfpathmoveto{\pgfqpoint{3.500575in}{4.720796in}}%
\pgfpathlineto{\pgfqpoint{3.510708in}{4.480860in}}%
\pgfpathlineto{\pgfqpoint{3.518179in}{4.642566in}}%
\pgfpathlineto{\pgfqpoint{3.551539in}{4.640338in}}%
\pgfpathlineto{\pgfqpoint{3.585613in}{4.513928in}}%
\pgfpathlineto{\pgfqpoint{3.576601in}{4.582667in}}%
\pgfpathlineto{\pgfqpoint{3.568879in}{4.440165in}}%
\pgfpathlineto{\pgfqpoint{3.534551in}{4.616598in}}%
\pgfpathlineto{\pgfqpoint{3.500575in}{4.720796in}}%
\pgfpathclose%
\pgfusepath{fill}%
\end{pgfscope}%
\begin{pgfscope}%
\pgfpathrectangle{\pgfqpoint{1.020000in}{0.880000in}}{\pgfqpoint{6.160000in}{6.160000in}}%
\pgfusepath{clip}%
\pgfsetbuttcap%
\pgfsetroundjoin%
\definecolor{currentfill}{rgb}{0.294718,0.393542,0.834384}%
\pgfsetfillcolor{currentfill}%
\pgfsetlinewidth{0.000000pt}%
\definecolor{currentstroke}{rgb}{0.000000,0.000000,0.000000}%
\pgfsetstrokecolor{currentstroke}%
\pgfsetdash{}{0pt}%
\pgfpathmoveto{\pgfqpoint{5.130305in}{3.060878in}}%
\pgfpathlineto{\pgfqpoint{5.140843in}{3.092528in}}%
\pgfpathlineto{\pgfqpoint{5.150724in}{3.053525in}}%
\pgfpathlineto{\pgfqpoint{5.183248in}{3.022029in}}%
\pgfpathlineto{\pgfqpoint{5.215178in}{2.937059in}}%
\pgfpathlineto{\pgfqpoint{5.206936in}{3.138158in}}%
\pgfpathlineto{\pgfqpoint{5.197518in}{3.225472in}}%
\pgfpathlineto{\pgfqpoint{5.163186in}{3.071247in}}%
\pgfpathlineto{\pgfqpoint{5.130305in}{3.060878in}}%
\pgfpathclose%
\pgfusepath{fill}%
\end{pgfscope}%
\begin{pgfscope}%
\pgfpathrectangle{\pgfqpoint{1.020000in}{0.880000in}}{\pgfqpoint{6.160000in}{6.160000in}}%
\pgfusepath{clip}%
\pgfsetbuttcap%
\pgfsetroundjoin%
\definecolor{currentfill}{rgb}{0.708720,0.805721,0.981117}%
\pgfsetfillcolor{currentfill}%
\pgfsetlinewidth{0.000000pt}%
\definecolor{currentstroke}{rgb}{0.000000,0.000000,0.000000}%
\pgfsetstrokecolor{currentstroke}%
\pgfsetdash{}{0pt}%
\pgfpathmoveto{\pgfqpoint{3.924495in}{3.824847in}}%
\pgfpathlineto{\pgfqpoint{3.933586in}{3.773499in}}%
\pgfpathlineto{\pgfqpoint{3.942355in}{3.860192in}}%
\pgfpathlineto{\pgfqpoint{3.975710in}{3.793401in}}%
\pgfpathlineto{\pgfqpoint{4.008588in}{3.964255in}}%
\pgfpathlineto{\pgfqpoint{3.999908in}{3.756880in}}%
\pgfpathlineto{\pgfqpoint{3.990784in}{3.810571in}}%
\pgfpathlineto{\pgfqpoint{3.957524in}{3.874401in}}%
\pgfpathlineto{\pgfqpoint{3.924495in}{3.824847in}}%
\pgfpathclose%
\pgfusepath{fill}%
\end{pgfscope}%
\begin{pgfscope}%
\pgfpathrectangle{\pgfqpoint{1.020000in}{0.880000in}}{\pgfqpoint{6.160000in}{6.160000in}}%
\pgfusepath{clip}%
\pgfsetbuttcap%
\pgfsetroundjoin%
\definecolor{currentfill}{rgb}{0.931831,0.519086,0.406480}%
\pgfsetfillcolor{currentfill}%
\pgfsetlinewidth{0.000000pt}%
\definecolor{currentstroke}{rgb}{0.000000,0.000000,0.000000}%
\pgfsetstrokecolor{currentstroke}%
\pgfsetdash{}{0pt}%
\pgfpathmoveto{\pgfqpoint{3.331651in}{4.964842in}}%
\pgfpathlineto{\pgfqpoint{3.339674in}{5.005004in}}%
\pgfpathlineto{\pgfqpoint{3.349243in}{4.860143in}}%
\pgfpathlineto{\pgfqpoint{3.382360in}{4.891760in}}%
\pgfpathlineto{\pgfqpoint{3.416550in}{4.781789in}}%
\pgfpathlineto{\pgfqpoint{3.407538in}{4.858752in}}%
\pgfpathlineto{\pgfqpoint{3.398579in}{4.929369in}}%
\pgfpathlineto{\pgfqpoint{3.365246in}{4.932801in}}%
\pgfpathlineto{\pgfqpoint{3.331651in}{4.964842in}}%
\pgfpathclose%
\pgfusepath{fill}%
\end{pgfscope}%
\begin{pgfscope}%
\pgfpathrectangle{\pgfqpoint{1.020000in}{0.880000in}}{\pgfqpoint{6.160000in}{6.160000in}}%
\pgfusepath{clip}%
\pgfsetbuttcap%
\pgfsetroundjoin%
\definecolor{currentfill}{rgb}{0.527132,0.664700,0.989065}%
\pgfsetfillcolor{currentfill}%
\pgfsetlinewidth{0.000000pt}%
\definecolor{currentstroke}{rgb}{0.000000,0.000000,0.000000}%
\pgfsetstrokecolor{currentstroke}%
\pgfsetdash{}{0pt}%
\pgfpathmoveto{\pgfqpoint{4.310713in}{3.581659in}}%
\pgfpathlineto{\pgfqpoint{4.320103in}{3.590341in}}%
\pgfpathlineto{\pgfqpoint{4.329411in}{3.528778in}}%
\pgfpathlineto{\pgfqpoint{4.362087in}{3.288700in}}%
\pgfpathlineto{\pgfqpoint{4.395559in}{3.508247in}}%
\pgfpathlineto{\pgfqpoint{4.386098in}{3.505089in}}%
\pgfpathlineto{\pgfqpoint{4.376538in}{3.440219in}}%
\pgfpathlineto{\pgfqpoint{4.343793in}{3.594531in}}%
\pgfpathlineto{\pgfqpoint{4.310713in}{3.581659in}}%
\pgfpathclose%
\pgfusepath{fill}%
\end{pgfscope}%
\begin{pgfscope}%
\pgfpathrectangle{\pgfqpoint{1.020000in}{0.880000in}}{\pgfqpoint{6.160000in}{6.160000in}}%
\pgfusepath{clip}%
\pgfsetbuttcap%
\pgfsetroundjoin%
\definecolor{currentfill}{rgb}{0.363461,0.484784,0.901019}%
\pgfsetfillcolor{currentfill}%
\pgfsetlinewidth{0.000000pt}%
\definecolor{currentstroke}{rgb}{0.000000,0.000000,0.000000}%
\pgfsetstrokecolor{currentstroke}%
\pgfsetdash{}{0pt}%
\pgfpathmoveto{\pgfqpoint{5.761682in}{3.121111in}}%
\pgfpathlineto{\pgfqpoint{5.775366in}{3.295967in}}%
\pgfpathlineto{\pgfqpoint{5.784076in}{3.157209in}}%
\pgfpathlineto{\pgfqpoint{5.815759in}{3.097955in}}%
\pgfpathlineto{\pgfqpoint{5.850584in}{3.230324in}}%
\pgfpathlineto{\pgfqpoint{5.839018in}{3.195801in}}%
\pgfpathlineto{\pgfqpoint{5.829452in}{3.280736in}}%
\pgfpathlineto{\pgfqpoint{5.796910in}{3.285048in}}%
\pgfpathlineto{\pgfqpoint{5.761682in}{3.121111in}}%
\pgfpathclose%
\pgfusepath{fill}%
\end{pgfscope}%
\begin{pgfscope}%
\pgfpathrectangle{\pgfqpoint{1.020000in}{0.880000in}}{\pgfqpoint{6.160000in}{6.160000in}}%
\pgfusepath{clip}%
\pgfsetbuttcap%
\pgfsetroundjoin%
\definecolor{currentfill}{rgb}{0.425199,0.559058,0.946061}%
\pgfsetfillcolor{currentfill}%
\pgfsetlinewidth{0.000000pt}%
\definecolor{currentstroke}{rgb}{0.000000,0.000000,0.000000}%
\pgfsetstrokecolor{currentstroke}%
\pgfsetdash{}{0pt}%
\pgfpathmoveto{\pgfqpoint{4.612403in}{3.373450in}}%
\pgfpathlineto{\pgfqpoint{4.621922in}{3.333466in}}%
\pgfpathlineto{\pgfqpoint{4.632105in}{3.438809in}}%
\pgfpathlineto{\pgfqpoint{4.664710in}{3.346652in}}%
\pgfpathlineto{\pgfqpoint{4.697611in}{3.327642in}}%
\pgfpathlineto{\pgfqpoint{4.687076in}{3.177078in}}%
\pgfpathlineto{\pgfqpoint{4.677293in}{3.168298in}}%
\pgfpathlineto{\pgfqpoint{4.645509in}{3.404390in}}%
\pgfpathlineto{\pgfqpoint{4.612403in}{3.373450in}}%
\pgfpathclose%
\pgfusepath{fill}%
\end{pgfscope}%
\begin{pgfscope}%
\pgfpathrectangle{\pgfqpoint{1.020000in}{0.880000in}}{\pgfqpoint{6.160000in}{6.160000in}}%
\pgfusepath{clip}%
\pgfsetbuttcap%
\pgfsetroundjoin%
\definecolor{currentfill}{rgb}{0.280550,0.373423,0.818011}%
\pgfsetfillcolor{currentfill}%
\pgfsetlinewidth{0.000000pt}%
\definecolor{currentstroke}{rgb}{0.000000,0.000000,0.000000}%
\pgfsetstrokecolor{currentstroke}%
\pgfsetdash{}{0pt}%
\pgfpathmoveto{\pgfqpoint{5.063886in}{2.970334in}}%
\pgfpathlineto{\pgfqpoint{5.074361in}{3.005318in}}%
\pgfpathlineto{\pgfqpoint{5.085389in}{3.098282in}}%
\pgfpathlineto{\pgfqpoint{5.117836in}{3.050555in}}%
\pgfpathlineto{\pgfqpoint{5.150724in}{3.053525in}}%
\pgfpathlineto{\pgfqpoint{5.140843in}{3.092528in}}%
\pgfpathlineto{\pgfqpoint{5.130305in}{3.060878in}}%
\pgfpathlineto{\pgfqpoint{5.097053in}{3.011530in}}%
\pgfpathlineto{\pgfqpoint{5.063886in}{2.970334in}}%
\pgfpathclose%
\pgfusepath{fill}%
\end{pgfscope}%
\begin{pgfscope}%
\pgfpathrectangle{\pgfqpoint{1.020000in}{0.880000in}}{\pgfqpoint{6.160000in}{6.160000in}}%
\pgfusepath{clip}%
\pgfsetbuttcap%
\pgfsetroundjoin%
\definecolor{currentfill}{rgb}{0.383662,0.510183,0.917831}%
\pgfsetfillcolor{currentfill}%
\pgfsetlinewidth{0.000000pt}%
\definecolor{currentstroke}{rgb}{0.000000,0.000000,0.000000}%
\pgfsetstrokecolor{currentstroke}%
\pgfsetdash{}{0pt}%
\pgfpathmoveto{\pgfqpoint{5.545639in}{3.207573in}}%
\pgfpathlineto{\pgfqpoint{5.557343in}{3.282520in}}%
\pgfpathlineto{\pgfqpoint{5.564912in}{3.056598in}}%
\pgfpathlineto{\pgfqpoint{5.601828in}{3.355517in}}%
\pgfpathlineto{\pgfqpoint{5.633653in}{3.291724in}}%
\pgfpathlineto{\pgfqpoint{5.621408in}{3.189139in}}%
\pgfpathlineto{\pgfqpoint{5.612596in}{3.323403in}}%
\pgfpathlineto{\pgfqpoint{5.579033in}{3.260113in}}%
\pgfpathlineto{\pgfqpoint{5.545639in}{3.207573in}}%
\pgfpathclose%
\pgfusepath{fill}%
\end{pgfscope}%
\begin{pgfscope}%
\pgfpathrectangle{\pgfqpoint{1.020000in}{0.880000in}}{\pgfqpoint{6.160000in}{6.160000in}}%
\pgfusepath{clip}%
\pgfsetbuttcap%
\pgfsetroundjoin%
\definecolor{currentfill}{rgb}{0.358415,0.478426,0.896795}%
\pgfsetfillcolor{currentfill}%
\pgfsetlinewidth{0.000000pt}%
\definecolor{currentstroke}{rgb}{0.000000,0.000000,0.000000}%
\pgfsetstrokecolor{currentstroke}%
\pgfsetdash{}{0pt}%
\pgfpathmoveto{\pgfqpoint{5.481775in}{3.323956in}}%
\pgfpathlineto{\pgfqpoint{5.488475in}{3.024678in}}%
\pgfpathlineto{\pgfqpoint{5.501925in}{3.239455in}}%
\pgfpathlineto{\pgfqpoint{5.533751in}{3.169704in}}%
\pgfpathlineto{\pgfqpoint{5.564912in}{3.056598in}}%
\pgfpathlineto{\pgfqpoint{5.557343in}{3.282520in}}%
\pgfpathlineto{\pgfqpoint{5.545639in}{3.207573in}}%
\pgfpathlineto{\pgfqpoint{5.513322in}{3.234695in}}%
\pgfpathlineto{\pgfqpoint{5.481775in}{3.323956in}}%
\pgfpathclose%
\pgfusepath{fill}%
\end{pgfscope}%
\begin{pgfscope}%
\pgfpathrectangle{\pgfqpoint{1.020000in}{0.880000in}}{\pgfqpoint{6.160000in}{6.160000in}}%
\pgfusepath{clip}%
\pgfsetbuttcap%
\pgfsetroundjoin%
\definecolor{currentfill}{rgb}{0.592356,0.722792,0.999434}%
\pgfsetfillcolor{currentfill}%
\pgfsetlinewidth{0.000000pt}%
\definecolor{currentstroke}{rgb}{0.000000,0.000000,0.000000}%
\pgfsetstrokecolor{currentstroke}%
\pgfsetdash{}{0pt}%
\pgfpathmoveto{\pgfqpoint{4.159824in}{3.664255in}}%
\pgfpathlineto{\pgfqpoint{4.169058in}{3.550288in}}%
\pgfpathlineto{\pgfqpoint{4.178278in}{3.684413in}}%
\pgfpathlineto{\pgfqpoint{4.211395in}{3.495047in}}%
\pgfpathlineto{\pgfqpoint{4.244587in}{3.663528in}}%
\pgfpathlineto{\pgfqpoint{4.235223in}{3.544223in}}%
\pgfpathlineto{\pgfqpoint{4.226037in}{3.741552in}}%
\pgfpathlineto{\pgfqpoint{4.192918in}{3.603864in}}%
\pgfpathlineto{\pgfqpoint{4.159824in}{3.664255in}}%
\pgfpathclose%
\pgfusepath{fill}%
\end{pgfscope}%
\begin{pgfscope}%
\pgfpathrectangle{\pgfqpoint{1.020000in}{0.880000in}}{\pgfqpoint{6.160000in}{6.160000in}}%
\pgfusepath{clip}%
\pgfsetbuttcap%
\pgfsetroundjoin%
\definecolor{currentfill}{rgb}{0.333490,0.446265,0.874452}%
\pgfsetfillcolor{currentfill}%
\pgfsetlinewidth{0.000000pt}%
\definecolor{currentstroke}{rgb}{0.000000,0.000000,0.000000}%
\pgfsetstrokecolor{currentstroke}%
\pgfsetdash{}{0pt}%
\pgfpathmoveto{\pgfqpoint{5.415259in}{3.244062in}}%
\pgfpathlineto{\pgfqpoint{5.423509in}{3.059344in}}%
\pgfpathlineto{\pgfqpoint{5.432700in}{2.950852in}}%
\pgfpathlineto{\pgfqpoint{5.467793in}{3.135392in}}%
\pgfpathlineto{\pgfqpoint{5.501925in}{3.239455in}}%
\pgfpathlineto{\pgfqpoint{5.488475in}{3.024678in}}%
\pgfpathlineto{\pgfqpoint{5.481775in}{3.323956in}}%
\pgfpathlineto{\pgfqpoint{5.447066in}{3.169802in}}%
\pgfpathlineto{\pgfqpoint{5.415259in}{3.244062in}}%
\pgfpathclose%
\pgfusepath{fill}%
\end{pgfscope}%
\begin{pgfscope}%
\pgfpathrectangle{\pgfqpoint{1.020000in}{0.880000in}}{\pgfqpoint{6.160000in}{6.160000in}}%
\pgfusepath{clip}%
\pgfsetbuttcap%
\pgfsetroundjoin%
\definecolor{currentfill}{rgb}{0.457046,0.594006,0.963029}%
\pgfsetfillcolor{currentfill}%
\pgfsetlinewidth{0.000000pt}%
\definecolor{currentstroke}{rgb}{0.000000,0.000000,0.000000}%
\pgfsetstrokecolor{currentstroke}%
\pgfsetdash{}{0pt}%
\pgfpathmoveto{\pgfqpoint{4.395559in}{3.508247in}}%
\pgfpathlineto{\pgfqpoint{4.405045in}{3.513249in}}%
\pgfpathlineto{\pgfqpoint{4.414312in}{3.414510in}}%
\pgfpathlineto{\pgfqpoint{4.446742in}{3.163478in}}%
\pgfpathlineto{\pgfqpoint{4.480249in}{3.337336in}}%
\pgfpathlineto{\pgfqpoint{4.470753in}{3.353259in}}%
\pgfpathlineto{\pgfqpoint{4.461357in}{3.399434in}}%
\pgfpathlineto{\pgfqpoint{4.428120in}{3.304708in}}%
\pgfpathlineto{\pgfqpoint{4.395559in}{3.508247in}}%
\pgfpathclose%
\pgfusepath{fill}%
\end{pgfscope}%
\begin{pgfscope}%
\pgfpathrectangle{\pgfqpoint{1.020000in}{0.880000in}}{\pgfqpoint{6.160000in}{6.160000in}}%
\pgfusepath{clip}%
\pgfsetbuttcap%
\pgfsetroundjoin%
\definecolor{currentfill}{rgb}{0.959518,0.766973,0.674145}%
\pgfsetfillcolor{currentfill}%
\pgfsetlinewidth{0.000000pt}%
\definecolor{currentstroke}{rgb}{0.000000,0.000000,0.000000}%
\pgfsetstrokecolor{currentstroke}%
\pgfsetdash{}{0pt}%
\pgfpathmoveto{\pgfqpoint{3.585613in}{4.513928in}}%
\pgfpathlineto{\pgfqpoint{3.594363in}{4.491497in}}%
\pgfpathlineto{\pgfqpoint{3.602884in}{4.511242in}}%
\pgfpathlineto{\pgfqpoint{3.636245in}{4.505270in}}%
\pgfpathlineto{\pgfqpoint{3.669750in}{4.466614in}}%
\pgfpathlineto{\pgfqpoint{3.661145in}{4.448669in}}%
\pgfpathlineto{\pgfqpoint{3.653202in}{4.309049in}}%
\pgfpathlineto{\pgfqpoint{3.618761in}{4.535043in}}%
\pgfpathlineto{\pgfqpoint{3.585613in}{4.513928in}}%
\pgfpathclose%
\pgfusepath{fill}%
\end{pgfscope}%
\begin{pgfscope}%
\pgfpathrectangle{\pgfqpoint{1.020000in}{0.880000in}}{\pgfqpoint{6.160000in}{6.160000in}}%
\pgfusepath{clip}%
\pgfsetbuttcap%
\pgfsetroundjoin%
\definecolor{currentfill}{rgb}{0.804965,0.851666,0.926165}%
\pgfsetfillcolor{currentfill}%
\pgfsetlinewidth{0.000000pt}%
\definecolor{currentstroke}{rgb}{0.000000,0.000000,0.000000}%
\pgfsetstrokecolor{currentstroke}%
\pgfsetdash{}{0pt}%
\pgfpathmoveto{\pgfqpoint{3.839310in}{4.148191in}}%
\pgfpathlineto{\pgfqpoint{3.848445in}{4.074503in}}%
\pgfpathlineto{\pgfqpoint{3.857384in}{4.062933in}}%
\pgfpathlineto{\pgfqpoint{3.890870in}{3.984638in}}%
\pgfpathlineto{\pgfqpoint{3.924495in}{3.824847in}}%
\pgfpathlineto{\pgfqpoint{3.914815in}{4.100774in}}%
\pgfpathlineto{\pgfqpoint{3.906166in}{3.991462in}}%
\pgfpathlineto{\pgfqpoint{3.872957in}{4.010225in}}%
\pgfpathlineto{\pgfqpoint{3.839310in}{4.148191in}}%
\pgfpathclose%
\pgfusepath{fill}%
\end{pgfscope}%
\begin{pgfscope}%
\pgfpathrectangle{\pgfqpoint{1.020000in}{0.880000in}}{\pgfqpoint{6.160000in}{6.160000in}}%
\pgfusepath{clip}%
\pgfsetbuttcap%
\pgfsetroundjoin%
\definecolor{currentfill}{rgb}{0.348323,0.465711,0.888346}%
\pgfsetfillcolor{currentfill}%
\pgfsetlinewidth{0.000000pt}%
\definecolor{currentstroke}{rgb}{0.000000,0.000000,0.000000}%
\pgfsetstrokecolor{currentstroke}%
\pgfsetdash{}{0pt}%
\pgfpathmoveto{\pgfqpoint{5.696906in}{3.150050in}}%
\pgfpathlineto{\pgfqpoint{5.706587in}{3.074442in}}%
\pgfpathlineto{\pgfqpoint{5.718307in}{3.130883in}}%
\pgfpathlineto{\pgfqpoint{5.752046in}{3.198024in}}%
\pgfpathlineto{\pgfqpoint{5.784076in}{3.157209in}}%
\pgfpathlineto{\pgfqpoint{5.775366in}{3.295967in}}%
\pgfpathlineto{\pgfqpoint{5.761682in}{3.121111in}}%
\pgfpathlineto{\pgfqpoint{5.731536in}{3.279239in}}%
\pgfpathlineto{\pgfqpoint{5.696906in}{3.150050in}}%
\pgfpathclose%
\pgfusepath{fill}%
\end{pgfscope}%
\begin{pgfscope}%
\pgfpathrectangle{\pgfqpoint{1.020000in}{0.880000in}}{\pgfqpoint{6.160000in}{6.160000in}}%
\pgfusepath{clip}%
\pgfsetbuttcap%
\pgfsetroundjoin%
\definecolor{currentfill}{rgb}{0.313946,0.420052,0.854993}%
\pgfsetfillcolor{currentfill}%
\pgfsetlinewidth{0.000000pt}%
\definecolor{currentstroke}{rgb}{0.000000,0.000000,0.000000}%
\pgfsetstrokecolor{currentstroke}%
\pgfsetdash{}{0pt}%
\pgfpathmoveto{\pgfqpoint{4.848308in}{3.186078in}}%
\pgfpathlineto{\pgfqpoint{4.858419in}{3.209781in}}%
\pgfpathlineto{\pgfqpoint{4.868345in}{3.202880in}}%
\pgfpathlineto{\pgfqpoint{4.899976in}{3.015227in}}%
\pgfpathlineto{\pgfqpoint{4.932870in}{3.014400in}}%
\pgfpathlineto{\pgfqpoint{4.923257in}{3.068912in}}%
\pgfpathlineto{\pgfqpoint{4.913294in}{3.074277in}}%
\pgfpathlineto{\pgfqpoint{4.880501in}{3.083291in}}%
\pgfpathlineto{\pgfqpoint{4.848308in}{3.186078in}}%
\pgfpathclose%
\pgfusepath{fill}%
\end{pgfscope}%
\begin{pgfscope}%
\pgfpathrectangle{\pgfqpoint{1.020000in}{0.880000in}}{\pgfqpoint{6.160000in}{6.160000in}}%
\pgfusepath{clip}%
\pgfsetbuttcap%
\pgfsetroundjoin%
\definecolor{currentfill}{rgb}{0.368507,0.491141,0.905243}%
\pgfsetfillcolor{currentfill}%
\pgfsetlinewidth{0.000000pt}%
\definecolor{currentstroke}{rgb}{0.000000,0.000000,0.000000}%
\pgfsetstrokecolor{currentstroke}%
\pgfsetdash{}{0pt}%
\pgfpathmoveto{\pgfqpoint{4.697611in}{3.327642in}}%
\pgfpathlineto{\pgfqpoint{4.707335in}{3.314158in}}%
\pgfpathlineto{\pgfqpoint{4.716900in}{3.267529in}}%
\pgfpathlineto{\pgfqpoint{4.748789in}{3.068826in}}%
\pgfpathlineto{\pgfqpoint{4.782081in}{3.128786in}}%
\pgfpathlineto{\pgfqpoint{4.772437in}{3.165341in}}%
\pgfpathlineto{\pgfqpoint{4.762718in}{3.186676in}}%
\pgfpathlineto{\pgfqpoint{4.729948in}{3.211728in}}%
\pgfpathlineto{\pgfqpoint{4.697611in}{3.327642in}}%
\pgfpathclose%
\pgfusepath{fill}%
\end{pgfscope}%
\begin{pgfscope}%
\pgfpathrectangle{\pgfqpoint{1.020000in}{0.880000in}}{\pgfqpoint{6.160000in}{6.160000in}}%
\pgfusepath{clip}%
\pgfsetbuttcap%
\pgfsetroundjoin%
\definecolor{currentfill}{rgb}{0.968500,0.673977,0.556649}%
\pgfsetfillcolor{currentfill}%
\pgfsetlinewidth{0.000000pt}%
\definecolor{currentstroke}{rgb}{0.000000,0.000000,0.000000}%
\pgfsetstrokecolor{currentstroke}%
\pgfsetdash{}{0pt}%
\pgfpathmoveto{\pgfqpoint{2.215196in}{4.752608in}}%
\pgfpathlineto{\pgfqpoint{2.226095in}{4.561475in}}%
\pgfpathlineto{\pgfqpoint{2.230789in}{4.691534in}}%
\pgfpathlineto{\pgfqpoint{2.264669in}{4.684086in}}%
\pgfpathlineto{\pgfqpoint{2.299118in}{4.644707in}}%
\pgfpathlineto{\pgfqpoint{2.291965in}{4.638275in}}%
\pgfpathlineto{\pgfqpoint{2.286919in}{4.520370in}}%
\pgfpathlineto{\pgfqpoint{2.249550in}{4.717990in}}%
\pgfpathlineto{\pgfqpoint{2.215196in}{4.752608in}}%
\pgfpathclose%
\pgfusepath{fill}%
\end{pgfscope}%
\begin{pgfscope}%
\pgfpathrectangle{\pgfqpoint{1.020000in}{0.880000in}}{\pgfqpoint{6.160000in}{6.160000in}}%
\pgfusepath{clip}%
\pgfsetbuttcap%
\pgfsetroundjoin%
\definecolor{currentfill}{rgb}{0.964911,0.640159,0.519806}%
\pgfsetfillcolor{currentfill}%
\pgfsetlinewidth{0.000000pt}%
\definecolor{currentstroke}{rgb}{0.000000,0.000000,0.000000}%
\pgfsetstrokecolor{currentstroke}%
\pgfsetdash{}{0pt}%
\pgfpathmoveto{\pgfqpoint{2.286919in}{4.520370in}}%
\pgfpathlineto{\pgfqpoint{2.291965in}{4.638275in}}%
\pgfpathlineto{\pgfqpoint{2.299118in}{4.644707in}}%
\pgfpathlineto{\pgfqpoint{2.331335in}{4.724943in}}%
\pgfpathlineto{\pgfqpoint{2.362716in}{4.853638in}}%
\pgfpathlineto{\pgfqpoint{2.356577in}{4.786184in}}%
\pgfpathlineto{\pgfqpoint{2.349988in}{4.745030in}}%
\pgfpathlineto{\pgfqpoint{2.315561in}{4.788154in}}%
\pgfpathlineto{\pgfqpoint{2.286919in}{4.520370in}}%
\pgfpathclose%
\pgfusepath{fill}%
\end{pgfscope}%
\begin{pgfscope}%
\pgfpathrectangle{\pgfqpoint{1.020000in}{0.880000in}}{\pgfqpoint{6.160000in}{6.160000in}}%
\pgfusepath{clip}%
\pgfsetbuttcap%
\pgfsetroundjoin%
\definecolor{currentfill}{rgb}{0.328604,0.439712,0.869587}%
\pgfsetfillcolor{currentfill}%
\pgfsetlinewidth{0.000000pt}%
\definecolor{currentstroke}{rgb}{0.000000,0.000000,0.000000}%
\pgfsetstrokecolor{currentstroke}%
\pgfsetdash{}{0pt}%
\pgfpathmoveto{\pgfqpoint{5.348825in}{3.168826in}}%
\pgfpathlineto{\pgfqpoint{5.358050in}{3.062524in}}%
\pgfpathlineto{\pgfqpoint{5.369981in}{3.183630in}}%
\pgfpathlineto{\pgfqpoint{5.401273in}{3.057784in}}%
\pgfpathlineto{\pgfqpoint{5.432700in}{2.950852in}}%
\pgfpathlineto{\pgfqpoint{5.423509in}{3.059344in}}%
\pgfpathlineto{\pgfqpoint{5.415259in}{3.244062in}}%
\pgfpathlineto{\pgfqpoint{5.383123in}{3.296694in}}%
\pgfpathlineto{\pgfqpoint{5.348825in}{3.168826in}}%
\pgfpathclose%
\pgfusepath{fill}%
\end{pgfscope}%
\begin{pgfscope}%
\pgfpathrectangle{\pgfqpoint{1.020000in}{0.880000in}}{\pgfqpoint{6.160000in}{6.160000in}}%
\pgfusepath{clip}%
\pgfsetbuttcap%
\pgfsetroundjoin%
\definecolor{currentfill}{rgb}{0.294718,0.393542,0.834384}%
\pgfsetfillcolor{currentfill}%
\pgfsetlinewidth{0.000000pt}%
\definecolor{currentstroke}{rgb}{0.000000,0.000000,0.000000}%
\pgfsetstrokecolor{currentstroke}%
\pgfsetdash{}{0pt}%
\pgfpathmoveto{\pgfqpoint{4.998508in}{3.001726in}}%
\pgfpathlineto{\pgfqpoint{5.010202in}{3.193850in}}%
\pgfpathlineto{\pgfqpoint{5.019390in}{3.080052in}}%
\pgfpathlineto{\pgfqpoint{5.052037in}{3.047968in}}%
\pgfpathlineto{\pgfqpoint{5.085389in}{3.098282in}}%
\pgfpathlineto{\pgfqpoint{5.074361in}{3.005318in}}%
\pgfpathlineto{\pgfqpoint{5.063886in}{2.970334in}}%
\pgfpathlineto{\pgfqpoint{5.031970in}{3.074994in}}%
\pgfpathlineto{\pgfqpoint{4.998508in}{3.001726in}}%
\pgfpathclose%
\pgfusepath{fill}%
\end{pgfscope}%
\begin{pgfscope}%
\pgfpathrectangle{\pgfqpoint{1.020000in}{0.880000in}}{\pgfqpoint{6.160000in}{6.160000in}}%
\pgfusepath{clip}%
\pgfsetbuttcap%
\pgfsetroundjoin%
\definecolor{currentfill}{rgb}{0.790562,0.231397,0.216242}%
\pgfsetfillcolor{currentfill}%
\pgfsetlinewidth{0.000000pt}%
\definecolor{currentstroke}{rgb}{0.000000,0.000000,0.000000}%
\pgfsetstrokecolor{currentstroke}%
\pgfsetdash{}{0pt}%
\pgfpathmoveto{\pgfqpoint{3.094357in}{5.375237in}}%
\pgfpathlineto{\pgfqpoint{3.102492in}{5.374668in}}%
\pgfpathlineto{\pgfqpoint{3.112774in}{5.171503in}}%
\pgfpathlineto{\pgfqpoint{3.145445in}{5.248574in}}%
\pgfpathlineto{\pgfqpoint{3.179821in}{5.155954in}}%
\pgfpathlineto{\pgfqpoint{3.170459in}{5.273517in}}%
\pgfpathlineto{\pgfqpoint{3.162672in}{5.233881in}}%
\pgfpathlineto{\pgfqpoint{3.131193in}{5.049009in}}%
\pgfpathlineto{\pgfqpoint{3.094357in}{5.375237in}}%
\pgfpathclose%
\pgfusepath{fill}%
\end{pgfscope}%
\begin{pgfscope}%
\pgfpathrectangle{\pgfqpoint{1.020000in}{0.880000in}}{\pgfqpoint{6.160000in}{6.160000in}}%
\pgfusepath{clip}%
\pgfsetbuttcap%
\pgfsetroundjoin%
\definecolor{currentfill}{rgb}{0.683056,0.790043,0.989768}%
\pgfsetfillcolor{currentfill}%
\pgfsetlinewidth{0.000000pt}%
\definecolor{currentstroke}{rgb}{0.000000,0.000000,0.000000}%
\pgfsetstrokecolor{currentstroke}%
\pgfsetdash{}{0pt}%
\pgfpathmoveto{\pgfqpoint{4.008588in}{3.964255in}}%
\pgfpathlineto{\pgfqpoint{4.018104in}{3.698678in}}%
\pgfpathlineto{\pgfqpoint{4.027000in}{3.812333in}}%
\pgfpathlineto{\pgfqpoint{4.060400in}{3.670463in}}%
\pgfpathlineto{\pgfqpoint{4.093482in}{3.757332in}}%
\pgfpathlineto{\pgfqpoint{4.084347in}{3.749929in}}%
\pgfpathlineto{\pgfqpoint{4.075107in}{3.858675in}}%
\pgfpathlineto{\pgfqpoint{4.042178in}{3.696371in}}%
\pgfpathlineto{\pgfqpoint{4.008588in}{3.964255in}}%
\pgfpathclose%
\pgfusepath{fill}%
\end{pgfscope}%
\begin{pgfscope}%
\pgfpathrectangle{\pgfqpoint{1.020000in}{0.880000in}}{\pgfqpoint{6.160000in}{6.160000in}}%
\pgfusepath{clip}%
\pgfsetbuttcap%
\pgfsetroundjoin%
\definecolor{currentfill}{rgb}{0.353369,0.472069,0.892570}%
\pgfsetfillcolor{currentfill}%
\pgfsetlinewidth{0.000000pt}%
\definecolor{currentstroke}{rgb}{0.000000,0.000000,0.000000}%
\pgfsetstrokecolor{currentstroke}%
\pgfsetdash{}{0pt}%
\pgfpathmoveto{\pgfqpoint{5.633653in}{3.291724in}}%
\pgfpathlineto{\pgfqpoint{5.643727in}{3.242789in}}%
\pgfpathlineto{\pgfqpoint{5.652516in}{3.105935in}}%
\pgfpathlineto{\pgfqpoint{5.686110in}{3.164373in}}%
\pgfpathlineto{\pgfqpoint{5.718307in}{3.130883in}}%
\pgfpathlineto{\pgfqpoint{5.706587in}{3.074442in}}%
\pgfpathlineto{\pgfqpoint{5.696906in}{3.150050in}}%
\pgfpathlineto{\pgfqpoint{5.665782in}{3.252502in}}%
\pgfpathlineto{\pgfqpoint{5.633653in}{3.291724in}}%
\pgfpathclose%
\pgfusepath{fill}%
\end{pgfscope}%
\begin{pgfscope}%
\pgfpathrectangle{\pgfqpoint{1.020000in}{0.880000in}}{\pgfqpoint{6.160000in}{6.160000in}}%
\pgfusepath{clip}%
\pgfsetbuttcap%
\pgfsetroundjoin%
\definecolor{currentfill}{rgb}{0.446431,0.582356,0.957373}%
\pgfsetfillcolor{currentfill}%
\pgfsetlinewidth{0.000000pt}%
\definecolor{currentstroke}{rgb}{0.000000,0.000000,0.000000}%
\pgfsetstrokecolor{currentstroke}%
\pgfsetdash{}{0pt}%
\pgfpathmoveto{\pgfqpoint{4.545984in}{3.253808in}}%
\pgfpathlineto{\pgfqpoint{4.555521in}{3.232004in}}%
\pgfpathlineto{\pgfqpoint{4.565758in}{3.388549in}}%
\pgfpathlineto{\pgfqpoint{4.598413in}{3.288573in}}%
\pgfpathlineto{\pgfqpoint{4.632105in}{3.438809in}}%
\pgfpathlineto{\pgfqpoint{4.621922in}{3.333466in}}%
\pgfpathlineto{\pgfqpoint{4.612403in}{3.373450in}}%
\pgfpathlineto{\pgfqpoint{4.579929in}{3.501594in}}%
\pgfpathlineto{\pgfqpoint{4.545984in}{3.253808in}}%
\pgfpathclose%
\pgfusepath{fill}%
\end{pgfscope}%
\begin{pgfscope}%
\pgfpathrectangle{\pgfqpoint{1.020000in}{0.880000in}}{\pgfqpoint{6.160000in}{6.160000in}}%
\pgfusepath{clip}%
\pgfsetbuttcap%
\pgfsetroundjoin%
\definecolor{currentfill}{rgb}{0.949454,0.572388,0.453443}%
\pgfsetfillcolor{currentfill}%
\pgfsetlinewidth{0.000000pt}%
\definecolor{currentstroke}{rgb}{0.000000,0.000000,0.000000}%
\pgfsetstrokecolor{currentstroke}%
\pgfsetdash{}{0pt}%
\pgfpathmoveto{\pgfqpoint{2.349988in}{4.745030in}}%
\pgfpathlineto{\pgfqpoint{2.356577in}{4.786184in}}%
\pgfpathlineto{\pgfqpoint{2.362716in}{4.853638in}}%
\pgfpathlineto{\pgfqpoint{2.398699in}{4.724826in}}%
\pgfpathlineto{\pgfqpoint{2.431467in}{4.775952in}}%
\pgfpathlineto{\pgfqpoint{2.419392in}{5.044464in}}%
\pgfpathlineto{\pgfqpoint{2.416447in}{4.787791in}}%
\pgfpathlineto{\pgfqpoint{2.383153in}{4.770186in}}%
\pgfpathlineto{\pgfqpoint{2.349988in}{4.745030in}}%
\pgfpathclose%
\pgfusepath{fill}%
\end{pgfscope}%
\begin{pgfscope}%
\pgfpathrectangle{\pgfqpoint{1.020000in}{0.880000in}}{\pgfqpoint{6.160000in}{6.160000in}}%
\pgfusepath{clip}%
\pgfsetbuttcap%
\pgfsetroundjoin%
\definecolor{currentfill}{rgb}{0.879622,0.858175,0.845844}%
\pgfsetfillcolor{currentfill}%
\pgfsetlinewidth{0.000000pt}%
\definecolor{currentstroke}{rgb}{0.000000,0.000000,0.000000}%
\pgfsetstrokecolor{currentstroke}%
\pgfsetdash{}{0pt}%
\pgfpathmoveto{\pgfqpoint{3.687610in}{4.382684in}}%
\pgfpathlineto{\pgfqpoint{3.696982in}{4.252504in}}%
\pgfpathlineto{\pgfqpoint{3.705600in}{4.280714in}}%
\pgfpathlineto{\pgfqpoint{3.739875in}{4.069377in}}%
\pgfpathlineto{\pgfqpoint{3.772673in}{4.181143in}}%
\pgfpathlineto{\pgfqpoint{3.764719in}{3.975707in}}%
\pgfpathlineto{\pgfqpoint{3.755267in}{4.135226in}}%
\pgfpathlineto{\pgfqpoint{3.721601in}{4.234584in}}%
\pgfpathlineto{\pgfqpoint{3.687610in}{4.382684in}}%
\pgfpathclose%
\pgfusepath{fill}%
\end{pgfscope}%
\begin{pgfscope}%
\pgfpathrectangle{\pgfqpoint{1.020000in}{0.880000in}}{\pgfqpoint{6.160000in}{6.160000in}}%
\pgfusepath{clip}%
\pgfsetbuttcap%
\pgfsetroundjoin%
\definecolor{currentfill}{rgb}{0.852378,0.346492,0.280346}%
\pgfsetfillcolor{currentfill}%
\pgfsetlinewidth{0.000000pt}%
\definecolor{currentstroke}{rgb}{0.000000,0.000000,0.000000}%
\pgfsetstrokecolor{currentstroke}%
\pgfsetdash{}{0pt}%
\pgfpathmoveto{\pgfqpoint{3.179821in}{5.155954in}}%
\pgfpathlineto{\pgfqpoint{3.186517in}{5.310580in}}%
\pgfpathlineto{\pgfqpoint{3.195282in}{5.256354in}}%
\pgfpathlineto{\pgfqpoint{3.229924in}{5.135620in}}%
\pgfpathlineto{\pgfqpoint{3.263750in}{5.094087in}}%
\pgfpathlineto{\pgfqpoint{3.257426in}{4.879175in}}%
\pgfpathlineto{\pgfqpoint{3.247616in}{5.048387in}}%
\pgfpathlineto{\pgfqpoint{3.215028in}{4.967690in}}%
\pgfpathlineto{\pgfqpoint{3.179821in}{5.155954in}}%
\pgfpathclose%
\pgfusepath{fill}%
\end{pgfscope}%
\begin{pgfscope}%
\pgfpathrectangle{\pgfqpoint{1.020000in}{0.880000in}}{\pgfqpoint{6.160000in}{6.160000in}}%
\pgfusepath{clip}%
\pgfsetbuttcap%
\pgfsetroundjoin%
\definecolor{currentfill}{rgb}{0.758112,0.168122,0.188827}%
\pgfsetfillcolor{currentfill}%
\pgfsetlinewidth{0.000000pt}%
\definecolor{currentstroke}{rgb}{0.000000,0.000000,0.000000}%
\pgfsetstrokecolor{currentstroke}%
\pgfsetdash{}{0pt}%
\pgfpathmoveto{\pgfqpoint{2.877310in}{5.380874in}}%
\pgfpathlineto{\pgfqpoint{2.887401in}{5.208910in}}%
\pgfpathlineto{\pgfqpoint{2.893043in}{5.387701in}}%
\pgfpathlineto{\pgfqpoint{2.927512in}{5.315438in}}%
\pgfpathlineto{\pgfqpoint{2.960609in}{5.352035in}}%
\pgfpathlineto{\pgfqpoint{2.953650in}{5.267381in}}%
\pgfpathlineto{\pgfqpoint{2.946724in}{5.182606in}}%
\pgfpathlineto{\pgfqpoint{2.912471in}{5.248749in}}%
\pgfpathlineto{\pgfqpoint{2.877310in}{5.380874in}}%
\pgfpathclose%
\pgfusepath{fill}%
\end{pgfscope}%
\begin{pgfscope}%
\pgfpathrectangle{\pgfqpoint{1.020000in}{0.880000in}}{\pgfqpoint{6.160000in}{6.160000in}}%
\pgfusepath{clip}%
\pgfsetbuttcap%
\pgfsetroundjoin%
\definecolor{currentfill}{rgb}{0.931831,0.519086,0.406480}%
\pgfsetfillcolor{currentfill}%
\pgfsetlinewidth{0.000000pt}%
\definecolor{currentstroke}{rgb}{0.000000,0.000000,0.000000}%
\pgfsetstrokecolor{currentstroke}%
\pgfsetdash{}{0pt}%
\pgfpathmoveto{\pgfqpoint{2.416447in}{4.787791in}}%
\pgfpathlineto{\pgfqpoint{2.419392in}{5.044464in}}%
\pgfpathlineto{\pgfqpoint{2.431467in}{4.775952in}}%
\pgfpathlineto{\pgfqpoint{2.463107in}{4.894481in}}%
\pgfpathlineto{\pgfqpoint{2.498427in}{4.795123in}}%
\pgfpathlineto{\pgfqpoint{2.489925in}{4.856268in}}%
\pgfpathlineto{\pgfqpoint{2.481388in}{4.919867in}}%
\pgfpathlineto{\pgfqpoint{2.446436in}{4.998250in}}%
\pgfpathlineto{\pgfqpoint{2.416447in}{4.787791in}}%
\pgfpathclose%
\pgfusepath{fill}%
\end{pgfscope}%
\begin{pgfscope}%
\pgfpathrectangle{\pgfqpoint{1.020000in}{0.880000in}}{\pgfqpoint{6.160000in}{6.160000in}}%
\pgfusepath{clip}%
\pgfsetbuttcap%
\pgfsetroundjoin%
\definecolor{currentfill}{rgb}{0.323718,0.433158,0.864722}%
\pgfsetfillcolor{currentfill}%
\pgfsetlinewidth{0.000000pt}%
\definecolor{currentstroke}{rgb}{0.000000,0.000000,0.000000}%
\pgfsetstrokecolor{currentstroke}%
\pgfsetdash{}{0pt}%
\pgfpathmoveto{\pgfqpoint{5.281937in}{3.050537in}}%
\pgfpathlineto{\pgfqpoint{5.293318in}{3.138279in}}%
\pgfpathlineto{\pgfqpoint{5.303633in}{3.127661in}}%
\pgfpathlineto{\pgfqpoint{5.336716in}{3.147745in}}%
\pgfpathlineto{\pgfqpoint{5.369981in}{3.183630in}}%
\pgfpathlineto{\pgfqpoint{5.358050in}{3.062524in}}%
\pgfpathlineto{\pgfqpoint{5.348825in}{3.168826in}}%
\pgfpathlineto{\pgfqpoint{5.314893in}{3.067639in}}%
\pgfpathlineto{\pgfqpoint{5.281937in}{3.050537in}}%
\pgfpathclose%
\pgfusepath{fill}%
\end{pgfscope}%
\begin{pgfscope}%
\pgfpathrectangle{\pgfqpoint{1.020000in}{0.880000in}}{\pgfqpoint{6.160000in}{6.160000in}}%
\pgfusepath{clip}%
\pgfsetbuttcap%
\pgfsetroundjoin%
\definecolor{currentfill}{rgb}{0.630089,0.752516,0.998508}%
\pgfsetfillcolor{currentfill}%
\pgfsetlinewidth{0.000000pt}%
\definecolor{currentstroke}{rgb}{0.000000,0.000000,0.000000}%
\pgfsetstrokecolor{currentstroke}%
\pgfsetdash{}{0pt}%
\pgfpathmoveto{\pgfqpoint{4.093482in}{3.757332in}}%
\pgfpathlineto{\pgfqpoint{4.102688in}{3.703174in}}%
\pgfpathlineto{\pgfqpoint{4.111882in}{3.678718in}}%
\pgfpathlineto{\pgfqpoint{4.145114in}{3.614256in}}%
\pgfpathlineto{\pgfqpoint{4.178278in}{3.684413in}}%
\pgfpathlineto{\pgfqpoint{4.169058in}{3.550288in}}%
\pgfpathlineto{\pgfqpoint{4.159824in}{3.664255in}}%
\pgfpathlineto{\pgfqpoint{4.126670in}{3.724908in}}%
\pgfpathlineto{\pgfqpoint{4.093482in}{3.757332in}}%
\pgfpathclose%
\pgfusepath{fill}%
\end{pgfscope}%
\begin{pgfscope}%
\pgfpathrectangle{\pgfqpoint{1.020000in}{0.880000in}}{\pgfqpoint{6.160000in}{6.160000in}}%
\pgfusepath{clip}%
\pgfsetbuttcap%
\pgfsetroundjoin%
\definecolor{currentfill}{rgb}{0.348323,0.465711,0.888346}%
\pgfsetfillcolor{currentfill}%
\pgfsetlinewidth{0.000000pt}%
\definecolor{currentstroke}{rgb}{0.000000,0.000000,0.000000}%
\pgfsetstrokecolor{currentstroke}%
\pgfsetdash{}{0pt}%
\pgfpathmoveto{\pgfqpoint{5.564912in}{3.056598in}}%
\pgfpathlineto{\pgfqpoint{5.574855in}{3.002369in}}%
\pgfpathlineto{\pgfqpoint{5.587259in}{3.122070in}}%
\pgfpathlineto{\pgfqpoint{5.620229in}{3.136824in}}%
\pgfpathlineto{\pgfqpoint{5.652516in}{3.105935in}}%
\pgfpathlineto{\pgfqpoint{5.643727in}{3.242789in}}%
\pgfpathlineto{\pgfqpoint{5.633653in}{3.291724in}}%
\pgfpathlineto{\pgfqpoint{5.601828in}{3.355517in}}%
\pgfpathlineto{\pgfqpoint{5.564912in}{3.056598in}}%
\pgfpathclose%
\pgfusepath{fill}%
\end{pgfscope}%
\begin{pgfscope}%
\pgfpathrectangle{\pgfqpoint{1.020000in}{0.880000in}}{\pgfqpoint{6.160000in}{6.160000in}}%
\pgfusepath{clip}%
\pgfsetbuttcap%
\pgfsetroundjoin%
\definecolor{currentfill}{rgb}{0.966922,0.651969,0.531997}%
\pgfsetfillcolor{currentfill}%
\pgfsetlinewidth{0.000000pt}%
\definecolor{currentstroke}{rgb}{0.000000,0.000000,0.000000}%
\pgfsetstrokecolor{currentstroke}%
\pgfsetdash{}{0pt}%
\pgfpathmoveto{\pgfqpoint{3.434291in}{4.668149in}}%
\pgfpathlineto{\pgfqpoint{3.441132in}{4.890342in}}%
\pgfpathlineto{\pgfqpoint{3.450318in}{4.795929in}}%
\pgfpathlineto{\pgfqpoint{3.485403in}{4.557010in}}%
\pgfpathlineto{\pgfqpoint{3.518179in}{4.642566in}}%
\pgfpathlineto{\pgfqpoint{3.510708in}{4.480860in}}%
\pgfpathlineto{\pgfqpoint{3.500575in}{4.720796in}}%
\pgfpathlineto{\pgfqpoint{3.467389in}{4.700788in}}%
\pgfpathlineto{\pgfqpoint{3.434291in}{4.668149in}}%
\pgfpathclose%
\pgfusepath{fill}%
\end{pgfscope}%
\begin{pgfscope}%
\pgfpathrectangle{\pgfqpoint{1.020000in}{0.880000in}}{\pgfqpoint{6.160000in}{6.160000in}}%
\pgfusepath{clip}%
\pgfsetbuttcap%
\pgfsetroundjoin%
\definecolor{currentfill}{rgb}{0.905783,0.455186,0.355336}%
\pgfsetfillcolor{currentfill}%
\pgfsetlinewidth{0.000000pt}%
\definecolor{currentstroke}{rgb}{0.000000,0.000000,0.000000}%
\pgfsetstrokecolor{currentstroke}%
\pgfsetdash{}{0pt}%
\pgfpathmoveto{\pgfqpoint{2.481388in}{4.919867in}}%
\pgfpathlineto{\pgfqpoint{2.489925in}{4.856268in}}%
\pgfpathlineto{\pgfqpoint{2.498427in}{4.795123in}}%
\pgfpathlineto{\pgfqpoint{2.525430in}{5.201105in}}%
\pgfpathlineto{\pgfqpoint{2.563256in}{4.944526in}}%
\pgfpathlineto{\pgfqpoint{2.553432in}{5.085871in}}%
\pgfpathlineto{\pgfqpoint{2.546404in}{5.053276in}}%
\pgfpathlineto{\pgfqpoint{2.514508in}{4.948611in}}%
\pgfpathlineto{\pgfqpoint{2.481388in}{4.919867in}}%
\pgfpathclose%
\pgfusepath{fill}%
\end{pgfscope}%
\begin{pgfscope}%
\pgfpathrectangle{\pgfqpoint{1.020000in}{0.880000in}}{\pgfqpoint{6.160000in}{6.160000in}}%
\pgfusepath{clip}%
\pgfsetbuttcap%
\pgfsetroundjoin%
\definecolor{currentfill}{rgb}{0.446431,0.582356,0.957373}%
\pgfsetfillcolor{currentfill}%
\pgfsetlinewidth{0.000000pt}%
\definecolor{currentstroke}{rgb}{0.000000,0.000000,0.000000}%
\pgfsetstrokecolor{currentstroke}%
\pgfsetdash{}{0pt}%
\pgfpathmoveto{\pgfqpoint{4.480249in}{3.337336in}}%
\pgfpathlineto{\pgfqpoint{4.489915in}{3.370962in}}%
\pgfpathlineto{\pgfqpoint{4.499492in}{3.367992in}}%
\pgfpathlineto{\pgfqpoint{4.532791in}{3.423931in}}%
\pgfpathlineto{\pgfqpoint{4.565758in}{3.388549in}}%
\pgfpathlineto{\pgfqpoint{4.555521in}{3.232004in}}%
\pgfpathlineto{\pgfqpoint{4.545984in}{3.253808in}}%
\pgfpathlineto{\pgfqpoint{4.513358in}{3.361454in}}%
\pgfpathlineto{\pgfqpoint{4.480249in}{3.337336in}}%
\pgfpathclose%
\pgfusepath{fill}%
\end{pgfscope}%
\begin{pgfscope}%
\pgfpathrectangle{\pgfqpoint{1.020000in}{0.880000in}}{\pgfqpoint{6.160000in}{6.160000in}}%
\pgfusepath{clip}%
\pgfsetbuttcap%
\pgfsetroundjoin%
\definecolor{currentfill}{rgb}{0.280550,0.373423,0.818011}%
\pgfsetfillcolor{currentfill}%
\pgfsetlinewidth{0.000000pt}%
\definecolor{currentstroke}{rgb}{0.000000,0.000000,0.000000}%
\pgfsetstrokecolor{currentstroke}%
\pgfsetdash{}{0pt}%
\pgfpathmoveto{\pgfqpoint{5.150724in}{3.053525in}}%
\pgfpathlineto{\pgfqpoint{5.160979in}{3.051311in}}%
\pgfpathlineto{\pgfqpoint{5.170187in}{2.941324in}}%
\pgfpathlineto{\pgfqpoint{5.201353in}{2.775366in}}%
\pgfpathlineto{\pgfqpoint{5.239057in}{3.239548in}}%
\pgfpathlineto{\pgfqpoint{5.228294in}{3.203214in}}%
\pgfpathlineto{\pgfqpoint{5.215178in}{2.937059in}}%
\pgfpathlineto{\pgfqpoint{5.183248in}{3.022029in}}%
\pgfpathlineto{\pgfqpoint{5.150724in}{3.053525in}}%
\pgfpathclose%
\pgfusepath{fill}%
\end{pgfscope}%
\begin{pgfscope}%
\pgfpathrectangle{\pgfqpoint{1.020000in}{0.880000in}}{\pgfqpoint{6.160000in}{6.160000in}}%
\pgfusepath{clip}%
\pgfsetbuttcap%
\pgfsetroundjoin%
\definecolor{currentfill}{rgb}{0.338377,0.452819,0.879317}%
\pgfsetfillcolor{currentfill}%
\pgfsetlinewidth{0.000000pt}%
\definecolor{currentstroke}{rgb}{0.000000,0.000000,0.000000}%
\pgfsetstrokecolor{currentstroke}%
\pgfsetdash{}{0pt}%
\pgfpathmoveto{\pgfqpoint{5.784076in}{3.157209in}}%
\pgfpathlineto{\pgfqpoint{5.792831in}{3.021593in}}%
\pgfpathlineto{\pgfqpoint{5.805090in}{3.101812in}}%
\pgfpathlineto{\pgfqpoint{5.837601in}{3.090537in}}%
\pgfpathlineto{\pgfqpoint{5.871675in}{3.173265in}}%
\pgfpathlineto{\pgfqpoint{5.861734in}{3.238245in}}%
\pgfpathlineto{\pgfqpoint{5.850584in}{3.230324in}}%
\pgfpathlineto{\pgfqpoint{5.815759in}{3.097955in}}%
\pgfpathlineto{\pgfqpoint{5.784076in}{3.157209in}}%
\pgfpathclose%
\pgfusepath{fill}%
\end{pgfscope}%
\begin{pgfscope}%
\pgfpathrectangle{\pgfqpoint{1.020000in}{0.880000in}}{\pgfqpoint{6.160000in}{6.160000in}}%
\pgfusepath{clip}%
\pgfsetbuttcap%
\pgfsetroundjoin%
\definecolor{currentfill}{rgb}{0.795938,0.241845,0.220830}%
\pgfsetfillcolor{currentfill}%
\pgfsetlinewidth{0.000000pt}%
\definecolor{currentstroke}{rgb}{0.000000,0.000000,0.000000}%
\pgfsetstrokecolor{currentstroke}%
\pgfsetdash{}{0pt}%
\pgfpathmoveto{\pgfqpoint{2.744772in}{5.260401in}}%
\pgfpathlineto{\pgfqpoint{2.754702in}{5.106120in}}%
\pgfpathlineto{\pgfqpoint{2.763021in}{5.066549in}}%
\pgfpathlineto{\pgfqpoint{2.794778in}{5.198008in}}%
\pgfpathlineto{\pgfqpoint{2.825366in}{5.422365in}}%
\pgfpathlineto{\pgfqpoint{2.819826in}{5.249363in}}%
\pgfpathlineto{\pgfqpoint{2.811211in}{5.307976in}}%
\pgfpathlineto{\pgfqpoint{2.780074in}{5.133379in}}%
\pgfpathlineto{\pgfqpoint{2.744772in}{5.260401in}}%
\pgfpathclose%
\pgfusepath{fill}%
\end{pgfscope}%
\begin{pgfscope}%
\pgfpathrectangle{\pgfqpoint{1.020000in}{0.880000in}}{\pgfqpoint{6.160000in}{6.160000in}}%
\pgfusepath{clip}%
\pgfsetbuttcap%
\pgfsetroundjoin%
\definecolor{currentfill}{rgb}{0.586921,0.718121,0.998874}%
\pgfsetfillcolor{currentfill}%
\pgfsetlinewidth{0.000000pt}%
\definecolor{currentstroke}{rgb}{0.000000,0.000000,0.000000}%
\pgfsetstrokecolor{currentstroke}%
\pgfsetdash{}{0pt}%
\pgfpathmoveto{\pgfqpoint{4.244587in}{3.663528in}}%
\pgfpathlineto{\pgfqpoint{4.253792in}{3.498504in}}%
\pgfpathlineto{\pgfqpoint{4.263186in}{3.585650in}}%
\pgfpathlineto{\pgfqpoint{4.296321in}{3.560376in}}%
\pgfpathlineto{\pgfqpoint{4.329411in}{3.528778in}}%
\pgfpathlineto{\pgfqpoint{4.320103in}{3.590341in}}%
\pgfpathlineto{\pgfqpoint{4.310713in}{3.581659in}}%
\pgfpathlineto{\pgfqpoint{4.277794in}{3.750427in}}%
\pgfpathlineto{\pgfqpoint{4.244587in}{3.663528in}}%
\pgfpathclose%
\pgfusepath{fill}%
\end{pgfscope}%
\begin{pgfscope}%
\pgfpathrectangle{\pgfqpoint{1.020000in}{0.880000in}}{\pgfqpoint{6.160000in}{6.160000in}}%
\pgfusepath{clip}%
\pgfsetbuttcap%
\pgfsetroundjoin%
\definecolor{currentfill}{rgb}{0.309060,0.413498,0.850128}%
\pgfsetfillcolor{currentfill}%
\pgfsetlinewidth{0.000000pt}%
\definecolor{currentstroke}{rgb}{0.000000,0.000000,0.000000}%
\pgfsetstrokecolor{currentstroke}%
\pgfsetdash{}{0pt}%
\pgfpathmoveto{\pgfqpoint{5.718307in}{3.130883in}}%
\pgfpathlineto{\pgfqpoint{5.728339in}{3.076157in}}%
\pgfpathlineto{\pgfqpoint{5.738696in}{3.041498in}}%
\pgfpathlineto{\pgfqpoint{5.770356in}{2.975280in}}%
\pgfpathlineto{\pgfqpoint{5.805090in}{3.101812in}}%
\pgfpathlineto{\pgfqpoint{5.792831in}{3.021593in}}%
\pgfpathlineto{\pgfqpoint{5.784076in}{3.157209in}}%
\pgfpathlineto{\pgfqpoint{5.752046in}{3.198024in}}%
\pgfpathlineto{\pgfqpoint{5.718307in}{3.130883in}}%
\pgfpathclose%
\pgfusepath{fill}%
\end{pgfscope}%
\begin{pgfscope}%
\pgfpathrectangle{\pgfqpoint{1.020000in}{0.880000in}}{\pgfqpoint{6.160000in}{6.160000in}}%
\pgfusepath{clip}%
\pgfsetbuttcap%
\pgfsetroundjoin%
\definecolor{currentfill}{rgb}{0.772706,0.838978,0.949319}%
\pgfsetfillcolor{currentfill}%
\pgfsetlinewidth{0.000000pt}%
\definecolor{currentstroke}{rgb}{0.000000,0.000000,0.000000}%
\pgfsetstrokecolor{currentstroke}%
\pgfsetdash{}{0pt}%
\pgfpathmoveto{\pgfqpoint{3.857384in}{4.062933in}}%
\pgfpathlineto{\pgfqpoint{3.866213in}{4.091850in}}%
\pgfpathlineto{\pgfqpoint{3.875098in}{4.109948in}}%
\pgfpathlineto{\pgfqpoint{3.909292in}{3.796839in}}%
\pgfpathlineto{\pgfqpoint{3.942355in}{3.860192in}}%
\pgfpathlineto{\pgfqpoint{3.933586in}{3.773499in}}%
\pgfpathlineto{\pgfqpoint{3.924495in}{3.824847in}}%
\pgfpathlineto{\pgfqpoint{3.890870in}{3.984638in}}%
\pgfpathlineto{\pgfqpoint{3.857384in}{4.062933in}}%
\pgfpathclose%
\pgfusepath{fill}%
\end{pgfscope}%
\begin{pgfscope}%
\pgfpathrectangle{\pgfqpoint{1.020000in}{0.880000in}}{\pgfqpoint{6.160000in}{6.160000in}}%
\pgfusepath{clip}%
\pgfsetbuttcap%
\pgfsetroundjoin%
\definecolor{currentfill}{rgb}{0.843358,0.861820,0.890017}%
\pgfsetfillcolor{currentfill}%
\pgfsetlinewidth{0.000000pt}%
\definecolor{currentstroke}{rgb}{0.000000,0.000000,0.000000}%
\pgfsetstrokecolor{currentstroke}%
\pgfsetdash{}{0pt}%
\pgfpathmoveto{\pgfqpoint{3.772673in}{4.181143in}}%
\pgfpathlineto{\pgfqpoint{3.781155in}{4.268001in}}%
\pgfpathlineto{\pgfqpoint{3.791213in}{3.959313in}}%
\pgfpathlineto{\pgfqpoint{3.824288in}{4.012040in}}%
\pgfpathlineto{\pgfqpoint{3.857384in}{4.062933in}}%
\pgfpathlineto{\pgfqpoint{3.848445in}{4.074503in}}%
\pgfpathlineto{\pgfqpoint{3.839310in}{4.148191in}}%
\pgfpathlineto{\pgfqpoint{3.806506in}{4.031203in}}%
\pgfpathlineto{\pgfqpoint{3.772673in}{4.181143in}}%
\pgfpathclose%
\pgfusepath{fill}%
\end{pgfscope}%
\begin{pgfscope}%
\pgfpathrectangle{\pgfqpoint{1.020000in}{0.880000in}}{\pgfqpoint{6.160000in}{6.160000in}}%
\pgfusepath{clip}%
\pgfsetbuttcap%
\pgfsetroundjoin%
\definecolor{currentfill}{rgb}{0.318832,0.426605,0.859857}%
\pgfsetfillcolor{currentfill}%
\pgfsetlinewidth{0.000000pt}%
\definecolor{currentstroke}{rgb}{0.000000,0.000000,0.000000}%
\pgfsetstrokecolor{currentstroke}%
\pgfsetdash{}{0pt}%
\pgfpathmoveto{\pgfqpoint{5.215178in}{2.937059in}}%
\pgfpathlineto{\pgfqpoint{5.228294in}{3.203214in}}%
\pgfpathlineto{\pgfqpoint{5.239057in}{3.239548in}}%
\pgfpathlineto{\pgfqpoint{5.269137in}{2.978599in}}%
\pgfpathlineto{\pgfqpoint{5.303633in}{3.127661in}}%
\pgfpathlineto{\pgfqpoint{5.293318in}{3.138279in}}%
\pgfpathlineto{\pgfqpoint{5.281937in}{3.050537in}}%
\pgfpathlineto{\pgfqpoint{5.250092in}{3.138085in}}%
\pgfpathlineto{\pgfqpoint{5.215178in}{2.937059in}}%
\pgfpathclose%
\pgfusepath{fill}%
\end{pgfscope}%
\begin{pgfscope}%
\pgfpathrectangle{\pgfqpoint{1.020000in}{0.880000in}}{\pgfqpoint{6.160000in}{6.160000in}}%
\pgfusepath{clip}%
\pgfsetbuttcap%
\pgfsetroundjoin%
\definecolor{currentfill}{rgb}{0.774337,0.199759,0.202535}%
\pgfsetfillcolor{currentfill}%
\pgfsetlinewidth{0.000000pt}%
\definecolor{currentstroke}{rgb}{0.000000,0.000000,0.000000}%
\pgfsetstrokecolor{currentstroke}%
\pgfsetdash{}{0pt}%
\pgfpathmoveto{\pgfqpoint{3.030114in}{5.134315in}}%
\pgfpathlineto{\pgfqpoint{3.036709in}{5.263045in}}%
\pgfpathlineto{\pgfqpoint{3.042270in}{5.487655in}}%
\pgfpathlineto{\pgfqpoint{3.079663in}{5.137938in}}%
\pgfpathlineto{\pgfqpoint{3.112774in}{5.171503in}}%
\pgfpathlineto{\pgfqpoint{3.102492in}{5.374668in}}%
\pgfpathlineto{\pgfqpoint{3.094357in}{5.375237in}}%
\pgfpathlineto{\pgfqpoint{3.063589in}{5.129172in}}%
\pgfpathlineto{\pgfqpoint{3.030114in}{5.134315in}}%
\pgfpathclose%
\pgfusepath{fill}%
\end{pgfscope}%
\begin{pgfscope}%
\pgfpathrectangle{\pgfqpoint{1.020000in}{0.880000in}}{\pgfqpoint{6.160000in}{6.160000in}}%
\pgfusepath{clip}%
\pgfsetbuttcap%
\pgfsetroundjoin%
\definecolor{currentfill}{rgb}{0.565182,0.699438,0.996635}%
\pgfsetfillcolor{currentfill}%
\pgfsetlinewidth{0.000000pt}%
\definecolor{currentstroke}{rgb}{0.000000,0.000000,0.000000}%
\pgfsetstrokecolor{currentstroke}%
\pgfsetdash{}{0pt}%
\pgfpathmoveto{\pgfqpoint{4.178278in}{3.684413in}}%
\pgfpathlineto{\pgfqpoint{4.187529in}{3.510599in}}%
\pgfpathlineto{\pgfqpoint{4.196783in}{3.456992in}}%
\pgfpathlineto{\pgfqpoint{4.229969in}{3.510913in}}%
\pgfpathlineto{\pgfqpoint{4.263186in}{3.585650in}}%
\pgfpathlineto{\pgfqpoint{4.253792in}{3.498504in}}%
\pgfpathlineto{\pgfqpoint{4.244587in}{3.663528in}}%
\pgfpathlineto{\pgfqpoint{4.211395in}{3.495047in}}%
\pgfpathlineto{\pgfqpoint{4.178278in}{3.684413in}}%
\pgfpathclose%
\pgfusepath{fill}%
\end{pgfscope}%
\begin{pgfscope}%
\pgfpathrectangle{\pgfqpoint{1.020000in}{0.880000in}}{\pgfqpoint{6.160000in}{6.160000in}}%
\pgfusepath{clip}%
\pgfsetbuttcap%
\pgfsetroundjoin%
\definecolor{currentfill}{rgb}{0.815508,0.277781,0.240294}%
\pgfsetfillcolor{currentfill}%
\pgfsetlinewidth{0.000000pt}%
\definecolor{currentstroke}{rgb}{0.000000,0.000000,0.000000}%
\pgfsetstrokecolor{currentstroke}%
\pgfsetdash{}{0pt}%
\pgfpathmoveto{\pgfqpoint{2.678853in}{5.176709in}}%
\pgfpathlineto{\pgfqpoint{2.684559in}{5.311624in}}%
\pgfpathlineto{\pgfqpoint{2.694421in}{5.164999in}}%
\pgfpathlineto{\pgfqpoint{2.728888in}{5.105932in}}%
\pgfpathlineto{\pgfqpoint{2.763021in}{5.066549in}}%
\pgfpathlineto{\pgfqpoint{2.754702in}{5.106120in}}%
\pgfpathlineto{\pgfqpoint{2.744772in}{5.260401in}}%
\pgfpathlineto{\pgfqpoint{2.711537in}{5.237301in}}%
\pgfpathlineto{\pgfqpoint{2.678853in}{5.176709in}}%
\pgfpathclose%
\pgfusepath{fill}%
\end{pgfscope}%
\begin{pgfscope}%
\pgfpathrectangle{\pgfqpoint{1.020000in}{0.880000in}}{\pgfqpoint{6.160000in}{6.160000in}}%
\pgfusepath{clip}%
\pgfsetbuttcap%
\pgfsetroundjoin%
\definecolor{currentfill}{rgb}{0.338377,0.452819,0.879317}%
\pgfsetfillcolor{currentfill}%
\pgfsetlinewidth{0.000000pt}%
\definecolor{currentstroke}{rgb}{0.000000,0.000000,0.000000}%
\pgfsetstrokecolor{currentstroke}%
\pgfsetdash{}{0pt}%
\pgfpathmoveto{\pgfqpoint{5.501925in}{3.239455in}}%
\pgfpathlineto{\pgfqpoint{5.511457in}{3.154971in}}%
\pgfpathlineto{\pgfqpoint{5.522543in}{3.185865in}}%
\pgfpathlineto{\pgfqpoint{5.555060in}{3.164051in}}%
\pgfpathlineto{\pgfqpoint{5.587259in}{3.122070in}}%
\pgfpathlineto{\pgfqpoint{5.574855in}{3.002369in}}%
\pgfpathlineto{\pgfqpoint{5.564912in}{3.056598in}}%
\pgfpathlineto{\pgfqpoint{5.533751in}{3.169704in}}%
\pgfpathlineto{\pgfqpoint{5.501925in}{3.239455in}}%
\pgfpathclose%
\pgfusepath{fill}%
\end{pgfscope}%
\begin{pgfscope}%
\pgfpathrectangle{\pgfqpoint{1.020000in}{0.880000in}}{\pgfqpoint{6.160000in}{6.160000in}}%
\pgfusepath{clip}%
\pgfsetbuttcap%
\pgfsetroundjoin%
\definecolor{currentfill}{rgb}{0.373552,0.497499,0.909467}%
\pgfsetfillcolor{currentfill}%
\pgfsetlinewidth{0.000000pt}%
\definecolor{currentstroke}{rgb}{0.000000,0.000000,0.000000}%
\pgfsetstrokecolor{currentstroke}%
\pgfsetdash{}{0pt}%
\pgfpathmoveto{\pgfqpoint{5.850584in}{3.230324in}}%
\pgfpathlineto{\pgfqpoint{5.861734in}{3.238245in}}%
\pgfpathlineto{\pgfqpoint{5.871675in}{3.173265in}}%
\pgfpathlineto{\pgfqpoint{5.902499in}{3.065548in}}%
\pgfpathlineto{\pgfqpoint{5.896995in}{3.389211in}}%
\pgfpathlineto{\pgfqpoint{5.881964in}{3.156433in}}%
\pgfpathlineto{\pgfqpoint{5.850584in}{3.230324in}}%
\pgfpathclose%
\pgfusepath{fill}%
\end{pgfscope}%
\begin{pgfscope}%
\pgfpathrectangle{\pgfqpoint{1.020000in}{0.880000in}}{\pgfqpoint{6.160000in}{6.160000in}}%
\pgfusepath{clip}%
\pgfsetbuttcap%
\pgfsetroundjoin%
\definecolor{currentfill}{rgb}{0.313946,0.420052,0.854993}%
\pgfsetfillcolor{currentfill}%
\pgfsetlinewidth{0.000000pt}%
\definecolor{currentstroke}{rgb}{0.000000,0.000000,0.000000}%
\pgfsetstrokecolor{currentstroke}%
\pgfsetdash{}{0pt}%
\pgfpathmoveto{\pgfqpoint{4.932870in}{3.014400in}}%
\pgfpathlineto{\pgfqpoint{4.944213in}{3.185348in}}%
\pgfpathlineto{\pgfqpoint{4.953557in}{3.090357in}}%
\pgfpathlineto{\pgfqpoint{4.987372in}{3.195329in}}%
\pgfpathlineto{\pgfqpoint{5.019390in}{3.080052in}}%
\pgfpathlineto{\pgfqpoint{5.010202in}{3.193850in}}%
\pgfpathlineto{\pgfqpoint{4.998508in}{3.001726in}}%
\pgfpathlineto{\pgfqpoint{4.965570in}{2.991454in}}%
\pgfpathlineto{\pgfqpoint{4.932870in}{3.014400in}}%
\pgfpathclose%
\pgfusepath{fill}%
\end{pgfscope}%
\begin{pgfscope}%
\pgfpathrectangle{\pgfqpoint{1.020000in}{0.880000in}}{\pgfqpoint{6.160000in}{6.160000in}}%
\pgfusepath{clip}%
\pgfsetbuttcap%
\pgfsetroundjoin%
\definecolor{currentfill}{rgb}{0.708720,0.805721,0.981117}%
\pgfsetfillcolor{currentfill}%
\pgfsetlinewidth{0.000000pt}%
\definecolor{currentstroke}{rgb}{0.000000,0.000000,0.000000}%
\pgfsetstrokecolor{currentstroke}%
\pgfsetdash{}{0pt}%
\pgfpathmoveto{\pgfqpoint{3.942355in}{3.860192in}}%
\pgfpathlineto{\pgfqpoint{3.951645in}{3.736478in}}%
\pgfpathlineto{\pgfqpoint{3.960466in}{3.822014in}}%
\pgfpathlineto{\pgfqpoint{3.993805in}{3.785115in}}%
\pgfpathlineto{\pgfqpoint{4.027000in}{3.812333in}}%
\pgfpathlineto{\pgfqpoint{4.018104in}{3.698678in}}%
\pgfpathlineto{\pgfqpoint{4.008588in}{3.964255in}}%
\pgfpathlineto{\pgfqpoint{3.975710in}{3.793401in}}%
\pgfpathlineto{\pgfqpoint{3.942355in}{3.860192in}}%
\pgfpathclose%
\pgfusepath{fill}%
\end{pgfscope}%
\begin{pgfscope}%
\pgfpathrectangle{\pgfqpoint{1.020000in}{0.880000in}}{\pgfqpoint{6.160000in}{6.160000in}}%
\pgfusepath{clip}%
\pgfsetbuttcap%
\pgfsetroundjoin%
\definecolor{currentfill}{rgb}{0.952761,0.782965,0.698646}%
\pgfsetfillcolor{currentfill}%
\pgfsetlinewidth{0.000000pt}%
\definecolor{currentstroke}{rgb}{0.000000,0.000000,0.000000}%
\pgfsetstrokecolor{currentstroke}%
\pgfsetdash{}{0pt}%
\pgfpathmoveto{\pgfqpoint{3.602884in}{4.511242in}}%
\pgfpathlineto{\pgfqpoint{3.611155in}{4.578981in}}%
\pgfpathlineto{\pgfqpoint{3.621052in}{4.360240in}}%
\pgfpathlineto{\pgfqpoint{3.655209in}{4.205068in}}%
\pgfpathlineto{\pgfqpoint{3.687610in}{4.382684in}}%
\pgfpathlineto{\pgfqpoint{3.679104in}{4.338670in}}%
\pgfpathlineto{\pgfqpoint{3.669750in}{4.466614in}}%
\pgfpathlineto{\pgfqpoint{3.636245in}{4.505270in}}%
\pgfpathlineto{\pgfqpoint{3.602884in}{4.511242in}}%
\pgfpathclose%
\pgfusepath{fill}%
\end{pgfscope}%
\begin{pgfscope}%
\pgfpathrectangle{\pgfqpoint{1.020000in}{0.880000in}}{\pgfqpoint{6.160000in}{6.160000in}}%
\pgfusepath{clip}%
\pgfsetbuttcap%
\pgfsetroundjoin%
\definecolor{currentfill}{rgb}{0.865391,0.371128,0.295769}%
\pgfsetfillcolor{currentfill}%
\pgfsetlinewidth{0.000000pt}%
\definecolor{currentstroke}{rgb}{0.000000,0.000000,0.000000}%
\pgfsetstrokecolor{currentstroke}%
\pgfsetdash{}{0pt}%
\pgfpathmoveto{\pgfqpoint{2.546404in}{5.053276in}}%
\pgfpathlineto{\pgfqpoint{2.553432in}{5.085871in}}%
\pgfpathlineto{\pgfqpoint{2.563256in}{4.944526in}}%
\pgfpathlineto{\pgfqpoint{2.593271in}{5.175662in}}%
\pgfpathlineto{\pgfqpoint{2.627413in}{5.145720in}}%
\pgfpathlineto{\pgfqpoint{2.621295in}{5.046805in}}%
\pgfpathlineto{\pgfqpoint{2.612932in}{5.095231in}}%
\pgfpathlineto{\pgfqpoint{2.580204in}{5.040465in}}%
\pgfpathlineto{\pgfqpoint{2.546404in}{5.053276in}}%
\pgfpathclose%
\pgfusepath{fill}%
\end{pgfscope}%
\begin{pgfscope}%
\pgfpathrectangle{\pgfqpoint{1.020000in}{0.880000in}}{\pgfqpoint{6.160000in}{6.160000in}}%
\pgfusepath{clip}%
\pgfsetbuttcap%
\pgfsetroundjoin%
\definecolor{currentfill}{rgb}{0.521696,0.659599,0.987736}%
\pgfsetfillcolor{currentfill}%
\pgfsetlinewidth{0.000000pt}%
\definecolor{currentstroke}{rgb}{0.000000,0.000000,0.000000}%
\pgfsetstrokecolor{currentstroke}%
\pgfsetdash{}{0pt}%
\pgfpathmoveto{\pgfqpoint{4.329411in}{3.528778in}}%
\pgfpathlineto{\pgfqpoint{4.338959in}{3.616433in}}%
\pgfpathlineto{\pgfqpoint{4.348150in}{3.464891in}}%
\pgfpathlineto{\pgfqpoint{4.381257in}{3.443728in}}%
\pgfpathlineto{\pgfqpoint{4.414312in}{3.414510in}}%
\pgfpathlineto{\pgfqpoint{4.405045in}{3.513249in}}%
\pgfpathlineto{\pgfqpoint{4.395559in}{3.508247in}}%
\pgfpathlineto{\pgfqpoint{4.362087in}{3.288700in}}%
\pgfpathlineto{\pgfqpoint{4.329411in}{3.528778in}}%
\pgfpathclose%
\pgfusepath{fill}%
\end{pgfscope}%
\begin{pgfscope}%
\pgfpathrectangle{\pgfqpoint{1.020000in}{0.880000in}}{\pgfqpoint{6.160000in}{6.160000in}}%
\pgfusepath{clip}%
\pgfsetbuttcap%
\pgfsetroundjoin%
\definecolor{currentfill}{rgb}{0.835027,0.313644,0.259783}%
\pgfsetfillcolor{currentfill}%
\pgfsetlinewidth{0.000000pt}%
\definecolor{currentstroke}{rgb}{0.000000,0.000000,0.000000}%
\pgfsetstrokecolor{currentstroke}%
\pgfsetdash{}{0pt}%
\pgfpathmoveto{\pgfqpoint{2.612932in}{5.095231in}}%
\pgfpathlineto{\pgfqpoint{2.621295in}{5.046805in}}%
\pgfpathlineto{\pgfqpoint{2.627413in}{5.145720in}}%
\pgfpathlineto{\pgfqpoint{2.662464in}{5.052450in}}%
\pgfpathlineto{\pgfqpoint{2.694421in}{5.164999in}}%
\pgfpathlineto{\pgfqpoint{2.684559in}{5.311624in}}%
\pgfpathlineto{\pgfqpoint{2.678853in}{5.176709in}}%
\pgfpathlineto{\pgfqpoint{2.646571in}{5.090838in}}%
\pgfpathlineto{\pgfqpoint{2.612932in}{5.095231in}}%
\pgfpathclose%
\pgfusepath{fill}%
\end{pgfscope}%
\begin{pgfscope}%
\pgfpathrectangle{\pgfqpoint{1.020000in}{0.880000in}}{\pgfqpoint{6.160000in}{6.160000in}}%
\pgfusepath{clip}%
\pgfsetbuttcap%
\pgfsetroundjoin%
\definecolor{currentfill}{rgb}{0.931831,0.519086,0.406480}%
\pgfsetfillcolor{currentfill}%
\pgfsetlinewidth{0.000000pt}%
\definecolor{currentstroke}{rgb}{0.000000,0.000000,0.000000}%
\pgfsetstrokecolor{currentstroke}%
\pgfsetdash{}{0pt}%
\pgfpathmoveto{\pgfqpoint{3.349243in}{4.860143in}}%
\pgfpathlineto{\pgfqpoint{3.356489in}{4.999825in}}%
\pgfpathlineto{\pgfqpoint{3.364194in}{5.087683in}}%
\pgfpathlineto{\pgfqpoint{3.398568in}{4.975398in}}%
\pgfpathlineto{\pgfqpoint{3.434291in}{4.668149in}}%
\pgfpathlineto{\pgfqpoint{3.424977in}{4.783685in}}%
\pgfpathlineto{\pgfqpoint{3.416550in}{4.781789in}}%
\pgfpathlineto{\pgfqpoint{3.382360in}{4.891760in}}%
\pgfpathlineto{\pgfqpoint{3.349243in}{4.860143in}}%
\pgfpathclose%
\pgfusepath{fill}%
\end{pgfscope}%
\begin{pgfscope}%
\pgfpathrectangle{\pgfqpoint{1.020000in}{0.880000in}}{\pgfqpoint{6.160000in}{6.160000in}}%
\pgfusepath{clip}%
\pgfsetbuttcap%
\pgfsetroundjoin%
\definecolor{currentfill}{rgb}{0.373552,0.497499,0.909467}%
\pgfsetfillcolor{currentfill}%
\pgfsetlinewidth{0.000000pt}%
\definecolor{currentstroke}{rgb}{0.000000,0.000000,0.000000}%
\pgfsetstrokecolor{currentstroke}%
\pgfsetdash{}{0pt}%
\pgfpathmoveto{\pgfqpoint{4.782081in}{3.128786in}}%
\pgfpathlineto{\pgfqpoint{4.793254in}{3.340423in}}%
\pgfpathlineto{\pgfqpoint{4.802401in}{3.215455in}}%
\pgfpathlineto{\pgfqpoint{4.834697in}{3.103933in}}%
\pgfpathlineto{\pgfqpoint{4.868345in}{3.202880in}}%
\pgfpathlineto{\pgfqpoint{4.858419in}{3.209781in}}%
\pgfpathlineto{\pgfqpoint{4.848308in}{3.186078in}}%
\pgfpathlineto{\pgfqpoint{4.815531in}{3.210552in}}%
\pgfpathlineto{\pgfqpoint{4.782081in}{3.128786in}}%
\pgfpathclose%
\pgfusepath{fill}%
\end{pgfscope}%
\begin{pgfscope}%
\pgfpathrectangle{\pgfqpoint{1.020000in}{0.880000in}}{\pgfqpoint{6.160000in}{6.160000in}}%
\pgfusepath{clip}%
\pgfsetbuttcap%
\pgfsetroundjoin%
\definecolor{currentfill}{rgb}{0.880896,0.402331,0.317115}%
\pgfsetfillcolor{currentfill}%
\pgfsetlinewidth{0.000000pt}%
\definecolor{currentstroke}{rgb}{0.000000,0.000000,0.000000}%
\pgfsetstrokecolor{currentstroke}%
\pgfsetdash{}{0pt}%
\pgfpathmoveto{\pgfqpoint{3.263750in}{5.094087in}}%
\pgfpathlineto{\pgfqpoint{3.271625in}{5.141824in}}%
\pgfpathlineto{\pgfqpoint{3.280469in}{5.083313in}}%
\pgfpathlineto{\pgfqpoint{3.313584in}{5.125974in}}%
\pgfpathlineto{\pgfqpoint{3.349243in}{4.860143in}}%
\pgfpathlineto{\pgfqpoint{3.339674in}{5.005004in}}%
\pgfpathlineto{\pgfqpoint{3.331651in}{4.964842in}}%
\pgfpathlineto{\pgfqpoint{3.297739in}{5.028624in}}%
\pgfpathlineto{\pgfqpoint{3.263750in}{5.094087in}}%
\pgfpathclose%
\pgfusepath{fill}%
\end{pgfscope}%
\begin{pgfscope}%
\pgfpathrectangle{\pgfqpoint{1.020000in}{0.880000in}}{\pgfqpoint{6.160000in}{6.160000in}}%
\pgfusepath{clip}%
\pgfsetbuttcap%
\pgfsetroundjoin%
\definecolor{currentfill}{rgb}{0.735077,0.104460,0.171492}%
\pgfsetfillcolor{currentfill}%
\pgfsetlinewidth{0.000000pt}%
\definecolor{currentstroke}{rgb}{0.000000,0.000000,0.000000}%
\pgfsetstrokecolor{currentstroke}%
\pgfsetdash{}{0pt}%
\pgfpathmoveto{\pgfqpoint{2.811211in}{5.307976in}}%
\pgfpathlineto{\pgfqpoint{2.819826in}{5.249363in}}%
\pgfpathlineto{\pgfqpoint{2.825366in}{5.422365in}}%
\pgfpathlineto{\pgfqpoint{2.858813in}{5.436360in}}%
\pgfpathlineto{\pgfqpoint{2.893043in}{5.387701in}}%
\pgfpathlineto{\pgfqpoint{2.887401in}{5.208910in}}%
\pgfpathlineto{\pgfqpoint{2.877310in}{5.380874in}}%
\pgfpathlineto{\pgfqpoint{2.846586in}{5.167273in}}%
\pgfpathlineto{\pgfqpoint{2.811211in}{5.307976in}}%
\pgfpathclose%
\pgfusepath{fill}%
\end{pgfscope}%
\begin{pgfscope}%
\pgfpathrectangle{\pgfqpoint{1.020000in}{0.880000in}}{\pgfqpoint{6.160000in}{6.160000in}}%
\pgfusepath{clip}%
\pgfsetbuttcap%
\pgfsetroundjoin%
\definecolor{currentfill}{rgb}{0.285273,0.380129,0.823469}%
\pgfsetfillcolor{currentfill}%
\pgfsetlinewidth{0.000000pt}%
\definecolor{currentstroke}{rgb}{0.000000,0.000000,0.000000}%
\pgfsetstrokecolor{currentstroke}%
\pgfsetdash{}{0pt}%
\pgfpathmoveto{\pgfqpoint{5.085389in}{3.098282in}}%
\pgfpathlineto{\pgfqpoint{5.095601in}{3.097591in}}%
\pgfpathlineto{\pgfqpoint{5.104463in}{2.948689in}}%
\pgfpathlineto{\pgfqpoint{5.137465in}{2.958517in}}%
\pgfpathlineto{\pgfqpoint{5.170187in}{2.941324in}}%
\pgfpathlineto{\pgfqpoint{5.160979in}{3.051311in}}%
\pgfpathlineto{\pgfqpoint{5.150724in}{3.053525in}}%
\pgfpathlineto{\pgfqpoint{5.117836in}{3.050555in}}%
\pgfpathlineto{\pgfqpoint{5.085389in}{3.098282in}}%
\pgfpathclose%
\pgfusepath{fill}%
\end{pgfscope}%
\begin{pgfscope}%
\pgfpathrectangle{\pgfqpoint{1.020000in}{0.880000in}}{\pgfqpoint{6.160000in}{6.160000in}}%
\pgfusepath{clip}%
\pgfsetbuttcap%
\pgfsetroundjoin%
\definecolor{currentfill}{rgb}{0.353369,0.472069,0.892570}%
\pgfsetfillcolor{currentfill}%
\pgfsetlinewidth{0.000000pt}%
\definecolor{currentstroke}{rgb}{0.000000,0.000000,0.000000}%
\pgfsetstrokecolor{currentstroke}%
\pgfsetdash{}{0pt}%
\pgfpathmoveto{\pgfqpoint{4.716900in}{3.267529in}}%
\pgfpathlineto{\pgfqpoint{4.725342in}{3.012464in}}%
\pgfpathlineto{\pgfqpoint{4.736072in}{3.175567in}}%
\pgfpathlineto{\pgfqpoint{4.768478in}{3.065910in}}%
\pgfpathlineto{\pgfqpoint{4.802401in}{3.215455in}}%
\pgfpathlineto{\pgfqpoint{4.793254in}{3.340423in}}%
\pgfpathlineto{\pgfqpoint{4.782081in}{3.128786in}}%
\pgfpathlineto{\pgfqpoint{4.748789in}{3.068826in}}%
\pgfpathlineto{\pgfqpoint{4.716900in}{3.267529in}}%
\pgfpathclose%
\pgfusepath{fill}%
\end{pgfscope}%
\begin{pgfscope}%
\pgfpathrectangle{\pgfqpoint{1.020000in}{0.880000in}}{\pgfqpoint{6.160000in}{6.160000in}}%
\pgfusepath{clip}%
\pgfsetbuttcap%
\pgfsetroundjoin%
\definecolor{currentfill}{rgb}{0.969683,0.690484,0.575138}%
\pgfsetfillcolor{currentfill}%
\pgfsetlinewidth{0.000000pt}%
\definecolor{currentstroke}{rgb}{0.000000,0.000000,0.000000}%
\pgfsetstrokecolor{currentstroke}%
\pgfsetdash{}{0pt}%
\pgfpathmoveto{\pgfqpoint{3.518179in}{4.642566in}}%
\pgfpathlineto{\pgfqpoint{3.526037in}{4.751472in}}%
\pgfpathlineto{\pgfqpoint{3.534646in}{4.748804in}}%
\pgfpathlineto{\pgfqpoint{3.569456in}{4.524250in}}%
\pgfpathlineto{\pgfqpoint{3.602884in}{4.511242in}}%
\pgfpathlineto{\pgfqpoint{3.594363in}{4.491497in}}%
\pgfpathlineto{\pgfqpoint{3.585613in}{4.513928in}}%
\pgfpathlineto{\pgfqpoint{3.551539in}{4.640338in}}%
\pgfpathlineto{\pgfqpoint{3.518179in}{4.642566in}}%
\pgfpathclose%
\pgfusepath{fill}%
\end{pgfscope}%
\begin{pgfscope}%
\pgfpathrectangle{\pgfqpoint{1.020000in}{0.880000in}}{\pgfqpoint{6.160000in}{6.160000in}}%
\pgfusepath{clip}%
\pgfsetbuttcap%
\pgfsetroundjoin%
\definecolor{currentfill}{rgb}{0.457046,0.594006,0.963029}%
\pgfsetfillcolor{currentfill}%
\pgfsetlinewidth{0.000000pt}%
\definecolor{currentstroke}{rgb}{0.000000,0.000000,0.000000}%
\pgfsetstrokecolor{currentstroke}%
\pgfsetdash{}{0pt}%
\pgfpathmoveto{\pgfqpoint{4.414312in}{3.414510in}}%
\pgfpathlineto{\pgfqpoint{4.423770in}{3.394836in}}%
\pgfpathlineto{\pgfqpoint{4.433092in}{3.313767in}}%
\pgfpathlineto{\pgfqpoint{4.466588in}{3.445137in}}%
\pgfpathlineto{\pgfqpoint{4.499492in}{3.367992in}}%
\pgfpathlineto{\pgfqpoint{4.489915in}{3.370962in}}%
\pgfpathlineto{\pgfqpoint{4.480249in}{3.337336in}}%
\pgfpathlineto{\pgfqpoint{4.446742in}{3.163478in}}%
\pgfpathlineto{\pgfqpoint{4.414312in}{3.414510in}}%
\pgfpathclose%
\pgfusepath{fill}%
\end{pgfscope}%
\begin{pgfscope}%
\pgfpathrectangle{\pgfqpoint{1.020000in}{0.880000in}}{\pgfqpoint{6.160000in}{6.160000in}}%
\pgfusepath{clip}%
\pgfsetbuttcap%
\pgfsetroundjoin%
\definecolor{currentfill}{rgb}{0.559747,0.694768,0.996075}%
\pgfsetfillcolor{currentfill}%
\pgfsetlinewidth{0.000000pt}%
\definecolor{currentstroke}{rgb}{0.000000,0.000000,0.000000}%
\pgfsetstrokecolor{currentstroke}%
\pgfsetdash{}{0pt}%
\pgfpathmoveto{\pgfqpoint{4.111882in}{3.678718in}}%
\pgfpathlineto{\pgfqpoint{4.121198in}{3.480142in}}%
\pgfpathlineto{\pgfqpoint{4.130411in}{3.446343in}}%
\pgfpathlineto{\pgfqpoint{4.163608in}{3.435340in}}%
\pgfpathlineto{\pgfqpoint{4.196783in}{3.456992in}}%
\pgfpathlineto{\pgfqpoint{4.187529in}{3.510599in}}%
\pgfpathlineto{\pgfqpoint{4.178278in}{3.684413in}}%
\pgfpathlineto{\pgfqpoint{4.145114in}{3.614256in}}%
\pgfpathlineto{\pgfqpoint{4.111882in}{3.678718in}}%
\pgfpathclose%
\pgfusepath{fill}%
\end{pgfscope}%
\begin{pgfscope}%
\pgfpathrectangle{\pgfqpoint{1.020000in}{0.880000in}}{\pgfqpoint{6.160000in}{6.160000in}}%
\pgfusepath{clip}%
\pgfsetbuttcap%
\pgfsetroundjoin%
\definecolor{currentfill}{rgb}{0.441123,0.576532,0.954545}%
\pgfsetfillcolor{currentfill}%
\pgfsetlinewidth{0.000000pt}%
\definecolor{currentstroke}{rgb}{0.000000,0.000000,0.000000}%
\pgfsetstrokecolor{currentstroke}%
\pgfsetdash{}{0pt}%
\pgfpathmoveto{\pgfqpoint{4.632105in}{3.438809in}}%
\pgfpathlineto{\pgfqpoint{4.641637in}{3.392968in}}%
\pgfpathlineto{\pgfqpoint{4.651052in}{3.319836in}}%
\pgfpathlineto{\pgfqpoint{4.683647in}{3.224449in}}%
\pgfpathlineto{\pgfqpoint{4.716900in}{3.267529in}}%
\pgfpathlineto{\pgfqpoint{4.707335in}{3.314158in}}%
\pgfpathlineto{\pgfqpoint{4.697611in}{3.327642in}}%
\pgfpathlineto{\pgfqpoint{4.664710in}{3.346652in}}%
\pgfpathlineto{\pgfqpoint{4.632105in}{3.438809in}}%
\pgfpathclose%
\pgfusepath{fill}%
\end{pgfscope}%
\begin{pgfscope}%
\pgfpathrectangle{\pgfqpoint{1.020000in}{0.880000in}}{\pgfqpoint{6.160000in}{6.160000in}}%
\pgfusepath{clip}%
\pgfsetbuttcap%
\pgfsetroundjoin%
\definecolor{currentfill}{rgb}{0.313946,0.420052,0.854993}%
\pgfsetfillcolor{currentfill}%
\pgfsetlinewidth{0.000000pt}%
\definecolor{currentstroke}{rgb}{0.000000,0.000000,0.000000}%
\pgfsetstrokecolor{currentstroke}%
\pgfsetdash{}{0pt}%
\pgfpathmoveto{\pgfqpoint{4.868345in}{3.202880in}}%
\pgfpathlineto{\pgfqpoint{4.878256in}{3.191015in}}%
\pgfpathlineto{\pgfqpoint{4.886643in}{2.960671in}}%
\pgfpathlineto{\pgfqpoint{4.920234in}{3.045258in}}%
\pgfpathlineto{\pgfqpoint{4.953557in}{3.090357in}}%
\pgfpathlineto{\pgfqpoint{4.944213in}{3.185348in}}%
\pgfpathlineto{\pgfqpoint{4.932870in}{3.014400in}}%
\pgfpathlineto{\pgfqpoint{4.899976in}{3.015227in}}%
\pgfpathlineto{\pgfqpoint{4.868345in}{3.202880in}}%
\pgfpathclose%
\pgfusepath{fill}%
\end{pgfscope}%
\begin{pgfscope}%
\pgfpathrectangle{\pgfqpoint{1.020000in}{0.880000in}}{\pgfqpoint{6.160000in}{6.160000in}}%
\pgfusepath{clip}%
\pgfsetbuttcap%
\pgfsetroundjoin%
\definecolor{currentfill}{rgb}{0.313946,0.420052,0.854993}%
\pgfsetfillcolor{currentfill}%
\pgfsetlinewidth{0.000000pt}%
\definecolor{currentstroke}{rgb}{0.000000,0.000000,0.000000}%
\pgfsetstrokecolor{currentstroke}%
\pgfsetdash{}{0pt}%
\pgfpathmoveto{\pgfqpoint{5.369981in}{3.183630in}}%
\pgfpathlineto{\pgfqpoint{5.378424in}{3.010165in}}%
\pgfpathlineto{\pgfqpoint{5.390190in}{3.112289in}}%
\pgfpathlineto{\pgfqpoint{5.422119in}{3.036622in}}%
\pgfpathlineto{\pgfqpoint{5.456589in}{3.165376in}}%
\pgfpathlineto{\pgfqpoint{5.445444in}{3.123048in}}%
\pgfpathlineto{\pgfqpoint{5.432700in}{2.950852in}}%
\pgfpathlineto{\pgfqpoint{5.401273in}{3.057784in}}%
\pgfpathlineto{\pgfqpoint{5.369981in}{3.183630in}}%
\pgfpathclose%
\pgfusepath{fill}%
\end{pgfscope}%
\begin{pgfscope}%
\pgfpathrectangle{\pgfqpoint{1.020000in}{0.880000in}}{\pgfqpoint{6.160000in}{6.160000in}}%
\pgfusepath{clip}%
\pgfsetbuttcap%
\pgfsetroundjoin%
\definecolor{currentfill}{rgb}{0.343278,0.459354,0.884122}%
\pgfsetfillcolor{currentfill}%
\pgfsetlinewidth{0.000000pt}%
\definecolor{currentstroke}{rgb}{0.000000,0.000000,0.000000}%
\pgfsetstrokecolor{currentstroke}%
\pgfsetdash{}{0pt}%
\pgfpathmoveto{\pgfqpoint{5.432700in}{2.950852in}}%
\pgfpathlineto{\pgfqpoint{5.445444in}{3.123048in}}%
\pgfpathlineto{\pgfqpoint{5.456589in}{3.165376in}}%
\pgfpathlineto{\pgfqpoint{5.489615in}{3.178832in}}%
\pgfpathlineto{\pgfqpoint{5.522543in}{3.185865in}}%
\pgfpathlineto{\pgfqpoint{5.511457in}{3.154971in}}%
\pgfpathlineto{\pgfqpoint{5.501925in}{3.239455in}}%
\pgfpathlineto{\pgfqpoint{5.467793in}{3.135392in}}%
\pgfpathlineto{\pgfqpoint{5.432700in}{2.950852in}}%
\pgfpathclose%
\pgfusepath{fill}%
\end{pgfscope}%
\begin{pgfscope}%
\pgfpathrectangle{\pgfqpoint{1.020000in}{0.880000in}}{\pgfqpoint{6.160000in}{6.160000in}}%
\pgfusepath{clip}%
\pgfsetbuttcap%
\pgfsetroundjoin%
\definecolor{currentfill}{rgb}{0.758112,0.168122,0.188827}%
\pgfsetfillcolor{currentfill}%
\pgfsetlinewidth{0.000000pt}%
\definecolor{currentstroke}{rgb}{0.000000,0.000000,0.000000}%
\pgfsetstrokecolor{currentstroke}%
\pgfsetdash{}{0pt}%
\pgfpathmoveto{\pgfqpoint{2.960609in}{5.352035in}}%
\pgfpathlineto{\pgfqpoint{2.970557in}{5.188247in}}%
\pgfpathlineto{\pgfqpoint{2.977520in}{5.275881in}}%
\pgfpathlineto{\pgfqpoint{3.009346in}{5.427281in}}%
\pgfpathlineto{\pgfqpoint{3.042270in}{5.487655in}}%
\pgfpathlineto{\pgfqpoint{3.036709in}{5.263045in}}%
\pgfpathlineto{\pgfqpoint{3.030114in}{5.134315in}}%
\pgfpathlineto{\pgfqpoint{2.996967in}{5.109421in}}%
\pgfpathlineto{\pgfqpoint{2.960609in}{5.352035in}}%
\pgfpathclose%
\pgfusepath{fill}%
\end{pgfscope}%
\begin{pgfscope}%
\pgfpathrectangle{\pgfqpoint{1.020000in}{0.880000in}}{\pgfqpoint{6.160000in}{6.160000in}}%
\pgfusepath{clip}%
\pgfsetbuttcap%
\pgfsetroundjoin%
\definecolor{currentfill}{rgb}{0.333490,0.446265,0.874452}%
\pgfsetfillcolor{currentfill}%
\pgfsetlinewidth{0.000000pt}%
\definecolor{currentstroke}{rgb}{0.000000,0.000000,0.000000}%
\pgfsetstrokecolor{currentstroke}%
\pgfsetdash{}{0pt}%
\pgfpathmoveto{\pgfqpoint{5.652516in}{3.105935in}}%
\pgfpathlineto{\pgfqpoint{5.663718in}{3.131972in}}%
\pgfpathlineto{\pgfqpoint{5.674883in}{3.153695in}}%
\pgfpathlineto{\pgfqpoint{5.708125in}{3.183154in}}%
\pgfpathlineto{\pgfqpoint{5.738696in}{3.041498in}}%
\pgfpathlineto{\pgfqpoint{5.728339in}{3.076157in}}%
\pgfpathlineto{\pgfqpoint{5.718307in}{3.130883in}}%
\pgfpathlineto{\pgfqpoint{5.686110in}{3.164373in}}%
\pgfpathlineto{\pgfqpoint{5.652516in}{3.105935in}}%
\pgfpathclose%
\pgfusepath{fill}%
\end{pgfscope}%
\begin{pgfscope}%
\pgfpathrectangle{\pgfqpoint{1.020000in}{0.880000in}}{\pgfqpoint{6.160000in}{6.160000in}}%
\pgfusepath{clip}%
\pgfsetbuttcap%
\pgfsetroundjoin%
\definecolor{currentfill}{rgb}{0.954853,0.591622,0.471337}%
\pgfsetfillcolor{currentfill}%
\pgfsetlinewidth{0.000000pt}%
\definecolor{currentstroke}{rgb}{0.000000,0.000000,0.000000}%
\pgfsetstrokecolor{currentstroke}%
\pgfsetdash{}{0pt}%
\pgfpathmoveto{\pgfqpoint{2.362716in}{4.853638in}}%
\pgfpathlineto{\pgfqpoint{2.371573in}{4.770438in}}%
\pgfpathlineto{\pgfqpoint{2.382151in}{4.590743in}}%
\pgfpathlineto{\pgfqpoint{2.413757in}{4.710163in}}%
\pgfpathlineto{\pgfqpoint{2.446490in}{4.767370in}}%
\pgfpathlineto{\pgfqpoint{2.436605in}{4.909137in}}%
\pgfpathlineto{\pgfqpoint{2.431467in}{4.775952in}}%
\pgfpathlineto{\pgfqpoint{2.398699in}{4.724826in}}%
\pgfpathlineto{\pgfqpoint{2.362716in}{4.853638in}}%
\pgfpathclose%
\pgfusepath{fill}%
\end{pgfscope}%
\begin{pgfscope}%
\pgfpathrectangle{\pgfqpoint{1.020000in}{0.880000in}}{\pgfqpoint{6.160000in}{6.160000in}}%
\pgfusepath{clip}%
\pgfsetbuttcap%
\pgfsetroundjoin%
\definecolor{currentfill}{rgb}{0.672538,0.782861,0.991982}%
\pgfsetfillcolor{currentfill}%
\pgfsetlinewidth{0.000000pt}%
\definecolor{currentstroke}{rgb}{0.000000,0.000000,0.000000}%
\pgfsetstrokecolor{currentstroke}%
\pgfsetdash{}{0pt}%
\pgfpathmoveto{\pgfqpoint{4.027000in}{3.812333in}}%
\pgfpathlineto{\pgfqpoint{4.036099in}{3.814558in}}%
\pgfpathlineto{\pgfqpoint{4.045314in}{3.746443in}}%
\pgfpathlineto{\pgfqpoint{4.078613in}{3.718150in}}%
\pgfpathlineto{\pgfqpoint{4.111882in}{3.678718in}}%
\pgfpathlineto{\pgfqpoint{4.102688in}{3.703174in}}%
\pgfpathlineto{\pgfqpoint{4.093482in}{3.757332in}}%
\pgfpathlineto{\pgfqpoint{4.060400in}{3.670463in}}%
\pgfpathlineto{\pgfqpoint{4.027000in}{3.812333in}}%
\pgfpathclose%
\pgfusepath{fill}%
\end{pgfscope}%
\begin{pgfscope}%
\pgfpathrectangle{\pgfqpoint{1.020000in}{0.880000in}}{\pgfqpoint{6.160000in}{6.160000in}}%
\pgfusepath{clip}%
\pgfsetbuttcap%
\pgfsetroundjoin%
\definecolor{currentfill}{rgb}{0.967317,0.657471,0.538160}%
\pgfsetfillcolor{currentfill}%
\pgfsetlinewidth{0.000000pt}%
\definecolor{currentstroke}{rgb}{0.000000,0.000000,0.000000}%
\pgfsetstrokecolor{currentstroke}%
\pgfsetdash{}{0pt}%
\pgfpathmoveto{\pgfqpoint{2.230789in}{4.691534in}}%
\pgfpathlineto{\pgfqpoint{2.240485in}{4.562919in}}%
\pgfpathlineto{\pgfqpoint{2.248229in}{4.536179in}}%
\pgfpathlineto{\pgfqpoint{2.279330in}{4.677191in}}%
\pgfpathlineto{\pgfqpoint{2.311708in}{4.753099in}}%
\pgfpathlineto{\pgfqpoint{2.305180in}{4.710844in}}%
\pgfpathlineto{\pgfqpoint{2.299118in}{4.644707in}}%
\pgfpathlineto{\pgfqpoint{2.264669in}{4.684086in}}%
\pgfpathlineto{\pgfqpoint{2.230789in}{4.691534in}}%
\pgfpathclose%
\pgfusepath{fill}%
\end{pgfscope}%
\begin{pgfscope}%
\pgfpathrectangle{\pgfqpoint{1.020000in}{0.880000in}}{\pgfqpoint{6.160000in}{6.160000in}}%
\pgfusepath{clip}%
\pgfsetbuttcap%
\pgfsetroundjoin%
\definecolor{currentfill}{rgb}{0.960490,0.616276,0.495467}%
\pgfsetfillcolor{currentfill}%
\pgfsetlinewidth{0.000000pt}%
\definecolor{currentstroke}{rgb}{0.000000,0.000000,0.000000}%
\pgfsetstrokecolor{currentstroke}%
\pgfsetdash{}{0pt}%
\pgfpathmoveto{\pgfqpoint{2.299118in}{4.644707in}}%
\pgfpathlineto{\pgfqpoint{2.305180in}{4.710844in}}%
\pgfpathlineto{\pgfqpoint{2.311708in}{4.753099in}}%
\pgfpathlineto{\pgfqpoint{2.345787in}{4.736956in}}%
\pgfpathlineto{\pgfqpoint{2.382151in}{4.590743in}}%
\pgfpathlineto{\pgfqpoint{2.371573in}{4.770438in}}%
\pgfpathlineto{\pgfqpoint{2.362716in}{4.853638in}}%
\pgfpathlineto{\pgfqpoint{2.331335in}{4.724943in}}%
\pgfpathlineto{\pgfqpoint{2.299118in}{4.644707in}}%
\pgfpathclose%
\pgfusepath{fill}%
\end{pgfscope}%
\begin{pgfscope}%
\pgfpathrectangle{\pgfqpoint{1.020000in}{0.880000in}}{\pgfqpoint{6.160000in}{6.160000in}}%
\pgfusepath{clip}%
\pgfsetbuttcap%
\pgfsetroundjoin%
\definecolor{currentfill}{rgb}{0.294718,0.393542,0.834384}%
\pgfsetfillcolor{currentfill}%
\pgfsetlinewidth{0.000000pt}%
\definecolor{currentstroke}{rgb}{0.000000,0.000000,0.000000}%
\pgfsetstrokecolor{currentstroke}%
\pgfsetdash{}{0pt}%
\pgfpathmoveto{\pgfqpoint{5.019390in}{3.080052in}}%
\pgfpathlineto{\pgfqpoint{5.028361in}{2.941005in}}%
\pgfpathlineto{\pgfqpoint{5.039298in}{3.031456in}}%
\pgfpathlineto{\pgfqpoint{5.072892in}{3.100819in}}%
\pgfpathlineto{\pgfqpoint{5.104463in}{2.948689in}}%
\pgfpathlineto{\pgfqpoint{5.095601in}{3.097591in}}%
\pgfpathlineto{\pgfqpoint{5.085389in}{3.098282in}}%
\pgfpathlineto{\pgfqpoint{5.052037in}{3.047968in}}%
\pgfpathlineto{\pgfqpoint{5.019390in}{3.080052in}}%
\pgfpathclose%
\pgfusepath{fill}%
\end{pgfscope}%
\begin{pgfscope}%
\pgfpathrectangle{\pgfqpoint{1.020000in}{0.880000in}}{\pgfqpoint{6.160000in}{6.160000in}}%
\pgfusepath{clip}%
\pgfsetbuttcap%
\pgfsetroundjoin%
\definecolor{currentfill}{rgb}{0.333490,0.446265,0.874452}%
\pgfsetfillcolor{currentfill}%
\pgfsetlinewidth{0.000000pt}%
\definecolor{currentstroke}{rgb}{0.000000,0.000000,0.000000}%
\pgfsetstrokecolor{currentstroke}%
\pgfsetdash{}{0pt}%
\pgfpathmoveto{\pgfqpoint{5.871675in}{3.173265in}}%
\pgfpathlineto{\pgfqpoint{5.882919in}{3.184201in}}%
\pgfpathlineto{\pgfqpoint{5.893261in}{3.141080in}}%
\pgfpathlineto{\pgfqpoint{5.923448in}{2.995328in}}%
\pgfpathlineto{\pgfqpoint{5.914307in}{3.107693in}}%
\pgfpathlineto{\pgfqpoint{5.902499in}{3.065548in}}%
\pgfpathlineto{\pgfqpoint{5.871675in}{3.173265in}}%
\pgfpathclose%
\pgfusepath{fill}%
\end{pgfscope}%
\begin{pgfscope}%
\pgfpathrectangle{\pgfqpoint{1.020000in}{0.880000in}}{\pgfqpoint{6.160000in}{6.160000in}}%
\pgfusepath{clip}%
\pgfsetbuttcap%
\pgfsetroundjoin%
\definecolor{currentfill}{rgb}{0.879622,0.858175,0.845844}%
\pgfsetfillcolor{currentfill}%
\pgfsetlinewidth{0.000000pt}%
\definecolor{currentstroke}{rgb}{0.000000,0.000000,0.000000}%
\pgfsetstrokecolor{currentstroke}%
\pgfsetdash{}{0pt}%
\pgfpathmoveto{\pgfqpoint{3.705600in}{4.280714in}}%
\pgfpathlineto{\pgfqpoint{3.714593in}{4.233250in}}%
\pgfpathlineto{\pgfqpoint{3.723417in}{4.224999in}}%
\pgfpathlineto{\pgfqpoint{3.757229in}{4.125221in}}%
\pgfpathlineto{\pgfqpoint{3.791213in}{3.959313in}}%
\pgfpathlineto{\pgfqpoint{3.781155in}{4.268001in}}%
\pgfpathlineto{\pgfqpoint{3.772673in}{4.181143in}}%
\pgfpathlineto{\pgfqpoint{3.739875in}{4.069377in}}%
\pgfpathlineto{\pgfqpoint{3.705600in}{4.280714in}}%
\pgfpathclose%
\pgfusepath{fill}%
\end{pgfscope}%
\begin{pgfscope}%
\pgfpathrectangle{\pgfqpoint{1.020000in}{0.880000in}}{\pgfqpoint{6.160000in}{6.160000in}}%
\pgfusepath{clip}%
\pgfsetbuttcap%
\pgfsetroundjoin%
\definecolor{currentfill}{rgb}{0.559747,0.694768,0.996075}%
\pgfsetfillcolor{currentfill}%
\pgfsetlinewidth{0.000000pt}%
\definecolor{currentstroke}{rgb}{0.000000,0.000000,0.000000}%
\pgfsetstrokecolor{currentstroke}%
\pgfsetdash{}{0pt}%
\pgfpathmoveto{\pgfqpoint{4.263186in}{3.585650in}}%
\pgfpathlineto{\pgfqpoint{4.272501in}{3.542618in}}%
\pgfpathlineto{\pgfqpoint{4.281723in}{3.393410in}}%
\pgfpathlineto{\pgfqpoint{4.315044in}{3.512479in}}%
\pgfpathlineto{\pgfqpoint{4.348150in}{3.464891in}}%
\pgfpathlineto{\pgfqpoint{4.338959in}{3.616433in}}%
\pgfpathlineto{\pgfqpoint{4.329411in}{3.528778in}}%
\pgfpathlineto{\pgfqpoint{4.296321in}{3.560376in}}%
\pgfpathlineto{\pgfqpoint{4.263186in}{3.585650in}}%
\pgfpathclose%
\pgfusepath{fill}%
\end{pgfscope}%
\begin{pgfscope}%
\pgfpathrectangle{\pgfqpoint{1.020000in}{0.880000in}}{\pgfqpoint{6.160000in}{6.160000in}}%
\pgfusepath{clip}%
\pgfsetbuttcap%
\pgfsetroundjoin%
\definecolor{currentfill}{rgb}{0.743754,0.825125,0.965798}%
\pgfsetfillcolor{currentfill}%
\pgfsetlinewidth{0.000000pt}%
\definecolor{currentstroke}{rgb}{0.000000,0.000000,0.000000}%
\pgfsetstrokecolor{currentstroke}%
\pgfsetdash{}{0pt}%
\pgfpathmoveto{\pgfqpoint{3.875098in}{4.109948in}}%
\pgfpathlineto{\pgfqpoint{3.884447in}{3.979520in}}%
\pgfpathlineto{\pgfqpoint{3.893895in}{3.813359in}}%
\pgfpathlineto{\pgfqpoint{3.927181in}{3.821045in}}%
\pgfpathlineto{\pgfqpoint{3.960466in}{3.822014in}}%
\pgfpathlineto{\pgfqpoint{3.951645in}{3.736478in}}%
\pgfpathlineto{\pgfqpoint{3.942355in}{3.860192in}}%
\pgfpathlineto{\pgfqpoint{3.909292in}{3.796839in}}%
\pgfpathlineto{\pgfqpoint{3.875098in}{4.109948in}}%
\pgfpathclose%
\pgfusepath{fill}%
\end{pgfscope}%
\begin{pgfscope}%
\pgfpathrectangle{\pgfqpoint{1.020000in}{0.880000in}}{\pgfqpoint{6.160000in}{6.160000in}}%
\pgfusepath{clip}%
\pgfsetbuttcap%
\pgfsetroundjoin%
\definecolor{currentfill}{rgb}{0.934305,0.525918,0.412286}%
\pgfsetfillcolor{currentfill}%
\pgfsetlinewidth{0.000000pt}%
\definecolor{currentstroke}{rgb}{0.000000,0.000000,0.000000}%
\pgfsetstrokecolor{currentstroke}%
\pgfsetdash{}{0pt}%
\pgfpathmoveto{\pgfqpoint{2.431467in}{4.775952in}}%
\pgfpathlineto{\pgfqpoint{2.436605in}{4.909137in}}%
\pgfpathlineto{\pgfqpoint{2.446490in}{4.767370in}}%
\pgfpathlineto{\pgfqpoint{2.478934in}{4.842832in}}%
\pgfpathlineto{\pgfqpoint{2.509534in}{5.032048in}}%
\pgfpathlineto{\pgfqpoint{2.504428in}{4.885558in}}%
\pgfpathlineto{\pgfqpoint{2.498427in}{4.795123in}}%
\pgfpathlineto{\pgfqpoint{2.463107in}{4.894481in}}%
\pgfpathlineto{\pgfqpoint{2.431467in}{4.775952in}}%
\pgfpathclose%
\pgfusepath{fill}%
\end{pgfscope}%
\begin{pgfscope}%
\pgfpathrectangle{\pgfqpoint{1.020000in}{0.880000in}}{\pgfqpoint{6.160000in}{6.160000in}}%
\pgfusepath{clip}%
\pgfsetbuttcap%
\pgfsetroundjoin%
\definecolor{currentfill}{rgb}{0.378598,0.503856,0.913692}%
\pgfsetfillcolor{currentfill}%
\pgfsetlinewidth{0.000000pt}%
\definecolor{currentstroke}{rgb}{0.000000,0.000000,0.000000}%
\pgfsetstrokecolor{currentstroke}%
\pgfsetdash{}{0pt}%
\pgfpathmoveto{\pgfqpoint{4.651052in}{3.319836in}}%
\pgfpathlineto{\pgfqpoint{4.659894in}{3.125487in}}%
\pgfpathlineto{\pgfqpoint{4.670719in}{3.340999in}}%
\pgfpathlineto{\pgfqpoint{4.702698in}{3.117356in}}%
\pgfpathlineto{\pgfqpoint{4.736072in}{3.175567in}}%
\pgfpathlineto{\pgfqpoint{4.725342in}{3.012464in}}%
\pgfpathlineto{\pgfqpoint{4.716900in}{3.267529in}}%
\pgfpathlineto{\pgfqpoint{4.683647in}{3.224449in}}%
\pgfpathlineto{\pgfqpoint{4.651052in}{3.319836in}}%
\pgfpathclose%
\pgfusepath{fill}%
\end{pgfscope}%
\begin{pgfscope}%
\pgfpathrectangle{\pgfqpoint{1.020000in}{0.880000in}}{\pgfqpoint{6.160000in}{6.160000in}}%
\pgfusepath{clip}%
\pgfsetbuttcap%
\pgfsetroundjoin%
\definecolor{currentfill}{rgb}{0.338377,0.452819,0.879317}%
\pgfsetfillcolor{currentfill}%
\pgfsetlinewidth{0.000000pt}%
\definecolor{currentstroke}{rgb}{0.000000,0.000000,0.000000}%
\pgfsetstrokecolor{currentstroke}%
\pgfsetdash{}{0pt}%
\pgfpathmoveto{\pgfqpoint{5.587259in}{3.122070in}}%
\pgfpathlineto{\pgfqpoint{5.597020in}{3.052158in}}%
\pgfpathlineto{\pgfqpoint{5.609252in}{3.154421in}}%
\pgfpathlineto{\pgfqpoint{5.641995in}{3.148421in}}%
\pgfpathlineto{\pgfqpoint{5.674883in}{3.153695in}}%
\pgfpathlineto{\pgfqpoint{5.663718in}{3.131972in}}%
\pgfpathlineto{\pgfqpoint{5.652516in}{3.105935in}}%
\pgfpathlineto{\pgfqpoint{5.620229in}{3.136824in}}%
\pgfpathlineto{\pgfqpoint{5.587259in}{3.122070in}}%
\pgfpathclose%
\pgfusepath{fill}%
\end{pgfscope}%
\begin{pgfscope}%
\pgfpathrectangle{\pgfqpoint{1.020000in}{0.880000in}}{\pgfqpoint{6.160000in}{6.160000in}}%
\pgfusepath{clip}%
\pgfsetbuttcap%
\pgfsetroundjoin%
\definecolor{currentfill}{rgb}{0.328604,0.439712,0.869587}%
\pgfsetfillcolor{currentfill}%
\pgfsetlinewidth{0.000000pt}%
\definecolor{currentstroke}{rgb}{0.000000,0.000000,0.000000}%
\pgfsetstrokecolor{currentstroke}%
\pgfsetdash{}{0pt}%
\pgfpathmoveto{\pgfqpoint{5.303633in}{3.127661in}}%
\pgfpathlineto{\pgfqpoint{5.314427in}{3.157482in}}%
\pgfpathlineto{\pgfqpoint{5.324443in}{3.116843in}}%
\pgfpathlineto{\pgfqpoint{5.355453in}{2.955419in}}%
\pgfpathlineto{\pgfqpoint{5.390190in}{3.112289in}}%
\pgfpathlineto{\pgfqpoint{5.378424in}{3.010165in}}%
\pgfpathlineto{\pgfqpoint{5.369981in}{3.183630in}}%
\pgfpathlineto{\pgfqpoint{5.336716in}{3.147745in}}%
\pgfpathlineto{\pgfqpoint{5.303633in}{3.127661in}}%
\pgfpathclose%
\pgfusepath{fill}%
\end{pgfscope}%
\begin{pgfscope}%
\pgfpathrectangle{\pgfqpoint{1.020000in}{0.880000in}}{\pgfqpoint{6.160000in}{6.160000in}}%
\pgfusepath{clip}%
\pgfsetbuttcap%
\pgfsetroundjoin%
\definecolor{currentfill}{rgb}{0.768929,0.189213,0.197965}%
\pgfsetfillcolor{currentfill}%
\pgfsetlinewidth{0.000000pt}%
\definecolor{currentstroke}{rgb}{0.000000,0.000000,0.000000}%
\pgfsetstrokecolor{currentstroke}%
\pgfsetdash{}{0pt}%
\pgfpathmoveto{\pgfqpoint{3.112774in}{5.171503in}}%
\pgfpathlineto{\pgfqpoint{3.121176in}{5.146974in}}%
\pgfpathlineto{\pgfqpoint{3.126698in}{5.401676in}}%
\pgfpathlineto{\pgfqpoint{3.160765in}{5.354840in}}%
\pgfpathlineto{\pgfqpoint{3.195282in}{5.256354in}}%
\pgfpathlineto{\pgfqpoint{3.186517in}{5.310580in}}%
\pgfpathlineto{\pgfqpoint{3.179821in}{5.155954in}}%
\pgfpathlineto{\pgfqpoint{3.145445in}{5.248574in}}%
\pgfpathlineto{\pgfqpoint{3.112774in}{5.171503in}}%
\pgfpathclose%
\pgfusepath{fill}%
\end{pgfscope}%
\begin{pgfscope}%
\pgfpathrectangle{\pgfqpoint{1.020000in}{0.880000in}}{\pgfqpoint{6.160000in}{6.160000in}}%
\pgfusepath{clip}%
\pgfsetbuttcap%
\pgfsetroundjoin%
\definecolor{currentfill}{rgb}{0.826784,0.858205,0.906953}%
\pgfsetfillcolor{currentfill}%
\pgfsetlinewidth{0.000000pt}%
\definecolor{currentstroke}{rgb}{0.000000,0.000000,0.000000}%
\pgfsetstrokecolor{currentstroke}%
\pgfsetdash{}{0pt}%
\pgfpathmoveto{\pgfqpoint{3.791213in}{3.959313in}}%
\pgfpathlineto{\pgfqpoint{3.799857in}{4.012379in}}%
\pgfpathlineto{\pgfqpoint{3.808669in}{4.027765in}}%
\pgfpathlineto{\pgfqpoint{3.842033in}{4.023632in}}%
\pgfpathlineto{\pgfqpoint{3.875098in}{4.109948in}}%
\pgfpathlineto{\pgfqpoint{3.866213in}{4.091850in}}%
\pgfpathlineto{\pgfqpoint{3.857384in}{4.062933in}}%
\pgfpathlineto{\pgfqpoint{3.824288in}{4.012040in}}%
\pgfpathlineto{\pgfqpoint{3.791213in}{3.959313in}}%
\pgfpathclose%
\pgfusepath{fill}%
\end{pgfscope}%
\begin{pgfscope}%
\pgfpathrectangle{\pgfqpoint{1.020000in}{0.880000in}}{\pgfqpoint{6.160000in}{6.160000in}}%
\pgfusepath{clip}%
\pgfsetbuttcap%
\pgfsetroundjoin%
\definecolor{currentfill}{rgb}{0.746838,0.140021,0.179996}%
\pgfsetfillcolor{currentfill}%
\pgfsetlinewidth{0.000000pt}%
\definecolor{currentstroke}{rgb}{0.000000,0.000000,0.000000}%
\pgfsetstrokecolor{currentstroke}%
\pgfsetdash{}{0pt}%
\pgfpathmoveto{\pgfqpoint{2.893043in}{5.387701in}}%
\pgfpathlineto{\pgfqpoint{2.901903in}{5.314075in}}%
\pgfpathlineto{\pgfqpoint{2.908584in}{5.415276in}}%
\pgfpathlineto{\pgfqpoint{2.946168in}{5.092170in}}%
\pgfpathlineto{\pgfqpoint{2.977520in}{5.275881in}}%
\pgfpathlineto{\pgfqpoint{2.970557in}{5.188247in}}%
\pgfpathlineto{\pgfqpoint{2.960609in}{5.352035in}}%
\pgfpathlineto{\pgfqpoint{2.927512in}{5.315438in}}%
\pgfpathlineto{\pgfqpoint{2.893043in}{5.387701in}}%
\pgfpathclose%
\pgfusepath{fill}%
\end{pgfscope}%
\begin{pgfscope}%
\pgfpathrectangle{\pgfqpoint{1.020000in}{0.880000in}}{\pgfqpoint{6.160000in}{6.160000in}}%
\pgfusepath{clip}%
\pgfsetbuttcap%
\pgfsetroundjoin%
\definecolor{currentfill}{rgb}{0.489246,0.627536,0.976896}%
\pgfsetfillcolor{currentfill}%
\pgfsetlinewidth{0.000000pt}%
\definecolor{currentstroke}{rgb}{0.000000,0.000000,0.000000}%
\pgfsetstrokecolor{currentstroke}%
\pgfsetdash{}{0pt}%
\pgfpathmoveto{\pgfqpoint{4.348150in}{3.464891in}}%
\pgfpathlineto{\pgfqpoint{4.357603in}{3.469015in}}%
\pgfpathlineto{\pgfqpoint{4.366848in}{3.348079in}}%
\pgfpathlineto{\pgfqpoint{4.399995in}{3.335399in}}%
\pgfpathlineto{\pgfqpoint{4.433092in}{3.313767in}}%
\pgfpathlineto{\pgfqpoint{4.423770in}{3.394836in}}%
\pgfpathlineto{\pgfqpoint{4.414312in}{3.414510in}}%
\pgfpathlineto{\pgfqpoint{4.381257in}{3.443728in}}%
\pgfpathlineto{\pgfqpoint{4.348150in}{3.464891in}}%
\pgfpathclose%
\pgfusepath{fill}%
\end{pgfscope}%
\begin{pgfscope}%
\pgfpathrectangle{\pgfqpoint{1.020000in}{0.880000in}}{\pgfqpoint{6.160000in}{6.160000in}}%
\pgfusepath{clip}%
\pgfsetbuttcap%
\pgfsetroundjoin%
\definecolor{currentfill}{rgb}{0.839365,0.321856,0.264924}%
\pgfsetfillcolor{currentfill}%
\pgfsetlinewidth{0.000000pt}%
\definecolor{currentstroke}{rgb}{0.000000,0.000000,0.000000}%
\pgfsetstrokecolor{currentstroke}%
\pgfsetdash{}{0pt}%
\pgfpathmoveto{\pgfqpoint{3.195282in}{5.256354in}}%
\pgfpathlineto{\pgfqpoint{3.204974in}{5.106683in}}%
\pgfpathlineto{\pgfqpoint{3.213900in}{5.036495in}}%
\pgfpathlineto{\pgfqpoint{3.247310in}{5.046508in}}%
\pgfpathlineto{\pgfqpoint{3.280469in}{5.083313in}}%
\pgfpathlineto{\pgfqpoint{3.271625in}{5.141824in}}%
\pgfpathlineto{\pgfqpoint{3.263750in}{5.094087in}}%
\pgfpathlineto{\pgfqpoint{3.229924in}{5.135620in}}%
\pgfpathlineto{\pgfqpoint{3.195282in}{5.256354in}}%
\pgfpathclose%
\pgfusepath{fill}%
\end{pgfscope}%
\begin{pgfscope}%
\pgfpathrectangle{\pgfqpoint{1.020000in}{0.880000in}}{\pgfqpoint{6.160000in}{6.160000in}}%
\pgfusepath{clip}%
\pgfsetbuttcap%
\pgfsetroundjoin%
\definecolor{currentfill}{rgb}{0.943432,0.802276,0.729172}%
\pgfsetfillcolor{currentfill}%
\pgfsetlinewidth{0.000000pt}%
\definecolor{currentstroke}{rgb}{0.000000,0.000000,0.000000}%
\pgfsetstrokecolor{currentstroke}%
\pgfsetdash{}{0pt}%
\pgfpathmoveto{\pgfqpoint{3.621052in}{4.360240in}}%
\pgfpathlineto{\pgfqpoint{3.628609in}{4.564456in}}%
\pgfpathlineto{\pgfqpoint{3.638096in}{4.421174in}}%
\pgfpathlineto{\pgfqpoint{3.671917in}{4.343799in}}%
\pgfpathlineto{\pgfqpoint{3.705600in}{4.280714in}}%
\pgfpathlineto{\pgfqpoint{3.696982in}{4.252504in}}%
\pgfpathlineto{\pgfqpoint{3.687610in}{4.382684in}}%
\pgfpathlineto{\pgfqpoint{3.655209in}{4.205068in}}%
\pgfpathlineto{\pgfqpoint{3.621052in}{4.360240in}}%
\pgfpathclose%
\pgfusepath{fill}%
\end{pgfscope}%
\begin{pgfscope}%
\pgfpathrectangle{\pgfqpoint{1.020000in}{0.880000in}}{\pgfqpoint{6.160000in}{6.160000in}}%
\pgfusepath{clip}%
\pgfsetbuttcap%
\pgfsetroundjoin%
\definecolor{currentfill}{rgb}{0.478462,0.616564,0.972721}%
\pgfsetfillcolor{currentfill}%
\pgfsetlinewidth{0.000000pt}%
\definecolor{currentstroke}{rgb}{0.000000,0.000000,0.000000}%
\pgfsetstrokecolor{currentstroke}%
\pgfsetdash{}{0pt}%
\pgfpathmoveto{\pgfqpoint{4.565758in}{3.388549in}}%
\pgfpathlineto{\pgfqpoint{4.575746in}{3.469110in}}%
\pgfpathlineto{\pgfqpoint{4.585214in}{3.411070in}}%
\pgfpathlineto{\pgfqpoint{4.617906in}{3.307661in}}%
\pgfpathlineto{\pgfqpoint{4.651052in}{3.319836in}}%
\pgfpathlineto{\pgfqpoint{4.641637in}{3.392968in}}%
\pgfpathlineto{\pgfqpoint{4.632105in}{3.438809in}}%
\pgfpathlineto{\pgfqpoint{4.598413in}{3.288573in}}%
\pgfpathlineto{\pgfqpoint{4.565758in}{3.388549in}}%
\pgfpathclose%
\pgfusepath{fill}%
\end{pgfscope}%
\begin{pgfscope}%
\pgfpathrectangle{\pgfqpoint{1.020000in}{0.880000in}}{\pgfqpoint{6.160000in}{6.160000in}}%
\pgfusepath{clip}%
\pgfsetbuttcap%
\pgfsetroundjoin%
\definecolor{currentfill}{rgb}{0.348323,0.465711,0.888346}%
\pgfsetfillcolor{currentfill}%
\pgfsetlinewidth{0.000000pt}%
\definecolor{currentstroke}{rgb}{0.000000,0.000000,0.000000}%
\pgfsetstrokecolor{currentstroke}%
\pgfsetdash{}{0pt}%
\pgfpathmoveto{\pgfqpoint{4.802401in}{3.215455in}}%
\pgfpathlineto{\pgfqpoint{4.811283in}{3.049073in}}%
\pgfpathlineto{\pgfqpoint{4.822507in}{3.250147in}}%
\pgfpathlineto{\pgfqpoint{4.854985in}{3.157718in}}%
\pgfpathlineto{\pgfqpoint{4.886643in}{2.960671in}}%
\pgfpathlineto{\pgfqpoint{4.878256in}{3.191015in}}%
\pgfpathlineto{\pgfqpoint{4.868345in}{3.202880in}}%
\pgfpathlineto{\pgfqpoint{4.834697in}{3.103933in}}%
\pgfpathlineto{\pgfqpoint{4.802401in}{3.215455in}}%
\pgfpathclose%
\pgfusepath{fill}%
\end{pgfscope}%
\begin{pgfscope}%
\pgfpathrectangle{\pgfqpoint{1.020000in}{0.880000in}}{\pgfqpoint{6.160000in}{6.160000in}}%
\pgfusepath{clip}%
\pgfsetbuttcap%
\pgfsetroundjoin%
\definecolor{currentfill}{rgb}{0.693321,0.796314,0.986308}%
\pgfsetfillcolor{currentfill}%
\pgfsetlinewidth{0.000000pt}%
\definecolor{currentstroke}{rgb}{0.000000,0.000000,0.000000}%
\pgfsetstrokecolor{currentstroke}%
\pgfsetdash{}{0pt}%
\pgfpathmoveto{\pgfqpoint{3.960466in}{3.822014in}}%
\pgfpathlineto{\pgfqpoint{3.969899in}{3.636917in}}%
\pgfpathlineto{\pgfqpoint{3.978693in}{3.758230in}}%
\pgfpathlineto{\pgfqpoint{4.012067in}{3.721871in}}%
\pgfpathlineto{\pgfqpoint{4.045314in}{3.746443in}}%
\pgfpathlineto{\pgfqpoint{4.036099in}{3.814558in}}%
\pgfpathlineto{\pgfqpoint{4.027000in}{3.812333in}}%
\pgfpathlineto{\pgfqpoint{3.993805in}{3.785115in}}%
\pgfpathlineto{\pgfqpoint{3.960466in}{3.822014in}}%
\pgfpathclose%
\pgfusepath{fill}%
\end{pgfscope}%
\begin{pgfscope}%
\pgfpathrectangle{\pgfqpoint{1.020000in}{0.880000in}}{\pgfqpoint{6.160000in}{6.160000in}}%
\pgfusepath{clip}%
\pgfsetbuttcap%
\pgfsetroundjoin%
\definecolor{currentfill}{rgb}{0.343278,0.459354,0.884122}%
\pgfsetfillcolor{currentfill}%
\pgfsetlinewidth{0.000000pt}%
\definecolor{currentstroke}{rgb}{0.000000,0.000000,0.000000}%
\pgfsetstrokecolor{currentstroke}%
\pgfsetdash{}{0pt}%
\pgfpathmoveto{\pgfqpoint{5.805090in}{3.101812in}}%
\pgfpathlineto{\pgfqpoint{5.816398in}{3.121391in}}%
\pgfpathlineto{\pgfqpoint{5.825565in}{3.009544in}}%
\pgfpathlineto{\pgfqpoint{5.861223in}{3.183631in}}%
\pgfpathlineto{\pgfqpoint{5.893261in}{3.141080in}}%
\pgfpathlineto{\pgfqpoint{5.882919in}{3.184201in}}%
\pgfpathlineto{\pgfqpoint{5.871675in}{3.173265in}}%
\pgfpathlineto{\pgfqpoint{5.837601in}{3.090537in}}%
\pgfpathlineto{\pgfqpoint{5.805090in}{3.101812in}}%
\pgfpathclose%
\pgfusepath{fill}%
\end{pgfscope}%
\begin{pgfscope}%
\pgfpathrectangle{\pgfqpoint{1.020000in}{0.880000in}}{\pgfqpoint{6.160000in}{6.160000in}}%
\pgfusepath{clip}%
\pgfsetbuttcap%
\pgfsetroundjoin%
\definecolor{currentfill}{rgb}{0.289996,0.386836,0.828926}%
\pgfsetfillcolor{currentfill}%
\pgfsetlinewidth{0.000000pt}%
\definecolor{currentstroke}{rgb}{0.000000,0.000000,0.000000}%
\pgfsetstrokecolor{currentstroke}%
\pgfsetdash{}{0pt}%
\pgfpathmoveto{\pgfqpoint{5.170187in}{2.941324in}}%
\pgfpathlineto{\pgfqpoint{5.181965in}{3.087935in}}%
\pgfpathlineto{\pgfqpoint{5.191271in}{2.985145in}}%
\pgfpathlineto{\pgfqpoint{5.224044in}{2.968636in}}%
\pgfpathlineto{\pgfqpoint{5.258340in}{3.096154in}}%
\pgfpathlineto{\pgfqpoint{5.247443in}{3.050715in}}%
\pgfpathlineto{\pgfqpoint{5.239057in}{3.239548in}}%
\pgfpathlineto{\pgfqpoint{5.201353in}{2.775366in}}%
\pgfpathlineto{\pgfqpoint{5.170187in}{2.941324in}}%
\pgfpathclose%
\pgfusepath{fill}%
\end{pgfscope}%
\begin{pgfscope}%
\pgfpathrectangle{\pgfqpoint{1.020000in}{0.880000in}}{\pgfqpoint{6.160000in}{6.160000in}}%
\pgfusepath{clip}%
\pgfsetbuttcap%
\pgfsetroundjoin%
\definecolor{currentfill}{rgb}{0.895885,0.433075,0.338681}%
\pgfsetfillcolor{currentfill}%
\pgfsetlinewidth{0.000000pt}%
\definecolor{currentstroke}{rgb}{0.000000,0.000000,0.000000}%
\pgfsetstrokecolor{currentstroke}%
\pgfsetdash{}{0pt}%
\pgfpathmoveto{\pgfqpoint{2.498427in}{4.795123in}}%
\pgfpathlineto{\pgfqpoint{2.504428in}{4.885558in}}%
\pgfpathlineto{\pgfqpoint{2.509534in}{5.032048in}}%
\pgfpathlineto{\pgfqpoint{2.543592in}{5.013677in}}%
\pgfpathlineto{\pgfqpoint{2.577115in}{5.027091in}}%
\pgfpathlineto{\pgfqpoint{2.571339in}{4.912479in}}%
\pgfpathlineto{\pgfqpoint{2.563256in}{4.944526in}}%
\pgfpathlineto{\pgfqpoint{2.525430in}{5.201105in}}%
\pgfpathlineto{\pgfqpoint{2.498427in}{4.795123in}}%
\pgfpathclose%
\pgfusepath{fill}%
\end{pgfscope}%
\begin{pgfscope}%
\pgfpathrectangle{\pgfqpoint{1.020000in}{0.880000in}}{\pgfqpoint{6.160000in}{6.160000in}}%
\pgfusepath{clip}%
\pgfsetbuttcap%
\pgfsetroundjoin%
\definecolor{currentfill}{rgb}{0.565182,0.699438,0.996635}%
\pgfsetfillcolor{currentfill}%
\pgfsetlinewidth{0.000000pt}%
\definecolor{currentstroke}{rgb}{0.000000,0.000000,0.000000}%
\pgfsetstrokecolor{currentstroke}%
\pgfsetdash{}{0pt}%
\pgfpathmoveto{\pgfqpoint{4.196783in}{3.456992in}}%
\pgfpathlineto{\pgfqpoint{4.206103in}{3.645712in}}%
\pgfpathlineto{\pgfqpoint{4.215410in}{3.622911in}}%
\pgfpathlineto{\pgfqpoint{4.248588in}{3.467416in}}%
\pgfpathlineto{\pgfqpoint{4.281723in}{3.393410in}}%
\pgfpathlineto{\pgfqpoint{4.272501in}{3.542618in}}%
\pgfpathlineto{\pgfqpoint{4.263186in}{3.585650in}}%
\pgfpathlineto{\pgfqpoint{4.229969in}{3.510913in}}%
\pgfpathlineto{\pgfqpoint{4.196783in}{3.456992in}}%
\pgfpathclose%
\pgfusepath{fill}%
\end{pgfscope}%
\begin{pgfscope}%
\pgfpathrectangle{\pgfqpoint{1.020000in}{0.880000in}}{\pgfqpoint{6.160000in}{6.160000in}}%
\pgfusepath{clip}%
\pgfsetbuttcap%
\pgfsetroundjoin%
\definecolor{currentfill}{rgb}{0.936780,0.532750,0.418093}%
\pgfsetfillcolor{currentfill}%
\pgfsetlinewidth{0.000000pt}%
\definecolor{currentstroke}{rgb}{0.000000,0.000000,0.000000}%
\pgfsetstrokecolor{currentstroke}%
\pgfsetdash{}{0pt}%
\pgfpathmoveto{\pgfqpoint{3.364194in}{5.087683in}}%
\pgfpathlineto{\pgfqpoint{3.375761in}{4.695150in}}%
\pgfpathlineto{\pgfqpoint{3.384685in}{4.631445in}}%
\pgfpathlineto{\pgfqpoint{3.415300in}{5.002567in}}%
\pgfpathlineto{\pgfqpoint{3.450318in}{4.795929in}}%
\pgfpathlineto{\pgfqpoint{3.441132in}{4.890342in}}%
\pgfpathlineto{\pgfqpoint{3.434291in}{4.668149in}}%
\pgfpathlineto{\pgfqpoint{3.398568in}{4.975398in}}%
\pgfpathlineto{\pgfqpoint{3.364194in}{5.087683in}}%
\pgfpathclose%
\pgfusepath{fill}%
\end{pgfscope}%
\begin{pgfscope}%
\pgfpathrectangle{\pgfqpoint{1.020000in}{0.880000in}}{\pgfqpoint{6.160000in}{6.160000in}}%
\pgfusepath{clip}%
\pgfsetbuttcap%
\pgfsetroundjoin%
\definecolor{currentfill}{rgb}{0.869655,0.379274,0.300941}%
\pgfsetfillcolor{currentfill}%
\pgfsetlinewidth{0.000000pt}%
\definecolor{currentstroke}{rgb}{0.000000,0.000000,0.000000}%
\pgfsetstrokecolor{currentstroke}%
\pgfsetdash{}{0pt}%
\pgfpathmoveto{\pgfqpoint{2.563256in}{4.944526in}}%
\pgfpathlineto{\pgfqpoint{2.571339in}{4.912479in}}%
\pgfpathlineto{\pgfqpoint{2.577115in}{5.027091in}}%
\pgfpathlineto{\pgfqpoint{2.609772in}{5.095788in}}%
\pgfpathlineto{\pgfqpoint{2.644634in}{5.020581in}}%
\pgfpathlineto{\pgfqpoint{2.636768in}{5.034033in}}%
\pgfpathlineto{\pgfqpoint{2.627413in}{5.145720in}}%
\pgfpathlineto{\pgfqpoint{2.593271in}{5.175662in}}%
\pgfpathlineto{\pgfqpoint{2.563256in}{4.944526in}}%
\pgfpathclose%
\pgfusepath{fill}%
\end{pgfscope}%
\begin{pgfscope}%
\pgfpathrectangle{\pgfqpoint{1.020000in}{0.880000in}}{\pgfqpoint{6.160000in}{6.160000in}}%
\pgfusepath{clip}%
\pgfsetbuttcap%
\pgfsetroundjoin%
\definecolor{currentfill}{rgb}{0.619318,0.744121,0.998931}%
\pgfsetfillcolor{currentfill}%
\pgfsetlinewidth{0.000000pt}%
\definecolor{currentstroke}{rgb}{0.000000,0.000000,0.000000}%
\pgfsetstrokecolor{currentstroke}%
\pgfsetdash{}{0pt}%
\pgfpathmoveto{\pgfqpoint{4.045314in}{3.746443in}}%
\pgfpathlineto{\pgfqpoint{4.054554in}{3.664645in}}%
\pgfpathlineto{\pgfqpoint{4.063603in}{3.746933in}}%
\pgfpathlineto{\pgfqpoint{4.097112in}{3.544107in}}%
\pgfpathlineto{\pgfqpoint{4.130411in}{3.446343in}}%
\pgfpathlineto{\pgfqpoint{4.121198in}{3.480142in}}%
\pgfpathlineto{\pgfqpoint{4.111882in}{3.678718in}}%
\pgfpathlineto{\pgfqpoint{4.078613in}{3.718150in}}%
\pgfpathlineto{\pgfqpoint{4.045314in}{3.746443in}}%
\pgfpathclose%
\pgfusepath{fill}%
\end{pgfscope}%
\begin{pgfscope}%
\pgfpathrectangle{\pgfqpoint{1.020000in}{0.880000in}}{\pgfqpoint{6.160000in}{6.160000in}}%
\pgfusepath{clip}%
\pgfsetbuttcap%
\pgfsetroundjoin%
\definecolor{currentfill}{rgb}{0.968533,0.715841,0.606097}%
\pgfsetfillcolor{currentfill}%
\pgfsetlinewidth{0.000000pt}%
\definecolor{currentstroke}{rgb}{0.000000,0.000000,0.000000}%
\pgfsetstrokecolor{currentstroke}%
\pgfsetdash{}{0pt}%
\pgfpathmoveto{\pgfqpoint{3.534646in}{4.748804in}}%
\pgfpathlineto{\pgfqpoint{3.544881in}{4.491598in}}%
\pgfpathlineto{\pgfqpoint{3.552609in}{4.632390in}}%
\pgfpathlineto{\pgfqpoint{3.586756in}{4.518300in}}%
\pgfpathlineto{\pgfqpoint{3.621052in}{4.360240in}}%
\pgfpathlineto{\pgfqpoint{3.611155in}{4.578981in}}%
\pgfpathlineto{\pgfqpoint{3.602884in}{4.511242in}}%
\pgfpathlineto{\pgfqpoint{3.569456in}{4.524250in}}%
\pgfpathlineto{\pgfqpoint{3.534646in}{4.748804in}}%
\pgfpathclose%
\pgfusepath{fill}%
\end{pgfscope}%
\begin{pgfscope}%
\pgfpathrectangle{\pgfqpoint{1.020000in}{0.880000in}}{\pgfqpoint{6.160000in}{6.160000in}}%
\pgfusepath{clip}%
\pgfsetbuttcap%
\pgfsetroundjoin%
\definecolor{currentfill}{rgb}{0.843703,0.330068,0.270065}%
\pgfsetfillcolor{currentfill}%
\pgfsetlinewidth{0.000000pt}%
\definecolor{currentstroke}{rgb}{0.000000,0.000000,0.000000}%
\pgfsetstrokecolor{currentstroke}%
\pgfsetdash{}{0pt}%
\pgfpathmoveto{\pgfqpoint{2.627413in}{5.145720in}}%
\pgfpathlineto{\pgfqpoint{2.636768in}{5.034033in}}%
\pgfpathlineto{\pgfqpoint{2.644634in}{5.020581in}}%
\pgfpathlineto{\pgfqpoint{2.677497in}{5.075656in}}%
\pgfpathlineto{\pgfqpoint{2.711043in}{5.084681in}}%
\pgfpathlineto{\pgfqpoint{2.700505in}{5.277501in}}%
\pgfpathlineto{\pgfqpoint{2.694421in}{5.164999in}}%
\pgfpathlineto{\pgfqpoint{2.662464in}{5.052450in}}%
\pgfpathlineto{\pgfqpoint{2.627413in}{5.145720in}}%
\pgfpathclose%
\pgfusepath{fill}%
\end{pgfscope}%
\begin{pgfscope}%
\pgfpathrectangle{\pgfqpoint{1.020000in}{0.880000in}}{\pgfqpoint{6.160000in}{6.160000in}}%
\pgfusepath{clip}%
\pgfsetbuttcap%
\pgfsetroundjoin%
\definecolor{currentfill}{rgb}{0.323718,0.433158,0.864722}%
\pgfsetfillcolor{currentfill}%
\pgfsetlinewidth{0.000000pt}%
\definecolor{currentstroke}{rgb}{0.000000,0.000000,0.000000}%
\pgfsetstrokecolor{currentstroke}%
\pgfsetdash{}{0pt}%
\pgfpathmoveto{\pgfqpoint{5.738696in}{3.041498in}}%
\pgfpathlineto{\pgfqpoint{5.753134in}{3.265460in}}%
\pgfpathlineto{\pgfqpoint{5.760841in}{3.059724in}}%
\pgfpathlineto{\pgfqpoint{5.794231in}{3.097087in}}%
\pgfpathlineto{\pgfqpoint{5.825565in}{3.009544in}}%
\pgfpathlineto{\pgfqpoint{5.816398in}{3.121391in}}%
\pgfpathlineto{\pgfqpoint{5.805090in}{3.101812in}}%
\pgfpathlineto{\pgfqpoint{5.770356in}{2.975280in}}%
\pgfpathlineto{\pgfqpoint{5.738696in}{3.041498in}}%
\pgfpathclose%
\pgfusepath{fill}%
\end{pgfscope}%
\begin{pgfscope}%
\pgfpathrectangle{\pgfqpoint{1.020000in}{0.880000in}}{\pgfqpoint{6.160000in}{6.160000in}}%
\pgfusepath{clip}%
\pgfsetbuttcap%
\pgfsetroundjoin%
\definecolor{currentfill}{rgb}{0.961595,0.622247,0.501551}%
\pgfsetfillcolor{currentfill}%
\pgfsetlinewidth{0.000000pt}%
\definecolor{currentstroke}{rgb}{0.000000,0.000000,0.000000}%
\pgfsetstrokecolor{currentstroke}%
\pgfsetdash{}{0pt}%
\pgfpathmoveto{\pgfqpoint{3.450318in}{4.795929in}}%
\pgfpathlineto{\pgfqpoint{3.458983in}{4.774718in}}%
\pgfpathlineto{\pgfqpoint{3.468760in}{4.597977in}}%
\pgfpathlineto{\pgfqpoint{3.501048in}{4.768402in}}%
\pgfpathlineto{\pgfqpoint{3.534646in}{4.748804in}}%
\pgfpathlineto{\pgfqpoint{3.526037in}{4.751472in}}%
\pgfpathlineto{\pgfqpoint{3.518179in}{4.642566in}}%
\pgfpathlineto{\pgfqpoint{3.485403in}{4.557010in}}%
\pgfpathlineto{\pgfqpoint{3.450318in}{4.795929in}}%
\pgfpathclose%
\pgfusepath{fill}%
\end{pgfscope}%
\begin{pgfscope}%
\pgfpathrectangle{\pgfqpoint{1.020000in}{0.880000in}}{\pgfqpoint{6.160000in}{6.160000in}}%
\pgfusepath{clip}%
\pgfsetbuttcap%
\pgfsetroundjoin%
\definecolor{currentfill}{rgb}{0.343278,0.459354,0.884122}%
\pgfsetfillcolor{currentfill}%
\pgfsetlinewidth{0.000000pt}%
\definecolor{currentstroke}{rgb}{0.000000,0.000000,0.000000}%
\pgfsetstrokecolor{currentstroke}%
\pgfsetdash{}{0pt}%
\pgfpathmoveto{\pgfqpoint{5.522543in}{3.185865in}}%
\pgfpathlineto{\pgfqpoint{5.532574in}{3.136643in}}%
\pgfpathlineto{\pgfqpoint{5.541675in}{3.018322in}}%
\pgfpathlineto{\pgfqpoint{5.576546in}{3.164811in}}%
\pgfpathlineto{\pgfqpoint{5.609252in}{3.154421in}}%
\pgfpathlineto{\pgfqpoint{5.597020in}{3.052158in}}%
\pgfpathlineto{\pgfqpoint{5.587259in}{3.122070in}}%
\pgfpathlineto{\pgfqpoint{5.555060in}{3.164051in}}%
\pgfpathlineto{\pgfqpoint{5.522543in}{3.185865in}}%
\pgfpathclose%
\pgfusepath{fill}%
\end{pgfscope}%
\begin{pgfscope}%
\pgfpathrectangle{\pgfqpoint{1.020000in}{0.880000in}}{\pgfqpoint{6.160000in}{6.160000in}}%
\pgfusepath{clip}%
\pgfsetbuttcap%
\pgfsetroundjoin%
\definecolor{currentfill}{rgb}{0.467678,0.605591,0.968546}%
\pgfsetfillcolor{currentfill}%
\pgfsetlinewidth{0.000000pt}%
\definecolor{currentstroke}{rgb}{0.000000,0.000000,0.000000}%
\pgfsetstrokecolor{currentstroke}%
\pgfsetdash{}{0pt}%
\pgfpathmoveto{\pgfqpoint{4.499492in}{3.367992in}}%
\pgfpathlineto{\pgfqpoint{4.509161in}{3.387234in}}%
\pgfpathlineto{\pgfqpoint{4.518074in}{3.174078in}}%
\pgfpathlineto{\pgfqpoint{4.551399in}{3.234145in}}%
\pgfpathlineto{\pgfqpoint{4.585214in}{3.411070in}}%
\pgfpathlineto{\pgfqpoint{4.575746in}{3.469110in}}%
\pgfpathlineto{\pgfqpoint{4.565758in}{3.388549in}}%
\pgfpathlineto{\pgfqpoint{4.532791in}{3.423931in}}%
\pgfpathlineto{\pgfqpoint{4.499492in}{3.367992in}}%
\pgfpathclose%
\pgfusepath{fill}%
\end{pgfscope}%
\begin{pgfscope}%
\pgfpathrectangle{\pgfqpoint{1.020000in}{0.880000in}}{\pgfqpoint{6.160000in}{6.160000in}}%
\pgfusepath{clip}%
\pgfsetbuttcap%
\pgfsetroundjoin%
\definecolor{currentfill}{rgb}{0.333490,0.446265,0.874452}%
\pgfsetfillcolor{currentfill}%
\pgfsetlinewidth{0.000000pt}%
\definecolor{currentstroke}{rgb}{0.000000,0.000000,0.000000}%
\pgfsetstrokecolor{currentstroke}%
\pgfsetdash{}{0pt}%
\pgfpathmoveto{\pgfqpoint{5.239057in}{3.239548in}}%
\pgfpathlineto{\pgfqpoint{5.247443in}{3.050715in}}%
\pgfpathlineto{\pgfqpoint{5.258340in}{3.096154in}}%
\pgfpathlineto{\pgfqpoint{5.291046in}{3.074845in}}%
\pgfpathlineto{\pgfqpoint{5.324443in}{3.116843in}}%
\pgfpathlineto{\pgfqpoint{5.314427in}{3.157482in}}%
\pgfpathlineto{\pgfqpoint{5.303633in}{3.127661in}}%
\pgfpathlineto{\pgfqpoint{5.269137in}{2.978599in}}%
\pgfpathlineto{\pgfqpoint{5.239057in}{3.239548in}}%
\pgfpathclose%
\pgfusepath{fill}%
\end{pgfscope}%
\begin{pgfscope}%
\pgfpathrectangle{\pgfqpoint{1.020000in}{0.880000in}}{\pgfqpoint{6.160000in}{6.160000in}}%
\pgfusepath{clip}%
\pgfsetbuttcap%
\pgfsetroundjoin%
\definecolor{currentfill}{rgb}{0.323718,0.433158,0.864722}%
\pgfsetfillcolor{currentfill}%
\pgfsetlinewidth{0.000000pt}%
\definecolor{currentstroke}{rgb}{0.000000,0.000000,0.000000}%
\pgfsetstrokecolor{currentstroke}%
\pgfsetdash{}{0pt}%
\pgfpathmoveto{\pgfqpoint{4.953557in}{3.090357in}}%
\pgfpathlineto{\pgfqpoint{4.964321in}{3.176578in}}%
\pgfpathlineto{\pgfqpoint{4.973885in}{3.106443in}}%
\pgfpathlineto{\pgfqpoint{5.006567in}{3.063240in}}%
\pgfpathlineto{\pgfqpoint{5.039298in}{3.031456in}}%
\pgfpathlineto{\pgfqpoint{5.028361in}{2.941005in}}%
\pgfpathlineto{\pgfqpoint{5.019390in}{3.080052in}}%
\pgfpathlineto{\pgfqpoint{4.987372in}{3.195329in}}%
\pgfpathlineto{\pgfqpoint{4.953557in}{3.090357in}}%
\pgfpathclose%
\pgfusepath{fill}%
\end{pgfscope}%
\begin{pgfscope}%
\pgfpathrectangle{\pgfqpoint{1.020000in}{0.880000in}}{\pgfqpoint{6.160000in}{6.160000in}}%
\pgfusepath{clip}%
\pgfsetbuttcap%
\pgfsetroundjoin%
\definecolor{currentfill}{rgb}{0.758112,0.168122,0.188827}%
\pgfsetfillcolor{currentfill}%
\pgfsetlinewidth{0.000000pt}%
\definecolor{currentstroke}{rgb}{0.000000,0.000000,0.000000}%
\pgfsetstrokecolor{currentstroke}%
\pgfsetdash{}{0pt}%
\pgfpathmoveto{\pgfqpoint{3.042270in}{5.487655in}}%
\pgfpathlineto{\pgfqpoint{3.052883in}{5.261742in}}%
\pgfpathlineto{\pgfqpoint{3.061034in}{5.257577in}}%
\pgfpathlineto{\pgfqpoint{3.094524in}{5.266775in}}%
\pgfpathlineto{\pgfqpoint{3.126698in}{5.401676in}}%
\pgfpathlineto{\pgfqpoint{3.121176in}{5.146974in}}%
\pgfpathlineto{\pgfqpoint{3.112774in}{5.171503in}}%
\pgfpathlineto{\pgfqpoint{3.079663in}{5.137938in}}%
\pgfpathlineto{\pgfqpoint{3.042270in}{5.487655in}}%
\pgfpathclose%
\pgfusepath{fill}%
\end{pgfscope}%
\begin{pgfscope}%
\pgfpathrectangle{\pgfqpoint{1.020000in}{0.880000in}}{\pgfqpoint{6.160000in}{6.160000in}}%
\pgfusepath{clip}%
\pgfsetbuttcap%
\pgfsetroundjoin%
\definecolor{currentfill}{rgb}{0.768929,0.189213,0.197965}%
\pgfsetfillcolor{currentfill}%
\pgfsetlinewidth{0.000000pt}%
\definecolor{currentstroke}{rgb}{0.000000,0.000000,0.000000}%
\pgfsetstrokecolor{currentstroke}%
\pgfsetdash{}{0pt}%
\pgfpathmoveto{\pgfqpoint{2.763021in}{5.066549in}}%
\pgfpathlineto{\pgfqpoint{2.767289in}{5.319218in}}%
\pgfpathlineto{\pgfqpoint{2.774971in}{5.328858in}}%
\pgfpathlineto{\pgfqpoint{2.811291in}{5.134777in}}%
\pgfpathlineto{\pgfqpoint{2.842361in}{5.328530in}}%
\pgfpathlineto{\pgfqpoint{2.836955in}{5.141519in}}%
\pgfpathlineto{\pgfqpoint{2.825366in}{5.422365in}}%
\pgfpathlineto{\pgfqpoint{2.794778in}{5.198008in}}%
\pgfpathlineto{\pgfqpoint{2.763021in}{5.066549in}}%
\pgfpathclose%
\pgfusepath{fill}%
\end{pgfscope}%
\begin{pgfscope}%
\pgfpathrectangle{\pgfqpoint{1.020000in}{0.880000in}}{\pgfqpoint{6.160000in}{6.160000in}}%
\pgfusepath{clip}%
\pgfsetbuttcap%
\pgfsetroundjoin%
\definecolor{currentfill}{rgb}{0.877149,0.394645,0.311724}%
\pgfsetfillcolor{currentfill}%
\pgfsetlinewidth{0.000000pt}%
\definecolor{currentstroke}{rgb}{0.000000,0.000000,0.000000}%
\pgfsetstrokecolor{currentstroke}%
\pgfsetdash{}{0pt}%
\pgfpathmoveto{\pgfqpoint{3.280469in}{5.083313in}}%
\pgfpathlineto{\pgfqpoint{3.289507in}{5.003609in}}%
\pgfpathlineto{\pgfqpoint{3.298081in}{4.977698in}}%
\pgfpathlineto{\pgfqpoint{3.330796in}{5.072389in}}%
\pgfpathlineto{\pgfqpoint{3.364194in}{5.087683in}}%
\pgfpathlineto{\pgfqpoint{3.356489in}{4.999825in}}%
\pgfpathlineto{\pgfqpoint{3.349243in}{4.860143in}}%
\pgfpathlineto{\pgfqpoint{3.313584in}{5.125974in}}%
\pgfpathlineto{\pgfqpoint{3.280469in}{5.083313in}}%
\pgfpathclose%
\pgfusepath{fill}%
\end{pgfscope}%
\begin{pgfscope}%
\pgfpathrectangle{\pgfqpoint{1.020000in}{0.880000in}}{\pgfqpoint{6.160000in}{6.160000in}}%
\pgfusepath{clip}%
\pgfsetbuttcap%
\pgfsetroundjoin%
\definecolor{currentfill}{rgb}{0.835345,0.860514,0.898970}%
\pgfsetfillcolor{currentfill}%
\pgfsetlinewidth{0.000000pt}%
\definecolor{currentstroke}{rgb}{0.000000,0.000000,0.000000}%
\pgfsetstrokecolor{currentstroke}%
\pgfsetdash{}{0pt}%
\pgfpathmoveto{\pgfqpoint{3.723417in}{4.224999in}}%
\pgfpathlineto{\pgfqpoint{3.733292in}{3.985676in}}%
\pgfpathlineto{\pgfqpoint{3.742014in}{4.003790in}}%
\pgfpathlineto{\pgfqpoint{3.775059in}{4.086953in}}%
\pgfpathlineto{\pgfqpoint{3.808669in}{4.027765in}}%
\pgfpathlineto{\pgfqpoint{3.799857in}{4.012379in}}%
\pgfpathlineto{\pgfqpoint{3.791213in}{3.959313in}}%
\pgfpathlineto{\pgfqpoint{3.757229in}{4.125221in}}%
\pgfpathlineto{\pgfqpoint{3.723417in}{4.224999in}}%
\pgfpathclose%
\pgfusepath{fill}%
\end{pgfscope}%
\begin{pgfscope}%
\pgfpathrectangle{\pgfqpoint{1.020000in}{0.880000in}}{\pgfqpoint{6.160000in}{6.160000in}}%
\pgfusepath{clip}%
\pgfsetbuttcap%
\pgfsetroundjoin%
\definecolor{currentfill}{rgb}{0.795938,0.241845,0.220830}%
\pgfsetfillcolor{currentfill}%
\pgfsetlinewidth{0.000000pt}%
\definecolor{currentstroke}{rgb}{0.000000,0.000000,0.000000}%
\pgfsetstrokecolor{currentstroke}%
\pgfsetdash{}{0pt}%
\pgfpathmoveto{\pgfqpoint{2.694421in}{5.164999in}}%
\pgfpathlineto{\pgfqpoint{2.700505in}{5.277501in}}%
\pgfpathlineto{\pgfqpoint{2.711043in}{5.084681in}}%
\pgfpathlineto{\pgfqpoint{2.742453in}{5.243779in}}%
\pgfpathlineto{\pgfqpoint{2.774971in}{5.328858in}}%
\pgfpathlineto{\pgfqpoint{2.767289in}{5.319218in}}%
\pgfpathlineto{\pgfqpoint{2.763021in}{5.066549in}}%
\pgfpathlineto{\pgfqpoint{2.728888in}{5.105932in}}%
\pgfpathlineto{\pgfqpoint{2.694421in}{5.164999in}}%
\pgfpathclose%
\pgfusepath{fill}%
\end{pgfscope}%
\begin{pgfscope}%
\pgfpathrectangle{\pgfqpoint{1.020000in}{0.880000in}}{\pgfqpoint{6.160000in}{6.160000in}}%
\pgfusepath{clip}%
\pgfsetbuttcap%
\pgfsetroundjoin%
\definecolor{currentfill}{rgb}{0.693321,0.796314,0.986308}%
\pgfsetfillcolor{currentfill}%
\pgfsetlinewidth{0.000000pt}%
\definecolor{currentstroke}{rgb}{0.000000,0.000000,0.000000}%
\pgfsetstrokecolor{currentstroke}%
\pgfsetdash{}{0pt}%
\pgfpathmoveto{\pgfqpoint{3.893895in}{3.813359in}}%
\pgfpathlineto{\pgfqpoint{3.903002in}{3.765451in}}%
\pgfpathlineto{\pgfqpoint{3.911953in}{3.778898in}}%
\pgfpathlineto{\pgfqpoint{3.945535in}{3.684393in}}%
\pgfpathlineto{\pgfqpoint{3.978693in}{3.758230in}}%
\pgfpathlineto{\pgfqpoint{3.969899in}{3.636917in}}%
\pgfpathlineto{\pgfqpoint{3.960466in}{3.822014in}}%
\pgfpathlineto{\pgfqpoint{3.927181in}{3.821045in}}%
\pgfpathlineto{\pgfqpoint{3.893895in}{3.813359in}}%
\pgfpathclose%
\pgfusepath{fill}%
\end{pgfscope}%
\begin{pgfscope}%
\pgfpathrectangle{\pgfqpoint{1.020000in}{0.880000in}}{\pgfqpoint{6.160000in}{6.160000in}}%
\pgfusepath{clip}%
\pgfsetbuttcap%
\pgfsetroundjoin%
\definecolor{currentfill}{rgb}{0.289996,0.386836,0.828926}%
\pgfsetfillcolor{currentfill}%
\pgfsetlinewidth{0.000000pt}%
\definecolor{currentstroke}{rgb}{0.000000,0.000000,0.000000}%
\pgfsetstrokecolor{currentstroke}%
\pgfsetdash{}{0pt}%
\pgfpathmoveto{\pgfqpoint{5.104463in}{2.948689in}}%
\pgfpathlineto{\pgfqpoint{5.117064in}{3.200309in}}%
\pgfpathlineto{\pgfqpoint{5.125477in}{3.001588in}}%
\pgfpathlineto{\pgfqpoint{5.158609in}{3.016117in}}%
\pgfpathlineto{\pgfqpoint{5.191271in}{2.985145in}}%
\pgfpathlineto{\pgfqpoint{5.181965in}{3.087935in}}%
\pgfpathlineto{\pgfqpoint{5.170187in}{2.941324in}}%
\pgfpathlineto{\pgfqpoint{5.137465in}{2.958517in}}%
\pgfpathlineto{\pgfqpoint{5.104463in}{2.948689in}}%
\pgfpathclose%
\pgfusepath{fill}%
\end{pgfscope}%
\begin{pgfscope}%
\pgfpathrectangle{\pgfqpoint{1.020000in}{0.880000in}}{\pgfqpoint{6.160000in}{6.160000in}}%
\pgfusepath{clip}%
\pgfsetbuttcap%
\pgfsetroundjoin%
\definecolor{currentfill}{rgb}{0.516260,0.654498,0.986407}%
\pgfsetfillcolor{currentfill}%
\pgfsetlinewidth{0.000000pt}%
\definecolor{currentstroke}{rgb}{0.000000,0.000000,0.000000}%
\pgfsetstrokecolor{currentstroke}%
\pgfsetdash{}{0pt}%
\pgfpathmoveto{\pgfqpoint{4.281723in}{3.393410in}}%
\pgfpathlineto{\pgfqpoint{4.291238in}{3.522749in}}%
\pgfpathlineto{\pgfqpoint{4.300526in}{3.422097in}}%
\pgfpathlineto{\pgfqpoint{4.333671in}{3.359763in}}%
\pgfpathlineto{\pgfqpoint{4.366848in}{3.348079in}}%
\pgfpathlineto{\pgfqpoint{4.357603in}{3.469015in}}%
\pgfpathlineto{\pgfqpoint{4.348150in}{3.464891in}}%
\pgfpathlineto{\pgfqpoint{4.315044in}{3.512479in}}%
\pgfpathlineto{\pgfqpoint{4.281723in}{3.393410in}}%
\pgfpathclose%
\pgfusepath{fill}%
\end{pgfscope}%
\begin{pgfscope}%
\pgfpathrectangle{\pgfqpoint{1.020000in}{0.880000in}}{\pgfqpoint{6.160000in}{6.160000in}}%
\pgfusepath{clip}%
\pgfsetbuttcap%
\pgfsetroundjoin%
\definecolor{currentfill}{rgb}{0.576051,0.708780,0.997755}%
\pgfsetfillcolor{currentfill}%
\pgfsetlinewidth{0.000000pt}%
\definecolor{currentstroke}{rgb}{0.000000,0.000000,0.000000}%
\pgfsetstrokecolor{currentstroke}%
\pgfsetdash{}{0pt}%
\pgfpathmoveto{\pgfqpoint{4.130411in}{3.446343in}}%
\pgfpathlineto{\pgfqpoint{4.139488in}{3.752456in}}%
\pgfpathlineto{\pgfqpoint{4.148842in}{3.475495in}}%
\pgfpathlineto{\pgfqpoint{4.182112in}{3.494525in}}%
\pgfpathlineto{\pgfqpoint{4.215410in}{3.622911in}}%
\pgfpathlineto{\pgfqpoint{4.206103in}{3.645712in}}%
\pgfpathlineto{\pgfqpoint{4.196783in}{3.456992in}}%
\pgfpathlineto{\pgfqpoint{4.163608in}{3.435340in}}%
\pgfpathlineto{\pgfqpoint{4.130411in}{3.446343in}}%
\pgfpathclose%
\pgfusepath{fill}%
\end{pgfscope}%
\begin{pgfscope}%
\pgfpathrectangle{\pgfqpoint{1.020000in}{0.880000in}}{\pgfqpoint{6.160000in}{6.160000in}}%
\pgfusepath{clip}%
\pgfsetbuttcap%
\pgfsetroundjoin%
\definecolor{currentfill}{rgb}{0.786721,0.844807,0.939810}%
\pgfsetfillcolor{currentfill}%
\pgfsetlinewidth{0.000000pt}%
\definecolor{currentstroke}{rgb}{0.000000,0.000000,0.000000}%
\pgfsetstrokecolor{currentstroke}%
\pgfsetdash{}{0pt}%
\pgfpathmoveto{\pgfqpoint{3.808669in}{4.027765in}}%
\pgfpathlineto{\pgfqpoint{3.818195in}{3.852560in}}%
\pgfpathlineto{\pgfqpoint{3.826946in}{3.892285in}}%
\pgfpathlineto{\pgfqpoint{3.860485in}{3.840102in}}%
\pgfpathlineto{\pgfqpoint{3.893895in}{3.813359in}}%
\pgfpathlineto{\pgfqpoint{3.884447in}{3.979520in}}%
\pgfpathlineto{\pgfqpoint{3.875098in}{4.109948in}}%
\pgfpathlineto{\pgfqpoint{3.842033in}{4.023632in}}%
\pgfpathlineto{\pgfqpoint{3.808669in}{4.027765in}}%
\pgfpathclose%
\pgfusepath{fill}%
\end{pgfscope}%
\begin{pgfscope}%
\pgfpathrectangle{\pgfqpoint{1.020000in}{0.880000in}}{\pgfqpoint{6.160000in}{6.160000in}}%
\pgfusepath{clip}%
\pgfsetbuttcap%
\pgfsetroundjoin%
\definecolor{currentfill}{rgb}{0.966017,0.646130,0.525890}%
\pgfsetfillcolor{currentfill}%
\pgfsetlinewidth{0.000000pt}%
\definecolor{currentstroke}{rgb}{0.000000,0.000000,0.000000}%
\pgfsetstrokecolor{currentstroke}%
\pgfsetdash{}{0pt}%
\pgfpathmoveto{\pgfqpoint{2.311708in}{4.753099in}}%
\pgfpathlineto{\pgfqpoint{2.320903in}{4.651594in}}%
\pgfpathlineto{\pgfqpoint{2.328163in}{4.655677in}}%
\pgfpathlineto{\pgfqpoint{2.362015in}{4.653894in}}%
\pgfpathlineto{\pgfqpoint{2.397287in}{4.570495in}}%
\pgfpathlineto{\pgfqpoint{2.387235in}{4.720310in}}%
\pgfpathlineto{\pgfqpoint{2.382151in}{4.590743in}}%
\pgfpathlineto{\pgfqpoint{2.345787in}{4.736956in}}%
\pgfpathlineto{\pgfqpoint{2.311708in}{4.753099in}}%
\pgfpathclose%
\pgfusepath{fill}%
\end{pgfscope}%
\begin{pgfscope}%
\pgfpathrectangle{\pgfqpoint{1.020000in}{0.880000in}}{\pgfqpoint{6.160000in}{6.160000in}}%
\pgfusepath{clip}%
\pgfsetbuttcap%
\pgfsetroundjoin%
\definecolor{currentfill}{rgb}{0.705673,0.015556,0.150233}%
\pgfsetfillcolor{currentfill}%
\pgfsetlinewidth{0.000000pt}%
\definecolor{currentstroke}{rgb}{0.000000,0.000000,0.000000}%
\pgfsetstrokecolor{currentstroke}%
\pgfsetdash{}{0pt}%
\pgfpathmoveto{\pgfqpoint{2.825366in}{5.422365in}}%
\pgfpathlineto{\pgfqpoint{2.836955in}{5.141519in}}%
\pgfpathlineto{\pgfqpoint{2.842361in}{5.328530in}}%
\pgfpathlineto{\pgfqpoint{2.875483in}{5.370717in}}%
\pgfpathlineto{\pgfqpoint{2.908584in}{5.415276in}}%
\pgfpathlineto{\pgfqpoint{2.901903in}{5.314075in}}%
\pgfpathlineto{\pgfqpoint{2.893043in}{5.387701in}}%
\pgfpathlineto{\pgfqpoint{2.858813in}{5.436360in}}%
\pgfpathlineto{\pgfqpoint{2.825366in}{5.422365in}}%
\pgfpathclose%
\pgfusepath{fill}%
\end{pgfscope}%
\begin{pgfscope}%
\pgfpathrectangle{\pgfqpoint{1.020000in}{0.880000in}}{\pgfqpoint{6.160000in}{6.160000in}}%
\pgfusepath{clip}%
\pgfsetbuttcap%
\pgfsetroundjoin%
\definecolor{currentfill}{rgb}{0.478462,0.616564,0.972721}%
\pgfsetfillcolor{currentfill}%
\pgfsetlinewidth{0.000000pt}%
\definecolor{currentstroke}{rgb}{0.000000,0.000000,0.000000}%
\pgfsetstrokecolor{currentstroke}%
\pgfsetdash{}{0pt}%
\pgfpathmoveto{\pgfqpoint{4.433092in}{3.313767in}}%
\pgfpathlineto{\pgfqpoint{4.443143in}{3.511817in}}%
\pgfpathlineto{\pgfqpoint{4.452357in}{3.377500in}}%
\pgfpathlineto{\pgfqpoint{4.485470in}{3.345491in}}%
\pgfpathlineto{\pgfqpoint{4.518074in}{3.174078in}}%
\pgfpathlineto{\pgfqpoint{4.509161in}{3.387234in}}%
\pgfpathlineto{\pgfqpoint{4.499492in}{3.367992in}}%
\pgfpathlineto{\pgfqpoint{4.466588in}{3.445137in}}%
\pgfpathlineto{\pgfqpoint{4.433092in}{3.313767in}}%
\pgfpathclose%
\pgfusepath{fill}%
\end{pgfscope}%
\begin{pgfscope}%
\pgfpathrectangle{\pgfqpoint{1.020000in}{0.880000in}}{\pgfqpoint{6.160000in}{6.160000in}}%
\pgfusepath{clip}%
\pgfsetbuttcap%
\pgfsetroundjoin%
\definecolor{currentfill}{rgb}{0.378598,0.503856,0.913692}%
\pgfsetfillcolor{currentfill}%
\pgfsetlinewidth{0.000000pt}%
\definecolor{currentstroke}{rgb}{0.000000,0.000000,0.000000}%
\pgfsetstrokecolor{currentstroke}%
\pgfsetdash{}{0pt}%
\pgfpathmoveto{\pgfqpoint{4.736072in}{3.175567in}}%
\pgfpathlineto{\pgfqpoint{4.746482in}{3.272594in}}%
\pgfpathlineto{\pgfqpoint{4.755764in}{3.166021in}}%
\pgfpathlineto{\pgfqpoint{4.789383in}{3.249483in}}%
\pgfpathlineto{\pgfqpoint{4.822507in}{3.250147in}}%
\pgfpathlineto{\pgfqpoint{4.811283in}{3.049073in}}%
\pgfpathlineto{\pgfqpoint{4.802401in}{3.215455in}}%
\pgfpathlineto{\pgfqpoint{4.768478in}{3.065910in}}%
\pgfpathlineto{\pgfqpoint{4.736072in}{3.175567in}}%
\pgfpathclose%
\pgfusepath{fill}%
\end{pgfscope}%
\begin{pgfscope}%
\pgfpathrectangle{\pgfqpoint{1.020000in}{0.880000in}}{\pgfqpoint{6.160000in}{6.160000in}}%
\pgfusepath{clip}%
\pgfsetbuttcap%
\pgfsetroundjoin%
\definecolor{currentfill}{rgb}{0.956653,0.598034,0.477302}%
\pgfsetfillcolor{currentfill}%
\pgfsetlinewidth{0.000000pt}%
\definecolor{currentstroke}{rgb}{0.000000,0.000000,0.000000}%
\pgfsetstrokecolor{currentstroke}%
\pgfsetdash{}{0pt}%
\pgfpathmoveto{\pgfqpoint{2.382151in}{4.590743in}}%
\pgfpathlineto{\pgfqpoint{2.387235in}{4.720310in}}%
\pgfpathlineto{\pgfqpoint{2.397287in}{4.570495in}}%
\pgfpathlineto{\pgfqpoint{2.425054in}{4.915946in}}%
\pgfpathlineto{\pgfqpoint{2.460509in}{4.821319in}}%
\pgfpathlineto{\pgfqpoint{2.453256in}{4.808013in}}%
\pgfpathlineto{\pgfqpoint{2.446490in}{4.767370in}}%
\pgfpathlineto{\pgfqpoint{2.413757in}{4.710163in}}%
\pgfpathlineto{\pgfqpoint{2.382151in}{4.590743in}}%
\pgfpathclose%
\pgfusepath{fill}%
\end{pgfscope}%
\begin{pgfscope}%
\pgfpathrectangle{\pgfqpoint{1.020000in}{0.880000in}}{\pgfqpoint{6.160000in}{6.160000in}}%
\pgfusepath{clip}%
\pgfsetbuttcap%
\pgfsetroundjoin%
\definecolor{currentfill}{rgb}{0.353369,0.472069,0.892570}%
\pgfsetfillcolor{currentfill}%
\pgfsetlinewidth{0.000000pt}%
\definecolor{currentstroke}{rgb}{0.000000,0.000000,0.000000}%
\pgfsetstrokecolor{currentstroke}%
\pgfsetdash{}{0pt}%
\pgfpathmoveto{\pgfqpoint{5.456589in}{3.165376in}}%
\pgfpathlineto{\pgfqpoint{5.467343in}{3.174825in}}%
\pgfpathlineto{\pgfqpoint{5.478745in}{3.232380in}}%
\pgfpathlineto{\pgfqpoint{5.508749in}{3.012208in}}%
\pgfpathlineto{\pgfqpoint{5.541675in}{3.018322in}}%
\pgfpathlineto{\pgfqpoint{5.532574in}{3.136643in}}%
\pgfpathlineto{\pgfqpoint{5.522543in}{3.185865in}}%
\pgfpathlineto{\pgfqpoint{5.489615in}{3.178832in}}%
\pgfpathlineto{\pgfqpoint{5.456589in}{3.165376in}}%
\pgfpathclose%
\pgfusepath{fill}%
\end{pgfscope}%
\begin{pgfscope}%
\pgfpathrectangle{\pgfqpoint{1.020000in}{0.880000in}}{\pgfqpoint{6.160000in}{6.160000in}}%
\pgfusepath{clip}%
\pgfsetbuttcap%
\pgfsetroundjoin%
\definecolor{currentfill}{rgb}{0.922681,0.828568,0.777054}%
\pgfsetfillcolor{currentfill}%
\pgfsetlinewidth{0.000000pt}%
\definecolor{currentstroke}{rgb}{0.000000,0.000000,0.000000}%
\pgfsetstrokecolor{currentstroke}%
\pgfsetdash{}{0pt}%
\pgfpathmoveto{\pgfqpoint{3.638096in}{4.421174in}}%
\pgfpathlineto{\pgfqpoint{3.648114in}{4.176782in}}%
\pgfpathlineto{\pgfqpoint{3.656960in}{4.151954in}}%
\pgfpathlineto{\pgfqpoint{3.689827in}{4.261801in}}%
\pgfpathlineto{\pgfqpoint{3.723417in}{4.224999in}}%
\pgfpathlineto{\pgfqpoint{3.714593in}{4.233250in}}%
\pgfpathlineto{\pgfqpoint{3.705600in}{4.280714in}}%
\pgfpathlineto{\pgfqpoint{3.671917in}{4.343799in}}%
\pgfpathlineto{\pgfqpoint{3.638096in}{4.421174in}}%
\pgfpathclose%
\pgfusepath{fill}%
\end{pgfscope}%
\begin{pgfscope}%
\pgfpathrectangle{\pgfqpoint{1.020000in}{0.880000in}}{\pgfqpoint{6.160000in}{6.160000in}}%
\pgfusepath{clip}%
\pgfsetbuttcap%
\pgfsetroundjoin%
\definecolor{currentfill}{rgb}{0.873402,0.386960,0.306332}%
\pgfsetfillcolor{currentfill}%
\pgfsetlinewidth{0.000000pt}%
\definecolor{currentstroke}{rgb}{0.000000,0.000000,0.000000}%
\pgfsetstrokecolor{currentstroke}%
\pgfsetdash{}{0pt}%
\pgfpathmoveto{\pgfqpoint{3.213900in}{5.036495in}}%
\pgfpathlineto{\pgfqpoint{3.221757in}{5.079941in}}%
\pgfpathlineto{\pgfqpoint{3.229803in}{5.105822in}}%
\pgfpathlineto{\pgfqpoint{3.265348in}{4.889766in}}%
\pgfpathlineto{\pgfqpoint{3.298081in}{4.977698in}}%
\pgfpathlineto{\pgfqpoint{3.289507in}{5.003609in}}%
\pgfpathlineto{\pgfqpoint{3.280469in}{5.083313in}}%
\pgfpathlineto{\pgfqpoint{3.247310in}{5.046508in}}%
\pgfpathlineto{\pgfqpoint{3.213900in}{5.036495in}}%
\pgfpathclose%
\pgfusepath{fill}%
\end{pgfscope}%
\begin{pgfscope}%
\pgfpathrectangle{\pgfqpoint{1.020000in}{0.880000in}}{\pgfqpoint{6.160000in}{6.160000in}}%
\pgfusepath{clip}%
\pgfsetbuttcap%
\pgfsetroundjoin%
\definecolor{currentfill}{rgb}{0.435815,0.570707,0.951717}%
\pgfsetfillcolor{currentfill}%
\pgfsetlinewidth{0.000000pt}%
\definecolor{currentstroke}{rgb}{0.000000,0.000000,0.000000}%
\pgfsetstrokecolor{currentstroke}%
\pgfsetdash{}{0pt}%
\pgfpathmoveto{\pgfqpoint{4.585214in}{3.411070in}}%
\pgfpathlineto{\pgfqpoint{4.594424in}{3.288400in}}%
\pgfpathlineto{\pgfqpoint{4.604078in}{3.271552in}}%
\pgfpathlineto{\pgfqpoint{4.637051in}{3.230365in}}%
\pgfpathlineto{\pgfqpoint{4.670719in}{3.340999in}}%
\pgfpathlineto{\pgfqpoint{4.659894in}{3.125487in}}%
\pgfpathlineto{\pgfqpoint{4.651052in}{3.319836in}}%
\pgfpathlineto{\pgfqpoint{4.617906in}{3.307661in}}%
\pgfpathlineto{\pgfqpoint{4.585214in}{3.411070in}}%
\pgfpathclose%
\pgfusepath{fill}%
\end{pgfscope}%
\begin{pgfscope}%
\pgfpathrectangle{\pgfqpoint{1.020000in}{0.880000in}}{\pgfqpoint{6.160000in}{6.160000in}}%
\pgfusepath{clip}%
\pgfsetbuttcap%
\pgfsetroundjoin%
\definecolor{currentfill}{rgb}{0.967711,0.662973,0.544323}%
\pgfsetfillcolor{currentfill}%
\pgfsetlinewidth{0.000000pt}%
\definecolor{currentstroke}{rgb}{0.000000,0.000000,0.000000}%
\pgfsetstrokecolor{currentstroke}%
\pgfsetdash{}{0pt}%
\pgfpathmoveto{\pgfqpoint{2.248229in}{4.536179in}}%
\pgfpathlineto{\pgfqpoint{2.253443in}{4.642650in}}%
\pgfpathlineto{\pgfqpoint{2.259987in}{4.680774in}}%
\pgfpathlineto{\pgfqpoint{2.297599in}{4.480129in}}%
\pgfpathlineto{\pgfqpoint{2.328163in}{4.655677in}}%
\pgfpathlineto{\pgfqpoint{2.320903in}{4.651594in}}%
\pgfpathlineto{\pgfqpoint{2.311708in}{4.753099in}}%
\pgfpathlineto{\pgfqpoint{2.279330in}{4.677191in}}%
\pgfpathlineto{\pgfqpoint{2.248229in}{4.536179in}}%
\pgfpathclose%
\pgfusepath{fill}%
\end{pgfscope}%
\begin{pgfscope}%
\pgfpathrectangle{\pgfqpoint{1.020000in}{0.880000in}}{\pgfqpoint{6.160000in}{6.160000in}}%
\pgfusepath{clip}%
\pgfsetbuttcap%
\pgfsetroundjoin%
\definecolor{currentfill}{rgb}{0.348323,0.465711,0.888346}%
\pgfsetfillcolor{currentfill}%
\pgfsetlinewidth{0.000000pt}%
\definecolor{currentstroke}{rgb}{0.000000,0.000000,0.000000}%
\pgfsetstrokecolor{currentstroke}%
\pgfsetdash{}{0pt}%
\pgfpathmoveto{\pgfqpoint{5.674883in}{3.153695in}}%
\pgfpathlineto{\pgfqpoint{5.687178in}{3.248525in}}%
\pgfpathlineto{\pgfqpoint{5.695219in}{3.060134in}}%
\pgfpathlineto{\pgfqpoint{5.726872in}{2.984629in}}%
\pgfpathlineto{\pgfqpoint{5.760841in}{3.059724in}}%
\pgfpathlineto{\pgfqpoint{5.753134in}{3.265460in}}%
\pgfpathlineto{\pgfqpoint{5.738696in}{3.041498in}}%
\pgfpathlineto{\pgfqpoint{5.708125in}{3.183154in}}%
\pgfpathlineto{\pgfqpoint{5.674883in}{3.153695in}}%
\pgfpathclose%
\pgfusepath{fill}%
\end{pgfscope}%
\begin{pgfscope}%
\pgfpathrectangle{\pgfqpoint{1.020000in}{0.880000in}}{\pgfqpoint{6.160000in}{6.160000in}}%
\pgfusepath{clip}%
\pgfsetbuttcap%
\pgfsetroundjoin%
\definecolor{currentfill}{rgb}{0.343278,0.459354,0.884122}%
\pgfsetfillcolor{currentfill}%
\pgfsetlinewidth{0.000000pt}%
\definecolor{currentstroke}{rgb}{0.000000,0.000000,0.000000}%
\pgfsetstrokecolor{currentstroke}%
\pgfsetdash{}{0pt}%
\pgfpathmoveto{\pgfqpoint{4.886643in}{2.960671in}}%
\pgfpathlineto{\pgfqpoint{4.898588in}{3.230401in}}%
\pgfpathlineto{\pgfqpoint{4.907841in}{3.119158in}}%
\pgfpathlineto{\pgfqpoint{4.941177in}{3.152536in}}%
\pgfpathlineto{\pgfqpoint{4.973885in}{3.106443in}}%
\pgfpathlineto{\pgfqpoint{4.964321in}{3.176578in}}%
\pgfpathlineto{\pgfqpoint{4.953557in}{3.090357in}}%
\pgfpathlineto{\pgfqpoint{4.920234in}{3.045258in}}%
\pgfpathlineto{\pgfqpoint{4.886643in}{2.960671in}}%
\pgfpathclose%
\pgfusepath{fill}%
\end{pgfscope}%
\begin{pgfscope}%
\pgfpathrectangle{\pgfqpoint{1.020000in}{0.880000in}}{\pgfqpoint{6.160000in}{6.160000in}}%
\pgfusepath{clip}%
\pgfsetbuttcap%
\pgfsetroundjoin%
\definecolor{currentfill}{rgb}{0.962708,0.753557,0.655601}%
\pgfsetfillcolor{currentfill}%
\pgfsetlinewidth{0.000000pt}%
\definecolor{currentstroke}{rgb}{0.000000,0.000000,0.000000}%
\pgfsetstrokecolor{currentstroke}%
\pgfsetdash{}{0pt}%
\pgfpathmoveto{\pgfqpoint{3.552609in}{4.632390in}}%
\pgfpathlineto{\pgfqpoint{3.562206in}{4.476159in}}%
\pgfpathlineto{\pgfqpoint{3.571786in}{4.321181in}}%
\pgfpathlineto{\pgfqpoint{3.605222in}{4.320901in}}%
\pgfpathlineto{\pgfqpoint{3.638096in}{4.421174in}}%
\pgfpathlineto{\pgfqpoint{3.628609in}{4.564456in}}%
\pgfpathlineto{\pgfqpoint{3.621052in}{4.360240in}}%
\pgfpathlineto{\pgfqpoint{3.586756in}{4.518300in}}%
\pgfpathlineto{\pgfqpoint{3.552609in}{4.632390in}}%
\pgfpathclose%
\pgfusepath{fill}%
\end{pgfscope}%
\begin{pgfscope}%
\pgfpathrectangle{\pgfqpoint{1.020000in}{0.880000in}}{\pgfqpoint{6.160000in}{6.160000in}}%
\pgfusepath{clip}%
\pgfsetbuttcap%
\pgfsetroundjoin%
\definecolor{currentfill}{rgb}{0.304174,0.406945,0.845263}%
\pgfsetfillcolor{currentfill}%
\pgfsetlinewidth{0.000000pt}%
\definecolor{currentstroke}{rgb}{0.000000,0.000000,0.000000}%
\pgfsetstrokecolor{currentstroke}%
\pgfsetdash{}{0pt}%
\pgfpathmoveto{\pgfqpoint{5.039298in}{3.031456in}}%
\pgfpathlineto{\pgfqpoint{5.050148in}{3.107229in}}%
\pgfpathlineto{\pgfqpoint{5.058727in}{2.920758in}}%
\pgfpathlineto{\pgfqpoint{5.092112in}{2.962615in}}%
\pgfpathlineto{\pgfqpoint{5.125477in}{3.001588in}}%
\pgfpathlineto{\pgfqpoint{5.117064in}{3.200309in}}%
\pgfpathlineto{\pgfqpoint{5.104463in}{2.948689in}}%
\pgfpathlineto{\pgfqpoint{5.072892in}{3.100819in}}%
\pgfpathlineto{\pgfqpoint{5.039298in}{3.031456in}}%
\pgfpathclose%
\pgfusepath{fill}%
\end{pgfscope}%
\begin{pgfscope}%
\pgfpathrectangle{\pgfqpoint{1.020000in}{0.880000in}}{\pgfqpoint{6.160000in}{6.160000in}}%
\pgfusepath{clip}%
\pgfsetbuttcap%
\pgfsetroundjoin%
\definecolor{currentfill}{rgb}{0.796064,0.848693,0.933471}%
\pgfsetfillcolor{currentfill}%
\pgfsetlinewidth{0.000000pt}%
\definecolor{currentstroke}{rgb}{0.000000,0.000000,0.000000}%
\pgfsetstrokecolor{currentstroke}%
\pgfsetdash{}{0pt}%
\pgfpathmoveto{\pgfqpoint{3.742014in}{4.003790in}}%
\pgfpathlineto{\pgfqpoint{3.751528in}{3.842825in}}%
\pgfpathlineto{\pgfqpoint{3.759857in}{3.960979in}}%
\pgfpathlineto{\pgfqpoint{3.793444in}{3.921239in}}%
\pgfpathlineto{\pgfqpoint{3.826946in}{3.892285in}}%
\pgfpathlineto{\pgfqpoint{3.818195in}{3.852560in}}%
\pgfpathlineto{\pgfqpoint{3.808669in}{4.027765in}}%
\pgfpathlineto{\pgfqpoint{3.775059in}{4.086953in}}%
\pgfpathlineto{\pgfqpoint{3.742014in}{4.003790in}}%
\pgfpathclose%
\pgfusepath{fill}%
\end{pgfscope}%
\begin{pgfscope}%
\pgfpathrectangle{\pgfqpoint{1.020000in}{0.880000in}}{\pgfqpoint{6.160000in}{6.160000in}}%
\pgfusepath{clip}%
\pgfsetbuttcap%
\pgfsetroundjoin%
\definecolor{currentfill}{rgb}{0.752704,0.157576,0.184258}%
\pgfsetfillcolor{currentfill}%
\pgfsetlinewidth{0.000000pt}%
\definecolor{currentstroke}{rgb}{0.000000,0.000000,0.000000}%
\pgfsetstrokecolor{currentstroke}%
\pgfsetdash{}{0pt}%
\pgfpathmoveto{\pgfqpoint{2.908584in}{5.415276in}}%
\pgfpathlineto{\pgfqpoint{2.919245in}{5.198615in}}%
\pgfpathlineto{\pgfqpoint{2.926020in}{5.295375in}}%
\pgfpathlineto{\pgfqpoint{2.960187in}{5.254714in}}%
\pgfpathlineto{\pgfqpoint{2.993271in}{5.302766in}}%
\pgfpathlineto{\pgfqpoint{2.985468in}{5.282377in}}%
\pgfpathlineto{\pgfqpoint{2.977520in}{5.275881in}}%
\pgfpathlineto{\pgfqpoint{2.946168in}{5.092170in}}%
\pgfpathlineto{\pgfqpoint{2.908584in}{5.415276in}}%
\pgfpathclose%
\pgfusepath{fill}%
\end{pgfscope}%
\begin{pgfscope}%
\pgfpathrectangle{\pgfqpoint{1.020000in}{0.880000in}}{\pgfqpoint{6.160000in}{6.160000in}}%
\pgfusepath{clip}%
\pgfsetbuttcap%
\pgfsetroundjoin%
\definecolor{currentfill}{rgb}{0.768929,0.189213,0.197965}%
\pgfsetfillcolor{currentfill}%
\pgfsetlinewidth{0.000000pt}%
\definecolor{currentstroke}{rgb}{0.000000,0.000000,0.000000}%
\pgfsetstrokecolor{currentstroke}%
\pgfsetdash{}{0pt}%
\pgfpathmoveto{\pgfqpoint{3.126698in}{5.401676in}}%
\pgfpathlineto{\pgfqpoint{3.136533in}{5.242434in}}%
\pgfpathlineto{\pgfqpoint{3.144507in}{5.264821in}}%
\pgfpathlineto{\pgfqpoint{3.178975in}{5.178657in}}%
\pgfpathlineto{\pgfqpoint{3.213900in}{5.036495in}}%
\pgfpathlineto{\pgfqpoint{3.204974in}{5.106683in}}%
\pgfpathlineto{\pgfqpoint{3.195282in}{5.256354in}}%
\pgfpathlineto{\pgfqpoint{3.160765in}{5.354840in}}%
\pgfpathlineto{\pgfqpoint{3.126698in}{5.401676in}}%
\pgfpathclose%
\pgfusepath{fill}%
\end{pgfscope}%
\begin{pgfscope}%
\pgfpathrectangle{\pgfqpoint{1.020000in}{0.880000in}}{\pgfqpoint{6.160000in}{6.160000in}}%
\pgfusepath{clip}%
\pgfsetbuttcap%
\pgfsetroundjoin%
\definecolor{currentfill}{rgb}{0.949454,0.572388,0.453443}%
\pgfsetfillcolor{currentfill}%
\pgfsetlinewidth{0.000000pt}%
\definecolor{currentstroke}{rgb}{0.000000,0.000000,0.000000}%
\pgfsetstrokecolor{currentstroke}%
\pgfsetdash{}{0pt}%
\pgfpathmoveto{\pgfqpoint{3.384685in}{4.631445in}}%
\pgfpathlineto{\pgfqpoint{3.392140in}{4.756879in}}%
\pgfpathlineto{\pgfqpoint{3.401400in}{4.653689in}}%
\pgfpathlineto{\pgfqpoint{3.432577in}{4.966920in}}%
\pgfpathlineto{\pgfqpoint{3.468760in}{4.597977in}}%
\pgfpathlineto{\pgfqpoint{3.458983in}{4.774718in}}%
\pgfpathlineto{\pgfqpoint{3.450318in}{4.795929in}}%
\pgfpathlineto{\pgfqpoint{3.415300in}{5.002567in}}%
\pgfpathlineto{\pgfqpoint{3.384685in}{4.631445in}}%
\pgfpathclose%
\pgfusepath{fill}%
\end{pgfscope}%
\begin{pgfscope}%
\pgfpathrectangle{\pgfqpoint{1.020000in}{0.880000in}}{\pgfqpoint{6.160000in}{6.160000in}}%
\pgfusepath{clip}%
\pgfsetbuttcap%
\pgfsetroundjoin%
\definecolor{currentfill}{rgb}{0.908908,0.462433,0.360950}%
\pgfsetfillcolor{currentfill}%
\pgfsetlinewidth{0.000000pt}%
\definecolor{currentstroke}{rgb}{0.000000,0.000000,0.000000}%
\pgfsetstrokecolor{currentstroke}%
\pgfsetdash{}{0pt}%
\pgfpathmoveto{\pgfqpoint{3.298081in}{4.977698in}}%
\pgfpathlineto{\pgfqpoint{3.306109in}{5.016540in}}%
\pgfpathlineto{\pgfqpoint{3.314742in}{4.987284in}}%
\pgfpathlineto{\pgfqpoint{3.348726in}{4.937894in}}%
\pgfpathlineto{\pgfqpoint{3.384685in}{4.631445in}}%
\pgfpathlineto{\pgfqpoint{3.375761in}{4.695150in}}%
\pgfpathlineto{\pgfqpoint{3.364194in}{5.087683in}}%
\pgfpathlineto{\pgfqpoint{3.330796in}{5.072389in}}%
\pgfpathlineto{\pgfqpoint{3.298081in}{4.977698in}}%
\pgfpathclose%
\pgfusepath{fill}%
\end{pgfscope}%
\begin{pgfscope}%
\pgfpathrectangle{\pgfqpoint{1.020000in}{0.880000in}}{\pgfqpoint{6.160000in}{6.160000in}}%
\pgfusepath{clip}%
\pgfsetbuttcap%
\pgfsetroundjoin%
\definecolor{currentfill}{rgb}{0.711554,0.033337,0.154485}%
\pgfsetfillcolor{currentfill}%
\pgfsetlinewidth{0.000000pt}%
\definecolor{currentstroke}{rgb}{0.000000,0.000000,0.000000}%
\pgfsetstrokecolor{currentstroke}%
\pgfsetdash{}{0pt}%
\pgfpathmoveto{\pgfqpoint{2.977520in}{5.275881in}}%
\pgfpathlineto{\pgfqpoint{2.985468in}{5.282377in}}%
\pgfpathlineto{\pgfqpoint{2.993271in}{5.302766in}}%
\pgfpathlineto{\pgfqpoint{3.026468in}{5.342014in}}%
\pgfpathlineto{\pgfqpoint{3.061034in}{5.257577in}}%
\pgfpathlineto{\pgfqpoint{3.052883in}{5.261742in}}%
\pgfpathlineto{\pgfqpoint{3.042270in}{5.487655in}}%
\pgfpathlineto{\pgfqpoint{3.009346in}{5.427281in}}%
\pgfpathlineto{\pgfqpoint{2.977520in}{5.275881in}}%
\pgfpathclose%
\pgfusepath{fill}%
\end{pgfscope}%
\begin{pgfscope}%
\pgfpathrectangle{\pgfqpoint{1.020000in}{0.880000in}}{\pgfqpoint{6.160000in}{6.160000in}}%
\pgfusepath{clip}%
\pgfsetbuttcap%
\pgfsetroundjoin%
\definecolor{currentfill}{rgb}{0.672538,0.782861,0.991982}%
\pgfsetfillcolor{currentfill}%
\pgfsetlinewidth{0.000000pt}%
\definecolor{currentstroke}{rgb}{0.000000,0.000000,0.000000}%
\pgfsetstrokecolor{currentstroke}%
\pgfsetdash{}{0pt}%
\pgfpathmoveto{\pgfqpoint{3.978693in}{3.758230in}}%
\pgfpathlineto{\pgfqpoint{3.987755in}{3.762458in}}%
\pgfpathlineto{\pgfqpoint{3.997147in}{3.600036in}}%
\pgfpathlineto{\pgfqpoint{4.030322in}{3.698843in}}%
\pgfpathlineto{\pgfqpoint{4.063603in}{3.746933in}}%
\pgfpathlineto{\pgfqpoint{4.054554in}{3.664645in}}%
\pgfpathlineto{\pgfqpoint{4.045314in}{3.746443in}}%
\pgfpathlineto{\pgfqpoint{4.012067in}{3.721871in}}%
\pgfpathlineto{\pgfqpoint{3.978693in}{3.758230in}}%
\pgfpathclose%
\pgfusepath{fill}%
\end{pgfscope}%
\begin{pgfscope}%
\pgfpathrectangle{\pgfqpoint{1.020000in}{0.880000in}}{\pgfqpoint{6.160000in}{6.160000in}}%
\pgfusepath{clip}%
\pgfsetbuttcap%
\pgfsetroundjoin%
\definecolor{currentfill}{rgb}{0.728970,0.817464,0.973188}%
\pgfsetfillcolor{currentfill}%
\pgfsetlinewidth{0.000000pt}%
\definecolor{currentstroke}{rgb}{0.000000,0.000000,0.000000}%
\pgfsetstrokecolor{currentstroke}%
\pgfsetdash{}{0pt}%
\pgfpathmoveto{\pgfqpoint{3.826946in}{3.892285in}}%
\pgfpathlineto{\pgfqpoint{3.835531in}{3.987643in}}%
\pgfpathlineto{\pgfqpoint{3.845619in}{3.647530in}}%
\pgfpathlineto{\pgfqpoint{3.878563in}{3.782570in}}%
\pgfpathlineto{\pgfqpoint{3.911953in}{3.778898in}}%
\pgfpathlineto{\pgfqpoint{3.903002in}{3.765451in}}%
\pgfpathlineto{\pgfqpoint{3.893895in}{3.813359in}}%
\pgfpathlineto{\pgfqpoint{3.860485in}{3.840102in}}%
\pgfpathlineto{\pgfqpoint{3.826946in}{3.892285in}}%
\pgfpathclose%
\pgfusepath{fill}%
\end{pgfscope}%
\begin{pgfscope}%
\pgfpathrectangle{\pgfqpoint{1.020000in}{0.880000in}}{\pgfqpoint{6.160000in}{6.160000in}}%
\pgfusepath{clip}%
\pgfsetbuttcap%
\pgfsetroundjoin%
\definecolor{currentfill}{rgb}{0.441123,0.576532,0.954545}%
\pgfsetfillcolor{currentfill}%
\pgfsetlinewidth{0.000000pt}%
\definecolor{currentstroke}{rgb}{0.000000,0.000000,0.000000}%
\pgfsetstrokecolor{currentstroke}%
\pgfsetdash{}{0pt}%
\pgfpathmoveto{\pgfqpoint{4.518074in}{3.174078in}}%
\pgfpathlineto{\pgfqpoint{4.528177in}{3.311632in}}%
\pgfpathlineto{\pgfqpoint{4.537828in}{3.308506in}}%
\pgfpathlineto{\pgfqpoint{4.571040in}{3.308207in}}%
\pgfpathlineto{\pgfqpoint{4.604078in}{3.271552in}}%
\pgfpathlineto{\pgfqpoint{4.594424in}{3.288400in}}%
\pgfpathlineto{\pgfqpoint{4.585214in}{3.411070in}}%
\pgfpathlineto{\pgfqpoint{4.551399in}{3.234145in}}%
\pgfpathlineto{\pgfqpoint{4.518074in}{3.174078in}}%
\pgfpathclose%
\pgfusepath{fill}%
\end{pgfscope}%
\begin{pgfscope}%
\pgfpathrectangle{\pgfqpoint{1.020000in}{0.880000in}}{\pgfqpoint{6.160000in}{6.160000in}}%
\pgfusepath{clip}%
\pgfsetbuttcap%
\pgfsetroundjoin%
\definecolor{currentfill}{rgb}{0.966922,0.651969,0.531997}%
\pgfsetfillcolor{currentfill}%
\pgfsetlinewidth{0.000000pt}%
\definecolor{currentstroke}{rgb}{0.000000,0.000000,0.000000}%
\pgfsetstrokecolor{currentstroke}%
\pgfsetdash{}{0pt}%
\pgfpathmoveto{\pgfqpoint{3.468760in}{4.597977in}}%
\pgfpathlineto{\pgfqpoint{3.476322in}{4.736957in}}%
\pgfpathlineto{\pgfqpoint{3.485597in}{4.633442in}}%
\pgfpathlineto{\pgfqpoint{3.519970in}{4.502381in}}%
\pgfpathlineto{\pgfqpoint{3.552609in}{4.632390in}}%
\pgfpathlineto{\pgfqpoint{3.544881in}{4.491598in}}%
\pgfpathlineto{\pgfqpoint{3.534646in}{4.748804in}}%
\pgfpathlineto{\pgfqpoint{3.501048in}{4.768402in}}%
\pgfpathlineto{\pgfqpoint{3.468760in}{4.597977in}}%
\pgfpathclose%
\pgfusepath{fill}%
\end{pgfscope}%
\begin{pgfscope}%
\pgfpathrectangle{\pgfqpoint{1.020000in}{0.880000in}}{\pgfqpoint{6.160000in}{6.160000in}}%
\pgfusepath{clip}%
\pgfsetbuttcap%
\pgfsetroundjoin%
\definecolor{currentfill}{rgb}{0.494638,0.633022,0.978983}%
\pgfsetfillcolor{currentfill}%
\pgfsetlinewidth{0.000000pt}%
\definecolor{currentstroke}{rgb}{0.000000,0.000000,0.000000}%
\pgfsetstrokecolor{currentstroke}%
\pgfsetdash{}{0pt}%
\pgfpathmoveto{\pgfqpoint{4.366848in}{3.348079in}}%
\pgfpathlineto{\pgfqpoint{4.376378in}{3.376814in}}%
\pgfpathlineto{\pgfqpoint{4.385968in}{3.422099in}}%
\pgfpathlineto{\pgfqpoint{4.419211in}{3.413065in}}%
\pgfpathlineto{\pgfqpoint{4.452357in}{3.377500in}}%
\pgfpathlineto{\pgfqpoint{4.443143in}{3.511817in}}%
\pgfpathlineto{\pgfqpoint{4.433092in}{3.313767in}}%
\pgfpathlineto{\pgfqpoint{4.399995in}{3.335399in}}%
\pgfpathlineto{\pgfqpoint{4.366848in}{3.348079in}}%
\pgfpathclose%
\pgfusepath{fill}%
\end{pgfscope}%
\begin{pgfscope}%
\pgfpathrectangle{\pgfqpoint{1.020000in}{0.880000in}}{\pgfqpoint{6.160000in}{6.160000in}}%
\pgfusepath{clip}%
\pgfsetbuttcap%
\pgfsetroundjoin%
\definecolor{currentfill}{rgb}{0.338377,0.452819,0.879317}%
\pgfsetfillcolor{currentfill}%
\pgfsetlinewidth{0.000000pt}%
\definecolor{currentstroke}{rgb}{0.000000,0.000000,0.000000}%
\pgfsetstrokecolor{currentstroke}%
\pgfsetdash{}{0pt}%
\pgfpathmoveto{\pgfqpoint{5.324443in}{3.116843in}}%
\pgfpathlineto{\pgfqpoint{5.334371in}{3.067427in}}%
\pgfpathlineto{\pgfqpoint{5.345124in}{3.087708in}}%
\pgfpathlineto{\pgfqpoint{5.377109in}{3.005999in}}%
\pgfpathlineto{\pgfqpoint{5.413245in}{3.267152in}}%
\pgfpathlineto{\pgfqpoint{5.401407in}{3.165860in}}%
\pgfpathlineto{\pgfqpoint{5.390190in}{3.112289in}}%
\pgfpathlineto{\pgfqpoint{5.355453in}{2.955419in}}%
\pgfpathlineto{\pgfqpoint{5.324443in}{3.116843in}}%
\pgfpathclose%
\pgfusepath{fill}%
\end{pgfscope}%
\begin{pgfscope}%
\pgfpathrectangle{\pgfqpoint{1.020000in}{0.880000in}}{\pgfqpoint{6.160000in}{6.160000in}}%
\pgfusepath{clip}%
\pgfsetbuttcap%
\pgfsetroundjoin%
\definecolor{currentfill}{rgb}{0.358415,0.478426,0.896795}%
\pgfsetfillcolor{currentfill}%
\pgfsetlinewidth{0.000000pt}%
\definecolor{currentstroke}{rgb}{0.000000,0.000000,0.000000}%
\pgfsetstrokecolor{currentstroke}%
\pgfsetdash{}{0pt}%
\pgfpathmoveto{\pgfqpoint{5.893261in}{3.141080in}}%
\pgfpathlineto{\pgfqpoint{5.908189in}{3.362811in}}%
\pgfpathlineto{\pgfqpoint{5.914215in}{3.067925in}}%
\pgfpathlineto{\pgfqpoint{5.947153in}{3.077204in}}%
\pgfpathlineto{\pgfqpoint{5.937271in}{3.149236in}}%
\pgfpathlineto{\pgfqpoint{5.923448in}{2.995328in}}%
\pgfpathlineto{\pgfqpoint{5.893261in}{3.141080in}}%
\pgfpathclose%
\pgfusepath{fill}%
\end{pgfscope}%
\begin{pgfscope}%
\pgfpathrectangle{\pgfqpoint{1.020000in}{0.880000in}}{\pgfqpoint{6.160000in}{6.160000in}}%
\pgfusepath{clip}%
\pgfsetbuttcap%
\pgfsetroundjoin%
\definecolor{currentfill}{rgb}{0.603162,0.731527,0.999565}%
\pgfsetfillcolor{currentfill}%
\pgfsetlinewidth{0.000000pt}%
\definecolor{currentstroke}{rgb}{0.000000,0.000000,0.000000}%
\pgfsetstrokecolor{currentstroke}%
\pgfsetdash{}{0pt}%
\pgfpathmoveto{\pgfqpoint{4.063603in}{3.746933in}}%
\pgfpathlineto{\pgfqpoint{4.072916in}{3.619923in}}%
\pgfpathlineto{\pgfqpoint{4.082143in}{3.574196in}}%
\pgfpathlineto{\pgfqpoint{4.115556in}{3.469793in}}%
\pgfpathlineto{\pgfqpoint{4.148842in}{3.475495in}}%
\pgfpathlineto{\pgfqpoint{4.139488in}{3.752456in}}%
\pgfpathlineto{\pgfqpoint{4.130411in}{3.446343in}}%
\pgfpathlineto{\pgfqpoint{4.097112in}{3.544107in}}%
\pgfpathlineto{\pgfqpoint{4.063603in}{3.746933in}}%
\pgfpathclose%
\pgfusepath{fill}%
\end{pgfscope}%
\begin{pgfscope}%
\pgfpathrectangle{\pgfqpoint{1.020000in}{0.880000in}}{\pgfqpoint{6.160000in}{6.160000in}}%
\pgfusepath{clip}%
\pgfsetbuttcap%
\pgfsetroundjoin%
\definecolor{currentfill}{rgb}{0.930669,0.818877,0.759146}%
\pgfsetfillcolor{currentfill}%
\pgfsetlinewidth{0.000000pt}%
\definecolor{currentstroke}{rgb}{0.000000,0.000000,0.000000}%
\pgfsetstrokecolor{currentstroke}%
\pgfsetdash{}{0pt}%
\pgfpathmoveto{\pgfqpoint{3.571786in}{4.321181in}}%
\pgfpathlineto{\pgfqpoint{3.580160in}{4.366485in}}%
\pgfpathlineto{\pgfqpoint{3.589702in}{4.218099in}}%
\pgfpathlineto{\pgfqpoint{3.622509in}{4.336758in}}%
\pgfpathlineto{\pgfqpoint{3.656960in}{4.151954in}}%
\pgfpathlineto{\pgfqpoint{3.648114in}{4.176782in}}%
\pgfpathlineto{\pgfqpoint{3.638096in}{4.421174in}}%
\pgfpathlineto{\pgfqpoint{3.605222in}{4.320901in}}%
\pgfpathlineto{\pgfqpoint{3.571786in}{4.321181in}}%
\pgfpathclose%
\pgfusepath{fill}%
\end{pgfscope}%
\begin{pgfscope}%
\pgfpathrectangle{\pgfqpoint{1.020000in}{0.880000in}}{\pgfqpoint{6.160000in}{6.160000in}}%
\pgfusepath{clip}%
\pgfsetbuttcap%
\pgfsetroundjoin%
\definecolor{currentfill}{rgb}{0.358415,0.478426,0.896795}%
\pgfsetfillcolor{currentfill}%
\pgfsetlinewidth{0.000000pt}%
\definecolor{currentstroke}{rgb}{0.000000,0.000000,0.000000}%
\pgfsetstrokecolor{currentstroke}%
\pgfsetdash{}{0pt}%
\pgfpathmoveto{\pgfqpoint{5.609252in}{3.154421in}}%
\pgfpathlineto{\pgfqpoint{5.619160in}{3.092338in}}%
\pgfpathlineto{\pgfqpoint{5.629252in}{3.042241in}}%
\pgfpathlineto{\pgfqpoint{5.664549in}{3.206599in}}%
\pgfpathlineto{\pgfqpoint{5.695219in}{3.060134in}}%
\pgfpathlineto{\pgfqpoint{5.687178in}{3.248525in}}%
\pgfpathlineto{\pgfqpoint{5.674883in}{3.153695in}}%
\pgfpathlineto{\pgfqpoint{5.641995in}{3.148421in}}%
\pgfpathlineto{\pgfqpoint{5.609252in}{3.154421in}}%
\pgfpathclose%
\pgfusepath{fill}%
\end{pgfscope}%
\begin{pgfscope}%
\pgfpathrectangle{\pgfqpoint{1.020000in}{0.880000in}}{\pgfqpoint{6.160000in}{6.160000in}}%
\pgfusepath{clip}%
\pgfsetbuttcap%
\pgfsetroundjoin%
\definecolor{currentfill}{rgb}{0.378598,0.503856,0.913692}%
\pgfsetfillcolor{currentfill}%
\pgfsetlinewidth{0.000000pt}%
\definecolor{currentstroke}{rgb}{0.000000,0.000000,0.000000}%
\pgfsetstrokecolor{currentstroke}%
\pgfsetdash{}{0pt}%
\pgfpathmoveto{\pgfqpoint{5.390190in}{3.112289in}}%
\pgfpathlineto{\pgfqpoint{5.401407in}{3.165860in}}%
\pgfpathlineto{\pgfqpoint{5.413245in}{3.267152in}}%
\pgfpathlineto{\pgfqpoint{5.445771in}{3.230820in}}%
\pgfpathlineto{\pgfqpoint{5.478745in}{3.232380in}}%
\pgfpathlineto{\pgfqpoint{5.467343in}{3.174825in}}%
\pgfpathlineto{\pgfqpoint{5.456589in}{3.165376in}}%
\pgfpathlineto{\pgfqpoint{5.422119in}{3.036622in}}%
\pgfpathlineto{\pgfqpoint{5.390190in}{3.112289in}}%
\pgfpathclose%
\pgfusepath{fill}%
\end{pgfscope}%
\begin{pgfscope}%
\pgfpathrectangle{\pgfqpoint{1.020000in}{0.880000in}}{\pgfqpoint{6.160000in}{6.160000in}}%
\pgfusepath{clip}%
\pgfsetbuttcap%
\pgfsetroundjoin%
\definecolor{currentfill}{rgb}{0.318832,0.426605,0.859857}%
\pgfsetfillcolor{currentfill}%
\pgfsetlinewidth{0.000000pt}%
\definecolor{currentstroke}{rgb}{0.000000,0.000000,0.000000}%
\pgfsetstrokecolor{currentstroke}%
\pgfsetdash{}{0pt}%
\pgfpathmoveto{\pgfqpoint{5.258340in}{3.096154in}}%
\pgfpathlineto{\pgfqpoint{5.266438in}{2.881812in}}%
\pgfpathlineto{\pgfqpoint{5.277790in}{2.965322in}}%
\pgfpathlineto{\pgfqpoint{5.312861in}{3.151423in}}%
\pgfpathlineto{\pgfqpoint{5.345124in}{3.087708in}}%
\pgfpathlineto{\pgfqpoint{5.334371in}{3.067427in}}%
\pgfpathlineto{\pgfqpoint{5.324443in}{3.116843in}}%
\pgfpathlineto{\pgfqpoint{5.291046in}{3.074845in}}%
\pgfpathlineto{\pgfqpoint{5.258340in}{3.096154in}}%
\pgfpathclose%
\pgfusepath{fill}%
\end{pgfscope}%
\begin{pgfscope}%
\pgfpathrectangle{\pgfqpoint{1.020000in}{0.880000in}}{\pgfqpoint{6.160000in}{6.160000in}}%
\pgfusepath{clip}%
\pgfsetbuttcap%
\pgfsetroundjoin%
\definecolor{currentfill}{rgb}{0.289996,0.386836,0.828926}%
\pgfsetfillcolor{currentfill}%
\pgfsetlinewidth{0.000000pt}%
\definecolor{currentstroke}{rgb}{0.000000,0.000000,0.000000}%
\pgfsetstrokecolor{currentstroke}%
\pgfsetdash{}{0pt}%
\pgfpathmoveto{\pgfqpoint{5.191271in}{2.985145in}}%
\pgfpathlineto{\pgfqpoint{5.201915in}{3.013110in}}%
\pgfpathlineto{\pgfqpoint{5.212739in}{3.055840in}}%
\pgfpathlineto{\pgfqpoint{5.245416in}{3.022529in}}%
\pgfpathlineto{\pgfqpoint{5.277790in}{2.965322in}}%
\pgfpathlineto{\pgfqpoint{5.266438in}{2.881812in}}%
\pgfpathlineto{\pgfqpoint{5.258340in}{3.096154in}}%
\pgfpathlineto{\pgfqpoint{5.224044in}{2.968636in}}%
\pgfpathlineto{\pgfqpoint{5.191271in}{2.985145in}}%
\pgfpathclose%
\pgfusepath{fill}%
\end{pgfscope}%
\begin{pgfscope}%
\pgfpathrectangle{\pgfqpoint{1.020000in}{0.880000in}}{\pgfqpoint{6.160000in}{6.160000in}}%
\pgfusepath{clip}%
\pgfsetbuttcap%
\pgfsetroundjoin%
\definecolor{currentfill}{rgb}{0.565182,0.699438,0.996635}%
\pgfsetfillcolor{currentfill}%
\pgfsetlinewidth{0.000000pt}%
\definecolor{currentstroke}{rgb}{0.000000,0.000000,0.000000}%
\pgfsetstrokecolor{currentstroke}%
\pgfsetdash{}{0pt}%
\pgfpathmoveto{\pgfqpoint{4.215410in}{3.622911in}}%
\pgfpathlineto{\pgfqpoint{4.224653in}{3.411128in}}%
\pgfpathlineto{\pgfqpoint{4.234053in}{3.546397in}}%
\pgfpathlineto{\pgfqpoint{4.267478in}{3.671885in}}%
\pgfpathlineto{\pgfqpoint{4.300526in}{3.422097in}}%
\pgfpathlineto{\pgfqpoint{4.291238in}{3.522749in}}%
\pgfpathlineto{\pgfqpoint{4.281723in}{3.393410in}}%
\pgfpathlineto{\pgfqpoint{4.248588in}{3.467416in}}%
\pgfpathlineto{\pgfqpoint{4.215410in}{3.622911in}}%
\pgfpathclose%
\pgfusepath{fill}%
\end{pgfscope}%
\begin{pgfscope}%
\pgfpathrectangle{\pgfqpoint{1.020000in}{0.880000in}}{\pgfqpoint{6.160000in}{6.160000in}}%
\pgfusepath{clip}%
\pgfsetbuttcap%
\pgfsetroundjoin%
\definecolor{currentfill}{rgb}{0.409611,0.540759,0.935545}%
\pgfsetfillcolor{currentfill}%
\pgfsetlinewidth{0.000000pt}%
\definecolor{currentstroke}{rgb}{0.000000,0.000000,0.000000}%
\pgfsetstrokecolor{currentstroke}%
\pgfsetdash{}{0pt}%
\pgfpathmoveto{\pgfqpoint{4.670719in}{3.340999in}}%
\pgfpathlineto{\pgfqpoint{4.680086in}{3.251043in}}%
\pgfpathlineto{\pgfqpoint{4.690237in}{3.314830in}}%
\pgfpathlineto{\pgfqpoint{4.722765in}{3.190468in}}%
\pgfpathlineto{\pgfqpoint{4.755764in}{3.166021in}}%
\pgfpathlineto{\pgfqpoint{4.746482in}{3.272594in}}%
\pgfpathlineto{\pgfqpoint{4.736072in}{3.175567in}}%
\pgfpathlineto{\pgfqpoint{4.702698in}{3.117356in}}%
\pgfpathlineto{\pgfqpoint{4.670719in}{3.340999in}}%
\pgfpathclose%
\pgfusepath{fill}%
\end{pgfscope}%
\begin{pgfscope}%
\pgfpathrectangle{\pgfqpoint{1.020000in}{0.880000in}}{\pgfqpoint{6.160000in}{6.160000in}}%
\pgfusepath{clip}%
\pgfsetbuttcap%
\pgfsetroundjoin%
\definecolor{currentfill}{rgb}{0.338377,0.452819,0.879317}%
\pgfsetfillcolor{currentfill}%
\pgfsetlinewidth{0.000000pt}%
\definecolor{currentstroke}{rgb}{0.000000,0.000000,0.000000}%
\pgfsetstrokecolor{currentstroke}%
\pgfsetdash{}{0pt}%
\pgfpathmoveto{\pgfqpoint{5.541675in}{3.018322in}}%
\pgfpathlineto{\pgfqpoint{5.553758in}{3.117146in}}%
\pgfpathlineto{\pgfqpoint{5.563798in}{3.065314in}}%
\pgfpathlineto{\pgfqpoint{5.596935in}{3.081749in}}%
\pgfpathlineto{\pgfqpoint{5.629252in}{3.042241in}}%
\pgfpathlineto{\pgfqpoint{5.619160in}{3.092338in}}%
\pgfpathlineto{\pgfqpoint{5.609252in}{3.154421in}}%
\pgfpathlineto{\pgfqpoint{5.576546in}{3.164811in}}%
\pgfpathlineto{\pgfqpoint{5.541675in}{3.018322in}}%
\pgfpathclose%
\pgfusepath{fill}%
\end{pgfscope}%
\begin{pgfscope}%
\pgfpathrectangle{\pgfqpoint{1.020000in}{0.880000in}}{\pgfqpoint{6.160000in}{6.160000in}}%
\pgfusepath{clip}%
\pgfsetbuttcap%
\pgfsetroundjoin%
\definecolor{currentfill}{rgb}{0.323718,0.433158,0.864722}%
\pgfsetfillcolor{currentfill}%
\pgfsetlinewidth{0.000000pt}%
\definecolor{currentstroke}{rgb}{0.000000,0.000000,0.000000}%
\pgfsetstrokecolor{currentstroke}%
\pgfsetdash{}{0pt}%
\pgfpathmoveto{\pgfqpoint{5.760841in}{3.059724in}}%
\pgfpathlineto{\pgfqpoint{5.770840in}{2.999098in}}%
\pgfpathlineto{\pgfqpoint{5.784044in}{3.137470in}}%
\pgfpathlineto{\pgfqpoint{5.814461in}{2.987522in}}%
\pgfpathlineto{\pgfqpoint{5.850094in}{3.156705in}}%
\pgfpathlineto{\pgfqpoint{5.837153in}{3.043556in}}%
\pgfpathlineto{\pgfqpoint{5.825565in}{3.009544in}}%
\pgfpathlineto{\pgfqpoint{5.794231in}{3.097087in}}%
\pgfpathlineto{\pgfqpoint{5.760841in}{3.059724in}}%
\pgfpathclose%
\pgfusepath{fill}%
\end{pgfscope}%
\begin{pgfscope}%
\pgfpathrectangle{\pgfqpoint{1.020000in}{0.880000in}}{\pgfqpoint{6.160000in}{6.160000in}}%
\pgfusepath{clip}%
\pgfsetbuttcap%
\pgfsetroundjoin%
\definecolor{currentfill}{rgb}{0.869655,0.379274,0.300941}%
\pgfsetfillcolor{currentfill}%
\pgfsetlinewidth{0.000000pt}%
\definecolor{currentstroke}{rgb}{0.000000,0.000000,0.000000}%
\pgfsetstrokecolor{currentstroke}%
\pgfsetdash{}{0pt}%
\pgfpathmoveto{\pgfqpoint{2.577115in}{5.027091in}}%
\pgfpathlineto{\pgfqpoint{2.586826in}{4.893659in}}%
\pgfpathlineto{\pgfqpoint{2.592667in}{5.007104in}}%
\pgfpathlineto{\pgfqpoint{2.628199in}{4.892738in}}%
\pgfpathlineto{\pgfqpoint{2.657458in}{5.190397in}}%
\pgfpathlineto{\pgfqpoint{2.650870in}{5.116226in}}%
\pgfpathlineto{\pgfqpoint{2.644634in}{5.020581in}}%
\pgfpathlineto{\pgfqpoint{2.609772in}{5.095788in}}%
\pgfpathlineto{\pgfqpoint{2.577115in}{5.027091in}}%
\pgfpathclose%
\pgfusepath{fill}%
\end{pgfscope}%
\begin{pgfscope}%
\pgfpathrectangle{\pgfqpoint{1.020000in}{0.880000in}}{\pgfqpoint{6.160000in}{6.160000in}}%
\pgfusepath{clip}%
\pgfsetbuttcap%
\pgfsetroundjoin%
\definecolor{currentfill}{rgb}{0.363461,0.484784,0.901019}%
\pgfsetfillcolor{currentfill}%
\pgfsetlinewidth{0.000000pt}%
\definecolor{currentstroke}{rgb}{0.000000,0.000000,0.000000}%
\pgfsetstrokecolor{currentstroke}%
\pgfsetdash{}{0pt}%
\pgfpathmoveto{\pgfqpoint{4.822507in}{3.250147in}}%
\pgfpathlineto{\pgfqpoint{4.831832in}{3.149852in}}%
\pgfpathlineto{\pgfqpoint{4.841497in}{3.100993in}}%
\pgfpathlineto{\pgfqpoint{4.875123in}{3.174613in}}%
\pgfpathlineto{\pgfqpoint{4.907841in}{3.119158in}}%
\pgfpathlineto{\pgfqpoint{4.898588in}{3.230401in}}%
\pgfpathlineto{\pgfqpoint{4.886643in}{2.960671in}}%
\pgfpathlineto{\pgfqpoint{4.854985in}{3.157718in}}%
\pgfpathlineto{\pgfqpoint{4.822507in}{3.250147in}}%
\pgfpathclose%
\pgfusepath{fill}%
\end{pgfscope}%
\begin{pgfscope}%
\pgfpathrectangle{\pgfqpoint{1.020000in}{0.880000in}}{\pgfqpoint{6.160000in}{6.160000in}}%
\pgfusepath{clip}%
\pgfsetbuttcap%
\pgfsetroundjoin%
\definecolor{currentfill}{rgb}{0.318832,0.426605,0.859857}%
\pgfsetfillcolor{currentfill}%
\pgfsetlinewidth{0.000000pt}%
\definecolor{currentstroke}{rgb}{0.000000,0.000000,0.000000}%
\pgfsetstrokecolor{currentstroke}%
\pgfsetdash{}{0pt}%
\pgfpathmoveto{\pgfqpoint{4.973885in}{3.106443in}}%
\pgfpathlineto{\pgfqpoint{4.983426in}{3.032624in}}%
\pgfpathlineto{\pgfqpoint{4.994373in}{3.130828in}}%
\pgfpathlineto{\pgfqpoint{5.026462in}{3.010096in}}%
\pgfpathlineto{\pgfqpoint{5.058727in}{2.920758in}}%
\pgfpathlineto{\pgfqpoint{5.050148in}{3.107229in}}%
\pgfpathlineto{\pgfqpoint{5.039298in}{3.031456in}}%
\pgfpathlineto{\pgfqpoint{5.006567in}{3.063240in}}%
\pgfpathlineto{\pgfqpoint{4.973885in}{3.106443in}}%
\pgfpathclose%
\pgfusepath{fill}%
\end{pgfscope}%
\begin{pgfscope}%
\pgfpathrectangle{\pgfqpoint{1.020000in}{0.880000in}}{\pgfqpoint{6.160000in}{6.160000in}}%
\pgfusepath{clip}%
\pgfsetbuttcap%
\pgfsetroundjoin%
\definecolor{currentfill}{rgb}{0.908908,0.462433,0.360950}%
\pgfsetfillcolor{currentfill}%
\pgfsetlinewidth{0.000000pt}%
\definecolor{currentstroke}{rgb}{0.000000,0.000000,0.000000}%
\pgfsetstrokecolor{currentstroke}%
\pgfsetdash{}{0pt}%
\pgfpathmoveto{\pgfqpoint{2.446490in}{4.767370in}}%
\pgfpathlineto{\pgfqpoint{2.453256in}{4.808013in}}%
\pgfpathlineto{\pgfqpoint{2.460509in}{4.821319in}}%
\pgfpathlineto{\pgfqpoint{2.493025in}{4.898190in}}%
\pgfpathlineto{\pgfqpoint{2.524189in}{5.059778in}}%
\pgfpathlineto{\pgfqpoint{2.515773in}{5.111770in}}%
\pgfpathlineto{\pgfqpoint{2.509534in}{5.032048in}}%
\pgfpathlineto{\pgfqpoint{2.478934in}{4.842832in}}%
\pgfpathlineto{\pgfqpoint{2.446490in}{4.767370in}}%
\pgfpathclose%
\pgfusepath{fill}%
\end{pgfscope}%
\begin{pgfscope}%
\pgfpathrectangle{\pgfqpoint{1.020000in}{0.880000in}}{\pgfqpoint{6.160000in}{6.160000in}}%
\pgfusepath{clip}%
\pgfsetbuttcap%
\pgfsetroundjoin%
\definecolor{currentfill}{rgb}{0.677823,0.786546,0.991005}%
\pgfsetfillcolor{currentfill}%
\pgfsetlinewidth{0.000000pt}%
\definecolor{currentstroke}{rgb}{0.000000,0.000000,0.000000}%
\pgfsetstrokecolor{currentstroke}%
\pgfsetdash{}{0pt}%
\pgfpathmoveto{\pgfqpoint{3.911953in}{3.778898in}}%
\pgfpathlineto{\pgfqpoint{3.921044in}{3.747023in}}%
\pgfpathlineto{\pgfqpoint{3.930394in}{3.617540in}}%
\pgfpathlineto{\pgfqpoint{3.963501in}{3.737646in}}%
\pgfpathlineto{\pgfqpoint{3.997147in}{3.600036in}}%
\pgfpathlineto{\pgfqpoint{3.987755in}{3.762458in}}%
\pgfpathlineto{\pgfqpoint{3.978693in}{3.758230in}}%
\pgfpathlineto{\pgfqpoint{3.945535in}{3.684393in}}%
\pgfpathlineto{\pgfqpoint{3.911953in}{3.778898in}}%
\pgfpathclose%
\pgfusepath{fill}%
\end{pgfscope}%
\begin{pgfscope}%
\pgfpathrectangle{\pgfqpoint{1.020000in}{0.880000in}}{\pgfqpoint{6.160000in}{6.160000in}}%
\pgfusepath{clip}%
\pgfsetbuttcap%
\pgfsetroundjoin%
\definecolor{currentfill}{rgb}{0.895882,0.849906,0.823499}%
\pgfsetfillcolor{currentfill}%
\pgfsetlinewidth{0.000000pt}%
\definecolor{currentstroke}{rgb}{0.000000,0.000000,0.000000}%
\pgfsetstrokecolor{currentstroke}%
\pgfsetdash{}{0pt}%
\pgfpathmoveto{\pgfqpoint{3.656960in}{4.151954in}}%
\pgfpathlineto{\pgfqpoint{3.665113in}{4.264756in}}%
\pgfpathlineto{\pgfqpoint{3.673860in}{4.267613in}}%
\pgfpathlineto{\pgfqpoint{3.707640in}{4.210100in}}%
\pgfpathlineto{\pgfqpoint{3.742014in}{4.003790in}}%
\pgfpathlineto{\pgfqpoint{3.733292in}{3.985676in}}%
\pgfpathlineto{\pgfqpoint{3.723417in}{4.224999in}}%
\pgfpathlineto{\pgfqpoint{3.689827in}{4.261801in}}%
\pgfpathlineto{\pgfqpoint{3.656960in}{4.151954in}}%
\pgfpathclose%
\pgfusepath{fill}%
\end{pgfscope}%
\begin{pgfscope}%
\pgfpathrectangle{\pgfqpoint{1.020000in}{0.880000in}}{\pgfqpoint{6.160000in}{6.160000in}}%
\pgfusepath{clip}%
\pgfsetbuttcap%
\pgfsetroundjoin%
\definecolor{currentfill}{rgb}{0.835027,0.313644,0.259783}%
\pgfsetfillcolor{currentfill}%
\pgfsetlinewidth{0.000000pt}%
\definecolor{currentstroke}{rgb}{0.000000,0.000000,0.000000}%
\pgfsetstrokecolor{currentstroke}%
\pgfsetdash{}{0pt}%
\pgfpathmoveto{\pgfqpoint{2.644634in}{5.020581in}}%
\pgfpathlineto{\pgfqpoint{2.650870in}{5.116226in}}%
\pgfpathlineto{\pgfqpoint{2.657458in}{5.190397in}}%
\pgfpathlineto{\pgfqpoint{2.692392in}{5.113068in}}%
\pgfpathlineto{\pgfqpoint{2.728050in}{4.980337in}}%
\pgfpathlineto{\pgfqpoint{2.716116in}{5.270351in}}%
\pgfpathlineto{\pgfqpoint{2.711043in}{5.084681in}}%
\pgfpathlineto{\pgfqpoint{2.677497in}{5.075656in}}%
\pgfpathlineto{\pgfqpoint{2.644634in}{5.020581in}}%
\pgfpathclose%
\pgfusepath{fill}%
\end{pgfscope}%
\begin{pgfscope}%
\pgfpathrectangle{\pgfqpoint{1.020000in}{0.880000in}}{\pgfqpoint{6.160000in}{6.160000in}}%
\pgfusepath{clip}%
\pgfsetbuttcap%
\pgfsetroundjoin%
\definecolor{currentfill}{rgb}{0.363461,0.484784,0.901019}%
\pgfsetfillcolor{currentfill}%
\pgfsetlinewidth{0.000000pt}%
\definecolor{currentstroke}{rgb}{0.000000,0.000000,0.000000}%
\pgfsetstrokecolor{currentstroke}%
\pgfsetdash{}{0pt}%
\pgfpathmoveto{\pgfqpoint{5.825565in}{3.009544in}}%
\pgfpathlineto{\pgfqpoint{5.837153in}{3.043556in}}%
\pgfpathlineto{\pgfqpoint{5.850094in}{3.156705in}}%
\pgfpathlineto{\pgfqpoint{5.882744in}{3.145449in}}%
\pgfpathlineto{\pgfqpoint{5.914215in}{3.067925in}}%
\pgfpathlineto{\pgfqpoint{5.908189in}{3.362811in}}%
\pgfpathlineto{\pgfqpoint{5.893261in}{3.141080in}}%
\pgfpathlineto{\pgfqpoint{5.861223in}{3.183631in}}%
\pgfpathlineto{\pgfqpoint{5.825565in}{3.009544in}}%
\pgfpathclose%
\pgfusepath{fill}%
\end{pgfscope}%
\begin{pgfscope}%
\pgfpathrectangle{\pgfqpoint{1.020000in}{0.880000in}}{\pgfqpoint{6.160000in}{6.160000in}}%
\pgfusepath{clip}%
\pgfsetbuttcap%
\pgfsetroundjoin%
\definecolor{currentfill}{rgb}{0.800830,0.250829,0.225696}%
\pgfsetfillcolor{currentfill}%
\pgfsetlinewidth{0.000000pt}%
\definecolor{currentstroke}{rgb}{0.000000,0.000000,0.000000}%
\pgfsetstrokecolor{currentstroke}%
\pgfsetdash{}{0pt}%
\pgfpathmoveto{\pgfqpoint{2.711043in}{5.084681in}}%
\pgfpathlineto{\pgfqpoint{2.716116in}{5.270351in}}%
\pgfpathlineto{\pgfqpoint{2.728050in}{4.980337in}}%
\pgfpathlineto{\pgfqpoint{2.760630in}{5.059759in}}%
\pgfpathlineto{\pgfqpoint{2.792653in}{5.182377in}}%
\pgfpathlineto{\pgfqpoint{2.784459in}{5.208388in}}%
\pgfpathlineto{\pgfqpoint{2.774971in}{5.328858in}}%
\pgfpathlineto{\pgfqpoint{2.742453in}{5.243779in}}%
\pgfpathlineto{\pgfqpoint{2.711043in}{5.084681in}}%
\pgfpathclose%
\pgfusepath{fill}%
\end{pgfscope}%
\begin{pgfscope}%
\pgfpathrectangle{\pgfqpoint{1.020000in}{0.880000in}}{\pgfqpoint{6.160000in}{6.160000in}}%
\pgfusepath{clip}%
\pgfsetbuttcap%
\pgfsetroundjoin%
\definecolor{currentfill}{rgb}{0.323718,0.433158,0.864722}%
\pgfsetfillcolor{currentfill}%
\pgfsetlinewidth{0.000000pt}%
\definecolor{currentstroke}{rgb}{0.000000,0.000000,0.000000}%
\pgfsetstrokecolor{currentstroke}%
\pgfsetdash{}{0pt}%
\pgfpathmoveto{\pgfqpoint{5.478745in}{3.232380in}}%
\pgfpathlineto{\pgfqpoint{5.488145in}{3.134288in}}%
\pgfpathlineto{\pgfqpoint{5.497439in}{3.027976in}}%
\pgfpathlineto{\pgfqpoint{5.528557in}{2.893993in}}%
\pgfpathlineto{\pgfqpoint{5.563798in}{3.065314in}}%
\pgfpathlineto{\pgfqpoint{5.553758in}{3.117146in}}%
\pgfpathlineto{\pgfqpoint{5.541675in}{3.018322in}}%
\pgfpathlineto{\pgfqpoint{5.508749in}{3.012208in}}%
\pgfpathlineto{\pgfqpoint{5.478745in}{3.232380in}}%
\pgfpathclose%
\pgfusepath{fill}%
\end{pgfscope}%
\begin{pgfscope}%
\pgfpathrectangle{\pgfqpoint{1.020000in}{0.880000in}}{\pgfqpoint{6.160000in}{6.160000in}}%
\pgfusepath{clip}%
\pgfsetbuttcap%
\pgfsetroundjoin%
\definecolor{currentfill}{rgb}{0.941728,0.546413,0.429707}%
\pgfsetfillcolor{currentfill}%
\pgfsetlinewidth{0.000000pt}%
\definecolor{currentstroke}{rgb}{0.000000,0.000000,0.000000}%
\pgfsetstrokecolor{currentstroke}%
\pgfsetdash{}{0pt}%
\pgfpathmoveto{\pgfqpoint{3.314742in}{4.987284in}}%
\pgfpathlineto{\pgfqpoint{3.324253in}{4.855961in}}%
\pgfpathlineto{\pgfqpoint{3.332859in}{4.832093in}}%
\pgfpathlineto{\pgfqpoint{3.366922in}{4.773546in}}%
\pgfpathlineto{\pgfqpoint{3.401400in}{4.653689in}}%
\pgfpathlineto{\pgfqpoint{3.392140in}{4.756879in}}%
\pgfpathlineto{\pgfqpoint{3.384685in}{4.631445in}}%
\pgfpathlineto{\pgfqpoint{3.348726in}{4.937894in}}%
\pgfpathlineto{\pgfqpoint{3.314742in}{4.987284in}}%
\pgfpathclose%
\pgfusepath{fill}%
\end{pgfscope}%
\begin{pgfscope}%
\pgfpathrectangle{\pgfqpoint{1.020000in}{0.880000in}}{\pgfqpoint{6.160000in}{6.160000in}}%
\pgfusepath{clip}%
\pgfsetbuttcap%
\pgfsetroundjoin%
\definecolor{currentfill}{rgb}{0.968203,0.720844,0.612293}%
\pgfsetfillcolor{currentfill}%
\pgfsetlinewidth{0.000000pt}%
\definecolor{currentstroke}{rgb}{0.000000,0.000000,0.000000}%
\pgfsetstrokecolor{currentstroke}%
\pgfsetdash{}{0pt}%
\pgfpathmoveto{\pgfqpoint{3.485597in}{4.633442in}}%
\pgfpathlineto{\pgfqpoint{3.494858in}{4.532104in}}%
\pgfpathlineto{\pgfqpoint{3.503770in}{4.482769in}}%
\pgfpathlineto{\pgfqpoint{3.537200in}{4.498096in}}%
\pgfpathlineto{\pgfqpoint{3.571786in}{4.321181in}}%
\pgfpathlineto{\pgfqpoint{3.562206in}{4.476159in}}%
\pgfpathlineto{\pgfqpoint{3.552609in}{4.632390in}}%
\pgfpathlineto{\pgfqpoint{3.519970in}{4.502381in}}%
\pgfpathlineto{\pgfqpoint{3.485597in}{4.633442in}}%
\pgfpathclose%
\pgfusepath{fill}%
\end{pgfscope}%
\begin{pgfscope}%
\pgfpathrectangle{\pgfqpoint{1.020000in}{0.880000in}}{\pgfqpoint{6.160000in}{6.160000in}}%
\pgfusepath{clip}%
\pgfsetbuttcap%
\pgfsetroundjoin%
\definecolor{currentfill}{rgb}{0.752704,0.157576,0.184258}%
\pgfsetfillcolor{currentfill}%
\pgfsetlinewidth{0.000000pt}%
\definecolor{currentstroke}{rgb}{0.000000,0.000000,0.000000}%
\pgfsetstrokecolor{currentstroke}%
\pgfsetdash{}{0pt}%
\pgfpathmoveto{\pgfqpoint{2.774971in}{5.328858in}}%
\pgfpathlineto{\pgfqpoint{2.784459in}{5.208388in}}%
\pgfpathlineto{\pgfqpoint{2.792653in}{5.182377in}}%
\pgfpathlineto{\pgfqpoint{2.825362in}{5.258181in}}%
\pgfpathlineto{\pgfqpoint{2.859085in}{5.258359in}}%
\pgfpathlineto{\pgfqpoint{2.850429in}{5.315443in}}%
\pgfpathlineto{\pgfqpoint{2.842361in}{5.328530in}}%
\pgfpathlineto{\pgfqpoint{2.811291in}{5.134777in}}%
\pgfpathlineto{\pgfqpoint{2.774971in}{5.328858in}}%
\pgfpathclose%
\pgfusepath{fill}%
\end{pgfscope}%
\begin{pgfscope}%
\pgfpathrectangle{\pgfqpoint{1.020000in}{0.880000in}}{\pgfqpoint{6.160000in}{6.160000in}}%
\pgfusepath{clip}%
\pgfsetbuttcap%
\pgfsetroundjoin%
\definecolor{currentfill}{rgb}{0.516260,0.654498,0.986407}%
\pgfsetfillcolor{currentfill}%
\pgfsetlinewidth{0.000000pt}%
\definecolor{currentstroke}{rgb}{0.000000,0.000000,0.000000}%
\pgfsetstrokecolor{currentstroke}%
\pgfsetdash{}{0pt}%
\pgfpathmoveto{\pgfqpoint{4.300526in}{3.422097in}}%
\pgfpathlineto{\pgfqpoint{4.309832in}{3.334127in}}%
\pgfpathlineto{\pgfqpoint{4.319610in}{3.590111in}}%
\pgfpathlineto{\pgfqpoint{4.352713in}{3.439140in}}%
\pgfpathlineto{\pgfqpoint{4.385968in}{3.422099in}}%
\pgfpathlineto{\pgfqpoint{4.376378in}{3.376814in}}%
\pgfpathlineto{\pgfqpoint{4.366848in}{3.348079in}}%
\pgfpathlineto{\pgfqpoint{4.333671in}{3.359763in}}%
\pgfpathlineto{\pgfqpoint{4.300526in}{3.422097in}}%
\pgfpathclose%
\pgfusepath{fill}%
\end{pgfscope}%
\begin{pgfscope}%
\pgfpathrectangle{\pgfqpoint{1.020000in}{0.880000in}}{\pgfqpoint{6.160000in}{6.160000in}}%
\pgfusepath{clip}%
\pgfsetbuttcap%
\pgfsetroundjoin%
\definecolor{currentfill}{rgb}{0.462354,0.599830,0.965857}%
\pgfsetfillcolor{currentfill}%
\pgfsetlinewidth{0.000000pt}%
\definecolor{currentstroke}{rgb}{0.000000,0.000000,0.000000}%
\pgfsetstrokecolor{currentstroke}%
\pgfsetdash{}{0pt}%
\pgfpathmoveto{\pgfqpoint{4.452357in}{3.377500in}}%
\pgfpathlineto{\pgfqpoint{4.461682in}{3.285131in}}%
\pgfpathlineto{\pgfqpoint{4.471624in}{3.405616in}}%
\pgfpathlineto{\pgfqpoint{4.504699in}{3.340636in}}%
\pgfpathlineto{\pgfqpoint{4.537828in}{3.308506in}}%
\pgfpathlineto{\pgfqpoint{4.528177in}{3.311632in}}%
\pgfpathlineto{\pgfqpoint{4.518074in}{3.174078in}}%
\pgfpathlineto{\pgfqpoint{4.485470in}{3.345491in}}%
\pgfpathlineto{\pgfqpoint{4.452357in}{3.377500in}}%
\pgfpathclose%
\pgfusepath{fill}%
\end{pgfscope}%
\begin{pgfscope}%
\pgfpathrectangle{\pgfqpoint{1.020000in}{0.880000in}}{\pgfqpoint{6.160000in}{6.160000in}}%
\pgfusepath{clip}%
\pgfsetbuttcap%
\pgfsetroundjoin%
\definecolor{currentfill}{rgb}{0.735077,0.104460,0.171492}%
\pgfsetfillcolor{currentfill}%
\pgfsetlinewidth{0.000000pt}%
\definecolor{currentstroke}{rgb}{0.000000,0.000000,0.000000}%
\pgfsetstrokecolor{currentstroke}%
\pgfsetdash{}{0pt}%
\pgfpathmoveto{\pgfqpoint{3.061034in}{5.257577in}}%
\pgfpathlineto{\pgfqpoint{3.068365in}{5.329803in}}%
\pgfpathlineto{\pgfqpoint{3.076989in}{5.285604in}}%
\pgfpathlineto{\pgfqpoint{3.111448in}{5.210017in}}%
\pgfpathlineto{\pgfqpoint{3.144507in}{5.264821in}}%
\pgfpathlineto{\pgfqpoint{3.136533in}{5.242434in}}%
\pgfpathlineto{\pgfqpoint{3.126698in}{5.401676in}}%
\pgfpathlineto{\pgfqpoint{3.094524in}{5.266775in}}%
\pgfpathlineto{\pgfqpoint{3.061034in}{5.257577in}}%
\pgfpathclose%
\pgfusepath{fill}%
\end{pgfscope}%
\begin{pgfscope}%
\pgfpathrectangle{\pgfqpoint{1.020000in}{0.880000in}}{\pgfqpoint{6.160000in}{6.160000in}}%
\pgfusepath{clip}%
\pgfsetbuttcap%
\pgfsetroundjoin%
\definecolor{currentfill}{rgb}{0.711554,0.033337,0.154485}%
\pgfsetfillcolor{currentfill}%
\pgfsetlinewidth{0.000000pt}%
\definecolor{currentstroke}{rgb}{0.000000,0.000000,0.000000}%
\pgfsetstrokecolor{currentstroke}%
\pgfsetdash{}{0pt}%
\pgfpathmoveto{\pgfqpoint{2.842361in}{5.328530in}}%
\pgfpathlineto{\pgfqpoint{2.850429in}{5.315443in}}%
\pgfpathlineto{\pgfqpoint{2.859085in}{5.258359in}}%
\pgfpathlineto{\pgfqpoint{2.891319in}{5.374431in}}%
\pgfpathlineto{\pgfqpoint{2.926020in}{5.295375in}}%
\pgfpathlineto{\pgfqpoint{2.919245in}{5.198615in}}%
\pgfpathlineto{\pgfqpoint{2.908584in}{5.415276in}}%
\pgfpathlineto{\pgfqpoint{2.875483in}{5.370717in}}%
\pgfpathlineto{\pgfqpoint{2.842361in}{5.328530in}}%
\pgfpathclose%
\pgfusepath{fill}%
\end{pgfscope}%
\begin{pgfscope}%
\pgfpathrectangle{\pgfqpoint{1.020000in}{0.880000in}}{\pgfqpoint{6.160000in}{6.160000in}}%
\pgfusepath{clip}%
\pgfsetbuttcap%
\pgfsetroundjoin%
\definecolor{currentfill}{rgb}{0.815508,0.277781,0.240294}%
\pgfsetfillcolor{currentfill}%
\pgfsetlinewidth{0.000000pt}%
\definecolor{currentstroke}{rgb}{0.000000,0.000000,0.000000}%
\pgfsetstrokecolor{currentstroke}%
\pgfsetdash{}{0pt}%
\pgfpathmoveto{\pgfqpoint{3.144507in}{5.264821in}}%
\pgfpathlineto{\pgfqpoint{3.154227in}{5.116738in}}%
\pgfpathlineto{\pgfqpoint{3.162452in}{5.116909in}}%
\pgfpathlineto{\pgfqpoint{3.195508in}{5.176320in}}%
\pgfpathlineto{\pgfqpoint{3.229803in}{5.105822in}}%
\pgfpathlineto{\pgfqpoint{3.221757in}{5.079941in}}%
\pgfpathlineto{\pgfqpoint{3.213900in}{5.036495in}}%
\pgfpathlineto{\pgfqpoint{3.178975in}{5.178657in}}%
\pgfpathlineto{\pgfqpoint{3.144507in}{5.264821in}}%
\pgfpathclose%
\pgfusepath{fill}%
\end{pgfscope}%
\begin{pgfscope}%
\pgfpathrectangle{\pgfqpoint{1.020000in}{0.880000in}}{\pgfqpoint{6.160000in}{6.160000in}}%
\pgfusepath{clip}%
\pgfsetbuttcap%
\pgfsetroundjoin%
\definecolor{currentfill}{rgb}{0.323718,0.433158,0.864722}%
\pgfsetfillcolor{currentfill}%
\pgfsetlinewidth{0.000000pt}%
\definecolor{currentstroke}{rgb}{0.000000,0.000000,0.000000}%
\pgfsetstrokecolor{currentstroke}%
\pgfsetdash{}{0pt}%
\pgfpathmoveto{\pgfqpoint{5.695219in}{3.060134in}}%
\pgfpathlineto{\pgfqpoint{5.706286in}{3.070980in}}%
\pgfpathlineto{\pgfqpoint{5.716990in}{3.056501in}}%
\pgfpathlineto{\pgfqpoint{5.750562in}{3.100069in}}%
\pgfpathlineto{\pgfqpoint{5.784044in}{3.137470in}}%
\pgfpathlineto{\pgfqpoint{5.770840in}{2.999098in}}%
\pgfpathlineto{\pgfqpoint{5.760841in}{3.059724in}}%
\pgfpathlineto{\pgfqpoint{5.726872in}{2.984629in}}%
\pgfpathlineto{\pgfqpoint{5.695219in}{3.060134in}}%
\pgfpathclose%
\pgfusepath{fill}%
\end{pgfscope}%
\begin{pgfscope}%
\pgfpathrectangle{\pgfqpoint{1.020000in}{0.880000in}}{\pgfqpoint{6.160000in}{6.160000in}}%
\pgfusepath{clip}%
\pgfsetbuttcap%
\pgfsetroundjoin%
\definecolor{currentfill}{rgb}{0.865391,0.371128,0.295769}%
\pgfsetfillcolor{currentfill}%
\pgfsetlinewidth{0.000000pt}%
\definecolor{currentstroke}{rgb}{0.000000,0.000000,0.000000}%
\pgfsetstrokecolor{currentstroke}%
\pgfsetdash{}{0pt}%
\pgfpathmoveto{\pgfqpoint{2.509534in}{5.032048in}}%
\pgfpathlineto{\pgfqpoint{2.515773in}{5.111770in}}%
\pgfpathlineto{\pgfqpoint{2.524189in}{5.059778in}}%
\pgfpathlineto{\pgfqpoint{2.557144in}{5.114909in}}%
\pgfpathlineto{\pgfqpoint{2.592667in}{5.007104in}}%
\pgfpathlineto{\pgfqpoint{2.586826in}{4.893659in}}%
\pgfpathlineto{\pgfqpoint{2.577115in}{5.027091in}}%
\pgfpathlineto{\pgfqpoint{2.543592in}{5.013677in}}%
\pgfpathlineto{\pgfqpoint{2.509534in}{5.032048in}}%
\pgfpathclose%
\pgfusepath{fill}%
\end{pgfscope}%
\begin{pgfscope}%
\pgfpathrectangle{\pgfqpoint{1.020000in}{0.880000in}}{\pgfqpoint{6.160000in}{6.160000in}}%
\pgfusepath{clip}%
\pgfsetbuttcap%
\pgfsetroundjoin%
\definecolor{currentfill}{rgb}{0.964911,0.640159,0.519806}%
\pgfsetfillcolor{currentfill}%
\pgfsetlinewidth{0.000000pt}%
\definecolor{currentstroke}{rgb}{0.000000,0.000000,0.000000}%
\pgfsetstrokecolor{currentstroke}%
\pgfsetdash{}{0pt}%
\pgfpathmoveto{\pgfqpoint{2.328163in}{4.655677in}}%
\pgfpathlineto{\pgfqpoint{2.336378in}{4.608294in}}%
\pgfpathlineto{\pgfqpoint{2.342101in}{4.698492in}}%
\pgfpathlineto{\pgfqpoint{2.378151in}{4.578050in}}%
\pgfpathlineto{\pgfqpoint{2.411799in}{4.589046in}}%
\pgfpathlineto{\pgfqpoint{2.400177in}{4.827564in}}%
\pgfpathlineto{\pgfqpoint{2.397287in}{4.570495in}}%
\pgfpathlineto{\pgfqpoint{2.362015in}{4.653894in}}%
\pgfpathlineto{\pgfqpoint{2.328163in}{4.655677in}}%
\pgfpathclose%
\pgfusepath{fill}%
\end{pgfscope}%
\begin{pgfscope}%
\pgfpathrectangle{\pgfqpoint{1.020000in}{0.880000in}}{\pgfqpoint{6.160000in}{6.160000in}}%
\pgfusepath{clip}%
\pgfsetbuttcap%
\pgfsetroundjoin%
\definecolor{currentfill}{rgb}{0.831148,0.859513,0.903110}%
\pgfsetfillcolor{currentfill}%
\pgfsetlinewidth{0.000000pt}%
\definecolor{currentstroke}{rgb}{0.000000,0.000000,0.000000}%
\pgfsetstrokecolor{currentstroke}%
\pgfsetdash{}{0pt}%
\pgfpathmoveto{\pgfqpoint{3.673860in}{4.267613in}}%
\pgfpathlineto{\pgfqpoint{3.684336in}{3.925898in}}%
\pgfpathlineto{\pgfqpoint{3.692291in}{4.092616in}}%
\pgfpathlineto{\pgfqpoint{3.727016in}{3.826548in}}%
\pgfpathlineto{\pgfqpoint{3.759857in}{3.960979in}}%
\pgfpathlineto{\pgfqpoint{3.751528in}{3.842825in}}%
\pgfpathlineto{\pgfqpoint{3.742014in}{4.003790in}}%
\pgfpathlineto{\pgfqpoint{3.707640in}{4.210100in}}%
\pgfpathlineto{\pgfqpoint{3.673860in}{4.267613in}}%
\pgfpathclose%
\pgfusepath{fill}%
\end{pgfscope}%
\begin{pgfscope}%
\pgfpathrectangle{\pgfqpoint{1.020000in}{0.880000in}}{\pgfqpoint{6.160000in}{6.160000in}}%
\pgfusepath{clip}%
\pgfsetbuttcap%
\pgfsetroundjoin%
\definecolor{currentfill}{rgb}{0.581486,0.713451,0.998314}%
\pgfsetfillcolor{currentfill}%
\pgfsetlinewidth{0.000000pt}%
\definecolor{currentstroke}{rgb}{0.000000,0.000000,0.000000}%
\pgfsetstrokecolor{currentstroke}%
\pgfsetdash{}{0pt}%
\pgfpathmoveto{\pgfqpoint{4.148842in}{3.475495in}}%
\pgfpathlineto{\pgfqpoint{4.158027in}{3.742992in}}%
\pgfpathlineto{\pgfqpoint{4.167365in}{3.489733in}}%
\pgfpathlineto{\pgfqpoint{4.200693in}{3.421351in}}%
\pgfpathlineto{\pgfqpoint{4.234053in}{3.546397in}}%
\pgfpathlineto{\pgfqpoint{4.224653in}{3.411128in}}%
\pgfpathlineto{\pgfqpoint{4.215410in}{3.622911in}}%
\pgfpathlineto{\pgfqpoint{4.182112in}{3.494525in}}%
\pgfpathlineto{\pgfqpoint{4.148842in}{3.475495in}}%
\pgfpathclose%
\pgfusepath{fill}%
\end{pgfscope}%
\begin{pgfscope}%
\pgfpathrectangle{\pgfqpoint{1.020000in}{0.880000in}}{\pgfqpoint{6.160000in}{6.160000in}}%
\pgfusepath{clip}%
\pgfsetbuttcap%
\pgfsetroundjoin%
\definecolor{currentfill}{rgb}{0.635474,0.756714,0.998297}%
\pgfsetfillcolor{currentfill}%
\pgfsetlinewidth{0.000000pt}%
\definecolor{currentstroke}{rgb}{0.000000,0.000000,0.000000}%
\pgfsetstrokecolor{currentstroke}%
\pgfsetdash{}{0pt}%
\pgfpathmoveto{\pgfqpoint{3.997147in}{3.600036in}}%
\pgfpathlineto{\pgfqpoint{4.006218in}{3.613777in}}%
\pgfpathlineto{\pgfqpoint{4.015309in}{3.628660in}}%
\pgfpathlineto{\pgfqpoint{4.048889in}{3.494325in}}%
\pgfpathlineto{\pgfqpoint{4.082143in}{3.574196in}}%
\pgfpathlineto{\pgfqpoint{4.072916in}{3.619923in}}%
\pgfpathlineto{\pgfqpoint{4.063603in}{3.746933in}}%
\pgfpathlineto{\pgfqpoint{4.030322in}{3.698843in}}%
\pgfpathlineto{\pgfqpoint{3.997147in}{3.600036in}}%
\pgfpathclose%
\pgfusepath{fill}%
\end{pgfscope}%
\begin{pgfscope}%
\pgfpathrectangle{\pgfqpoint{1.020000in}{0.880000in}}{\pgfqpoint{6.160000in}{6.160000in}}%
\pgfusepath{clip}%
\pgfsetbuttcap%
\pgfsetroundjoin%
\definecolor{currentfill}{rgb}{0.968500,0.673977,0.556649}%
\pgfsetfillcolor{currentfill}%
\pgfsetlinewidth{0.000000pt}%
\definecolor{currentstroke}{rgb}{0.000000,0.000000,0.000000}%
\pgfsetstrokecolor{currentstroke}%
\pgfsetdash{}{0pt}%
\pgfpathmoveto{\pgfqpoint{2.259987in}{4.680774in}}%
\pgfpathlineto{\pgfqpoint{2.268905in}{4.594527in}}%
\pgfpathlineto{\pgfqpoint{2.275338in}{4.640237in}}%
\pgfpathlineto{\pgfqpoint{2.313583in}{4.406657in}}%
\pgfpathlineto{\pgfqpoint{2.342101in}{4.698492in}}%
\pgfpathlineto{\pgfqpoint{2.336378in}{4.608294in}}%
\pgfpathlineto{\pgfqpoint{2.328163in}{4.655677in}}%
\pgfpathlineto{\pgfqpoint{2.297599in}{4.480129in}}%
\pgfpathlineto{\pgfqpoint{2.259987in}{4.680774in}}%
\pgfpathclose%
\pgfusepath{fill}%
\end{pgfscope}%
\begin{pgfscope}%
\pgfpathrectangle{\pgfqpoint{1.020000in}{0.880000in}}{\pgfqpoint{6.160000in}{6.160000in}}%
\pgfusepath{clip}%
\pgfsetbuttcap%
\pgfsetroundjoin%
\definecolor{currentfill}{rgb}{0.425199,0.559058,0.946061}%
\pgfsetfillcolor{currentfill}%
\pgfsetlinewidth{0.000000pt}%
\definecolor{currentstroke}{rgb}{0.000000,0.000000,0.000000}%
\pgfsetstrokecolor{currentstroke}%
\pgfsetdash{}{0pt}%
\pgfpathmoveto{\pgfqpoint{4.604078in}{3.271552in}}%
\pgfpathlineto{\pgfqpoint{4.613746in}{3.254005in}}%
\pgfpathlineto{\pgfqpoint{4.623176in}{3.178773in}}%
\pgfpathlineto{\pgfqpoint{4.656380in}{3.180683in}}%
\pgfpathlineto{\pgfqpoint{4.690237in}{3.314830in}}%
\pgfpathlineto{\pgfqpoint{4.680086in}{3.251043in}}%
\pgfpathlineto{\pgfqpoint{4.670719in}{3.340999in}}%
\pgfpathlineto{\pgfqpoint{4.637051in}{3.230365in}}%
\pgfpathlineto{\pgfqpoint{4.604078in}{3.271552in}}%
\pgfpathclose%
\pgfusepath{fill}%
\end{pgfscope}%
\begin{pgfscope}%
\pgfpathrectangle{\pgfqpoint{1.020000in}{0.880000in}}{\pgfqpoint{6.160000in}{6.160000in}}%
\pgfusepath{clip}%
\pgfsetbuttcap%
\pgfsetroundjoin%
\definecolor{currentfill}{rgb}{0.959385,0.610306,0.489382}%
\pgfsetfillcolor{currentfill}%
\pgfsetlinewidth{0.000000pt}%
\definecolor{currentstroke}{rgb}{0.000000,0.000000,0.000000}%
\pgfsetstrokecolor{currentstroke}%
\pgfsetdash{}{0pt}%
\pgfpathmoveto{\pgfqpoint{3.401400in}{4.653689in}}%
\pgfpathlineto{\pgfqpoint{3.408524in}{4.830100in}}%
\pgfpathlineto{\pgfqpoint{3.419478in}{4.505796in}}%
\pgfpathlineto{\pgfqpoint{3.451578in}{4.700908in}}%
\pgfpathlineto{\pgfqpoint{3.485597in}{4.633442in}}%
\pgfpathlineto{\pgfqpoint{3.476322in}{4.736957in}}%
\pgfpathlineto{\pgfqpoint{3.468760in}{4.597977in}}%
\pgfpathlineto{\pgfqpoint{3.432577in}{4.966920in}}%
\pgfpathlineto{\pgfqpoint{3.401400in}{4.653689in}}%
\pgfpathclose%
\pgfusepath{fill}%
\end{pgfscope}%
\begin{pgfscope}%
\pgfpathrectangle{\pgfqpoint{1.020000in}{0.880000in}}{\pgfqpoint{6.160000in}{6.160000in}}%
\pgfusepath{clip}%
\pgfsetbuttcap%
\pgfsetroundjoin%
\definecolor{currentfill}{rgb}{0.740957,0.122240,0.175744}%
\pgfsetfillcolor{currentfill}%
\pgfsetlinewidth{0.000000pt}%
\definecolor{currentstroke}{rgb}{0.000000,0.000000,0.000000}%
\pgfsetstrokecolor{currentstroke}%
\pgfsetdash{}{0pt}%
\pgfpathmoveto{\pgfqpoint{2.993271in}{5.302766in}}%
\pgfpathlineto{\pgfqpoint{3.002262in}{5.222533in}}%
\pgfpathlineto{\pgfqpoint{3.009776in}{5.270802in}}%
\pgfpathlineto{\pgfqpoint{3.044677in}{5.163244in}}%
\pgfpathlineto{\pgfqpoint{3.076989in}{5.285604in}}%
\pgfpathlineto{\pgfqpoint{3.068365in}{5.329803in}}%
\pgfpathlineto{\pgfqpoint{3.061034in}{5.257577in}}%
\pgfpathlineto{\pgfqpoint{3.026468in}{5.342014in}}%
\pgfpathlineto{\pgfqpoint{2.993271in}{5.302766in}}%
\pgfpathclose%
\pgfusepath{fill}%
\end{pgfscope}%
\begin{pgfscope}%
\pgfpathrectangle{\pgfqpoint{1.020000in}{0.880000in}}{\pgfqpoint{6.160000in}{6.160000in}}%
\pgfusepath{clip}%
\pgfsetbuttcap%
\pgfsetroundjoin%
\definecolor{currentfill}{rgb}{0.703587,0.802586,0.982847}%
\pgfsetfillcolor{currentfill}%
\pgfsetlinewidth{0.000000pt}%
\definecolor{currentstroke}{rgb}{0.000000,0.000000,0.000000}%
\pgfsetstrokecolor{currentstroke}%
\pgfsetdash{}{0pt}%
\pgfpathmoveto{\pgfqpoint{3.845619in}{3.647530in}}%
\pgfpathlineto{\pgfqpoint{3.853801in}{3.875800in}}%
\pgfpathlineto{\pgfqpoint{3.862635in}{3.918552in}}%
\pgfpathlineto{\pgfqpoint{3.897033in}{3.607702in}}%
\pgfpathlineto{\pgfqpoint{3.930394in}{3.617540in}}%
\pgfpathlineto{\pgfqpoint{3.921044in}{3.747023in}}%
\pgfpathlineto{\pgfqpoint{3.911953in}{3.778898in}}%
\pgfpathlineto{\pgfqpoint{3.878563in}{3.782570in}}%
\pgfpathlineto{\pgfqpoint{3.845619in}{3.647530in}}%
\pgfpathclose%
\pgfusepath{fill}%
\end{pgfscope}%
\begin{pgfscope}%
\pgfpathrectangle{\pgfqpoint{1.020000in}{0.880000in}}{\pgfqpoint{6.160000in}{6.160000in}}%
\pgfusepath{clip}%
\pgfsetbuttcap%
\pgfsetroundjoin%
\definecolor{currentfill}{rgb}{0.861054,0.362916,0.290628}%
\pgfsetfillcolor{currentfill}%
\pgfsetlinewidth{0.000000pt}%
\definecolor{currentstroke}{rgb}{0.000000,0.000000,0.000000}%
\pgfsetstrokecolor{currentstroke}%
\pgfsetdash{}{0pt}%
\pgfpathmoveto{\pgfqpoint{3.229803in}{5.105822in}}%
\pgfpathlineto{\pgfqpoint{3.239448in}{4.961983in}}%
\pgfpathlineto{\pgfqpoint{3.244636in}{5.301262in}}%
\pgfpathlineto{\pgfqpoint{3.280855in}{5.020009in}}%
\pgfpathlineto{\pgfqpoint{3.314742in}{4.987284in}}%
\pgfpathlineto{\pgfqpoint{3.306109in}{5.016540in}}%
\pgfpathlineto{\pgfqpoint{3.298081in}{4.977698in}}%
\pgfpathlineto{\pgfqpoint{3.265348in}{4.889766in}}%
\pgfpathlineto{\pgfqpoint{3.229803in}{5.105822in}}%
\pgfpathclose%
\pgfusepath{fill}%
\end{pgfscope}%
\begin{pgfscope}%
\pgfpathrectangle{\pgfqpoint{1.020000in}{0.880000in}}{\pgfqpoint{6.160000in}{6.160000in}}%
\pgfusepath{clip}%
\pgfsetbuttcap%
\pgfsetroundjoin%
\definecolor{currentfill}{rgb}{0.333490,0.446265,0.874452}%
\pgfsetfillcolor{currentfill}%
\pgfsetlinewidth{0.000000pt}%
\definecolor{currentstroke}{rgb}{0.000000,0.000000,0.000000}%
\pgfsetstrokecolor{currentstroke}%
\pgfsetdash{}{0pt}%
\pgfpathmoveto{\pgfqpoint{5.125477in}{3.001588in}}%
\pgfpathlineto{\pgfqpoint{5.137220in}{3.152942in}}%
\pgfpathlineto{\pgfqpoint{5.147438in}{3.141039in}}%
\pgfpathlineto{\pgfqpoint{5.180747in}{3.161897in}}%
\pgfpathlineto{\pgfqpoint{5.212739in}{3.055840in}}%
\pgfpathlineto{\pgfqpoint{5.201915in}{3.013110in}}%
\pgfpathlineto{\pgfqpoint{5.191271in}{2.985145in}}%
\pgfpathlineto{\pgfqpoint{5.158609in}{3.016117in}}%
\pgfpathlineto{\pgfqpoint{5.125477in}{3.001588in}}%
\pgfpathclose%
\pgfusepath{fill}%
\end{pgfscope}%
\begin{pgfscope}%
\pgfpathrectangle{\pgfqpoint{1.020000in}{0.880000in}}{\pgfqpoint{6.160000in}{6.160000in}}%
\pgfusepath{clip}%
\pgfsetbuttcap%
\pgfsetroundjoin%
\definecolor{currentfill}{rgb}{0.399231,0.528528,0.928459}%
\pgfsetfillcolor{currentfill}%
\pgfsetlinewidth{0.000000pt}%
\definecolor{currentstroke}{rgb}{0.000000,0.000000,0.000000}%
\pgfsetstrokecolor{currentstroke}%
\pgfsetdash{}{0pt}%
\pgfpathmoveto{\pgfqpoint{4.755764in}{3.166021in}}%
\pgfpathlineto{\pgfqpoint{4.765928in}{3.210557in}}%
\pgfpathlineto{\pgfqpoint{4.776771in}{3.364365in}}%
\pgfpathlineto{\pgfqpoint{4.808360in}{3.100747in}}%
\pgfpathlineto{\pgfqpoint{4.841497in}{3.100993in}}%
\pgfpathlineto{\pgfqpoint{4.831832in}{3.149852in}}%
\pgfpathlineto{\pgfqpoint{4.822507in}{3.250147in}}%
\pgfpathlineto{\pgfqpoint{4.789383in}{3.249483in}}%
\pgfpathlineto{\pgfqpoint{4.755764in}{3.166021in}}%
\pgfpathclose%
\pgfusepath{fill}%
\end{pgfscope}%
\begin{pgfscope}%
\pgfpathrectangle{\pgfqpoint{1.020000in}{0.880000in}}{\pgfqpoint{6.160000in}{6.160000in}}%
\pgfusepath{clip}%
\pgfsetbuttcap%
\pgfsetroundjoin%
\definecolor{currentfill}{rgb}{0.947654,0.565976,0.447478}%
\pgfsetfillcolor{currentfill}%
\pgfsetlinewidth{0.000000pt}%
\definecolor{currentstroke}{rgb}{0.000000,0.000000,0.000000}%
\pgfsetstrokecolor{currentstroke}%
\pgfsetdash{}{0pt}%
\pgfpathmoveto{\pgfqpoint{2.397287in}{4.570495in}}%
\pgfpathlineto{\pgfqpoint{2.400177in}{4.827564in}}%
\pgfpathlineto{\pgfqpoint{2.411799in}{4.589046in}}%
\pgfpathlineto{\pgfqpoint{2.443673in}{4.702575in}}%
\pgfpathlineto{\pgfqpoint{2.474571in}{4.877449in}}%
\pgfpathlineto{\pgfqpoint{2.468077in}{4.817029in}}%
\pgfpathlineto{\pgfqpoint{2.460509in}{4.821319in}}%
\pgfpathlineto{\pgfqpoint{2.425054in}{4.915946in}}%
\pgfpathlineto{\pgfqpoint{2.397287in}{4.570495in}}%
\pgfpathclose%
\pgfusepath{fill}%
\end{pgfscope}%
\begin{pgfscope}%
\pgfpathrectangle{\pgfqpoint{1.020000in}{0.880000in}}{\pgfqpoint{6.160000in}{6.160000in}}%
\pgfusepath{clip}%
\pgfsetbuttcap%
\pgfsetroundjoin%
\definecolor{currentfill}{rgb}{0.956371,0.775144,0.686416}%
\pgfsetfillcolor{currentfill}%
\pgfsetlinewidth{0.000000pt}%
\definecolor{currentstroke}{rgb}{0.000000,0.000000,0.000000}%
\pgfsetstrokecolor{currentstroke}%
\pgfsetdash{}{0pt}%
\pgfpathmoveto{\pgfqpoint{3.503770in}{4.482769in}}%
\pgfpathlineto{\pgfqpoint{3.513022in}{4.383640in}}%
\pgfpathlineto{\pgfqpoint{3.522401in}{4.265397in}}%
\pgfpathlineto{\pgfqpoint{3.554047in}{4.566836in}}%
\pgfpathlineto{\pgfqpoint{3.589702in}{4.218099in}}%
\pgfpathlineto{\pgfqpoint{3.580160in}{4.366485in}}%
\pgfpathlineto{\pgfqpoint{3.571786in}{4.321181in}}%
\pgfpathlineto{\pgfqpoint{3.537200in}{4.498096in}}%
\pgfpathlineto{\pgfqpoint{3.503770in}{4.482769in}}%
\pgfpathclose%
\pgfusepath{fill}%
\end{pgfscope}%
\begin{pgfscope}%
\pgfpathrectangle{\pgfqpoint{1.020000in}{0.880000in}}{\pgfqpoint{6.160000in}{6.160000in}}%
\pgfusepath{clip}%
\pgfsetbuttcap%
\pgfsetroundjoin%
\definecolor{currentfill}{rgb}{0.796064,0.848693,0.933471}%
\pgfsetfillcolor{currentfill}%
\pgfsetlinewidth{0.000000pt}%
\definecolor{currentstroke}{rgb}{0.000000,0.000000,0.000000}%
\pgfsetstrokecolor{currentstroke}%
\pgfsetdash{}{0pt}%
\pgfpathmoveto{\pgfqpoint{3.759857in}{3.960979in}}%
\pgfpathlineto{\pgfqpoint{3.767882in}{4.162471in}}%
\pgfpathlineto{\pgfqpoint{3.777669in}{3.940639in}}%
\pgfpathlineto{\pgfqpoint{3.811239in}{3.919285in}}%
\pgfpathlineto{\pgfqpoint{3.845619in}{3.647530in}}%
\pgfpathlineto{\pgfqpoint{3.835531in}{3.987643in}}%
\pgfpathlineto{\pgfqpoint{3.826946in}{3.892285in}}%
\pgfpathlineto{\pgfqpoint{3.793444in}{3.921239in}}%
\pgfpathlineto{\pgfqpoint{3.759857in}{3.960979in}}%
\pgfpathclose%
\pgfusepath{fill}%
\end{pgfscope}%
\begin{pgfscope}%
\pgfpathrectangle{\pgfqpoint{1.020000in}{0.880000in}}{\pgfqpoint{6.160000in}{6.160000in}}%
\pgfusepath{clip}%
\pgfsetbuttcap%
\pgfsetroundjoin%
\definecolor{currentfill}{rgb}{0.430507,0.564883,0.948889}%
\pgfsetfillcolor{currentfill}%
\pgfsetlinewidth{0.000000pt}%
\definecolor{currentstroke}{rgb}{0.000000,0.000000,0.000000}%
\pgfsetstrokecolor{currentstroke}%
\pgfsetdash{}{0pt}%
\pgfpathmoveto{\pgfqpoint{4.537828in}{3.308506in}}%
\pgfpathlineto{\pgfqpoint{4.547730in}{3.368807in}}%
\pgfpathlineto{\pgfqpoint{4.556645in}{3.156759in}}%
\pgfpathlineto{\pgfqpoint{4.589820in}{3.143916in}}%
\pgfpathlineto{\pgfqpoint{4.623176in}{3.178773in}}%
\pgfpathlineto{\pgfqpoint{4.613746in}{3.254005in}}%
\pgfpathlineto{\pgfqpoint{4.604078in}{3.271552in}}%
\pgfpathlineto{\pgfqpoint{4.571040in}{3.308207in}}%
\pgfpathlineto{\pgfqpoint{4.537828in}{3.308506in}}%
\pgfpathclose%
\pgfusepath{fill}%
\end{pgfscope}%
\begin{pgfscope}%
\pgfpathrectangle{\pgfqpoint{1.020000in}{0.880000in}}{\pgfqpoint{6.160000in}{6.160000in}}%
\pgfusepath{clip}%
\pgfsetbuttcap%
\pgfsetroundjoin%
\definecolor{currentfill}{rgb}{0.746838,0.140021,0.179996}%
\pgfsetfillcolor{currentfill}%
\pgfsetlinewidth{0.000000pt}%
\definecolor{currentstroke}{rgb}{0.000000,0.000000,0.000000}%
\pgfsetstrokecolor{currentstroke}%
\pgfsetdash{}{0pt}%
\pgfpathmoveto{\pgfqpoint{2.926020in}{5.295375in}}%
\pgfpathlineto{\pgfqpoint{2.934921in}{5.221318in}}%
\pgfpathlineto{\pgfqpoint{2.943207in}{5.198272in}}%
\pgfpathlineto{\pgfqpoint{2.975928in}{5.281901in}}%
\pgfpathlineto{\pgfqpoint{3.009776in}{5.270802in}}%
\pgfpathlineto{\pgfqpoint{3.002262in}{5.222533in}}%
\pgfpathlineto{\pgfqpoint{2.993271in}{5.302766in}}%
\pgfpathlineto{\pgfqpoint{2.960187in}{5.254714in}}%
\pgfpathlineto{\pgfqpoint{2.926020in}{5.295375in}}%
\pgfpathclose%
\pgfusepath{fill}%
\end{pgfscope}%
\begin{pgfscope}%
\pgfpathrectangle{\pgfqpoint{1.020000in}{0.880000in}}{\pgfqpoint{6.160000in}{6.160000in}}%
\pgfusepath{clip}%
\pgfsetbuttcap%
\pgfsetroundjoin%
\definecolor{currentfill}{rgb}{0.925563,0.825517,0.771136}%
\pgfsetfillcolor{currentfill}%
\pgfsetlinewidth{0.000000pt}%
\definecolor{currentstroke}{rgb}{0.000000,0.000000,0.000000}%
\pgfsetstrokecolor{currentstroke}%
\pgfsetdash{}{0pt}%
\pgfpathmoveto{\pgfqpoint{3.589702in}{4.218099in}}%
\pgfpathlineto{\pgfqpoint{3.597334in}{4.397443in}}%
\pgfpathlineto{\pgfqpoint{3.607169in}{4.200146in}}%
\pgfpathlineto{\pgfqpoint{3.640972in}{4.148717in}}%
\pgfpathlineto{\pgfqpoint{3.673860in}{4.267613in}}%
\pgfpathlineto{\pgfqpoint{3.665113in}{4.264756in}}%
\pgfpathlineto{\pgfqpoint{3.656960in}{4.151954in}}%
\pgfpathlineto{\pgfqpoint{3.622509in}{4.336758in}}%
\pgfpathlineto{\pgfqpoint{3.589702in}{4.218099in}}%
\pgfpathclose%
\pgfusepath{fill}%
\end{pgfscope}%
\begin{pgfscope}%
\pgfpathrectangle{\pgfqpoint{1.020000in}{0.880000in}}{\pgfqpoint{6.160000in}{6.160000in}}%
\pgfusepath{clip}%
\pgfsetbuttcap%
\pgfsetroundjoin%
\definecolor{currentfill}{rgb}{0.489246,0.627536,0.976896}%
\pgfsetfillcolor{currentfill}%
\pgfsetlinewidth{0.000000pt}%
\definecolor{currentstroke}{rgb}{0.000000,0.000000,0.000000}%
\pgfsetstrokecolor{currentstroke}%
\pgfsetdash{}{0pt}%
\pgfpathmoveto{\pgfqpoint{4.385968in}{3.422099in}}%
\pgfpathlineto{\pgfqpoint{4.395040in}{3.210038in}}%
\pgfpathlineto{\pgfqpoint{4.404802in}{3.318064in}}%
\pgfpathlineto{\pgfqpoint{4.438369in}{3.424695in}}%
\pgfpathlineto{\pgfqpoint{4.471624in}{3.405616in}}%
\pgfpathlineto{\pgfqpoint{4.461682in}{3.285131in}}%
\pgfpathlineto{\pgfqpoint{4.452357in}{3.377500in}}%
\pgfpathlineto{\pgfqpoint{4.419211in}{3.413065in}}%
\pgfpathlineto{\pgfqpoint{4.385968in}{3.422099in}}%
\pgfpathclose%
\pgfusepath{fill}%
\end{pgfscope}%
\begin{pgfscope}%
\pgfpathrectangle{\pgfqpoint{1.020000in}{0.880000in}}{\pgfqpoint{6.160000in}{6.160000in}}%
\pgfusepath{clip}%
\pgfsetbuttcap%
\pgfsetroundjoin%
\definecolor{currentfill}{rgb}{0.323718,0.433158,0.864722}%
\pgfsetfillcolor{currentfill}%
\pgfsetlinewidth{0.000000pt}%
\definecolor{currentstroke}{rgb}{0.000000,0.000000,0.000000}%
\pgfsetstrokecolor{currentstroke}%
\pgfsetdash{}{0pt}%
\pgfpathmoveto{\pgfqpoint{5.058727in}{2.920758in}}%
\pgfpathlineto{\pgfqpoint{5.070712in}{3.118710in}}%
\pgfpathlineto{\pgfqpoint{5.079614in}{2.967373in}}%
\pgfpathlineto{\pgfqpoint{5.113862in}{3.092253in}}%
\pgfpathlineto{\pgfqpoint{5.147438in}{3.141039in}}%
\pgfpathlineto{\pgfqpoint{5.137220in}{3.152942in}}%
\pgfpathlineto{\pgfqpoint{5.125477in}{3.001588in}}%
\pgfpathlineto{\pgfqpoint{5.092112in}{2.962615in}}%
\pgfpathlineto{\pgfqpoint{5.058727in}{2.920758in}}%
\pgfpathclose%
\pgfusepath{fill}%
\end{pgfscope}%
\begin{pgfscope}%
\pgfpathrectangle{\pgfqpoint{1.020000in}{0.880000in}}{\pgfqpoint{6.160000in}{6.160000in}}%
\pgfusepath{clip}%
\pgfsetbuttcap%
\pgfsetroundjoin%
\definecolor{currentfill}{rgb}{0.348323,0.465711,0.888346}%
\pgfsetfillcolor{currentfill}%
\pgfsetlinewidth{0.000000pt}%
\definecolor{currentstroke}{rgb}{0.000000,0.000000,0.000000}%
\pgfsetstrokecolor{currentstroke}%
\pgfsetdash{}{0pt}%
\pgfpathmoveto{\pgfqpoint{5.629252in}{3.042241in}}%
\pgfpathlineto{\pgfqpoint{5.642432in}{3.202523in}}%
\pgfpathlineto{\pgfqpoint{5.651928in}{3.108735in}}%
\pgfpathlineto{\pgfqpoint{5.683818in}{3.038857in}}%
\pgfpathlineto{\pgfqpoint{5.716990in}{3.056501in}}%
\pgfpathlineto{\pgfqpoint{5.706286in}{3.070980in}}%
\pgfpathlineto{\pgfqpoint{5.695219in}{3.060134in}}%
\pgfpathlineto{\pgfqpoint{5.664549in}{3.206599in}}%
\pgfpathlineto{\pgfqpoint{5.629252in}{3.042241in}}%
\pgfpathclose%
\pgfusepath{fill}%
\end{pgfscope}%
\begin{pgfscope}%
\pgfpathrectangle{\pgfqpoint{1.020000in}{0.880000in}}{\pgfqpoint{6.160000in}{6.160000in}}%
\pgfusepath{clip}%
\pgfsetbuttcap%
\pgfsetroundjoin%
\definecolor{currentfill}{rgb}{0.969289,0.684982,0.568975}%
\pgfsetfillcolor{currentfill}%
\pgfsetlinewidth{0.000000pt}%
\definecolor{currentstroke}{rgb}{0.000000,0.000000,0.000000}%
\pgfsetstrokecolor{currentstroke}%
\pgfsetdash{}{0pt}%
\pgfpathmoveto{\pgfqpoint{3.419478in}{4.505796in}}%
\pgfpathlineto{\pgfqpoint{3.427198in}{4.609540in}}%
\pgfpathlineto{\pgfqpoint{3.435741in}{4.606533in}}%
\pgfpathlineto{\pgfqpoint{3.470338in}{4.466457in}}%
\pgfpathlineto{\pgfqpoint{3.503770in}{4.482769in}}%
\pgfpathlineto{\pgfqpoint{3.494858in}{4.532104in}}%
\pgfpathlineto{\pgfqpoint{3.485597in}{4.633442in}}%
\pgfpathlineto{\pgfqpoint{3.451578in}{4.700908in}}%
\pgfpathlineto{\pgfqpoint{3.419478in}{4.505796in}}%
\pgfpathclose%
\pgfusepath{fill}%
\end{pgfscope}%
\begin{pgfscope}%
\pgfpathrectangle{\pgfqpoint{1.020000in}{0.880000in}}{\pgfqpoint{6.160000in}{6.160000in}}%
\pgfusepath{clip}%
\pgfsetbuttcap%
\pgfsetroundjoin%
\definecolor{currentfill}{rgb}{0.603162,0.731527,0.999565}%
\pgfsetfillcolor{currentfill}%
\pgfsetlinewidth{0.000000pt}%
\definecolor{currentstroke}{rgb}{0.000000,0.000000,0.000000}%
\pgfsetstrokecolor{currentstroke}%
\pgfsetdash{}{0pt}%
\pgfpathmoveto{\pgfqpoint{4.082143in}{3.574196in}}%
\pgfpathlineto{\pgfqpoint{4.091398in}{3.511321in}}%
\pgfpathlineto{\pgfqpoint{4.100597in}{3.525192in}}%
\pgfpathlineto{\pgfqpoint{4.133925in}{3.652213in}}%
\pgfpathlineto{\pgfqpoint{4.167365in}{3.489733in}}%
\pgfpathlineto{\pgfqpoint{4.158027in}{3.742992in}}%
\pgfpathlineto{\pgfqpoint{4.148842in}{3.475495in}}%
\pgfpathlineto{\pgfqpoint{4.115556in}{3.469793in}}%
\pgfpathlineto{\pgfqpoint{4.082143in}{3.574196in}}%
\pgfpathclose%
\pgfusepath{fill}%
\end{pgfscope}%
\begin{pgfscope}%
\pgfpathrectangle{\pgfqpoint{1.020000in}{0.880000in}}{\pgfqpoint{6.160000in}{6.160000in}}%
\pgfusepath{clip}%
\pgfsetbuttcap%
\pgfsetroundjoin%
\definecolor{currentfill}{rgb}{0.378598,0.503856,0.913692}%
\pgfsetfillcolor{currentfill}%
\pgfsetlinewidth{0.000000pt}%
\definecolor{currentstroke}{rgb}{0.000000,0.000000,0.000000}%
\pgfsetstrokecolor{currentstroke}%
\pgfsetdash{}{0pt}%
\pgfpathmoveto{\pgfqpoint{4.907841in}{3.119158in}}%
\pgfpathlineto{\pgfqpoint{4.919488in}{3.333103in}}%
\pgfpathlineto{\pgfqpoint{4.928076in}{3.128737in}}%
\pgfpathlineto{\pgfqpoint{4.961579in}{3.174073in}}%
\pgfpathlineto{\pgfqpoint{4.994373in}{3.130828in}}%
\pgfpathlineto{\pgfqpoint{4.983426in}{3.032624in}}%
\pgfpathlineto{\pgfqpoint{4.973885in}{3.106443in}}%
\pgfpathlineto{\pgfqpoint{4.941177in}{3.152536in}}%
\pgfpathlineto{\pgfqpoint{4.907841in}{3.119158in}}%
\pgfpathclose%
\pgfusepath{fill}%
\end{pgfscope}%
\begin{pgfscope}%
\pgfpathrectangle{\pgfqpoint{1.020000in}{0.880000in}}{\pgfqpoint{6.160000in}{6.160000in}}%
\pgfusepath{clip}%
\pgfsetbuttcap%
\pgfsetroundjoin%
\definecolor{currentfill}{rgb}{0.373552,0.497499,0.909467}%
\pgfsetfillcolor{currentfill}%
\pgfsetlinewidth{0.000000pt}%
\definecolor{currentstroke}{rgb}{0.000000,0.000000,0.000000}%
\pgfsetstrokecolor{currentstroke}%
\pgfsetdash{}{0pt}%
\pgfpathmoveto{\pgfqpoint{5.413245in}{3.267152in}}%
\pgfpathlineto{\pgfqpoint{5.421591in}{3.083791in}}%
\pgfpathlineto{\pgfqpoint{5.432564in}{3.111426in}}%
\pgfpathlineto{\pgfqpoint{5.464425in}{3.023070in}}%
\pgfpathlineto{\pgfqpoint{5.497439in}{3.027976in}}%
\pgfpathlineto{\pgfqpoint{5.488145in}{3.134288in}}%
\pgfpathlineto{\pgfqpoint{5.478745in}{3.232380in}}%
\pgfpathlineto{\pgfqpoint{5.445771in}{3.230820in}}%
\pgfpathlineto{\pgfqpoint{5.413245in}{3.267152in}}%
\pgfpathclose%
\pgfusepath{fill}%
\end{pgfscope}%
\begin{pgfscope}%
\pgfpathrectangle{\pgfqpoint{1.020000in}{0.880000in}}{\pgfqpoint{6.160000in}{6.160000in}}%
\pgfusepath{clip}%
\pgfsetbuttcap%
\pgfsetroundjoin%
\definecolor{currentfill}{rgb}{0.918282,0.484173,0.377794}%
\pgfsetfillcolor{currentfill}%
\pgfsetlinewidth{0.000000pt}%
\definecolor{currentstroke}{rgb}{0.000000,0.000000,0.000000}%
\pgfsetstrokecolor{currentstroke}%
\pgfsetdash{}{0pt}%
\pgfpathmoveto{\pgfqpoint{2.460509in}{4.821319in}}%
\pgfpathlineto{\pgfqpoint{2.468077in}{4.817029in}}%
\pgfpathlineto{\pgfqpoint{2.474571in}{4.877449in}}%
\pgfpathlineto{\pgfqpoint{2.507986in}{4.905670in}}%
\pgfpathlineto{\pgfqpoint{2.543524in}{4.802645in}}%
\pgfpathlineto{\pgfqpoint{2.536039in}{4.796992in}}%
\pgfpathlineto{\pgfqpoint{2.524189in}{5.059778in}}%
\pgfpathlineto{\pgfqpoint{2.493025in}{4.898190in}}%
\pgfpathlineto{\pgfqpoint{2.460509in}{4.821319in}}%
\pgfpathclose%
\pgfusepath{fill}%
\end{pgfscope}%
\begin{pgfscope}%
\pgfpathrectangle{\pgfqpoint{1.020000in}{0.880000in}}{\pgfqpoint{6.160000in}{6.160000in}}%
\pgfusepath{clip}%
\pgfsetbuttcap%
\pgfsetroundjoin%
\definecolor{currentfill}{rgb}{0.363461,0.484784,0.901019}%
\pgfsetfillcolor{currentfill}%
\pgfsetlinewidth{0.000000pt}%
\definecolor{currentstroke}{rgb}{0.000000,0.000000,0.000000}%
\pgfsetstrokecolor{currentstroke}%
\pgfsetdash{}{0pt}%
\pgfpathmoveto{\pgfqpoint{5.914215in}{3.067925in}}%
\pgfpathlineto{\pgfqpoint{5.927395in}{3.184569in}}%
\pgfpathlineto{\pgfqpoint{5.935610in}{3.016282in}}%
\pgfpathlineto{\pgfqpoint{5.972811in}{3.260697in}}%
\pgfpathlineto{\pgfqpoint{5.959529in}{3.144792in}}%
\pgfpathlineto{\pgfqpoint{5.947153in}{3.077204in}}%
\pgfpathlineto{\pgfqpoint{5.914215in}{3.067925in}}%
\pgfpathclose%
\pgfusepath{fill}%
\end{pgfscope}%
\begin{pgfscope}%
\pgfpathrectangle{\pgfqpoint{1.020000in}{0.880000in}}{\pgfqpoint{6.160000in}{6.160000in}}%
\pgfusepath{clip}%
\pgfsetbuttcap%
\pgfsetroundjoin%
\definecolor{currentfill}{rgb}{0.348323,0.465711,0.888346}%
\pgfsetfillcolor{currentfill}%
\pgfsetlinewidth{0.000000pt}%
\definecolor{currentstroke}{rgb}{0.000000,0.000000,0.000000}%
\pgfsetstrokecolor{currentstroke}%
\pgfsetdash{}{0pt}%
\pgfpathmoveto{\pgfqpoint{5.345124in}{3.087708in}}%
\pgfpathlineto{\pgfqpoint{5.356325in}{3.143841in}}%
\pgfpathlineto{\pgfqpoint{5.364094in}{2.907920in}}%
\pgfpathlineto{\pgfqpoint{5.399905in}{3.140777in}}%
\pgfpathlineto{\pgfqpoint{5.432564in}{3.111426in}}%
\pgfpathlineto{\pgfqpoint{5.421591in}{3.083791in}}%
\pgfpathlineto{\pgfqpoint{5.413245in}{3.267152in}}%
\pgfpathlineto{\pgfqpoint{5.377109in}{3.005999in}}%
\pgfpathlineto{\pgfqpoint{5.345124in}{3.087708in}}%
\pgfpathclose%
\pgfusepath{fill}%
\end{pgfscope}%
\begin{pgfscope}%
\pgfpathrectangle{\pgfqpoint{1.020000in}{0.880000in}}{\pgfqpoint{6.160000in}{6.160000in}}%
\pgfusepath{clip}%
\pgfsetbuttcap%
\pgfsetroundjoin%
\definecolor{currentfill}{rgb}{0.667253,0.779176,0.992959}%
\pgfsetfillcolor{currentfill}%
\pgfsetlinewidth{0.000000pt}%
\definecolor{currentstroke}{rgb}{0.000000,0.000000,0.000000}%
\pgfsetstrokecolor{currentstroke}%
\pgfsetdash{}{0pt}%
\pgfpathmoveto{\pgfqpoint{3.930394in}{3.617540in}}%
\pgfpathlineto{\pgfqpoint{3.939035in}{3.778195in}}%
\pgfpathlineto{\pgfqpoint{3.948251in}{3.713631in}}%
\pgfpathlineto{\pgfqpoint{3.981810in}{3.668262in}}%
\pgfpathlineto{\pgfqpoint{4.015309in}{3.628660in}}%
\pgfpathlineto{\pgfqpoint{4.006218in}{3.613777in}}%
\pgfpathlineto{\pgfqpoint{3.997147in}{3.600036in}}%
\pgfpathlineto{\pgfqpoint{3.963501in}{3.737646in}}%
\pgfpathlineto{\pgfqpoint{3.930394in}{3.617540in}}%
\pgfpathclose%
\pgfusepath{fill}%
\end{pgfscope}%
\begin{pgfscope}%
\pgfpathrectangle{\pgfqpoint{1.020000in}{0.880000in}}{\pgfqpoint{6.160000in}{6.160000in}}%
\pgfusepath{clip}%
\pgfsetbuttcap%
\pgfsetroundjoin%
\definecolor{currentfill}{rgb}{0.768929,0.189213,0.197965}%
\pgfsetfillcolor{currentfill}%
\pgfsetlinewidth{0.000000pt}%
\definecolor{currentstroke}{rgb}{0.000000,0.000000,0.000000}%
\pgfsetstrokecolor{currentstroke}%
\pgfsetdash{}{0pt}%
\pgfpathmoveto{\pgfqpoint{3.076989in}{5.285604in}}%
\pgfpathlineto{\pgfqpoint{3.084977in}{5.301398in}}%
\pgfpathlineto{\pgfqpoint{3.093965in}{5.225462in}}%
\pgfpathlineto{\pgfqpoint{3.128196in}{5.175093in}}%
\pgfpathlineto{\pgfqpoint{3.162452in}{5.116909in}}%
\pgfpathlineto{\pgfqpoint{3.154227in}{5.116738in}}%
\pgfpathlineto{\pgfqpoint{3.144507in}{5.264821in}}%
\pgfpathlineto{\pgfqpoint{3.111448in}{5.210017in}}%
\pgfpathlineto{\pgfqpoint{3.076989in}{5.285604in}}%
\pgfpathclose%
\pgfusepath{fill}%
\end{pgfscope}%
\begin{pgfscope}%
\pgfpathrectangle{\pgfqpoint{1.020000in}{0.880000in}}{\pgfqpoint{6.160000in}{6.160000in}}%
\pgfusepath{clip}%
\pgfsetbuttcap%
\pgfsetroundjoin%
\definecolor{currentfill}{rgb}{0.947654,0.565976,0.447478}%
\pgfsetfillcolor{currentfill}%
\pgfsetlinewidth{0.000000pt}%
\definecolor{currentstroke}{rgb}{0.000000,0.000000,0.000000}%
\pgfsetstrokecolor{currentstroke}%
\pgfsetdash{}{0pt}%
\pgfpathmoveto{\pgfqpoint{3.332859in}{4.832093in}}%
\pgfpathlineto{\pgfqpoint{3.340465in}{4.930254in}}%
\pgfpathlineto{\pgfqpoint{3.349659in}{4.839336in}}%
\pgfpathlineto{\pgfqpoint{3.384714in}{4.662409in}}%
\pgfpathlineto{\pgfqpoint{3.419478in}{4.505796in}}%
\pgfpathlineto{\pgfqpoint{3.408524in}{4.830100in}}%
\pgfpathlineto{\pgfqpoint{3.401400in}{4.653689in}}%
\pgfpathlineto{\pgfqpoint{3.366922in}{4.773546in}}%
\pgfpathlineto{\pgfqpoint{3.332859in}{4.832093in}}%
\pgfpathclose%
\pgfusepath{fill}%
\end{pgfscope}%
\begin{pgfscope}%
\pgfpathrectangle{\pgfqpoint{1.020000in}{0.880000in}}{\pgfqpoint{6.160000in}{6.160000in}}%
\pgfusepath{clip}%
\pgfsetbuttcap%
\pgfsetroundjoin%
\definecolor{currentfill}{rgb}{0.810616,0.268797,0.235428}%
\pgfsetfillcolor{currentfill}%
\pgfsetlinewidth{0.000000pt}%
\definecolor{currentstroke}{rgb}{0.000000,0.000000,0.000000}%
\pgfsetstrokecolor{currentstroke}%
\pgfsetdash{}{0pt}%
\pgfpathmoveto{\pgfqpoint{2.728050in}{4.980337in}}%
\pgfpathlineto{\pgfqpoint{2.733806in}{5.121777in}}%
\pgfpathlineto{\pgfqpoint{2.741719in}{5.113751in}}%
\pgfpathlineto{\pgfqpoint{2.774126in}{5.214515in}}%
\pgfpathlineto{\pgfqpoint{2.807991in}{5.210923in}}%
\pgfpathlineto{\pgfqpoint{2.800806in}{5.160346in}}%
\pgfpathlineto{\pgfqpoint{2.792653in}{5.182377in}}%
\pgfpathlineto{\pgfqpoint{2.760630in}{5.059759in}}%
\pgfpathlineto{\pgfqpoint{2.728050in}{4.980337in}}%
\pgfpathclose%
\pgfusepath{fill}%
\end{pgfscope}%
\begin{pgfscope}%
\pgfpathrectangle{\pgfqpoint{1.020000in}{0.880000in}}{\pgfqpoint{6.160000in}{6.160000in}}%
\pgfusepath{clip}%
\pgfsetbuttcap%
\pgfsetroundjoin%
\definecolor{currentfill}{rgb}{0.746838,0.140021,0.179996}%
\pgfsetfillcolor{currentfill}%
\pgfsetlinewidth{0.000000pt}%
\definecolor{currentstroke}{rgb}{0.000000,0.000000,0.000000}%
\pgfsetstrokecolor{currentstroke}%
\pgfsetdash{}{0pt}%
\pgfpathmoveto{\pgfqpoint{2.859085in}{5.258359in}}%
\pgfpathlineto{\pgfqpoint{2.865655in}{5.363203in}}%
\pgfpathlineto{\pgfqpoint{2.876893in}{5.107271in}}%
\pgfpathlineto{\pgfqpoint{2.909736in}{5.177421in}}%
\pgfpathlineto{\pgfqpoint{2.943207in}{5.198272in}}%
\pgfpathlineto{\pgfqpoint{2.934921in}{5.221318in}}%
\pgfpathlineto{\pgfqpoint{2.926020in}{5.295375in}}%
\pgfpathlineto{\pgfqpoint{2.891319in}{5.374431in}}%
\pgfpathlineto{\pgfqpoint{2.859085in}{5.258359in}}%
\pgfpathclose%
\pgfusepath{fill}%
\end{pgfscope}%
\begin{pgfscope}%
\pgfpathrectangle{\pgfqpoint{1.020000in}{0.880000in}}{\pgfqpoint{6.160000in}{6.160000in}}%
\pgfusepath{clip}%
\pgfsetbuttcap%
\pgfsetroundjoin%
\definecolor{currentfill}{rgb}{0.576051,0.708780,0.997755}%
\pgfsetfillcolor{currentfill}%
\pgfsetlinewidth{0.000000pt}%
\definecolor{currentstroke}{rgb}{0.000000,0.000000,0.000000}%
\pgfsetstrokecolor{currentstroke}%
\pgfsetdash{}{0pt}%
\pgfpathmoveto{\pgfqpoint{4.234053in}{3.546397in}}%
\pgfpathlineto{\pgfqpoint{4.243516in}{3.710964in}}%
\pgfpathlineto{\pgfqpoint{4.252655in}{3.336301in}}%
\pgfpathlineto{\pgfqpoint{4.286041in}{3.406363in}}%
\pgfpathlineto{\pgfqpoint{4.319610in}{3.590111in}}%
\pgfpathlineto{\pgfqpoint{4.309832in}{3.334127in}}%
\pgfpathlineto{\pgfqpoint{4.300526in}{3.422097in}}%
\pgfpathlineto{\pgfqpoint{4.267478in}{3.671885in}}%
\pgfpathlineto{\pgfqpoint{4.234053in}{3.546397in}}%
\pgfpathclose%
\pgfusepath{fill}%
\end{pgfscope}%
\begin{pgfscope}%
\pgfpathrectangle{\pgfqpoint{1.020000in}{0.880000in}}{\pgfqpoint{6.160000in}{6.160000in}}%
\pgfusepath{clip}%
\pgfsetbuttcap%
\pgfsetroundjoin%
\definecolor{currentfill}{rgb}{0.425199,0.559058,0.946061}%
\pgfsetfillcolor{currentfill}%
\pgfsetlinewidth{0.000000pt}%
\definecolor{currentstroke}{rgb}{0.000000,0.000000,0.000000}%
\pgfsetstrokecolor{currentstroke}%
\pgfsetdash{}{0pt}%
\pgfpathmoveto{\pgfqpoint{4.690237in}{3.314830in}}%
\pgfpathlineto{\pgfqpoint{4.699700in}{3.239376in}}%
\pgfpathlineto{\pgfqpoint{4.709123in}{3.154882in}}%
\pgfpathlineto{\pgfqpoint{4.742532in}{3.190061in}}%
\pgfpathlineto{\pgfqpoint{4.776771in}{3.364365in}}%
\pgfpathlineto{\pgfqpoint{4.765928in}{3.210557in}}%
\pgfpathlineto{\pgfqpoint{4.755764in}{3.166021in}}%
\pgfpathlineto{\pgfqpoint{4.722765in}{3.190468in}}%
\pgfpathlineto{\pgfqpoint{4.690237in}{3.314830in}}%
\pgfpathclose%
\pgfusepath{fill}%
\end{pgfscope}%
\begin{pgfscope}%
\pgfpathrectangle{\pgfqpoint{1.020000in}{0.880000in}}{\pgfqpoint{6.160000in}{6.160000in}}%
\pgfusepath{clip}%
\pgfsetbuttcap%
\pgfsetroundjoin%
\definecolor{currentfill}{rgb}{0.323718,0.433158,0.864722}%
\pgfsetfillcolor{currentfill}%
\pgfsetlinewidth{0.000000pt}%
\definecolor{currentstroke}{rgb}{0.000000,0.000000,0.000000}%
\pgfsetstrokecolor{currentstroke}%
\pgfsetdash{}{0pt}%
\pgfpathmoveto{\pgfqpoint{5.497439in}{3.027976in}}%
\pgfpathlineto{\pgfqpoint{5.509247in}{3.110732in}}%
\pgfpathlineto{\pgfqpoint{5.519965in}{3.109164in}}%
\pgfpathlineto{\pgfqpoint{5.550561in}{2.932583in}}%
\pgfpathlineto{\pgfqpoint{5.584943in}{3.034847in}}%
\pgfpathlineto{\pgfqpoint{5.575787in}{3.152214in}}%
\pgfpathlineto{\pgfqpoint{5.563798in}{3.065314in}}%
\pgfpathlineto{\pgfqpoint{5.528557in}{2.893993in}}%
\pgfpathlineto{\pgfqpoint{5.497439in}{3.027976in}}%
\pgfpathclose%
\pgfusepath{fill}%
\end{pgfscope}%
\begin{pgfscope}%
\pgfpathrectangle{\pgfqpoint{1.020000in}{0.880000in}}{\pgfqpoint{6.160000in}{6.160000in}}%
\pgfusepath{clip}%
\pgfsetbuttcap%
\pgfsetroundjoin%
\definecolor{currentfill}{rgb}{0.961645,0.758029,0.661782}%
\pgfsetfillcolor{currentfill}%
\pgfsetlinewidth{0.000000pt}%
\definecolor{currentstroke}{rgb}{0.000000,0.000000,0.000000}%
\pgfsetstrokecolor{currentstroke}%
\pgfsetdash{}{0pt}%
\pgfpathmoveto{\pgfqpoint{3.435741in}{4.606533in}}%
\pgfpathlineto{\pgfqpoint{3.444844in}{4.529349in}}%
\pgfpathlineto{\pgfqpoint{3.455358in}{4.257352in}}%
\pgfpathlineto{\pgfqpoint{3.488389in}{4.333861in}}%
\pgfpathlineto{\pgfqpoint{3.522401in}{4.265397in}}%
\pgfpathlineto{\pgfqpoint{3.513022in}{4.383640in}}%
\pgfpathlineto{\pgfqpoint{3.503770in}{4.482769in}}%
\pgfpathlineto{\pgfqpoint{3.470338in}{4.466457in}}%
\pgfpathlineto{\pgfqpoint{3.435741in}{4.606533in}}%
\pgfpathclose%
\pgfusepath{fill}%
\end{pgfscope}%
\begin{pgfscope}%
\pgfpathrectangle{\pgfqpoint{1.020000in}{0.880000in}}{\pgfqpoint{6.160000in}{6.160000in}}%
\pgfusepath{clip}%
\pgfsetbuttcap%
\pgfsetroundjoin%
\definecolor{currentfill}{rgb}{0.358415,0.478426,0.896795}%
\pgfsetfillcolor{currentfill}%
\pgfsetlinewidth{0.000000pt}%
\definecolor{currentstroke}{rgb}{0.000000,0.000000,0.000000}%
\pgfsetstrokecolor{currentstroke}%
\pgfsetdash{}{0pt}%
\pgfpathmoveto{\pgfqpoint{5.563798in}{3.065314in}}%
\pgfpathlineto{\pgfqpoint{5.575787in}{3.152214in}}%
\pgfpathlineto{\pgfqpoint{5.584943in}{3.034847in}}%
\pgfpathlineto{\pgfqpoint{5.619212in}{3.125885in}}%
\pgfpathlineto{\pgfqpoint{5.651928in}{3.108735in}}%
\pgfpathlineto{\pgfqpoint{5.642432in}{3.202523in}}%
\pgfpathlineto{\pgfqpoint{5.629252in}{3.042241in}}%
\pgfpathlineto{\pgfqpoint{5.596935in}{3.081749in}}%
\pgfpathlineto{\pgfqpoint{5.563798in}{3.065314in}}%
\pgfpathclose%
\pgfusepath{fill}%
\end{pgfscope}%
\begin{pgfscope}%
\pgfpathrectangle{\pgfqpoint{1.020000in}{0.880000in}}{\pgfqpoint{6.160000in}{6.160000in}}%
\pgfusepath{clip}%
\pgfsetbuttcap%
\pgfsetroundjoin%
\definecolor{currentfill}{rgb}{0.825294,0.295749,0.250025}%
\pgfsetfillcolor{currentfill}%
\pgfsetlinewidth{0.000000pt}%
\definecolor{currentstroke}{rgb}{0.000000,0.000000,0.000000}%
\pgfsetstrokecolor{currentstroke}%
\pgfsetdash{}{0pt}%
\pgfpathmoveto{\pgfqpoint{2.657458in}{5.190397in}}%
\pgfpathlineto{\pgfqpoint{2.666406in}{5.108190in}}%
\pgfpathlineto{\pgfqpoint{2.675416in}{5.022121in}}%
\pgfpathlineto{\pgfqpoint{2.707173in}{5.163670in}}%
\pgfpathlineto{\pgfqpoint{2.741719in}{5.113751in}}%
\pgfpathlineto{\pgfqpoint{2.733806in}{5.121777in}}%
\pgfpathlineto{\pgfqpoint{2.728050in}{4.980337in}}%
\pgfpathlineto{\pgfqpoint{2.692392in}{5.113068in}}%
\pgfpathlineto{\pgfqpoint{2.657458in}{5.190397in}}%
\pgfpathclose%
\pgfusepath{fill}%
\end{pgfscope}%
\begin{pgfscope}%
\pgfpathrectangle{\pgfqpoint{1.020000in}{0.880000in}}{\pgfqpoint{6.160000in}{6.160000in}}%
\pgfusepath{clip}%
\pgfsetbuttcap%
\pgfsetroundjoin%
\definecolor{currentfill}{rgb}{0.333490,0.446265,0.874452}%
\pgfsetfillcolor{currentfill}%
\pgfsetlinewidth{0.000000pt}%
\definecolor{currentstroke}{rgb}{0.000000,0.000000,0.000000}%
\pgfsetstrokecolor{currentstroke}%
\pgfsetdash{}{0pt}%
\pgfpathmoveto{\pgfqpoint{4.994373in}{3.130828in}}%
\pgfpathlineto{\pgfqpoint{5.004517in}{3.126525in}}%
\pgfpathlineto{\pgfqpoint{5.014298in}{3.076157in}}%
\pgfpathlineto{\pgfqpoint{5.047703in}{3.104696in}}%
\pgfpathlineto{\pgfqpoint{5.079614in}{2.967373in}}%
\pgfpathlineto{\pgfqpoint{5.070712in}{3.118710in}}%
\pgfpathlineto{\pgfqpoint{5.058727in}{2.920758in}}%
\pgfpathlineto{\pgfqpoint{5.026462in}{3.010096in}}%
\pgfpathlineto{\pgfqpoint{4.994373in}{3.130828in}}%
\pgfpathclose%
\pgfusepath{fill}%
\end{pgfscope}%
\begin{pgfscope}%
\pgfpathrectangle{\pgfqpoint{1.020000in}{0.880000in}}{\pgfqpoint{6.160000in}{6.160000in}}%
\pgfusepath{clip}%
\pgfsetbuttcap%
\pgfsetroundjoin%
\definecolor{currentfill}{rgb}{0.348323,0.465711,0.888346}%
\pgfsetfillcolor{currentfill}%
\pgfsetlinewidth{0.000000pt}%
\definecolor{currentstroke}{rgb}{0.000000,0.000000,0.000000}%
\pgfsetstrokecolor{currentstroke}%
\pgfsetdash{}{0pt}%
\pgfpathmoveto{\pgfqpoint{5.277790in}{2.965322in}}%
\pgfpathlineto{\pgfqpoint{5.291193in}{3.230058in}}%
\pgfpathlineto{\pgfqpoint{5.300460in}{3.118861in}}%
\pgfpathlineto{\pgfqpoint{5.332850in}{3.059092in}}%
\pgfpathlineto{\pgfqpoint{5.364094in}{2.907920in}}%
\pgfpathlineto{\pgfqpoint{5.356325in}{3.143841in}}%
\pgfpathlineto{\pgfqpoint{5.345124in}{3.087708in}}%
\pgfpathlineto{\pgfqpoint{5.312861in}{3.151423in}}%
\pgfpathlineto{\pgfqpoint{5.277790in}{2.965322in}}%
\pgfpathclose%
\pgfusepath{fill}%
\end{pgfscope}%
\begin{pgfscope}%
\pgfpathrectangle{\pgfqpoint{1.020000in}{0.880000in}}{\pgfqpoint{6.160000in}{6.160000in}}%
\pgfusepath{clip}%
\pgfsetbuttcap%
\pgfsetroundjoin%
\definecolor{currentfill}{rgb}{0.969851,0.695830,0.581312}%
\pgfsetfillcolor{currentfill}%
\pgfsetlinewidth{0.000000pt}%
\definecolor{currentstroke}{rgb}{0.000000,0.000000,0.000000}%
\pgfsetstrokecolor{currentstroke}%
\pgfsetdash{}{0pt}%
\pgfpathmoveto{\pgfqpoint{2.275338in}{4.640237in}}%
\pgfpathlineto{\pgfqpoint{2.286570in}{4.431521in}}%
\pgfpathlineto{\pgfqpoint{2.292345in}{4.513808in}}%
\pgfpathlineto{\pgfqpoint{2.326436in}{4.505854in}}%
\pgfpathlineto{\pgfqpoint{2.359562in}{4.550071in}}%
\pgfpathlineto{\pgfqpoint{2.352258in}{4.545347in}}%
\pgfpathlineto{\pgfqpoint{2.342101in}{4.698492in}}%
\pgfpathlineto{\pgfqpoint{2.313583in}{4.406657in}}%
\pgfpathlineto{\pgfqpoint{2.275338in}{4.640237in}}%
\pgfpathclose%
\pgfusepath{fill}%
\end{pgfscope}%
\begin{pgfscope}%
\pgfpathrectangle{\pgfqpoint{1.020000in}{0.880000in}}{\pgfqpoint{6.160000in}{6.160000in}}%
\pgfusepath{clip}%
\pgfsetbuttcap%
\pgfsetroundjoin%
\definecolor{currentfill}{rgb}{0.758112,0.168122,0.188827}%
\pgfsetfillcolor{currentfill}%
\pgfsetlinewidth{0.000000pt}%
\definecolor{currentstroke}{rgb}{0.000000,0.000000,0.000000}%
\pgfsetstrokecolor{currentstroke}%
\pgfsetdash{}{0pt}%
\pgfpathmoveto{\pgfqpoint{2.792653in}{5.182377in}}%
\pgfpathlineto{\pgfqpoint{2.800806in}{5.160346in}}%
\pgfpathlineto{\pgfqpoint{2.807991in}{5.210923in}}%
\pgfpathlineto{\pgfqpoint{2.841354in}{5.243644in}}%
\pgfpathlineto{\pgfqpoint{2.876893in}{5.107271in}}%
\pgfpathlineto{\pgfqpoint{2.865655in}{5.363203in}}%
\pgfpathlineto{\pgfqpoint{2.859085in}{5.258359in}}%
\pgfpathlineto{\pgfqpoint{2.825362in}{5.258181in}}%
\pgfpathlineto{\pgfqpoint{2.792653in}{5.182377in}}%
\pgfpathclose%
\pgfusepath{fill}%
\end{pgfscope}%
\begin{pgfscope}%
\pgfpathrectangle{\pgfqpoint{1.020000in}{0.880000in}}{\pgfqpoint{6.160000in}{6.160000in}}%
\pgfusepath{clip}%
\pgfsetbuttcap%
\pgfsetroundjoin%
\definecolor{currentfill}{rgb}{0.608547,0.735725,0.999354}%
\pgfsetfillcolor{currentfill}%
\pgfsetlinewidth{0.000000pt}%
\definecolor{currentstroke}{rgb}{0.000000,0.000000,0.000000}%
\pgfsetstrokecolor{currentstroke}%
\pgfsetdash{}{0pt}%
\pgfpathmoveto{\pgfqpoint{4.015309in}{3.628660in}}%
\pgfpathlineto{\pgfqpoint{4.024497in}{3.595888in}}%
\pgfpathlineto{\pgfqpoint{4.033723in}{3.546547in}}%
\pgfpathlineto{\pgfqpoint{4.067062in}{3.631834in}}%
\pgfpathlineto{\pgfqpoint{4.100597in}{3.525192in}}%
\pgfpathlineto{\pgfqpoint{4.091398in}{3.511321in}}%
\pgfpathlineto{\pgfqpoint{4.082143in}{3.574196in}}%
\pgfpathlineto{\pgfqpoint{4.048889in}{3.494325in}}%
\pgfpathlineto{\pgfqpoint{4.015309in}{3.628660in}}%
\pgfpathclose%
\pgfusepath{fill}%
\end{pgfscope}%
\begin{pgfscope}%
\pgfpathrectangle{\pgfqpoint{1.020000in}{0.880000in}}{\pgfqpoint{6.160000in}{6.160000in}}%
\pgfusepath{clip}%
\pgfsetbuttcap%
\pgfsetroundjoin%
\definecolor{currentfill}{rgb}{0.848040,0.338280,0.275206}%
\pgfsetfillcolor{currentfill}%
\pgfsetlinewidth{0.000000pt}%
\definecolor{currentstroke}{rgb}{0.000000,0.000000,0.000000}%
\pgfsetstrokecolor{currentstroke}%
\pgfsetdash{}{0pt}%
\pgfpathmoveto{\pgfqpoint{2.592667in}{5.007104in}}%
\pgfpathlineto{\pgfqpoint{2.599368in}{5.067491in}}%
\pgfpathlineto{\pgfqpoint{2.606049in}{5.130782in}}%
\pgfpathlineto{\pgfqpoint{2.642508in}{4.961293in}}%
\pgfpathlineto{\pgfqpoint{2.675416in}{5.022121in}}%
\pgfpathlineto{\pgfqpoint{2.666406in}{5.108190in}}%
\pgfpathlineto{\pgfqpoint{2.657458in}{5.190397in}}%
\pgfpathlineto{\pgfqpoint{2.628199in}{4.892738in}}%
\pgfpathlineto{\pgfqpoint{2.592667in}{5.007104in}}%
\pgfpathclose%
\pgfusepath{fill}%
\end{pgfscope}%
\begin{pgfscope}%
\pgfpathrectangle{\pgfqpoint{1.020000in}{0.880000in}}{\pgfqpoint{6.160000in}{6.160000in}}%
\pgfusepath{clip}%
\pgfsetbuttcap%
\pgfsetroundjoin%
\definecolor{currentfill}{rgb}{0.378598,0.503856,0.913692}%
\pgfsetfillcolor{currentfill}%
\pgfsetlinewidth{0.000000pt}%
\definecolor{currentstroke}{rgb}{0.000000,0.000000,0.000000}%
\pgfsetstrokecolor{currentstroke}%
\pgfsetdash{}{0pt}%
\pgfpathmoveto{\pgfqpoint{5.850094in}{3.156705in}}%
\pgfpathlineto{\pgfqpoint{5.861010in}{3.146598in}}%
\pgfpathlineto{\pgfqpoint{5.872847in}{3.189586in}}%
\pgfpathlineto{\pgfqpoint{5.906229in}{3.216577in}}%
\pgfpathlineto{\pgfqpoint{5.935610in}{3.016282in}}%
\pgfpathlineto{\pgfqpoint{5.927395in}{3.184569in}}%
\pgfpathlineto{\pgfqpoint{5.914215in}{3.067925in}}%
\pgfpathlineto{\pgfqpoint{5.882744in}{3.145449in}}%
\pgfpathlineto{\pgfqpoint{5.850094in}{3.156705in}}%
\pgfpathclose%
\pgfusepath{fill}%
\end{pgfscope}%
\begin{pgfscope}%
\pgfpathrectangle{\pgfqpoint{1.020000in}{0.880000in}}{\pgfqpoint{6.160000in}{6.160000in}}%
\pgfusepath{clip}%
\pgfsetbuttcap%
\pgfsetroundjoin%
\definecolor{currentfill}{rgb}{0.967317,0.657471,0.538160}%
\pgfsetfillcolor{currentfill}%
\pgfsetlinewidth{0.000000pt}%
\definecolor{currentstroke}{rgb}{0.000000,0.000000,0.000000}%
\pgfsetstrokecolor{currentstroke}%
\pgfsetdash{}{0pt}%
\pgfpathmoveto{\pgfqpoint{2.342101in}{4.698492in}}%
\pgfpathlineto{\pgfqpoint{2.352258in}{4.545347in}}%
\pgfpathlineto{\pgfqpoint{2.359562in}{4.550071in}}%
\pgfpathlineto{\pgfqpoint{2.391935in}{4.637378in}}%
\pgfpathlineto{\pgfqpoint{2.424932in}{4.690949in}}%
\pgfpathlineto{\pgfqpoint{2.419532in}{4.572445in}}%
\pgfpathlineto{\pgfqpoint{2.411799in}{4.589046in}}%
\pgfpathlineto{\pgfqpoint{2.378151in}{4.578050in}}%
\pgfpathlineto{\pgfqpoint{2.342101in}{4.698492in}}%
\pgfpathclose%
\pgfusepath{fill}%
\end{pgfscope}%
\begin{pgfscope}%
\pgfpathrectangle{\pgfqpoint{1.020000in}{0.880000in}}{\pgfqpoint{6.160000in}{6.160000in}}%
\pgfusepath{clip}%
\pgfsetbuttcap%
\pgfsetroundjoin%
\definecolor{currentfill}{rgb}{0.409611,0.540759,0.935545}%
\pgfsetfillcolor{currentfill}%
\pgfsetlinewidth{0.000000pt}%
\definecolor{currentstroke}{rgb}{0.000000,0.000000,0.000000}%
\pgfsetstrokecolor{currentstroke}%
\pgfsetdash{}{0pt}%
\pgfpathmoveto{\pgfqpoint{4.623176in}{3.178773in}}%
\pgfpathlineto{\pgfqpoint{4.633102in}{3.211948in}}%
\pgfpathlineto{\pgfqpoint{4.642439in}{3.111399in}}%
\pgfpathlineto{\pgfqpoint{4.676185in}{3.214512in}}%
\pgfpathlineto{\pgfqpoint{4.709123in}{3.154882in}}%
\pgfpathlineto{\pgfqpoint{4.699700in}{3.239376in}}%
\pgfpathlineto{\pgfqpoint{4.690237in}{3.314830in}}%
\pgfpathlineto{\pgfqpoint{4.656380in}{3.180683in}}%
\pgfpathlineto{\pgfqpoint{4.623176in}{3.178773in}}%
\pgfpathclose%
\pgfusepath{fill}%
\end{pgfscope}%
\begin{pgfscope}%
\pgfpathrectangle{\pgfqpoint{1.020000in}{0.880000in}}{\pgfqpoint{6.160000in}{6.160000in}}%
\pgfusepath{clip}%
\pgfsetbuttcap%
\pgfsetroundjoin%
\definecolor{currentfill}{rgb}{0.887752,0.854040,0.834671}%
\pgfsetfillcolor{currentfill}%
\pgfsetlinewidth{0.000000pt}%
\definecolor{currentstroke}{rgb}{0.000000,0.000000,0.000000}%
\pgfsetstrokecolor{currentstroke}%
\pgfsetdash{}{0pt}%
\pgfpathmoveto{\pgfqpoint{3.607169in}{4.200146in}}%
\pgfpathlineto{\pgfqpoint{3.615366in}{4.291007in}}%
\pgfpathlineto{\pgfqpoint{3.624546in}{4.210609in}}%
\pgfpathlineto{\pgfqpoint{3.659649in}{3.921117in}}%
\pgfpathlineto{\pgfqpoint{3.692291in}{4.092616in}}%
\pgfpathlineto{\pgfqpoint{3.684336in}{3.925898in}}%
\pgfpathlineto{\pgfqpoint{3.673860in}{4.267613in}}%
\pgfpathlineto{\pgfqpoint{3.640972in}{4.148717in}}%
\pgfpathlineto{\pgfqpoint{3.607169in}{4.200146in}}%
\pgfpathclose%
\pgfusepath{fill}%
\end{pgfscope}%
\begin{pgfscope}%
\pgfpathrectangle{\pgfqpoint{1.020000in}{0.880000in}}{\pgfqpoint{6.160000in}{6.160000in}}%
\pgfusepath{clip}%
\pgfsetbuttcap%
\pgfsetroundjoin%
\definecolor{currentfill}{rgb}{0.869655,0.379274,0.300941}%
\pgfsetfillcolor{currentfill}%
\pgfsetlinewidth{0.000000pt}%
\definecolor{currentstroke}{rgb}{0.000000,0.000000,0.000000}%
\pgfsetstrokecolor{currentstroke}%
\pgfsetdash{}{0pt}%
\pgfpathmoveto{\pgfqpoint{2.524189in}{5.059778in}}%
\pgfpathlineto{\pgfqpoint{2.536039in}{4.796992in}}%
\pgfpathlineto{\pgfqpoint{2.543524in}{4.802645in}}%
\pgfpathlineto{\pgfqpoint{2.573590in}{5.039420in}}%
\pgfpathlineto{\pgfqpoint{2.606049in}{5.130782in}}%
\pgfpathlineto{\pgfqpoint{2.599368in}{5.067491in}}%
\pgfpathlineto{\pgfqpoint{2.592667in}{5.007104in}}%
\pgfpathlineto{\pgfqpoint{2.557144in}{5.114909in}}%
\pgfpathlineto{\pgfqpoint{2.524189in}{5.059778in}}%
\pgfpathclose%
\pgfusepath{fill}%
\end{pgfscope}%
\begin{pgfscope}%
\pgfpathrectangle{\pgfqpoint{1.020000in}{0.880000in}}{\pgfqpoint{6.160000in}{6.160000in}}%
\pgfusepath{clip}%
\pgfsetbuttcap%
\pgfsetroundjoin%
\definecolor{currentfill}{rgb}{0.516260,0.654498,0.986407}%
\pgfsetfillcolor{currentfill}%
\pgfsetlinewidth{0.000000pt}%
\definecolor{currentstroke}{rgb}{0.000000,0.000000,0.000000}%
\pgfsetstrokecolor{currentstroke}%
\pgfsetdash{}{0pt}%
\pgfpathmoveto{\pgfqpoint{4.319610in}{3.590111in}}%
\pgfpathlineto{\pgfqpoint{4.328868in}{3.442925in}}%
\pgfpathlineto{\pgfqpoint{4.338163in}{3.323787in}}%
\pgfpathlineto{\pgfqpoint{4.371619in}{3.387329in}}%
\pgfpathlineto{\pgfqpoint{4.404802in}{3.318064in}}%
\pgfpathlineto{\pgfqpoint{4.395040in}{3.210038in}}%
\pgfpathlineto{\pgfqpoint{4.385968in}{3.422099in}}%
\pgfpathlineto{\pgfqpoint{4.352713in}{3.439140in}}%
\pgfpathlineto{\pgfqpoint{4.319610in}{3.590111in}}%
\pgfpathclose%
\pgfusepath{fill}%
\end{pgfscope}%
\begin{pgfscope}%
\pgfpathrectangle{\pgfqpoint{1.020000in}{0.880000in}}{\pgfqpoint{6.160000in}{6.160000in}}%
\pgfusepath{clip}%
\pgfsetbuttcap%
\pgfsetroundjoin%
\definecolor{currentfill}{rgb}{0.570616,0.704109,0.997195}%
\pgfsetfillcolor{currentfill}%
\pgfsetlinewidth{0.000000pt}%
\definecolor{currentstroke}{rgb}{0.000000,0.000000,0.000000}%
\pgfsetstrokecolor{currentstroke}%
\pgfsetdash{}{0pt}%
\pgfpathmoveto{\pgfqpoint{4.167365in}{3.489733in}}%
\pgfpathlineto{\pgfqpoint{4.176657in}{3.510509in}}%
\pgfpathlineto{\pgfqpoint{4.185969in}{3.412897in}}%
\pgfpathlineto{\pgfqpoint{4.219375in}{3.487038in}}%
\pgfpathlineto{\pgfqpoint{4.252655in}{3.336301in}}%
\pgfpathlineto{\pgfqpoint{4.243516in}{3.710964in}}%
\pgfpathlineto{\pgfqpoint{4.234053in}{3.546397in}}%
\pgfpathlineto{\pgfqpoint{4.200693in}{3.421351in}}%
\pgfpathlineto{\pgfqpoint{4.167365in}{3.489733in}}%
\pgfpathclose%
\pgfusepath{fill}%
\end{pgfscope}%
\begin{pgfscope}%
\pgfpathrectangle{\pgfqpoint{1.020000in}{0.880000in}}{\pgfqpoint{6.160000in}{6.160000in}}%
\pgfusepath{clip}%
\pgfsetbuttcap%
\pgfsetroundjoin%
\definecolor{currentfill}{rgb}{0.399231,0.528528,0.928459}%
\pgfsetfillcolor{currentfill}%
\pgfsetlinewidth{0.000000pt}%
\definecolor{currentstroke}{rgb}{0.000000,0.000000,0.000000}%
\pgfsetstrokecolor{currentstroke}%
\pgfsetdash{}{0pt}%
\pgfpathmoveto{\pgfqpoint{4.841497in}{3.100993in}}%
\pgfpathlineto{\pgfqpoint{4.852108in}{3.191355in}}%
\pgfpathlineto{\pgfqpoint{4.862118in}{3.187703in}}%
\pgfpathlineto{\pgfqpoint{4.895203in}{3.170329in}}%
\pgfpathlineto{\pgfqpoint{4.928076in}{3.128737in}}%
\pgfpathlineto{\pgfqpoint{4.919488in}{3.333103in}}%
\pgfpathlineto{\pgfqpoint{4.907841in}{3.119158in}}%
\pgfpathlineto{\pgfqpoint{4.875123in}{3.174613in}}%
\pgfpathlineto{\pgfqpoint{4.841497in}{3.100993in}}%
\pgfpathclose%
\pgfusepath{fill}%
\end{pgfscope}%
\begin{pgfscope}%
\pgfpathrectangle{\pgfqpoint{1.020000in}{0.880000in}}{\pgfqpoint{6.160000in}{6.160000in}}%
\pgfusepath{clip}%
\pgfsetbuttcap%
\pgfsetroundjoin%
\definecolor{currentfill}{rgb}{0.880896,0.402331,0.317115}%
\pgfsetfillcolor{currentfill}%
\pgfsetlinewidth{0.000000pt}%
\definecolor{currentstroke}{rgb}{0.000000,0.000000,0.000000}%
\pgfsetstrokecolor{currentstroke}%
\pgfsetdash{}{0pt}%
\pgfpathmoveto{\pgfqpoint{3.244636in}{5.301262in}}%
\pgfpathlineto{\pgfqpoint{3.254753in}{5.108716in}}%
\pgfpathlineto{\pgfqpoint{3.264658in}{4.938051in}}%
\pgfpathlineto{\pgfqpoint{3.300327in}{4.708308in}}%
\pgfpathlineto{\pgfqpoint{3.332859in}{4.832093in}}%
\pgfpathlineto{\pgfqpoint{3.324253in}{4.855961in}}%
\pgfpathlineto{\pgfqpoint{3.314742in}{4.987284in}}%
\pgfpathlineto{\pgfqpoint{3.280855in}{5.020009in}}%
\pgfpathlineto{\pgfqpoint{3.244636in}{5.301262in}}%
\pgfpathclose%
\pgfusepath{fill}%
\end{pgfscope}%
\begin{pgfscope}%
\pgfpathrectangle{\pgfqpoint{1.020000in}{0.880000in}}{\pgfqpoint{6.160000in}{6.160000in}}%
\pgfusepath{clip}%
\pgfsetbuttcap%
\pgfsetroundjoin%
\definecolor{currentfill}{rgb}{0.810616,0.268797,0.235428}%
\pgfsetfillcolor{currentfill}%
\pgfsetlinewidth{0.000000pt}%
\definecolor{currentstroke}{rgb}{0.000000,0.000000,0.000000}%
\pgfsetstrokecolor{currentstroke}%
\pgfsetdash{}{0pt}%
\pgfpathmoveto{\pgfqpoint{3.162452in}{5.116909in}}%
\pgfpathlineto{\pgfqpoint{3.171039in}{5.082415in}}%
\pgfpathlineto{\pgfqpoint{3.179746in}{5.037026in}}%
\pgfpathlineto{\pgfqpoint{3.211758in}{5.210881in}}%
\pgfpathlineto{\pgfqpoint{3.244636in}{5.301262in}}%
\pgfpathlineto{\pgfqpoint{3.239448in}{4.961983in}}%
\pgfpathlineto{\pgfqpoint{3.229803in}{5.105822in}}%
\pgfpathlineto{\pgfqpoint{3.195508in}{5.176320in}}%
\pgfpathlineto{\pgfqpoint{3.162452in}{5.116909in}}%
\pgfpathclose%
\pgfusepath{fill}%
\end{pgfscope}%
\begin{pgfscope}%
\pgfpathrectangle{\pgfqpoint{1.020000in}{0.880000in}}{\pgfqpoint{6.160000in}{6.160000in}}%
\pgfusepath{clip}%
\pgfsetbuttcap%
\pgfsetroundjoin%
\definecolor{currentfill}{rgb}{0.758112,0.168122,0.188827}%
\pgfsetfillcolor{currentfill}%
\pgfsetlinewidth{0.000000pt}%
\definecolor{currentstroke}{rgb}{0.000000,0.000000,0.000000}%
\pgfsetstrokecolor{currentstroke}%
\pgfsetdash{}{0pt}%
\pgfpathmoveto{\pgfqpoint{3.009776in}{5.270802in}}%
\pgfpathlineto{\pgfqpoint{3.019629in}{5.116764in}}%
\pgfpathlineto{\pgfqpoint{3.027234in}{5.159916in}}%
\pgfpathlineto{\pgfqpoint{3.060229in}{5.226208in}}%
\pgfpathlineto{\pgfqpoint{3.093965in}{5.225462in}}%
\pgfpathlineto{\pgfqpoint{3.084977in}{5.301398in}}%
\pgfpathlineto{\pgfqpoint{3.076989in}{5.285604in}}%
\pgfpathlineto{\pgfqpoint{3.044677in}{5.163244in}}%
\pgfpathlineto{\pgfqpoint{3.009776in}{5.270802in}}%
\pgfpathclose%
\pgfusepath{fill}%
\end{pgfscope}%
\begin{pgfscope}%
\pgfpathrectangle{\pgfqpoint{1.020000in}{0.880000in}}{\pgfqpoint{6.160000in}{6.160000in}}%
\pgfusepath{clip}%
\pgfsetbuttcap%
\pgfsetroundjoin%
\definecolor{currentfill}{rgb}{0.943432,0.802276,0.729172}%
\pgfsetfillcolor{currentfill}%
\pgfsetlinewidth{0.000000pt}%
\definecolor{currentstroke}{rgb}{0.000000,0.000000,0.000000}%
\pgfsetstrokecolor{currentstroke}%
\pgfsetdash{}{0pt}%
\pgfpathmoveto{\pgfqpoint{3.522401in}{4.265397in}}%
\pgfpathlineto{\pgfqpoint{3.530399in}{4.360247in}}%
\pgfpathlineto{\pgfqpoint{3.540057in}{4.200505in}}%
\pgfpathlineto{\pgfqpoint{3.573371in}{4.242036in}}%
\pgfpathlineto{\pgfqpoint{3.607169in}{4.200146in}}%
\pgfpathlineto{\pgfqpoint{3.597334in}{4.397443in}}%
\pgfpathlineto{\pgfqpoint{3.589702in}{4.218099in}}%
\pgfpathlineto{\pgfqpoint{3.554047in}{4.566836in}}%
\pgfpathlineto{\pgfqpoint{3.522401in}{4.265397in}}%
\pgfpathclose%
\pgfusepath{fill}%
\end{pgfscope}%
\begin{pgfscope}%
\pgfpathrectangle{\pgfqpoint{1.020000in}{0.880000in}}{\pgfqpoint{6.160000in}{6.160000in}}%
\pgfusepath{clip}%
\pgfsetbuttcap%
\pgfsetroundjoin%
\definecolor{currentfill}{rgb}{0.835345,0.860514,0.898970}%
\pgfsetfillcolor{currentfill}%
\pgfsetlinewidth{0.000000pt}%
\definecolor{currentstroke}{rgb}{0.000000,0.000000,0.000000}%
\pgfsetstrokecolor{currentstroke}%
\pgfsetdash{}{0pt}%
\pgfpathmoveto{\pgfqpoint{3.692291in}{4.092616in}}%
\pgfpathlineto{\pgfqpoint{3.701872in}{3.929843in}}%
\pgfpathlineto{\pgfqpoint{3.710317in}{4.006597in}}%
\pgfpathlineto{\pgfqpoint{3.743501in}{4.090939in}}%
\pgfpathlineto{\pgfqpoint{3.777669in}{3.940639in}}%
\pgfpathlineto{\pgfqpoint{3.767882in}{4.162471in}}%
\pgfpathlineto{\pgfqpoint{3.759857in}{3.960979in}}%
\pgfpathlineto{\pgfqpoint{3.727016in}{3.826548in}}%
\pgfpathlineto{\pgfqpoint{3.692291in}{4.092616in}}%
\pgfpathclose%
\pgfusepath{fill}%
\end{pgfscope}%
\begin{pgfscope}%
\pgfpathrectangle{\pgfqpoint{1.020000in}{0.880000in}}{\pgfqpoint{6.160000in}{6.160000in}}%
\pgfusepath{clip}%
\pgfsetbuttcap%
\pgfsetroundjoin%
\definecolor{currentfill}{rgb}{0.763363,0.835092,0.955658}%
\pgfsetfillcolor{currentfill}%
\pgfsetlinewidth{0.000000pt}%
\definecolor{currentstroke}{rgb}{0.000000,0.000000,0.000000}%
\pgfsetstrokecolor{currentstroke}%
\pgfsetdash{}{0pt}%
\pgfpathmoveto{\pgfqpoint{3.777669in}{3.940639in}}%
\pgfpathlineto{\pgfqpoint{3.787092in}{3.805808in}}%
\pgfpathlineto{\pgfqpoint{3.796161in}{3.761614in}}%
\pgfpathlineto{\pgfqpoint{3.829096in}{3.921050in}}%
\pgfpathlineto{\pgfqpoint{3.862635in}{3.918552in}}%
\pgfpathlineto{\pgfqpoint{3.853801in}{3.875800in}}%
\pgfpathlineto{\pgfqpoint{3.845619in}{3.647530in}}%
\pgfpathlineto{\pgfqpoint{3.811239in}{3.919285in}}%
\pgfpathlineto{\pgfqpoint{3.777669in}{3.940639in}}%
\pgfpathclose%
\pgfusepath{fill}%
\end{pgfscope}%
\begin{pgfscope}%
\pgfpathrectangle{\pgfqpoint{1.020000in}{0.880000in}}{\pgfqpoint{6.160000in}{6.160000in}}%
\pgfusepath{clip}%
\pgfsetbuttcap%
\pgfsetroundjoin%
\definecolor{currentfill}{rgb}{0.368507,0.491141,0.905243}%
\pgfsetfillcolor{currentfill}%
\pgfsetlinewidth{0.000000pt}%
\definecolor{currentstroke}{rgb}{0.000000,0.000000,0.000000}%
\pgfsetstrokecolor{currentstroke}%
\pgfsetdash{}{0pt}%
\pgfpathmoveto{\pgfqpoint{5.784044in}{3.137470in}}%
\pgfpathlineto{\pgfqpoint{5.792844in}{3.000088in}}%
\pgfpathlineto{\pgfqpoint{5.806406in}{3.155726in}}%
\pgfpathlineto{\pgfqpoint{5.839210in}{3.147297in}}%
\pgfpathlineto{\pgfqpoint{5.872847in}{3.189586in}}%
\pgfpathlineto{\pgfqpoint{5.861010in}{3.146598in}}%
\pgfpathlineto{\pgfqpoint{5.850094in}{3.156705in}}%
\pgfpathlineto{\pgfqpoint{5.814461in}{2.987522in}}%
\pgfpathlineto{\pgfqpoint{5.784044in}{3.137470in}}%
\pgfpathclose%
\pgfusepath{fill}%
\end{pgfscope}%
\begin{pgfscope}%
\pgfpathrectangle{\pgfqpoint{1.020000in}{0.880000in}}{\pgfqpoint{6.160000in}{6.160000in}}%
\pgfusepath{clip}%
\pgfsetbuttcap%
\pgfsetroundjoin%
\definecolor{currentfill}{rgb}{0.688188,0.793178,0.988038}%
\pgfsetfillcolor{currentfill}%
\pgfsetlinewidth{0.000000pt}%
\definecolor{currentstroke}{rgb}{0.000000,0.000000,0.000000}%
\pgfsetstrokecolor{currentstroke}%
\pgfsetdash{}{0pt}%
\pgfpathmoveto{\pgfqpoint{3.862635in}{3.918552in}}%
\pgfpathlineto{\pgfqpoint{3.873086in}{3.448120in}}%
\pgfpathlineto{\pgfqpoint{3.881090in}{3.768780in}}%
\pgfpathlineto{\pgfqpoint{3.914677in}{3.745957in}}%
\pgfpathlineto{\pgfqpoint{3.948251in}{3.713631in}}%
\pgfpathlineto{\pgfqpoint{3.939035in}{3.778195in}}%
\pgfpathlineto{\pgfqpoint{3.930394in}{3.617540in}}%
\pgfpathlineto{\pgfqpoint{3.897033in}{3.607702in}}%
\pgfpathlineto{\pgfqpoint{3.862635in}{3.918552in}}%
\pgfpathclose%
\pgfusepath{fill}%
\end{pgfscope}%
\begin{pgfscope}%
\pgfpathrectangle{\pgfqpoint{1.020000in}{0.880000in}}{\pgfqpoint{6.160000in}{6.160000in}}%
\pgfusepath{clip}%
\pgfsetbuttcap%
\pgfsetroundjoin%
\definecolor{currentfill}{rgb}{0.956653,0.598034,0.477302}%
\pgfsetfillcolor{currentfill}%
\pgfsetlinewidth{0.000000pt}%
\definecolor{currentstroke}{rgb}{0.000000,0.000000,0.000000}%
\pgfsetstrokecolor{currentstroke}%
\pgfsetdash{}{0pt}%
\pgfpathmoveto{\pgfqpoint{2.411799in}{4.589046in}}%
\pgfpathlineto{\pgfqpoint{2.419532in}{4.572445in}}%
\pgfpathlineto{\pgfqpoint{2.424932in}{4.690949in}}%
\pgfpathlineto{\pgfqpoint{2.459200in}{4.670812in}}%
\pgfpathlineto{\pgfqpoint{2.490723in}{4.813143in}}%
\pgfpathlineto{\pgfqpoint{2.485353in}{4.683563in}}%
\pgfpathlineto{\pgfqpoint{2.474571in}{4.877449in}}%
\pgfpathlineto{\pgfqpoint{2.443673in}{4.702575in}}%
\pgfpathlineto{\pgfqpoint{2.411799in}{4.589046in}}%
\pgfpathclose%
\pgfusepath{fill}%
\end{pgfscope}%
\begin{pgfscope}%
\pgfpathrectangle{\pgfqpoint{1.020000in}{0.880000in}}{\pgfqpoint{6.160000in}{6.160000in}}%
\pgfusepath{clip}%
\pgfsetbuttcap%
\pgfsetroundjoin%
\definecolor{currentfill}{rgb}{0.313946,0.420052,0.854993}%
\pgfsetfillcolor{currentfill}%
\pgfsetlinewidth{0.000000pt}%
\definecolor{currentstroke}{rgb}{0.000000,0.000000,0.000000}%
\pgfsetstrokecolor{currentstroke}%
\pgfsetdash{}{0pt}%
\pgfpathmoveto{\pgfqpoint{5.935610in}{3.016282in}}%
\pgfpathlineto{\pgfqpoint{5.943977in}{2.857373in}}%
\pgfpathlineto{\pgfqpoint{5.957797in}{3.005060in}}%
\pgfpathlineto{\pgfqpoint{5.990380in}{2.988749in}}%
\pgfpathlineto{\pgfqpoint{5.978675in}{2.962649in}}%
\pgfpathlineto{\pgfqpoint{5.972811in}{3.260697in}}%
\pgfpathlineto{\pgfqpoint{5.935610in}{3.016282in}}%
\pgfpathclose%
\pgfusepath{fill}%
\end{pgfscope}%
\begin{pgfscope}%
\pgfpathrectangle{\pgfqpoint{1.020000in}{0.880000in}}{\pgfqpoint{6.160000in}{6.160000in}}%
\pgfusepath{clip}%
\pgfsetbuttcap%
\pgfsetroundjoin%
\definecolor{currentfill}{rgb}{0.348323,0.465711,0.888346}%
\pgfsetfillcolor{currentfill}%
\pgfsetlinewidth{0.000000pt}%
\definecolor{currentstroke}{rgb}{0.000000,0.000000,0.000000}%
\pgfsetstrokecolor{currentstroke}%
\pgfsetdash{}{0pt}%
\pgfpathmoveto{\pgfqpoint{5.212739in}{3.055840in}}%
\pgfpathlineto{\pgfqpoint{5.223228in}{3.063446in}}%
\pgfpathlineto{\pgfqpoint{5.233521in}{3.050271in}}%
\pgfpathlineto{\pgfqpoint{5.267226in}{3.106358in}}%
\pgfpathlineto{\pgfqpoint{5.300460in}{3.118861in}}%
\pgfpathlineto{\pgfqpoint{5.291193in}{3.230058in}}%
\pgfpathlineto{\pgfqpoint{5.277790in}{2.965322in}}%
\pgfpathlineto{\pgfqpoint{5.245416in}{3.022529in}}%
\pgfpathlineto{\pgfqpoint{5.212739in}{3.055840in}}%
\pgfpathclose%
\pgfusepath{fill}%
\end{pgfscope}%
\begin{pgfscope}%
\pgfpathrectangle{\pgfqpoint{1.020000in}{0.880000in}}{\pgfqpoint{6.160000in}{6.160000in}}%
\pgfusepath{clip}%
\pgfsetbuttcap%
\pgfsetroundjoin%
\definecolor{currentfill}{rgb}{0.343278,0.459354,0.884122}%
\pgfsetfillcolor{currentfill}%
\pgfsetlinewidth{0.000000pt}%
\definecolor{currentstroke}{rgb}{0.000000,0.000000,0.000000}%
\pgfsetstrokecolor{currentstroke}%
\pgfsetdash{}{0pt}%
\pgfpathmoveto{\pgfqpoint{5.716990in}{3.056501in}}%
\pgfpathlineto{\pgfqpoint{5.727827in}{3.049343in}}%
\pgfpathlineto{\pgfqpoint{5.737670in}{2.977083in}}%
\pgfpathlineto{\pgfqpoint{5.772124in}{3.073042in}}%
\pgfpathlineto{\pgfqpoint{5.806406in}{3.155726in}}%
\pgfpathlineto{\pgfqpoint{5.792844in}{3.000088in}}%
\pgfpathlineto{\pgfqpoint{5.784044in}{3.137470in}}%
\pgfpathlineto{\pgfqpoint{5.750562in}{3.100069in}}%
\pgfpathlineto{\pgfqpoint{5.716990in}{3.056501in}}%
\pgfpathclose%
\pgfusepath{fill}%
\end{pgfscope}%
\begin{pgfscope}%
\pgfpathrectangle{\pgfqpoint{1.020000in}{0.880000in}}{\pgfqpoint{6.160000in}{6.160000in}}%
\pgfusepath{clip}%
\pgfsetbuttcap%
\pgfsetroundjoin%
\definecolor{currentfill}{rgb}{0.494638,0.633022,0.978983}%
\pgfsetfillcolor{currentfill}%
\pgfsetlinewidth{0.000000pt}%
\definecolor{currentstroke}{rgb}{0.000000,0.000000,0.000000}%
\pgfsetstrokecolor{currentstroke}%
\pgfsetdash{}{0pt}%
\pgfpathmoveto{\pgfqpoint{4.471624in}{3.405616in}}%
\pgfpathlineto{\pgfqpoint{4.481061in}{3.341752in}}%
\pgfpathlineto{\pgfqpoint{4.490849in}{3.389669in}}%
\pgfpathlineto{\pgfqpoint{4.524407in}{3.452316in}}%
\pgfpathlineto{\pgfqpoint{4.556645in}{3.156759in}}%
\pgfpathlineto{\pgfqpoint{4.547730in}{3.368807in}}%
\pgfpathlineto{\pgfqpoint{4.537828in}{3.308506in}}%
\pgfpathlineto{\pgfqpoint{4.504699in}{3.340636in}}%
\pgfpathlineto{\pgfqpoint{4.471624in}{3.405616in}}%
\pgfpathclose%
\pgfusepath{fill}%
\end{pgfscope}%
\begin{pgfscope}%
\pgfpathrectangle{\pgfqpoint{1.020000in}{0.880000in}}{\pgfqpoint{6.160000in}{6.160000in}}%
\pgfusepath{clip}%
\pgfsetbuttcap%
\pgfsetroundjoin%
\definecolor{currentfill}{rgb}{0.409611,0.540759,0.935545}%
\pgfsetfillcolor{currentfill}%
\pgfsetlinewidth{0.000000pt}%
\definecolor{currentstroke}{rgb}{0.000000,0.000000,0.000000}%
\pgfsetstrokecolor{currentstroke}%
\pgfsetdash{}{0pt}%
\pgfpathmoveto{\pgfqpoint{4.556645in}{3.156759in}}%
\pgfpathlineto{\pgfqpoint{4.566860in}{3.288574in}}%
\pgfpathlineto{\pgfqpoint{4.576348in}{3.223996in}}%
\pgfpathlineto{\pgfqpoint{4.609683in}{3.228644in}}%
\pgfpathlineto{\pgfqpoint{4.642439in}{3.111399in}}%
\pgfpathlineto{\pgfqpoint{4.633102in}{3.211948in}}%
\pgfpathlineto{\pgfqpoint{4.623176in}{3.178773in}}%
\pgfpathlineto{\pgfqpoint{4.589820in}{3.143916in}}%
\pgfpathlineto{\pgfqpoint{4.556645in}{3.156759in}}%
\pgfpathclose%
\pgfusepath{fill}%
\end{pgfscope}%
\begin{pgfscope}%
\pgfpathrectangle{\pgfqpoint{1.020000in}{0.880000in}}{\pgfqpoint{6.160000in}{6.160000in}}%
\pgfusepath{clip}%
\pgfsetbuttcap%
\pgfsetroundjoin%
\definecolor{currentfill}{rgb}{0.516260,0.654498,0.986407}%
\pgfsetfillcolor{currentfill}%
\pgfsetlinewidth{0.000000pt}%
\definecolor{currentstroke}{rgb}{0.000000,0.000000,0.000000}%
\pgfsetstrokecolor{currentstroke}%
\pgfsetdash{}{0pt}%
\pgfpathmoveto{\pgfqpoint{4.252655in}{3.336301in}}%
\pgfpathlineto{\pgfqpoint{4.262142in}{3.467502in}}%
\pgfpathlineto{\pgfqpoint{4.271344in}{3.223977in}}%
\pgfpathlineto{\pgfqpoint{4.304750in}{3.277052in}}%
\pgfpathlineto{\pgfqpoint{4.338163in}{3.323787in}}%
\pgfpathlineto{\pgfqpoint{4.328868in}{3.442925in}}%
\pgfpathlineto{\pgfqpoint{4.319610in}{3.590111in}}%
\pgfpathlineto{\pgfqpoint{4.286041in}{3.406363in}}%
\pgfpathlineto{\pgfqpoint{4.252655in}{3.336301in}}%
\pgfpathclose%
\pgfusepath{fill}%
\end{pgfscope}%
\begin{pgfscope}%
\pgfpathrectangle{\pgfqpoint{1.020000in}{0.880000in}}{\pgfqpoint{6.160000in}{6.160000in}}%
\pgfusepath{clip}%
\pgfsetbuttcap%
\pgfsetroundjoin%
\definecolor{currentfill}{rgb}{0.962701,0.628218,0.507636}%
\pgfsetfillcolor{currentfill}%
\pgfsetlinewidth{0.000000pt}%
\definecolor{currentstroke}{rgb}{0.000000,0.000000,0.000000}%
\pgfsetstrokecolor{currentstroke}%
\pgfsetdash{}{0pt}%
\pgfpathmoveto{\pgfqpoint{3.349659in}{4.839336in}}%
\pgfpathlineto{\pgfqpoint{3.360107in}{4.594610in}}%
\pgfpathlineto{\pgfqpoint{3.367918in}{4.674502in}}%
\pgfpathlineto{\pgfqpoint{3.401570in}{4.676857in}}%
\pgfpathlineto{\pgfqpoint{3.435741in}{4.606533in}}%
\pgfpathlineto{\pgfqpoint{3.427198in}{4.609540in}}%
\pgfpathlineto{\pgfqpoint{3.419478in}{4.505796in}}%
\pgfpathlineto{\pgfqpoint{3.384714in}{4.662409in}}%
\pgfpathlineto{\pgfqpoint{3.349659in}{4.839336in}}%
\pgfpathclose%
\pgfusepath{fill}%
\end{pgfscope}%
\begin{pgfscope}%
\pgfpathrectangle{\pgfqpoint{1.020000in}{0.880000in}}{\pgfqpoint{6.160000in}{6.160000in}}%
\pgfusepath{clip}%
\pgfsetbuttcap%
\pgfsetroundjoin%
\definecolor{currentfill}{rgb}{0.936780,0.532750,0.418093}%
\pgfsetfillcolor{currentfill}%
\pgfsetlinewidth{0.000000pt}%
\definecolor{currentstroke}{rgb}{0.000000,0.000000,0.000000}%
\pgfsetstrokecolor{currentstroke}%
\pgfsetdash{}{0pt}%
\pgfpathmoveto{\pgfqpoint{2.474571in}{4.877449in}}%
\pgfpathlineto{\pgfqpoint{2.485353in}{4.683563in}}%
\pgfpathlineto{\pgfqpoint{2.490723in}{4.813143in}}%
\pgfpathlineto{\pgfqpoint{2.529034in}{4.545142in}}%
\pgfpathlineto{\pgfqpoint{2.558771in}{4.799750in}}%
\pgfpathlineto{\pgfqpoint{2.549769in}{4.886323in}}%
\pgfpathlineto{\pgfqpoint{2.543524in}{4.802645in}}%
\pgfpathlineto{\pgfqpoint{2.507986in}{4.905670in}}%
\pgfpathlineto{\pgfqpoint{2.474571in}{4.877449in}}%
\pgfpathclose%
\pgfusepath{fill}%
\end{pgfscope}%
\begin{pgfscope}%
\pgfpathrectangle{\pgfqpoint{1.020000in}{0.880000in}}{\pgfqpoint{6.160000in}{6.160000in}}%
\pgfusepath{clip}%
\pgfsetbuttcap%
\pgfsetroundjoin%
\definecolor{currentfill}{rgb}{0.763520,0.178667,0.193396}%
\pgfsetfillcolor{currentfill}%
\pgfsetlinewidth{0.000000pt}%
\definecolor{currentstroke}{rgb}{0.000000,0.000000,0.000000}%
\pgfsetstrokecolor{currentstroke}%
\pgfsetdash{}{0pt}%
\pgfpathmoveto{\pgfqpoint{2.943207in}{5.198272in}}%
\pgfpathlineto{\pgfqpoint{2.950911in}{5.224272in}}%
\pgfpathlineto{\pgfqpoint{2.959137in}{5.208684in}}%
\pgfpathlineto{\pgfqpoint{2.992777in}{5.220797in}}%
\pgfpathlineto{\pgfqpoint{3.027234in}{5.159916in}}%
\pgfpathlineto{\pgfqpoint{3.019629in}{5.116764in}}%
\pgfpathlineto{\pgfqpoint{3.009776in}{5.270802in}}%
\pgfpathlineto{\pgfqpoint{2.975928in}{5.281901in}}%
\pgfpathlineto{\pgfqpoint{2.943207in}{5.198272in}}%
\pgfpathclose%
\pgfusepath{fill}%
\end{pgfscope}%
\begin{pgfscope}%
\pgfpathrectangle{\pgfqpoint{1.020000in}{0.880000in}}{\pgfqpoint{6.160000in}{6.160000in}}%
\pgfusepath{clip}%
\pgfsetbuttcap%
\pgfsetroundjoin%
\definecolor{currentfill}{rgb}{0.404421,0.534643,0.932002}%
\pgfsetfillcolor{currentfill}%
\pgfsetlinewidth{0.000000pt}%
\definecolor{currentstroke}{rgb}{0.000000,0.000000,0.000000}%
\pgfsetstrokecolor{currentstroke}%
\pgfsetdash{}{0pt}%
\pgfpathmoveto{\pgfqpoint{4.776771in}{3.364365in}}%
\pgfpathlineto{\pgfqpoint{4.785031in}{3.081763in}}%
\pgfpathlineto{\pgfqpoint{4.796087in}{3.260487in}}%
\pgfpathlineto{\pgfqpoint{4.828608in}{3.144295in}}%
\pgfpathlineto{\pgfqpoint{4.862118in}{3.187703in}}%
\pgfpathlineto{\pgfqpoint{4.852108in}{3.191355in}}%
\pgfpathlineto{\pgfqpoint{4.841497in}{3.100993in}}%
\pgfpathlineto{\pgfqpoint{4.808360in}{3.100747in}}%
\pgfpathlineto{\pgfqpoint{4.776771in}{3.364365in}}%
\pgfpathclose%
\pgfusepath{fill}%
\end{pgfscope}%
\begin{pgfscope}%
\pgfpathrectangle{\pgfqpoint{1.020000in}{0.880000in}}{\pgfqpoint{6.160000in}{6.160000in}}%
\pgfusepath{clip}%
\pgfsetbuttcap%
\pgfsetroundjoin%
\definecolor{currentfill}{rgb}{0.581486,0.713451,0.998314}%
\pgfsetfillcolor{currentfill}%
\pgfsetlinewidth{0.000000pt}%
\definecolor{currentstroke}{rgb}{0.000000,0.000000,0.000000}%
\pgfsetstrokecolor{currentstroke}%
\pgfsetdash{}{0pt}%
\pgfpathmoveto{\pgfqpoint{4.100597in}{3.525192in}}%
\pgfpathlineto{\pgfqpoint{4.109781in}{3.589287in}}%
\pgfpathlineto{\pgfqpoint{4.119113in}{3.471477in}}%
\pgfpathlineto{\pgfqpoint{4.152581in}{3.377597in}}%
\pgfpathlineto{\pgfqpoint{4.185969in}{3.412897in}}%
\pgfpathlineto{\pgfqpoint{4.176657in}{3.510509in}}%
\pgfpathlineto{\pgfqpoint{4.167365in}{3.489733in}}%
\pgfpathlineto{\pgfqpoint{4.133925in}{3.652213in}}%
\pgfpathlineto{\pgfqpoint{4.100597in}{3.525192in}}%
\pgfpathclose%
\pgfusepath{fill}%
\end{pgfscope}%
\begin{pgfscope}%
\pgfpathrectangle{\pgfqpoint{1.020000in}{0.880000in}}{\pgfqpoint{6.160000in}{6.160000in}}%
\pgfusepath{clip}%
\pgfsetbuttcap%
\pgfsetroundjoin%
\definecolor{currentfill}{rgb}{0.947345,0.794696,0.716991}%
\pgfsetfillcolor{currentfill}%
\pgfsetlinewidth{0.000000pt}%
\definecolor{currentstroke}{rgb}{0.000000,0.000000,0.000000}%
\pgfsetstrokecolor{currentstroke}%
\pgfsetdash{}{0pt}%
\pgfpathmoveto{\pgfqpoint{3.455358in}{4.257352in}}%
\pgfpathlineto{\pgfqpoint{3.461969in}{4.528558in}}%
\pgfpathlineto{\pgfqpoint{3.472329in}{4.276841in}}%
\pgfpathlineto{\pgfqpoint{3.505657in}{4.321853in}}%
\pgfpathlineto{\pgfqpoint{3.540057in}{4.200505in}}%
\pgfpathlineto{\pgfqpoint{3.530399in}{4.360247in}}%
\pgfpathlineto{\pgfqpoint{3.522401in}{4.265397in}}%
\pgfpathlineto{\pgfqpoint{3.488389in}{4.333861in}}%
\pgfpathlineto{\pgfqpoint{3.455358in}{4.257352in}}%
\pgfpathclose%
\pgfusepath{fill}%
\end{pgfscope}%
\begin{pgfscope}%
\pgfpathrectangle{\pgfqpoint{1.020000in}{0.880000in}}{\pgfqpoint{6.160000in}{6.160000in}}%
\pgfusepath{clip}%
\pgfsetbuttcap%
\pgfsetroundjoin%
\definecolor{currentfill}{rgb}{0.843358,0.861820,0.890017}%
\pgfsetfillcolor{currentfill}%
\pgfsetlinewidth{0.000000pt}%
\definecolor{currentstroke}{rgb}{0.000000,0.000000,0.000000}%
\pgfsetstrokecolor{currentstroke}%
\pgfsetdash{}{0pt}%
\pgfpathmoveto{\pgfqpoint{3.624546in}{4.210609in}}%
\pgfpathlineto{\pgfqpoint{3.634460in}{3.996616in}}%
\pgfpathlineto{\pgfqpoint{3.643184in}{4.000697in}}%
\pgfpathlineto{\pgfqpoint{3.677023in}{3.951220in}}%
\pgfpathlineto{\pgfqpoint{3.710317in}{4.006597in}}%
\pgfpathlineto{\pgfqpoint{3.701872in}{3.929843in}}%
\pgfpathlineto{\pgfqpoint{3.692291in}{4.092616in}}%
\pgfpathlineto{\pgfqpoint{3.659649in}{3.921117in}}%
\pgfpathlineto{\pgfqpoint{3.624546in}{4.210609in}}%
\pgfpathclose%
\pgfusepath{fill}%
\end{pgfscope}%
\begin{pgfscope}%
\pgfpathrectangle{\pgfqpoint{1.020000in}{0.880000in}}{\pgfqpoint{6.160000in}{6.160000in}}%
\pgfusepath{clip}%
\pgfsetbuttcap%
\pgfsetroundjoin%
\definecolor{currentfill}{rgb}{0.363461,0.484784,0.901019}%
\pgfsetfillcolor{currentfill}%
\pgfsetlinewidth{0.000000pt}%
\definecolor{currentstroke}{rgb}{0.000000,0.000000,0.000000}%
\pgfsetstrokecolor{currentstroke}%
\pgfsetdash{}{0pt}%
\pgfpathmoveto{\pgfqpoint{5.432564in}{3.111426in}}%
\pgfpathlineto{\pgfqpoint{5.442874in}{3.084287in}}%
\pgfpathlineto{\pgfqpoint{5.454440in}{3.154424in}}%
\pgfpathlineto{\pgfqpoint{5.487830in}{3.178404in}}%
\pgfpathlineto{\pgfqpoint{5.519965in}{3.109164in}}%
\pgfpathlineto{\pgfqpoint{5.509247in}{3.110732in}}%
\pgfpathlineto{\pgfqpoint{5.497439in}{3.027976in}}%
\pgfpathlineto{\pgfqpoint{5.464425in}{3.023070in}}%
\pgfpathlineto{\pgfqpoint{5.432564in}{3.111426in}}%
\pgfpathclose%
\pgfusepath{fill}%
\end{pgfscope}%
\begin{pgfscope}%
\pgfpathrectangle{\pgfqpoint{1.020000in}{0.880000in}}{\pgfqpoint{6.160000in}{6.160000in}}%
\pgfusepath{clip}%
\pgfsetbuttcap%
\pgfsetroundjoin%
\definecolor{currentfill}{rgb}{0.333490,0.446265,0.874452}%
\pgfsetfillcolor{currentfill}%
\pgfsetlinewidth{0.000000pt}%
\definecolor{currentstroke}{rgb}{0.000000,0.000000,0.000000}%
\pgfsetstrokecolor{currentstroke}%
\pgfsetdash{}{0pt}%
\pgfpathmoveto{\pgfqpoint{5.364094in}{2.907920in}}%
\pgfpathlineto{\pgfqpoint{5.375513in}{2.979022in}}%
\pgfpathlineto{\pgfqpoint{5.385537in}{2.931598in}}%
\pgfpathlineto{\pgfqpoint{5.420190in}{3.061478in}}%
\pgfpathlineto{\pgfqpoint{5.454440in}{3.154424in}}%
\pgfpathlineto{\pgfqpoint{5.442874in}{3.084287in}}%
\pgfpathlineto{\pgfqpoint{5.432564in}{3.111426in}}%
\pgfpathlineto{\pgfqpoint{5.399905in}{3.140777in}}%
\pgfpathlineto{\pgfqpoint{5.364094in}{2.907920in}}%
\pgfpathclose%
\pgfusepath{fill}%
\end{pgfscope}%
\begin{pgfscope}%
\pgfpathrectangle{\pgfqpoint{1.020000in}{0.880000in}}{\pgfqpoint{6.160000in}{6.160000in}}%
\pgfusepath{clip}%
\pgfsetbuttcap%
\pgfsetroundjoin%
\definecolor{currentfill}{rgb}{0.790562,0.231397,0.216242}%
\pgfsetfillcolor{currentfill}%
\pgfsetlinewidth{0.000000pt}%
\definecolor{currentstroke}{rgb}{0.000000,0.000000,0.000000}%
\pgfsetstrokecolor{currentstroke}%
\pgfsetdash{}{0pt}%
\pgfpathmoveto{\pgfqpoint{3.093965in}{5.225462in}}%
\pgfpathlineto{\pgfqpoint{3.101574in}{5.279867in}}%
\pgfpathlineto{\pgfqpoint{3.111271in}{5.138671in}}%
\pgfpathlineto{\pgfqpoint{3.144361in}{5.202827in}}%
\pgfpathlineto{\pgfqpoint{3.179746in}{5.037026in}}%
\pgfpathlineto{\pgfqpoint{3.171039in}{5.082415in}}%
\pgfpathlineto{\pgfqpoint{3.162452in}{5.116909in}}%
\pgfpathlineto{\pgfqpoint{3.128196in}{5.175093in}}%
\pgfpathlineto{\pgfqpoint{3.093965in}{5.225462in}}%
\pgfpathclose%
\pgfusepath{fill}%
\end{pgfscope}%
\begin{pgfscope}%
\pgfpathrectangle{\pgfqpoint{1.020000in}{0.880000in}}{\pgfqpoint{6.160000in}{6.160000in}}%
\pgfusepath{clip}%
\pgfsetbuttcap%
\pgfsetroundjoin%
\definecolor{currentfill}{rgb}{0.373552,0.497499,0.909467}%
\pgfsetfillcolor{currentfill}%
\pgfsetlinewidth{0.000000pt}%
\definecolor{currentstroke}{rgb}{0.000000,0.000000,0.000000}%
\pgfsetstrokecolor{currentstroke}%
\pgfsetdash{}{0pt}%
\pgfpathmoveto{\pgfqpoint{4.928076in}{3.128737in}}%
\pgfpathlineto{\pgfqpoint{4.938804in}{3.209771in}}%
\pgfpathlineto{\pgfqpoint{4.948583in}{3.162100in}}%
\pgfpathlineto{\pgfqpoint{4.980164in}{2.956626in}}%
\pgfpathlineto{\pgfqpoint{5.014298in}{3.076157in}}%
\pgfpathlineto{\pgfqpoint{5.004517in}{3.126525in}}%
\pgfpathlineto{\pgfqpoint{4.994373in}{3.130828in}}%
\pgfpathlineto{\pgfqpoint{4.961579in}{3.174073in}}%
\pgfpathlineto{\pgfqpoint{4.928076in}{3.128737in}}%
\pgfpathclose%
\pgfusepath{fill}%
\end{pgfscope}%
\begin{pgfscope}%
\pgfpathrectangle{\pgfqpoint{1.020000in}{0.880000in}}{\pgfqpoint{6.160000in}{6.160000in}}%
\pgfusepath{clip}%
\pgfsetbuttcap%
\pgfsetroundjoin%
\definecolor{currentfill}{rgb}{0.353369,0.472069,0.892570}%
\pgfsetfillcolor{currentfill}%
\pgfsetlinewidth{0.000000pt}%
\definecolor{currentstroke}{rgb}{0.000000,0.000000,0.000000}%
\pgfsetstrokecolor{currentstroke}%
\pgfsetdash{}{0pt}%
\pgfpathmoveto{\pgfqpoint{5.651928in}{3.108735in}}%
\pgfpathlineto{\pgfqpoint{5.665326in}{3.277704in}}%
\pgfpathlineto{\pgfqpoint{5.672337in}{3.015484in}}%
\pgfpathlineto{\pgfqpoint{5.706718in}{3.107408in}}%
\pgfpathlineto{\pgfqpoint{5.737670in}{2.977083in}}%
\pgfpathlineto{\pgfqpoint{5.727827in}{3.049343in}}%
\pgfpathlineto{\pgfqpoint{5.716990in}{3.056501in}}%
\pgfpathlineto{\pgfqpoint{5.683818in}{3.038857in}}%
\pgfpathlineto{\pgfqpoint{5.651928in}{3.108735in}}%
\pgfpathclose%
\pgfusepath{fill}%
\end{pgfscope}%
\begin{pgfscope}%
\pgfpathrectangle{\pgfqpoint{1.020000in}{0.880000in}}{\pgfqpoint{6.160000in}{6.160000in}}%
\pgfusepath{clip}%
\pgfsetbuttcap%
\pgfsetroundjoin%
\definecolor{currentfill}{rgb}{0.313946,0.420052,0.854993}%
\pgfsetfillcolor{currentfill}%
\pgfsetlinewidth{0.000000pt}%
\definecolor{currentstroke}{rgb}{0.000000,0.000000,0.000000}%
\pgfsetstrokecolor{currentstroke}%
\pgfsetdash{}{0pt}%
\pgfpathmoveto{\pgfqpoint{5.300460in}{3.118861in}}%
\pgfpathlineto{\pgfqpoint{5.309844in}{3.017957in}}%
\pgfpathlineto{\pgfqpoint{5.319054in}{2.901765in}}%
\pgfpathlineto{\pgfqpoint{5.354141in}{3.073313in}}%
\pgfpathlineto{\pgfqpoint{5.385537in}{2.931598in}}%
\pgfpathlineto{\pgfqpoint{5.375513in}{2.979022in}}%
\pgfpathlineto{\pgfqpoint{5.364094in}{2.907920in}}%
\pgfpathlineto{\pgfqpoint{5.332850in}{3.059092in}}%
\pgfpathlineto{\pgfqpoint{5.300460in}{3.118861in}}%
\pgfpathclose%
\pgfusepath{fill}%
\end{pgfscope}%
\begin{pgfscope}%
\pgfpathrectangle{\pgfqpoint{1.020000in}{0.880000in}}{\pgfqpoint{6.160000in}{6.160000in}}%
\pgfusepath{clip}%
\pgfsetbuttcap%
\pgfsetroundjoin%
\definecolor{currentfill}{rgb}{0.763520,0.178667,0.193396}%
\pgfsetfillcolor{currentfill}%
\pgfsetlinewidth{0.000000pt}%
\definecolor{currentstroke}{rgb}{0.000000,0.000000,0.000000}%
\pgfsetstrokecolor{currentstroke}%
\pgfsetdash{}{0pt}%
\pgfpathmoveto{\pgfqpoint{2.876893in}{5.107271in}}%
\pgfpathlineto{\pgfqpoint{2.883817in}{5.187268in}}%
\pgfpathlineto{\pgfqpoint{2.890624in}{5.278917in}}%
\pgfpathlineto{\pgfqpoint{2.924805in}{5.251618in}}%
\pgfpathlineto{\pgfqpoint{2.959137in}{5.208684in}}%
\pgfpathlineto{\pgfqpoint{2.950911in}{5.224272in}}%
\pgfpathlineto{\pgfqpoint{2.943207in}{5.198272in}}%
\pgfpathlineto{\pgfqpoint{2.909736in}{5.177421in}}%
\pgfpathlineto{\pgfqpoint{2.876893in}{5.107271in}}%
\pgfpathclose%
\pgfusepath{fill}%
\end{pgfscope}%
\begin{pgfscope}%
\pgfpathrectangle{\pgfqpoint{1.020000in}{0.880000in}}{\pgfqpoint{6.160000in}{6.160000in}}%
\pgfusepath{clip}%
\pgfsetbuttcap%
\pgfsetroundjoin%
\definecolor{currentfill}{rgb}{0.656683,0.771806,0.994914}%
\pgfsetfillcolor{currentfill}%
\pgfsetlinewidth{0.000000pt}%
\definecolor{currentstroke}{rgb}{0.000000,0.000000,0.000000}%
\pgfsetstrokecolor{currentstroke}%
\pgfsetdash{}{0pt}%
\pgfpathmoveto{\pgfqpoint{3.948251in}{3.713631in}}%
\pgfpathlineto{\pgfqpoint{3.957445in}{3.662261in}}%
\pgfpathlineto{\pgfqpoint{3.966514in}{3.673296in}}%
\pgfpathlineto{\pgfqpoint{4.000159in}{3.604892in}}%
\pgfpathlineto{\pgfqpoint{4.033723in}{3.546547in}}%
\pgfpathlineto{\pgfqpoint{4.024497in}{3.595888in}}%
\pgfpathlineto{\pgfqpoint{4.015309in}{3.628660in}}%
\pgfpathlineto{\pgfqpoint{3.981810in}{3.668262in}}%
\pgfpathlineto{\pgfqpoint{3.948251in}{3.713631in}}%
\pgfpathclose%
\pgfusepath{fill}%
\end{pgfscope}%
\begin{pgfscope}%
\pgfpathrectangle{\pgfqpoint{1.020000in}{0.880000in}}{\pgfqpoint{6.160000in}{6.160000in}}%
\pgfusepath{clip}%
\pgfsetbuttcap%
\pgfsetroundjoin%
\definecolor{currentfill}{rgb}{0.915157,0.476927,0.372179}%
\pgfsetfillcolor{currentfill}%
\pgfsetlinewidth{0.000000pt}%
\definecolor{currentstroke}{rgb}{0.000000,0.000000,0.000000}%
\pgfsetstrokecolor{currentstroke}%
\pgfsetdash{}{0pt}%
\pgfpathmoveto{\pgfqpoint{3.264658in}{4.938051in}}%
\pgfpathlineto{\pgfqpoint{3.272724in}{4.971366in}}%
\pgfpathlineto{\pgfqpoint{3.281402in}{4.938327in}}%
\pgfpathlineto{\pgfqpoint{3.316677in}{4.757867in}}%
\pgfpathlineto{\pgfqpoint{3.349659in}{4.839336in}}%
\pgfpathlineto{\pgfqpoint{3.340465in}{4.930254in}}%
\pgfpathlineto{\pgfqpoint{3.332859in}{4.832093in}}%
\pgfpathlineto{\pgfqpoint{3.300327in}{4.708308in}}%
\pgfpathlineto{\pgfqpoint{3.264658in}{4.938051in}}%
\pgfpathclose%
\pgfusepath{fill}%
\end{pgfscope}%
\begin{pgfscope}%
\pgfpathrectangle{\pgfqpoint{1.020000in}{0.880000in}}{\pgfqpoint{6.160000in}{6.160000in}}%
\pgfusepath{clip}%
\pgfsetbuttcap%
\pgfsetroundjoin%
\definecolor{currentfill}{rgb}{0.899534,0.440692,0.344107}%
\pgfsetfillcolor{currentfill}%
\pgfsetlinewidth{0.000000pt}%
\definecolor{currentstroke}{rgb}{0.000000,0.000000,0.000000}%
\pgfsetstrokecolor{currentstroke}%
\pgfsetdash{}{0pt}%
\pgfpathmoveto{\pgfqpoint{2.543524in}{4.802645in}}%
\pgfpathlineto{\pgfqpoint{2.549769in}{4.886323in}}%
\pgfpathlineto{\pgfqpoint{2.558771in}{4.799750in}}%
\pgfpathlineto{\pgfqpoint{2.590715in}{4.923279in}}%
\pgfpathlineto{\pgfqpoint{2.626765in}{4.783493in}}%
\pgfpathlineto{\pgfqpoint{2.617046in}{4.916052in}}%
\pgfpathlineto{\pgfqpoint{2.606049in}{5.130782in}}%
\pgfpathlineto{\pgfqpoint{2.573590in}{5.039420in}}%
\pgfpathlineto{\pgfqpoint{2.543524in}{4.802645in}}%
\pgfpathclose%
\pgfusepath{fill}%
\end{pgfscope}%
\begin{pgfscope}%
\pgfpathrectangle{\pgfqpoint{1.020000in}{0.880000in}}{\pgfqpoint{6.160000in}{6.160000in}}%
\pgfusepath{clip}%
\pgfsetbuttcap%
\pgfsetroundjoin%
\definecolor{currentfill}{rgb}{0.527132,0.664700,0.989065}%
\pgfsetfillcolor{currentfill}%
\pgfsetlinewidth{0.000000pt}%
\definecolor{currentstroke}{rgb}{0.000000,0.000000,0.000000}%
\pgfsetstrokecolor{currentstroke}%
\pgfsetdash{}{0pt}%
\pgfpathmoveto{\pgfqpoint{4.185969in}{3.412897in}}%
\pgfpathlineto{\pgfqpoint{4.195297in}{3.435698in}}%
\pgfpathlineto{\pgfqpoint{4.204630in}{3.365781in}}%
\pgfpathlineto{\pgfqpoint{4.238114in}{3.465302in}}%
\pgfpathlineto{\pgfqpoint{4.271344in}{3.223977in}}%
\pgfpathlineto{\pgfqpoint{4.262142in}{3.467502in}}%
\pgfpathlineto{\pgfqpoint{4.252655in}{3.336301in}}%
\pgfpathlineto{\pgfqpoint{4.219375in}{3.487038in}}%
\pgfpathlineto{\pgfqpoint{4.185969in}{3.412897in}}%
\pgfpathclose%
\pgfusepath{fill}%
\end{pgfscope}%
\begin{pgfscope}%
\pgfpathrectangle{\pgfqpoint{1.020000in}{0.880000in}}{\pgfqpoint{6.160000in}{6.160000in}}%
\pgfusepath{clip}%
\pgfsetbuttcap%
\pgfsetroundjoin%
\definecolor{currentfill}{rgb}{0.343278,0.459354,0.884122}%
\pgfsetfillcolor{currentfill}%
\pgfsetlinewidth{0.000000pt}%
\definecolor{currentstroke}{rgb}{0.000000,0.000000,0.000000}%
\pgfsetstrokecolor{currentstroke}%
\pgfsetdash{}{0pt}%
\pgfpathmoveto{\pgfqpoint{5.872847in}{3.189586in}}%
\pgfpathlineto{\pgfqpoint{5.883083in}{3.136694in}}%
\pgfpathlineto{\pgfqpoint{5.892077in}{3.010767in}}%
\pgfpathlineto{\pgfqpoint{5.926109in}{3.074304in}}%
\pgfpathlineto{\pgfqpoint{5.957797in}{3.005060in}}%
\pgfpathlineto{\pgfqpoint{5.943977in}{2.857373in}}%
\pgfpathlineto{\pgfqpoint{5.935610in}{3.016282in}}%
\pgfpathlineto{\pgfqpoint{5.906229in}{3.216577in}}%
\pgfpathlineto{\pgfqpoint{5.872847in}{3.189586in}}%
\pgfpathclose%
\pgfusepath{fill}%
\end{pgfscope}%
\begin{pgfscope}%
\pgfpathrectangle{\pgfqpoint{1.020000in}{0.880000in}}{\pgfqpoint{6.160000in}{6.160000in}}%
\pgfusepath{clip}%
\pgfsetbuttcap%
\pgfsetroundjoin%
\definecolor{currentfill}{rgb}{0.516260,0.654498,0.986407}%
\pgfsetfillcolor{currentfill}%
\pgfsetlinewidth{0.000000pt}%
\definecolor{currentstroke}{rgb}{0.000000,0.000000,0.000000}%
\pgfsetstrokecolor{currentstroke}%
\pgfsetdash{}{0pt}%
\pgfpathmoveto{\pgfqpoint{4.404802in}{3.318064in}}%
\pgfpathlineto{\pgfqpoint{4.414476in}{3.369969in}}%
\pgfpathlineto{\pgfqpoint{4.424034in}{3.361728in}}%
\pgfpathlineto{\pgfqpoint{4.457492in}{3.392283in}}%
\pgfpathlineto{\pgfqpoint{4.490849in}{3.389669in}}%
\pgfpathlineto{\pgfqpoint{4.481061in}{3.341752in}}%
\pgfpathlineto{\pgfqpoint{4.471624in}{3.405616in}}%
\pgfpathlineto{\pgfqpoint{4.438369in}{3.424695in}}%
\pgfpathlineto{\pgfqpoint{4.404802in}{3.318064in}}%
\pgfpathclose%
\pgfusepath{fill}%
\end{pgfscope}%
\begin{pgfscope}%
\pgfpathrectangle{\pgfqpoint{1.020000in}{0.880000in}}{\pgfqpoint{6.160000in}{6.160000in}}%
\pgfusepath{clip}%
\pgfsetbuttcap%
\pgfsetroundjoin%
\definecolor{currentfill}{rgb}{0.414801,0.546874,0.939088}%
\pgfsetfillcolor{currentfill}%
\pgfsetlinewidth{0.000000pt}%
\definecolor{currentstroke}{rgb}{0.000000,0.000000,0.000000}%
\pgfsetstrokecolor{currentstroke}%
\pgfsetdash{}{0pt}%
\pgfpathmoveto{\pgfqpoint{4.709123in}{3.154882in}}%
\pgfpathlineto{\pgfqpoint{4.719274in}{3.205327in}}%
\pgfpathlineto{\pgfqpoint{4.728287in}{3.041990in}}%
\pgfpathlineto{\pgfqpoint{4.762526in}{3.214280in}}%
\pgfpathlineto{\pgfqpoint{4.796087in}{3.260487in}}%
\pgfpathlineto{\pgfqpoint{4.785031in}{3.081763in}}%
\pgfpathlineto{\pgfqpoint{4.776771in}{3.364365in}}%
\pgfpathlineto{\pgfqpoint{4.742532in}{3.190061in}}%
\pgfpathlineto{\pgfqpoint{4.709123in}{3.154882in}}%
\pgfpathclose%
\pgfusepath{fill}%
\end{pgfscope}%
\begin{pgfscope}%
\pgfpathrectangle{\pgfqpoint{1.020000in}{0.880000in}}{\pgfqpoint{6.160000in}{6.160000in}}%
\pgfusepath{clip}%
\pgfsetbuttcap%
\pgfsetroundjoin%
\definecolor{currentfill}{rgb}{0.378598,0.503856,0.913692}%
\pgfsetfillcolor{currentfill}%
\pgfsetlinewidth{0.000000pt}%
\definecolor{currentstroke}{rgb}{0.000000,0.000000,0.000000}%
\pgfsetstrokecolor{currentstroke}%
\pgfsetdash{}{0pt}%
\pgfpathmoveto{\pgfqpoint{5.147438in}{3.141039in}}%
\pgfpathlineto{\pgfqpoint{5.158183in}{3.181035in}}%
\pgfpathlineto{\pgfqpoint{5.167330in}{3.056199in}}%
\pgfpathlineto{\pgfqpoint{5.202609in}{3.266073in}}%
\pgfpathlineto{\pgfqpoint{5.233521in}{3.050271in}}%
\pgfpathlineto{\pgfqpoint{5.223228in}{3.063446in}}%
\pgfpathlineto{\pgfqpoint{5.212739in}{3.055840in}}%
\pgfpathlineto{\pgfqpoint{5.180747in}{3.161897in}}%
\pgfpathlineto{\pgfqpoint{5.147438in}{3.141039in}}%
\pgfpathclose%
\pgfusepath{fill}%
\end{pgfscope}%
\begin{pgfscope}%
\pgfpathrectangle{\pgfqpoint{1.020000in}{0.880000in}}{\pgfqpoint{6.160000in}{6.160000in}}%
\pgfusepath{clip}%
\pgfsetbuttcap%
\pgfsetroundjoin%
\definecolor{currentfill}{rgb}{0.865391,0.371128,0.295769}%
\pgfsetfillcolor{currentfill}%
\pgfsetlinewidth{0.000000pt}%
\definecolor{currentstroke}{rgb}{0.000000,0.000000,0.000000}%
\pgfsetstrokecolor{currentstroke}%
\pgfsetdash{}{0pt}%
\pgfpathmoveto{\pgfqpoint{2.606049in}{5.130782in}}%
\pgfpathlineto{\pgfqpoint{2.617046in}{4.916052in}}%
\pgfpathlineto{\pgfqpoint{2.626765in}{4.783493in}}%
\pgfpathlineto{\pgfqpoint{2.656748in}{5.040050in}}%
\pgfpathlineto{\pgfqpoint{2.692136in}{4.938998in}}%
\pgfpathlineto{\pgfqpoint{2.680986in}{5.169138in}}%
\pgfpathlineto{\pgfqpoint{2.675416in}{5.022121in}}%
\pgfpathlineto{\pgfqpoint{2.642508in}{4.961293in}}%
\pgfpathlineto{\pgfqpoint{2.606049in}{5.130782in}}%
\pgfpathclose%
\pgfusepath{fill}%
\end{pgfscope}%
\begin{pgfscope}%
\pgfpathrectangle{\pgfqpoint{1.020000in}{0.880000in}}{\pgfqpoint{6.160000in}{6.160000in}}%
\pgfusepath{clip}%
\pgfsetbuttcap%
\pgfsetroundjoin%
\definecolor{currentfill}{rgb}{0.728970,0.817464,0.973188}%
\pgfsetfillcolor{currentfill}%
\pgfsetlinewidth{0.000000pt}%
\definecolor{currentstroke}{rgb}{0.000000,0.000000,0.000000}%
\pgfsetstrokecolor{currentstroke}%
\pgfsetdash{}{0pt}%
\pgfpathmoveto{\pgfqpoint{3.796161in}{3.761614in}}%
\pgfpathlineto{\pgfqpoint{3.804592in}{3.889368in}}%
\pgfpathlineto{\pgfqpoint{3.814458in}{3.638744in}}%
\pgfpathlineto{\pgfqpoint{3.847597in}{3.753076in}}%
\pgfpathlineto{\pgfqpoint{3.881090in}{3.768780in}}%
\pgfpathlineto{\pgfqpoint{3.873086in}{3.448120in}}%
\pgfpathlineto{\pgfqpoint{3.862635in}{3.918552in}}%
\pgfpathlineto{\pgfqpoint{3.829096in}{3.921050in}}%
\pgfpathlineto{\pgfqpoint{3.796161in}{3.761614in}}%
\pgfpathclose%
\pgfusepath{fill}%
\end{pgfscope}%
\begin{pgfscope}%
\pgfpathrectangle{\pgfqpoint{1.020000in}{0.880000in}}{\pgfqpoint{6.160000in}{6.160000in}}%
\pgfusepath{clip}%
\pgfsetbuttcap%
\pgfsetroundjoin%
\definecolor{currentfill}{rgb}{0.928116,0.822197,0.765141}%
\pgfsetfillcolor{currentfill}%
\pgfsetlinewidth{0.000000pt}%
\definecolor{currentstroke}{rgb}{0.000000,0.000000,0.000000}%
\pgfsetstrokecolor{currentstroke}%
\pgfsetdash{}{0pt}%
\pgfpathmoveto{\pgfqpoint{3.540057in}{4.200505in}}%
\pgfpathlineto{\pgfqpoint{3.549246in}{4.113784in}}%
\pgfpathlineto{\pgfqpoint{3.556859in}{4.281240in}}%
\pgfpathlineto{\pgfqpoint{3.590628in}{4.262212in}}%
\pgfpathlineto{\pgfqpoint{3.624546in}{4.210609in}}%
\pgfpathlineto{\pgfqpoint{3.615366in}{4.291007in}}%
\pgfpathlineto{\pgfqpoint{3.607169in}{4.200146in}}%
\pgfpathlineto{\pgfqpoint{3.573371in}{4.242036in}}%
\pgfpathlineto{\pgfqpoint{3.540057in}{4.200505in}}%
\pgfpathclose%
\pgfusepath{fill}%
\end{pgfscope}%
\begin{pgfscope}%
\pgfpathrectangle{\pgfqpoint{1.020000in}{0.880000in}}{\pgfqpoint{6.160000in}{6.160000in}}%
\pgfusepath{clip}%
\pgfsetbuttcap%
\pgfsetroundjoin%
\definecolor{currentfill}{rgb}{0.804965,0.851666,0.926165}%
\pgfsetfillcolor{currentfill}%
\pgfsetlinewidth{0.000000pt}%
\definecolor{currentstroke}{rgb}{0.000000,0.000000,0.000000}%
\pgfsetstrokecolor{currentstroke}%
\pgfsetdash{}{0pt}%
\pgfpathmoveto{\pgfqpoint{3.710317in}{4.006597in}}%
\pgfpathlineto{\pgfqpoint{3.719277in}{3.978712in}}%
\pgfpathlineto{\pgfqpoint{3.727976in}{4.011567in}}%
\pgfpathlineto{\pgfqpoint{3.762533in}{3.788181in}}%
\pgfpathlineto{\pgfqpoint{3.796161in}{3.761614in}}%
\pgfpathlineto{\pgfqpoint{3.787092in}{3.805808in}}%
\pgfpathlineto{\pgfqpoint{3.777669in}{3.940639in}}%
\pgfpathlineto{\pgfqpoint{3.743501in}{4.090939in}}%
\pgfpathlineto{\pgfqpoint{3.710317in}{4.006597in}}%
\pgfpathclose%
\pgfusepath{fill}%
\end{pgfscope}%
\begin{pgfscope}%
\pgfpathrectangle{\pgfqpoint{1.020000in}{0.880000in}}{\pgfqpoint{6.160000in}{6.160000in}}%
\pgfusepath{clip}%
\pgfsetbuttcap%
\pgfsetroundjoin%
\definecolor{currentfill}{rgb}{0.353369,0.472069,0.892570}%
\pgfsetfillcolor{currentfill}%
\pgfsetlinewidth{0.000000pt}%
\definecolor{currentstroke}{rgb}{0.000000,0.000000,0.000000}%
\pgfsetstrokecolor{currentstroke}%
\pgfsetdash{}{0pt}%
\pgfpathmoveto{\pgfqpoint{5.079614in}{2.967373in}}%
\pgfpathlineto{\pgfqpoint{5.090807in}{3.068579in}}%
\pgfpathlineto{\pgfqpoint{5.101004in}{3.056927in}}%
\pgfpathlineto{\pgfqpoint{5.134099in}{3.048399in}}%
\pgfpathlineto{\pgfqpoint{5.167330in}{3.056199in}}%
\pgfpathlineto{\pgfqpoint{5.158183in}{3.181035in}}%
\pgfpathlineto{\pgfqpoint{5.147438in}{3.141039in}}%
\pgfpathlineto{\pgfqpoint{5.113862in}{3.092253in}}%
\pgfpathlineto{\pgfqpoint{5.079614in}{2.967373in}}%
\pgfpathclose%
\pgfusepath{fill}%
\end{pgfscope}%
\begin{pgfscope}%
\pgfpathrectangle{\pgfqpoint{1.020000in}{0.880000in}}{\pgfqpoint{6.160000in}{6.160000in}}%
\pgfusepath{clip}%
\pgfsetbuttcap%
\pgfsetroundjoin%
\definecolor{currentfill}{rgb}{0.348323,0.465711,0.888346}%
\pgfsetfillcolor{currentfill}%
\pgfsetlinewidth{0.000000pt}%
\definecolor{currentstroke}{rgb}{0.000000,0.000000,0.000000}%
\pgfsetstrokecolor{currentstroke}%
\pgfsetdash{}{0pt}%
\pgfpathmoveto{\pgfqpoint{5.584943in}{3.034847in}}%
\pgfpathlineto{\pgfqpoint{5.597301in}{3.143389in}}%
\pgfpathlineto{\pgfqpoint{5.605511in}{2.958424in}}%
\pgfpathlineto{\pgfqpoint{5.637492in}{2.889028in}}%
\pgfpathlineto{\pgfqpoint{5.672337in}{3.015484in}}%
\pgfpathlineto{\pgfqpoint{5.665326in}{3.277704in}}%
\pgfpathlineto{\pgfqpoint{5.651928in}{3.108735in}}%
\pgfpathlineto{\pgfqpoint{5.619212in}{3.125885in}}%
\pgfpathlineto{\pgfqpoint{5.584943in}{3.034847in}}%
\pgfpathclose%
\pgfusepath{fill}%
\end{pgfscope}%
\begin{pgfscope}%
\pgfpathrectangle{\pgfqpoint{1.020000in}{0.880000in}}{\pgfqpoint{6.160000in}{6.160000in}}%
\pgfusepath{clip}%
\pgfsetbuttcap%
\pgfsetroundjoin%
\definecolor{currentfill}{rgb}{0.969522,0.700833,0.587508}%
\pgfsetfillcolor{currentfill}%
\pgfsetlinewidth{0.000000pt}%
\definecolor{currentstroke}{rgb}{0.000000,0.000000,0.000000}%
\pgfsetstrokecolor{currentstroke}%
\pgfsetdash{}{0pt}%
\pgfpathmoveto{\pgfqpoint{2.292345in}{4.513808in}}%
\pgfpathlineto{\pgfqpoint{2.298850in}{4.558324in}}%
\pgfpathlineto{\pgfqpoint{2.307992in}{4.461998in}}%
\pgfpathlineto{\pgfqpoint{2.342896in}{4.412093in}}%
\pgfpathlineto{\pgfqpoint{2.374839in}{4.524809in}}%
\pgfpathlineto{\pgfqpoint{2.367460in}{4.522621in}}%
\pgfpathlineto{\pgfqpoint{2.359562in}{4.550071in}}%
\pgfpathlineto{\pgfqpoint{2.326436in}{4.505854in}}%
\pgfpathlineto{\pgfqpoint{2.292345in}{4.513808in}}%
\pgfpathclose%
\pgfusepath{fill}%
\end{pgfscope}%
\begin{pgfscope}%
\pgfpathrectangle{\pgfqpoint{1.020000in}{0.880000in}}{\pgfqpoint{6.160000in}{6.160000in}}%
\pgfusepath{clip}%
\pgfsetbuttcap%
\pgfsetroundjoin%
\definecolor{currentfill}{rgb}{0.763520,0.178667,0.193396}%
\pgfsetfillcolor{currentfill}%
\pgfsetlinewidth{0.000000pt}%
\definecolor{currentstroke}{rgb}{0.000000,0.000000,0.000000}%
\pgfsetstrokecolor{currentstroke}%
\pgfsetdash{}{0pt}%
\pgfpathmoveto{\pgfqpoint{2.807991in}{5.210923in}}%
\pgfpathlineto{\pgfqpoint{2.816718in}{5.148476in}}%
\pgfpathlineto{\pgfqpoint{2.823694in}{5.217692in}}%
\pgfpathlineto{\pgfqpoint{2.857789in}{5.199846in}}%
\pgfpathlineto{\pgfqpoint{2.890624in}{5.278917in}}%
\pgfpathlineto{\pgfqpoint{2.883817in}{5.187268in}}%
\pgfpathlineto{\pgfqpoint{2.876893in}{5.107271in}}%
\pgfpathlineto{\pgfqpoint{2.841354in}{5.243644in}}%
\pgfpathlineto{\pgfqpoint{2.807991in}{5.210923in}}%
\pgfpathclose%
\pgfusepath{fill}%
\end{pgfscope}%
\begin{pgfscope}%
\pgfpathrectangle{\pgfqpoint{1.020000in}{0.880000in}}{\pgfqpoint{6.160000in}{6.160000in}}%
\pgfusepath{clip}%
\pgfsetbuttcap%
\pgfsetroundjoin%
\definecolor{currentfill}{rgb}{0.953054,0.585211,0.465373}%
\pgfsetfillcolor{currentfill}%
\pgfsetlinewidth{0.000000pt}%
\definecolor{currentstroke}{rgb}{0.000000,0.000000,0.000000}%
\pgfsetstrokecolor{currentstroke}%
\pgfsetdash{}{0pt}%
\pgfpathmoveto{\pgfqpoint{3.281402in}{4.938327in}}%
\pgfpathlineto{\pgfqpoint{3.291644in}{4.728924in}}%
\pgfpathlineto{\pgfqpoint{3.300755in}{4.647373in}}%
\pgfpathlineto{\pgfqpoint{3.335853in}{4.480829in}}%
\pgfpathlineto{\pgfqpoint{3.367918in}{4.674502in}}%
\pgfpathlineto{\pgfqpoint{3.360107in}{4.594610in}}%
\pgfpathlineto{\pgfqpoint{3.349659in}{4.839336in}}%
\pgfpathlineto{\pgfqpoint{3.316677in}{4.757867in}}%
\pgfpathlineto{\pgfqpoint{3.281402in}{4.938327in}}%
\pgfpathclose%
\pgfusepath{fill}%
\end{pgfscope}%
\begin{pgfscope}%
\pgfpathrectangle{\pgfqpoint{1.020000in}{0.880000in}}{\pgfqpoint{6.160000in}{6.160000in}}%
\pgfusepath{clip}%
\pgfsetbuttcap%
\pgfsetroundjoin%
\definecolor{currentfill}{rgb}{0.825294,0.295749,0.250025}%
\pgfsetfillcolor{currentfill}%
\pgfsetlinewidth{0.000000pt}%
\definecolor{currentstroke}{rgb}{0.000000,0.000000,0.000000}%
\pgfsetstrokecolor{currentstroke}%
\pgfsetdash{}{0pt}%
\pgfpathmoveto{\pgfqpoint{3.179746in}{5.037026in}}%
\pgfpathlineto{\pgfqpoint{3.187744in}{5.064917in}}%
\pgfpathlineto{\pgfqpoint{3.196879in}{4.978080in}}%
\pgfpathlineto{\pgfqpoint{3.230571in}{4.980903in}}%
\pgfpathlineto{\pgfqpoint{3.264658in}{4.938051in}}%
\pgfpathlineto{\pgfqpoint{3.254753in}{5.108716in}}%
\pgfpathlineto{\pgfqpoint{3.244636in}{5.301262in}}%
\pgfpathlineto{\pgfqpoint{3.211758in}{5.210881in}}%
\pgfpathlineto{\pgfqpoint{3.179746in}{5.037026in}}%
\pgfpathclose%
\pgfusepath{fill}%
\end{pgfscope}%
\begin{pgfscope}%
\pgfpathrectangle{\pgfqpoint{1.020000in}{0.880000in}}{\pgfqpoint{6.160000in}{6.160000in}}%
\pgfusepath{clip}%
\pgfsetbuttcap%
\pgfsetroundjoin%
\definecolor{currentfill}{rgb}{0.968500,0.673977,0.556649}%
\pgfsetfillcolor{currentfill}%
\pgfsetlinewidth{0.000000pt}%
\definecolor{currentstroke}{rgb}{0.000000,0.000000,0.000000}%
\pgfsetstrokecolor{currentstroke}%
\pgfsetdash{}{0pt}%
\pgfpathmoveto{\pgfqpoint{3.367918in}{4.674502in}}%
\pgfpathlineto{\pgfqpoint{3.377015in}{4.597004in}}%
\pgfpathlineto{\pgfqpoint{3.385736in}{4.567668in}}%
\pgfpathlineto{\pgfqpoint{3.419539in}{4.554208in}}%
\pgfpathlineto{\pgfqpoint{3.455358in}{4.257352in}}%
\pgfpathlineto{\pgfqpoint{3.444844in}{4.529349in}}%
\pgfpathlineto{\pgfqpoint{3.435741in}{4.606533in}}%
\pgfpathlineto{\pgfqpoint{3.401570in}{4.676857in}}%
\pgfpathlineto{\pgfqpoint{3.367918in}{4.674502in}}%
\pgfpathclose%
\pgfusepath{fill}%
\end{pgfscope}%
\begin{pgfscope}%
\pgfpathrectangle{\pgfqpoint{1.020000in}{0.880000in}}{\pgfqpoint{6.160000in}{6.160000in}}%
\pgfusepath{clip}%
\pgfsetbuttcap%
\pgfsetroundjoin%
\definecolor{currentfill}{rgb}{0.404421,0.534643,0.932002}%
\pgfsetfillcolor{currentfill}%
\pgfsetlinewidth{0.000000pt}%
\definecolor{currentstroke}{rgb}{0.000000,0.000000,0.000000}%
\pgfsetstrokecolor{currentstroke}%
\pgfsetdash{}{0pt}%
\pgfpathmoveto{\pgfqpoint{4.642439in}{3.111399in}}%
\pgfpathlineto{\pgfqpoint{4.652787in}{3.225674in}}%
\pgfpathlineto{\pgfqpoint{4.662577in}{3.214556in}}%
\pgfpathlineto{\pgfqpoint{4.695696in}{3.172055in}}%
\pgfpathlineto{\pgfqpoint{4.728287in}{3.041990in}}%
\pgfpathlineto{\pgfqpoint{4.719274in}{3.205327in}}%
\pgfpathlineto{\pgfqpoint{4.709123in}{3.154882in}}%
\pgfpathlineto{\pgfqpoint{4.676185in}{3.214512in}}%
\pgfpathlineto{\pgfqpoint{4.642439in}{3.111399in}}%
\pgfpathclose%
\pgfusepath{fill}%
\end{pgfscope}%
\begin{pgfscope}%
\pgfpathrectangle{\pgfqpoint{1.020000in}{0.880000in}}{\pgfqpoint{6.160000in}{6.160000in}}%
\pgfusepath{clip}%
\pgfsetbuttcap%
\pgfsetroundjoin%
\definecolor{currentfill}{rgb}{0.967711,0.662973,0.544323}%
\pgfsetfillcolor{currentfill}%
\pgfsetlinewidth{0.000000pt}%
\definecolor{currentstroke}{rgb}{0.000000,0.000000,0.000000}%
\pgfsetstrokecolor{currentstroke}%
\pgfsetdash{}{0pt}%
\pgfpathmoveto{\pgfqpoint{2.359562in}{4.550071in}}%
\pgfpathlineto{\pgfqpoint{2.367460in}{4.522621in}}%
\pgfpathlineto{\pgfqpoint{2.374839in}{4.524809in}}%
\pgfpathlineto{\pgfqpoint{2.408896in}{4.520207in}}%
\pgfpathlineto{\pgfqpoint{2.440442in}{4.660163in}}%
\pgfpathlineto{\pgfqpoint{2.435042in}{4.538968in}}%
\pgfpathlineto{\pgfqpoint{2.424932in}{4.690949in}}%
\pgfpathlineto{\pgfqpoint{2.391935in}{4.637378in}}%
\pgfpathlineto{\pgfqpoint{2.359562in}{4.550071in}}%
\pgfpathclose%
\pgfusepath{fill}%
\end{pgfscope}%
\begin{pgfscope}%
\pgfpathrectangle{\pgfqpoint{1.020000in}{0.880000in}}{\pgfqpoint{6.160000in}{6.160000in}}%
\pgfusepath{clip}%
\pgfsetbuttcap%
\pgfsetroundjoin%
\definecolor{currentfill}{rgb}{0.613933,0.739923,0.999142}%
\pgfsetfillcolor{currentfill}%
\pgfsetlinewidth{0.000000pt}%
\definecolor{currentstroke}{rgb}{0.000000,0.000000,0.000000}%
\pgfsetstrokecolor{currentstroke}%
\pgfsetdash{}{0pt}%
\pgfpathmoveto{\pgfqpoint{4.033723in}{3.546547in}}%
\pgfpathlineto{\pgfqpoint{4.042837in}{3.585006in}}%
\pgfpathlineto{\pgfqpoint{4.052106in}{3.525255in}}%
\pgfpathlineto{\pgfqpoint{4.085609in}{3.517655in}}%
\pgfpathlineto{\pgfqpoint{4.119113in}{3.471477in}}%
\pgfpathlineto{\pgfqpoint{4.109781in}{3.589287in}}%
\pgfpathlineto{\pgfqpoint{4.100597in}{3.525192in}}%
\pgfpathlineto{\pgfqpoint{4.067062in}{3.631834in}}%
\pgfpathlineto{\pgfqpoint{4.033723in}{3.546547in}}%
\pgfpathclose%
\pgfusepath{fill}%
\end{pgfscope}%
\begin{pgfscope}%
\pgfpathrectangle{\pgfqpoint{1.020000in}{0.880000in}}{\pgfqpoint{6.160000in}{6.160000in}}%
\pgfusepath{clip}%
\pgfsetbuttcap%
\pgfsetroundjoin%
\definecolor{currentfill}{rgb}{0.815508,0.277781,0.240294}%
\pgfsetfillcolor{currentfill}%
\pgfsetlinewidth{0.000000pt}%
\definecolor{currentstroke}{rgb}{0.000000,0.000000,0.000000}%
\pgfsetstrokecolor{currentstroke}%
\pgfsetdash{}{0pt}%
\pgfpathmoveto{\pgfqpoint{2.675416in}{5.022121in}}%
\pgfpathlineto{\pgfqpoint{2.680986in}{5.169138in}}%
\pgfpathlineto{\pgfqpoint{2.692136in}{4.938998in}}%
\pgfpathlineto{\pgfqpoint{2.724181in}{5.064682in}}%
\pgfpathlineto{\pgfqpoint{2.758087in}{5.062286in}}%
\pgfpathlineto{\pgfqpoint{2.747960in}{5.225365in}}%
\pgfpathlineto{\pgfqpoint{2.741719in}{5.113751in}}%
\pgfpathlineto{\pgfqpoint{2.707173in}{5.163670in}}%
\pgfpathlineto{\pgfqpoint{2.675416in}{5.022121in}}%
\pgfpathclose%
\pgfusepath{fill}%
\end{pgfscope}%
\begin{pgfscope}%
\pgfpathrectangle{\pgfqpoint{1.020000in}{0.880000in}}{\pgfqpoint{6.160000in}{6.160000in}}%
\pgfusepath{clip}%
\pgfsetbuttcap%
\pgfsetroundjoin%
\definecolor{currentfill}{rgb}{0.510824,0.649397,0.985079}%
\pgfsetfillcolor{currentfill}%
\pgfsetlinewidth{0.000000pt}%
\definecolor{currentstroke}{rgb}{0.000000,0.000000,0.000000}%
\pgfsetstrokecolor{currentstroke}%
\pgfsetdash{}{0pt}%
\pgfpathmoveto{\pgfqpoint{4.338163in}{3.323787in}}%
\pgfpathlineto{\pgfqpoint{4.347775in}{3.398752in}}%
\pgfpathlineto{\pgfqpoint{4.357163in}{3.325702in}}%
\pgfpathlineto{\pgfqpoint{4.390805in}{3.441481in}}%
\pgfpathlineto{\pgfqpoint{4.424034in}{3.361728in}}%
\pgfpathlineto{\pgfqpoint{4.414476in}{3.369969in}}%
\pgfpathlineto{\pgfqpoint{4.404802in}{3.318064in}}%
\pgfpathlineto{\pgfqpoint{4.371619in}{3.387329in}}%
\pgfpathlineto{\pgfqpoint{4.338163in}{3.323787in}}%
\pgfpathclose%
\pgfusepath{fill}%
\end{pgfscope}%
\begin{pgfscope}%
\pgfpathrectangle{\pgfqpoint{1.020000in}{0.880000in}}{\pgfqpoint{6.160000in}{6.160000in}}%
\pgfusepath{clip}%
\pgfsetbuttcap%
\pgfsetroundjoin%
\definecolor{currentfill}{rgb}{0.785153,0.220851,0.211673}%
\pgfsetfillcolor{currentfill}%
\pgfsetlinewidth{0.000000pt}%
\definecolor{currentstroke}{rgb}{0.000000,0.000000,0.000000}%
\pgfsetstrokecolor{currentstroke}%
\pgfsetdash{}{0pt}%
\pgfpathmoveto{\pgfqpoint{2.741719in}{5.113751in}}%
\pgfpathlineto{\pgfqpoint{2.747960in}{5.225365in}}%
\pgfpathlineto{\pgfqpoint{2.758087in}{5.062286in}}%
\pgfpathlineto{\pgfqpoint{2.791600in}{5.086990in}}%
\pgfpathlineto{\pgfqpoint{2.823694in}{5.217692in}}%
\pgfpathlineto{\pgfqpoint{2.816718in}{5.148476in}}%
\pgfpathlineto{\pgfqpoint{2.807991in}{5.210923in}}%
\pgfpathlineto{\pgfqpoint{2.774126in}{5.214515in}}%
\pgfpathlineto{\pgfqpoint{2.741719in}{5.113751in}}%
\pgfpathclose%
\pgfusepath{fill}%
\end{pgfscope}%
\begin{pgfscope}%
\pgfpathrectangle{\pgfqpoint{1.020000in}{0.880000in}}{\pgfqpoint{6.160000in}{6.160000in}}%
\pgfusepath{clip}%
\pgfsetbuttcap%
\pgfsetroundjoin%
\definecolor{currentfill}{rgb}{0.343278,0.459354,0.884122}%
\pgfsetfillcolor{currentfill}%
\pgfsetlinewidth{0.000000pt}%
\definecolor{currentstroke}{rgb}{0.000000,0.000000,0.000000}%
\pgfsetstrokecolor{currentstroke}%
\pgfsetdash{}{0pt}%
\pgfpathmoveto{\pgfqpoint{5.014298in}{3.076157in}}%
\pgfpathlineto{\pgfqpoint{5.023880in}{3.000818in}}%
\pgfpathlineto{\pgfqpoint{5.034997in}{3.104380in}}%
\pgfpathlineto{\pgfqpoint{5.067747in}{3.050168in}}%
\pgfpathlineto{\pgfqpoint{5.101004in}{3.056927in}}%
\pgfpathlineto{\pgfqpoint{5.090807in}{3.068579in}}%
\pgfpathlineto{\pgfqpoint{5.079614in}{2.967373in}}%
\pgfpathlineto{\pgfqpoint{5.047703in}{3.104696in}}%
\pgfpathlineto{\pgfqpoint{5.014298in}{3.076157in}}%
\pgfpathclose%
\pgfusepath{fill}%
\end{pgfscope}%
\begin{pgfscope}%
\pgfpathrectangle{\pgfqpoint{1.020000in}{0.880000in}}{\pgfqpoint{6.160000in}{6.160000in}}%
\pgfusepath{clip}%
\pgfsetbuttcap%
\pgfsetroundjoin%
\definecolor{currentfill}{rgb}{0.843703,0.330068,0.270065}%
\pgfsetfillcolor{currentfill}%
\pgfsetlinewidth{0.000000pt}%
\definecolor{currentstroke}{rgb}{0.000000,0.000000,0.000000}%
\pgfsetstrokecolor{currentstroke}%
\pgfsetdash{}{0pt}%
\pgfpathmoveto{\pgfqpoint{3.111271in}{5.138671in}}%
\pgfpathlineto{\pgfqpoint{3.120920in}{5.001446in}}%
\pgfpathlineto{\pgfqpoint{3.129352in}{4.981118in}}%
\pgfpathlineto{\pgfqpoint{3.163795in}{4.912991in}}%
\pgfpathlineto{\pgfqpoint{3.196879in}{4.978080in}}%
\pgfpathlineto{\pgfqpoint{3.187744in}{5.064917in}}%
\pgfpathlineto{\pgfqpoint{3.179746in}{5.037026in}}%
\pgfpathlineto{\pgfqpoint{3.144361in}{5.202827in}}%
\pgfpathlineto{\pgfqpoint{3.111271in}{5.138671in}}%
\pgfpathclose%
\pgfusepath{fill}%
\end{pgfscope}%
\begin{pgfscope}%
\pgfpathrectangle{\pgfqpoint{1.020000in}{0.880000in}}{\pgfqpoint{6.160000in}{6.160000in}}%
\pgfusepath{clip}%
\pgfsetbuttcap%
\pgfsetroundjoin%
\definecolor{currentfill}{rgb}{0.763520,0.178667,0.193396}%
\pgfsetfillcolor{currentfill}%
\pgfsetlinewidth{0.000000pt}%
\definecolor{currentstroke}{rgb}{0.000000,0.000000,0.000000}%
\pgfsetstrokecolor{currentstroke}%
\pgfsetdash{}{0pt}%
\pgfpathmoveto{\pgfqpoint{3.027234in}{5.159916in}}%
\pgfpathlineto{\pgfqpoint{3.033351in}{5.337074in}}%
\pgfpathlineto{\pgfqpoint{3.043515in}{5.157081in}}%
\pgfpathlineto{\pgfqpoint{3.078636in}{5.034737in}}%
\pgfpathlineto{\pgfqpoint{3.111271in}{5.138671in}}%
\pgfpathlineto{\pgfqpoint{3.101574in}{5.279867in}}%
\pgfpathlineto{\pgfqpoint{3.093965in}{5.225462in}}%
\pgfpathlineto{\pgfqpoint{3.060229in}{5.226208in}}%
\pgfpathlineto{\pgfqpoint{3.027234in}{5.159916in}}%
\pgfpathclose%
\pgfusepath{fill}%
\end{pgfscope}%
\begin{pgfscope}%
\pgfpathrectangle{\pgfqpoint{1.020000in}{0.880000in}}{\pgfqpoint{6.160000in}{6.160000in}}%
\pgfusepath{clip}%
\pgfsetbuttcap%
\pgfsetroundjoin%
\definecolor{currentfill}{rgb}{0.399231,0.528528,0.928459}%
\pgfsetfillcolor{currentfill}%
\pgfsetlinewidth{0.000000pt}%
\definecolor{currentstroke}{rgb}{0.000000,0.000000,0.000000}%
\pgfsetstrokecolor{currentstroke}%
\pgfsetdash{}{0pt}%
\pgfpathmoveto{\pgfqpoint{4.862118in}{3.187703in}}%
\pgfpathlineto{\pgfqpoint{4.872041in}{3.168418in}}%
\pgfpathlineto{\pgfqpoint{4.880590in}{2.950642in}}%
\pgfpathlineto{\pgfqpoint{4.916308in}{3.294322in}}%
\pgfpathlineto{\pgfqpoint{4.948583in}{3.162100in}}%
\pgfpathlineto{\pgfqpoint{4.938804in}{3.209771in}}%
\pgfpathlineto{\pgfqpoint{4.928076in}{3.128737in}}%
\pgfpathlineto{\pgfqpoint{4.895203in}{3.170329in}}%
\pgfpathlineto{\pgfqpoint{4.862118in}{3.187703in}}%
\pgfpathclose%
\pgfusepath{fill}%
\end{pgfscope}%
\begin{pgfscope}%
\pgfpathrectangle{\pgfqpoint{1.020000in}{0.880000in}}{\pgfqpoint{6.160000in}{6.160000in}}%
\pgfusepath{clip}%
\pgfsetbuttcap%
\pgfsetroundjoin%
\definecolor{currentfill}{rgb}{0.473070,0.611077,0.970634}%
\pgfsetfillcolor{currentfill}%
\pgfsetlinewidth{0.000000pt}%
\definecolor{currentstroke}{rgb}{0.000000,0.000000,0.000000}%
\pgfsetstrokecolor{currentstroke}%
\pgfsetdash{}{0pt}%
\pgfpathmoveto{\pgfqpoint{4.490849in}{3.389669in}}%
\pgfpathlineto{\pgfqpoint{4.500401in}{3.353565in}}%
\pgfpathlineto{\pgfqpoint{4.509868in}{3.287650in}}%
\pgfpathlineto{\pgfqpoint{4.542752in}{3.151181in}}%
\pgfpathlineto{\pgfqpoint{4.576348in}{3.223996in}}%
\pgfpathlineto{\pgfqpoint{4.566860in}{3.288574in}}%
\pgfpathlineto{\pgfqpoint{4.556645in}{3.156759in}}%
\pgfpathlineto{\pgfqpoint{4.524407in}{3.452316in}}%
\pgfpathlineto{\pgfqpoint{4.490849in}{3.389669in}}%
\pgfpathclose%
\pgfusepath{fill}%
\end{pgfscope}%
\begin{pgfscope}%
\pgfpathrectangle{\pgfqpoint{1.020000in}{0.880000in}}{\pgfqpoint{6.160000in}{6.160000in}}%
\pgfusepath{clip}%
\pgfsetbuttcap%
\pgfsetroundjoin%
\definecolor{currentfill}{rgb}{0.348323,0.465711,0.888346}%
\pgfsetfillcolor{currentfill}%
\pgfsetlinewidth{0.000000pt}%
\definecolor{currentstroke}{rgb}{0.000000,0.000000,0.000000}%
\pgfsetstrokecolor{currentstroke}%
\pgfsetdash{}{0pt}%
\pgfpathmoveto{\pgfqpoint{5.519965in}{3.109164in}}%
\pgfpathlineto{\pgfqpoint{5.530720in}{3.108567in}}%
\pgfpathlineto{\pgfqpoint{5.540957in}{3.068392in}}%
\pgfpathlineto{\pgfqpoint{5.574594in}{3.108791in}}%
\pgfpathlineto{\pgfqpoint{5.605511in}{2.958424in}}%
\pgfpathlineto{\pgfqpoint{5.597301in}{3.143389in}}%
\pgfpathlineto{\pgfqpoint{5.584943in}{3.034847in}}%
\pgfpathlineto{\pgfqpoint{5.550561in}{2.932583in}}%
\pgfpathlineto{\pgfqpoint{5.519965in}{3.109164in}}%
\pgfpathclose%
\pgfusepath{fill}%
\end{pgfscope}%
\begin{pgfscope}%
\pgfpathrectangle{\pgfqpoint{1.020000in}{0.880000in}}{\pgfqpoint{6.160000in}{6.160000in}}%
\pgfusepath{clip}%
\pgfsetbuttcap%
\pgfsetroundjoin%
\definecolor{currentfill}{rgb}{0.958279,0.604335,0.483297}%
\pgfsetfillcolor{currentfill}%
\pgfsetlinewidth{0.000000pt}%
\definecolor{currentstroke}{rgb}{0.000000,0.000000,0.000000}%
\pgfsetstrokecolor{currentstroke}%
\pgfsetdash{}{0pt}%
\pgfpathmoveto{\pgfqpoint{2.424932in}{4.690949in}}%
\pgfpathlineto{\pgfqpoint{2.435042in}{4.538968in}}%
\pgfpathlineto{\pgfqpoint{2.440442in}{4.660163in}}%
\pgfpathlineto{\pgfqpoint{2.475285in}{4.609125in}}%
\pgfpathlineto{\pgfqpoint{2.509145in}{4.614304in}}%
\pgfpathlineto{\pgfqpoint{2.499186in}{4.758662in}}%
\pgfpathlineto{\pgfqpoint{2.490723in}{4.813143in}}%
\pgfpathlineto{\pgfqpoint{2.459200in}{4.670812in}}%
\pgfpathlineto{\pgfqpoint{2.424932in}{4.690949in}}%
\pgfpathclose%
\pgfusepath{fill}%
\end{pgfscope}%
\begin{pgfscope}%
\pgfpathrectangle{\pgfqpoint{1.020000in}{0.880000in}}{\pgfqpoint{6.160000in}{6.160000in}}%
\pgfusepath{clip}%
\pgfsetbuttcap%
\pgfsetroundjoin%
\definecolor{currentfill}{rgb}{0.494638,0.633022,0.978983}%
\pgfsetfillcolor{currentfill}%
\pgfsetlinewidth{0.000000pt}%
\definecolor{currentstroke}{rgb}{0.000000,0.000000,0.000000}%
\pgfsetstrokecolor{currentstroke}%
\pgfsetdash{}{0pt}%
\pgfpathmoveto{\pgfqpoint{4.271344in}{3.223977in}}%
\pgfpathlineto{\pgfqpoint{4.280902in}{3.372725in}}%
\pgfpathlineto{\pgfqpoint{4.290402in}{3.421580in}}%
\pgfpathlineto{\pgfqpoint{4.323703in}{3.298174in}}%
\pgfpathlineto{\pgfqpoint{4.357163in}{3.325702in}}%
\pgfpathlineto{\pgfqpoint{4.347775in}{3.398752in}}%
\pgfpathlineto{\pgfqpoint{4.338163in}{3.323787in}}%
\pgfpathlineto{\pgfqpoint{4.304750in}{3.277052in}}%
\pgfpathlineto{\pgfqpoint{4.271344in}{3.223977in}}%
\pgfpathclose%
\pgfusepath{fill}%
\end{pgfscope}%
\begin{pgfscope}%
\pgfpathrectangle{\pgfqpoint{1.020000in}{0.880000in}}{\pgfqpoint{6.160000in}{6.160000in}}%
\pgfusepath{clip}%
\pgfsetbuttcap%
\pgfsetroundjoin%
\definecolor{currentfill}{rgb}{0.309060,0.413498,0.850128}%
\pgfsetfillcolor{currentfill}%
\pgfsetlinewidth{0.000000pt}%
\definecolor{currentstroke}{rgb}{0.000000,0.000000,0.000000}%
\pgfsetstrokecolor{currentstroke}%
\pgfsetdash{}{0pt}%
\pgfpathmoveto{\pgfqpoint{5.957797in}{3.005060in}}%
\pgfpathlineto{\pgfqpoint{5.969182in}{3.013799in}}%
\pgfpathlineto{\pgfqpoint{5.980003in}{2.989885in}}%
\pgfpathlineto{\pgfqpoint{6.013020in}{2.994134in}}%
\pgfpathlineto{\pgfqpoint{6.000291in}{2.914969in}}%
\pgfpathlineto{\pgfqpoint{5.990380in}{2.988749in}}%
\pgfpathlineto{\pgfqpoint{5.957797in}{3.005060in}}%
\pgfpathclose%
\pgfusepath{fill}%
\end{pgfscope}%
\begin{pgfscope}%
\pgfpathrectangle{\pgfqpoint{1.020000in}{0.880000in}}{\pgfqpoint{6.160000in}{6.160000in}}%
\pgfusepath{clip}%
\pgfsetbuttcap%
\pgfsetroundjoin%
\definecolor{currentfill}{rgb}{0.554312,0.690097,0.995516}%
\pgfsetfillcolor{currentfill}%
\pgfsetlinewidth{0.000000pt}%
\definecolor{currentstroke}{rgb}{0.000000,0.000000,0.000000}%
\pgfsetstrokecolor{currentstroke}%
\pgfsetdash{}{0pt}%
\pgfpathmoveto{\pgfqpoint{4.119113in}{3.471477in}}%
\pgfpathlineto{\pgfqpoint{4.128404in}{3.424649in}}%
\pgfpathlineto{\pgfqpoint{4.137682in}{3.431002in}}%
\pgfpathlineto{\pgfqpoint{4.171159in}{3.548036in}}%
\pgfpathlineto{\pgfqpoint{4.204630in}{3.365781in}}%
\pgfpathlineto{\pgfqpoint{4.195297in}{3.435698in}}%
\pgfpathlineto{\pgfqpoint{4.185969in}{3.412897in}}%
\pgfpathlineto{\pgfqpoint{4.152581in}{3.377597in}}%
\pgfpathlineto{\pgfqpoint{4.119113in}{3.471477in}}%
\pgfpathclose%
\pgfusepath{fill}%
\end{pgfscope}%
\begin{pgfscope}%
\pgfpathrectangle{\pgfqpoint{1.020000in}{0.880000in}}{\pgfqpoint{6.160000in}{6.160000in}}%
\pgfusepath{clip}%
\pgfsetbuttcap%
\pgfsetroundjoin%
\definecolor{currentfill}{rgb}{0.309060,0.413498,0.850128}%
\pgfsetfillcolor{currentfill}%
\pgfsetlinewidth{0.000000pt}%
\definecolor{currentstroke}{rgb}{0.000000,0.000000,0.000000}%
\pgfsetstrokecolor{currentstroke}%
\pgfsetdash{}{0pt}%
\pgfpathmoveto{\pgfqpoint{5.672337in}{3.015484in}}%
\pgfpathlineto{\pgfqpoint{5.681559in}{2.902289in}}%
\pgfpathlineto{\pgfqpoint{5.693668in}{2.979500in}}%
\pgfpathlineto{\pgfqpoint{5.724290in}{2.824337in}}%
\pgfpathlineto{\pgfqpoint{5.760861in}{3.052564in}}%
\pgfpathlineto{\pgfqpoint{5.748562in}{2.971103in}}%
\pgfpathlineto{\pgfqpoint{5.737670in}{2.977083in}}%
\pgfpathlineto{\pgfqpoint{5.706718in}{3.107408in}}%
\pgfpathlineto{\pgfqpoint{5.672337in}{3.015484in}}%
\pgfpathclose%
\pgfusepath{fill}%
\end{pgfscope}%
\begin{pgfscope}%
\pgfpathrectangle{\pgfqpoint{1.020000in}{0.880000in}}{\pgfqpoint{6.160000in}{6.160000in}}%
\pgfusepath{clip}%
\pgfsetbuttcap%
\pgfsetroundjoin%
\definecolor{currentfill}{rgb}{0.693321,0.796314,0.986308}%
\pgfsetfillcolor{currentfill}%
\pgfsetlinewidth{0.000000pt}%
\definecolor{currentstroke}{rgb}{0.000000,0.000000,0.000000}%
\pgfsetstrokecolor{currentstroke}%
\pgfsetdash{}{0pt}%
\pgfpathmoveto{\pgfqpoint{3.881090in}{3.768780in}}%
\pgfpathlineto{\pgfqpoint{3.890409in}{3.665807in}}%
\pgfpathlineto{\pgfqpoint{3.899831in}{3.526203in}}%
\pgfpathlineto{\pgfqpoint{3.932850in}{3.723063in}}%
\pgfpathlineto{\pgfqpoint{3.966514in}{3.673296in}}%
\pgfpathlineto{\pgfqpoint{3.957445in}{3.662261in}}%
\pgfpathlineto{\pgfqpoint{3.948251in}{3.713631in}}%
\pgfpathlineto{\pgfqpoint{3.914677in}{3.745957in}}%
\pgfpathlineto{\pgfqpoint{3.881090in}{3.768780in}}%
\pgfpathclose%
\pgfusepath{fill}%
\end{pgfscope}%
\begin{pgfscope}%
\pgfpathrectangle{\pgfqpoint{1.020000in}{0.880000in}}{\pgfqpoint{6.160000in}{6.160000in}}%
\pgfusepath{clip}%
\pgfsetbuttcap%
\pgfsetroundjoin%
\definecolor{currentfill}{rgb}{0.818056,0.855590,0.914638}%
\pgfsetfillcolor{currentfill}%
\pgfsetlinewidth{0.000000pt}%
\definecolor{currentstroke}{rgb}{0.000000,0.000000,0.000000}%
\pgfsetstrokecolor{currentstroke}%
\pgfsetdash{}{0pt}%
\pgfpathmoveto{\pgfqpoint{3.643184in}{4.000697in}}%
\pgfpathlineto{\pgfqpoint{3.653139in}{3.775382in}}%
\pgfpathlineto{\pgfqpoint{3.661302in}{3.891007in}}%
\pgfpathlineto{\pgfqpoint{3.694934in}{3.888458in}}%
\pgfpathlineto{\pgfqpoint{3.727976in}{4.011567in}}%
\pgfpathlineto{\pgfqpoint{3.719277in}{3.978712in}}%
\pgfpathlineto{\pgfqpoint{3.710317in}{4.006597in}}%
\pgfpathlineto{\pgfqpoint{3.677023in}{3.951220in}}%
\pgfpathlineto{\pgfqpoint{3.643184in}{4.000697in}}%
\pgfpathclose%
\pgfusepath{fill}%
\end{pgfscope}%
\begin{pgfscope}%
\pgfpathrectangle{\pgfqpoint{1.020000in}{0.880000in}}{\pgfqpoint{6.160000in}{6.160000in}}%
\pgfusepath{clip}%
\pgfsetbuttcap%
\pgfsetroundjoin%
\definecolor{currentfill}{rgb}{0.941728,0.546413,0.429707}%
\pgfsetfillcolor{currentfill}%
\pgfsetlinewidth{0.000000pt}%
\definecolor{currentstroke}{rgb}{0.000000,0.000000,0.000000}%
\pgfsetstrokecolor{currentstroke}%
\pgfsetdash{}{0pt}%
\pgfpathmoveto{\pgfqpoint{2.490723in}{4.813143in}}%
\pgfpathlineto{\pgfqpoint{2.499186in}{4.758662in}}%
\pgfpathlineto{\pgfqpoint{2.509145in}{4.614304in}}%
\pgfpathlineto{\pgfqpoint{2.537632in}{4.949863in}}%
\pgfpathlineto{\pgfqpoint{2.573592in}{4.827611in}}%
\pgfpathlineto{\pgfqpoint{2.568009in}{4.698542in}}%
\pgfpathlineto{\pgfqpoint{2.558771in}{4.799750in}}%
\pgfpathlineto{\pgfqpoint{2.529034in}{4.545142in}}%
\pgfpathlineto{\pgfqpoint{2.490723in}{4.813143in}}%
\pgfpathclose%
\pgfusepath{fill}%
\end{pgfscope}%
\begin{pgfscope}%
\pgfpathrectangle{\pgfqpoint{1.020000in}{0.880000in}}{\pgfqpoint{6.160000in}{6.160000in}}%
\pgfusepath{clip}%
\pgfsetbuttcap%
\pgfsetroundjoin%
\definecolor{currentfill}{rgb}{0.935774,0.812237,0.747156}%
\pgfsetfillcolor{currentfill}%
\pgfsetlinewidth{0.000000pt}%
\definecolor{currentstroke}{rgb}{0.000000,0.000000,0.000000}%
\pgfsetstrokecolor{currentstroke}%
\pgfsetdash{}{0pt}%
\pgfpathmoveto{\pgfqpoint{3.472329in}{4.276841in}}%
\pgfpathlineto{\pgfqpoint{3.480734in}{4.302545in}}%
\pgfpathlineto{\pgfqpoint{3.488969in}{4.356120in}}%
\pgfpathlineto{\pgfqpoint{3.524157in}{4.132151in}}%
\pgfpathlineto{\pgfqpoint{3.556859in}{4.281240in}}%
\pgfpathlineto{\pgfqpoint{3.549246in}{4.113784in}}%
\pgfpathlineto{\pgfqpoint{3.540057in}{4.200505in}}%
\pgfpathlineto{\pgfqpoint{3.505657in}{4.321853in}}%
\pgfpathlineto{\pgfqpoint{3.472329in}{4.276841in}}%
\pgfpathclose%
\pgfusepath{fill}%
\end{pgfscope}%
\begin{pgfscope}%
\pgfpathrectangle{\pgfqpoint{1.020000in}{0.880000in}}{\pgfqpoint{6.160000in}{6.160000in}}%
\pgfusepath{clip}%
\pgfsetbuttcap%
\pgfsetroundjoin%
\definecolor{currentfill}{rgb}{0.926883,0.505422,0.394866}%
\pgfsetfillcolor{currentfill}%
\pgfsetlinewidth{0.000000pt}%
\definecolor{currentstroke}{rgb}{0.000000,0.000000,0.000000}%
\pgfsetstrokecolor{currentstroke}%
\pgfsetdash{}{0pt}%
\pgfpathmoveto{\pgfqpoint{2.558771in}{4.799750in}}%
\pgfpathlineto{\pgfqpoint{2.568009in}{4.698542in}}%
\pgfpathlineto{\pgfqpoint{2.573592in}{4.827611in}}%
\pgfpathlineto{\pgfqpoint{2.609022in}{4.732921in}}%
\pgfpathlineto{\pgfqpoint{2.640885in}{4.867338in}}%
\pgfpathlineto{\pgfqpoint{2.632626in}{4.903173in}}%
\pgfpathlineto{\pgfqpoint{2.626765in}{4.783493in}}%
\pgfpathlineto{\pgfqpoint{2.590715in}{4.923279in}}%
\pgfpathlineto{\pgfqpoint{2.558771in}{4.799750in}}%
\pgfpathclose%
\pgfusepath{fill}%
\end{pgfscope}%
\begin{pgfscope}%
\pgfpathrectangle{\pgfqpoint{1.020000in}{0.880000in}}{\pgfqpoint{6.160000in}{6.160000in}}%
\pgfusepath{clip}%
\pgfsetbuttcap%
\pgfsetroundjoin%
\definecolor{currentfill}{rgb}{0.353369,0.472069,0.892570}%
\pgfsetfillcolor{currentfill}%
\pgfsetlinewidth{0.000000pt}%
\definecolor{currentstroke}{rgb}{0.000000,0.000000,0.000000}%
\pgfsetstrokecolor{currentstroke}%
\pgfsetdash{}{0pt}%
\pgfpathmoveto{\pgfqpoint{5.233521in}{3.050271in}}%
\pgfpathlineto{\pgfqpoint{5.244578in}{3.107020in}}%
\pgfpathlineto{\pgfqpoint{5.255822in}{3.177896in}}%
\pgfpathlineto{\pgfqpoint{5.287664in}{3.055174in}}%
\pgfpathlineto{\pgfqpoint{5.319054in}{2.901765in}}%
\pgfpathlineto{\pgfqpoint{5.309844in}{3.017957in}}%
\pgfpathlineto{\pgfqpoint{5.300460in}{3.118861in}}%
\pgfpathlineto{\pgfqpoint{5.267226in}{3.106358in}}%
\pgfpathlineto{\pgfqpoint{5.233521in}{3.050271in}}%
\pgfpathclose%
\pgfusepath{fill}%
\end{pgfscope}%
\begin{pgfscope}%
\pgfpathrectangle{\pgfqpoint{1.020000in}{0.880000in}}{\pgfqpoint{6.160000in}{6.160000in}}%
\pgfusepath{clip}%
\pgfsetbuttcap%
\pgfsetroundjoin%
\definecolor{currentfill}{rgb}{0.848040,0.338280,0.275206}%
\pgfsetfillcolor{currentfill}%
\pgfsetlinewidth{0.000000pt}%
\definecolor{currentstroke}{rgb}{0.000000,0.000000,0.000000}%
\pgfsetstrokecolor{currentstroke}%
\pgfsetdash{}{0pt}%
\pgfpathmoveto{\pgfqpoint{3.043515in}{5.157081in}}%
\pgfpathlineto{\pgfqpoint{3.052685in}{5.065307in}}%
\pgfpathlineto{\pgfqpoint{3.062156in}{4.946537in}}%
\pgfpathlineto{\pgfqpoint{3.096565in}{4.888524in}}%
\pgfpathlineto{\pgfqpoint{3.129352in}{4.981118in}}%
\pgfpathlineto{\pgfqpoint{3.120920in}{5.001446in}}%
\pgfpathlineto{\pgfqpoint{3.111271in}{5.138671in}}%
\pgfpathlineto{\pgfqpoint{3.078636in}{5.034737in}}%
\pgfpathlineto{\pgfqpoint{3.043515in}{5.157081in}}%
\pgfpathclose%
\pgfusepath{fill}%
\end{pgfscope}%
\begin{pgfscope}%
\pgfpathrectangle{\pgfqpoint{1.020000in}{0.880000in}}{\pgfqpoint{6.160000in}{6.160000in}}%
\pgfusepath{clip}%
\pgfsetbuttcap%
\pgfsetroundjoin%
\definecolor{currentfill}{rgb}{0.388852,0.516298,0.921373}%
\pgfsetfillcolor{currentfill}%
\pgfsetlinewidth{0.000000pt}%
\definecolor{currentstroke}{rgb}{0.000000,0.000000,0.000000}%
\pgfsetstrokecolor{currentstroke}%
\pgfsetdash{}{0pt}%
\pgfpathmoveto{\pgfqpoint{5.806406in}{3.155726in}}%
\pgfpathlineto{\pgfqpoint{5.817100in}{3.132769in}}%
\pgfpathlineto{\pgfqpoint{5.829566in}{3.216107in}}%
\pgfpathlineto{\pgfqpoint{5.859406in}{3.026711in}}%
\pgfpathlineto{\pgfqpoint{5.892077in}{3.010767in}}%
\pgfpathlineto{\pgfqpoint{5.883083in}{3.136694in}}%
\pgfpathlineto{\pgfqpoint{5.872847in}{3.189586in}}%
\pgfpathlineto{\pgfqpoint{5.839210in}{3.147297in}}%
\pgfpathlineto{\pgfqpoint{5.806406in}{3.155726in}}%
\pgfpathclose%
\pgfusepath{fill}%
\end{pgfscope}%
\begin{pgfscope}%
\pgfpathrectangle{\pgfqpoint{1.020000in}{0.880000in}}{\pgfqpoint{6.160000in}{6.160000in}}%
\pgfusepath{clip}%
\pgfsetbuttcap%
\pgfsetroundjoin%
\definecolor{currentfill}{rgb}{0.353369,0.472069,0.892570}%
\pgfsetfillcolor{currentfill}%
\pgfsetlinewidth{0.000000pt}%
\definecolor{currentstroke}{rgb}{0.000000,0.000000,0.000000}%
\pgfsetstrokecolor{currentstroke}%
\pgfsetdash{}{0pt}%
\pgfpathmoveto{\pgfqpoint{5.737670in}{2.977083in}}%
\pgfpathlineto{\pgfqpoint{5.748562in}{2.971103in}}%
\pgfpathlineto{\pgfqpoint{5.760861in}{3.052564in}}%
\pgfpathlineto{\pgfqpoint{5.792474in}{2.965958in}}%
\pgfpathlineto{\pgfqpoint{5.829566in}{3.216107in}}%
\pgfpathlineto{\pgfqpoint{5.817100in}{3.132769in}}%
\pgfpathlineto{\pgfqpoint{5.806406in}{3.155726in}}%
\pgfpathlineto{\pgfqpoint{5.772124in}{3.073042in}}%
\pgfpathlineto{\pgfqpoint{5.737670in}{2.977083in}}%
\pgfpathclose%
\pgfusepath{fill}%
\end{pgfscope}%
\begin{pgfscope}%
\pgfpathrectangle{\pgfqpoint{1.020000in}{0.880000in}}{\pgfqpoint{6.160000in}{6.160000in}}%
\pgfusepath{clip}%
\pgfsetbuttcap%
\pgfsetroundjoin%
\definecolor{currentfill}{rgb}{0.967317,0.657471,0.538160}%
\pgfsetfillcolor{currentfill}%
\pgfsetlinewidth{0.000000pt}%
\definecolor{currentstroke}{rgb}{0.000000,0.000000,0.000000}%
\pgfsetstrokecolor{currentstroke}%
\pgfsetdash{}{0pt}%
\pgfpathmoveto{\pgfqpoint{3.300755in}{4.647373in}}%
\pgfpathlineto{\pgfqpoint{3.310101in}{4.539063in}}%
\pgfpathlineto{\pgfqpoint{3.317778in}{4.625927in}}%
\pgfpathlineto{\pgfqpoint{3.352566in}{4.500780in}}%
\pgfpathlineto{\pgfqpoint{3.385736in}{4.567668in}}%
\pgfpathlineto{\pgfqpoint{3.377015in}{4.597004in}}%
\pgfpathlineto{\pgfqpoint{3.367918in}{4.674502in}}%
\pgfpathlineto{\pgfqpoint{3.335853in}{4.480829in}}%
\pgfpathlineto{\pgfqpoint{3.300755in}{4.647373in}}%
\pgfpathclose%
\pgfusepath{fill}%
\end{pgfscope}%
\begin{pgfscope}%
\pgfpathrectangle{\pgfqpoint{1.020000in}{0.880000in}}{\pgfqpoint{6.160000in}{6.160000in}}%
\pgfusepath{clip}%
\pgfsetbuttcap%
\pgfsetroundjoin%
\definecolor{currentfill}{rgb}{0.895882,0.849906,0.823499}%
\pgfsetfillcolor{currentfill}%
\pgfsetlinewidth{0.000000pt}%
\definecolor{currentstroke}{rgb}{0.000000,0.000000,0.000000}%
\pgfsetstrokecolor{currentstroke}%
\pgfsetdash{}{0pt}%
\pgfpathmoveto{\pgfqpoint{3.556859in}{4.281240in}}%
\pgfpathlineto{\pgfqpoint{3.566578in}{4.112618in}}%
\pgfpathlineto{\pgfqpoint{3.576076in}{3.978781in}}%
\pgfpathlineto{\pgfqpoint{3.608841in}{4.129090in}}%
\pgfpathlineto{\pgfqpoint{3.643184in}{4.000697in}}%
\pgfpathlineto{\pgfqpoint{3.634460in}{3.996616in}}%
\pgfpathlineto{\pgfqpoint{3.624546in}{4.210609in}}%
\pgfpathlineto{\pgfqpoint{3.590628in}{4.262212in}}%
\pgfpathlineto{\pgfqpoint{3.556859in}{4.281240in}}%
\pgfpathclose%
\pgfusepath{fill}%
\end{pgfscope}%
\begin{pgfscope}%
\pgfpathrectangle{\pgfqpoint{1.020000in}{0.880000in}}{\pgfqpoint{6.160000in}{6.160000in}}%
\pgfusepath{clip}%
\pgfsetbuttcap%
\pgfsetroundjoin%
\definecolor{currentfill}{rgb}{0.419991,0.552989,0.942630}%
\pgfsetfillcolor{currentfill}%
\pgfsetlinewidth{0.000000pt}%
\definecolor{currentstroke}{rgb}{0.000000,0.000000,0.000000}%
\pgfsetstrokecolor{currentstroke}%
\pgfsetdash{}{0pt}%
\pgfpathmoveto{\pgfqpoint{4.576348in}{3.223996in}}%
\pgfpathlineto{\pgfqpoint{4.585766in}{3.140054in}}%
\pgfpathlineto{\pgfqpoint{4.595689in}{3.177012in}}%
\pgfpathlineto{\pgfqpoint{4.629214in}{3.213127in}}%
\pgfpathlineto{\pgfqpoint{4.662577in}{3.214556in}}%
\pgfpathlineto{\pgfqpoint{4.652787in}{3.225674in}}%
\pgfpathlineto{\pgfqpoint{4.642439in}{3.111399in}}%
\pgfpathlineto{\pgfqpoint{4.609683in}{3.228644in}}%
\pgfpathlineto{\pgfqpoint{4.576348in}{3.223996in}}%
\pgfpathclose%
\pgfusepath{fill}%
\end{pgfscope}%
\begin{pgfscope}%
\pgfpathrectangle{\pgfqpoint{1.020000in}{0.880000in}}{\pgfqpoint{6.160000in}{6.160000in}}%
\pgfusepath{clip}%
\pgfsetbuttcap%
\pgfsetroundjoin%
\definecolor{currentfill}{rgb}{0.966962,0.735670,0.630877}%
\pgfsetfillcolor{currentfill}%
\pgfsetlinewidth{0.000000pt}%
\definecolor{currentstroke}{rgb}{0.000000,0.000000,0.000000}%
\pgfsetstrokecolor{currentstroke}%
\pgfsetdash{}{0pt}%
\pgfpathmoveto{\pgfqpoint{3.385736in}{4.567668in}}%
\pgfpathlineto{\pgfqpoint{3.395515in}{4.404542in}}%
\pgfpathlineto{\pgfqpoint{3.404199in}{4.381906in}}%
\pgfpathlineto{\pgfqpoint{3.437216in}{4.475120in}}%
\pgfpathlineto{\pgfqpoint{3.472329in}{4.276841in}}%
\pgfpathlineto{\pgfqpoint{3.461969in}{4.528558in}}%
\pgfpathlineto{\pgfqpoint{3.455358in}{4.257352in}}%
\pgfpathlineto{\pgfqpoint{3.419539in}{4.554208in}}%
\pgfpathlineto{\pgfqpoint{3.385736in}{4.567668in}}%
\pgfpathclose%
\pgfusepath{fill}%
\end{pgfscope}%
\begin{pgfscope}%
\pgfpathrectangle{\pgfqpoint{1.020000in}{0.880000in}}{\pgfqpoint{6.160000in}{6.160000in}}%
\pgfusepath{clip}%
\pgfsetbuttcap%
\pgfsetroundjoin%
\definecolor{currentfill}{rgb}{0.763363,0.835092,0.955658}%
\pgfsetfillcolor{currentfill}%
\pgfsetlinewidth{0.000000pt}%
\definecolor{currentstroke}{rgb}{0.000000,0.000000,0.000000}%
\pgfsetstrokecolor{currentstroke}%
\pgfsetdash{}{0pt}%
\pgfpathmoveto{\pgfqpoint{3.727976in}{4.011567in}}%
\pgfpathlineto{\pgfqpoint{3.737612in}{3.838635in}}%
\pgfpathlineto{\pgfqpoint{3.746631in}{3.805069in}}%
\pgfpathlineto{\pgfqpoint{3.780073in}{3.848792in}}%
\pgfpathlineto{\pgfqpoint{3.814458in}{3.638744in}}%
\pgfpathlineto{\pgfqpoint{3.804592in}{3.889368in}}%
\pgfpathlineto{\pgfqpoint{3.796161in}{3.761614in}}%
\pgfpathlineto{\pgfqpoint{3.762533in}{3.788181in}}%
\pgfpathlineto{\pgfqpoint{3.727976in}{4.011567in}}%
\pgfpathclose%
\pgfusepath{fill}%
\end{pgfscope}%
\begin{pgfscope}%
\pgfpathrectangle{\pgfqpoint{1.020000in}{0.880000in}}{\pgfqpoint{6.160000in}{6.160000in}}%
\pgfusepath{clip}%
\pgfsetbuttcap%
\pgfsetroundjoin%
\definecolor{currentfill}{rgb}{0.963772,0.749086,0.649420}%
\pgfsetfillcolor{currentfill}%
\pgfsetlinewidth{0.000000pt}%
\definecolor{currentstroke}{rgb}{0.000000,0.000000,0.000000}%
\pgfsetstrokecolor{currentstroke}%
\pgfsetdash{}{0pt}%
\pgfpathmoveto{\pgfqpoint{2.307992in}{4.461998in}}%
\pgfpathlineto{\pgfqpoint{2.320274in}{4.195851in}}%
\pgfpathlineto{\pgfqpoint{2.326229in}{4.272082in}}%
\pgfpathlineto{\pgfqpoint{2.359849in}{4.293015in}}%
\pgfpathlineto{\pgfqpoint{2.390033in}{4.507303in}}%
\pgfpathlineto{\pgfqpoint{2.382240in}{4.526663in}}%
\pgfpathlineto{\pgfqpoint{2.374839in}{4.524809in}}%
\pgfpathlineto{\pgfqpoint{2.342896in}{4.412093in}}%
\pgfpathlineto{\pgfqpoint{2.307992in}{4.461998in}}%
\pgfpathclose%
\pgfusepath{fill}%
\end{pgfscope}%
\begin{pgfscope}%
\pgfpathrectangle{\pgfqpoint{1.020000in}{0.880000in}}{\pgfqpoint{6.160000in}{6.160000in}}%
\pgfusepath{clip}%
\pgfsetbuttcap%
\pgfsetroundjoin%
\definecolor{currentfill}{rgb}{0.358415,0.478426,0.896795}%
\pgfsetfillcolor{currentfill}%
\pgfsetlinewidth{0.000000pt}%
\definecolor{currentstroke}{rgb}{0.000000,0.000000,0.000000}%
\pgfsetstrokecolor{currentstroke}%
\pgfsetdash{}{0pt}%
\pgfpathmoveto{\pgfqpoint{4.948583in}{3.162100in}}%
\pgfpathlineto{\pgfqpoint{4.958236in}{3.096656in}}%
\pgfpathlineto{\pgfqpoint{4.968323in}{3.085247in}}%
\pgfpathlineto{\pgfqpoint{5.001868in}{3.119326in}}%
\pgfpathlineto{\pgfqpoint{5.034997in}{3.104380in}}%
\pgfpathlineto{\pgfqpoint{5.023880in}{3.000818in}}%
\pgfpathlineto{\pgfqpoint{5.014298in}{3.076157in}}%
\pgfpathlineto{\pgfqpoint{4.980164in}{2.956626in}}%
\pgfpathlineto{\pgfqpoint{4.948583in}{3.162100in}}%
\pgfpathclose%
\pgfusepath{fill}%
\end{pgfscope}%
\begin{pgfscope}%
\pgfpathrectangle{\pgfqpoint{1.020000in}{0.880000in}}{\pgfqpoint{6.160000in}{6.160000in}}%
\pgfusepath{clip}%
\pgfsetbuttcap%
\pgfsetroundjoin%
\definecolor{currentfill}{rgb}{0.409611,0.540759,0.935545}%
\pgfsetfillcolor{currentfill}%
\pgfsetlinewidth{0.000000pt}%
\definecolor{currentstroke}{rgb}{0.000000,0.000000,0.000000}%
\pgfsetstrokecolor{currentstroke}%
\pgfsetdash{}{0pt}%
\pgfpathmoveto{\pgfqpoint{4.796087in}{3.260487in}}%
\pgfpathlineto{\pgfqpoint{4.806243in}{3.285689in}}%
\pgfpathlineto{\pgfqpoint{4.815933in}{3.233317in}}%
\pgfpathlineto{\pgfqpoint{4.848589in}{3.132451in}}%
\pgfpathlineto{\pgfqpoint{4.880590in}{2.950642in}}%
\pgfpathlineto{\pgfqpoint{4.872041in}{3.168418in}}%
\pgfpathlineto{\pgfqpoint{4.862118in}{3.187703in}}%
\pgfpathlineto{\pgfqpoint{4.828608in}{3.144295in}}%
\pgfpathlineto{\pgfqpoint{4.796087in}{3.260487in}}%
\pgfpathclose%
\pgfusepath{fill}%
\end{pgfscope}%
\begin{pgfscope}%
\pgfpathrectangle{\pgfqpoint{1.020000in}{0.880000in}}{\pgfqpoint{6.160000in}{6.160000in}}%
\pgfusepath{clip}%
\pgfsetbuttcap%
\pgfsetroundjoin%
\definecolor{currentfill}{rgb}{0.873402,0.386960,0.306332}%
\pgfsetfillcolor{currentfill}%
\pgfsetlinewidth{0.000000pt}%
\definecolor{currentstroke}{rgb}{0.000000,0.000000,0.000000}%
\pgfsetstrokecolor{currentstroke}%
\pgfsetdash{}{0pt}%
\pgfpathmoveto{\pgfqpoint{3.196879in}{4.978080in}}%
\pgfpathlineto{\pgfqpoint{3.205500in}{4.944955in}}%
\pgfpathlineto{\pgfqpoint{3.213930in}{4.933017in}}%
\pgfpathlineto{\pgfqpoint{3.246773in}{5.033523in}}%
\pgfpathlineto{\pgfqpoint{3.281402in}{4.938327in}}%
\pgfpathlineto{\pgfqpoint{3.272724in}{4.971366in}}%
\pgfpathlineto{\pgfqpoint{3.264658in}{4.938051in}}%
\pgfpathlineto{\pgfqpoint{3.230571in}{4.980903in}}%
\pgfpathlineto{\pgfqpoint{3.196879in}{4.978080in}}%
\pgfpathclose%
\pgfusepath{fill}%
\end{pgfscope}%
\begin{pgfscope}%
\pgfpathrectangle{\pgfqpoint{1.020000in}{0.880000in}}{\pgfqpoint{6.160000in}{6.160000in}}%
\pgfusepath{clip}%
\pgfsetbuttcap%
\pgfsetroundjoin%
\definecolor{currentfill}{rgb}{0.693321,0.796314,0.986308}%
\pgfsetfillcolor{currentfill}%
\pgfsetlinewidth{0.000000pt}%
\definecolor{currentstroke}{rgb}{0.000000,0.000000,0.000000}%
\pgfsetstrokecolor{currentstroke}%
\pgfsetdash{}{0pt}%
\pgfpathmoveto{\pgfqpoint{3.814458in}{3.638744in}}%
\pgfpathlineto{\pgfqpoint{3.823159in}{3.705418in}}%
\pgfpathlineto{\pgfqpoint{3.831990in}{3.743479in}}%
\pgfpathlineto{\pgfqpoint{3.865898in}{3.653431in}}%
\pgfpathlineto{\pgfqpoint{3.899831in}{3.526203in}}%
\pgfpathlineto{\pgfqpoint{3.890409in}{3.665807in}}%
\pgfpathlineto{\pgfqpoint{3.881090in}{3.768780in}}%
\pgfpathlineto{\pgfqpoint{3.847597in}{3.753076in}}%
\pgfpathlineto{\pgfqpoint{3.814458in}{3.638744in}}%
\pgfpathclose%
\pgfusepath{fill}%
\end{pgfscope}%
\begin{pgfscope}%
\pgfpathrectangle{\pgfqpoint{1.020000in}{0.880000in}}{\pgfqpoint{6.160000in}{6.160000in}}%
\pgfusepath{clip}%
\pgfsetbuttcap%
\pgfsetroundjoin%
\definecolor{currentfill}{rgb}{0.323718,0.433158,0.864722}%
\pgfsetfillcolor{currentfill}%
\pgfsetlinewidth{0.000000pt}%
\definecolor{currentstroke}{rgb}{0.000000,0.000000,0.000000}%
\pgfsetstrokecolor{currentstroke}%
\pgfsetdash{}{0pt}%
\pgfpathmoveto{\pgfqpoint{5.892077in}{3.010767in}}%
\pgfpathlineto{\pgfqpoint{5.902652in}{2.976603in}}%
\pgfpathlineto{\pgfqpoint{5.912934in}{2.924603in}}%
\pgfpathlineto{\pgfqpoint{5.947951in}{3.040951in}}%
\pgfpathlineto{\pgfqpoint{5.980003in}{2.989885in}}%
\pgfpathlineto{\pgfqpoint{5.969182in}{3.013799in}}%
\pgfpathlineto{\pgfqpoint{5.957797in}{3.005060in}}%
\pgfpathlineto{\pgfqpoint{5.926109in}{3.074304in}}%
\pgfpathlineto{\pgfqpoint{5.892077in}{3.010767in}}%
\pgfpathclose%
\pgfusepath{fill}%
\end{pgfscope}%
\begin{pgfscope}%
\pgfpathrectangle{\pgfqpoint{1.020000in}{0.880000in}}{\pgfqpoint{6.160000in}{6.160000in}}%
\pgfusepath{clip}%
\pgfsetbuttcap%
\pgfsetroundjoin%
\definecolor{currentfill}{rgb}{0.884643,0.410017,0.322507}%
\pgfsetfillcolor{currentfill}%
\pgfsetlinewidth{0.000000pt}%
\definecolor{currentstroke}{rgb}{0.000000,0.000000,0.000000}%
\pgfsetstrokecolor{currentstroke}%
\pgfsetdash{}{0pt}%
\pgfpathmoveto{\pgfqpoint{2.626765in}{4.783493in}}%
\pgfpathlineto{\pgfqpoint{2.632626in}{4.903173in}}%
\pgfpathlineto{\pgfqpoint{2.640885in}{4.867338in}}%
\pgfpathlineto{\pgfqpoint{2.673388in}{4.963472in}}%
\pgfpathlineto{\pgfqpoint{2.707702in}{4.937981in}}%
\pgfpathlineto{\pgfqpoint{2.698729in}{5.019377in}}%
\pgfpathlineto{\pgfqpoint{2.692136in}{4.938998in}}%
\pgfpathlineto{\pgfqpoint{2.656748in}{5.040050in}}%
\pgfpathlineto{\pgfqpoint{2.626765in}{4.783493in}}%
\pgfpathclose%
\pgfusepath{fill}%
\end{pgfscope}%
\begin{pgfscope}%
\pgfpathrectangle{\pgfqpoint{1.020000in}{0.880000in}}{\pgfqpoint{6.160000in}{6.160000in}}%
\pgfusepath{clip}%
\pgfsetbuttcap%
\pgfsetroundjoin%
\definecolor{currentfill}{rgb}{0.861054,0.362916,0.290628}%
\pgfsetfillcolor{currentfill}%
\pgfsetlinewidth{0.000000pt}%
\definecolor{currentstroke}{rgb}{0.000000,0.000000,0.000000}%
\pgfsetstrokecolor{currentstroke}%
\pgfsetdash{}{0pt}%
\pgfpathmoveto{\pgfqpoint{2.692136in}{4.938998in}}%
\pgfpathlineto{\pgfqpoint{2.698729in}{5.019377in}}%
\pgfpathlineto{\pgfqpoint{2.707702in}{4.937981in}}%
\pgfpathlineto{\pgfqpoint{2.741557in}{4.942304in}}%
\pgfpathlineto{\pgfqpoint{2.772588in}{5.148990in}}%
\pgfpathlineto{\pgfqpoint{2.768975in}{4.843993in}}%
\pgfpathlineto{\pgfqpoint{2.758087in}{5.062286in}}%
\pgfpathlineto{\pgfqpoint{2.724181in}{5.064682in}}%
\pgfpathlineto{\pgfqpoint{2.692136in}{4.938998in}}%
\pgfpathclose%
\pgfusepath{fill}%
\end{pgfscope}%
\begin{pgfscope}%
\pgfpathrectangle{\pgfqpoint{1.020000in}{0.880000in}}{\pgfqpoint{6.160000in}{6.160000in}}%
\pgfusepath{clip}%
\pgfsetbuttcap%
\pgfsetroundjoin%
\definecolor{currentfill}{rgb}{0.656683,0.771806,0.994914}%
\pgfsetfillcolor{currentfill}%
\pgfsetlinewidth{0.000000pt}%
\definecolor{currentstroke}{rgb}{0.000000,0.000000,0.000000}%
\pgfsetstrokecolor{currentstroke}%
\pgfsetdash{}{0pt}%
\pgfpathmoveto{\pgfqpoint{3.966514in}{3.673296in}}%
\pgfpathlineto{\pgfqpoint{3.975572in}{3.700230in}}%
\pgfpathlineto{\pgfqpoint{3.984934in}{3.585695in}}%
\pgfpathlineto{\pgfqpoint{4.018387in}{3.648052in}}%
\pgfpathlineto{\pgfqpoint{4.052106in}{3.525255in}}%
\pgfpathlineto{\pgfqpoint{4.042837in}{3.585006in}}%
\pgfpathlineto{\pgfqpoint{4.033723in}{3.546547in}}%
\pgfpathlineto{\pgfqpoint{4.000159in}{3.604892in}}%
\pgfpathlineto{\pgfqpoint{3.966514in}{3.673296in}}%
\pgfpathclose%
\pgfusepath{fill}%
\end{pgfscope}%
\begin{pgfscope}%
\pgfpathrectangle{\pgfqpoint{1.020000in}{0.880000in}}{\pgfqpoint{6.160000in}{6.160000in}}%
\pgfusepath{clip}%
\pgfsetbuttcap%
\pgfsetroundjoin%
\definecolor{currentfill}{rgb}{0.729196,0.086679,0.167240}%
\pgfsetfillcolor{currentfill}%
\pgfsetlinewidth{0.000000pt}%
\definecolor{currentstroke}{rgb}{0.000000,0.000000,0.000000}%
\pgfsetstrokecolor{currentstroke}%
\pgfsetdash{}{0pt}%
\pgfpathmoveto{\pgfqpoint{2.959137in}{5.208684in}}%
\pgfpathlineto{\pgfqpoint{2.966857in}{5.236626in}}%
\pgfpathlineto{\pgfqpoint{2.972833in}{5.413321in}}%
\pgfpathlineto{\pgfqpoint{3.009376in}{5.185547in}}%
\pgfpathlineto{\pgfqpoint{3.043515in}{5.157081in}}%
\pgfpathlineto{\pgfqpoint{3.033351in}{5.337074in}}%
\pgfpathlineto{\pgfqpoint{3.027234in}{5.159916in}}%
\pgfpathlineto{\pgfqpoint{2.992777in}{5.220797in}}%
\pgfpathlineto{\pgfqpoint{2.959137in}{5.208684in}}%
\pgfpathclose%
\pgfusepath{fill}%
\end{pgfscope}%
\begin{pgfscope}%
\pgfpathrectangle{\pgfqpoint{1.020000in}{0.880000in}}{\pgfqpoint{6.160000in}{6.160000in}}%
\pgfusepath{clip}%
\pgfsetbuttcap%
\pgfsetroundjoin%
\definecolor{currentfill}{rgb}{0.328604,0.439712,0.869587}%
\pgfsetfillcolor{currentfill}%
\pgfsetlinewidth{0.000000pt}%
\definecolor{currentstroke}{rgb}{0.000000,0.000000,0.000000}%
\pgfsetstrokecolor{currentstroke}%
\pgfsetdash{}{0pt}%
\pgfpathmoveto{\pgfqpoint{5.605511in}{2.958424in}}%
\pgfpathlineto{\pgfqpoint{5.617971in}{3.069849in}}%
\pgfpathlineto{\pgfqpoint{5.629309in}{3.100619in}}%
\pgfpathlineto{\pgfqpoint{5.663244in}{3.156349in}}%
\pgfpathlineto{\pgfqpoint{5.693668in}{2.979500in}}%
\pgfpathlineto{\pgfqpoint{5.681559in}{2.902289in}}%
\pgfpathlineto{\pgfqpoint{5.672337in}{3.015484in}}%
\pgfpathlineto{\pgfqpoint{5.637492in}{2.889028in}}%
\pgfpathlineto{\pgfqpoint{5.605511in}{2.958424in}}%
\pgfpathclose%
\pgfusepath{fill}%
\end{pgfscope}%
\begin{pgfscope}%
\pgfpathrectangle{\pgfqpoint{1.020000in}{0.880000in}}{\pgfqpoint{6.160000in}{6.160000in}}%
\pgfusepath{clip}%
\pgfsetbuttcap%
\pgfsetroundjoin%
\definecolor{currentfill}{rgb}{0.880896,0.402331,0.317115}%
\pgfsetfillcolor{currentfill}%
\pgfsetlinewidth{0.000000pt}%
\definecolor{currentstroke}{rgb}{0.000000,0.000000,0.000000}%
\pgfsetstrokecolor{currentstroke}%
\pgfsetdash{}{0pt}%
\pgfpathmoveto{\pgfqpoint{3.129352in}{4.981118in}}%
\pgfpathlineto{\pgfqpoint{3.139286in}{4.816156in}}%
\pgfpathlineto{\pgfqpoint{3.145323in}{5.031815in}}%
\pgfpathlineto{\pgfqpoint{3.180025in}{4.944801in}}%
\pgfpathlineto{\pgfqpoint{3.213930in}{4.933017in}}%
\pgfpathlineto{\pgfqpoint{3.205500in}{4.944955in}}%
\pgfpathlineto{\pgfqpoint{3.196879in}{4.978080in}}%
\pgfpathlineto{\pgfqpoint{3.163795in}{4.912991in}}%
\pgfpathlineto{\pgfqpoint{3.129352in}{4.981118in}}%
\pgfpathclose%
\pgfusepath{fill}%
\end{pgfscope}%
\begin{pgfscope}%
\pgfpathrectangle{\pgfqpoint{1.020000in}{0.880000in}}{\pgfqpoint{6.160000in}{6.160000in}}%
\pgfusepath{clip}%
\pgfsetbuttcap%
\pgfsetroundjoin%
\definecolor{currentfill}{rgb}{0.815508,0.277781,0.240294}%
\pgfsetfillcolor{currentfill}%
\pgfsetlinewidth{0.000000pt}%
\definecolor{currentstroke}{rgb}{0.000000,0.000000,0.000000}%
\pgfsetstrokecolor{currentstroke}%
\pgfsetdash{}{0pt}%
\pgfpathmoveto{\pgfqpoint{2.758087in}{5.062286in}}%
\pgfpathlineto{\pgfqpoint{2.768975in}{4.843993in}}%
\pgfpathlineto{\pgfqpoint{2.772588in}{5.148990in}}%
\pgfpathlineto{\pgfqpoint{2.805832in}{5.200880in}}%
\pgfpathlineto{\pgfqpoint{2.843292in}{4.934530in}}%
\pgfpathlineto{\pgfqpoint{2.832221in}{5.172072in}}%
\pgfpathlineto{\pgfqpoint{2.823694in}{5.217692in}}%
\pgfpathlineto{\pgfqpoint{2.791600in}{5.086990in}}%
\pgfpathlineto{\pgfqpoint{2.758087in}{5.062286in}}%
\pgfpathclose%
\pgfusepath{fill}%
\end{pgfscope}%
\begin{pgfscope}%
\pgfpathrectangle{\pgfqpoint{1.020000in}{0.880000in}}{\pgfqpoint{6.160000in}{6.160000in}}%
\pgfusepath{clip}%
\pgfsetbuttcap%
\pgfsetroundjoin%
\definecolor{currentfill}{rgb}{0.373552,0.497499,0.909467}%
\pgfsetfillcolor{currentfill}%
\pgfsetlinewidth{0.000000pt}%
\definecolor{currentstroke}{rgb}{0.000000,0.000000,0.000000}%
\pgfsetstrokecolor{currentstroke}%
\pgfsetdash{}{0pt}%
\pgfpathmoveto{\pgfqpoint{5.167330in}{3.056199in}}%
\pgfpathlineto{\pgfqpoint{5.177653in}{3.049586in}}%
\pgfpathlineto{\pgfqpoint{5.187487in}{2.992430in}}%
\pgfpathlineto{\pgfqpoint{5.221147in}{3.038464in}}%
\pgfpathlineto{\pgfqpoint{5.255822in}{3.177896in}}%
\pgfpathlineto{\pgfqpoint{5.244578in}{3.107020in}}%
\pgfpathlineto{\pgfqpoint{5.233521in}{3.050271in}}%
\pgfpathlineto{\pgfqpoint{5.202609in}{3.266073in}}%
\pgfpathlineto{\pgfqpoint{5.167330in}{3.056199in}}%
\pgfpathclose%
\pgfusepath{fill}%
\end{pgfscope}%
\begin{pgfscope}%
\pgfpathrectangle{\pgfqpoint{1.020000in}{0.880000in}}{\pgfqpoint{6.160000in}{6.160000in}}%
\pgfusepath{clip}%
\pgfsetbuttcap%
\pgfsetroundjoin%
\definecolor{currentfill}{rgb}{0.902849,0.844796,0.811970}%
\pgfsetfillcolor{currentfill}%
\pgfsetlinewidth{0.000000pt}%
\definecolor{currentstroke}{rgb}{0.000000,0.000000,0.000000}%
\pgfsetstrokecolor{currentstroke}%
\pgfsetdash{}{0pt}%
\pgfpathmoveto{\pgfqpoint{3.488969in}{4.356120in}}%
\pgfpathlineto{\pgfqpoint{3.499294in}{4.106051in}}%
\pgfpathlineto{\pgfqpoint{3.508199in}{4.064027in}}%
\pgfpathlineto{\pgfqpoint{3.542269in}{4.004560in}}%
\pgfpathlineto{\pgfqpoint{3.576076in}{3.978781in}}%
\pgfpathlineto{\pgfqpoint{3.566578in}{4.112618in}}%
\pgfpathlineto{\pgfqpoint{3.556859in}{4.281240in}}%
\pgfpathlineto{\pgfqpoint{3.524157in}{4.132151in}}%
\pgfpathlineto{\pgfqpoint{3.488969in}{4.356120in}}%
\pgfpathclose%
\pgfusepath{fill}%
\end{pgfscope}%
\begin{pgfscope}%
\pgfpathrectangle{\pgfqpoint{1.020000in}{0.880000in}}{\pgfqpoint{6.160000in}{6.160000in}}%
\pgfusepath{clip}%
\pgfsetbuttcap%
\pgfsetroundjoin%
\definecolor{currentfill}{rgb}{0.839351,0.861167,0.894494}%
\pgfsetfillcolor{currentfill}%
\pgfsetlinewidth{0.000000pt}%
\definecolor{currentstroke}{rgb}{0.000000,0.000000,0.000000}%
\pgfsetstrokecolor{currentstroke}%
\pgfsetdash{}{0pt}%
\pgfpathmoveto{\pgfqpoint{3.576076in}{3.978781in}}%
\pgfpathlineto{\pgfqpoint{3.584389in}{4.043258in}}%
\pgfpathlineto{\pgfqpoint{3.592898in}{4.079315in}}%
\pgfpathlineto{\pgfqpoint{3.627541in}{3.912859in}}%
\pgfpathlineto{\pgfqpoint{3.661302in}{3.891007in}}%
\pgfpathlineto{\pgfqpoint{3.653139in}{3.775382in}}%
\pgfpathlineto{\pgfqpoint{3.643184in}{4.000697in}}%
\pgfpathlineto{\pgfqpoint{3.608841in}{4.129090in}}%
\pgfpathlineto{\pgfqpoint{3.576076in}{3.978781in}}%
\pgfpathclose%
\pgfusepath{fill}%
\end{pgfscope}%
\begin{pgfscope}%
\pgfpathrectangle{\pgfqpoint{1.020000in}{0.880000in}}{\pgfqpoint{6.160000in}{6.160000in}}%
\pgfusepath{clip}%
\pgfsetbuttcap%
\pgfsetroundjoin%
\definecolor{currentfill}{rgb}{0.785153,0.220851,0.211673}%
\pgfsetfillcolor{currentfill}%
\pgfsetlinewidth{0.000000pt}%
\definecolor{currentstroke}{rgb}{0.000000,0.000000,0.000000}%
\pgfsetstrokecolor{currentstroke}%
\pgfsetdash{}{0pt}%
\pgfpathmoveto{\pgfqpoint{2.823694in}{5.217692in}}%
\pgfpathlineto{\pgfqpoint{2.832221in}{5.172072in}}%
\pgfpathlineto{\pgfqpoint{2.843292in}{4.934530in}}%
\pgfpathlineto{\pgfqpoint{2.875353in}{5.072444in}}%
\pgfpathlineto{\pgfqpoint{2.908666in}{5.116216in}}%
\pgfpathlineto{\pgfqpoint{2.899865in}{5.179939in}}%
\pgfpathlineto{\pgfqpoint{2.890624in}{5.278917in}}%
\pgfpathlineto{\pgfqpoint{2.857789in}{5.199846in}}%
\pgfpathlineto{\pgfqpoint{2.823694in}{5.217692in}}%
\pgfpathclose%
\pgfusepath{fill}%
\end{pgfscope}%
\begin{pgfscope}%
\pgfpathrectangle{\pgfqpoint{1.020000in}{0.880000in}}{\pgfqpoint{6.160000in}{6.160000in}}%
\pgfusepath{clip}%
\pgfsetbuttcap%
\pgfsetroundjoin%
\definecolor{currentfill}{rgb}{0.388852,0.516298,0.921373}%
\pgfsetfillcolor{currentfill}%
\pgfsetlinewidth{0.000000pt}%
\definecolor{currentstroke}{rgb}{0.000000,0.000000,0.000000}%
\pgfsetstrokecolor{currentstroke}%
\pgfsetdash{}{0pt}%
\pgfpathmoveto{\pgfqpoint{5.454440in}{3.154424in}}%
\pgfpathlineto{\pgfqpoint{5.463094in}{2.995038in}}%
\pgfpathlineto{\pgfqpoint{5.475662in}{3.138506in}}%
\pgfpathlineto{\pgfqpoint{5.509860in}{3.218442in}}%
\pgfpathlineto{\pgfqpoint{5.540957in}{3.068392in}}%
\pgfpathlineto{\pgfqpoint{5.530720in}{3.108567in}}%
\pgfpathlineto{\pgfqpoint{5.519965in}{3.109164in}}%
\pgfpathlineto{\pgfqpoint{5.487830in}{3.178404in}}%
\pgfpathlineto{\pgfqpoint{5.454440in}{3.154424in}}%
\pgfpathclose%
\pgfusepath{fill}%
\end{pgfscope}%
\begin{pgfscope}%
\pgfpathrectangle{\pgfqpoint{1.020000in}{0.880000in}}{\pgfqpoint{6.160000in}{6.160000in}}%
\pgfusepath{clip}%
\pgfsetbuttcap%
\pgfsetroundjoin%
\definecolor{currentfill}{rgb}{0.538004,0.674902,0.991722}%
\pgfsetfillcolor{currentfill}%
\pgfsetlinewidth{0.000000pt}%
\definecolor{currentstroke}{rgb}{0.000000,0.000000,0.000000}%
\pgfsetstrokecolor{currentstroke}%
\pgfsetdash{}{0pt}%
\pgfpathmoveto{\pgfqpoint{4.204630in}{3.365781in}}%
\pgfpathlineto{\pgfqpoint{4.214041in}{3.526981in}}%
\pgfpathlineto{\pgfqpoint{4.223386in}{3.400212in}}%
\pgfpathlineto{\pgfqpoint{4.256849in}{3.340726in}}%
\pgfpathlineto{\pgfqpoint{4.290402in}{3.421580in}}%
\pgfpathlineto{\pgfqpoint{4.280902in}{3.372725in}}%
\pgfpathlineto{\pgfqpoint{4.271344in}{3.223977in}}%
\pgfpathlineto{\pgfqpoint{4.238114in}{3.465302in}}%
\pgfpathlineto{\pgfqpoint{4.204630in}{3.365781in}}%
\pgfpathclose%
\pgfusepath{fill}%
\end{pgfscope}%
\begin{pgfscope}%
\pgfpathrectangle{\pgfqpoint{1.020000in}{0.880000in}}{\pgfqpoint{6.160000in}{6.160000in}}%
\pgfusepath{clip}%
\pgfsetbuttcap%
\pgfsetroundjoin%
\definecolor{currentfill}{rgb}{0.328604,0.439712,0.869587}%
\pgfsetfillcolor{currentfill}%
\pgfsetlinewidth{0.000000pt}%
\definecolor{currentstroke}{rgb}{0.000000,0.000000,0.000000}%
\pgfsetstrokecolor{currentstroke}%
\pgfsetdash{}{0pt}%
\pgfpathmoveto{\pgfqpoint{5.319054in}{2.901765in}}%
\pgfpathlineto{\pgfqpoint{5.330291in}{2.961645in}}%
\pgfpathlineto{\pgfqpoint{5.341410in}{3.008350in}}%
\pgfpathlineto{\pgfqpoint{5.373439in}{2.912912in}}%
\pgfpathlineto{\pgfqpoint{5.409117in}{3.120995in}}%
\pgfpathlineto{\pgfqpoint{5.398782in}{3.147385in}}%
\pgfpathlineto{\pgfqpoint{5.385537in}{2.931598in}}%
\pgfpathlineto{\pgfqpoint{5.354141in}{3.073313in}}%
\pgfpathlineto{\pgfqpoint{5.319054in}{2.901765in}}%
\pgfpathclose%
\pgfusepath{fill}%
\end{pgfscope}%
\begin{pgfscope}%
\pgfpathrectangle{\pgfqpoint{1.020000in}{0.880000in}}{\pgfqpoint{6.160000in}{6.160000in}}%
\pgfusepath{clip}%
\pgfsetbuttcap%
\pgfsetroundjoin%
\definecolor{currentfill}{rgb}{0.740957,0.122240,0.175744}%
\pgfsetfillcolor{currentfill}%
\pgfsetlinewidth{0.000000pt}%
\definecolor{currentstroke}{rgb}{0.000000,0.000000,0.000000}%
\pgfsetstrokecolor{currentstroke}%
\pgfsetdash{}{0pt}%
\pgfpathmoveto{\pgfqpoint{2.890624in}{5.278917in}}%
\pgfpathlineto{\pgfqpoint{2.899865in}{5.179939in}}%
\pgfpathlineto{\pgfqpoint{2.908666in}{5.116216in}}%
\pgfpathlineto{\pgfqpoint{2.943099in}{5.068950in}}%
\pgfpathlineto{\pgfqpoint{2.972833in}{5.413321in}}%
\pgfpathlineto{\pgfqpoint{2.966857in}{5.236626in}}%
\pgfpathlineto{\pgfqpoint{2.959137in}{5.208684in}}%
\pgfpathlineto{\pgfqpoint{2.924805in}{5.251618in}}%
\pgfpathlineto{\pgfqpoint{2.890624in}{5.278917in}}%
\pgfpathclose%
\pgfusepath{fill}%
\end{pgfscope}%
\begin{pgfscope}%
\pgfpathrectangle{\pgfqpoint{1.020000in}{0.880000in}}{\pgfqpoint{6.160000in}{6.160000in}}%
\pgfusepath{clip}%
\pgfsetbuttcap%
\pgfsetroundjoin%
\definecolor{currentfill}{rgb}{0.353369,0.472069,0.892570}%
\pgfsetfillcolor{currentfill}%
\pgfsetlinewidth{0.000000pt}%
\definecolor{currentstroke}{rgb}{0.000000,0.000000,0.000000}%
\pgfsetstrokecolor{currentstroke}%
\pgfsetdash{}{0pt}%
\pgfpathmoveto{\pgfqpoint{5.101004in}{3.056927in}}%
\pgfpathlineto{\pgfqpoint{5.110706in}{2.990141in}}%
\pgfpathlineto{\pgfqpoint{5.122723in}{3.168170in}}%
\pgfpathlineto{\pgfqpoint{5.154806in}{3.045629in}}%
\pgfpathlineto{\pgfqpoint{5.187487in}{2.992430in}}%
\pgfpathlineto{\pgfqpoint{5.177653in}{3.049586in}}%
\pgfpathlineto{\pgfqpoint{5.167330in}{3.056199in}}%
\pgfpathlineto{\pgfqpoint{5.134099in}{3.048399in}}%
\pgfpathlineto{\pgfqpoint{5.101004in}{3.056927in}}%
\pgfpathclose%
\pgfusepath{fill}%
\end{pgfscope}%
\begin{pgfscope}%
\pgfpathrectangle{\pgfqpoint{1.020000in}{0.880000in}}{\pgfqpoint{6.160000in}{6.160000in}}%
\pgfusepath{clip}%
\pgfsetbuttcap%
\pgfsetroundjoin%
\definecolor{currentfill}{rgb}{0.597777,0.727330,0.999777}%
\pgfsetfillcolor{currentfill}%
\pgfsetlinewidth{0.000000pt}%
\definecolor{currentstroke}{rgb}{0.000000,0.000000,0.000000}%
\pgfsetstrokecolor{currentstroke}%
\pgfsetdash{}{0pt}%
\pgfpathmoveto{\pgfqpoint{4.052106in}{3.525255in}}%
\pgfpathlineto{\pgfqpoint{4.061144in}{3.662085in}}%
\pgfpathlineto{\pgfqpoint{4.070577in}{3.487307in}}%
\pgfpathlineto{\pgfqpoint{4.104123in}{3.492688in}}%
\pgfpathlineto{\pgfqpoint{4.137682in}{3.431002in}}%
\pgfpathlineto{\pgfqpoint{4.128404in}{3.424649in}}%
\pgfpathlineto{\pgfqpoint{4.119113in}{3.471477in}}%
\pgfpathlineto{\pgfqpoint{4.085609in}{3.517655in}}%
\pgfpathlineto{\pgfqpoint{4.052106in}{3.525255in}}%
\pgfpathclose%
\pgfusepath{fill}%
\end{pgfscope}%
\begin{pgfscope}%
\pgfpathrectangle{\pgfqpoint{1.020000in}{0.880000in}}{\pgfqpoint{6.160000in}{6.160000in}}%
\pgfusepath{clip}%
\pgfsetbuttcap%
\pgfsetroundjoin%
\definecolor{currentfill}{rgb}{0.527132,0.664700,0.989065}%
\pgfsetfillcolor{currentfill}%
\pgfsetlinewidth{0.000000pt}%
\definecolor{currentstroke}{rgb}{0.000000,0.000000,0.000000}%
\pgfsetstrokecolor{currentstroke}%
\pgfsetdash{}{0pt}%
\pgfpathmoveto{\pgfqpoint{4.424034in}{3.361728in}}%
\pgfpathlineto{\pgfqpoint{4.433806in}{3.431261in}}%
\pgfpathlineto{\pgfqpoint{4.443196in}{3.340648in}}%
\pgfpathlineto{\pgfqpoint{4.476885in}{3.427791in}}%
\pgfpathlineto{\pgfqpoint{4.509868in}{3.287650in}}%
\pgfpathlineto{\pgfqpoint{4.500401in}{3.353565in}}%
\pgfpathlineto{\pgfqpoint{4.490849in}{3.389669in}}%
\pgfpathlineto{\pgfqpoint{4.457492in}{3.392283in}}%
\pgfpathlineto{\pgfqpoint{4.424034in}{3.361728in}}%
\pgfpathclose%
\pgfusepath{fill}%
\end{pgfscope}%
\begin{pgfscope}%
\pgfpathrectangle{\pgfqpoint{1.020000in}{0.880000in}}{\pgfqpoint{6.160000in}{6.160000in}}%
\pgfusepath{clip}%
\pgfsetbuttcap%
\pgfsetroundjoin%
\definecolor{currentfill}{rgb}{0.368507,0.491141,0.905243}%
\pgfsetfillcolor{currentfill}%
\pgfsetlinewidth{0.000000pt}%
\definecolor{currentstroke}{rgb}{0.000000,0.000000,0.000000}%
\pgfsetstrokecolor{currentstroke}%
\pgfsetdash{}{0pt}%
\pgfpathmoveto{\pgfqpoint{5.385537in}{2.931598in}}%
\pgfpathlineto{\pgfqpoint{5.398782in}{3.147385in}}%
\pgfpathlineto{\pgfqpoint{5.409117in}{3.120995in}}%
\pgfpathlineto{\pgfqpoint{5.441928in}{3.092619in}}%
\pgfpathlineto{\pgfqpoint{5.475662in}{3.138506in}}%
\pgfpathlineto{\pgfqpoint{5.463094in}{2.995038in}}%
\pgfpathlineto{\pgfqpoint{5.454440in}{3.154424in}}%
\pgfpathlineto{\pgfqpoint{5.420190in}{3.061478in}}%
\pgfpathlineto{\pgfqpoint{5.385537in}{2.931598in}}%
\pgfpathclose%
\pgfusepath{fill}%
\end{pgfscope}%
\begin{pgfscope}%
\pgfpathrectangle{\pgfqpoint{1.020000in}{0.880000in}}{\pgfqpoint{6.160000in}{6.160000in}}%
\pgfusepath{clip}%
\pgfsetbuttcap%
\pgfsetroundjoin%
\definecolor{currentfill}{rgb}{0.968533,0.715841,0.606097}%
\pgfsetfillcolor{currentfill}%
\pgfsetlinewidth{0.000000pt}%
\definecolor{currentstroke}{rgb}{0.000000,0.000000,0.000000}%
\pgfsetstrokecolor{currentstroke}%
\pgfsetdash{}{0pt}%
\pgfpathmoveto{\pgfqpoint{3.317778in}{4.625927in}}%
\pgfpathlineto{\pgfqpoint{3.327167in}{4.514318in}}%
\pgfpathlineto{\pgfqpoint{3.337526in}{4.287072in}}%
\pgfpathlineto{\pgfqpoint{3.370257in}{4.408977in}}%
\pgfpathlineto{\pgfqpoint{3.404199in}{4.381906in}}%
\pgfpathlineto{\pgfqpoint{3.395515in}{4.404542in}}%
\pgfpathlineto{\pgfqpoint{3.385736in}{4.567668in}}%
\pgfpathlineto{\pgfqpoint{3.352566in}{4.500780in}}%
\pgfpathlineto{\pgfqpoint{3.317778in}{4.625927in}}%
\pgfpathclose%
\pgfusepath{fill}%
\end{pgfscope}%
\begin{pgfscope}%
\pgfpathrectangle{\pgfqpoint{1.020000in}{0.880000in}}{\pgfqpoint{6.160000in}{6.160000in}}%
\pgfusepath{clip}%
\pgfsetbuttcap%
\pgfsetroundjoin%
\definecolor{currentfill}{rgb}{0.333490,0.446265,0.874452}%
\pgfsetfillcolor{currentfill}%
\pgfsetlinewidth{0.000000pt}%
\definecolor{currentstroke}{rgb}{0.000000,0.000000,0.000000}%
\pgfsetstrokecolor{currentstroke}%
\pgfsetdash{}{0pt}%
\pgfpathmoveto{\pgfqpoint{5.829566in}{3.216107in}}%
\pgfpathlineto{\pgfqpoint{5.836509in}{2.964393in}}%
\pgfpathlineto{\pgfqpoint{5.848216in}{2.999553in}}%
\pgfpathlineto{\pgfqpoint{5.880995in}{2.985373in}}%
\pgfpathlineto{\pgfqpoint{5.912934in}{2.924603in}}%
\pgfpathlineto{\pgfqpoint{5.902652in}{2.976603in}}%
\pgfpathlineto{\pgfqpoint{5.892077in}{3.010767in}}%
\pgfpathlineto{\pgfqpoint{5.859406in}{3.026711in}}%
\pgfpathlineto{\pgfqpoint{5.829566in}{3.216107in}}%
\pgfpathclose%
\pgfusepath{fill}%
\end{pgfscope}%
\begin{pgfscope}%
\pgfpathrectangle{\pgfqpoint{1.020000in}{0.880000in}}{\pgfqpoint{6.160000in}{6.160000in}}%
\pgfusepath{clip}%
\pgfsetbuttcap%
\pgfsetroundjoin%
\definecolor{currentfill}{rgb}{0.952761,0.782965,0.698646}%
\pgfsetfillcolor{currentfill}%
\pgfsetlinewidth{0.000000pt}%
\definecolor{currentstroke}{rgb}{0.000000,0.000000,0.000000}%
\pgfsetstrokecolor{currentstroke}%
\pgfsetdash{}{0pt}%
\pgfpathmoveto{\pgfqpoint{3.404199in}{4.381906in}}%
\pgfpathlineto{\pgfqpoint{3.414340in}{4.170390in}}%
\pgfpathlineto{\pgfqpoint{3.421299in}{4.377991in}}%
\pgfpathlineto{\pgfqpoint{3.456665in}{4.156683in}}%
\pgfpathlineto{\pgfqpoint{3.488969in}{4.356120in}}%
\pgfpathlineto{\pgfqpoint{3.480734in}{4.302545in}}%
\pgfpathlineto{\pgfqpoint{3.472329in}{4.276841in}}%
\pgfpathlineto{\pgfqpoint{3.437216in}{4.475120in}}%
\pgfpathlineto{\pgfqpoint{3.404199in}{4.381906in}}%
\pgfpathclose%
\pgfusepath{fill}%
\end{pgfscope}%
\begin{pgfscope}%
\pgfpathrectangle{\pgfqpoint{1.020000in}{0.880000in}}{\pgfqpoint{6.160000in}{6.160000in}}%
\pgfusepath{clip}%
\pgfsetbuttcap%
\pgfsetroundjoin%
\definecolor{currentfill}{rgb}{0.967711,0.662973,0.544323}%
\pgfsetfillcolor{currentfill}%
\pgfsetlinewidth{0.000000pt}%
\definecolor{currentstroke}{rgb}{0.000000,0.000000,0.000000}%
\pgfsetstrokecolor{currentstroke}%
\pgfsetdash{}{0pt}%
\pgfpathmoveto{\pgfqpoint{2.374839in}{4.524809in}}%
\pgfpathlineto{\pgfqpoint{2.382240in}{4.526663in}}%
\pgfpathlineto{\pgfqpoint{2.390033in}{4.507303in}}%
\pgfpathlineto{\pgfqpoint{2.422746in}{4.583462in}}%
\pgfpathlineto{\pgfqpoint{2.459522in}{4.423363in}}%
\pgfpathlineto{\pgfqpoint{2.447248in}{4.701292in}}%
\pgfpathlineto{\pgfqpoint{2.440442in}{4.660163in}}%
\pgfpathlineto{\pgfqpoint{2.408896in}{4.520207in}}%
\pgfpathlineto{\pgfqpoint{2.374839in}{4.524809in}}%
\pgfpathclose%
\pgfusepath{fill}%
\end{pgfscope}%
\begin{pgfscope}%
\pgfpathrectangle{\pgfqpoint{1.020000in}{0.880000in}}{\pgfqpoint{6.160000in}{6.160000in}}%
\pgfusepath{clip}%
\pgfsetbuttcap%
\pgfsetroundjoin%
\definecolor{currentfill}{rgb}{0.962701,0.628218,0.507636}%
\pgfsetfillcolor{currentfill}%
\pgfsetlinewidth{0.000000pt}%
\definecolor{currentstroke}{rgb}{0.000000,0.000000,0.000000}%
\pgfsetstrokecolor{currentstroke}%
\pgfsetdash{}{0pt}%
\pgfpathmoveto{\pgfqpoint{2.440442in}{4.660163in}}%
\pgfpathlineto{\pgfqpoint{2.447248in}{4.701292in}}%
\pgfpathlineto{\pgfqpoint{2.459522in}{4.423363in}}%
\pgfpathlineto{\pgfqpoint{2.488722in}{4.708453in}}%
\pgfpathlineto{\pgfqpoint{2.523721in}{4.650252in}}%
\pgfpathlineto{\pgfqpoint{2.517291in}{4.579623in}}%
\pgfpathlineto{\pgfqpoint{2.509145in}{4.614304in}}%
\pgfpathlineto{\pgfqpoint{2.475285in}{4.609125in}}%
\pgfpathlineto{\pgfqpoint{2.440442in}{4.660163in}}%
\pgfpathclose%
\pgfusepath{fill}%
\end{pgfscope}%
\begin{pgfscope}%
\pgfpathrectangle{\pgfqpoint{1.020000in}{0.880000in}}{\pgfqpoint{6.160000in}{6.160000in}}%
\pgfusepath{clip}%
\pgfsetbuttcap%
\pgfsetroundjoin%
\definecolor{currentfill}{rgb}{0.877149,0.394645,0.311724}%
\pgfsetfillcolor{currentfill}%
\pgfsetlinewidth{0.000000pt}%
\definecolor{currentstroke}{rgb}{0.000000,0.000000,0.000000}%
\pgfsetstrokecolor{currentstroke}%
\pgfsetdash{}{0pt}%
\pgfpathmoveto{\pgfqpoint{3.062156in}{4.946537in}}%
\pgfpathlineto{\pgfqpoint{3.069159in}{5.052714in}}%
\pgfpathlineto{\pgfqpoint{3.080599in}{4.754349in}}%
\pgfpathlineto{\pgfqpoint{3.111050in}{5.070828in}}%
\pgfpathlineto{\pgfqpoint{3.145323in}{5.031815in}}%
\pgfpathlineto{\pgfqpoint{3.139286in}{4.816156in}}%
\pgfpathlineto{\pgfqpoint{3.129352in}{4.981118in}}%
\pgfpathlineto{\pgfqpoint{3.096565in}{4.888524in}}%
\pgfpathlineto{\pgfqpoint{3.062156in}{4.946537in}}%
\pgfpathclose%
\pgfusepath{fill}%
\end{pgfscope}%
\begin{pgfscope}%
\pgfpathrectangle{\pgfqpoint{1.020000in}{0.880000in}}{\pgfqpoint{6.160000in}{6.160000in}}%
\pgfusepath{clip}%
\pgfsetbuttcap%
\pgfsetroundjoin%
\definecolor{currentfill}{rgb}{0.667253,0.779176,0.992959}%
\pgfsetfillcolor{currentfill}%
\pgfsetlinewidth{0.000000pt}%
\definecolor{currentstroke}{rgb}{0.000000,0.000000,0.000000}%
\pgfsetstrokecolor{currentstroke}%
\pgfsetdash{}{0pt}%
\pgfpathmoveto{\pgfqpoint{3.899831in}{3.526203in}}%
\pgfpathlineto{\pgfqpoint{3.908766in}{3.562331in}}%
\pgfpathlineto{\pgfqpoint{3.917667in}{3.621039in}}%
\pgfpathlineto{\pgfqpoint{3.951325in}{3.599776in}}%
\pgfpathlineto{\pgfqpoint{3.984934in}{3.585695in}}%
\pgfpathlineto{\pgfqpoint{3.975572in}{3.700230in}}%
\pgfpathlineto{\pgfqpoint{3.966514in}{3.673296in}}%
\pgfpathlineto{\pgfqpoint{3.932850in}{3.723063in}}%
\pgfpathlineto{\pgfqpoint{3.899831in}{3.526203in}}%
\pgfpathclose%
\pgfusepath{fill}%
\end{pgfscope}%
\begin{pgfscope}%
\pgfpathrectangle{\pgfqpoint{1.020000in}{0.880000in}}{\pgfqpoint{6.160000in}{6.160000in}}%
\pgfusepath{clip}%
\pgfsetbuttcap%
\pgfsetroundjoin%
\definecolor{currentfill}{rgb}{0.388852,0.516298,0.921373}%
\pgfsetfillcolor{currentfill}%
\pgfsetlinewidth{0.000000pt}%
\definecolor{currentstroke}{rgb}{0.000000,0.000000,0.000000}%
\pgfsetstrokecolor{currentstroke}%
\pgfsetdash{}{0pt}%
\pgfpathmoveto{\pgfqpoint{4.880590in}{2.950642in}}%
\pgfpathlineto{\pgfqpoint{4.891483in}{3.065615in}}%
\pgfpathlineto{\pgfqpoint{4.902388in}{3.176109in}}%
\pgfpathlineto{\pgfqpoint{4.935429in}{3.137169in}}%
\pgfpathlineto{\pgfqpoint{4.968323in}{3.085247in}}%
\pgfpathlineto{\pgfqpoint{4.958236in}{3.096656in}}%
\pgfpathlineto{\pgfqpoint{4.948583in}{3.162100in}}%
\pgfpathlineto{\pgfqpoint{4.916308in}{3.294322in}}%
\pgfpathlineto{\pgfqpoint{4.880590in}{2.950642in}}%
\pgfpathclose%
\pgfusepath{fill}%
\end{pgfscope}%
\begin{pgfscope}%
\pgfpathrectangle{\pgfqpoint{1.020000in}{0.880000in}}{\pgfqpoint{6.160000in}{6.160000in}}%
\pgfusepath{clip}%
\pgfsetbuttcap%
\pgfsetroundjoin%
\definecolor{currentfill}{rgb}{0.908908,0.462433,0.360950}%
\pgfsetfillcolor{currentfill}%
\pgfsetlinewidth{0.000000pt}%
\definecolor{currentstroke}{rgb}{0.000000,0.000000,0.000000}%
\pgfsetstrokecolor{currentstroke}%
\pgfsetdash{}{0pt}%
\pgfpathmoveto{\pgfqpoint{3.213930in}{4.933017in}}%
\pgfpathlineto{\pgfqpoint{3.223792in}{4.771569in}}%
\pgfpathlineto{\pgfqpoint{3.230541in}{4.940696in}}%
\pgfpathlineto{\pgfqpoint{3.264986in}{4.873661in}}%
\pgfpathlineto{\pgfqpoint{3.300755in}{4.647373in}}%
\pgfpathlineto{\pgfqpoint{3.291644in}{4.728924in}}%
\pgfpathlineto{\pgfqpoint{3.281402in}{4.938327in}}%
\pgfpathlineto{\pgfqpoint{3.246773in}{5.033523in}}%
\pgfpathlineto{\pgfqpoint{3.213930in}{4.933017in}}%
\pgfpathclose%
\pgfusepath{fill}%
\end{pgfscope}%
\begin{pgfscope}%
\pgfpathrectangle{\pgfqpoint{1.020000in}{0.880000in}}{\pgfqpoint{6.160000in}{6.160000in}}%
\pgfusepath{clip}%
\pgfsetbuttcap%
\pgfsetroundjoin%
\definecolor{currentfill}{rgb}{0.516260,0.654498,0.986407}%
\pgfsetfillcolor{currentfill}%
\pgfsetlinewidth{0.000000pt}%
\definecolor{currentstroke}{rgb}{0.000000,0.000000,0.000000}%
\pgfsetstrokecolor{currentstroke}%
\pgfsetdash{}{0pt}%
\pgfpathmoveto{\pgfqpoint{4.357163in}{3.325702in}}%
\pgfpathlineto{\pgfqpoint{4.366504in}{3.223243in}}%
\pgfpathlineto{\pgfqpoint{4.376380in}{3.397525in}}%
\pgfpathlineto{\pgfqpoint{4.409527in}{3.245336in}}%
\pgfpathlineto{\pgfqpoint{4.443196in}{3.340648in}}%
\pgfpathlineto{\pgfqpoint{4.433806in}{3.431261in}}%
\pgfpathlineto{\pgfqpoint{4.424034in}{3.361728in}}%
\pgfpathlineto{\pgfqpoint{4.390805in}{3.441481in}}%
\pgfpathlineto{\pgfqpoint{4.357163in}{3.325702in}}%
\pgfpathclose%
\pgfusepath{fill}%
\end{pgfscope}%
\begin{pgfscope}%
\pgfpathrectangle{\pgfqpoint{1.020000in}{0.880000in}}{\pgfqpoint{6.160000in}{6.160000in}}%
\pgfusepath{clip}%
\pgfsetbuttcap%
\pgfsetroundjoin%
\definecolor{currentfill}{rgb}{0.800830,0.250829,0.225696}%
\pgfsetfillcolor{currentfill}%
\pgfsetlinewidth{0.000000pt}%
\definecolor{currentstroke}{rgb}{0.000000,0.000000,0.000000}%
\pgfsetstrokecolor{currentstroke}%
\pgfsetdash{}{0pt}%
\pgfpathmoveto{\pgfqpoint{2.972833in}{5.413321in}}%
\pgfpathlineto{\pgfqpoint{2.983195in}{5.221724in}}%
\pgfpathlineto{\pgfqpoint{2.994202in}{4.973943in}}%
\pgfpathlineto{\pgfqpoint{3.028754in}{4.911037in}}%
\pgfpathlineto{\pgfqpoint{3.062156in}{4.946537in}}%
\pgfpathlineto{\pgfqpoint{3.052685in}{5.065307in}}%
\pgfpathlineto{\pgfqpoint{3.043515in}{5.157081in}}%
\pgfpathlineto{\pgfqpoint{3.009376in}{5.185547in}}%
\pgfpathlineto{\pgfqpoint{2.972833in}{5.413321in}}%
\pgfpathclose%
\pgfusepath{fill}%
\end{pgfscope}%
\begin{pgfscope}%
\pgfpathrectangle{\pgfqpoint{1.020000in}{0.880000in}}{\pgfqpoint{6.160000in}{6.160000in}}%
\pgfusepath{clip}%
\pgfsetbuttcap%
\pgfsetroundjoin%
\definecolor{currentfill}{rgb}{0.462354,0.599830,0.965857}%
\pgfsetfillcolor{currentfill}%
\pgfsetlinewidth{0.000000pt}%
\definecolor{currentstroke}{rgb}{0.000000,0.000000,0.000000}%
\pgfsetstrokecolor{currentstroke}%
\pgfsetdash{}{0pt}%
\pgfpathmoveto{\pgfqpoint{4.509868in}{3.287650in}}%
\pgfpathlineto{\pgfqpoint{4.519635in}{3.308314in}}%
\pgfpathlineto{\pgfqpoint{4.529434in}{3.330827in}}%
\pgfpathlineto{\pgfqpoint{4.563225in}{3.419239in}}%
\pgfpathlineto{\pgfqpoint{4.595689in}{3.177012in}}%
\pgfpathlineto{\pgfqpoint{4.585766in}{3.140054in}}%
\pgfpathlineto{\pgfqpoint{4.576348in}{3.223996in}}%
\pgfpathlineto{\pgfqpoint{4.542752in}{3.151181in}}%
\pgfpathlineto{\pgfqpoint{4.509868in}{3.287650in}}%
\pgfpathclose%
\pgfusepath{fill}%
\end{pgfscope}%
\begin{pgfscope}%
\pgfpathrectangle{\pgfqpoint{1.020000in}{0.880000in}}{\pgfqpoint{6.160000in}{6.160000in}}%
\pgfusepath{clip}%
\pgfsetbuttcap%
\pgfsetroundjoin%
\definecolor{currentfill}{rgb}{0.800601,0.850358,0.930008}%
\pgfsetfillcolor{currentfill}%
\pgfsetlinewidth{0.000000pt}%
\definecolor{currentstroke}{rgb}{0.000000,0.000000,0.000000}%
\pgfsetstrokecolor{currentstroke}%
\pgfsetdash{}{0pt}%
\pgfpathmoveto{\pgfqpoint{3.661302in}{3.891007in}}%
\pgfpathlineto{\pgfqpoint{3.670117in}{3.886547in}}%
\pgfpathlineto{\pgfqpoint{3.679022in}{3.867790in}}%
\pgfpathlineto{\pgfqpoint{3.712743in}{3.858659in}}%
\pgfpathlineto{\pgfqpoint{3.746631in}{3.805069in}}%
\pgfpathlineto{\pgfqpoint{3.737612in}{3.838635in}}%
\pgfpathlineto{\pgfqpoint{3.727976in}{4.011567in}}%
\pgfpathlineto{\pgfqpoint{3.694934in}{3.888458in}}%
\pgfpathlineto{\pgfqpoint{3.661302in}{3.891007in}}%
\pgfpathclose%
\pgfusepath{fill}%
\end{pgfscope}%
\begin{pgfscope}%
\pgfpathrectangle{\pgfqpoint{1.020000in}{0.880000in}}{\pgfqpoint{6.160000in}{6.160000in}}%
\pgfusepath{clip}%
\pgfsetbuttcap%
\pgfsetroundjoin%
\definecolor{currentfill}{rgb}{0.949454,0.572388,0.453443}%
\pgfsetfillcolor{currentfill}%
\pgfsetlinewidth{0.000000pt}%
\definecolor{currentstroke}{rgb}{0.000000,0.000000,0.000000}%
\pgfsetstrokecolor{currentstroke}%
\pgfsetdash{}{0pt}%
\pgfpathmoveto{\pgfqpoint{2.509145in}{4.614304in}}%
\pgfpathlineto{\pgfqpoint{2.517291in}{4.579623in}}%
\pgfpathlineto{\pgfqpoint{2.523721in}{4.650252in}}%
\pgfpathlineto{\pgfqpoint{2.557673in}{4.653609in}}%
\pgfpathlineto{\pgfqpoint{2.593579in}{4.531599in}}%
\pgfpathlineto{\pgfqpoint{2.581577in}{4.806914in}}%
\pgfpathlineto{\pgfqpoint{2.573592in}{4.827611in}}%
\pgfpathlineto{\pgfqpoint{2.537632in}{4.949863in}}%
\pgfpathlineto{\pgfqpoint{2.509145in}{4.614304in}}%
\pgfpathclose%
\pgfusepath{fill}%
\end{pgfscope}%
\begin{pgfscope}%
\pgfpathrectangle{\pgfqpoint{1.020000in}{0.880000in}}{\pgfqpoint{6.160000in}{6.160000in}}%
\pgfusepath{clip}%
\pgfsetbuttcap%
\pgfsetroundjoin%
\definecolor{currentfill}{rgb}{0.348323,0.465711,0.888346}%
\pgfsetfillcolor{currentfill}%
\pgfsetlinewidth{0.000000pt}%
\definecolor{currentstroke}{rgb}{0.000000,0.000000,0.000000}%
\pgfsetstrokecolor{currentstroke}%
\pgfsetdash{}{0pt}%
\pgfpathmoveto{\pgfqpoint{5.540957in}{3.068392in}}%
\pgfpathlineto{\pgfqpoint{5.551092in}{3.019670in}}%
\pgfpathlineto{\pgfqpoint{5.560631in}{2.927032in}}%
\pgfpathlineto{\pgfqpoint{5.595291in}{3.037778in}}%
\pgfpathlineto{\pgfqpoint{5.629309in}{3.100619in}}%
\pgfpathlineto{\pgfqpoint{5.617971in}{3.069849in}}%
\pgfpathlineto{\pgfqpoint{5.605511in}{2.958424in}}%
\pgfpathlineto{\pgfqpoint{5.574594in}{3.108791in}}%
\pgfpathlineto{\pgfqpoint{5.540957in}{3.068392in}}%
\pgfpathclose%
\pgfusepath{fill}%
\end{pgfscope}%
\begin{pgfscope}%
\pgfpathrectangle{\pgfqpoint{1.020000in}{0.880000in}}{\pgfqpoint{6.160000in}{6.160000in}}%
\pgfusepath{clip}%
\pgfsetbuttcap%
\pgfsetroundjoin%
\definecolor{currentfill}{rgb}{0.934305,0.525918,0.412286}%
\pgfsetfillcolor{currentfill}%
\pgfsetlinewidth{0.000000pt}%
\definecolor{currentstroke}{rgb}{0.000000,0.000000,0.000000}%
\pgfsetstrokecolor{currentstroke}%
\pgfsetdash{}{0pt}%
\pgfpathmoveto{\pgfqpoint{2.573592in}{4.827611in}}%
\pgfpathlineto{\pgfqpoint{2.581577in}{4.806914in}}%
\pgfpathlineto{\pgfqpoint{2.593579in}{4.531599in}}%
\pgfpathlineto{\pgfqpoint{2.623632in}{4.783283in}}%
\pgfpathlineto{\pgfqpoint{2.656986in}{4.825582in}}%
\pgfpathlineto{\pgfqpoint{2.649879in}{4.783657in}}%
\pgfpathlineto{\pgfqpoint{2.640885in}{4.867338in}}%
\pgfpathlineto{\pgfqpoint{2.609022in}{4.732921in}}%
\pgfpathlineto{\pgfqpoint{2.573592in}{4.827611in}}%
\pgfpathclose%
\pgfusepath{fill}%
\end{pgfscope}%
\begin{pgfscope}%
\pgfpathrectangle{\pgfqpoint{1.020000in}{0.880000in}}{\pgfqpoint{6.160000in}{6.160000in}}%
\pgfusepath{clip}%
\pgfsetbuttcap%
\pgfsetroundjoin%
\definecolor{currentfill}{rgb}{0.333490,0.446265,0.874452}%
\pgfsetfillcolor{currentfill}%
\pgfsetlinewidth{0.000000pt}%
\definecolor{currentstroke}{rgb}{0.000000,0.000000,0.000000}%
\pgfsetstrokecolor{currentstroke}%
\pgfsetdash{}{0pt}%
\pgfpathmoveto{\pgfqpoint{5.760861in}{3.052564in}}%
\pgfpathlineto{\pgfqpoint{5.771614in}{3.034418in}}%
\pgfpathlineto{\pgfqpoint{5.780007in}{2.868131in}}%
\pgfpathlineto{\pgfqpoint{5.813999in}{2.927707in}}%
\pgfpathlineto{\pgfqpoint{5.848216in}{2.999553in}}%
\pgfpathlineto{\pgfqpoint{5.836509in}{2.964393in}}%
\pgfpathlineto{\pgfqpoint{5.829566in}{3.216107in}}%
\pgfpathlineto{\pgfqpoint{5.792474in}{2.965958in}}%
\pgfpathlineto{\pgfqpoint{5.760861in}{3.052564in}}%
\pgfpathclose%
\pgfusepath{fill}%
\end{pgfscope}%
\begin{pgfscope}%
\pgfpathrectangle{\pgfqpoint{1.020000in}{0.880000in}}{\pgfqpoint{6.160000in}{6.160000in}}%
\pgfusepath{clip}%
\pgfsetbuttcap%
\pgfsetroundjoin%
\definecolor{currentfill}{rgb}{0.457046,0.594006,0.963029}%
\pgfsetfillcolor{currentfill}%
\pgfsetlinewidth{0.000000pt}%
\definecolor{currentstroke}{rgb}{0.000000,0.000000,0.000000}%
\pgfsetstrokecolor{currentstroke}%
\pgfsetdash{}{0pt}%
\pgfpathmoveto{\pgfqpoint{4.728287in}{3.041990in}}%
\pgfpathlineto{\pgfqpoint{4.739565in}{3.287420in}}%
\pgfpathlineto{\pgfqpoint{4.749770in}{3.331088in}}%
\pgfpathlineto{\pgfqpoint{4.782639in}{3.242425in}}%
\pgfpathlineto{\pgfqpoint{4.815933in}{3.233317in}}%
\pgfpathlineto{\pgfqpoint{4.806243in}{3.285689in}}%
\pgfpathlineto{\pgfqpoint{4.796087in}{3.260487in}}%
\pgfpathlineto{\pgfqpoint{4.762526in}{3.214280in}}%
\pgfpathlineto{\pgfqpoint{4.728287in}{3.041990in}}%
\pgfpathclose%
\pgfusepath{fill}%
\end{pgfscope}%
\begin{pgfscope}%
\pgfpathrectangle{\pgfqpoint{1.020000in}{0.880000in}}{\pgfqpoint{6.160000in}{6.160000in}}%
\pgfusepath{clip}%
\pgfsetbuttcap%
\pgfsetroundjoin%
\definecolor{currentfill}{rgb}{0.304174,0.406945,0.845263}%
\pgfsetfillcolor{currentfill}%
\pgfsetlinewidth{0.000000pt}%
\definecolor{currentstroke}{rgb}{0.000000,0.000000,0.000000}%
\pgfsetstrokecolor{currentstroke}%
\pgfsetdash{}{0pt}%
\pgfpathmoveto{\pgfqpoint{5.693668in}{2.979500in}}%
\pgfpathlineto{\pgfqpoint{5.704042in}{2.940567in}}%
\pgfpathlineto{\pgfqpoint{5.716777in}{3.054004in}}%
\pgfpathlineto{\pgfqpoint{5.745906in}{2.800378in}}%
\pgfpathlineto{\pgfqpoint{5.780007in}{2.868131in}}%
\pgfpathlineto{\pgfqpoint{5.771614in}{3.034418in}}%
\pgfpathlineto{\pgfqpoint{5.760861in}{3.052564in}}%
\pgfpathlineto{\pgfqpoint{5.724290in}{2.824337in}}%
\pgfpathlineto{\pgfqpoint{5.693668in}{2.979500in}}%
\pgfpathclose%
\pgfusepath{fill}%
\end{pgfscope}%
\begin{pgfscope}%
\pgfpathrectangle{\pgfqpoint{1.020000in}{0.880000in}}{\pgfqpoint{6.160000in}{6.160000in}}%
\pgfusepath{clip}%
\pgfsetbuttcap%
\pgfsetroundjoin%
\definecolor{currentfill}{rgb}{0.309060,0.413498,0.850128}%
\pgfsetfillcolor{currentfill}%
\pgfsetlinewidth{0.000000pt}%
\definecolor{currentstroke}{rgb}{0.000000,0.000000,0.000000}%
\pgfsetstrokecolor{currentstroke}%
\pgfsetdash{}{0pt}%
\pgfpathmoveto{\pgfqpoint{5.980003in}{2.989885in}}%
\pgfpathlineto{\pgfqpoint{5.986850in}{2.746049in}}%
\pgfpathlineto{\pgfqpoint{6.001979in}{2.957146in}}%
\pgfpathlineto{\pgfqpoint{6.035877in}{3.005929in}}%
\pgfpathlineto{\pgfqpoint{6.025741in}{3.070623in}}%
\pgfpathlineto{\pgfqpoint{6.013020in}{2.994134in}}%
\pgfpathlineto{\pgfqpoint{5.980003in}{2.989885in}}%
\pgfpathclose%
\pgfusepath{fill}%
\end{pgfscope}%
\begin{pgfscope}%
\pgfpathrectangle{\pgfqpoint{1.020000in}{0.880000in}}{\pgfqpoint{6.160000in}{6.160000in}}%
\pgfusepath{clip}%
\pgfsetbuttcap%
\pgfsetroundjoin%
\definecolor{currentfill}{rgb}{0.952761,0.782965,0.698646}%
\pgfsetfillcolor{currentfill}%
\pgfsetlinewidth{0.000000pt}%
\definecolor{currentstroke}{rgb}{0.000000,0.000000,0.000000}%
\pgfsetstrokecolor{currentstroke}%
\pgfsetdash{}{0pt}%
\pgfpathmoveto{\pgfqpoint{3.337526in}{4.287072in}}%
\pgfpathlineto{\pgfqpoint{3.345735in}{4.316202in}}%
\pgfpathlineto{\pgfqpoint{3.354308in}{4.303974in}}%
\pgfpathlineto{\pgfqpoint{3.388747in}{4.220923in}}%
\pgfpathlineto{\pgfqpoint{3.421299in}{4.377991in}}%
\pgfpathlineto{\pgfqpoint{3.414340in}{4.170390in}}%
\pgfpathlineto{\pgfqpoint{3.404199in}{4.381906in}}%
\pgfpathlineto{\pgfqpoint{3.370257in}{4.408977in}}%
\pgfpathlineto{\pgfqpoint{3.337526in}{4.287072in}}%
\pgfpathclose%
\pgfusepath{fill}%
\end{pgfscope}%
\begin{pgfscope}%
\pgfpathrectangle{\pgfqpoint{1.020000in}{0.880000in}}{\pgfqpoint{6.160000in}{6.160000in}}%
\pgfusepath{clip}%
\pgfsetbuttcap%
\pgfsetroundjoin%
\definecolor{currentfill}{rgb}{0.430507,0.564883,0.948889}%
\pgfsetfillcolor{currentfill}%
\pgfsetlinewidth{0.000000pt}%
\definecolor{currentstroke}{rgb}{0.000000,0.000000,0.000000}%
\pgfsetstrokecolor{currentstroke}%
\pgfsetdash{}{0pt}%
\pgfpathmoveto{\pgfqpoint{4.662577in}{3.214556in}}%
\pgfpathlineto{\pgfqpoint{4.672316in}{3.189281in}}%
\pgfpathlineto{\pgfqpoint{4.682041in}{3.157826in}}%
\pgfpathlineto{\pgfqpoint{4.715184in}{3.112770in}}%
\pgfpathlineto{\pgfqpoint{4.749770in}{3.331088in}}%
\pgfpathlineto{\pgfqpoint{4.739565in}{3.287420in}}%
\pgfpathlineto{\pgfqpoint{4.728287in}{3.041990in}}%
\pgfpathlineto{\pgfqpoint{4.695696in}{3.172055in}}%
\pgfpathlineto{\pgfqpoint{4.662577in}{3.214556in}}%
\pgfpathclose%
\pgfusepath{fill}%
\end{pgfscope}%
\begin{pgfscope}%
\pgfpathrectangle{\pgfqpoint{1.020000in}{0.880000in}}{\pgfqpoint{6.160000in}{6.160000in}}%
\pgfusepath{clip}%
\pgfsetbuttcap%
\pgfsetroundjoin%
\definecolor{currentfill}{rgb}{0.505423,0.643995,0.983157}%
\pgfsetfillcolor{currentfill}%
\pgfsetlinewidth{0.000000pt}%
\definecolor{currentstroke}{rgb}{0.000000,0.000000,0.000000}%
\pgfsetstrokecolor{currentstroke}%
\pgfsetdash{}{0pt}%
\pgfpathmoveto{\pgfqpoint{4.290402in}{3.421580in}}%
\pgfpathlineto{\pgfqpoint{4.299818in}{3.372978in}}%
\pgfpathlineto{\pgfqpoint{4.309081in}{3.194771in}}%
\pgfpathlineto{\pgfqpoint{4.342780in}{3.342283in}}%
\pgfpathlineto{\pgfqpoint{4.376380in}{3.397525in}}%
\pgfpathlineto{\pgfqpoint{4.366504in}{3.223243in}}%
\pgfpathlineto{\pgfqpoint{4.357163in}{3.325702in}}%
\pgfpathlineto{\pgfqpoint{4.323703in}{3.298174in}}%
\pgfpathlineto{\pgfqpoint{4.290402in}{3.421580in}}%
\pgfpathclose%
\pgfusepath{fill}%
\end{pgfscope}%
\begin{pgfscope}%
\pgfpathrectangle{\pgfqpoint{1.020000in}{0.880000in}}{\pgfqpoint{6.160000in}{6.160000in}}%
\pgfusepath{clip}%
\pgfsetbuttcap%
\pgfsetroundjoin%
\definecolor{currentfill}{rgb}{0.958176,0.771234,0.680301}%
\pgfsetfillcolor{currentfill}%
\pgfsetlinewidth{0.000000pt}%
\definecolor{currentstroke}{rgb}{0.000000,0.000000,0.000000}%
\pgfsetstrokecolor{currentstroke}%
\pgfsetdash{}{0pt}%
\pgfpathmoveto{\pgfqpoint{2.326229in}{4.272082in}}%
\pgfpathlineto{\pgfqpoint{2.337000in}{4.087178in}}%
\pgfpathlineto{\pgfqpoint{2.340159in}{4.318287in}}%
\pgfpathlineto{\pgfqpoint{2.373604in}{4.353810in}}%
\pgfpathlineto{\pgfqpoint{2.406965in}{4.394356in}}%
\pgfpathlineto{\pgfqpoint{2.398097in}{4.473392in}}%
\pgfpathlineto{\pgfqpoint{2.390033in}{4.507303in}}%
\pgfpathlineto{\pgfqpoint{2.359849in}{4.293015in}}%
\pgfpathlineto{\pgfqpoint{2.326229in}{4.272082in}}%
\pgfpathclose%
\pgfusepath{fill}%
\end{pgfscope}%
\begin{pgfscope}%
\pgfpathrectangle{\pgfqpoint{1.020000in}{0.880000in}}{\pgfqpoint{6.160000in}{6.160000in}}%
\pgfusepath{clip}%
\pgfsetbuttcap%
\pgfsetroundjoin%
\definecolor{currentfill}{rgb}{0.383662,0.510183,0.917831}%
\pgfsetfillcolor{currentfill}%
\pgfsetlinewidth{0.000000pt}%
\definecolor{currentstroke}{rgb}{0.000000,0.000000,0.000000}%
\pgfsetstrokecolor{currentstroke}%
\pgfsetdash{}{0pt}%
\pgfpathmoveto{\pgfqpoint{4.815933in}{3.233317in}}%
\pgfpathlineto{\pgfqpoint{4.825205in}{3.114342in}}%
\pgfpathlineto{\pgfqpoint{4.835279in}{3.118262in}}%
\pgfpathlineto{\pgfqpoint{4.867972in}{3.021870in}}%
\pgfpathlineto{\pgfqpoint{4.902388in}{3.176109in}}%
\pgfpathlineto{\pgfqpoint{4.891483in}{3.065615in}}%
\pgfpathlineto{\pgfqpoint{4.880590in}{2.950642in}}%
\pgfpathlineto{\pgfqpoint{4.848589in}{3.132451in}}%
\pgfpathlineto{\pgfqpoint{4.815933in}{3.233317in}}%
\pgfpathclose%
\pgfusepath{fill}%
\end{pgfscope}%
\begin{pgfscope}%
\pgfpathrectangle{\pgfqpoint{1.020000in}{0.880000in}}{\pgfqpoint{6.160000in}{6.160000in}}%
\pgfusepath{clip}%
\pgfsetbuttcap%
\pgfsetroundjoin%
\definecolor{currentfill}{rgb}{0.743754,0.825125,0.965798}%
\pgfsetfillcolor{currentfill}%
\pgfsetlinewidth{0.000000pt}%
\definecolor{currentstroke}{rgb}{0.000000,0.000000,0.000000}%
\pgfsetstrokecolor{currentstroke}%
\pgfsetdash{}{0pt}%
\pgfpathmoveto{\pgfqpoint{3.746631in}{3.805069in}}%
\pgfpathlineto{\pgfqpoint{3.755586in}{3.789531in}}%
\pgfpathlineto{\pgfqpoint{3.764329in}{3.828684in}}%
\pgfpathlineto{\pgfqpoint{3.798357in}{3.740920in}}%
\pgfpathlineto{\pgfqpoint{3.831990in}{3.743479in}}%
\pgfpathlineto{\pgfqpoint{3.823159in}{3.705418in}}%
\pgfpathlineto{\pgfqpoint{3.814458in}{3.638744in}}%
\pgfpathlineto{\pgfqpoint{3.780073in}{3.848792in}}%
\pgfpathlineto{\pgfqpoint{3.746631in}{3.805069in}}%
\pgfpathclose%
\pgfusepath{fill}%
\end{pgfscope}%
\begin{pgfscope}%
\pgfpathrectangle{\pgfqpoint{1.020000in}{0.880000in}}{\pgfqpoint{6.160000in}{6.160000in}}%
\pgfusepath{clip}%
\pgfsetbuttcap%
\pgfsetroundjoin%
\definecolor{currentfill}{rgb}{0.949454,0.572388,0.453443}%
\pgfsetfillcolor{currentfill}%
\pgfsetlinewidth{0.000000pt}%
\definecolor{currentstroke}{rgb}{0.000000,0.000000,0.000000}%
\pgfsetstrokecolor{currentstroke}%
\pgfsetdash{}{0pt}%
\pgfpathmoveto{\pgfqpoint{3.230541in}{4.940696in}}%
\pgfpathlineto{\pgfqpoint{3.241760in}{4.634355in}}%
\pgfpathlineto{\pgfqpoint{3.249336in}{4.719473in}}%
\pgfpathlineto{\pgfqpoint{3.284576in}{4.560921in}}%
\pgfpathlineto{\pgfqpoint{3.317778in}{4.625927in}}%
\pgfpathlineto{\pgfqpoint{3.310101in}{4.539063in}}%
\pgfpathlineto{\pgfqpoint{3.300755in}{4.647373in}}%
\pgfpathlineto{\pgfqpoint{3.264986in}{4.873661in}}%
\pgfpathlineto{\pgfqpoint{3.230541in}{4.940696in}}%
\pgfpathclose%
\pgfusepath{fill}%
\end{pgfscope}%
\begin{pgfscope}%
\pgfpathrectangle{\pgfqpoint{1.020000in}{0.880000in}}{\pgfqpoint{6.160000in}{6.160000in}}%
\pgfusepath{clip}%
\pgfsetbuttcap%
\pgfsetroundjoin%
\definecolor{currentfill}{rgb}{0.299441,0.400248,0.839842}%
\pgfsetfillcolor{currentfill}%
\pgfsetlinewidth{0.000000pt}%
\definecolor{currentstroke}{rgb}{0.000000,0.000000,0.000000}%
\pgfsetstrokecolor{currentstroke}%
\pgfsetdash{}{0pt}%
\pgfpathmoveto{\pgfqpoint{5.912934in}{2.924603in}}%
\pgfpathlineto{\pgfqpoint{5.925052in}{2.976930in}}%
\pgfpathlineto{\pgfqpoint{5.935124in}{2.910990in}}%
\pgfpathlineto{\pgfqpoint{5.969104in}{2.964757in}}%
\pgfpathlineto{\pgfqpoint{6.001979in}{2.957146in}}%
\pgfpathlineto{\pgfqpoint{5.986850in}{2.746049in}}%
\pgfpathlineto{\pgfqpoint{5.980003in}{2.989885in}}%
\pgfpathlineto{\pgfqpoint{5.947951in}{3.040951in}}%
\pgfpathlineto{\pgfqpoint{5.912934in}{2.924603in}}%
\pgfpathclose%
\pgfusepath{fill}%
\end{pgfscope}%
\begin{pgfscope}%
\pgfpathrectangle{\pgfqpoint{1.020000in}{0.880000in}}{\pgfqpoint{6.160000in}{6.160000in}}%
\pgfusepath{clip}%
\pgfsetbuttcap%
\pgfsetroundjoin%
\definecolor{currentfill}{rgb}{0.570616,0.704109,0.997195}%
\pgfsetfillcolor{currentfill}%
\pgfsetlinewidth{0.000000pt}%
\definecolor{currentstroke}{rgb}{0.000000,0.000000,0.000000}%
\pgfsetstrokecolor{currentstroke}%
\pgfsetdash{}{0pt}%
\pgfpathmoveto{\pgfqpoint{4.137682in}{3.431002in}}%
\pgfpathlineto{\pgfqpoint{4.146989in}{3.415249in}}%
\pgfpathlineto{\pgfqpoint{4.156309in}{3.417244in}}%
\pgfpathlineto{\pgfqpoint{4.189862in}{3.435907in}}%
\pgfpathlineto{\pgfqpoint{4.223386in}{3.400212in}}%
\pgfpathlineto{\pgfqpoint{4.214041in}{3.526981in}}%
\pgfpathlineto{\pgfqpoint{4.204630in}{3.365781in}}%
\pgfpathlineto{\pgfqpoint{4.171159in}{3.548036in}}%
\pgfpathlineto{\pgfqpoint{4.137682in}{3.431002in}}%
\pgfpathclose%
\pgfusepath{fill}%
\end{pgfscope}%
\begin{pgfscope}%
\pgfpathrectangle{\pgfqpoint{1.020000in}{0.880000in}}{\pgfqpoint{6.160000in}{6.160000in}}%
\pgfusepath{clip}%
\pgfsetbuttcap%
\pgfsetroundjoin%
\definecolor{currentfill}{rgb}{0.378598,0.503856,0.913692}%
\pgfsetfillcolor{currentfill}%
\pgfsetlinewidth{0.000000pt}%
\definecolor{currentstroke}{rgb}{0.000000,0.000000,0.000000}%
\pgfsetstrokecolor{currentstroke}%
\pgfsetdash{}{0pt}%
\pgfpathmoveto{\pgfqpoint{5.034997in}{3.104380in}}%
\pgfpathlineto{\pgfqpoint{5.045514in}{3.133634in}}%
\pgfpathlineto{\pgfqpoint{5.055738in}{3.126077in}}%
\pgfpathlineto{\pgfqpoint{5.088813in}{3.100865in}}%
\pgfpathlineto{\pgfqpoint{5.122723in}{3.168170in}}%
\pgfpathlineto{\pgfqpoint{5.110706in}{2.990141in}}%
\pgfpathlineto{\pgfqpoint{5.101004in}{3.056927in}}%
\pgfpathlineto{\pgfqpoint{5.067747in}{3.050168in}}%
\pgfpathlineto{\pgfqpoint{5.034997in}{3.104380in}}%
\pgfpathclose%
\pgfusepath{fill}%
\end{pgfscope}%
\begin{pgfscope}%
\pgfpathrectangle{\pgfqpoint{1.020000in}{0.880000in}}{\pgfqpoint{6.160000in}{6.160000in}}%
\pgfusepath{clip}%
\pgfsetbuttcap%
\pgfsetroundjoin%
\definecolor{currentfill}{rgb}{0.902659,0.447939,0.349721}%
\pgfsetfillcolor{currentfill}%
\pgfsetlinewidth{0.000000pt}%
\definecolor{currentstroke}{rgb}{0.000000,0.000000,0.000000}%
\pgfsetstrokecolor{currentstroke}%
\pgfsetdash{}{0pt}%
\pgfpathmoveto{\pgfqpoint{3.145323in}{5.031815in}}%
\pgfpathlineto{\pgfqpoint{3.155556in}{4.838531in}}%
\pgfpathlineto{\pgfqpoint{3.165377in}{4.684881in}}%
\pgfpathlineto{\pgfqpoint{3.198317in}{4.772571in}}%
\pgfpathlineto{\pgfqpoint{3.230541in}{4.940696in}}%
\pgfpathlineto{\pgfqpoint{3.223792in}{4.771569in}}%
\pgfpathlineto{\pgfqpoint{3.213930in}{4.933017in}}%
\pgfpathlineto{\pgfqpoint{3.180025in}{4.944801in}}%
\pgfpathlineto{\pgfqpoint{3.145323in}{5.031815in}}%
\pgfpathclose%
\pgfusepath{fill}%
\end{pgfscope}%
\begin{pgfscope}%
\pgfpathrectangle{\pgfqpoint{1.020000in}{0.880000in}}{\pgfqpoint{6.160000in}{6.160000in}}%
\pgfusepath{clip}%
\pgfsetbuttcap%
\pgfsetroundjoin%
\definecolor{currentfill}{rgb}{0.925563,0.825517,0.771136}%
\pgfsetfillcolor{currentfill}%
\pgfsetlinewidth{0.000000pt}%
\definecolor{currentstroke}{rgb}{0.000000,0.000000,0.000000}%
\pgfsetstrokecolor{currentstroke}%
\pgfsetdash{}{0pt}%
\pgfpathmoveto{\pgfqpoint{3.421299in}{4.377991in}}%
\pgfpathlineto{\pgfqpoint{3.431530in}{4.154341in}}%
\pgfpathlineto{\pgfqpoint{3.440072in}{4.156810in}}%
\pgfpathlineto{\pgfqpoint{3.473810in}{4.160296in}}%
\pgfpathlineto{\pgfqpoint{3.508199in}{4.064027in}}%
\pgfpathlineto{\pgfqpoint{3.499294in}{4.106051in}}%
\pgfpathlineto{\pgfqpoint{3.488969in}{4.356120in}}%
\pgfpathlineto{\pgfqpoint{3.456665in}{4.156683in}}%
\pgfpathlineto{\pgfqpoint{3.421299in}{4.377991in}}%
\pgfpathclose%
\pgfusepath{fill}%
\end{pgfscope}%
\begin{pgfscope}%
\pgfpathrectangle{\pgfqpoint{1.020000in}{0.880000in}}{\pgfqpoint{6.160000in}{6.160000in}}%
\pgfusepath{clip}%
\pgfsetbuttcap%
\pgfsetroundjoin%
\definecolor{currentfill}{rgb}{0.865391,0.371128,0.295769}%
\pgfsetfillcolor{currentfill}%
\pgfsetlinewidth{0.000000pt}%
\definecolor{currentstroke}{rgb}{0.000000,0.000000,0.000000}%
\pgfsetstrokecolor{currentstroke}%
\pgfsetdash{}{0pt}%
\pgfpathmoveto{\pgfqpoint{2.994202in}{4.973943in}}%
\pgfpathlineto{\pgfqpoint{3.003249in}{4.892779in}}%
\pgfpathlineto{\pgfqpoint{3.009116in}{5.087282in}}%
\pgfpathlineto{\pgfqpoint{3.042897in}{5.101490in}}%
\pgfpathlineto{\pgfqpoint{3.080599in}{4.754349in}}%
\pgfpathlineto{\pgfqpoint{3.069159in}{5.052714in}}%
\pgfpathlineto{\pgfqpoint{3.062156in}{4.946537in}}%
\pgfpathlineto{\pgfqpoint{3.028754in}{4.911037in}}%
\pgfpathlineto{\pgfqpoint{2.994202in}{4.973943in}}%
\pgfpathclose%
\pgfusepath{fill}%
\end{pgfscope}%
\begin{pgfscope}%
\pgfpathrectangle{\pgfqpoint{1.020000in}{0.880000in}}{\pgfqpoint{6.160000in}{6.160000in}}%
\pgfusepath{clip}%
\pgfsetbuttcap%
\pgfsetroundjoin%
\definecolor{currentfill}{rgb}{0.368507,0.491141,0.905243}%
\pgfsetfillcolor{currentfill}%
\pgfsetlinewidth{0.000000pt}%
\definecolor{currentstroke}{rgb}{0.000000,0.000000,0.000000}%
\pgfsetstrokecolor{currentstroke}%
\pgfsetdash{}{0pt}%
\pgfpathmoveto{\pgfqpoint{5.255822in}{3.177896in}}%
\pgfpathlineto{\pgfqpoint{5.266736in}{3.214959in}}%
\pgfpathlineto{\pgfqpoint{5.276573in}{3.151054in}}%
\pgfpathlineto{\pgfqpoint{5.308670in}{3.048230in}}%
\pgfpathlineto{\pgfqpoint{5.341410in}{3.008350in}}%
\pgfpathlineto{\pgfqpoint{5.330291in}{2.961645in}}%
\pgfpathlineto{\pgfqpoint{5.319054in}{2.901765in}}%
\pgfpathlineto{\pgfqpoint{5.287664in}{3.055174in}}%
\pgfpathlineto{\pgfqpoint{5.255822in}{3.177896in}}%
\pgfpathclose%
\pgfusepath{fill}%
\end{pgfscope}%
\begin{pgfscope}%
\pgfpathrectangle{\pgfqpoint{1.020000in}{0.880000in}}{\pgfqpoint{6.160000in}{6.160000in}}%
\pgfusepath{clip}%
\pgfsetbuttcap%
\pgfsetroundjoin%
\definecolor{currentfill}{rgb}{0.848040,0.338280,0.275206}%
\pgfsetfillcolor{currentfill}%
\pgfsetlinewidth{0.000000pt}%
\definecolor{currentstroke}{rgb}{0.000000,0.000000,0.000000}%
\pgfsetstrokecolor{currentstroke}%
\pgfsetdash{}{0pt}%
\pgfpathmoveto{\pgfqpoint{2.772588in}{5.148990in}}%
\pgfpathlineto{\pgfqpoint{2.782046in}{5.034839in}}%
\pgfpathlineto{\pgfqpoint{2.792051in}{4.880652in}}%
\pgfpathlineto{\pgfqpoint{2.826758in}{4.822130in}}%
\pgfpathlineto{\pgfqpoint{2.858747in}{4.967415in}}%
\pgfpathlineto{\pgfqpoint{2.850061in}{5.023263in}}%
\pgfpathlineto{\pgfqpoint{2.843292in}{4.934530in}}%
\pgfpathlineto{\pgfqpoint{2.805832in}{5.200880in}}%
\pgfpathlineto{\pgfqpoint{2.772588in}{5.148990in}}%
\pgfpathclose%
\pgfusepath{fill}%
\end{pgfscope}%
\begin{pgfscope}%
\pgfpathrectangle{\pgfqpoint{1.020000in}{0.880000in}}{\pgfqpoint{6.160000in}{6.160000in}}%
\pgfusepath{clip}%
\pgfsetbuttcap%
\pgfsetroundjoin%
\definecolor{currentfill}{rgb}{0.683056,0.790043,0.989768}%
\pgfsetfillcolor{currentfill}%
\pgfsetlinewidth{0.000000pt}%
\definecolor{currentstroke}{rgb}{0.000000,0.000000,0.000000}%
\pgfsetstrokecolor{currentstroke}%
\pgfsetdash{}{0pt}%
\pgfpathmoveto{\pgfqpoint{3.831990in}{3.743479in}}%
\pgfpathlineto{\pgfqpoint{3.841093in}{3.709766in}}%
\pgfpathlineto{\pgfqpoint{3.850482in}{3.593671in}}%
\pgfpathlineto{\pgfqpoint{3.883613in}{3.763644in}}%
\pgfpathlineto{\pgfqpoint{3.917667in}{3.621039in}}%
\pgfpathlineto{\pgfqpoint{3.908766in}{3.562331in}}%
\pgfpathlineto{\pgfqpoint{3.899831in}{3.526203in}}%
\pgfpathlineto{\pgfqpoint{3.865898in}{3.653431in}}%
\pgfpathlineto{\pgfqpoint{3.831990in}{3.743479in}}%
\pgfpathclose%
\pgfusepath{fill}%
\end{pgfscope}%
\begin{pgfscope}%
\pgfpathrectangle{\pgfqpoint{1.020000in}{0.880000in}}{\pgfqpoint{6.160000in}{6.160000in}}%
\pgfusepath{clip}%
\pgfsetbuttcap%
\pgfsetroundjoin%
\definecolor{currentfill}{rgb}{0.635474,0.756714,0.998297}%
\pgfsetfillcolor{currentfill}%
\pgfsetlinewidth{0.000000pt}%
\definecolor{currentstroke}{rgb}{0.000000,0.000000,0.000000}%
\pgfsetstrokecolor{currentstroke}%
\pgfsetdash{}{0pt}%
\pgfpathmoveto{\pgfqpoint{3.984934in}{3.585695in}}%
\pgfpathlineto{\pgfqpoint{3.993964in}{3.645855in}}%
\pgfpathlineto{\pgfqpoint{4.003446in}{3.472030in}}%
\pgfpathlineto{\pgfqpoint{4.037131in}{3.402802in}}%
\pgfpathlineto{\pgfqpoint{4.070577in}{3.487307in}}%
\pgfpathlineto{\pgfqpoint{4.061144in}{3.662085in}}%
\pgfpathlineto{\pgfqpoint{4.052106in}{3.525255in}}%
\pgfpathlineto{\pgfqpoint{4.018387in}{3.648052in}}%
\pgfpathlineto{\pgfqpoint{3.984934in}{3.585695in}}%
\pgfpathclose%
\pgfusepath{fill}%
\end{pgfscope}%
\begin{pgfscope}%
\pgfpathrectangle{\pgfqpoint{1.020000in}{0.880000in}}{\pgfqpoint{6.160000in}{6.160000in}}%
\pgfusepath{clip}%
\pgfsetbuttcap%
\pgfsetroundjoin%
\definecolor{currentfill}{rgb}{0.918282,0.484173,0.377794}%
\pgfsetfillcolor{currentfill}%
\pgfsetlinewidth{0.000000pt}%
\definecolor{currentstroke}{rgb}{0.000000,0.000000,0.000000}%
\pgfsetstrokecolor{currentstroke}%
\pgfsetdash{}{0pt}%
\pgfpathmoveto{\pgfqpoint{3.080599in}{4.754349in}}%
\pgfpathlineto{\pgfqpoint{3.087933in}{4.833239in}}%
\pgfpathlineto{\pgfqpoint{3.098299in}{4.632032in}}%
\pgfpathlineto{\pgfqpoint{3.131146in}{4.725221in}}%
\pgfpathlineto{\pgfqpoint{3.165377in}{4.684881in}}%
\pgfpathlineto{\pgfqpoint{3.155556in}{4.838531in}}%
\pgfpathlineto{\pgfqpoint{3.145323in}{5.031815in}}%
\pgfpathlineto{\pgfqpoint{3.111050in}{5.070828in}}%
\pgfpathlineto{\pgfqpoint{3.080599in}{4.754349in}}%
\pgfpathclose%
\pgfusepath{fill}%
\end{pgfscope}%
\begin{pgfscope}%
\pgfpathrectangle{\pgfqpoint{1.020000in}{0.880000in}}{\pgfqpoint{6.160000in}{6.160000in}}%
\pgfusepath{clip}%
\pgfsetbuttcap%
\pgfsetroundjoin%
\definecolor{currentfill}{rgb}{0.875557,0.860242,0.851430}%
\pgfsetfillcolor{currentfill}%
\pgfsetlinewidth{0.000000pt}%
\definecolor{currentstroke}{rgb}{0.000000,0.000000,0.000000}%
\pgfsetstrokecolor{currentstroke}%
\pgfsetdash{}{0pt}%
\pgfpathmoveto{\pgfqpoint{3.508199in}{4.064027in}}%
\pgfpathlineto{\pgfqpoint{3.516491in}{4.116274in}}%
\pgfpathlineto{\pgfqpoint{3.525293in}{4.094674in}}%
\pgfpathlineto{\pgfqpoint{3.559433in}{4.034551in}}%
\pgfpathlineto{\pgfqpoint{3.592898in}{4.079315in}}%
\pgfpathlineto{\pgfqpoint{3.584389in}{4.043258in}}%
\pgfpathlineto{\pgfqpoint{3.576076in}{3.978781in}}%
\pgfpathlineto{\pgfqpoint{3.542269in}{4.004560in}}%
\pgfpathlineto{\pgfqpoint{3.508199in}{4.064027in}}%
\pgfpathclose%
\pgfusepath{fill}%
\end{pgfscope}%
\begin{pgfscope}%
\pgfpathrectangle{\pgfqpoint{1.020000in}{0.880000in}}{\pgfqpoint{6.160000in}{6.160000in}}%
\pgfusepath{clip}%
\pgfsetbuttcap%
\pgfsetroundjoin%
\definecolor{currentfill}{rgb}{0.810616,0.268797,0.235428}%
\pgfsetfillcolor{currentfill}%
\pgfsetlinewidth{0.000000pt}%
\definecolor{currentstroke}{rgb}{0.000000,0.000000,0.000000}%
\pgfsetstrokecolor{currentstroke}%
\pgfsetdash{}{0pt}%
\pgfpathmoveto{\pgfqpoint{2.843292in}{4.934530in}}%
\pgfpathlineto{\pgfqpoint{2.850061in}{5.023263in}}%
\pgfpathlineto{\pgfqpoint{2.858747in}{4.967415in}}%
\pgfpathlineto{\pgfqpoint{2.890126in}{5.166772in}}%
\pgfpathlineto{\pgfqpoint{2.923621in}{5.204233in}}%
\pgfpathlineto{\pgfqpoint{2.916462in}{5.133711in}}%
\pgfpathlineto{\pgfqpoint{2.908666in}{5.116216in}}%
\pgfpathlineto{\pgfqpoint{2.875353in}{5.072444in}}%
\pgfpathlineto{\pgfqpoint{2.843292in}{4.934530in}}%
\pgfpathclose%
\pgfusepath{fill}%
\end{pgfscope}%
\begin{pgfscope}%
\pgfpathrectangle{\pgfqpoint{1.020000in}{0.880000in}}{\pgfqpoint{6.160000in}{6.160000in}}%
\pgfusepath{clip}%
\pgfsetbuttcap%
\pgfsetroundjoin%
\definecolor{currentfill}{rgb}{0.888390,0.417703,0.327898}%
\pgfsetfillcolor{currentfill}%
\pgfsetlinewidth{0.000000pt}%
\definecolor{currentstroke}{rgb}{0.000000,0.000000,0.000000}%
\pgfsetstrokecolor{currentstroke}%
\pgfsetdash{}{0pt}%
\pgfpathmoveto{\pgfqpoint{2.640885in}{4.867338in}}%
\pgfpathlineto{\pgfqpoint{2.649879in}{4.783657in}}%
\pgfpathlineto{\pgfqpoint{2.656986in}{4.825582in}}%
\pgfpathlineto{\pgfqpoint{2.690852in}{4.833699in}}%
\pgfpathlineto{\pgfqpoint{2.723038in}{4.957535in}}%
\pgfpathlineto{\pgfqpoint{2.713944in}{5.045803in}}%
\pgfpathlineto{\pgfqpoint{2.707702in}{4.937981in}}%
\pgfpathlineto{\pgfqpoint{2.673388in}{4.963472in}}%
\pgfpathlineto{\pgfqpoint{2.640885in}{4.867338in}}%
\pgfpathclose%
\pgfusepath{fill}%
\end{pgfscope}%
\begin{pgfscope}%
\pgfpathrectangle{\pgfqpoint{1.020000in}{0.880000in}}{\pgfqpoint{6.160000in}{6.160000in}}%
\pgfusepath{clip}%
\pgfsetbuttcap%
\pgfsetroundjoin%
\definecolor{currentfill}{rgb}{0.968863,0.710838,0.599901}%
\pgfsetfillcolor{currentfill}%
\pgfsetlinewidth{0.000000pt}%
\definecolor{currentstroke}{rgb}{0.000000,0.000000,0.000000}%
\pgfsetstrokecolor{currentstroke}%
\pgfsetdash{}{0pt}%
\pgfpathmoveto{\pgfqpoint{2.390033in}{4.507303in}}%
\pgfpathlineto{\pgfqpoint{2.398097in}{4.473392in}}%
\pgfpathlineto{\pgfqpoint{2.406965in}{4.394356in}}%
\pgfpathlineto{\pgfqpoint{2.442767in}{4.293814in}}%
\pgfpathlineto{\pgfqpoint{2.474162in}{4.448568in}}%
\pgfpathlineto{\pgfqpoint{2.465128in}{4.536316in}}%
\pgfpathlineto{\pgfqpoint{2.459522in}{4.423363in}}%
\pgfpathlineto{\pgfqpoint{2.422746in}{4.583462in}}%
\pgfpathlineto{\pgfqpoint{2.390033in}{4.507303in}}%
\pgfpathclose%
\pgfusepath{fill}%
\end{pgfscope}%
\begin{pgfscope}%
\pgfpathrectangle{\pgfqpoint{1.020000in}{0.880000in}}{\pgfqpoint{6.160000in}{6.160000in}}%
\pgfusepath{clip}%
\pgfsetbuttcap%
\pgfsetroundjoin%
\definecolor{currentfill}{rgb}{0.822420,0.856898,0.910795}%
\pgfsetfillcolor{currentfill}%
\pgfsetlinewidth{0.000000pt}%
\definecolor{currentstroke}{rgb}{0.000000,0.000000,0.000000}%
\pgfsetstrokecolor{currentstroke}%
\pgfsetdash{}{0pt}%
\pgfpathmoveto{\pgfqpoint{3.592898in}{4.079315in}}%
\pgfpathlineto{\pgfqpoint{3.601981in}{4.020484in}}%
\pgfpathlineto{\pgfqpoint{3.611415in}{3.901252in}}%
\pgfpathlineto{\pgfqpoint{3.645791in}{3.780481in}}%
\pgfpathlineto{\pgfqpoint{3.679022in}{3.867790in}}%
\pgfpathlineto{\pgfqpoint{3.670117in}{3.886547in}}%
\pgfpathlineto{\pgfqpoint{3.661302in}{3.891007in}}%
\pgfpathlineto{\pgfqpoint{3.627541in}{3.912859in}}%
\pgfpathlineto{\pgfqpoint{3.592898in}{4.079315in}}%
\pgfpathclose%
\pgfusepath{fill}%
\end{pgfscope}%
\begin{pgfscope}%
\pgfpathrectangle{\pgfqpoint{1.020000in}{0.880000in}}{\pgfqpoint{6.160000in}{6.160000in}}%
\pgfusepath{clip}%
\pgfsetbuttcap%
\pgfsetroundjoin%
\definecolor{currentfill}{rgb}{0.856716,0.354704,0.285487}%
\pgfsetfillcolor{currentfill}%
\pgfsetlinewidth{0.000000pt}%
\definecolor{currentstroke}{rgb}{0.000000,0.000000,0.000000}%
\pgfsetstrokecolor{currentstroke}%
\pgfsetdash{}{0pt}%
\pgfpathmoveto{\pgfqpoint{2.707702in}{4.937981in}}%
\pgfpathlineto{\pgfqpoint{2.713944in}{5.045803in}}%
\pgfpathlineto{\pgfqpoint{2.723038in}{4.957535in}}%
\pgfpathlineto{\pgfqpoint{2.758242in}{4.871108in}}%
\pgfpathlineto{\pgfqpoint{2.792051in}{4.880652in}}%
\pgfpathlineto{\pgfqpoint{2.782046in}{5.034839in}}%
\pgfpathlineto{\pgfqpoint{2.772588in}{5.148990in}}%
\pgfpathlineto{\pgfqpoint{2.741557in}{4.942304in}}%
\pgfpathlineto{\pgfqpoint{2.707702in}{4.937981in}}%
\pgfpathclose%
\pgfusepath{fill}%
\end{pgfscope}%
\begin{pgfscope}%
\pgfpathrectangle{\pgfqpoint{1.020000in}{0.880000in}}{\pgfqpoint{6.160000in}{6.160000in}}%
\pgfusepath{clip}%
\pgfsetbuttcap%
\pgfsetroundjoin%
\definecolor{currentfill}{rgb}{0.768929,0.189213,0.197965}%
\pgfsetfillcolor{currentfill}%
\pgfsetlinewidth{0.000000pt}%
\definecolor{currentstroke}{rgb}{0.000000,0.000000,0.000000}%
\pgfsetstrokecolor{currentstroke}%
\pgfsetdash{}{0pt}%
\pgfpathmoveto{\pgfqpoint{2.908666in}{5.116216in}}%
\pgfpathlineto{\pgfqpoint{2.916462in}{5.133711in}}%
\pgfpathlineto{\pgfqpoint{2.923621in}{5.204233in}}%
\pgfpathlineto{\pgfqpoint{2.958563in}{5.122033in}}%
\pgfpathlineto{\pgfqpoint{2.994202in}{4.973943in}}%
\pgfpathlineto{\pgfqpoint{2.983195in}{5.221724in}}%
\pgfpathlineto{\pgfqpoint{2.972833in}{5.413321in}}%
\pgfpathlineto{\pgfqpoint{2.943099in}{5.068950in}}%
\pgfpathlineto{\pgfqpoint{2.908666in}{5.116216in}}%
\pgfpathclose%
\pgfusepath{fill}%
\end{pgfscope}%
\begin{pgfscope}%
\pgfpathrectangle{\pgfqpoint{1.020000in}{0.880000in}}{\pgfqpoint{6.160000in}{6.160000in}}%
\pgfusepath{clip}%
\pgfsetbuttcap%
\pgfsetroundjoin%
\definecolor{currentfill}{rgb}{0.446431,0.582356,0.957373}%
\pgfsetfillcolor{currentfill}%
\pgfsetlinewidth{0.000000pt}%
\definecolor{currentstroke}{rgb}{0.000000,0.000000,0.000000}%
\pgfsetstrokecolor{currentstroke}%
\pgfsetdash{}{0pt}%
\pgfpathmoveto{\pgfqpoint{4.595689in}{3.177012in}}%
\pgfpathlineto{\pgfqpoint{4.605467in}{3.173227in}}%
\pgfpathlineto{\pgfqpoint{4.615958in}{3.329603in}}%
\pgfpathlineto{\pgfqpoint{4.649076in}{3.251762in}}%
\pgfpathlineto{\pgfqpoint{4.682041in}{3.157826in}}%
\pgfpathlineto{\pgfqpoint{4.672316in}{3.189281in}}%
\pgfpathlineto{\pgfqpoint{4.662577in}{3.214556in}}%
\pgfpathlineto{\pgfqpoint{4.629214in}{3.213127in}}%
\pgfpathlineto{\pgfqpoint{4.595689in}{3.177012in}}%
\pgfpathclose%
\pgfusepath{fill}%
\end{pgfscope}%
\begin{pgfscope}%
\pgfpathrectangle{\pgfqpoint{1.020000in}{0.880000in}}{\pgfqpoint{6.160000in}{6.160000in}}%
\pgfusepath{clip}%
\pgfsetbuttcap%
\pgfsetroundjoin%
\definecolor{currentfill}{rgb}{0.967317,0.657471,0.538160}%
\pgfsetfillcolor{currentfill}%
\pgfsetlinewidth{0.000000pt}%
\definecolor{currentstroke}{rgb}{0.000000,0.000000,0.000000}%
\pgfsetstrokecolor{currentstroke}%
\pgfsetdash{}{0pt}%
\pgfpathmoveto{\pgfqpoint{2.459522in}{4.423363in}}%
\pgfpathlineto{\pgfqpoint{2.465128in}{4.536316in}}%
\pgfpathlineto{\pgfqpoint{2.474162in}{4.448568in}}%
\pgfpathlineto{\pgfqpoint{2.508236in}{4.446869in}}%
\pgfpathlineto{\pgfqpoint{2.539081in}{4.642644in}}%
\pgfpathlineto{\pgfqpoint{2.533451in}{4.520382in}}%
\pgfpathlineto{\pgfqpoint{2.523721in}{4.650252in}}%
\pgfpathlineto{\pgfqpoint{2.488722in}{4.708453in}}%
\pgfpathlineto{\pgfqpoint{2.459522in}{4.423363in}}%
\pgfpathclose%
\pgfusepath{fill}%
\end{pgfscope}%
\begin{pgfscope}%
\pgfpathrectangle{\pgfqpoint{1.020000in}{0.880000in}}{\pgfqpoint{6.160000in}{6.160000in}}%
\pgfusepath{clip}%
\pgfsetbuttcap%
\pgfsetroundjoin%
\definecolor{currentfill}{rgb}{0.388852,0.516298,0.921373}%
\pgfsetfillcolor{currentfill}%
\pgfsetlinewidth{0.000000pt}%
\definecolor{currentstroke}{rgb}{0.000000,0.000000,0.000000}%
\pgfsetstrokecolor{currentstroke}%
\pgfsetdash{}{0pt}%
\pgfpathmoveto{\pgfqpoint{4.968323in}{3.085247in}}%
\pgfpathlineto{\pgfqpoint{4.978293in}{3.056761in}}%
\pgfpathlineto{\pgfqpoint{4.989109in}{3.131019in}}%
\pgfpathlineto{\pgfqpoint{5.021501in}{3.017703in}}%
\pgfpathlineto{\pgfqpoint{5.055738in}{3.126077in}}%
\pgfpathlineto{\pgfqpoint{5.045514in}{3.133634in}}%
\pgfpathlineto{\pgfqpoint{5.034997in}{3.104380in}}%
\pgfpathlineto{\pgfqpoint{5.001868in}{3.119326in}}%
\pgfpathlineto{\pgfqpoint{4.968323in}{3.085247in}}%
\pgfpathclose%
\pgfusepath{fill}%
\end{pgfscope}%
\begin{pgfscope}%
\pgfpathrectangle{\pgfqpoint{1.020000in}{0.880000in}}{\pgfqpoint{6.160000in}{6.160000in}}%
\pgfusepath{clip}%
\pgfsetbuttcap%
\pgfsetroundjoin%
\definecolor{currentfill}{rgb}{0.944055,0.553153,0.435548}%
\pgfsetfillcolor{currentfill}%
\pgfsetlinewidth{0.000000pt}%
\definecolor{currentstroke}{rgb}{0.000000,0.000000,0.000000}%
\pgfsetstrokecolor{currentstroke}%
\pgfsetdash{}{0pt}%
\pgfpathmoveto{\pgfqpoint{3.165377in}{4.684881in}}%
\pgfpathlineto{\pgfqpoint{3.173255in}{4.725321in}}%
\pgfpathlineto{\pgfqpoint{3.182706in}{4.609124in}}%
\pgfpathlineto{\pgfqpoint{3.216054in}{4.660008in}}%
\pgfpathlineto{\pgfqpoint{3.249336in}{4.719473in}}%
\pgfpathlineto{\pgfqpoint{3.241760in}{4.634355in}}%
\pgfpathlineto{\pgfqpoint{3.230541in}{4.940696in}}%
\pgfpathlineto{\pgfqpoint{3.198317in}{4.772571in}}%
\pgfpathlineto{\pgfqpoint{3.165377in}{4.684881in}}%
\pgfpathclose%
\pgfusepath{fill}%
\end{pgfscope}%
\begin{pgfscope}%
\pgfpathrectangle{\pgfqpoint{1.020000in}{0.880000in}}{\pgfqpoint{6.160000in}{6.160000in}}%
\pgfusepath{clip}%
\pgfsetbuttcap%
\pgfsetroundjoin%
\definecolor{currentfill}{rgb}{0.383662,0.510183,0.917831}%
\pgfsetfillcolor{currentfill}%
\pgfsetlinewidth{0.000000pt}%
\definecolor{currentstroke}{rgb}{0.000000,0.000000,0.000000}%
\pgfsetstrokecolor{currentstroke}%
\pgfsetdash{}{0pt}%
\pgfpathmoveto{\pgfqpoint{5.475662in}{3.138506in}}%
\pgfpathlineto{\pgfqpoint{5.486616in}{3.154358in}}%
\pgfpathlineto{\pgfqpoint{5.496310in}{3.072535in}}%
\pgfpathlineto{\pgfqpoint{5.530139in}{3.120725in}}%
\pgfpathlineto{\pgfqpoint{5.560631in}{2.927032in}}%
\pgfpathlineto{\pgfqpoint{5.551092in}{3.019670in}}%
\pgfpathlineto{\pgfqpoint{5.540957in}{3.068392in}}%
\pgfpathlineto{\pgfqpoint{5.509860in}{3.218442in}}%
\pgfpathlineto{\pgfqpoint{5.475662in}{3.138506in}}%
\pgfpathclose%
\pgfusepath{fill}%
\end{pgfscope}%
\begin{pgfscope}%
\pgfpathrectangle{\pgfqpoint{1.020000in}{0.880000in}}{\pgfqpoint{6.160000in}{6.160000in}}%
\pgfusepath{clip}%
\pgfsetbuttcap%
\pgfsetroundjoin%
\definecolor{currentfill}{rgb}{0.318832,0.426605,0.859857}%
\pgfsetfillcolor{currentfill}%
\pgfsetlinewidth{0.000000pt}%
\definecolor{currentstroke}{rgb}{0.000000,0.000000,0.000000}%
\pgfsetstrokecolor{currentstroke}%
\pgfsetdash{}{0pt}%
\pgfpathmoveto{\pgfqpoint{5.848216in}{2.999553in}}%
\pgfpathlineto{\pgfqpoint{5.856064in}{2.803545in}}%
\pgfpathlineto{\pgfqpoint{5.872192in}{3.097294in}}%
\pgfpathlineto{\pgfqpoint{5.903564in}{2.996312in}}%
\pgfpathlineto{\pgfqpoint{5.935124in}{2.910990in}}%
\pgfpathlineto{\pgfqpoint{5.925052in}{2.976930in}}%
\pgfpathlineto{\pgfqpoint{5.912934in}{2.924603in}}%
\pgfpathlineto{\pgfqpoint{5.880995in}{2.985373in}}%
\pgfpathlineto{\pgfqpoint{5.848216in}{2.999553in}}%
\pgfpathclose%
\pgfusepath{fill}%
\end{pgfscope}%
\begin{pgfscope}%
\pgfpathrectangle{\pgfqpoint{1.020000in}{0.880000in}}{\pgfqpoint{6.160000in}{6.160000in}}%
\pgfusepath{clip}%
\pgfsetbuttcap%
\pgfsetroundjoin%
\definecolor{currentfill}{rgb}{0.323718,0.433158,0.864722}%
\pgfsetfillcolor{currentfill}%
\pgfsetlinewidth{0.000000pt}%
\definecolor{currentstroke}{rgb}{0.000000,0.000000,0.000000}%
\pgfsetstrokecolor{currentstroke}%
\pgfsetdash{}{0pt}%
\pgfpathmoveto{\pgfqpoint{5.341410in}{3.008350in}}%
\pgfpathlineto{\pgfqpoint{5.352671in}{3.064334in}}%
\pgfpathlineto{\pgfqpoint{5.363148in}{3.051208in}}%
\pgfpathlineto{\pgfqpoint{5.392280in}{2.713061in}}%
\pgfpathlineto{\pgfqpoint{5.428050in}{2.925070in}}%
\pgfpathlineto{\pgfqpoint{5.418258in}{2.996889in}}%
\pgfpathlineto{\pgfqpoint{5.409117in}{3.120995in}}%
\pgfpathlineto{\pgfqpoint{5.373439in}{2.912912in}}%
\pgfpathlineto{\pgfqpoint{5.341410in}{3.008350in}}%
\pgfpathclose%
\pgfusepath{fill}%
\end{pgfscope}%
\begin{pgfscope}%
\pgfpathrectangle{\pgfqpoint{1.020000in}{0.880000in}}{\pgfqpoint{6.160000in}{6.160000in}}%
\pgfusepath{clip}%
\pgfsetbuttcap%
\pgfsetroundjoin%
\definecolor{currentfill}{rgb}{0.966017,0.646130,0.525890}%
\pgfsetfillcolor{currentfill}%
\pgfsetlinewidth{0.000000pt}%
\definecolor{currentstroke}{rgb}{0.000000,0.000000,0.000000}%
\pgfsetstrokecolor{currentstroke}%
\pgfsetdash{}{0pt}%
\pgfpathmoveto{\pgfqpoint{3.249336in}{4.719473in}}%
\pgfpathlineto{\pgfqpoint{3.258834in}{4.597723in}}%
\pgfpathlineto{\pgfqpoint{3.266489in}{4.679267in}}%
\pgfpathlineto{\pgfqpoint{3.301234in}{4.580482in}}%
\pgfpathlineto{\pgfqpoint{3.337526in}{4.287072in}}%
\pgfpathlineto{\pgfqpoint{3.327167in}{4.514318in}}%
\pgfpathlineto{\pgfqpoint{3.317778in}{4.625927in}}%
\pgfpathlineto{\pgfqpoint{3.284576in}{4.560921in}}%
\pgfpathlineto{\pgfqpoint{3.249336in}{4.719473in}}%
\pgfpathclose%
\pgfusepath{fill}%
\end{pgfscope}%
\begin{pgfscope}%
\pgfpathrectangle{\pgfqpoint{1.020000in}{0.880000in}}{\pgfqpoint{6.160000in}{6.160000in}}%
\pgfusepath{clip}%
\pgfsetbuttcap%
\pgfsetroundjoin%
\definecolor{currentfill}{rgb}{0.586921,0.718121,0.998874}%
\pgfsetfillcolor{currentfill}%
\pgfsetlinewidth{0.000000pt}%
\definecolor{currentstroke}{rgb}{0.000000,0.000000,0.000000}%
\pgfsetstrokecolor{currentstroke}%
\pgfsetdash{}{0pt}%
\pgfpathmoveto{\pgfqpoint{4.070577in}{3.487307in}}%
\pgfpathlineto{\pgfqpoint{4.079810in}{3.494851in}}%
\pgfpathlineto{\pgfqpoint{4.089117in}{3.446511in}}%
\pgfpathlineto{\pgfqpoint{4.122678in}{3.514354in}}%
\pgfpathlineto{\pgfqpoint{4.156309in}{3.417244in}}%
\pgfpathlineto{\pgfqpoint{4.146989in}{3.415249in}}%
\pgfpathlineto{\pgfqpoint{4.137682in}{3.431002in}}%
\pgfpathlineto{\pgfqpoint{4.104123in}{3.492688in}}%
\pgfpathlineto{\pgfqpoint{4.070577in}{3.487307in}}%
\pgfpathclose%
\pgfusepath{fill}%
\end{pgfscope}%
\begin{pgfscope}%
\pgfpathrectangle{\pgfqpoint{1.020000in}{0.880000in}}{\pgfqpoint{6.160000in}{6.160000in}}%
\pgfusepath{clip}%
\pgfsetbuttcap%
\pgfsetroundjoin%
\definecolor{currentfill}{rgb}{0.378598,0.503856,0.913692}%
\pgfsetfillcolor{currentfill}%
\pgfsetlinewidth{0.000000pt}%
\definecolor{currentstroke}{rgb}{0.000000,0.000000,0.000000}%
\pgfsetstrokecolor{currentstroke}%
\pgfsetdash{}{0pt}%
\pgfpathmoveto{\pgfqpoint{5.187487in}{2.992430in}}%
\pgfpathlineto{\pgfqpoint{5.197305in}{2.932637in}}%
\pgfpathlineto{\pgfqpoint{5.208541in}{3.009637in}}%
\pgfpathlineto{\pgfqpoint{5.242559in}{3.081848in}}%
\pgfpathlineto{\pgfqpoint{5.276573in}{3.151054in}}%
\pgfpathlineto{\pgfqpoint{5.266736in}{3.214959in}}%
\pgfpathlineto{\pgfqpoint{5.255822in}{3.177896in}}%
\pgfpathlineto{\pgfqpoint{5.221147in}{3.038464in}}%
\pgfpathlineto{\pgfqpoint{5.187487in}{2.992430in}}%
\pgfpathclose%
\pgfusepath{fill}%
\end{pgfscope}%
\begin{pgfscope}%
\pgfpathrectangle{\pgfqpoint{1.020000in}{0.880000in}}{\pgfqpoint{6.160000in}{6.160000in}}%
\pgfusepath{clip}%
\pgfsetbuttcap%
\pgfsetroundjoin%
\definecolor{currentfill}{rgb}{0.538004,0.674902,0.991722}%
\pgfsetfillcolor{currentfill}%
\pgfsetlinewidth{0.000000pt}%
\definecolor{currentstroke}{rgb}{0.000000,0.000000,0.000000}%
\pgfsetstrokecolor{currentstroke}%
\pgfsetdash{}{0pt}%
\pgfpathmoveto{\pgfqpoint{4.223386in}{3.400212in}}%
\pgfpathlineto{\pgfqpoint{4.232863in}{3.547136in}}%
\pgfpathlineto{\pgfqpoint{4.242170in}{3.323513in}}%
\pgfpathlineto{\pgfqpoint{4.275749in}{3.357906in}}%
\pgfpathlineto{\pgfqpoint{4.309081in}{3.194771in}}%
\pgfpathlineto{\pgfqpoint{4.299818in}{3.372978in}}%
\pgfpathlineto{\pgfqpoint{4.290402in}{3.421580in}}%
\pgfpathlineto{\pgfqpoint{4.256849in}{3.340726in}}%
\pgfpathlineto{\pgfqpoint{4.223386in}{3.400212in}}%
\pgfpathclose%
\pgfusepath{fill}%
\end{pgfscope}%
\begin{pgfscope}%
\pgfpathrectangle{\pgfqpoint{1.020000in}{0.880000in}}{\pgfqpoint{6.160000in}{6.160000in}}%
\pgfusepath{clip}%
\pgfsetbuttcap%
\pgfsetroundjoin%
\definecolor{currentfill}{rgb}{0.358415,0.478426,0.896795}%
\pgfsetfillcolor{currentfill}%
\pgfsetlinewidth{0.000000pt}%
\definecolor{currentstroke}{rgb}{0.000000,0.000000,0.000000}%
\pgfsetstrokecolor{currentstroke}%
\pgfsetdash{}{0pt}%
\pgfpathmoveto{\pgfqpoint{5.409117in}{3.120995in}}%
\pgfpathlineto{\pgfqpoint{5.418258in}{2.996889in}}%
\pgfpathlineto{\pgfqpoint{5.428050in}{2.925070in}}%
\pgfpathlineto{\pgfqpoint{5.460313in}{2.854608in}}%
\pgfpathlineto{\pgfqpoint{5.496310in}{3.072535in}}%
\pgfpathlineto{\pgfqpoint{5.486616in}{3.154358in}}%
\pgfpathlineto{\pgfqpoint{5.475662in}{3.138506in}}%
\pgfpathlineto{\pgfqpoint{5.441928in}{3.092619in}}%
\pgfpathlineto{\pgfqpoint{5.409117in}{3.120995in}}%
\pgfpathclose%
\pgfusepath{fill}%
\end{pgfscope}%
\begin{pgfscope}%
\pgfpathrectangle{\pgfqpoint{1.020000in}{0.880000in}}{\pgfqpoint{6.160000in}{6.160000in}}%
\pgfusepath{clip}%
\pgfsetbuttcap%
\pgfsetroundjoin%
\definecolor{currentfill}{rgb}{0.348323,0.465711,0.888346}%
\pgfsetfillcolor{currentfill}%
\pgfsetlinewidth{0.000000pt}%
\definecolor{currentstroke}{rgb}{0.000000,0.000000,0.000000}%
\pgfsetstrokecolor{currentstroke}%
\pgfsetdash{}{0pt}%
\pgfpathmoveto{\pgfqpoint{5.122723in}{3.168170in}}%
\pgfpathlineto{\pgfqpoint{5.132017in}{3.054105in}}%
\pgfpathlineto{\pgfqpoint{5.141262in}{2.935116in}}%
\pgfpathlineto{\pgfqpoint{5.175520in}{3.034760in}}%
\pgfpathlineto{\pgfqpoint{5.208541in}{3.009637in}}%
\pgfpathlineto{\pgfqpoint{5.197305in}{2.932637in}}%
\pgfpathlineto{\pgfqpoint{5.187487in}{2.992430in}}%
\pgfpathlineto{\pgfqpoint{5.154806in}{3.045629in}}%
\pgfpathlineto{\pgfqpoint{5.122723in}{3.168170in}}%
\pgfpathclose%
\pgfusepath{fill}%
\end{pgfscope}%
\begin{pgfscope}%
\pgfpathrectangle{\pgfqpoint{1.020000in}{0.880000in}}{\pgfqpoint{6.160000in}{6.160000in}}%
\pgfusepath{clip}%
\pgfsetbuttcap%
\pgfsetroundjoin%
\definecolor{currentfill}{rgb}{0.521696,0.659599,0.987736}%
\pgfsetfillcolor{currentfill}%
\pgfsetlinewidth{0.000000pt}%
\definecolor{currentstroke}{rgb}{0.000000,0.000000,0.000000}%
\pgfsetstrokecolor{currentstroke}%
\pgfsetdash{}{0pt}%
\pgfpathmoveto{\pgfqpoint{4.443196in}{3.340648in}}%
\pgfpathlineto{\pgfqpoint{4.452885in}{3.359968in}}%
\pgfpathlineto{\pgfqpoint{4.462381in}{3.301260in}}%
\pgfpathlineto{\pgfqpoint{4.496137in}{3.387907in}}%
\pgfpathlineto{\pgfqpoint{4.529434in}{3.330827in}}%
\pgfpathlineto{\pgfqpoint{4.519635in}{3.308314in}}%
\pgfpathlineto{\pgfqpoint{4.509868in}{3.287650in}}%
\pgfpathlineto{\pgfqpoint{4.476885in}{3.427791in}}%
\pgfpathlineto{\pgfqpoint{4.443196in}{3.340648in}}%
\pgfpathclose%
\pgfusepath{fill}%
\end{pgfscope}%
\begin{pgfscope}%
\pgfpathrectangle{\pgfqpoint{1.020000in}{0.880000in}}{\pgfqpoint{6.160000in}{6.160000in}}%
\pgfusepath{clip}%
\pgfsetbuttcap%
\pgfsetroundjoin%
\definecolor{currentfill}{rgb}{0.959385,0.610306,0.489382}%
\pgfsetfillcolor{currentfill}%
\pgfsetlinewidth{0.000000pt}%
\definecolor{currentstroke}{rgb}{0.000000,0.000000,0.000000}%
\pgfsetstrokecolor{currentstroke}%
\pgfsetdash{}{0pt}%
\pgfpathmoveto{\pgfqpoint{2.523721in}{4.650252in}}%
\pgfpathlineto{\pgfqpoint{2.533451in}{4.520382in}}%
\pgfpathlineto{\pgfqpoint{2.539081in}{4.642644in}}%
\pgfpathlineto{\pgfqpoint{2.573609in}{4.613866in}}%
\pgfpathlineto{\pgfqpoint{2.606048in}{4.716766in}}%
\pgfpathlineto{\pgfqpoint{2.598318in}{4.718799in}}%
\pgfpathlineto{\pgfqpoint{2.593579in}{4.531599in}}%
\pgfpathlineto{\pgfqpoint{2.557673in}{4.653609in}}%
\pgfpathlineto{\pgfqpoint{2.523721in}{4.650252in}}%
\pgfpathclose%
\pgfusepath{fill}%
\end{pgfscope}%
\begin{pgfscope}%
\pgfpathrectangle{\pgfqpoint{1.020000in}{0.880000in}}{\pgfqpoint{6.160000in}{6.160000in}}%
\pgfusepath{clip}%
\pgfsetbuttcap%
\pgfsetroundjoin%
\definecolor{currentfill}{rgb}{0.949454,0.572388,0.453443}%
\pgfsetfillcolor{currentfill}%
\pgfsetlinewidth{0.000000pt}%
\definecolor{currentstroke}{rgb}{0.000000,0.000000,0.000000}%
\pgfsetstrokecolor{currentstroke}%
\pgfsetdash{}{0pt}%
\pgfpathmoveto{\pgfqpoint{3.098299in}{4.632032in}}%
\pgfpathlineto{\pgfqpoint{3.105082in}{4.766263in}}%
\pgfpathlineto{\pgfqpoint{3.113975in}{4.704242in}}%
\pgfpathlineto{\pgfqpoint{3.148427in}{4.650778in}}%
\pgfpathlineto{\pgfqpoint{3.182706in}{4.609124in}}%
\pgfpathlineto{\pgfqpoint{3.173255in}{4.725321in}}%
\pgfpathlineto{\pgfqpoint{3.165377in}{4.684881in}}%
\pgfpathlineto{\pgfqpoint{3.131146in}{4.725221in}}%
\pgfpathlineto{\pgfqpoint{3.098299in}{4.632032in}}%
\pgfpathclose%
\pgfusepath{fill}%
\end{pgfscope}%
\begin{pgfscope}%
\pgfpathrectangle{\pgfqpoint{1.020000in}{0.880000in}}{\pgfqpoint{6.160000in}{6.160000in}}%
\pgfusepath{clip}%
\pgfsetbuttcap%
\pgfsetroundjoin%
\definecolor{currentfill}{rgb}{0.651398,0.768121,0.995891}%
\pgfsetfillcolor{currentfill}%
\pgfsetlinewidth{0.000000pt}%
\definecolor{currentstroke}{rgb}{0.000000,0.000000,0.000000}%
\pgfsetstrokecolor{currentstroke}%
\pgfsetdash{}{0pt}%
\pgfpathmoveto{\pgfqpoint{3.917667in}{3.621039in}}%
\pgfpathlineto{\pgfqpoint{3.926875in}{3.571433in}}%
\pgfpathlineto{\pgfqpoint{3.936023in}{3.550675in}}%
\pgfpathlineto{\pgfqpoint{3.969677in}{3.548982in}}%
\pgfpathlineto{\pgfqpoint{4.003446in}{3.472030in}}%
\pgfpathlineto{\pgfqpoint{3.993964in}{3.645855in}}%
\pgfpathlineto{\pgfqpoint{3.984934in}{3.585695in}}%
\pgfpathlineto{\pgfqpoint{3.951325in}{3.599776in}}%
\pgfpathlineto{\pgfqpoint{3.917667in}{3.621039in}}%
\pgfpathclose%
\pgfusepath{fill}%
\end{pgfscope}%
\begin{pgfscope}%
\pgfpathrectangle{\pgfqpoint{1.020000in}{0.880000in}}{\pgfqpoint{6.160000in}{6.160000in}}%
\pgfusepath{clip}%
\pgfsetbuttcap%
\pgfsetroundjoin%
\definecolor{currentfill}{rgb}{0.895885,0.433075,0.338681}%
\pgfsetfillcolor{currentfill}%
\pgfsetlinewidth{0.000000pt}%
\definecolor{currentstroke}{rgb}{0.000000,0.000000,0.000000}%
\pgfsetstrokecolor{currentstroke}%
\pgfsetdash{}{0pt}%
\pgfpathmoveto{\pgfqpoint{2.792051in}{4.880652in}}%
\pgfpathlineto{\pgfqpoint{2.797708in}{5.044874in}}%
\pgfpathlineto{\pgfqpoint{2.809461in}{4.762494in}}%
\pgfpathlineto{\pgfqpoint{2.842137in}{4.858902in}}%
\pgfpathlineto{\pgfqpoint{2.878232in}{4.693495in}}%
\pgfpathlineto{\pgfqpoint{2.866306in}{4.999044in}}%
\pgfpathlineto{\pgfqpoint{2.858747in}{4.967415in}}%
\pgfpathlineto{\pgfqpoint{2.826758in}{4.822130in}}%
\pgfpathlineto{\pgfqpoint{2.792051in}{4.880652in}}%
\pgfpathclose%
\pgfusepath{fill}%
\end{pgfscope}%
\begin{pgfscope}%
\pgfpathrectangle{\pgfqpoint{1.020000in}{0.880000in}}{\pgfqpoint{6.160000in}{6.160000in}}%
\pgfusepath{clip}%
\pgfsetbuttcap%
\pgfsetroundjoin%
\definecolor{currentfill}{rgb}{0.304174,0.406945,0.845263}%
\pgfsetfillcolor{currentfill}%
\pgfsetlinewidth{0.000000pt}%
\definecolor{currentstroke}{rgb}{0.000000,0.000000,0.000000}%
\pgfsetstrokecolor{currentstroke}%
\pgfsetdash{}{0pt}%
\pgfpathmoveto{\pgfqpoint{5.780007in}{2.868131in}}%
\pgfpathlineto{\pgfqpoint{5.794288in}{3.066796in}}%
\pgfpathlineto{\pgfqpoint{5.799738in}{2.718856in}}%
\pgfpathlineto{\pgfqpoint{5.836893in}{2.967164in}}%
\pgfpathlineto{\pgfqpoint{5.872192in}{3.097294in}}%
\pgfpathlineto{\pgfqpoint{5.856064in}{2.803545in}}%
\pgfpathlineto{\pgfqpoint{5.848216in}{2.999553in}}%
\pgfpathlineto{\pgfqpoint{5.813999in}{2.927707in}}%
\pgfpathlineto{\pgfqpoint{5.780007in}{2.868131in}}%
\pgfpathclose%
\pgfusepath{fill}%
\end{pgfscope}%
\begin{pgfscope}%
\pgfpathrectangle{\pgfqpoint{1.020000in}{0.880000in}}{\pgfqpoint{6.160000in}{6.160000in}}%
\pgfusepath{clip}%
\pgfsetbuttcap%
\pgfsetroundjoin%
\definecolor{currentfill}{rgb}{0.945854,0.559565,0.441513}%
\pgfsetfillcolor{currentfill}%
\pgfsetlinewidth{0.000000pt}%
\definecolor{currentstroke}{rgb}{0.000000,0.000000,0.000000}%
\pgfsetstrokecolor{currentstroke}%
\pgfsetdash{}{0pt}%
\pgfpathmoveto{\pgfqpoint{2.593579in}{4.531599in}}%
\pgfpathlineto{\pgfqpoint{2.598318in}{4.718799in}}%
\pgfpathlineto{\pgfqpoint{2.606048in}{4.716766in}}%
\pgfpathlineto{\pgfqpoint{2.640296in}{4.704145in}}%
\pgfpathlineto{\pgfqpoint{2.675669in}{4.614061in}}%
\pgfpathlineto{\pgfqpoint{2.666450in}{4.711657in}}%
\pgfpathlineto{\pgfqpoint{2.656986in}{4.825582in}}%
\pgfpathlineto{\pgfqpoint{2.623632in}{4.783283in}}%
\pgfpathlineto{\pgfqpoint{2.593579in}{4.531599in}}%
\pgfpathclose%
\pgfusepath{fill}%
\end{pgfscope}%
\begin{pgfscope}%
\pgfpathrectangle{\pgfqpoint{1.020000in}{0.880000in}}{\pgfqpoint{6.160000in}{6.160000in}}%
\pgfusepath{clip}%
\pgfsetbuttcap%
\pgfsetroundjoin%
\definecolor{currentfill}{rgb}{0.964835,0.744614,0.643239}%
\pgfsetfillcolor{currentfill}%
\pgfsetlinewidth{0.000000pt}%
\definecolor{currentstroke}{rgb}{0.000000,0.000000,0.000000}%
\pgfsetstrokecolor{currentstroke}%
\pgfsetdash{}{0pt}%
\pgfpathmoveto{\pgfqpoint{3.266489in}{4.679267in}}%
\pgfpathlineto{\pgfqpoint{3.277439in}{4.397087in}}%
\pgfpathlineto{\pgfqpoint{3.286820in}{4.288011in}}%
\pgfpathlineto{\pgfqpoint{3.321207in}{4.221726in}}%
\pgfpathlineto{\pgfqpoint{3.354308in}{4.303974in}}%
\pgfpathlineto{\pgfqpoint{3.345735in}{4.316202in}}%
\pgfpathlineto{\pgfqpoint{3.337526in}{4.287072in}}%
\pgfpathlineto{\pgfqpoint{3.301234in}{4.580482in}}%
\pgfpathlineto{\pgfqpoint{3.266489in}{4.679267in}}%
\pgfpathclose%
\pgfusepath{fill}%
\end{pgfscope}%
\begin{pgfscope}%
\pgfpathrectangle{\pgfqpoint{1.020000in}{0.880000in}}{\pgfqpoint{6.160000in}{6.160000in}}%
\pgfusepath{clip}%
\pgfsetbuttcap%
\pgfsetroundjoin%
\definecolor{currentfill}{rgb}{0.931831,0.519086,0.406480}%
\pgfsetfillcolor{currentfill}%
\pgfsetlinewidth{0.000000pt}%
\definecolor{currentstroke}{rgb}{0.000000,0.000000,0.000000}%
\pgfsetstrokecolor{currentstroke}%
\pgfsetdash{}{0pt}%
\pgfpathmoveto{\pgfqpoint{2.656986in}{4.825582in}}%
\pgfpathlineto{\pgfqpoint{2.666450in}{4.711657in}}%
\pgfpathlineto{\pgfqpoint{2.675669in}{4.614061in}}%
\pgfpathlineto{\pgfqpoint{2.710061in}{4.585865in}}%
\pgfpathlineto{\pgfqpoint{2.741533in}{4.760127in}}%
\pgfpathlineto{\pgfqpoint{2.732762in}{4.825498in}}%
\pgfpathlineto{\pgfqpoint{2.723038in}{4.957535in}}%
\pgfpathlineto{\pgfqpoint{2.690852in}{4.833699in}}%
\pgfpathlineto{\pgfqpoint{2.656986in}{4.825582in}}%
\pgfpathclose%
\pgfusepath{fill}%
\end{pgfscope}%
\begin{pgfscope}%
\pgfpathrectangle{\pgfqpoint{1.020000in}{0.880000in}}{\pgfqpoint{6.160000in}{6.160000in}}%
\pgfusepath{clip}%
\pgfsetbuttcap%
\pgfsetroundjoin%
\definecolor{currentfill}{rgb}{0.708720,0.805721,0.981117}%
\pgfsetfillcolor{currentfill}%
\pgfsetlinewidth{0.000000pt}%
\definecolor{currentstroke}{rgb}{0.000000,0.000000,0.000000}%
\pgfsetstrokecolor{currentstroke}%
\pgfsetdash{}{0pt}%
\pgfpathmoveto{\pgfqpoint{3.764329in}{3.828684in}}%
\pgfpathlineto{\pgfqpoint{3.774265in}{3.582079in}}%
\pgfpathlineto{\pgfqpoint{3.782816in}{3.675742in}}%
\pgfpathlineto{\pgfqpoint{3.817022in}{3.539317in}}%
\pgfpathlineto{\pgfqpoint{3.850482in}{3.593671in}}%
\pgfpathlineto{\pgfqpoint{3.841093in}{3.709766in}}%
\pgfpathlineto{\pgfqpoint{3.831990in}{3.743479in}}%
\pgfpathlineto{\pgfqpoint{3.798357in}{3.740920in}}%
\pgfpathlineto{\pgfqpoint{3.764329in}{3.828684in}}%
\pgfpathclose%
\pgfusepath{fill}%
\end{pgfscope}%
\begin{pgfscope}%
\pgfpathrectangle{\pgfqpoint{1.020000in}{0.880000in}}{\pgfqpoint{6.160000in}{6.160000in}}%
\pgfusepath{clip}%
\pgfsetbuttcap%
\pgfsetroundjoin%
\definecolor{currentfill}{rgb}{0.358415,0.478426,0.896795}%
\pgfsetfillcolor{currentfill}%
\pgfsetlinewidth{0.000000pt}%
\definecolor{currentstroke}{rgb}{0.000000,0.000000,0.000000}%
\pgfsetstrokecolor{currentstroke}%
\pgfsetdash{}{0pt}%
\pgfpathmoveto{\pgfqpoint{5.629309in}{3.100619in}}%
\pgfpathlineto{\pgfqpoint{5.639248in}{3.033554in}}%
\pgfpathlineto{\pgfqpoint{5.649711in}{3.001585in}}%
\pgfpathlineto{\pgfqpoint{5.683443in}{3.040899in}}%
\pgfpathlineto{\pgfqpoint{5.716777in}{3.054004in}}%
\pgfpathlineto{\pgfqpoint{5.704042in}{2.940567in}}%
\pgfpathlineto{\pgfqpoint{5.693668in}{2.979500in}}%
\pgfpathlineto{\pgfqpoint{5.663244in}{3.156349in}}%
\pgfpathlineto{\pgfqpoint{5.629309in}{3.100619in}}%
\pgfpathclose%
\pgfusepath{fill}%
\end{pgfscope}%
\begin{pgfscope}%
\pgfpathrectangle{\pgfqpoint{1.020000in}{0.880000in}}{\pgfqpoint{6.160000in}{6.160000in}}%
\pgfusepath{clip}%
\pgfsetbuttcap%
\pgfsetroundjoin%
\definecolor{currentfill}{rgb}{0.786721,0.844807,0.939810}%
\pgfsetfillcolor{currentfill}%
\pgfsetlinewidth{0.000000pt}%
\definecolor{currentstroke}{rgb}{0.000000,0.000000,0.000000}%
\pgfsetstrokecolor{currentstroke}%
\pgfsetdash{}{0pt}%
\pgfpathmoveto{\pgfqpoint{3.679022in}{3.867790in}}%
\pgfpathlineto{\pgfqpoint{3.687873in}{3.863179in}}%
\pgfpathlineto{\pgfqpoint{3.696131in}{3.984473in}}%
\pgfpathlineto{\pgfqpoint{3.731115in}{3.718347in}}%
\pgfpathlineto{\pgfqpoint{3.764329in}{3.828684in}}%
\pgfpathlineto{\pgfqpoint{3.755586in}{3.789531in}}%
\pgfpathlineto{\pgfqpoint{3.746631in}{3.805069in}}%
\pgfpathlineto{\pgfqpoint{3.712743in}{3.858659in}}%
\pgfpathlineto{\pgfqpoint{3.679022in}{3.867790in}}%
\pgfpathclose%
\pgfusepath{fill}%
\end{pgfscope}%
\begin{pgfscope}%
\pgfpathrectangle{\pgfqpoint{1.020000in}{0.880000in}}{\pgfqpoint{6.160000in}{6.160000in}}%
\pgfusepath{clip}%
\pgfsetbuttcap%
\pgfsetroundjoin%
\definecolor{currentfill}{rgb}{0.947345,0.794696,0.716991}%
\pgfsetfillcolor{currentfill}%
\pgfsetlinewidth{0.000000pt}%
\definecolor{currentstroke}{rgb}{0.000000,0.000000,0.000000}%
\pgfsetstrokecolor{currentstroke}%
\pgfsetdash{}{0pt}%
\pgfpathmoveto{\pgfqpoint{2.340159in}{4.318287in}}%
\pgfpathlineto{\pgfqpoint{2.351002in}{4.129823in}}%
\pgfpathlineto{\pgfqpoint{2.359046in}{4.095211in}}%
\pgfpathlineto{\pgfqpoint{2.389772in}{4.283493in}}%
\pgfpathlineto{\pgfqpoint{2.423861in}{4.285124in}}%
\pgfpathlineto{\pgfqpoint{2.417087in}{4.243954in}}%
\pgfpathlineto{\pgfqpoint{2.406965in}{4.394356in}}%
\pgfpathlineto{\pgfqpoint{2.373604in}{4.353810in}}%
\pgfpathlineto{\pgfqpoint{2.340159in}{4.318287in}}%
\pgfpathclose%
\pgfusepath{fill}%
\end{pgfscope}%
\begin{pgfscope}%
\pgfpathrectangle{\pgfqpoint{1.020000in}{0.880000in}}{\pgfqpoint{6.160000in}{6.160000in}}%
\pgfusepath{clip}%
\pgfsetbuttcap%
\pgfsetroundjoin%
\definecolor{currentfill}{rgb}{0.839365,0.321856,0.264924}%
\pgfsetfillcolor{currentfill}%
\pgfsetlinewidth{0.000000pt}%
\definecolor{currentstroke}{rgb}{0.000000,0.000000,0.000000}%
\pgfsetstrokecolor{currentstroke}%
\pgfsetdash{}{0pt}%
\pgfpathmoveto{\pgfqpoint{2.923621in}{5.204233in}}%
\pgfpathlineto{\pgfqpoint{2.932425in}{5.143311in}}%
\pgfpathlineto{\pgfqpoint{2.945125in}{4.764140in}}%
\pgfpathlineto{\pgfqpoint{2.977058in}{4.927262in}}%
\pgfpathlineto{\pgfqpoint{3.009116in}{5.087282in}}%
\pgfpathlineto{\pgfqpoint{3.003249in}{4.892779in}}%
\pgfpathlineto{\pgfqpoint{2.994202in}{4.973943in}}%
\pgfpathlineto{\pgfqpoint{2.958563in}{5.122033in}}%
\pgfpathlineto{\pgfqpoint{2.923621in}{5.204233in}}%
\pgfpathclose%
\pgfusepath{fill}%
\end{pgfscope}%
\begin{pgfscope}%
\pgfpathrectangle{\pgfqpoint{1.020000in}{0.880000in}}{\pgfqpoint{6.160000in}{6.160000in}}%
\pgfusepath{clip}%
\pgfsetbuttcap%
\pgfsetroundjoin%
\definecolor{currentfill}{rgb}{0.943432,0.802276,0.729172}%
\pgfsetfillcolor{currentfill}%
\pgfsetlinewidth{0.000000pt}%
\definecolor{currentstroke}{rgb}{0.000000,0.000000,0.000000}%
\pgfsetstrokecolor{currentstroke}%
\pgfsetdash{}{0pt}%
\pgfpathmoveto{\pgfqpoint{3.354308in}{4.303974in}}%
\pgfpathlineto{\pgfqpoint{3.363415in}{4.228004in}}%
\pgfpathlineto{\pgfqpoint{3.372116in}{4.203039in}}%
\pgfpathlineto{\pgfqpoint{3.404923in}{4.334079in}}%
\pgfpathlineto{\pgfqpoint{3.440072in}{4.156810in}}%
\pgfpathlineto{\pgfqpoint{3.431530in}{4.154341in}}%
\pgfpathlineto{\pgfqpoint{3.421299in}{4.377991in}}%
\pgfpathlineto{\pgfqpoint{3.388747in}{4.220923in}}%
\pgfpathlineto{\pgfqpoint{3.354308in}{4.303974in}}%
\pgfpathclose%
\pgfusepath{fill}%
\end{pgfscope}%
\begin{pgfscope}%
\pgfpathrectangle{\pgfqpoint{1.020000in}{0.880000in}}{\pgfqpoint{6.160000in}{6.160000in}}%
\pgfusepath{clip}%
\pgfsetbuttcap%
\pgfsetroundjoin%
\definecolor{currentfill}{rgb}{0.959518,0.766973,0.674145}%
\pgfsetfillcolor{currentfill}%
\pgfsetlinewidth{0.000000pt}%
\definecolor{currentstroke}{rgb}{0.000000,0.000000,0.000000}%
\pgfsetstrokecolor{currentstroke}%
\pgfsetdash{}{0pt}%
\pgfpathmoveto{\pgfqpoint{2.406965in}{4.394356in}}%
\pgfpathlineto{\pgfqpoint{2.417087in}{4.243954in}}%
\pgfpathlineto{\pgfqpoint{2.423861in}{4.285124in}}%
\pgfpathlineto{\pgfqpoint{2.457196in}{4.329775in}}%
\pgfpathlineto{\pgfqpoint{2.492985in}{4.228632in}}%
\pgfpathlineto{\pgfqpoint{2.482596in}{4.396661in}}%
\pgfpathlineto{\pgfqpoint{2.474162in}{4.448568in}}%
\pgfpathlineto{\pgfqpoint{2.442767in}{4.293814in}}%
\pgfpathlineto{\pgfqpoint{2.406965in}{4.394356in}}%
\pgfpathclose%
\pgfusepath{fill}%
\end{pgfscope}%
\begin{pgfscope}%
\pgfpathrectangle{\pgfqpoint{1.020000in}{0.880000in}}{\pgfqpoint{6.160000in}{6.160000in}}%
\pgfusepath{clip}%
\pgfsetbuttcap%
\pgfsetroundjoin%
\definecolor{currentfill}{rgb}{0.581486,0.713451,0.998314}%
\pgfsetfillcolor{currentfill}%
\pgfsetlinewidth{0.000000pt}%
\definecolor{currentstroke}{rgb}{0.000000,0.000000,0.000000}%
\pgfsetstrokecolor{currentstroke}%
\pgfsetdash{}{0pt}%
\pgfpathmoveto{\pgfqpoint{4.003446in}{3.472030in}}%
\pgfpathlineto{\pgfqpoint{4.012705in}{3.421486in}}%
\pgfpathlineto{\pgfqpoint{4.022051in}{3.323227in}}%
\pgfpathlineto{\pgfqpoint{4.055401in}{3.519999in}}%
\pgfpathlineto{\pgfqpoint{4.089117in}{3.446511in}}%
\pgfpathlineto{\pgfqpoint{4.079810in}{3.494851in}}%
\pgfpathlineto{\pgfqpoint{4.070577in}{3.487307in}}%
\pgfpathlineto{\pgfqpoint{4.037131in}{3.402802in}}%
\pgfpathlineto{\pgfqpoint{4.003446in}{3.472030in}}%
\pgfpathclose%
\pgfusepath{fill}%
\end{pgfscope}%
\begin{pgfscope}%
\pgfpathrectangle{\pgfqpoint{1.020000in}{0.880000in}}{\pgfqpoint{6.160000in}{6.160000in}}%
\pgfusepath{clip}%
\pgfsetbuttcap%
\pgfsetroundjoin%
\definecolor{currentfill}{rgb}{0.441123,0.576532,0.954545}%
\pgfsetfillcolor{currentfill}%
\pgfsetlinewidth{0.000000pt}%
\definecolor{currentstroke}{rgb}{0.000000,0.000000,0.000000}%
\pgfsetstrokecolor{currentstroke}%
\pgfsetdash{}{0pt}%
\pgfpathmoveto{\pgfqpoint{4.749770in}{3.331088in}}%
\pgfpathlineto{\pgfqpoint{4.759189in}{3.233407in}}%
\pgfpathlineto{\pgfqpoint{4.768897in}{3.184677in}}%
\pgfpathlineto{\pgfqpoint{4.801437in}{3.043198in}}%
\pgfpathlineto{\pgfqpoint{4.835279in}{3.118262in}}%
\pgfpathlineto{\pgfqpoint{4.825205in}{3.114342in}}%
\pgfpathlineto{\pgfqpoint{4.815933in}{3.233317in}}%
\pgfpathlineto{\pgfqpoint{4.782639in}{3.242425in}}%
\pgfpathlineto{\pgfqpoint{4.749770in}{3.331088in}}%
\pgfpathclose%
\pgfusepath{fill}%
\end{pgfscope}%
\begin{pgfscope}%
\pgfpathrectangle{\pgfqpoint{1.020000in}{0.880000in}}{\pgfqpoint{6.160000in}{6.160000in}}%
\pgfusepath{clip}%
\pgfsetbuttcap%
\pgfsetroundjoin%
\definecolor{currentfill}{rgb}{0.899543,0.847500,0.817789}%
\pgfsetfillcolor{currentfill}%
\pgfsetlinewidth{0.000000pt}%
\definecolor{currentstroke}{rgb}{0.000000,0.000000,0.000000}%
\pgfsetstrokecolor{currentstroke}%
\pgfsetdash{}{0pt}%
\pgfpathmoveto{\pgfqpoint{3.440072in}{4.156810in}}%
\pgfpathlineto{\pgfqpoint{3.448975in}{4.112418in}}%
\pgfpathlineto{\pgfqpoint{3.457177in}{4.166839in}}%
\pgfpathlineto{\pgfqpoint{3.492763in}{3.911568in}}%
\pgfpathlineto{\pgfqpoint{3.525293in}{4.094674in}}%
\pgfpathlineto{\pgfqpoint{3.516491in}{4.116274in}}%
\pgfpathlineto{\pgfqpoint{3.508199in}{4.064027in}}%
\pgfpathlineto{\pgfqpoint{3.473810in}{4.160296in}}%
\pgfpathlineto{\pgfqpoint{3.440072in}{4.156810in}}%
\pgfpathclose%
\pgfusepath{fill}%
\end{pgfscope}%
\begin{pgfscope}%
\pgfpathrectangle{\pgfqpoint{1.020000in}{0.880000in}}{\pgfqpoint{6.160000in}{6.160000in}}%
\pgfusepath{clip}%
\pgfsetbuttcap%
\pgfsetroundjoin%
\definecolor{currentfill}{rgb}{0.510824,0.649397,0.985079}%
\pgfsetfillcolor{currentfill}%
\pgfsetlinewidth{0.000000pt}%
\definecolor{currentstroke}{rgb}{0.000000,0.000000,0.000000}%
\pgfsetstrokecolor{currentstroke}%
\pgfsetdash{}{0pt}%
\pgfpathmoveto{\pgfqpoint{4.376380in}{3.397525in}}%
\pgfpathlineto{\pgfqpoint{4.385725in}{3.283109in}}%
\pgfpathlineto{\pgfqpoint{4.395471in}{3.358438in}}%
\pgfpathlineto{\pgfqpoint{4.428788in}{3.265867in}}%
\pgfpathlineto{\pgfqpoint{4.462381in}{3.301260in}}%
\pgfpathlineto{\pgfqpoint{4.452885in}{3.359968in}}%
\pgfpathlineto{\pgfqpoint{4.443196in}{3.340648in}}%
\pgfpathlineto{\pgfqpoint{4.409527in}{3.245336in}}%
\pgfpathlineto{\pgfqpoint{4.376380in}{3.397525in}}%
\pgfpathclose%
\pgfusepath{fill}%
\end{pgfscope}%
\begin{pgfscope}%
\pgfpathrectangle{\pgfqpoint{1.020000in}{0.880000in}}{\pgfqpoint{6.160000in}{6.160000in}}%
\pgfusepath{clip}%
\pgfsetbuttcap%
\pgfsetroundjoin%
\definecolor{currentfill}{rgb}{0.895885,0.433075,0.338681}%
\pgfsetfillcolor{currentfill}%
\pgfsetlinewidth{0.000000pt}%
\definecolor{currentstroke}{rgb}{0.000000,0.000000,0.000000}%
\pgfsetstrokecolor{currentstroke}%
\pgfsetdash{}{0pt}%
\pgfpathmoveto{\pgfqpoint{3.009116in}{5.087282in}}%
\pgfpathlineto{\pgfqpoint{3.018943in}{4.940740in}}%
\pgfpathlineto{\pgfqpoint{3.029267in}{4.749954in}}%
\pgfpathlineto{\pgfqpoint{3.061350in}{4.913756in}}%
\pgfpathlineto{\pgfqpoint{3.098299in}{4.632032in}}%
\pgfpathlineto{\pgfqpoint{3.087933in}{4.833239in}}%
\pgfpathlineto{\pgfqpoint{3.080599in}{4.754349in}}%
\pgfpathlineto{\pgfqpoint{3.042897in}{5.101490in}}%
\pgfpathlineto{\pgfqpoint{3.009116in}{5.087282in}}%
\pgfpathclose%
\pgfusepath{fill}%
\end{pgfscope}%
\begin{pgfscope}%
\pgfpathrectangle{\pgfqpoint{1.020000in}{0.880000in}}{\pgfqpoint{6.160000in}{6.160000in}}%
\pgfusepath{clip}%
\pgfsetbuttcap%
\pgfsetroundjoin%
\definecolor{currentfill}{rgb}{0.363461,0.484784,0.901019}%
\pgfsetfillcolor{currentfill}%
\pgfsetlinewidth{0.000000pt}%
\definecolor{currentstroke}{rgb}{0.000000,0.000000,0.000000}%
\pgfsetstrokecolor{currentstroke}%
\pgfsetdash{}{0pt}%
\pgfpathmoveto{\pgfqpoint{4.835279in}{3.118262in}}%
\pgfpathlineto{\pgfqpoint{4.844539in}{2.996330in}}%
\pgfpathlineto{\pgfqpoint{4.853680in}{2.857242in}}%
\pgfpathlineto{\pgfqpoint{4.888645in}{3.087648in}}%
\pgfpathlineto{\pgfqpoint{4.921562in}{3.023732in}}%
\pgfpathlineto{\pgfqpoint{4.911815in}{3.078507in}}%
\pgfpathlineto{\pgfqpoint{4.902388in}{3.176109in}}%
\pgfpathlineto{\pgfqpoint{4.867972in}{3.021870in}}%
\pgfpathlineto{\pgfqpoint{4.835279in}{3.118262in}}%
\pgfpathclose%
\pgfusepath{fill}%
\end{pgfscope}%
\begin{pgfscope}%
\pgfpathrectangle{\pgfqpoint{1.020000in}{0.880000in}}{\pgfqpoint{6.160000in}{6.160000in}}%
\pgfusepath{clip}%
\pgfsetbuttcap%
\pgfsetroundjoin%
\definecolor{currentfill}{rgb}{0.399231,0.528528,0.928459}%
\pgfsetfillcolor{currentfill}%
\pgfsetlinewidth{0.000000pt}%
\definecolor{currentstroke}{rgb}{0.000000,0.000000,0.000000}%
\pgfsetstrokecolor{currentstroke}%
\pgfsetdash{}{0pt}%
\pgfpathmoveto{\pgfqpoint{4.902388in}{3.176109in}}%
\pgfpathlineto{\pgfqpoint{4.911815in}{3.078507in}}%
\pgfpathlineto{\pgfqpoint{4.921562in}{3.023732in}}%
\pgfpathlineto{\pgfqpoint{4.956087in}{3.175395in}}%
\pgfpathlineto{\pgfqpoint{4.989109in}{3.131019in}}%
\pgfpathlineto{\pgfqpoint{4.978293in}{3.056761in}}%
\pgfpathlineto{\pgfqpoint{4.968323in}{3.085247in}}%
\pgfpathlineto{\pgfqpoint{4.935429in}{3.137169in}}%
\pgfpathlineto{\pgfqpoint{4.902388in}{3.176109in}}%
\pgfpathclose%
\pgfusepath{fill}%
\end{pgfscope}%
\begin{pgfscope}%
\pgfpathrectangle{\pgfqpoint{1.020000in}{0.880000in}}{\pgfqpoint{6.160000in}{6.160000in}}%
\pgfusepath{clip}%
\pgfsetbuttcap%
\pgfsetroundjoin%
\definecolor{currentfill}{rgb}{0.852378,0.346492,0.280346}%
\pgfsetfillcolor{currentfill}%
\pgfsetlinewidth{0.000000pt}%
\definecolor{currentstroke}{rgb}{0.000000,0.000000,0.000000}%
\pgfsetstrokecolor{currentstroke}%
\pgfsetdash{}{0pt}%
\pgfpathmoveto{\pgfqpoint{2.858747in}{4.967415in}}%
\pgfpathlineto{\pgfqpoint{2.866306in}{4.999044in}}%
\pgfpathlineto{\pgfqpoint{2.878232in}{4.693495in}}%
\pgfpathlineto{\pgfqpoint{2.909427in}{4.907994in}}%
\pgfpathlineto{\pgfqpoint{2.945125in}{4.764140in}}%
\pgfpathlineto{\pgfqpoint{2.932425in}{5.143311in}}%
\pgfpathlineto{\pgfqpoint{2.923621in}{5.204233in}}%
\pgfpathlineto{\pgfqpoint{2.890126in}{5.166772in}}%
\pgfpathlineto{\pgfqpoint{2.858747in}{4.967415in}}%
\pgfpathclose%
\pgfusepath{fill}%
\end{pgfscope}%
\begin{pgfscope}%
\pgfpathrectangle{\pgfqpoint{1.020000in}{0.880000in}}{\pgfqpoint{6.160000in}{6.160000in}}%
\pgfusepath{clip}%
\pgfsetbuttcap%
\pgfsetroundjoin%
\definecolor{currentfill}{rgb}{0.908908,0.462433,0.360950}%
\pgfsetfillcolor{currentfill}%
\pgfsetlinewidth{0.000000pt}%
\definecolor{currentstroke}{rgb}{0.000000,0.000000,0.000000}%
\pgfsetstrokecolor{currentstroke}%
\pgfsetdash{}{0pt}%
\pgfpathmoveto{\pgfqpoint{2.945125in}{4.764140in}}%
\pgfpathlineto{\pgfqpoint{2.953555in}{4.733084in}}%
\pgfpathlineto{\pgfqpoint{2.961647in}{4.731114in}}%
\pgfpathlineto{\pgfqpoint{2.995722in}{4.718559in}}%
\pgfpathlineto{\pgfqpoint{3.029267in}{4.749954in}}%
\pgfpathlineto{\pgfqpoint{3.018943in}{4.940740in}}%
\pgfpathlineto{\pgfqpoint{3.009116in}{5.087282in}}%
\pgfpathlineto{\pgfqpoint{2.977058in}{4.927262in}}%
\pgfpathlineto{\pgfqpoint{2.945125in}{4.764140in}}%
\pgfpathclose%
\pgfusepath{fill}%
\end{pgfscope}%
\begin{pgfscope}%
\pgfpathrectangle{\pgfqpoint{1.020000in}{0.880000in}}{\pgfqpoint{6.160000in}{6.160000in}}%
\pgfusepath{clip}%
\pgfsetbuttcap%
\pgfsetroundjoin%
\definecolor{currentfill}{rgb}{0.963806,0.634188,0.513721}%
\pgfsetfillcolor{currentfill}%
\pgfsetlinewidth{0.000000pt}%
\definecolor{currentstroke}{rgb}{0.000000,0.000000,0.000000}%
\pgfsetstrokecolor{currentstroke}%
\pgfsetdash{}{0pt}%
\pgfpathmoveto{\pgfqpoint{3.182706in}{4.609124in}}%
\pgfpathlineto{\pgfqpoint{3.192029in}{4.505892in}}%
\pgfpathlineto{\pgfqpoint{3.201530in}{4.384328in}}%
\pgfpathlineto{\pgfqpoint{3.234116in}{4.516328in}}%
\pgfpathlineto{\pgfqpoint{3.266489in}{4.679267in}}%
\pgfpathlineto{\pgfqpoint{3.258834in}{4.597723in}}%
\pgfpathlineto{\pgfqpoint{3.249336in}{4.719473in}}%
\pgfpathlineto{\pgfqpoint{3.216054in}{4.660008in}}%
\pgfpathlineto{\pgfqpoint{3.182706in}{4.609124in}}%
\pgfpathclose%
\pgfusepath{fill}%
\end{pgfscope}%
\begin{pgfscope}%
\pgfpathrectangle{\pgfqpoint{1.020000in}{0.880000in}}{\pgfqpoint{6.160000in}{6.160000in}}%
\pgfusepath{clip}%
\pgfsetbuttcap%
\pgfsetroundjoin%
\definecolor{currentfill}{rgb}{0.299441,0.400248,0.839842}%
\pgfsetfillcolor{currentfill}%
\pgfsetlinewidth{0.000000pt}%
\definecolor{currentstroke}{rgb}{0.000000,0.000000,0.000000}%
\pgfsetstrokecolor{currentstroke}%
\pgfsetdash{}{0pt}%
\pgfpathmoveto{\pgfqpoint{5.716777in}{3.054004in}}%
\pgfpathlineto{\pgfqpoint{5.726547in}{2.973074in}}%
\pgfpathlineto{\pgfqpoint{5.738093in}{3.005506in}}%
\pgfpathlineto{\pgfqpoint{5.769008in}{2.864357in}}%
\pgfpathlineto{\pgfqpoint{5.799738in}{2.718856in}}%
\pgfpathlineto{\pgfqpoint{5.794288in}{3.066796in}}%
\pgfpathlineto{\pgfqpoint{5.780007in}{2.868131in}}%
\pgfpathlineto{\pgfqpoint{5.745906in}{2.800378in}}%
\pgfpathlineto{\pgfqpoint{5.716777in}{3.054004in}}%
\pgfpathclose%
\pgfusepath{fill}%
\end{pgfscope}%
\begin{pgfscope}%
\pgfpathrectangle{\pgfqpoint{1.020000in}{0.880000in}}{\pgfqpoint{6.160000in}{6.160000in}}%
\pgfusepath{clip}%
\pgfsetbuttcap%
\pgfsetroundjoin%
\definecolor{currentfill}{rgb}{0.967874,0.725847,0.618489}%
\pgfsetfillcolor{currentfill}%
\pgfsetlinewidth{0.000000pt}%
\definecolor{currentstroke}{rgb}{0.000000,0.000000,0.000000}%
\pgfsetstrokecolor{currentstroke}%
\pgfsetdash{}{0pt}%
\pgfpathmoveto{\pgfqpoint{2.474162in}{4.448568in}}%
\pgfpathlineto{\pgfqpoint{2.482596in}{4.396661in}}%
\pgfpathlineto{\pgfqpoint{2.492985in}{4.228632in}}%
\pgfpathlineto{\pgfqpoint{2.523109in}{4.466867in}}%
\pgfpathlineto{\pgfqpoint{2.558866in}{4.363892in}}%
\pgfpathlineto{\pgfqpoint{2.551160in}{4.368142in}}%
\pgfpathlineto{\pgfqpoint{2.539081in}{4.642644in}}%
\pgfpathlineto{\pgfqpoint{2.508236in}{4.446869in}}%
\pgfpathlineto{\pgfqpoint{2.474162in}{4.448568in}}%
\pgfpathclose%
\pgfusepath{fill}%
\end{pgfscope}%
\begin{pgfscope}%
\pgfpathrectangle{\pgfqpoint{1.020000in}{0.880000in}}{\pgfqpoint{6.160000in}{6.160000in}}%
\pgfusepath{clip}%
\pgfsetbuttcap%
\pgfsetroundjoin%
\definecolor{currentfill}{rgb}{0.892138,0.425389,0.333289}%
\pgfsetfillcolor{currentfill}%
\pgfsetlinewidth{0.000000pt}%
\definecolor{currentstroke}{rgb}{0.000000,0.000000,0.000000}%
\pgfsetstrokecolor{currentstroke}%
\pgfsetdash{}{0pt}%
\pgfpathmoveto{\pgfqpoint{2.723038in}{4.957535in}}%
\pgfpathlineto{\pgfqpoint{2.732762in}{4.825498in}}%
\pgfpathlineto{\pgfqpoint{2.741533in}{4.760127in}}%
\pgfpathlineto{\pgfqpoint{2.772943in}{4.945806in}}%
\pgfpathlineto{\pgfqpoint{2.809461in}{4.762494in}}%
\pgfpathlineto{\pgfqpoint{2.797708in}{5.044874in}}%
\pgfpathlineto{\pgfqpoint{2.792051in}{4.880652in}}%
\pgfpathlineto{\pgfqpoint{2.758242in}{4.871108in}}%
\pgfpathlineto{\pgfqpoint{2.723038in}{4.957535in}}%
\pgfpathclose%
\pgfusepath{fill}%
\end{pgfscope}%
\begin{pgfscope}%
\pgfpathrectangle{\pgfqpoint{1.020000in}{0.880000in}}{\pgfqpoint{6.160000in}{6.160000in}}%
\pgfusepath{clip}%
\pgfsetbuttcap%
\pgfsetroundjoin%
\definecolor{currentfill}{rgb}{0.358415,0.478426,0.896795}%
\pgfsetfillcolor{currentfill}%
\pgfsetlinewidth{0.000000pt}%
\definecolor{currentstroke}{rgb}{0.000000,0.000000,0.000000}%
\pgfsetstrokecolor{currentstroke}%
\pgfsetdash{}{0pt}%
\pgfpathmoveto{\pgfqpoint{5.560631in}{2.927032in}}%
\pgfpathlineto{\pgfqpoint{5.573677in}{3.085492in}}%
\pgfpathlineto{\pgfqpoint{5.582699in}{2.953663in}}%
\pgfpathlineto{\pgfqpoint{5.618013in}{3.103105in}}%
\pgfpathlineto{\pgfqpoint{5.649711in}{3.001585in}}%
\pgfpathlineto{\pgfqpoint{5.639248in}{3.033554in}}%
\pgfpathlineto{\pgfqpoint{5.629309in}{3.100619in}}%
\pgfpathlineto{\pgfqpoint{5.595291in}{3.037778in}}%
\pgfpathlineto{\pgfqpoint{5.560631in}{2.927032in}}%
\pgfpathclose%
\pgfusepath{fill}%
\end{pgfscope}%
\begin{pgfscope}%
\pgfpathrectangle{\pgfqpoint{1.020000in}{0.880000in}}{\pgfqpoint{6.160000in}{6.160000in}}%
\pgfusepath{clip}%
\pgfsetbuttcap%
\pgfsetroundjoin%
\definecolor{currentfill}{rgb}{0.921406,0.491420,0.383408}%
\pgfsetfillcolor{currentfill}%
\pgfsetlinewidth{0.000000pt}%
\definecolor{currentstroke}{rgb}{0.000000,0.000000,0.000000}%
\pgfsetstrokecolor{currentstroke}%
\pgfsetdash{}{0pt}%
\pgfpathmoveto{\pgfqpoint{2.878232in}{4.693495in}}%
\pgfpathlineto{\pgfqpoint{2.884818in}{4.802397in}}%
\pgfpathlineto{\pgfqpoint{2.891678in}{4.892527in}}%
\pgfpathlineto{\pgfqpoint{2.926577in}{4.821705in}}%
\pgfpathlineto{\pgfqpoint{2.961647in}{4.731114in}}%
\pgfpathlineto{\pgfqpoint{2.953555in}{4.733084in}}%
\pgfpathlineto{\pgfqpoint{2.945125in}{4.764140in}}%
\pgfpathlineto{\pgfqpoint{2.909427in}{4.907994in}}%
\pgfpathlineto{\pgfqpoint{2.878232in}{4.693495in}}%
\pgfpathclose%
\pgfusepath{fill}%
\end{pgfscope}%
\begin{pgfscope}%
\pgfpathrectangle{\pgfqpoint{1.020000in}{0.880000in}}{\pgfqpoint{6.160000in}{6.160000in}}%
\pgfusepath{clip}%
\pgfsetbuttcap%
\pgfsetroundjoin%
\definecolor{currentfill}{rgb}{0.318832,0.426605,0.859857}%
\pgfsetfillcolor{currentfill}%
\pgfsetlinewidth{0.000000pt}%
\definecolor{currentstroke}{rgb}{0.000000,0.000000,0.000000}%
\pgfsetstrokecolor{currentstroke}%
\pgfsetdash{}{0pt}%
\pgfpathmoveto{\pgfqpoint{6.001979in}{2.957146in}}%
\pgfpathlineto{\pgfqpoint{6.014198in}{3.006073in}}%
\pgfpathlineto{\pgfqpoint{6.024129in}{2.929453in}}%
\pgfpathlineto{\pgfqpoint{6.056244in}{2.880022in}}%
\pgfpathlineto{\pgfqpoint{6.046100in}{2.945362in}}%
\pgfpathlineto{\pgfqpoint{6.035877in}{3.005929in}}%
\pgfpathlineto{\pgfqpoint{6.001979in}{2.957146in}}%
\pgfpathclose%
\pgfusepath{fill}%
\end{pgfscope}%
\begin{pgfscope}%
\pgfpathrectangle{\pgfqpoint{1.020000in}{0.880000in}}{\pgfqpoint{6.160000in}{6.160000in}}%
\pgfusepath{clip}%
\pgfsetbuttcap%
\pgfsetroundjoin%
\definecolor{currentfill}{rgb}{0.941728,0.546413,0.429707}%
\pgfsetfillcolor{currentfill}%
\pgfsetlinewidth{0.000000pt}%
\definecolor{currentstroke}{rgb}{0.000000,0.000000,0.000000}%
\pgfsetstrokecolor{currentstroke}%
\pgfsetdash{}{0pt}%
\pgfpathmoveto{\pgfqpoint{3.029267in}{4.749954in}}%
\pgfpathlineto{\pgfqpoint{3.038672in}{4.639331in}}%
\pgfpathlineto{\pgfqpoint{3.046747in}{4.647019in}}%
\pgfpathlineto{\pgfqpoint{3.080390in}{4.672978in}}%
\pgfpathlineto{\pgfqpoint{3.113975in}{4.704242in}}%
\pgfpathlineto{\pgfqpoint{3.105082in}{4.766263in}}%
\pgfpathlineto{\pgfqpoint{3.098299in}{4.632032in}}%
\pgfpathlineto{\pgfqpoint{3.061350in}{4.913756in}}%
\pgfpathlineto{\pgfqpoint{3.029267in}{4.749954in}}%
\pgfpathclose%
\pgfusepath{fill}%
\end{pgfscope}%
\begin{pgfscope}%
\pgfpathrectangle{\pgfqpoint{1.020000in}{0.880000in}}{\pgfqpoint{6.160000in}{6.160000in}}%
\pgfusepath{clip}%
\pgfsetbuttcap%
\pgfsetroundjoin%
\definecolor{currentfill}{rgb}{0.796064,0.848693,0.933471}%
\pgfsetfillcolor{currentfill}%
\pgfsetlinewidth{0.000000pt}%
\definecolor{currentstroke}{rgb}{0.000000,0.000000,0.000000}%
\pgfsetstrokecolor{currentstroke}%
\pgfsetdash{}{0pt}%
\pgfpathmoveto{\pgfqpoint{3.611415in}{3.901252in}}%
\pgfpathlineto{\pgfqpoint{3.620392in}{3.863254in}}%
\pgfpathlineto{\pgfqpoint{3.629761in}{3.756558in}}%
\pgfpathlineto{\pgfqpoint{3.663324in}{3.791973in}}%
\pgfpathlineto{\pgfqpoint{3.696131in}{3.984473in}}%
\pgfpathlineto{\pgfqpoint{3.687873in}{3.863179in}}%
\pgfpathlineto{\pgfqpoint{3.679022in}{3.867790in}}%
\pgfpathlineto{\pgfqpoint{3.645791in}{3.780481in}}%
\pgfpathlineto{\pgfqpoint{3.611415in}{3.901252in}}%
\pgfpathclose%
\pgfusepath{fill}%
\end{pgfscope}%
\begin{pgfscope}%
\pgfpathrectangle{\pgfqpoint{1.020000in}{0.880000in}}{\pgfqpoint{6.160000in}{6.160000in}}%
\pgfusepath{clip}%
\pgfsetbuttcap%
\pgfsetroundjoin%
\definecolor{currentfill}{rgb}{0.446431,0.582356,0.957373}%
\pgfsetfillcolor{currentfill}%
\pgfsetlinewidth{0.000000pt}%
\definecolor{currentstroke}{rgb}{0.000000,0.000000,0.000000}%
\pgfsetstrokecolor{currentstroke}%
\pgfsetdash{}{0pt}%
\pgfpathmoveto{\pgfqpoint{4.682041in}{3.157826in}}%
\pgfpathlineto{\pgfqpoint{4.691829in}{3.135962in}}%
\pgfpathlineto{\pgfqpoint{4.701654in}{3.117835in}}%
\pgfpathlineto{\pgfqpoint{4.735902in}{3.264561in}}%
\pgfpathlineto{\pgfqpoint{4.768897in}{3.184677in}}%
\pgfpathlineto{\pgfqpoint{4.759189in}{3.233407in}}%
\pgfpathlineto{\pgfqpoint{4.749770in}{3.331088in}}%
\pgfpathlineto{\pgfqpoint{4.715184in}{3.112770in}}%
\pgfpathlineto{\pgfqpoint{4.682041in}{3.157826in}}%
\pgfpathclose%
\pgfusepath{fill}%
\end{pgfscope}%
\begin{pgfscope}%
\pgfpathrectangle{\pgfqpoint{1.020000in}{0.880000in}}{\pgfqpoint{6.160000in}{6.160000in}}%
\pgfusepath{clip}%
\pgfsetbuttcap%
\pgfsetroundjoin%
\definecolor{currentfill}{rgb}{0.500031,0.638508,0.981070}%
\pgfsetfillcolor{currentfill}%
\pgfsetlinewidth{0.000000pt}%
\definecolor{currentstroke}{rgb}{0.000000,0.000000,0.000000}%
\pgfsetstrokecolor{currentstroke}%
\pgfsetdash{}{0pt}%
\pgfpathmoveto{\pgfqpoint{4.529434in}{3.330827in}}%
\pgfpathlineto{\pgfqpoint{4.539382in}{3.388335in}}%
\pgfpathlineto{\pgfqpoint{4.548331in}{3.164196in}}%
\pgfpathlineto{\pgfqpoint{4.582496in}{3.339274in}}%
\pgfpathlineto{\pgfqpoint{4.615958in}{3.329603in}}%
\pgfpathlineto{\pgfqpoint{4.605467in}{3.173227in}}%
\pgfpathlineto{\pgfqpoint{4.595689in}{3.177012in}}%
\pgfpathlineto{\pgfqpoint{4.563225in}{3.419239in}}%
\pgfpathlineto{\pgfqpoint{4.529434in}{3.330827in}}%
\pgfpathclose%
\pgfusepath{fill}%
\end{pgfscope}%
\begin{pgfscope}%
\pgfpathrectangle{\pgfqpoint{1.020000in}{0.880000in}}{\pgfqpoint{6.160000in}{6.160000in}}%
\pgfusepath{clip}%
\pgfsetbuttcap%
\pgfsetroundjoin%
\definecolor{currentfill}{rgb}{0.867428,0.864377,0.862602}%
\pgfsetfillcolor{currentfill}%
\pgfsetlinewidth{0.000000pt}%
\definecolor{currentstroke}{rgb}{0.000000,0.000000,0.000000}%
\pgfsetstrokecolor{currentstroke}%
\pgfsetdash{}{0pt}%
\pgfpathmoveto{\pgfqpoint{3.525293in}{4.094674in}}%
\pgfpathlineto{\pgfqpoint{3.534325in}{4.039577in}}%
\pgfpathlineto{\pgfqpoint{3.543715in}{3.929821in}}%
\pgfpathlineto{\pgfqpoint{3.577032in}{4.006346in}}%
\pgfpathlineto{\pgfqpoint{3.611415in}{3.901252in}}%
\pgfpathlineto{\pgfqpoint{3.601981in}{4.020484in}}%
\pgfpathlineto{\pgfqpoint{3.592898in}{4.079315in}}%
\pgfpathlineto{\pgfqpoint{3.559433in}{4.034551in}}%
\pgfpathlineto{\pgfqpoint{3.525293in}{4.094674in}}%
\pgfpathclose%
\pgfusepath{fill}%
\end{pgfscope}%
\begin{pgfscope}%
\pgfpathrectangle{\pgfqpoint{1.020000in}{0.880000in}}{\pgfqpoint{6.160000in}{6.160000in}}%
\pgfusepath{clip}%
\pgfsetbuttcap%
\pgfsetroundjoin%
\definecolor{currentfill}{rgb}{0.521696,0.659599,0.987736}%
\pgfsetfillcolor{currentfill}%
\pgfsetlinewidth{0.000000pt}%
\definecolor{currentstroke}{rgb}{0.000000,0.000000,0.000000}%
\pgfsetstrokecolor{currentstroke}%
\pgfsetdash{}{0pt}%
\pgfpathmoveto{\pgfqpoint{4.309081in}{3.194771in}}%
\pgfpathlineto{\pgfqpoint{4.318891in}{3.429425in}}%
\pgfpathlineto{\pgfqpoint{4.328140in}{3.233481in}}%
\pgfpathlineto{\pgfqpoint{4.362003in}{3.412434in}}%
\pgfpathlineto{\pgfqpoint{4.395471in}{3.358438in}}%
\pgfpathlineto{\pgfqpoint{4.385725in}{3.283109in}}%
\pgfpathlineto{\pgfqpoint{4.376380in}{3.397525in}}%
\pgfpathlineto{\pgfqpoint{4.342780in}{3.342283in}}%
\pgfpathlineto{\pgfqpoint{4.309081in}{3.194771in}}%
\pgfpathclose%
\pgfusepath{fill}%
\end{pgfscope}%
\begin{pgfscope}%
\pgfpathrectangle{\pgfqpoint{1.020000in}{0.880000in}}{\pgfqpoint{6.160000in}{6.160000in}}%
\pgfusepath{clip}%
\pgfsetbuttcap%
\pgfsetroundjoin%
\definecolor{currentfill}{rgb}{0.967711,0.662973,0.544323}%
\pgfsetfillcolor{currentfill}%
\pgfsetlinewidth{0.000000pt}%
\definecolor{currentstroke}{rgb}{0.000000,0.000000,0.000000}%
\pgfsetstrokecolor{currentstroke}%
\pgfsetdash{}{0pt}%
\pgfpathmoveto{\pgfqpoint{2.539081in}{4.642644in}}%
\pgfpathlineto{\pgfqpoint{2.551160in}{4.368142in}}%
\pgfpathlineto{\pgfqpoint{2.558866in}{4.363892in}}%
\pgfpathlineto{\pgfqpoint{2.590069in}{4.544227in}}%
\pgfpathlineto{\pgfqpoint{2.625344in}{4.467046in}}%
\pgfpathlineto{\pgfqpoint{2.617238in}{4.492467in}}%
\pgfpathlineto{\pgfqpoint{2.606048in}{4.716766in}}%
\pgfpathlineto{\pgfqpoint{2.573609in}{4.613866in}}%
\pgfpathlineto{\pgfqpoint{2.539081in}{4.642644in}}%
\pgfpathclose%
\pgfusepath{fill}%
\end{pgfscope}%
\begin{pgfscope}%
\pgfpathrectangle{\pgfqpoint{1.020000in}{0.880000in}}{\pgfqpoint{6.160000in}{6.160000in}}%
\pgfusepath{clip}%
\pgfsetbuttcap%
\pgfsetroundjoin%
\definecolor{currentfill}{rgb}{0.576051,0.708780,0.997755}%
\pgfsetfillcolor{currentfill}%
\pgfsetlinewidth{0.000000pt}%
\definecolor{currentstroke}{rgb}{0.000000,0.000000,0.000000}%
\pgfsetstrokecolor{currentstroke}%
\pgfsetdash{}{0pt}%
\pgfpathmoveto{\pgfqpoint{4.156309in}{3.417244in}}%
\pgfpathlineto{\pgfqpoint{4.165641in}{3.462488in}}%
\pgfpathlineto{\pgfqpoint{4.175013in}{3.366225in}}%
\pgfpathlineto{\pgfqpoint{4.208640in}{3.468797in}}%
\pgfpathlineto{\pgfqpoint{4.242170in}{3.323513in}}%
\pgfpathlineto{\pgfqpoint{4.232863in}{3.547136in}}%
\pgfpathlineto{\pgfqpoint{4.223386in}{3.400212in}}%
\pgfpathlineto{\pgfqpoint{4.189862in}{3.435907in}}%
\pgfpathlineto{\pgfqpoint{4.156309in}{3.417244in}}%
\pgfpathclose%
\pgfusepath{fill}%
\end{pgfscope}%
\begin{pgfscope}%
\pgfpathrectangle{\pgfqpoint{1.020000in}{0.880000in}}{\pgfqpoint{6.160000in}{6.160000in}}%
\pgfusepath{clip}%
\pgfsetbuttcap%
\pgfsetroundjoin%
\definecolor{currentfill}{rgb}{0.962701,0.628218,0.507636}%
\pgfsetfillcolor{currentfill}%
\pgfsetlinewidth{0.000000pt}%
\definecolor{currentstroke}{rgb}{0.000000,0.000000,0.000000}%
\pgfsetstrokecolor{currentstroke}%
\pgfsetdash{}{0pt}%
\pgfpathmoveto{\pgfqpoint{2.606048in}{4.716766in}}%
\pgfpathlineto{\pgfqpoint{2.617238in}{4.492467in}}%
\pgfpathlineto{\pgfqpoint{2.625344in}{4.467046in}}%
\pgfpathlineto{\pgfqpoint{2.661180in}{4.347874in}}%
\pgfpathlineto{\pgfqpoint{2.692249in}{4.545239in}}%
\pgfpathlineto{\pgfqpoint{2.680127in}{4.837760in}}%
\pgfpathlineto{\pgfqpoint{2.675669in}{4.614061in}}%
\pgfpathlineto{\pgfqpoint{2.640296in}{4.704145in}}%
\pgfpathlineto{\pgfqpoint{2.606048in}{4.716766in}}%
\pgfpathclose%
\pgfusepath{fill}%
\end{pgfscope}%
\begin{pgfscope}%
\pgfpathrectangle{\pgfqpoint{1.020000in}{0.880000in}}{\pgfqpoint{6.160000in}{6.160000in}}%
\pgfusepath{clip}%
\pgfsetbuttcap%
\pgfsetroundjoin%
\definecolor{currentfill}{rgb}{0.597777,0.727330,0.999777}%
\pgfsetfillcolor{currentfill}%
\pgfsetlinewidth{0.000000pt}%
\definecolor{currentstroke}{rgb}{0.000000,0.000000,0.000000}%
\pgfsetstrokecolor{currentstroke}%
\pgfsetdash{}{0pt}%
\pgfpathmoveto{\pgfqpoint{3.936023in}{3.550675in}}%
\pgfpathlineto{\pgfqpoint{3.945277in}{3.491494in}}%
\pgfpathlineto{\pgfqpoint{3.954358in}{3.512202in}}%
\pgfpathlineto{\pgfqpoint{3.988220in}{3.431191in}}%
\pgfpathlineto{\pgfqpoint{4.022051in}{3.323227in}}%
\pgfpathlineto{\pgfqpoint{4.012705in}{3.421486in}}%
\pgfpathlineto{\pgfqpoint{4.003446in}{3.472030in}}%
\pgfpathlineto{\pgfqpoint{3.969677in}{3.548982in}}%
\pgfpathlineto{\pgfqpoint{3.936023in}{3.550675in}}%
\pgfpathclose%
\pgfusepath{fill}%
\end{pgfscope}%
\begin{pgfscope}%
\pgfpathrectangle{\pgfqpoint{1.020000in}{0.880000in}}{\pgfqpoint{6.160000in}{6.160000in}}%
\pgfusepath{clip}%
\pgfsetbuttcap%
\pgfsetroundjoin%
\definecolor{currentfill}{rgb}{0.962701,0.628218,0.507636}%
\pgfsetfillcolor{currentfill}%
\pgfsetlinewidth{0.000000pt}%
\definecolor{currentstroke}{rgb}{0.000000,0.000000,0.000000}%
\pgfsetstrokecolor{currentstroke}%
\pgfsetdash{}{0pt}%
\pgfpathmoveto{\pgfqpoint{3.113975in}{4.704242in}}%
\pgfpathlineto{\pgfqpoint{3.121086in}{4.813066in}}%
\pgfpathlineto{\pgfqpoint{3.133136in}{4.450118in}}%
\pgfpathlineto{\pgfqpoint{3.165991in}{4.552703in}}%
\pgfpathlineto{\pgfqpoint{3.201530in}{4.384328in}}%
\pgfpathlineto{\pgfqpoint{3.192029in}{4.505892in}}%
\pgfpathlineto{\pgfqpoint{3.182706in}{4.609124in}}%
\pgfpathlineto{\pgfqpoint{3.148427in}{4.650778in}}%
\pgfpathlineto{\pgfqpoint{3.113975in}{4.704242in}}%
\pgfpathclose%
\pgfusepath{fill}%
\end{pgfscope}%
\begin{pgfscope}%
\pgfpathrectangle{\pgfqpoint{1.020000in}{0.880000in}}{\pgfqpoint{6.160000in}{6.160000in}}%
\pgfusepath{clip}%
\pgfsetbuttcap%
\pgfsetroundjoin%
\definecolor{currentfill}{rgb}{0.926883,0.505422,0.394866}%
\pgfsetfillcolor{currentfill}%
\pgfsetlinewidth{0.000000pt}%
\definecolor{currentstroke}{rgb}{0.000000,0.000000,0.000000}%
\pgfsetstrokecolor{currentstroke}%
\pgfsetdash{}{0pt}%
\pgfpathmoveto{\pgfqpoint{2.809461in}{4.762494in}}%
\pgfpathlineto{\pgfqpoint{2.816429in}{4.833085in}}%
\pgfpathlineto{\pgfqpoint{2.827170in}{4.624267in}}%
\pgfpathlineto{\pgfqpoint{2.859912in}{4.718357in}}%
\pgfpathlineto{\pgfqpoint{2.891678in}{4.892527in}}%
\pgfpathlineto{\pgfqpoint{2.884818in}{4.802397in}}%
\pgfpathlineto{\pgfqpoint{2.878232in}{4.693495in}}%
\pgfpathlineto{\pgfqpoint{2.842137in}{4.858902in}}%
\pgfpathlineto{\pgfqpoint{2.809461in}{4.762494in}}%
\pgfpathclose%
\pgfusepath{fill}%
\end{pgfscope}%
\begin{pgfscope}%
\pgfpathrectangle{\pgfqpoint{1.020000in}{0.880000in}}{\pgfqpoint{6.160000in}{6.160000in}}%
\pgfusepath{clip}%
\pgfsetbuttcap%
\pgfsetroundjoin%
\definecolor{currentfill}{rgb}{0.950956,0.786875,0.704761}%
\pgfsetfillcolor{currentfill}%
\pgfsetlinewidth{0.000000pt}%
\definecolor{currentstroke}{rgb}{0.000000,0.000000,0.000000}%
\pgfsetstrokecolor{currentstroke}%
\pgfsetdash{}{0pt}%
\pgfpathmoveto{\pgfqpoint{3.286820in}{4.288011in}}%
\pgfpathlineto{\pgfqpoint{3.294743in}{4.344630in}}%
\pgfpathlineto{\pgfqpoint{3.303466in}{4.312553in}}%
\pgfpathlineto{\pgfqpoint{3.337404in}{4.307324in}}%
\pgfpathlineto{\pgfqpoint{3.372116in}{4.203039in}}%
\pgfpathlineto{\pgfqpoint{3.363415in}{4.228004in}}%
\pgfpathlineto{\pgfqpoint{3.354308in}{4.303974in}}%
\pgfpathlineto{\pgfqpoint{3.321207in}{4.221726in}}%
\pgfpathlineto{\pgfqpoint{3.286820in}{4.288011in}}%
\pgfpathclose%
\pgfusepath{fill}%
\end{pgfscope}%
\begin{pgfscope}%
\pgfpathrectangle{\pgfqpoint{1.020000in}{0.880000in}}{\pgfqpoint{6.160000in}{6.160000in}}%
\pgfusepath{clip}%
\pgfsetbuttcap%
\pgfsetroundjoin%
\definecolor{currentfill}{rgb}{0.953054,0.585211,0.465373}%
\pgfsetfillcolor{currentfill}%
\pgfsetlinewidth{0.000000pt}%
\definecolor{currentstroke}{rgb}{0.000000,0.000000,0.000000}%
\pgfsetstrokecolor{currentstroke}%
\pgfsetdash{}{0pt}%
\pgfpathmoveto{\pgfqpoint{2.675669in}{4.614061in}}%
\pgfpathlineto{\pgfqpoint{2.680127in}{4.837760in}}%
\pgfpathlineto{\pgfqpoint{2.692249in}{4.545239in}}%
\pgfpathlineto{\pgfqpoint{2.727298in}{4.474136in}}%
\pgfpathlineto{\pgfqpoint{2.759276in}{4.616476in}}%
\pgfpathlineto{\pgfqpoint{2.748507in}{4.822186in}}%
\pgfpathlineto{\pgfqpoint{2.741533in}{4.760127in}}%
\pgfpathlineto{\pgfqpoint{2.710061in}{4.585865in}}%
\pgfpathlineto{\pgfqpoint{2.675669in}{4.614061in}}%
\pgfpathclose%
\pgfusepath{fill}%
\end{pgfscope}%
\begin{pgfscope}%
\pgfpathrectangle{\pgfqpoint{1.020000in}{0.880000in}}{\pgfqpoint{6.160000in}{6.160000in}}%
\pgfusepath{clip}%
\pgfsetbuttcap%
\pgfsetroundjoin%
\definecolor{currentfill}{rgb}{0.388852,0.516298,0.921373}%
\pgfsetfillcolor{currentfill}%
\pgfsetlinewidth{0.000000pt}%
\definecolor{currentstroke}{rgb}{0.000000,0.000000,0.000000}%
\pgfsetstrokecolor{currentstroke}%
\pgfsetdash{}{0pt}%
\pgfpathmoveto{\pgfqpoint{5.055738in}{3.126077in}}%
\pgfpathlineto{\pgfqpoint{5.065451in}{3.058606in}}%
\pgfpathlineto{\pgfqpoint{5.076048in}{3.088881in}}%
\pgfpathlineto{\pgfqpoint{5.109947in}{3.147195in}}%
\pgfpathlineto{\pgfqpoint{5.141262in}{2.935116in}}%
\pgfpathlineto{\pgfqpoint{5.132017in}{3.054105in}}%
\pgfpathlineto{\pgfqpoint{5.122723in}{3.168170in}}%
\pgfpathlineto{\pgfqpoint{5.088813in}{3.100865in}}%
\pgfpathlineto{\pgfqpoint{5.055738in}{3.126077in}}%
\pgfpathclose%
\pgfusepath{fill}%
\end{pgfscope}%
\begin{pgfscope}%
\pgfpathrectangle{\pgfqpoint{1.020000in}{0.880000in}}{\pgfqpoint{6.160000in}{6.160000in}}%
\pgfusepath{clip}%
\pgfsetbuttcap%
\pgfsetroundjoin%
\definecolor{currentfill}{rgb}{0.318832,0.426605,0.859857}%
\pgfsetfillcolor{currentfill}%
\pgfsetlinewidth{0.000000pt}%
\definecolor{currentstroke}{rgb}{0.000000,0.000000,0.000000}%
\pgfsetstrokecolor{currentstroke}%
\pgfsetdash{}{0pt}%
\pgfpathmoveto{\pgfqpoint{5.935124in}{2.910990in}}%
\pgfpathlineto{\pgfqpoint{5.945513in}{2.862368in}}%
\pgfpathlineto{\pgfqpoint{5.958435in}{2.955088in}}%
\pgfpathlineto{\pgfqpoint{5.992007in}{2.981412in}}%
\pgfpathlineto{\pgfqpoint{6.024129in}{2.929453in}}%
\pgfpathlineto{\pgfqpoint{6.014198in}{3.006073in}}%
\pgfpathlineto{\pgfqpoint{6.001979in}{2.957146in}}%
\pgfpathlineto{\pgfqpoint{5.969104in}{2.964757in}}%
\pgfpathlineto{\pgfqpoint{5.935124in}{2.910990in}}%
\pgfpathclose%
\pgfusepath{fill}%
\end{pgfscope}%
\begin{pgfscope}%
\pgfpathrectangle{\pgfqpoint{1.020000in}{0.880000in}}{\pgfqpoint{6.160000in}{6.160000in}}%
\pgfusepath{clip}%
\pgfsetbuttcap%
\pgfsetroundjoin%
\definecolor{currentfill}{rgb}{0.373552,0.497499,0.909467}%
\pgfsetfillcolor{currentfill}%
\pgfsetlinewidth{0.000000pt}%
\definecolor{currentstroke}{rgb}{0.000000,0.000000,0.000000}%
\pgfsetstrokecolor{currentstroke}%
\pgfsetdash{}{0pt}%
\pgfpathmoveto{\pgfqpoint{5.276573in}{3.151054in}}%
\pgfpathlineto{\pgfqpoint{5.285009in}{2.959695in}}%
\pgfpathlineto{\pgfqpoint{5.297289in}{3.113937in}}%
\pgfpathlineto{\pgfqpoint{5.329519in}{3.020053in}}%
\pgfpathlineto{\pgfqpoint{5.363148in}{3.051208in}}%
\pgfpathlineto{\pgfqpoint{5.352671in}{3.064334in}}%
\pgfpathlineto{\pgfqpoint{5.341410in}{3.008350in}}%
\pgfpathlineto{\pgfqpoint{5.308670in}{3.048230in}}%
\pgfpathlineto{\pgfqpoint{5.276573in}{3.151054in}}%
\pgfpathclose%
\pgfusepath{fill}%
\end{pgfscope}%
\begin{pgfscope}%
\pgfpathrectangle{\pgfqpoint{1.020000in}{0.880000in}}{\pgfqpoint{6.160000in}{6.160000in}}%
\pgfusepath{clip}%
\pgfsetbuttcap%
\pgfsetroundjoin%
\definecolor{currentfill}{rgb}{0.738826,0.822572,0.968261}%
\pgfsetfillcolor{currentfill}%
\pgfsetlinewidth{0.000000pt}%
\definecolor{currentstroke}{rgb}{0.000000,0.000000,0.000000}%
\pgfsetstrokecolor{currentstroke}%
\pgfsetdash{}{0pt}%
\pgfpathmoveto{\pgfqpoint{3.696131in}{3.984473in}}%
\pgfpathlineto{\pgfqpoint{3.706667in}{3.635447in}}%
\pgfpathlineto{\pgfqpoint{3.715169in}{3.712956in}}%
\pgfpathlineto{\pgfqpoint{3.749059in}{3.682879in}}%
\pgfpathlineto{\pgfqpoint{3.782816in}{3.675742in}}%
\pgfpathlineto{\pgfqpoint{3.774265in}{3.582079in}}%
\pgfpathlineto{\pgfqpoint{3.764329in}{3.828684in}}%
\pgfpathlineto{\pgfqpoint{3.731115in}{3.718347in}}%
\pgfpathlineto{\pgfqpoint{3.696131in}{3.984473in}}%
\pgfpathclose%
\pgfusepath{fill}%
\end{pgfscope}%
\begin{pgfscope}%
\pgfpathrectangle{\pgfqpoint{1.020000in}{0.880000in}}{\pgfqpoint{6.160000in}{6.160000in}}%
\pgfusepath{clip}%
\pgfsetbuttcap%
\pgfsetroundjoin%
\definecolor{currentfill}{rgb}{0.693321,0.796314,0.986308}%
\pgfsetfillcolor{currentfill}%
\pgfsetlinewidth{0.000000pt}%
\definecolor{currentstroke}{rgb}{0.000000,0.000000,0.000000}%
\pgfsetstrokecolor{currentstroke}%
\pgfsetdash{}{0pt}%
\pgfpathmoveto{\pgfqpoint{3.850482in}{3.593671in}}%
\pgfpathlineto{\pgfqpoint{3.858818in}{3.802839in}}%
\pgfpathlineto{\pgfqpoint{3.868232in}{3.687176in}}%
\pgfpathlineto{\pgfqpoint{3.902497in}{3.498815in}}%
\pgfpathlineto{\pgfqpoint{3.936023in}{3.550675in}}%
\pgfpathlineto{\pgfqpoint{3.926875in}{3.571433in}}%
\pgfpathlineto{\pgfqpoint{3.917667in}{3.621039in}}%
\pgfpathlineto{\pgfqpoint{3.883613in}{3.763644in}}%
\pgfpathlineto{\pgfqpoint{3.850482in}{3.593671in}}%
\pgfpathclose%
\pgfusepath{fill}%
\end{pgfscope}%
\begin{pgfscope}%
\pgfpathrectangle{\pgfqpoint{1.020000in}{0.880000in}}{\pgfqpoint{6.160000in}{6.160000in}}%
\pgfusepath{clip}%
\pgfsetbuttcap%
\pgfsetroundjoin%
\definecolor{currentfill}{rgb}{0.394042,0.522413,0.924916}%
\pgfsetfillcolor{currentfill}%
\pgfsetlinewidth{0.000000pt}%
\definecolor{currentstroke}{rgb}{0.000000,0.000000,0.000000}%
\pgfsetstrokecolor{currentstroke}%
\pgfsetdash{}{0pt}%
\pgfpathmoveto{\pgfqpoint{4.768897in}{3.184677in}}%
\pgfpathlineto{\pgfqpoint{4.778728in}{3.154409in}}%
\pgfpathlineto{\pgfqpoint{4.788893in}{3.176817in}}%
\pgfpathlineto{\pgfqpoint{4.822264in}{3.159582in}}%
\pgfpathlineto{\pgfqpoint{4.853680in}{2.857242in}}%
\pgfpathlineto{\pgfqpoint{4.844539in}{2.996330in}}%
\pgfpathlineto{\pgfqpoint{4.835279in}{3.118262in}}%
\pgfpathlineto{\pgfqpoint{4.801437in}{3.043198in}}%
\pgfpathlineto{\pgfqpoint{4.768897in}{3.184677in}}%
\pgfpathclose%
\pgfusepath{fill}%
\end{pgfscope}%
\begin{pgfscope}%
\pgfpathrectangle{\pgfqpoint{1.020000in}{0.880000in}}{\pgfqpoint{6.160000in}{6.160000in}}%
\pgfusepath{clip}%
\pgfsetbuttcap%
\pgfsetroundjoin%
\definecolor{currentfill}{rgb}{0.969851,0.695830,0.581312}%
\pgfsetfillcolor{currentfill}%
\pgfsetlinewidth{0.000000pt}%
\definecolor{currentstroke}{rgb}{0.000000,0.000000,0.000000}%
\pgfsetstrokecolor{currentstroke}%
\pgfsetdash{}{0pt}%
\pgfpathmoveto{\pgfqpoint{3.201530in}{4.384328in}}%
\pgfpathlineto{\pgfqpoint{3.208294in}{4.546366in}}%
\pgfpathlineto{\pgfqpoint{3.216761in}{4.534939in}}%
\pgfpathlineto{\pgfqpoint{3.252512in}{4.338886in}}%
\pgfpathlineto{\pgfqpoint{3.286820in}{4.288011in}}%
\pgfpathlineto{\pgfqpoint{3.277439in}{4.397087in}}%
\pgfpathlineto{\pgfqpoint{3.266489in}{4.679267in}}%
\pgfpathlineto{\pgfqpoint{3.234116in}{4.516328in}}%
\pgfpathlineto{\pgfqpoint{3.201530in}{4.384328in}}%
\pgfpathclose%
\pgfusepath{fill}%
\end{pgfscope}%
\begin{pgfscope}%
\pgfpathrectangle{\pgfqpoint{1.020000in}{0.880000in}}{\pgfqpoint{6.160000in}{6.160000in}}%
\pgfusepath{clip}%
\pgfsetbuttcap%
\pgfsetroundjoin%
\definecolor{currentfill}{rgb}{0.968203,0.720844,0.612293}%
\pgfsetfillcolor{currentfill}%
\pgfsetlinewidth{0.000000pt}%
\definecolor{currentstroke}{rgb}{0.000000,0.000000,0.000000}%
\pgfsetstrokecolor{currentstroke}%
\pgfsetdash{}{0pt}%
\pgfpathmoveto{\pgfqpoint{2.625344in}{4.467046in}}%
\pgfpathlineto{\pgfqpoint{2.635053in}{4.337918in}}%
\pgfpathlineto{\pgfqpoint{2.641290in}{4.436012in}}%
\pgfpathlineto{\pgfqpoint{2.674847in}{4.472170in}}%
\pgfpathlineto{\pgfqpoint{2.709340in}{4.444412in}}%
\pgfpathlineto{\pgfqpoint{2.704395in}{4.249047in}}%
\pgfpathlineto{\pgfqpoint{2.692249in}{4.545239in}}%
\pgfpathlineto{\pgfqpoint{2.661180in}{4.347874in}}%
\pgfpathlineto{\pgfqpoint{2.625344in}{4.467046in}}%
\pgfpathclose%
\pgfusepath{fill}%
\end{pgfscope}%
\begin{pgfscope}%
\pgfpathrectangle{\pgfqpoint{1.020000in}{0.880000in}}{\pgfqpoint{6.160000in}{6.160000in}}%
\pgfusepath{clip}%
\pgfsetbuttcap%
\pgfsetroundjoin%
\definecolor{currentfill}{rgb}{0.373552,0.497499,0.909467}%
\pgfsetfillcolor{currentfill}%
\pgfsetlinewidth{0.000000pt}%
\definecolor{currentstroke}{rgb}{0.000000,0.000000,0.000000}%
\pgfsetstrokecolor{currentstroke}%
\pgfsetdash{}{0pt}%
\pgfpathmoveto{\pgfqpoint{5.496310in}{3.072535in}}%
\pgfpathlineto{\pgfqpoint{5.507168in}{3.077924in}}%
\pgfpathlineto{\pgfqpoint{5.518262in}{3.099143in}}%
\pgfpathlineto{\pgfqpoint{5.550963in}{3.059081in}}%
\pgfpathlineto{\pgfqpoint{5.582699in}{2.953663in}}%
\pgfpathlineto{\pgfqpoint{5.573677in}{3.085492in}}%
\pgfpathlineto{\pgfqpoint{5.560631in}{2.927032in}}%
\pgfpathlineto{\pgfqpoint{5.530139in}{3.120725in}}%
\pgfpathlineto{\pgfqpoint{5.496310in}{3.072535in}}%
\pgfpathclose%
\pgfusepath{fill}%
\end{pgfscope}%
\begin{pgfscope}%
\pgfpathrectangle{\pgfqpoint{1.020000in}{0.880000in}}{\pgfqpoint{6.160000in}{6.160000in}}%
\pgfusepath{clip}%
\pgfsetbuttcap%
\pgfsetroundjoin%
\definecolor{currentfill}{rgb}{0.451739,0.588181,0.960201}%
\pgfsetfillcolor{currentfill}%
\pgfsetlinewidth{0.000000pt}%
\definecolor{currentstroke}{rgb}{0.000000,0.000000,0.000000}%
\pgfsetstrokecolor{currentstroke}%
\pgfsetdash{}{0pt}%
\pgfpathmoveto{\pgfqpoint{4.615958in}{3.329603in}}%
\pgfpathlineto{\pgfqpoint{4.624907in}{3.125914in}}%
\pgfpathlineto{\pgfqpoint{4.635211in}{3.225584in}}%
\pgfpathlineto{\pgfqpoint{4.668632in}{3.206913in}}%
\pgfpathlineto{\pgfqpoint{4.701654in}{3.117835in}}%
\pgfpathlineto{\pgfqpoint{4.691829in}{3.135962in}}%
\pgfpathlineto{\pgfqpoint{4.682041in}{3.157826in}}%
\pgfpathlineto{\pgfqpoint{4.649076in}{3.251762in}}%
\pgfpathlineto{\pgfqpoint{4.615958in}{3.329603in}}%
\pgfpathclose%
\pgfusepath{fill}%
\end{pgfscope}%
\begin{pgfscope}%
\pgfpathrectangle{\pgfqpoint{1.020000in}{0.880000in}}{\pgfqpoint{6.160000in}{6.160000in}}%
\pgfusepath{clip}%
\pgfsetbuttcap%
\pgfsetroundjoin%
\definecolor{currentfill}{rgb}{0.929357,0.512254,0.400673}%
\pgfsetfillcolor{currentfill}%
\pgfsetlinewidth{0.000000pt}%
\definecolor{currentstroke}{rgb}{0.000000,0.000000,0.000000}%
\pgfsetstrokecolor{currentstroke}%
\pgfsetdash{}{0pt}%
\pgfpathmoveto{\pgfqpoint{2.741533in}{4.760127in}}%
\pgfpathlineto{\pgfqpoint{2.748507in}{4.822186in}}%
\pgfpathlineto{\pgfqpoint{2.759276in}{4.616476in}}%
\pgfpathlineto{\pgfqpoint{2.791864in}{4.719917in}}%
\pgfpathlineto{\pgfqpoint{2.827170in}{4.624267in}}%
\pgfpathlineto{\pgfqpoint{2.816429in}{4.833085in}}%
\pgfpathlineto{\pgfqpoint{2.809461in}{4.762494in}}%
\pgfpathlineto{\pgfqpoint{2.772943in}{4.945806in}}%
\pgfpathlineto{\pgfqpoint{2.741533in}{4.760127in}}%
\pgfpathclose%
\pgfusepath{fill}%
\end{pgfscope}%
\begin{pgfscope}%
\pgfpathrectangle{\pgfqpoint{1.020000in}{0.880000in}}{\pgfqpoint{6.160000in}{6.160000in}}%
\pgfusepath{clip}%
\pgfsetbuttcap%
\pgfsetroundjoin%
\definecolor{currentfill}{rgb}{0.505423,0.643995,0.983157}%
\pgfsetfillcolor{currentfill}%
\pgfsetlinewidth{0.000000pt}%
\definecolor{currentstroke}{rgb}{0.000000,0.000000,0.000000}%
\pgfsetstrokecolor{currentstroke}%
\pgfsetdash{}{0pt}%
\pgfpathmoveto{\pgfqpoint{4.462381in}{3.301260in}}%
\pgfpathlineto{\pgfqpoint{4.472258in}{3.371558in}}%
\pgfpathlineto{\pgfqpoint{4.481518in}{3.223496in}}%
\pgfpathlineto{\pgfqpoint{4.514886in}{3.176147in}}%
\pgfpathlineto{\pgfqpoint{4.548331in}{3.164196in}}%
\pgfpathlineto{\pgfqpoint{4.539382in}{3.388335in}}%
\pgfpathlineto{\pgfqpoint{4.529434in}{3.330827in}}%
\pgfpathlineto{\pgfqpoint{4.496137in}{3.387907in}}%
\pgfpathlineto{\pgfqpoint{4.462381in}{3.301260in}}%
\pgfpathclose%
\pgfusepath{fill}%
\end{pgfscope}%
\begin{pgfscope}%
\pgfpathrectangle{\pgfqpoint{1.020000in}{0.880000in}}{\pgfqpoint{6.160000in}{6.160000in}}%
\pgfusepath{clip}%
\pgfsetbuttcap%
\pgfsetroundjoin%
\definecolor{currentfill}{rgb}{0.936780,0.532750,0.418093}%
\pgfsetfillcolor{currentfill}%
\pgfsetlinewidth{0.000000pt}%
\definecolor{currentstroke}{rgb}{0.000000,0.000000,0.000000}%
\pgfsetstrokecolor{currentstroke}%
\pgfsetdash{}{0pt}%
\pgfpathmoveto{\pgfqpoint{2.961647in}{4.731114in}}%
\pgfpathlineto{\pgfqpoint{2.968973in}{4.794518in}}%
\pgfpathlineto{\pgfqpoint{2.979436in}{4.596483in}}%
\pgfpathlineto{\pgfqpoint{3.009305in}{4.949335in}}%
\pgfpathlineto{\pgfqpoint{3.046747in}{4.647019in}}%
\pgfpathlineto{\pgfqpoint{3.038672in}{4.639331in}}%
\pgfpathlineto{\pgfqpoint{3.029267in}{4.749954in}}%
\pgfpathlineto{\pgfqpoint{2.995722in}{4.718559in}}%
\pgfpathlineto{\pgfqpoint{2.961647in}{4.731114in}}%
\pgfpathclose%
\pgfusepath{fill}%
\end{pgfscope}%
\begin{pgfscope}%
\pgfpathrectangle{\pgfqpoint{1.020000in}{0.880000in}}{\pgfqpoint{6.160000in}{6.160000in}}%
\pgfusepath{clip}%
\pgfsetbuttcap%
\pgfsetroundjoin%
\definecolor{currentfill}{rgb}{0.527132,0.664700,0.989065}%
\pgfsetfillcolor{currentfill}%
\pgfsetlinewidth{0.000000pt}%
\definecolor{currentstroke}{rgb}{0.000000,0.000000,0.000000}%
\pgfsetstrokecolor{currentstroke}%
\pgfsetdash{}{0pt}%
\pgfpathmoveto{\pgfqpoint{4.242170in}{3.323513in}}%
\pgfpathlineto{\pgfqpoint{4.251651in}{3.387633in}}%
\pgfpathlineto{\pgfqpoint{4.261094in}{3.361106in}}%
\pgfpathlineto{\pgfqpoint{4.294735in}{3.374307in}}%
\pgfpathlineto{\pgfqpoint{4.328140in}{3.233481in}}%
\pgfpathlineto{\pgfqpoint{4.318891in}{3.429425in}}%
\pgfpathlineto{\pgfqpoint{4.309081in}{3.194771in}}%
\pgfpathlineto{\pgfqpoint{4.275749in}{3.357906in}}%
\pgfpathlineto{\pgfqpoint{4.242170in}{3.323513in}}%
\pgfpathclose%
\pgfusepath{fill}%
\end{pgfscope}%
\begin{pgfscope}%
\pgfpathrectangle{\pgfqpoint{1.020000in}{0.880000in}}{\pgfqpoint{6.160000in}{6.160000in}}%
\pgfusepath{clip}%
\pgfsetbuttcap%
\pgfsetroundjoin%
\definecolor{currentfill}{rgb}{0.933221,0.815557,0.753151}%
\pgfsetfillcolor{currentfill}%
\pgfsetlinewidth{0.000000pt}%
\definecolor{currentstroke}{rgb}{0.000000,0.000000,0.000000}%
\pgfsetstrokecolor{currentstroke}%
\pgfsetdash{}{0pt}%
\pgfpathmoveto{\pgfqpoint{2.359046in}{4.095211in}}%
\pgfpathlineto{\pgfqpoint{2.362548in}{4.312770in}}%
\pgfpathlineto{\pgfqpoint{2.374449in}{4.065743in}}%
\pgfpathlineto{\pgfqpoint{2.408056in}{4.095553in}}%
\pgfpathlineto{\pgfqpoint{2.438073in}{4.332904in}}%
\pgfpathlineto{\pgfqpoint{2.434843in}{4.085319in}}%
\pgfpathlineto{\pgfqpoint{2.423861in}{4.285124in}}%
\pgfpathlineto{\pgfqpoint{2.389772in}{4.283493in}}%
\pgfpathlineto{\pgfqpoint{2.359046in}{4.095211in}}%
\pgfpathclose%
\pgfusepath{fill}%
\end{pgfscope}%
\begin{pgfscope}%
\pgfpathrectangle{\pgfqpoint{1.020000in}{0.880000in}}{\pgfqpoint{6.160000in}{6.160000in}}%
\pgfusepath{clip}%
\pgfsetbuttcap%
\pgfsetroundjoin%
\definecolor{currentfill}{rgb}{0.928116,0.822197,0.765141}%
\pgfsetfillcolor{currentfill}%
\pgfsetlinewidth{0.000000pt}%
\definecolor{currentstroke}{rgb}{0.000000,0.000000,0.000000}%
\pgfsetstrokecolor{currentstroke}%
\pgfsetdash{}{0pt}%
\pgfpathmoveto{\pgfqpoint{3.372116in}{4.203039in}}%
\pgfpathlineto{\pgfqpoint{3.380214in}{4.255645in}}%
\pgfpathlineto{\pgfqpoint{3.390073in}{4.088102in}}%
\pgfpathlineto{\pgfqpoint{3.423993in}{4.078286in}}%
\pgfpathlineto{\pgfqpoint{3.457177in}{4.166839in}}%
\pgfpathlineto{\pgfqpoint{3.448975in}{4.112418in}}%
\pgfpathlineto{\pgfqpoint{3.440072in}{4.156810in}}%
\pgfpathlineto{\pgfqpoint{3.404923in}{4.334079in}}%
\pgfpathlineto{\pgfqpoint{3.372116in}{4.203039in}}%
\pgfpathclose%
\pgfusepath{fill}%
\end{pgfscope}%
\begin{pgfscope}%
\pgfpathrectangle{\pgfqpoint{1.020000in}{0.880000in}}{\pgfqpoint{6.160000in}{6.160000in}}%
\pgfusepath{clip}%
\pgfsetbuttcap%
\pgfsetroundjoin%
\definecolor{currentfill}{rgb}{0.333490,0.446265,0.874452}%
\pgfsetfillcolor{currentfill}%
\pgfsetlinewidth{0.000000pt}%
\definecolor{currentstroke}{rgb}{0.000000,0.000000,0.000000}%
\pgfsetstrokecolor{currentstroke}%
\pgfsetdash{}{0pt}%
\pgfpathmoveto{\pgfqpoint{5.649711in}{3.001585in}}%
\pgfpathlineto{\pgfqpoint{5.658082in}{2.827446in}}%
\pgfpathlineto{\pgfqpoint{5.670998in}{2.958376in}}%
\pgfpathlineto{\pgfqpoint{5.703437in}{2.909251in}}%
\pgfpathlineto{\pgfqpoint{5.738093in}{3.005506in}}%
\pgfpathlineto{\pgfqpoint{5.726547in}{2.973074in}}%
\pgfpathlineto{\pgfqpoint{5.716777in}{3.054004in}}%
\pgfpathlineto{\pgfqpoint{5.683443in}{3.040899in}}%
\pgfpathlineto{\pgfqpoint{5.649711in}{3.001585in}}%
\pgfpathclose%
\pgfusepath{fill}%
\end{pgfscope}%
\begin{pgfscope}%
\pgfpathrectangle{\pgfqpoint{1.020000in}{0.880000in}}{\pgfqpoint{6.160000in}{6.160000in}}%
\pgfusepath{clip}%
\pgfsetbuttcap%
\pgfsetroundjoin%
\definecolor{currentfill}{rgb}{0.875557,0.860242,0.851430}%
\pgfsetfillcolor{currentfill}%
\pgfsetlinewidth{0.000000pt}%
\definecolor{currentstroke}{rgb}{0.000000,0.000000,0.000000}%
\pgfsetstrokecolor{currentstroke}%
\pgfsetdash{}{0pt}%
\pgfpathmoveto{\pgfqpoint{3.457177in}{4.166839in}}%
\pgfpathlineto{\pgfqpoint{3.466825in}{4.022073in}}%
\pgfpathlineto{\pgfqpoint{3.475541in}{4.008695in}}%
\pgfpathlineto{\pgfqpoint{3.509319in}{4.018892in}}%
\pgfpathlineto{\pgfqpoint{3.543715in}{3.929821in}}%
\pgfpathlineto{\pgfqpoint{3.534325in}{4.039577in}}%
\pgfpathlineto{\pgfqpoint{3.525293in}{4.094674in}}%
\pgfpathlineto{\pgfqpoint{3.492763in}{3.911568in}}%
\pgfpathlineto{\pgfqpoint{3.457177in}{4.166839in}}%
\pgfpathclose%
\pgfusepath{fill}%
\end{pgfscope}%
\begin{pgfscope}%
\pgfpathrectangle{\pgfqpoint{1.020000in}{0.880000in}}{\pgfqpoint{6.160000in}{6.160000in}}%
\pgfusepath{clip}%
\pgfsetbuttcap%
\pgfsetroundjoin%
\definecolor{currentfill}{rgb}{0.950956,0.786875,0.704761}%
\pgfsetfillcolor{currentfill}%
\pgfsetlinewidth{0.000000pt}%
\definecolor{currentstroke}{rgb}{0.000000,0.000000,0.000000}%
\pgfsetstrokecolor{currentstroke}%
\pgfsetdash{}{0pt}%
\pgfpathmoveto{\pgfqpoint{2.423861in}{4.285124in}}%
\pgfpathlineto{\pgfqpoint{2.434843in}{4.085319in}}%
\pgfpathlineto{\pgfqpoint{2.438073in}{4.332904in}}%
\pgfpathlineto{\pgfqpoint{2.474198in}{4.218361in}}%
\pgfpathlineto{\pgfqpoint{2.508415in}{4.212740in}}%
\pgfpathlineto{\pgfqpoint{2.497323in}{4.422521in}}%
\pgfpathlineto{\pgfqpoint{2.492985in}{4.228632in}}%
\pgfpathlineto{\pgfqpoint{2.457196in}{4.329775in}}%
\pgfpathlineto{\pgfqpoint{2.423861in}{4.285124in}}%
\pgfpathclose%
\pgfusepath{fill}%
\end{pgfscope}%
\begin{pgfscope}%
\pgfpathrectangle{\pgfqpoint{1.020000in}{0.880000in}}{\pgfqpoint{6.160000in}{6.160000in}}%
\pgfusepath{clip}%
\pgfsetbuttcap%
\pgfsetroundjoin%
\definecolor{currentfill}{rgb}{0.373552,0.497499,0.909467}%
\pgfsetfillcolor{currentfill}%
\pgfsetlinewidth{0.000000pt}%
\definecolor{currentstroke}{rgb}{0.000000,0.000000,0.000000}%
\pgfsetstrokecolor{currentstroke}%
\pgfsetdash{}{0pt}%
\pgfpathmoveto{\pgfqpoint{5.208541in}{3.009637in}}%
\pgfpathlineto{\pgfqpoint{5.217920in}{2.904450in}}%
\pgfpathlineto{\pgfqpoint{5.229519in}{3.010464in}}%
\pgfpathlineto{\pgfqpoint{5.264514in}{3.165239in}}%
\pgfpathlineto{\pgfqpoint{5.297289in}{3.113937in}}%
\pgfpathlineto{\pgfqpoint{5.285009in}{2.959695in}}%
\pgfpathlineto{\pgfqpoint{5.276573in}{3.151054in}}%
\pgfpathlineto{\pgfqpoint{5.242559in}{3.081848in}}%
\pgfpathlineto{\pgfqpoint{5.208541in}{3.009637in}}%
\pgfpathclose%
\pgfusepath{fill}%
\end{pgfscope}%
\begin{pgfscope}%
\pgfpathrectangle{\pgfqpoint{1.020000in}{0.880000in}}{\pgfqpoint{6.160000in}{6.160000in}}%
\pgfusepath{clip}%
\pgfsetbuttcap%
\pgfsetroundjoin%
\definecolor{currentfill}{rgb}{0.328604,0.439712,0.869587}%
\pgfsetfillcolor{currentfill}%
\pgfsetlinewidth{0.000000pt}%
\definecolor{currentstroke}{rgb}{0.000000,0.000000,0.000000}%
\pgfsetstrokecolor{currentstroke}%
\pgfsetdash{}{0pt}%
\pgfpathmoveto{\pgfqpoint{5.872192in}{3.097294in}}%
\pgfpathlineto{\pgfqpoint{5.879445in}{2.865553in}}%
\pgfpathlineto{\pgfqpoint{5.893201in}{3.014148in}}%
\pgfpathlineto{\pgfqpoint{5.925123in}{2.943764in}}%
\pgfpathlineto{\pgfqpoint{5.958435in}{2.955088in}}%
\pgfpathlineto{\pgfqpoint{5.945513in}{2.862368in}}%
\pgfpathlineto{\pgfqpoint{5.935124in}{2.910990in}}%
\pgfpathlineto{\pgfqpoint{5.903564in}{2.996312in}}%
\pgfpathlineto{\pgfqpoint{5.872192in}{3.097294in}}%
\pgfpathclose%
\pgfusepath{fill}%
\end{pgfscope}%
\begin{pgfscope}%
\pgfpathrectangle{\pgfqpoint{1.020000in}{0.880000in}}{\pgfqpoint{6.160000in}{6.160000in}}%
\pgfusepath{clip}%
\pgfsetbuttcap%
\pgfsetroundjoin%
\definecolor{currentfill}{rgb}{0.353369,0.472069,0.892570}%
\pgfsetfillcolor{currentfill}%
\pgfsetlinewidth{0.000000pt}%
\definecolor{currentstroke}{rgb}{0.000000,0.000000,0.000000}%
\pgfsetstrokecolor{currentstroke}%
\pgfsetdash{}{0pt}%
\pgfpathmoveto{\pgfqpoint{5.141262in}{2.935116in}}%
\pgfpathlineto{\pgfqpoint{5.152633in}{3.034887in}}%
\pgfpathlineto{\pgfqpoint{5.162385in}{2.965567in}}%
\pgfpathlineto{\pgfqpoint{5.197349in}{3.125112in}}%
\pgfpathlineto{\pgfqpoint{5.229519in}{3.010464in}}%
\pgfpathlineto{\pgfqpoint{5.217920in}{2.904450in}}%
\pgfpathlineto{\pgfqpoint{5.208541in}{3.009637in}}%
\pgfpathlineto{\pgfqpoint{5.175520in}{3.034760in}}%
\pgfpathlineto{\pgfqpoint{5.141262in}{2.935116in}}%
\pgfpathclose%
\pgfusepath{fill}%
\end{pgfscope}%
\begin{pgfscope}%
\pgfpathrectangle{\pgfqpoint{1.020000in}{0.880000in}}{\pgfqpoint{6.160000in}{6.160000in}}%
\pgfusepath{clip}%
\pgfsetbuttcap%
\pgfsetroundjoin%
\definecolor{currentfill}{rgb}{0.404421,0.534643,0.932002}%
\pgfsetfillcolor{currentfill}%
\pgfsetlinewidth{0.000000pt}%
\definecolor{currentstroke}{rgb}{0.000000,0.000000,0.000000}%
\pgfsetstrokecolor{currentstroke}%
\pgfsetdash{}{0pt}%
\pgfpathmoveto{\pgfqpoint{4.989109in}{3.131019in}}%
\pgfpathlineto{\pgfqpoint{4.999505in}{3.149384in}}%
\pgfpathlineto{\pgfqpoint{5.009462in}{3.111588in}}%
\pgfpathlineto{\pgfqpoint{5.043204in}{3.150731in}}%
\pgfpathlineto{\pgfqpoint{5.076048in}{3.088881in}}%
\pgfpathlineto{\pgfqpoint{5.065451in}{3.058606in}}%
\pgfpathlineto{\pgfqpoint{5.055738in}{3.126077in}}%
\pgfpathlineto{\pgfqpoint{5.021501in}{3.017703in}}%
\pgfpathlineto{\pgfqpoint{4.989109in}{3.131019in}}%
\pgfpathclose%
\pgfusepath{fill}%
\end{pgfscope}%
\begin{pgfscope}%
\pgfpathrectangle{\pgfqpoint{1.020000in}{0.880000in}}{\pgfqpoint{6.160000in}{6.160000in}}%
\pgfusepath{clip}%
\pgfsetbuttcap%
\pgfsetroundjoin%
\definecolor{currentfill}{rgb}{0.338377,0.452819,0.879317}%
\pgfsetfillcolor{currentfill}%
\pgfsetlinewidth{0.000000pt}%
\definecolor{currentstroke}{rgb}{0.000000,0.000000,0.000000}%
\pgfsetstrokecolor{currentstroke}%
\pgfsetdash{}{0pt}%
\pgfpathmoveto{\pgfqpoint{5.363148in}{3.051208in}}%
\pgfpathlineto{\pgfqpoint{5.373601in}{3.034468in}}%
\pgfpathlineto{\pgfqpoint{5.383167in}{2.942511in}}%
\pgfpathlineto{\pgfqpoint{5.416667in}{2.960346in}}%
\pgfpathlineto{\pgfqpoint{5.451209in}{3.060680in}}%
\pgfpathlineto{\pgfqpoint{5.441410in}{3.135493in}}%
\pgfpathlineto{\pgfqpoint{5.428050in}{2.925070in}}%
\pgfpathlineto{\pgfqpoint{5.392280in}{2.713061in}}%
\pgfpathlineto{\pgfqpoint{5.363148in}{3.051208in}}%
\pgfpathclose%
\pgfusepath{fill}%
\end{pgfscope}%
\begin{pgfscope}%
\pgfpathrectangle{\pgfqpoint{1.020000in}{0.880000in}}{\pgfqpoint{6.160000in}{6.160000in}}%
\pgfusepath{clip}%
\pgfsetbuttcap%
\pgfsetroundjoin%
\definecolor{currentfill}{rgb}{0.323718,0.433158,0.864722}%
\pgfsetfillcolor{currentfill}%
\pgfsetlinewidth{0.000000pt}%
\definecolor{currentstroke}{rgb}{0.000000,0.000000,0.000000}%
\pgfsetstrokecolor{currentstroke}%
\pgfsetdash{}{0pt}%
\pgfpathmoveto{\pgfqpoint{5.582699in}{2.953663in}}%
\pgfpathlineto{\pgfqpoint{5.592607in}{2.884734in}}%
\pgfpathlineto{\pgfqpoint{5.603753in}{2.901988in}}%
\pgfpathlineto{\pgfqpoint{5.637304in}{2.925249in}}%
\pgfpathlineto{\pgfqpoint{5.670998in}{2.958376in}}%
\pgfpathlineto{\pgfqpoint{5.658082in}{2.827446in}}%
\pgfpathlineto{\pgfqpoint{5.649711in}{3.001585in}}%
\pgfpathlineto{\pgfqpoint{5.618013in}{3.103105in}}%
\pgfpathlineto{\pgfqpoint{5.582699in}{2.953663in}}%
\pgfpathclose%
\pgfusepath{fill}%
\end{pgfscope}%
\begin{pgfscope}%
\pgfpathrectangle{\pgfqpoint{1.020000in}{0.880000in}}{\pgfqpoint{6.160000in}{6.160000in}}%
\pgfusepath{clip}%
\pgfsetbuttcap%
\pgfsetroundjoin%
\definecolor{currentfill}{rgb}{0.373552,0.497499,0.909467}%
\pgfsetfillcolor{currentfill}%
\pgfsetlinewidth{0.000000pt}%
\definecolor{currentstroke}{rgb}{0.000000,0.000000,0.000000}%
\pgfsetstrokecolor{currentstroke}%
\pgfsetdash{}{0pt}%
\pgfpathmoveto{\pgfqpoint{5.428050in}{2.925070in}}%
\pgfpathlineto{\pgfqpoint{5.441410in}{3.135493in}}%
\pgfpathlineto{\pgfqpoint{5.451209in}{3.060680in}}%
\pgfpathlineto{\pgfqpoint{5.484886in}{3.091153in}}%
\pgfpathlineto{\pgfqpoint{5.518262in}{3.099143in}}%
\pgfpathlineto{\pgfqpoint{5.507168in}{3.077924in}}%
\pgfpathlineto{\pgfqpoint{5.496310in}{3.072535in}}%
\pgfpathlineto{\pgfqpoint{5.460313in}{2.854608in}}%
\pgfpathlineto{\pgfqpoint{5.428050in}{2.925070in}}%
\pgfpathclose%
\pgfusepath{fill}%
\end{pgfscope}%
\begin{pgfscope}%
\pgfpathrectangle{\pgfqpoint{1.020000in}{0.880000in}}{\pgfqpoint{6.160000in}{6.160000in}}%
\pgfusepath{clip}%
\pgfsetbuttcap%
\pgfsetroundjoin%
\definecolor{currentfill}{rgb}{0.958176,0.771234,0.680301}%
\pgfsetfillcolor{currentfill}%
\pgfsetlinewidth{0.000000pt}%
\definecolor{currentstroke}{rgb}{0.000000,0.000000,0.000000}%
\pgfsetstrokecolor{currentstroke}%
\pgfsetdash{}{0pt}%
\pgfpathmoveto{\pgfqpoint{3.216761in}{4.534939in}}%
\pgfpathlineto{\pgfqpoint{3.228516in}{4.178271in}}%
\pgfpathlineto{\pgfqpoint{3.236292in}{4.240768in}}%
\pgfpathlineto{\pgfqpoint{3.270381in}{4.221028in}}%
\pgfpathlineto{\pgfqpoint{3.303466in}{4.312553in}}%
\pgfpathlineto{\pgfqpoint{3.294743in}{4.344630in}}%
\pgfpathlineto{\pgfqpoint{3.286820in}{4.288011in}}%
\pgfpathlineto{\pgfqpoint{3.252512in}{4.338886in}}%
\pgfpathlineto{\pgfqpoint{3.216761in}{4.534939in}}%
\pgfpathclose%
\pgfusepath{fill}%
\end{pgfscope}%
\begin{pgfscope}%
\pgfpathrectangle{\pgfqpoint{1.020000in}{0.880000in}}{\pgfqpoint{6.160000in}{6.160000in}}%
\pgfusepath{clip}%
\pgfsetbuttcap%
\pgfsetroundjoin%
\definecolor{currentfill}{rgb}{0.831148,0.859513,0.903110}%
\pgfsetfillcolor{currentfill}%
\pgfsetlinewidth{0.000000pt}%
\definecolor{currentstroke}{rgb}{0.000000,0.000000,0.000000}%
\pgfsetstrokecolor{currentstroke}%
\pgfsetdash{}{0pt}%
\pgfpathmoveto{\pgfqpoint{3.543715in}{3.929821in}}%
\pgfpathlineto{\pgfqpoint{3.552360in}{3.938559in}}%
\pgfpathlineto{\pgfqpoint{3.560873in}{3.971908in}}%
\pgfpathlineto{\pgfqpoint{3.595159in}{3.899061in}}%
\pgfpathlineto{\pgfqpoint{3.629761in}{3.756558in}}%
\pgfpathlineto{\pgfqpoint{3.620392in}{3.863254in}}%
\pgfpathlineto{\pgfqpoint{3.611415in}{3.901252in}}%
\pgfpathlineto{\pgfqpoint{3.577032in}{4.006346in}}%
\pgfpathlineto{\pgfqpoint{3.543715in}{3.929821in}}%
\pgfpathclose%
\pgfusepath{fill}%
\end{pgfscope}%
\begin{pgfscope}%
\pgfpathrectangle{\pgfqpoint{1.020000in}{0.880000in}}{\pgfqpoint{6.160000in}{6.160000in}}%
\pgfusepath{clip}%
\pgfsetbuttcap%
\pgfsetroundjoin%
\definecolor{currentfill}{rgb}{0.966922,0.651969,0.531997}%
\pgfsetfillcolor{currentfill}%
\pgfsetlinewidth{0.000000pt}%
\definecolor{currentstroke}{rgb}{0.000000,0.000000,0.000000}%
\pgfsetstrokecolor{currentstroke}%
\pgfsetdash{}{0pt}%
\pgfpathmoveto{\pgfqpoint{2.692249in}{4.545239in}}%
\pgfpathlineto{\pgfqpoint{2.704395in}{4.249047in}}%
\pgfpathlineto{\pgfqpoint{2.709340in}{4.444412in}}%
\pgfpathlineto{\pgfqpoint{2.740952in}{4.616032in}}%
\pgfpathlineto{\pgfqpoint{2.774199in}{4.677048in}}%
\pgfpathlineto{\pgfqpoint{2.766041in}{4.695802in}}%
\pgfpathlineto{\pgfqpoint{2.759276in}{4.616476in}}%
\pgfpathlineto{\pgfqpoint{2.727298in}{4.474136in}}%
\pgfpathlineto{\pgfqpoint{2.692249in}{4.545239in}}%
\pgfpathclose%
\pgfusepath{fill}%
\end{pgfscope}%
\begin{pgfscope}%
\pgfpathrectangle{\pgfqpoint{1.020000in}{0.880000in}}{\pgfqpoint{6.160000in}{6.160000in}}%
\pgfusepath{clip}%
\pgfsetbuttcap%
\pgfsetroundjoin%
\definecolor{currentfill}{rgb}{0.608547,0.735725,0.999354}%
\pgfsetfillcolor{currentfill}%
\pgfsetlinewidth{0.000000pt}%
\definecolor{currentstroke}{rgb}{0.000000,0.000000,0.000000}%
\pgfsetstrokecolor{currentstroke}%
\pgfsetdash{}{0pt}%
\pgfpathmoveto{\pgfqpoint{4.089117in}{3.446511in}}%
\pgfpathlineto{\pgfqpoint{4.098300in}{3.555796in}}%
\pgfpathlineto{\pgfqpoint{4.107578in}{3.593420in}}%
\pgfpathlineto{\pgfqpoint{4.141360in}{3.440174in}}%
\pgfpathlineto{\pgfqpoint{4.175013in}{3.366225in}}%
\pgfpathlineto{\pgfqpoint{4.165641in}{3.462488in}}%
\pgfpathlineto{\pgfqpoint{4.156309in}{3.417244in}}%
\pgfpathlineto{\pgfqpoint{4.122678in}{3.514354in}}%
\pgfpathlineto{\pgfqpoint{4.089117in}{3.446511in}}%
\pgfpathclose%
\pgfusepath{fill}%
\end{pgfscope}%
\begin{pgfscope}%
\pgfpathrectangle{\pgfqpoint{1.020000in}{0.880000in}}{\pgfqpoint{6.160000in}{6.160000in}}%
\pgfusepath{clip}%
\pgfsetbuttcap%
\pgfsetroundjoin%
\definecolor{currentfill}{rgb}{0.968533,0.715841,0.606097}%
\pgfsetfillcolor{currentfill}%
\pgfsetlinewidth{0.000000pt}%
\definecolor{currentstroke}{rgb}{0.000000,0.000000,0.000000}%
\pgfsetstrokecolor{currentstroke}%
\pgfsetdash{}{0pt}%
\pgfpathmoveto{\pgfqpoint{2.558866in}{4.363892in}}%
\pgfpathlineto{\pgfqpoint{2.565747in}{4.412024in}}%
\pgfpathlineto{\pgfqpoint{2.572082in}{4.495929in}}%
\pgfpathlineto{\pgfqpoint{2.608899in}{4.325458in}}%
\pgfpathlineto{\pgfqpoint{2.641290in}{4.436012in}}%
\pgfpathlineto{\pgfqpoint{2.635053in}{4.337918in}}%
\pgfpathlineto{\pgfqpoint{2.625344in}{4.467046in}}%
\pgfpathlineto{\pgfqpoint{2.590069in}{4.544227in}}%
\pgfpathlineto{\pgfqpoint{2.558866in}{4.363892in}}%
\pgfpathclose%
\pgfusepath{fill}%
\end{pgfscope}%
\begin{pgfscope}%
\pgfpathrectangle{\pgfqpoint{1.020000in}{0.880000in}}{\pgfqpoint{6.160000in}{6.160000in}}%
\pgfusepath{clip}%
\pgfsetbuttcap%
\pgfsetroundjoin%
\definecolor{currentfill}{rgb}{0.763363,0.835092,0.955658}%
\pgfsetfillcolor{currentfill}%
\pgfsetlinewidth{0.000000pt}%
\definecolor{currentstroke}{rgb}{0.000000,0.000000,0.000000}%
\pgfsetstrokecolor{currentstroke}%
\pgfsetdash{}{0pt}%
\pgfpathmoveto{\pgfqpoint{3.629761in}{3.756558in}}%
\pgfpathlineto{\pgfqpoint{3.638250in}{3.811777in}}%
\pgfpathlineto{\pgfqpoint{3.647755in}{3.682034in}}%
\pgfpathlineto{\pgfqpoint{3.681199in}{3.751033in}}%
\pgfpathlineto{\pgfqpoint{3.715169in}{3.712956in}}%
\pgfpathlineto{\pgfqpoint{3.706667in}{3.635447in}}%
\pgfpathlineto{\pgfqpoint{3.696131in}{3.984473in}}%
\pgfpathlineto{\pgfqpoint{3.663324in}{3.791973in}}%
\pgfpathlineto{\pgfqpoint{3.629761in}{3.756558in}}%
\pgfpathclose%
\pgfusepath{fill}%
\end{pgfscope}%
\begin{pgfscope}%
\pgfpathrectangle{\pgfqpoint{1.020000in}{0.880000in}}{\pgfqpoint{6.160000in}{6.160000in}}%
\pgfusepath{clip}%
\pgfsetbuttcap%
\pgfsetroundjoin%
\definecolor{currentfill}{rgb}{0.718985,0.811993,0.977656}%
\pgfsetfillcolor{currentfill}%
\pgfsetlinewidth{0.000000pt}%
\definecolor{currentstroke}{rgb}{0.000000,0.000000,0.000000}%
\pgfsetstrokecolor{currentstroke}%
\pgfsetdash{}{0pt}%
\pgfpathmoveto{\pgfqpoint{3.782816in}{3.675742in}}%
\pgfpathlineto{\pgfqpoint{3.791314in}{3.792404in}}%
\pgfpathlineto{\pgfqpoint{3.801287in}{3.532690in}}%
\pgfpathlineto{\pgfqpoint{3.834257in}{3.749962in}}%
\pgfpathlineto{\pgfqpoint{3.868232in}{3.687176in}}%
\pgfpathlineto{\pgfqpoint{3.858818in}{3.802839in}}%
\pgfpathlineto{\pgfqpoint{3.850482in}{3.593671in}}%
\pgfpathlineto{\pgfqpoint{3.817022in}{3.539317in}}%
\pgfpathlineto{\pgfqpoint{3.782816in}{3.675742in}}%
\pgfpathclose%
\pgfusepath{fill}%
\end{pgfscope}%
\begin{pgfscope}%
\pgfpathrectangle{\pgfqpoint{1.020000in}{0.880000in}}{\pgfqpoint{6.160000in}{6.160000in}}%
\pgfusepath{clip}%
\pgfsetbuttcap%
\pgfsetroundjoin%
\definecolor{currentfill}{rgb}{0.318832,0.426605,0.859857}%
\pgfsetfillcolor{currentfill}%
\pgfsetlinewidth{0.000000pt}%
\definecolor{currentstroke}{rgb}{0.000000,0.000000,0.000000}%
\pgfsetstrokecolor{currentstroke}%
\pgfsetdash{}{0pt}%
\pgfpathmoveto{\pgfqpoint{5.799738in}{2.718856in}}%
\pgfpathlineto{\pgfqpoint{5.815450in}{3.000255in}}%
\pgfpathlineto{\pgfqpoint{5.826140in}{2.972039in}}%
\pgfpathlineto{\pgfqpoint{5.857017in}{2.835311in}}%
\pgfpathlineto{\pgfqpoint{5.893201in}{3.014148in}}%
\pgfpathlineto{\pgfqpoint{5.879445in}{2.865553in}}%
\pgfpathlineto{\pgfqpoint{5.872192in}{3.097294in}}%
\pgfpathlineto{\pgfqpoint{5.836893in}{2.967164in}}%
\pgfpathlineto{\pgfqpoint{5.799738in}{2.718856in}}%
\pgfpathclose%
\pgfusepath{fill}%
\end{pgfscope}%
\begin{pgfscope}%
\pgfpathrectangle{\pgfqpoint{1.020000in}{0.880000in}}{\pgfqpoint{6.160000in}{6.160000in}}%
\pgfusepath{clip}%
\pgfsetbuttcap%
\pgfsetroundjoin%
\definecolor{currentfill}{rgb}{0.934305,0.525918,0.412286}%
\pgfsetfillcolor{currentfill}%
\pgfsetlinewidth{0.000000pt}%
\definecolor{currentstroke}{rgb}{0.000000,0.000000,0.000000}%
\pgfsetstrokecolor{currentstroke}%
\pgfsetdash{}{0pt}%
\pgfpathmoveto{\pgfqpoint{2.891678in}{4.892527in}}%
\pgfpathlineto{\pgfqpoint{2.902081in}{4.704990in}}%
\pgfpathlineto{\pgfqpoint{2.911402in}{4.602528in}}%
\pgfpathlineto{\pgfqpoint{2.943941in}{4.721229in}}%
\pgfpathlineto{\pgfqpoint{2.979436in}{4.596483in}}%
\pgfpathlineto{\pgfqpoint{2.968973in}{4.794518in}}%
\pgfpathlineto{\pgfqpoint{2.961647in}{4.731114in}}%
\pgfpathlineto{\pgfqpoint{2.926577in}{4.821705in}}%
\pgfpathlineto{\pgfqpoint{2.891678in}{4.892527in}}%
\pgfpathclose%
\pgfusepath{fill}%
\end{pgfscope}%
\begin{pgfscope}%
\pgfpathrectangle{\pgfqpoint{1.020000in}{0.880000in}}{\pgfqpoint{6.160000in}{6.160000in}}%
\pgfusepath{clip}%
\pgfsetbuttcap%
\pgfsetroundjoin%
\definecolor{currentfill}{rgb}{0.966962,0.735670,0.630877}%
\pgfsetfillcolor{currentfill}%
\pgfsetlinewidth{0.000000pt}%
\definecolor{currentstroke}{rgb}{0.000000,0.000000,0.000000}%
\pgfsetstrokecolor{currentstroke}%
\pgfsetdash{}{0pt}%
\pgfpathmoveto{\pgfqpoint{2.492985in}{4.228632in}}%
\pgfpathlineto{\pgfqpoint{2.497323in}{4.422521in}}%
\pgfpathlineto{\pgfqpoint{2.508415in}{4.212740in}}%
\pgfpathlineto{\pgfqpoint{2.538946in}{4.432049in}}%
\pgfpathlineto{\pgfqpoint{2.572082in}{4.495929in}}%
\pgfpathlineto{\pgfqpoint{2.565747in}{4.412024in}}%
\pgfpathlineto{\pgfqpoint{2.558866in}{4.363892in}}%
\pgfpathlineto{\pgfqpoint{2.523109in}{4.466867in}}%
\pgfpathlineto{\pgfqpoint{2.492985in}{4.228632in}}%
\pgfpathclose%
\pgfusepath{fill}%
\end{pgfscope}%
\begin{pgfscope}%
\pgfpathrectangle{\pgfqpoint{1.020000in}{0.880000in}}{\pgfqpoint{6.160000in}{6.160000in}}%
\pgfusepath{clip}%
\pgfsetbuttcap%
\pgfsetroundjoin%
\definecolor{currentfill}{rgb}{0.399231,0.528528,0.928459}%
\pgfsetfillcolor{currentfill}%
\pgfsetlinewidth{0.000000pt}%
\definecolor{currentstroke}{rgb}{0.000000,0.000000,0.000000}%
\pgfsetstrokecolor{currentstroke}%
\pgfsetdash{}{0pt}%
\pgfpathmoveto{\pgfqpoint{4.921562in}{3.023732in}}%
\pgfpathlineto{\pgfqpoint{4.932123in}{3.075796in}}%
\pgfpathlineto{\pgfqpoint{4.941733in}{2.998759in}}%
\pgfpathlineto{\pgfqpoint{4.975430in}{3.035262in}}%
\pgfpathlineto{\pgfqpoint{5.009462in}{3.111588in}}%
\pgfpathlineto{\pgfqpoint{4.999505in}{3.149384in}}%
\pgfpathlineto{\pgfqpoint{4.989109in}{3.131019in}}%
\pgfpathlineto{\pgfqpoint{4.956087in}{3.175395in}}%
\pgfpathlineto{\pgfqpoint{4.921562in}{3.023732in}}%
\pgfpathclose%
\pgfusepath{fill}%
\end{pgfscope}%
\begin{pgfscope}%
\pgfpathrectangle{\pgfqpoint{1.020000in}{0.880000in}}{\pgfqpoint{6.160000in}{6.160000in}}%
\pgfusepath{clip}%
\pgfsetbuttcap%
\pgfsetroundjoin%
\definecolor{currentfill}{rgb}{0.935774,0.812237,0.747156}%
\pgfsetfillcolor{currentfill}%
\pgfsetlinewidth{0.000000pt}%
\definecolor{currentstroke}{rgb}{0.000000,0.000000,0.000000}%
\pgfsetstrokecolor{currentstroke}%
\pgfsetdash{}{0pt}%
\pgfpathmoveto{\pgfqpoint{3.303466in}{4.312553in}}%
\pgfpathlineto{\pgfqpoint{3.313824in}{4.092638in}}%
\pgfpathlineto{\pgfqpoint{3.322580in}{4.057829in}}%
\pgfpathlineto{\pgfqpoint{3.355147in}{4.217190in}}%
\pgfpathlineto{\pgfqpoint{3.390073in}{4.088102in}}%
\pgfpathlineto{\pgfqpoint{3.380214in}{4.255645in}}%
\pgfpathlineto{\pgfqpoint{3.372116in}{4.203039in}}%
\pgfpathlineto{\pgfqpoint{3.337404in}{4.307324in}}%
\pgfpathlineto{\pgfqpoint{3.303466in}{4.312553in}}%
\pgfpathclose%
\pgfusepath{fill}%
\end{pgfscope}%
\begin{pgfscope}%
\pgfpathrectangle{\pgfqpoint{1.020000in}{0.880000in}}{\pgfqpoint{6.160000in}{6.160000in}}%
\pgfusepath{clip}%
\pgfsetbuttcap%
\pgfsetroundjoin%
\definecolor{currentfill}{rgb}{0.521696,0.659599,0.987736}%
\pgfsetfillcolor{currentfill}%
\pgfsetlinewidth{0.000000pt}%
\definecolor{currentstroke}{rgb}{0.000000,0.000000,0.000000}%
\pgfsetstrokecolor{currentstroke}%
\pgfsetdash{}{0pt}%
\pgfpathmoveto{\pgfqpoint{4.395471in}{3.358438in}}%
\pgfpathlineto{\pgfqpoint{4.405030in}{3.333810in}}%
\pgfpathlineto{\pgfqpoint{4.414747in}{3.370327in}}%
\pgfpathlineto{\pgfqpoint{4.448192in}{3.306241in}}%
\pgfpathlineto{\pgfqpoint{4.481518in}{3.223496in}}%
\pgfpathlineto{\pgfqpoint{4.472258in}{3.371558in}}%
\pgfpathlineto{\pgfqpoint{4.462381in}{3.301260in}}%
\pgfpathlineto{\pgfqpoint{4.428788in}{3.265867in}}%
\pgfpathlineto{\pgfqpoint{4.395471in}{3.358438in}}%
\pgfpathclose%
\pgfusepath{fill}%
\end{pgfscope}%
\begin{pgfscope}%
\pgfpathrectangle{\pgfqpoint{1.020000in}{0.880000in}}{\pgfqpoint{6.160000in}{6.160000in}}%
\pgfusepath{clip}%
\pgfsetbuttcap%
\pgfsetroundjoin%
\definecolor{currentfill}{rgb}{0.949454,0.572388,0.453443}%
\pgfsetfillcolor{currentfill}%
\pgfsetlinewidth{0.000000pt}%
\definecolor{currentstroke}{rgb}{0.000000,0.000000,0.000000}%
\pgfsetstrokecolor{currentstroke}%
\pgfsetdash{}{0pt}%
\pgfpathmoveto{\pgfqpoint{3.046747in}{4.647019in}}%
\pgfpathlineto{\pgfqpoint{3.055147in}{4.627289in}}%
\pgfpathlineto{\pgfqpoint{3.063669in}{4.597773in}}%
\pgfpathlineto{\pgfqpoint{3.095816in}{4.768200in}}%
\pgfpathlineto{\pgfqpoint{3.133136in}{4.450118in}}%
\pgfpathlineto{\pgfqpoint{3.121086in}{4.813066in}}%
\pgfpathlineto{\pgfqpoint{3.113975in}{4.704242in}}%
\pgfpathlineto{\pgfqpoint{3.080390in}{4.672978in}}%
\pgfpathlineto{\pgfqpoint{3.046747in}{4.647019in}}%
\pgfpathclose%
\pgfusepath{fill}%
\end{pgfscope}%
\begin{pgfscope}%
\pgfpathrectangle{\pgfqpoint{1.020000in}{0.880000in}}{\pgfqpoint{6.160000in}{6.160000in}}%
\pgfusepath{clip}%
\pgfsetbuttcap%
\pgfsetroundjoin%
\definecolor{currentfill}{rgb}{0.373552,0.497499,0.909467}%
\pgfsetfillcolor{currentfill}%
\pgfsetlinewidth{0.000000pt}%
\definecolor{currentstroke}{rgb}{0.000000,0.000000,0.000000}%
\pgfsetstrokecolor{currentstroke}%
\pgfsetdash{}{0pt}%
\pgfpathmoveto{\pgfqpoint{4.853680in}{2.857242in}}%
\pgfpathlineto{\pgfqpoint{4.864536in}{2.969537in}}%
\pgfpathlineto{\pgfqpoint{4.875704in}{3.120388in}}%
\pgfpathlineto{\pgfqpoint{4.909475in}{3.159346in}}%
\pgfpathlineto{\pgfqpoint{4.941733in}{2.998759in}}%
\pgfpathlineto{\pgfqpoint{4.932123in}{3.075796in}}%
\pgfpathlineto{\pgfqpoint{4.921562in}{3.023732in}}%
\pgfpathlineto{\pgfqpoint{4.888645in}{3.087648in}}%
\pgfpathlineto{\pgfqpoint{4.853680in}{2.857242in}}%
\pgfpathclose%
\pgfusepath{fill}%
\end{pgfscope}%
\begin{pgfscope}%
\pgfpathrectangle{\pgfqpoint{1.020000in}{0.880000in}}{\pgfqpoint{6.160000in}{6.160000in}}%
\pgfusepath{clip}%
\pgfsetbuttcap%
\pgfsetroundjoin%
\definecolor{currentfill}{rgb}{0.473070,0.611077,0.970634}%
\pgfsetfillcolor{currentfill}%
\pgfsetlinewidth{0.000000pt}%
\definecolor{currentstroke}{rgb}{0.000000,0.000000,0.000000}%
\pgfsetstrokecolor{currentstroke}%
\pgfsetdash{}{0pt}%
\pgfpathmoveto{\pgfqpoint{4.548331in}{3.164196in}}%
\pgfpathlineto{\pgfqpoint{4.558453in}{3.258162in}}%
\pgfpathlineto{\pgfqpoint{4.568231in}{3.252474in}}%
\pgfpathlineto{\pgfqpoint{4.601241in}{3.121151in}}%
\pgfpathlineto{\pgfqpoint{4.635211in}{3.225584in}}%
\pgfpathlineto{\pgfqpoint{4.624907in}{3.125914in}}%
\pgfpathlineto{\pgfqpoint{4.615958in}{3.329603in}}%
\pgfpathlineto{\pgfqpoint{4.582496in}{3.339274in}}%
\pgfpathlineto{\pgfqpoint{4.548331in}{3.164196in}}%
\pgfpathclose%
\pgfusepath{fill}%
\end{pgfscope}%
\begin{pgfscope}%
\pgfpathrectangle{\pgfqpoint{1.020000in}{0.880000in}}{\pgfqpoint{6.160000in}{6.160000in}}%
\pgfusepath{clip}%
\pgfsetbuttcap%
\pgfsetroundjoin%
\definecolor{currentfill}{rgb}{0.967711,0.662973,0.544323}%
\pgfsetfillcolor{currentfill}%
\pgfsetlinewidth{0.000000pt}%
\definecolor{currentstroke}{rgb}{0.000000,0.000000,0.000000}%
\pgfsetstrokecolor{currentstroke}%
\pgfsetdash{}{0pt}%
\pgfpathmoveto{\pgfqpoint{3.133136in}{4.450118in}}%
\pgfpathlineto{\pgfqpoint{3.140567in}{4.531042in}}%
\pgfpathlineto{\pgfqpoint{3.148970in}{4.519921in}}%
\pgfpathlineto{\pgfqpoint{3.182545in}{4.560627in}}%
\pgfpathlineto{\pgfqpoint{3.216761in}{4.534939in}}%
\pgfpathlineto{\pgfqpoint{3.208294in}{4.546366in}}%
\pgfpathlineto{\pgfqpoint{3.201530in}{4.384328in}}%
\pgfpathlineto{\pgfqpoint{3.165991in}{4.552703in}}%
\pgfpathlineto{\pgfqpoint{3.133136in}{4.450118in}}%
\pgfpathclose%
\pgfusepath{fill}%
\end{pgfscope}%
\begin{pgfscope}%
\pgfpathrectangle{\pgfqpoint{1.020000in}{0.880000in}}{\pgfqpoint{6.160000in}{6.160000in}}%
\pgfusepath{clip}%
\pgfsetbuttcap%
\pgfsetroundjoin%
\definecolor{currentfill}{rgb}{0.656683,0.771806,0.994914}%
\pgfsetfillcolor{currentfill}%
\pgfsetlinewidth{0.000000pt}%
\definecolor{currentstroke}{rgb}{0.000000,0.000000,0.000000}%
\pgfsetstrokecolor{currentstroke}%
\pgfsetdash{}{0pt}%
\pgfpathmoveto{\pgfqpoint{3.868232in}{3.687176in}}%
\pgfpathlineto{\pgfqpoint{3.877339in}{3.671533in}}%
\pgfpathlineto{\pgfqpoint{3.886705in}{3.574201in}}%
\pgfpathlineto{\pgfqpoint{3.920831in}{3.436909in}}%
\pgfpathlineto{\pgfqpoint{3.954358in}{3.512202in}}%
\pgfpathlineto{\pgfqpoint{3.945277in}{3.491494in}}%
\pgfpathlineto{\pgfqpoint{3.936023in}{3.550675in}}%
\pgfpathlineto{\pgfqpoint{3.902497in}{3.498815in}}%
\pgfpathlineto{\pgfqpoint{3.868232in}{3.687176in}}%
\pgfpathclose%
\pgfusepath{fill}%
\end{pgfscope}%
\begin{pgfscope}%
\pgfpathrectangle{\pgfqpoint{1.020000in}{0.880000in}}{\pgfqpoint{6.160000in}{6.160000in}}%
\pgfusepath{clip}%
\pgfsetbuttcap%
\pgfsetroundjoin%
\definecolor{currentfill}{rgb}{0.430507,0.564883,0.948889}%
\pgfsetfillcolor{currentfill}%
\pgfsetlinewidth{0.000000pt}%
\definecolor{currentstroke}{rgb}{0.000000,0.000000,0.000000}%
\pgfsetstrokecolor{currentstroke}%
\pgfsetdash{}{0pt}%
\pgfpathmoveto{\pgfqpoint{4.701654in}{3.117835in}}%
\pgfpathlineto{\pgfqpoint{4.711247in}{3.052665in}}%
\pgfpathlineto{\pgfqpoint{4.721416in}{3.092572in}}%
\pgfpathlineto{\pgfqpoint{4.754882in}{3.087867in}}%
\pgfpathlineto{\pgfqpoint{4.788893in}{3.176817in}}%
\pgfpathlineto{\pgfqpoint{4.778728in}{3.154409in}}%
\pgfpathlineto{\pgfqpoint{4.768897in}{3.184677in}}%
\pgfpathlineto{\pgfqpoint{4.735902in}{3.264561in}}%
\pgfpathlineto{\pgfqpoint{4.701654in}{3.117835in}}%
\pgfpathclose%
\pgfusepath{fill}%
\end{pgfscope}%
\begin{pgfscope}%
\pgfpathrectangle{\pgfqpoint{1.020000in}{0.880000in}}{\pgfqpoint{6.160000in}{6.160000in}}%
\pgfusepath{clip}%
\pgfsetbuttcap%
\pgfsetroundjoin%
\definecolor{currentfill}{rgb}{0.891817,0.851973,0.829085}%
\pgfsetfillcolor{currentfill}%
\pgfsetlinewidth{0.000000pt}%
\definecolor{currentstroke}{rgb}{0.000000,0.000000,0.000000}%
\pgfsetstrokecolor{currentstroke}%
\pgfsetdash{}{0pt}%
\pgfpathmoveto{\pgfqpoint{3.390073in}{4.088102in}}%
\pgfpathlineto{\pgfqpoint{3.398995in}{4.039512in}}%
\pgfpathlineto{\pgfqpoint{3.407626in}{4.029976in}}%
\pgfpathlineto{\pgfqpoint{3.441739in}{4.000138in}}%
\pgfpathlineto{\pgfqpoint{3.475541in}{4.008695in}}%
\pgfpathlineto{\pgfqpoint{3.466825in}{4.022073in}}%
\pgfpathlineto{\pgfqpoint{3.457177in}{4.166839in}}%
\pgfpathlineto{\pgfqpoint{3.423993in}{4.078286in}}%
\pgfpathlineto{\pgfqpoint{3.390073in}{4.088102in}}%
\pgfpathclose%
\pgfusepath{fill}%
\end{pgfscope}%
\begin{pgfscope}%
\pgfpathrectangle{\pgfqpoint{1.020000in}{0.880000in}}{\pgfqpoint{6.160000in}{6.160000in}}%
\pgfusepath{clip}%
\pgfsetbuttcap%
\pgfsetroundjoin%
\definecolor{currentfill}{rgb}{0.708720,0.805721,0.981117}%
\pgfsetfillcolor{currentfill}%
\pgfsetlinewidth{0.000000pt}%
\definecolor{currentstroke}{rgb}{0.000000,0.000000,0.000000}%
\pgfsetstrokecolor{currentstroke}%
\pgfsetdash{}{0pt}%
\pgfpathmoveto{\pgfqpoint{3.715169in}{3.712956in}}%
\pgfpathlineto{\pgfqpoint{3.725066in}{3.492865in}}%
\pgfpathlineto{\pgfqpoint{3.733792in}{3.529284in}}%
\pgfpathlineto{\pgfqpoint{3.766364in}{3.816372in}}%
\pgfpathlineto{\pgfqpoint{3.801287in}{3.532690in}}%
\pgfpathlineto{\pgfqpoint{3.791314in}{3.792404in}}%
\pgfpathlineto{\pgfqpoint{3.782816in}{3.675742in}}%
\pgfpathlineto{\pgfqpoint{3.749059in}{3.682879in}}%
\pgfpathlineto{\pgfqpoint{3.715169in}{3.712956in}}%
\pgfpathclose%
\pgfusepath{fill}%
\end{pgfscope}%
\begin{pgfscope}%
\pgfpathrectangle{\pgfqpoint{1.020000in}{0.880000in}}{\pgfqpoint{6.160000in}{6.160000in}}%
\pgfusepath{clip}%
\pgfsetbuttcap%
\pgfsetroundjoin%
\definecolor{currentfill}{rgb}{0.936780,0.532750,0.418093}%
\pgfsetfillcolor{currentfill}%
\pgfsetlinewidth{0.000000pt}%
\definecolor{currentstroke}{rgb}{0.000000,0.000000,0.000000}%
\pgfsetstrokecolor{currentstroke}%
\pgfsetdash{}{0pt}%
\pgfpathmoveto{\pgfqpoint{2.827170in}{4.624267in}}%
\pgfpathlineto{\pgfqpoint{2.833257in}{4.763484in}}%
\pgfpathlineto{\pgfqpoint{2.841701in}{4.727764in}}%
\pgfpathlineto{\pgfqpoint{2.876056in}{4.705964in}}%
\pgfpathlineto{\pgfqpoint{2.911402in}{4.602528in}}%
\pgfpathlineto{\pgfqpoint{2.902081in}{4.704990in}}%
\pgfpathlineto{\pgfqpoint{2.891678in}{4.892527in}}%
\pgfpathlineto{\pgfqpoint{2.859912in}{4.718357in}}%
\pgfpathlineto{\pgfqpoint{2.827170in}{4.624267in}}%
\pgfpathclose%
\pgfusepath{fill}%
\end{pgfscope}%
\begin{pgfscope}%
\pgfpathrectangle{\pgfqpoint{1.020000in}{0.880000in}}{\pgfqpoint{6.160000in}{6.160000in}}%
\pgfusepath{clip}%
\pgfsetbuttcap%
\pgfsetroundjoin%
\definecolor{currentfill}{rgb}{0.313946,0.420052,0.854993}%
\pgfsetfillcolor{currentfill}%
\pgfsetlinewidth{0.000000pt}%
\definecolor{currentstroke}{rgb}{0.000000,0.000000,0.000000}%
\pgfsetstrokecolor{currentstroke}%
\pgfsetdash{}{0pt}%
\pgfpathmoveto{\pgfqpoint{5.738093in}{3.005506in}}%
\pgfpathlineto{\pgfqpoint{5.747016in}{2.869141in}}%
\pgfpathlineto{\pgfqpoint{5.759491in}{2.957504in}}%
\pgfpathlineto{\pgfqpoint{5.793221in}{2.989409in}}%
\pgfpathlineto{\pgfqpoint{5.826140in}{2.972039in}}%
\pgfpathlineto{\pgfqpoint{5.815450in}{3.000255in}}%
\pgfpathlineto{\pgfqpoint{5.799738in}{2.718856in}}%
\pgfpathlineto{\pgfqpoint{5.769008in}{2.864357in}}%
\pgfpathlineto{\pgfqpoint{5.738093in}{3.005506in}}%
\pgfpathclose%
\pgfusepath{fill}%
\end{pgfscope}%
\begin{pgfscope}%
\pgfpathrectangle{\pgfqpoint{1.020000in}{0.880000in}}{\pgfqpoint{6.160000in}{6.160000in}}%
\pgfusepath{clip}%
\pgfsetbuttcap%
\pgfsetroundjoin%
\definecolor{currentfill}{rgb}{0.930669,0.818877,0.759146}%
\pgfsetfillcolor{currentfill}%
\pgfsetlinewidth{0.000000pt}%
\definecolor{currentstroke}{rgb}{0.000000,0.000000,0.000000}%
\pgfsetstrokecolor{currentstroke}%
\pgfsetdash{}{0pt}%
\pgfpathmoveto{\pgfqpoint{3.236292in}{4.240768in}}%
\pgfpathlineto{\pgfqpoint{3.244744in}{4.233630in}}%
\pgfpathlineto{\pgfqpoint{3.254946in}{4.038657in}}%
\pgfpathlineto{\pgfqpoint{3.288113in}{4.122102in}}%
\pgfpathlineto{\pgfqpoint{3.322580in}{4.057829in}}%
\pgfpathlineto{\pgfqpoint{3.313824in}{4.092638in}}%
\pgfpathlineto{\pgfqpoint{3.303466in}{4.312553in}}%
\pgfpathlineto{\pgfqpoint{3.270381in}{4.221028in}}%
\pgfpathlineto{\pgfqpoint{3.236292in}{4.240768in}}%
\pgfpathclose%
\pgfusepath{fill}%
\end{pgfscope}%
\begin{pgfscope}%
\pgfpathrectangle{\pgfqpoint{1.020000in}{0.880000in}}{\pgfqpoint{6.160000in}{6.160000in}}%
\pgfusepath{clip}%
\pgfsetbuttcap%
\pgfsetroundjoin%
\definecolor{currentfill}{rgb}{0.373552,0.497499,0.909467}%
\pgfsetfillcolor{currentfill}%
\pgfsetlinewidth{0.000000pt}%
\definecolor{currentstroke}{rgb}{0.000000,0.000000,0.000000}%
\pgfsetstrokecolor{currentstroke}%
\pgfsetdash{}{0pt}%
\pgfpathmoveto{\pgfqpoint{5.076048in}{3.088881in}}%
\pgfpathlineto{\pgfqpoint{5.085754in}{3.017729in}}%
\pgfpathlineto{\pgfqpoint{5.096799in}{3.091714in}}%
\pgfpathlineto{\pgfqpoint{5.129121in}{2.975799in}}%
\pgfpathlineto{\pgfqpoint{5.162385in}{2.965567in}}%
\pgfpathlineto{\pgfqpoint{5.152633in}{3.034887in}}%
\pgfpathlineto{\pgfqpoint{5.141262in}{2.935116in}}%
\pgfpathlineto{\pgfqpoint{5.109947in}{3.147195in}}%
\pgfpathlineto{\pgfqpoint{5.076048in}{3.088881in}}%
\pgfpathclose%
\pgfusepath{fill}%
\end{pgfscope}%
\begin{pgfscope}%
\pgfpathrectangle{\pgfqpoint{1.020000in}{0.880000in}}{\pgfqpoint{6.160000in}{6.160000in}}%
\pgfusepath{clip}%
\pgfsetbuttcap%
\pgfsetroundjoin%
\definecolor{currentfill}{rgb}{0.962701,0.628218,0.507636}%
\pgfsetfillcolor{currentfill}%
\pgfsetlinewidth{0.000000pt}%
\definecolor{currentstroke}{rgb}{0.000000,0.000000,0.000000}%
\pgfsetstrokecolor{currentstroke}%
\pgfsetdash{}{0pt}%
\pgfpathmoveto{\pgfqpoint{2.911402in}{4.602528in}}%
\pgfpathlineto{\pgfqpoint{2.921239in}{4.458834in}}%
\pgfpathlineto{\pgfqpoint{2.929440in}{4.446454in}}%
\pgfpathlineto{\pgfqpoint{2.963389in}{4.451173in}}%
\pgfpathlineto{\pgfqpoint{2.995591in}{4.603245in}}%
\pgfpathlineto{\pgfqpoint{2.986294in}{4.702337in}}%
\pgfpathlineto{\pgfqpoint{2.979436in}{4.596483in}}%
\pgfpathlineto{\pgfqpoint{2.943941in}{4.721229in}}%
\pgfpathlineto{\pgfqpoint{2.911402in}{4.602528in}}%
\pgfpathclose%
\pgfusepath{fill}%
\end{pgfscope}%
\begin{pgfscope}%
\pgfpathrectangle{\pgfqpoint{1.020000in}{0.880000in}}{\pgfqpoint{6.160000in}{6.160000in}}%
\pgfusepath{clip}%
\pgfsetbuttcap%
\pgfsetroundjoin%
\definecolor{currentfill}{rgb}{0.945854,0.559565,0.441513}%
\pgfsetfillcolor{currentfill}%
\pgfsetlinewidth{0.000000pt}%
\definecolor{currentstroke}{rgb}{0.000000,0.000000,0.000000}%
\pgfsetstrokecolor{currentstroke}%
\pgfsetdash{}{0pt}%
\pgfpathmoveto{\pgfqpoint{2.759276in}{4.616476in}}%
\pgfpathlineto{\pgfqpoint{2.766041in}{4.695802in}}%
\pgfpathlineto{\pgfqpoint{2.774199in}{4.677048in}}%
\pgfpathlineto{\pgfqpoint{2.810146in}{4.540770in}}%
\pgfpathlineto{\pgfqpoint{2.841701in}{4.727764in}}%
\pgfpathlineto{\pgfqpoint{2.833257in}{4.763484in}}%
\pgfpathlineto{\pgfqpoint{2.827170in}{4.624267in}}%
\pgfpathlineto{\pgfqpoint{2.791864in}{4.719917in}}%
\pgfpathlineto{\pgfqpoint{2.759276in}{4.616476in}}%
\pgfpathclose%
\pgfusepath{fill}%
\end{pgfscope}%
\begin{pgfscope}%
\pgfpathrectangle{\pgfqpoint{1.020000in}{0.880000in}}{\pgfqpoint{6.160000in}{6.160000in}}%
\pgfusepath{clip}%
\pgfsetbuttcap%
\pgfsetroundjoin%
\definecolor{currentfill}{rgb}{0.328604,0.439712,0.869587}%
\pgfsetfillcolor{currentfill}%
\pgfsetlinewidth{0.000000pt}%
\definecolor{currentstroke}{rgb}{0.000000,0.000000,0.000000}%
\pgfsetstrokecolor{currentstroke}%
\pgfsetdash{}{0pt}%
\pgfpathmoveto{\pgfqpoint{6.024129in}{2.929453in}}%
\pgfpathlineto{\pgfqpoint{6.036857in}{3.002688in}}%
\pgfpathlineto{\pgfqpoint{6.046618in}{2.915262in}}%
\pgfpathlineto{\pgfqpoint{6.079620in}{2.909841in}}%
\pgfpathlineto{\pgfqpoint{6.070599in}{3.036444in}}%
\pgfpathlineto{\pgfqpoint{6.056244in}{2.880022in}}%
\pgfpathlineto{\pgfqpoint{6.024129in}{2.929453in}}%
\pgfpathclose%
\pgfusepath{fill}%
\end{pgfscope}%
\begin{pgfscope}%
\pgfpathrectangle{\pgfqpoint{1.020000in}{0.880000in}}{\pgfqpoint{6.160000in}{6.160000in}}%
\pgfusepath{clip}%
\pgfsetbuttcap%
\pgfsetroundjoin%
\definecolor{currentfill}{rgb}{0.930669,0.818877,0.759146}%
\pgfsetfillcolor{currentfill}%
\pgfsetlinewidth{0.000000pt}%
\definecolor{currentstroke}{rgb}{0.000000,0.000000,0.000000}%
\pgfsetstrokecolor{currentstroke}%
\pgfsetdash{}{0pt}%
\pgfpathmoveto{\pgfqpoint{2.438073in}{4.332904in}}%
\pgfpathlineto{\pgfqpoint{2.449788in}{4.090852in}}%
\pgfpathlineto{\pgfqpoint{2.458012in}{4.051250in}}%
\pgfpathlineto{\pgfqpoint{2.492680in}{4.020714in}}%
\pgfpathlineto{\pgfqpoint{2.523945in}{4.194375in}}%
\pgfpathlineto{\pgfqpoint{2.516111in}{4.207280in}}%
\pgfpathlineto{\pgfqpoint{2.508415in}{4.212740in}}%
\pgfpathlineto{\pgfqpoint{2.474198in}{4.218361in}}%
\pgfpathlineto{\pgfqpoint{2.438073in}{4.332904in}}%
\pgfpathclose%
\pgfusepath{fill}%
\end{pgfscope}%
\begin{pgfscope}%
\pgfpathrectangle{\pgfqpoint{1.020000in}{0.880000in}}{\pgfqpoint{6.160000in}{6.160000in}}%
\pgfusepath{clip}%
\pgfsetbuttcap%
\pgfsetroundjoin%
\definecolor{currentfill}{rgb}{0.843358,0.861820,0.890017}%
\pgfsetfillcolor{currentfill}%
\pgfsetlinewidth{0.000000pt}%
\definecolor{currentstroke}{rgb}{0.000000,0.000000,0.000000}%
\pgfsetstrokecolor{currentstroke}%
\pgfsetdash{}{0pt}%
\pgfpathmoveto{\pgfqpoint{3.475541in}{4.008695in}}%
\pgfpathlineto{\pgfqpoint{3.485107in}{3.875545in}}%
\pgfpathlineto{\pgfqpoint{3.494223in}{3.807214in}}%
\pgfpathlineto{\pgfqpoint{3.527284in}{3.927187in}}%
\pgfpathlineto{\pgfqpoint{3.560873in}{3.971908in}}%
\pgfpathlineto{\pgfqpoint{3.552360in}{3.938559in}}%
\pgfpathlineto{\pgfqpoint{3.543715in}{3.929821in}}%
\pgfpathlineto{\pgfqpoint{3.509319in}{4.018892in}}%
\pgfpathlineto{\pgfqpoint{3.475541in}{4.008695in}}%
\pgfpathclose%
\pgfusepath{fill}%
\end{pgfscope}%
\begin{pgfscope}%
\pgfpathrectangle{\pgfqpoint{1.020000in}{0.880000in}}{\pgfqpoint{6.160000in}{6.160000in}}%
\pgfusepath{clip}%
\pgfsetbuttcap%
\pgfsetroundjoin%
\definecolor{currentfill}{rgb}{0.968533,0.715841,0.606097}%
\pgfsetfillcolor{currentfill}%
\pgfsetlinewidth{0.000000pt}%
\definecolor{currentstroke}{rgb}{0.000000,0.000000,0.000000}%
\pgfsetstrokecolor{currentstroke}%
\pgfsetdash{}{0pt}%
\pgfpathmoveto{\pgfqpoint{2.709340in}{4.444412in}}%
\pgfpathlineto{\pgfqpoint{2.719186in}{4.305297in}}%
\pgfpathlineto{\pgfqpoint{2.726984in}{4.307682in}}%
\pgfpathlineto{\pgfqpoint{2.760560in}{4.343433in}}%
\pgfpathlineto{\pgfqpoint{2.794120in}{4.380660in}}%
\pgfpathlineto{\pgfqpoint{2.788677in}{4.202629in}}%
\pgfpathlineto{\pgfqpoint{2.774199in}{4.677048in}}%
\pgfpathlineto{\pgfqpoint{2.740952in}{4.616032in}}%
\pgfpathlineto{\pgfqpoint{2.709340in}{4.444412in}}%
\pgfpathclose%
\pgfusepath{fill}%
\end{pgfscope}%
\begin{pgfscope}%
\pgfpathrectangle{\pgfqpoint{1.020000in}{0.880000in}}{\pgfqpoint{6.160000in}{6.160000in}}%
\pgfusepath{clip}%
\pgfsetbuttcap%
\pgfsetroundjoin%
\definecolor{currentfill}{rgb}{0.630089,0.752516,0.998508}%
\pgfsetfillcolor{currentfill}%
\pgfsetlinewidth{0.000000pt}%
\definecolor{currentstroke}{rgb}{0.000000,0.000000,0.000000}%
\pgfsetstrokecolor{currentstroke}%
\pgfsetdash{}{0pt}%
\pgfpathmoveto{\pgfqpoint{4.022051in}{3.323227in}}%
\pgfpathlineto{\pgfqpoint{4.030982in}{3.499460in}}%
\pgfpathlineto{\pgfqpoint{4.040040in}{3.622376in}}%
\pgfpathlineto{\pgfqpoint{4.074013in}{3.435255in}}%
\pgfpathlineto{\pgfqpoint{4.107578in}{3.593420in}}%
\pgfpathlineto{\pgfqpoint{4.098300in}{3.555796in}}%
\pgfpathlineto{\pgfqpoint{4.089117in}{3.446511in}}%
\pgfpathlineto{\pgfqpoint{4.055401in}{3.519999in}}%
\pgfpathlineto{\pgfqpoint{4.022051in}{3.323227in}}%
\pgfpathclose%
\pgfusepath{fill}%
\end{pgfscope}%
\begin{pgfscope}%
\pgfpathrectangle{\pgfqpoint{1.020000in}{0.880000in}}{\pgfqpoint{6.160000in}{6.160000in}}%
\pgfusepath{clip}%
\pgfsetbuttcap%
\pgfsetroundjoin%
\definecolor{currentfill}{rgb}{0.363461,0.484784,0.901019}%
\pgfsetfillcolor{currentfill}%
\pgfsetlinewidth{0.000000pt}%
\definecolor{currentstroke}{rgb}{0.000000,0.000000,0.000000}%
\pgfsetstrokecolor{currentstroke}%
\pgfsetdash{}{0pt}%
\pgfpathmoveto{\pgfqpoint{5.297289in}{3.113937in}}%
\pgfpathlineto{\pgfqpoint{5.307251in}{3.057827in}}%
\pgfpathlineto{\pgfqpoint{5.316352in}{2.925152in}}%
\pgfpathlineto{\pgfqpoint{5.350473in}{2.994092in}}%
\pgfpathlineto{\pgfqpoint{5.383167in}{2.942511in}}%
\pgfpathlineto{\pgfqpoint{5.373601in}{3.034468in}}%
\pgfpathlineto{\pgfqpoint{5.363148in}{3.051208in}}%
\pgfpathlineto{\pgfqpoint{5.329519in}{3.020053in}}%
\pgfpathlineto{\pgfqpoint{5.297289in}{3.113937in}}%
\pgfpathclose%
\pgfusepath{fill}%
\end{pgfscope}%
\begin{pgfscope}%
\pgfpathrectangle{\pgfqpoint{1.020000in}{0.880000in}}{\pgfqpoint{6.160000in}{6.160000in}}%
\pgfusepath{clip}%
\pgfsetbuttcap%
\pgfsetroundjoin%
\definecolor{currentfill}{rgb}{0.949454,0.572388,0.453443}%
\pgfsetfillcolor{currentfill}%
\pgfsetlinewidth{0.000000pt}%
\definecolor{currentstroke}{rgb}{0.000000,0.000000,0.000000}%
\pgfsetstrokecolor{currentstroke}%
\pgfsetdash{}{0pt}%
\pgfpathmoveto{\pgfqpoint{2.979436in}{4.596483in}}%
\pgfpathlineto{\pgfqpoint{2.986294in}{4.702337in}}%
\pgfpathlineto{\pgfqpoint{2.995591in}{4.603245in}}%
\pgfpathlineto{\pgfqpoint{3.030820in}{4.497066in}}%
\pgfpathlineto{\pgfqpoint{3.063669in}{4.597773in}}%
\pgfpathlineto{\pgfqpoint{3.055147in}{4.627289in}}%
\pgfpathlineto{\pgfqpoint{3.046747in}{4.647019in}}%
\pgfpathlineto{\pgfqpoint{3.009305in}{4.949335in}}%
\pgfpathlineto{\pgfqpoint{2.979436in}{4.596483in}}%
\pgfpathclose%
\pgfusepath{fill}%
\end{pgfscope}%
\begin{pgfscope}%
\pgfpathrectangle{\pgfqpoint{1.020000in}{0.880000in}}{\pgfqpoint{6.160000in}{6.160000in}}%
\pgfusepath{clip}%
\pgfsetbuttcap%
\pgfsetroundjoin%
\definecolor{currentfill}{rgb}{0.313946,0.420052,0.854993}%
\pgfsetfillcolor{currentfill}%
\pgfsetlinewidth{0.000000pt}%
\definecolor{currentstroke}{rgb}{0.000000,0.000000,0.000000}%
\pgfsetstrokecolor{currentstroke}%
\pgfsetdash{}{0pt}%
\pgfpathmoveto{\pgfqpoint{5.958435in}{2.955088in}}%
\pgfpathlineto{\pgfqpoint{5.968935in}{2.910328in}}%
\pgfpathlineto{\pgfqpoint{5.978623in}{2.819837in}}%
\pgfpathlineto{\pgfqpoint{6.011496in}{2.806763in}}%
\pgfpathlineto{\pgfqpoint{6.046618in}{2.915262in}}%
\pgfpathlineto{\pgfqpoint{6.036857in}{3.002688in}}%
\pgfpathlineto{\pgfqpoint{6.024129in}{2.929453in}}%
\pgfpathlineto{\pgfqpoint{5.992007in}{2.981412in}}%
\pgfpathlineto{\pgfqpoint{5.958435in}{2.955088in}}%
\pgfpathclose%
\pgfusepath{fill}%
\end{pgfscope}%
\begin{pgfscope}%
\pgfpathrectangle{\pgfqpoint{1.020000in}{0.880000in}}{\pgfqpoint{6.160000in}{6.160000in}}%
\pgfusepath{clip}%
\pgfsetbuttcap%
\pgfsetroundjoin%
\definecolor{currentfill}{rgb}{0.965899,0.740142,0.637058}%
\pgfsetfillcolor{currentfill}%
\pgfsetlinewidth{0.000000pt}%
\definecolor{currentstroke}{rgb}{0.000000,0.000000,0.000000}%
\pgfsetstrokecolor{currentstroke}%
\pgfsetdash{}{0pt}%
\pgfpathmoveto{\pgfqpoint{2.641290in}{4.436012in}}%
\pgfpathlineto{\pgfqpoint{2.651035in}{4.305239in}}%
\pgfpathlineto{\pgfqpoint{2.659395in}{4.265861in}}%
\pgfpathlineto{\pgfqpoint{2.692490in}{4.334328in}}%
\pgfpathlineto{\pgfqpoint{2.726984in}{4.307682in}}%
\pgfpathlineto{\pgfqpoint{2.719186in}{4.305297in}}%
\pgfpathlineto{\pgfqpoint{2.709340in}{4.444412in}}%
\pgfpathlineto{\pgfqpoint{2.674847in}{4.472170in}}%
\pgfpathlineto{\pgfqpoint{2.641290in}{4.436012in}}%
\pgfpathclose%
\pgfusepath{fill}%
\end{pgfscope}%
\begin{pgfscope}%
\pgfpathrectangle{\pgfqpoint{1.020000in}{0.880000in}}{\pgfqpoint{6.160000in}{6.160000in}}%
\pgfusepath{clip}%
\pgfsetbuttcap%
\pgfsetroundjoin%
\definecolor{currentfill}{rgb}{0.919376,0.831273,0.782874}%
\pgfsetfillcolor{currentfill}%
\pgfsetlinewidth{0.000000pt}%
\definecolor{currentstroke}{rgb}{0.000000,0.000000,0.000000}%
\pgfsetstrokecolor{currentstroke}%
\pgfsetdash{}{0pt}%
\pgfpathmoveto{\pgfqpoint{2.374449in}{4.065743in}}%
\pgfpathlineto{\pgfqpoint{2.379277in}{4.212742in}}%
\pgfpathlineto{\pgfqpoint{2.390860in}{3.982864in}}%
\pgfpathlineto{\pgfqpoint{2.421706in}{4.173093in}}%
\pgfpathlineto{\pgfqpoint{2.458012in}{4.051250in}}%
\pgfpathlineto{\pgfqpoint{2.449788in}{4.090852in}}%
\pgfpathlineto{\pgfqpoint{2.438073in}{4.332904in}}%
\pgfpathlineto{\pgfqpoint{2.408056in}{4.095553in}}%
\pgfpathlineto{\pgfqpoint{2.374449in}{4.065743in}}%
\pgfpathclose%
\pgfusepath{fill}%
\end{pgfscope}%
\begin{pgfscope}%
\pgfpathrectangle{\pgfqpoint{1.020000in}{0.880000in}}{\pgfqpoint{6.160000in}{6.160000in}}%
\pgfusepath{clip}%
\pgfsetbuttcap%
\pgfsetroundjoin%
\definecolor{currentfill}{rgb}{0.786721,0.844807,0.939810}%
\pgfsetfillcolor{currentfill}%
\pgfsetlinewidth{0.000000pt}%
\definecolor{currentstroke}{rgb}{0.000000,0.000000,0.000000}%
\pgfsetstrokecolor{currentstroke}%
\pgfsetdash{}{0pt}%
\pgfpathmoveto{\pgfqpoint{3.560873in}{3.971908in}}%
\pgfpathlineto{\pgfqpoint{3.570525in}{3.822496in}}%
\pgfpathlineto{\pgfqpoint{3.579943in}{3.710837in}}%
\pgfpathlineto{\pgfqpoint{3.613486in}{3.763059in}}%
\pgfpathlineto{\pgfqpoint{3.647755in}{3.682034in}}%
\pgfpathlineto{\pgfqpoint{3.638250in}{3.811777in}}%
\pgfpathlineto{\pgfqpoint{3.629761in}{3.756558in}}%
\pgfpathlineto{\pgfqpoint{3.595159in}{3.899061in}}%
\pgfpathlineto{\pgfqpoint{3.560873in}{3.971908in}}%
\pgfpathclose%
\pgfusepath{fill}%
\end{pgfscope}%
\begin{pgfscope}%
\pgfpathrectangle{\pgfqpoint{1.020000in}{0.880000in}}{\pgfqpoint{6.160000in}{6.160000in}}%
\pgfusepath{clip}%
\pgfsetbuttcap%
\pgfsetroundjoin%
\definecolor{currentfill}{rgb}{0.969683,0.690484,0.575138}%
\pgfsetfillcolor{currentfill}%
\pgfsetlinewidth{0.000000pt}%
\definecolor{currentstroke}{rgb}{0.000000,0.000000,0.000000}%
\pgfsetstrokecolor{currentstroke}%
\pgfsetdash{}{0pt}%
\pgfpathmoveto{\pgfqpoint{2.995591in}{4.603245in}}%
\pgfpathlineto{\pgfqpoint{3.004661in}{4.523779in}}%
\pgfpathlineto{\pgfqpoint{3.015395in}{4.300993in}}%
\pgfpathlineto{\pgfqpoint{3.050017in}{4.244244in}}%
\pgfpathlineto{\pgfqpoint{3.082799in}{4.350859in}}%
\pgfpathlineto{\pgfqpoint{3.072616in}{4.530690in}}%
\pgfpathlineto{\pgfqpoint{3.063669in}{4.597773in}}%
\pgfpathlineto{\pgfqpoint{3.030820in}{4.497066in}}%
\pgfpathlineto{\pgfqpoint{2.995591in}{4.603245in}}%
\pgfpathclose%
\pgfusepath{fill}%
\end{pgfscope}%
\begin{pgfscope}%
\pgfpathrectangle{\pgfqpoint{1.020000in}{0.880000in}}{\pgfqpoint{6.160000in}{6.160000in}}%
\pgfusepath{clip}%
\pgfsetbuttcap%
\pgfsetroundjoin%
\definecolor{currentfill}{rgb}{0.586921,0.718121,0.998874}%
\pgfsetfillcolor{currentfill}%
\pgfsetlinewidth{0.000000pt}%
\definecolor{currentstroke}{rgb}{0.000000,0.000000,0.000000}%
\pgfsetstrokecolor{currentstroke}%
\pgfsetdash{}{0pt}%
\pgfpathmoveto{\pgfqpoint{4.175013in}{3.366225in}}%
\pgfpathlineto{\pgfqpoint{4.184389in}{3.478034in}}%
\pgfpathlineto{\pgfqpoint{4.193799in}{3.520851in}}%
\pgfpathlineto{\pgfqpoint{4.227504in}{3.487011in}}%
\pgfpathlineto{\pgfqpoint{4.261094in}{3.361106in}}%
\pgfpathlineto{\pgfqpoint{4.251651in}{3.387633in}}%
\pgfpathlineto{\pgfqpoint{4.242170in}{3.323513in}}%
\pgfpathlineto{\pgfqpoint{4.208640in}{3.468797in}}%
\pgfpathlineto{\pgfqpoint{4.175013in}{3.366225in}}%
\pgfpathclose%
\pgfusepath{fill}%
\end{pgfscope}%
\begin{pgfscope}%
\pgfpathrectangle{\pgfqpoint{1.020000in}{0.880000in}}{\pgfqpoint{6.160000in}{6.160000in}}%
\pgfusepath{clip}%
\pgfsetbuttcap%
\pgfsetroundjoin%
\definecolor{currentfill}{rgb}{0.358415,0.478426,0.896795}%
\pgfsetfillcolor{currentfill}%
\pgfsetlinewidth{0.000000pt}%
\definecolor{currentstroke}{rgb}{0.000000,0.000000,0.000000}%
\pgfsetstrokecolor{currentstroke}%
\pgfsetdash{}{0pt}%
\pgfpathmoveto{\pgfqpoint{5.518262in}{3.099143in}}%
\pgfpathlineto{\pgfqpoint{5.528423in}{3.049199in}}%
\pgfpathlineto{\pgfqpoint{5.538959in}{3.025885in}}%
\pgfpathlineto{\pgfqpoint{5.572488in}{3.042920in}}%
\pgfpathlineto{\pgfqpoint{5.603753in}{2.901988in}}%
\pgfpathlineto{\pgfqpoint{5.592607in}{2.884734in}}%
\pgfpathlineto{\pgfqpoint{5.582699in}{2.953663in}}%
\pgfpathlineto{\pgfqpoint{5.550963in}{3.059081in}}%
\pgfpathlineto{\pgfqpoint{5.518262in}{3.099143in}}%
\pgfpathclose%
\pgfusepath{fill}%
\end{pgfscope}%
\begin{pgfscope}%
\pgfpathrectangle{\pgfqpoint{1.020000in}{0.880000in}}{\pgfqpoint{6.160000in}{6.160000in}}%
\pgfusepath{clip}%
\pgfsetbuttcap%
\pgfsetroundjoin%
\definecolor{currentfill}{rgb}{0.831148,0.859513,0.903110}%
\pgfsetfillcolor{currentfill}%
\pgfsetlinewidth{0.000000pt}%
\definecolor{currentstroke}{rgb}{0.000000,0.000000,0.000000}%
\pgfsetstrokecolor{currentstroke}%
\pgfsetdash{}{0pt}%
\pgfpathmoveto{\pgfqpoint{3.407626in}{4.029976in}}%
\pgfpathlineto{\pgfqpoint{3.416620in}{3.974937in}}%
\pgfpathlineto{\pgfqpoint{3.426043in}{3.864146in}}%
\pgfpathlineto{\pgfqpoint{3.461423in}{3.659220in}}%
\pgfpathlineto{\pgfqpoint{3.494223in}{3.807214in}}%
\pgfpathlineto{\pgfqpoint{3.485107in}{3.875545in}}%
\pgfpathlineto{\pgfqpoint{3.475541in}{4.008695in}}%
\pgfpathlineto{\pgfqpoint{3.441739in}{4.000138in}}%
\pgfpathlineto{\pgfqpoint{3.407626in}{4.029976in}}%
\pgfpathclose%
\pgfusepath{fill}%
\end{pgfscope}%
\begin{pgfscope}%
\pgfpathrectangle{\pgfqpoint{1.020000in}{0.880000in}}{\pgfqpoint{6.160000in}{6.160000in}}%
\pgfusepath{clip}%
\pgfsetbuttcap%
\pgfsetroundjoin%
\definecolor{currentfill}{rgb}{0.966922,0.651969,0.531997}%
\pgfsetfillcolor{currentfill}%
\pgfsetlinewidth{0.000000pt}%
\definecolor{currentstroke}{rgb}{0.000000,0.000000,0.000000}%
\pgfsetstrokecolor{currentstroke}%
\pgfsetdash{}{0pt}%
\pgfpathmoveto{\pgfqpoint{3.063669in}{4.597773in}}%
\pgfpathlineto{\pgfqpoint{3.072616in}{4.530690in}}%
\pgfpathlineto{\pgfqpoint{3.082799in}{4.350859in}}%
\pgfpathlineto{\pgfqpoint{3.115914in}{4.430834in}}%
\pgfpathlineto{\pgfqpoint{3.148970in}{4.519921in}}%
\pgfpathlineto{\pgfqpoint{3.140567in}{4.531042in}}%
\pgfpathlineto{\pgfqpoint{3.133136in}{4.450118in}}%
\pgfpathlineto{\pgfqpoint{3.095816in}{4.768200in}}%
\pgfpathlineto{\pgfqpoint{3.063669in}{4.597773in}}%
\pgfpathclose%
\pgfusepath{fill}%
\end{pgfscope}%
\begin{pgfscope}%
\pgfpathrectangle{\pgfqpoint{1.020000in}{0.880000in}}{\pgfqpoint{6.160000in}{6.160000in}}%
\pgfusepath{clip}%
\pgfsetbuttcap%
\pgfsetroundjoin%
\definecolor{currentfill}{rgb}{0.489246,0.627536,0.976896}%
\pgfsetfillcolor{currentfill}%
\pgfsetlinewidth{0.000000pt}%
\definecolor{currentstroke}{rgb}{0.000000,0.000000,0.000000}%
\pgfsetstrokecolor{currentstroke}%
\pgfsetdash{}{0pt}%
\pgfpathmoveto{\pgfqpoint{4.481518in}{3.223496in}}%
\pgfpathlineto{\pgfqpoint{4.491507in}{3.314363in}}%
\pgfpathlineto{\pgfqpoint{4.501185in}{3.295314in}}%
\pgfpathlineto{\pgfqpoint{4.534646in}{3.251416in}}%
\pgfpathlineto{\pgfqpoint{4.568231in}{3.252474in}}%
\pgfpathlineto{\pgfqpoint{4.558453in}{3.258162in}}%
\pgfpathlineto{\pgfqpoint{4.548331in}{3.164196in}}%
\pgfpathlineto{\pgfqpoint{4.514886in}{3.176147in}}%
\pgfpathlineto{\pgfqpoint{4.481518in}{3.223496in}}%
\pgfpathclose%
\pgfusepath{fill}%
\end{pgfscope}%
\begin{pgfscope}%
\pgfpathrectangle{\pgfqpoint{1.020000in}{0.880000in}}{\pgfqpoint{6.160000in}{6.160000in}}%
\pgfusepath{clip}%
\pgfsetbuttcap%
\pgfsetroundjoin%
\definecolor{currentfill}{rgb}{0.962701,0.628218,0.507636}%
\pgfsetfillcolor{currentfill}%
\pgfsetlinewidth{0.000000pt}%
\definecolor{currentstroke}{rgb}{0.000000,0.000000,0.000000}%
\pgfsetstrokecolor{currentstroke}%
\pgfsetdash{}{0pt}%
\pgfpathmoveto{\pgfqpoint{2.841701in}{4.727764in}}%
\pgfpathlineto{\pgfqpoint{2.852406in}{4.521079in}}%
\pgfpathlineto{\pgfqpoint{2.859376in}{4.598612in}}%
\pgfpathlineto{\pgfqpoint{2.895842in}{4.412847in}}%
\pgfpathlineto{\pgfqpoint{2.929440in}{4.446454in}}%
\pgfpathlineto{\pgfqpoint{2.921239in}{4.458834in}}%
\pgfpathlineto{\pgfqpoint{2.911402in}{4.602528in}}%
\pgfpathlineto{\pgfqpoint{2.876056in}{4.705964in}}%
\pgfpathlineto{\pgfqpoint{2.841701in}{4.727764in}}%
\pgfpathclose%
\pgfusepath{fill}%
\end{pgfscope}%
\begin{pgfscope}%
\pgfpathrectangle{\pgfqpoint{1.020000in}{0.880000in}}{\pgfqpoint{6.160000in}{6.160000in}}%
\pgfusepath{clip}%
\pgfsetbuttcap%
\pgfsetroundjoin%
\definecolor{currentfill}{rgb}{0.441123,0.576532,0.954545}%
\pgfsetfillcolor{currentfill}%
\pgfsetlinewidth{0.000000pt}%
\definecolor{currentstroke}{rgb}{0.000000,0.000000,0.000000}%
\pgfsetstrokecolor{currentstroke}%
\pgfsetdash{}{0pt}%
\pgfpathmoveto{\pgfqpoint{4.635211in}{3.225584in}}%
\pgfpathlineto{\pgfqpoint{4.644942in}{3.192712in}}%
\pgfpathlineto{\pgfqpoint{4.654945in}{3.214310in}}%
\pgfpathlineto{\pgfqpoint{4.687983in}{3.108379in}}%
\pgfpathlineto{\pgfqpoint{4.721416in}{3.092572in}}%
\pgfpathlineto{\pgfqpoint{4.711247in}{3.052665in}}%
\pgfpathlineto{\pgfqpoint{4.701654in}{3.117835in}}%
\pgfpathlineto{\pgfqpoint{4.668632in}{3.206913in}}%
\pgfpathlineto{\pgfqpoint{4.635211in}{3.225584in}}%
\pgfpathclose%
\pgfusepath{fill}%
\end{pgfscope}%
\begin{pgfscope}%
\pgfpathrectangle{\pgfqpoint{1.020000in}{0.880000in}}{\pgfqpoint{6.160000in}{6.160000in}}%
\pgfusepath{clip}%
\pgfsetbuttcap%
\pgfsetroundjoin%
\definecolor{currentfill}{rgb}{0.318832,0.426605,0.859857}%
\pgfsetfillcolor{currentfill}%
\pgfsetlinewidth{0.000000pt}%
\definecolor{currentstroke}{rgb}{0.000000,0.000000,0.000000}%
\pgfsetstrokecolor{currentstroke}%
\pgfsetdash{}{0pt}%
\pgfpathmoveto{\pgfqpoint{5.670998in}{2.958376in}}%
\pgfpathlineto{\pgfqpoint{5.680787in}{2.878205in}}%
\pgfpathlineto{\pgfqpoint{5.693187in}{2.969708in}}%
\pgfpathlineto{\pgfqpoint{5.723690in}{2.791972in}}%
\pgfpathlineto{\pgfqpoint{5.759491in}{2.957504in}}%
\pgfpathlineto{\pgfqpoint{5.747016in}{2.869141in}}%
\pgfpathlineto{\pgfqpoint{5.738093in}{3.005506in}}%
\pgfpathlineto{\pgfqpoint{5.703437in}{2.909251in}}%
\pgfpathlineto{\pgfqpoint{5.670998in}{2.958376in}}%
\pgfpathclose%
\pgfusepath{fill}%
\end{pgfscope}%
\begin{pgfscope}%
\pgfpathrectangle{\pgfqpoint{1.020000in}{0.880000in}}{\pgfqpoint{6.160000in}{6.160000in}}%
\pgfusepath{clip}%
\pgfsetbuttcap%
\pgfsetroundjoin%
\definecolor{currentfill}{rgb}{0.962708,0.753557,0.655601}%
\pgfsetfillcolor{currentfill}%
\pgfsetlinewidth{0.000000pt}%
\definecolor{currentstroke}{rgb}{0.000000,0.000000,0.000000}%
\pgfsetstrokecolor{currentstroke}%
\pgfsetdash{}{0pt}%
\pgfpathmoveto{\pgfqpoint{2.572082in}{4.495929in}}%
\pgfpathlineto{\pgfqpoint{2.583186in}{4.281133in}}%
\pgfpathlineto{\pgfqpoint{2.592096in}{4.204154in}}%
\pgfpathlineto{\pgfqpoint{2.624817in}{4.295082in}}%
\pgfpathlineto{\pgfqpoint{2.659395in}{4.265861in}}%
\pgfpathlineto{\pgfqpoint{2.651035in}{4.305239in}}%
\pgfpathlineto{\pgfqpoint{2.641290in}{4.436012in}}%
\pgfpathlineto{\pgfqpoint{2.608899in}{4.325458in}}%
\pgfpathlineto{\pgfqpoint{2.572082in}{4.495929in}}%
\pgfpathclose%
\pgfusepath{fill}%
\end{pgfscope}%
\begin{pgfscope}%
\pgfpathrectangle{\pgfqpoint{1.020000in}{0.880000in}}{\pgfqpoint{6.160000in}{6.160000in}}%
\pgfusepath{clip}%
\pgfsetbuttcap%
\pgfsetroundjoin%
\definecolor{currentfill}{rgb}{0.968863,0.710838,0.599901}%
\pgfsetfillcolor{currentfill}%
\pgfsetlinewidth{0.000000pt}%
\definecolor{currentstroke}{rgb}{0.000000,0.000000,0.000000}%
\pgfsetstrokecolor{currentstroke}%
\pgfsetdash{}{0pt}%
\pgfpathmoveto{\pgfqpoint{2.929440in}{4.446454in}}%
\pgfpathlineto{\pgfqpoint{2.939727in}{4.266163in}}%
\pgfpathlineto{\pgfqpoint{2.946523in}{4.369823in}}%
\pgfpathlineto{\pgfqpoint{2.980755in}{4.354279in}}%
\pgfpathlineto{\pgfqpoint{3.015395in}{4.300993in}}%
\pgfpathlineto{\pgfqpoint{3.004661in}{4.523779in}}%
\pgfpathlineto{\pgfqpoint{2.995591in}{4.603245in}}%
\pgfpathlineto{\pgfqpoint{2.963389in}{4.451173in}}%
\pgfpathlineto{\pgfqpoint{2.929440in}{4.446454in}}%
\pgfpathclose%
\pgfusepath{fill}%
\end{pgfscope}%
\begin{pgfscope}%
\pgfpathrectangle{\pgfqpoint{1.020000in}{0.880000in}}{\pgfqpoint{6.160000in}{6.160000in}}%
\pgfusepath{clip}%
\pgfsetbuttcap%
\pgfsetroundjoin%
\definecolor{currentfill}{rgb}{0.967544,0.730850,0.624685}%
\pgfsetfillcolor{currentfill}%
\pgfsetlinewidth{0.000000pt}%
\definecolor{currentstroke}{rgb}{0.000000,0.000000,0.000000}%
\pgfsetstrokecolor{currentstroke}%
\pgfsetdash{}{0pt}%
\pgfpathmoveto{\pgfqpoint{3.148970in}{4.519921in}}%
\pgfpathlineto{\pgfqpoint{3.158047in}{4.444024in}}%
\pgfpathlineto{\pgfqpoint{3.168953in}{4.186879in}}%
\pgfpathlineto{\pgfqpoint{3.201029in}{4.377727in}}%
\pgfpathlineto{\pgfqpoint{3.236292in}{4.240768in}}%
\pgfpathlineto{\pgfqpoint{3.228516in}{4.178271in}}%
\pgfpathlineto{\pgfqpoint{3.216761in}{4.534939in}}%
\pgfpathlineto{\pgfqpoint{3.182545in}{4.560627in}}%
\pgfpathlineto{\pgfqpoint{3.148970in}{4.519921in}}%
\pgfpathclose%
\pgfusepath{fill}%
\end{pgfscope}%
\begin{pgfscope}%
\pgfpathrectangle{\pgfqpoint{1.020000in}{0.880000in}}{\pgfqpoint{6.160000in}{6.160000in}}%
\pgfusepath{clip}%
\pgfsetbuttcap%
\pgfsetroundjoin%
\definecolor{currentfill}{rgb}{0.724041,0.814910,0.975651}%
\pgfsetfillcolor{currentfill}%
\pgfsetlinewidth{0.000000pt}%
\definecolor{currentstroke}{rgb}{0.000000,0.000000,0.000000}%
\pgfsetstrokecolor{currentstroke}%
\pgfsetdash{}{0pt}%
\pgfpathmoveto{\pgfqpoint{3.647755in}{3.682034in}}%
\pgfpathlineto{\pgfqpoint{3.656042in}{3.783604in}}%
\pgfpathlineto{\pgfqpoint{3.665241in}{3.715647in}}%
\pgfpathlineto{\pgfqpoint{3.699267in}{3.682715in}}%
\pgfpathlineto{\pgfqpoint{3.733792in}{3.529284in}}%
\pgfpathlineto{\pgfqpoint{3.725066in}{3.492865in}}%
\pgfpathlineto{\pgfqpoint{3.715169in}{3.712956in}}%
\pgfpathlineto{\pgfqpoint{3.681199in}{3.751033in}}%
\pgfpathlineto{\pgfqpoint{3.647755in}{3.682034in}}%
\pgfpathclose%
\pgfusepath{fill}%
\end{pgfscope}%
\begin{pgfscope}%
\pgfpathrectangle{\pgfqpoint{1.020000in}{0.880000in}}{\pgfqpoint{6.160000in}{6.160000in}}%
\pgfusepath{clip}%
\pgfsetbuttcap%
\pgfsetroundjoin%
\definecolor{currentfill}{rgb}{0.967317,0.657471,0.538160}%
\pgfsetfillcolor{currentfill}%
\pgfsetlinewidth{0.000000pt}%
\definecolor{currentstroke}{rgb}{0.000000,0.000000,0.000000}%
\pgfsetstrokecolor{currentstroke}%
\pgfsetdash{}{0pt}%
\pgfpathmoveto{\pgfqpoint{2.774199in}{4.677048in}}%
\pgfpathlineto{\pgfqpoint{2.788677in}{4.202629in}}%
\pgfpathlineto{\pgfqpoint{2.794120in}{4.380660in}}%
\pgfpathlineto{\pgfqpoint{2.826953in}{4.472373in}}%
\pgfpathlineto{\pgfqpoint{2.859376in}{4.598612in}}%
\pgfpathlineto{\pgfqpoint{2.852406in}{4.521079in}}%
\pgfpathlineto{\pgfqpoint{2.841701in}{4.727764in}}%
\pgfpathlineto{\pgfqpoint{2.810146in}{4.540770in}}%
\pgfpathlineto{\pgfqpoint{2.774199in}{4.677048in}}%
\pgfpathclose%
\pgfusepath{fill}%
\end{pgfscope}%
\begin{pgfscope}%
\pgfpathrectangle{\pgfqpoint{1.020000in}{0.880000in}}{\pgfqpoint{6.160000in}{6.160000in}}%
\pgfusepath{clip}%
\pgfsetbuttcap%
\pgfsetroundjoin%
\definecolor{currentfill}{rgb}{0.906154,0.842091,0.806151}%
\pgfsetfillcolor{currentfill}%
\pgfsetlinewidth{0.000000pt}%
\definecolor{currentstroke}{rgb}{0.000000,0.000000,0.000000}%
\pgfsetstrokecolor{currentstroke}%
\pgfsetdash{}{0pt}%
\pgfpathmoveto{\pgfqpoint{3.322580in}{4.057829in}}%
\pgfpathlineto{\pgfqpoint{3.330322in}{4.144233in}}%
\pgfpathlineto{\pgfqpoint{3.339549in}{4.057424in}}%
\pgfpathlineto{\pgfqpoint{3.373888in}{4.008183in}}%
\pgfpathlineto{\pgfqpoint{3.407626in}{4.029976in}}%
\pgfpathlineto{\pgfqpoint{3.398995in}{4.039512in}}%
\pgfpathlineto{\pgfqpoint{3.390073in}{4.088102in}}%
\pgfpathlineto{\pgfqpoint{3.355147in}{4.217190in}}%
\pgfpathlineto{\pgfqpoint{3.322580in}{4.057829in}}%
\pgfpathclose%
\pgfusepath{fill}%
\end{pgfscope}%
\begin{pgfscope}%
\pgfpathrectangle{\pgfqpoint{1.020000in}{0.880000in}}{\pgfqpoint{6.160000in}{6.160000in}}%
\pgfusepath{clip}%
\pgfsetbuttcap%
\pgfsetroundjoin%
\definecolor{currentfill}{rgb}{0.554312,0.690097,0.995516}%
\pgfsetfillcolor{currentfill}%
\pgfsetlinewidth{0.000000pt}%
\definecolor{currentstroke}{rgb}{0.000000,0.000000,0.000000}%
\pgfsetstrokecolor{currentstroke}%
\pgfsetdash{}{0pt}%
\pgfpathmoveto{\pgfqpoint{4.328140in}{3.233481in}}%
\pgfpathlineto{\pgfqpoint{4.338109in}{3.513035in}}%
\pgfpathlineto{\pgfqpoint{4.347509in}{3.401636in}}%
\pgfpathlineto{\pgfqpoint{4.380876in}{3.251008in}}%
\pgfpathlineto{\pgfqpoint{4.414747in}{3.370327in}}%
\pgfpathlineto{\pgfqpoint{4.405030in}{3.333810in}}%
\pgfpathlineto{\pgfqpoint{4.395471in}{3.358438in}}%
\pgfpathlineto{\pgfqpoint{4.362003in}{3.412434in}}%
\pgfpathlineto{\pgfqpoint{4.328140in}{3.233481in}}%
\pgfpathclose%
\pgfusepath{fill}%
\end{pgfscope}%
\begin{pgfscope}%
\pgfpathrectangle{\pgfqpoint{1.020000in}{0.880000in}}{\pgfqpoint{6.160000in}{6.160000in}}%
\pgfusepath{clip}%
\pgfsetbuttcap%
\pgfsetroundjoin%
\definecolor{currentfill}{rgb}{0.399231,0.528528,0.928459}%
\pgfsetfillcolor{currentfill}%
\pgfsetlinewidth{0.000000pt}%
\definecolor{currentstroke}{rgb}{0.000000,0.000000,0.000000}%
\pgfsetstrokecolor{currentstroke}%
\pgfsetdash{}{0pt}%
\pgfpathmoveto{\pgfqpoint{4.788893in}{3.176817in}}%
\pgfpathlineto{\pgfqpoint{4.798361in}{3.082027in}}%
\pgfpathlineto{\pgfqpoint{4.808459in}{3.087243in}}%
\pgfpathlineto{\pgfqpoint{4.842463in}{3.160850in}}%
\pgfpathlineto{\pgfqpoint{4.875704in}{3.120388in}}%
\pgfpathlineto{\pgfqpoint{4.864536in}{2.969537in}}%
\pgfpathlineto{\pgfqpoint{4.853680in}{2.857242in}}%
\pgfpathlineto{\pgfqpoint{4.822264in}{3.159582in}}%
\pgfpathlineto{\pgfqpoint{4.788893in}{3.176817in}}%
\pgfpathclose%
\pgfusepath{fill}%
\end{pgfscope}%
\begin{pgfscope}%
\pgfpathrectangle{\pgfqpoint{1.020000in}{0.880000in}}{\pgfqpoint{6.160000in}{6.160000in}}%
\pgfusepath{clip}%
\pgfsetbuttcap%
\pgfsetroundjoin%
\definecolor{currentfill}{rgb}{0.959518,0.766973,0.674145}%
\pgfsetfillcolor{currentfill}%
\pgfsetlinewidth{0.000000pt}%
\definecolor{currentstroke}{rgb}{0.000000,0.000000,0.000000}%
\pgfsetstrokecolor{currentstroke}%
\pgfsetdash{}{0pt}%
\pgfpathmoveto{\pgfqpoint{2.508415in}{4.212740in}}%
\pgfpathlineto{\pgfqpoint{2.516111in}{4.207280in}}%
\pgfpathlineto{\pgfqpoint{2.523945in}{4.194375in}}%
\pgfpathlineto{\pgfqpoint{2.556009in}{4.324489in}}%
\pgfpathlineto{\pgfqpoint{2.592096in}{4.204154in}}%
\pgfpathlineto{\pgfqpoint{2.583186in}{4.281133in}}%
\pgfpathlineto{\pgfqpoint{2.572082in}{4.495929in}}%
\pgfpathlineto{\pgfqpoint{2.538946in}{4.432049in}}%
\pgfpathlineto{\pgfqpoint{2.508415in}{4.212740in}}%
\pgfpathclose%
\pgfusepath{fill}%
\end{pgfscope}%
\begin{pgfscope}%
\pgfpathrectangle{\pgfqpoint{1.020000in}{0.880000in}}{\pgfqpoint{6.160000in}{6.160000in}}%
\pgfusepath{clip}%
\pgfsetbuttcap%
\pgfsetroundjoin%
\definecolor{currentfill}{rgb}{0.318832,0.426605,0.859857}%
\pgfsetfillcolor{currentfill}%
\pgfsetlinewidth{0.000000pt}%
\definecolor{currentstroke}{rgb}{0.000000,0.000000,0.000000}%
\pgfsetstrokecolor{currentstroke}%
\pgfsetdash{}{0pt}%
\pgfpathmoveto{\pgfqpoint{5.893201in}{3.014148in}}%
\pgfpathlineto{\pgfqpoint{5.903047in}{2.933141in}}%
\pgfpathlineto{\pgfqpoint{5.914059in}{2.918863in}}%
\pgfpathlineto{\pgfqpoint{5.945256in}{2.806661in}}%
\pgfpathlineto{\pgfqpoint{5.978623in}{2.819837in}}%
\pgfpathlineto{\pgfqpoint{5.968935in}{2.910328in}}%
\pgfpathlineto{\pgfqpoint{5.958435in}{2.955088in}}%
\pgfpathlineto{\pgfqpoint{5.925123in}{2.943764in}}%
\pgfpathlineto{\pgfqpoint{5.893201in}{3.014148in}}%
\pgfpathclose%
\pgfusepath{fill}%
\end{pgfscope}%
\begin{pgfscope}%
\pgfpathrectangle{\pgfqpoint{1.020000in}{0.880000in}}{\pgfqpoint{6.160000in}{6.160000in}}%
\pgfusepath{clip}%
\pgfsetbuttcap%
\pgfsetroundjoin%
\definecolor{currentfill}{rgb}{0.635474,0.756714,0.998297}%
\pgfsetfillcolor{currentfill}%
\pgfsetlinewidth{0.000000pt}%
\definecolor{currentstroke}{rgb}{0.000000,0.000000,0.000000}%
\pgfsetstrokecolor{currentstroke}%
\pgfsetdash{}{0pt}%
\pgfpathmoveto{\pgfqpoint{3.954358in}{3.512202in}}%
\pgfpathlineto{\pgfqpoint{3.963364in}{3.577050in}}%
\pgfpathlineto{\pgfqpoint{3.972393in}{3.646659in}}%
\pgfpathlineto{\pgfqpoint{4.006611in}{3.419299in}}%
\pgfpathlineto{\pgfqpoint{4.040040in}{3.622376in}}%
\pgfpathlineto{\pgfqpoint{4.030982in}{3.499460in}}%
\pgfpathlineto{\pgfqpoint{4.022051in}{3.323227in}}%
\pgfpathlineto{\pgfqpoint{3.988220in}{3.431191in}}%
\pgfpathlineto{\pgfqpoint{3.954358in}{3.512202in}}%
\pgfpathclose%
\pgfusepath{fill}%
\end{pgfscope}%
\begin{pgfscope}%
\pgfpathrectangle{\pgfqpoint{1.020000in}{0.880000in}}{\pgfqpoint{6.160000in}{6.160000in}}%
\pgfusepath{clip}%
\pgfsetbuttcap%
\pgfsetroundjoin%
\definecolor{currentfill}{rgb}{0.958176,0.771234,0.680301}%
\pgfsetfillcolor{currentfill}%
\pgfsetlinewidth{0.000000pt}%
\definecolor{currentstroke}{rgb}{0.000000,0.000000,0.000000}%
\pgfsetstrokecolor{currentstroke}%
\pgfsetdash{}{0pt}%
\pgfpathmoveto{\pgfqpoint{3.015395in}{4.300993in}}%
\pgfpathlineto{\pgfqpoint{3.022646in}{4.379495in}}%
\pgfpathlineto{\pgfqpoint{3.033847in}{4.114579in}}%
\pgfpathlineto{\pgfqpoint{3.066911in}{4.198084in}}%
\pgfpathlineto{\pgfqpoint{3.099142in}{4.362413in}}%
\pgfpathlineto{\pgfqpoint{3.090813in}{4.370372in}}%
\pgfpathlineto{\pgfqpoint{3.082799in}{4.350859in}}%
\pgfpathlineto{\pgfqpoint{3.050017in}{4.244244in}}%
\pgfpathlineto{\pgfqpoint{3.015395in}{4.300993in}}%
\pgfpathclose%
\pgfusepath{fill}%
\end{pgfscope}%
\begin{pgfscope}%
\pgfpathrectangle{\pgfqpoint{1.020000in}{0.880000in}}{\pgfqpoint{6.160000in}{6.160000in}}%
\pgfusepath{clip}%
\pgfsetbuttcap%
\pgfsetroundjoin%
\definecolor{currentfill}{rgb}{0.378598,0.503856,0.913692}%
\pgfsetfillcolor{currentfill}%
\pgfsetlinewidth{0.000000pt}%
\definecolor{currentstroke}{rgb}{0.000000,0.000000,0.000000}%
\pgfsetstrokecolor{currentstroke}%
\pgfsetdash{}{0pt}%
\pgfpathmoveto{\pgfqpoint{5.229519in}{3.010464in}}%
\pgfpathlineto{\pgfqpoint{5.239967in}{3.003744in}}%
\pgfpathlineto{\pgfqpoint{5.250521in}{3.005000in}}%
\pgfpathlineto{\pgfqpoint{5.283394in}{2.959109in}}%
\pgfpathlineto{\pgfqpoint{5.316352in}{2.925152in}}%
\pgfpathlineto{\pgfqpoint{5.307251in}{3.057827in}}%
\pgfpathlineto{\pgfqpoint{5.297289in}{3.113937in}}%
\pgfpathlineto{\pgfqpoint{5.264514in}{3.165239in}}%
\pgfpathlineto{\pgfqpoint{5.229519in}{3.010464in}}%
\pgfpathclose%
\pgfusepath{fill}%
\end{pgfscope}%
\begin{pgfscope}%
\pgfpathrectangle{\pgfqpoint{1.020000in}{0.880000in}}{\pgfqpoint{6.160000in}{6.160000in}}%
\pgfusepath{clip}%
\pgfsetbuttcap%
\pgfsetroundjoin%
\definecolor{currentfill}{rgb}{0.966962,0.735670,0.630877}%
\pgfsetfillcolor{currentfill}%
\pgfsetlinewidth{0.000000pt}%
\definecolor{currentstroke}{rgb}{0.000000,0.000000,0.000000}%
\pgfsetstrokecolor{currentstroke}%
\pgfsetdash{}{0pt}%
\pgfpathmoveto{\pgfqpoint{3.082799in}{4.350859in}}%
\pgfpathlineto{\pgfqpoint{3.090813in}{4.370372in}}%
\pgfpathlineto{\pgfqpoint{3.099142in}{4.362413in}}%
\pgfpathlineto{\pgfqpoint{3.134391in}{4.245481in}}%
\pgfpathlineto{\pgfqpoint{3.168953in}{4.186879in}}%
\pgfpathlineto{\pgfqpoint{3.158047in}{4.444024in}}%
\pgfpathlineto{\pgfqpoint{3.148970in}{4.519921in}}%
\pgfpathlineto{\pgfqpoint{3.115914in}{4.430834in}}%
\pgfpathlineto{\pgfqpoint{3.082799in}{4.350859in}}%
\pgfpathclose%
\pgfusepath{fill}%
\end{pgfscope}%
\begin{pgfscope}%
\pgfpathrectangle{\pgfqpoint{1.020000in}{0.880000in}}{\pgfqpoint{6.160000in}{6.160000in}}%
\pgfusepath{clip}%
\pgfsetbuttcap%
\pgfsetroundjoin%
\definecolor{currentfill}{rgb}{0.968533,0.715841,0.606097}%
\pgfsetfillcolor{currentfill}%
\pgfsetlinewidth{0.000000pt}%
\definecolor{currentstroke}{rgb}{0.000000,0.000000,0.000000}%
\pgfsetstrokecolor{currentstroke}%
\pgfsetdash{}{0pt}%
\pgfpathmoveto{\pgfqpoint{2.859376in}{4.598612in}}%
\pgfpathlineto{\pgfqpoint{2.871527in}{4.280113in}}%
\pgfpathlineto{\pgfqpoint{2.878100in}{4.390495in}}%
\pgfpathlineto{\pgfqpoint{2.911503in}{4.445465in}}%
\pgfpathlineto{\pgfqpoint{2.946523in}{4.369823in}}%
\pgfpathlineto{\pgfqpoint{2.939727in}{4.266163in}}%
\pgfpathlineto{\pgfqpoint{2.929440in}{4.446454in}}%
\pgfpathlineto{\pgfqpoint{2.895842in}{4.412847in}}%
\pgfpathlineto{\pgfqpoint{2.859376in}{4.598612in}}%
\pgfpathclose%
\pgfusepath{fill}%
\end{pgfscope}%
\begin{pgfscope}%
\pgfpathrectangle{\pgfqpoint{1.020000in}{0.880000in}}{\pgfqpoint{6.160000in}{6.160000in}}%
\pgfusepath{clip}%
\pgfsetbuttcap%
\pgfsetroundjoin%
\definecolor{currentfill}{rgb}{0.718985,0.811993,0.977656}%
\pgfsetfillcolor{currentfill}%
\pgfsetlinewidth{0.000000pt}%
\definecolor{currentstroke}{rgb}{0.000000,0.000000,0.000000}%
\pgfsetstrokecolor{currentstroke}%
\pgfsetdash{}{0pt}%
\pgfpathmoveto{\pgfqpoint{3.579943in}{3.710837in}}%
\pgfpathlineto{\pgfqpoint{3.589591in}{3.560324in}}%
\pgfpathlineto{\pgfqpoint{3.598404in}{3.551520in}}%
\pgfpathlineto{\pgfqpoint{3.632439in}{3.518345in}}%
\pgfpathlineto{\pgfqpoint{3.665241in}{3.715647in}}%
\pgfpathlineto{\pgfqpoint{3.656042in}{3.783604in}}%
\pgfpathlineto{\pgfqpoint{3.647755in}{3.682034in}}%
\pgfpathlineto{\pgfqpoint{3.613486in}{3.763059in}}%
\pgfpathlineto{\pgfqpoint{3.579943in}{3.710837in}}%
\pgfpathclose%
\pgfusepath{fill}%
\end{pgfscope}%
\begin{pgfscope}%
\pgfpathrectangle{\pgfqpoint{1.020000in}{0.880000in}}{\pgfqpoint{6.160000in}{6.160000in}}%
\pgfusepath{clip}%
\pgfsetbuttcap%
\pgfsetroundjoin%
\definecolor{currentfill}{rgb}{0.949151,0.790785,0.710876}%
\pgfsetfillcolor{currentfill}%
\pgfsetlinewidth{0.000000pt}%
\definecolor{currentstroke}{rgb}{0.000000,0.000000,0.000000}%
\pgfsetstrokecolor{currentstroke}%
\pgfsetdash{}{0pt}%
\pgfpathmoveto{\pgfqpoint{2.659395in}{4.265861in}}%
\pgfpathlineto{\pgfqpoint{2.669291in}{4.125030in}}%
\pgfpathlineto{\pgfqpoint{2.676569in}{4.158879in}}%
\pgfpathlineto{\pgfqpoint{2.711629in}{4.095823in}}%
\pgfpathlineto{\pgfqpoint{2.743230in}{4.270542in}}%
\pgfpathlineto{\pgfqpoint{2.735240in}{4.279389in}}%
\pgfpathlineto{\pgfqpoint{2.726984in}{4.307682in}}%
\pgfpathlineto{\pgfqpoint{2.692490in}{4.334328in}}%
\pgfpathlineto{\pgfqpoint{2.659395in}{4.265861in}}%
\pgfpathclose%
\pgfusepath{fill}%
\end{pgfscope}%
\begin{pgfscope}%
\pgfpathrectangle{\pgfqpoint{1.020000in}{0.880000in}}{\pgfqpoint{6.160000in}{6.160000in}}%
\pgfusepath{clip}%
\pgfsetbuttcap%
\pgfsetroundjoin%
\definecolor{currentfill}{rgb}{0.404421,0.534643,0.932002}%
\pgfsetfillcolor{currentfill}%
\pgfsetlinewidth{0.000000pt}%
\definecolor{currentstroke}{rgb}{0.000000,0.000000,0.000000}%
\pgfsetstrokecolor{currentstroke}%
\pgfsetdash{}{0pt}%
\pgfpathmoveto{\pgfqpoint{5.009462in}{3.111588in}}%
\pgfpathlineto{\pgfqpoint{5.021104in}{3.272922in}}%
\pgfpathlineto{\pgfqpoint{5.029368in}{3.030133in}}%
\pgfpathlineto{\pgfqpoint{5.061620in}{2.895274in}}%
\pgfpathlineto{\pgfqpoint{5.096799in}{3.091714in}}%
\pgfpathlineto{\pgfqpoint{5.085754in}{3.017729in}}%
\pgfpathlineto{\pgfqpoint{5.076048in}{3.088881in}}%
\pgfpathlineto{\pgfqpoint{5.043204in}{3.150731in}}%
\pgfpathlineto{\pgfqpoint{5.009462in}{3.111588in}}%
\pgfpathclose%
\pgfusepath{fill}%
\end{pgfscope}%
\begin{pgfscope}%
\pgfpathrectangle{\pgfqpoint{1.020000in}{0.880000in}}{\pgfqpoint{6.160000in}{6.160000in}}%
\pgfusepath{clip}%
\pgfsetbuttcap%
\pgfsetroundjoin%
\definecolor{currentfill}{rgb}{0.959518,0.766973,0.674145}%
\pgfsetfillcolor{currentfill}%
\pgfsetlinewidth{0.000000pt}%
\definecolor{currentstroke}{rgb}{0.000000,0.000000,0.000000}%
\pgfsetstrokecolor{currentstroke}%
\pgfsetdash{}{0pt}%
\pgfpathmoveto{\pgfqpoint{2.726984in}{4.307682in}}%
\pgfpathlineto{\pgfqpoint{2.735240in}{4.279389in}}%
\pgfpathlineto{\pgfqpoint{2.743230in}{4.270542in}}%
\pgfpathlineto{\pgfqpoint{2.778201in}{4.210617in}}%
\pgfpathlineto{\pgfqpoint{2.811957in}{4.235564in}}%
\pgfpathlineto{\pgfqpoint{2.803015in}{4.309643in}}%
\pgfpathlineto{\pgfqpoint{2.794120in}{4.380660in}}%
\pgfpathlineto{\pgfqpoint{2.760560in}{4.343433in}}%
\pgfpathlineto{\pgfqpoint{2.726984in}{4.307682in}}%
\pgfpathclose%
\pgfusepath{fill}%
\end{pgfscope}%
\begin{pgfscope}%
\pgfpathrectangle{\pgfqpoint{1.020000in}{0.880000in}}{\pgfqpoint{6.160000in}{6.160000in}}%
\pgfusepath{clip}%
\pgfsetbuttcap%
\pgfsetroundjoin%
\definecolor{currentfill}{rgb}{0.800601,0.850358,0.930008}%
\pgfsetfillcolor{currentfill}%
\pgfsetlinewidth{0.000000pt}%
\definecolor{currentstroke}{rgb}{0.000000,0.000000,0.000000}%
\pgfsetstrokecolor{currentstroke}%
\pgfsetdash{}{0pt}%
\pgfpathmoveto{\pgfqpoint{3.494223in}{3.807214in}}%
\pgfpathlineto{\pgfqpoint{3.502409in}{3.876919in}}%
\pgfpathlineto{\pgfqpoint{3.512339in}{3.690824in}}%
\pgfpathlineto{\pgfqpoint{3.545880in}{3.742977in}}%
\pgfpathlineto{\pgfqpoint{3.579943in}{3.710837in}}%
\pgfpathlineto{\pgfqpoint{3.570525in}{3.822496in}}%
\pgfpathlineto{\pgfqpoint{3.560873in}{3.971908in}}%
\pgfpathlineto{\pgfqpoint{3.527284in}{3.927187in}}%
\pgfpathlineto{\pgfqpoint{3.494223in}{3.807214in}}%
\pgfpathclose%
\pgfusepath{fill}%
\end{pgfscope}%
\begin{pgfscope}%
\pgfpathrectangle{\pgfqpoint{1.020000in}{0.880000in}}{\pgfqpoint{6.160000in}{6.160000in}}%
\pgfusepath{clip}%
\pgfsetbuttcap%
\pgfsetroundjoin%
\definecolor{currentfill}{rgb}{0.947345,0.794696,0.716991}%
\pgfsetfillcolor{currentfill}%
\pgfsetlinewidth{0.000000pt}%
\definecolor{currentstroke}{rgb}{0.000000,0.000000,0.000000}%
\pgfsetstrokecolor{currentstroke}%
\pgfsetdash{}{0pt}%
\pgfpathmoveto{\pgfqpoint{3.168953in}{4.186879in}}%
\pgfpathlineto{\pgfqpoint{3.176796in}{4.234092in}}%
\pgfpathlineto{\pgfqpoint{3.184404in}{4.307576in}}%
\pgfpathlineto{\pgfqpoint{3.219738in}{4.172335in}}%
\pgfpathlineto{\pgfqpoint{3.254946in}{4.038657in}}%
\pgfpathlineto{\pgfqpoint{3.244744in}{4.233630in}}%
\pgfpathlineto{\pgfqpoint{3.236292in}{4.240768in}}%
\pgfpathlineto{\pgfqpoint{3.201029in}{4.377727in}}%
\pgfpathlineto{\pgfqpoint{3.168953in}{4.186879in}}%
\pgfpathclose%
\pgfusepath{fill}%
\end{pgfscope}%
\begin{pgfscope}%
\pgfpathrectangle{\pgfqpoint{1.020000in}{0.880000in}}{\pgfqpoint{6.160000in}{6.160000in}}%
\pgfusepath{clip}%
\pgfsetbuttcap%
\pgfsetroundjoin%
\definecolor{currentfill}{rgb}{0.966962,0.735670,0.630877}%
\pgfsetfillcolor{currentfill}%
\pgfsetlinewidth{0.000000pt}%
\definecolor{currentstroke}{rgb}{0.000000,0.000000,0.000000}%
\pgfsetstrokecolor{currentstroke}%
\pgfsetdash{}{0pt}%
\pgfpathmoveto{\pgfqpoint{2.794120in}{4.380660in}}%
\pgfpathlineto{\pgfqpoint{2.803015in}{4.309643in}}%
\pgfpathlineto{\pgfqpoint{2.811957in}{4.235564in}}%
\pgfpathlineto{\pgfqpoint{2.846656in}{4.189106in}}%
\pgfpathlineto{\pgfqpoint{2.878100in}{4.390495in}}%
\pgfpathlineto{\pgfqpoint{2.871527in}{4.280113in}}%
\pgfpathlineto{\pgfqpoint{2.859376in}{4.598612in}}%
\pgfpathlineto{\pgfqpoint{2.826953in}{4.472373in}}%
\pgfpathlineto{\pgfqpoint{2.794120in}{4.380660in}}%
\pgfpathclose%
\pgfusepath{fill}%
\end{pgfscope}%
\begin{pgfscope}%
\pgfpathrectangle{\pgfqpoint{1.020000in}{0.880000in}}{\pgfqpoint{6.160000in}{6.160000in}}%
\pgfusepath{clip}%
\pgfsetbuttcap%
\pgfsetroundjoin%
\definecolor{currentfill}{rgb}{0.954566,0.779055,0.692531}%
\pgfsetfillcolor{currentfill}%
\pgfsetlinewidth{0.000000pt}%
\definecolor{currentstroke}{rgb}{0.000000,0.000000,0.000000}%
\pgfsetstrokecolor{currentstroke}%
\pgfsetdash{}{0pt}%
\pgfpathmoveto{\pgfqpoint{2.946523in}{4.369823in}}%
\pgfpathlineto{\pgfqpoint{2.956710in}{4.197612in}}%
\pgfpathlineto{\pgfqpoint{2.964965in}{4.184428in}}%
\pgfpathlineto{\pgfqpoint{2.998701in}{4.211339in}}%
\pgfpathlineto{\pgfqpoint{3.033847in}{4.114579in}}%
\pgfpathlineto{\pgfqpoint{3.022646in}{4.379495in}}%
\pgfpathlineto{\pgfqpoint{3.015395in}{4.300993in}}%
\pgfpathlineto{\pgfqpoint{2.980755in}{4.354279in}}%
\pgfpathlineto{\pgfqpoint{2.946523in}{4.369823in}}%
\pgfpathclose%
\pgfusepath{fill}%
\end{pgfscope}%
\begin{pgfscope}%
\pgfpathrectangle{\pgfqpoint{1.020000in}{0.880000in}}{\pgfqpoint{6.160000in}{6.160000in}}%
\pgfusepath{clip}%
\pgfsetbuttcap%
\pgfsetroundjoin%
\definecolor{currentfill}{rgb}{0.328604,0.439712,0.869587}%
\pgfsetfillcolor{currentfill}%
\pgfsetlinewidth{0.000000pt}%
\definecolor{currentstroke}{rgb}{0.000000,0.000000,0.000000}%
\pgfsetstrokecolor{currentstroke}%
\pgfsetdash{}{0pt}%
\pgfpathmoveto{\pgfqpoint{5.603753in}{2.901988in}}%
\pgfpathlineto{\pgfqpoint{5.614874in}{2.915528in}}%
\pgfpathlineto{\pgfqpoint{5.626118in}{2.935755in}}%
\pgfpathlineto{\pgfqpoint{5.659508in}{2.942662in}}%
\pgfpathlineto{\pgfqpoint{5.693187in}{2.969708in}}%
\pgfpathlineto{\pgfqpoint{5.680787in}{2.878205in}}%
\pgfpathlineto{\pgfqpoint{5.670998in}{2.958376in}}%
\pgfpathlineto{\pgfqpoint{5.637304in}{2.925249in}}%
\pgfpathlineto{\pgfqpoint{5.603753in}{2.901988in}}%
\pgfpathclose%
\pgfusepath{fill}%
\end{pgfscope}%
\begin{pgfscope}%
\pgfpathrectangle{\pgfqpoint{1.020000in}{0.880000in}}{\pgfqpoint{6.160000in}{6.160000in}}%
\pgfusepath{clip}%
\pgfsetbuttcap%
\pgfsetroundjoin%
\definecolor{currentfill}{rgb}{0.902849,0.844796,0.811970}%
\pgfsetfillcolor{currentfill}%
\pgfsetlinewidth{0.000000pt}%
\definecolor{currentstroke}{rgb}{0.000000,0.000000,0.000000}%
\pgfsetstrokecolor{currentstroke}%
\pgfsetdash{}{0pt}%
\pgfpathmoveto{\pgfqpoint{2.458012in}{4.051250in}}%
\pgfpathlineto{\pgfqpoint{2.466350in}{4.005519in}}%
\pgfpathlineto{\pgfqpoint{2.470786in}{4.189962in}}%
\pgfpathlineto{\pgfqpoint{2.508390in}{3.989789in}}%
\pgfpathlineto{\pgfqpoint{2.540914in}{4.091462in}}%
\pgfpathlineto{\pgfqpoint{2.534983in}{3.986856in}}%
\pgfpathlineto{\pgfqpoint{2.523945in}{4.194375in}}%
\pgfpathlineto{\pgfqpoint{2.492680in}{4.020714in}}%
\pgfpathlineto{\pgfqpoint{2.458012in}{4.051250in}}%
\pgfpathclose%
\pgfusepath{fill}%
\end{pgfscope}%
\begin{pgfscope}%
\pgfpathrectangle{\pgfqpoint{1.020000in}{0.880000in}}{\pgfqpoint{6.160000in}{6.160000in}}%
\pgfusepath{clip}%
\pgfsetbuttcap%
\pgfsetroundjoin%
\definecolor{currentfill}{rgb}{0.613933,0.739923,0.999142}%
\pgfsetfillcolor{currentfill}%
\pgfsetlinewidth{0.000000pt}%
\definecolor{currentstroke}{rgb}{0.000000,0.000000,0.000000}%
\pgfsetstrokecolor{currentstroke}%
\pgfsetdash{}{0pt}%
\pgfpathmoveto{\pgfqpoint{4.107578in}{3.593420in}}%
\pgfpathlineto{\pgfqpoint{4.117015in}{3.431910in}}%
\pgfpathlineto{\pgfqpoint{4.126334in}{3.460088in}}%
\pgfpathlineto{\pgfqpoint{4.160098in}{3.352569in}}%
\pgfpathlineto{\pgfqpoint{4.193799in}{3.520851in}}%
\pgfpathlineto{\pgfqpoint{4.184389in}{3.478034in}}%
\pgfpathlineto{\pgfqpoint{4.175013in}{3.366225in}}%
\pgfpathlineto{\pgfqpoint{4.141360in}{3.440174in}}%
\pgfpathlineto{\pgfqpoint{4.107578in}{3.593420in}}%
\pgfpathclose%
\pgfusepath{fill}%
\end{pgfscope}%
\begin{pgfscope}%
\pgfpathrectangle{\pgfqpoint{1.020000in}{0.880000in}}{\pgfqpoint{6.160000in}{6.160000in}}%
\pgfusepath{clip}%
\pgfsetbuttcap%
\pgfsetroundjoin%
\definecolor{currentfill}{rgb}{0.318832,0.426605,0.859857}%
\pgfsetfillcolor{currentfill}%
\pgfsetlinewidth{0.000000pt}%
\definecolor{currentstroke}{rgb}{0.000000,0.000000,0.000000}%
\pgfsetstrokecolor{currentstroke}%
\pgfsetdash{}{0pt}%
\pgfpathmoveto{\pgfqpoint{5.826140in}{2.972039in}}%
\pgfpathlineto{\pgfqpoint{5.836457in}{2.920388in}}%
\pgfpathlineto{\pgfqpoint{5.846485in}{2.850680in}}%
\pgfpathlineto{\pgfqpoint{5.879658in}{2.848851in}}%
\pgfpathlineto{\pgfqpoint{5.914059in}{2.918863in}}%
\pgfpathlineto{\pgfqpoint{5.903047in}{2.933141in}}%
\pgfpathlineto{\pgfqpoint{5.893201in}{3.014148in}}%
\pgfpathlineto{\pgfqpoint{5.857017in}{2.835311in}}%
\pgfpathlineto{\pgfqpoint{5.826140in}{2.972039in}}%
\pgfpathclose%
\pgfusepath{fill}%
\end{pgfscope}%
\begin{pgfscope}%
\pgfpathrectangle{\pgfqpoint{1.020000in}{0.880000in}}{\pgfqpoint{6.160000in}{6.160000in}}%
\pgfusepath{clip}%
\pgfsetbuttcap%
\pgfsetroundjoin%
\definecolor{currentfill}{rgb}{0.383662,0.510183,0.917831}%
\pgfsetfillcolor{currentfill}%
\pgfsetlinewidth{0.000000pt}%
\definecolor{currentstroke}{rgb}{0.000000,0.000000,0.000000}%
\pgfsetstrokecolor{currentstroke}%
\pgfsetdash{}{0pt}%
\pgfpathmoveto{\pgfqpoint{5.451209in}{3.060680in}}%
\pgfpathlineto{\pgfqpoint{5.461987in}{3.061622in}}%
\pgfpathlineto{\pgfqpoint{5.471394in}{2.954763in}}%
\pgfpathlineto{\pgfqpoint{5.505177in}{2.990521in}}%
\pgfpathlineto{\pgfqpoint{5.538959in}{3.025885in}}%
\pgfpathlineto{\pgfqpoint{5.528423in}{3.049199in}}%
\pgfpathlineto{\pgfqpoint{5.518262in}{3.099143in}}%
\pgfpathlineto{\pgfqpoint{5.484886in}{3.091153in}}%
\pgfpathlineto{\pgfqpoint{5.451209in}{3.060680in}}%
\pgfpathclose%
\pgfusepath{fill}%
\end{pgfscope}%
\begin{pgfscope}%
\pgfpathrectangle{\pgfqpoint{1.020000in}{0.880000in}}{\pgfqpoint{6.160000in}{6.160000in}}%
\pgfusepath{clip}%
\pgfsetbuttcap%
\pgfsetroundjoin%
\definecolor{currentfill}{rgb}{0.462354,0.599830,0.965857}%
\pgfsetfillcolor{currentfill}%
\pgfsetlinewidth{0.000000pt}%
\definecolor{currentstroke}{rgb}{0.000000,0.000000,0.000000}%
\pgfsetstrokecolor{currentstroke}%
\pgfsetdash{}{0pt}%
\pgfpathmoveto{\pgfqpoint{4.568231in}{3.252474in}}%
\pgfpathlineto{\pgfqpoint{4.577549in}{3.125720in}}%
\pgfpathlineto{\pgfqpoint{4.587701in}{3.205729in}}%
\pgfpathlineto{\pgfqpoint{4.620956in}{3.124479in}}%
\pgfpathlineto{\pgfqpoint{4.654945in}{3.214310in}}%
\pgfpathlineto{\pgfqpoint{4.644942in}{3.192712in}}%
\pgfpathlineto{\pgfqpoint{4.635211in}{3.225584in}}%
\pgfpathlineto{\pgfqpoint{4.601241in}{3.121151in}}%
\pgfpathlineto{\pgfqpoint{4.568231in}{3.252474in}}%
\pgfpathclose%
\pgfusepath{fill}%
\end{pgfscope}%
\begin{pgfscope}%
\pgfpathrectangle{\pgfqpoint{1.020000in}{0.880000in}}{\pgfqpoint{6.160000in}{6.160000in}}%
\pgfusepath{clip}%
\pgfsetbuttcap%
\pgfsetroundjoin%
\definecolor{currentfill}{rgb}{0.855378,0.863778,0.876587}%
\pgfsetfillcolor{currentfill}%
\pgfsetlinewidth{0.000000pt}%
\definecolor{currentstroke}{rgb}{0.000000,0.000000,0.000000}%
\pgfsetstrokecolor{currentstroke}%
\pgfsetdash{}{0pt}%
\pgfpathmoveto{\pgfqpoint{3.339549in}{4.057424in}}%
\pgfpathlineto{\pgfqpoint{3.349788in}{3.849027in}}%
\pgfpathlineto{\pgfqpoint{3.358431in}{3.832884in}}%
\pgfpathlineto{\pgfqpoint{3.391670in}{3.921075in}}%
\pgfpathlineto{\pgfqpoint{3.426043in}{3.864146in}}%
\pgfpathlineto{\pgfqpoint{3.416620in}{3.974937in}}%
\pgfpathlineto{\pgfqpoint{3.407626in}{4.029976in}}%
\pgfpathlineto{\pgfqpoint{3.373888in}{4.008183in}}%
\pgfpathlineto{\pgfqpoint{3.339549in}{4.057424in}}%
\pgfpathclose%
\pgfusepath{fill}%
\end{pgfscope}%
\begin{pgfscope}%
\pgfpathrectangle{\pgfqpoint{1.020000in}{0.880000in}}{\pgfqpoint{6.160000in}{6.160000in}}%
\pgfusepath{clip}%
\pgfsetbuttcap%
\pgfsetroundjoin%
\definecolor{currentfill}{rgb}{0.916071,0.833977,0.788693}%
\pgfsetfillcolor{currentfill}%
\pgfsetlinewidth{0.000000pt}%
\definecolor{currentstroke}{rgb}{0.000000,0.000000,0.000000}%
\pgfsetstrokecolor{currentstroke}%
\pgfsetdash{}{0pt}%
\pgfpathmoveto{\pgfqpoint{3.254946in}{4.038657in}}%
\pgfpathlineto{\pgfqpoint{3.262121in}{4.173157in}}%
\pgfpathlineto{\pgfqpoint{3.271870in}{4.027891in}}%
\pgfpathlineto{\pgfqpoint{3.304478in}{4.184451in}}%
\pgfpathlineto{\pgfqpoint{3.339549in}{4.057424in}}%
\pgfpathlineto{\pgfqpoint{3.330322in}{4.144233in}}%
\pgfpathlineto{\pgfqpoint{3.322580in}{4.057829in}}%
\pgfpathlineto{\pgfqpoint{3.288113in}{4.122102in}}%
\pgfpathlineto{\pgfqpoint{3.254946in}{4.038657in}}%
\pgfpathclose%
\pgfusepath{fill}%
\end{pgfscope}%
\begin{pgfscope}%
\pgfpathrectangle{\pgfqpoint{1.020000in}{0.880000in}}{\pgfqpoint{6.160000in}{6.160000in}}%
\pgfusepath{clip}%
\pgfsetbuttcap%
\pgfsetroundjoin%
\definecolor{currentfill}{rgb}{0.516260,0.654498,0.986407}%
\pgfsetfillcolor{currentfill}%
\pgfsetlinewidth{0.000000pt}%
\definecolor{currentstroke}{rgb}{0.000000,0.000000,0.000000}%
\pgfsetstrokecolor{currentstroke}%
\pgfsetdash{}{0pt}%
\pgfpathmoveto{\pgfqpoint{4.414747in}{3.370327in}}%
\pgfpathlineto{\pgfqpoint{4.424362in}{3.352868in}}%
\pgfpathlineto{\pgfqpoint{4.433583in}{3.171769in}}%
\pgfpathlineto{\pgfqpoint{4.467369in}{3.232034in}}%
\pgfpathlineto{\pgfqpoint{4.501185in}{3.295314in}}%
\pgfpathlineto{\pgfqpoint{4.491507in}{3.314363in}}%
\pgfpathlineto{\pgfqpoint{4.481518in}{3.223496in}}%
\pgfpathlineto{\pgfqpoint{4.448192in}{3.306241in}}%
\pgfpathlineto{\pgfqpoint{4.414747in}{3.370327in}}%
\pgfpathclose%
\pgfusepath{fill}%
\end{pgfscope}%
\begin{pgfscope}%
\pgfpathrectangle{\pgfqpoint{1.020000in}{0.880000in}}{\pgfqpoint{6.160000in}{6.160000in}}%
\pgfusepath{clip}%
\pgfsetbuttcap%
\pgfsetroundjoin%
\definecolor{currentfill}{rgb}{0.570616,0.704109,0.997195}%
\pgfsetfillcolor{currentfill}%
\pgfsetlinewidth{0.000000pt}%
\definecolor{currentstroke}{rgb}{0.000000,0.000000,0.000000}%
\pgfsetstrokecolor{currentstroke}%
\pgfsetdash{}{0pt}%
\pgfpathmoveto{\pgfqpoint{4.261094in}{3.361106in}}%
\pgfpathlineto{\pgfqpoint{4.270693in}{3.495527in}}%
\pgfpathlineto{\pgfqpoint{4.279966in}{3.245493in}}%
\pgfpathlineto{\pgfqpoint{4.313835in}{3.411150in}}%
\pgfpathlineto{\pgfqpoint{4.347509in}{3.401636in}}%
\pgfpathlineto{\pgfqpoint{4.338109in}{3.513035in}}%
\pgfpathlineto{\pgfqpoint{4.328140in}{3.233481in}}%
\pgfpathlineto{\pgfqpoint{4.294735in}{3.374307in}}%
\pgfpathlineto{\pgfqpoint{4.261094in}{3.361106in}}%
\pgfpathclose%
\pgfusepath{fill}%
\end{pgfscope}%
\begin{pgfscope}%
\pgfpathrectangle{\pgfqpoint{1.020000in}{0.880000in}}{\pgfqpoint{6.160000in}{6.160000in}}%
\pgfusepath{clip}%
\pgfsetbuttcap%
\pgfsetroundjoin%
\definecolor{currentfill}{rgb}{0.945540,0.798606,0.723105}%
\pgfsetfillcolor{currentfill}%
\pgfsetlinewidth{0.000000pt}%
\definecolor{currentstroke}{rgb}{0.000000,0.000000,0.000000}%
\pgfsetstrokecolor{currentstroke}%
\pgfsetdash{}{0pt}%
\pgfpathmoveto{\pgfqpoint{2.592096in}{4.204154in}}%
\pgfpathlineto{\pgfqpoint{2.597749in}{4.334815in}}%
\pgfpathlineto{\pgfqpoint{2.607806in}{4.186104in}}%
\pgfpathlineto{\pgfqpoint{2.642667in}{4.142125in}}%
\pgfpathlineto{\pgfqpoint{2.676569in}{4.158879in}}%
\pgfpathlineto{\pgfqpoint{2.669291in}{4.125030in}}%
\pgfpathlineto{\pgfqpoint{2.659395in}{4.265861in}}%
\pgfpathlineto{\pgfqpoint{2.624817in}{4.295082in}}%
\pgfpathlineto{\pgfqpoint{2.592096in}{4.204154in}}%
\pgfpathclose%
\pgfusepath{fill}%
\end{pgfscope}%
\begin{pgfscope}%
\pgfpathrectangle{\pgfqpoint{1.020000in}{0.880000in}}{\pgfqpoint{6.160000in}{6.160000in}}%
\pgfusepath{clip}%
\pgfsetbuttcap%
\pgfsetroundjoin%
\definecolor{currentfill}{rgb}{0.646113,0.764436,0.996868}%
\pgfsetfillcolor{currentfill}%
\pgfsetlinewidth{0.000000pt}%
\definecolor{currentstroke}{rgb}{0.000000,0.000000,0.000000}%
\pgfsetstrokecolor{currentstroke}%
\pgfsetdash{}{0pt}%
\pgfpathmoveto{\pgfqpoint{3.886705in}{3.574201in}}%
\pgfpathlineto{\pgfqpoint{3.895974in}{3.511165in}}%
\pgfpathlineto{\pgfqpoint{3.905216in}{3.460452in}}%
\pgfpathlineto{\pgfqpoint{3.939111in}{3.419102in}}%
\pgfpathlineto{\pgfqpoint{3.972393in}{3.646659in}}%
\pgfpathlineto{\pgfqpoint{3.963364in}{3.577050in}}%
\pgfpathlineto{\pgfqpoint{3.954358in}{3.512202in}}%
\pgfpathlineto{\pgfqpoint{3.920831in}{3.436909in}}%
\pgfpathlineto{\pgfqpoint{3.886705in}{3.574201in}}%
\pgfpathclose%
\pgfusepath{fill}%
\end{pgfscope}%
\begin{pgfscope}%
\pgfpathrectangle{\pgfqpoint{1.020000in}{0.880000in}}{\pgfqpoint{6.160000in}{6.160000in}}%
\pgfusepath{clip}%
\pgfsetbuttcap%
\pgfsetroundjoin%
\definecolor{currentfill}{rgb}{0.733898,0.820018,0.970724}%
\pgfsetfillcolor{currentfill}%
\pgfsetlinewidth{0.000000pt}%
\definecolor{currentstroke}{rgb}{0.000000,0.000000,0.000000}%
\pgfsetstrokecolor{currentstroke}%
\pgfsetdash{}{0pt}%
\pgfpathmoveto{\pgfqpoint{3.801287in}{3.532690in}}%
\pgfpathlineto{\pgfqpoint{3.809473in}{3.745439in}}%
\pgfpathlineto{\pgfqpoint{3.818660in}{3.698910in}}%
\pgfpathlineto{\pgfqpoint{3.852274in}{3.769872in}}%
\pgfpathlineto{\pgfqpoint{3.886705in}{3.574201in}}%
\pgfpathlineto{\pgfqpoint{3.877339in}{3.671533in}}%
\pgfpathlineto{\pgfqpoint{3.868232in}{3.687176in}}%
\pgfpathlineto{\pgfqpoint{3.834257in}{3.749962in}}%
\pgfpathlineto{\pgfqpoint{3.801287in}{3.532690in}}%
\pgfpathclose%
\pgfusepath{fill}%
\end{pgfscope}%
\begin{pgfscope}%
\pgfpathrectangle{\pgfqpoint{1.020000in}{0.880000in}}{\pgfqpoint{6.160000in}{6.160000in}}%
\pgfusepath{clip}%
\pgfsetbuttcap%
\pgfsetroundjoin%
\definecolor{currentfill}{rgb}{0.940879,0.805596,0.735167}%
\pgfsetfillcolor{currentfill}%
\pgfsetlinewidth{0.000000pt}%
\definecolor{currentstroke}{rgb}{0.000000,0.000000,0.000000}%
\pgfsetstrokecolor{currentstroke}%
\pgfsetdash{}{0pt}%
\pgfpathmoveto{\pgfqpoint{2.811957in}{4.235564in}}%
\pgfpathlineto{\pgfqpoint{2.821502in}{4.117291in}}%
\pgfpathlineto{\pgfqpoint{2.829676in}{4.100867in}}%
\pgfpathlineto{\pgfqpoint{2.862297in}{4.214101in}}%
\pgfpathlineto{\pgfqpoint{2.897789in}{4.106878in}}%
\pgfpathlineto{\pgfqpoint{2.888962in}{4.169695in}}%
\pgfpathlineto{\pgfqpoint{2.878100in}{4.390495in}}%
\pgfpathlineto{\pgfqpoint{2.846656in}{4.189106in}}%
\pgfpathlineto{\pgfqpoint{2.811957in}{4.235564in}}%
\pgfpathclose%
\pgfusepath{fill}%
\end{pgfscope}%
\begin{pgfscope}%
\pgfpathrectangle{\pgfqpoint{1.020000in}{0.880000in}}{\pgfqpoint{6.160000in}{6.160000in}}%
\pgfusepath{clip}%
\pgfsetbuttcap%
\pgfsetroundjoin%
\definecolor{currentfill}{rgb}{0.800601,0.850358,0.930008}%
\pgfsetfillcolor{currentfill}%
\pgfsetlinewidth{0.000000pt}%
\definecolor{currentstroke}{rgb}{0.000000,0.000000,0.000000}%
\pgfsetstrokecolor{currentstroke}%
\pgfsetdash{}{0pt}%
\pgfpathmoveto{\pgfqpoint{3.426043in}{3.864146in}}%
\pgfpathlineto{\pgfqpoint{3.434491in}{3.884234in}}%
\pgfpathlineto{\pgfqpoint{3.443131in}{3.881089in}}%
\pgfpathlineto{\pgfqpoint{3.477300in}{3.853882in}}%
\pgfpathlineto{\pgfqpoint{3.512339in}{3.690824in}}%
\pgfpathlineto{\pgfqpoint{3.502409in}{3.876919in}}%
\pgfpathlineto{\pgfqpoint{3.494223in}{3.807214in}}%
\pgfpathlineto{\pgfqpoint{3.461423in}{3.659220in}}%
\pgfpathlineto{\pgfqpoint{3.426043in}{3.864146in}}%
\pgfpathclose%
\pgfusepath{fill}%
\end{pgfscope}%
\begin{pgfscope}%
\pgfpathrectangle{\pgfqpoint{1.020000in}{0.880000in}}{\pgfqpoint{6.160000in}{6.160000in}}%
\pgfusepath{clip}%
\pgfsetbuttcap%
\pgfsetroundjoin%
\definecolor{currentfill}{rgb}{0.373552,0.497499,0.909467}%
\pgfsetfillcolor{currentfill}%
\pgfsetlinewidth{0.000000pt}%
\definecolor{currentstroke}{rgb}{0.000000,0.000000,0.000000}%
\pgfsetstrokecolor{currentstroke}%
\pgfsetdash{}{0pt}%
\pgfpathmoveto{\pgfqpoint{5.383167in}{2.942511in}}%
\pgfpathlineto{\pgfqpoint{5.395613in}{3.086959in}}%
\pgfpathlineto{\pgfqpoint{5.406333in}{3.086461in}}%
\pgfpathlineto{\pgfqpoint{5.438413in}{2.982309in}}%
\pgfpathlineto{\pgfqpoint{5.471394in}{2.954763in}}%
\pgfpathlineto{\pgfqpoint{5.461987in}{3.061622in}}%
\pgfpathlineto{\pgfqpoint{5.451209in}{3.060680in}}%
\pgfpathlineto{\pgfqpoint{5.416667in}{2.960346in}}%
\pgfpathlineto{\pgfqpoint{5.383167in}{2.942511in}}%
\pgfpathclose%
\pgfusepath{fill}%
\end{pgfscope}%
\begin{pgfscope}%
\pgfpathrectangle{\pgfqpoint{1.020000in}{0.880000in}}{\pgfqpoint{6.160000in}{6.160000in}}%
\pgfusepath{clip}%
\pgfsetbuttcap%
\pgfsetroundjoin%
\definecolor{currentfill}{rgb}{0.938326,0.808917,0.741162}%
\pgfsetfillcolor{currentfill}%
\pgfsetlinewidth{0.000000pt}%
\definecolor{currentstroke}{rgb}{0.000000,0.000000,0.000000}%
\pgfsetstrokecolor{currentstroke}%
\pgfsetdash{}{0pt}%
\pgfpathmoveto{\pgfqpoint{2.523945in}{4.194375in}}%
\pgfpathlineto{\pgfqpoint{2.534983in}{3.986856in}}%
\pgfpathlineto{\pgfqpoint{2.540914in}{4.091462in}}%
\pgfpathlineto{\pgfqpoint{2.574856in}{4.107383in}}%
\pgfpathlineto{\pgfqpoint{2.607806in}{4.186104in}}%
\pgfpathlineto{\pgfqpoint{2.597749in}{4.334815in}}%
\pgfpathlineto{\pgfqpoint{2.592096in}{4.204154in}}%
\pgfpathlineto{\pgfqpoint{2.556009in}{4.324489in}}%
\pgfpathlineto{\pgfqpoint{2.523945in}{4.194375in}}%
\pgfpathclose%
\pgfusepath{fill}%
\end{pgfscope}%
\begin{pgfscope}%
\pgfpathrectangle{\pgfqpoint{1.020000in}{0.880000in}}{\pgfqpoint{6.160000in}{6.160000in}}%
\pgfusepath{clip}%
\pgfsetbuttcap%
\pgfsetroundjoin%
\definecolor{currentfill}{rgb}{0.399231,0.528528,0.928459}%
\pgfsetfillcolor{currentfill}%
\pgfsetlinewidth{0.000000pt}%
\definecolor{currentstroke}{rgb}{0.000000,0.000000,0.000000}%
\pgfsetstrokecolor{currentstroke}%
\pgfsetdash{}{0pt}%
\pgfpathmoveto{\pgfqpoint{5.162385in}{2.965567in}}%
\pgfpathlineto{\pgfqpoint{5.173971in}{3.079984in}}%
\pgfpathlineto{\pgfqpoint{5.184678in}{3.102211in}}%
\pgfpathlineto{\pgfqpoint{5.218807in}{3.167327in}}%
\pgfpathlineto{\pgfqpoint{5.250521in}{3.005000in}}%
\pgfpathlineto{\pgfqpoint{5.239967in}{3.003744in}}%
\pgfpathlineto{\pgfqpoint{5.229519in}{3.010464in}}%
\pgfpathlineto{\pgfqpoint{5.197349in}{3.125112in}}%
\pgfpathlineto{\pgfqpoint{5.162385in}{2.965567in}}%
\pgfpathclose%
\pgfusepath{fill}%
\end{pgfscope}%
\begin{pgfscope}%
\pgfpathrectangle{\pgfqpoint{1.020000in}{0.880000in}}{\pgfqpoint{6.160000in}{6.160000in}}%
\pgfusepath{clip}%
\pgfsetbuttcap%
\pgfsetroundjoin%
\definecolor{currentfill}{rgb}{0.419991,0.552989,0.942630}%
\pgfsetfillcolor{currentfill}%
\pgfsetlinewidth{0.000000pt}%
\definecolor{currentstroke}{rgb}{0.000000,0.000000,0.000000}%
\pgfsetstrokecolor{currentstroke}%
\pgfsetdash{}{0pt}%
\pgfpathmoveto{\pgfqpoint{4.941733in}{2.998759in}}%
\pgfpathlineto{\pgfqpoint{4.952895in}{3.122040in}}%
\pgfpathlineto{\pgfqpoint{4.962250in}{3.008271in}}%
\pgfpathlineto{\pgfqpoint{4.997278in}{3.198431in}}%
\pgfpathlineto{\pgfqpoint{5.029368in}{3.030133in}}%
\pgfpathlineto{\pgfqpoint{5.021104in}{3.272922in}}%
\pgfpathlineto{\pgfqpoint{5.009462in}{3.111588in}}%
\pgfpathlineto{\pgfqpoint{4.975430in}{3.035262in}}%
\pgfpathlineto{\pgfqpoint{4.941733in}{2.998759in}}%
\pgfpathclose%
\pgfusepath{fill}%
\end{pgfscope}%
\begin{pgfscope}%
\pgfpathrectangle{\pgfqpoint{1.020000in}{0.880000in}}{\pgfqpoint{6.160000in}{6.160000in}}%
\pgfusepath{clip}%
\pgfsetbuttcap%
\pgfsetroundjoin%
\definecolor{currentfill}{rgb}{0.733898,0.820018,0.970724}%
\pgfsetfillcolor{currentfill}%
\pgfsetlinewidth{0.000000pt}%
\definecolor{currentstroke}{rgb}{0.000000,0.000000,0.000000}%
\pgfsetstrokecolor{currentstroke}%
\pgfsetdash{}{0pt}%
\pgfpathmoveto{\pgfqpoint{3.512339in}{3.690824in}}%
\pgfpathlineto{\pgfqpoint{3.520421in}{3.782230in}}%
\pgfpathlineto{\pgfqpoint{3.529858in}{3.671285in}}%
\pgfpathlineto{\pgfqpoint{3.563394in}{3.735488in}}%
\pgfpathlineto{\pgfqpoint{3.598404in}{3.551520in}}%
\pgfpathlineto{\pgfqpoint{3.589591in}{3.560324in}}%
\pgfpathlineto{\pgfqpoint{3.579943in}{3.710837in}}%
\pgfpathlineto{\pgfqpoint{3.545880in}{3.742977in}}%
\pgfpathlineto{\pgfqpoint{3.512339in}{3.690824in}}%
\pgfpathclose%
\pgfusepath{fill}%
\end{pgfscope}%
\begin{pgfscope}%
\pgfpathrectangle{\pgfqpoint{1.020000in}{0.880000in}}{\pgfqpoint{6.160000in}{6.160000in}}%
\pgfusepath{clip}%
\pgfsetbuttcap%
\pgfsetroundjoin%
\definecolor{currentfill}{rgb}{0.958176,0.771234,0.680301}%
\pgfsetfillcolor{currentfill}%
\pgfsetlinewidth{0.000000pt}%
\definecolor{currentstroke}{rgb}{0.000000,0.000000,0.000000}%
\pgfsetstrokecolor{currentstroke}%
\pgfsetdash{}{0pt}%
\pgfpathmoveto{\pgfqpoint{2.878100in}{4.390495in}}%
\pgfpathlineto{\pgfqpoint{2.888962in}{4.169695in}}%
\pgfpathlineto{\pgfqpoint{2.897789in}{4.106878in}}%
\pgfpathlineto{\pgfqpoint{2.928967in}{4.339300in}}%
\pgfpathlineto{\pgfqpoint{2.964965in}{4.184428in}}%
\pgfpathlineto{\pgfqpoint{2.956710in}{4.197612in}}%
\pgfpathlineto{\pgfqpoint{2.946523in}{4.369823in}}%
\pgfpathlineto{\pgfqpoint{2.911503in}{4.445465in}}%
\pgfpathlineto{\pgfqpoint{2.878100in}{4.390495in}}%
\pgfpathclose%
\pgfusepath{fill}%
\end{pgfscope}%
\begin{pgfscope}%
\pgfpathrectangle{\pgfqpoint{1.020000in}{0.880000in}}{\pgfqpoint{6.160000in}{6.160000in}}%
\pgfusepath{clip}%
\pgfsetbuttcap%
\pgfsetroundjoin%
\definecolor{currentfill}{rgb}{0.441123,0.576532,0.954545}%
\pgfsetfillcolor{currentfill}%
\pgfsetlinewidth{0.000000pt}%
\definecolor{currentstroke}{rgb}{0.000000,0.000000,0.000000}%
\pgfsetstrokecolor{currentstroke}%
\pgfsetdash{}{0pt}%
\pgfpathmoveto{\pgfqpoint{4.721416in}{3.092572in}}%
\pgfpathlineto{\pgfqpoint{4.732066in}{3.214664in}}%
\pgfpathlineto{\pgfqpoint{4.741614in}{3.131661in}}%
\pgfpathlineto{\pgfqpoint{4.775912in}{3.253949in}}%
\pgfpathlineto{\pgfqpoint{4.808459in}{3.087243in}}%
\pgfpathlineto{\pgfqpoint{4.798361in}{3.082027in}}%
\pgfpathlineto{\pgfqpoint{4.788893in}{3.176817in}}%
\pgfpathlineto{\pgfqpoint{4.754882in}{3.087867in}}%
\pgfpathlineto{\pgfqpoint{4.721416in}{3.092572in}}%
\pgfpathclose%
\pgfusepath{fill}%
\end{pgfscope}%
\begin{pgfscope}%
\pgfpathrectangle{\pgfqpoint{1.020000in}{0.880000in}}{\pgfqpoint{6.160000in}{6.160000in}}%
\pgfusepath{clip}%
\pgfsetbuttcap%
\pgfsetroundjoin%
\definecolor{currentfill}{rgb}{0.353369,0.472069,0.892570}%
\pgfsetfillcolor{currentfill}%
\pgfsetlinewidth{0.000000pt}%
\definecolor{currentstroke}{rgb}{0.000000,0.000000,0.000000}%
\pgfsetstrokecolor{currentstroke}%
\pgfsetdash{}{0pt}%
\pgfpathmoveto{\pgfqpoint{5.316352in}{2.925152in}}%
\pgfpathlineto{\pgfqpoint{5.326345in}{2.870513in}}%
\pgfpathlineto{\pgfqpoint{5.338009in}{2.958983in}}%
\pgfpathlineto{\pgfqpoint{5.371274in}{2.948557in}}%
\pgfpathlineto{\pgfqpoint{5.406333in}{3.086461in}}%
\pgfpathlineto{\pgfqpoint{5.395613in}{3.086959in}}%
\pgfpathlineto{\pgfqpoint{5.383167in}{2.942511in}}%
\pgfpathlineto{\pgfqpoint{5.350473in}{2.994092in}}%
\pgfpathlineto{\pgfqpoint{5.316352in}{2.925152in}}%
\pgfpathclose%
\pgfusepath{fill}%
\end{pgfscope}%
\begin{pgfscope}%
\pgfpathrectangle{\pgfqpoint{1.020000in}{0.880000in}}{\pgfqpoint{6.160000in}{6.160000in}}%
\pgfusepath{clip}%
\pgfsetbuttcap%
\pgfsetroundjoin%
\definecolor{currentfill}{rgb}{0.909460,0.839386,0.800331}%
\pgfsetfillcolor{currentfill}%
\pgfsetlinewidth{0.000000pt}%
\definecolor{currentstroke}{rgb}{0.000000,0.000000,0.000000}%
\pgfsetstrokecolor{currentstroke}%
\pgfsetdash{}{0pt}%
\pgfpathmoveto{\pgfqpoint{2.390860in}{3.982864in}}%
\pgfpathlineto{\pgfqpoint{2.396133in}{4.107404in}}%
\pgfpathlineto{\pgfqpoint{2.404444in}{4.062072in}}%
\pgfpathlineto{\pgfqpoint{2.438998in}{4.045590in}}%
\pgfpathlineto{\pgfqpoint{2.470786in}{4.189962in}}%
\pgfpathlineto{\pgfqpoint{2.466350in}{4.005519in}}%
\pgfpathlineto{\pgfqpoint{2.458012in}{4.051250in}}%
\pgfpathlineto{\pgfqpoint{2.421706in}{4.173093in}}%
\pgfpathlineto{\pgfqpoint{2.390860in}{3.982864in}}%
\pgfpathclose%
\pgfusepath{fill}%
\end{pgfscope}%
\begin{pgfscope}%
\pgfpathrectangle{\pgfqpoint{1.020000in}{0.880000in}}{\pgfqpoint{6.160000in}{6.160000in}}%
\pgfusepath{clip}%
\pgfsetbuttcap%
\pgfsetroundjoin%
\definecolor{currentfill}{rgb}{0.378598,0.503856,0.913692}%
\pgfsetfillcolor{currentfill}%
\pgfsetlinewidth{0.000000pt}%
\definecolor{currentstroke}{rgb}{0.000000,0.000000,0.000000}%
\pgfsetstrokecolor{currentstroke}%
\pgfsetdash{}{0pt}%
\pgfpathmoveto{\pgfqpoint{5.096799in}{3.091714in}}%
\pgfpathlineto{\pgfqpoint{5.106020in}{2.964932in}}%
\pgfpathlineto{\pgfqpoint{5.116526in}{2.975658in}}%
\pgfpathlineto{\pgfqpoint{5.150525in}{3.032192in}}%
\pgfpathlineto{\pgfqpoint{5.184678in}{3.102211in}}%
\pgfpathlineto{\pgfqpoint{5.173971in}{3.079984in}}%
\pgfpathlineto{\pgfqpoint{5.162385in}{2.965567in}}%
\pgfpathlineto{\pgfqpoint{5.129121in}{2.975799in}}%
\pgfpathlineto{\pgfqpoint{5.096799in}{3.091714in}}%
\pgfpathclose%
\pgfusepath{fill}%
\end{pgfscope}%
\begin{pgfscope}%
\pgfpathrectangle{\pgfqpoint{1.020000in}{0.880000in}}{\pgfqpoint{6.160000in}{6.160000in}}%
\pgfusepath{clip}%
\pgfsetbuttcap%
\pgfsetroundjoin%
\definecolor{currentfill}{rgb}{0.919376,0.831273,0.782874}%
\pgfsetfillcolor{currentfill}%
\pgfsetlinewidth{0.000000pt}%
\definecolor{currentstroke}{rgb}{0.000000,0.000000,0.000000}%
\pgfsetstrokecolor{currentstroke}%
\pgfsetdash{}{0pt}%
\pgfpathmoveto{\pgfqpoint{3.184404in}{4.307576in}}%
\pgfpathlineto{\pgfqpoint{3.194660in}{4.114447in}}%
\pgfpathlineto{\pgfqpoint{3.204429in}{3.969800in}}%
\pgfpathlineto{\pgfqpoint{3.237891in}{4.026397in}}%
\pgfpathlineto{\pgfqpoint{3.271870in}{4.027891in}}%
\pgfpathlineto{\pgfqpoint{3.262121in}{4.173157in}}%
\pgfpathlineto{\pgfqpoint{3.254946in}{4.038657in}}%
\pgfpathlineto{\pgfqpoint{3.219738in}{4.172335in}}%
\pgfpathlineto{\pgfqpoint{3.184404in}{4.307576in}}%
\pgfpathclose%
\pgfusepath{fill}%
\end{pgfscope}%
\begin{pgfscope}%
\pgfpathrectangle{\pgfqpoint{1.020000in}{0.880000in}}{\pgfqpoint{6.160000in}{6.160000in}}%
\pgfusepath{clip}%
\pgfsetbuttcap%
\pgfsetroundjoin%
\definecolor{currentfill}{rgb}{0.353369,0.472069,0.892570}%
\pgfsetfillcolor{currentfill}%
\pgfsetlinewidth{0.000000pt}%
\definecolor{currentstroke}{rgb}{0.000000,0.000000,0.000000}%
\pgfsetstrokecolor{currentstroke}%
\pgfsetdash{}{0pt}%
\pgfpathmoveto{\pgfqpoint{5.029368in}{3.030133in}}%
\pgfpathlineto{\pgfqpoint{5.038881in}{2.935812in}}%
\pgfpathlineto{\pgfqpoint{5.048829in}{2.891195in}}%
\pgfpathlineto{\pgfqpoint{5.083089in}{2.979461in}}%
\pgfpathlineto{\pgfqpoint{5.116526in}{2.975658in}}%
\pgfpathlineto{\pgfqpoint{5.106020in}{2.964932in}}%
\pgfpathlineto{\pgfqpoint{5.096799in}{3.091714in}}%
\pgfpathlineto{\pgfqpoint{5.061620in}{2.895274in}}%
\pgfpathlineto{\pgfqpoint{5.029368in}{3.030133in}}%
\pgfpathclose%
\pgfusepath{fill}%
\end{pgfscope}%
\begin{pgfscope}%
\pgfpathrectangle{\pgfqpoint{1.020000in}{0.880000in}}{\pgfqpoint{6.160000in}{6.160000in}}%
\pgfusepath{clip}%
\pgfsetbuttcap%
\pgfsetroundjoin%
\definecolor{currentfill}{rgb}{0.318832,0.426605,0.859857}%
\pgfsetfillcolor{currentfill}%
\pgfsetlinewidth{0.000000pt}%
\definecolor{currentstroke}{rgb}{0.000000,0.000000,0.000000}%
\pgfsetstrokecolor{currentstroke}%
\pgfsetdash{}{0pt}%
\pgfpathmoveto{\pgfqpoint{5.978623in}{2.819837in}}%
\pgfpathlineto{\pgfqpoint{5.989794in}{2.810689in}}%
\pgfpathlineto{\pgfqpoint{6.004240in}{2.979407in}}%
\pgfpathlineto{\pgfqpoint{6.034977in}{2.845480in}}%
\pgfpathlineto{\pgfqpoint{6.071432in}{3.018897in}}%
\pgfpathlineto{\pgfqpoint{6.059331in}{2.984310in}}%
\pgfpathlineto{\pgfqpoint{6.046618in}{2.915262in}}%
\pgfpathlineto{\pgfqpoint{6.011496in}{2.806763in}}%
\pgfpathlineto{\pgfqpoint{5.978623in}{2.819837in}}%
\pgfpathclose%
\pgfusepath{fill}%
\end{pgfscope}%
\begin{pgfscope}%
\pgfpathrectangle{\pgfqpoint{1.020000in}{0.880000in}}{\pgfqpoint{6.160000in}{6.160000in}}%
\pgfusepath{clip}%
\pgfsetbuttcap%
\pgfsetroundjoin%
\definecolor{currentfill}{rgb}{0.906154,0.842091,0.806151}%
\pgfsetfillcolor{currentfill}%
\pgfsetlinewidth{0.000000pt}%
\definecolor{currentstroke}{rgb}{0.000000,0.000000,0.000000}%
\pgfsetstrokecolor{currentstroke}%
\pgfsetdash{}{0pt}%
\pgfpathmoveto{\pgfqpoint{2.607806in}{4.186104in}}%
\pgfpathlineto{\pgfqpoint{2.617354in}{4.069740in}}%
\pgfpathlineto{\pgfqpoint{2.627020in}{3.945642in}}%
\pgfpathlineto{\pgfqpoint{2.659921in}{4.029745in}}%
\pgfpathlineto{\pgfqpoint{2.694748in}{3.986278in}}%
\pgfpathlineto{\pgfqpoint{2.686447in}{4.019440in}}%
\pgfpathlineto{\pgfqpoint{2.676569in}{4.158879in}}%
\pgfpathlineto{\pgfqpoint{2.642667in}{4.142125in}}%
\pgfpathlineto{\pgfqpoint{2.607806in}{4.186104in}}%
\pgfpathclose%
\pgfusepath{fill}%
\end{pgfscope}%
\begin{pgfscope}%
\pgfpathrectangle{\pgfqpoint{1.020000in}{0.880000in}}{\pgfqpoint{6.160000in}{6.160000in}}%
\pgfusepath{clip}%
\pgfsetbuttcap%
\pgfsetroundjoin%
\definecolor{currentfill}{rgb}{0.353369,0.472069,0.892570}%
\pgfsetfillcolor{currentfill}%
\pgfsetlinewidth{0.000000pt}%
\definecolor{currentstroke}{rgb}{0.000000,0.000000,0.000000}%
\pgfsetstrokecolor{currentstroke}%
\pgfsetdash{}{0pt}%
\pgfpathmoveto{\pgfqpoint{5.538959in}{3.025885in}}%
\pgfpathlineto{\pgfqpoint{5.548994in}{2.964792in}}%
\pgfpathlineto{\pgfqpoint{5.559619in}{2.945565in}}%
\pgfpathlineto{\pgfqpoint{5.593881in}{3.011304in}}%
\pgfpathlineto{\pgfqpoint{5.626118in}{2.935755in}}%
\pgfpathlineto{\pgfqpoint{5.614874in}{2.915528in}}%
\pgfpathlineto{\pgfqpoint{5.603753in}{2.901988in}}%
\pgfpathlineto{\pgfqpoint{5.572488in}{3.042920in}}%
\pgfpathlineto{\pgfqpoint{5.538959in}{3.025885in}}%
\pgfpathclose%
\pgfusepath{fill}%
\end{pgfscope}%
\begin{pgfscope}%
\pgfpathrectangle{\pgfqpoint{1.020000in}{0.880000in}}{\pgfqpoint{6.160000in}{6.160000in}}%
\pgfusepath{clip}%
\pgfsetbuttcap%
\pgfsetroundjoin%
\definecolor{currentfill}{rgb}{0.943432,0.802276,0.729172}%
\pgfsetfillcolor{currentfill}%
\pgfsetlinewidth{0.000000pt}%
\definecolor{currentstroke}{rgb}{0.000000,0.000000,0.000000}%
\pgfsetstrokecolor{currentstroke}%
\pgfsetdash{}{0pt}%
\pgfpathmoveto{\pgfqpoint{2.743230in}{4.270542in}}%
\pgfpathlineto{\pgfqpoint{2.750104in}{4.341421in}}%
\pgfpathlineto{\pgfqpoint{2.761359in}{4.104042in}}%
\pgfpathlineto{\pgfqpoint{2.795114in}{4.132477in}}%
\pgfpathlineto{\pgfqpoint{2.829676in}{4.100867in}}%
\pgfpathlineto{\pgfqpoint{2.821502in}{4.117291in}}%
\pgfpathlineto{\pgfqpoint{2.811957in}{4.235564in}}%
\pgfpathlineto{\pgfqpoint{2.778201in}{4.210617in}}%
\pgfpathlineto{\pgfqpoint{2.743230in}{4.270542in}}%
\pgfpathclose%
\pgfusepath{fill}%
\end{pgfscope}%
\begin{pgfscope}%
\pgfpathrectangle{\pgfqpoint{1.020000in}{0.880000in}}{\pgfqpoint{6.160000in}{6.160000in}}%
\pgfusepath{clip}%
\pgfsetbuttcap%
\pgfsetroundjoin%
\definecolor{currentfill}{rgb}{0.409611,0.540759,0.935545}%
\pgfsetfillcolor{currentfill}%
\pgfsetlinewidth{0.000000pt}%
\definecolor{currentstroke}{rgb}{0.000000,0.000000,0.000000}%
\pgfsetstrokecolor{currentstroke}%
\pgfsetdash{}{0pt}%
\pgfpathmoveto{\pgfqpoint{4.875704in}{3.120388in}}%
\pgfpathlineto{\pgfqpoint{4.884420in}{2.916059in}}%
\pgfpathlineto{\pgfqpoint{4.895931in}{3.105252in}}%
\pgfpathlineto{\pgfqpoint{4.930006in}{3.175458in}}%
\pgfpathlineto{\pgfqpoint{4.962250in}{3.008271in}}%
\pgfpathlineto{\pgfqpoint{4.952895in}{3.122040in}}%
\pgfpathlineto{\pgfqpoint{4.941733in}{2.998759in}}%
\pgfpathlineto{\pgfqpoint{4.909475in}{3.159346in}}%
\pgfpathlineto{\pgfqpoint{4.875704in}{3.120388in}}%
\pgfpathclose%
\pgfusepath{fill}%
\end{pgfscope}%
\begin{pgfscope}%
\pgfpathrectangle{\pgfqpoint{1.020000in}{0.880000in}}{\pgfqpoint{6.160000in}{6.160000in}}%
\pgfusepath{clip}%
\pgfsetbuttcap%
\pgfsetroundjoin%
\definecolor{currentfill}{rgb}{0.933221,0.815557,0.753151}%
\pgfsetfillcolor{currentfill}%
\pgfsetlinewidth{0.000000pt}%
\definecolor{currentstroke}{rgb}{0.000000,0.000000,0.000000}%
\pgfsetstrokecolor{currentstroke}%
\pgfsetdash{}{0pt}%
\pgfpathmoveto{\pgfqpoint{2.964965in}{4.184428in}}%
\pgfpathlineto{\pgfqpoint{2.973644in}{4.137165in}}%
\pgfpathlineto{\pgfqpoint{2.981742in}{4.139401in}}%
\pgfpathlineto{\pgfqpoint{3.017517in}{3.995297in}}%
\pgfpathlineto{\pgfqpoint{3.049361in}{4.190168in}}%
\pgfpathlineto{\pgfqpoint{3.040241in}{4.271556in}}%
\pgfpathlineto{\pgfqpoint{3.033847in}{4.114579in}}%
\pgfpathlineto{\pgfqpoint{2.998701in}{4.211339in}}%
\pgfpathlineto{\pgfqpoint{2.964965in}{4.184428in}}%
\pgfpathclose%
\pgfusepath{fill}%
\end{pgfscope}%
\begin{pgfscope}%
\pgfpathrectangle{\pgfqpoint{1.020000in}{0.880000in}}{\pgfqpoint{6.160000in}{6.160000in}}%
\pgfusepath{clip}%
\pgfsetbuttcap%
\pgfsetroundjoin%
\definecolor{currentfill}{rgb}{0.928116,0.822197,0.765141}%
\pgfsetfillcolor{currentfill}%
\pgfsetlinewidth{0.000000pt}%
\definecolor{currentstroke}{rgb}{0.000000,0.000000,0.000000}%
\pgfsetstrokecolor{currentstroke}%
\pgfsetdash{}{0pt}%
\pgfpathmoveto{\pgfqpoint{2.676569in}{4.158879in}}%
\pgfpathlineto{\pgfqpoint{2.686447in}{4.019440in}}%
\pgfpathlineto{\pgfqpoint{2.694748in}{3.986278in}}%
\pgfpathlineto{\pgfqpoint{2.727483in}{4.083892in}}%
\pgfpathlineto{\pgfqpoint{2.761359in}{4.104042in}}%
\pgfpathlineto{\pgfqpoint{2.750104in}{4.341421in}}%
\pgfpathlineto{\pgfqpoint{2.743230in}{4.270542in}}%
\pgfpathlineto{\pgfqpoint{2.711629in}{4.095823in}}%
\pgfpathlineto{\pgfqpoint{2.676569in}{4.158879in}}%
\pgfpathclose%
\pgfusepath{fill}%
\end{pgfscope}%
\begin{pgfscope}%
\pgfpathrectangle{\pgfqpoint{1.020000in}{0.880000in}}{\pgfqpoint{6.160000in}{6.160000in}}%
\pgfusepath{clip}%
\pgfsetbuttcap%
\pgfsetroundjoin%
\definecolor{currentfill}{rgb}{0.949151,0.790785,0.710876}%
\pgfsetfillcolor{currentfill}%
\pgfsetlinewidth{0.000000pt}%
\definecolor{currentstroke}{rgb}{0.000000,0.000000,0.000000}%
\pgfsetstrokecolor{currentstroke}%
\pgfsetdash{}{0pt}%
\pgfpathmoveto{\pgfqpoint{3.033847in}{4.114579in}}%
\pgfpathlineto{\pgfqpoint{3.040241in}{4.271556in}}%
\pgfpathlineto{\pgfqpoint{3.049361in}{4.190168in}}%
\pgfpathlineto{\pgfqpoint{3.084910in}{4.055679in}}%
\pgfpathlineto{\pgfqpoint{3.116148in}{4.318320in}}%
\pgfpathlineto{\pgfqpoint{3.108740in}{4.236999in}}%
\pgfpathlineto{\pgfqpoint{3.099142in}{4.362413in}}%
\pgfpathlineto{\pgfqpoint{3.066911in}{4.198084in}}%
\pgfpathlineto{\pgfqpoint{3.033847in}{4.114579in}}%
\pgfpathclose%
\pgfusepath{fill}%
\end{pgfscope}%
\begin{pgfscope}%
\pgfpathrectangle{\pgfqpoint{1.020000in}{0.880000in}}{\pgfqpoint{6.160000in}{6.160000in}}%
\pgfusepath{clip}%
\pgfsetbuttcap%
\pgfsetroundjoin%
\definecolor{currentfill}{rgb}{0.959518,0.766973,0.674145}%
\pgfsetfillcolor{currentfill}%
\pgfsetlinewidth{0.000000pt}%
\definecolor{currentstroke}{rgb}{0.000000,0.000000,0.000000}%
\pgfsetstrokecolor{currentstroke}%
\pgfsetdash{}{0pt}%
\pgfpathmoveto{\pgfqpoint{3.099142in}{4.362413in}}%
\pgfpathlineto{\pgfqpoint{3.108740in}{4.236999in}}%
\pgfpathlineto{\pgfqpoint{3.116148in}{4.318320in}}%
\pgfpathlineto{\pgfqpoint{3.149977in}{4.343343in}}%
\pgfpathlineto{\pgfqpoint{3.184404in}{4.307576in}}%
\pgfpathlineto{\pgfqpoint{3.176796in}{4.234092in}}%
\pgfpathlineto{\pgfqpoint{3.168953in}{4.186879in}}%
\pgfpathlineto{\pgfqpoint{3.134391in}{4.245481in}}%
\pgfpathlineto{\pgfqpoint{3.099142in}{4.362413in}}%
\pgfpathclose%
\pgfusepath{fill}%
\end{pgfscope}%
\begin{pgfscope}%
\pgfpathrectangle{\pgfqpoint{1.020000in}{0.880000in}}{\pgfqpoint{6.160000in}{6.160000in}}%
\pgfusepath{clip}%
\pgfsetbuttcap%
\pgfsetroundjoin%
\definecolor{currentfill}{rgb}{0.309060,0.413498,0.850128}%
\pgfsetfillcolor{currentfill}%
\pgfsetlinewidth{0.000000pt}%
\definecolor{currentstroke}{rgb}{0.000000,0.000000,0.000000}%
\pgfsetstrokecolor{currentstroke}%
\pgfsetdash{}{0pt}%
\pgfpathmoveto{\pgfqpoint{5.693187in}{2.969708in}}%
\pgfpathlineto{\pgfqpoint{5.702715in}{2.870381in}}%
\pgfpathlineto{\pgfqpoint{5.712394in}{2.780705in}}%
\pgfpathlineto{\pgfqpoint{5.746196in}{2.815189in}}%
\pgfpathlineto{\pgfqpoint{5.781834in}{2.963730in}}%
\pgfpathlineto{\pgfqpoint{5.769694in}{2.900660in}}%
\pgfpathlineto{\pgfqpoint{5.759491in}{2.957504in}}%
\pgfpathlineto{\pgfqpoint{5.723690in}{2.791972in}}%
\pgfpathlineto{\pgfqpoint{5.693187in}{2.969708in}}%
\pgfpathclose%
\pgfusepath{fill}%
\end{pgfscope}%
\begin{pgfscope}%
\pgfpathrectangle{\pgfqpoint{1.020000in}{0.880000in}}{\pgfqpoint{6.160000in}{6.160000in}}%
\pgfusepath{clip}%
\pgfsetbuttcap%
\pgfsetroundjoin%
\definecolor{currentfill}{rgb}{0.338377,0.452819,0.879317}%
\pgfsetfillcolor{currentfill}%
\pgfsetlinewidth{0.000000pt}%
\definecolor{currentstroke}{rgb}{0.000000,0.000000,0.000000}%
\pgfsetstrokecolor{currentstroke}%
\pgfsetdash{}{0pt}%
\pgfpathmoveto{\pgfqpoint{5.759491in}{2.957504in}}%
\pgfpathlineto{\pgfqpoint{5.769694in}{2.900660in}}%
\pgfpathlineto{\pgfqpoint{5.781834in}{2.963730in}}%
\pgfpathlineto{\pgfqpoint{5.815367in}{2.979095in}}%
\pgfpathlineto{\pgfqpoint{5.846485in}{2.850680in}}%
\pgfpathlineto{\pgfqpoint{5.836457in}{2.920388in}}%
\pgfpathlineto{\pgfqpoint{5.826140in}{2.972039in}}%
\pgfpathlineto{\pgfqpoint{5.793221in}{2.989409in}}%
\pgfpathlineto{\pgfqpoint{5.759491in}{2.957504in}}%
\pgfpathclose%
\pgfusepath{fill}%
\end{pgfscope}%
\begin{pgfscope}%
\pgfpathrectangle{\pgfqpoint{1.020000in}{0.880000in}}{\pgfqpoint{6.160000in}{6.160000in}}%
\pgfusepath{clip}%
\pgfsetbuttcap%
\pgfsetroundjoin%
\definecolor{currentfill}{rgb}{0.348323,0.465711,0.888346}%
\pgfsetfillcolor{currentfill}%
\pgfsetlinewidth{0.000000pt}%
\definecolor{currentstroke}{rgb}{0.000000,0.000000,0.000000}%
\pgfsetstrokecolor{currentstroke}%
\pgfsetdash{}{0pt}%
\pgfpathmoveto{\pgfqpoint{6.046618in}{2.915262in}}%
\pgfpathlineto{\pgfqpoint{6.059331in}{2.984310in}}%
\pgfpathlineto{\pgfqpoint{6.071432in}{3.018897in}}%
\pgfpathlineto{\pgfqpoint{6.104451in}{3.009294in}}%
\pgfpathlineto{\pgfqpoint{6.090882in}{2.900349in}}%
\pgfpathlineto{\pgfqpoint{6.079620in}{2.909841in}}%
\pgfpathlineto{\pgfqpoint{6.046618in}{2.915262in}}%
\pgfpathclose%
\pgfusepath{fill}%
\end{pgfscope}%
\begin{pgfscope}%
\pgfpathrectangle{\pgfqpoint{1.020000in}{0.880000in}}{\pgfqpoint{6.160000in}{6.160000in}}%
\pgfusepath{clip}%
\pgfsetbuttcap%
\pgfsetroundjoin%
\definecolor{currentfill}{rgb}{0.743754,0.825125,0.965798}%
\pgfsetfillcolor{currentfill}%
\pgfsetlinewidth{0.000000pt}%
\definecolor{currentstroke}{rgb}{0.000000,0.000000,0.000000}%
\pgfsetstrokecolor{currentstroke}%
\pgfsetdash{}{0pt}%
\pgfpathmoveto{\pgfqpoint{3.733792in}{3.529284in}}%
\pgfpathlineto{\pgfqpoint{3.742112in}{3.663701in}}%
\pgfpathlineto{\pgfqpoint{3.750707in}{3.744297in}}%
\pgfpathlineto{\pgfqpoint{3.784377in}{3.803035in}}%
\pgfpathlineto{\pgfqpoint{3.818660in}{3.698910in}}%
\pgfpathlineto{\pgfqpoint{3.809473in}{3.745439in}}%
\pgfpathlineto{\pgfqpoint{3.801287in}{3.532690in}}%
\pgfpathlineto{\pgfqpoint{3.766364in}{3.816372in}}%
\pgfpathlineto{\pgfqpoint{3.733792in}{3.529284in}}%
\pgfpathclose%
\pgfusepath{fill}%
\end{pgfscope}%
\begin{pgfscope}%
\pgfpathrectangle{\pgfqpoint{1.020000in}{0.880000in}}{\pgfqpoint{6.160000in}{6.160000in}}%
\pgfusepath{clip}%
\pgfsetbuttcap%
\pgfsetroundjoin%
\definecolor{currentfill}{rgb}{0.922681,0.828568,0.777054}%
\pgfsetfillcolor{currentfill}%
\pgfsetlinewidth{0.000000pt}%
\definecolor{currentstroke}{rgb}{0.000000,0.000000,0.000000}%
\pgfsetstrokecolor{currentstroke}%
\pgfsetdash{}{0pt}%
\pgfpathmoveto{\pgfqpoint{2.897789in}{4.106878in}}%
\pgfpathlineto{\pgfqpoint{2.905326in}{4.146339in}}%
\pgfpathlineto{\pgfqpoint{2.917017in}{3.858143in}}%
\pgfpathlineto{\pgfqpoint{2.949752in}{3.965319in}}%
\pgfpathlineto{\pgfqpoint{2.981742in}{4.139401in}}%
\pgfpathlineto{\pgfqpoint{2.973644in}{4.137165in}}%
\pgfpathlineto{\pgfqpoint{2.964965in}{4.184428in}}%
\pgfpathlineto{\pgfqpoint{2.928967in}{4.339300in}}%
\pgfpathlineto{\pgfqpoint{2.897789in}{4.106878in}}%
\pgfpathclose%
\pgfusepath{fill}%
\end{pgfscope}%
\begin{pgfscope}%
\pgfpathrectangle{\pgfqpoint{1.020000in}{0.880000in}}{\pgfqpoint{6.160000in}{6.160000in}}%
\pgfusepath{clip}%
\pgfsetbuttcap%
\pgfsetroundjoin%
\definecolor{currentfill}{rgb}{0.489246,0.627536,0.976896}%
\pgfsetfillcolor{currentfill}%
\pgfsetlinewidth{0.000000pt}%
\definecolor{currentstroke}{rgb}{0.000000,0.000000,0.000000}%
\pgfsetstrokecolor{currentstroke}%
\pgfsetdash{}{0pt}%
\pgfpathmoveto{\pgfqpoint{4.501185in}{3.295314in}}%
\pgfpathlineto{\pgfqpoint{4.510484in}{3.155821in}}%
\pgfpathlineto{\pgfqpoint{4.520501in}{3.230359in}}%
\pgfpathlineto{\pgfqpoint{4.554233in}{3.249792in}}%
\pgfpathlineto{\pgfqpoint{4.587701in}{3.205729in}}%
\pgfpathlineto{\pgfqpoint{4.577549in}{3.125720in}}%
\pgfpathlineto{\pgfqpoint{4.568231in}{3.252474in}}%
\pgfpathlineto{\pgfqpoint{4.534646in}{3.251416in}}%
\pgfpathlineto{\pgfqpoint{4.501185in}{3.295314in}}%
\pgfpathclose%
\pgfusepath{fill}%
\end{pgfscope}%
\begin{pgfscope}%
\pgfpathrectangle{\pgfqpoint{1.020000in}{0.880000in}}{\pgfqpoint{6.160000in}{6.160000in}}%
\pgfusepath{clip}%
\pgfsetbuttcap%
\pgfsetroundjoin%
\definecolor{currentfill}{rgb}{0.875557,0.860242,0.851430}%
\pgfsetfillcolor{currentfill}%
\pgfsetlinewidth{0.000000pt}%
\definecolor{currentstroke}{rgb}{0.000000,0.000000,0.000000}%
\pgfsetstrokecolor{currentstroke}%
\pgfsetdash{}{0pt}%
\pgfpathmoveto{\pgfqpoint{3.271870in}{4.027891in}}%
\pgfpathlineto{\pgfqpoint{3.279732in}{4.092260in}}%
\pgfpathlineto{\pgfqpoint{3.289955in}{3.894326in}}%
\pgfpathlineto{\pgfqpoint{3.323723in}{3.920982in}}%
\pgfpathlineto{\pgfqpoint{3.358431in}{3.832884in}}%
\pgfpathlineto{\pgfqpoint{3.349788in}{3.849027in}}%
\pgfpathlineto{\pgfqpoint{3.339549in}{4.057424in}}%
\pgfpathlineto{\pgfqpoint{3.304478in}{4.184451in}}%
\pgfpathlineto{\pgfqpoint{3.271870in}{4.027891in}}%
\pgfpathclose%
\pgfusepath{fill}%
\end{pgfscope}%
\begin{pgfscope}%
\pgfpathrectangle{\pgfqpoint{1.020000in}{0.880000in}}{\pgfqpoint{6.160000in}{6.160000in}}%
\pgfusepath{clip}%
\pgfsetbuttcap%
\pgfsetroundjoin%
\definecolor{currentfill}{rgb}{0.646113,0.764436,0.996868}%
\pgfsetfillcolor{currentfill}%
\pgfsetlinewidth{0.000000pt}%
\definecolor{currentstroke}{rgb}{0.000000,0.000000,0.000000}%
\pgfsetstrokecolor{currentstroke}%
\pgfsetdash{}{0pt}%
\pgfpathmoveto{\pgfqpoint{4.040040in}{3.622376in}}%
\pgfpathlineto{\pgfqpoint{4.049376in}{3.570091in}}%
\pgfpathlineto{\pgfqpoint{4.058806in}{3.449778in}}%
\pgfpathlineto{\pgfqpoint{4.092535in}{3.502646in}}%
\pgfpathlineto{\pgfqpoint{4.126334in}{3.460088in}}%
\pgfpathlineto{\pgfqpoint{4.117015in}{3.431910in}}%
\pgfpathlineto{\pgfqpoint{4.107578in}{3.593420in}}%
\pgfpathlineto{\pgfqpoint{4.074013in}{3.435255in}}%
\pgfpathlineto{\pgfqpoint{4.040040in}{3.622376in}}%
\pgfpathclose%
\pgfusepath{fill}%
\end{pgfscope}%
\begin{pgfscope}%
\pgfpathrectangle{\pgfqpoint{1.020000in}{0.880000in}}{\pgfqpoint{6.160000in}{6.160000in}}%
\pgfusepath{clip}%
\pgfsetbuttcap%
\pgfsetroundjoin%
\definecolor{currentfill}{rgb}{0.353369,0.472069,0.892570}%
\pgfsetfillcolor{currentfill}%
\pgfsetlinewidth{0.000000pt}%
\definecolor{currentstroke}{rgb}{0.000000,0.000000,0.000000}%
\pgfsetstrokecolor{currentstroke}%
\pgfsetdash{}{0pt}%
\pgfpathmoveto{\pgfqpoint{5.250521in}{3.005000in}}%
\pgfpathlineto{\pgfqpoint{5.263266in}{3.206504in}}%
\pgfpathlineto{\pgfqpoint{5.270509in}{2.898674in}}%
\pgfpathlineto{\pgfqpoint{5.303864in}{2.893732in}}%
\pgfpathlineto{\pgfqpoint{5.338009in}{2.958983in}}%
\pgfpathlineto{\pgfqpoint{5.326345in}{2.870513in}}%
\pgfpathlineto{\pgfqpoint{5.316352in}{2.925152in}}%
\pgfpathlineto{\pgfqpoint{5.283394in}{2.959109in}}%
\pgfpathlineto{\pgfqpoint{5.250521in}{3.005000in}}%
\pgfpathclose%
\pgfusepath{fill}%
\end{pgfscope}%
\begin{pgfscope}%
\pgfpathrectangle{\pgfqpoint{1.020000in}{0.880000in}}{\pgfqpoint{6.160000in}{6.160000in}}%
\pgfusepath{clip}%
\pgfsetbuttcap%
\pgfsetroundjoin%
\definecolor{currentfill}{rgb}{0.441123,0.576532,0.954545}%
\pgfsetfillcolor{currentfill}%
\pgfsetlinewidth{0.000000pt}%
\definecolor{currentstroke}{rgb}{0.000000,0.000000,0.000000}%
\pgfsetstrokecolor{currentstroke}%
\pgfsetdash{}{0pt}%
\pgfpathmoveto{\pgfqpoint{4.654945in}{3.214310in}}%
\pgfpathlineto{\pgfqpoint{4.664880in}{3.216486in}}%
\pgfpathlineto{\pgfqpoint{4.674026in}{3.054036in}}%
\pgfpathlineto{\pgfqpoint{4.707655in}{3.062022in}}%
\pgfpathlineto{\pgfqpoint{4.741614in}{3.131661in}}%
\pgfpathlineto{\pgfqpoint{4.732066in}{3.214664in}}%
\pgfpathlineto{\pgfqpoint{4.721416in}{3.092572in}}%
\pgfpathlineto{\pgfqpoint{4.687983in}{3.108379in}}%
\pgfpathlineto{\pgfqpoint{4.654945in}{3.214310in}}%
\pgfpathclose%
\pgfusepath{fill}%
\end{pgfscope}%
\begin{pgfscope}%
\pgfpathrectangle{\pgfqpoint{1.020000in}{0.880000in}}{\pgfqpoint{6.160000in}{6.160000in}}%
\pgfusepath{clip}%
\pgfsetbuttcap%
\pgfsetroundjoin%
\definecolor{currentfill}{rgb}{0.724041,0.814910,0.975651}%
\pgfsetfillcolor{currentfill}%
\pgfsetlinewidth{0.000000pt}%
\definecolor{currentstroke}{rgb}{0.000000,0.000000,0.000000}%
\pgfsetstrokecolor{currentstroke}%
\pgfsetdash{}{0pt}%
\pgfpathmoveto{\pgfqpoint{3.665241in}{3.715647in}}%
\pgfpathlineto{\pgfqpoint{3.674764in}{3.585081in}}%
\pgfpathlineto{\pgfqpoint{3.683693in}{3.572934in}}%
\pgfpathlineto{\pgfqpoint{3.716935in}{3.711568in}}%
\pgfpathlineto{\pgfqpoint{3.750707in}{3.744297in}}%
\pgfpathlineto{\pgfqpoint{3.742112in}{3.663701in}}%
\pgfpathlineto{\pgfqpoint{3.733792in}{3.529284in}}%
\pgfpathlineto{\pgfqpoint{3.699267in}{3.682715in}}%
\pgfpathlineto{\pgfqpoint{3.665241in}{3.715647in}}%
\pgfpathclose%
\pgfusepath{fill}%
\end{pgfscope}%
\begin{pgfscope}%
\pgfpathrectangle{\pgfqpoint{1.020000in}{0.880000in}}{\pgfqpoint{6.160000in}{6.160000in}}%
\pgfusepath{clip}%
\pgfsetbuttcap%
\pgfsetroundjoin%
\definecolor{currentfill}{rgb}{0.909460,0.839386,0.800331}%
\pgfsetfillcolor{currentfill}%
\pgfsetlinewidth{0.000000pt}%
\definecolor{currentstroke}{rgb}{0.000000,0.000000,0.000000}%
\pgfsetstrokecolor{currentstroke}%
\pgfsetdash{}{0pt}%
\pgfpathmoveto{\pgfqpoint{2.829676in}{4.100867in}}%
\pgfpathlineto{\pgfqpoint{2.839390in}{3.970260in}}%
\pgfpathlineto{\pgfqpoint{2.844554in}{4.181870in}}%
\pgfpathlineto{\pgfqpoint{2.881269in}{3.987713in}}%
\pgfpathlineto{\pgfqpoint{2.917017in}{3.858143in}}%
\pgfpathlineto{\pgfqpoint{2.905326in}{4.146339in}}%
\pgfpathlineto{\pgfqpoint{2.897789in}{4.106878in}}%
\pgfpathlineto{\pgfqpoint{2.862297in}{4.214101in}}%
\pgfpathlineto{\pgfqpoint{2.829676in}{4.100867in}}%
\pgfpathclose%
\pgfusepath{fill}%
\end{pgfscope}%
\begin{pgfscope}%
\pgfpathrectangle{\pgfqpoint{1.020000in}{0.880000in}}{\pgfqpoint{6.160000in}{6.160000in}}%
\pgfusepath{clip}%
\pgfsetbuttcap%
\pgfsetroundjoin%
\definecolor{currentfill}{rgb}{0.608547,0.735725,0.999354}%
\pgfsetfillcolor{currentfill}%
\pgfsetlinewidth{0.000000pt}%
\definecolor{currentstroke}{rgb}{0.000000,0.000000,0.000000}%
\pgfsetstrokecolor{currentstroke}%
\pgfsetdash{}{0pt}%
\pgfpathmoveto{\pgfqpoint{4.193799in}{3.520851in}}%
\pgfpathlineto{\pgfqpoint{4.203221in}{3.480660in}}%
\pgfpathlineto{\pgfqpoint{4.212656in}{3.442466in}}%
\pgfpathlineto{\pgfqpoint{4.246406in}{3.425927in}}%
\pgfpathlineto{\pgfqpoint{4.279966in}{3.245493in}}%
\pgfpathlineto{\pgfqpoint{4.270693in}{3.495527in}}%
\pgfpathlineto{\pgfqpoint{4.261094in}{3.361106in}}%
\pgfpathlineto{\pgfqpoint{4.227504in}{3.487011in}}%
\pgfpathlineto{\pgfqpoint{4.193799in}{3.520851in}}%
\pgfpathclose%
\pgfusepath{fill}%
\end{pgfscope}%
\begin{pgfscope}%
\pgfpathrectangle{\pgfqpoint{1.020000in}{0.880000in}}{\pgfqpoint{6.160000in}{6.160000in}}%
\pgfusepath{clip}%
\pgfsetbuttcap%
\pgfsetroundjoin%
\definecolor{currentfill}{rgb}{0.822420,0.856898,0.910795}%
\pgfsetfillcolor{currentfill}%
\pgfsetlinewidth{0.000000pt}%
\definecolor{currentstroke}{rgb}{0.000000,0.000000,0.000000}%
\pgfsetstrokecolor{currentstroke}%
\pgfsetdash{}{0pt}%
\pgfpathmoveto{\pgfqpoint{3.358431in}{3.832884in}}%
\pgfpathlineto{\pgfqpoint{3.367477in}{3.768992in}}%
\pgfpathlineto{\pgfqpoint{3.375437in}{3.841115in}}%
\pgfpathlineto{\pgfqpoint{3.409631in}{3.816409in}}%
\pgfpathlineto{\pgfqpoint{3.443131in}{3.881089in}}%
\pgfpathlineto{\pgfqpoint{3.434491in}{3.884234in}}%
\pgfpathlineto{\pgfqpoint{3.426043in}{3.864146in}}%
\pgfpathlineto{\pgfqpoint{3.391670in}{3.921075in}}%
\pgfpathlineto{\pgfqpoint{3.358431in}{3.832884in}}%
\pgfpathclose%
\pgfusepath{fill}%
\end{pgfscope}%
\begin{pgfscope}%
\pgfpathrectangle{\pgfqpoint{1.020000in}{0.880000in}}{\pgfqpoint{6.160000in}{6.160000in}}%
\pgfusepath{clip}%
\pgfsetbuttcap%
\pgfsetroundjoin%
\definecolor{currentfill}{rgb}{0.879622,0.858175,0.845844}%
\pgfsetfillcolor{currentfill}%
\pgfsetlinewidth{0.000000pt}%
\definecolor{currentstroke}{rgb}{0.000000,0.000000,0.000000}%
\pgfsetstrokecolor{currentstroke}%
\pgfsetdash{}{0pt}%
\pgfpathmoveto{\pgfqpoint{2.470786in}{4.189962in}}%
\pgfpathlineto{\pgfqpoint{2.482738in}{3.932767in}}%
\pgfpathlineto{\pgfqpoint{2.491537in}{3.861449in}}%
\pgfpathlineto{\pgfqpoint{2.526300in}{3.828954in}}%
\pgfpathlineto{\pgfqpoint{2.559241in}{3.906609in}}%
\pgfpathlineto{\pgfqpoint{2.547679in}{4.146657in}}%
\pgfpathlineto{\pgfqpoint{2.540914in}{4.091462in}}%
\pgfpathlineto{\pgfqpoint{2.508390in}{3.989789in}}%
\pgfpathlineto{\pgfqpoint{2.470786in}{4.189962in}}%
\pgfpathclose%
\pgfusepath{fill}%
\end{pgfscope}%
\begin{pgfscope}%
\pgfpathrectangle{\pgfqpoint{1.020000in}{0.880000in}}{\pgfqpoint{6.160000in}{6.160000in}}%
\pgfusepath{clip}%
\pgfsetbuttcap%
\pgfsetroundjoin%
\definecolor{currentfill}{rgb}{0.353369,0.472069,0.892570}%
\pgfsetfillcolor{currentfill}%
\pgfsetlinewidth{0.000000pt}%
\definecolor{currentstroke}{rgb}{0.000000,0.000000,0.000000}%
\pgfsetstrokecolor{currentstroke}%
\pgfsetdash{}{0pt}%
\pgfpathmoveto{\pgfqpoint{5.471394in}{2.954763in}}%
\pgfpathlineto{\pgfqpoint{5.482120in}{2.949047in}}%
\pgfpathlineto{\pgfqpoint{5.491799in}{2.862068in}}%
\pgfpathlineto{\pgfqpoint{5.526744in}{2.980769in}}%
\pgfpathlineto{\pgfqpoint{5.559619in}{2.945565in}}%
\pgfpathlineto{\pgfqpoint{5.548994in}{2.964792in}}%
\pgfpathlineto{\pgfqpoint{5.538959in}{3.025885in}}%
\pgfpathlineto{\pgfqpoint{5.505177in}{2.990521in}}%
\pgfpathlineto{\pgfqpoint{5.471394in}{2.954763in}}%
\pgfpathclose%
\pgfusepath{fill}%
\end{pgfscope}%
\begin{pgfscope}%
\pgfpathrectangle{\pgfqpoint{1.020000in}{0.880000in}}{\pgfqpoint{6.160000in}{6.160000in}}%
\pgfusepath{clip}%
\pgfsetbuttcap%
\pgfsetroundjoin%
\definecolor{currentfill}{rgb}{0.565182,0.699438,0.996635}%
\pgfsetfillcolor{currentfill}%
\pgfsetlinewidth{0.000000pt}%
\definecolor{currentstroke}{rgb}{0.000000,0.000000,0.000000}%
\pgfsetstrokecolor{currentstroke}%
\pgfsetdash{}{0pt}%
\pgfpathmoveto{\pgfqpoint{4.347509in}{3.401636in}}%
\pgfpathlineto{\pgfqpoint{4.357286in}{3.507611in}}%
\pgfpathlineto{\pgfqpoint{4.366733in}{3.411526in}}%
\pgfpathlineto{\pgfqpoint{4.400347in}{3.355176in}}%
\pgfpathlineto{\pgfqpoint{4.433583in}{3.171769in}}%
\pgfpathlineto{\pgfqpoint{4.424362in}{3.352868in}}%
\pgfpathlineto{\pgfqpoint{4.414747in}{3.370327in}}%
\pgfpathlineto{\pgfqpoint{4.380876in}{3.251008in}}%
\pgfpathlineto{\pgfqpoint{4.347509in}{3.401636in}}%
\pgfpathclose%
\pgfusepath{fill}%
\end{pgfscope}%
\begin{pgfscope}%
\pgfpathrectangle{\pgfqpoint{1.020000in}{0.880000in}}{\pgfqpoint{6.160000in}{6.160000in}}%
\pgfusepath{clip}%
\pgfsetbuttcap%
\pgfsetroundjoin%
\definecolor{currentfill}{rgb}{0.318832,0.426605,0.859857}%
\pgfsetfillcolor{currentfill}%
\pgfsetlinewidth{0.000000pt}%
\definecolor{currentstroke}{rgb}{0.000000,0.000000,0.000000}%
\pgfsetstrokecolor{currentstroke}%
\pgfsetdash{}{0pt}%
\pgfpathmoveto{\pgfqpoint{5.914059in}{2.918863in}}%
\pgfpathlineto{\pgfqpoint{5.927896in}{3.064688in}}%
\pgfpathlineto{\pgfqpoint{5.936031in}{2.883759in}}%
\pgfpathlineto{\pgfqpoint{5.968919in}{2.864150in}}%
\pgfpathlineto{\pgfqpoint{6.004240in}{2.979407in}}%
\pgfpathlineto{\pgfqpoint{5.989794in}{2.810689in}}%
\pgfpathlineto{\pgfqpoint{5.978623in}{2.819837in}}%
\pgfpathlineto{\pgfqpoint{5.945256in}{2.806661in}}%
\pgfpathlineto{\pgfqpoint{5.914059in}{2.918863in}}%
\pgfpathclose%
\pgfusepath{fill}%
\end{pgfscope}%
\begin{pgfscope}%
\pgfpathrectangle{\pgfqpoint{1.020000in}{0.880000in}}{\pgfqpoint{6.160000in}{6.160000in}}%
\pgfusepath{clip}%
\pgfsetbuttcap%
\pgfsetroundjoin%
\definecolor{currentfill}{rgb}{0.909460,0.839386,0.800331}%
\pgfsetfillcolor{currentfill}%
\pgfsetlinewidth{0.000000pt}%
\definecolor{currentstroke}{rgb}{0.000000,0.000000,0.000000}%
\pgfsetstrokecolor{currentstroke}%
\pgfsetdash{}{0pt}%
\pgfpathmoveto{\pgfqpoint{2.540914in}{4.091462in}}%
\pgfpathlineto{\pgfqpoint{2.547679in}{4.146657in}}%
\pgfpathlineto{\pgfqpoint{2.559241in}{3.906609in}}%
\pgfpathlineto{\pgfqpoint{2.590504in}{4.092374in}}%
\pgfpathlineto{\pgfqpoint{2.627020in}{3.945642in}}%
\pgfpathlineto{\pgfqpoint{2.617354in}{4.069740in}}%
\pgfpathlineto{\pgfqpoint{2.607806in}{4.186104in}}%
\pgfpathlineto{\pgfqpoint{2.574856in}{4.107383in}}%
\pgfpathlineto{\pgfqpoint{2.540914in}{4.091462in}}%
\pgfpathclose%
\pgfusepath{fill}%
\end{pgfscope}%
\begin{pgfscope}%
\pgfpathrectangle{\pgfqpoint{1.020000in}{0.880000in}}{\pgfqpoint{6.160000in}{6.160000in}}%
\pgfusepath{clip}%
\pgfsetbuttcap%
\pgfsetroundjoin%
\definecolor{currentfill}{rgb}{0.414801,0.546874,0.939088}%
\pgfsetfillcolor{currentfill}%
\pgfsetlinewidth{0.000000pt}%
\definecolor{currentstroke}{rgb}{0.000000,0.000000,0.000000}%
\pgfsetstrokecolor{currentstroke}%
\pgfsetdash{}{0pt}%
\pgfpathmoveto{\pgfqpoint{4.808459in}{3.087243in}}%
\pgfpathlineto{\pgfqpoint{4.817878in}{2.982383in}}%
\pgfpathlineto{\pgfqpoint{4.828834in}{3.114706in}}%
\pgfpathlineto{\pgfqpoint{4.862423in}{3.114187in}}%
\pgfpathlineto{\pgfqpoint{4.895931in}{3.105252in}}%
\pgfpathlineto{\pgfqpoint{4.884420in}{2.916059in}}%
\pgfpathlineto{\pgfqpoint{4.875704in}{3.120388in}}%
\pgfpathlineto{\pgfqpoint{4.842463in}{3.160850in}}%
\pgfpathlineto{\pgfqpoint{4.808459in}{3.087243in}}%
\pgfpathclose%
\pgfusepath{fill}%
\end{pgfscope}%
\begin{pgfscope}%
\pgfpathrectangle{\pgfqpoint{1.020000in}{0.880000in}}{\pgfqpoint{6.160000in}{6.160000in}}%
\pgfusepath{clip}%
\pgfsetbuttcap%
\pgfsetroundjoin%
\definecolor{currentfill}{rgb}{0.804965,0.851666,0.926165}%
\pgfsetfillcolor{currentfill}%
\pgfsetlinewidth{0.000000pt}%
\definecolor{currentstroke}{rgb}{0.000000,0.000000,0.000000}%
\pgfsetstrokecolor{currentstroke}%
\pgfsetdash{}{0pt}%
\pgfpathmoveto{\pgfqpoint{3.289955in}{3.894326in}}%
\pgfpathlineto{\pgfqpoint{3.299319in}{3.792178in}}%
\pgfpathlineto{\pgfqpoint{3.308412in}{3.721222in}}%
\pgfpathlineto{\pgfqpoint{3.342521in}{3.708842in}}%
\pgfpathlineto{\pgfqpoint{3.375437in}{3.841115in}}%
\pgfpathlineto{\pgfqpoint{3.367477in}{3.768992in}}%
\pgfpathlineto{\pgfqpoint{3.358431in}{3.832884in}}%
\pgfpathlineto{\pgfqpoint{3.323723in}{3.920982in}}%
\pgfpathlineto{\pgfqpoint{3.289955in}{3.894326in}}%
\pgfpathclose%
\pgfusepath{fill}%
\end{pgfscope}%
\begin{pgfscope}%
\pgfpathrectangle{\pgfqpoint{1.020000in}{0.880000in}}{\pgfqpoint{6.160000in}{6.160000in}}%
\pgfusepath{clip}%
\pgfsetbuttcap%
\pgfsetroundjoin%
\definecolor{currentfill}{rgb}{0.863392,0.865084,0.867634}%
\pgfsetfillcolor{currentfill}%
\pgfsetlinewidth{0.000000pt}%
\definecolor{currentstroke}{rgb}{0.000000,0.000000,0.000000}%
\pgfsetstrokecolor{currentstroke}%
\pgfsetdash{}{0pt}%
\pgfpathmoveto{\pgfqpoint{2.404444in}{4.062072in}}%
\pgfpathlineto{\pgfqpoint{2.416795in}{3.787718in}}%
\pgfpathlineto{\pgfqpoint{2.424049in}{3.802762in}}%
\pgfpathlineto{\pgfqpoint{2.455755in}{3.950688in}}%
\pgfpathlineto{\pgfqpoint{2.491537in}{3.861449in}}%
\pgfpathlineto{\pgfqpoint{2.482738in}{3.932767in}}%
\pgfpathlineto{\pgfqpoint{2.470786in}{4.189962in}}%
\pgfpathlineto{\pgfqpoint{2.438998in}{4.045590in}}%
\pgfpathlineto{\pgfqpoint{2.404444in}{4.062072in}}%
\pgfpathclose%
\pgfusepath{fill}%
\end{pgfscope}%
\begin{pgfscope}%
\pgfpathrectangle{\pgfqpoint{1.020000in}{0.880000in}}{\pgfqpoint{6.160000in}{6.160000in}}%
\pgfusepath{clip}%
\pgfsetbuttcap%
\pgfsetroundjoin%
\definecolor{currentfill}{rgb}{0.871493,0.862309,0.857016}%
\pgfsetfillcolor{currentfill}%
\pgfsetlinewidth{0.000000pt}%
\definecolor{currentstroke}{rgb}{0.000000,0.000000,0.000000}%
\pgfsetstrokecolor{currentstroke}%
\pgfsetdash{}{0pt}%
\pgfpathmoveto{\pgfqpoint{3.204429in}{3.969800in}}%
\pgfpathlineto{\pgfqpoint{3.212413in}{4.009533in}}%
\pgfpathlineto{\pgfqpoint{3.222185in}{3.864867in}}%
\pgfpathlineto{\pgfqpoint{3.256089in}{3.878101in}}%
\pgfpathlineto{\pgfqpoint{3.289955in}{3.894326in}}%
\pgfpathlineto{\pgfqpoint{3.279732in}{4.092260in}}%
\pgfpathlineto{\pgfqpoint{3.271870in}{4.027891in}}%
\pgfpathlineto{\pgfqpoint{3.237891in}{4.026397in}}%
\pgfpathlineto{\pgfqpoint{3.204429in}{3.969800in}}%
\pgfpathclose%
\pgfusepath{fill}%
\end{pgfscope}%
\begin{pgfscope}%
\pgfpathrectangle{\pgfqpoint{1.020000in}{0.880000in}}{\pgfqpoint{6.160000in}{6.160000in}}%
\pgfusepath{clip}%
\pgfsetbuttcap%
\pgfsetroundjoin%
\definecolor{currentfill}{rgb}{0.718985,0.811993,0.977656}%
\pgfsetfillcolor{currentfill}%
\pgfsetlinewidth{0.000000pt}%
\definecolor{currentstroke}{rgb}{0.000000,0.000000,0.000000}%
\pgfsetstrokecolor{currentstroke}%
\pgfsetdash{}{0pt}%
\pgfpathmoveto{\pgfqpoint{3.598404in}{3.551520in}}%
\pgfpathlineto{\pgfqpoint{3.605767in}{3.797496in}}%
\pgfpathlineto{\pgfqpoint{3.615791in}{3.585042in}}%
\pgfpathlineto{\pgfqpoint{3.648823in}{3.753833in}}%
\pgfpathlineto{\pgfqpoint{3.683693in}{3.572934in}}%
\pgfpathlineto{\pgfqpoint{3.674764in}{3.585081in}}%
\pgfpathlineto{\pgfqpoint{3.665241in}{3.715647in}}%
\pgfpathlineto{\pgfqpoint{3.632439in}{3.518345in}}%
\pgfpathlineto{\pgfqpoint{3.598404in}{3.551520in}}%
\pgfpathclose%
\pgfusepath{fill}%
\end{pgfscope}%
\begin{pgfscope}%
\pgfpathrectangle{\pgfqpoint{1.020000in}{0.880000in}}{\pgfqpoint{6.160000in}{6.160000in}}%
\pgfusepath{clip}%
\pgfsetbuttcap%
\pgfsetroundjoin%
\definecolor{currentfill}{rgb}{0.373552,0.497499,0.909467}%
\pgfsetfillcolor{currentfill}%
\pgfsetlinewidth{0.000000pt}%
\definecolor{currentstroke}{rgb}{0.000000,0.000000,0.000000}%
\pgfsetstrokecolor{currentstroke}%
\pgfsetdash{}{0pt}%
\pgfpathmoveto{\pgfqpoint{4.962250in}{3.008271in}}%
\pgfpathlineto{\pgfqpoint{4.972016in}{2.946508in}}%
\pgfpathlineto{\pgfqpoint{4.982759in}{3.005213in}}%
\pgfpathlineto{\pgfqpoint{5.016291in}{3.004117in}}%
\pgfpathlineto{\pgfqpoint{5.048829in}{2.891195in}}%
\pgfpathlineto{\pgfqpoint{5.038881in}{2.935812in}}%
\pgfpathlineto{\pgfqpoint{5.029368in}{3.030133in}}%
\pgfpathlineto{\pgfqpoint{4.997278in}{3.198431in}}%
\pgfpathlineto{\pgfqpoint{4.962250in}{3.008271in}}%
\pgfpathclose%
\pgfusepath{fill}%
\end{pgfscope}%
\begin{pgfscope}%
\pgfpathrectangle{\pgfqpoint{1.020000in}{0.880000in}}{\pgfqpoint{6.160000in}{6.160000in}}%
\pgfusepath{clip}%
\pgfsetbuttcap%
\pgfsetroundjoin%
\definecolor{currentfill}{rgb}{0.809329,0.852974,0.922323}%
\pgfsetfillcolor{currentfill}%
\pgfsetlinewidth{0.000000pt}%
\definecolor{currentstroke}{rgb}{0.000000,0.000000,0.000000}%
\pgfsetstrokecolor{currentstroke}%
\pgfsetdash{}{0pt}%
\pgfpathmoveto{\pgfqpoint{2.491537in}{3.861449in}}%
\pgfpathlineto{\pgfqpoint{2.499777in}{3.823741in}}%
\pgfpathlineto{\pgfqpoint{2.510078in}{3.663253in}}%
\pgfpathlineto{\pgfqpoint{2.543704in}{3.700077in}}%
\pgfpathlineto{\pgfqpoint{2.575143in}{3.873809in}}%
\pgfpathlineto{\pgfqpoint{2.567178in}{3.890656in}}%
\pgfpathlineto{\pgfqpoint{2.559241in}{3.906609in}}%
\pgfpathlineto{\pgfqpoint{2.526300in}{3.828954in}}%
\pgfpathlineto{\pgfqpoint{2.491537in}{3.861449in}}%
\pgfpathclose%
\pgfusepath{fill}%
\end{pgfscope}%
\begin{pgfscope}%
\pgfpathrectangle{\pgfqpoint{1.020000in}{0.880000in}}{\pgfqpoint{6.160000in}{6.160000in}}%
\pgfusepath{clip}%
\pgfsetbuttcap%
\pgfsetroundjoin%
\definecolor{currentfill}{rgb}{0.323718,0.433158,0.864722}%
\pgfsetfillcolor{currentfill}%
\pgfsetlinewidth{0.000000pt}%
\definecolor{currentstroke}{rgb}{0.000000,0.000000,0.000000}%
\pgfsetstrokecolor{currentstroke}%
\pgfsetdash{}{0pt}%
\pgfpathmoveto{\pgfqpoint{5.626118in}{2.935755in}}%
\pgfpathlineto{\pgfqpoint{5.636032in}{2.863004in}}%
\pgfpathlineto{\pgfqpoint{5.647392in}{2.887955in}}%
\pgfpathlineto{\pgfqpoint{5.681701in}{2.952577in}}%
\pgfpathlineto{\pgfqpoint{5.712394in}{2.780705in}}%
\pgfpathlineto{\pgfqpoint{5.702715in}{2.870381in}}%
\pgfpathlineto{\pgfqpoint{5.693187in}{2.969708in}}%
\pgfpathlineto{\pgfqpoint{5.659508in}{2.942662in}}%
\pgfpathlineto{\pgfqpoint{5.626118in}{2.935755in}}%
\pgfpathclose%
\pgfusepath{fill}%
\end{pgfscope}%
\begin{pgfscope}%
\pgfpathrectangle{\pgfqpoint{1.020000in}{0.880000in}}{\pgfqpoint{6.160000in}{6.160000in}}%
\pgfusepath{clip}%
\pgfsetbuttcap%
\pgfsetroundjoin%
\definecolor{currentfill}{rgb}{0.909460,0.839386,0.800331}%
\pgfsetfillcolor{currentfill}%
\pgfsetlinewidth{0.000000pt}%
\definecolor{currentstroke}{rgb}{0.000000,0.000000,0.000000}%
\pgfsetstrokecolor{currentstroke}%
\pgfsetdash{}{0pt}%
\pgfpathmoveto{\pgfqpoint{2.761359in}{4.104042in}}%
\pgfpathlineto{\pgfqpoint{2.769377in}{4.095736in}}%
\pgfpathlineto{\pgfqpoint{2.778473in}{4.011336in}}%
\pgfpathlineto{\pgfqpoint{2.814062in}{3.908165in}}%
\pgfpathlineto{\pgfqpoint{2.844554in}{4.181870in}}%
\pgfpathlineto{\pgfqpoint{2.839390in}{3.970260in}}%
\pgfpathlineto{\pgfqpoint{2.829676in}{4.100867in}}%
\pgfpathlineto{\pgfqpoint{2.795114in}{4.132477in}}%
\pgfpathlineto{\pgfqpoint{2.761359in}{4.104042in}}%
\pgfpathclose%
\pgfusepath{fill}%
\end{pgfscope}%
\begin{pgfscope}%
\pgfpathrectangle{\pgfqpoint{1.020000in}{0.880000in}}{\pgfqpoint{6.160000in}{6.160000in}}%
\pgfusepath{clip}%
\pgfsetbuttcap%
\pgfsetroundjoin%
\definecolor{currentfill}{rgb}{0.446431,0.582356,0.957373}%
\pgfsetfillcolor{currentfill}%
\pgfsetlinewidth{0.000000pt}%
\definecolor{currentstroke}{rgb}{0.000000,0.000000,0.000000}%
\pgfsetstrokecolor{currentstroke}%
\pgfsetdash{}{0pt}%
\pgfpathmoveto{\pgfqpoint{4.587701in}{3.205729in}}%
\pgfpathlineto{\pgfqpoint{4.596939in}{3.056489in}}%
\pgfpathlineto{\pgfqpoint{4.607146in}{3.137236in}}%
\pgfpathlineto{\pgfqpoint{4.640620in}{3.097758in}}%
\pgfpathlineto{\pgfqpoint{4.674026in}{3.054036in}}%
\pgfpathlineto{\pgfqpoint{4.664880in}{3.216486in}}%
\pgfpathlineto{\pgfqpoint{4.654945in}{3.214310in}}%
\pgfpathlineto{\pgfqpoint{4.620956in}{3.124479in}}%
\pgfpathlineto{\pgfqpoint{4.587701in}{3.205729in}}%
\pgfpathclose%
\pgfusepath{fill}%
\end{pgfscope}%
\begin{pgfscope}%
\pgfpathrectangle{\pgfqpoint{1.020000in}{0.880000in}}{\pgfqpoint{6.160000in}{6.160000in}}%
\pgfusepath{clip}%
\pgfsetbuttcap%
\pgfsetroundjoin%
\definecolor{currentfill}{rgb}{0.343278,0.459354,0.884122}%
\pgfsetfillcolor{currentfill}%
\pgfsetlinewidth{0.000000pt}%
\definecolor{currentstroke}{rgb}{0.000000,0.000000,0.000000}%
\pgfsetstrokecolor{currentstroke}%
\pgfsetdash{}{0pt}%
\pgfpathmoveto{\pgfqpoint{5.406333in}{3.086461in}}%
\pgfpathlineto{\pgfqpoint{5.416283in}{3.021840in}}%
\pgfpathlineto{\pgfqpoint{5.425252in}{2.877970in}}%
\pgfpathlineto{\pgfqpoint{5.457485in}{2.787877in}}%
\pgfpathlineto{\pgfqpoint{5.491799in}{2.862068in}}%
\pgfpathlineto{\pgfqpoint{5.482120in}{2.949047in}}%
\pgfpathlineto{\pgfqpoint{5.471394in}{2.954763in}}%
\pgfpathlineto{\pgfqpoint{5.438413in}{2.982309in}}%
\pgfpathlineto{\pgfqpoint{5.406333in}{3.086461in}}%
\pgfpathclose%
\pgfusepath{fill}%
\end{pgfscope}%
\begin{pgfscope}%
\pgfpathrectangle{\pgfqpoint{1.020000in}{0.880000in}}{\pgfqpoint{6.160000in}{6.160000in}}%
\pgfusepath{clip}%
\pgfsetbuttcap%
\pgfsetroundjoin%
\definecolor{currentfill}{rgb}{0.935774,0.812237,0.747156}%
\pgfsetfillcolor{currentfill}%
\pgfsetlinewidth{0.000000pt}%
\definecolor{currentstroke}{rgb}{0.000000,0.000000,0.000000}%
\pgfsetstrokecolor{currentstroke}%
\pgfsetdash{}{0pt}%
\pgfpathmoveto{\pgfqpoint{3.116148in}{4.318320in}}%
\pgfpathlineto{\pgfqpoint{3.126488in}{4.123254in}}%
\pgfpathlineto{\pgfqpoint{3.135365in}{4.067279in}}%
\pgfpathlineto{\pgfqpoint{3.170209in}{3.990278in}}%
\pgfpathlineto{\pgfqpoint{3.204429in}{3.969800in}}%
\pgfpathlineto{\pgfqpoint{3.194660in}{4.114447in}}%
\pgfpathlineto{\pgfqpoint{3.184404in}{4.307576in}}%
\pgfpathlineto{\pgfqpoint{3.149977in}{4.343343in}}%
\pgfpathlineto{\pgfqpoint{3.116148in}{4.318320in}}%
\pgfpathclose%
\pgfusepath{fill}%
\end{pgfscope}%
\begin{pgfscope}%
\pgfpathrectangle{\pgfqpoint{1.020000in}{0.880000in}}{\pgfqpoint{6.160000in}{6.160000in}}%
\pgfusepath{clip}%
\pgfsetbuttcap%
\pgfsetroundjoin%
\definecolor{currentfill}{rgb}{0.919376,0.831273,0.782874}%
\pgfsetfillcolor{currentfill}%
\pgfsetlinewidth{0.000000pt}%
\definecolor{currentstroke}{rgb}{0.000000,0.000000,0.000000}%
\pgfsetstrokecolor{currentstroke}%
\pgfsetdash{}{0pt}%
\pgfpathmoveto{\pgfqpoint{3.049361in}{4.190168in}}%
\pgfpathlineto{\pgfqpoint{3.059910in}{3.981147in}}%
\pgfpathlineto{\pgfqpoint{3.067510in}{4.036583in}}%
\pgfpathlineto{\pgfqpoint{3.102027in}{3.997579in}}%
\pgfpathlineto{\pgfqpoint{3.135365in}{4.067279in}}%
\pgfpathlineto{\pgfqpoint{3.126488in}{4.123254in}}%
\pgfpathlineto{\pgfqpoint{3.116148in}{4.318320in}}%
\pgfpathlineto{\pgfqpoint{3.084910in}{4.055679in}}%
\pgfpathlineto{\pgfqpoint{3.049361in}{4.190168in}}%
\pgfpathclose%
\pgfusepath{fill}%
\end{pgfscope}%
\begin{pgfscope}%
\pgfpathrectangle{\pgfqpoint{1.020000in}{0.880000in}}{\pgfqpoint{6.160000in}{6.160000in}}%
\pgfusepath{clip}%
\pgfsetbuttcap%
\pgfsetroundjoin%
\definecolor{currentfill}{rgb}{0.875557,0.860242,0.851430}%
\pgfsetfillcolor{currentfill}%
\pgfsetlinewidth{0.000000pt}%
\definecolor{currentstroke}{rgb}{0.000000,0.000000,0.000000}%
\pgfsetstrokecolor{currentstroke}%
\pgfsetdash{}{0pt}%
\pgfpathmoveto{\pgfqpoint{2.917017in}{3.858143in}}%
\pgfpathlineto{\pgfqpoint{2.924322in}{3.917719in}}%
\pgfpathlineto{\pgfqpoint{2.932812in}{3.884300in}}%
\pgfpathlineto{\pgfqpoint{2.965192in}{4.028556in}}%
\pgfpathlineto{\pgfqpoint{2.999582in}{4.008701in}}%
\pgfpathlineto{\pgfqpoint{2.991406in}{4.010993in}}%
\pgfpathlineto{\pgfqpoint{2.981742in}{4.139401in}}%
\pgfpathlineto{\pgfqpoint{2.949752in}{3.965319in}}%
\pgfpathlineto{\pgfqpoint{2.917017in}{3.858143in}}%
\pgfpathclose%
\pgfusepath{fill}%
\end{pgfscope}%
\begin{pgfscope}%
\pgfpathrectangle{\pgfqpoint{1.020000in}{0.880000in}}{\pgfqpoint{6.160000in}{6.160000in}}%
\pgfusepath{clip}%
\pgfsetbuttcap%
\pgfsetroundjoin%
\definecolor{currentfill}{rgb}{0.902849,0.844796,0.811970}%
\pgfsetfillcolor{currentfill}%
\pgfsetlinewidth{0.000000pt}%
\definecolor{currentstroke}{rgb}{0.000000,0.000000,0.000000}%
\pgfsetstrokecolor{currentstroke}%
\pgfsetdash{}{0pt}%
\pgfpathmoveto{\pgfqpoint{2.981742in}{4.139401in}}%
\pgfpathlineto{\pgfqpoint{2.991406in}{4.010993in}}%
\pgfpathlineto{\pgfqpoint{2.999582in}{4.008701in}}%
\pgfpathlineto{\pgfqpoint{3.033860in}{3.995619in}}%
\pgfpathlineto{\pgfqpoint{3.067510in}{4.036583in}}%
\pgfpathlineto{\pgfqpoint{3.059910in}{3.981147in}}%
\pgfpathlineto{\pgfqpoint{3.049361in}{4.190168in}}%
\pgfpathlineto{\pgfqpoint{3.017517in}{3.995297in}}%
\pgfpathlineto{\pgfqpoint{2.981742in}{4.139401in}}%
\pgfpathclose%
\pgfusepath{fill}%
\end{pgfscope}%
\begin{pgfscope}%
\pgfpathrectangle{\pgfqpoint{1.020000in}{0.880000in}}{\pgfqpoint{6.160000in}{6.160000in}}%
\pgfusepath{clip}%
\pgfsetbuttcap%
\pgfsetroundjoin%
\definecolor{currentfill}{rgb}{0.667253,0.779176,0.992959}%
\pgfsetfillcolor{currentfill}%
\pgfsetlinewidth{0.000000pt}%
\definecolor{currentstroke}{rgb}{0.000000,0.000000,0.000000}%
\pgfsetstrokecolor{currentstroke}%
\pgfsetdash{}{0pt}%
\pgfpathmoveto{\pgfqpoint{3.972393in}{3.646659in}}%
\pgfpathlineto{\pgfqpoint{3.982020in}{3.434718in}}%
\pgfpathlineto{\pgfqpoint{3.991064in}{3.516414in}}%
\pgfpathlineto{\pgfqpoint{4.024712in}{3.637583in}}%
\pgfpathlineto{\pgfqpoint{4.058806in}{3.449778in}}%
\pgfpathlineto{\pgfqpoint{4.049376in}{3.570091in}}%
\pgfpathlineto{\pgfqpoint{4.040040in}{3.622376in}}%
\pgfpathlineto{\pgfqpoint{4.006611in}{3.419299in}}%
\pgfpathlineto{\pgfqpoint{3.972393in}{3.646659in}}%
\pgfpathclose%
\pgfusepath{fill}%
\end{pgfscope}%
\begin{pgfscope}%
\pgfpathrectangle{\pgfqpoint{1.020000in}{0.880000in}}{\pgfqpoint{6.160000in}{6.160000in}}%
\pgfusepath{clip}%
\pgfsetbuttcap%
\pgfsetroundjoin%
\definecolor{currentfill}{rgb}{0.786721,0.844807,0.939810}%
\pgfsetfillcolor{currentfill}%
\pgfsetlinewidth{0.000000pt}%
\definecolor{currentstroke}{rgb}{0.000000,0.000000,0.000000}%
\pgfsetstrokecolor{currentstroke}%
\pgfsetdash{}{0pt}%
\pgfpathmoveto{\pgfqpoint{3.443131in}{3.881089in}}%
\pgfpathlineto{\pgfqpoint{3.452733in}{3.748305in}}%
\pgfpathlineto{\pgfqpoint{3.461443in}{3.738879in}}%
\pgfpathlineto{\pgfqpoint{3.495396in}{3.745368in}}%
\pgfpathlineto{\pgfqpoint{3.529858in}{3.671285in}}%
\pgfpathlineto{\pgfqpoint{3.520421in}{3.782230in}}%
\pgfpathlineto{\pgfqpoint{3.512339in}{3.690824in}}%
\pgfpathlineto{\pgfqpoint{3.477300in}{3.853882in}}%
\pgfpathlineto{\pgfqpoint{3.443131in}{3.881089in}}%
\pgfpathclose%
\pgfusepath{fill}%
\end{pgfscope}%
\begin{pgfscope}%
\pgfpathrectangle{\pgfqpoint{1.020000in}{0.880000in}}{\pgfqpoint{6.160000in}{6.160000in}}%
\pgfusepath{clip}%
\pgfsetbuttcap%
\pgfsetroundjoin%
\definecolor{currentfill}{rgb}{0.510824,0.649397,0.985079}%
\pgfsetfillcolor{currentfill}%
\pgfsetlinewidth{0.000000pt}%
\definecolor{currentstroke}{rgb}{0.000000,0.000000,0.000000}%
\pgfsetstrokecolor{currentstroke}%
\pgfsetdash{}{0pt}%
\pgfpathmoveto{\pgfqpoint{4.433583in}{3.171769in}}%
\pgfpathlineto{\pgfqpoint{4.443564in}{3.287969in}}%
\pgfpathlineto{\pgfqpoint{4.453379in}{3.326432in}}%
\pgfpathlineto{\pgfqpoint{4.487129in}{3.331867in}}%
\pgfpathlineto{\pgfqpoint{4.520501in}{3.230359in}}%
\pgfpathlineto{\pgfqpoint{4.510484in}{3.155821in}}%
\pgfpathlineto{\pgfqpoint{4.501185in}{3.295314in}}%
\pgfpathlineto{\pgfqpoint{4.467369in}{3.232034in}}%
\pgfpathlineto{\pgfqpoint{4.433583in}{3.171769in}}%
\pgfpathclose%
\pgfusepath{fill}%
\end{pgfscope}%
\begin{pgfscope}%
\pgfpathrectangle{\pgfqpoint{1.020000in}{0.880000in}}{\pgfqpoint{6.160000in}{6.160000in}}%
\pgfusepath{clip}%
\pgfsetbuttcap%
\pgfsetroundjoin%
\definecolor{currentfill}{rgb}{0.724041,0.814910,0.975651}%
\pgfsetfillcolor{currentfill}%
\pgfsetlinewidth{0.000000pt}%
\definecolor{currentstroke}{rgb}{0.000000,0.000000,0.000000}%
\pgfsetstrokecolor{currentstroke}%
\pgfsetdash{}{0pt}%
\pgfpathmoveto{\pgfqpoint{3.818660in}{3.698910in}}%
\pgfpathlineto{\pgfqpoint{3.827596in}{3.725814in}}%
\pgfpathlineto{\pgfqpoint{3.836401in}{3.797848in}}%
\pgfpathlineto{\pgfqpoint{3.871039in}{3.576752in}}%
\pgfpathlineto{\pgfqpoint{3.905216in}{3.460452in}}%
\pgfpathlineto{\pgfqpoint{3.895974in}{3.511165in}}%
\pgfpathlineto{\pgfqpoint{3.886705in}{3.574201in}}%
\pgfpathlineto{\pgfqpoint{3.852274in}{3.769872in}}%
\pgfpathlineto{\pgfqpoint{3.818660in}{3.698910in}}%
\pgfpathclose%
\pgfusepath{fill}%
\end{pgfscope}%
\begin{pgfscope}%
\pgfpathrectangle{\pgfqpoint{1.020000in}{0.880000in}}{\pgfqpoint{6.160000in}{6.160000in}}%
\pgfusepath{clip}%
\pgfsetbuttcap%
\pgfsetroundjoin%
\definecolor{currentfill}{rgb}{0.875557,0.860242,0.851430}%
\pgfsetfillcolor{currentfill}%
\pgfsetlinewidth{0.000000pt}%
\definecolor{currentstroke}{rgb}{0.000000,0.000000,0.000000}%
\pgfsetstrokecolor{currentstroke}%
\pgfsetdash{}{0pt}%
\pgfpathmoveto{\pgfqpoint{2.627020in}{3.945642in}}%
\pgfpathlineto{\pgfqpoint{2.635317in}{3.910277in}}%
\pgfpathlineto{\pgfqpoint{2.644450in}{3.821060in}}%
\pgfpathlineto{\pgfqpoint{2.675291in}{4.045259in}}%
\pgfpathlineto{\pgfqpoint{2.710366in}{3.989321in}}%
\pgfpathlineto{\pgfqpoint{2.701571in}{4.054169in}}%
\pgfpathlineto{\pgfqpoint{2.694748in}{3.986278in}}%
\pgfpathlineto{\pgfqpoint{2.659921in}{4.029745in}}%
\pgfpathlineto{\pgfqpoint{2.627020in}{3.945642in}}%
\pgfpathclose%
\pgfusepath{fill}%
\end{pgfscope}%
\begin{pgfscope}%
\pgfpathrectangle{\pgfqpoint{1.020000in}{0.880000in}}{\pgfqpoint{6.160000in}{6.160000in}}%
\pgfusepath{clip}%
\pgfsetbuttcap%
\pgfsetroundjoin%
\definecolor{currentfill}{rgb}{0.328604,0.439712,0.869587}%
\pgfsetfillcolor{currentfill}%
\pgfsetlinewidth{0.000000pt}%
\definecolor{currentstroke}{rgb}{0.000000,0.000000,0.000000}%
\pgfsetstrokecolor{currentstroke}%
\pgfsetdash{}{0pt}%
\pgfpathmoveto{\pgfqpoint{5.846485in}{2.850680in}}%
\pgfpathlineto{\pgfqpoint{5.859066in}{2.932040in}}%
\pgfpathlineto{\pgfqpoint{5.868010in}{2.796365in}}%
\pgfpathlineto{\pgfqpoint{5.903204in}{2.908803in}}%
\pgfpathlineto{\pgfqpoint{5.936031in}{2.883759in}}%
\pgfpathlineto{\pgfqpoint{5.927896in}{3.064688in}}%
\pgfpathlineto{\pgfqpoint{5.914059in}{2.918863in}}%
\pgfpathlineto{\pgfqpoint{5.879658in}{2.848851in}}%
\pgfpathlineto{\pgfqpoint{5.846485in}{2.850680in}}%
\pgfpathclose%
\pgfusepath{fill}%
\end{pgfscope}%
\begin{pgfscope}%
\pgfpathrectangle{\pgfqpoint{1.020000in}{0.880000in}}{\pgfqpoint{6.160000in}{6.160000in}}%
\pgfusepath{clip}%
\pgfsetbuttcap%
\pgfsetroundjoin%
\definecolor{currentfill}{rgb}{0.630089,0.752516,0.998508}%
\pgfsetfillcolor{currentfill}%
\pgfsetlinewidth{0.000000pt}%
\definecolor{currentstroke}{rgb}{0.000000,0.000000,0.000000}%
\pgfsetstrokecolor{currentstroke}%
\pgfsetdash{}{0pt}%
\pgfpathmoveto{\pgfqpoint{4.126334in}{3.460088in}}%
\pgfpathlineto{\pgfqpoint{4.135689in}{3.464078in}}%
\pgfpathlineto{\pgfqpoint{4.145056in}{3.493548in}}%
\pgfpathlineto{\pgfqpoint{4.178876in}{3.453799in}}%
\pgfpathlineto{\pgfqpoint{4.212656in}{3.442466in}}%
\pgfpathlineto{\pgfqpoint{4.203221in}{3.480660in}}%
\pgfpathlineto{\pgfqpoint{4.193799in}{3.520851in}}%
\pgfpathlineto{\pgfqpoint{4.160098in}{3.352569in}}%
\pgfpathlineto{\pgfqpoint{4.126334in}{3.460088in}}%
\pgfpathclose%
\pgfusepath{fill}%
\end{pgfscope}%
\begin{pgfscope}%
\pgfpathrectangle{\pgfqpoint{1.020000in}{0.880000in}}{\pgfqpoint{6.160000in}{6.160000in}}%
\pgfusepath{clip}%
\pgfsetbuttcap%
\pgfsetroundjoin%
\definecolor{currentfill}{rgb}{0.809329,0.852974,0.922323}%
\pgfsetfillcolor{currentfill}%
\pgfsetlinewidth{0.000000pt}%
\definecolor{currentstroke}{rgb}{0.000000,0.000000,0.000000}%
\pgfsetstrokecolor{currentstroke}%
\pgfsetdash{}{0pt}%
\pgfpathmoveto{\pgfqpoint{3.222185in}{3.864867in}}%
\pgfpathlineto{\pgfqpoint{3.231166in}{3.803004in}}%
\pgfpathlineto{\pgfqpoint{3.240303in}{3.725381in}}%
\pgfpathlineto{\pgfqpoint{3.273303in}{3.840888in}}%
\pgfpathlineto{\pgfqpoint{3.308412in}{3.721222in}}%
\pgfpathlineto{\pgfqpoint{3.299319in}{3.792178in}}%
\pgfpathlineto{\pgfqpoint{3.289955in}{3.894326in}}%
\pgfpathlineto{\pgfqpoint{3.256089in}{3.878101in}}%
\pgfpathlineto{\pgfqpoint{3.222185in}{3.864867in}}%
\pgfpathclose%
\pgfusepath{fill}%
\end{pgfscope}%
\begin{pgfscope}%
\pgfpathrectangle{\pgfqpoint{1.020000in}{0.880000in}}{\pgfqpoint{6.160000in}{6.160000in}}%
\pgfusepath{clip}%
\pgfsetbuttcap%
\pgfsetroundjoin%
\definecolor{currentfill}{rgb}{0.855378,0.863778,0.876587}%
\pgfsetfillcolor{currentfill}%
\pgfsetlinewidth{0.000000pt}%
\definecolor{currentstroke}{rgb}{0.000000,0.000000,0.000000}%
\pgfsetstrokecolor{currentstroke}%
\pgfsetdash{}{0pt}%
\pgfpathmoveto{\pgfqpoint{2.559241in}{3.906609in}}%
\pgfpathlineto{\pgfqpoint{2.567178in}{3.890656in}}%
\pgfpathlineto{\pgfqpoint{2.575143in}{3.873809in}}%
\pgfpathlineto{\pgfqpoint{2.608467in}{3.933526in}}%
\pgfpathlineto{\pgfqpoint{2.644450in}{3.821060in}}%
\pgfpathlineto{\pgfqpoint{2.635317in}{3.910277in}}%
\pgfpathlineto{\pgfqpoint{2.627020in}{3.945642in}}%
\pgfpathlineto{\pgfqpoint{2.590504in}{4.092374in}}%
\pgfpathlineto{\pgfqpoint{2.559241in}{3.906609in}}%
\pgfpathclose%
\pgfusepath{fill}%
\end{pgfscope}%
\begin{pgfscope}%
\pgfpathrectangle{\pgfqpoint{1.020000in}{0.880000in}}{\pgfqpoint{6.160000in}{6.160000in}}%
\pgfusepath{clip}%
\pgfsetbuttcap%
\pgfsetroundjoin%
\definecolor{currentfill}{rgb}{0.875557,0.860242,0.851430}%
\pgfsetfillcolor{currentfill}%
\pgfsetlinewidth{0.000000pt}%
\definecolor{currentstroke}{rgb}{0.000000,0.000000,0.000000}%
\pgfsetstrokecolor{currentstroke}%
\pgfsetdash{}{0pt}%
\pgfpathmoveto{\pgfqpoint{3.135365in}{4.067279in}}%
\pgfpathlineto{\pgfqpoint{3.144745in}{3.963532in}}%
\pgfpathlineto{\pgfqpoint{3.152999in}{3.969593in}}%
\pgfpathlineto{\pgfqpoint{3.187342in}{3.945266in}}%
\pgfpathlineto{\pgfqpoint{3.222185in}{3.864867in}}%
\pgfpathlineto{\pgfqpoint{3.212413in}{4.009533in}}%
\pgfpathlineto{\pgfqpoint{3.204429in}{3.969800in}}%
\pgfpathlineto{\pgfqpoint{3.170209in}{3.990278in}}%
\pgfpathlineto{\pgfqpoint{3.135365in}{4.067279in}}%
\pgfpathclose%
\pgfusepath{fill}%
\end{pgfscope}%
\begin{pgfscope}%
\pgfpathrectangle{\pgfqpoint{1.020000in}{0.880000in}}{\pgfqpoint{6.160000in}{6.160000in}}%
\pgfusepath{clip}%
\pgfsetbuttcap%
\pgfsetroundjoin%
\definecolor{currentfill}{rgb}{0.576051,0.708780,0.997755}%
\pgfsetfillcolor{currentfill}%
\pgfsetlinewidth{0.000000pt}%
\definecolor{currentstroke}{rgb}{0.000000,0.000000,0.000000}%
\pgfsetstrokecolor{currentstroke}%
\pgfsetdash{}{0pt}%
\pgfpathmoveto{\pgfqpoint{4.279966in}{3.245493in}}%
\pgfpathlineto{\pgfqpoint{4.289666in}{3.422867in}}%
\pgfpathlineto{\pgfqpoint{4.299146in}{3.363381in}}%
\pgfpathlineto{\pgfqpoint{4.332630in}{3.179764in}}%
\pgfpathlineto{\pgfqpoint{4.366733in}{3.411526in}}%
\pgfpathlineto{\pgfqpoint{4.357286in}{3.507611in}}%
\pgfpathlineto{\pgfqpoint{4.347509in}{3.401636in}}%
\pgfpathlineto{\pgfqpoint{4.313835in}{3.411150in}}%
\pgfpathlineto{\pgfqpoint{4.279966in}{3.245493in}}%
\pgfpathclose%
\pgfusepath{fill}%
\end{pgfscope}%
\begin{pgfscope}%
\pgfpathrectangle{\pgfqpoint{1.020000in}{0.880000in}}{\pgfqpoint{6.160000in}{6.160000in}}%
\pgfusepath{clip}%
\pgfsetbuttcap%
\pgfsetroundjoin%
\definecolor{currentfill}{rgb}{0.906154,0.842091,0.806151}%
\pgfsetfillcolor{currentfill}%
\pgfsetlinewidth{0.000000pt}%
\definecolor{currentstroke}{rgb}{0.000000,0.000000,0.000000}%
\pgfsetstrokecolor{currentstroke}%
\pgfsetdash{}{0pt}%
\pgfpathmoveto{\pgfqpoint{2.694748in}{3.986278in}}%
\pgfpathlineto{\pgfqpoint{2.701571in}{4.054169in}}%
\pgfpathlineto{\pgfqpoint{2.710366in}{3.989321in}}%
\pgfpathlineto{\pgfqpoint{2.742973in}{4.101876in}}%
\pgfpathlineto{\pgfqpoint{2.778473in}{4.011336in}}%
\pgfpathlineto{\pgfqpoint{2.769377in}{4.095736in}}%
\pgfpathlineto{\pgfqpoint{2.761359in}{4.104042in}}%
\pgfpathlineto{\pgfqpoint{2.727483in}{4.083892in}}%
\pgfpathlineto{\pgfqpoint{2.694748in}{3.986278in}}%
\pgfpathclose%
\pgfusepath{fill}%
\end{pgfscope}%
\begin{pgfscope}%
\pgfpathrectangle{\pgfqpoint{1.020000in}{0.880000in}}{\pgfqpoint{6.160000in}{6.160000in}}%
\pgfusepath{clip}%
\pgfsetbuttcap%
\pgfsetroundjoin%
\definecolor{currentfill}{rgb}{0.425199,0.559058,0.946061}%
\pgfsetfillcolor{currentfill}%
\pgfsetlinewidth{0.000000pt}%
\definecolor{currentstroke}{rgb}{0.000000,0.000000,0.000000}%
\pgfsetstrokecolor{currentstroke}%
\pgfsetdash{}{0pt}%
\pgfpathmoveto{\pgfqpoint{4.741614in}{3.131661in}}%
\pgfpathlineto{\pgfqpoint{4.751330in}{3.077363in}}%
\pgfpathlineto{\pgfqpoint{4.761878in}{3.164631in}}%
\pgfpathlineto{\pgfqpoint{4.794138in}{2.938209in}}%
\pgfpathlineto{\pgfqpoint{4.828834in}{3.114706in}}%
\pgfpathlineto{\pgfqpoint{4.817878in}{2.982383in}}%
\pgfpathlineto{\pgfqpoint{4.808459in}{3.087243in}}%
\pgfpathlineto{\pgfqpoint{4.775912in}{3.253949in}}%
\pgfpathlineto{\pgfqpoint{4.741614in}{3.131661in}}%
\pgfpathclose%
\pgfusepath{fill}%
\end{pgfscope}%
\begin{pgfscope}%
\pgfpathrectangle{\pgfqpoint{1.020000in}{0.880000in}}{\pgfqpoint{6.160000in}{6.160000in}}%
\pgfusepath{clip}%
\pgfsetbuttcap%
\pgfsetroundjoin%
\definecolor{currentfill}{rgb}{0.871493,0.862309,0.857016}%
\pgfsetfillcolor{currentfill}%
\pgfsetlinewidth{0.000000pt}%
\definecolor{currentstroke}{rgb}{0.000000,0.000000,0.000000}%
\pgfsetstrokecolor{currentstroke}%
\pgfsetdash{}{0pt}%
\pgfpathmoveto{\pgfqpoint{2.844554in}{4.181870in}}%
\pgfpathlineto{\pgfqpoint{2.855602in}{3.951767in}}%
\pgfpathlineto{\pgfqpoint{2.864256in}{3.902550in}}%
\pgfpathlineto{\pgfqpoint{2.896941in}{4.019040in}}%
\pgfpathlineto{\pgfqpoint{2.932812in}{3.884300in}}%
\pgfpathlineto{\pgfqpoint{2.924322in}{3.917719in}}%
\pgfpathlineto{\pgfqpoint{2.917017in}{3.858143in}}%
\pgfpathlineto{\pgfqpoint{2.881269in}{3.987713in}}%
\pgfpathlineto{\pgfqpoint{2.844554in}{4.181870in}}%
\pgfpathclose%
\pgfusepath{fill}%
\end{pgfscope}%
\begin{pgfscope}%
\pgfpathrectangle{\pgfqpoint{1.020000in}{0.880000in}}{\pgfqpoint{6.160000in}{6.160000in}}%
\pgfusepath{clip}%
\pgfsetbuttcap%
\pgfsetroundjoin%
\definecolor{currentfill}{rgb}{0.328604,0.439712,0.869587}%
\pgfsetfillcolor{currentfill}%
\pgfsetlinewidth{0.000000pt}%
\definecolor{currentstroke}{rgb}{0.000000,0.000000,0.000000}%
\pgfsetstrokecolor{currentstroke}%
\pgfsetdash{}{0pt}%
\pgfpathmoveto{\pgfqpoint{5.559619in}{2.945565in}}%
\pgfpathlineto{\pgfqpoint{5.570384in}{2.934953in}}%
\pgfpathlineto{\pgfqpoint{5.581004in}{2.912537in}}%
\pgfpathlineto{\pgfqpoint{5.611818in}{2.733833in}}%
\pgfpathlineto{\pgfqpoint{5.647392in}{2.887955in}}%
\pgfpathlineto{\pgfqpoint{5.636032in}{2.863004in}}%
\pgfpathlineto{\pgfqpoint{5.626118in}{2.935755in}}%
\pgfpathlineto{\pgfqpoint{5.593881in}{3.011304in}}%
\pgfpathlineto{\pgfqpoint{5.559619in}{2.945565in}}%
\pgfpathclose%
\pgfusepath{fill}%
\end{pgfscope}%
\begin{pgfscope}%
\pgfpathrectangle{\pgfqpoint{1.020000in}{0.880000in}}{\pgfqpoint{6.160000in}{6.160000in}}%
\pgfusepath{clip}%
\pgfsetbuttcap%
\pgfsetroundjoin%
\definecolor{currentfill}{rgb}{0.667253,0.779176,0.992959}%
\pgfsetfillcolor{currentfill}%
\pgfsetlinewidth{0.000000pt}%
\definecolor{currentstroke}{rgb}{0.000000,0.000000,0.000000}%
\pgfsetstrokecolor{currentstroke}%
\pgfsetdash{}{0pt}%
\pgfpathmoveto{\pgfqpoint{3.905216in}{3.460452in}}%
\pgfpathlineto{\pgfqpoint{3.914100in}{3.547153in}}%
\pgfpathlineto{\pgfqpoint{3.923200in}{3.564112in}}%
\pgfpathlineto{\pgfqpoint{3.956997in}{3.608117in}}%
\pgfpathlineto{\pgfqpoint{3.991064in}{3.516414in}}%
\pgfpathlineto{\pgfqpoint{3.982020in}{3.434718in}}%
\pgfpathlineto{\pgfqpoint{3.972393in}{3.646659in}}%
\pgfpathlineto{\pgfqpoint{3.939111in}{3.419102in}}%
\pgfpathlineto{\pgfqpoint{3.905216in}{3.460452in}}%
\pgfpathclose%
\pgfusepath{fill}%
\end{pgfscope}%
\begin{pgfscope}%
\pgfpathrectangle{\pgfqpoint{1.020000in}{0.880000in}}{\pgfqpoint{6.160000in}{6.160000in}}%
\pgfusepath{clip}%
\pgfsetbuttcap%
\pgfsetroundjoin%
\definecolor{currentfill}{rgb}{0.818056,0.855590,0.914638}%
\pgfsetfillcolor{currentfill}%
\pgfsetlinewidth{0.000000pt}%
\definecolor{currentstroke}{rgb}{0.000000,0.000000,0.000000}%
\pgfsetstrokecolor{currentstroke}%
\pgfsetdash{}{0pt}%
\pgfpathmoveto{\pgfqpoint{2.424049in}{3.802762in}}%
\pgfpathlineto{\pgfqpoint{2.430636in}{3.857112in}}%
\pgfpathlineto{\pgfqpoint{2.437374in}{3.904136in}}%
\pgfpathlineto{\pgfqpoint{2.473688in}{3.788946in}}%
\pgfpathlineto{\pgfqpoint{2.510078in}{3.663253in}}%
\pgfpathlineto{\pgfqpoint{2.499777in}{3.823741in}}%
\pgfpathlineto{\pgfqpoint{2.491537in}{3.861449in}}%
\pgfpathlineto{\pgfqpoint{2.455755in}{3.950688in}}%
\pgfpathlineto{\pgfqpoint{2.424049in}{3.802762in}}%
\pgfpathclose%
\pgfusepath{fill}%
\end{pgfscope}%
\begin{pgfscope}%
\pgfpathrectangle{\pgfqpoint{1.020000in}{0.880000in}}{\pgfqpoint{6.160000in}{6.160000in}}%
\pgfusepath{clip}%
\pgfsetbuttcap%
\pgfsetroundjoin%
\definecolor{currentfill}{rgb}{0.388852,0.516298,0.921373}%
\pgfsetfillcolor{currentfill}%
\pgfsetlinewidth{0.000000pt}%
\definecolor{currentstroke}{rgb}{0.000000,0.000000,0.000000}%
\pgfsetstrokecolor{currentstroke}%
\pgfsetdash{}{0pt}%
\pgfpathmoveto{\pgfqpoint{4.895931in}{3.105252in}}%
\pgfpathlineto{\pgfqpoint{4.905043in}{2.955231in}}%
\pgfpathlineto{\pgfqpoint{4.915146in}{2.941269in}}%
\pgfpathlineto{\pgfqpoint{4.949290in}{3.017143in}}%
\pgfpathlineto{\pgfqpoint{4.982759in}{3.005213in}}%
\pgfpathlineto{\pgfqpoint{4.972016in}{2.946508in}}%
\pgfpathlineto{\pgfqpoint{4.962250in}{3.008271in}}%
\pgfpathlineto{\pgfqpoint{4.930006in}{3.175458in}}%
\pgfpathlineto{\pgfqpoint{4.895931in}{3.105252in}}%
\pgfpathclose%
\pgfusepath{fill}%
\end{pgfscope}%
\begin{pgfscope}%
\pgfpathrectangle{\pgfqpoint{1.020000in}{0.880000in}}{\pgfqpoint{6.160000in}{6.160000in}}%
\pgfusepath{clip}%
\pgfsetbuttcap%
\pgfsetroundjoin%
\definecolor{currentfill}{rgb}{0.323718,0.433158,0.864722}%
\pgfsetfillcolor{currentfill}%
\pgfsetlinewidth{0.000000pt}%
\definecolor{currentstroke}{rgb}{0.000000,0.000000,0.000000}%
\pgfsetstrokecolor{currentstroke}%
\pgfsetdash{}{0pt}%
\pgfpathmoveto{\pgfqpoint{6.071432in}{3.018897in}}%
\pgfpathlineto{\pgfqpoint{6.080709in}{2.903741in}}%
\pgfpathlineto{\pgfqpoint{6.090602in}{2.820794in}}%
\pgfpathlineto{\pgfqpoint{6.124812in}{2.872331in}}%
\pgfpathlineto{\pgfqpoint{6.110765in}{2.741817in}}%
\pgfpathlineto{\pgfqpoint{6.104451in}{3.009294in}}%
\pgfpathlineto{\pgfqpoint{6.071432in}{3.018897in}}%
\pgfpathclose%
\pgfusepath{fill}%
\end{pgfscope}%
\begin{pgfscope}%
\pgfpathrectangle{\pgfqpoint{1.020000in}{0.880000in}}{\pgfqpoint{6.160000in}{6.160000in}}%
\pgfusepath{clip}%
\pgfsetbuttcap%
\pgfsetroundjoin%
\definecolor{currentfill}{rgb}{0.738826,0.822572,0.968261}%
\pgfsetfillcolor{currentfill}%
\pgfsetlinewidth{0.000000pt}%
\definecolor{currentstroke}{rgb}{0.000000,0.000000,0.000000}%
\pgfsetstrokecolor{currentstroke}%
\pgfsetdash{}{0pt}%
\pgfpathmoveto{\pgfqpoint{3.529858in}{3.671285in}}%
\pgfpathlineto{\pgfqpoint{3.538742in}{3.645986in}}%
\pgfpathlineto{\pgfqpoint{3.547205in}{3.689445in}}%
\pgfpathlineto{\pgfqpoint{3.581390in}{3.659917in}}%
\pgfpathlineto{\pgfqpoint{3.615791in}{3.585042in}}%
\pgfpathlineto{\pgfqpoint{3.605767in}{3.797496in}}%
\pgfpathlineto{\pgfqpoint{3.598404in}{3.551520in}}%
\pgfpathlineto{\pgfqpoint{3.563394in}{3.735488in}}%
\pgfpathlineto{\pgfqpoint{3.529858in}{3.671285in}}%
\pgfpathclose%
\pgfusepath{fill}%
\end{pgfscope}%
\begin{pgfscope}%
\pgfpathrectangle{\pgfqpoint{1.020000in}{0.880000in}}{\pgfqpoint{6.160000in}{6.160000in}}%
\pgfusepath{clip}%
\pgfsetbuttcap%
\pgfsetroundjoin%
\definecolor{currentfill}{rgb}{0.871493,0.862309,0.857016}%
\pgfsetfillcolor{currentfill}%
\pgfsetlinewidth{0.000000pt}%
\definecolor{currentstroke}{rgb}{0.000000,0.000000,0.000000}%
\pgfsetstrokecolor{currentstroke}%
\pgfsetdash{}{0pt}%
\pgfpathmoveto{\pgfqpoint{3.067510in}{4.036583in}}%
\pgfpathlineto{\pgfqpoint{3.077160in}{3.908246in}}%
\pgfpathlineto{\pgfqpoint{3.085827in}{3.869496in}}%
\pgfpathlineto{\pgfqpoint{3.120106in}{3.853499in}}%
\pgfpathlineto{\pgfqpoint{3.152999in}{3.969593in}}%
\pgfpathlineto{\pgfqpoint{3.144745in}{3.963532in}}%
\pgfpathlineto{\pgfqpoint{3.135365in}{4.067279in}}%
\pgfpathlineto{\pgfqpoint{3.102027in}{3.997579in}}%
\pgfpathlineto{\pgfqpoint{3.067510in}{4.036583in}}%
\pgfpathclose%
\pgfusepath{fill}%
\end{pgfscope}%
\begin{pgfscope}%
\pgfpathrectangle{\pgfqpoint{1.020000in}{0.880000in}}{\pgfqpoint{6.160000in}{6.160000in}}%
\pgfusepath{clip}%
\pgfsetbuttcap%
\pgfsetroundjoin%
\definecolor{currentfill}{rgb}{0.328604,0.439712,0.869587}%
\pgfsetfillcolor{currentfill}%
\pgfsetlinewidth{0.000000pt}%
\definecolor{currentstroke}{rgb}{0.000000,0.000000,0.000000}%
\pgfsetstrokecolor{currentstroke}%
\pgfsetdash{}{0pt}%
\pgfpathmoveto{\pgfqpoint{5.781834in}{2.963730in}}%
\pgfpathlineto{\pgfqpoint{5.790725in}{2.823558in}}%
\pgfpathlineto{\pgfqpoint{5.801295in}{2.786941in}}%
\pgfpathlineto{\pgfqpoint{5.838733in}{3.036640in}}%
\pgfpathlineto{\pgfqpoint{5.868010in}{2.796365in}}%
\pgfpathlineto{\pgfqpoint{5.859066in}{2.932040in}}%
\pgfpathlineto{\pgfqpoint{5.846485in}{2.850680in}}%
\pgfpathlineto{\pgfqpoint{5.815367in}{2.979095in}}%
\pgfpathlineto{\pgfqpoint{5.781834in}{2.963730in}}%
\pgfpathclose%
\pgfusepath{fill}%
\end{pgfscope}%
\begin{pgfscope}%
\pgfpathrectangle{\pgfqpoint{1.020000in}{0.880000in}}{\pgfqpoint{6.160000in}{6.160000in}}%
\pgfusepath{clip}%
\pgfsetbuttcap%
\pgfsetroundjoin%
\definecolor{currentfill}{rgb}{0.383662,0.510183,0.917831}%
\pgfsetfillcolor{currentfill}%
\pgfsetlinewidth{0.000000pt}%
\definecolor{currentstroke}{rgb}{0.000000,0.000000,0.000000}%
\pgfsetstrokecolor{currentstroke}%
\pgfsetdash{}{0pt}%
\pgfpathmoveto{\pgfqpoint{5.116526in}{2.975658in}}%
\pgfpathlineto{\pgfqpoint{5.126566in}{2.934726in}}%
\pgfpathlineto{\pgfqpoint{5.136800in}{2.912384in}}%
\pgfpathlineto{\pgfqpoint{5.169962in}{2.878826in}}%
\pgfpathlineto{\pgfqpoint{5.204760in}{3.008910in}}%
\pgfpathlineto{\pgfqpoint{5.195933in}{3.175818in}}%
\pgfpathlineto{\pgfqpoint{5.184678in}{3.102211in}}%
\pgfpathlineto{\pgfqpoint{5.150525in}{3.032192in}}%
\pgfpathlineto{\pgfqpoint{5.116526in}{2.975658in}}%
\pgfpathclose%
\pgfusepath{fill}%
\end{pgfscope}%
\begin{pgfscope}%
\pgfpathrectangle{\pgfqpoint{1.020000in}{0.880000in}}{\pgfqpoint{6.160000in}{6.160000in}}%
\pgfusepath{clip}%
\pgfsetbuttcap%
\pgfsetroundjoin%
\definecolor{currentfill}{rgb}{0.851372,0.863125,0.881064}%
\pgfsetfillcolor{currentfill}%
\pgfsetlinewidth{0.000000pt}%
\definecolor{currentstroke}{rgb}{0.000000,0.000000,0.000000}%
\pgfsetstrokecolor{currentstroke}%
\pgfsetdash{}{0pt}%
\pgfpathmoveto{\pgfqpoint{2.999582in}{4.008701in}}%
\pgfpathlineto{\pgfqpoint{3.010118in}{3.805963in}}%
\pgfpathlineto{\pgfqpoint{3.017878in}{3.841059in}}%
\pgfpathlineto{\pgfqpoint{3.052792in}{3.772340in}}%
\pgfpathlineto{\pgfqpoint{3.085827in}{3.869496in}}%
\pgfpathlineto{\pgfqpoint{3.077160in}{3.908246in}}%
\pgfpathlineto{\pgfqpoint{3.067510in}{4.036583in}}%
\pgfpathlineto{\pgfqpoint{3.033860in}{3.995619in}}%
\pgfpathlineto{\pgfqpoint{2.999582in}{4.008701in}}%
\pgfpathclose%
\pgfusepath{fill}%
\end{pgfscope}%
\begin{pgfscope}%
\pgfpathrectangle{\pgfqpoint{1.020000in}{0.880000in}}{\pgfqpoint{6.160000in}{6.160000in}}%
\pgfusepath{clip}%
\pgfsetbuttcap%
\pgfsetroundjoin%
\definecolor{currentfill}{rgb}{0.363461,0.484784,0.901019}%
\pgfsetfillcolor{currentfill}%
\pgfsetlinewidth{0.000000pt}%
\definecolor{currentstroke}{rgb}{0.000000,0.000000,0.000000}%
\pgfsetstrokecolor{currentstroke}%
\pgfsetdash{}{0pt}%
\pgfpathmoveto{\pgfqpoint{5.338009in}{2.958983in}}%
\pgfpathlineto{\pgfqpoint{5.348497in}{2.943446in}}%
\pgfpathlineto{\pgfqpoint{5.360028in}{3.014454in}}%
\pgfpathlineto{\pgfqpoint{5.392227in}{2.909492in}}%
\pgfpathlineto{\pgfqpoint{5.425252in}{2.877970in}}%
\pgfpathlineto{\pgfqpoint{5.416283in}{3.021840in}}%
\pgfpathlineto{\pgfqpoint{5.406333in}{3.086461in}}%
\pgfpathlineto{\pgfqpoint{5.371274in}{2.948557in}}%
\pgfpathlineto{\pgfqpoint{5.338009in}{2.958983in}}%
\pgfpathclose%
\pgfusepath{fill}%
\end{pgfscope}%
\begin{pgfscope}%
\pgfpathrectangle{\pgfqpoint{1.020000in}{0.880000in}}{\pgfqpoint{6.160000in}{6.160000in}}%
\pgfusepath{clip}%
\pgfsetbuttcap%
\pgfsetroundjoin%
\definecolor{currentfill}{rgb}{0.419991,0.552989,0.942630}%
\pgfsetfillcolor{currentfill}%
\pgfsetlinewidth{0.000000pt}%
\definecolor{currentstroke}{rgb}{0.000000,0.000000,0.000000}%
\pgfsetstrokecolor{currentstroke}%
\pgfsetdash{}{0pt}%
\pgfpathmoveto{\pgfqpoint{5.184678in}{3.102211in}}%
\pgfpathlineto{\pgfqpoint{5.195933in}{3.175818in}}%
\pgfpathlineto{\pgfqpoint{5.204760in}{3.008910in}}%
\pgfpathlineto{\pgfqpoint{5.238268in}{3.010824in}}%
\pgfpathlineto{\pgfqpoint{5.270509in}{2.898674in}}%
\pgfpathlineto{\pgfqpoint{5.263266in}{3.206504in}}%
\pgfpathlineto{\pgfqpoint{5.250521in}{3.005000in}}%
\pgfpathlineto{\pgfqpoint{5.218807in}{3.167327in}}%
\pgfpathlineto{\pgfqpoint{5.184678in}{3.102211in}}%
\pgfpathclose%
\pgfusepath{fill}%
\end{pgfscope}%
\begin{pgfscope}%
\pgfpathrectangle{\pgfqpoint{1.020000in}{0.880000in}}{\pgfqpoint{6.160000in}{6.160000in}}%
\pgfusepath{clip}%
\pgfsetbuttcap%
\pgfsetroundjoin%
\definecolor{currentfill}{rgb}{0.871493,0.862309,0.857016}%
\pgfsetfillcolor{currentfill}%
\pgfsetlinewidth{0.000000pt}%
\definecolor{currentstroke}{rgb}{0.000000,0.000000,0.000000}%
\pgfsetstrokecolor{currentstroke}%
\pgfsetdash{}{0pt}%
\pgfpathmoveto{\pgfqpoint{2.778473in}{4.011336in}}%
\pgfpathlineto{\pgfqpoint{2.786037in}{4.037489in}}%
\pgfpathlineto{\pgfqpoint{2.796254in}{3.873043in}}%
\pgfpathlineto{\pgfqpoint{2.832184in}{3.744961in}}%
\pgfpathlineto{\pgfqpoint{2.864256in}{3.902550in}}%
\pgfpathlineto{\pgfqpoint{2.855602in}{3.951767in}}%
\pgfpathlineto{\pgfqpoint{2.844554in}{4.181870in}}%
\pgfpathlineto{\pgfqpoint{2.814062in}{3.908165in}}%
\pgfpathlineto{\pgfqpoint{2.778473in}{4.011336in}}%
\pgfpathclose%
\pgfusepath{fill}%
\end{pgfscope}%
\begin{pgfscope}%
\pgfpathrectangle{\pgfqpoint{1.020000in}{0.880000in}}{\pgfqpoint{6.160000in}{6.160000in}}%
\pgfusepath{clip}%
\pgfsetbuttcap%
\pgfsetroundjoin%
\definecolor{currentfill}{rgb}{0.328604,0.439712,0.869587}%
\pgfsetfillcolor{currentfill}%
\pgfsetlinewidth{0.000000pt}%
\definecolor{currentstroke}{rgb}{0.000000,0.000000,0.000000}%
\pgfsetstrokecolor{currentstroke}%
\pgfsetdash{}{0pt}%
\pgfpathmoveto{\pgfqpoint{5.491799in}{2.862068in}}%
\pgfpathlineto{\pgfqpoint{5.502068in}{2.819172in}}%
\pgfpathlineto{\pgfqpoint{5.513172in}{2.837697in}}%
\pgfpathlineto{\pgfqpoint{5.546841in}{2.857315in}}%
\pgfpathlineto{\pgfqpoint{5.581004in}{2.912537in}}%
\pgfpathlineto{\pgfqpoint{5.570384in}{2.934953in}}%
\pgfpathlineto{\pgfqpoint{5.559619in}{2.945565in}}%
\pgfpathlineto{\pgfqpoint{5.526744in}{2.980769in}}%
\pgfpathlineto{\pgfqpoint{5.491799in}{2.862068in}}%
\pgfpathclose%
\pgfusepath{fill}%
\end{pgfscope}%
\begin{pgfscope}%
\pgfpathrectangle{\pgfqpoint{1.020000in}{0.880000in}}{\pgfqpoint{6.160000in}{6.160000in}}%
\pgfusepath{clip}%
\pgfsetbuttcap%
\pgfsetroundjoin%
\definecolor{currentfill}{rgb}{0.822420,0.856898,0.910795}%
\pgfsetfillcolor{currentfill}%
\pgfsetlinewidth{0.000000pt}%
\definecolor{currentstroke}{rgb}{0.000000,0.000000,0.000000}%
\pgfsetstrokecolor{currentstroke}%
\pgfsetdash{}{0pt}%
\pgfpathmoveto{\pgfqpoint{2.644450in}{3.821060in}}%
\pgfpathlineto{\pgfqpoint{2.653713in}{3.723507in}}%
\pgfpathlineto{\pgfqpoint{2.660477in}{3.790806in}}%
\pgfpathlineto{\pgfqpoint{2.695132in}{3.764070in}}%
\pgfpathlineto{\pgfqpoint{2.728557in}{3.819565in}}%
\pgfpathlineto{\pgfqpoint{2.721281in}{3.779452in}}%
\pgfpathlineto{\pgfqpoint{2.710366in}{3.989321in}}%
\pgfpathlineto{\pgfqpoint{2.675291in}{4.045259in}}%
\pgfpathlineto{\pgfqpoint{2.644450in}{3.821060in}}%
\pgfpathclose%
\pgfusepath{fill}%
\end{pgfscope}%
\begin{pgfscope}%
\pgfpathrectangle{\pgfqpoint{1.020000in}{0.880000in}}{\pgfqpoint{6.160000in}{6.160000in}}%
\pgfusepath{clip}%
\pgfsetbuttcap%
\pgfsetroundjoin%
\definecolor{currentfill}{rgb}{0.299441,0.400248,0.839842}%
\pgfsetfillcolor{currentfill}%
\pgfsetlinewidth{0.000000pt}%
\definecolor{currentstroke}{rgb}{0.000000,0.000000,0.000000}%
\pgfsetstrokecolor{currentstroke}%
\pgfsetdash{}{0pt}%
\pgfpathmoveto{\pgfqpoint{5.425252in}{2.877970in}}%
\pgfpathlineto{\pgfqpoint{5.436950in}{2.951343in}}%
\pgfpathlineto{\pgfqpoint{5.447366in}{2.920710in}}%
\pgfpathlineto{\pgfqpoint{5.477273in}{2.647660in}}%
\pgfpathlineto{\pgfqpoint{5.513172in}{2.837697in}}%
\pgfpathlineto{\pgfqpoint{5.502068in}{2.819172in}}%
\pgfpathlineto{\pgfqpoint{5.491799in}{2.862068in}}%
\pgfpathlineto{\pgfqpoint{5.457485in}{2.787877in}}%
\pgfpathlineto{\pgfqpoint{5.425252in}{2.877970in}}%
\pgfpathclose%
\pgfusepath{fill}%
\end{pgfscope}%
\begin{pgfscope}%
\pgfpathrectangle{\pgfqpoint{1.020000in}{0.880000in}}{\pgfqpoint{6.160000in}{6.160000in}}%
\pgfusepath{clip}%
\pgfsetbuttcap%
\pgfsetroundjoin%
\definecolor{currentfill}{rgb}{0.313946,0.420052,0.854993}%
\pgfsetfillcolor{currentfill}%
\pgfsetlinewidth{0.000000pt}%
\definecolor{currentstroke}{rgb}{0.000000,0.000000,0.000000}%
\pgfsetstrokecolor{currentstroke}%
\pgfsetdash{}{0pt}%
\pgfpathmoveto{\pgfqpoint{5.712394in}{2.780705in}}%
\pgfpathlineto{\pgfqpoint{5.727214in}{3.022454in}}%
\pgfpathlineto{\pgfqpoint{5.735416in}{2.835369in}}%
\pgfpathlineto{\pgfqpoint{5.770035in}{2.915299in}}%
\pgfpathlineto{\pgfqpoint{5.801295in}{2.786941in}}%
\pgfpathlineto{\pgfqpoint{5.790725in}{2.823558in}}%
\pgfpathlineto{\pgfqpoint{5.781834in}{2.963730in}}%
\pgfpathlineto{\pgfqpoint{5.746196in}{2.815189in}}%
\pgfpathlineto{\pgfqpoint{5.712394in}{2.780705in}}%
\pgfpathclose%
\pgfusepath{fill}%
\end{pgfscope}%
\begin{pgfscope}%
\pgfpathrectangle{\pgfqpoint{1.020000in}{0.880000in}}{\pgfqpoint{6.160000in}{6.160000in}}%
\pgfusepath{clip}%
\pgfsetbuttcap%
\pgfsetroundjoin%
\definecolor{currentfill}{rgb}{0.489246,0.627536,0.976896}%
\pgfsetfillcolor{currentfill}%
\pgfsetlinewidth{0.000000pt}%
\definecolor{currentstroke}{rgb}{0.000000,0.000000,0.000000}%
\pgfsetstrokecolor{currentstroke}%
\pgfsetdash{}{0pt}%
\pgfpathmoveto{\pgfqpoint{4.520501in}{3.230359in}}%
\pgfpathlineto{\pgfqpoint{4.529862in}{3.106215in}}%
\pgfpathlineto{\pgfqpoint{4.540496in}{3.338711in}}%
\pgfpathlineto{\pgfqpoint{4.574015in}{3.276166in}}%
\pgfpathlineto{\pgfqpoint{4.607146in}{3.137236in}}%
\pgfpathlineto{\pgfqpoint{4.596939in}{3.056489in}}%
\pgfpathlineto{\pgfqpoint{4.587701in}{3.205729in}}%
\pgfpathlineto{\pgfqpoint{4.554233in}{3.249792in}}%
\pgfpathlineto{\pgfqpoint{4.520501in}{3.230359in}}%
\pgfpathclose%
\pgfusepath{fill}%
\end{pgfscope}%
\begin{pgfscope}%
\pgfpathrectangle{\pgfqpoint{1.020000in}{0.880000in}}{\pgfqpoint{6.160000in}{6.160000in}}%
\pgfusepath{clip}%
\pgfsetbuttcap%
\pgfsetroundjoin%
\definecolor{currentfill}{rgb}{0.441123,0.576532,0.954545}%
\pgfsetfillcolor{currentfill}%
\pgfsetlinewidth{0.000000pt}%
\definecolor{currentstroke}{rgb}{0.000000,0.000000,0.000000}%
\pgfsetstrokecolor{currentstroke}%
\pgfsetdash{}{0pt}%
\pgfpathmoveto{\pgfqpoint{4.674026in}{3.054036in}}%
\pgfpathlineto{\pgfqpoint{4.684603in}{3.178400in}}%
\pgfpathlineto{\pgfqpoint{4.693975in}{3.058802in}}%
\pgfpathlineto{\pgfqpoint{4.728071in}{3.139573in}}%
\pgfpathlineto{\pgfqpoint{4.761878in}{3.164631in}}%
\pgfpathlineto{\pgfqpoint{4.751330in}{3.077363in}}%
\pgfpathlineto{\pgfqpoint{4.741614in}{3.131661in}}%
\pgfpathlineto{\pgfqpoint{4.707655in}{3.062022in}}%
\pgfpathlineto{\pgfqpoint{4.674026in}{3.054036in}}%
\pgfpathclose%
\pgfusepath{fill}%
\end{pgfscope}%
\begin{pgfscope}%
\pgfpathrectangle{\pgfqpoint{1.020000in}{0.880000in}}{\pgfqpoint{6.160000in}{6.160000in}}%
\pgfusepath{clip}%
\pgfsetbuttcap%
\pgfsetroundjoin%
\definecolor{currentfill}{rgb}{0.373552,0.497499,0.909467}%
\pgfsetfillcolor{currentfill}%
\pgfsetlinewidth{0.000000pt}%
\definecolor{currentstroke}{rgb}{0.000000,0.000000,0.000000}%
\pgfsetstrokecolor{currentstroke}%
\pgfsetdash{}{0pt}%
\pgfpathmoveto{\pgfqpoint{5.048829in}{2.891195in}}%
\pgfpathlineto{\pgfqpoint{5.060433in}{3.033066in}}%
\pgfpathlineto{\pgfqpoint{5.071195in}{3.074759in}}%
\pgfpathlineto{\pgfqpoint{5.104328in}{3.025298in}}%
\pgfpathlineto{\pgfqpoint{5.136800in}{2.912384in}}%
\pgfpathlineto{\pgfqpoint{5.126566in}{2.934726in}}%
\pgfpathlineto{\pgfqpoint{5.116526in}{2.975658in}}%
\pgfpathlineto{\pgfqpoint{5.083089in}{2.979461in}}%
\pgfpathlineto{\pgfqpoint{5.048829in}{2.891195in}}%
\pgfpathclose%
\pgfusepath{fill}%
\end{pgfscope}%
\begin{pgfscope}%
\pgfpathrectangle{\pgfqpoint{1.020000in}{0.880000in}}{\pgfqpoint{6.160000in}{6.160000in}}%
\pgfusepath{clip}%
\pgfsetbuttcap%
\pgfsetroundjoin%
\definecolor{currentfill}{rgb}{0.809329,0.852974,0.922323}%
\pgfsetfillcolor{currentfill}%
\pgfsetlinewidth{0.000000pt}%
\definecolor{currentstroke}{rgb}{0.000000,0.000000,0.000000}%
\pgfsetstrokecolor{currentstroke}%
\pgfsetdash{}{0pt}%
\pgfpathmoveto{\pgfqpoint{3.375437in}{3.841115in}}%
\pgfpathlineto{\pgfqpoint{3.384197in}{3.816488in}}%
\pgfpathlineto{\pgfqpoint{3.392644in}{3.833353in}}%
\pgfpathlineto{\pgfqpoint{3.426954in}{3.801480in}}%
\pgfpathlineto{\pgfqpoint{3.461443in}{3.738879in}}%
\pgfpathlineto{\pgfqpoint{3.452733in}{3.748305in}}%
\pgfpathlineto{\pgfqpoint{3.443131in}{3.881089in}}%
\pgfpathlineto{\pgfqpoint{3.409631in}{3.816409in}}%
\pgfpathlineto{\pgfqpoint{3.375437in}{3.841115in}}%
\pgfpathclose%
\pgfusepath{fill}%
\end{pgfscope}%
\begin{pgfscope}%
\pgfpathrectangle{\pgfqpoint{1.020000in}{0.880000in}}{\pgfqpoint{6.160000in}{6.160000in}}%
\pgfusepath{clip}%
\pgfsetbuttcap%
\pgfsetroundjoin%
\definecolor{currentfill}{rgb}{0.353369,0.472069,0.892570}%
\pgfsetfillcolor{currentfill}%
\pgfsetlinewidth{0.000000pt}%
\definecolor{currentstroke}{rgb}{0.000000,0.000000,0.000000}%
\pgfsetstrokecolor{currentstroke}%
\pgfsetdash{}{0pt}%
\pgfpathmoveto{\pgfqpoint{5.270509in}{2.898674in}}%
\pgfpathlineto{\pgfqpoint{5.281045in}{2.893040in}}%
\pgfpathlineto{\pgfqpoint{5.293583in}{3.065086in}}%
\pgfpathlineto{\pgfqpoint{5.325152in}{2.894109in}}%
\pgfpathlineto{\pgfqpoint{5.360028in}{3.014454in}}%
\pgfpathlineto{\pgfqpoint{5.348497in}{2.943446in}}%
\pgfpathlineto{\pgfqpoint{5.338009in}{2.958983in}}%
\pgfpathlineto{\pgfqpoint{5.303864in}{2.893732in}}%
\pgfpathlineto{\pgfqpoint{5.270509in}{2.898674in}}%
\pgfpathclose%
\pgfusepath{fill}%
\end{pgfscope}%
\begin{pgfscope}%
\pgfpathrectangle{\pgfqpoint{1.020000in}{0.880000in}}{\pgfqpoint{6.160000in}{6.160000in}}%
\pgfusepath{clip}%
\pgfsetbuttcap%
\pgfsetroundjoin%
\definecolor{currentfill}{rgb}{0.333490,0.446265,0.874452}%
\pgfsetfillcolor{currentfill}%
\pgfsetlinewidth{0.000000pt}%
\definecolor{currentstroke}{rgb}{0.000000,0.000000,0.000000}%
\pgfsetstrokecolor{currentstroke}%
\pgfsetdash{}{0pt}%
\pgfpathmoveto{\pgfqpoint{6.004240in}{2.979407in}}%
\pgfpathlineto{\pgfqpoint{6.015970in}{2.997113in}}%
\pgfpathlineto{\pgfqpoint{6.024344in}{2.832038in}}%
\pgfpathlineto{\pgfqpoint{6.058078in}{2.857851in}}%
\pgfpathlineto{\pgfqpoint{6.090602in}{2.820794in}}%
\pgfpathlineto{\pgfqpoint{6.080709in}{2.903741in}}%
\pgfpathlineto{\pgfqpoint{6.071432in}{3.018897in}}%
\pgfpathlineto{\pgfqpoint{6.034977in}{2.845480in}}%
\pgfpathlineto{\pgfqpoint{6.004240in}{2.979407in}}%
\pgfpathclose%
\pgfusepath{fill}%
\end{pgfscope}%
\begin{pgfscope}%
\pgfpathrectangle{\pgfqpoint{1.020000in}{0.880000in}}{\pgfqpoint{6.160000in}{6.160000in}}%
\pgfusepath{clip}%
\pgfsetbuttcap%
\pgfsetroundjoin%
\definecolor{currentfill}{rgb}{0.554312,0.690097,0.995516}%
\pgfsetfillcolor{currentfill}%
\pgfsetlinewidth{0.000000pt}%
\definecolor{currentstroke}{rgb}{0.000000,0.000000,0.000000}%
\pgfsetstrokecolor{currentstroke}%
\pgfsetdash{}{0pt}%
\pgfpathmoveto{\pgfqpoint{4.366733in}{3.411526in}}%
\pgfpathlineto{\pgfqpoint{4.376292in}{3.372038in}}%
\pgfpathlineto{\pgfqpoint{4.385532in}{3.168453in}}%
\pgfpathlineto{\pgfqpoint{4.419791in}{3.396111in}}%
\pgfpathlineto{\pgfqpoint{4.453379in}{3.326432in}}%
\pgfpathlineto{\pgfqpoint{4.443564in}{3.287969in}}%
\pgfpathlineto{\pgfqpoint{4.433583in}{3.171769in}}%
\pgfpathlineto{\pgfqpoint{4.400347in}{3.355176in}}%
\pgfpathlineto{\pgfqpoint{4.366733in}{3.411526in}}%
\pgfpathclose%
\pgfusepath{fill}%
\end{pgfscope}%
\begin{pgfscope}%
\pgfpathrectangle{\pgfqpoint{1.020000in}{0.880000in}}{\pgfqpoint{6.160000in}{6.160000in}}%
\pgfusepath{clip}%
\pgfsetbuttcap%
\pgfsetroundjoin%
\definecolor{currentfill}{rgb}{0.835345,0.860514,0.898970}%
\pgfsetfillcolor{currentfill}%
\pgfsetlinewidth{0.000000pt}%
\definecolor{currentstroke}{rgb}{0.000000,0.000000,0.000000}%
\pgfsetstrokecolor{currentstroke}%
\pgfsetdash{}{0pt}%
\pgfpathmoveto{\pgfqpoint{3.152999in}{3.969593in}}%
\pgfpathlineto{\pgfqpoint{3.161960in}{3.908032in}}%
\pgfpathlineto{\pgfqpoint{3.170694in}{3.869743in}}%
\pgfpathlineto{\pgfqpoint{3.205645in}{3.786085in}}%
\pgfpathlineto{\pgfqpoint{3.240303in}{3.725381in}}%
\pgfpathlineto{\pgfqpoint{3.231166in}{3.803004in}}%
\pgfpathlineto{\pgfqpoint{3.222185in}{3.864867in}}%
\pgfpathlineto{\pgfqpoint{3.187342in}{3.945266in}}%
\pgfpathlineto{\pgfqpoint{3.152999in}{3.969593in}}%
\pgfpathclose%
\pgfusepath{fill}%
\end{pgfscope}%
\begin{pgfscope}%
\pgfpathrectangle{\pgfqpoint{1.020000in}{0.880000in}}{\pgfqpoint{6.160000in}{6.160000in}}%
\pgfusepath{clip}%
\pgfsetbuttcap%
\pgfsetroundjoin%
\definecolor{currentfill}{rgb}{0.791392,0.846750,0.936641}%
\pgfsetfillcolor{currentfill}%
\pgfsetlinewidth{0.000000pt}%
\definecolor{currentstroke}{rgb}{0.000000,0.000000,0.000000}%
\pgfsetstrokecolor{currentstroke}%
\pgfsetdash{}{0pt}%
\pgfpathmoveto{\pgfqpoint{3.308412in}{3.721222in}}%
\pgfpathlineto{\pgfqpoint{3.315247in}{3.912366in}}%
\pgfpathlineto{\pgfqpoint{3.325852in}{3.668778in}}%
\pgfpathlineto{\pgfqpoint{3.360173in}{3.636312in}}%
\pgfpathlineto{\pgfqpoint{3.392644in}{3.833353in}}%
\pgfpathlineto{\pgfqpoint{3.384197in}{3.816488in}}%
\pgfpathlineto{\pgfqpoint{3.375437in}{3.841115in}}%
\pgfpathlineto{\pgfqpoint{3.342521in}{3.708842in}}%
\pgfpathlineto{\pgfqpoint{3.308412in}{3.721222in}}%
\pgfpathclose%
\pgfusepath{fill}%
\end{pgfscope}%
\begin{pgfscope}%
\pgfpathrectangle{\pgfqpoint{1.020000in}{0.880000in}}{\pgfqpoint{6.160000in}{6.160000in}}%
\pgfusepath{clip}%
\pgfsetbuttcap%
\pgfsetroundjoin%
\definecolor{currentfill}{rgb}{0.608547,0.735725,0.999354}%
\pgfsetfillcolor{currentfill}%
\pgfsetlinewidth{0.000000pt}%
\definecolor{currentstroke}{rgb}{0.000000,0.000000,0.000000}%
\pgfsetstrokecolor{currentstroke}%
\pgfsetdash{}{0pt}%
\pgfpathmoveto{\pgfqpoint{4.212656in}{3.442466in}}%
\pgfpathlineto{\pgfqpoint{4.222127in}{3.460563in}}%
\pgfpathlineto{\pgfqpoint{4.231568in}{3.370614in}}%
\pgfpathlineto{\pgfqpoint{4.265540in}{3.576396in}}%
\pgfpathlineto{\pgfqpoint{4.299146in}{3.363381in}}%
\pgfpathlineto{\pgfqpoint{4.289666in}{3.422867in}}%
\pgfpathlineto{\pgfqpoint{4.279966in}{3.245493in}}%
\pgfpathlineto{\pgfqpoint{4.246406in}{3.425927in}}%
\pgfpathlineto{\pgfqpoint{4.212656in}{3.442466in}}%
\pgfpathclose%
\pgfusepath{fill}%
\end{pgfscope}%
\begin{pgfscope}%
\pgfpathrectangle{\pgfqpoint{1.020000in}{0.880000in}}{\pgfqpoint{6.160000in}{6.160000in}}%
\pgfusepath{clip}%
\pgfsetbuttcap%
\pgfsetroundjoin%
\definecolor{currentfill}{rgb}{0.867428,0.864377,0.862602}%
\pgfsetfillcolor{currentfill}%
\pgfsetlinewidth{0.000000pt}%
\definecolor{currentstroke}{rgb}{0.000000,0.000000,0.000000}%
\pgfsetstrokecolor{currentstroke}%
\pgfsetdash{}{0pt}%
\pgfpathmoveto{\pgfqpoint{2.710366in}{3.989321in}}%
\pgfpathlineto{\pgfqpoint{2.721281in}{3.779452in}}%
\pgfpathlineto{\pgfqpoint{2.728557in}{3.819565in}}%
\pgfpathlineto{\pgfqpoint{2.761866in}{3.884496in}}%
\pgfpathlineto{\pgfqpoint{2.796254in}{3.873043in}}%
\pgfpathlineto{\pgfqpoint{2.786037in}{4.037489in}}%
\pgfpathlineto{\pgfqpoint{2.778473in}{4.011336in}}%
\pgfpathlineto{\pgfqpoint{2.742973in}{4.101876in}}%
\pgfpathlineto{\pgfqpoint{2.710366in}{3.989321in}}%
\pgfpathclose%
\pgfusepath{fill}%
\end{pgfscope}%
\begin{pgfscope}%
\pgfpathrectangle{\pgfqpoint{1.020000in}{0.880000in}}{\pgfqpoint{6.160000in}{6.160000in}}%
\pgfusepath{clip}%
\pgfsetbuttcap%
\pgfsetroundjoin%
\definecolor{currentfill}{rgb}{0.404421,0.534643,0.932002}%
\pgfsetfillcolor{currentfill}%
\pgfsetlinewidth{0.000000pt}%
\definecolor{currentstroke}{rgb}{0.000000,0.000000,0.000000}%
\pgfsetstrokecolor{currentstroke}%
\pgfsetdash{}{0pt}%
\pgfpathmoveto{\pgfqpoint{4.828834in}{3.114706in}}%
\pgfpathlineto{\pgfqpoint{4.839310in}{3.166728in}}%
\pgfpathlineto{\pgfqpoint{4.848519in}{3.024105in}}%
\pgfpathlineto{\pgfqpoint{4.881290in}{2.901978in}}%
\pgfpathlineto{\pgfqpoint{4.915146in}{2.941269in}}%
\pgfpathlineto{\pgfqpoint{4.905043in}{2.955231in}}%
\pgfpathlineto{\pgfqpoint{4.895931in}{3.105252in}}%
\pgfpathlineto{\pgfqpoint{4.862423in}{3.114187in}}%
\pgfpathlineto{\pgfqpoint{4.828834in}{3.114706in}}%
\pgfpathclose%
\pgfusepath{fill}%
\end{pgfscope}%
\begin{pgfscope}%
\pgfpathrectangle{\pgfqpoint{1.020000in}{0.880000in}}{\pgfqpoint{6.160000in}{6.160000in}}%
\pgfusepath{clip}%
\pgfsetbuttcap%
\pgfsetroundjoin%
\definecolor{currentfill}{rgb}{0.667253,0.779176,0.992959}%
\pgfsetfillcolor{currentfill}%
\pgfsetlinewidth{0.000000pt}%
\definecolor{currentstroke}{rgb}{0.000000,0.000000,0.000000}%
\pgfsetstrokecolor{currentstroke}%
\pgfsetdash{}{0pt}%
\pgfpathmoveto{\pgfqpoint{4.058806in}{3.449778in}}%
\pgfpathlineto{\pgfqpoint{4.068027in}{3.509211in}}%
\pgfpathlineto{\pgfqpoint{4.077225in}{3.620567in}}%
\pgfpathlineto{\pgfqpoint{4.111088in}{3.673568in}}%
\pgfpathlineto{\pgfqpoint{4.145056in}{3.493548in}}%
\pgfpathlineto{\pgfqpoint{4.135689in}{3.464078in}}%
\pgfpathlineto{\pgfqpoint{4.126334in}{3.460088in}}%
\pgfpathlineto{\pgfqpoint{4.092535in}{3.502646in}}%
\pgfpathlineto{\pgfqpoint{4.058806in}{3.449778in}}%
\pgfpathclose%
\pgfusepath{fill}%
\end{pgfscope}%
\begin{pgfscope}%
\pgfpathrectangle{\pgfqpoint{1.020000in}{0.880000in}}{\pgfqpoint{6.160000in}{6.160000in}}%
\pgfusepath{clip}%
\pgfsetbuttcap%
\pgfsetroundjoin%
\definecolor{currentfill}{rgb}{0.822420,0.856898,0.910795}%
\pgfsetfillcolor{currentfill}%
\pgfsetlinewidth{0.000000pt}%
\definecolor{currentstroke}{rgb}{0.000000,0.000000,0.000000}%
\pgfsetstrokecolor{currentstroke}%
\pgfsetdash{}{0pt}%
\pgfpathmoveto{\pgfqpoint{2.575143in}{3.873809in}}%
\pgfpathlineto{\pgfqpoint{2.583223in}{3.850675in}}%
\pgfpathlineto{\pgfqpoint{2.591635in}{3.807334in}}%
\pgfpathlineto{\pgfqpoint{2.625196in}{3.855377in}}%
\pgfpathlineto{\pgfqpoint{2.660477in}{3.790806in}}%
\pgfpathlineto{\pgfqpoint{2.653713in}{3.723507in}}%
\pgfpathlineto{\pgfqpoint{2.644450in}{3.821060in}}%
\pgfpathlineto{\pgfqpoint{2.608467in}{3.933526in}}%
\pgfpathlineto{\pgfqpoint{2.575143in}{3.873809in}}%
\pgfpathclose%
\pgfusepath{fill}%
\end{pgfscope}%
\begin{pgfscope}%
\pgfpathrectangle{\pgfqpoint{1.020000in}{0.880000in}}{\pgfqpoint{6.160000in}{6.160000in}}%
\pgfusepath{clip}%
\pgfsetbuttcap%
\pgfsetroundjoin%
\definecolor{currentfill}{rgb}{0.768034,0.837035,0.952488}%
\pgfsetfillcolor{currentfill}%
\pgfsetlinewidth{0.000000pt}%
\definecolor{currentstroke}{rgb}{0.000000,0.000000,0.000000}%
\pgfsetstrokecolor{currentstroke}%
\pgfsetdash{}{0pt}%
\pgfpathmoveto{\pgfqpoint{3.750707in}{3.744297in}}%
\pgfpathlineto{\pgfqpoint{3.759816in}{3.711138in}}%
\pgfpathlineto{\pgfqpoint{3.769097in}{3.639564in}}%
\pgfpathlineto{\pgfqpoint{3.803105in}{3.620587in}}%
\pgfpathlineto{\pgfqpoint{3.836401in}{3.797848in}}%
\pgfpathlineto{\pgfqpoint{3.827596in}{3.725814in}}%
\pgfpathlineto{\pgfqpoint{3.818660in}{3.698910in}}%
\pgfpathlineto{\pgfqpoint{3.784377in}{3.803035in}}%
\pgfpathlineto{\pgfqpoint{3.750707in}{3.744297in}}%
\pgfpathclose%
\pgfusepath{fill}%
\end{pgfscope}%
\begin{pgfscope}%
\pgfpathrectangle{\pgfqpoint{1.020000in}{0.880000in}}{\pgfqpoint{6.160000in}{6.160000in}}%
\pgfusepath{clip}%
\pgfsetbuttcap%
\pgfsetroundjoin%
\definecolor{currentfill}{rgb}{0.796064,0.848693,0.933471}%
\pgfsetfillcolor{currentfill}%
\pgfsetlinewidth{0.000000pt}%
\definecolor{currentstroke}{rgb}{0.000000,0.000000,0.000000}%
\pgfsetstrokecolor{currentstroke}%
\pgfsetdash{}{0pt}%
\pgfpathmoveto{\pgfqpoint{2.510078in}{3.663253in}}%
\pgfpathlineto{\pgfqpoint{2.515720in}{3.782609in}}%
\pgfpathlineto{\pgfqpoint{2.521428in}{3.900129in}}%
\pgfpathlineto{\pgfqpoint{2.560402in}{3.616430in}}%
\pgfpathlineto{\pgfqpoint{2.591635in}{3.807334in}}%
\pgfpathlineto{\pgfqpoint{2.583223in}{3.850675in}}%
\pgfpathlineto{\pgfqpoint{2.575143in}{3.873809in}}%
\pgfpathlineto{\pgfqpoint{2.543704in}{3.700077in}}%
\pgfpathlineto{\pgfqpoint{2.510078in}{3.663253in}}%
\pgfpathclose%
\pgfusepath{fill}%
\end{pgfscope}%
\begin{pgfscope}%
\pgfpathrectangle{\pgfqpoint{1.020000in}{0.880000in}}{\pgfqpoint{6.160000in}{6.160000in}}%
\pgfusepath{clip}%
\pgfsetbuttcap%
\pgfsetroundjoin%
\definecolor{currentfill}{rgb}{0.693321,0.796314,0.986308}%
\pgfsetfillcolor{currentfill}%
\pgfsetlinewidth{0.000000pt}%
\definecolor{currentstroke}{rgb}{0.000000,0.000000,0.000000}%
\pgfsetstrokecolor{currentstroke}%
\pgfsetdash{}{0pt}%
\pgfpathmoveto{\pgfqpoint{3.836401in}{3.797848in}}%
\pgfpathlineto{\pgfqpoint{3.846177in}{3.591119in}}%
\pgfpathlineto{\pgfqpoint{3.855477in}{3.524266in}}%
\pgfpathlineto{\pgfqpoint{3.889553in}{3.472349in}}%
\pgfpathlineto{\pgfqpoint{3.923200in}{3.564112in}}%
\pgfpathlineto{\pgfqpoint{3.914100in}{3.547153in}}%
\pgfpathlineto{\pgfqpoint{3.905216in}{3.460452in}}%
\pgfpathlineto{\pgfqpoint{3.871039in}{3.576752in}}%
\pgfpathlineto{\pgfqpoint{3.836401in}{3.797848in}}%
\pgfpathclose%
\pgfusepath{fill}%
\end{pgfscope}%
\begin{pgfscope}%
\pgfpathrectangle{\pgfqpoint{1.020000in}{0.880000in}}{\pgfqpoint{6.160000in}{6.160000in}}%
\pgfusepath{clip}%
\pgfsetbuttcap%
\pgfsetroundjoin%
\definecolor{currentfill}{rgb}{0.543440,0.680003,0.993051}%
\pgfsetfillcolor{currentfill}%
\pgfsetlinewidth{0.000000pt}%
\definecolor{currentstroke}{rgb}{0.000000,0.000000,0.000000}%
\pgfsetstrokecolor{currentstroke}%
\pgfsetdash{}{0pt}%
\pgfpathmoveto{\pgfqpoint{4.299146in}{3.363381in}}%
\pgfpathlineto{\pgfqpoint{4.308496in}{3.193730in}}%
\pgfpathlineto{\pgfqpoint{4.318104in}{3.222982in}}%
\pgfpathlineto{\pgfqpoint{4.352191in}{3.404817in}}%
\pgfpathlineto{\pgfqpoint{4.385532in}{3.168453in}}%
\pgfpathlineto{\pgfqpoint{4.376292in}{3.372038in}}%
\pgfpathlineto{\pgfqpoint{4.366733in}{3.411526in}}%
\pgfpathlineto{\pgfqpoint{4.332630in}{3.179764in}}%
\pgfpathlineto{\pgfqpoint{4.299146in}{3.363381in}}%
\pgfpathclose%
\pgfusepath{fill}%
\end{pgfscope}%
\begin{pgfscope}%
\pgfpathrectangle{\pgfqpoint{1.020000in}{0.880000in}}{\pgfqpoint{6.160000in}{6.160000in}}%
\pgfusepath{clip}%
\pgfsetbuttcap%
\pgfsetroundjoin%
\definecolor{currentfill}{rgb}{0.289996,0.386836,0.828926}%
\pgfsetfillcolor{currentfill}%
\pgfsetlinewidth{0.000000pt}%
\definecolor{currentstroke}{rgb}{0.000000,0.000000,0.000000}%
\pgfsetstrokecolor{currentstroke}%
\pgfsetdash{}{0pt}%
\pgfpathmoveto{\pgfqpoint{5.581004in}{2.912537in}}%
\pgfpathlineto{\pgfqpoint{5.590334in}{2.797762in}}%
\pgfpathlineto{\pgfqpoint{5.602283in}{2.866535in}}%
\pgfpathlineto{\pgfqpoint{5.633946in}{2.745177in}}%
\pgfpathlineto{\pgfqpoint{5.667829in}{2.779232in}}%
\pgfpathlineto{\pgfqpoint{5.656136in}{2.734603in}}%
\pgfpathlineto{\pgfqpoint{5.647392in}{2.887955in}}%
\pgfpathlineto{\pgfqpoint{5.611818in}{2.733833in}}%
\pgfpathlineto{\pgfqpoint{5.581004in}{2.912537in}}%
\pgfpathclose%
\pgfusepath{fill}%
\end{pgfscope}%
\begin{pgfscope}%
\pgfpathrectangle{\pgfqpoint{1.020000in}{0.880000in}}{\pgfqpoint{6.160000in}{6.160000in}}%
\pgfusepath{clip}%
\pgfsetbuttcap%
\pgfsetroundjoin%
\definecolor{currentfill}{rgb}{0.867428,0.864377,0.862602}%
\pgfsetfillcolor{currentfill}%
\pgfsetlinewidth{0.000000pt}%
\definecolor{currentstroke}{rgb}{0.000000,0.000000,0.000000}%
\pgfsetstrokecolor{currentstroke}%
\pgfsetdash{}{0pt}%
\pgfpathmoveto{\pgfqpoint{2.932812in}{3.884300in}}%
\pgfpathlineto{\pgfqpoint{2.938841in}{4.050811in}}%
\pgfpathlineto{\pgfqpoint{2.948375in}{3.935631in}}%
\pgfpathlineto{\pgfqpoint{2.982723in}{3.924291in}}%
\pgfpathlineto{\pgfqpoint{3.017878in}{3.841059in}}%
\pgfpathlineto{\pgfqpoint{3.010118in}{3.805963in}}%
\pgfpathlineto{\pgfqpoint{2.999582in}{4.008701in}}%
\pgfpathlineto{\pgfqpoint{2.965192in}{4.028556in}}%
\pgfpathlineto{\pgfqpoint{2.932812in}{3.884300in}}%
\pgfpathclose%
\pgfusepath{fill}%
\end{pgfscope}%
\begin{pgfscope}%
\pgfpathrectangle{\pgfqpoint{1.020000in}{0.880000in}}{\pgfqpoint{6.160000in}{6.160000in}}%
\pgfusepath{clip}%
\pgfsetbuttcap%
\pgfsetroundjoin%
\definecolor{currentfill}{rgb}{0.333490,0.446265,0.874452}%
\pgfsetfillcolor{currentfill}%
\pgfsetlinewidth{0.000000pt}%
\definecolor{currentstroke}{rgb}{0.000000,0.000000,0.000000}%
\pgfsetstrokecolor{currentstroke}%
\pgfsetdash{}{0pt}%
\pgfpathmoveto{\pgfqpoint{5.936031in}{2.883759in}}%
\pgfpathlineto{\pgfqpoint{5.947242in}{2.877158in}}%
\pgfpathlineto{\pgfqpoint{5.957813in}{2.833395in}}%
\pgfpathlineto{\pgfqpoint{5.991880in}{2.876226in}}%
\pgfpathlineto{\pgfqpoint{6.024344in}{2.832038in}}%
\pgfpathlineto{\pgfqpoint{6.015970in}{2.997113in}}%
\pgfpathlineto{\pgfqpoint{6.004240in}{2.979407in}}%
\pgfpathlineto{\pgfqpoint{5.968919in}{2.864150in}}%
\pgfpathlineto{\pgfqpoint{5.936031in}{2.883759in}}%
\pgfpathclose%
\pgfusepath{fill}%
\end{pgfscope}%
\begin{pgfscope}%
\pgfpathrectangle{\pgfqpoint{1.020000in}{0.880000in}}{\pgfqpoint{6.160000in}{6.160000in}}%
\pgfusepath{clip}%
\pgfsetbuttcap%
\pgfsetroundjoin%
\definecolor{currentfill}{rgb}{0.323718,0.433158,0.864722}%
\pgfsetfillcolor{currentfill}%
\pgfsetlinewidth{0.000000pt}%
\definecolor{currentstroke}{rgb}{0.000000,0.000000,0.000000}%
\pgfsetstrokecolor{currentstroke}%
\pgfsetdash{}{0pt}%
\pgfpathmoveto{\pgfqpoint{5.647392in}{2.887955in}}%
\pgfpathlineto{\pgfqpoint{5.656136in}{2.734603in}}%
\pgfpathlineto{\pgfqpoint{5.667829in}{2.779232in}}%
\pgfpathlineto{\pgfqpoint{5.704561in}{2.999583in}}%
\pgfpathlineto{\pgfqpoint{5.735416in}{2.835369in}}%
\pgfpathlineto{\pgfqpoint{5.727214in}{3.022454in}}%
\pgfpathlineto{\pgfqpoint{5.712394in}{2.780705in}}%
\pgfpathlineto{\pgfqpoint{5.681701in}{2.952577in}}%
\pgfpathlineto{\pgfqpoint{5.647392in}{2.887955in}}%
\pgfpathclose%
\pgfusepath{fill}%
\end{pgfscope}%
\begin{pgfscope}%
\pgfpathrectangle{\pgfqpoint{1.020000in}{0.880000in}}{\pgfqpoint{6.160000in}{6.160000in}}%
\pgfusepath{clip}%
\pgfsetbuttcap%
\pgfsetroundjoin%
\definecolor{currentfill}{rgb}{0.822420,0.856898,0.910795}%
\pgfsetfillcolor{currentfill}%
\pgfsetlinewidth{0.000000pt}%
\definecolor{currentstroke}{rgb}{0.000000,0.000000,0.000000}%
\pgfsetstrokecolor{currentstroke}%
\pgfsetdash{}{0pt}%
\pgfpathmoveto{\pgfqpoint{2.796254in}{3.873043in}}%
\pgfpathlineto{\pgfqpoint{2.803369in}{3.933858in}}%
\pgfpathlineto{\pgfqpoint{2.813358in}{3.786122in}}%
\pgfpathlineto{\pgfqpoint{2.849120in}{3.672100in}}%
\pgfpathlineto{\pgfqpoint{2.881985in}{3.773876in}}%
\pgfpathlineto{\pgfqpoint{2.871647in}{3.950909in}}%
\pgfpathlineto{\pgfqpoint{2.864256in}{3.902550in}}%
\pgfpathlineto{\pgfqpoint{2.832184in}{3.744961in}}%
\pgfpathlineto{\pgfqpoint{2.796254in}{3.873043in}}%
\pgfpathclose%
\pgfusepath{fill}%
\end{pgfscope}%
\begin{pgfscope}%
\pgfpathrectangle{\pgfqpoint{1.020000in}{0.880000in}}{\pgfqpoint{6.160000in}{6.160000in}}%
\pgfusepath{clip}%
\pgfsetbuttcap%
\pgfsetroundjoin%
\definecolor{currentfill}{rgb}{0.758539,0.832787,0.958408}%
\pgfsetfillcolor{currentfill}%
\pgfsetlinewidth{0.000000pt}%
\definecolor{currentstroke}{rgb}{0.000000,0.000000,0.000000}%
\pgfsetstrokecolor{currentstroke}%
\pgfsetdash{}{0pt}%
\pgfpathmoveto{\pgfqpoint{3.683693in}{3.572934in}}%
\pgfpathlineto{\pgfqpoint{3.692313in}{3.627248in}}%
\pgfpathlineto{\pgfqpoint{3.700908in}{3.692889in}}%
\pgfpathlineto{\pgfqpoint{3.734215in}{3.846544in}}%
\pgfpathlineto{\pgfqpoint{3.769097in}{3.639564in}}%
\pgfpathlineto{\pgfqpoint{3.759816in}{3.711138in}}%
\pgfpathlineto{\pgfqpoint{3.750707in}{3.744297in}}%
\pgfpathlineto{\pgfqpoint{3.716935in}{3.711568in}}%
\pgfpathlineto{\pgfqpoint{3.683693in}{3.572934in}}%
\pgfpathclose%
\pgfusepath{fill}%
\end{pgfscope}%
\begin{pgfscope}%
\pgfpathrectangle{\pgfqpoint{1.020000in}{0.880000in}}{\pgfqpoint{6.160000in}{6.160000in}}%
\pgfusepath{clip}%
\pgfsetbuttcap%
\pgfsetroundjoin%
\definecolor{currentfill}{rgb}{0.782049,0.842864,0.942980}%
\pgfsetfillcolor{currentfill}%
\pgfsetlinewidth{0.000000pt}%
\definecolor{currentstroke}{rgb}{0.000000,0.000000,0.000000}%
\pgfsetstrokecolor{currentstroke}%
\pgfsetdash{}{0pt}%
\pgfpathmoveto{\pgfqpoint{3.461443in}{3.738879in}}%
\pgfpathlineto{\pgfqpoint{3.469122in}{3.876890in}}%
\pgfpathlineto{\pgfqpoint{3.479205in}{3.678696in}}%
\pgfpathlineto{\pgfqpoint{3.512280in}{3.823674in}}%
\pgfpathlineto{\pgfqpoint{3.547205in}{3.689445in}}%
\pgfpathlineto{\pgfqpoint{3.538742in}{3.645986in}}%
\pgfpathlineto{\pgfqpoint{3.529858in}{3.671285in}}%
\pgfpathlineto{\pgfqpoint{3.495396in}{3.745368in}}%
\pgfpathlineto{\pgfqpoint{3.461443in}{3.738879in}}%
\pgfpathclose%
\pgfusepath{fill}%
\end{pgfscope}%
\begin{pgfscope}%
\pgfpathrectangle{\pgfqpoint{1.020000in}{0.880000in}}{\pgfqpoint{6.160000in}{6.160000in}}%
\pgfusepath{clip}%
\pgfsetbuttcap%
\pgfsetroundjoin%
\definecolor{currentfill}{rgb}{0.313946,0.420052,0.854993}%
\pgfsetfillcolor{currentfill}%
\pgfsetlinewidth{0.000000pt}%
\definecolor{currentstroke}{rgb}{0.000000,0.000000,0.000000}%
\pgfsetstrokecolor{currentstroke}%
\pgfsetdash{}{0pt}%
\pgfpathmoveto{\pgfqpoint{5.868010in}{2.796365in}}%
\pgfpathlineto{\pgfqpoint{5.882407in}{2.980561in}}%
\pgfpathlineto{\pgfqpoint{5.890782in}{2.810435in}}%
\pgfpathlineto{\pgfqpoint{5.923391in}{2.769809in}}%
\pgfpathlineto{\pgfqpoint{5.957813in}{2.833395in}}%
\pgfpathlineto{\pgfqpoint{5.947242in}{2.877158in}}%
\pgfpathlineto{\pgfqpoint{5.936031in}{2.883759in}}%
\pgfpathlineto{\pgfqpoint{5.903204in}{2.908803in}}%
\pgfpathlineto{\pgfqpoint{5.868010in}{2.796365in}}%
\pgfpathclose%
\pgfusepath{fill}%
\end{pgfscope}%
\begin{pgfscope}%
\pgfpathrectangle{\pgfqpoint{1.020000in}{0.880000in}}{\pgfqpoint{6.160000in}{6.160000in}}%
\pgfusepath{clip}%
\pgfsetbuttcap%
\pgfsetroundjoin%
\definecolor{currentfill}{rgb}{0.800601,0.850358,0.930008}%
\pgfsetfillcolor{currentfill}%
\pgfsetlinewidth{0.000000pt}%
\definecolor{currentstroke}{rgb}{0.000000,0.000000,0.000000}%
\pgfsetstrokecolor{currentstroke}%
\pgfsetdash{}{0pt}%
\pgfpathmoveto{\pgfqpoint{3.240303in}{3.725381in}}%
\pgfpathlineto{\pgfqpoint{3.248074in}{3.794916in}}%
\pgfpathlineto{\pgfqpoint{3.257007in}{3.741806in}}%
\pgfpathlineto{\pgfqpoint{3.290651in}{3.795345in}}%
\pgfpathlineto{\pgfqpoint{3.325852in}{3.668778in}}%
\pgfpathlineto{\pgfqpoint{3.315247in}{3.912366in}}%
\pgfpathlineto{\pgfqpoint{3.308412in}{3.721222in}}%
\pgfpathlineto{\pgfqpoint{3.273303in}{3.840888in}}%
\pgfpathlineto{\pgfqpoint{3.240303in}{3.725381in}}%
\pgfpathclose%
\pgfusepath{fill}%
\end{pgfscope}%
\begin{pgfscope}%
\pgfpathrectangle{\pgfqpoint{1.020000in}{0.880000in}}{\pgfqpoint{6.160000in}{6.160000in}}%
\pgfusepath{clip}%
\pgfsetbuttcap%
\pgfsetroundjoin%
\definecolor{currentfill}{rgb}{0.859385,0.864431,0.872111}%
\pgfsetfillcolor{currentfill}%
\pgfsetlinewidth{0.000000pt}%
\definecolor{currentstroke}{rgb}{0.000000,0.000000,0.000000}%
\pgfsetstrokecolor{currentstroke}%
\pgfsetdash{}{0pt}%
\pgfpathmoveto{\pgfqpoint{2.864256in}{3.902550in}}%
\pgfpathlineto{\pgfqpoint{2.871647in}{3.950909in}}%
\pgfpathlineto{\pgfqpoint{2.881985in}{3.773876in}}%
\pgfpathlineto{\pgfqpoint{2.916038in}{3.785437in}}%
\pgfpathlineto{\pgfqpoint{2.948375in}{3.935631in}}%
\pgfpathlineto{\pgfqpoint{2.938841in}{4.050811in}}%
\pgfpathlineto{\pgfqpoint{2.932812in}{3.884300in}}%
\pgfpathlineto{\pgfqpoint{2.896941in}{4.019040in}}%
\pgfpathlineto{\pgfqpoint{2.864256in}{3.902550in}}%
\pgfpathclose%
\pgfusepath{fill}%
\end{pgfscope}%
\begin{pgfscope}%
\pgfpathrectangle{\pgfqpoint{1.020000in}{0.880000in}}{\pgfqpoint{6.160000in}{6.160000in}}%
\pgfusepath{clip}%
\pgfsetbuttcap%
\pgfsetroundjoin%
\definecolor{currentfill}{rgb}{0.738826,0.822572,0.968261}%
\pgfsetfillcolor{currentfill}%
\pgfsetlinewidth{0.000000pt}%
\definecolor{currentstroke}{rgb}{0.000000,0.000000,0.000000}%
\pgfsetstrokecolor{currentstroke}%
\pgfsetdash{}{0pt}%
\pgfpathmoveto{\pgfqpoint{3.615791in}{3.585042in}}%
\pgfpathlineto{\pgfqpoint{3.623672in}{3.753354in}}%
\pgfpathlineto{\pgfqpoint{3.632774in}{3.707967in}}%
\pgfpathlineto{\pgfqpoint{3.667764in}{3.523634in}}%
\pgfpathlineto{\pgfqpoint{3.700908in}{3.692889in}}%
\pgfpathlineto{\pgfqpoint{3.692313in}{3.627248in}}%
\pgfpathlineto{\pgfqpoint{3.683693in}{3.572934in}}%
\pgfpathlineto{\pgfqpoint{3.648823in}{3.753833in}}%
\pgfpathlineto{\pgfqpoint{3.615791in}{3.585042in}}%
\pgfpathclose%
\pgfusepath{fill}%
\end{pgfscope}%
\begin{pgfscope}%
\pgfpathrectangle{\pgfqpoint{1.020000in}{0.880000in}}{\pgfqpoint{6.160000in}{6.160000in}}%
\pgfusepath{clip}%
\pgfsetbuttcap%
\pgfsetroundjoin%
\definecolor{currentfill}{rgb}{0.467678,0.605591,0.968546}%
\pgfsetfillcolor{currentfill}%
\pgfsetlinewidth{0.000000pt}%
\definecolor{currentstroke}{rgb}{0.000000,0.000000,0.000000}%
\pgfsetstrokecolor{currentstroke}%
\pgfsetdash{}{0pt}%
\pgfpathmoveto{\pgfqpoint{4.607146in}{3.137236in}}%
\pgfpathlineto{\pgfqpoint{4.616746in}{3.069916in}}%
\pgfpathlineto{\pgfqpoint{4.627578in}{3.277559in}}%
\pgfpathlineto{\pgfqpoint{4.661327in}{3.273350in}}%
\pgfpathlineto{\pgfqpoint{4.693975in}{3.058802in}}%
\pgfpathlineto{\pgfqpoint{4.684603in}{3.178400in}}%
\pgfpathlineto{\pgfqpoint{4.674026in}{3.054036in}}%
\pgfpathlineto{\pgfqpoint{4.640620in}{3.097758in}}%
\pgfpathlineto{\pgfqpoint{4.607146in}{3.137236in}}%
\pgfpathclose%
\pgfusepath{fill}%
\end{pgfscope}%
\begin{pgfscope}%
\pgfpathrectangle{\pgfqpoint{1.020000in}{0.880000in}}{\pgfqpoint{6.160000in}{6.160000in}}%
\pgfusepath{clip}%
\pgfsetbuttcap%
\pgfsetroundjoin%
\definecolor{currentfill}{rgb}{0.672538,0.782861,0.991982}%
\pgfsetfillcolor{currentfill}%
\pgfsetlinewidth{0.000000pt}%
\definecolor{currentstroke}{rgb}{0.000000,0.000000,0.000000}%
\pgfsetstrokecolor{currentstroke}%
\pgfsetdash{}{0pt}%
\pgfpathmoveto{\pgfqpoint{3.991064in}{3.516414in}}%
\pgfpathlineto{\pgfqpoint{4.000241in}{3.544800in}}%
\pgfpathlineto{\pgfqpoint{4.009612in}{3.475653in}}%
\pgfpathlineto{\pgfqpoint{4.043552in}{3.442058in}}%
\pgfpathlineto{\pgfqpoint{4.077225in}{3.620567in}}%
\pgfpathlineto{\pgfqpoint{4.068027in}{3.509211in}}%
\pgfpathlineto{\pgfqpoint{4.058806in}{3.449778in}}%
\pgfpathlineto{\pgfqpoint{4.024712in}{3.637583in}}%
\pgfpathlineto{\pgfqpoint{3.991064in}{3.516414in}}%
\pgfpathclose%
\pgfusepath{fill}%
\end{pgfscope}%
\begin{pgfscope}%
\pgfpathrectangle{\pgfqpoint{1.020000in}{0.880000in}}{\pgfqpoint{6.160000in}{6.160000in}}%
\pgfusepath{clip}%
\pgfsetbuttcap%
\pgfsetroundjoin%
\definecolor{currentfill}{rgb}{0.782049,0.842864,0.942980}%
\pgfsetfillcolor{currentfill}%
\pgfsetlinewidth{0.000000pt}%
\definecolor{currentstroke}{rgb}{0.000000,0.000000,0.000000}%
\pgfsetstrokecolor{currentstroke}%
\pgfsetdash{}{0pt}%
\pgfpathmoveto{\pgfqpoint{2.660477in}{3.790806in}}%
\pgfpathlineto{\pgfqpoint{2.669283in}{3.724600in}}%
\pgfpathlineto{\pgfqpoint{2.677213in}{3.717093in}}%
\pgfpathlineto{\pgfqpoint{2.712765in}{3.631991in}}%
\pgfpathlineto{\pgfqpoint{2.745599in}{3.730965in}}%
\pgfpathlineto{\pgfqpoint{2.737913in}{3.716976in}}%
\pgfpathlineto{\pgfqpoint{2.728557in}{3.819565in}}%
\pgfpathlineto{\pgfqpoint{2.695132in}{3.764070in}}%
\pgfpathlineto{\pgfqpoint{2.660477in}{3.790806in}}%
\pgfpathclose%
\pgfusepath{fill}%
\end{pgfscope}%
\begin{pgfscope}%
\pgfpathrectangle{\pgfqpoint{1.020000in}{0.880000in}}{\pgfqpoint{6.160000in}{6.160000in}}%
\pgfusepath{clip}%
\pgfsetbuttcap%
\pgfsetroundjoin%
\definecolor{currentfill}{rgb}{0.399231,0.528528,0.928459}%
\pgfsetfillcolor{currentfill}%
\pgfsetlinewidth{0.000000pt}%
\definecolor{currentstroke}{rgb}{0.000000,0.000000,0.000000}%
\pgfsetstrokecolor{currentstroke}%
\pgfsetdash{}{0pt}%
\pgfpathmoveto{\pgfqpoint{4.982759in}{3.005213in}}%
\pgfpathlineto{\pgfqpoint{4.993192in}{3.021534in}}%
\pgfpathlineto{\pgfqpoint{5.003927in}{3.071520in}}%
\pgfpathlineto{\pgfqpoint{5.037368in}{3.049533in}}%
\pgfpathlineto{\pgfqpoint{5.071195in}{3.074759in}}%
\pgfpathlineto{\pgfqpoint{5.060433in}{3.033066in}}%
\pgfpathlineto{\pgfqpoint{5.048829in}{2.891195in}}%
\pgfpathlineto{\pgfqpoint{5.016291in}{3.004117in}}%
\pgfpathlineto{\pgfqpoint{4.982759in}{3.005213in}}%
\pgfpathclose%
\pgfusepath{fill}%
\end{pgfscope}%
\begin{pgfscope}%
\pgfpathrectangle{\pgfqpoint{1.020000in}{0.880000in}}{\pgfqpoint{6.160000in}{6.160000in}}%
\pgfusepath{clip}%
\pgfsetbuttcap%
\pgfsetroundjoin%
\definecolor{currentfill}{rgb}{0.383662,0.510183,0.917831}%
\pgfsetfillcolor{currentfill}%
\pgfsetlinewidth{0.000000pt}%
\definecolor{currentstroke}{rgb}{0.000000,0.000000,0.000000}%
\pgfsetstrokecolor{currentstroke}%
\pgfsetdash{}{0pt}%
\pgfpathmoveto{\pgfqpoint{5.204760in}{3.008910in}}%
\pgfpathlineto{\pgfqpoint{5.214650in}{2.945795in}}%
\pgfpathlineto{\pgfqpoint{5.226033in}{3.024447in}}%
\pgfpathlineto{\pgfqpoint{5.259698in}{3.034294in}}%
\pgfpathlineto{\pgfqpoint{5.293583in}{3.065086in}}%
\pgfpathlineto{\pgfqpoint{5.281045in}{2.893040in}}%
\pgfpathlineto{\pgfqpoint{5.270509in}{2.898674in}}%
\pgfpathlineto{\pgfqpoint{5.238268in}{3.010824in}}%
\pgfpathlineto{\pgfqpoint{5.204760in}{3.008910in}}%
\pgfpathclose%
\pgfusepath{fill}%
\end{pgfscope}%
\begin{pgfscope}%
\pgfpathrectangle{\pgfqpoint{1.020000in}{0.880000in}}{\pgfqpoint{6.160000in}{6.160000in}}%
\pgfusepath{clip}%
\pgfsetbuttcap%
\pgfsetroundjoin%
\definecolor{currentfill}{rgb}{0.630089,0.752516,0.998508}%
\pgfsetfillcolor{currentfill}%
\pgfsetlinewidth{0.000000pt}%
\definecolor{currentstroke}{rgb}{0.000000,0.000000,0.000000}%
\pgfsetstrokecolor{currentstroke}%
\pgfsetdash{}{0pt}%
\pgfpathmoveto{\pgfqpoint{4.145056in}{3.493548in}}%
\pgfpathlineto{\pgfqpoint{4.154477in}{3.422649in}}%
\pgfpathlineto{\pgfqpoint{4.163879in}{3.486854in}}%
\pgfpathlineto{\pgfqpoint{4.197751in}{3.407248in}}%
\pgfpathlineto{\pgfqpoint{4.231568in}{3.370614in}}%
\pgfpathlineto{\pgfqpoint{4.222127in}{3.460563in}}%
\pgfpathlineto{\pgfqpoint{4.212656in}{3.442466in}}%
\pgfpathlineto{\pgfqpoint{4.178876in}{3.453799in}}%
\pgfpathlineto{\pgfqpoint{4.145056in}{3.493548in}}%
\pgfpathclose%
\pgfusepath{fill}%
\end{pgfscope}%
\begin{pgfscope}%
\pgfpathrectangle{\pgfqpoint{1.020000in}{0.880000in}}{\pgfqpoint{6.160000in}{6.160000in}}%
\pgfusepath{clip}%
\pgfsetbuttcap%
\pgfsetroundjoin%
\definecolor{currentfill}{rgb}{0.826784,0.858205,0.906953}%
\pgfsetfillcolor{currentfill}%
\pgfsetlinewidth{0.000000pt}%
\definecolor{currentstroke}{rgb}{0.000000,0.000000,0.000000}%
\pgfsetstrokecolor{currentstroke}%
\pgfsetdash{}{0pt}%
\pgfpathmoveto{\pgfqpoint{3.017878in}{3.841059in}}%
\pgfpathlineto{\pgfqpoint{3.025319in}{3.905856in}}%
\pgfpathlineto{\pgfqpoint{3.036036in}{3.686890in}}%
\pgfpathlineto{\pgfqpoint{3.069381in}{3.759916in}}%
\pgfpathlineto{\pgfqpoint{3.103437in}{3.769613in}}%
\pgfpathlineto{\pgfqpoint{3.091935in}{4.067818in}}%
\pgfpathlineto{\pgfqpoint{3.085827in}{3.869496in}}%
\pgfpathlineto{\pgfqpoint{3.052792in}{3.772340in}}%
\pgfpathlineto{\pgfqpoint{3.017878in}{3.841059in}}%
\pgfpathclose%
\pgfusepath{fill}%
\end{pgfscope}%
\begin{pgfscope}%
\pgfpathrectangle{\pgfqpoint{1.020000in}{0.880000in}}{\pgfqpoint{6.160000in}{6.160000in}}%
\pgfusepath{clip}%
\pgfsetbuttcap%
\pgfsetroundjoin%
\definecolor{currentfill}{rgb}{0.804965,0.851666,0.926165}%
\pgfsetfillcolor{currentfill}%
\pgfsetlinewidth{0.000000pt}%
\definecolor{currentstroke}{rgb}{0.000000,0.000000,0.000000}%
\pgfsetstrokecolor{currentstroke}%
\pgfsetdash{}{0pt}%
\pgfpathmoveto{\pgfqpoint{2.437374in}{3.904136in}}%
\pgfpathlineto{\pgfqpoint{2.445737in}{3.858550in}}%
\pgfpathlineto{\pgfqpoint{2.455190in}{3.750314in}}%
\pgfpathlineto{\pgfqpoint{2.491411in}{3.640558in}}%
\pgfpathlineto{\pgfqpoint{2.521428in}{3.900129in}}%
\pgfpathlineto{\pgfqpoint{2.515720in}{3.782609in}}%
\pgfpathlineto{\pgfqpoint{2.510078in}{3.663253in}}%
\pgfpathlineto{\pgfqpoint{2.473688in}{3.788946in}}%
\pgfpathlineto{\pgfqpoint{2.437374in}{3.904136in}}%
\pgfpathclose%
\pgfusepath{fill}%
\end{pgfscope}%
\begin{pgfscope}%
\pgfpathrectangle{\pgfqpoint{1.020000in}{0.880000in}}{\pgfqpoint{6.160000in}{6.160000in}}%
\pgfusepath{clip}%
\pgfsetbuttcap%
\pgfsetroundjoin%
\definecolor{currentfill}{rgb}{0.738826,0.822572,0.968261}%
\pgfsetfillcolor{currentfill}%
\pgfsetlinewidth{0.000000pt}%
\definecolor{currentstroke}{rgb}{0.000000,0.000000,0.000000}%
\pgfsetstrokecolor{currentstroke}%
\pgfsetdash{}{0pt}%
\pgfpathmoveto{\pgfqpoint{3.325852in}{3.668778in}}%
\pgfpathlineto{\pgfqpoint{3.334320in}{3.674709in}}%
\pgfpathlineto{\pgfqpoint{3.343268in}{3.625635in}}%
\pgfpathlineto{\pgfqpoint{3.377299in}{3.635166in}}%
\pgfpathlineto{\pgfqpoint{3.411836in}{3.577316in}}%
\pgfpathlineto{\pgfqpoint{3.403515in}{3.541862in}}%
\pgfpathlineto{\pgfqpoint{3.392644in}{3.833353in}}%
\pgfpathlineto{\pgfqpoint{3.360173in}{3.636312in}}%
\pgfpathlineto{\pgfqpoint{3.325852in}{3.668778in}}%
\pgfpathclose%
\pgfusepath{fill}%
\end{pgfscope}%
\begin{pgfscope}%
\pgfpathrectangle{\pgfqpoint{1.020000in}{0.880000in}}{\pgfqpoint{6.160000in}{6.160000in}}%
\pgfusepath{clip}%
\pgfsetbuttcap%
\pgfsetroundjoin%
\definecolor{currentfill}{rgb}{0.772706,0.838978,0.949319}%
\pgfsetfillcolor{currentfill}%
\pgfsetlinewidth{0.000000pt}%
\definecolor{currentstroke}{rgb}{0.000000,0.000000,0.000000}%
\pgfsetstrokecolor{currentstroke}%
\pgfsetdash{}{0pt}%
\pgfpathmoveto{\pgfqpoint{3.392644in}{3.833353in}}%
\pgfpathlineto{\pgfqpoint{3.403515in}{3.541862in}}%
\pgfpathlineto{\pgfqpoint{3.411836in}{3.577316in}}%
\pgfpathlineto{\pgfqpoint{3.445387in}{3.645150in}}%
\pgfpathlineto{\pgfqpoint{3.479205in}{3.678696in}}%
\pgfpathlineto{\pgfqpoint{3.469122in}{3.876890in}}%
\pgfpathlineto{\pgfqpoint{3.461443in}{3.738879in}}%
\pgfpathlineto{\pgfqpoint{3.426954in}{3.801480in}}%
\pgfpathlineto{\pgfqpoint{3.392644in}{3.833353in}}%
\pgfpathclose%
\pgfusepath{fill}%
\end{pgfscope}%
\begin{pgfscope}%
\pgfpathrectangle{\pgfqpoint{1.020000in}{0.880000in}}{\pgfqpoint{6.160000in}{6.160000in}}%
\pgfusepath{clip}%
\pgfsetbuttcap%
\pgfsetroundjoin%
\definecolor{currentfill}{rgb}{0.859385,0.864431,0.872111}%
\pgfsetfillcolor{currentfill}%
\pgfsetlinewidth{0.000000pt}%
\definecolor{currentstroke}{rgb}{0.000000,0.000000,0.000000}%
\pgfsetstrokecolor{currentstroke}%
\pgfsetdash{}{0pt}%
\pgfpathmoveto{\pgfqpoint{3.085827in}{3.869496in}}%
\pgfpathlineto{\pgfqpoint{3.091935in}{4.067818in}}%
\pgfpathlineto{\pgfqpoint{3.103437in}{3.769613in}}%
\pgfpathlineto{\pgfqpoint{3.135729in}{3.947206in}}%
\pgfpathlineto{\pgfqpoint{3.170694in}{3.869743in}}%
\pgfpathlineto{\pgfqpoint{3.161960in}{3.908032in}}%
\pgfpathlineto{\pgfqpoint{3.152999in}{3.969593in}}%
\pgfpathlineto{\pgfqpoint{3.120106in}{3.853499in}}%
\pgfpathlineto{\pgfqpoint{3.085827in}{3.869496in}}%
\pgfpathclose%
\pgfusepath{fill}%
\end{pgfscope}%
\begin{pgfscope}%
\pgfpathrectangle{\pgfqpoint{1.020000in}{0.880000in}}{\pgfqpoint{6.160000in}{6.160000in}}%
\pgfusepath{clip}%
\pgfsetbuttcap%
\pgfsetroundjoin%
\definecolor{currentfill}{rgb}{0.343278,0.459354,0.884122}%
\pgfsetfillcolor{currentfill}%
\pgfsetlinewidth{0.000000pt}%
\definecolor{currentstroke}{rgb}{0.000000,0.000000,0.000000}%
\pgfsetstrokecolor{currentstroke}%
\pgfsetdash{}{0pt}%
\pgfpathmoveto{\pgfqpoint{5.360028in}{3.014454in}}%
\pgfpathlineto{\pgfqpoint{5.369071in}{2.873582in}}%
\pgfpathlineto{\pgfqpoint{5.379172in}{2.821290in}}%
\pgfpathlineto{\pgfqpoint{5.413992in}{2.929958in}}%
\pgfpathlineto{\pgfqpoint{5.447366in}{2.920710in}}%
\pgfpathlineto{\pgfqpoint{5.436950in}{2.951343in}}%
\pgfpathlineto{\pgfqpoint{5.425252in}{2.877970in}}%
\pgfpathlineto{\pgfqpoint{5.392227in}{2.909492in}}%
\pgfpathlineto{\pgfqpoint{5.360028in}{3.014454in}}%
\pgfpathclose%
\pgfusepath{fill}%
\end{pgfscope}%
\begin{pgfscope}%
\pgfpathrectangle{\pgfqpoint{1.020000in}{0.880000in}}{\pgfqpoint{6.160000in}{6.160000in}}%
\pgfusepath{clip}%
\pgfsetbuttcap%
\pgfsetroundjoin%
\definecolor{currentfill}{rgb}{0.822420,0.856898,0.910795}%
\pgfsetfillcolor{currentfill}%
\pgfsetlinewidth{0.000000pt}%
\definecolor{currentstroke}{rgb}{0.000000,0.000000,0.000000}%
\pgfsetstrokecolor{currentstroke}%
\pgfsetdash{}{0pt}%
\pgfpathmoveto{\pgfqpoint{2.728557in}{3.819565in}}%
\pgfpathlineto{\pgfqpoint{2.737913in}{3.716976in}}%
\pgfpathlineto{\pgfqpoint{2.745599in}{3.730965in}}%
\pgfpathlineto{\pgfqpoint{2.778009in}{3.863630in}}%
\pgfpathlineto{\pgfqpoint{2.813358in}{3.786122in}}%
\pgfpathlineto{\pgfqpoint{2.803369in}{3.933858in}}%
\pgfpathlineto{\pgfqpoint{2.796254in}{3.873043in}}%
\pgfpathlineto{\pgfqpoint{2.761866in}{3.884496in}}%
\pgfpathlineto{\pgfqpoint{2.728557in}{3.819565in}}%
\pgfpathclose%
\pgfusepath{fill}%
\end{pgfscope}%
\begin{pgfscope}%
\pgfpathrectangle{\pgfqpoint{1.020000in}{0.880000in}}{\pgfqpoint{6.160000in}{6.160000in}}%
\pgfusepath{clip}%
\pgfsetbuttcap%
\pgfsetroundjoin%
\definecolor{currentfill}{rgb}{0.299441,0.400248,0.839842}%
\pgfsetfillcolor{currentfill}%
\pgfsetlinewidth{0.000000pt}%
\definecolor{currentstroke}{rgb}{0.000000,0.000000,0.000000}%
\pgfsetstrokecolor{currentstroke}%
\pgfsetdash{}{0pt}%
\pgfpathmoveto{\pgfqpoint{5.513172in}{2.837697in}}%
\pgfpathlineto{\pgfqpoint{5.523893in}{2.825765in}}%
\pgfpathlineto{\pgfqpoint{5.533919in}{2.761162in}}%
\pgfpathlineto{\pgfqpoint{5.566598in}{2.706498in}}%
\pgfpathlineto{\pgfqpoint{5.602283in}{2.866535in}}%
\pgfpathlineto{\pgfqpoint{5.590334in}{2.797762in}}%
\pgfpathlineto{\pgfqpoint{5.581004in}{2.912537in}}%
\pgfpathlineto{\pgfqpoint{5.546841in}{2.857315in}}%
\pgfpathlineto{\pgfqpoint{5.513172in}{2.837697in}}%
\pgfpathclose%
\pgfusepath{fill}%
\end{pgfscope}%
\begin{pgfscope}%
\pgfpathrectangle{\pgfqpoint{1.020000in}{0.880000in}}{\pgfqpoint{6.160000in}{6.160000in}}%
\pgfusepath{clip}%
\pgfsetbuttcap%
\pgfsetroundjoin%
\definecolor{currentfill}{rgb}{0.318832,0.426605,0.859857}%
\pgfsetfillcolor{currentfill}%
\pgfsetlinewidth{0.000000pt}%
\definecolor{currentstroke}{rgb}{0.000000,0.000000,0.000000}%
\pgfsetstrokecolor{currentstroke}%
\pgfsetdash{}{0pt}%
\pgfpathmoveto{\pgfqpoint{6.090602in}{2.820794in}}%
\pgfpathlineto{\pgfqpoint{6.099630in}{2.692710in}}%
\pgfpathlineto{\pgfqpoint{6.116161in}{2.951200in}}%
\pgfpathlineto{\pgfqpoint{6.148724in}{2.913270in}}%
\pgfpathlineto{\pgfqpoint{6.136786in}{2.894500in}}%
\pgfpathlineto{\pgfqpoint{6.124812in}{2.872331in}}%
\pgfpathlineto{\pgfqpoint{6.090602in}{2.820794in}}%
\pgfpathclose%
\pgfusepath{fill}%
\end{pgfscope}%
\begin{pgfscope}%
\pgfpathrectangle{\pgfqpoint{1.020000in}{0.880000in}}{\pgfqpoint{6.160000in}{6.160000in}}%
\pgfusepath{clip}%
\pgfsetbuttcap%
\pgfsetroundjoin%
\definecolor{currentfill}{rgb}{0.777378,0.840921,0.946149}%
\pgfsetfillcolor{currentfill}%
\pgfsetlinewidth{0.000000pt}%
\definecolor{currentstroke}{rgb}{0.000000,0.000000,0.000000}%
\pgfsetstrokecolor{currentstroke}%
\pgfsetdash{}{0pt}%
\pgfpathmoveto{\pgfqpoint{2.591635in}{3.807334in}}%
\pgfpathlineto{\pgfqpoint{2.601814in}{3.652708in}}%
\pgfpathlineto{\pgfqpoint{2.610226in}{3.610102in}}%
\pgfpathlineto{\pgfqpoint{2.644465in}{3.614535in}}%
\pgfpathlineto{\pgfqpoint{2.677213in}{3.717093in}}%
\pgfpathlineto{\pgfqpoint{2.669283in}{3.724600in}}%
\pgfpathlineto{\pgfqpoint{2.660477in}{3.790806in}}%
\pgfpathlineto{\pgfqpoint{2.625196in}{3.855377in}}%
\pgfpathlineto{\pgfqpoint{2.591635in}{3.807334in}}%
\pgfpathclose%
\pgfusepath{fill}%
\end{pgfscope}%
\begin{pgfscope}%
\pgfpathrectangle{\pgfqpoint{1.020000in}{0.880000in}}{\pgfqpoint{6.160000in}{6.160000in}}%
\pgfusepath{clip}%
\pgfsetbuttcap%
\pgfsetroundjoin%
\definecolor{currentfill}{rgb}{0.548876,0.685104,0.994379}%
\pgfsetfillcolor{currentfill}%
\pgfsetlinewidth{0.000000pt}%
\definecolor{currentstroke}{rgb}{0.000000,0.000000,0.000000}%
\pgfsetstrokecolor{currentstroke}%
\pgfsetdash{}{0pt}%
\pgfpathmoveto{\pgfqpoint{4.453379in}{3.326432in}}%
\pgfpathlineto{\pgfqpoint{4.462813in}{3.218452in}}%
\pgfpathlineto{\pgfqpoint{4.473218in}{3.445774in}}%
\pgfpathlineto{\pgfqpoint{4.506536in}{3.284544in}}%
\pgfpathlineto{\pgfqpoint{4.540496in}{3.338711in}}%
\pgfpathlineto{\pgfqpoint{4.529862in}{3.106215in}}%
\pgfpathlineto{\pgfqpoint{4.520501in}{3.230359in}}%
\pgfpathlineto{\pgfqpoint{4.487129in}{3.331867in}}%
\pgfpathlineto{\pgfqpoint{4.453379in}{3.326432in}}%
\pgfpathclose%
\pgfusepath{fill}%
\end{pgfscope}%
\begin{pgfscope}%
\pgfpathrectangle{\pgfqpoint{1.020000in}{0.880000in}}{\pgfqpoint{6.160000in}{6.160000in}}%
\pgfusepath{clip}%
\pgfsetbuttcap%
\pgfsetroundjoin%
\definecolor{currentfill}{rgb}{0.373552,0.497499,0.909467}%
\pgfsetfillcolor{currentfill}%
\pgfsetlinewidth{0.000000pt}%
\definecolor{currentstroke}{rgb}{0.000000,0.000000,0.000000}%
\pgfsetstrokecolor{currentstroke}%
\pgfsetdash{}{0pt}%
\pgfpathmoveto{\pgfqpoint{5.136800in}{2.912384in}}%
\pgfpathlineto{\pgfqpoint{5.148683in}{3.058477in}}%
\pgfpathlineto{\pgfqpoint{5.158637in}{3.002179in}}%
\pgfpathlineto{\pgfqpoint{5.191048in}{2.885682in}}%
\pgfpathlineto{\pgfqpoint{5.226033in}{3.024447in}}%
\pgfpathlineto{\pgfqpoint{5.214650in}{2.945795in}}%
\pgfpathlineto{\pgfqpoint{5.204760in}{3.008910in}}%
\pgfpathlineto{\pgfqpoint{5.169962in}{2.878826in}}%
\pgfpathlineto{\pgfqpoint{5.136800in}{2.912384in}}%
\pgfpathclose%
\pgfusepath{fill}%
\end{pgfscope}%
\begin{pgfscope}%
\pgfpathrectangle{\pgfqpoint{1.020000in}{0.880000in}}{\pgfqpoint{6.160000in}{6.160000in}}%
\pgfusepath{clip}%
\pgfsetbuttcap%
\pgfsetroundjoin%
\definecolor{currentfill}{rgb}{0.724041,0.814910,0.975651}%
\pgfsetfillcolor{currentfill}%
\pgfsetlinewidth{0.000000pt}%
\definecolor{currentstroke}{rgb}{0.000000,0.000000,0.000000}%
\pgfsetstrokecolor{currentstroke}%
\pgfsetdash{}{0pt}%
\pgfpathmoveto{\pgfqpoint{3.769097in}{3.639564in}}%
\pgfpathlineto{\pgfqpoint{3.777697in}{3.738019in}}%
\pgfpathlineto{\pgfqpoint{3.787489in}{3.544306in}}%
\pgfpathlineto{\pgfqpoint{3.821887in}{3.426160in}}%
\pgfpathlineto{\pgfqpoint{3.855477in}{3.524266in}}%
\pgfpathlineto{\pgfqpoint{3.846177in}{3.591119in}}%
\pgfpathlineto{\pgfqpoint{3.836401in}{3.797848in}}%
\pgfpathlineto{\pgfqpoint{3.803105in}{3.620587in}}%
\pgfpathlineto{\pgfqpoint{3.769097in}{3.639564in}}%
\pgfpathclose%
\pgfusepath{fill}%
\end{pgfscope}%
\begin{pgfscope}%
\pgfpathrectangle{\pgfqpoint{1.020000in}{0.880000in}}{\pgfqpoint{6.160000in}{6.160000in}}%
\pgfusepath{clip}%
\pgfsetbuttcap%
\pgfsetroundjoin%
\definecolor{currentfill}{rgb}{0.309060,0.413498,0.850128}%
\pgfsetfillcolor{currentfill}%
\pgfsetlinewidth{0.000000pt}%
\definecolor{currentstroke}{rgb}{0.000000,0.000000,0.000000}%
\pgfsetstrokecolor{currentstroke}%
\pgfsetdash{}{0pt}%
\pgfpathmoveto{\pgfqpoint{6.024344in}{2.832038in}}%
\pgfpathlineto{\pgfqpoint{6.036212in}{2.855204in}}%
\pgfpathlineto{\pgfqpoint{6.048028in}{2.873985in}}%
\pgfpathlineto{\pgfqpoint{6.080115in}{2.809051in}}%
\pgfpathlineto{\pgfqpoint{6.116161in}{2.951200in}}%
\pgfpathlineto{\pgfqpoint{6.099630in}{2.692710in}}%
\pgfpathlineto{\pgfqpoint{6.090602in}{2.820794in}}%
\pgfpathlineto{\pgfqpoint{6.058078in}{2.857851in}}%
\pgfpathlineto{\pgfqpoint{6.024344in}{2.832038in}}%
\pgfpathclose%
\pgfusepath{fill}%
\end{pgfscope}%
\begin{pgfscope}%
\pgfpathrectangle{\pgfqpoint{1.020000in}{0.880000in}}{\pgfqpoint{6.160000in}{6.160000in}}%
\pgfusepath{clip}%
\pgfsetbuttcap%
\pgfsetroundjoin%
\definecolor{currentfill}{rgb}{0.809329,0.852974,0.922323}%
\pgfsetfillcolor{currentfill}%
\pgfsetlinewidth{0.000000pt}%
\definecolor{currentstroke}{rgb}{0.000000,0.000000,0.000000}%
\pgfsetstrokecolor{currentstroke}%
\pgfsetdash{}{0pt}%
\pgfpathmoveto{\pgfqpoint{3.170694in}{3.869743in}}%
\pgfpathlineto{\pgfqpoint{3.179239in}{3.851470in}}%
\pgfpathlineto{\pgfqpoint{3.188418in}{3.770762in}}%
\pgfpathlineto{\pgfqpoint{3.222826in}{3.746008in}}%
\pgfpathlineto{\pgfqpoint{3.257007in}{3.741806in}}%
\pgfpathlineto{\pgfqpoint{3.248074in}{3.794916in}}%
\pgfpathlineto{\pgfqpoint{3.240303in}{3.725381in}}%
\pgfpathlineto{\pgfqpoint{3.205645in}{3.786085in}}%
\pgfpathlineto{\pgfqpoint{3.170694in}{3.869743in}}%
\pgfpathclose%
\pgfusepath{fill}%
\end{pgfscope}%
\begin{pgfscope}%
\pgfpathrectangle{\pgfqpoint{1.020000in}{0.880000in}}{\pgfqpoint{6.160000in}{6.160000in}}%
\pgfusepath{clip}%
\pgfsetbuttcap%
\pgfsetroundjoin%
\definecolor{currentfill}{rgb}{0.758539,0.832787,0.958408}%
\pgfsetfillcolor{currentfill}%
\pgfsetlinewidth{0.000000pt}%
\definecolor{currentstroke}{rgb}{0.000000,0.000000,0.000000}%
\pgfsetstrokecolor{currentstroke}%
\pgfsetdash{}{0pt}%
\pgfpathmoveto{\pgfqpoint{2.521428in}{3.900129in}}%
\pgfpathlineto{\pgfqpoint{2.534337in}{3.582315in}}%
\pgfpathlineto{\pgfqpoint{2.539920in}{3.709607in}}%
\pgfpathlineto{\pgfqpoint{2.576805in}{3.553641in}}%
\pgfpathlineto{\pgfqpoint{2.610226in}{3.610102in}}%
\pgfpathlineto{\pgfqpoint{2.601814in}{3.652708in}}%
\pgfpathlineto{\pgfqpoint{2.591635in}{3.807334in}}%
\pgfpathlineto{\pgfqpoint{2.560402in}{3.616430in}}%
\pgfpathlineto{\pgfqpoint{2.521428in}{3.900129in}}%
\pgfpathclose%
\pgfusepath{fill}%
\end{pgfscope}%
\begin{pgfscope}%
\pgfpathrectangle{\pgfqpoint{1.020000in}{0.880000in}}{\pgfqpoint{6.160000in}{6.160000in}}%
\pgfusepath{clip}%
\pgfsetbuttcap%
\pgfsetroundjoin%
\definecolor{currentfill}{rgb}{0.404421,0.534643,0.932002}%
\pgfsetfillcolor{currentfill}%
\pgfsetlinewidth{0.000000pt}%
\definecolor{currentstroke}{rgb}{0.000000,0.000000,0.000000}%
\pgfsetstrokecolor{currentstroke}%
\pgfsetdash{}{0pt}%
\pgfpathmoveto{\pgfqpoint{4.915146in}{2.941269in}}%
\pgfpathlineto{\pgfqpoint{4.925939in}{3.017148in}}%
\pgfpathlineto{\pgfqpoint{4.937224in}{3.153221in}}%
\pgfpathlineto{\pgfqpoint{4.969111in}{2.923844in}}%
\pgfpathlineto{\pgfqpoint{5.003927in}{3.071520in}}%
\pgfpathlineto{\pgfqpoint{4.993192in}{3.021534in}}%
\pgfpathlineto{\pgfqpoint{4.982759in}{3.005213in}}%
\pgfpathlineto{\pgfqpoint{4.949290in}{3.017143in}}%
\pgfpathlineto{\pgfqpoint{4.915146in}{2.941269in}}%
\pgfpathclose%
\pgfusepath{fill}%
\end{pgfscope}%
\begin{pgfscope}%
\pgfpathrectangle{\pgfqpoint{1.020000in}{0.880000in}}{\pgfqpoint{6.160000in}{6.160000in}}%
\pgfusepath{clip}%
\pgfsetbuttcap%
\pgfsetroundjoin%
\definecolor{currentfill}{rgb}{0.748682,0.827679,0.963334}%
\pgfsetfillcolor{currentfill}%
\pgfsetlinewidth{0.000000pt}%
\definecolor{currentstroke}{rgb}{0.000000,0.000000,0.000000}%
\pgfsetstrokecolor{currentstroke}%
\pgfsetdash{}{0pt}%
\pgfpathmoveto{\pgfqpoint{3.547205in}{3.689445in}}%
\pgfpathlineto{\pgfqpoint{3.556463in}{3.609992in}}%
\pgfpathlineto{\pgfqpoint{3.565957in}{3.493213in}}%
\pgfpathlineto{\pgfqpoint{3.598161in}{3.802714in}}%
\pgfpathlineto{\pgfqpoint{3.632774in}{3.707967in}}%
\pgfpathlineto{\pgfqpoint{3.623672in}{3.753354in}}%
\pgfpathlineto{\pgfqpoint{3.615791in}{3.585042in}}%
\pgfpathlineto{\pgfqpoint{3.581390in}{3.659917in}}%
\pgfpathlineto{\pgfqpoint{3.547205in}{3.689445in}}%
\pgfpathclose%
\pgfusepath{fill}%
\end{pgfscope}%
\begin{pgfscope}%
\pgfpathrectangle{\pgfqpoint{1.020000in}{0.880000in}}{\pgfqpoint{6.160000in}{6.160000in}}%
\pgfusepath{clip}%
\pgfsetbuttcap%
\pgfsetroundjoin%
\definecolor{currentfill}{rgb}{0.304174,0.406945,0.845263}%
\pgfsetfillcolor{currentfill}%
\pgfsetlinewidth{0.000000pt}%
\definecolor{currentstroke}{rgb}{0.000000,0.000000,0.000000}%
\pgfsetstrokecolor{currentstroke}%
\pgfsetdash{}{0pt}%
\pgfpathmoveto{\pgfqpoint{5.735416in}{2.835369in}}%
\pgfpathlineto{\pgfqpoint{5.745340in}{2.758918in}}%
\pgfpathlineto{\pgfqpoint{5.757857in}{2.845685in}}%
\pgfpathlineto{\pgfqpoint{5.789654in}{2.745759in}}%
\pgfpathlineto{\pgfqpoint{5.823447in}{2.772078in}}%
\pgfpathlineto{\pgfqpoint{5.814712in}{2.922950in}}%
\pgfpathlineto{\pgfqpoint{5.801295in}{2.786941in}}%
\pgfpathlineto{\pgfqpoint{5.770035in}{2.915299in}}%
\pgfpathlineto{\pgfqpoint{5.735416in}{2.835369in}}%
\pgfpathclose%
\pgfusepath{fill}%
\end{pgfscope}%
\begin{pgfscope}%
\pgfpathrectangle{\pgfqpoint{1.020000in}{0.880000in}}{\pgfqpoint{6.160000in}{6.160000in}}%
\pgfusepath{clip}%
\pgfsetbuttcap%
\pgfsetroundjoin%
\definecolor{currentfill}{rgb}{0.333490,0.446265,0.874452}%
\pgfsetfillcolor{currentfill}%
\pgfsetlinewidth{0.000000pt}%
\definecolor{currentstroke}{rgb}{0.000000,0.000000,0.000000}%
\pgfsetstrokecolor{currentstroke}%
\pgfsetdash{}{0pt}%
\pgfpathmoveto{\pgfqpoint{5.801295in}{2.786941in}}%
\pgfpathlineto{\pgfqpoint{5.814712in}{2.922950in}}%
\pgfpathlineto{\pgfqpoint{5.823447in}{2.772078in}}%
\pgfpathlineto{\pgfqpoint{5.860031in}{2.964130in}}%
\pgfpathlineto{\pgfqpoint{5.890782in}{2.810435in}}%
\pgfpathlineto{\pgfqpoint{5.882407in}{2.980561in}}%
\pgfpathlineto{\pgfqpoint{5.868010in}{2.796365in}}%
\pgfpathlineto{\pgfqpoint{5.838733in}{3.036640in}}%
\pgfpathlineto{\pgfqpoint{5.801295in}{2.786941in}}%
\pgfpathclose%
\pgfusepath{fill}%
\end{pgfscope}%
\begin{pgfscope}%
\pgfpathrectangle{\pgfqpoint{1.020000in}{0.880000in}}{\pgfqpoint{6.160000in}{6.160000in}}%
\pgfusepath{clip}%
\pgfsetbuttcap%
\pgfsetroundjoin%
\definecolor{currentfill}{rgb}{0.688188,0.793178,0.988038}%
\pgfsetfillcolor{currentfill}%
\pgfsetlinewidth{0.000000pt}%
\definecolor{currentstroke}{rgb}{0.000000,0.000000,0.000000}%
\pgfsetstrokecolor{currentstroke}%
\pgfsetdash{}{0pt}%
\pgfpathmoveto{\pgfqpoint{3.923200in}{3.564112in}}%
\pgfpathlineto{\pgfqpoint{3.932394in}{3.552700in}}%
\pgfpathlineto{\pgfqpoint{3.941571in}{3.555739in}}%
\pgfpathlineto{\pgfqpoint{3.975606in}{3.519942in}}%
\pgfpathlineto{\pgfqpoint{4.009612in}{3.475653in}}%
\pgfpathlineto{\pgfqpoint{4.000241in}{3.544800in}}%
\pgfpathlineto{\pgfqpoint{3.991064in}{3.516414in}}%
\pgfpathlineto{\pgfqpoint{3.956997in}{3.608117in}}%
\pgfpathlineto{\pgfqpoint{3.923200in}{3.564112in}}%
\pgfpathclose%
\pgfusepath{fill}%
\end{pgfscope}%
\begin{pgfscope}%
\pgfpathrectangle{\pgfqpoint{1.020000in}{0.880000in}}{\pgfqpoint{6.160000in}{6.160000in}}%
\pgfusepath{clip}%
\pgfsetbuttcap%
\pgfsetroundjoin%
\definecolor{currentfill}{rgb}{0.467678,0.605591,0.968546}%
\pgfsetfillcolor{currentfill}%
\pgfsetlinewidth{0.000000pt}%
\definecolor{currentstroke}{rgb}{0.000000,0.000000,0.000000}%
\pgfsetstrokecolor{currentstroke}%
\pgfsetdash{}{0pt}%
\pgfpathmoveto{\pgfqpoint{4.761878in}{3.164631in}}%
\pgfpathlineto{\pgfqpoint{4.772398in}{3.240447in}}%
\pgfpathlineto{\pgfqpoint{4.781926in}{3.144920in}}%
\pgfpathlineto{\pgfqpoint{4.816435in}{3.270919in}}%
\pgfpathlineto{\pgfqpoint{4.848519in}{3.024105in}}%
\pgfpathlineto{\pgfqpoint{4.839310in}{3.166728in}}%
\pgfpathlineto{\pgfqpoint{4.828834in}{3.114706in}}%
\pgfpathlineto{\pgfqpoint{4.794138in}{2.938209in}}%
\pgfpathlineto{\pgfqpoint{4.761878in}{3.164631in}}%
\pgfpathclose%
\pgfusepath{fill}%
\end{pgfscope}%
\begin{pgfscope}%
\pgfpathrectangle{\pgfqpoint{1.020000in}{0.880000in}}{\pgfqpoint{6.160000in}{6.160000in}}%
\pgfusepath{clip}%
\pgfsetbuttcap%
\pgfsetroundjoin%
\definecolor{currentfill}{rgb}{0.835345,0.860514,0.898970}%
\pgfsetfillcolor{currentfill}%
\pgfsetlinewidth{0.000000pt}%
\definecolor{currentstroke}{rgb}{0.000000,0.000000,0.000000}%
\pgfsetstrokecolor{currentstroke}%
\pgfsetdash{}{0pt}%
\pgfpathmoveto{\pgfqpoint{2.948375in}{3.935631in}}%
\pgfpathlineto{\pgfqpoint{2.955981in}{3.978273in}}%
\pgfpathlineto{\pgfqpoint{2.967681in}{3.685354in}}%
\pgfpathlineto{\pgfqpoint{3.000247in}{3.823689in}}%
\pgfpathlineto{\pgfqpoint{3.036036in}{3.686890in}}%
\pgfpathlineto{\pgfqpoint{3.025319in}{3.905856in}}%
\pgfpathlineto{\pgfqpoint{3.017878in}{3.841059in}}%
\pgfpathlineto{\pgfqpoint{2.982723in}{3.924291in}}%
\pgfpathlineto{\pgfqpoint{2.948375in}{3.935631in}}%
\pgfpathclose%
\pgfusepath{fill}%
\end{pgfscope}%
\begin{pgfscope}%
\pgfpathrectangle{\pgfqpoint{1.020000in}{0.880000in}}{\pgfqpoint{6.160000in}{6.160000in}}%
\pgfusepath{clip}%
\pgfsetbuttcap%
\pgfsetroundjoin%
\definecolor{currentfill}{rgb}{0.768034,0.837035,0.952488}%
\pgfsetfillcolor{currentfill}%
\pgfsetlinewidth{0.000000pt}%
\definecolor{currentstroke}{rgb}{0.000000,0.000000,0.000000}%
\pgfsetstrokecolor{currentstroke}%
\pgfsetdash{}{0pt}%
\pgfpathmoveto{\pgfqpoint{3.036036in}{3.686890in}}%
\pgfpathlineto{\pgfqpoint{3.045452in}{3.580754in}}%
\pgfpathlineto{\pgfqpoint{3.052957in}{3.644143in}}%
\pgfpathlineto{\pgfqpoint{3.085903in}{3.759625in}}%
\pgfpathlineto{\pgfqpoint{3.121809in}{3.601355in}}%
\pgfpathlineto{\pgfqpoint{3.112011in}{3.742575in}}%
\pgfpathlineto{\pgfqpoint{3.103437in}{3.769613in}}%
\pgfpathlineto{\pgfqpoint{3.069381in}{3.759916in}}%
\pgfpathlineto{\pgfqpoint{3.036036in}{3.686890in}}%
\pgfpathclose%
\pgfusepath{fill}%
\end{pgfscope}%
\begin{pgfscope}%
\pgfpathrectangle{\pgfqpoint{1.020000in}{0.880000in}}{\pgfqpoint{6.160000in}{6.160000in}}%
\pgfusepath{clip}%
\pgfsetbuttcap%
\pgfsetroundjoin%
\definecolor{currentfill}{rgb}{0.592356,0.722792,0.999434}%
\pgfsetfillcolor{currentfill}%
\pgfsetlinewidth{0.000000pt}%
\definecolor{currentstroke}{rgb}{0.000000,0.000000,0.000000}%
\pgfsetstrokecolor{currentstroke}%
\pgfsetdash{}{0pt}%
\pgfpathmoveto{\pgfqpoint{4.231568in}{3.370614in}}%
\pgfpathlineto{\pgfqpoint{4.241033in}{3.314081in}}%
\pgfpathlineto{\pgfqpoint{4.250663in}{3.489050in}}%
\pgfpathlineto{\pgfqpoint{4.284457in}{3.374321in}}%
\pgfpathlineto{\pgfqpoint{4.318104in}{3.222982in}}%
\pgfpathlineto{\pgfqpoint{4.308496in}{3.193730in}}%
\pgfpathlineto{\pgfqpoint{4.299146in}{3.363381in}}%
\pgfpathlineto{\pgfqpoint{4.265540in}{3.576396in}}%
\pgfpathlineto{\pgfqpoint{4.231568in}{3.370614in}}%
\pgfpathclose%
\pgfusepath{fill}%
\end{pgfscope}%
\begin{pgfscope}%
\pgfpathrectangle{\pgfqpoint{1.020000in}{0.880000in}}{\pgfqpoint{6.160000in}{6.160000in}}%
\pgfusepath{clip}%
\pgfsetbuttcap%
\pgfsetroundjoin%
\definecolor{currentfill}{rgb}{0.748682,0.827679,0.963334}%
\pgfsetfillcolor{currentfill}%
\pgfsetlinewidth{0.000000pt}%
\definecolor{currentstroke}{rgb}{0.000000,0.000000,0.000000}%
\pgfsetstrokecolor{currentstroke}%
\pgfsetdash{}{0pt}%
\pgfpathmoveto{\pgfqpoint{3.257007in}{3.741806in}}%
\pgfpathlineto{\pgfqpoint{3.265037in}{3.788413in}}%
\pgfpathlineto{\pgfqpoint{3.277259in}{3.375372in}}%
\pgfpathlineto{\pgfqpoint{3.309278in}{3.609807in}}%
\pgfpathlineto{\pgfqpoint{3.343268in}{3.625635in}}%
\pgfpathlineto{\pgfqpoint{3.334320in}{3.674709in}}%
\pgfpathlineto{\pgfqpoint{3.325852in}{3.668778in}}%
\pgfpathlineto{\pgfqpoint{3.290651in}{3.795345in}}%
\pgfpathlineto{\pgfqpoint{3.257007in}{3.741806in}}%
\pgfpathclose%
\pgfusepath{fill}%
\end{pgfscope}%
\begin{pgfscope}%
\pgfpathrectangle{\pgfqpoint{1.020000in}{0.880000in}}{\pgfqpoint{6.160000in}{6.160000in}}%
\pgfusepath{clip}%
\pgfsetbuttcap%
\pgfsetroundjoin%
\definecolor{currentfill}{rgb}{0.294718,0.393542,0.834384}%
\pgfsetfillcolor{currentfill}%
\pgfsetlinewidth{0.000000pt}%
\definecolor{currentstroke}{rgb}{0.000000,0.000000,0.000000}%
\pgfsetstrokecolor{currentstroke}%
\pgfsetdash{}{0pt}%
\pgfpathmoveto{\pgfqpoint{5.957813in}{2.833395in}}%
\pgfpathlineto{\pgfqpoint{5.968865in}{2.815589in}}%
\pgfpathlineto{\pgfqpoint{5.977890in}{2.684593in}}%
\pgfpathlineto{\pgfqpoint{6.011223in}{2.686192in}}%
\pgfpathlineto{\pgfqpoint{6.048028in}{2.873985in}}%
\pgfpathlineto{\pgfqpoint{6.036212in}{2.855204in}}%
\pgfpathlineto{\pgfqpoint{6.024344in}{2.832038in}}%
\pgfpathlineto{\pgfqpoint{5.991880in}{2.876226in}}%
\pgfpathlineto{\pgfqpoint{5.957813in}{2.833395in}}%
\pgfpathclose%
\pgfusepath{fill}%
\end{pgfscope}%
\begin{pgfscope}%
\pgfpathrectangle{\pgfqpoint{1.020000in}{0.880000in}}{\pgfqpoint{6.160000in}{6.160000in}}%
\pgfusepath{clip}%
\pgfsetbuttcap%
\pgfsetroundjoin%
\definecolor{currentfill}{rgb}{0.304174,0.406945,0.845263}%
\pgfsetfillcolor{currentfill}%
\pgfsetlinewidth{0.000000pt}%
\definecolor{currentstroke}{rgb}{0.000000,0.000000,0.000000}%
\pgfsetstrokecolor{currentstroke}%
\pgfsetdash{}{0pt}%
\pgfpathmoveto{\pgfqpoint{5.447366in}{2.920710in}}%
\pgfpathlineto{\pgfqpoint{5.458373in}{2.934894in}}%
\pgfpathlineto{\pgfqpoint{5.466838in}{2.750342in}}%
\pgfpathlineto{\pgfqpoint{5.501973in}{2.875320in}}%
\pgfpathlineto{\pgfqpoint{5.533919in}{2.761162in}}%
\pgfpathlineto{\pgfqpoint{5.523893in}{2.825765in}}%
\pgfpathlineto{\pgfqpoint{5.513172in}{2.837697in}}%
\pgfpathlineto{\pgfqpoint{5.477273in}{2.647660in}}%
\pgfpathlineto{\pgfqpoint{5.447366in}{2.920710in}}%
\pgfpathclose%
\pgfusepath{fill}%
\end{pgfscope}%
\begin{pgfscope}%
\pgfpathrectangle{\pgfqpoint{1.020000in}{0.880000in}}{\pgfqpoint{6.160000in}{6.160000in}}%
\pgfusepath{clip}%
\pgfsetbuttcap%
\pgfsetroundjoin%
\definecolor{currentfill}{rgb}{0.388852,0.516298,0.921373}%
\pgfsetfillcolor{currentfill}%
\pgfsetlinewidth{0.000000pt}%
\definecolor{currentstroke}{rgb}{0.000000,0.000000,0.000000}%
\pgfsetstrokecolor{currentstroke}%
\pgfsetdash{}{0pt}%
\pgfpathmoveto{\pgfqpoint{4.848519in}{3.024105in}}%
\pgfpathlineto{\pgfqpoint{4.857665in}{2.873416in}}%
\pgfpathlineto{\pgfqpoint{4.869392in}{3.098791in}}%
\pgfpathlineto{\pgfqpoint{4.901851in}{2.924110in}}%
\pgfpathlineto{\pgfqpoint{4.937224in}{3.153221in}}%
\pgfpathlineto{\pgfqpoint{4.925939in}{3.017148in}}%
\pgfpathlineto{\pgfqpoint{4.915146in}{2.941269in}}%
\pgfpathlineto{\pgfqpoint{4.881290in}{2.901978in}}%
\pgfpathlineto{\pgfqpoint{4.848519in}{3.024105in}}%
\pgfpathclose%
\pgfusepath{fill}%
\end{pgfscope}%
\begin{pgfscope}%
\pgfpathrectangle{\pgfqpoint{1.020000in}{0.880000in}}{\pgfqpoint{6.160000in}{6.160000in}}%
\pgfusepath{clip}%
\pgfsetbuttcap%
\pgfsetroundjoin%
\definecolor{currentfill}{rgb}{0.796064,0.848693,0.933471}%
\pgfsetfillcolor{currentfill}%
\pgfsetlinewidth{0.000000pt}%
\definecolor{currentstroke}{rgb}{0.000000,0.000000,0.000000}%
\pgfsetstrokecolor{currentstroke}%
\pgfsetdash{}{0pt}%
\pgfpathmoveto{\pgfqpoint{2.813358in}{3.786122in}}%
\pgfpathlineto{\pgfqpoint{2.822511in}{3.699576in}}%
\pgfpathlineto{\pgfqpoint{2.831511in}{3.624554in}}%
\pgfpathlineto{\pgfqpoint{2.863510in}{3.796270in}}%
\pgfpathlineto{\pgfqpoint{2.898369in}{3.752304in}}%
\pgfpathlineto{\pgfqpoint{2.888497in}{3.892852in}}%
\pgfpathlineto{\pgfqpoint{2.881985in}{3.773876in}}%
\pgfpathlineto{\pgfqpoint{2.849120in}{3.672100in}}%
\pgfpathlineto{\pgfqpoint{2.813358in}{3.786122in}}%
\pgfpathclose%
\pgfusepath{fill}%
\end{pgfscope}%
\begin{pgfscope}%
\pgfpathrectangle{\pgfqpoint{1.020000in}{0.880000in}}{\pgfqpoint{6.160000in}{6.160000in}}%
\pgfusepath{clip}%
\pgfsetbuttcap%
\pgfsetroundjoin%
\definecolor{currentfill}{rgb}{0.656683,0.771806,0.994914}%
\pgfsetfillcolor{currentfill}%
\pgfsetlinewidth{0.000000pt}%
\definecolor{currentstroke}{rgb}{0.000000,0.000000,0.000000}%
\pgfsetstrokecolor{currentstroke}%
\pgfsetdash{}{0pt}%
\pgfpathmoveto{\pgfqpoint{4.077225in}{3.620567in}}%
\pgfpathlineto{\pgfqpoint{4.086820in}{3.361362in}}%
\pgfpathlineto{\pgfqpoint{4.096085in}{3.449469in}}%
\pgfpathlineto{\pgfqpoint{4.130058in}{3.331675in}}%
\pgfpathlineto{\pgfqpoint{4.163879in}{3.486854in}}%
\pgfpathlineto{\pgfqpoint{4.154477in}{3.422649in}}%
\pgfpathlineto{\pgfqpoint{4.145056in}{3.493548in}}%
\pgfpathlineto{\pgfqpoint{4.111088in}{3.673568in}}%
\pgfpathlineto{\pgfqpoint{4.077225in}{3.620567in}}%
\pgfpathclose%
\pgfusepath{fill}%
\end{pgfscope}%
\begin{pgfscope}%
\pgfpathrectangle{\pgfqpoint{1.020000in}{0.880000in}}{\pgfqpoint{6.160000in}{6.160000in}}%
\pgfusepath{clip}%
\pgfsetbuttcap%
\pgfsetroundjoin%
\definecolor{currentfill}{rgb}{0.804965,0.851666,0.926165}%
\pgfsetfillcolor{currentfill}%
\pgfsetlinewidth{0.000000pt}%
\definecolor{currentstroke}{rgb}{0.000000,0.000000,0.000000}%
\pgfsetstrokecolor{currentstroke}%
\pgfsetdash{}{0pt}%
\pgfpathmoveto{\pgfqpoint{3.103437in}{3.769613in}}%
\pgfpathlineto{\pgfqpoint{3.112011in}{3.742575in}}%
\pgfpathlineto{\pgfqpoint{3.121809in}{3.601355in}}%
\pgfpathlineto{\pgfqpoint{3.155322in}{3.663898in}}%
\pgfpathlineto{\pgfqpoint{3.188418in}{3.770762in}}%
\pgfpathlineto{\pgfqpoint{3.179239in}{3.851470in}}%
\pgfpathlineto{\pgfqpoint{3.170694in}{3.869743in}}%
\pgfpathlineto{\pgfqpoint{3.135729in}{3.947206in}}%
\pgfpathlineto{\pgfqpoint{3.103437in}{3.769613in}}%
\pgfpathclose%
\pgfusepath{fill}%
\end{pgfscope}%
\begin{pgfscope}%
\pgfpathrectangle{\pgfqpoint{1.020000in}{0.880000in}}{\pgfqpoint{6.160000in}{6.160000in}}%
\pgfusepath{clip}%
\pgfsetbuttcap%
\pgfsetroundjoin%
\definecolor{currentfill}{rgb}{0.748682,0.827679,0.963334}%
\pgfsetfillcolor{currentfill}%
\pgfsetlinewidth{0.000000pt}%
\definecolor{currentstroke}{rgb}{0.000000,0.000000,0.000000}%
\pgfsetstrokecolor{currentstroke}%
\pgfsetdash{}{0pt}%
\pgfpathmoveto{\pgfqpoint{2.967681in}{3.685354in}}%
\pgfpathlineto{\pgfqpoint{2.976751in}{3.608286in}}%
\pgfpathlineto{\pgfqpoint{2.983586in}{3.718495in}}%
\pgfpathlineto{\pgfqpoint{3.020210in}{3.516418in}}%
\pgfpathlineto{\pgfqpoint{3.052957in}{3.644143in}}%
\pgfpathlineto{\pgfqpoint{3.045452in}{3.580754in}}%
\pgfpathlineto{\pgfqpoint{3.036036in}{3.686890in}}%
\pgfpathlineto{\pgfqpoint{3.000247in}{3.823689in}}%
\pgfpathlineto{\pgfqpoint{2.967681in}{3.685354in}}%
\pgfpathclose%
\pgfusepath{fill}%
\end{pgfscope}%
\begin{pgfscope}%
\pgfpathrectangle{\pgfqpoint{1.020000in}{0.880000in}}{\pgfqpoint{6.160000in}{6.160000in}}%
\pgfusepath{clip}%
\pgfsetbuttcap%
\pgfsetroundjoin%
\definecolor{currentfill}{rgb}{0.363461,0.484784,0.901019}%
\pgfsetfillcolor{currentfill}%
\pgfsetlinewidth{0.000000pt}%
\definecolor{currentstroke}{rgb}{0.000000,0.000000,0.000000}%
\pgfsetstrokecolor{currentstroke}%
\pgfsetdash{}{0pt}%
\pgfpathmoveto{\pgfqpoint{5.293583in}{3.065086in}}%
\pgfpathlineto{\pgfqpoint{5.303290in}{2.980184in}}%
\pgfpathlineto{\pgfqpoint{5.313116in}{2.905338in}}%
\pgfpathlineto{\pgfqpoint{5.347342in}{2.963637in}}%
\pgfpathlineto{\pgfqpoint{5.379172in}{2.821290in}}%
\pgfpathlineto{\pgfqpoint{5.369071in}{2.873582in}}%
\pgfpathlineto{\pgfqpoint{5.360028in}{3.014454in}}%
\pgfpathlineto{\pgfqpoint{5.325152in}{2.894109in}}%
\pgfpathlineto{\pgfqpoint{5.293583in}{3.065086in}}%
\pgfpathclose%
\pgfusepath{fill}%
\end{pgfscope}%
\begin{pgfscope}%
\pgfpathrectangle{\pgfqpoint{1.020000in}{0.880000in}}{\pgfqpoint{6.160000in}{6.160000in}}%
\pgfusepath{clip}%
\pgfsetbuttcap%
\pgfsetroundjoin%
\definecolor{currentfill}{rgb}{0.772706,0.838978,0.949319}%
\pgfsetfillcolor{currentfill}%
\pgfsetlinewidth{0.000000pt}%
\definecolor{currentstroke}{rgb}{0.000000,0.000000,0.000000}%
\pgfsetstrokecolor{currentstroke}%
\pgfsetdash{}{0pt}%
\pgfpathmoveto{\pgfqpoint{2.745599in}{3.730965in}}%
\pgfpathlineto{\pgfqpoint{2.756683in}{3.507946in}}%
\pgfpathlineto{\pgfqpoint{2.762154in}{3.679572in}}%
\pgfpathlineto{\pgfqpoint{2.795450in}{3.753400in}}%
\pgfpathlineto{\pgfqpoint{2.831511in}{3.624554in}}%
\pgfpathlineto{\pgfqpoint{2.822511in}{3.699576in}}%
\pgfpathlineto{\pgfqpoint{2.813358in}{3.786122in}}%
\pgfpathlineto{\pgfqpoint{2.778009in}{3.863630in}}%
\pgfpathlineto{\pgfqpoint{2.745599in}{3.730965in}}%
\pgfpathclose%
\pgfusepath{fill}%
\end{pgfscope}%
\begin{pgfscope}%
\pgfpathrectangle{\pgfqpoint{1.020000in}{0.880000in}}{\pgfqpoint{6.160000in}{6.160000in}}%
\pgfusepath{clip}%
\pgfsetbuttcap%
\pgfsetroundjoin%
\definecolor{currentfill}{rgb}{0.688188,0.793178,0.988038}%
\pgfsetfillcolor{currentfill}%
\pgfsetlinewidth{0.000000pt}%
\definecolor{currentstroke}{rgb}{0.000000,0.000000,0.000000}%
\pgfsetstrokecolor{currentstroke}%
\pgfsetdash{}{0pt}%
\pgfpathmoveto{\pgfqpoint{3.855477in}{3.524266in}}%
\pgfpathlineto{\pgfqpoint{3.864843in}{3.438782in}}%
\pgfpathlineto{\pgfqpoint{3.873458in}{3.597477in}}%
\pgfpathlineto{\pgfqpoint{3.907505in}{3.586534in}}%
\pgfpathlineto{\pgfqpoint{3.941571in}{3.555739in}}%
\pgfpathlineto{\pgfqpoint{3.932394in}{3.552700in}}%
\pgfpathlineto{\pgfqpoint{3.923200in}{3.564112in}}%
\pgfpathlineto{\pgfqpoint{3.889553in}{3.472349in}}%
\pgfpathlineto{\pgfqpoint{3.855477in}{3.524266in}}%
\pgfpathclose%
\pgfusepath{fill}%
\end{pgfscope}%
\begin{pgfscope}%
\pgfpathrectangle{\pgfqpoint{1.020000in}{0.880000in}}{\pgfqpoint{6.160000in}{6.160000in}}%
\pgfusepath{clip}%
\pgfsetbuttcap%
\pgfsetroundjoin%
\definecolor{currentfill}{rgb}{0.394042,0.522413,0.924916}%
\pgfsetfillcolor{currentfill}%
\pgfsetlinewidth{0.000000pt}%
\definecolor{currentstroke}{rgb}{0.000000,0.000000,0.000000}%
\pgfsetstrokecolor{currentstroke}%
\pgfsetdash{}{0pt}%
\pgfpathmoveto{\pgfqpoint{5.071195in}{3.074759in}}%
\pgfpathlineto{\pgfqpoint{5.080447in}{2.945348in}}%
\pgfpathlineto{\pgfqpoint{5.090978in}{2.956755in}}%
\pgfpathlineto{\pgfqpoint{5.124466in}{2.943088in}}%
\pgfpathlineto{\pgfqpoint{5.158637in}{3.002179in}}%
\pgfpathlineto{\pgfqpoint{5.148683in}{3.058477in}}%
\pgfpathlineto{\pgfqpoint{5.136800in}{2.912384in}}%
\pgfpathlineto{\pgfqpoint{5.104328in}{3.025298in}}%
\pgfpathlineto{\pgfqpoint{5.071195in}{3.074759in}}%
\pgfpathclose%
\pgfusepath{fill}%
\end{pgfscope}%
\begin{pgfscope}%
\pgfpathrectangle{\pgfqpoint{1.020000in}{0.880000in}}{\pgfqpoint{6.160000in}{6.160000in}}%
\pgfusepath{clip}%
\pgfsetbuttcap%
\pgfsetroundjoin%
\definecolor{currentfill}{rgb}{0.748682,0.827679,0.963334}%
\pgfsetfillcolor{currentfill}%
\pgfsetlinewidth{0.000000pt}%
\definecolor{currentstroke}{rgb}{0.000000,0.000000,0.000000}%
\pgfsetstrokecolor{currentstroke}%
\pgfsetdash{}{0pt}%
\pgfpathmoveto{\pgfqpoint{2.677213in}{3.717093in}}%
\pgfpathlineto{\pgfqpoint{2.685056in}{3.716509in}}%
\pgfpathlineto{\pgfqpoint{2.695550in}{3.538684in}}%
\pgfpathlineto{\pgfqpoint{2.727390in}{3.708889in}}%
\pgfpathlineto{\pgfqpoint{2.762154in}{3.679572in}}%
\pgfpathlineto{\pgfqpoint{2.756683in}{3.507946in}}%
\pgfpathlineto{\pgfqpoint{2.745599in}{3.730965in}}%
\pgfpathlineto{\pgfqpoint{2.712765in}{3.631991in}}%
\pgfpathlineto{\pgfqpoint{2.677213in}{3.717093in}}%
\pgfpathclose%
\pgfusepath{fill}%
\end{pgfscope}%
\begin{pgfscope}%
\pgfpathrectangle{\pgfqpoint{1.020000in}{0.880000in}}{\pgfqpoint{6.160000in}{6.160000in}}%
\pgfusepath{clip}%
\pgfsetbuttcap%
\pgfsetroundjoin%
\definecolor{currentfill}{rgb}{0.289996,0.386836,0.828926}%
\pgfsetfillcolor{currentfill}%
\pgfsetlinewidth{0.000000pt}%
\definecolor{currentstroke}{rgb}{0.000000,0.000000,0.000000}%
\pgfsetstrokecolor{currentstroke}%
\pgfsetdash{}{0pt}%
\pgfpathmoveto{\pgfqpoint{5.890782in}{2.810435in}}%
\pgfpathlineto{\pgfqpoint{5.901475in}{2.775132in}}%
\pgfpathlineto{\pgfqpoint{5.915144in}{2.909835in}}%
\pgfpathlineto{\pgfqpoint{5.945359in}{2.729231in}}%
\pgfpathlineto{\pgfqpoint{5.977890in}{2.684593in}}%
\pgfpathlineto{\pgfqpoint{5.968865in}{2.815589in}}%
\pgfpathlineto{\pgfqpoint{5.957813in}{2.833395in}}%
\pgfpathlineto{\pgfqpoint{5.923391in}{2.769809in}}%
\pgfpathlineto{\pgfqpoint{5.890782in}{2.810435in}}%
\pgfpathclose%
\pgfusepath{fill}%
\end{pgfscope}%
\begin{pgfscope}%
\pgfpathrectangle{\pgfqpoint{1.020000in}{0.880000in}}{\pgfqpoint{6.160000in}{6.160000in}}%
\pgfusepath{clip}%
\pgfsetbuttcap%
\pgfsetroundjoin%
\definecolor{currentfill}{rgb}{0.758539,0.832787,0.958408}%
\pgfsetfillcolor{currentfill}%
\pgfsetlinewidth{0.000000pt}%
\definecolor{currentstroke}{rgb}{0.000000,0.000000,0.000000}%
\pgfsetstrokecolor{currentstroke}%
\pgfsetdash{}{0pt}%
\pgfpathmoveto{\pgfqpoint{3.479205in}{3.678696in}}%
\pgfpathlineto{\pgfqpoint{3.487561in}{3.727207in}}%
\pgfpathlineto{\pgfqpoint{3.497017in}{3.619056in}}%
\pgfpathlineto{\pgfqpoint{3.530420in}{3.724934in}}%
\pgfpathlineto{\pgfqpoint{3.565957in}{3.493213in}}%
\pgfpathlineto{\pgfqpoint{3.556463in}{3.609992in}}%
\pgfpathlineto{\pgfqpoint{3.547205in}{3.689445in}}%
\pgfpathlineto{\pgfqpoint{3.512280in}{3.823674in}}%
\pgfpathlineto{\pgfqpoint{3.479205in}{3.678696in}}%
\pgfpathclose%
\pgfusepath{fill}%
\end{pgfscope}%
\begin{pgfscope}%
\pgfpathrectangle{\pgfqpoint{1.020000in}{0.880000in}}{\pgfqpoint{6.160000in}{6.160000in}}%
\pgfusepath{clip}%
\pgfsetbuttcap%
\pgfsetroundjoin%
\definecolor{currentfill}{rgb}{0.570616,0.704109,0.997195}%
\pgfsetfillcolor{currentfill}%
\pgfsetlinewidth{0.000000pt}%
\definecolor{currentstroke}{rgb}{0.000000,0.000000,0.000000}%
\pgfsetstrokecolor{currentstroke}%
\pgfsetdash{}{0pt}%
\pgfpathmoveto{\pgfqpoint{4.385532in}{3.168453in}}%
\pgfpathlineto{\pgfqpoint{4.395662in}{3.392672in}}%
\pgfpathlineto{\pgfqpoint{4.404996in}{3.229765in}}%
\pgfpathlineto{\pgfqpoint{4.439145in}{3.363080in}}%
\pgfpathlineto{\pgfqpoint{4.473218in}{3.445774in}}%
\pgfpathlineto{\pgfqpoint{4.462813in}{3.218452in}}%
\pgfpathlineto{\pgfqpoint{4.453379in}{3.326432in}}%
\pgfpathlineto{\pgfqpoint{4.419791in}{3.396111in}}%
\pgfpathlineto{\pgfqpoint{4.385532in}{3.168453in}}%
\pgfpathclose%
\pgfusepath{fill}%
\end{pgfscope}%
\begin{pgfscope}%
\pgfpathrectangle{\pgfqpoint{1.020000in}{0.880000in}}{\pgfqpoint{6.160000in}{6.160000in}}%
\pgfusepath{clip}%
\pgfsetbuttcap%
\pgfsetroundjoin%
\definecolor{currentfill}{rgb}{0.839351,0.861167,0.894494}%
\pgfsetfillcolor{currentfill}%
\pgfsetlinewidth{0.000000pt}%
\definecolor{currentstroke}{rgb}{0.000000,0.000000,0.000000}%
\pgfsetstrokecolor{currentstroke}%
\pgfsetdash{}{0pt}%
\pgfpathmoveto{\pgfqpoint{2.881985in}{3.773876in}}%
\pgfpathlineto{\pgfqpoint{2.888497in}{3.892852in}}%
\pgfpathlineto{\pgfqpoint{2.898369in}{3.752304in}}%
\pgfpathlineto{\pgfqpoint{2.929466in}{4.006080in}}%
\pgfpathlineto{\pgfqpoint{2.967681in}{3.685354in}}%
\pgfpathlineto{\pgfqpoint{2.955981in}{3.978273in}}%
\pgfpathlineto{\pgfqpoint{2.948375in}{3.935631in}}%
\pgfpathlineto{\pgfqpoint{2.916038in}{3.785437in}}%
\pgfpathlineto{\pgfqpoint{2.881985in}{3.773876in}}%
\pgfpathclose%
\pgfusepath{fill}%
\end{pgfscope}%
\begin{pgfscope}%
\pgfpathrectangle{\pgfqpoint{1.020000in}{0.880000in}}{\pgfqpoint{6.160000in}{6.160000in}}%
\pgfusepath{clip}%
\pgfsetbuttcap%
\pgfsetroundjoin%
\definecolor{currentfill}{rgb}{0.473070,0.611077,0.970634}%
\pgfsetfillcolor{currentfill}%
\pgfsetlinewidth{0.000000pt}%
\definecolor{currentstroke}{rgb}{0.000000,0.000000,0.000000}%
\pgfsetstrokecolor{currentstroke}%
\pgfsetdash{}{0pt}%
\pgfpathmoveto{\pgfqpoint{4.693975in}{3.058802in}}%
\pgfpathlineto{\pgfqpoint{4.705431in}{3.338255in}}%
\pgfpathlineto{\pgfqpoint{4.713187in}{2.908655in}}%
\pgfpathlineto{\pgfqpoint{4.748163in}{3.139176in}}%
\pgfpathlineto{\pgfqpoint{4.781926in}{3.144920in}}%
\pgfpathlineto{\pgfqpoint{4.772398in}{3.240447in}}%
\pgfpathlineto{\pgfqpoint{4.761878in}{3.164631in}}%
\pgfpathlineto{\pgfqpoint{4.728071in}{3.139573in}}%
\pgfpathlineto{\pgfqpoint{4.693975in}{3.058802in}}%
\pgfpathclose%
\pgfusepath{fill}%
\end{pgfscope}%
\begin{pgfscope}%
\pgfpathrectangle{\pgfqpoint{1.020000in}{0.880000in}}{\pgfqpoint{6.160000in}{6.160000in}}%
\pgfusepath{clip}%
\pgfsetbuttcap%
\pgfsetroundjoin%
\definecolor{currentfill}{rgb}{0.777378,0.840921,0.946149}%
\pgfsetfillcolor{currentfill}%
\pgfsetlinewidth{0.000000pt}%
\definecolor{currentstroke}{rgb}{0.000000,0.000000,0.000000}%
\pgfsetstrokecolor{currentstroke}%
\pgfsetdash{}{0pt}%
\pgfpathmoveto{\pgfqpoint{2.455190in}{3.750314in}}%
\pgfpathlineto{\pgfqpoint{2.461253in}{3.839707in}}%
\pgfpathlineto{\pgfqpoint{2.473033in}{3.595955in}}%
\pgfpathlineto{\pgfqpoint{2.506162in}{3.671032in}}%
\pgfpathlineto{\pgfqpoint{2.539920in}{3.709607in}}%
\pgfpathlineto{\pgfqpoint{2.534337in}{3.582315in}}%
\pgfpathlineto{\pgfqpoint{2.521428in}{3.900129in}}%
\pgfpathlineto{\pgfqpoint{2.491411in}{3.640558in}}%
\pgfpathlineto{\pgfqpoint{2.455190in}{3.750314in}}%
\pgfpathclose%
\pgfusepath{fill}%
\end{pgfscope}%
\begin{pgfscope}%
\pgfpathrectangle{\pgfqpoint{1.020000in}{0.880000in}}{\pgfqpoint{6.160000in}{6.160000in}}%
\pgfusepath{clip}%
\pgfsetbuttcap%
\pgfsetroundjoin%
\definecolor{currentfill}{rgb}{0.733898,0.820018,0.970724}%
\pgfsetfillcolor{currentfill}%
\pgfsetlinewidth{0.000000pt}%
\definecolor{currentstroke}{rgb}{0.000000,0.000000,0.000000}%
\pgfsetstrokecolor{currentstroke}%
\pgfsetdash{}{0pt}%
\pgfpathmoveto{\pgfqpoint{2.610226in}{3.610102in}}%
\pgfpathlineto{\pgfqpoint{2.616007in}{3.736559in}}%
\pgfpathlineto{\pgfqpoint{2.628592in}{3.427736in}}%
\pgfpathlineto{\pgfqpoint{2.660464in}{3.588322in}}%
\pgfpathlineto{\pgfqpoint{2.695550in}{3.538684in}}%
\pgfpathlineto{\pgfqpoint{2.685056in}{3.716509in}}%
\pgfpathlineto{\pgfqpoint{2.677213in}{3.717093in}}%
\pgfpathlineto{\pgfqpoint{2.644465in}{3.614535in}}%
\pgfpathlineto{\pgfqpoint{2.610226in}{3.610102in}}%
\pgfpathclose%
\pgfusepath{fill}%
\end{pgfscope}%
\begin{pgfscope}%
\pgfpathrectangle{\pgfqpoint{1.020000in}{0.880000in}}{\pgfqpoint{6.160000in}{6.160000in}}%
\pgfusepath{clip}%
\pgfsetbuttcap%
\pgfsetroundjoin%
\definecolor{currentfill}{rgb}{0.656683,0.771806,0.994914}%
\pgfsetfillcolor{currentfill}%
\pgfsetlinewidth{0.000000pt}%
\definecolor{currentstroke}{rgb}{0.000000,0.000000,0.000000}%
\pgfsetstrokecolor{currentstroke}%
\pgfsetdash{}{0pt}%
\pgfpathmoveto{\pgfqpoint{4.009612in}{3.475653in}}%
\pgfpathlineto{\pgfqpoint{4.018863in}{3.484010in}}%
\pgfpathlineto{\pgfqpoint{4.028259in}{3.412339in}}%
\pgfpathlineto{\pgfqpoint{4.062092in}{3.499674in}}%
\pgfpathlineto{\pgfqpoint{4.096085in}{3.449469in}}%
\pgfpathlineto{\pgfqpoint{4.086820in}{3.361362in}}%
\pgfpathlineto{\pgfqpoint{4.077225in}{3.620567in}}%
\pgfpathlineto{\pgfqpoint{4.043552in}{3.442058in}}%
\pgfpathlineto{\pgfqpoint{4.009612in}{3.475653in}}%
\pgfpathclose%
\pgfusepath{fill}%
\end{pgfscope}%
\begin{pgfscope}%
\pgfpathrectangle{\pgfqpoint{1.020000in}{0.880000in}}{\pgfqpoint{6.160000in}{6.160000in}}%
\pgfusepath{clip}%
\pgfsetbuttcap%
\pgfsetroundjoin%
\definecolor{currentfill}{rgb}{0.743754,0.825125,0.965798}%
\pgfsetfillcolor{currentfill}%
\pgfsetlinewidth{0.000000pt}%
\definecolor{currentstroke}{rgb}{0.000000,0.000000,0.000000}%
\pgfsetstrokecolor{currentstroke}%
\pgfsetdash{}{0pt}%
\pgfpathmoveto{\pgfqpoint{3.411836in}{3.577316in}}%
\pgfpathlineto{\pgfqpoint{3.419075in}{3.757948in}}%
\pgfpathlineto{\pgfqpoint{3.429179in}{3.564795in}}%
\pgfpathlineto{\pgfqpoint{3.463684in}{3.510552in}}%
\pgfpathlineto{\pgfqpoint{3.497017in}{3.619056in}}%
\pgfpathlineto{\pgfqpoint{3.487561in}{3.727207in}}%
\pgfpathlineto{\pgfqpoint{3.479205in}{3.678696in}}%
\pgfpathlineto{\pgfqpoint{3.445387in}{3.645150in}}%
\pgfpathlineto{\pgfqpoint{3.411836in}{3.577316in}}%
\pgfpathclose%
\pgfusepath{fill}%
\end{pgfscope}%
\begin{pgfscope}%
\pgfpathrectangle{\pgfqpoint{1.020000in}{0.880000in}}{\pgfqpoint{6.160000in}{6.160000in}}%
\pgfusepath{clip}%
\pgfsetbuttcap%
\pgfsetroundjoin%
\definecolor{currentfill}{rgb}{0.758539,0.832787,0.958408}%
\pgfsetfillcolor{currentfill}%
\pgfsetlinewidth{0.000000pt}%
\definecolor{currentstroke}{rgb}{0.000000,0.000000,0.000000}%
\pgfsetstrokecolor{currentstroke}%
\pgfsetdash{}{0pt}%
\pgfpathmoveto{\pgfqpoint{3.700908in}{3.692889in}}%
\pgfpathlineto{\pgfqpoint{3.710524in}{3.548571in}}%
\pgfpathlineto{\pgfqpoint{3.719315in}{3.581161in}}%
\pgfpathlineto{\pgfqpoint{3.752720in}{3.725455in}}%
\pgfpathlineto{\pgfqpoint{3.787489in}{3.544306in}}%
\pgfpathlineto{\pgfqpoint{3.777697in}{3.738019in}}%
\pgfpathlineto{\pgfqpoint{3.769097in}{3.639564in}}%
\pgfpathlineto{\pgfqpoint{3.734215in}{3.846544in}}%
\pgfpathlineto{\pgfqpoint{3.700908in}{3.692889in}}%
\pgfpathclose%
\pgfusepath{fill}%
\end{pgfscope}%
\begin{pgfscope}%
\pgfpathrectangle{\pgfqpoint{1.020000in}{0.880000in}}{\pgfqpoint{6.160000in}{6.160000in}}%
\pgfusepath{clip}%
\pgfsetbuttcap%
\pgfsetroundjoin%
\definecolor{currentfill}{rgb}{0.532568,0.669801,0.990393}%
\pgfsetfillcolor{currentfill}%
\pgfsetlinewidth{0.000000pt}%
\definecolor{currentstroke}{rgb}{0.000000,0.000000,0.000000}%
\pgfsetstrokecolor{currentstroke}%
\pgfsetdash{}{0pt}%
\pgfpathmoveto{\pgfqpoint{4.540496in}{3.338711in}}%
\pgfpathlineto{\pgfqpoint{4.550025in}{3.252540in}}%
\pgfpathlineto{\pgfqpoint{4.560325in}{3.369307in}}%
\pgfpathlineto{\pgfqpoint{4.593750in}{3.268278in}}%
\pgfpathlineto{\pgfqpoint{4.627578in}{3.277559in}}%
\pgfpathlineto{\pgfqpoint{4.616746in}{3.069916in}}%
\pgfpathlineto{\pgfqpoint{4.607146in}{3.137236in}}%
\pgfpathlineto{\pgfqpoint{4.574015in}{3.276166in}}%
\pgfpathlineto{\pgfqpoint{4.540496in}{3.338711in}}%
\pgfpathclose%
\pgfusepath{fill}%
\end{pgfscope}%
\begin{pgfscope}%
\pgfpathrectangle{\pgfqpoint{1.020000in}{0.880000in}}{\pgfqpoint{6.160000in}{6.160000in}}%
\pgfusepath{clip}%
\pgfsetbuttcap%
\pgfsetroundjoin%
\definecolor{currentfill}{rgb}{0.328604,0.439712,0.869587}%
\pgfsetfillcolor{currentfill}%
\pgfsetlinewidth{0.000000pt}%
\definecolor{currentstroke}{rgb}{0.000000,0.000000,0.000000}%
\pgfsetstrokecolor{currentstroke}%
\pgfsetdash{}{0pt}%
\pgfpathmoveto{\pgfqpoint{5.667829in}{2.779232in}}%
\pgfpathlineto{\pgfqpoint{5.678925in}{2.781993in}}%
\pgfpathlineto{\pgfqpoint{5.692450in}{2.943306in}}%
\pgfpathlineto{\pgfqpoint{5.725909in}{2.941497in}}%
\pgfpathlineto{\pgfqpoint{5.757857in}{2.845685in}}%
\pgfpathlineto{\pgfqpoint{5.745340in}{2.758918in}}%
\pgfpathlineto{\pgfqpoint{5.735416in}{2.835369in}}%
\pgfpathlineto{\pgfqpoint{5.704561in}{2.999583in}}%
\pgfpathlineto{\pgfqpoint{5.667829in}{2.779232in}}%
\pgfpathclose%
\pgfusepath{fill}%
\end{pgfscope}%
\begin{pgfscope}%
\pgfpathrectangle{\pgfqpoint{1.020000in}{0.880000in}}{\pgfqpoint{6.160000in}{6.160000in}}%
\pgfusepath{clip}%
\pgfsetbuttcap%
\pgfsetroundjoin%
\definecolor{currentfill}{rgb}{0.768034,0.837035,0.952488}%
\pgfsetfillcolor{currentfill}%
\pgfsetlinewidth{0.000000pt}%
\definecolor{currentstroke}{rgb}{0.000000,0.000000,0.000000}%
\pgfsetstrokecolor{currentstroke}%
\pgfsetdash{}{0pt}%
\pgfpathmoveto{\pgfqpoint{3.188418in}{3.770762in}}%
\pgfpathlineto{\pgfqpoint{3.197227in}{3.728164in}}%
\pgfpathlineto{\pgfqpoint{3.206834in}{3.604742in}}%
\pgfpathlineto{\pgfqpoint{3.240019in}{3.710893in}}%
\pgfpathlineto{\pgfqpoint{3.277259in}{3.375372in}}%
\pgfpathlineto{\pgfqpoint{3.265037in}{3.788413in}}%
\pgfpathlineto{\pgfqpoint{3.257007in}{3.741806in}}%
\pgfpathlineto{\pgfqpoint{3.222826in}{3.746008in}}%
\pgfpathlineto{\pgfqpoint{3.188418in}{3.770762in}}%
\pgfpathclose%
\pgfusepath{fill}%
\end{pgfscope}%
\begin{pgfscope}%
\pgfpathrectangle{\pgfqpoint{1.020000in}{0.880000in}}{\pgfqpoint{6.160000in}{6.160000in}}%
\pgfusepath{clip}%
\pgfsetbuttcap%
\pgfsetroundjoin%
\definecolor{currentfill}{rgb}{0.368507,0.491141,0.905243}%
\pgfsetfillcolor{currentfill}%
\pgfsetlinewidth{0.000000pt}%
\definecolor{currentstroke}{rgb}{0.000000,0.000000,0.000000}%
\pgfsetstrokecolor{currentstroke}%
\pgfsetdash{}{0pt}%
\pgfpathmoveto{\pgfqpoint{5.226033in}{3.024447in}}%
\pgfpathlineto{\pgfqpoint{5.234857in}{2.857321in}}%
\pgfpathlineto{\pgfqpoint{5.245297in}{2.842681in}}%
\pgfpathlineto{\pgfqpoint{5.278108in}{2.774367in}}%
\pgfpathlineto{\pgfqpoint{5.313116in}{2.905338in}}%
\pgfpathlineto{\pgfqpoint{5.303290in}{2.980184in}}%
\pgfpathlineto{\pgfqpoint{5.293583in}{3.065086in}}%
\pgfpathlineto{\pgfqpoint{5.259698in}{3.034294in}}%
\pgfpathlineto{\pgfqpoint{5.226033in}{3.024447in}}%
\pgfpathclose%
\pgfusepath{fill}%
\end{pgfscope}%
\begin{pgfscope}%
\pgfpathrectangle{\pgfqpoint{1.020000in}{0.880000in}}{\pgfqpoint{6.160000in}{6.160000in}}%
\pgfusepath{clip}%
\pgfsetbuttcap%
\pgfsetroundjoin%
\definecolor{currentfill}{rgb}{0.338377,0.452819,0.879317}%
\pgfsetfillcolor{currentfill}%
\pgfsetlinewidth{0.000000pt}%
\definecolor{currentstroke}{rgb}{0.000000,0.000000,0.000000}%
\pgfsetstrokecolor{currentstroke}%
\pgfsetdash{}{0pt}%
\pgfpathmoveto{\pgfqpoint{5.158637in}{3.002179in}}%
\pgfpathlineto{\pgfqpoint{5.167350in}{2.819491in}}%
\pgfpathlineto{\pgfqpoint{5.177627in}{2.794535in}}%
\pgfpathlineto{\pgfqpoint{5.211570in}{2.828865in}}%
\pgfpathlineto{\pgfqpoint{5.245297in}{2.842681in}}%
\pgfpathlineto{\pgfqpoint{5.234857in}{2.857321in}}%
\pgfpathlineto{\pgfqpoint{5.226033in}{3.024447in}}%
\pgfpathlineto{\pgfqpoint{5.191048in}{2.885682in}}%
\pgfpathlineto{\pgfqpoint{5.158637in}{3.002179in}}%
\pgfpathclose%
\pgfusepath{fill}%
\end{pgfscope}%
\begin{pgfscope}%
\pgfpathrectangle{\pgfqpoint{1.020000in}{0.880000in}}{\pgfqpoint{6.160000in}{6.160000in}}%
\pgfusepath{clip}%
\pgfsetbuttcap%
\pgfsetroundjoin%
\definecolor{currentfill}{rgb}{0.713852,0.808857,0.979386}%
\pgfsetfillcolor{currentfill}%
\pgfsetlinewidth{0.000000pt}%
\definecolor{currentstroke}{rgb}{0.000000,0.000000,0.000000}%
\pgfsetstrokecolor{currentstroke}%
\pgfsetdash{}{0pt}%
\pgfpathmoveto{\pgfqpoint{2.539920in}{3.709607in}}%
\pgfpathlineto{\pgfqpoint{2.547923in}{3.690743in}}%
\pgfpathlineto{\pgfqpoint{2.560322in}{3.401893in}}%
\pgfpathlineto{\pgfqpoint{2.592770in}{3.521316in}}%
\pgfpathlineto{\pgfqpoint{2.628592in}{3.427736in}}%
\pgfpathlineto{\pgfqpoint{2.616007in}{3.736559in}}%
\pgfpathlineto{\pgfqpoint{2.610226in}{3.610102in}}%
\pgfpathlineto{\pgfqpoint{2.576805in}{3.553641in}}%
\pgfpathlineto{\pgfqpoint{2.539920in}{3.709607in}}%
\pgfpathclose%
\pgfusepath{fill}%
\end{pgfscope}%
\begin{pgfscope}%
\pgfpathrectangle{\pgfqpoint{1.020000in}{0.880000in}}{\pgfqpoint{6.160000in}{6.160000in}}%
\pgfusepath{clip}%
\pgfsetbuttcap%
\pgfsetroundjoin%
\definecolor{currentfill}{rgb}{0.309060,0.413498,0.850128}%
\pgfsetfillcolor{currentfill}%
\pgfsetlinewidth{0.000000pt}%
\definecolor{currentstroke}{rgb}{0.000000,0.000000,0.000000}%
\pgfsetstrokecolor{currentstroke}%
\pgfsetdash{}{0pt}%
\pgfpathmoveto{\pgfqpoint{5.602283in}{2.866535in}}%
\pgfpathlineto{\pgfqpoint{5.612120in}{2.785747in}}%
\pgfpathlineto{\pgfqpoint{5.622572in}{2.747043in}}%
\pgfpathlineto{\pgfqpoint{5.658117in}{2.887552in}}%
\pgfpathlineto{\pgfqpoint{5.692450in}{2.943306in}}%
\pgfpathlineto{\pgfqpoint{5.678925in}{2.781993in}}%
\pgfpathlineto{\pgfqpoint{5.667829in}{2.779232in}}%
\pgfpathlineto{\pgfqpoint{5.633946in}{2.745177in}}%
\pgfpathlineto{\pgfqpoint{5.602283in}{2.866535in}}%
\pgfpathclose%
\pgfusepath{fill}%
\end{pgfscope}%
\begin{pgfscope}%
\pgfpathrectangle{\pgfqpoint{1.020000in}{0.880000in}}{\pgfqpoint{6.160000in}{6.160000in}}%
\pgfusepath{clip}%
\pgfsetbuttcap%
\pgfsetroundjoin%
\definecolor{currentfill}{rgb}{0.748682,0.827679,0.963334}%
\pgfsetfillcolor{currentfill}%
\pgfsetlinewidth{0.000000pt}%
\definecolor{currentstroke}{rgb}{0.000000,0.000000,0.000000}%
\pgfsetstrokecolor{currentstroke}%
\pgfsetdash{}{0pt}%
\pgfpathmoveto{\pgfqpoint{3.121809in}{3.601355in}}%
\pgfpathlineto{\pgfqpoint{3.129988in}{3.613625in}}%
\pgfpathlineto{\pgfqpoint{3.139028in}{3.545200in}}%
\pgfpathlineto{\pgfqpoint{3.172652in}{3.602695in}}%
\pgfpathlineto{\pgfqpoint{3.206834in}{3.604742in}}%
\pgfpathlineto{\pgfqpoint{3.197227in}{3.728164in}}%
\pgfpathlineto{\pgfqpoint{3.188418in}{3.770762in}}%
\pgfpathlineto{\pgfqpoint{3.155322in}{3.663898in}}%
\pgfpathlineto{\pgfqpoint{3.121809in}{3.601355in}}%
\pgfpathclose%
\pgfusepath{fill}%
\end{pgfscope}%
\begin{pgfscope}%
\pgfpathrectangle{\pgfqpoint{1.020000in}{0.880000in}}{\pgfqpoint{6.160000in}{6.160000in}}%
\pgfusepath{clip}%
\pgfsetbuttcap%
\pgfsetroundjoin%
\definecolor{currentfill}{rgb}{0.570616,0.704109,0.997195}%
\pgfsetfillcolor{currentfill}%
\pgfsetlinewidth{0.000000pt}%
\definecolor{currentstroke}{rgb}{0.000000,0.000000,0.000000}%
\pgfsetstrokecolor{currentstroke}%
\pgfsetdash{}{0pt}%
\pgfpathmoveto{\pgfqpoint{4.318104in}{3.222982in}}%
\pgfpathlineto{\pgfqpoint{4.328052in}{3.469472in}}%
\pgfpathlineto{\pgfqpoint{4.337367in}{3.261585in}}%
\pgfpathlineto{\pgfqpoint{4.371362in}{3.333060in}}%
\pgfpathlineto{\pgfqpoint{4.404996in}{3.229765in}}%
\pgfpathlineto{\pgfqpoint{4.395662in}{3.392672in}}%
\pgfpathlineto{\pgfqpoint{4.385532in}{3.168453in}}%
\pgfpathlineto{\pgfqpoint{4.352191in}{3.404817in}}%
\pgfpathlineto{\pgfqpoint{4.318104in}{3.222982in}}%
\pgfpathclose%
\pgfusepath{fill}%
\end{pgfscope}%
\begin{pgfscope}%
\pgfpathrectangle{\pgfqpoint{1.020000in}{0.880000in}}{\pgfqpoint{6.160000in}{6.160000in}}%
\pgfusepath{clip}%
\pgfsetbuttcap%
\pgfsetroundjoin%
\definecolor{currentfill}{rgb}{0.394042,0.522413,0.924916}%
\pgfsetfillcolor{currentfill}%
\pgfsetlinewidth{0.000000pt}%
\definecolor{currentstroke}{rgb}{0.000000,0.000000,0.000000}%
\pgfsetstrokecolor{currentstroke}%
\pgfsetdash{}{0pt}%
\pgfpathmoveto{\pgfqpoint{5.003927in}{3.071520in}}%
\pgfpathlineto{\pgfqpoint{5.013275in}{2.951238in}}%
\pgfpathlineto{\pgfqpoint{5.023823in}{2.973377in}}%
\pgfpathlineto{\pgfqpoint{5.056845in}{2.900277in}}%
\pgfpathlineto{\pgfqpoint{5.090978in}{2.956755in}}%
\pgfpathlineto{\pgfqpoint{5.080447in}{2.945348in}}%
\pgfpathlineto{\pgfqpoint{5.071195in}{3.074759in}}%
\pgfpathlineto{\pgfqpoint{5.037368in}{3.049533in}}%
\pgfpathlineto{\pgfqpoint{5.003927in}{3.071520in}}%
\pgfpathclose%
\pgfusepath{fill}%
\end{pgfscope}%
\begin{pgfscope}%
\pgfpathrectangle{\pgfqpoint{1.020000in}{0.880000in}}{\pgfqpoint{6.160000in}{6.160000in}}%
\pgfusepath{clip}%
\pgfsetbuttcap%
\pgfsetroundjoin%
\definecolor{currentfill}{rgb}{0.743754,0.825125,0.965798}%
\pgfsetfillcolor{currentfill}%
\pgfsetlinewidth{0.000000pt}%
\definecolor{currentstroke}{rgb}{0.000000,0.000000,0.000000}%
\pgfsetstrokecolor{currentstroke}%
\pgfsetdash{}{0pt}%
\pgfpathmoveto{\pgfqpoint{3.632774in}{3.707967in}}%
\pgfpathlineto{\pgfqpoint{3.641305in}{3.770484in}}%
\pgfpathlineto{\pgfqpoint{3.651211in}{3.579885in}}%
\pgfpathlineto{\pgfqpoint{3.684684in}{3.698861in}}%
\pgfpathlineto{\pgfqpoint{3.719315in}{3.581161in}}%
\pgfpathlineto{\pgfqpoint{3.710524in}{3.548571in}}%
\pgfpathlineto{\pgfqpoint{3.700908in}{3.692889in}}%
\pgfpathlineto{\pgfqpoint{3.667764in}{3.523634in}}%
\pgfpathlineto{\pgfqpoint{3.632774in}{3.707967in}}%
\pgfpathclose%
\pgfusepath{fill}%
\end{pgfscope}%
\begin{pgfscope}%
\pgfpathrectangle{\pgfqpoint{1.020000in}{0.880000in}}{\pgfqpoint{6.160000in}{6.160000in}}%
\pgfusepath{clip}%
\pgfsetbuttcap%
\pgfsetroundjoin%
\definecolor{currentfill}{rgb}{0.309060,0.413498,0.850128}%
\pgfsetfillcolor{currentfill}%
\pgfsetlinewidth{0.000000pt}%
\definecolor{currentstroke}{rgb}{0.000000,0.000000,0.000000}%
\pgfsetstrokecolor{currentstroke}%
\pgfsetdash{}{0pt}%
\pgfpathmoveto{\pgfqpoint{6.116161in}{2.951200in}}%
\pgfpathlineto{\pgfqpoint{6.123928in}{2.756782in}}%
\pgfpathlineto{\pgfqpoint{6.135709in}{2.767594in}}%
\pgfpathlineto{\pgfqpoint{6.167373in}{2.684984in}}%
\pgfpathlineto{\pgfqpoint{6.158206in}{2.807042in}}%
\pgfpathlineto{\pgfqpoint{6.148724in}{2.913270in}}%
\pgfpathlineto{\pgfqpoint{6.116161in}{2.951200in}}%
\pgfpathclose%
\pgfusepath{fill}%
\end{pgfscope}%
\begin{pgfscope}%
\pgfpathrectangle{\pgfqpoint{1.020000in}{0.880000in}}{\pgfqpoint{6.160000in}{6.160000in}}%
\pgfusepath{clip}%
\pgfsetbuttcap%
\pgfsetroundjoin%
\definecolor{currentfill}{rgb}{0.338377,0.452819,0.879317}%
\pgfsetfillcolor{currentfill}%
\pgfsetlinewidth{0.000000pt}%
\definecolor{currentstroke}{rgb}{0.000000,0.000000,0.000000}%
\pgfsetstrokecolor{currentstroke}%
\pgfsetdash{}{0pt}%
\pgfpathmoveto{\pgfqpoint{5.379172in}{2.821290in}}%
\pgfpathlineto{\pgfqpoint{5.391447in}{2.947011in}}%
\pgfpathlineto{\pgfqpoint{5.401643in}{2.898894in}}%
\pgfpathlineto{\pgfqpoint{5.433911in}{2.795612in}}%
\pgfpathlineto{\pgfqpoint{5.466838in}{2.750342in}}%
\pgfpathlineto{\pgfqpoint{5.458373in}{2.934894in}}%
\pgfpathlineto{\pgfqpoint{5.447366in}{2.920710in}}%
\pgfpathlineto{\pgfqpoint{5.413992in}{2.929958in}}%
\pgfpathlineto{\pgfqpoint{5.379172in}{2.821290in}}%
\pgfpathclose%
\pgfusepath{fill}%
\end{pgfscope}%
\begin{pgfscope}%
\pgfpathrectangle{\pgfqpoint{1.020000in}{0.880000in}}{\pgfqpoint{6.160000in}{6.160000in}}%
\pgfusepath{clip}%
\pgfsetbuttcap%
\pgfsetroundjoin%
\definecolor{currentfill}{rgb}{0.478462,0.616564,0.972721}%
\pgfsetfillcolor{currentfill}%
\pgfsetlinewidth{0.000000pt}%
\definecolor{currentstroke}{rgb}{0.000000,0.000000,0.000000}%
\pgfsetstrokecolor{currentstroke}%
\pgfsetdash{}{0pt}%
\pgfpathmoveto{\pgfqpoint{4.627578in}{3.277559in}}%
\pgfpathlineto{\pgfqpoint{4.636694in}{3.094178in}}%
\pgfpathlineto{\pgfqpoint{4.646793in}{3.125623in}}%
\pgfpathlineto{\pgfqpoint{4.680093in}{3.028380in}}%
\pgfpathlineto{\pgfqpoint{4.713187in}{2.908655in}}%
\pgfpathlineto{\pgfqpoint{4.705431in}{3.338255in}}%
\pgfpathlineto{\pgfqpoint{4.693975in}{3.058802in}}%
\pgfpathlineto{\pgfqpoint{4.661327in}{3.273350in}}%
\pgfpathlineto{\pgfqpoint{4.627578in}{3.277559in}}%
\pgfpathclose%
\pgfusepath{fill}%
\end{pgfscope}%
\begin{pgfscope}%
\pgfpathrectangle{\pgfqpoint{1.020000in}{0.880000in}}{\pgfqpoint{6.160000in}{6.160000in}}%
\pgfusepath{clip}%
\pgfsetbuttcap%
\pgfsetroundjoin%
\definecolor{currentfill}{rgb}{0.261805,0.346484,0.795658}%
\pgfsetfillcolor{currentfill}%
\pgfsetlinewidth{0.000000pt}%
\definecolor{currentstroke}{rgb}{0.000000,0.000000,0.000000}%
\pgfsetstrokecolor{currentstroke}%
\pgfsetdash{}{0pt}%
\pgfpathmoveto{\pgfqpoint{5.977890in}{2.684593in}}%
\pgfpathlineto{\pgfqpoint{5.988346in}{2.632461in}}%
\pgfpathlineto{\pgfqpoint{6.000243in}{2.658379in}}%
\pgfpathlineto{\pgfqpoint{6.035631in}{2.767115in}}%
\pgfpathlineto{\pgfqpoint{6.067262in}{2.675183in}}%
\pgfpathlineto{\pgfqpoint{6.057119in}{2.746757in}}%
\pgfpathlineto{\pgfqpoint{6.048028in}{2.873985in}}%
\pgfpathlineto{\pgfqpoint{6.011223in}{2.686192in}}%
\pgfpathlineto{\pgfqpoint{5.977890in}{2.684593in}}%
\pgfpathclose%
\pgfusepath{fill}%
\end{pgfscope}%
\begin{pgfscope}%
\pgfpathrectangle{\pgfqpoint{1.020000in}{0.880000in}}{\pgfqpoint{6.160000in}{6.160000in}}%
\pgfusepath{clip}%
\pgfsetbuttcap%
\pgfsetroundjoin%
\definecolor{currentfill}{rgb}{0.708720,0.805721,0.981117}%
\pgfsetfillcolor{currentfill}%
\pgfsetlinewidth{0.000000pt}%
\definecolor{currentstroke}{rgb}{0.000000,0.000000,0.000000}%
\pgfsetstrokecolor{currentstroke}%
\pgfsetdash{}{0pt}%
\pgfpathmoveto{\pgfqpoint{3.787489in}{3.544306in}}%
\pgfpathlineto{\pgfqpoint{3.796207in}{3.624638in}}%
\pgfpathlineto{\pgfqpoint{3.805133in}{3.658979in}}%
\pgfpathlineto{\pgfqpoint{3.839004in}{3.719148in}}%
\pgfpathlineto{\pgfqpoint{3.873458in}{3.597477in}}%
\pgfpathlineto{\pgfqpoint{3.864843in}{3.438782in}}%
\pgfpathlineto{\pgfqpoint{3.855477in}{3.524266in}}%
\pgfpathlineto{\pgfqpoint{3.821887in}{3.426160in}}%
\pgfpathlineto{\pgfqpoint{3.787489in}{3.544306in}}%
\pgfpathclose%
\pgfusepath{fill}%
\end{pgfscope}%
\begin{pgfscope}%
\pgfpathrectangle{\pgfqpoint{1.020000in}{0.880000in}}{\pgfqpoint{6.160000in}{6.160000in}}%
\pgfusepath{clip}%
\pgfsetbuttcap%
\pgfsetroundjoin%
\definecolor{currentfill}{rgb}{0.388852,0.516298,0.921373}%
\pgfsetfillcolor{currentfill}%
\pgfsetlinewidth{0.000000pt}%
\definecolor{currentstroke}{rgb}{0.000000,0.000000,0.000000}%
\pgfsetstrokecolor{currentstroke}%
\pgfsetdash{}{0pt}%
\pgfpathmoveto{\pgfqpoint{4.937224in}{3.153221in}}%
\pgfpathlineto{\pgfqpoint{4.946873in}{3.069206in}}%
\pgfpathlineto{\pgfqpoint{4.954713in}{2.751661in}}%
\pgfpathlineto{\pgfqpoint{4.989362in}{2.877495in}}%
\pgfpathlineto{\pgfqpoint{5.023823in}{2.973377in}}%
\pgfpathlineto{\pgfqpoint{5.013275in}{2.951238in}}%
\pgfpathlineto{\pgfqpoint{5.003927in}{3.071520in}}%
\pgfpathlineto{\pgfqpoint{4.969111in}{2.923844in}}%
\pgfpathlineto{\pgfqpoint{4.937224in}{3.153221in}}%
\pgfpathclose%
\pgfusepath{fill}%
\end{pgfscope}%
\begin{pgfscope}%
\pgfpathrectangle{\pgfqpoint{1.020000in}{0.880000in}}{\pgfqpoint{6.160000in}{6.160000in}}%
\pgfusepath{clip}%
\pgfsetbuttcap%
\pgfsetroundjoin%
\definecolor{currentfill}{rgb}{0.786721,0.844807,0.939810}%
\pgfsetfillcolor{currentfill}%
\pgfsetlinewidth{0.000000pt}%
\definecolor{currentstroke}{rgb}{0.000000,0.000000,0.000000}%
\pgfsetstrokecolor{currentstroke}%
\pgfsetdash{}{0pt}%
\pgfpathmoveto{\pgfqpoint{2.898369in}{3.752304in}}%
\pgfpathlineto{\pgfqpoint{2.906488in}{3.749114in}}%
\pgfpathlineto{\pgfqpoint{2.915431in}{3.682121in}}%
\pgfpathlineto{\pgfqpoint{2.951175in}{3.565168in}}%
\pgfpathlineto{\pgfqpoint{2.983586in}{3.718495in}}%
\pgfpathlineto{\pgfqpoint{2.976751in}{3.608286in}}%
\pgfpathlineto{\pgfqpoint{2.967681in}{3.685354in}}%
\pgfpathlineto{\pgfqpoint{2.929466in}{4.006080in}}%
\pgfpathlineto{\pgfqpoint{2.898369in}{3.752304in}}%
\pgfpathclose%
\pgfusepath{fill}%
\end{pgfscope}%
\begin{pgfscope}%
\pgfpathrectangle{\pgfqpoint{1.020000in}{0.880000in}}{\pgfqpoint{6.160000in}{6.160000in}}%
\pgfusepath{clip}%
\pgfsetbuttcap%
\pgfsetroundjoin%
\definecolor{currentfill}{rgb}{0.743754,0.825125,0.965798}%
\pgfsetfillcolor{currentfill}%
\pgfsetlinewidth{0.000000pt}%
\definecolor{currentstroke}{rgb}{0.000000,0.000000,0.000000}%
\pgfsetstrokecolor{currentstroke}%
\pgfsetdash{}{0pt}%
\pgfpathmoveto{\pgfqpoint{3.052957in}{3.644143in}}%
\pgfpathlineto{\pgfqpoint{3.061612in}{3.607262in}}%
\pgfpathlineto{\pgfqpoint{3.070677in}{3.534608in}}%
\pgfpathlineto{\pgfqpoint{3.102444in}{3.764239in}}%
\pgfpathlineto{\pgfqpoint{3.139028in}{3.545200in}}%
\pgfpathlineto{\pgfqpoint{3.129988in}{3.613625in}}%
\pgfpathlineto{\pgfqpoint{3.121809in}{3.601355in}}%
\pgfpathlineto{\pgfqpoint{3.085903in}{3.759625in}}%
\pgfpathlineto{\pgfqpoint{3.052957in}{3.644143in}}%
\pgfpathclose%
\pgfusepath{fill}%
\end{pgfscope}%
\begin{pgfscope}%
\pgfpathrectangle{\pgfqpoint{1.020000in}{0.880000in}}{\pgfqpoint{6.160000in}{6.160000in}}%
\pgfusepath{clip}%
\pgfsetbuttcap%
\pgfsetroundjoin%
\definecolor{currentfill}{rgb}{0.323718,0.433158,0.864722}%
\pgfsetfillcolor{currentfill}%
\pgfsetlinewidth{0.000000pt}%
\definecolor{currentstroke}{rgb}{0.000000,0.000000,0.000000}%
\pgfsetstrokecolor{currentstroke}%
\pgfsetdash{}{0pt}%
\pgfpathmoveto{\pgfqpoint{5.823447in}{2.772078in}}%
\pgfpathlineto{\pgfqpoint{5.834330in}{2.750954in}}%
\pgfpathlineto{\pgfqpoint{5.847130in}{2.843265in}}%
\pgfpathlineto{\pgfqpoint{5.881435in}{2.894049in}}%
\pgfpathlineto{\pgfqpoint{5.915144in}{2.909835in}}%
\pgfpathlineto{\pgfqpoint{5.901475in}{2.775132in}}%
\pgfpathlineto{\pgfqpoint{5.890782in}{2.810435in}}%
\pgfpathlineto{\pgfqpoint{5.860031in}{2.964130in}}%
\pgfpathlineto{\pgfqpoint{5.823447in}{2.772078in}}%
\pgfpathclose%
\pgfusepath{fill}%
\end{pgfscope}%
\begin{pgfscope}%
\pgfpathrectangle{\pgfqpoint{1.020000in}{0.880000in}}{\pgfqpoint{6.160000in}{6.160000in}}%
\pgfusepath{clip}%
\pgfsetbuttcap%
\pgfsetroundjoin%
\definecolor{currentfill}{rgb}{0.309060,0.413498,0.850128}%
\pgfsetfillcolor{currentfill}%
\pgfsetlinewidth{0.000000pt}%
\definecolor{currentstroke}{rgb}{0.000000,0.000000,0.000000}%
\pgfsetstrokecolor{currentstroke}%
\pgfsetdash{}{0pt}%
\pgfpathmoveto{\pgfqpoint{5.533919in}{2.761162in}}%
\pgfpathlineto{\pgfqpoint{5.545901in}{2.838544in}}%
\pgfpathlineto{\pgfqpoint{5.556381in}{2.804329in}}%
\pgfpathlineto{\pgfqpoint{5.592909in}{3.016779in}}%
\pgfpathlineto{\pgfqpoint{5.622572in}{2.747043in}}%
\pgfpathlineto{\pgfqpoint{5.612120in}{2.785747in}}%
\pgfpathlineto{\pgfqpoint{5.602283in}{2.866535in}}%
\pgfpathlineto{\pgfqpoint{5.566598in}{2.706498in}}%
\pgfpathlineto{\pgfqpoint{5.533919in}{2.761162in}}%
\pgfpathclose%
\pgfusepath{fill}%
\end{pgfscope}%
\begin{pgfscope}%
\pgfpathrectangle{\pgfqpoint{1.020000in}{0.880000in}}{\pgfqpoint{6.160000in}{6.160000in}}%
\pgfusepath{clip}%
\pgfsetbuttcap%
\pgfsetroundjoin%
\definecolor{currentfill}{rgb}{0.763363,0.835092,0.955658}%
\pgfsetfillcolor{currentfill}%
\pgfsetlinewidth{0.000000pt}%
\definecolor{currentstroke}{rgb}{0.000000,0.000000,0.000000}%
\pgfsetstrokecolor{currentstroke}%
\pgfsetdash{}{0pt}%
\pgfpathmoveto{\pgfqpoint{2.831511in}{3.624554in}}%
\pgfpathlineto{\pgfqpoint{2.838772in}{3.679598in}}%
\pgfpathlineto{\pgfqpoint{2.847342in}{3.638368in}}%
\pgfpathlineto{\pgfqpoint{2.884196in}{3.444178in}}%
\pgfpathlineto{\pgfqpoint{2.915431in}{3.682121in}}%
\pgfpathlineto{\pgfqpoint{2.906488in}{3.749114in}}%
\pgfpathlineto{\pgfqpoint{2.898369in}{3.752304in}}%
\pgfpathlineto{\pgfqpoint{2.863510in}{3.796270in}}%
\pgfpathlineto{\pgfqpoint{2.831511in}{3.624554in}}%
\pgfpathclose%
\pgfusepath{fill}%
\end{pgfscope}%
\begin{pgfscope}%
\pgfpathrectangle{\pgfqpoint{1.020000in}{0.880000in}}{\pgfqpoint{6.160000in}{6.160000in}}%
\pgfusepath{clip}%
\pgfsetbuttcap%
\pgfsetroundjoin%
\definecolor{currentfill}{rgb}{0.758539,0.832787,0.958408}%
\pgfsetfillcolor{currentfill}%
\pgfsetlinewidth{0.000000pt}%
\definecolor{currentstroke}{rgb}{0.000000,0.000000,0.000000}%
\pgfsetstrokecolor{currentstroke}%
\pgfsetdash{}{0pt}%
\pgfpathmoveto{\pgfqpoint{3.343268in}{3.625635in}}%
\pgfpathlineto{\pgfqpoint{3.351360in}{3.681009in}}%
\pgfpathlineto{\pgfqpoint{3.359716in}{3.707575in}}%
\pgfpathlineto{\pgfqpoint{3.393686in}{3.737298in}}%
\pgfpathlineto{\pgfqpoint{3.429179in}{3.564795in}}%
\pgfpathlineto{\pgfqpoint{3.419075in}{3.757948in}}%
\pgfpathlineto{\pgfqpoint{3.411836in}{3.577316in}}%
\pgfpathlineto{\pgfqpoint{3.377299in}{3.635166in}}%
\pgfpathlineto{\pgfqpoint{3.343268in}{3.625635in}}%
\pgfpathclose%
\pgfusepath{fill}%
\end{pgfscope}%
\begin{pgfscope}%
\pgfpathrectangle{\pgfqpoint{1.020000in}{0.880000in}}{\pgfqpoint{6.160000in}{6.160000in}}%
\pgfusepath{clip}%
\pgfsetbuttcap%
\pgfsetroundjoin%
\definecolor{currentfill}{rgb}{0.309060,0.413498,0.850128}%
\pgfsetfillcolor{currentfill}%
\pgfsetlinewidth{0.000000pt}%
\definecolor{currentstroke}{rgb}{0.000000,0.000000,0.000000}%
\pgfsetstrokecolor{currentstroke}%
\pgfsetdash{}{0pt}%
\pgfpathmoveto{\pgfqpoint{6.048028in}{2.873985in}}%
\pgfpathlineto{\pgfqpoint{6.057119in}{2.746757in}}%
\pgfpathlineto{\pgfqpoint{6.067262in}{2.675183in}}%
\pgfpathlineto{\pgfqpoint{6.104618in}{2.883809in}}%
\pgfpathlineto{\pgfqpoint{6.135709in}{2.767594in}}%
\pgfpathlineto{\pgfqpoint{6.123928in}{2.756782in}}%
\pgfpathlineto{\pgfqpoint{6.116161in}{2.951200in}}%
\pgfpathlineto{\pgfqpoint{6.080115in}{2.809051in}}%
\pgfpathlineto{\pgfqpoint{6.048028in}{2.873985in}}%
\pgfpathclose%
\pgfusepath{fill}%
\end{pgfscope}%
\begin{pgfscope}%
\pgfpathrectangle{\pgfqpoint{1.020000in}{0.880000in}}{\pgfqpoint{6.160000in}{6.160000in}}%
\pgfusepath{clip}%
\pgfsetbuttcap%
\pgfsetroundjoin%
\definecolor{currentfill}{rgb}{0.294718,0.393542,0.834384}%
\pgfsetfillcolor{currentfill}%
\pgfsetlinewidth{0.000000pt}%
\definecolor{currentstroke}{rgb}{0.000000,0.000000,0.000000}%
\pgfsetstrokecolor{currentstroke}%
\pgfsetdash{}{0pt}%
\pgfpathmoveto{\pgfqpoint{5.757857in}{2.845685in}}%
\pgfpathlineto{\pgfqpoint{5.770269in}{2.923179in}}%
\pgfpathlineto{\pgfqpoint{5.779363in}{2.791484in}}%
\pgfpathlineto{\pgfqpoint{5.809718in}{2.602150in}}%
\pgfpathlineto{\pgfqpoint{5.847130in}{2.843265in}}%
\pgfpathlineto{\pgfqpoint{5.834330in}{2.750954in}}%
\pgfpathlineto{\pgfqpoint{5.823447in}{2.772078in}}%
\pgfpathlineto{\pgfqpoint{5.789654in}{2.745759in}}%
\pgfpathlineto{\pgfqpoint{5.757857in}{2.845685in}}%
\pgfpathclose%
\pgfusepath{fill}%
\end{pgfscope}%
\begin{pgfscope}%
\pgfpathrectangle{\pgfqpoint{1.020000in}{0.880000in}}{\pgfqpoint{6.160000in}{6.160000in}}%
\pgfusepath{clip}%
\pgfsetbuttcap%
\pgfsetroundjoin%
\definecolor{currentfill}{rgb}{0.446431,0.582356,0.957373}%
\pgfsetfillcolor{currentfill}%
\pgfsetlinewidth{0.000000pt}%
\definecolor{currentstroke}{rgb}{0.000000,0.000000,0.000000}%
\pgfsetstrokecolor{currentstroke}%
\pgfsetdash{}{0pt}%
\pgfpathmoveto{\pgfqpoint{4.781926in}{3.144920in}}%
\pgfpathlineto{\pgfqpoint{4.790659in}{2.918745in}}%
\pgfpathlineto{\pgfqpoint{4.801920in}{3.103639in}}%
\pgfpathlineto{\pgfqpoint{4.836456in}{3.221653in}}%
\pgfpathlineto{\pgfqpoint{4.869392in}{3.098791in}}%
\pgfpathlineto{\pgfqpoint{4.857665in}{2.873416in}}%
\pgfpathlineto{\pgfqpoint{4.848519in}{3.024105in}}%
\pgfpathlineto{\pgfqpoint{4.816435in}{3.270919in}}%
\pgfpathlineto{\pgfqpoint{4.781926in}{3.144920in}}%
\pgfpathclose%
\pgfusepath{fill}%
\end{pgfscope}%
\begin{pgfscope}%
\pgfpathrectangle{\pgfqpoint{1.020000in}{0.880000in}}{\pgfqpoint{6.160000in}{6.160000in}}%
\pgfusepath{clip}%
\pgfsetbuttcap%
\pgfsetroundjoin%
\definecolor{currentfill}{rgb}{0.724041,0.814910,0.975651}%
\pgfsetfillcolor{currentfill}%
\pgfsetlinewidth{0.000000pt}%
\definecolor{currentstroke}{rgb}{0.000000,0.000000,0.000000}%
\pgfsetstrokecolor{currentstroke}%
\pgfsetdash{}{0pt}%
\pgfpathmoveto{\pgfqpoint{2.983586in}{3.718495in}}%
\pgfpathlineto{\pgfqpoint{2.993324in}{3.587337in}}%
\pgfpathlineto{\pgfqpoint{3.003198in}{3.444271in}}%
\pgfpathlineto{\pgfqpoint{3.034801in}{3.675450in}}%
\pgfpathlineto{\pgfqpoint{3.070677in}{3.534608in}}%
\pgfpathlineto{\pgfqpoint{3.061612in}{3.607262in}}%
\pgfpathlineto{\pgfqpoint{3.052957in}{3.644143in}}%
\pgfpathlineto{\pgfqpoint{3.020210in}{3.516418in}}%
\pgfpathlineto{\pgfqpoint{2.983586in}{3.718495in}}%
\pgfpathclose%
\pgfusepath{fill}%
\end{pgfscope}%
\begin{pgfscope}%
\pgfpathrectangle{\pgfqpoint{1.020000in}{0.880000in}}{\pgfqpoint{6.160000in}{6.160000in}}%
\pgfusepath{clip}%
\pgfsetbuttcap%
\pgfsetroundjoin%
\definecolor{currentfill}{rgb}{0.358415,0.478426,0.896795}%
\pgfsetfillcolor{currentfill}%
\pgfsetlinewidth{0.000000pt}%
\definecolor{currentstroke}{rgb}{0.000000,0.000000,0.000000}%
\pgfsetstrokecolor{currentstroke}%
\pgfsetdash{}{0pt}%
\pgfpathmoveto{\pgfqpoint{5.090978in}{2.956755in}}%
\pgfpathlineto{\pgfqpoint{5.100330in}{2.837602in}}%
\pgfpathlineto{\pgfqpoint{5.111816in}{2.947483in}}%
\pgfpathlineto{\pgfqpoint{5.145467in}{2.944424in}}%
\pgfpathlineto{\pgfqpoint{5.177627in}{2.794535in}}%
\pgfpathlineto{\pgfqpoint{5.167350in}{2.819491in}}%
\pgfpathlineto{\pgfqpoint{5.158637in}{3.002179in}}%
\pgfpathlineto{\pgfqpoint{5.124466in}{2.943088in}}%
\pgfpathlineto{\pgfqpoint{5.090978in}{2.956755in}}%
\pgfpathclose%
\pgfusepath{fill}%
\end{pgfscope}%
\begin{pgfscope}%
\pgfpathrectangle{\pgfqpoint{1.020000in}{0.880000in}}{\pgfqpoint{6.160000in}{6.160000in}}%
\pgfusepath{clip}%
\pgfsetbuttcap%
\pgfsetroundjoin%
\definecolor{currentfill}{rgb}{0.667253,0.779176,0.992959}%
\pgfsetfillcolor{currentfill}%
\pgfsetlinewidth{0.000000pt}%
\definecolor{currentstroke}{rgb}{0.000000,0.000000,0.000000}%
\pgfsetstrokecolor{currentstroke}%
\pgfsetdash{}{0pt}%
\pgfpathmoveto{\pgfqpoint{4.163879in}{3.486854in}}%
\pgfpathlineto{\pgfqpoint{4.173299in}{3.696877in}}%
\pgfpathlineto{\pgfqpoint{4.182781in}{3.552635in}}%
\pgfpathlineto{\pgfqpoint{4.216745in}{3.525640in}}%
\pgfpathlineto{\pgfqpoint{4.250663in}{3.489050in}}%
\pgfpathlineto{\pgfqpoint{4.241033in}{3.314081in}}%
\pgfpathlineto{\pgfqpoint{4.231568in}{3.370614in}}%
\pgfpathlineto{\pgfqpoint{4.197751in}{3.407248in}}%
\pgfpathlineto{\pgfqpoint{4.163879in}{3.486854in}}%
\pgfpathclose%
\pgfusepath{fill}%
\end{pgfscope}%
\begin{pgfscope}%
\pgfpathrectangle{\pgfqpoint{1.020000in}{0.880000in}}{\pgfqpoint{6.160000in}{6.160000in}}%
\pgfusepath{clip}%
\pgfsetbuttcap%
\pgfsetroundjoin%
\definecolor{currentfill}{rgb}{0.688188,0.793178,0.988038}%
\pgfsetfillcolor{currentfill}%
\pgfsetlinewidth{0.000000pt}%
\definecolor{currentstroke}{rgb}{0.000000,0.000000,0.000000}%
\pgfsetstrokecolor{currentstroke}%
\pgfsetdash{}{0pt}%
\pgfpathmoveto{\pgfqpoint{3.941571in}{3.555739in}}%
\pgfpathlineto{\pgfqpoint{3.950880in}{3.510457in}}%
\pgfpathlineto{\pgfqpoint{3.959775in}{3.656147in}}%
\pgfpathlineto{\pgfqpoint{3.994097in}{3.519333in}}%
\pgfpathlineto{\pgfqpoint{4.028259in}{3.412339in}}%
\pgfpathlineto{\pgfqpoint{4.018863in}{3.484010in}}%
\pgfpathlineto{\pgfqpoint{4.009612in}{3.475653in}}%
\pgfpathlineto{\pgfqpoint{3.975606in}{3.519942in}}%
\pgfpathlineto{\pgfqpoint{3.941571in}{3.555739in}}%
\pgfpathclose%
\pgfusepath{fill}%
\end{pgfscope}%
\begin{pgfscope}%
\pgfpathrectangle{\pgfqpoint{1.020000in}{0.880000in}}{\pgfqpoint{6.160000in}{6.160000in}}%
\pgfusepath{clip}%
\pgfsetbuttcap%
\pgfsetroundjoin%
\definecolor{currentfill}{rgb}{0.294718,0.393542,0.834384}%
\pgfsetfillcolor{currentfill}%
\pgfsetlinewidth{0.000000pt}%
\definecolor{currentstroke}{rgb}{0.000000,0.000000,0.000000}%
\pgfsetstrokecolor{currentstroke}%
\pgfsetdash{}{0pt}%
\pgfpathmoveto{\pgfqpoint{5.466838in}{2.750342in}}%
\pgfpathlineto{\pgfqpoint{5.477772in}{2.756467in}}%
\pgfpathlineto{\pgfqpoint{5.487731in}{2.686589in}}%
\pgfpathlineto{\pgfqpoint{5.522188in}{2.755922in}}%
\pgfpathlineto{\pgfqpoint{5.556381in}{2.804329in}}%
\pgfpathlineto{\pgfqpoint{5.545901in}{2.838544in}}%
\pgfpathlineto{\pgfqpoint{5.533919in}{2.761162in}}%
\pgfpathlineto{\pgfqpoint{5.501973in}{2.875320in}}%
\pgfpathlineto{\pgfqpoint{5.466838in}{2.750342in}}%
\pgfpathclose%
\pgfusepath{fill}%
\end{pgfscope}%
\begin{pgfscope}%
\pgfpathrectangle{\pgfqpoint{1.020000in}{0.880000in}}{\pgfqpoint{6.160000in}{6.160000in}}%
\pgfusepath{clip}%
\pgfsetbuttcap%
\pgfsetroundjoin%
\definecolor{currentfill}{rgb}{0.748682,0.827679,0.963334}%
\pgfsetfillcolor{currentfill}%
\pgfsetlinewidth{0.000000pt}%
\definecolor{currentstroke}{rgb}{0.000000,0.000000,0.000000}%
\pgfsetstrokecolor{currentstroke}%
\pgfsetdash{}{0pt}%
\pgfpathmoveto{\pgfqpoint{2.762154in}{3.679572in}}%
\pgfpathlineto{\pgfqpoint{2.772974in}{3.474815in}}%
\pgfpathlineto{\pgfqpoint{2.778847in}{3.621990in}}%
\pgfpathlineto{\pgfqpoint{2.813466in}{3.603558in}}%
\pgfpathlineto{\pgfqpoint{2.847342in}{3.638368in}}%
\pgfpathlineto{\pgfqpoint{2.838772in}{3.679598in}}%
\pgfpathlineto{\pgfqpoint{2.831511in}{3.624554in}}%
\pgfpathlineto{\pgfqpoint{2.795450in}{3.753400in}}%
\pgfpathlineto{\pgfqpoint{2.762154in}{3.679572in}}%
\pgfpathclose%
\pgfusepath{fill}%
\end{pgfscope}%
\begin{pgfscope}%
\pgfpathrectangle{\pgfqpoint{1.020000in}{0.880000in}}{\pgfqpoint{6.160000in}{6.160000in}}%
\pgfusepath{clip}%
\pgfsetbuttcap%
\pgfsetroundjoin%
\definecolor{currentfill}{rgb}{0.570616,0.704109,0.997195}%
\pgfsetfillcolor{currentfill}%
\pgfsetlinewidth{0.000000pt}%
\definecolor{currentstroke}{rgb}{0.000000,0.000000,0.000000}%
\pgfsetstrokecolor{currentstroke}%
\pgfsetdash{}{0pt}%
\pgfpathmoveto{\pgfqpoint{4.473218in}{3.445774in}}%
\pgfpathlineto{\pgfqpoint{4.482424in}{3.251038in}}%
\pgfpathlineto{\pgfqpoint{4.492270in}{3.268841in}}%
\pgfpathlineto{\pgfqpoint{4.525856in}{3.192463in}}%
\pgfpathlineto{\pgfqpoint{4.560325in}{3.369307in}}%
\pgfpathlineto{\pgfqpoint{4.550025in}{3.252540in}}%
\pgfpathlineto{\pgfqpoint{4.540496in}{3.338711in}}%
\pgfpathlineto{\pgfqpoint{4.506536in}{3.284544in}}%
\pgfpathlineto{\pgfqpoint{4.473218in}{3.445774in}}%
\pgfpathclose%
\pgfusepath{fill}%
\end{pgfscope}%
\begin{pgfscope}%
\pgfpathrectangle{\pgfqpoint{1.020000in}{0.880000in}}{\pgfqpoint{6.160000in}{6.160000in}}%
\pgfusepath{clip}%
\pgfsetbuttcap%
\pgfsetroundjoin%
\definecolor{currentfill}{rgb}{0.743754,0.825125,0.965798}%
\pgfsetfillcolor{currentfill}%
\pgfsetlinewidth{0.000000pt}%
\definecolor{currentstroke}{rgb}{0.000000,0.000000,0.000000}%
\pgfsetstrokecolor{currentstroke}%
\pgfsetdash{}{0pt}%
\pgfpathmoveto{\pgfqpoint{2.473033in}{3.595955in}}%
\pgfpathlineto{\pgfqpoint{2.477626in}{3.773865in}}%
\pgfpathlineto{\pgfqpoint{2.487584in}{3.637309in}}%
\pgfpathlineto{\pgfqpoint{2.524327in}{3.500075in}}%
\pgfpathlineto{\pgfqpoint{2.560322in}{3.401893in}}%
\pgfpathlineto{\pgfqpoint{2.547923in}{3.690743in}}%
\pgfpathlineto{\pgfqpoint{2.539920in}{3.709607in}}%
\pgfpathlineto{\pgfqpoint{2.506162in}{3.671032in}}%
\pgfpathlineto{\pgfqpoint{2.473033in}{3.595955in}}%
\pgfpathclose%
\pgfusepath{fill}%
\end{pgfscope}%
\begin{pgfscope}%
\pgfpathrectangle{\pgfqpoint{1.020000in}{0.880000in}}{\pgfqpoint{6.160000in}{6.160000in}}%
\pgfusepath{clip}%
\pgfsetbuttcap%
\pgfsetroundjoin%
\definecolor{currentfill}{rgb}{0.728970,0.817464,0.973188}%
\pgfsetfillcolor{currentfill}%
\pgfsetlinewidth{0.000000pt}%
\definecolor{currentstroke}{rgb}{0.000000,0.000000,0.000000}%
\pgfsetstrokecolor{currentstroke}%
\pgfsetdash{}{0pt}%
\pgfpathmoveto{\pgfqpoint{2.695550in}{3.538684in}}%
\pgfpathlineto{\pgfqpoint{2.702601in}{3.593012in}}%
\pgfpathlineto{\pgfqpoint{2.710993in}{3.557883in}}%
\pgfpathlineto{\pgfqpoint{2.745408in}{3.555892in}}%
\pgfpathlineto{\pgfqpoint{2.778847in}{3.621990in}}%
\pgfpathlineto{\pgfqpoint{2.772974in}{3.474815in}}%
\pgfpathlineto{\pgfqpoint{2.762154in}{3.679572in}}%
\pgfpathlineto{\pgfqpoint{2.727390in}{3.708889in}}%
\pgfpathlineto{\pgfqpoint{2.695550in}{3.538684in}}%
\pgfpathclose%
\pgfusepath{fill}%
\end{pgfscope}%
\begin{pgfscope}%
\pgfpathrectangle{\pgfqpoint{1.020000in}{0.880000in}}{\pgfqpoint{6.160000in}{6.160000in}}%
\pgfusepath{clip}%
\pgfsetbuttcap%
\pgfsetroundjoin%
\definecolor{currentfill}{rgb}{0.753611,0.830233,0.960871}%
\pgfsetfillcolor{currentfill}%
\pgfsetlinewidth{0.000000pt}%
\definecolor{currentstroke}{rgb}{0.000000,0.000000,0.000000}%
\pgfsetstrokecolor{currentstroke}%
\pgfsetdash{}{0pt}%
\pgfpathmoveto{\pgfqpoint{3.277259in}{3.375372in}}%
\pgfpathlineto{\pgfqpoint{3.282235in}{3.763462in}}%
\pgfpathlineto{\pgfqpoint{3.291961in}{3.627139in}}%
\pgfpathlineto{\pgfqpoint{3.324960in}{3.769602in}}%
\pgfpathlineto{\pgfqpoint{3.359716in}{3.707575in}}%
\pgfpathlineto{\pgfqpoint{3.351360in}{3.681009in}}%
\pgfpathlineto{\pgfqpoint{3.343268in}{3.625635in}}%
\pgfpathlineto{\pgfqpoint{3.309278in}{3.609807in}}%
\pgfpathlineto{\pgfqpoint{3.277259in}{3.375372in}}%
\pgfpathclose%
\pgfusepath{fill}%
\end{pgfscope}%
\begin{pgfscope}%
\pgfpathrectangle{\pgfqpoint{1.020000in}{0.880000in}}{\pgfqpoint{6.160000in}{6.160000in}}%
\pgfusepath{clip}%
\pgfsetbuttcap%
\pgfsetroundjoin%
\definecolor{currentfill}{rgb}{0.358415,0.478426,0.896795}%
\pgfsetfillcolor{currentfill}%
\pgfsetlinewidth{0.000000pt}%
\definecolor{currentstroke}{rgb}{0.000000,0.000000,0.000000}%
\pgfsetstrokecolor{currentstroke}%
\pgfsetdash{}{0pt}%
\pgfpathmoveto{\pgfqpoint{5.313116in}{2.905338in}}%
\pgfpathlineto{\pgfqpoint{5.322493in}{2.790815in}}%
\pgfpathlineto{\pgfqpoint{5.334947in}{2.942166in}}%
\pgfpathlineto{\pgfqpoint{5.368779in}{2.959711in}}%
\pgfpathlineto{\pgfqpoint{5.401643in}{2.898894in}}%
\pgfpathlineto{\pgfqpoint{5.391447in}{2.947011in}}%
\pgfpathlineto{\pgfqpoint{5.379172in}{2.821290in}}%
\pgfpathlineto{\pgfqpoint{5.347342in}{2.963637in}}%
\pgfpathlineto{\pgfqpoint{5.313116in}{2.905338in}}%
\pgfpathclose%
\pgfusepath{fill}%
\end{pgfscope}%
\begin{pgfscope}%
\pgfpathrectangle{\pgfqpoint{1.020000in}{0.880000in}}{\pgfqpoint{6.160000in}{6.160000in}}%
\pgfusepath{clip}%
\pgfsetbuttcap%
\pgfsetroundjoin%
\definecolor{currentfill}{rgb}{0.772706,0.838978,0.949319}%
\pgfsetfillcolor{currentfill}%
\pgfsetlinewidth{0.000000pt}%
\definecolor{currentstroke}{rgb}{0.000000,0.000000,0.000000}%
\pgfsetstrokecolor{currentstroke}%
\pgfsetdash{}{0pt}%
\pgfpathmoveto{\pgfqpoint{3.565957in}{3.493213in}}%
\pgfpathlineto{\pgfqpoint{3.573223in}{3.742322in}}%
\pgfpathlineto{\pgfqpoint{3.581925in}{3.762466in}}%
\pgfpathlineto{\pgfqpoint{3.616896in}{3.620494in}}%
\pgfpathlineto{\pgfqpoint{3.651211in}{3.579885in}}%
\pgfpathlineto{\pgfqpoint{3.641305in}{3.770484in}}%
\pgfpathlineto{\pgfqpoint{3.632774in}{3.707967in}}%
\pgfpathlineto{\pgfqpoint{3.598161in}{3.802714in}}%
\pgfpathlineto{\pgfqpoint{3.565957in}{3.493213in}}%
\pgfpathclose%
\pgfusepath{fill}%
\end{pgfscope}%
\begin{pgfscope}%
\pgfpathrectangle{\pgfqpoint{1.020000in}{0.880000in}}{\pgfqpoint{6.160000in}{6.160000in}}%
\pgfusepath{clip}%
\pgfsetbuttcap%
\pgfsetroundjoin%
\definecolor{currentfill}{rgb}{0.399231,0.528528,0.928459}%
\pgfsetfillcolor{currentfill}%
\pgfsetlinewidth{0.000000pt}%
\definecolor{currentstroke}{rgb}{0.000000,0.000000,0.000000}%
\pgfsetstrokecolor{currentstroke}%
\pgfsetdash{}{0pt}%
\pgfpathmoveto{\pgfqpoint{4.869392in}{3.098791in}}%
\pgfpathlineto{\pgfqpoint{4.878403in}{2.925608in}}%
\pgfpathlineto{\pgfqpoint{4.889499in}{3.048598in}}%
\pgfpathlineto{\pgfqpoint{4.922021in}{2.880517in}}%
\pgfpathlineto{\pgfqpoint{4.954713in}{2.751661in}}%
\pgfpathlineto{\pgfqpoint{4.946873in}{3.069206in}}%
\pgfpathlineto{\pgfqpoint{4.937224in}{3.153221in}}%
\pgfpathlineto{\pgfqpoint{4.901851in}{2.924110in}}%
\pgfpathlineto{\pgfqpoint{4.869392in}{3.098791in}}%
\pgfpathclose%
\pgfusepath{fill}%
\end{pgfscope}%
\begin{pgfscope}%
\pgfpathrectangle{\pgfqpoint{1.020000in}{0.880000in}}{\pgfqpoint{6.160000in}{6.160000in}}%
\pgfusepath{clip}%
\pgfsetbuttcap%
\pgfsetroundjoin%
\definecolor{currentfill}{rgb}{0.708720,0.805721,0.981117}%
\pgfsetfillcolor{currentfill}%
\pgfsetlinewidth{0.000000pt}%
\definecolor{currentstroke}{rgb}{0.000000,0.000000,0.000000}%
\pgfsetstrokecolor{currentstroke}%
\pgfsetdash{}{0pt}%
\pgfpathmoveto{\pgfqpoint{2.628592in}{3.427736in}}%
\pgfpathlineto{\pgfqpoint{2.632283in}{3.691793in}}%
\pgfpathlineto{\pgfqpoint{2.642588in}{3.529936in}}%
\pgfpathlineto{\pgfqpoint{2.677511in}{3.496394in}}%
\pgfpathlineto{\pgfqpoint{2.710993in}{3.557883in}}%
\pgfpathlineto{\pgfqpoint{2.702601in}{3.593012in}}%
\pgfpathlineto{\pgfqpoint{2.695550in}{3.538684in}}%
\pgfpathlineto{\pgfqpoint{2.660464in}{3.588322in}}%
\pgfpathlineto{\pgfqpoint{2.628592in}{3.427736in}}%
\pgfpathclose%
\pgfusepath{fill}%
\end{pgfscope}%
\begin{pgfscope}%
\pgfpathrectangle{\pgfqpoint{1.020000in}{0.880000in}}{\pgfqpoint{6.160000in}{6.160000in}}%
\pgfusepath{clip}%
\pgfsetbuttcap%
\pgfsetroundjoin%
\definecolor{currentfill}{rgb}{0.738826,0.822572,0.968261}%
\pgfsetfillcolor{currentfill}%
\pgfsetlinewidth{0.000000pt}%
\definecolor{currentstroke}{rgb}{0.000000,0.000000,0.000000}%
\pgfsetstrokecolor{currentstroke}%
\pgfsetdash{}{0pt}%
\pgfpathmoveto{\pgfqpoint{3.206834in}{3.604742in}}%
\pgfpathlineto{\pgfqpoint{3.215219in}{3.607637in}}%
\pgfpathlineto{\pgfqpoint{3.222801in}{3.696282in}}%
\pgfpathlineto{\pgfqpoint{3.259116in}{3.476209in}}%
\pgfpathlineto{\pgfqpoint{3.291961in}{3.627139in}}%
\pgfpathlineto{\pgfqpoint{3.282235in}{3.763462in}}%
\pgfpathlineto{\pgfqpoint{3.277259in}{3.375372in}}%
\pgfpathlineto{\pgfqpoint{3.240019in}{3.710893in}}%
\pgfpathlineto{\pgfqpoint{3.206834in}{3.604742in}}%
\pgfpathclose%
\pgfusepath{fill}%
\end{pgfscope}%
\begin{pgfscope}%
\pgfpathrectangle{\pgfqpoint{1.020000in}{0.880000in}}{\pgfqpoint{6.160000in}{6.160000in}}%
\pgfusepath{clip}%
\pgfsetbuttcap%
\pgfsetroundjoin%
\definecolor{currentfill}{rgb}{0.285273,0.380129,0.823469}%
\pgfsetfillcolor{currentfill}%
\pgfsetlinewidth{0.000000pt}%
\definecolor{currentstroke}{rgb}{0.000000,0.000000,0.000000}%
\pgfsetstrokecolor{currentstroke}%
\pgfsetdash{}{0pt}%
\pgfpathmoveto{\pgfqpoint{5.915144in}{2.909835in}}%
\pgfpathlineto{\pgfqpoint{5.923531in}{2.739936in}}%
\pgfpathlineto{\pgfqpoint{5.937087in}{2.863878in}}%
\pgfpathlineto{\pgfqpoint{5.969519in}{2.806169in}}%
\pgfpathlineto{\pgfqpoint{6.000243in}{2.658379in}}%
\pgfpathlineto{\pgfqpoint{5.988346in}{2.632461in}}%
\pgfpathlineto{\pgfqpoint{5.977890in}{2.684593in}}%
\pgfpathlineto{\pgfqpoint{5.945359in}{2.729231in}}%
\pgfpathlineto{\pgfqpoint{5.915144in}{2.909835in}}%
\pgfpathclose%
\pgfusepath{fill}%
\end{pgfscope}%
\begin{pgfscope}%
\pgfpathrectangle{\pgfqpoint{1.020000in}{0.880000in}}{\pgfqpoint{6.160000in}{6.160000in}}%
\pgfusepath{clip}%
\pgfsetbuttcap%
\pgfsetroundjoin%
\definecolor{currentfill}{rgb}{0.724041,0.814910,0.975651}%
\pgfsetfillcolor{currentfill}%
\pgfsetlinewidth{0.000000pt}%
\definecolor{currentstroke}{rgb}{0.000000,0.000000,0.000000}%
\pgfsetstrokecolor{currentstroke}%
\pgfsetdash{}{0pt}%
\pgfpathmoveto{\pgfqpoint{2.915431in}{3.682121in}}%
\pgfpathlineto{\pgfqpoint{2.925707in}{3.509560in}}%
\pgfpathlineto{\pgfqpoint{2.931206in}{3.718763in}}%
\pgfpathlineto{\pgfqpoint{2.968641in}{3.467761in}}%
\pgfpathlineto{\pgfqpoint{3.003198in}{3.444271in}}%
\pgfpathlineto{\pgfqpoint{2.993324in}{3.587337in}}%
\pgfpathlineto{\pgfqpoint{2.983586in}{3.718495in}}%
\pgfpathlineto{\pgfqpoint{2.951175in}{3.565168in}}%
\pgfpathlineto{\pgfqpoint{2.915431in}{3.682121in}}%
\pgfpathclose%
\pgfusepath{fill}%
\end{pgfscope}%
\begin{pgfscope}%
\pgfpathrectangle{\pgfqpoint{1.020000in}{0.880000in}}{\pgfqpoint{6.160000in}{6.160000in}}%
\pgfusepath{clip}%
\pgfsetbuttcap%
\pgfsetroundjoin%
\definecolor{currentfill}{rgb}{0.613933,0.739923,0.999142}%
\pgfsetfillcolor{currentfill}%
\pgfsetlinewidth{0.000000pt}%
\definecolor{currentstroke}{rgb}{0.000000,0.000000,0.000000}%
\pgfsetstrokecolor{currentstroke}%
\pgfsetdash{}{0pt}%
\pgfpathmoveto{\pgfqpoint{4.250663in}{3.489050in}}%
\pgfpathlineto{\pgfqpoint{4.260128in}{3.378438in}}%
\pgfpathlineto{\pgfqpoint{4.269694in}{3.386564in}}%
\pgfpathlineto{\pgfqpoint{4.303687in}{3.427420in}}%
\pgfpathlineto{\pgfqpoint{4.337367in}{3.261585in}}%
\pgfpathlineto{\pgfqpoint{4.328052in}{3.469472in}}%
\pgfpathlineto{\pgfqpoint{4.318104in}{3.222982in}}%
\pgfpathlineto{\pgfqpoint{4.284457in}{3.374321in}}%
\pgfpathlineto{\pgfqpoint{4.250663in}{3.489050in}}%
\pgfpathclose%
\pgfusepath{fill}%
\end{pgfscope}%
\begin{pgfscope}%
\pgfpathrectangle{\pgfqpoint{1.020000in}{0.880000in}}{\pgfqpoint{6.160000in}{6.160000in}}%
\pgfusepath{clip}%
\pgfsetbuttcap%
\pgfsetroundjoin%
\definecolor{currentfill}{rgb}{0.358415,0.478426,0.896795}%
\pgfsetfillcolor{currentfill}%
\pgfsetlinewidth{0.000000pt}%
\definecolor{currentstroke}{rgb}{0.000000,0.000000,0.000000}%
\pgfsetstrokecolor{currentstroke}%
\pgfsetdash{}{0pt}%
\pgfpathmoveto{\pgfqpoint{5.023823in}{2.973377in}}%
\pgfpathlineto{\pgfqpoint{5.034430in}{2.999226in}}%
\pgfpathlineto{\pgfqpoint{5.043901in}{2.890464in}}%
\pgfpathlineto{\pgfqpoint{5.076001in}{2.712368in}}%
\pgfpathlineto{\pgfqpoint{5.111816in}{2.947483in}}%
\pgfpathlineto{\pgfqpoint{5.100330in}{2.837602in}}%
\pgfpathlineto{\pgfqpoint{5.090978in}{2.956755in}}%
\pgfpathlineto{\pgfqpoint{5.056845in}{2.900277in}}%
\pgfpathlineto{\pgfqpoint{5.023823in}{2.973377in}}%
\pgfpathclose%
\pgfusepath{fill}%
\end{pgfscope}%
\begin{pgfscope}%
\pgfpathrectangle{\pgfqpoint{1.020000in}{0.880000in}}{\pgfqpoint{6.160000in}{6.160000in}}%
\pgfusepath{clip}%
\pgfsetbuttcap%
\pgfsetroundjoin%
\definecolor{currentfill}{rgb}{0.667253,0.779176,0.992959}%
\pgfsetfillcolor{currentfill}%
\pgfsetlinewidth{0.000000pt}%
\definecolor{currentstroke}{rgb}{0.000000,0.000000,0.000000}%
\pgfsetstrokecolor{currentstroke}%
\pgfsetdash{}{0pt}%
\pgfpathmoveto{\pgfqpoint{4.096085in}{3.449469in}}%
\pgfpathlineto{\pgfqpoint{4.105572in}{3.298256in}}%
\pgfpathlineto{\pgfqpoint{4.114859in}{3.427988in}}%
\pgfpathlineto{\pgfqpoint{4.148780in}{3.574033in}}%
\pgfpathlineto{\pgfqpoint{4.182781in}{3.552635in}}%
\pgfpathlineto{\pgfqpoint{4.173299in}{3.696877in}}%
\pgfpathlineto{\pgfqpoint{4.163879in}{3.486854in}}%
\pgfpathlineto{\pgfqpoint{4.130058in}{3.331675in}}%
\pgfpathlineto{\pgfqpoint{4.096085in}{3.449469in}}%
\pgfpathclose%
\pgfusepath{fill}%
\end{pgfscope}%
\begin{pgfscope}%
\pgfpathrectangle{\pgfqpoint{1.020000in}{0.880000in}}{\pgfqpoint{6.160000in}{6.160000in}}%
\pgfusepath{clip}%
\pgfsetbuttcap%
\pgfsetroundjoin%
\definecolor{currentfill}{rgb}{0.343278,0.459354,0.884122}%
\pgfsetfillcolor{currentfill}%
\pgfsetlinewidth{0.000000pt}%
\definecolor{currentstroke}{rgb}{0.000000,0.000000,0.000000}%
\pgfsetstrokecolor{currentstroke}%
\pgfsetdash{}{0pt}%
\pgfpathmoveto{\pgfqpoint{5.245297in}{2.842681in}}%
\pgfpathlineto{\pgfqpoint{5.257078in}{2.950497in}}%
\pgfpathlineto{\pgfqpoint{5.267670in}{2.944981in}}%
\pgfpathlineto{\pgfqpoint{5.300013in}{2.827283in}}%
\pgfpathlineto{\pgfqpoint{5.334947in}{2.942166in}}%
\pgfpathlineto{\pgfqpoint{5.322493in}{2.790815in}}%
\pgfpathlineto{\pgfqpoint{5.313116in}{2.905338in}}%
\pgfpathlineto{\pgfqpoint{5.278108in}{2.774367in}}%
\pgfpathlineto{\pgfqpoint{5.245297in}{2.842681in}}%
\pgfpathclose%
\pgfusepath{fill}%
\end{pgfscope}%
\begin{pgfscope}%
\pgfpathrectangle{\pgfqpoint{1.020000in}{0.880000in}}{\pgfqpoint{6.160000in}{6.160000in}}%
\pgfusepath{clip}%
\pgfsetbuttcap%
\pgfsetroundjoin%
\definecolor{currentfill}{rgb}{0.688188,0.793178,0.988038}%
\pgfsetfillcolor{currentfill}%
\pgfsetlinewidth{0.000000pt}%
\definecolor{currentstroke}{rgb}{0.000000,0.000000,0.000000}%
\pgfsetstrokecolor{currentstroke}%
\pgfsetdash{}{0pt}%
\pgfpathmoveto{\pgfqpoint{2.560322in}{3.401893in}}%
\pgfpathlineto{\pgfqpoint{2.567545in}{3.431692in}}%
\pgfpathlineto{\pgfqpoint{2.570548in}{3.725256in}}%
\pgfpathlineto{\pgfqpoint{2.610374in}{3.388545in}}%
\pgfpathlineto{\pgfqpoint{2.642588in}{3.529936in}}%
\pgfpathlineto{\pgfqpoint{2.632283in}{3.691793in}}%
\pgfpathlineto{\pgfqpoint{2.628592in}{3.427736in}}%
\pgfpathlineto{\pgfqpoint{2.592770in}{3.521316in}}%
\pgfpathlineto{\pgfqpoint{2.560322in}{3.401893in}}%
\pgfpathclose%
\pgfusepath{fill}%
\end{pgfscope}%
\begin{pgfscope}%
\pgfpathrectangle{\pgfqpoint{1.020000in}{0.880000in}}{\pgfqpoint{6.160000in}{6.160000in}}%
\pgfusepath{clip}%
\pgfsetbuttcap%
\pgfsetroundjoin%
\definecolor{currentfill}{rgb}{0.516260,0.654498,0.986407}%
\pgfsetfillcolor{currentfill}%
\pgfsetlinewidth{0.000000pt}%
\definecolor{currentstroke}{rgb}{0.000000,0.000000,0.000000}%
\pgfsetstrokecolor{currentstroke}%
\pgfsetdash{}{0pt}%
\pgfpathmoveto{\pgfqpoint{4.560325in}{3.369307in}}%
\pgfpathlineto{\pgfqpoint{4.569468in}{3.175919in}}%
\pgfpathlineto{\pgfqpoint{4.579083in}{3.105640in}}%
\pgfpathlineto{\pgfqpoint{4.613153in}{3.163867in}}%
\pgfpathlineto{\pgfqpoint{4.646793in}{3.125623in}}%
\pgfpathlineto{\pgfqpoint{4.636694in}{3.094178in}}%
\pgfpathlineto{\pgfqpoint{4.627578in}{3.277559in}}%
\pgfpathlineto{\pgfqpoint{4.593750in}{3.268278in}}%
\pgfpathlineto{\pgfqpoint{4.560325in}{3.369307in}}%
\pgfpathclose%
\pgfusepath{fill}%
\end{pgfscope}%
\begin{pgfscope}%
\pgfpathrectangle{\pgfqpoint{1.020000in}{0.880000in}}{\pgfqpoint{6.160000in}{6.160000in}}%
\pgfusepath{clip}%
\pgfsetbuttcap%
\pgfsetroundjoin%
\definecolor{currentfill}{rgb}{0.467678,0.605591,0.968546}%
\pgfsetfillcolor{currentfill}%
\pgfsetlinewidth{0.000000pt}%
\definecolor{currentstroke}{rgb}{0.000000,0.000000,0.000000}%
\pgfsetstrokecolor{currentstroke}%
\pgfsetdash{}{0pt}%
\pgfpathmoveto{\pgfqpoint{4.713187in}{2.908655in}}%
\pgfpathlineto{\pgfqpoint{4.724987in}{3.235956in}}%
\pgfpathlineto{\pgfqpoint{4.734302in}{3.097932in}}%
\pgfpathlineto{\pgfqpoint{4.769265in}{3.295929in}}%
\pgfpathlineto{\pgfqpoint{4.801920in}{3.103639in}}%
\pgfpathlineto{\pgfqpoint{4.790659in}{2.918745in}}%
\pgfpathlineto{\pgfqpoint{4.781926in}{3.144920in}}%
\pgfpathlineto{\pgfqpoint{4.748163in}{3.139176in}}%
\pgfpathlineto{\pgfqpoint{4.713187in}{2.908655in}}%
\pgfpathclose%
\pgfusepath{fill}%
\end{pgfscope}%
\begin{pgfscope}%
\pgfpathrectangle{\pgfqpoint{1.020000in}{0.880000in}}{\pgfqpoint{6.160000in}{6.160000in}}%
\pgfusepath{clip}%
\pgfsetbuttcap%
\pgfsetroundjoin%
\definecolor{currentfill}{rgb}{0.304174,0.406945,0.845263}%
\pgfsetfillcolor{currentfill}%
\pgfsetlinewidth{0.000000pt}%
\definecolor{currentstroke}{rgb}{0.000000,0.000000,0.000000}%
\pgfsetstrokecolor{currentstroke}%
\pgfsetdash{}{0pt}%
\pgfpathmoveto{\pgfqpoint{5.401643in}{2.898894in}}%
\pgfpathlineto{\pgfqpoint{5.412269in}{2.884565in}}%
\pgfpathlineto{\pgfqpoint{5.421307in}{2.741131in}}%
\pgfpathlineto{\pgfqpoint{5.455523in}{2.790734in}}%
\pgfpathlineto{\pgfqpoint{5.487731in}{2.686589in}}%
\pgfpathlineto{\pgfqpoint{5.477772in}{2.756467in}}%
\pgfpathlineto{\pgfqpoint{5.466838in}{2.750342in}}%
\pgfpathlineto{\pgfqpoint{5.433911in}{2.795612in}}%
\pgfpathlineto{\pgfqpoint{5.401643in}{2.898894in}}%
\pgfpathclose%
\pgfusepath{fill}%
\end{pgfscope}%
\begin{pgfscope}%
\pgfpathrectangle{\pgfqpoint{1.020000in}{0.880000in}}{\pgfqpoint{6.160000in}{6.160000in}}%
\pgfusepath{clip}%
\pgfsetbuttcap%
\pgfsetroundjoin%
\definecolor{currentfill}{rgb}{0.646113,0.764436,0.996868}%
\pgfsetfillcolor{currentfill}%
\pgfsetlinewidth{0.000000pt}%
\definecolor{currentstroke}{rgb}{0.000000,0.000000,0.000000}%
\pgfsetstrokecolor{currentstroke}%
\pgfsetdash{}{0pt}%
\pgfpathmoveto{\pgfqpoint{4.028259in}{3.412339in}}%
\pgfpathlineto{\pgfqpoint{4.037466in}{3.475109in}}%
\pgfpathlineto{\pgfqpoint{4.046904in}{3.392013in}}%
\pgfpathlineto{\pgfqpoint{4.080786in}{3.507988in}}%
\pgfpathlineto{\pgfqpoint{4.114859in}{3.427988in}}%
\pgfpathlineto{\pgfqpoint{4.105572in}{3.298256in}}%
\pgfpathlineto{\pgfqpoint{4.096085in}{3.449469in}}%
\pgfpathlineto{\pgfqpoint{4.062092in}{3.499674in}}%
\pgfpathlineto{\pgfqpoint{4.028259in}{3.412339in}}%
\pgfpathclose%
\pgfusepath{fill}%
\end{pgfscope}%
\begin{pgfscope}%
\pgfpathrectangle{\pgfqpoint{1.020000in}{0.880000in}}{\pgfqpoint{6.160000in}{6.160000in}}%
\pgfusepath{clip}%
\pgfsetbuttcap%
\pgfsetroundjoin%
\definecolor{currentfill}{rgb}{0.581486,0.713451,0.998314}%
\pgfsetfillcolor{currentfill}%
\pgfsetlinewidth{0.000000pt}%
\definecolor{currentstroke}{rgb}{0.000000,0.000000,0.000000}%
\pgfsetstrokecolor{currentstroke}%
\pgfsetdash{}{0pt}%
\pgfpathmoveto{\pgfqpoint{4.404996in}{3.229765in}}%
\pgfpathlineto{\pgfqpoint{4.414558in}{3.170075in}}%
\pgfpathlineto{\pgfqpoint{4.424899in}{3.428592in}}%
\pgfpathlineto{\pgfqpoint{4.458561in}{3.326524in}}%
\pgfpathlineto{\pgfqpoint{4.492270in}{3.268841in}}%
\pgfpathlineto{\pgfqpoint{4.482424in}{3.251038in}}%
\pgfpathlineto{\pgfqpoint{4.473218in}{3.445774in}}%
\pgfpathlineto{\pgfqpoint{4.439145in}{3.363080in}}%
\pgfpathlineto{\pgfqpoint{4.404996in}{3.229765in}}%
\pgfpathclose%
\pgfusepath{fill}%
\end{pgfscope}%
\begin{pgfscope}%
\pgfpathrectangle{\pgfqpoint{1.020000in}{0.880000in}}{\pgfqpoint{6.160000in}{6.160000in}}%
\pgfusepath{clip}%
\pgfsetbuttcap%
\pgfsetroundjoin%
\definecolor{currentfill}{rgb}{0.768034,0.837035,0.952488}%
\pgfsetfillcolor{currentfill}%
\pgfsetlinewidth{0.000000pt}%
\definecolor{currentstroke}{rgb}{0.000000,0.000000,0.000000}%
\pgfsetstrokecolor{currentstroke}%
\pgfsetdash{}{0pt}%
\pgfpathmoveto{\pgfqpoint{3.497017in}{3.619056in}}%
\pgfpathlineto{\pgfqpoint{3.505790in}{3.611793in}}%
\pgfpathlineto{\pgfqpoint{3.514096in}{3.677288in}}%
\pgfpathlineto{\pgfqpoint{3.547956in}{3.727850in}}%
\pgfpathlineto{\pgfqpoint{3.581925in}{3.762466in}}%
\pgfpathlineto{\pgfqpoint{3.573223in}{3.742322in}}%
\pgfpathlineto{\pgfqpoint{3.565957in}{3.493213in}}%
\pgfpathlineto{\pgfqpoint{3.530420in}{3.724934in}}%
\pgfpathlineto{\pgfqpoint{3.497017in}{3.619056in}}%
\pgfpathclose%
\pgfusepath{fill}%
\end{pgfscope}%
\begin{pgfscope}%
\pgfpathrectangle{\pgfqpoint{1.020000in}{0.880000in}}{\pgfqpoint{6.160000in}{6.160000in}}%
\pgfusepath{clip}%
\pgfsetbuttcap%
\pgfsetroundjoin%
\definecolor{currentfill}{rgb}{0.323718,0.433158,0.864722}%
\pgfsetfillcolor{currentfill}%
\pgfsetlinewidth{0.000000pt}%
\definecolor{currentstroke}{rgb}{0.000000,0.000000,0.000000}%
\pgfsetstrokecolor{currentstroke}%
\pgfsetdash{}{0pt}%
\pgfpathmoveto{\pgfqpoint{5.692450in}{2.943306in}}%
\pgfpathlineto{\pgfqpoint{5.701813in}{2.828188in}}%
\pgfpathlineto{\pgfqpoint{5.712498in}{2.799102in}}%
\pgfpathlineto{\pgfqpoint{5.742692in}{2.588699in}}%
\pgfpathlineto{\pgfqpoint{5.779363in}{2.791484in}}%
\pgfpathlineto{\pgfqpoint{5.770269in}{2.923179in}}%
\pgfpathlineto{\pgfqpoint{5.757857in}{2.845685in}}%
\pgfpathlineto{\pgfqpoint{5.725909in}{2.941497in}}%
\pgfpathlineto{\pgfqpoint{5.692450in}{2.943306in}}%
\pgfpathclose%
\pgfusepath{fill}%
\end{pgfscope}%
\begin{pgfscope}%
\pgfpathrectangle{\pgfqpoint{1.020000in}{0.880000in}}{\pgfqpoint{6.160000in}{6.160000in}}%
\pgfusepath{clip}%
\pgfsetbuttcap%
\pgfsetroundjoin%
\definecolor{currentfill}{rgb}{0.743754,0.825125,0.965798}%
\pgfsetfillcolor{currentfill}%
\pgfsetlinewidth{0.000000pt}%
\definecolor{currentstroke}{rgb}{0.000000,0.000000,0.000000}%
\pgfsetstrokecolor{currentstroke}%
\pgfsetdash{}{0pt}%
\pgfpathmoveto{\pgfqpoint{3.429179in}{3.564795in}}%
\pgfpathlineto{\pgfqpoint{3.437242in}{3.643960in}}%
\pgfpathlineto{\pgfqpoint{3.446051in}{3.625893in}}%
\pgfpathlineto{\pgfqpoint{3.479825in}{3.687265in}}%
\pgfpathlineto{\pgfqpoint{3.514096in}{3.677288in}}%
\pgfpathlineto{\pgfqpoint{3.505790in}{3.611793in}}%
\pgfpathlineto{\pgfqpoint{3.497017in}{3.619056in}}%
\pgfpathlineto{\pgfqpoint{3.463684in}{3.510552in}}%
\pgfpathlineto{\pgfqpoint{3.429179in}{3.564795in}}%
\pgfpathclose%
\pgfusepath{fill}%
\end{pgfscope}%
\begin{pgfscope}%
\pgfpathrectangle{\pgfqpoint{1.020000in}{0.880000in}}{\pgfqpoint{6.160000in}{6.160000in}}%
\pgfusepath{clip}%
\pgfsetbuttcap%
\pgfsetroundjoin%
\definecolor{currentfill}{rgb}{0.758539,0.832787,0.958408}%
\pgfsetfillcolor{currentfill}%
\pgfsetlinewidth{0.000000pt}%
\definecolor{currentstroke}{rgb}{0.000000,0.000000,0.000000}%
\pgfsetstrokecolor{currentstroke}%
\pgfsetdash{}{0pt}%
\pgfpathmoveto{\pgfqpoint{3.719315in}{3.581161in}}%
\pgfpathlineto{\pgfqpoint{3.727699in}{3.708691in}}%
\pgfpathlineto{\pgfqpoint{3.737153in}{3.605526in}}%
\pgfpathlineto{\pgfqpoint{3.770712in}{3.737409in}}%
\pgfpathlineto{\pgfqpoint{3.805133in}{3.658979in}}%
\pgfpathlineto{\pgfqpoint{3.796207in}{3.624638in}}%
\pgfpathlineto{\pgfqpoint{3.787489in}{3.544306in}}%
\pgfpathlineto{\pgfqpoint{3.752720in}{3.725455in}}%
\pgfpathlineto{\pgfqpoint{3.719315in}{3.581161in}}%
\pgfpathclose%
\pgfusepath{fill}%
\end{pgfscope}%
\begin{pgfscope}%
\pgfpathrectangle{\pgfqpoint{1.020000in}{0.880000in}}{\pgfqpoint{6.160000in}{6.160000in}}%
\pgfusepath{clip}%
\pgfsetbuttcap%
\pgfsetroundjoin%
\definecolor{currentfill}{rgb}{0.724041,0.814910,0.975651}%
\pgfsetfillcolor{currentfill}%
\pgfsetlinewidth{0.000000pt}%
\definecolor{currentstroke}{rgb}{0.000000,0.000000,0.000000}%
\pgfsetstrokecolor{currentstroke}%
\pgfsetdash{}{0pt}%
\pgfpathmoveto{\pgfqpoint{2.847342in}{3.638368in}}%
\pgfpathlineto{\pgfqpoint{2.856024in}{3.589519in}}%
\pgfpathlineto{\pgfqpoint{2.864562in}{3.552380in}}%
\pgfpathlineto{\pgfqpoint{2.900193in}{3.454135in}}%
\pgfpathlineto{\pgfqpoint{2.931206in}{3.718763in}}%
\pgfpathlineto{\pgfqpoint{2.925707in}{3.509560in}}%
\pgfpathlineto{\pgfqpoint{2.915431in}{3.682121in}}%
\pgfpathlineto{\pgfqpoint{2.884196in}{3.444178in}}%
\pgfpathlineto{\pgfqpoint{2.847342in}{3.638368in}}%
\pgfpathclose%
\pgfusepath{fill}%
\end{pgfscope}%
\begin{pgfscope}%
\pgfpathrectangle{\pgfqpoint{1.020000in}{0.880000in}}{\pgfqpoint{6.160000in}{6.160000in}}%
\pgfusepath{clip}%
\pgfsetbuttcap%
\pgfsetroundjoin%
\definecolor{currentfill}{rgb}{0.733898,0.820018,0.970724}%
\pgfsetfillcolor{currentfill}%
\pgfsetlinewidth{0.000000pt}%
\definecolor{currentstroke}{rgb}{0.000000,0.000000,0.000000}%
\pgfsetstrokecolor{currentstroke}%
\pgfsetdash{}{0pt}%
\pgfpathmoveto{\pgfqpoint{3.139028in}{3.545200in}}%
\pgfpathlineto{\pgfqpoint{3.147192in}{3.562041in}}%
\pgfpathlineto{\pgfqpoint{3.154319in}{3.682041in}}%
\pgfpathlineto{\pgfqpoint{3.190512in}{3.493136in}}%
\pgfpathlineto{\pgfqpoint{3.222801in}{3.696282in}}%
\pgfpathlineto{\pgfqpoint{3.215219in}{3.607637in}}%
\pgfpathlineto{\pgfqpoint{3.206834in}{3.604742in}}%
\pgfpathlineto{\pgfqpoint{3.172652in}{3.602695in}}%
\pgfpathlineto{\pgfqpoint{3.139028in}{3.545200in}}%
\pgfpathclose%
\pgfusepath{fill}%
\end{pgfscope}%
\begin{pgfscope}%
\pgfpathrectangle{\pgfqpoint{1.020000in}{0.880000in}}{\pgfqpoint{6.160000in}{6.160000in}}%
\pgfusepath{clip}%
\pgfsetbuttcap%
\pgfsetroundjoin%
\definecolor{currentfill}{rgb}{0.718985,0.811993,0.977656}%
\pgfsetfillcolor{currentfill}%
\pgfsetlinewidth{0.000000pt}%
\definecolor{currentstroke}{rgb}{0.000000,0.000000,0.000000}%
\pgfsetstrokecolor{currentstroke}%
\pgfsetdash{}{0pt}%
\pgfpathmoveto{\pgfqpoint{3.873458in}{3.597477in}}%
\pgfpathlineto{\pgfqpoint{3.882985in}{3.468890in}}%
\pgfpathlineto{\pgfqpoint{3.891729in}{3.607645in}}%
\pgfpathlineto{\pgfqpoint{3.926024in}{3.526899in}}%
\pgfpathlineto{\pgfqpoint{3.959775in}{3.656147in}}%
\pgfpathlineto{\pgfqpoint{3.950880in}{3.510457in}}%
\pgfpathlineto{\pgfqpoint{3.941571in}{3.555739in}}%
\pgfpathlineto{\pgfqpoint{3.907505in}{3.586534in}}%
\pgfpathlineto{\pgfqpoint{3.873458in}{3.597477in}}%
\pgfpathclose%
\pgfusepath{fill}%
\end{pgfscope}%
\begin{pgfscope}%
\pgfpathrectangle{\pgfqpoint{1.020000in}{0.880000in}}{\pgfqpoint{6.160000in}{6.160000in}}%
\pgfusepath{clip}%
\pgfsetbuttcap%
\pgfsetroundjoin%
\definecolor{currentfill}{rgb}{0.294718,0.393542,0.834384}%
\pgfsetfillcolor{currentfill}%
\pgfsetlinewidth{0.000000pt}%
\definecolor{currentstroke}{rgb}{0.000000,0.000000,0.000000}%
\pgfsetstrokecolor{currentstroke}%
\pgfsetdash{}{0pt}%
\pgfpathmoveto{\pgfqpoint{6.135709in}{2.767594in}}%
\pgfpathlineto{\pgfqpoint{6.147346in}{2.769762in}}%
\pgfpathlineto{\pgfqpoint{6.159819in}{2.812671in}}%
\pgfpathlineto{\pgfqpoint{6.192879in}{2.796033in}}%
\pgfpathlineto{\pgfqpoint{6.180973in}{2.783566in}}%
\pgfpathlineto{\pgfqpoint{6.167373in}{2.684984in}}%
\pgfpathlineto{\pgfqpoint{6.135709in}{2.767594in}}%
\pgfpathclose%
\pgfusepath{fill}%
\end{pgfscope}%
\begin{pgfscope}%
\pgfpathrectangle{\pgfqpoint{1.020000in}{0.880000in}}{\pgfqpoint{6.160000in}{6.160000in}}%
\pgfusepath{clip}%
\pgfsetbuttcap%
\pgfsetroundjoin%
\definecolor{currentfill}{rgb}{0.462354,0.599830,0.965857}%
\pgfsetfillcolor{currentfill}%
\pgfsetlinewidth{0.000000pt}%
\definecolor{currentstroke}{rgb}{0.000000,0.000000,0.000000}%
\pgfsetstrokecolor{currentstroke}%
\pgfsetdash{}{0pt}%
\pgfpathmoveto{\pgfqpoint{4.646793in}{3.125623in}}%
\pgfpathlineto{\pgfqpoint{4.657104in}{3.195690in}}%
\pgfpathlineto{\pgfqpoint{4.666373in}{3.043619in}}%
\pgfpathlineto{\pgfqpoint{4.700583in}{3.118007in}}%
\pgfpathlineto{\pgfqpoint{4.734302in}{3.097932in}}%
\pgfpathlineto{\pgfqpoint{4.724987in}{3.235956in}}%
\pgfpathlineto{\pgfqpoint{4.713187in}{2.908655in}}%
\pgfpathlineto{\pgfqpoint{4.680093in}{3.028380in}}%
\pgfpathlineto{\pgfqpoint{4.646793in}{3.125623in}}%
\pgfpathclose%
\pgfusepath{fill}%
\end{pgfscope}%
\begin{pgfscope}%
\pgfpathrectangle{\pgfqpoint{1.020000in}{0.880000in}}{\pgfqpoint{6.160000in}{6.160000in}}%
\pgfusepath{clip}%
\pgfsetbuttcap%
\pgfsetroundjoin%
\definecolor{currentfill}{rgb}{0.743754,0.825125,0.965798}%
\pgfsetfillcolor{currentfill}%
\pgfsetlinewidth{0.000000pt}%
\definecolor{currentstroke}{rgb}{0.000000,0.000000,0.000000}%
\pgfsetstrokecolor{currentstroke}%
\pgfsetdash{}{0pt}%
\pgfpathmoveto{\pgfqpoint{3.651211in}{3.579885in}}%
\pgfpathlineto{\pgfqpoint{3.660091in}{3.582808in}}%
\pgfpathlineto{\pgfqpoint{3.668931in}{3.597445in}}%
\pgfpathlineto{\pgfqpoint{3.703440in}{3.520300in}}%
\pgfpathlineto{\pgfqpoint{3.737153in}{3.605526in}}%
\pgfpathlineto{\pgfqpoint{3.727699in}{3.708691in}}%
\pgfpathlineto{\pgfqpoint{3.719315in}{3.581161in}}%
\pgfpathlineto{\pgfqpoint{3.684684in}{3.698861in}}%
\pgfpathlineto{\pgfqpoint{3.651211in}{3.579885in}}%
\pgfpathclose%
\pgfusepath{fill}%
\end{pgfscope}%
\begin{pgfscope}%
\pgfpathrectangle{\pgfqpoint{1.020000in}{0.880000in}}{\pgfqpoint{6.160000in}{6.160000in}}%
\pgfusepath{clip}%
\pgfsetbuttcap%
\pgfsetroundjoin%
\definecolor{currentfill}{rgb}{0.348323,0.465711,0.888346}%
\pgfsetfillcolor{currentfill}%
\pgfsetlinewidth{0.000000pt}%
\definecolor{currentstroke}{rgb}{0.000000,0.000000,0.000000}%
\pgfsetstrokecolor{currentstroke}%
\pgfsetdash{}{0pt}%
\pgfpathmoveto{\pgfqpoint{5.177627in}{2.794535in}}%
\pgfpathlineto{\pgfqpoint{5.189671in}{2.942671in}}%
\pgfpathlineto{\pgfqpoint{5.198617in}{2.782328in}}%
\pgfpathlineto{\pgfqpoint{5.233299in}{2.879982in}}%
\pgfpathlineto{\pgfqpoint{5.267670in}{2.944981in}}%
\pgfpathlineto{\pgfqpoint{5.257078in}{2.950497in}}%
\pgfpathlineto{\pgfqpoint{5.245297in}{2.842681in}}%
\pgfpathlineto{\pgfqpoint{5.211570in}{2.828865in}}%
\pgfpathlineto{\pgfqpoint{5.177627in}{2.794535in}}%
\pgfpathclose%
\pgfusepath{fill}%
\end{pgfscope}%
\begin{pgfscope}%
\pgfpathrectangle{\pgfqpoint{1.020000in}{0.880000in}}{\pgfqpoint{6.160000in}{6.160000in}}%
\pgfusepath{clip}%
\pgfsetbuttcap%
\pgfsetroundjoin%
\definecolor{currentfill}{rgb}{0.368507,0.491141,0.905243}%
\pgfsetfillcolor{currentfill}%
\pgfsetlinewidth{0.000000pt}%
\definecolor{currentstroke}{rgb}{0.000000,0.000000,0.000000}%
\pgfsetstrokecolor{currentstroke}%
\pgfsetdash{}{0pt}%
\pgfpathmoveto{\pgfqpoint{4.954713in}{2.751661in}}%
\pgfpathlineto{\pgfqpoint{4.966149in}{2.896425in}}%
\pgfpathlineto{\pgfqpoint{4.976931in}{2.952662in}}%
\pgfpathlineto{\pgfqpoint{5.010737in}{2.957743in}}%
\pgfpathlineto{\pgfqpoint{5.043901in}{2.890464in}}%
\pgfpathlineto{\pgfqpoint{5.034430in}{2.999226in}}%
\pgfpathlineto{\pgfqpoint{5.023823in}{2.973377in}}%
\pgfpathlineto{\pgfqpoint{4.989362in}{2.877495in}}%
\pgfpathlineto{\pgfqpoint{4.954713in}{2.751661in}}%
\pgfpathclose%
\pgfusepath{fill}%
\end{pgfscope}%
\begin{pgfscope}%
\pgfpathrectangle{\pgfqpoint{1.020000in}{0.880000in}}{\pgfqpoint{6.160000in}{6.160000in}}%
\pgfusepath{clip}%
\pgfsetbuttcap%
\pgfsetroundjoin%
\definecolor{currentfill}{rgb}{0.703587,0.802586,0.982847}%
\pgfsetfillcolor{currentfill}%
\pgfsetlinewidth{0.000000pt}%
\definecolor{currentstroke}{rgb}{0.000000,0.000000,0.000000}%
\pgfsetstrokecolor{currentstroke}%
\pgfsetdash{}{0pt}%
\pgfpathmoveto{\pgfqpoint{2.487584in}{3.637309in}}%
\pgfpathlineto{\pgfqpoint{2.497032in}{3.530769in}}%
\pgfpathlineto{\pgfqpoint{2.503691in}{3.590507in}}%
\pgfpathlineto{\pgfqpoint{2.540908in}{3.426425in}}%
\pgfpathlineto{\pgfqpoint{2.570548in}{3.725256in}}%
\pgfpathlineto{\pgfqpoint{2.567545in}{3.431692in}}%
\pgfpathlineto{\pgfqpoint{2.560322in}{3.401893in}}%
\pgfpathlineto{\pgfqpoint{2.524327in}{3.500075in}}%
\pgfpathlineto{\pgfqpoint{2.487584in}{3.637309in}}%
\pgfpathclose%
\pgfusepath{fill}%
\end{pgfscope}%
\begin{pgfscope}%
\pgfpathrectangle{\pgfqpoint{1.020000in}{0.880000in}}{\pgfqpoint{6.160000in}{6.160000in}}%
\pgfusepath{clip}%
\pgfsetbuttcap%
\pgfsetroundjoin%
\definecolor{currentfill}{rgb}{0.323718,0.433158,0.864722}%
\pgfsetfillcolor{currentfill}%
\pgfsetlinewidth{0.000000pt}%
\definecolor{currentstroke}{rgb}{0.000000,0.000000,0.000000}%
\pgfsetstrokecolor{currentstroke}%
\pgfsetdash{}{0pt}%
\pgfpathmoveto{\pgfqpoint{5.622572in}{2.747043in}}%
\pgfpathlineto{\pgfqpoint{5.634751in}{2.825500in}}%
\pgfpathlineto{\pgfqpoint{5.645965in}{2.835539in}}%
\pgfpathlineto{\pgfqpoint{5.677661in}{2.711914in}}%
\pgfpathlineto{\pgfqpoint{5.712498in}{2.799102in}}%
\pgfpathlineto{\pgfqpoint{5.701813in}{2.828188in}}%
\pgfpathlineto{\pgfqpoint{5.692450in}{2.943306in}}%
\pgfpathlineto{\pgfqpoint{5.658117in}{2.887552in}}%
\pgfpathlineto{\pgfqpoint{5.622572in}{2.747043in}}%
\pgfpathclose%
\pgfusepath{fill}%
\end{pgfscope}%
\begin{pgfscope}%
\pgfpathrectangle{\pgfqpoint{1.020000in}{0.880000in}}{\pgfqpoint{6.160000in}{6.160000in}}%
\pgfusepath{clip}%
\pgfsetbuttcap%
\pgfsetroundjoin%
\definecolor{currentfill}{rgb}{0.435815,0.570707,0.951717}%
\pgfsetfillcolor{currentfill}%
\pgfsetlinewidth{0.000000pt}%
\definecolor{currentstroke}{rgb}{0.000000,0.000000,0.000000}%
\pgfsetstrokecolor{currentstroke}%
\pgfsetdash{}{0pt}%
\pgfpathmoveto{\pgfqpoint{4.801920in}{3.103639in}}%
\pgfpathlineto{\pgfqpoint{4.811367in}{2.991564in}}%
\pgfpathlineto{\pgfqpoint{4.821335in}{2.960973in}}%
\pgfpathlineto{\pgfqpoint{4.855065in}{2.953234in}}%
\pgfpathlineto{\pgfqpoint{4.889499in}{3.048598in}}%
\pgfpathlineto{\pgfqpoint{4.878403in}{2.925608in}}%
\pgfpathlineto{\pgfqpoint{4.869392in}{3.098791in}}%
\pgfpathlineto{\pgfqpoint{4.836456in}{3.221653in}}%
\pgfpathlineto{\pgfqpoint{4.801920in}{3.103639in}}%
\pgfpathclose%
\pgfusepath{fill}%
\end{pgfscope}%
\begin{pgfscope}%
\pgfpathrectangle{\pgfqpoint{1.020000in}{0.880000in}}{\pgfqpoint{6.160000in}{6.160000in}}%
\pgfusepath{clip}%
\pgfsetbuttcap%
\pgfsetroundjoin%
\definecolor{currentfill}{rgb}{0.543440,0.680003,0.993051}%
\pgfsetfillcolor{currentfill}%
\pgfsetlinewidth{0.000000pt}%
\definecolor{currentstroke}{rgb}{0.000000,0.000000,0.000000}%
\pgfsetstrokecolor{currentstroke}%
\pgfsetdash{}{0pt}%
\pgfpathmoveto{\pgfqpoint{4.492270in}{3.268841in}}%
\pgfpathlineto{\pgfqpoint{4.502101in}{3.273733in}}%
\pgfpathlineto{\pgfqpoint{4.511771in}{3.223483in}}%
\pgfpathlineto{\pgfqpoint{4.545697in}{3.230676in}}%
\pgfpathlineto{\pgfqpoint{4.579083in}{3.105640in}}%
\pgfpathlineto{\pgfqpoint{4.569468in}{3.175919in}}%
\pgfpathlineto{\pgfqpoint{4.560325in}{3.369307in}}%
\pgfpathlineto{\pgfqpoint{4.525856in}{3.192463in}}%
\pgfpathlineto{\pgfqpoint{4.492270in}{3.268841in}}%
\pgfpathclose%
\pgfusepath{fill}%
\end{pgfscope}%
\begin{pgfscope}%
\pgfpathrectangle{\pgfqpoint{1.020000in}{0.880000in}}{\pgfqpoint{6.160000in}{6.160000in}}%
\pgfusepath{clip}%
\pgfsetbuttcap%
\pgfsetroundjoin%
\definecolor{currentfill}{rgb}{0.304174,0.406945,0.845263}%
\pgfsetfillcolor{currentfill}%
\pgfsetlinewidth{0.000000pt}%
\definecolor{currentstroke}{rgb}{0.000000,0.000000,0.000000}%
\pgfsetstrokecolor{currentstroke}%
\pgfsetdash{}{0pt}%
\pgfpathmoveto{\pgfqpoint{6.067262in}{2.675183in}}%
\pgfpathlineto{\pgfqpoint{6.079863in}{2.731886in}}%
\pgfpathlineto{\pgfqpoint{6.094432in}{2.889019in}}%
\pgfpathlineto{\pgfqpoint{6.124291in}{2.704703in}}%
\pgfpathlineto{\pgfqpoint{6.159819in}{2.812671in}}%
\pgfpathlineto{\pgfqpoint{6.147346in}{2.769762in}}%
\pgfpathlineto{\pgfqpoint{6.135709in}{2.767594in}}%
\pgfpathlineto{\pgfqpoint{6.104618in}{2.883809in}}%
\pgfpathlineto{\pgfqpoint{6.067262in}{2.675183in}}%
\pgfpathclose%
\pgfusepath{fill}%
\end{pgfscope}%
\begin{pgfscope}%
\pgfpathrectangle{\pgfqpoint{1.020000in}{0.880000in}}{\pgfqpoint{6.160000in}{6.160000in}}%
\pgfusepath{clip}%
\pgfsetbuttcap%
\pgfsetroundjoin%
\definecolor{currentfill}{rgb}{0.703587,0.802586,0.982847}%
\pgfsetfillcolor{currentfill}%
\pgfsetlinewidth{0.000000pt}%
\definecolor{currentstroke}{rgb}{0.000000,0.000000,0.000000}%
\pgfsetstrokecolor{currentstroke}%
\pgfsetdash{}{0pt}%
\pgfpathmoveto{\pgfqpoint{2.710993in}{3.557883in}}%
\pgfpathlineto{\pgfqpoint{2.720014in}{3.480526in}}%
\pgfpathlineto{\pgfqpoint{2.729878in}{3.345539in}}%
\pgfpathlineto{\pgfqpoint{2.760691in}{3.596926in}}%
\pgfpathlineto{\pgfqpoint{2.794991in}{3.607672in}}%
\pgfpathlineto{\pgfqpoint{2.789245in}{3.447351in}}%
\pgfpathlineto{\pgfqpoint{2.778847in}{3.621990in}}%
\pgfpathlineto{\pgfqpoint{2.745408in}{3.555892in}}%
\pgfpathlineto{\pgfqpoint{2.710993in}{3.557883in}}%
\pgfpathclose%
\pgfusepath{fill}%
\end{pgfscope}%
\begin{pgfscope}%
\pgfpathrectangle{\pgfqpoint{1.020000in}{0.880000in}}{\pgfqpoint{6.160000in}{6.160000in}}%
\pgfusepath{clip}%
\pgfsetbuttcap%
\pgfsetroundjoin%
\definecolor{currentfill}{rgb}{0.323718,0.433158,0.864722}%
\pgfsetfillcolor{currentfill}%
\pgfsetlinewidth{0.000000pt}%
\definecolor{currentstroke}{rgb}{0.000000,0.000000,0.000000}%
\pgfsetstrokecolor{currentstroke}%
\pgfsetdash{}{0pt}%
\pgfpathmoveto{\pgfqpoint{5.847130in}{2.843265in}}%
\pgfpathlineto{\pgfqpoint{5.856840in}{2.749440in}}%
\pgfpathlineto{\pgfqpoint{5.868783in}{2.787086in}}%
\pgfpathlineto{\pgfqpoint{5.901532in}{2.744550in}}%
\pgfpathlineto{\pgfqpoint{5.937087in}{2.863878in}}%
\pgfpathlineto{\pgfqpoint{5.923531in}{2.739936in}}%
\pgfpathlineto{\pgfqpoint{5.915144in}{2.909835in}}%
\pgfpathlineto{\pgfqpoint{5.881435in}{2.894049in}}%
\pgfpathlineto{\pgfqpoint{5.847130in}{2.843265in}}%
\pgfpathclose%
\pgfusepath{fill}%
\end{pgfscope}%
\begin{pgfscope}%
\pgfpathrectangle{\pgfqpoint{1.020000in}{0.880000in}}{\pgfqpoint{6.160000in}{6.160000in}}%
\pgfusepath{clip}%
\pgfsetbuttcap%
\pgfsetroundjoin%
\definecolor{currentfill}{rgb}{0.294718,0.393542,0.834384}%
\pgfsetfillcolor{currentfill}%
\pgfsetlinewidth{0.000000pt}%
\definecolor{currentstroke}{rgb}{0.000000,0.000000,0.000000}%
\pgfsetstrokecolor{currentstroke}%
\pgfsetdash{}{0pt}%
\pgfpathmoveto{\pgfqpoint{5.779363in}{2.791484in}}%
\pgfpathlineto{\pgfqpoint{5.789595in}{2.730501in}}%
\pgfpathlineto{\pgfqpoint{5.801941in}{2.798636in}}%
\pgfpathlineto{\pgfqpoint{5.834686in}{2.751488in}}%
\pgfpathlineto{\pgfqpoint{5.868783in}{2.787086in}}%
\pgfpathlineto{\pgfqpoint{5.856840in}{2.749440in}}%
\pgfpathlineto{\pgfqpoint{5.847130in}{2.843265in}}%
\pgfpathlineto{\pgfqpoint{5.809718in}{2.602150in}}%
\pgfpathlineto{\pgfqpoint{5.779363in}{2.791484in}}%
\pgfpathclose%
\pgfusepath{fill}%
\end{pgfscope}%
\begin{pgfscope}%
\pgfpathrectangle{\pgfqpoint{1.020000in}{0.880000in}}{\pgfqpoint{6.160000in}{6.160000in}}%
\pgfusepath{clip}%
\pgfsetbuttcap%
\pgfsetroundjoin%
\definecolor{currentfill}{rgb}{0.733898,0.820018,0.970724}%
\pgfsetfillcolor{currentfill}%
\pgfsetlinewidth{0.000000pt}%
\definecolor{currentstroke}{rgb}{0.000000,0.000000,0.000000}%
\pgfsetstrokecolor{currentstroke}%
\pgfsetdash{}{0pt}%
\pgfpathmoveto{\pgfqpoint{2.778847in}{3.621990in}}%
\pgfpathlineto{\pgfqpoint{2.789245in}{3.447351in}}%
\pgfpathlineto{\pgfqpoint{2.794991in}{3.607672in}}%
\pgfpathlineto{\pgfqpoint{2.828933in}{3.643894in}}%
\pgfpathlineto{\pgfqpoint{2.864562in}{3.552380in}}%
\pgfpathlineto{\pgfqpoint{2.856024in}{3.589519in}}%
\pgfpathlineto{\pgfqpoint{2.847342in}{3.638368in}}%
\pgfpathlineto{\pgfqpoint{2.813466in}{3.603558in}}%
\pgfpathlineto{\pgfqpoint{2.778847in}{3.621990in}}%
\pgfpathclose%
\pgfusepath{fill}%
\end{pgfscope}%
\begin{pgfscope}%
\pgfpathrectangle{\pgfqpoint{1.020000in}{0.880000in}}{\pgfqpoint{6.160000in}{6.160000in}}%
\pgfusepath{clip}%
\pgfsetbuttcap%
\pgfsetroundjoin%
\definecolor{currentfill}{rgb}{0.603162,0.731527,0.999565}%
\pgfsetfillcolor{currentfill}%
\pgfsetlinewidth{0.000000pt}%
\definecolor{currentstroke}{rgb}{0.000000,0.000000,0.000000}%
\pgfsetstrokecolor{currentstroke}%
\pgfsetdash{}{0pt}%
\pgfpathmoveto{\pgfqpoint{4.337367in}{3.261585in}}%
\pgfpathlineto{\pgfqpoint{4.347369in}{3.483866in}}%
\pgfpathlineto{\pgfqpoint{4.357126in}{3.529132in}}%
\pgfpathlineto{\pgfqpoint{4.390656in}{3.293957in}}%
\pgfpathlineto{\pgfqpoint{4.424899in}{3.428592in}}%
\pgfpathlineto{\pgfqpoint{4.414558in}{3.170075in}}%
\pgfpathlineto{\pgfqpoint{4.404996in}{3.229765in}}%
\pgfpathlineto{\pgfqpoint{4.371362in}{3.333060in}}%
\pgfpathlineto{\pgfqpoint{4.337367in}{3.261585in}}%
\pgfpathclose%
\pgfusepath{fill}%
\end{pgfscope}%
\begin{pgfscope}%
\pgfpathrectangle{\pgfqpoint{1.020000in}{0.880000in}}{\pgfqpoint{6.160000in}{6.160000in}}%
\pgfusepath{clip}%
\pgfsetbuttcap%
\pgfsetroundjoin%
\definecolor{currentfill}{rgb}{0.363461,0.484784,0.901019}%
\pgfsetfillcolor{currentfill}%
\pgfsetlinewidth{0.000000pt}%
\definecolor{currentstroke}{rgb}{0.000000,0.000000,0.000000}%
\pgfsetstrokecolor{currentstroke}%
\pgfsetdash{}{0pt}%
\pgfpathmoveto{\pgfqpoint{4.889499in}{3.048598in}}%
\pgfpathlineto{\pgfqpoint{4.899001in}{2.943137in}}%
\pgfpathlineto{\pgfqpoint{4.909394in}{2.959906in}}%
\pgfpathlineto{\pgfqpoint{4.941819in}{2.778779in}}%
\pgfpathlineto{\pgfqpoint{4.976931in}{2.952662in}}%
\pgfpathlineto{\pgfqpoint{4.966149in}{2.896425in}}%
\pgfpathlineto{\pgfqpoint{4.954713in}{2.751661in}}%
\pgfpathlineto{\pgfqpoint{4.922021in}{2.880517in}}%
\pgfpathlineto{\pgfqpoint{4.889499in}{3.048598in}}%
\pgfpathclose%
\pgfusepath{fill}%
\end{pgfscope}%
\begin{pgfscope}%
\pgfpathrectangle{\pgfqpoint{1.020000in}{0.880000in}}{\pgfqpoint{6.160000in}{6.160000in}}%
\pgfusepath{clip}%
\pgfsetbuttcap%
\pgfsetroundjoin%
\definecolor{currentfill}{rgb}{0.285273,0.380129,0.823469}%
\pgfsetfillcolor{currentfill}%
\pgfsetlinewidth{0.000000pt}%
\definecolor{currentstroke}{rgb}{0.000000,0.000000,0.000000}%
\pgfsetstrokecolor{currentstroke}%
\pgfsetdash{}{0pt}%
\pgfpathmoveto{\pgfqpoint{6.000243in}{2.658379in}}%
\pgfpathlineto{\pgfqpoint{6.011674in}{2.657311in}}%
\pgfpathlineto{\pgfqpoint{6.024258in}{2.717131in}}%
\pgfpathlineto{\pgfqpoint{6.059075in}{2.789726in}}%
\pgfpathlineto{\pgfqpoint{6.094432in}{2.889019in}}%
\pgfpathlineto{\pgfqpoint{6.079863in}{2.731886in}}%
\pgfpathlineto{\pgfqpoint{6.067262in}{2.675183in}}%
\pgfpathlineto{\pgfqpoint{6.035631in}{2.767115in}}%
\pgfpathlineto{\pgfqpoint{6.000243in}{2.658379in}}%
\pgfpathclose%
\pgfusepath{fill}%
\end{pgfscope}%
\begin{pgfscope}%
\pgfpathrectangle{\pgfqpoint{1.020000in}{0.880000in}}{\pgfqpoint{6.160000in}{6.160000in}}%
\pgfusepath{clip}%
\pgfsetbuttcap%
\pgfsetroundjoin%
\definecolor{currentfill}{rgb}{0.758539,0.832787,0.958408}%
\pgfsetfillcolor{currentfill}%
\pgfsetlinewidth{0.000000pt}%
\definecolor{currentstroke}{rgb}{0.000000,0.000000,0.000000}%
\pgfsetstrokecolor{currentstroke}%
\pgfsetdash{}{0pt}%
\pgfpathmoveto{\pgfqpoint{3.070677in}{3.534608in}}%
\pgfpathlineto{\pgfqpoint{3.076753in}{3.733332in}}%
\pgfpathlineto{\pgfqpoint{3.084331in}{3.799631in}}%
\pgfpathlineto{\pgfqpoint{3.122352in}{3.458282in}}%
\pgfpathlineto{\pgfqpoint{3.154319in}{3.682041in}}%
\pgfpathlineto{\pgfqpoint{3.147192in}{3.562041in}}%
\pgfpathlineto{\pgfqpoint{3.139028in}{3.545200in}}%
\pgfpathlineto{\pgfqpoint{3.102444in}{3.764239in}}%
\pgfpathlineto{\pgfqpoint{3.070677in}{3.534608in}}%
\pgfpathclose%
\pgfusepath{fill}%
\end{pgfscope}%
\begin{pgfscope}%
\pgfpathrectangle{\pgfqpoint{1.020000in}{0.880000in}}{\pgfqpoint{6.160000in}{6.160000in}}%
\pgfusepath{clip}%
\pgfsetbuttcap%
\pgfsetroundjoin%
\definecolor{currentfill}{rgb}{0.343278,0.459354,0.884122}%
\pgfsetfillcolor{currentfill}%
\pgfsetlinewidth{0.000000pt}%
\definecolor{currentstroke}{rgb}{0.000000,0.000000,0.000000}%
\pgfsetstrokecolor{currentstroke}%
\pgfsetdash{}{0pt}%
\pgfpathmoveto{\pgfqpoint{5.334947in}{2.942166in}}%
\pgfpathlineto{\pgfqpoint{5.344431in}{2.834510in}}%
\pgfpathlineto{\pgfqpoint{5.354787in}{2.800812in}}%
\pgfpathlineto{\pgfqpoint{5.387837in}{2.751973in}}%
\pgfpathlineto{\pgfqpoint{5.421307in}{2.741131in}}%
\pgfpathlineto{\pgfqpoint{5.412269in}{2.884565in}}%
\pgfpathlineto{\pgfqpoint{5.401643in}{2.898894in}}%
\pgfpathlineto{\pgfqpoint{5.368779in}{2.959711in}}%
\pgfpathlineto{\pgfqpoint{5.334947in}{2.942166in}}%
\pgfpathclose%
\pgfusepath{fill}%
\end{pgfscope}%
\begin{pgfscope}%
\pgfpathrectangle{\pgfqpoint{1.020000in}{0.880000in}}{\pgfqpoint{6.160000in}{6.160000in}}%
\pgfusepath{clip}%
\pgfsetbuttcap%
\pgfsetroundjoin%
\definecolor{currentfill}{rgb}{0.693321,0.796314,0.986308}%
\pgfsetfillcolor{currentfill}%
\pgfsetlinewidth{0.000000pt}%
\definecolor{currentstroke}{rgb}{0.000000,0.000000,0.000000}%
\pgfsetstrokecolor{currentstroke}%
\pgfsetdash{}{0pt}%
\pgfpathmoveto{\pgfqpoint{3.959775in}{3.656147in}}%
\pgfpathlineto{\pgfqpoint{3.969278in}{3.539649in}}%
\pgfpathlineto{\pgfqpoint{3.978576in}{3.522362in}}%
\pgfpathlineto{\pgfqpoint{4.012665in}{3.520181in}}%
\pgfpathlineto{\pgfqpoint{4.046904in}{3.392013in}}%
\pgfpathlineto{\pgfqpoint{4.037466in}{3.475109in}}%
\pgfpathlineto{\pgfqpoint{4.028259in}{3.412339in}}%
\pgfpathlineto{\pgfqpoint{3.994097in}{3.519333in}}%
\pgfpathlineto{\pgfqpoint{3.959775in}{3.656147in}}%
\pgfpathclose%
\pgfusepath{fill}%
\end{pgfscope}%
\begin{pgfscope}%
\pgfpathrectangle{\pgfqpoint{1.020000in}{0.880000in}}{\pgfqpoint{6.160000in}{6.160000in}}%
\pgfusepath{clip}%
\pgfsetbuttcap%
\pgfsetroundjoin%
\definecolor{currentfill}{rgb}{0.299441,0.400248,0.839842}%
\pgfsetfillcolor{currentfill}%
\pgfsetlinewidth{0.000000pt}%
\definecolor{currentstroke}{rgb}{0.000000,0.000000,0.000000}%
\pgfsetstrokecolor{currentstroke}%
\pgfsetdash{}{0pt}%
\pgfpathmoveto{\pgfqpoint{5.487731in}{2.686589in}}%
\pgfpathlineto{\pgfqpoint{5.499187in}{2.728964in}}%
\pgfpathlineto{\pgfqpoint{5.510360in}{2.747775in}}%
\pgfpathlineto{\pgfqpoint{5.543492in}{2.712912in}}%
\pgfpathlineto{\pgfqpoint{5.578899in}{2.843339in}}%
\pgfpathlineto{\pgfqpoint{5.567830in}{2.838471in}}%
\pgfpathlineto{\pgfqpoint{5.556381in}{2.804329in}}%
\pgfpathlineto{\pgfqpoint{5.522188in}{2.755922in}}%
\pgfpathlineto{\pgfqpoint{5.487731in}{2.686589in}}%
\pgfpathclose%
\pgfusepath{fill}%
\end{pgfscope}%
\begin{pgfscope}%
\pgfpathrectangle{\pgfqpoint{1.020000in}{0.880000in}}{\pgfqpoint{6.160000in}{6.160000in}}%
\pgfusepath{clip}%
\pgfsetbuttcap%
\pgfsetroundjoin%
\definecolor{currentfill}{rgb}{0.693321,0.796314,0.986308}%
\pgfsetfillcolor{currentfill}%
\pgfsetlinewidth{0.000000pt}%
\definecolor{currentstroke}{rgb}{0.000000,0.000000,0.000000}%
\pgfsetstrokecolor{currentstroke}%
\pgfsetdash{}{0pt}%
\pgfpathmoveto{\pgfqpoint{2.642588in}{3.529936in}}%
\pgfpathlineto{\pgfqpoint{2.648952in}{3.624854in}}%
\pgfpathlineto{\pgfqpoint{2.659363in}{3.456265in}}%
\pgfpathlineto{\pgfqpoint{2.692703in}{3.531660in}}%
\pgfpathlineto{\pgfqpoint{2.729878in}{3.345539in}}%
\pgfpathlineto{\pgfqpoint{2.720014in}{3.480526in}}%
\pgfpathlineto{\pgfqpoint{2.710993in}{3.557883in}}%
\pgfpathlineto{\pgfqpoint{2.677511in}{3.496394in}}%
\pgfpathlineto{\pgfqpoint{2.642588in}{3.529936in}}%
\pgfpathclose%
\pgfusepath{fill}%
\end{pgfscope}%
\begin{pgfscope}%
\pgfpathrectangle{\pgfqpoint{1.020000in}{0.880000in}}{\pgfqpoint{6.160000in}{6.160000in}}%
\pgfusepath{clip}%
\pgfsetbuttcap%
\pgfsetroundjoin%
\definecolor{currentfill}{rgb}{0.763363,0.835092,0.955658}%
\pgfsetfillcolor{currentfill}%
\pgfsetlinewidth{0.000000pt}%
\definecolor{currentstroke}{rgb}{0.000000,0.000000,0.000000}%
\pgfsetstrokecolor{currentstroke}%
\pgfsetdash{}{0pt}%
\pgfpathmoveto{\pgfqpoint{3.581925in}{3.762466in}}%
\pgfpathlineto{\pgfqpoint{3.591278in}{3.676229in}}%
\pgfpathlineto{\pgfqpoint{3.599836in}{3.727282in}}%
\pgfpathlineto{\pgfqpoint{3.634885in}{3.576617in}}%
\pgfpathlineto{\pgfqpoint{3.668931in}{3.597445in}}%
\pgfpathlineto{\pgfqpoint{3.660091in}{3.582808in}}%
\pgfpathlineto{\pgfqpoint{3.651211in}{3.579885in}}%
\pgfpathlineto{\pgfqpoint{3.616896in}{3.620494in}}%
\pgfpathlineto{\pgfqpoint{3.581925in}{3.762466in}}%
\pgfpathclose%
\pgfusepath{fill}%
\end{pgfscope}%
\begin{pgfscope}%
\pgfpathrectangle{\pgfqpoint{1.020000in}{0.880000in}}{\pgfqpoint{6.160000in}{6.160000in}}%
\pgfusepath{clip}%
\pgfsetbuttcap%
\pgfsetroundjoin%
\definecolor{currentfill}{rgb}{0.677823,0.786546,0.991005}%
\pgfsetfillcolor{currentfill}%
\pgfsetlinewidth{0.000000pt}%
\definecolor{currentstroke}{rgb}{0.000000,0.000000,0.000000}%
\pgfsetstrokecolor{currentstroke}%
\pgfsetdash{}{0pt}%
\pgfpathmoveto{\pgfqpoint{4.182781in}{3.552635in}}%
\pgfpathlineto{\pgfqpoint{4.192263in}{3.485869in}}%
\pgfpathlineto{\pgfqpoint{4.201763in}{3.474609in}}%
\pgfpathlineto{\pgfqpoint{4.235794in}{3.507291in}}%
\pgfpathlineto{\pgfqpoint{4.269694in}{3.386564in}}%
\pgfpathlineto{\pgfqpoint{4.260128in}{3.378438in}}%
\pgfpathlineto{\pgfqpoint{4.250663in}{3.489050in}}%
\pgfpathlineto{\pgfqpoint{4.216745in}{3.525640in}}%
\pgfpathlineto{\pgfqpoint{4.182781in}{3.552635in}}%
\pgfpathclose%
\pgfusepath{fill}%
\end{pgfscope}%
\begin{pgfscope}%
\pgfpathrectangle{\pgfqpoint{1.020000in}{0.880000in}}{\pgfqpoint{6.160000in}{6.160000in}}%
\pgfusepath{clip}%
\pgfsetbuttcap%
\pgfsetroundjoin%
\definecolor{currentfill}{rgb}{0.708720,0.805721,0.981117}%
\pgfsetfillcolor{currentfill}%
\pgfsetlinewidth{0.000000pt}%
\definecolor{currentstroke}{rgb}{0.000000,0.000000,0.000000}%
\pgfsetstrokecolor{currentstroke}%
\pgfsetdash{}{0pt}%
\pgfpathmoveto{\pgfqpoint{2.931206in}{3.718763in}}%
\pgfpathlineto{\pgfqpoint{2.941308in}{3.561044in}}%
\pgfpathlineto{\pgfqpoint{2.951488in}{3.396336in}}%
\pgfpathlineto{\pgfqpoint{2.985790in}{3.399998in}}%
\pgfpathlineto{\pgfqpoint{3.017929in}{3.587990in}}%
\pgfpathlineto{\pgfqpoint{3.008624in}{3.680407in}}%
\pgfpathlineto{\pgfqpoint{3.003198in}{3.444271in}}%
\pgfpathlineto{\pgfqpoint{2.968641in}{3.467761in}}%
\pgfpathlineto{\pgfqpoint{2.931206in}{3.718763in}}%
\pgfpathclose%
\pgfusepath{fill}%
\end{pgfscope}%
\begin{pgfscope}%
\pgfpathrectangle{\pgfqpoint{1.020000in}{0.880000in}}{\pgfqpoint{6.160000in}{6.160000in}}%
\pgfusepath{clip}%
\pgfsetbuttcap%
\pgfsetroundjoin%
\definecolor{currentfill}{rgb}{0.373552,0.497499,0.909467}%
\pgfsetfillcolor{currentfill}%
\pgfsetlinewidth{0.000000pt}%
\definecolor{currentstroke}{rgb}{0.000000,0.000000,0.000000}%
\pgfsetstrokecolor{currentstroke}%
\pgfsetdash{}{0pt}%
\pgfpathmoveto{\pgfqpoint{5.111816in}{2.947483in}}%
\pgfpathlineto{\pgfqpoint{5.120811in}{2.788420in}}%
\pgfpathlineto{\pgfqpoint{5.133507in}{3.018435in}}%
\pgfpathlineto{\pgfqpoint{5.167478in}{3.038533in}}%
\pgfpathlineto{\pgfqpoint{5.198617in}{2.782328in}}%
\pgfpathlineto{\pgfqpoint{5.189671in}{2.942671in}}%
\pgfpathlineto{\pgfqpoint{5.177627in}{2.794535in}}%
\pgfpathlineto{\pgfqpoint{5.145467in}{2.944424in}}%
\pgfpathlineto{\pgfqpoint{5.111816in}{2.947483in}}%
\pgfpathclose%
\pgfusepath{fill}%
\end{pgfscope}%
\begin{pgfscope}%
\pgfpathrectangle{\pgfqpoint{1.020000in}{0.880000in}}{\pgfqpoint{6.160000in}{6.160000in}}%
\pgfusepath{clip}%
\pgfsetbuttcap%
\pgfsetroundjoin%
\definecolor{currentfill}{rgb}{0.338377,0.452819,0.879317}%
\pgfsetfillcolor{currentfill}%
\pgfsetlinewidth{0.000000pt}%
\definecolor{currentstroke}{rgb}{0.000000,0.000000,0.000000}%
\pgfsetstrokecolor{currentstroke}%
\pgfsetdash{}{0pt}%
\pgfpathmoveto{\pgfqpoint{5.556381in}{2.804329in}}%
\pgfpathlineto{\pgfqpoint{5.567830in}{2.838471in}}%
\pgfpathlineto{\pgfqpoint{5.578899in}{2.843339in}}%
\pgfpathlineto{\pgfqpoint{5.611931in}{2.803753in}}%
\pgfpathlineto{\pgfqpoint{5.645965in}{2.835539in}}%
\pgfpathlineto{\pgfqpoint{5.634751in}{2.825500in}}%
\pgfpathlineto{\pgfqpoint{5.622572in}{2.747043in}}%
\pgfpathlineto{\pgfqpoint{5.592909in}{3.016779in}}%
\pgfpathlineto{\pgfqpoint{5.556381in}{2.804329in}}%
\pgfpathclose%
\pgfusepath{fill}%
\end{pgfscope}%
\begin{pgfscope}%
\pgfpathrectangle{\pgfqpoint{1.020000in}{0.880000in}}{\pgfqpoint{6.160000in}{6.160000in}}%
\pgfusepath{clip}%
\pgfsetbuttcap%
\pgfsetroundjoin%
\definecolor{currentfill}{rgb}{0.494638,0.633022,0.978983}%
\pgfsetfillcolor{currentfill}%
\pgfsetlinewidth{0.000000pt}%
\definecolor{currentstroke}{rgb}{0.000000,0.000000,0.000000}%
\pgfsetstrokecolor{currentstroke}%
\pgfsetdash{}{0pt}%
\pgfpathmoveto{\pgfqpoint{4.579083in}{3.105640in}}%
\pgfpathlineto{\pgfqpoint{4.589296in}{3.180543in}}%
\pgfpathlineto{\pgfqpoint{4.598959in}{3.114551in}}%
\pgfpathlineto{\pgfqpoint{4.633200in}{3.192643in}}%
\pgfpathlineto{\pgfqpoint{4.666373in}{3.043619in}}%
\pgfpathlineto{\pgfqpoint{4.657104in}{3.195690in}}%
\pgfpathlineto{\pgfqpoint{4.646793in}{3.125623in}}%
\pgfpathlineto{\pgfqpoint{4.613153in}{3.163867in}}%
\pgfpathlineto{\pgfqpoint{4.579083in}{3.105640in}}%
\pgfpathclose%
\pgfusepath{fill}%
\end{pgfscope}%
\begin{pgfscope}%
\pgfpathrectangle{\pgfqpoint{1.020000in}{0.880000in}}{\pgfqpoint{6.160000in}{6.160000in}}%
\pgfusepath{clip}%
\pgfsetbuttcap%
\pgfsetroundjoin%
\definecolor{currentfill}{rgb}{0.289996,0.386836,0.828926}%
\pgfsetfillcolor{currentfill}%
\pgfsetlinewidth{0.000000pt}%
\definecolor{currentstroke}{rgb}{0.000000,0.000000,0.000000}%
\pgfsetstrokecolor{currentstroke}%
\pgfsetdash{}{0pt}%
\pgfpathmoveto{\pgfqpoint{5.712498in}{2.799102in}}%
\pgfpathlineto{\pgfqpoint{5.723238in}{2.772450in}}%
\pgfpathlineto{\pgfqpoint{5.733175in}{2.693167in}}%
\pgfpathlineto{\pgfqpoint{5.768514in}{2.806341in}}%
\pgfpathlineto{\pgfqpoint{5.801941in}{2.798636in}}%
\pgfpathlineto{\pgfqpoint{5.789595in}{2.730501in}}%
\pgfpathlineto{\pgfqpoint{5.779363in}{2.791484in}}%
\pgfpathlineto{\pgfqpoint{5.742692in}{2.588699in}}%
\pgfpathlineto{\pgfqpoint{5.712498in}{2.799102in}}%
\pgfpathclose%
\pgfusepath{fill}%
\end{pgfscope}%
\begin{pgfscope}%
\pgfpathrectangle{\pgfqpoint{1.020000in}{0.880000in}}{\pgfqpoint{6.160000in}{6.160000in}}%
\pgfusepath{clip}%
\pgfsetbuttcap%
\pgfsetroundjoin%
\definecolor{currentfill}{rgb}{0.786721,0.844807,0.939810}%
\pgfsetfillcolor{currentfill}%
\pgfsetlinewidth{0.000000pt}%
\definecolor{currentstroke}{rgb}{0.000000,0.000000,0.000000}%
\pgfsetstrokecolor{currentstroke}%
\pgfsetdash{}{0pt}%
\pgfpathmoveto{\pgfqpoint{3.359716in}{3.707575in}}%
\pgfpathlineto{\pgfqpoint{3.367830in}{3.766799in}}%
\pgfpathlineto{\pgfqpoint{3.376916in}{3.708749in}}%
\pgfpathlineto{\pgfqpoint{3.410713in}{3.770579in}}%
\pgfpathlineto{\pgfqpoint{3.446051in}{3.625893in}}%
\pgfpathlineto{\pgfqpoint{3.437242in}{3.643960in}}%
\pgfpathlineto{\pgfqpoint{3.429179in}{3.564795in}}%
\pgfpathlineto{\pgfqpoint{3.393686in}{3.737298in}}%
\pgfpathlineto{\pgfqpoint{3.359716in}{3.707575in}}%
\pgfpathclose%
\pgfusepath{fill}%
\end{pgfscope}%
\begin{pgfscope}%
\pgfpathrectangle{\pgfqpoint{1.020000in}{0.880000in}}{\pgfqpoint{6.160000in}{6.160000in}}%
\pgfusepath{clip}%
\pgfsetbuttcap%
\pgfsetroundjoin%
\definecolor{currentfill}{rgb}{0.728970,0.817464,0.973188}%
\pgfsetfillcolor{currentfill}%
\pgfsetlinewidth{0.000000pt}%
\definecolor{currentstroke}{rgb}{0.000000,0.000000,0.000000}%
\pgfsetstrokecolor{currentstroke}%
\pgfsetdash{}{0pt}%
\pgfpathmoveto{\pgfqpoint{3.222801in}{3.696282in}}%
\pgfpathlineto{\pgfqpoint{3.232194in}{3.597629in}}%
\pgfpathlineto{\pgfqpoint{3.242090in}{3.445725in}}%
\pgfpathlineto{\pgfqpoint{3.275001in}{3.595113in}}%
\pgfpathlineto{\pgfqpoint{3.310097in}{3.503589in}}%
\pgfpathlineto{\pgfqpoint{3.300417in}{3.634370in}}%
\pgfpathlineto{\pgfqpoint{3.291961in}{3.627139in}}%
\pgfpathlineto{\pgfqpoint{3.259116in}{3.476209in}}%
\pgfpathlineto{\pgfqpoint{3.222801in}{3.696282in}}%
\pgfpathclose%
\pgfusepath{fill}%
\end{pgfscope}%
\begin{pgfscope}%
\pgfpathrectangle{\pgfqpoint{1.020000in}{0.880000in}}{\pgfqpoint{6.160000in}{6.160000in}}%
\pgfusepath{clip}%
\pgfsetbuttcap%
\pgfsetroundjoin%
\definecolor{currentfill}{rgb}{0.758539,0.832787,0.958408}%
\pgfsetfillcolor{currentfill}%
\pgfsetlinewidth{0.000000pt}%
\definecolor{currentstroke}{rgb}{0.000000,0.000000,0.000000}%
\pgfsetstrokecolor{currentstroke}%
\pgfsetdash{}{0pt}%
\pgfpathmoveto{\pgfqpoint{3.805133in}{3.658979in}}%
\pgfpathlineto{\pgfqpoint{3.814334in}{3.625514in}}%
\pgfpathlineto{\pgfqpoint{3.822902in}{3.772053in}}%
\pgfpathlineto{\pgfqpoint{3.857692in}{3.585735in}}%
\pgfpathlineto{\pgfqpoint{3.891729in}{3.607645in}}%
\pgfpathlineto{\pgfqpoint{3.882985in}{3.468890in}}%
\pgfpathlineto{\pgfqpoint{3.873458in}{3.597477in}}%
\pgfpathlineto{\pgfqpoint{3.839004in}{3.719148in}}%
\pgfpathlineto{\pgfqpoint{3.805133in}{3.658979in}}%
\pgfpathclose%
\pgfusepath{fill}%
\end{pgfscope}%
\begin{pgfscope}%
\pgfpathrectangle{\pgfqpoint{1.020000in}{0.880000in}}{\pgfqpoint{6.160000in}{6.160000in}}%
\pgfusepath{clip}%
\pgfsetbuttcap%
\pgfsetroundjoin%
\definecolor{currentfill}{rgb}{0.753611,0.830233,0.960871}%
\pgfsetfillcolor{currentfill}%
\pgfsetlinewidth{0.000000pt}%
\definecolor{currentstroke}{rgb}{0.000000,0.000000,0.000000}%
\pgfsetstrokecolor{currentstroke}%
\pgfsetdash{}{0pt}%
\pgfpathmoveto{\pgfqpoint{3.003198in}{3.444271in}}%
\pgfpathlineto{\pgfqpoint{3.008624in}{3.680407in}}%
\pgfpathlineto{\pgfqpoint{3.017929in}{3.587990in}}%
\pgfpathlineto{\pgfqpoint{3.053038in}{3.522548in}}%
\pgfpathlineto{\pgfqpoint{3.084331in}{3.799631in}}%
\pgfpathlineto{\pgfqpoint{3.076753in}{3.733332in}}%
\pgfpathlineto{\pgfqpoint{3.070677in}{3.534608in}}%
\pgfpathlineto{\pgfqpoint{3.034801in}{3.675450in}}%
\pgfpathlineto{\pgfqpoint{3.003198in}{3.444271in}}%
\pgfpathclose%
\pgfusepath{fill}%
\end{pgfscope}%
\begin{pgfscope}%
\pgfpathrectangle{\pgfqpoint{1.020000in}{0.880000in}}{\pgfqpoint{6.160000in}{6.160000in}}%
\pgfusepath{clip}%
\pgfsetbuttcap%
\pgfsetroundjoin%
\definecolor{currentfill}{rgb}{0.693321,0.796314,0.986308}%
\pgfsetfillcolor{currentfill}%
\pgfsetlinewidth{0.000000pt}%
\definecolor{currentstroke}{rgb}{0.000000,0.000000,0.000000}%
\pgfsetstrokecolor{currentstroke}%
\pgfsetdash{}{0pt}%
\pgfpathmoveto{\pgfqpoint{2.864562in}{3.552380in}}%
\pgfpathlineto{\pgfqpoint{2.875236in}{3.352843in}}%
\pgfpathlineto{\pgfqpoint{2.880266in}{3.586063in}}%
\pgfpathlineto{\pgfqpoint{2.916928in}{3.411642in}}%
\pgfpathlineto{\pgfqpoint{2.951488in}{3.396336in}}%
\pgfpathlineto{\pgfqpoint{2.941308in}{3.561044in}}%
\pgfpathlineto{\pgfqpoint{2.931206in}{3.718763in}}%
\pgfpathlineto{\pgfqpoint{2.900193in}{3.454135in}}%
\pgfpathlineto{\pgfqpoint{2.864562in}{3.552380in}}%
\pgfpathclose%
\pgfusepath{fill}%
\end{pgfscope}%
\begin{pgfscope}%
\pgfpathrectangle{\pgfqpoint{1.020000in}{0.880000in}}{\pgfqpoint{6.160000in}{6.160000in}}%
\pgfusepath{clip}%
\pgfsetbuttcap%
\pgfsetroundjoin%
\definecolor{currentfill}{rgb}{0.358415,0.478426,0.896795}%
\pgfsetfillcolor{currentfill}%
\pgfsetlinewidth{0.000000pt}%
\definecolor{currentstroke}{rgb}{0.000000,0.000000,0.000000}%
\pgfsetstrokecolor{currentstroke}%
\pgfsetdash{}{0pt}%
\pgfpathmoveto{\pgfqpoint{5.043901in}{2.890464in}}%
\pgfpathlineto{\pgfqpoint{5.055251in}{2.996352in}}%
\pgfpathlineto{\pgfqpoint{5.063563in}{2.754420in}}%
\pgfpathlineto{\pgfqpoint{5.098554in}{2.892259in}}%
\pgfpathlineto{\pgfqpoint{5.133507in}{3.018435in}}%
\pgfpathlineto{\pgfqpoint{5.120811in}{2.788420in}}%
\pgfpathlineto{\pgfqpoint{5.111816in}{2.947483in}}%
\pgfpathlineto{\pgfqpoint{5.076001in}{2.712368in}}%
\pgfpathlineto{\pgfqpoint{5.043901in}{2.890464in}}%
\pgfpathclose%
\pgfusepath{fill}%
\end{pgfscope}%
\begin{pgfscope}%
\pgfpathrectangle{\pgfqpoint{1.020000in}{0.880000in}}{\pgfqpoint{6.160000in}{6.160000in}}%
\pgfusepath{clip}%
\pgfsetbuttcap%
\pgfsetroundjoin%
\definecolor{currentfill}{rgb}{0.683056,0.790043,0.989768}%
\pgfsetfillcolor{currentfill}%
\pgfsetlinewidth{0.000000pt}%
\definecolor{currentstroke}{rgb}{0.000000,0.000000,0.000000}%
\pgfsetstrokecolor{currentstroke}%
\pgfsetdash{}{0pt}%
\pgfpathmoveto{\pgfqpoint{2.570548in}{3.725256in}}%
\pgfpathlineto{\pgfqpoint{2.583710in}{3.387900in}}%
\pgfpathlineto{\pgfqpoint{2.591954in}{3.357079in}}%
\pgfpathlineto{\pgfqpoint{2.625990in}{3.384971in}}%
\pgfpathlineto{\pgfqpoint{2.659363in}{3.456265in}}%
\pgfpathlineto{\pgfqpoint{2.648952in}{3.624854in}}%
\pgfpathlineto{\pgfqpoint{2.642588in}{3.529936in}}%
\pgfpathlineto{\pgfqpoint{2.610374in}{3.388545in}}%
\pgfpathlineto{\pgfqpoint{2.570548in}{3.725256in}}%
\pgfpathclose%
\pgfusepath{fill}%
\end{pgfscope}%
\begin{pgfscope}%
\pgfpathrectangle{\pgfqpoint{1.020000in}{0.880000in}}{\pgfqpoint{6.160000in}{6.160000in}}%
\pgfusepath{clip}%
\pgfsetbuttcap%
\pgfsetroundjoin%
\definecolor{currentfill}{rgb}{0.294718,0.393542,0.834384}%
\pgfsetfillcolor{currentfill}%
\pgfsetlinewidth{0.000000pt}%
\definecolor{currentstroke}{rgb}{0.000000,0.000000,0.000000}%
\pgfsetstrokecolor{currentstroke}%
\pgfsetdash{}{0pt}%
\pgfpathmoveto{\pgfqpoint{5.421307in}{2.741131in}}%
\pgfpathlineto{\pgfqpoint{5.432344in}{2.757390in}}%
\pgfpathlineto{\pgfqpoint{5.445008in}{2.899684in}}%
\pgfpathlineto{\pgfqpoint{5.475572in}{2.658812in}}%
\pgfpathlineto{\pgfqpoint{5.510360in}{2.747775in}}%
\pgfpathlineto{\pgfqpoint{5.499187in}{2.728964in}}%
\pgfpathlineto{\pgfqpoint{5.487731in}{2.686589in}}%
\pgfpathlineto{\pgfqpoint{5.455523in}{2.790734in}}%
\pgfpathlineto{\pgfqpoint{5.421307in}{2.741131in}}%
\pgfpathclose%
\pgfusepath{fill}%
\end{pgfscope}%
\begin{pgfscope}%
\pgfpathrectangle{\pgfqpoint{1.020000in}{0.880000in}}{\pgfqpoint{6.160000in}{6.160000in}}%
\pgfusepath{clip}%
\pgfsetbuttcap%
\pgfsetroundjoin%
\definecolor{currentfill}{rgb}{0.294718,0.393542,0.834384}%
\pgfsetfillcolor{currentfill}%
\pgfsetlinewidth{0.000000pt}%
\definecolor{currentstroke}{rgb}{0.000000,0.000000,0.000000}%
\pgfsetstrokecolor{currentstroke}%
\pgfsetdash{}{0pt}%
\pgfpathmoveto{\pgfqpoint{5.937087in}{2.863878in}}%
\pgfpathlineto{\pgfqpoint{5.946172in}{2.733037in}}%
\pgfpathlineto{\pgfqpoint{5.961363in}{2.944189in}}%
\pgfpathlineto{\pgfqpoint{5.988938in}{2.615266in}}%
\pgfpathlineto{\pgfqpoint{6.024258in}{2.717131in}}%
\pgfpathlineto{\pgfqpoint{6.011674in}{2.657311in}}%
\pgfpathlineto{\pgfqpoint{6.000243in}{2.658379in}}%
\pgfpathlineto{\pgfqpoint{5.969519in}{2.806169in}}%
\pgfpathlineto{\pgfqpoint{5.937087in}{2.863878in}}%
\pgfpathclose%
\pgfusepath{fill}%
\end{pgfscope}%
\begin{pgfscope}%
\pgfpathrectangle{\pgfqpoint{1.020000in}{0.880000in}}{\pgfqpoint{6.160000in}{6.160000in}}%
\pgfusepath{clip}%
\pgfsetbuttcap%
\pgfsetroundjoin%
\definecolor{currentfill}{rgb}{0.348323,0.465711,0.888346}%
\pgfsetfillcolor{currentfill}%
\pgfsetlinewidth{0.000000pt}%
\definecolor{currentstroke}{rgb}{0.000000,0.000000,0.000000}%
\pgfsetstrokecolor{currentstroke}%
\pgfsetdash{}{0pt}%
\pgfpathmoveto{\pgfqpoint{5.267670in}{2.944981in}}%
\pgfpathlineto{\pgfqpoint{5.278030in}{2.916335in}}%
\pgfpathlineto{\pgfqpoint{5.287013in}{2.762205in}}%
\pgfpathlineto{\pgfqpoint{5.321331in}{2.818810in}}%
\pgfpathlineto{\pgfqpoint{5.354787in}{2.800812in}}%
\pgfpathlineto{\pgfqpoint{5.344431in}{2.834510in}}%
\pgfpathlineto{\pgfqpoint{5.334947in}{2.942166in}}%
\pgfpathlineto{\pgfqpoint{5.300013in}{2.827283in}}%
\pgfpathlineto{\pgfqpoint{5.267670in}{2.944981in}}%
\pgfpathclose%
\pgfusepath{fill}%
\end{pgfscope}%
\begin{pgfscope}%
\pgfpathrectangle{\pgfqpoint{1.020000in}{0.880000in}}{\pgfqpoint{6.160000in}{6.160000in}}%
\pgfusepath{clip}%
\pgfsetbuttcap%
\pgfsetroundjoin%
\definecolor{currentfill}{rgb}{0.724041,0.814910,0.975651}%
\pgfsetfillcolor{currentfill}%
\pgfsetlinewidth{0.000000pt}%
\definecolor{currentstroke}{rgb}{0.000000,0.000000,0.000000}%
\pgfsetstrokecolor{currentstroke}%
\pgfsetdash{}{0pt}%
\pgfpathmoveto{\pgfqpoint{3.154319in}{3.682041in}}%
\pgfpathlineto{\pgfqpoint{3.164982in}{3.458341in}}%
\pgfpathlineto{\pgfqpoint{3.173411in}{3.454105in}}%
\pgfpathlineto{\pgfqpoint{3.206172in}{3.612956in}}%
\pgfpathlineto{\pgfqpoint{3.242090in}{3.445725in}}%
\pgfpathlineto{\pgfqpoint{3.232194in}{3.597629in}}%
\pgfpathlineto{\pgfqpoint{3.222801in}{3.696282in}}%
\pgfpathlineto{\pgfqpoint{3.190512in}{3.493136in}}%
\pgfpathlineto{\pgfqpoint{3.154319in}{3.682041in}}%
\pgfpathclose%
\pgfusepath{fill}%
\end{pgfscope}%
\begin{pgfscope}%
\pgfpathrectangle{\pgfqpoint{1.020000in}{0.880000in}}{\pgfqpoint{6.160000in}{6.160000in}}%
\pgfusepath{clip}%
\pgfsetbuttcap%
\pgfsetroundjoin%
\definecolor{currentfill}{rgb}{0.782049,0.842864,0.942980}%
\pgfsetfillcolor{currentfill}%
\pgfsetlinewidth{0.000000pt}%
\definecolor{currentstroke}{rgb}{0.000000,0.000000,0.000000}%
\pgfsetstrokecolor{currentstroke}%
\pgfsetdash{}{0pt}%
\pgfpathmoveto{\pgfqpoint{3.291961in}{3.627139in}}%
\pgfpathlineto{\pgfqpoint{3.300417in}{3.634370in}}%
\pgfpathlineto{\pgfqpoint{3.310097in}{3.503589in}}%
\pgfpathlineto{\pgfqpoint{3.342733in}{3.695410in}}%
\pgfpathlineto{\pgfqpoint{3.376916in}{3.708749in}}%
\pgfpathlineto{\pgfqpoint{3.367830in}{3.766799in}}%
\pgfpathlineto{\pgfqpoint{3.359716in}{3.707575in}}%
\pgfpathlineto{\pgfqpoint{3.324960in}{3.769602in}}%
\pgfpathlineto{\pgfqpoint{3.291961in}{3.627139in}}%
\pgfpathclose%
\pgfusepath{fill}%
\end{pgfscope}%
\begin{pgfscope}%
\pgfpathrectangle{\pgfqpoint{1.020000in}{0.880000in}}{\pgfqpoint{6.160000in}{6.160000in}}%
\pgfusepath{clip}%
\pgfsetbuttcap%
\pgfsetroundjoin%
\definecolor{currentfill}{rgb}{0.586921,0.718121,0.998874}%
\pgfsetfillcolor{currentfill}%
\pgfsetlinewidth{0.000000pt}%
\definecolor{currentstroke}{rgb}{0.000000,0.000000,0.000000}%
\pgfsetstrokecolor{currentstroke}%
\pgfsetdash{}{0pt}%
\pgfpathmoveto{\pgfqpoint{4.424899in}{3.428592in}}%
\pgfpathlineto{\pgfqpoint{4.433805in}{3.087926in}}%
\pgfpathlineto{\pgfqpoint{4.444470in}{3.437998in}}%
\pgfpathlineto{\pgfqpoint{4.478172in}{3.332153in}}%
\pgfpathlineto{\pgfqpoint{4.511771in}{3.223483in}}%
\pgfpathlineto{\pgfqpoint{4.502101in}{3.273733in}}%
\pgfpathlineto{\pgfqpoint{4.492270in}{3.268841in}}%
\pgfpathlineto{\pgfqpoint{4.458561in}{3.326524in}}%
\pgfpathlineto{\pgfqpoint{4.424899in}{3.428592in}}%
\pgfpathclose%
\pgfusepath{fill}%
\end{pgfscope}%
\begin{pgfscope}%
\pgfpathrectangle{\pgfqpoint{1.020000in}{0.880000in}}{\pgfqpoint{6.160000in}{6.160000in}}%
\pgfusepath{clip}%
\pgfsetbuttcap%
\pgfsetroundjoin%
\definecolor{currentfill}{rgb}{0.483854,0.622050,0.974808}%
\pgfsetfillcolor{currentfill}%
\pgfsetlinewidth{0.000000pt}%
\definecolor{currentstroke}{rgb}{0.000000,0.000000,0.000000}%
\pgfsetstrokecolor{currentstroke}%
\pgfsetdash{}{0pt}%
\pgfpathmoveto{\pgfqpoint{4.734302in}{3.097932in}}%
\pgfpathlineto{\pgfqpoint{4.744911in}{3.190929in}}%
\pgfpathlineto{\pgfqpoint{4.754405in}{3.082095in}}%
\pgfpathlineto{\pgfqpoint{4.788593in}{3.135498in}}%
\pgfpathlineto{\pgfqpoint{4.821335in}{2.960973in}}%
\pgfpathlineto{\pgfqpoint{4.811367in}{2.991564in}}%
\pgfpathlineto{\pgfqpoint{4.801920in}{3.103639in}}%
\pgfpathlineto{\pgfqpoint{4.769265in}{3.295929in}}%
\pgfpathlineto{\pgfqpoint{4.734302in}{3.097932in}}%
\pgfpathclose%
\pgfusepath{fill}%
\end{pgfscope}%
\begin{pgfscope}%
\pgfpathrectangle{\pgfqpoint{1.020000in}{0.880000in}}{\pgfqpoint{6.160000in}{6.160000in}}%
\pgfusepath{clip}%
\pgfsetbuttcap%
\pgfsetroundjoin%
\definecolor{currentfill}{rgb}{0.299441,0.400248,0.839842}%
\pgfsetfillcolor{currentfill}%
\pgfsetlinewidth{0.000000pt}%
\definecolor{currentstroke}{rgb}{0.000000,0.000000,0.000000}%
\pgfsetstrokecolor{currentstroke}%
\pgfsetdash{}{0pt}%
\pgfpathmoveto{\pgfqpoint{5.645965in}{2.835539in}}%
\pgfpathlineto{\pgfqpoint{5.654684in}{2.675547in}}%
\pgfpathlineto{\pgfqpoint{5.666807in}{2.744217in}}%
\pgfpathlineto{\pgfqpoint{5.701400in}{2.809692in}}%
\pgfpathlineto{\pgfqpoint{5.733175in}{2.693167in}}%
\pgfpathlineto{\pgfqpoint{5.723238in}{2.772450in}}%
\pgfpathlineto{\pgfqpoint{5.712498in}{2.799102in}}%
\pgfpathlineto{\pgfqpoint{5.677661in}{2.711914in}}%
\pgfpathlineto{\pgfqpoint{5.645965in}{2.835539in}}%
\pgfpathclose%
\pgfusepath{fill}%
\end{pgfscope}%
\begin{pgfscope}%
\pgfpathrectangle{\pgfqpoint{1.020000in}{0.880000in}}{\pgfqpoint{6.160000in}{6.160000in}}%
\pgfusepath{clip}%
\pgfsetbuttcap%
\pgfsetroundjoin%
\definecolor{currentfill}{rgb}{0.289996,0.386836,0.828926}%
\pgfsetfillcolor{currentfill}%
\pgfsetlinewidth{0.000000pt}%
\definecolor{currentstroke}{rgb}{0.000000,0.000000,0.000000}%
\pgfsetstrokecolor{currentstroke}%
\pgfsetdash{}{0pt}%
\pgfpathmoveto{\pgfqpoint{6.159819in}{2.812671in}}%
\pgfpathlineto{\pgfqpoint{6.166736in}{2.576064in}}%
\pgfpathlineto{\pgfqpoint{6.182390in}{2.775130in}}%
\pgfpathlineto{\pgfqpoint{6.211906in}{2.582887in}}%
\pgfpathlineto{\pgfqpoint{6.205464in}{2.840425in}}%
\pgfpathlineto{\pgfqpoint{6.192879in}{2.796033in}}%
\pgfpathlineto{\pgfqpoint{6.159819in}{2.812671in}}%
\pgfpathclose%
\pgfusepath{fill}%
\end{pgfscope}%
\begin{pgfscope}%
\pgfpathrectangle{\pgfqpoint{1.020000in}{0.880000in}}{\pgfqpoint{6.160000in}{6.160000in}}%
\pgfusepath{clip}%
\pgfsetbuttcap%
\pgfsetroundjoin%
\definecolor{currentfill}{rgb}{0.693321,0.796314,0.986308}%
\pgfsetfillcolor{currentfill}%
\pgfsetlinewidth{0.000000pt}%
\definecolor{currentstroke}{rgb}{0.000000,0.000000,0.000000}%
\pgfsetstrokecolor{currentstroke}%
\pgfsetdash{}{0pt}%
\pgfpathmoveto{\pgfqpoint{4.114859in}{3.427988in}}%
\pgfpathlineto{\pgfqpoint{4.124200in}{3.543526in}}%
\pgfpathlineto{\pgfqpoint{4.133675in}{3.471009in}}%
\pgfpathlineto{\pgfqpoint{4.167730in}{3.465686in}}%
\pgfpathlineto{\pgfqpoint{4.201763in}{3.474609in}}%
\pgfpathlineto{\pgfqpoint{4.192263in}{3.485869in}}%
\pgfpathlineto{\pgfqpoint{4.182781in}{3.552635in}}%
\pgfpathlineto{\pgfqpoint{4.148780in}{3.574033in}}%
\pgfpathlineto{\pgfqpoint{4.114859in}{3.427988in}}%
\pgfpathclose%
\pgfusepath{fill}%
\end{pgfscope}%
\begin{pgfscope}%
\pgfpathrectangle{\pgfqpoint{1.020000in}{0.880000in}}{\pgfqpoint{6.160000in}{6.160000in}}%
\pgfusepath{clip}%
\pgfsetbuttcap%
\pgfsetroundjoin%
\definecolor{currentfill}{rgb}{0.768034,0.837035,0.952488}%
\pgfsetfillcolor{currentfill}%
\pgfsetlinewidth{0.000000pt}%
\definecolor{currentstroke}{rgb}{0.000000,0.000000,0.000000}%
\pgfsetstrokecolor{currentstroke}%
\pgfsetdash{}{0pt}%
\pgfpathmoveto{\pgfqpoint{3.737153in}{3.605526in}}%
\pgfpathlineto{\pgfqpoint{3.746166in}{3.603658in}}%
\pgfpathlineto{\pgfqpoint{3.755060in}{3.634011in}}%
\pgfpathlineto{\pgfqpoint{3.789731in}{3.508999in}}%
\pgfpathlineto{\pgfqpoint{3.822902in}{3.772053in}}%
\pgfpathlineto{\pgfqpoint{3.814334in}{3.625514in}}%
\pgfpathlineto{\pgfqpoint{3.805133in}{3.658979in}}%
\pgfpathlineto{\pgfqpoint{3.770712in}{3.737409in}}%
\pgfpathlineto{\pgfqpoint{3.737153in}{3.605526in}}%
\pgfpathclose%
\pgfusepath{fill}%
\end{pgfscope}%
\begin{pgfscope}%
\pgfpathrectangle{\pgfqpoint{1.020000in}{0.880000in}}{\pgfqpoint{6.160000in}{6.160000in}}%
\pgfusepath{clip}%
\pgfsetbuttcap%
\pgfsetroundjoin%
\definecolor{currentfill}{rgb}{0.688188,0.793178,0.988038}%
\pgfsetfillcolor{currentfill}%
\pgfsetlinewidth{0.000000pt}%
\definecolor{currentstroke}{rgb}{0.000000,0.000000,0.000000}%
\pgfsetstrokecolor{currentstroke}%
\pgfsetdash{}{0pt}%
\pgfpathmoveto{\pgfqpoint{2.503691in}{3.590507in}}%
\pgfpathlineto{\pgfqpoint{2.513993in}{3.433609in}}%
\pgfpathlineto{\pgfqpoint{2.519547in}{3.561963in}}%
\pgfpathlineto{\pgfqpoint{2.556835in}{3.395716in}}%
\pgfpathlineto{\pgfqpoint{2.591954in}{3.357079in}}%
\pgfpathlineto{\pgfqpoint{2.583710in}{3.387900in}}%
\pgfpathlineto{\pgfqpoint{2.570548in}{3.725256in}}%
\pgfpathlineto{\pgfqpoint{2.540908in}{3.426425in}}%
\pgfpathlineto{\pgfqpoint{2.503691in}{3.590507in}}%
\pgfpathclose%
\pgfusepath{fill}%
\end{pgfscope}%
\begin{pgfscope}%
\pgfpathrectangle{\pgfqpoint{1.020000in}{0.880000in}}{\pgfqpoint{6.160000in}{6.160000in}}%
\pgfusepath{clip}%
\pgfsetbuttcap%
\pgfsetroundjoin%
\definecolor{currentfill}{rgb}{0.656683,0.771806,0.994914}%
\pgfsetfillcolor{currentfill}%
\pgfsetlinewidth{0.000000pt}%
\definecolor{currentstroke}{rgb}{0.000000,0.000000,0.000000}%
\pgfsetstrokecolor{currentstroke}%
\pgfsetdash{}{0pt}%
\pgfpathmoveto{\pgfqpoint{4.269694in}{3.386564in}}%
\pgfpathlineto{\pgfqpoint{4.279358in}{3.472244in}}%
\pgfpathlineto{\pgfqpoint{4.288815in}{3.327427in}}%
\pgfpathlineto{\pgfqpoint{4.323010in}{3.473673in}}%
\pgfpathlineto{\pgfqpoint{4.357126in}{3.529132in}}%
\pgfpathlineto{\pgfqpoint{4.347369in}{3.483866in}}%
\pgfpathlineto{\pgfqpoint{4.337367in}{3.261585in}}%
\pgfpathlineto{\pgfqpoint{4.303687in}{3.427420in}}%
\pgfpathlineto{\pgfqpoint{4.269694in}{3.386564in}}%
\pgfpathclose%
\pgfusepath{fill}%
\end{pgfscope}%
\begin{pgfscope}%
\pgfpathrectangle{\pgfqpoint{1.020000in}{0.880000in}}{\pgfqpoint{6.160000in}{6.160000in}}%
\pgfusepath{clip}%
\pgfsetbuttcap%
\pgfsetroundjoin%
\definecolor{currentfill}{rgb}{0.748682,0.827679,0.963334}%
\pgfsetfillcolor{currentfill}%
\pgfsetlinewidth{0.000000pt}%
\definecolor{currentstroke}{rgb}{0.000000,0.000000,0.000000}%
\pgfsetstrokecolor{currentstroke}%
\pgfsetdash{}{0pt}%
\pgfpathmoveto{\pgfqpoint{3.668931in}{3.597445in}}%
\pgfpathlineto{\pgfqpoint{3.677626in}{3.645179in}}%
\pgfpathlineto{\pgfqpoint{3.687119in}{3.537387in}}%
\pgfpathlineto{\pgfqpoint{3.720625in}{3.685183in}}%
\pgfpathlineto{\pgfqpoint{3.755060in}{3.634011in}}%
\pgfpathlineto{\pgfqpoint{3.746166in}{3.603658in}}%
\pgfpathlineto{\pgfqpoint{3.737153in}{3.605526in}}%
\pgfpathlineto{\pgfqpoint{3.703440in}{3.520300in}}%
\pgfpathlineto{\pgfqpoint{3.668931in}{3.597445in}}%
\pgfpathclose%
\pgfusepath{fill}%
\end{pgfscope}%
\begin{pgfscope}%
\pgfpathrectangle{\pgfqpoint{1.020000in}{0.880000in}}{\pgfqpoint{6.160000in}{6.160000in}}%
\pgfusepath{clip}%
\pgfsetbuttcap%
\pgfsetroundjoin%
\definecolor{currentfill}{rgb}{0.796064,0.848693,0.933471}%
\pgfsetfillcolor{currentfill}%
\pgfsetlinewidth{0.000000pt}%
\definecolor{currentstroke}{rgb}{0.000000,0.000000,0.000000}%
\pgfsetstrokecolor{currentstroke}%
\pgfsetdash{}{0pt}%
\pgfpathmoveto{\pgfqpoint{3.514096in}{3.677288in}}%
\pgfpathlineto{\pgfqpoint{3.523197in}{3.626578in}}%
\pgfpathlineto{\pgfqpoint{3.531957in}{3.629920in}}%
\pgfpathlineto{\pgfqpoint{3.565288in}{3.776379in}}%
\pgfpathlineto{\pgfqpoint{3.599836in}{3.727282in}}%
\pgfpathlineto{\pgfqpoint{3.591278in}{3.676229in}}%
\pgfpathlineto{\pgfqpoint{3.581925in}{3.762466in}}%
\pgfpathlineto{\pgfqpoint{3.547956in}{3.727850in}}%
\pgfpathlineto{\pgfqpoint{3.514096in}{3.677288in}}%
\pgfpathclose%
\pgfusepath{fill}%
\end{pgfscope}%
\begin{pgfscope}%
\pgfpathrectangle{\pgfqpoint{1.020000in}{0.880000in}}{\pgfqpoint{6.160000in}{6.160000in}}%
\pgfusepath{clip}%
\pgfsetbuttcap%
\pgfsetroundjoin%
\definecolor{currentfill}{rgb}{0.733898,0.820018,0.970724}%
\pgfsetfillcolor{currentfill}%
\pgfsetlinewidth{0.000000pt}%
\definecolor{currentstroke}{rgb}{0.000000,0.000000,0.000000}%
\pgfsetstrokecolor{currentstroke}%
\pgfsetdash{}{0pt}%
\pgfpathmoveto{\pgfqpoint{3.891729in}{3.607645in}}%
\pgfpathlineto{\pgfqpoint{3.900865in}{3.623532in}}%
\pgfpathlineto{\pgfqpoint{3.910331in}{3.526690in}}%
\pgfpathlineto{\pgfqpoint{3.944328in}{3.581502in}}%
\pgfpathlineto{\pgfqpoint{3.978576in}{3.522362in}}%
\pgfpathlineto{\pgfqpoint{3.969278in}{3.539649in}}%
\pgfpathlineto{\pgfqpoint{3.959775in}{3.656147in}}%
\pgfpathlineto{\pgfqpoint{3.926024in}{3.526899in}}%
\pgfpathlineto{\pgfqpoint{3.891729in}{3.607645in}}%
\pgfpathclose%
\pgfusepath{fill}%
\end{pgfscope}%
\begin{pgfscope}%
\pgfpathrectangle{\pgfqpoint{1.020000in}{0.880000in}}{\pgfqpoint{6.160000in}{6.160000in}}%
\pgfusepath{clip}%
\pgfsetbuttcap%
\pgfsetroundjoin%
\definecolor{currentfill}{rgb}{0.358415,0.478426,0.896795}%
\pgfsetfillcolor{currentfill}%
\pgfsetlinewidth{0.000000pt}%
\definecolor{currentstroke}{rgb}{0.000000,0.000000,0.000000}%
\pgfsetstrokecolor{currentstroke}%
\pgfsetdash{}{0pt}%
\pgfpathmoveto{\pgfqpoint{5.198617in}{2.782328in}}%
\pgfpathlineto{\pgfqpoint{5.210373in}{2.895184in}}%
\pgfpathlineto{\pgfqpoint{5.220409in}{2.838564in}}%
\pgfpathlineto{\pgfqpoint{5.255124in}{2.929404in}}%
\pgfpathlineto{\pgfqpoint{5.287013in}{2.762205in}}%
\pgfpathlineto{\pgfqpoint{5.278030in}{2.916335in}}%
\pgfpathlineto{\pgfqpoint{5.267670in}{2.944981in}}%
\pgfpathlineto{\pgfqpoint{5.233299in}{2.879982in}}%
\pgfpathlineto{\pgfqpoint{5.198617in}{2.782328in}}%
\pgfpathclose%
\pgfusepath{fill}%
\end{pgfscope}%
\begin{pgfscope}%
\pgfpathrectangle{\pgfqpoint{1.020000in}{0.880000in}}{\pgfqpoint{6.160000in}{6.160000in}}%
\pgfusepath{clip}%
\pgfsetbuttcap%
\pgfsetroundjoin%
\definecolor{currentfill}{rgb}{0.718985,0.811993,0.977656}%
\pgfsetfillcolor{currentfill}%
\pgfsetlinewidth{0.000000pt}%
\definecolor{currentstroke}{rgb}{0.000000,0.000000,0.000000}%
\pgfsetstrokecolor{currentstroke}%
\pgfsetdash{}{0pt}%
\pgfpathmoveto{\pgfqpoint{2.794991in}{3.607672in}}%
\pgfpathlineto{\pgfqpoint{2.803733in}{3.553586in}}%
\pgfpathlineto{\pgfqpoint{2.811525in}{3.569572in}}%
\pgfpathlineto{\pgfqpoint{2.847991in}{3.421492in}}%
\pgfpathlineto{\pgfqpoint{2.880266in}{3.586063in}}%
\pgfpathlineto{\pgfqpoint{2.875236in}{3.352843in}}%
\pgfpathlineto{\pgfqpoint{2.864562in}{3.552380in}}%
\pgfpathlineto{\pgfqpoint{2.828933in}{3.643894in}}%
\pgfpathlineto{\pgfqpoint{2.794991in}{3.607672in}}%
\pgfpathclose%
\pgfusepath{fill}%
\end{pgfscope}%
\begin{pgfscope}%
\pgfpathrectangle{\pgfqpoint{1.020000in}{0.880000in}}{\pgfqpoint{6.160000in}{6.160000in}}%
\pgfusepath{clip}%
\pgfsetbuttcap%
\pgfsetroundjoin%
\definecolor{currentfill}{rgb}{0.430507,0.564883,0.948889}%
\pgfsetfillcolor{currentfill}%
\pgfsetlinewidth{0.000000pt}%
\definecolor{currentstroke}{rgb}{0.000000,0.000000,0.000000}%
\pgfsetstrokecolor{currentstroke}%
\pgfsetdash{}{0pt}%
\pgfpathmoveto{\pgfqpoint{4.821335in}{2.960973in}}%
\pgfpathlineto{\pgfqpoint{4.831612in}{2.975336in}}%
\pgfpathlineto{\pgfqpoint{4.842081in}{3.014899in}}%
\pgfpathlineto{\pgfqpoint{4.877047in}{3.172803in}}%
\pgfpathlineto{\pgfqpoint{4.909394in}{2.959906in}}%
\pgfpathlineto{\pgfqpoint{4.899001in}{2.943137in}}%
\pgfpathlineto{\pgfqpoint{4.889499in}{3.048598in}}%
\pgfpathlineto{\pgfqpoint{4.855065in}{2.953234in}}%
\pgfpathlineto{\pgfqpoint{4.821335in}{2.960973in}}%
\pgfpathclose%
\pgfusepath{fill}%
\end{pgfscope}%
\begin{pgfscope}%
\pgfpathrectangle{\pgfqpoint{1.020000in}{0.880000in}}{\pgfqpoint{6.160000in}{6.160000in}}%
\pgfusepath{clip}%
\pgfsetbuttcap%
\pgfsetroundjoin%
\definecolor{currentfill}{rgb}{0.532568,0.669801,0.990393}%
\pgfsetfillcolor{currentfill}%
\pgfsetlinewidth{0.000000pt}%
\definecolor{currentstroke}{rgb}{0.000000,0.000000,0.000000}%
\pgfsetstrokecolor{currentstroke}%
\pgfsetdash{}{0pt}%
\pgfpathmoveto{\pgfqpoint{4.511771in}{3.223483in}}%
\pgfpathlineto{\pgfqpoint{4.521826in}{3.282953in}}%
\pgfpathlineto{\pgfqpoint{4.531441in}{3.207001in}}%
\pgfpathlineto{\pgfqpoint{4.565476in}{3.226073in}}%
\pgfpathlineto{\pgfqpoint{4.598959in}{3.114551in}}%
\pgfpathlineto{\pgfqpoint{4.589296in}{3.180543in}}%
\pgfpathlineto{\pgfqpoint{4.579083in}{3.105640in}}%
\pgfpathlineto{\pgfqpoint{4.545697in}{3.230676in}}%
\pgfpathlineto{\pgfqpoint{4.511771in}{3.223483in}}%
\pgfpathclose%
\pgfusepath{fill}%
\end{pgfscope}%
\begin{pgfscope}%
\pgfpathrectangle{\pgfqpoint{1.020000in}{0.880000in}}{\pgfqpoint{6.160000in}{6.160000in}}%
\pgfusepath{clip}%
\pgfsetbuttcap%
\pgfsetroundjoin%
\definecolor{currentfill}{rgb}{0.733898,0.820018,0.970724}%
\pgfsetfillcolor{currentfill}%
\pgfsetlinewidth{0.000000pt}%
\definecolor{currentstroke}{rgb}{0.000000,0.000000,0.000000}%
\pgfsetstrokecolor{currentstroke}%
\pgfsetdash{}{0pt}%
\pgfpathmoveto{\pgfqpoint{3.084331in}{3.799631in}}%
\pgfpathlineto{\pgfqpoint{3.095246in}{3.561366in}}%
\pgfpathlineto{\pgfqpoint{3.102706in}{3.641654in}}%
\pgfpathlineto{\pgfqpoint{3.138847in}{3.476361in}}%
\pgfpathlineto{\pgfqpoint{3.173411in}{3.454105in}}%
\pgfpathlineto{\pgfqpoint{3.164982in}{3.458341in}}%
\pgfpathlineto{\pgfqpoint{3.154319in}{3.682041in}}%
\pgfpathlineto{\pgfqpoint{3.122352in}{3.458282in}}%
\pgfpathlineto{\pgfqpoint{3.084331in}{3.799631in}}%
\pgfpathclose%
\pgfusepath{fill}%
\end{pgfscope}%
\begin{pgfscope}%
\pgfpathrectangle{\pgfqpoint{1.020000in}{0.880000in}}{\pgfqpoint{6.160000in}{6.160000in}}%
\pgfusepath{clip}%
\pgfsetbuttcap%
\pgfsetroundjoin%
\definecolor{currentfill}{rgb}{0.777378,0.840921,0.946149}%
\pgfsetfillcolor{currentfill}%
\pgfsetlinewidth{0.000000pt}%
\definecolor{currentstroke}{rgb}{0.000000,0.000000,0.000000}%
\pgfsetstrokecolor{currentstroke}%
\pgfsetdash{}{0pt}%
\pgfpathmoveto{\pgfqpoint{3.446051in}{3.625893in}}%
\pgfpathlineto{\pgfqpoint{3.453179in}{3.841328in}}%
\pgfpathlineto{\pgfqpoint{3.464859in}{3.430919in}}%
\pgfpathlineto{\pgfqpoint{3.497040in}{3.726922in}}%
\pgfpathlineto{\pgfqpoint{3.531957in}{3.629920in}}%
\pgfpathlineto{\pgfqpoint{3.523197in}{3.626578in}}%
\pgfpathlineto{\pgfqpoint{3.514096in}{3.677288in}}%
\pgfpathlineto{\pgfqpoint{3.479825in}{3.687265in}}%
\pgfpathlineto{\pgfqpoint{3.446051in}{3.625893in}}%
\pgfpathclose%
\pgfusepath{fill}%
\end{pgfscope}%
\begin{pgfscope}%
\pgfpathrectangle{\pgfqpoint{1.020000in}{0.880000in}}{\pgfqpoint{6.160000in}{6.160000in}}%
\pgfusepath{clip}%
\pgfsetbuttcap%
\pgfsetroundjoin%
\definecolor{currentfill}{rgb}{0.661968,0.775491,0.993937}%
\pgfsetfillcolor{currentfill}%
\pgfsetlinewidth{0.000000pt}%
\definecolor{currentstroke}{rgb}{0.000000,0.000000,0.000000}%
\pgfsetstrokecolor{currentstroke}%
\pgfsetdash{}{0pt}%
\pgfpathmoveto{\pgfqpoint{2.659363in}{3.456265in}}%
\pgfpathlineto{\pgfqpoint{2.669418in}{3.310474in}}%
\pgfpathlineto{\pgfqpoint{2.675468in}{3.429978in}}%
\pgfpathlineto{\pgfqpoint{2.710814in}{3.374121in}}%
\pgfpathlineto{\pgfqpoint{2.745011in}{3.394085in}}%
\pgfpathlineto{\pgfqpoint{2.734564in}{3.568175in}}%
\pgfpathlineto{\pgfqpoint{2.729878in}{3.345539in}}%
\pgfpathlineto{\pgfqpoint{2.692703in}{3.531660in}}%
\pgfpathlineto{\pgfqpoint{2.659363in}{3.456265in}}%
\pgfpathclose%
\pgfusepath{fill}%
\end{pgfscope}%
\begin{pgfscope}%
\pgfpathrectangle{\pgfqpoint{1.020000in}{0.880000in}}{\pgfqpoint{6.160000in}{6.160000in}}%
\pgfusepath{clip}%
\pgfsetbuttcap%
\pgfsetroundjoin%
\definecolor{currentfill}{rgb}{0.294718,0.393542,0.834384}%
\pgfsetfillcolor{currentfill}%
\pgfsetlinewidth{0.000000pt}%
\definecolor{currentstroke}{rgb}{0.000000,0.000000,0.000000}%
\pgfsetstrokecolor{currentstroke}%
\pgfsetdash{}{0pt}%
\pgfpathmoveto{\pgfqpoint{6.094432in}{2.889019in}}%
\pgfpathlineto{\pgfqpoint{6.101594in}{2.659797in}}%
\pgfpathlineto{\pgfqpoint{6.112136in}{2.605801in}}%
\pgfpathlineto{\pgfqpoint{6.151535in}{2.907303in}}%
\pgfpathlineto{\pgfqpoint{6.182390in}{2.775130in}}%
\pgfpathlineto{\pgfqpoint{6.166736in}{2.576064in}}%
\pgfpathlineto{\pgfqpoint{6.159819in}{2.812671in}}%
\pgfpathlineto{\pgfqpoint{6.124291in}{2.704703in}}%
\pgfpathlineto{\pgfqpoint{6.094432in}{2.889019in}}%
\pgfpathclose%
\pgfusepath{fill}%
\end{pgfscope}%
\begin{pgfscope}%
\pgfpathrectangle{\pgfqpoint{1.020000in}{0.880000in}}{\pgfqpoint{6.160000in}{6.160000in}}%
\pgfusepath{clip}%
\pgfsetbuttcap%
\pgfsetroundjoin%
\definecolor{currentfill}{rgb}{0.313946,0.420052,0.854993}%
\pgfsetfillcolor{currentfill}%
\pgfsetlinewidth{0.000000pt}%
\definecolor{currentstroke}{rgb}{0.000000,0.000000,0.000000}%
\pgfsetstrokecolor{currentstroke}%
\pgfsetdash{}{0pt}%
\pgfpathmoveto{\pgfqpoint{5.868783in}{2.787086in}}%
\pgfpathlineto{\pgfqpoint{5.880984in}{2.837884in}}%
\pgfpathlineto{\pgfqpoint{5.888984in}{2.642224in}}%
\pgfpathlineto{\pgfqpoint{5.922815in}{2.660958in}}%
\pgfpathlineto{\pgfqpoint{5.961363in}{2.944189in}}%
\pgfpathlineto{\pgfqpoint{5.946172in}{2.733037in}}%
\pgfpathlineto{\pgfqpoint{5.937087in}{2.863878in}}%
\pgfpathlineto{\pgfqpoint{5.901532in}{2.744550in}}%
\pgfpathlineto{\pgfqpoint{5.868783in}{2.787086in}}%
\pgfpathclose%
\pgfusepath{fill}%
\end{pgfscope}%
\begin{pgfscope}%
\pgfpathrectangle{\pgfqpoint{1.020000in}{0.880000in}}{\pgfqpoint{6.160000in}{6.160000in}}%
\pgfusepath{clip}%
\pgfsetbuttcap%
\pgfsetroundjoin%
\definecolor{currentfill}{rgb}{0.494638,0.633022,0.978983}%
\pgfsetfillcolor{currentfill}%
\pgfsetlinewidth{0.000000pt}%
\definecolor{currentstroke}{rgb}{0.000000,0.000000,0.000000}%
\pgfsetstrokecolor{currentstroke}%
\pgfsetdash{}{0pt}%
\pgfpathmoveto{\pgfqpoint{4.666373in}{3.043619in}}%
\pgfpathlineto{\pgfqpoint{4.676915in}{3.150520in}}%
\pgfpathlineto{\pgfqpoint{4.686723in}{3.103465in}}%
\pgfpathlineto{\pgfqpoint{4.720918in}{3.155938in}}%
\pgfpathlineto{\pgfqpoint{4.754405in}{3.082095in}}%
\pgfpathlineto{\pgfqpoint{4.744911in}{3.190929in}}%
\pgfpathlineto{\pgfqpoint{4.734302in}{3.097932in}}%
\pgfpathlineto{\pgfqpoint{4.700583in}{3.118007in}}%
\pgfpathlineto{\pgfqpoint{4.666373in}{3.043619in}}%
\pgfpathclose%
\pgfusepath{fill}%
\end{pgfscope}%
\begin{pgfscope}%
\pgfpathrectangle{\pgfqpoint{1.020000in}{0.880000in}}{\pgfqpoint{6.160000in}{6.160000in}}%
\pgfusepath{clip}%
\pgfsetbuttcap%
\pgfsetroundjoin%
\definecolor{currentfill}{rgb}{0.635474,0.756714,0.998297}%
\pgfsetfillcolor{currentfill}%
\pgfsetlinewidth{0.000000pt}%
\definecolor{currentstroke}{rgb}{0.000000,0.000000,0.000000}%
\pgfsetstrokecolor{currentstroke}%
\pgfsetdash{}{0pt}%
\pgfpathmoveto{\pgfqpoint{2.591954in}{3.357079in}}%
\pgfpathlineto{\pgfqpoint{2.597407in}{3.503172in}}%
\pgfpathlineto{\pgfqpoint{2.608072in}{3.321275in}}%
\pgfpathlineto{\pgfqpoint{2.643704in}{3.249691in}}%
\pgfpathlineto{\pgfqpoint{2.675468in}{3.429978in}}%
\pgfpathlineto{\pgfqpoint{2.669418in}{3.310474in}}%
\pgfpathlineto{\pgfqpoint{2.659363in}{3.456265in}}%
\pgfpathlineto{\pgfqpoint{2.625990in}{3.384971in}}%
\pgfpathlineto{\pgfqpoint{2.591954in}{3.357079in}}%
\pgfpathclose%
\pgfusepath{fill}%
\end{pgfscope}%
\begin{pgfscope}%
\pgfpathrectangle{\pgfqpoint{1.020000in}{0.880000in}}{\pgfqpoint{6.160000in}{6.160000in}}%
\pgfusepath{clip}%
\pgfsetbuttcap%
\pgfsetroundjoin%
\definecolor{currentfill}{rgb}{0.323718,0.433158,0.864722}%
\pgfsetfillcolor{currentfill}%
\pgfsetlinewidth{0.000000pt}%
\definecolor{currentstroke}{rgb}{0.000000,0.000000,0.000000}%
\pgfsetstrokecolor{currentstroke}%
\pgfsetdash{}{0pt}%
\pgfpathmoveto{\pgfqpoint{5.354787in}{2.800812in}}%
\pgfpathlineto{\pgfqpoint{5.363952in}{2.665625in}}%
\pgfpathlineto{\pgfqpoint{5.378244in}{2.957953in}}%
\pgfpathlineto{\pgfqpoint{5.410129in}{2.805904in}}%
\pgfpathlineto{\pgfqpoint{5.445008in}{2.899684in}}%
\pgfpathlineto{\pgfqpoint{5.432344in}{2.757390in}}%
\pgfpathlineto{\pgfqpoint{5.421307in}{2.741131in}}%
\pgfpathlineto{\pgfqpoint{5.387837in}{2.751973in}}%
\pgfpathlineto{\pgfqpoint{5.354787in}{2.800812in}}%
\pgfpathclose%
\pgfusepath{fill}%
\end{pgfscope}%
\begin{pgfscope}%
\pgfpathrectangle{\pgfqpoint{1.020000in}{0.880000in}}{\pgfqpoint{6.160000in}{6.160000in}}%
\pgfusepath{clip}%
\pgfsetbuttcap%
\pgfsetroundjoin%
\definecolor{currentfill}{rgb}{0.698454,0.799450,0.984577}%
\pgfsetfillcolor{currentfill}%
\pgfsetlinewidth{0.000000pt}%
\definecolor{currentstroke}{rgb}{0.000000,0.000000,0.000000}%
\pgfsetstrokecolor{currentstroke}%
\pgfsetdash{}{0pt}%
\pgfpathmoveto{\pgfqpoint{2.729878in}{3.345539in}}%
\pgfpathlineto{\pgfqpoint{2.734564in}{3.568175in}}%
\pgfpathlineto{\pgfqpoint{2.745011in}{3.394085in}}%
\pgfpathlineto{\pgfqpoint{2.779962in}{3.359764in}}%
\pgfpathlineto{\pgfqpoint{2.811525in}{3.569572in}}%
\pgfpathlineto{\pgfqpoint{2.803733in}{3.553586in}}%
\pgfpathlineto{\pgfqpoint{2.794991in}{3.607672in}}%
\pgfpathlineto{\pgfqpoint{2.760691in}{3.596926in}}%
\pgfpathlineto{\pgfqpoint{2.729878in}{3.345539in}}%
\pgfpathclose%
\pgfusepath{fill}%
\end{pgfscope}%
\begin{pgfscope}%
\pgfpathrectangle{\pgfqpoint{1.020000in}{0.880000in}}{\pgfqpoint{6.160000in}{6.160000in}}%
\pgfusepath{clip}%
\pgfsetbuttcap%
\pgfsetroundjoin%
\definecolor{currentfill}{rgb}{0.656683,0.771806,0.994914}%
\pgfsetfillcolor{currentfill}%
\pgfsetlinewidth{0.000000pt}%
\definecolor{currentstroke}{rgb}{0.000000,0.000000,0.000000}%
\pgfsetstrokecolor{currentstroke}%
\pgfsetdash{}{0pt}%
\pgfpathmoveto{\pgfqpoint{4.201763in}{3.474609in}}%
\pgfpathlineto{\pgfqpoint{4.211263in}{3.396083in}}%
\pgfpathlineto{\pgfqpoint{4.220773in}{3.329878in}}%
\pgfpathlineto{\pgfqpoint{4.254871in}{3.421540in}}%
\pgfpathlineto{\pgfqpoint{4.288815in}{3.327427in}}%
\pgfpathlineto{\pgfqpoint{4.279358in}{3.472244in}}%
\pgfpathlineto{\pgfqpoint{4.269694in}{3.386564in}}%
\pgfpathlineto{\pgfqpoint{4.235794in}{3.507291in}}%
\pgfpathlineto{\pgfqpoint{4.201763in}{3.474609in}}%
\pgfpathclose%
\pgfusepath{fill}%
\end{pgfscope}%
\begin{pgfscope}%
\pgfpathrectangle{\pgfqpoint{1.020000in}{0.880000in}}{\pgfqpoint{6.160000in}{6.160000in}}%
\pgfusepath{clip}%
\pgfsetbuttcap%
\pgfsetroundjoin%
\definecolor{currentfill}{rgb}{0.703587,0.802586,0.982847}%
\pgfsetfillcolor{currentfill}%
\pgfsetlinewidth{0.000000pt}%
\definecolor{currentstroke}{rgb}{0.000000,0.000000,0.000000}%
\pgfsetstrokecolor{currentstroke}%
\pgfsetdash{}{0pt}%
\pgfpathmoveto{\pgfqpoint{4.046904in}{3.392013in}}%
\pgfpathlineto{\pgfqpoint{4.056092in}{3.509222in}}%
\pgfpathlineto{\pgfqpoint{4.065373in}{3.583254in}}%
\pgfpathlineto{\pgfqpoint{4.099451in}{3.650162in}}%
\pgfpathlineto{\pgfqpoint{4.133675in}{3.471009in}}%
\pgfpathlineto{\pgfqpoint{4.124200in}{3.543526in}}%
\pgfpathlineto{\pgfqpoint{4.114859in}{3.427988in}}%
\pgfpathlineto{\pgfqpoint{4.080786in}{3.507988in}}%
\pgfpathlineto{\pgfqpoint{4.046904in}{3.392013in}}%
\pgfpathclose%
\pgfusepath{fill}%
\end{pgfscope}%
\begin{pgfscope}%
\pgfpathrectangle{\pgfqpoint{1.020000in}{0.880000in}}{\pgfqpoint{6.160000in}{6.160000in}}%
\pgfusepath{clip}%
\pgfsetbuttcap%
\pgfsetroundjoin%
\definecolor{currentfill}{rgb}{0.688188,0.793178,0.988038}%
\pgfsetfillcolor{currentfill}%
\pgfsetlinewidth{0.000000pt}%
\definecolor{currentstroke}{rgb}{0.000000,0.000000,0.000000}%
\pgfsetstrokecolor{currentstroke}%
\pgfsetdash{}{0pt}%
\pgfpathmoveto{\pgfqpoint{2.951488in}{3.396336in}}%
\pgfpathlineto{\pgfqpoint{2.956308in}{3.668780in}}%
\pgfpathlineto{\pgfqpoint{2.966857in}{3.475093in}}%
\pgfpathlineto{\pgfqpoint{3.002018in}{3.414322in}}%
\pgfpathlineto{\pgfqpoint{3.035839in}{3.464787in}}%
\pgfpathlineto{\pgfqpoint{3.028063in}{3.423983in}}%
\pgfpathlineto{\pgfqpoint{3.017929in}{3.587990in}}%
\pgfpathlineto{\pgfqpoint{2.985790in}{3.399998in}}%
\pgfpathlineto{\pgfqpoint{2.951488in}{3.396336in}}%
\pgfpathclose%
\pgfusepath{fill}%
\end{pgfscope}%
\begin{pgfscope}%
\pgfpathrectangle{\pgfqpoint{1.020000in}{0.880000in}}{\pgfqpoint{6.160000in}{6.160000in}}%
\pgfusepath{clip}%
\pgfsetbuttcap%
\pgfsetroundjoin%
\definecolor{currentfill}{rgb}{0.404421,0.534643,0.932002}%
\pgfsetfillcolor{currentfill}%
\pgfsetlinewidth{0.000000pt}%
\definecolor{currentstroke}{rgb}{0.000000,0.000000,0.000000}%
\pgfsetstrokecolor{currentstroke}%
\pgfsetdash{}{0pt}%
\pgfpathmoveto{\pgfqpoint{4.976931in}{2.952662in}}%
\pgfpathlineto{\pgfqpoint{4.988590in}{3.113230in}}%
\pgfpathlineto{\pgfqpoint{4.999026in}{3.117505in}}%
\pgfpathlineto{\pgfqpoint{5.030346in}{2.816214in}}%
\pgfpathlineto{\pgfqpoint{5.063563in}{2.754420in}}%
\pgfpathlineto{\pgfqpoint{5.055251in}{2.996352in}}%
\pgfpathlineto{\pgfqpoint{5.043901in}{2.890464in}}%
\pgfpathlineto{\pgfqpoint{5.010737in}{2.957743in}}%
\pgfpathlineto{\pgfqpoint{4.976931in}{2.952662in}}%
\pgfpathclose%
\pgfusepath{fill}%
\end{pgfscope}%
\begin{pgfscope}%
\pgfpathrectangle{\pgfqpoint{1.020000in}{0.880000in}}{\pgfqpoint{6.160000in}{6.160000in}}%
\pgfusepath{clip}%
\pgfsetbuttcap%
\pgfsetroundjoin%
\definecolor{currentfill}{rgb}{0.353369,0.472069,0.892570}%
\pgfsetfillcolor{currentfill}%
\pgfsetlinewidth{0.000000pt}%
\definecolor{currentstroke}{rgb}{0.000000,0.000000,0.000000}%
\pgfsetstrokecolor{currentstroke}%
\pgfsetdash{}{0pt}%
\pgfpathmoveto{\pgfqpoint{5.133507in}{3.018435in}}%
\pgfpathlineto{\pgfqpoint{5.142982in}{2.907070in}}%
\pgfpathlineto{\pgfqpoint{5.151637in}{2.711807in}}%
\pgfpathlineto{\pgfqpoint{5.184865in}{2.661548in}}%
\pgfpathlineto{\pgfqpoint{5.220409in}{2.838564in}}%
\pgfpathlineto{\pgfqpoint{5.210373in}{2.895184in}}%
\pgfpathlineto{\pgfqpoint{5.198617in}{2.782328in}}%
\pgfpathlineto{\pgfqpoint{5.167478in}{3.038533in}}%
\pgfpathlineto{\pgfqpoint{5.133507in}{3.018435in}}%
\pgfpathclose%
\pgfusepath{fill}%
\end{pgfscope}%
\begin{pgfscope}%
\pgfpathrectangle{\pgfqpoint{1.020000in}{0.880000in}}{\pgfqpoint{6.160000in}{6.160000in}}%
\pgfusepath{clip}%
\pgfsetbuttcap%
\pgfsetroundjoin%
\definecolor{currentfill}{rgb}{0.323718,0.433158,0.864722}%
\pgfsetfillcolor{currentfill}%
\pgfsetlinewidth{0.000000pt}%
\definecolor{currentstroke}{rgb}{0.000000,0.000000,0.000000}%
\pgfsetstrokecolor{currentstroke}%
\pgfsetdash{}{0pt}%
\pgfpathmoveto{\pgfqpoint{5.578899in}{2.843339in}}%
\pgfpathlineto{\pgfqpoint{5.590706in}{2.898577in}}%
\pgfpathlineto{\pgfqpoint{5.598582in}{2.675605in}}%
\pgfpathlineto{\pgfqpoint{5.634373in}{2.824983in}}%
\pgfpathlineto{\pgfqpoint{5.666807in}{2.744217in}}%
\pgfpathlineto{\pgfqpoint{5.654684in}{2.675547in}}%
\pgfpathlineto{\pgfqpoint{5.645965in}{2.835539in}}%
\pgfpathlineto{\pgfqpoint{5.611931in}{2.803753in}}%
\pgfpathlineto{\pgfqpoint{5.578899in}{2.843339in}}%
\pgfpathclose%
\pgfusepath{fill}%
\end{pgfscope}%
\begin{pgfscope}%
\pgfpathrectangle{\pgfqpoint{1.020000in}{0.880000in}}{\pgfqpoint{6.160000in}{6.160000in}}%
\pgfusepath{clip}%
\pgfsetbuttcap%
\pgfsetroundjoin%
\definecolor{currentfill}{rgb}{0.309060,0.413498,0.850128}%
\pgfsetfillcolor{currentfill}%
\pgfsetlinewidth{0.000000pt}%
\definecolor{currentstroke}{rgb}{0.000000,0.000000,0.000000}%
\pgfsetstrokecolor{currentstroke}%
\pgfsetdash{}{0pt}%
\pgfpathmoveto{\pgfqpoint{5.801941in}{2.798636in}}%
\pgfpathlineto{\pgfqpoint{5.815114in}{2.914708in}}%
\pgfpathlineto{\pgfqpoint{5.820761in}{2.572072in}}%
\pgfpathlineto{\pgfqpoint{5.857665in}{2.772896in}}%
\pgfpathlineto{\pgfqpoint{5.888984in}{2.642224in}}%
\pgfpathlineto{\pgfqpoint{5.880984in}{2.837884in}}%
\pgfpathlineto{\pgfqpoint{5.868783in}{2.787086in}}%
\pgfpathlineto{\pgfqpoint{5.834686in}{2.751488in}}%
\pgfpathlineto{\pgfqpoint{5.801941in}{2.798636in}}%
\pgfpathclose%
\pgfusepath{fill}%
\end{pgfscope}%
\begin{pgfscope}%
\pgfpathrectangle{\pgfqpoint{1.020000in}{0.880000in}}{\pgfqpoint{6.160000in}{6.160000in}}%
\pgfusepath{clip}%
\pgfsetbuttcap%
\pgfsetroundjoin%
\definecolor{currentfill}{rgb}{0.758539,0.832787,0.958408}%
\pgfsetfillcolor{currentfill}%
\pgfsetlinewidth{0.000000pt}%
\definecolor{currentstroke}{rgb}{0.000000,0.000000,0.000000}%
\pgfsetstrokecolor{currentstroke}%
\pgfsetdash{}{0pt}%
\pgfpathmoveto{\pgfqpoint{3.599836in}{3.727282in}}%
\pgfpathlineto{\pgfqpoint{3.609314in}{3.622639in}}%
\pgfpathlineto{\pgfqpoint{3.618886in}{3.501405in}}%
\pgfpathlineto{\pgfqpoint{3.652205in}{3.670825in}}%
\pgfpathlineto{\pgfqpoint{3.687119in}{3.537387in}}%
\pgfpathlineto{\pgfqpoint{3.677626in}{3.645179in}}%
\pgfpathlineto{\pgfqpoint{3.668931in}{3.597445in}}%
\pgfpathlineto{\pgfqpoint{3.634885in}{3.576617in}}%
\pgfpathlineto{\pgfqpoint{3.599836in}{3.727282in}}%
\pgfpathclose%
\pgfusepath{fill}%
\end{pgfscope}%
\begin{pgfscope}%
\pgfpathrectangle{\pgfqpoint{1.020000in}{0.880000in}}{\pgfqpoint{6.160000in}{6.160000in}}%
\pgfusepath{clip}%
\pgfsetbuttcap%
\pgfsetroundjoin%
\definecolor{currentfill}{rgb}{0.733898,0.820018,0.970724}%
\pgfsetfillcolor{currentfill}%
\pgfsetlinewidth{0.000000pt}%
\definecolor{currentstroke}{rgb}{0.000000,0.000000,0.000000}%
\pgfsetstrokecolor{currentstroke}%
\pgfsetdash{}{0pt}%
\pgfpathmoveto{\pgfqpoint{3.017929in}{3.587990in}}%
\pgfpathlineto{\pgfqpoint{3.028063in}{3.423983in}}%
\pgfpathlineto{\pgfqpoint{3.035839in}{3.464787in}}%
\pgfpathlineto{\pgfqpoint{3.069696in}{3.513332in}}%
\pgfpathlineto{\pgfqpoint{3.102706in}{3.641654in}}%
\pgfpathlineto{\pgfqpoint{3.095246in}{3.561366in}}%
\pgfpathlineto{\pgfqpoint{3.084331in}{3.799631in}}%
\pgfpathlineto{\pgfqpoint{3.053038in}{3.522548in}}%
\pgfpathlineto{\pgfqpoint{3.017929in}{3.587990in}}%
\pgfpathclose%
\pgfusepath{fill}%
\end{pgfscope}%
\begin{pgfscope}%
\pgfpathrectangle{\pgfqpoint{1.020000in}{0.880000in}}{\pgfqpoint{6.160000in}{6.160000in}}%
\pgfusepath{clip}%
\pgfsetbuttcap%
\pgfsetroundjoin%
\definecolor{currentfill}{rgb}{0.624703,0.748318,0.998719}%
\pgfsetfillcolor{currentfill}%
\pgfsetlinewidth{0.000000pt}%
\definecolor{currentstroke}{rgb}{0.000000,0.000000,0.000000}%
\pgfsetstrokecolor{currentstroke}%
\pgfsetdash{}{0pt}%
\pgfpathmoveto{\pgfqpoint{4.357126in}{3.529132in}}%
\pgfpathlineto{\pgfqpoint{4.366330in}{3.256263in}}%
\pgfpathlineto{\pgfqpoint{4.376171in}{3.329601in}}%
\pgfpathlineto{\pgfqpoint{4.410507in}{3.469514in}}%
\pgfpathlineto{\pgfqpoint{4.444470in}{3.437998in}}%
\pgfpathlineto{\pgfqpoint{4.433805in}{3.087926in}}%
\pgfpathlineto{\pgfqpoint{4.424899in}{3.428592in}}%
\pgfpathlineto{\pgfqpoint{4.390656in}{3.293957in}}%
\pgfpathlineto{\pgfqpoint{4.357126in}{3.529132in}}%
\pgfpathclose%
\pgfusepath{fill}%
\end{pgfscope}%
\begin{pgfscope}%
\pgfpathrectangle{\pgfqpoint{1.020000in}{0.880000in}}{\pgfqpoint{6.160000in}{6.160000in}}%
\pgfusepath{clip}%
\pgfsetbuttcap%
\pgfsetroundjoin%
\definecolor{currentfill}{rgb}{0.677823,0.786546,0.991005}%
\pgfsetfillcolor{currentfill}%
\pgfsetlinewidth{0.000000pt}%
\definecolor{currentstroke}{rgb}{0.000000,0.000000,0.000000}%
\pgfsetstrokecolor{currentstroke}%
\pgfsetdash{}{0pt}%
\pgfpathmoveto{\pgfqpoint{2.880266in}{3.586063in}}%
\pgfpathlineto{\pgfqpoint{2.889260in}{3.516963in}}%
\pgfpathlineto{\pgfqpoint{2.900513in}{3.272521in}}%
\pgfpathlineto{\pgfqpoint{2.934870in}{3.277100in}}%
\pgfpathlineto{\pgfqpoint{2.966857in}{3.475093in}}%
\pgfpathlineto{\pgfqpoint{2.956308in}{3.668780in}}%
\pgfpathlineto{\pgfqpoint{2.951488in}{3.396336in}}%
\pgfpathlineto{\pgfqpoint{2.916928in}{3.411642in}}%
\pgfpathlineto{\pgfqpoint{2.880266in}{3.586063in}}%
\pgfpathclose%
\pgfusepath{fill}%
\end{pgfscope}%
\begin{pgfscope}%
\pgfpathrectangle{\pgfqpoint{1.020000in}{0.880000in}}{\pgfqpoint{6.160000in}{6.160000in}}%
\pgfusepath{clip}%
\pgfsetbuttcap%
\pgfsetroundjoin%
\definecolor{currentfill}{rgb}{0.758539,0.832787,0.958408}%
\pgfsetfillcolor{currentfill}%
\pgfsetlinewidth{0.000000pt}%
\definecolor{currentstroke}{rgb}{0.000000,0.000000,0.000000}%
\pgfsetstrokecolor{currentstroke}%
\pgfsetdash{}{0pt}%
\pgfpathmoveto{\pgfqpoint{3.310097in}{3.503589in}}%
\pgfpathlineto{\pgfqpoint{3.318384in}{3.533564in}}%
\pgfpathlineto{\pgfqpoint{3.326900in}{3.539480in}}%
\pgfpathlineto{\pgfqpoint{3.362599in}{3.372287in}}%
\pgfpathlineto{\pgfqpoint{3.392841in}{3.881312in}}%
\pgfpathlineto{\pgfqpoint{3.386223in}{3.623962in}}%
\pgfpathlineto{\pgfqpoint{3.376916in}{3.708749in}}%
\pgfpathlineto{\pgfqpoint{3.342733in}{3.695410in}}%
\pgfpathlineto{\pgfqpoint{3.310097in}{3.503589in}}%
\pgfpathclose%
\pgfusepath{fill}%
\end{pgfscope}%
\begin{pgfscope}%
\pgfpathrectangle{\pgfqpoint{1.020000in}{0.880000in}}{\pgfqpoint{6.160000in}{6.160000in}}%
\pgfusepath{clip}%
\pgfsetbuttcap%
\pgfsetroundjoin%
\definecolor{currentfill}{rgb}{0.304174,0.406945,0.845263}%
\pgfsetfillcolor{currentfill}%
\pgfsetlinewidth{0.000000pt}%
\definecolor{currentstroke}{rgb}{0.000000,0.000000,0.000000}%
\pgfsetstrokecolor{currentstroke}%
\pgfsetdash{}{0pt}%
\pgfpathmoveto{\pgfqpoint{6.024258in}{2.717131in}}%
\pgfpathlineto{\pgfqpoint{6.038597in}{2.868840in}}%
\pgfpathlineto{\pgfqpoint{6.050839in}{2.905693in}}%
\pgfpathlineto{\pgfqpoint{6.078265in}{2.583939in}}%
\pgfpathlineto{\pgfqpoint{6.112136in}{2.605801in}}%
\pgfpathlineto{\pgfqpoint{6.101594in}{2.659797in}}%
\pgfpathlineto{\pgfqpoint{6.094432in}{2.889019in}}%
\pgfpathlineto{\pgfqpoint{6.059075in}{2.789726in}}%
\pgfpathlineto{\pgfqpoint{6.024258in}{2.717131in}}%
\pgfpathclose%
\pgfusepath{fill}%
\end{pgfscope}%
\begin{pgfscope}%
\pgfpathrectangle{\pgfqpoint{1.020000in}{0.880000in}}{\pgfqpoint{6.160000in}{6.160000in}}%
\pgfusepath{clip}%
\pgfsetbuttcap%
\pgfsetroundjoin%
\definecolor{currentfill}{rgb}{0.313946,0.420052,0.854993}%
\pgfsetfillcolor{currentfill}%
\pgfsetlinewidth{0.000000pt}%
\definecolor{currentstroke}{rgb}{0.000000,0.000000,0.000000}%
\pgfsetstrokecolor{currentstroke}%
\pgfsetdash{}{0pt}%
\pgfpathmoveto{\pgfqpoint{5.510360in}{2.747775in}}%
\pgfpathlineto{\pgfqpoint{5.519849in}{2.639467in}}%
\pgfpathlineto{\pgfqpoint{5.531134in}{2.663347in}}%
\pgfpathlineto{\pgfqpoint{5.568665in}{2.942299in}}%
\pgfpathlineto{\pgfqpoint{5.598582in}{2.675605in}}%
\pgfpathlineto{\pgfqpoint{5.590706in}{2.898577in}}%
\pgfpathlineto{\pgfqpoint{5.578899in}{2.843339in}}%
\pgfpathlineto{\pgfqpoint{5.543492in}{2.712912in}}%
\pgfpathlineto{\pgfqpoint{5.510360in}{2.747775in}}%
\pgfpathclose%
\pgfusepath{fill}%
\end{pgfscope}%
\begin{pgfscope}%
\pgfpathrectangle{\pgfqpoint{1.020000in}{0.880000in}}{\pgfqpoint{6.160000in}{6.160000in}}%
\pgfusepath{clip}%
\pgfsetbuttcap%
\pgfsetroundjoin%
\definecolor{currentfill}{rgb}{0.733898,0.820018,0.970724}%
\pgfsetfillcolor{currentfill}%
\pgfsetlinewidth{0.000000pt}%
\definecolor{currentstroke}{rgb}{0.000000,0.000000,0.000000}%
\pgfsetstrokecolor{currentstroke}%
\pgfsetdash{}{0pt}%
\pgfpathmoveto{\pgfqpoint{3.242090in}{3.445725in}}%
\pgfpathlineto{\pgfqpoint{3.248279in}{3.690497in}}%
\pgfpathlineto{\pgfqpoint{3.258151in}{3.543253in}}%
\pgfpathlineto{\pgfqpoint{3.292161in}{3.583624in}}%
\pgfpathlineto{\pgfqpoint{3.326900in}{3.539480in}}%
\pgfpathlineto{\pgfqpoint{3.318384in}{3.533564in}}%
\pgfpathlineto{\pgfqpoint{3.310097in}{3.503589in}}%
\pgfpathlineto{\pgfqpoint{3.275001in}{3.595113in}}%
\pgfpathlineto{\pgfqpoint{3.242090in}{3.445725in}}%
\pgfpathclose%
\pgfusepath{fill}%
\end{pgfscope}%
\begin{pgfscope}%
\pgfpathrectangle{\pgfqpoint{1.020000in}{0.880000in}}{\pgfqpoint{6.160000in}{6.160000in}}%
\pgfusepath{clip}%
\pgfsetbuttcap%
\pgfsetroundjoin%
\definecolor{currentfill}{rgb}{0.656683,0.771806,0.994914}%
\pgfsetfillcolor{currentfill}%
\pgfsetlinewidth{0.000000pt}%
\definecolor{currentstroke}{rgb}{0.000000,0.000000,0.000000}%
\pgfsetstrokecolor{currentstroke}%
\pgfsetdash{}{0pt}%
\pgfpathmoveto{\pgfqpoint{2.811525in}{3.569572in}}%
\pgfpathlineto{\pgfqpoint{2.823014in}{3.315530in}}%
\pgfpathlineto{\pgfqpoint{2.832142in}{3.234137in}}%
\pgfpathlineto{\pgfqpoint{2.864982in}{3.355602in}}%
\pgfpathlineto{\pgfqpoint{2.900513in}{3.272521in}}%
\pgfpathlineto{\pgfqpoint{2.889260in}{3.516963in}}%
\pgfpathlineto{\pgfqpoint{2.880266in}{3.586063in}}%
\pgfpathlineto{\pgfqpoint{2.847991in}{3.421492in}}%
\pgfpathlineto{\pgfqpoint{2.811525in}{3.569572in}}%
\pgfpathclose%
\pgfusepath{fill}%
\end{pgfscope}%
\begin{pgfscope}%
\pgfpathrectangle{\pgfqpoint{1.020000in}{0.880000in}}{\pgfqpoint{6.160000in}{6.160000in}}%
\pgfusepath{clip}%
\pgfsetbuttcap%
\pgfsetroundjoin%
\definecolor{currentfill}{rgb}{0.763363,0.835092,0.955658}%
\pgfsetfillcolor{currentfill}%
\pgfsetlinewidth{0.000000pt}%
\definecolor{currentstroke}{rgb}{0.000000,0.000000,0.000000}%
\pgfsetstrokecolor{currentstroke}%
\pgfsetdash{}{0pt}%
\pgfpathmoveto{\pgfqpoint{3.822902in}{3.772053in}}%
\pgfpathlineto{\pgfqpoint{3.832847in}{3.538302in}}%
\pgfpathlineto{\pgfqpoint{3.841479in}{3.683300in}}%
\pgfpathlineto{\pgfqpoint{3.876069in}{3.563436in}}%
\pgfpathlineto{\pgfqpoint{3.910331in}{3.526690in}}%
\pgfpathlineto{\pgfqpoint{3.900865in}{3.623532in}}%
\pgfpathlineto{\pgfqpoint{3.891729in}{3.607645in}}%
\pgfpathlineto{\pgfqpoint{3.857692in}{3.585735in}}%
\pgfpathlineto{\pgfqpoint{3.822902in}{3.772053in}}%
\pgfpathclose%
\pgfusepath{fill}%
\end{pgfscope}%
\begin{pgfscope}%
\pgfpathrectangle{\pgfqpoint{1.020000in}{0.880000in}}{\pgfqpoint{6.160000in}{6.160000in}}%
\pgfusepath{clip}%
\pgfsetbuttcap%
\pgfsetroundjoin%
\definecolor{currentfill}{rgb}{0.646113,0.764436,0.996868}%
\pgfsetfillcolor{currentfill}%
\pgfsetlinewidth{0.000000pt}%
\definecolor{currentstroke}{rgb}{0.000000,0.000000,0.000000}%
\pgfsetstrokecolor{currentstroke}%
\pgfsetdash{}{0pt}%
\pgfpathmoveto{\pgfqpoint{2.519547in}{3.561963in}}%
\pgfpathlineto{\pgfqpoint{2.533571in}{3.180396in}}%
\pgfpathlineto{\pgfqpoint{2.536782in}{3.453188in}}%
\pgfpathlineto{\pgfqpoint{2.572805in}{3.365843in}}%
\pgfpathlineto{\pgfqpoint{2.608072in}{3.321275in}}%
\pgfpathlineto{\pgfqpoint{2.597407in}{3.503172in}}%
\pgfpathlineto{\pgfqpoint{2.591954in}{3.357079in}}%
\pgfpathlineto{\pgfqpoint{2.556835in}{3.395716in}}%
\pgfpathlineto{\pgfqpoint{2.519547in}{3.561963in}}%
\pgfpathclose%
\pgfusepath{fill}%
\end{pgfscope}%
\begin{pgfscope}%
\pgfpathrectangle{\pgfqpoint{1.020000in}{0.880000in}}{\pgfqpoint{6.160000in}{6.160000in}}%
\pgfusepath{clip}%
\pgfsetbuttcap%
\pgfsetroundjoin%
\definecolor{currentfill}{rgb}{0.338377,0.452819,0.879317}%
\pgfsetfillcolor{currentfill}%
\pgfsetlinewidth{0.000000pt}%
\definecolor{currentstroke}{rgb}{0.000000,0.000000,0.000000}%
\pgfsetstrokecolor{currentstroke}%
\pgfsetdash{}{0pt}%
\pgfpathmoveto{\pgfqpoint{5.287013in}{2.762205in}}%
\pgfpathlineto{\pgfqpoint{5.298790in}{2.858063in}}%
\pgfpathlineto{\pgfqpoint{5.309458in}{2.852079in}}%
\pgfpathlineto{\pgfqpoint{5.342348in}{2.776833in}}%
\pgfpathlineto{\pgfqpoint{5.378244in}{2.957953in}}%
\pgfpathlineto{\pgfqpoint{5.363952in}{2.665625in}}%
\pgfpathlineto{\pgfqpoint{5.354787in}{2.800812in}}%
\pgfpathlineto{\pgfqpoint{5.321331in}{2.818810in}}%
\pgfpathlineto{\pgfqpoint{5.287013in}{2.762205in}}%
\pgfpathclose%
\pgfusepath{fill}%
\end{pgfscope}%
\begin{pgfscope}%
\pgfpathrectangle{\pgfqpoint{1.020000in}{0.880000in}}{\pgfqpoint{6.160000in}{6.160000in}}%
\pgfusepath{clip}%
\pgfsetbuttcap%
\pgfsetroundjoin%
\definecolor{currentfill}{rgb}{0.796064,0.848693,0.933471}%
\pgfsetfillcolor{currentfill}%
\pgfsetlinewidth{0.000000pt}%
\definecolor{currentstroke}{rgb}{0.000000,0.000000,0.000000}%
\pgfsetstrokecolor{currentstroke}%
\pgfsetdash{}{0pt}%
\pgfpathmoveto{\pgfqpoint{3.376916in}{3.708749in}}%
\pgfpathlineto{\pgfqpoint{3.386223in}{3.623962in}}%
\pgfpathlineto{\pgfqpoint{3.392841in}{3.881312in}}%
\pgfpathlineto{\pgfqpoint{3.429288in}{3.609541in}}%
\pgfpathlineto{\pgfqpoint{3.464859in}{3.430919in}}%
\pgfpathlineto{\pgfqpoint{3.453179in}{3.841328in}}%
\pgfpathlineto{\pgfqpoint{3.446051in}{3.625893in}}%
\pgfpathlineto{\pgfqpoint{3.410713in}{3.770579in}}%
\pgfpathlineto{\pgfqpoint{3.376916in}{3.708749in}}%
\pgfpathclose%
\pgfusepath{fill}%
\end{pgfscope}%
\begin{pgfscope}%
\pgfpathrectangle{\pgfqpoint{1.020000in}{0.880000in}}{\pgfqpoint{6.160000in}{6.160000in}}%
\pgfusepath{clip}%
\pgfsetbuttcap%
\pgfsetroundjoin%
\definecolor{currentfill}{rgb}{0.718985,0.811993,0.977656}%
\pgfsetfillcolor{currentfill}%
\pgfsetlinewidth{0.000000pt}%
\definecolor{currentstroke}{rgb}{0.000000,0.000000,0.000000}%
\pgfsetstrokecolor{currentstroke}%
\pgfsetdash{}{0pt}%
\pgfpathmoveto{\pgfqpoint{3.978576in}{3.522362in}}%
\pgfpathlineto{\pgfqpoint{3.987737in}{3.583543in}}%
\pgfpathlineto{\pgfqpoint{3.997009in}{3.602737in}}%
\pgfpathlineto{\pgfqpoint{4.031329in}{3.511339in}}%
\pgfpathlineto{\pgfqpoint{4.065373in}{3.583254in}}%
\pgfpathlineto{\pgfqpoint{4.056092in}{3.509222in}}%
\pgfpathlineto{\pgfqpoint{4.046904in}{3.392013in}}%
\pgfpathlineto{\pgfqpoint{4.012665in}{3.520181in}}%
\pgfpathlineto{\pgfqpoint{3.978576in}{3.522362in}}%
\pgfpathclose%
\pgfusepath{fill}%
\end{pgfscope}%
\begin{pgfscope}%
\pgfpathrectangle{\pgfqpoint{1.020000in}{0.880000in}}{\pgfqpoint{6.160000in}{6.160000in}}%
\pgfusepath{clip}%
\pgfsetbuttcap%
\pgfsetroundjoin%
\definecolor{currentfill}{rgb}{0.309060,0.413498,0.850128}%
\pgfsetfillcolor{currentfill}%
\pgfsetlinewidth{0.000000pt}%
\definecolor{currentstroke}{rgb}{0.000000,0.000000,0.000000}%
\pgfsetstrokecolor{currentstroke}%
\pgfsetdash{}{0pt}%
\pgfpathmoveto{\pgfqpoint{5.733175in}{2.693167in}}%
\pgfpathlineto{\pgfqpoint{5.744524in}{2.703312in}}%
\pgfpathlineto{\pgfqpoint{5.756765in}{2.768021in}}%
\pgfpathlineto{\pgfqpoint{5.790167in}{2.754177in}}%
\pgfpathlineto{\pgfqpoint{5.820761in}{2.572072in}}%
\pgfpathlineto{\pgfqpoint{5.815114in}{2.914708in}}%
\pgfpathlineto{\pgfqpoint{5.801941in}{2.798636in}}%
\pgfpathlineto{\pgfqpoint{5.768514in}{2.806341in}}%
\pgfpathlineto{\pgfqpoint{5.733175in}{2.693167in}}%
\pgfpathclose%
\pgfusepath{fill}%
\end{pgfscope}%
\begin{pgfscope}%
\pgfpathrectangle{\pgfqpoint{1.020000in}{0.880000in}}{\pgfqpoint{6.160000in}{6.160000in}}%
\pgfusepath{clip}%
\pgfsetbuttcap%
\pgfsetroundjoin%
\definecolor{currentfill}{rgb}{0.510824,0.649397,0.985079}%
\pgfsetfillcolor{currentfill}%
\pgfsetlinewidth{0.000000pt}%
\definecolor{currentstroke}{rgb}{0.000000,0.000000,0.000000}%
\pgfsetstrokecolor{currentstroke}%
\pgfsetdash{}{0pt}%
\pgfpathmoveto{\pgfqpoint{4.598959in}{3.114551in}}%
\pgfpathlineto{\pgfqpoint{4.609016in}{3.139054in}}%
\pgfpathlineto{\pgfqpoint{4.619495in}{3.253915in}}%
\pgfpathlineto{\pgfqpoint{4.653094in}{3.167786in}}%
\pgfpathlineto{\pgfqpoint{4.686723in}{3.103465in}}%
\pgfpathlineto{\pgfqpoint{4.676915in}{3.150520in}}%
\pgfpathlineto{\pgfqpoint{4.666373in}{3.043619in}}%
\pgfpathlineto{\pgfqpoint{4.633200in}{3.192643in}}%
\pgfpathlineto{\pgfqpoint{4.598959in}{3.114551in}}%
\pgfpathclose%
\pgfusepath{fill}%
\end{pgfscope}%
\begin{pgfscope}%
\pgfpathrectangle{\pgfqpoint{1.020000in}{0.880000in}}{\pgfqpoint{6.160000in}{6.160000in}}%
\pgfusepath{clip}%
\pgfsetbuttcap%
\pgfsetroundjoin%
\definecolor{currentfill}{rgb}{0.266381,0.353304,0.801637}%
\pgfsetfillcolor{currentfill}%
\pgfsetlinewidth{0.000000pt}%
\definecolor{currentstroke}{rgb}{0.000000,0.000000,0.000000}%
\pgfsetstrokecolor{currentstroke}%
\pgfsetdash{}{0pt}%
\pgfpathmoveto{\pgfqpoint{6.182390in}{2.775130in}}%
\pgfpathlineto{\pgfqpoint{6.195764in}{2.857985in}}%
\pgfpathlineto{\pgfqpoint{6.203112in}{2.642651in}}%
\pgfpathlineto{\pgfqpoint{6.235737in}{2.601744in}}%
\pgfpathlineto{\pgfqpoint{6.222965in}{2.551378in}}%
\pgfpathlineto{\pgfqpoint{6.211906in}{2.582887in}}%
\pgfpathlineto{\pgfqpoint{6.182390in}{2.775130in}}%
\pgfpathclose%
\pgfusepath{fill}%
\end{pgfscope}%
\begin{pgfscope}%
\pgfpathrectangle{\pgfqpoint{1.020000in}{0.880000in}}{\pgfqpoint{6.160000in}{6.160000in}}%
\pgfusepath{clip}%
\pgfsetbuttcap%
\pgfsetroundjoin%
\definecolor{currentfill}{rgb}{0.462354,0.599830,0.965857}%
\pgfsetfillcolor{currentfill}%
\pgfsetlinewidth{0.000000pt}%
\definecolor{currentstroke}{rgb}{0.000000,0.000000,0.000000}%
\pgfsetstrokecolor{currentstroke}%
\pgfsetdash{}{0pt}%
\pgfpathmoveto{\pgfqpoint{4.754405in}{3.082095in}}%
\pgfpathlineto{\pgfqpoint{4.764096in}{3.007242in}}%
\pgfpathlineto{\pgfqpoint{4.775391in}{3.201272in}}%
\pgfpathlineto{\pgfqpoint{4.808380in}{3.044594in}}%
\pgfpathlineto{\pgfqpoint{4.842081in}{3.014899in}}%
\pgfpathlineto{\pgfqpoint{4.831612in}{2.975336in}}%
\pgfpathlineto{\pgfqpoint{4.821335in}{2.960973in}}%
\pgfpathlineto{\pgfqpoint{4.788593in}{3.135498in}}%
\pgfpathlineto{\pgfqpoint{4.754405in}{3.082095in}}%
\pgfpathclose%
\pgfusepath{fill}%
\end{pgfscope}%
\begin{pgfscope}%
\pgfpathrectangle{\pgfqpoint{1.020000in}{0.880000in}}{\pgfqpoint{6.160000in}{6.160000in}}%
\pgfusepath{clip}%
\pgfsetbuttcap%
\pgfsetroundjoin%
\definecolor{currentfill}{rgb}{0.299441,0.400248,0.839842}%
\pgfsetfillcolor{currentfill}%
\pgfsetlinewidth{0.000000pt}%
\definecolor{currentstroke}{rgb}{0.000000,0.000000,0.000000}%
\pgfsetstrokecolor{currentstroke}%
\pgfsetdash{}{0pt}%
\pgfpathmoveto{\pgfqpoint{5.445008in}{2.899684in}}%
\pgfpathlineto{\pgfqpoint{5.454672in}{2.803197in}}%
\pgfpathlineto{\pgfqpoint{5.463573in}{2.647395in}}%
\pgfpathlineto{\pgfqpoint{5.499111in}{2.787543in}}%
\pgfpathlineto{\pgfqpoint{5.531134in}{2.663347in}}%
\pgfpathlineto{\pgfqpoint{5.519849in}{2.639467in}}%
\pgfpathlineto{\pgfqpoint{5.510360in}{2.747775in}}%
\pgfpathlineto{\pgfqpoint{5.475572in}{2.658812in}}%
\pgfpathlineto{\pgfqpoint{5.445008in}{2.899684in}}%
\pgfpathclose%
\pgfusepath{fill}%
\end{pgfscope}%
\begin{pgfscope}%
\pgfpathrectangle{\pgfqpoint{1.020000in}{0.880000in}}{\pgfqpoint{6.160000in}{6.160000in}}%
\pgfusepath{clip}%
\pgfsetbuttcap%
\pgfsetroundjoin%
\definecolor{currentfill}{rgb}{0.368507,0.491141,0.905243}%
\pgfsetfillcolor{currentfill}%
\pgfsetlinewidth{0.000000pt}%
\definecolor{currentstroke}{rgb}{0.000000,0.000000,0.000000}%
\pgfsetstrokecolor{currentstroke}%
\pgfsetdash{}{0pt}%
\pgfpathmoveto{\pgfqpoint{5.063563in}{2.754420in}}%
\pgfpathlineto{\pgfqpoint{5.075662in}{2.937884in}}%
\pgfpathlineto{\pgfqpoint{5.085310in}{2.844436in}}%
\pgfpathlineto{\pgfqpoint{5.119508in}{2.884528in}}%
\pgfpathlineto{\pgfqpoint{5.151637in}{2.711807in}}%
\pgfpathlineto{\pgfqpoint{5.142982in}{2.907070in}}%
\pgfpathlineto{\pgfqpoint{5.133507in}{3.018435in}}%
\pgfpathlineto{\pgfqpoint{5.098554in}{2.892259in}}%
\pgfpathlineto{\pgfqpoint{5.063563in}{2.754420in}}%
\pgfpathclose%
\pgfusepath{fill}%
\end{pgfscope}%
\begin{pgfscope}%
\pgfpathrectangle{\pgfqpoint{1.020000in}{0.880000in}}{\pgfqpoint{6.160000in}{6.160000in}}%
\pgfusepath{clip}%
\pgfsetbuttcap%
\pgfsetroundjoin%
\definecolor{currentfill}{rgb}{0.718985,0.811993,0.977656}%
\pgfsetfillcolor{currentfill}%
\pgfsetlinewidth{0.000000pt}%
\definecolor{currentstroke}{rgb}{0.000000,0.000000,0.000000}%
\pgfsetstrokecolor{currentstroke}%
\pgfsetdash{}{0pt}%
\pgfpathmoveto{\pgfqpoint{3.173411in}{3.454105in}}%
\pgfpathlineto{\pgfqpoint{3.181597in}{3.475813in}}%
\pgfpathlineto{\pgfqpoint{3.189888in}{3.489183in}}%
\pgfpathlineto{\pgfqpoint{3.223940in}{3.524601in}}%
\pgfpathlineto{\pgfqpoint{3.258151in}{3.543253in}}%
\pgfpathlineto{\pgfqpoint{3.248279in}{3.690497in}}%
\pgfpathlineto{\pgfqpoint{3.242090in}{3.445725in}}%
\pgfpathlineto{\pgfqpoint{3.206172in}{3.612956in}}%
\pgfpathlineto{\pgfqpoint{3.173411in}{3.454105in}}%
\pgfpathclose%
\pgfusepath{fill}%
\end{pgfscope}%
\begin{pgfscope}%
\pgfpathrectangle{\pgfqpoint{1.020000in}{0.880000in}}{\pgfqpoint{6.160000in}{6.160000in}}%
\pgfusepath{clip}%
\pgfsetbuttcap%
\pgfsetroundjoin%
\definecolor{currentfill}{rgb}{0.425199,0.559058,0.946061}%
\pgfsetfillcolor{currentfill}%
\pgfsetlinewidth{0.000000pt}%
\definecolor{currentstroke}{rgb}{0.000000,0.000000,0.000000}%
\pgfsetstrokecolor{currentstroke}%
\pgfsetdash{}{0pt}%
\pgfpathmoveto{\pgfqpoint{4.909394in}{2.959906in}}%
\pgfpathlineto{\pgfqpoint{4.919581in}{2.945281in}}%
\pgfpathlineto{\pgfqpoint{4.929401in}{2.879071in}}%
\pgfpathlineto{\pgfqpoint{4.964777in}{3.073972in}}%
\pgfpathlineto{\pgfqpoint{4.999026in}{3.117505in}}%
\pgfpathlineto{\pgfqpoint{4.988590in}{3.113230in}}%
\pgfpathlineto{\pgfqpoint{4.976931in}{2.952662in}}%
\pgfpathlineto{\pgfqpoint{4.941819in}{2.778779in}}%
\pgfpathlineto{\pgfqpoint{4.909394in}{2.959906in}}%
\pgfpathclose%
\pgfusepath{fill}%
\end{pgfscope}%
\begin{pgfscope}%
\pgfpathrectangle{\pgfqpoint{1.020000in}{0.880000in}}{\pgfqpoint{6.160000in}{6.160000in}}%
\pgfusepath{clip}%
\pgfsetbuttcap%
\pgfsetroundjoin%
\definecolor{currentfill}{rgb}{0.289996,0.386836,0.828926}%
\pgfsetfillcolor{currentfill}%
\pgfsetlinewidth{0.000000pt}%
\definecolor{currentstroke}{rgb}{0.000000,0.000000,0.000000}%
\pgfsetstrokecolor{currentstroke}%
\pgfsetdash{}{0pt}%
\pgfpathmoveto{\pgfqpoint{5.666807in}{2.744217in}}%
\pgfpathlineto{\pgfqpoint{5.676140in}{2.624858in}}%
\pgfpathlineto{\pgfqpoint{5.689287in}{2.757162in}}%
\pgfpathlineto{\pgfqpoint{5.720952in}{2.628225in}}%
\pgfpathlineto{\pgfqpoint{5.756765in}{2.768021in}}%
\pgfpathlineto{\pgfqpoint{5.744524in}{2.703312in}}%
\pgfpathlineto{\pgfqpoint{5.733175in}{2.693167in}}%
\pgfpathlineto{\pgfqpoint{5.701400in}{2.809692in}}%
\pgfpathlineto{\pgfqpoint{5.666807in}{2.744217in}}%
\pgfpathclose%
\pgfusepath{fill}%
\end{pgfscope}%
\begin{pgfscope}%
\pgfpathrectangle{\pgfqpoint{1.020000in}{0.880000in}}{\pgfqpoint{6.160000in}{6.160000in}}%
\pgfusepath{clip}%
\pgfsetbuttcap%
\pgfsetroundjoin%
\definecolor{currentfill}{rgb}{0.304174,0.406945,0.845263}%
\pgfsetfillcolor{currentfill}%
\pgfsetlinewidth{0.000000pt}%
\definecolor{currentstroke}{rgb}{0.000000,0.000000,0.000000}%
\pgfsetstrokecolor{currentstroke}%
\pgfsetdash{}{0pt}%
\pgfpathmoveto{\pgfqpoint{5.961363in}{2.944189in}}%
\pgfpathlineto{\pgfqpoint{5.968207in}{2.687064in}}%
\pgfpathlineto{\pgfqpoint{5.977725in}{2.579609in}}%
\pgfpathlineto{\pgfqpoint{6.012236in}{2.634147in}}%
\pgfpathlineto{\pgfqpoint{6.050839in}{2.905693in}}%
\pgfpathlineto{\pgfqpoint{6.038597in}{2.868840in}}%
\pgfpathlineto{\pgfqpoint{6.024258in}{2.717131in}}%
\pgfpathlineto{\pgfqpoint{5.988938in}{2.615266in}}%
\pgfpathlineto{\pgfqpoint{5.961363in}{2.944189in}}%
\pgfpathclose%
\pgfusepath{fill}%
\end{pgfscope}%
\begin{pgfscope}%
\pgfpathrectangle{\pgfqpoint{1.020000in}{0.880000in}}{\pgfqpoint{6.160000in}{6.160000in}}%
\pgfusepath{clip}%
\pgfsetbuttcap%
\pgfsetroundjoin%
\definecolor{currentfill}{rgb}{0.651398,0.768121,0.995891}%
\pgfsetfillcolor{currentfill}%
\pgfsetlinewidth{0.000000pt}%
\definecolor{currentstroke}{rgb}{0.000000,0.000000,0.000000}%
\pgfsetstrokecolor{currentstroke}%
\pgfsetdash{}{0pt}%
\pgfpathmoveto{\pgfqpoint{2.745011in}{3.394085in}}%
\pgfpathlineto{\pgfqpoint{2.751436in}{3.500340in}}%
\pgfpathlineto{\pgfqpoint{2.761401in}{3.360191in}}%
\pgfpathlineto{\pgfqpoint{2.794864in}{3.436847in}}%
\pgfpathlineto{\pgfqpoint{2.832142in}{3.234137in}}%
\pgfpathlineto{\pgfqpoint{2.823014in}{3.315530in}}%
\pgfpathlineto{\pgfqpoint{2.811525in}{3.569572in}}%
\pgfpathlineto{\pgfqpoint{2.779962in}{3.359764in}}%
\pgfpathlineto{\pgfqpoint{2.745011in}{3.394085in}}%
\pgfpathclose%
\pgfusepath{fill}%
\end{pgfscope}%
\begin{pgfscope}%
\pgfpathrectangle{\pgfqpoint{1.020000in}{0.880000in}}{\pgfqpoint{6.160000in}{6.160000in}}%
\pgfusepath{clip}%
\pgfsetbuttcap%
\pgfsetroundjoin%
\definecolor{currentfill}{rgb}{0.753611,0.830233,0.960871}%
\pgfsetfillcolor{currentfill}%
\pgfsetlinewidth{0.000000pt}%
\definecolor{currentstroke}{rgb}{0.000000,0.000000,0.000000}%
\pgfsetstrokecolor{currentstroke}%
\pgfsetdash{}{0pt}%
\pgfpathmoveto{\pgfqpoint{3.755060in}{3.634011in}}%
\pgfpathlineto{\pgfqpoint{3.764648in}{3.504565in}}%
\pgfpathlineto{\pgfqpoint{3.773850in}{3.468521in}}%
\pgfpathlineto{\pgfqpoint{3.807336in}{3.656381in}}%
\pgfpathlineto{\pgfqpoint{3.841479in}{3.683300in}}%
\pgfpathlineto{\pgfqpoint{3.832847in}{3.538302in}}%
\pgfpathlineto{\pgfqpoint{3.822902in}{3.772053in}}%
\pgfpathlineto{\pgfqpoint{3.789731in}{3.508999in}}%
\pgfpathlineto{\pgfqpoint{3.755060in}{3.634011in}}%
\pgfpathclose%
\pgfusepath{fill}%
\end{pgfscope}%
\begin{pgfscope}%
\pgfpathrectangle{\pgfqpoint{1.020000in}{0.880000in}}{\pgfqpoint{6.160000in}{6.160000in}}%
\pgfusepath{clip}%
\pgfsetbuttcap%
\pgfsetroundjoin%
\definecolor{currentfill}{rgb}{0.603162,0.731527,0.999565}%
\pgfsetfillcolor{currentfill}%
\pgfsetlinewidth{0.000000pt}%
\definecolor{currentstroke}{rgb}{0.000000,0.000000,0.000000}%
\pgfsetstrokecolor{currentstroke}%
\pgfsetdash{}{0pt}%
\pgfpathmoveto{\pgfqpoint{4.444470in}{3.437998in}}%
\pgfpathlineto{\pgfqpoint{4.453889in}{3.295124in}}%
\pgfpathlineto{\pgfqpoint{4.463734in}{3.306501in}}%
\pgfpathlineto{\pgfqpoint{4.497845in}{3.329744in}}%
\pgfpathlineto{\pgfqpoint{4.531441in}{3.207001in}}%
\pgfpathlineto{\pgfqpoint{4.521826in}{3.282953in}}%
\pgfpathlineto{\pgfqpoint{4.511771in}{3.223483in}}%
\pgfpathlineto{\pgfqpoint{4.478172in}{3.332153in}}%
\pgfpathlineto{\pgfqpoint{4.444470in}{3.437998in}}%
\pgfpathclose%
\pgfusepath{fill}%
\end{pgfscope}%
\begin{pgfscope}%
\pgfpathrectangle{\pgfqpoint{1.020000in}{0.880000in}}{\pgfqpoint{6.160000in}{6.160000in}}%
\pgfusepath{clip}%
\pgfsetbuttcap%
\pgfsetroundjoin%
\definecolor{currentfill}{rgb}{0.640828,0.760752,0.997846}%
\pgfsetfillcolor{currentfill}%
\pgfsetlinewidth{0.000000pt}%
\definecolor{currentstroke}{rgb}{0.000000,0.000000,0.000000}%
\pgfsetstrokecolor{currentstroke}%
\pgfsetdash{}{0pt}%
\pgfpathmoveto{\pgfqpoint{2.675468in}{3.429978in}}%
\pgfpathlineto{\pgfqpoint{2.684422in}{3.358236in}}%
\pgfpathlineto{\pgfqpoint{2.693362in}{3.287786in}}%
\pgfpathlineto{\pgfqpoint{2.728067in}{3.276812in}}%
\pgfpathlineto{\pgfqpoint{2.761401in}{3.360191in}}%
\pgfpathlineto{\pgfqpoint{2.751436in}{3.500340in}}%
\pgfpathlineto{\pgfqpoint{2.745011in}{3.394085in}}%
\pgfpathlineto{\pgfqpoint{2.710814in}{3.374121in}}%
\pgfpathlineto{\pgfqpoint{2.675468in}{3.429978in}}%
\pgfpathclose%
\pgfusepath{fill}%
\end{pgfscope}%
\begin{pgfscope}%
\pgfpathrectangle{\pgfqpoint{1.020000in}{0.880000in}}{\pgfqpoint{6.160000in}{6.160000in}}%
\pgfusepath{clip}%
\pgfsetbuttcap%
\pgfsetroundjoin%
\definecolor{currentfill}{rgb}{0.791392,0.846750,0.936641}%
\pgfsetfillcolor{currentfill}%
\pgfsetlinewidth{0.000000pt}%
\definecolor{currentstroke}{rgb}{0.000000,0.000000,0.000000}%
\pgfsetstrokecolor{currentstroke}%
\pgfsetdash{}{0pt}%
\pgfpathmoveto{\pgfqpoint{3.531957in}{3.629920in}}%
\pgfpathlineto{\pgfqpoint{3.540593in}{3.655688in}}%
\pgfpathlineto{\pgfqpoint{3.548732in}{3.763855in}}%
\pgfpathlineto{\pgfqpoint{3.583751in}{3.651440in}}%
\pgfpathlineto{\pgfqpoint{3.618886in}{3.501405in}}%
\pgfpathlineto{\pgfqpoint{3.609314in}{3.622639in}}%
\pgfpathlineto{\pgfqpoint{3.599836in}{3.727282in}}%
\pgfpathlineto{\pgfqpoint{3.565288in}{3.776379in}}%
\pgfpathlineto{\pgfqpoint{3.531957in}{3.629920in}}%
\pgfpathclose%
\pgfusepath{fill}%
\end{pgfscope}%
\begin{pgfscope}%
\pgfpathrectangle{\pgfqpoint{1.020000in}{0.880000in}}{\pgfqpoint{6.160000in}{6.160000in}}%
\pgfusepath{clip}%
\pgfsetbuttcap%
\pgfsetroundjoin%
\definecolor{currentfill}{rgb}{0.661968,0.775491,0.993937}%
\pgfsetfillcolor{currentfill}%
\pgfsetlinewidth{0.000000pt}%
\definecolor{currentstroke}{rgb}{0.000000,0.000000,0.000000}%
\pgfsetstrokecolor{currentstroke}%
\pgfsetdash{}{0pt}%
\pgfpathmoveto{\pgfqpoint{4.288815in}{3.327427in}}%
\pgfpathlineto{\pgfqpoint{4.298725in}{3.581169in}}%
\pgfpathlineto{\pgfqpoint{4.308209in}{3.438795in}}%
\pgfpathlineto{\pgfqpoint{4.342137in}{3.332710in}}%
\pgfpathlineto{\pgfqpoint{4.376171in}{3.329601in}}%
\pgfpathlineto{\pgfqpoint{4.366330in}{3.256263in}}%
\pgfpathlineto{\pgfqpoint{4.357126in}{3.529132in}}%
\pgfpathlineto{\pgfqpoint{4.323010in}{3.473673in}}%
\pgfpathlineto{\pgfqpoint{4.288815in}{3.327427in}}%
\pgfpathclose%
\pgfusepath{fill}%
\end{pgfscope}%
\begin{pgfscope}%
\pgfpathrectangle{\pgfqpoint{1.020000in}{0.880000in}}{\pgfqpoint{6.160000in}{6.160000in}}%
\pgfusepath{clip}%
\pgfsetbuttcap%
\pgfsetroundjoin%
\definecolor{currentfill}{rgb}{0.630089,0.752516,0.998508}%
\pgfsetfillcolor{currentfill}%
\pgfsetlinewidth{0.000000pt}%
\definecolor{currentstroke}{rgb}{0.000000,0.000000,0.000000}%
\pgfsetstrokecolor{currentstroke}%
\pgfsetdash{}{0pt}%
\pgfpathmoveto{\pgfqpoint{2.900513in}{3.272521in}}%
\pgfpathlineto{\pgfqpoint{2.909854in}{3.176084in}}%
\pgfpathlineto{\pgfqpoint{2.915812in}{3.346568in}}%
\pgfpathlineto{\pgfqpoint{2.948841in}{3.466529in}}%
\pgfpathlineto{\pgfqpoint{2.984041in}{3.410677in}}%
\pgfpathlineto{\pgfqpoint{2.976687in}{3.339862in}}%
\pgfpathlineto{\pgfqpoint{2.966857in}{3.475093in}}%
\pgfpathlineto{\pgfqpoint{2.934870in}{3.277100in}}%
\pgfpathlineto{\pgfqpoint{2.900513in}{3.272521in}}%
\pgfpathclose%
\pgfusepath{fill}%
\end{pgfscope}%
\begin{pgfscope}%
\pgfpathrectangle{\pgfqpoint{1.020000in}{0.880000in}}{\pgfqpoint{6.160000in}{6.160000in}}%
\pgfusepath{clip}%
\pgfsetbuttcap%
\pgfsetroundjoin%
\definecolor{currentfill}{rgb}{0.724041,0.814910,0.975651}%
\pgfsetfillcolor{currentfill}%
\pgfsetlinewidth{0.000000pt}%
\definecolor{currentstroke}{rgb}{0.000000,0.000000,0.000000}%
\pgfsetstrokecolor{currentstroke}%
\pgfsetdash{}{0pt}%
\pgfpathmoveto{\pgfqpoint{3.102706in}{3.641654in}}%
\pgfpathlineto{\pgfqpoint{3.111776in}{3.574325in}}%
\pgfpathlineto{\pgfqpoint{3.121207in}{3.473588in}}%
\pgfpathlineto{\pgfqpoint{3.153881in}{3.644399in}}%
\pgfpathlineto{\pgfqpoint{3.189888in}{3.489183in}}%
\pgfpathlineto{\pgfqpoint{3.181597in}{3.475813in}}%
\pgfpathlineto{\pgfqpoint{3.173411in}{3.454105in}}%
\pgfpathlineto{\pgfqpoint{3.138847in}{3.476361in}}%
\pgfpathlineto{\pgfqpoint{3.102706in}{3.641654in}}%
\pgfpathclose%
\pgfusepath{fill}%
\end{pgfscope}%
\begin{pgfscope}%
\pgfpathrectangle{\pgfqpoint{1.020000in}{0.880000in}}{\pgfqpoint{6.160000in}{6.160000in}}%
\pgfusepath{clip}%
\pgfsetbuttcap%
\pgfsetroundjoin%
\definecolor{currentfill}{rgb}{0.285273,0.380129,0.823469}%
\pgfsetfillcolor{currentfill}%
\pgfsetlinewidth{0.000000pt}%
\definecolor{currentstroke}{rgb}{0.000000,0.000000,0.000000}%
\pgfsetstrokecolor{currentstroke}%
\pgfsetdash{}{0pt}%
\pgfpathmoveto{\pgfqpoint{6.112136in}{2.605801in}}%
\pgfpathlineto{\pgfqpoint{6.125054in}{2.672565in}}%
\pgfpathlineto{\pgfqpoint{6.135184in}{2.595675in}}%
\pgfpathlineto{\pgfqpoint{6.167335in}{2.528467in}}%
\pgfpathlineto{\pgfqpoint{6.203112in}{2.642651in}}%
\pgfpathlineto{\pgfqpoint{6.195764in}{2.857985in}}%
\pgfpathlineto{\pgfqpoint{6.182390in}{2.775130in}}%
\pgfpathlineto{\pgfqpoint{6.151535in}{2.907303in}}%
\pgfpathlineto{\pgfqpoint{6.112136in}{2.605801in}}%
\pgfpathclose%
\pgfusepath{fill}%
\end{pgfscope}%
\begin{pgfscope}%
\pgfpathrectangle{\pgfqpoint{1.020000in}{0.880000in}}{\pgfqpoint{6.160000in}{6.160000in}}%
\pgfusepath{clip}%
\pgfsetbuttcap%
\pgfsetroundjoin%
\definecolor{currentfill}{rgb}{0.304174,0.406945,0.845263}%
\pgfsetfillcolor{currentfill}%
\pgfsetlinewidth{0.000000pt}%
\definecolor{currentstroke}{rgb}{0.000000,0.000000,0.000000}%
\pgfsetstrokecolor{currentstroke}%
\pgfsetdash{}{0pt}%
\pgfpathmoveto{\pgfqpoint{5.888984in}{2.642224in}}%
\pgfpathlineto{\pgfqpoint{5.902614in}{2.772971in}}%
\pgfpathlineto{\pgfqpoint{5.917541in}{2.975054in}}%
\pgfpathlineto{\pgfqpoint{5.944978in}{2.623502in}}%
\pgfpathlineto{\pgfqpoint{5.977725in}{2.579609in}}%
\pgfpathlineto{\pgfqpoint{5.968207in}{2.687064in}}%
\pgfpathlineto{\pgfqpoint{5.961363in}{2.944189in}}%
\pgfpathlineto{\pgfqpoint{5.922815in}{2.660958in}}%
\pgfpathlineto{\pgfqpoint{5.888984in}{2.642224in}}%
\pgfpathclose%
\pgfusepath{fill}%
\end{pgfscope}%
\begin{pgfscope}%
\pgfpathrectangle{\pgfqpoint{1.020000in}{0.880000in}}{\pgfqpoint{6.160000in}{6.160000in}}%
\pgfusepath{clip}%
\pgfsetbuttcap%
\pgfsetroundjoin%
\definecolor{currentfill}{rgb}{0.363461,0.484784,0.901019}%
\pgfsetfillcolor{currentfill}%
\pgfsetlinewidth{0.000000pt}%
\definecolor{currentstroke}{rgb}{0.000000,0.000000,0.000000}%
\pgfsetstrokecolor{currentstroke}%
\pgfsetdash{}{0pt}%
\pgfpathmoveto{\pgfqpoint{5.220409in}{2.838564in}}%
\pgfpathlineto{\pgfqpoint{5.231952in}{2.924008in}}%
\pgfpathlineto{\pgfqpoint{5.242035in}{2.868311in}}%
\pgfpathlineto{\pgfqpoint{5.274722in}{2.765744in}}%
\pgfpathlineto{\pgfqpoint{5.309458in}{2.852079in}}%
\pgfpathlineto{\pgfqpoint{5.298790in}{2.858063in}}%
\pgfpathlineto{\pgfqpoint{5.287013in}{2.762205in}}%
\pgfpathlineto{\pgfqpoint{5.255124in}{2.929404in}}%
\pgfpathlineto{\pgfqpoint{5.220409in}{2.838564in}}%
\pgfpathclose%
\pgfusepath{fill}%
\end{pgfscope}%
\begin{pgfscope}%
\pgfpathrectangle{\pgfqpoint{1.020000in}{0.880000in}}{\pgfqpoint{6.160000in}{6.160000in}}%
\pgfusepath{clip}%
\pgfsetbuttcap%
\pgfsetroundjoin%
\definecolor{currentfill}{rgb}{0.603162,0.731527,0.999565}%
\pgfsetfillcolor{currentfill}%
\pgfsetlinewidth{0.000000pt}%
\definecolor{currentstroke}{rgb}{0.000000,0.000000,0.000000}%
\pgfsetstrokecolor{currentstroke}%
\pgfsetdash{}{0pt}%
\pgfpathmoveto{\pgfqpoint{2.832142in}{3.234137in}}%
\pgfpathlineto{\pgfqpoint{2.837198in}{3.455616in}}%
\pgfpathlineto{\pgfqpoint{2.847528in}{3.286866in}}%
\pgfpathlineto{\pgfqpoint{2.882132in}{3.281251in}}%
\pgfpathlineto{\pgfqpoint{2.915812in}{3.346568in}}%
\pgfpathlineto{\pgfqpoint{2.909854in}{3.176084in}}%
\pgfpathlineto{\pgfqpoint{2.900513in}{3.272521in}}%
\pgfpathlineto{\pgfqpoint{2.864982in}{3.355602in}}%
\pgfpathlineto{\pgfqpoint{2.832142in}{3.234137in}}%
\pgfpathclose%
\pgfusepath{fill}%
\end{pgfscope}%
\begin{pgfscope}%
\pgfpathrectangle{\pgfqpoint{1.020000in}{0.880000in}}{\pgfqpoint{6.160000in}{6.160000in}}%
\pgfusepath{clip}%
\pgfsetbuttcap%
\pgfsetroundjoin%
\definecolor{currentfill}{rgb}{0.724041,0.814910,0.975651}%
\pgfsetfillcolor{currentfill}%
\pgfsetlinewidth{0.000000pt}%
\definecolor{currentstroke}{rgb}{0.000000,0.000000,0.000000}%
\pgfsetstrokecolor{currentstroke}%
\pgfsetdash{}{0pt}%
\pgfpathmoveto{\pgfqpoint{4.133675in}{3.471009in}}%
\pgfpathlineto{\pgfqpoint{4.143043in}{3.685009in}}%
\pgfpathlineto{\pgfqpoint{4.152536in}{3.685756in}}%
\pgfpathlineto{\pgfqpoint{4.186731in}{3.675727in}}%
\pgfpathlineto{\pgfqpoint{4.220773in}{3.329878in}}%
\pgfpathlineto{\pgfqpoint{4.211263in}{3.396083in}}%
\pgfpathlineto{\pgfqpoint{4.201763in}{3.474609in}}%
\pgfpathlineto{\pgfqpoint{4.167730in}{3.465686in}}%
\pgfpathlineto{\pgfqpoint{4.133675in}{3.471009in}}%
\pgfpathclose%
\pgfusepath{fill}%
\end{pgfscope}%
\begin{pgfscope}%
\pgfpathrectangle{\pgfqpoint{1.020000in}{0.880000in}}{\pgfqpoint{6.160000in}{6.160000in}}%
\pgfusepath{clip}%
\pgfsetbuttcap%
\pgfsetroundjoin%
\definecolor{currentfill}{rgb}{0.703587,0.802586,0.982847}%
\pgfsetfillcolor{currentfill}%
\pgfsetlinewidth{0.000000pt}%
\definecolor{currentstroke}{rgb}{0.000000,0.000000,0.000000}%
\pgfsetstrokecolor{currentstroke}%
\pgfsetdash{}{0pt}%
\pgfpathmoveto{\pgfqpoint{3.035839in}{3.464787in}}%
\pgfpathlineto{\pgfqpoint{3.043873in}{3.485163in}}%
\pgfpathlineto{\pgfqpoint{3.052799in}{3.428429in}}%
\pgfpathlineto{\pgfqpoint{3.088537in}{3.311287in}}%
\pgfpathlineto{\pgfqpoint{3.121207in}{3.473588in}}%
\pgfpathlineto{\pgfqpoint{3.111776in}{3.574325in}}%
\pgfpathlineto{\pgfqpoint{3.102706in}{3.641654in}}%
\pgfpathlineto{\pgfqpoint{3.069696in}{3.513332in}}%
\pgfpathlineto{\pgfqpoint{3.035839in}{3.464787in}}%
\pgfpathclose%
\pgfusepath{fill}%
\end{pgfscope}%
\begin{pgfscope}%
\pgfpathrectangle{\pgfqpoint{1.020000in}{0.880000in}}{\pgfqpoint{6.160000in}{6.160000in}}%
\pgfusepath{clip}%
\pgfsetbuttcap%
\pgfsetroundjoin%
\definecolor{currentfill}{rgb}{0.763363,0.835092,0.955658}%
\pgfsetfillcolor{currentfill}%
\pgfsetlinewidth{0.000000pt}%
\definecolor{currentstroke}{rgb}{0.000000,0.000000,0.000000}%
\pgfsetstrokecolor{currentstroke}%
\pgfsetdash{}{0pt}%
\pgfpathmoveto{\pgfqpoint{3.687119in}{3.537387in}}%
\pgfpathlineto{\pgfqpoint{3.695838in}{3.588868in}}%
\pgfpathlineto{\pgfqpoint{3.704446in}{3.669435in}}%
\pgfpathlineto{\pgfqpoint{3.738543in}{3.715545in}}%
\pgfpathlineto{\pgfqpoint{3.773850in}{3.468521in}}%
\pgfpathlineto{\pgfqpoint{3.764648in}{3.504565in}}%
\pgfpathlineto{\pgfqpoint{3.755060in}{3.634011in}}%
\pgfpathlineto{\pgfqpoint{3.720625in}{3.685183in}}%
\pgfpathlineto{\pgfqpoint{3.687119in}{3.537387in}}%
\pgfpathclose%
\pgfusepath{fill}%
\end{pgfscope}%
\begin{pgfscope}%
\pgfpathrectangle{\pgfqpoint{1.020000in}{0.880000in}}{\pgfqpoint{6.160000in}{6.160000in}}%
\pgfusepath{clip}%
\pgfsetbuttcap%
\pgfsetroundjoin%
\definecolor{currentfill}{rgb}{0.630089,0.752516,0.998508}%
\pgfsetfillcolor{currentfill}%
\pgfsetlinewidth{0.000000pt}%
\definecolor{currentstroke}{rgb}{0.000000,0.000000,0.000000}%
\pgfsetstrokecolor{currentstroke}%
\pgfsetdash{}{0pt}%
\pgfpathmoveto{\pgfqpoint{2.608072in}{3.321275in}}%
\pgfpathlineto{\pgfqpoint{2.614755in}{3.392383in}}%
\pgfpathlineto{\pgfqpoint{2.621376in}{3.469306in}}%
\pgfpathlineto{\pgfqpoint{2.659277in}{3.256302in}}%
\pgfpathlineto{\pgfqpoint{2.693362in}{3.287786in}}%
\pgfpathlineto{\pgfqpoint{2.684422in}{3.358236in}}%
\pgfpathlineto{\pgfqpoint{2.675468in}{3.429978in}}%
\pgfpathlineto{\pgfqpoint{2.643704in}{3.249691in}}%
\pgfpathlineto{\pgfqpoint{2.608072in}{3.321275in}}%
\pgfpathclose%
\pgfusepath{fill}%
\end{pgfscope}%
\begin{pgfscope}%
\pgfpathrectangle{\pgfqpoint{1.020000in}{0.880000in}}{\pgfqpoint{6.160000in}{6.160000in}}%
\pgfusepath{clip}%
\pgfsetbuttcap%
\pgfsetroundjoin%
\definecolor{currentfill}{rgb}{0.266381,0.353304,0.801637}%
\pgfsetfillcolor{currentfill}%
\pgfsetlinewidth{0.000000pt}%
\definecolor{currentstroke}{rgb}{0.000000,0.000000,0.000000}%
\pgfsetstrokecolor{currentstroke}%
\pgfsetdash{}{0pt}%
\pgfpathmoveto{\pgfqpoint{6.050839in}{2.905693in}}%
\pgfpathlineto{\pgfqpoint{6.057630in}{2.652198in}}%
\pgfpathlineto{\pgfqpoint{6.066690in}{2.519889in}}%
\pgfpathlineto{\pgfqpoint{6.104221in}{2.727639in}}%
\pgfpathlineto{\pgfqpoint{6.135184in}{2.595675in}}%
\pgfpathlineto{\pgfqpoint{6.125054in}{2.672565in}}%
\pgfpathlineto{\pgfqpoint{6.112136in}{2.605801in}}%
\pgfpathlineto{\pgfqpoint{6.078265in}{2.583939in}}%
\pgfpathlineto{\pgfqpoint{6.050839in}{2.905693in}}%
\pgfpathclose%
\pgfusepath{fill}%
\end{pgfscope}%
\begin{pgfscope}%
\pgfpathrectangle{\pgfqpoint{1.020000in}{0.880000in}}{\pgfqpoint{6.160000in}{6.160000in}}%
\pgfusepath{clip}%
\pgfsetbuttcap%
\pgfsetroundjoin%
\definecolor{currentfill}{rgb}{0.743754,0.825125,0.965798}%
\pgfsetfillcolor{currentfill}%
\pgfsetlinewidth{0.000000pt}%
\definecolor{currentstroke}{rgb}{0.000000,0.000000,0.000000}%
\pgfsetstrokecolor{currentstroke}%
\pgfsetdash{}{0pt}%
\pgfpathmoveto{\pgfqpoint{3.910331in}{3.526690in}}%
\pgfpathlineto{\pgfqpoint{3.919300in}{3.619341in}}%
\pgfpathlineto{\pgfqpoint{3.928922in}{3.469601in}}%
\pgfpathlineto{\pgfqpoint{3.962724in}{3.640362in}}%
\pgfpathlineto{\pgfqpoint{3.997009in}{3.602737in}}%
\pgfpathlineto{\pgfqpoint{3.987737in}{3.583543in}}%
\pgfpathlineto{\pgfqpoint{3.978576in}{3.522362in}}%
\pgfpathlineto{\pgfqpoint{3.944328in}{3.581502in}}%
\pgfpathlineto{\pgfqpoint{3.910331in}{3.526690in}}%
\pgfpathclose%
\pgfusepath{fill}%
\end{pgfscope}%
\begin{pgfscope}%
\pgfpathrectangle{\pgfqpoint{1.020000in}{0.880000in}}{\pgfqpoint{6.160000in}{6.160000in}}%
\pgfusepath{clip}%
\pgfsetbuttcap%
\pgfsetroundjoin%
\definecolor{currentfill}{rgb}{0.677823,0.786546,0.991005}%
\pgfsetfillcolor{currentfill}%
\pgfsetlinewidth{0.000000pt}%
\definecolor{currentstroke}{rgb}{0.000000,0.000000,0.000000}%
\pgfsetstrokecolor{currentstroke}%
\pgfsetdash{}{0pt}%
\pgfpathmoveto{\pgfqpoint{2.966857in}{3.475093in}}%
\pgfpathlineto{\pgfqpoint{2.976687in}{3.339862in}}%
\pgfpathlineto{\pgfqpoint{2.984041in}{3.410677in}}%
\pgfpathlineto{\pgfqpoint{3.017917in}{3.463363in}}%
\pgfpathlineto{\pgfqpoint{3.052799in}{3.428429in}}%
\pgfpathlineto{\pgfqpoint{3.043873in}{3.485163in}}%
\pgfpathlineto{\pgfqpoint{3.035839in}{3.464787in}}%
\pgfpathlineto{\pgfqpoint{3.002018in}{3.414322in}}%
\pgfpathlineto{\pgfqpoint{2.966857in}{3.475093in}}%
\pgfpathclose%
\pgfusepath{fill}%
\end{pgfscope}%
\begin{pgfscope}%
\pgfpathrectangle{\pgfqpoint{1.020000in}{0.880000in}}{\pgfqpoint{6.160000in}{6.160000in}}%
\pgfusepath{clip}%
\pgfsetbuttcap%
\pgfsetroundjoin%
\definecolor{currentfill}{rgb}{0.782049,0.842864,0.942980}%
\pgfsetfillcolor{currentfill}%
\pgfsetlinewidth{0.000000pt}%
\definecolor{currentstroke}{rgb}{0.000000,0.000000,0.000000}%
\pgfsetstrokecolor{currentstroke}%
\pgfsetdash{}{0pt}%
\pgfpathmoveto{\pgfqpoint{3.464859in}{3.430919in}}%
\pgfpathlineto{\pgfqpoint{3.471250in}{3.758389in}}%
\pgfpathlineto{\pgfqpoint{3.481121in}{3.600255in}}%
\pgfpathlineto{\pgfqpoint{3.515597in}{3.578853in}}%
\pgfpathlineto{\pgfqpoint{3.548732in}{3.763855in}}%
\pgfpathlineto{\pgfqpoint{3.540593in}{3.655688in}}%
\pgfpathlineto{\pgfqpoint{3.531957in}{3.629920in}}%
\pgfpathlineto{\pgfqpoint{3.497040in}{3.726922in}}%
\pgfpathlineto{\pgfqpoint{3.464859in}{3.430919in}}%
\pgfpathclose%
\pgfusepath{fill}%
\end{pgfscope}%
\begin{pgfscope}%
\pgfpathrectangle{\pgfqpoint{1.020000in}{0.880000in}}{\pgfqpoint{6.160000in}{6.160000in}}%
\pgfusepath{clip}%
\pgfsetbuttcap%
\pgfsetroundjoin%
\definecolor{currentfill}{rgb}{0.543440,0.680003,0.993051}%
\pgfsetfillcolor{currentfill}%
\pgfsetlinewidth{0.000000pt}%
\definecolor{currentstroke}{rgb}{0.000000,0.000000,0.000000}%
\pgfsetstrokecolor{currentstroke}%
\pgfsetdash{}{0pt}%
\pgfpathmoveto{\pgfqpoint{4.531441in}{3.207001in}}%
\pgfpathlineto{\pgfqpoint{4.540904in}{3.087046in}}%
\pgfpathlineto{\pgfqpoint{4.551409in}{3.250042in}}%
\pgfpathlineto{\pgfqpoint{4.585481in}{3.256867in}}%
\pgfpathlineto{\pgfqpoint{4.619495in}{3.253915in}}%
\pgfpathlineto{\pgfqpoint{4.609016in}{3.139054in}}%
\pgfpathlineto{\pgfqpoint{4.598959in}{3.114551in}}%
\pgfpathlineto{\pgfqpoint{4.565476in}{3.226073in}}%
\pgfpathlineto{\pgfqpoint{4.531441in}{3.207001in}}%
\pgfpathclose%
\pgfusepath{fill}%
\end{pgfscope}%
\begin{pgfscope}%
\pgfpathrectangle{\pgfqpoint{1.020000in}{0.880000in}}{\pgfqpoint{6.160000in}{6.160000in}}%
\pgfusepath{clip}%
\pgfsetbuttcap%
\pgfsetroundjoin%
\definecolor{currentfill}{rgb}{0.441123,0.576532,0.954545}%
\pgfsetfillcolor{currentfill}%
\pgfsetlinewidth{0.000000pt}%
\definecolor{currentstroke}{rgb}{0.000000,0.000000,0.000000}%
\pgfsetstrokecolor{currentstroke}%
\pgfsetdash{}{0pt}%
\pgfpathmoveto{\pgfqpoint{4.842081in}{3.014899in}}%
\pgfpathlineto{\pgfqpoint{4.851581in}{2.905663in}}%
\pgfpathlineto{\pgfqpoint{4.863098in}{3.092214in}}%
\pgfpathlineto{\pgfqpoint{4.896791in}{3.054804in}}%
\pgfpathlineto{\pgfqpoint{4.929401in}{2.879071in}}%
\pgfpathlineto{\pgfqpoint{4.919581in}{2.945281in}}%
\pgfpathlineto{\pgfqpoint{4.909394in}{2.959906in}}%
\pgfpathlineto{\pgfqpoint{4.877047in}{3.172803in}}%
\pgfpathlineto{\pgfqpoint{4.842081in}{3.014899in}}%
\pgfpathclose%
\pgfusepath{fill}%
\end{pgfscope}%
\begin{pgfscope}%
\pgfpathrectangle{\pgfqpoint{1.020000in}{0.880000in}}{\pgfqpoint{6.160000in}{6.160000in}}%
\pgfusepath{clip}%
\pgfsetbuttcap%
\pgfsetroundjoin%
\definecolor{currentfill}{rgb}{0.309060,0.413498,0.850128}%
\pgfsetfillcolor{currentfill}%
\pgfsetlinewidth{0.000000pt}%
\definecolor{currentstroke}{rgb}{0.000000,0.000000,0.000000}%
\pgfsetstrokecolor{currentstroke}%
\pgfsetdash{}{0pt}%
\pgfpathmoveto{\pgfqpoint{5.598582in}{2.675605in}}%
\pgfpathlineto{\pgfqpoint{5.610340in}{2.724510in}}%
\pgfpathlineto{\pgfqpoint{5.622638in}{2.808291in}}%
\pgfpathlineto{\pgfqpoint{5.655874in}{2.775453in}}%
\pgfpathlineto{\pgfqpoint{5.689287in}{2.757162in}}%
\pgfpathlineto{\pgfqpoint{5.676140in}{2.624858in}}%
\pgfpathlineto{\pgfqpoint{5.666807in}{2.744217in}}%
\pgfpathlineto{\pgfqpoint{5.634373in}{2.824983in}}%
\pgfpathlineto{\pgfqpoint{5.598582in}{2.675605in}}%
\pgfpathclose%
\pgfusepath{fill}%
\end{pgfscope}%
\begin{pgfscope}%
\pgfpathrectangle{\pgfqpoint{1.020000in}{0.880000in}}{\pgfqpoint{6.160000in}{6.160000in}}%
\pgfusepath{clip}%
\pgfsetbuttcap%
\pgfsetroundjoin%
\definecolor{currentfill}{rgb}{0.343278,0.459354,0.884122}%
\pgfsetfillcolor{currentfill}%
\pgfsetlinewidth{0.000000pt}%
\definecolor{currentstroke}{rgb}{0.000000,0.000000,0.000000}%
\pgfsetstrokecolor{currentstroke}%
\pgfsetdash{}{0pt}%
\pgfpathmoveto{\pgfqpoint{5.151637in}{2.711807in}}%
\pgfpathlineto{\pgfqpoint{5.162849in}{2.777638in}}%
\pgfpathlineto{\pgfqpoint{5.172549in}{2.688119in}}%
\pgfpathlineto{\pgfqpoint{5.210076in}{3.050307in}}%
\pgfpathlineto{\pgfqpoint{5.242035in}{2.868311in}}%
\pgfpathlineto{\pgfqpoint{5.231952in}{2.924008in}}%
\pgfpathlineto{\pgfqpoint{5.220409in}{2.838564in}}%
\pgfpathlineto{\pgfqpoint{5.184865in}{2.661548in}}%
\pgfpathlineto{\pgfqpoint{5.151637in}{2.711807in}}%
\pgfpathclose%
\pgfusepath{fill}%
\end{pgfscope}%
\begin{pgfscope}%
\pgfpathrectangle{\pgfqpoint{1.020000in}{0.880000in}}{\pgfqpoint{6.160000in}{6.160000in}}%
\pgfusepath{clip}%
\pgfsetbuttcap%
\pgfsetroundjoin%
\definecolor{currentfill}{rgb}{0.394042,0.522413,0.924916}%
\pgfsetfillcolor{currentfill}%
\pgfsetlinewidth{0.000000pt}%
\definecolor{currentstroke}{rgb}{0.000000,0.000000,0.000000}%
\pgfsetstrokecolor{currentstroke}%
\pgfsetdash{}{0pt}%
\pgfpathmoveto{\pgfqpoint{4.999026in}{3.117505in}}%
\pgfpathlineto{\pgfqpoint{5.007529in}{2.885620in}}%
\pgfpathlineto{\pgfqpoint{5.019289in}{3.044938in}}%
\pgfpathlineto{\pgfqpoint{5.051956in}{2.900332in}}%
\pgfpathlineto{\pgfqpoint{5.085310in}{2.844436in}}%
\pgfpathlineto{\pgfqpoint{5.075662in}{2.937884in}}%
\pgfpathlineto{\pgfqpoint{5.063563in}{2.754420in}}%
\pgfpathlineto{\pgfqpoint{5.030346in}{2.816214in}}%
\pgfpathlineto{\pgfqpoint{4.999026in}{3.117505in}}%
\pgfpathclose%
\pgfusepath{fill}%
\end{pgfscope}%
\begin{pgfscope}%
\pgfpathrectangle{\pgfqpoint{1.020000in}{0.880000in}}{\pgfqpoint{6.160000in}{6.160000in}}%
\pgfusepath{clip}%
\pgfsetbuttcap%
\pgfsetroundjoin%
\definecolor{currentfill}{rgb}{0.505423,0.643995,0.983157}%
\pgfsetfillcolor{currentfill}%
\pgfsetlinewidth{0.000000pt}%
\definecolor{currentstroke}{rgb}{0.000000,0.000000,0.000000}%
\pgfsetstrokecolor{currentstroke}%
\pgfsetdash{}{0pt}%
\pgfpathmoveto{\pgfqpoint{4.686723in}{3.103465in}}%
\pgfpathlineto{\pgfqpoint{4.696474in}{3.042864in}}%
\pgfpathlineto{\pgfqpoint{4.707536in}{3.230116in}}%
\pgfpathlineto{\pgfqpoint{4.740732in}{3.082211in}}%
\pgfpathlineto{\pgfqpoint{4.775391in}{3.201272in}}%
\pgfpathlineto{\pgfqpoint{4.764096in}{3.007242in}}%
\pgfpathlineto{\pgfqpoint{4.754405in}{3.082095in}}%
\pgfpathlineto{\pgfqpoint{4.720918in}{3.155938in}}%
\pgfpathlineto{\pgfqpoint{4.686723in}{3.103465in}}%
\pgfpathclose%
\pgfusepath{fill}%
\end{pgfscope}%
\begin{pgfscope}%
\pgfpathrectangle{\pgfqpoint{1.020000in}{0.880000in}}{\pgfqpoint{6.160000in}{6.160000in}}%
\pgfusepath{clip}%
\pgfsetbuttcap%
\pgfsetroundjoin%
\definecolor{currentfill}{rgb}{0.261805,0.346484,0.795658}%
\pgfsetfillcolor{currentfill}%
\pgfsetlinewidth{0.000000pt}%
\definecolor{currentstroke}{rgb}{0.000000,0.000000,0.000000}%
\pgfsetstrokecolor{currentstroke}%
\pgfsetdash{}{0pt}%
\pgfpathmoveto{\pgfqpoint{5.977725in}{2.579609in}}%
\pgfpathlineto{\pgfqpoint{5.989758in}{2.610216in}}%
\pgfpathlineto{\pgfqpoint{6.001215in}{2.607619in}}%
\pgfpathlineto{\pgfqpoint{6.034608in}{2.597512in}}%
\pgfpathlineto{\pgfqpoint{6.066690in}{2.519889in}}%
\pgfpathlineto{\pgfqpoint{6.057630in}{2.652198in}}%
\pgfpathlineto{\pgfqpoint{6.050839in}{2.905693in}}%
\pgfpathlineto{\pgfqpoint{6.012236in}{2.634147in}}%
\pgfpathlineto{\pgfqpoint{5.977725in}{2.579609in}}%
\pgfpathclose%
\pgfusepath{fill}%
\end{pgfscope}%
\begin{pgfscope}%
\pgfpathrectangle{\pgfqpoint{1.020000in}{0.880000in}}{\pgfqpoint{6.160000in}{6.160000in}}%
\pgfusepath{clip}%
\pgfsetbuttcap%
\pgfsetroundjoin%
\definecolor{currentfill}{rgb}{0.309060,0.413498,0.850128}%
\pgfsetfillcolor{currentfill}%
\pgfsetlinewidth{0.000000pt}%
\definecolor{currentstroke}{rgb}{0.000000,0.000000,0.000000}%
\pgfsetstrokecolor{currentstroke}%
\pgfsetdash{}{0pt}%
\pgfpathmoveto{\pgfqpoint{5.820761in}{2.572072in}}%
\pgfpathlineto{\pgfqpoint{5.834896in}{2.743044in}}%
\pgfpathlineto{\pgfqpoint{5.845390in}{2.692762in}}%
\pgfpathlineto{\pgfqpoint{5.880103in}{2.756348in}}%
\pgfpathlineto{\pgfqpoint{5.917541in}{2.975054in}}%
\pgfpathlineto{\pgfqpoint{5.902614in}{2.772971in}}%
\pgfpathlineto{\pgfqpoint{5.888984in}{2.642224in}}%
\pgfpathlineto{\pgfqpoint{5.857665in}{2.772896in}}%
\pgfpathlineto{\pgfqpoint{5.820761in}{2.572072in}}%
\pgfpathclose%
\pgfusepath{fill}%
\end{pgfscope}%
\begin{pgfscope}%
\pgfpathrectangle{\pgfqpoint{1.020000in}{0.880000in}}{\pgfqpoint{6.160000in}{6.160000in}}%
\pgfusepath{clip}%
\pgfsetbuttcap%
\pgfsetroundjoin%
\definecolor{currentfill}{rgb}{0.672538,0.782861,0.991982}%
\pgfsetfillcolor{currentfill}%
\pgfsetlinewidth{0.000000pt}%
\definecolor{currentstroke}{rgb}{0.000000,0.000000,0.000000}%
\pgfsetstrokecolor{currentstroke}%
\pgfsetdash{}{0pt}%
\pgfpathmoveto{\pgfqpoint{4.220773in}{3.329878in}}%
\pgfpathlineto{\pgfqpoint{4.230389in}{3.471576in}}%
\pgfpathlineto{\pgfqpoint{4.239965in}{3.458446in}}%
\pgfpathlineto{\pgfqpoint{4.274027in}{3.367317in}}%
\pgfpathlineto{\pgfqpoint{4.308209in}{3.438795in}}%
\pgfpathlineto{\pgfqpoint{4.298725in}{3.581169in}}%
\pgfpathlineto{\pgfqpoint{4.288815in}{3.327427in}}%
\pgfpathlineto{\pgfqpoint{4.254871in}{3.421540in}}%
\pgfpathlineto{\pgfqpoint{4.220773in}{3.329878in}}%
\pgfpathclose%
\pgfusepath{fill}%
\end{pgfscope}%
\begin{pgfscope}%
\pgfpathrectangle{\pgfqpoint{1.020000in}{0.880000in}}{\pgfqpoint{6.160000in}{6.160000in}}%
\pgfusepath{clip}%
\pgfsetbuttcap%
\pgfsetroundjoin%
\definecolor{currentfill}{rgb}{0.348323,0.465711,0.888346}%
\pgfsetfillcolor{currentfill}%
\pgfsetlinewidth{0.000000pt}%
\definecolor{currentstroke}{rgb}{0.000000,0.000000,0.000000}%
\pgfsetstrokecolor{currentstroke}%
\pgfsetdash{}{0pt}%
\pgfpathmoveto{\pgfqpoint{5.378244in}{2.957953in}}%
\pgfpathlineto{\pgfqpoint{5.386784in}{2.768388in}}%
\pgfpathlineto{\pgfqpoint{5.396898in}{2.708958in}}%
\pgfpathlineto{\pgfqpoint{5.433719in}{2.953801in}}%
\pgfpathlineto{\pgfqpoint{5.463573in}{2.647395in}}%
\pgfpathlineto{\pgfqpoint{5.454672in}{2.803197in}}%
\pgfpathlineto{\pgfqpoint{5.445008in}{2.899684in}}%
\pgfpathlineto{\pgfqpoint{5.410129in}{2.805904in}}%
\pgfpathlineto{\pgfqpoint{5.378244in}{2.957953in}}%
\pgfpathclose%
\pgfusepath{fill}%
\end{pgfscope}%
\begin{pgfscope}%
\pgfpathrectangle{\pgfqpoint{1.020000in}{0.880000in}}{\pgfqpoint{6.160000in}{6.160000in}}%
\pgfusepath{clip}%
\pgfsetbuttcap%
\pgfsetroundjoin%
\definecolor{currentfill}{rgb}{0.724041,0.814910,0.975651}%
\pgfsetfillcolor{currentfill}%
\pgfsetlinewidth{0.000000pt}%
\definecolor{currentstroke}{rgb}{0.000000,0.000000,0.000000}%
\pgfsetstrokecolor{currentstroke}%
\pgfsetdash{}{0pt}%
\pgfpathmoveto{\pgfqpoint{3.258151in}{3.543253in}}%
\pgfpathlineto{\pgfqpoint{3.266488in}{3.562918in}}%
\pgfpathlineto{\pgfqpoint{3.276475in}{3.403092in}}%
\pgfpathlineto{\pgfqpoint{3.310122in}{3.488259in}}%
\pgfpathlineto{\pgfqpoint{3.344290in}{3.515683in}}%
\pgfpathlineto{\pgfqpoint{3.335990in}{3.479921in}}%
\pgfpathlineto{\pgfqpoint{3.326900in}{3.539480in}}%
\pgfpathlineto{\pgfqpoint{3.292161in}{3.583624in}}%
\pgfpathlineto{\pgfqpoint{3.258151in}{3.543253in}}%
\pgfpathclose%
\pgfusepath{fill}%
\end{pgfscope}%
\begin{pgfscope}%
\pgfpathrectangle{\pgfqpoint{1.020000in}{0.880000in}}{\pgfqpoint{6.160000in}{6.160000in}}%
\pgfusepath{clip}%
\pgfsetbuttcap%
\pgfsetroundjoin%
\definecolor{currentfill}{rgb}{0.768034,0.837035,0.952488}%
\pgfsetfillcolor{currentfill}%
\pgfsetlinewidth{0.000000pt}%
\definecolor{currentstroke}{rgb}{0.000000,0.000000,0.000000}%
\pgfsetstrokecolor{currentstroke}%
\pgfsetdash{}{0pt}%
\pgfpathmoveto{\pgfqpoint{3.618886in}{3.501405in}}%
\pgfpathlineto{\pgfqpoint{3.626724in}{3.692208in}}%
\pgfpathlineto{\pgfqpoint{3.635860in}{3.655060in}}%
\pgfpathlineto{\pgfqpoint{3.670847in}{3.528155in}}%
\pgfpathlineto{\pgfqpoint{3.704446in}{3.669435in}}%
\pgfpathlineto{\pgfqpoint{3.695838in}{3.588868in}}%
\pgfpathlineto{\pgfqpoint{3.687119in}{3.537387in}}%
\pgfpathlineto{\pgfqpoint{3.652205in}{3.670825in}}%
\pgfpathlineto{\pgfqpoint{3.618886in}{3.501405in}}%
\pgfpathclose%
\pgfusepath{fill}%
\end{pgfscope}%
\begin{pgfscope}%
\pgfpathrectangle{\pgfqpoint{1.020000in}{0.880000in}}{\pgfqpoint{6.160000in}{6.160000in}}%
\pgfusepath{clip}%
\pgfsetbuttcap%
\pgfsetroundjoin%
\definecolor{currentfill}{rgb}{0.758539,0.832787,0.958408}%
\pgfsetfillcolor{currentfill}%
\pgfsetlinewidth{0.000000pt}%
\definecolor{currentstroke}{rgb}{0.000000,0.000000,0.000000}%
\pgfsetstrokecolor{currentstroke}%
\pgfsetdash{}{0pt}%
\pgfpathmoveto{\pgfqpoint{3.326900in}{3.539480in}}%
\pgfpathlineto{\pgfqpoint{3.335990in}{3.479921in}}%
\pgfpathlineto{\pgfqpoint{3.344290in}{3.515683in}}%
\pgfpathlineto{\pgfqpoint{3.378859in}{3.493129in}}%
\pgfpathlineto{\pgfqpoint{3.411322in}{3.739449in}}%
\pgfpathlineto{\pgfqpoint{3.403209in}{3.665137in}}%
\pgfpathlineto{\pgfqpoint{3.392841in}{3.881312in}}%
\pgfpathlineto{\pgfqpoint{3.362599in}{3.372287in}}%
\pgfpathlineto{\pgfqpoint{3.326900in}{3.539480in}}%
\pgfpathclose%
\pgfusepath{fill}%
\end{pgfscope}%
\begin{pgfscope}%
\pgfpathrectangle{\pgfqpoint{1.020000in}{0.880000in}}{\pgfqpoint{6.160000in}{6.160000in}}%
\pgfusepath{clip}%
\pgfsetbuttcap%
\pgfsetroundjoin%
\definecolor{currentfill}{rgb}{0.640828,0.760752,0.997846}%
\pgfsetfillcolor{currentfill}%
\pgfsetlinewidth{0.000000pt}%
\definecolor{currentstroke}{rgb}{0.000000,0.000000,0.000000}%
\pgfsetstrokecolor{currentstroke}%
\pgfsetdash{}{0pt}%
\pgfpathmoveto{\pgfqpoint{4.376171in}{3.329601in}}%
\pgfpathlineto{\pgfqpoint{4.386040in}{3.399198in}}%
\pgfpathlineto{\pgfqpoint{4.395823in}{3.412385in}}%
\pgfpathlineto{\pgfqpoint{4.429410in}{3.198300in}}%
\pgfpathlineto{\pgfqpoint{4.463734in}{3.306501in}}%
\pgfpathlineto{\pgfqpoint{4.453889in}{3.295124in}}%
\pgfpathlineto{\pgfqpoint{4.444470in}{3.437998in}}%
\pgfpathlineto{\pgfqpoint{4.410507in}{3.469514in}}%
\pgfpathlineto{\pgfqpoint{4.376171in}{3.329601in}}%
\pgfpathclose%
\pgfusepath{fill}%
\end{pgfscope}%
\begin{pgfscope}%
\pgfpathrectangle{\pgfqpoint{1.020000in}{0.880000in}}{\pgfqpoint{6.160000in}{6.160000in}}%
\pgfusepath{clip}%
\pgfsetbuttcap%
\pgfsetroundjoin%
\definecolor{currentfill}{rgb}{0.348323,0.465711,0.888346}%
\pgfsetfillcolor{currentfill}%
\pgfsetlinewidth{0.000000pt}%
\definecolor{currentstroke}{rgb}{0.000000,0.000000,0.000000}%
\pgfsetstrokecolor{currentstroke}%
\pgfsetdash{}{0pt}%
\pgfpathmoveto{\pgfqpoint{5.085310in}{2.844436in}}%
\pgfpathlineto{\pgfqpoint{5.096884in}{2.960466in}}%
\pgfpathlineto{\pgfqpoint{5.105811in}{2.786347in}}%
\pgfpathlineto{\pgfqpoint{5.140477in}{2.869121in}}%
\pgfpathlineto{\pgfqpoint{5.172549in}{2.688119in}}%
\pgfpathlineto{\pgfqpoint{5.162849in}{2.777638in}}%
\pgfpathlineto{\pgfqpoint{5.151637in}{2.711807in}}%
\pgfpathlineto{\pgfqpoint{5.119508in}{2.884528in}}%
\pgfpathlineto{\pgfqpoint{5.085310in}{2.844436in}}%
\pgfpathclose%
\pgfusepath{fill}%
\end{pgfscope}%
\begin{pgfscope}%
\pgfpathrectangle{\pgfqpoint{1.020000in}{0.880000in}}{\pgfqpoint{6.160000in}{6.160000in}}%
\pgfusepath{clip}%
\pgfsetbuttcap%
\pgfsetroundjoin%
\definecolor{currentfill}{rgb}{0.333490,0.446265,0.874452}%
\pgfsetfillcolor{currentfill}%
\pgfsetlinewidth{0.000000pt}%
\definecolor{currentstroke}{rgb}{0.000000,0.000000,0.000000}%
\pgfsetstrokecolor{currentstroke}%
\pgfsetdash{}{0pt}%
\pgfpathmoveto{\pgfqpoint{5.309458in}{2.852079in}}%
\pgfpathlineto{\pgfqpoint{5.319192in}{2.762473in}}%
\pgfpathlineto{\pgfqpoint{5.329183in}{2.694793in}}%
\pgfpathlineto{\pgfqpoint{5.363773in}{2.762859in}}%
\pgfpathlineto{\pgfqpoint{5.396898in}{2.708958in}}%
\pgfpathlineto{\pgfqpoint{5.386784in}{2.768388in}}%
\pgfpathlineto{\pgfqpoint{5.378244in}{2.957953in}}%
\pgfpathlineto{\pgfqpoint{5.342348in}{2.776833in}}%
\pgfpathlineto{\pgfqpoint{5.309458in}{2.852079in}}%
\pgfpathclose%
\pgfusepath{fill}%
\end{pgfscope}%
\begin{pgfscope}%
\pgfpathrectangle{\pgfqpoint{1.020000in}{0.880000in}}{\pgfqpoint{6.160000in}{6.160000in}}%
\pgfusepath{clip}%
\pgfsetbuttcap%
\pgfsetroundjoin%
\definecolor{currentfill}{rgb}{0.624703,0.748318,0.998719}%
\pgfsetfillcolor{currentfill}%
\pgfsetlinewidth{0.000000pt}%
\definecolor{currentstroke}{rgb}{0.000000,0.000000,0.000000}%
\pgfsetstrokecolor{currentstroke}%
\pgfsetdash{}{0pt}%
\pgfpathmoveto{\pgfqpoint{2.761401in}{3.360191in}}%
\pgfpathlineto{\pgfqpoint{2.770547in}{3.277526in}}%
\pgfpathlineto{\pgfqpoint{2.777425in}{3.356406in}}%
\pgfpathlineto{\pgfqpoint{2.813464in}{3.251293in}}%
\pgfpathlineto{\pgfqpoint{2.847528in}{3.286866in}}%
\pgfpathlineto{\pgfqpoint{2.837198in}{3.455616in}}%
\pgfpathlineto{\pgfqpoint{2.832142in}{3.234137in}}%
\pgfpathlineto{\pgfqpoint{2.794864in}{3.436847in}}%
\pgfpathlineto{\pgfqpoint{2.761401in}{3.360191in}}%
\pgfpathclose%
\pgfusepath{fill}%
\end{pgfscope}%
\begin{pgfscope}%
\pgfpathrectangle{\pgfqpoint{1.020000in}{0.880000in}}{\pgfqpoint{6.160000in}{6.160000in}}%
\pgfusepath{clip}%
\pgfsetbuttcap%
\pgfsetroundjoin%
\definecolor{currentfill}{rgb}{0.796064,0.848693,0.933471}%
\pgfsetfillcolor{currentfill}%
\pgfsetlinewidth{0.000000pt}%
\definecolor{currentstroke}{rgb}{0.000000,0.000000,0.000000}%
\pgfsetstrokecolor{currentstroke}%
\pgfsetdash{}{0pt}%
\pgfpathmoveto{\pgfqpoint{3.392841in}{3.881312in}}%
\pgfpathlineto{\pgfqpoint{3.403209in}{3.665137in}}%
\pgfpathlineto{\pgfqpoint{3.411322in}{3.739449in}}%
\pgfpathlineto{\pgfqpoint{3.446542in}{3.630891in}}%
\pgfpathlineto{\pgfqpoint{3.481121in}{3.600255in}}%
\pgfpathlineto{\pgfqpoint{3.471250in}{3.758389in}}%
\pgfpathlineto{\pgfqpoint{3.464859in}{3.430919in}}%
\pgfpathlineto{\pgfqpoint{3.429288in}{3.609541in}}%
\pgfpathlineto{\pgfqpoint{3.392841in}{3.881312in}}%
\pgfpathclose%
\pgfusepath{fill}%
\end{pgfscope}%
\begin{pgfscope}%
\pgfpathrectangle{\pgfqpoint{1.020000in}{0.880000in}}{\pgfqpoint{6.160000in}{6.160000in}}%
\pgfusepath{clip}%
\pgfsetbuttcap%
\pgfsetroundjoin%
\definecolor{currentfill}{rgb}{0.651398,0.768121,0.995891}%
\pgfsetfillcolor{currentfill}%
\pgfsetlinewidth{0.000000pt}%
\definecolor{currentstroke}{rgb}{0.000000,0.000000,0.000000}%
\pgfsetstrokecolor{currentstroke}%
\pgfsetdash{}{0pt}%
\pgfpathmoveto{\pgfqpoint{2.536782in}{3.453188in}}%
\pgfpathlineto{\pgfqpoint{2.546317in}{3.343552in}}%
\pgfpathlineto{\pgfqpoint{2.555520in}{3.254287in}}%
\pgfpathlineto{\pgfqpoint{2.588129in}{3.380130in}}%
\pgfpathlineto{\pgfqpoint{2.621376in}{3.469306in}}%
\pgfpathlineto{\pgfqpoint{2.614755in}{3.392383in}}%
\pgfpathlineto{\pgfqpoint{2.608072in}{3.321275in}}%
\pgfpathlineto{\pgfqpoint{2.572805in}{3.365843in}}%
\pgfpathlineto{\pgfqpoint{2.536782in}{3.453188in}}%
\pgfpathclose%
\pgfusepath{fill}%
\end{pgfscope}%
\begin{pgfscope}%
\pgfpathrectangle{\pgfqpoint{1.020000in}{0.880000in}}{\pgfqpoint{6.160000in}{6.160000in}}%
\pgfusepath{clip}%
\pgfsetbuttcap%
\pgfsetroundjoin%
\definecolor{currentfill}{rgb}{0.510824,0.649397,0.985079}%
\pgfsetfillcolor{currentfill}%
\pgfsetlinewidth{0.000000pt}%
\definecolor{currentstroke}{rgb}{0.000000,0.000000,0.000000}%
\pgfsetstrokecolor{currentstroke}%
\pgfsetdash{}{0pt}%
\pgfpathmoveto{\pgfqpoint{4.619495in}{3.253915in}}%
\pgfpathlineto{\pgfqpoint{4.629092in}{3.161702in}}%
\pgfpathlineto{\pgfqpoint{4.638965in}{3.128792in}}%
\pgfpathlineto{\pgfqpoint{4.671833in}{2.892354in}}%
\pgfpathlineto{\pgfqpoint{4.707536in}{3.230116in}}%
\pgfpathlineto{\pgfqpoint{4.696474in}{3.042864in}}%
\pgfpathlineto{\pgfqpoint{4.686723in}{3.103465in}}%
\pgfpathlineto{\pgfqpoint{4.653094in}{3.167786in}}%
\pgfpathlineto{\pgfqpoint{4.619495in}{3.253915in}}%
\pgfpathclose%
\pgfusepath{fill}%
\end{pgfscope}%
\begin{pgfscope}%
\pgfpathrectangle{\pgfqpoint{1.020000in}{0.880000in}}{\pgfqpoint{6.160000in}{6.160000in}}%
\pgfusepath{clip}%
\pgfsetbuttcap%
\pgfsetroundjoin%
\definecolor{currentfill}{rgb}{0.328604,0.439712,0.869587}%
\pgfsetfillcolor{currentfill}%
\pgfsetlinewidth{0.000000pt}%
\definecolor{currentstroke}{rgb}{0.000000,0.000000,0.000000}%
\pgfsetstrokecolor{currentstroke}%
\pgfsetdash{}{0pt}%
\pgfpathmoveto{\pgfqpoint{5.531134in}{2.663347in}}%
\pgfpathlineto{\pgfqpoint{5.544974in}{2.871708in}}%
\pgfpathlineto{\pgfqpoint{5.553082in}{2.659747in}}%
\pgfpathlineto{\pgfqpoint{5.588636in}{2.789950in}}%
\pgfpathlineto{\pgfqpoint{5.622638in}{2.808291in}}%
\pgfpathlineto{\pgfqpoint{5.610340in}{2.724510in}}%
\pgfpathlineto{\pgfqpoint{5.598582in}{2.675605in}}%
\pgfpathlineto{\pgfqpoint{5.568665in}{2.942299in}}%
\pgfpathlineto{\pgfqpoint{5.531134in}{2.663347in}}%
\pgfpathclose%
\pgfusepath{fill}%
\end{pgfscope}%
\begin{pgfscope}%
\pgfpathrectangle{\pgfqpoint{1.020000in}{0.880000in}}{\pgfqpoint{6.160000in}{6.160000in}}%
\pgfusepath{clip}%
\pgfsetbuttcap%
\pgfsetroundjoin%
\definecolor{currentfill}{rgb}{0.313946,0.420052,0.854993}%
\pgfsetfillcolor{currentfill}%
\pgfsetlinewidth{0.000000pt}%
\definecolor{currentstroke}{rgb}{0.000000,0.000000,0.000000}%
\pgfsetstrokecolor{currentstroke}%
\pgfsetdash{}{0pt}%
\pgfpathmoveto{\pgfqpoint{5.756765in}{2.768021in}}%
\pgfpathlineto{\pgfqpoint{5.767019in}{2.705702in}}%
\pgfpathlineto{\pgfqpoint{5.780332in}{2.833142in}}%
\pgfpathlineto{\pgfqpoint{5.814005in}{2.830581in}}%
\pgfpathlineto{\pgfqpoint{5.845390in}{2.692762in}}%
\pgfpathlineto{\pgfqpoint{5.834896in}{2.743044in}}%
\pgfpathlineto{\pgfqpoint{5.820761in}{2.572072in}}%
\pgfpathlineto{\pgfqpoint{5.790167in}{2.754177in}}%
\pgfpathlineto{\pgfqpoint{5.756765in}{2.768021in}}%
\pgfpathclose%
\pgfusepath{fill}%
\end{pgfscope}%
\begin{pgfscope}%
\pgfpathrectangle{\pgfqpoint{1.020000in}{0.880000in}}{\pgfqpoint{6.160000in}{6.160000in}}%
\pgfusepath{clip}%
\pgfsetbuttcap%
\pgfsetroundjoin%
\definecolor{currentfill}{rgb}{0.758539,0.832787,0.958408}%
\pgfsetfillcolor{currentfill}%
\pgfsetlinewidth{0.000000pt}%
\definecolor{currentstroke}{rgb}{0.000000,0.000000,0.000000}%
\pgfsetstrokecolor{currentstroke}%
\pgfsetdash{}{0pt}%
\pgfpathmoveto{\pgfqpoint{4.065373in}{3.583254in}}%
\pgfpathlineto{\pgfqpoint{4.074852in}{3.506586in}}%
\pgfpathlineto{\pgfqpoint{4.084325in}{3.441710in}}%
\pgfpathlineto{\pgfqpoint{4.118364in}{3.618903in}}%
\pgfpathlineto{\pgfqpoint{4.152536in}{3.685756in}}%
\pgfpathlineto{\pgfqpoint{4.143043in}{3.685009in}}%
\pgfpathlineto{\pgfqpoint{4.133675in}{3.471009in}}%
\pgfpathlineto{\pgfqpoint{4.099451in}{3.650162in}}%
\pgfpathlineto{\pgfqpoint{4.065373in}{3.583254in}}%
\pgfpathclose%
\pgfusepath{fill}%
\end{pgfscope}%
\begin{pgfscope}%
\pgfpathrectangle{\pgfqpoint{1.020000in}{0.880000in}}{\pgfqpoint{6.160000in}{6.160000in}}%
\pgfusepath{clip}%
\pgfsetbuttcap%
\pgfsetroundjoin%
\definecolor{currentfill}{rgb}{0.419991,0.552989,0.942630}%
\pgfsetfillcolor{currentfill}%
\pgfsetlinewidth{0.000000pt}%
\definecolor{currentstroke}{rgb}{0.000000,0.000000,0.000000}%
\pgfsetstrokecolor{currentstroke}%
\pgfsetdash{}{0pt}%
\pgfpathmoveto{\pgfqpoint{4.929401in}{2.879071in}}%
\pgfpathlineto{\pgfqpoint{4.939715in}{2.876740in}}%
\pgfpathlineto{\pgfqpoint{4.948784in}{2.710017in}}%
\pgfpathlineto{\pgfqpoint{4.984944in}{2.995980in}}%
\pgfpathlineto{\pgfqpoint{5.019289in}{3.044938in}}%
\pgfpathlineto{\pgfqpoint{5.007529in}{2.885620in}}%
\pgfpathlineto{\pgfqpoint{4.999026in}{3.117505in}}%
\pgfpathlineto{\pgfqpoint{4.964777in}{3.073972in}}%
\pgfpathlineto{\pgfqpoint{4.929401in}{2.879071in}}%
\pgfpathclose%
\pgfusepath{fill}%
\end{pgfscope}%
\begin{pgfscope}%
\pgfpathrectangle{\pgfqpoint{1.020000in}{0.880000in}}{\pgfqpoint{6.160000in}{6.160000in}}%
\pgfusepath{clip}%
\pgfsetbuttcap%
\pgfsetroundjoin%
\definecolor{currentfill}{rgb}{0.661968,0.775491,0.993937}%
\pgfsetfillcolor{currentfill}%
\pgfsetlinewidth{0.000000pt}%
\definecolor{currentstroke}{rgb}{0.000000,0.000000,0.000000}%
\pgfsetstrokecolor{currentstroke}%
\pgfsetdash{}{0pt}%
\pgfpathmoveto{\pgfqpoint{3.052799in}{3.428429in}}%
\pgfpathlineto{\pgfqpoint{3.063335in}{3.229078in}}%
\pgfpathlineto{\pgfqpoint{3.068687in}{3.493721in}}%
\pgfpathlineto{\pgfqpoint{3.106338in}{3.206030in}}%
\pgfpathlineto{\pgfqpoint{3.138239in}{3.447742in}}%
\pgfpathlineto{\pgfqpoint{3.128946in}{3.533699in}}%
\pgfpathlineto{\pgfqpoint{3.121207in}{3.473588in}}%
\pgfpathlineto{\pgfqpoint{3.088537in}{3.311287in}}%
\pgfpathlineto{\pgfqpoint{3.052799in}{3.428429in}}%
\pgfpathclose%
\pgfusepath{fill}%
\end{pgfscope}%
\begin{pgfscope}%
\pgfpathrectangle{\pgfqpoint{1.020000in}{0.880000in}}{\pgfqpoint{6.160000in}{6.160000in}}%
\pgfusepath{clip}%
\pgfsetbuttcap%
\pgfsetroundjoin%
\definecolor{currentfill}{rgb}{0.323718,0.433158,0.864722}%
\pgfsetfillcolor{currentfill}%
\pgfsetlinewidth{0.000000pt}%
\definecolor{currentstroke}{rgb}{0.000000,0.000000,0.000000}%
\pgfsetstrokecolor{currentstroke}%
\pgfsetdash{}{0pt}%
\pgfpathmoveto{\pgfqpoint{5.242035in}{2.868311in}}%
\pgfpathlineto{\pgfqpoint{5.252720in}{2.867543in}}%
\pgfpathlineto{\pgfqpoint{5.260702in}{2.615855in}}%
\pgfpathlineto{\pgfqpoint{5.294814in}{2.644155in}}%
\pgfpathlineto{\pgfqpoint{5.329183in}{2.694793in}}%
\pgfpathlineto{\pgfqpoint{5.319192in}{2.762473in}}%
\pgfpathlineto{\pgfqpoint{5.309458in}{2.852079in}}%
\pgfpathlineto{\pgfqpoint{5.274722in}{2.765744in}}%
\pgfpathlineto{\pgfqpoint{5.242035in}{2.868311in}}%
\pgfpathclose%
\pgfusepath{fill}%
\end{pgfscope}%
\begin{pgfscope}%
\pgfpathrectangle{\pgfqpoint{1.020000in}{0.880000in}}{\pgfqpoint{6.160000in}{6.160000in}}%
\pgfusepath{clip}%
\pgfsetbuttcap%
\pgfsetroundjoin%
\definecolor{currentfill}{rgb}{0.619318,0.744121,0.998931}%
\pgfsetfillcolor{currentfill}%
\pgfsetlinewidth{0.000000pt}%
\definecolor{currentstroke}{rgb}{0.000000,0.000000,0.000000}%
\pgfsetstrokecolor{currentstroke}%
\pgfsetdash{}{0pt}%
\pgfpathmoveto{\pgfqpoint{2.693362in}{3.287786in}}%
\pgfpathlineto{\pgfqpoint{2.698205in}{3.493883in}}%
\pgfpathlineto{\pgfqpoint{2.709896in}{3.239405in}}%
\pgfpathlineto{\pgfqpoint{2.745381in}{3.177847in}}%
\pgfpathlineto{\pgfqpoint{2.777425in}{3.356406in}}%
\pgfpathlineto{\pgfqpoint{2.770547in}{3.277526in}}%
\pgfpathlineto{\pgfqpoint{2.761401in}{3.360191in}}%
\pgfpathlineto{\pgfqpoint{2.728067in}{3.276812in}}%
\pgfpathlineto{\pgfqpoint{2.693362in}{3.287786in}}%
\pgfpathclose%
\pgfusepath{fill}%
\end{pgfscope}%
\begin{pgfscope}%
\pgfpathrectangle{\pgfqpoint{1.020000in}{0.880000in}}{\pgfqpoint{6.160000in}{6.160000in}}%
\pgfusepath{clip}%
\pgfsetbuttcap%
\pgfsetroundjoin%
\definecolor{currentfill}{rgb}{0.782049,0.842864,0.942980}%
\pgfsetfillcolor{currentfill}%
\pgfsetlinewidth{0.000000pt}%
\definecolor{currentstroke}{rgb}{0.000000,0.000000,0.000000}%
\pgfsetstrokecolor{currentstroke}%
\pgfsetdash{}{0pt}%
\pgfpathmoveto{\pgfqpoint{3.841479in}{3.683300in}}%
\pgfpathlineto{\pgfqpoint{3.850574in}{3.702367in}}%
\pgfpathlineto{\pgfqpoint{3.859741in}{3.706788in}}%
\pgfpathlineto{\pgfqpoint{3.893940in}{3.741607in}}%
\pgfpathlineto{\pgfqpoint{3.928922in}{3.469601in}}%
\pgfpathlineto{\pgfqpoint{3.919300in}{3.619341in}}%
\pgfpathlineto{\pgfqpoint{3.910331in}{3.526690in}}%
\pgfpathlineto{\pgfqpoint{3.876069in}{3.563436in}}%
\pgfpathlineto{\pgfqpoint{3.841479in}{3.683300in}}%
\pgfpathclose%
\pgfusepath{fill}%
\end{pgfscope}%
\begin{pgfscope}%
\pgfpathrectangle{\pgfqpoint{1.020000in}{0.880000in}}{\pgfqpoint{6.160000in}{6.160000in}}%
\pgfusepath{clip}%
\pgfsetbuttcap%
\pgfsetroundjoin%
\definecolor{currentfill}{rgb}{0.309060,0.413498,0.850128}%
\pgfsetfillcolor{currentfill}%
\pgfsetlinewidth{0.000000pt}%
\definecolor{currentstroke}{rgb}{0.000000,0.000000,0.000000}%
\pgfsetstrokecolor{currentstroke}%
\pgfsetdash{}{0pt}%
\pgfpathmoveto{\pgfqpoint{5.463573in}{2.647395in}}%
\pgfpathlineto{\pgfqpoint{5.476785in}{2.823711in}}%
\pgfpathlineto{\pgfqpoint{5.485570in}{2.657871in}}%
\pgfpathlineto{\pgfqpoint{5.520143in}{2.718737in}}%
\pgfpathlineto{\pgfqpoint{5.553082in}{2.659747in}}%
\pgfpathlineto{\pgfqpoint{5.544974in}{2.871708in}}%
\pgfpathlineto{\pgfqpoint{5.531134in}{2.663347in}}%
\pgfpathlineto{\pgfqpoint{5.499111in}{2.787543in}}%
\pgfpathlineto{\pgfqpoint{5.463573in}{2.647395in}}%
\pgfpathclose%
\pgfusepath{fill}%
\end{pgfscope}%
\begin{pgfscope}%
\pgfpathrectangle{\pgfqpoint{1.020000in}{0.880000in}}{\pgfqpoint{6.160000in}{6.160000in}}%
\pgfusepath{clip}%
\pgfsetbuttcap%
\pgfsetroundjoin%
\definecolor{currentfill}{rgb}{0.478462,0.616564,0.972721}%
\pgfsetfillcolor{currentfill}%
\pgfsetlinewidth{0.000000pt}%
\definecolor{currentstroke}{rgb}{0.000000,0.000000,0.000000}%
\pgfsetstrokecolor{currentstroke}%
\pgfsetdash{}{0pt}%
\pgfpathmoveto{\pgfqpoint{4.775391in}{3.201272in}}%
\pgfpathlineto{\pgfqpoint{4.784607in}{3.040882in}}%
\pgfpathlineto{\pgfqpoint{4.794624in}{3.012638in}}%
\pgfpathlineto{\pgfqpoint{4.829224in}{3.108858in}}%
\pgfpathlineto{\pgfqpoint{4.863098in}{3.092214in}}%
\pgfpathlineto{\pgfqpoint{4.851581in}{2.905663in}}%
\pgfpathlineto{\pgfqpoint{4.842081in}{3.014899in}}%
\pgfpathlineto{\pgfqpoint{4.808380in}{3.044594in}}%
\pgfpathlineto{\pgfqpoint{4.775391in}{3.201272in}}%
\pgfpathclose%
\pgfusepath{fill}%
\end{pgfscope}%
\begin{pgfscope}%
\pgfpathrectangle{\pgfqpoint{1.020000in}{0.880000in}}{\pgfqpoint{6.160000in}{6.160000in}}%
\pgfusepath{clip}%
\pgfsetbuttcap%
\pgfsetroundjoin%
\definecolor{currentfill}{rgb}{0.275827,0.366717,0.812553}%
\pgfsetfillcolor{currentfill}%
\pgfsetlinewidth{0.000000pt}%
\definecolor{currentstroke}{rgb}{0.000000,0.000000,0.000000}%
\pgfsetstrokecolor{currentstroke}%
\pgfsetdash{}{0pt}%
\pgfpathmoveto{\pgfqpoint{5.917541in}{2.975054in}}%
\pgfpathlineto{\pgfqpoint{5.920847in}{2.510432in}}%
\pgfpathlineto{\pgfqpoint{5.937521in}{2.807703in}}%
\pgfpathlineto{\pgfqpoint{5.967323in}{2.591547in}}%
\pgfpathlineto{\pgfqpoint{6.001215in}{2.607619in}}%
\pgfpathlineto{\pgfqpoint{5.989758in}{2.610216in}}%
\pgfpathlineto{\pgfqpoint{5.977725in}{2.579609in}}%
\pgfpathlineto{\pgfqpoint{5.944978in}{2.623502in}}%
\pgfpathlineto{\pgfqpoint{5.917541in}{2.975054in}}%
\pgfpathclose%
\pgfusepath{fill}%
\end{pgfscope}%
\begin{pgfscope}%
\pgfpathrectangle{\pgfqpoint{1.020000in}{0.880000in}}{\pgfqpoint{6.160000in}{6.160000in}}%
\pgfusepath{clip}%
\pgfsetbuttcap%
\pgfsetroundjoin%
\definecolor{currentfill}{rgb}{0.738826,0.822572,0.968261}%
\pgfsetfillcolor{currentfill}%
\pgfsetlinewidth{0.000000pt}%
\definecolor{currentstroke}{rgb}{0.000000,0.000000,0.000000}%
\pgfsetstrokecolor{currentstroke}%
\pgfsetdash{}{0pt}%
\pgfpathmoveto{\pgfqpoint{3.997009in}{3.602737in}}%
\pgfpathlineto{\pgfqpoint{4.006357in}{3.591621in}}%
\pgfpathlineto{\pgfqpoint{4.015870in}{3.493338in}}%
\pgfpathlineto{\pgfqpoint{4.049944in}{3.596791in}}%
\pgfpathlineto{\pgfqpoint{4.084325in}{3.441710in}}%
\pgfpathlineto{\pgfqpoint{4.074852in}{3.506586in}}%
\pgfpathlineto{\pgfqpoint{4.065373in}{3.583254in}}%
\pgfpathlineto{\pgfqpoint{4.031329in}{3.511339in}}%
\pgfpathlineto{\pgfqpoint{3.997009in}{3.602737in}}%
\pgfpathclose%
\pgfusepath{fill}%
\end{pgfscope}%
\begin{pgfscope}%
\pgfpathrectangle{\pgfqpoint{1.020000in}{0.880000in}}{\pgfqpoint{6.160000in}{6.160000in}}%
\pgfusepath{clip}%
\pgfsetbuttcap%
\pgfsetroundjoin%
\definecolor{currentfill}{rgb}{0.733898,0.820018,0.970724}%
\pgfsetfillcolor{currentfill}%
\pgfsetlinewidth{0.000000pt}%
\definecolor{currentstroke}{rgb}{0.000000,0.000000,0.000000}%
\pgfsetstrokecolor{currentstroke}%
\pgfsetdash{}{0pt}%
\pgfpathmoveto{\pgfqpoint{3.189888in}{3.489183in}}%
\pgfpathlineto{\pgfqpoint{3.197287in}{3.595266in}}%
\pgfpathlineto{\pgfqpoint{3.206204in}{3.549402in}}%
\pgfpathlineto{\pgfqpoint{3.240703in}{3.547642in}}%
\pgfpathlineto{\pgfqpoint{3.276475in}{3.403092in}}%
\pgfpathlineto{\pgfqpoint{3.266488in}{3.562918in}}%
\pgfpathlineto{\pgfqpoint{3.258151in}{3.543253in}}%
\pgfpathlineto{\pgfqpoint{3.223940in}{3.524601in}}%
\pgfpathlineto{\pgfqpoint{3.189888in}{3.489183in}}%
\pgfpathclose%
\pgfusepath{fill}%
\end{pgfscope}%
\begin{pgfscope}%
\pgfpathrectangle{\pgfqpoint{1.020000in}{0.880000in}}{\pgfqpoint{6.160000in}{6.160000in}}%
\pgfusepath{clip}%
\pgfsetbuttcap%
\pgfsetroundjoin%
\definecolor{currentfill}{rgb}{0.592356,0.722792,0.999434}%
\pgfsetfillcolor{currentfill}%
\pgfsetlinewidth{0.000000pt}%
\definecolor{currentstroke}{rgb}{0.000000,0.000000,0.000000}%
\pgfsetstrokecolor{currentstroke}%
\pgfsetdash{}{0pt}%
\pgfpathmoveto{\pgfqpoint{4.463734in}{3.306501in}}%
\pgfpathlineto{\pgfqpoint{4.473467in}{3.271018in}}%
\pgfpathlineto{\pgfqpoint{4.483656in}{3.382570in}}%
\pgfpathlineto{\pgfqpoint{4.517272in}{3.228539in}}%
\pgfpathlineto{\pgfqpoint{4.551409in}{3.250042in}}%
\pgfpathlineto{\pgfqpoint{4.540904in}{3.087046in}}%
\pgfpathlineto{\pgfqpoint{4.531441in}{3.207001in}}%
\pgfpathlineto{\pgfqpoint{4.497845in}{3.329744in}}%
\pgfpathlineto{\pgfqpoint{4.463734in}{3.306501in}}%
\pgfpathclose%
\pgfusepath{fill}%
\end{pgfscope}%
\begin{pgfscope}%
\pgfpathrectangle{\pgfqpoint{1.020000in}{0.880000in}}{\pgfqpoint{6.160000in}{6.160000in}}%
\pgfusepath{clip}%
\pgfsetbuttcap%
\pgfsetroundjoin%
\definecolor{currentfill}{rgb}{0.289996,0.386836,0.828926}%
\pgfsetfillcolor{currentfill}%
\pgfsetlinewidth{0.000000pt}%
\definecolor{currentstroke}{rgb}{0.000000,0.000000,0.000000}%
\pgfsetstrokecolor{currentstroke}%
\pgfsetdash{}{0pt}%
\pgfpathmoveto{\pgfqpoint{6.203112in}{2.642651in}}%
\pgfpathlineto{\pgfqpoint{6.217358in}{2.765397in}}%
\pgfpathlineto{\pgfqpoint{6.226370in}{2.631355in}}%
\pgfpathlineto{\pgfqpoint{6.262270in}{2.744392in}}%
\pgfpathlineto{\pgfqpoint{6.250086in}{2.726126in}}%
\pgfpathlineto{\pgfqpoint{6.235737in}{2.601744in}}%
\pgfpathlineto{\pgfqpoint{6.203112in}{2.642651in}}%
\pgfpathclose%
\pgfusepath{fill}%
\end{pgfscope}%
\begin{pgfscope}%
\pgfpathrectangle{\pgfqpoint{1.020000in}{0.880000in}}{\pgfqpoint{6.160000in}{6.160000in}}%
\pgfusepath{clip}%
\pgfsetbuttcap%
\pgfsetroundjoin%
\definecolor{currentfill}{rgb}{0.309060,0.413498,0.850128}%
\pgfsetfillcolor{currentfill}%
\pgfsetlinewidth{0.000000pt}%
\definecolor{currentstroke}{rgb}{0.000000,0.000000,0.000000}%
\pgfsetstrokecolor{currentstroke}%
\pgfsetdash{}{0pt}%
\pgfpathmoveto{\pgfqpoint{5.689287in}{2.757162in}}%
\pgfpathlineto{\pgfqpoint{5.699205in}{2.674570in}}%
\pgfpathlineto{\pgfqpoint{5.711082in}{2.718939in}}%
\pgfpathlineto{\pgfqpoint{5.744800in}{2.718979in}}%
\pgfpathlineto{\pgfqpoint{5.780332in}{2.833142in}}%
\pgfpathlineto{\pgfqpoint{5.767019in}{2.705702in}}%
\pgfpathlineto{\pgfqpoint{5.756765in}{2.768021in}}%
\pgfpathlineto{\pgfqpoint{5.720952in}{2.628225in}}%
\pgfpathlineto{\pgfqpoint{5.689287in}{2.757162in}}%
\pgfpathclose%
\pgfusepath{fill}%
\end{pgfscope}%
\begin{pgfscope}%
\pgfpathrectangle{\pgfqpoint{1.020000in}{0.880000in}}{\pgfqpoint{6.160000in}{6.160000in}}%
\pgfusepath{clip}%
\pgfsetbuttcap%
\pgfsetroundjoin%
\definecolor{currentfill}{rgb}{0.275827,0.366717,0.812553}%
\pgfsetfillcolor{currentfill}%
\pgfsetlinewidth{0.000000pt}%
\definecolor{currentstroke}{rgb}{0.000000,0.000000,0.000000}%
\pgfsetstrokecolor{currentstroke}%
\pgfsetdash{}{0pt}%
\pgfpathmoveto{\pgfqpoint{6.135184in}{2.595675in}}%
\pgfpathlineto{\pgfqpoint{6.151748in}{2.843282in}}%
\pgfpathlineto{\pgfqpoint{6.158078in}{2.572921in}}%
\pgfpathlineto{\pgfqpoint{6.193343in}{2.657352in}}%
\pgfpathlineto{\pgfqpoint{6.226370in}{2.631355in}}%
\pgfpathlineto{\pgfqpoint{6.217358in}{2.765397in}}%
\pgfpathlineto{\pgfqpoint{6.203112in}{2.642651in}}%
\pgfpathlineto{\pgfqpoint{6.167335in}{2.528467in}}%
\pgfpathlineto{\pgfqpoint{6.135184in}{2.595675in}}%
\pgfpathclose%
\pgfusepath{fill}%
\end{pgfscope}%
\begin{pgfscope}%
\pgfpathrectangle{\pgfqpoint{1.020000in}{0.880000in}}{\pgfqpoint{6.160000in}{6.160000in}}%
\pgfusepath{clip}%
\pgfsetbuttcap%
\pgfsetroundjoin%
\definecolor{currentfill}{rgb}{0.419991,0.552989,0.942630}%
\pgfsetfillcolor{currentfill}%
\pgfsetlinewidth{0.000000pt}%
\definecolor{currentstroke}{rgb}{0.000000,0.000000,0.000000}%
\pgfsetstrokecolor{currentstroke}%
\pgfsetdash{}{0pt}%
\pgfpathmoveto{\pgfqpoint{4.863098in}{3.092214in}}%
\pgfpathlineto{\pgfqpoint{4.872966in}{3.031883in}}%
\pgfpathlineto{\pgfqpoint{4.882692in}{2.949890in}}%
\pgfpathlineto{\pgfqpoint{4.916430in}{2.916908in}}%
\pgfpathlineto{\pgfqpoint{4.948784in}{2.710017in}}%
\pgfpathlineto{\pgfqpoint{4.939715in}{2.876740in}}%
\pgfpathlineto{\pgfqpoint{4.929401in}{2.879071in}}%
\pgfpathlineto{\pgfqpoint{4.896791in}{3.054804in}}%
\pgfpathlineto{\pgfqpoint{4.863098in}{3.092214in}}%
\pgfpathclose%
\pgfusepath{fill}%
\end{pgfscope}%
\begin{pgfscope}%
\pgfpathrectangle{\pgfqpoint{1.020000in}{0.880000in}}{\pgfqpoint{6.160000in}{6.160000in}}%
\pgfusepath{clip}%
\pgfsetbuttcap%
\pgfsetroundjoin%
\definecolor{currentfill}{rgb}{0.661968,0.775491,0.993937}%
\pgfsetfillcolor{currentfill}%
\pgfsetlinewidth{0.000000pt}%
\definecolor{currentstroke}{rgb}{0.000000,0.000000,0.000000}%
\pgfsetstrokecolor{currentstroke}%
\pgfsetdash{}{0pt}%
\pgfpathmoveto{\pgfqpoint{4.308209in}{3.438795in}}%
\pgfpathlineto{\pgfqpoint{4.317810in}{3.387176in}}%
\pgfpathlineto{\pgfqpoint{4.327567in}{3.437178in}}%
\pgfpathlineto{\pgfqpoint{4.361639in}{3.385042in}}%
\pgfpathlineto{\pgfqpoint{4.395823in}{3.412385in}}%
\pgfpathlineto{\pgfqpoint{4.386040in}{3.399198in}}%
\pgfpathlineto{\pgfqpoint{4.376171in}{3.329601in}}%
\pgfpathlineto{\pgfqpoint{4.342137in}{3.332710in}}%
\pgfpathlineto{\pgfqpoint{4.308209in}{3.438795in}}%
\pgfpathclose%
\pgfusepath{fill}%
\end{pgfscope}%
\begin{pgfscope}%
\pgfpathrectangle{\pgfqpoint{1.020000in}{0.880000in}}{\pgfqpoint{6.160000in}{6.160000in}}%
\pgfusepath{clip}%
\pgfsetbuttcap%
\pgfsetroundjoin%
\definecolor{currentfill}{rgb}{0.661968,0.775491,0.993937}%
\pgfsetfillcolor{currentfill}%
\pgfsetlinewidth{0.000000pt}%
\definecolor{currentstroke}{rgb}{0.000000,0.000000,0.000000}%
\pgfsetstrokecolor{currentstroke}%
\pgfsetdash{}{0pt}%
\pgfpathmoveto{\pgfqpoint{2.984041in}{3.410677in}}%
\pgfpathlineto{\pgfqpoint{2.991263in}{3.495004in}}%
\pgfpathlineto{\pgfqpoint{3.002368in}{3.253861in}}%
\pgfpathlineto{\pgfqpoint{3.036181in}{3.314156in}}%
\pgfpathlineto{\pgfqpoint{3.068687in}{3.493721in}}%
\pgfpathlineto{\pgfqpoint{3.063335in}{3.229078in}}%
\pgfpathlineto{\pgfqpoint{3.052799in}{3.428429in}}%
\pgfpathlineto{\pgfqpoint{3.017917in}{3.463363in}}%
\pgfpathlineto{\pgfqpoint{2.984041in}{3.410677in}}%
\pgfpathclose%
\pgfusepath{fill}%
\end{pgfscope}%
\begin{pgfscope}%
\pgfpathrectangle{\pgfqpoint{1.020000in}{0.880000in}}{\pgfqpoint{6.160000in}{6.160000in}}%
\pgfusepath{clip}%
\pgfsetbuttcap%
\pgfsetroundjoin%
\definecolor{currentfill}{rgb}{0.619318,0.744121,0.998931}%
\pgfsetfillcolor{currentfill}%
\pgfsetlinewidth{0.000000pt}%
\definecolor{currentstroke}{rgb}{0.000000,0.000000,0.000000}%
\pgfsetstrokecolor{currentstroke}%
\pgfsetdash{}{0pt}%
\pgfpathmoveto{\pgfqpoint{2.621376in}{3.469306in}}%
\pgfpathlineto{\pgfqpoint{2.632990in}{3.226663in}}%
\pgfpathlineto{\pgfqpoint{2.643525in}{3.052434in}}%
\pgfpathlineto{\pgfqpoint{2.672348in}{3.433236in}}%
\pgfpathlineto{\pgfqpoint{2.709896in}{3.239405in}}%
\pgfpathlineto{\pgfqpoint{2.698205in}{3.493883in}}%
\pgfpathlineto{\pgfqpoint{2.693362in}{3.287786in}}%
\pgfpathlineto{\pgfqpoint{2.659277in}{3.256302in}}%
\pgfpathlineto{\pgfqpoint{2.621376in}{3.469306in}}%
\pgfpathclose%
\pgfusepath{fill}%
\end{pgfscope}%
\begin{pgfscope}%
\pgfpathrectangle{\pgfqpoint{1.020000in}{0.880000in}}{\pgfqpoint{6.160000in}{6.160000in}}%
\pgfusepath{clip}%
\pgfsetbuttcap%
\pgfsetroundjoin%
\definecolor{currentfill}{rgb}{0.586921,0.718121,0.998874}%
\pgfsetfillcolor{currentfill}%
\pgfsetlinewidth{0.000000pt}%
\definecolor{currentstroke}{rgb}{0.000000,0.000000,0.000000}%
\pgfsetstrokecolor{currentstroke}%
\pgfsetdash{}{0pt}%
\pgfpathmoveto{\pgfqpoint{2.555520in}{3.254287in}}%
\pgfpathlineto{\pgfqpoint{2.565292in}{3.130132in}}%
\pgfpathlineto{\pgfqpoint{2.570259in}{3.303360in}}%
\pgfpathlineto{\pgfqpoint{2.606860in}{3.183358in}}%
\pgfpathlineto{\pgfqpoint{2.643525in}{3.052434in}}%
\pgfpathlineto{\pgfqpoint{2.632990in}{3.226663in}}%
\pgfpathlineto{\pgfqpoint{2.621376in}{3.469306in}}%
\pgfpathlineto{\pgfqpoint{2.588129in}{3.380130in}}%
\pgfpathlineto{\pgfqpoint{2.555520in}{3.254287in}}%
\pgfpathclose%
\pgfusepath{fill}%
\end{pgfscope}%
\begin{pgfscope}%
\pgfpathrectangle{\pgfqpoint{1.020000in}{0.880000in}}{\pgfqpoint{6.160000in}{6.160000in}}%
\pgfusepath{clip}%
\pgfsetbuttcap%
\pgfsetroundjoin%
\definecolor{currentfill}{rgb}{0.728970,0.817464,0.973188}%
\pgfsetfillcolor{currentfill}%
\pgfsetlinewidth{0.000000pt}%
\definecolor{currentstroke}{rgb}{0.000000,0.000000,0.000000}%
\pgfsetstrokecolor{currentstroke}%
\pgfsetdash{}{0pt}%
\pgfpathmoveto{\pgfqpoint{3.121207in}{3.473588in}}%
\pgfpathlineto{\pgfqpoint{3.128946in}{3.533699in}}%
\pgfpathlineto{\pgfqpoint{3.138239in}{3.447742in}}%
\pgfpathlineto{\pgfqpoint{3.173117in}{3.409254in}}%
\pgfpathlineto{\pgfqpoint{3.206204in}{3.549402in}}%
\pgfpathlineto{\pgfqpoint{3.197287in}{3.595266in}}%
\pgfpathlineto{\pgfqpoint{3.189888in}{3.489183in}}%
\pgfpathlineto{\pgfqpoint{3.153881in}{3.644399in}}%
\pgfpathlineto{\pgfqpoint{3.121207in}{3.473588in}}%
\pgfpathclose%
\pgfusepath{fill}%
\end{pgfscope}%
\begin{pgfscope}%
\pgfpathrectangle{\pgfqpoint{1.020000in}{0.880000in}}{\pgfqpoint{6.160000in}{6.160000in}}%
\pgfusepath{clip}%
\pgfsetbuttcap%
\pgfsetroundjoin%
\definecolor{currentfill}{rgb}{0.483854,0.622050,0.974808}%
\pgfsetfillcolor{currentfill}%
\pgfsetlinewidth{0.000000pt}%
\definecolor{currentstroke}{rgb}{0.000000,0.000000,0.000000}%
\pgfsetstrokecolor{currentstroke}%
\pgfsetdash{}{0pt}%
\pgfpathmoveto{\pgfqpoint{4.707536in}{3.230116in}}%
\pgfpathlineto{\pgfqpoint{4.716336in}{2.983632in}}%
\pgfpathlineto{\pgfqpoint{4.726082in}{2.915714in}}%
\pgfpathlineto{\pgfqpoint{4.760437in}{2.979671in}}%
\pgfpathlineto{\pgfqpoint{4.794624in}{3.012638in}}%
\pgfpathlineto{\pgfqpoint{4.784607in}{3.040882in}}%
\pgfpathlineto{\pgfqpoint{4.775391in}{3.201272in}}%
\pgfpathlineto{\pgfqpoint{4.740732in}{3.082211in}}%
\pgfpathlineto{\pgfqpoint{4.707536in}{3.230116in}}%
\pgfpathclose%
\pgfusepath{fill}%
\end{pgfscope}%
\begin{pgfscope}%
\pgfpathrectangle{\pgfqpoint{1.020000in}{0.880000in}}{\pgfqpoint{6.160000in}{6.160000in}}%
\pgfusepath{clip}%
\pgfsetbuttcap%
\pgfsetroundjoin%
\definecolor{currentfill}{rgb}{0.640828,0.760752,0.997846}%
\pgfsetfillcolor{currentfill}%
\pgfsetlinewidth{0.000000pt}%
\definecolor{currentstroke}{rgb}{0.000000,0.000000,0.000000}%
\pgfsetstrokecolor{currentstroke}%
\pgfsetdash{}{0pt}%
\pgfpathmoveto{\pgfqpoint{2.915812in}{3.346568in}}%
\pgfpathlineto{\pgfqpoint{2.925299in}{3.240587in}}%
\pgfpathlineto{\pgfqpoint{2.933192in}{3.261934in}}%
\pgfpathlineto{\pgfqpoint{2.967856in}{3.252660in}}%
\pgfpathlineto{\pgfqpoint{3.002368in}{3.253861in}}%
\pgfpathlineto{\pgfqpoint{2.991263in}{3.495004in}}%
\pgfpathlineto{\pgfqpoint{2.984041in}{3.410677in}}%
\pgfpathlineto{\pgfqpoint{2.948841in}{3.466529in}}%
\pgfpathlineto{\pgfqpoint{2.915812in}{3.346568in}}%
\pgfpathclose%
\pgfusepath{fill}%
\end{pgfscope}%
\begin{pgfscope}%
\pgfpathrectangle{\pgfqpoint{1.020000in}{0.880000in}}{\pgfqpoint{6.160000in}{6.160000in}}%
\pgfusepath{clip}%
\pgfsetbuttcap%
\pgfsetroundjoin%
\definecolor{currentfill}{rgb}{0.809329,0.852974,0.922323}%
\pgfsetfillcolor{currentfill}%
\pgfsetlinewidth{0.000000pt}%
\definecolor{currentstroke}{rgb}{0.000000,0.000000,0.000000}%
\pgfsetstrokecolor{currentstroke}%
\pgfsetdash{}{0pt}%
\pgfpathmoveto{\pgfqpoint{3.548732in}{3.763855in}}%
\pgfpathlineto{\pgfqpoint{3.558136in}{3.675201in}}%
\pgfpathlineto{\pgfqpoint{3.566583in}{3.742788in}}%
\pgfpathlineto{\pgfqpoint{3.600853in}{3.766767in}}%
\pgfpathlineto{\pgfqpoint{3.635860in}{3.655060in}}%
\pgfpathlineto{\pgfqpoint{3.626724in}{3.692208in}}%
\pgfpathlineto{\pgfqpoint{3.618886in}{3.501405in}}%
\pgfpathlineto{\pgfqpoint{3.583751in}{3.651440in}}%
\pgfpathlineto{\pgfqpoint{3.548732in}{3.763855in}}%
\pgfpathclose%
\pgfusepath{fill}%
\end{pgfscope}%
\begin{pgfscope}%
\pgfpathrectangle{\pgfqpoint{1.020000in}{0.880000in}}{\pgfqpoint{6.160000in}{6.160000in}}%
\pgfusepath{clip}%
\pgfsetbuttcap%
\pgfsetroundjoin%
\definecolor{currentfill}{rgb}{0.613933,0.739923,0.999142}%
\pgfsetfillcolor{currentfill}%
\pgfsetlinewidth{0.000000pt}%
\definecolor{currentstroke}{rgb}{0.000000,0.000000,0.000000}%
\pgfsetstrokecolor{currentstroke}%
\pgfsetdash{}{0pt}%
\pgfpathmoveto{\pgfqpoint{2.847528in}{3.286866in}}%
\pgfpathlineto{\pgfqpoint{2.854032in}{3.405171in}}%
\pgfpathlineto{\pgfqpoint{2.865911in}{3.119422in}}%
\pgfpathlineto{\pgfqpoint{2.896541in}{3.423542in}}%
\pgfpathlineto{\pgfqpoint{2.933192in}{3.261934in}}%
\pgfpathlineto{\pgfqpoint{2.925299in}{3.240587in}}%
\pgfpathlineto{\pgfqpoint{2.915812in}{3.346568in}}%
\pgfpathlineto{\pgfqpoint{2.882132in}{3.281251in}}%
\pgfpathlineto{\pgfqpoint{2.847528in}{3.286866in}}%
\pgfpathclose%
\pgfusepath{fill}%
\end{pgfscope}%
\begin{pgfscope}%
\pgfpathrectangle{\pgfqpoint{1.020000in}{0.880000in}}{\pgfqpoint{6.160000in}{6.160000in}}%
\pgfusepath{clip}%
\pgfsetbuttcap%
\pgfsetroundjoin%
\definecolor{currentfill}{rgb}{0.388852,0.516298,0.921373}%
\pgfsetfillcolor{currentfill}%
\pgfsetlinewidth{0.000000pt}%
\definecolor{currentstroke}{rgb}{0.000000,0.000000,0.000000}%
\pgfsetstrokecolor{currentstroke}%
\pgfsetdash{}{0pt}%
\pgfpathmoveto{\pgfqpoint{5.019289in}{3.044938in}}%
\pgfpathlineto{\pgfqpoint{5.028364in}{2.881648in}}%
\pgfpathlineto{\pgfqpoint{5.038399in}{2.831348in}}%
\pgfpathlineto{\pgfqpoint{5.072078in}{2.803825in}}%
\pgfpathlineto{\pgfqpoint{5.105811in}{2.786347in}}%
\pgfpathlineto{\pgfqpoint{5.096884in}{2.960466in}}%
\pgfpathlineto{\pgfqpoint{5.085310in}{2.844436in}}%
\pgfpathlineto{\pgfqpoint{5.051956in}{2.900332in}}%
\pgfpathlineto{\pgfqpoint{5.019289in}{3.044938in}}%
\pgfpathclose%
\pgfusepath{fill}%
\end{pgfscope}%
\begin{pgfscope}%
\pgfpathrectangle{\pgfqpoint{1.020000in}{0.880000in}}{\pgfqpoint{6.160000in}{6.160000in}}%
\pgfusepath{clip}%
\pgfsetbuttcap%
\pgfsetroundjoin%
\definecolor{currentfill}{rgb}{0.323718,0.433158,0.864722}%
\pgfsetfillcolor{currentfill}%
\pgfsetlinewidth{0.000000pt}%
\definecolor{currentstroke}{rgb}{0.000000,0.000000,0.000000}%
\pgfsetstrokecolor{currentstroke}%
\pgfsetdash{}{0pt}%
\pgfpathmoveto{\pgfqpoint{5.396898in}{2.708958in}}%
\pgfpathlineto{\pgfqpoint{5.407804in}{2.712895in}}%
\pgfpathlineto{\pgfqpoint{5.419301in}{2.762330in}}%
\pgfpathlineto{\pgfqpoint{5.452883in}{2.742904in}}%
\pgfpathlineto{\pgfqpoint{5.485570in}{2.657871in}}%
\pgfpathlineto{\pgfqpoint{5.476785in}{2.823711in}}%
\pgfpathlineto{\pgfqpoint{5.463573in}{2.647395in}}%
\pgfpathlineto{\pgfqpoint{5.433719in}{2.953801in}}%
\pgfpathlineto{\pgfqpoint{5.396898in}{2.708958in}}%
\pgfpathclose%
\pgfusepath{fill}%
\end{pgfscope}%
\begin{pgfscope}%
\pgfpathrectangle{\pgfqpoint{1.020000in}{0.880000in}}{\pgfqpoint{6.160000in}{6.160000in}}%
\pgfusepath{clip}%
\pgfsetbuttcap%
\pgfsetroundjoin%
\definecolor{currentfill}{rgb}{0.738826,0.822572,0.968261}%
\pgfsetfillcolor{currentfill}%
\pgfsetlinewidth{0.000000pt}%
\definecolor{currentstroke}{rgb}{0.000000,0.000000,0.000000}%
\pgfsetstrokecolor{currentstroke}%
\pgfsetdash{}{0pt}%
\pgfpathmoveto{\pgfqpoint{4.152536in}{3.685756in}}%
\pgfpathlineto{\pgfqpoint{4.162071in}{3.590532in}}%
\pgfpathlineto{\pgfqpoint{4.171600in}{3.572925in}}%
\pgfpathlineto{\pgfqpoint{4.205797in}{3.448634in}}%
\pgfpathlineto{\pgfqpoint{4.239965in}{3.458446in}}%
\pgfpathlineto{\pgfqpoint{4.230389in}{3.471576in}}%
\pgfpathlineto{\pgfqpoint{4.220773in}{3.329878in}}%
\pgfpathlineto{\pgfqpoint{4.186731in}{3.675727in}}%
\pgfpathlineto{\pgfqpoint{4.152536in}{3.685756in}}%
\pgfpathclose%
\pgfusepath{fill}%
\end{pgfscope}%
\begin{pgfscope}%
\pgfpathrectangle{\pgfqpoint{1.020000in}{0.880000in}}{\pgfqpoint{6.160000in}{6.160000in}}%
\pgfusepath{clip}%
\pgfsetbuttcap%
\pgfsetroundjoin%
\definecolor{currentfill}{rgb}{0.252663,0.332837,0.783665}%
\pgfsetfillcolor{currentfill}%
\pgfsetlinewidth{0.000000pt}%
\definecolor{currentstroke}{rgb}{0.000000,0.000000,0.000000}%
\pgfsetstrokecolor{currentstroke}%
\pgfsetdash{}{0pt}%
\pgfpathmoveto{\pgfqpoint{6.001215in}{2.607619in}}%
\pgfpathlineto{\pgfqpoint{6.010603in}{2.491584in}}%
\pgfpathlineto{\pgfqpoint{6.023814in}{2.581536in}}%
\pgfpathlineto{\pgfqpoint{6.057159in}{2.565956in}}%
\pgfpathlineto{\pgfqpoint{6.092802in}{2.671376in}}%
\pgfpathlineto{\pgfqpoint{6.082218in}{2.726008in}}%
\pgfpathlineto{\pgfqpoint{6.066690in}{2.519889in}}%
\pgfpathlineto{\pgfqpoint{6.034608in}{2.597512in}}%
\pgfpathlineto{\pgfqpoint{6.001215in}{2.607619in}}%
\pgfpathclose%
\pgfusepath{fill}%
\end{pgfscope}%
\begin{pgfscope}%
\pgfpathrectangle{\pgfqpoint{1.020000in}{0.880000in}}{\pgfqpoint{6.160000in}{6.160000in}}%
\pgfusepath{clip}%
\pgfsetbuttcap%
\pgfsetroundjoin%
\definecolor{currentfill}{rgb}{0.280550,0.373423,0.818011}%
\pgfsetfillcolor{currentfill}%
\pgfsetlinewidth{0.000000pt}%
\definecolor{currentstroke}{rgb}{0.000000,0.000000,0.000000}%
\pgfsetstrokecolor{currentstroke}%
\pgfsetdash{}{0pt}%
\pgfpathmoveto{\pgfqpoint{6.066690in}{2.519889in}}%
\pgfpathlineto{\pgfqpoint{6.082218in}{2.726008in}}%
\pgfpathlineto{\pgfqpoint{6.092802in}{2.671376in}}%
\pgfpathlineto{\pgfqpoint{6.125116in}{2.604227in}}%
\pgfpathlineto{\pgfqpoint{6.158078in}{2.572921in}}%
\pgfpathlineto{\pgfqpoint{6.151748in}{2.843282in}}%
\pgfpathlineto{\pgfqpoint{6.135184in}{2.595675in}}%
\pgfpathlineto{\pgfqpoint{6.104221in}{2.727639in}}%
\pgfpathlineto{\pgfqpoint{6.066690in}{2.519889in}}%
\pgfpathclose%
\pgfusepath{fill}%
\end{pgfscope}%
\begin{pgfscope}%
\pgfpathrectangle{\pgfqpoint{1.020000in}{0.880000in}}{\pgfqpoint{6.160000in}{6.160000in}}%
\pgfusepath{clip}%
\pgfsetbuttcap%
\pgfsetroundjoin%
\definecolor{currentfill}{rgb}{0.804965,0.851666,0.926165}%
\pgfsetfillcolor{currentfill}%
\pgfsetlinewidth{0.000000pt}%
\definecolor{currentstroke}{rgb}{0.000000,0.000000,0.000000}%
\pgfsetstrokecolor{currentstroke}%
\pgfsetdash{}{0pt}%
\pgfpathmoveto{\pgfqpoint{3.773850in}{3.468521in}}%
\pgfpathlineto{\pgfqpoint{3.781785in}{3.752246in}}%
\pgfpathlineto{\pgfqpoint{3.791111in}{3.698644in}}%
\pgfpathlineto{\pgfqpoint{3.825545in}{3.671891in}}%
\pgfpathlineto{\pgfqpoint{3.859741in}{3.706788in}}%
\pgfpathlineto{\pgfqpoint{3.850574in}{3.702367in}}%
\pgfpathlineto{\pgfqpoint{3.841479in}{3.683300in}}%
\pgfpathlineto{\pgfqpoint{3.807336in}{3.656381in}}%
\pgfpathlineto{\pgfqpoint{3.773850in}{3.468521in}}%
\pgfpathclose%
\pgfusepath{fill}%
\end{pgfscope}%
\begin{pgfscope}%
\pgfpathrectangle{\pgfqpoint{1.020000in}{0.880000in}}{\pgfqpoint{6.160000in}{6.160000in}}%
\pgfusepath{clip}%
\pgfsetbuttcap%
\pgfsetroundjoin%
\definecolor{currentfill}{rgb}{0.353369,0.472069,0.892570}%
\pgfsetfillcolor{currentfill}%
\pgfsetlinewidth{0.000000pt}%
\definecolor{currentstroke}{rgb}{0.000000,0.000000,0.000000}%
\pgfsetstrokecolor{currentstroke}%
\pgfsetdash{}{0pt}%
\pgfpathmoveto{\pgfqpoint{5.172549in}{2.688119in}}%
\pgfpathlineto{\pgfqpoint{5.184700in}{2.841435in}}%
\pgfpathlineto{\pgfqpoint{5.194787in}{2.786620in}}%
\pgfpathlineto{\pgfqpoint{5.227891in}{2.711419in}}%
\pgfpathlineto{\pgfqpoint{5.260702in}{2.615855in}}%
\pgfpathlineto{\pgfqpoint{5.252720in}{2.867543in}}%
\pgfpathlineto{\pgfqpoint{5.242035in}{2.868311in}}%
\pgfpathlineto{\pgfqpoint{5.210076in}{3.050307in}}%
\pgfpathlineto{\pgfqpoint{5.172549in}{2.688119in}}%
\pgfpathclose%
\pgfusepath{fill}%
\end{pgfscope}%
\begin{pgfscope}%
\pgfpathrectangle{\pgfqpoint{1.020000in}{0.880000in}}{\pgfqpoint{6.160000in}{6.160000in}}%
\pgfusepath{clip}%
\pgfsetbuttcap%
\pgfsetroundjoin%
\definecolor{currentfill}{rgb}{0.473070,0.611077,0.970634}%
\pgfsetfillcolor{currentfill}%
\pgfsetlinewidth{0.000000pt}%
\definecolor{currentstroke}{rgb}{0.000000,0.000000,0.000000}%
\pgfsetstrokecolor{currentstroke}%
\pgfsetdash{}{0pt}%
\pgfpathmoveto{\pgfqpoint{4.638965in}{3.128792in}}%
\pgfpathlineto{\pgfqpoint{4.648591in}{3.039985in}}%
\pgfpathlineto{\pgfqpoint{4.659119in}{3.139382in}}%
\pgfpathlineto{\pgfqpoint{4.692194in}{2.938559in}}%
\pgfpathlineto{\pgfqpoint{4.726082in}{2.915714in}}%
\pgfpathlineto{\pgfqpoint{4.716336in}{2.983632in}}%
\pgfpathlineto{\pgfqpoint{4.707536in}{3.230116in}}%
\pgfpathlineto{\pgfqpoint{4.671833in}{2.892354in}}%
\pgfpathlineto{\pgfqpoint{4.638965in}{3.128792in}}%
\pgfpathclose%
\pgfusepath{fill}%
\end{pgfscope}%
\begin{pgfscope}%
\pgfpathrectangle{\pgfqpoint{1.020000in}{0.880000in}}{\pgfqpoint{6.160000in}{6.160000in}}%
\pgfusepath{clip}%
\pgfsetbuttcap%
\pgfsetroundjoin%
\definecolor{currentfill}{rgb}{0.597777,0.727330,0.999777}%
\pgfsetfillcolor{currentfill}%
\pgfsetlinewidth{0.000000pt}%
\definecolor{currentstroke}{rgb}{0.000000,0.000000,0.000000}%
\pgfsetstrokecolor{currentstroke}%
\pgfsetdash{}{0pt}%
\pgfpathmoveto{\pgfqpoint{2.777425in}{3.356406in}}%
\pgfpathlineto{\pgfqpoint{2.787075in}{3.239312in}}%
\pgfpathlineto{\pgfqpoint{2.794905in}{3.252935in}}%
\pgfpathlineto{\pgfqpoint{2.830346in}{3.193241in}}%
\pgfpathlineto{\pgfqpoint{2.865911in}{3.119422in}}%
\pgfpathlineto{\pgfqpoint{2.854032in}{3.405171in}}%
\pgfpathlineto{\pgfqpoint{2.847528in}{3.286866in}}%
\pgfpathlineto{\pgfqpoint{2.813464in}{3.251293in}}%
\pgfpathlineto{\pgfqpoint{2.777425in}{3.356406in}}%
\pgfpathclose%
\pgfusepath{fill}%
\end{pgfscope}%
\begin{pgfscope}%
\pgfpathrectangle{\pgfqpoint{1.020000in}{0.880000in}}{\pgfqpoint{6.160000in}{6.160000in}}%
\pgfusepath{clip}%
\pgfsetbuttcap%
\pgfsetroundjoin%
\definecolor{currentfill}{rgb}{0.378598,0.503856,0.913692}%
\pgfsetfillcolor{currentfill}%
\pgfsetlinewidth{0.000000pt}%
\definecolor{currentstroke}{rgb}{0.000000,0.000000,0.000000}%
\pgfsetstrokecolor{currentstroke}%
\pgfsetdash{}{0pt}%
\pgfpathmoveto{\pgfqpoint{4.948784in}{2.710017in}}%
\pgfpathlineto{\pgfqpoint{4.959234in}{2.721676in}}%
\pgfpathlineto{\pgfqpoint{4.970379in}{2.817793in}}%
\pgfpathlineto{\pgfqpoint{5.004443in}{2.830090in}}%
\pgfpathlineto{\pgfqpoint{5.038399in}{2.831348in}}%
\pgfpathlineto{\pgfqpoint{5.028364in}{2.881648in}}%
\pgfpathlineto{\pgfqpoint{5.019289in}{3.044938in}}%
\pgfpathlineto{\pgfqpoint{4.984944in}{2.995980in}}%
\pgfpathlineto{\pgfqpoint{4.948784in}{2.710017in}}%
\pgfpathclose%
\pgfusepath{fill}%
\end{pgfscope}%
\begin{pgfscope}%
\pgfpathrectangle{\pgfqpoint{1.020000in}{0.880000in}}{\pgfqpoint{6.160000in}{6.160000in}}%
\pgfusepath{clip}%
\pgfsetbuttcap%
\pgfsetroundjoin%
\definecolor{currentfill}{rgb}{0.328604,0.439712,0.869587}%
\pgfsetfillcolor{currentfill}%
\pgfsetlinewidth{0.000000pt}%
\definecolor{currentstroke}{rgb}{0.000000,0.000000,0.000000}%
\pgfsetstrokecolor{currentstroke}%
\pgfsetdash{}{0pt}%
\pgfpathmoveto{\pgfqpoint{5.622638in}{2.808291in}}%
\pgfpathlineto{\pgfqpoint{5.633056in}{2.760275in}}%
\pgfpathlineto{\pgfqpoint{5.643609in}{2.720510in}}%
\pgfpathlineto{\pgfqpoint{5.678167in}{2.773586in}}%
\pgfpathlineto{\pgfqpoint{5.711082in}{2.718939in}}%
\pgfpathlineto{\pgfqpoint{5.699205in}{2.674570in}}%
\pgfpathlineto{\pgfqpoint{5.689287in}{2.757162in}}%
\pgfpathlineto{\pgfqpoint{5.655874in}{2.775453in}}%
\pgfpathlineto{\pgfqpoint{5.622638in}{2.808291in}}%
\pgfpathclose%
\pgfusepath{fill}%
\end{pgfscope}%
\begin{pgfscope}%
\pgfpathrectangle{\pgfqpoint{1.020000in}{0.880000in}}{\pgfqpoint{6.160000in}{6.160000in}}%
\pgfusepath{clip}%
\pgfsetbuttcap%
\pgfsetroundjoin%
\definecolor{currentfill}{rgb}{0.243520,0.319189,0.771672}%
\pgfsetfillcolor{currentfill}%
\pgfsetlinewidth{0.000000pt}%
\definecolor{currentstroke}{rgb}{0.000000,0.000000,0.000000}%
\pgfsetstrokecolor{currentstroke}%
\pgfsetdash{}{0pt}%
\pgfpathmoveto{\pgfqpoint{6.226370in}{2.631355in}}%
\pgfpathlineto{\pgfqpoint{6.236943in}{2.572941in}}%
\pgfpathlineto{\pgfqpoint{6.246665in}{2.473093in}}%
\pgfpathlineto{\pgfqpoint{6.280494in}{2.486456in}}%
\pgfpathlineto{\pgfqpoint{6.269529in}{2.527306in}}%
\pgfpathlineto{\pgfqpoint{6.262270in}{2.744392in}}%
\pgfpathlineto{\pgfqpoint{6.226370in}{2.631355in}}%
\pgfpathclose%
\pgfusepath{fill}%
\end{pgfscope}%
\begin{pgfscope}%
\pgfpathrectangle{\pgfqpoint{1.020000in}{0.880000in}}{\pgfqpoint{6.160000in}{6.160000in}}%
\pgfusepath{clip}%
\pgfsetbuttcap%
\pgfsetroundjoin%
\definecolor{currentfill}{rgb}{0.318832,0.426605,0.859857}%
\pgfsetfillcolor{currentfill}%
\pgfsetlinewidth{0.000000pt}%
\definecolor{currentstroke}{rgb}{0.000000,0.000000,0.000000}%
\pgfsetstrokecolor{currentstroke}%
\pgfsetdash{}{0pt}%
\pgfpathmoveto{\pgfqpoint{5.329183in}{2.694793in}}%
\pgfpathlineto{\pgfqpoint{5.341258in}{2.805121in}}%
\pgfpathlineto{\pgfqpoint{5.351872in}{2.787223in}}%
\pgfpathlineto{\pgfqpoint{5.383597in}{2.609203in}}%
\pgfpathlineto{\pgfqpoint{5.419301in}{2.762330in}}%
\pgfpathlineto{\pgfqpoint{5.407804in}{2.712895in}}%
\pgfpathlineto{\pgfqpoint{5.396898in}{2.708958in}}%
\pgfpathlineto{\pgfqpoint{5.363773in}{2.762859in}}%
\pgfpathlineto{\pgfqpoint{5.329183in}{2.694793in}}%
\pgfpathclose%
\pgfusepath{fill}%
\end{pgfscope}%
\begin{pgfscope}%
\pgfpathrectangle{\pgfqpoint{1.020000in}{0.880000in}}{\pgfqpoint{6.160000in}{6.160000in}}%
\pgfusepath{clip}%
\pgfsetbuttcap%
\pgfsetroundjoin%
\definecolor{currentfill}{rgb}{0.748682,0.827679,0.963334}%
\pgfsetfillcolor{currentfill}%
\pgfsetlinewidth{0.000000pt}%
\definecolor{currentstroke}{rgb}{0.000000,0.000000,0.000000}%
\pgfsetstrokecolor{currentstroke}%
\pgfsetdash{}{0pt}%
\pgfpathmoveto{\pgfqpoint{3.344290in}{3.515683in}}%
\pgfpathlineto{\pgfqpoint{3.352451in}{3.571252in}}%
\pgfpathlineto{\pgfqpoint{3.361834in}{3.481381in}}%
\pgfpathlineto{\pgfqpoint{3.396601in}{3.440366in}}%
\pgfpathlineto{\pgfqpoint{3.429687in}{3.616344in}}%
\pgfpathlineto{\pgfqpoint{3.421746in}{3.514238in}}%
\pgfpathlineto{\pgfqpoint{3.411322in}{3.739449in}}%
\pgfpathlineto{\pgfqpoint{3.378859in}{3.493129in}}%
\pgfpathlineto{\pgfqpoint{3.344290in}{3.515683in}}%
\pgfpathclose%
\pgfusepath{fill}%
\end{pgfscope}%
\begin{pgfscope}%
\pgfpathrectangle{\pgfqpoint{1.020000in}{0.880000in}}{\pgfqpoint{6.160000in}{6.160000in}}%
\pgfusepath{clip}%
\pgfsetbuttcap%
\pgfsetroundjoin%
\definecolor{currentfill}{rgb}{0.809329,0.852974,0.922323}%
\pgfsetfillcolor{currentfill}%
\pgfsetlinewidth{0.000000pt}%
\definecolor{currentstroke}{rgb}{0.000000,0.000000,0.000000}%
\pgfsetstrokecolor{currentstroke}%
\pgfsetdash{}{0pt}%
\pgfpathmoveto{\pgfqpoint{3.481121in}{3.600255in}}%
\pgfpathlineto{\pgfqpoint{3.489503in}{3.656051in}}%
\pgfpathlineto{\pgfqpoint{3.498505in}{3.625172in}}%
\pgfpathlineto{\pgfqpoint{3.532122in}{3.747484in}}%
\pgfpathlineto{\pgfqpoint{3.566583in}{3.742788in}}%
\pgfpathlineto{\pgfqpoint{3.558136in}{3.675201in}}%
\pgfpathlineto{\pgfqpoint{3.548732in}{3.763855in}}%
\pgfpathlineto{\pgfqpoint{3.515597in}{3.578853in}}%
\pgfpathlineto{\pgfqpoint{3.481121in}{3.600255in}}%
\pgfpathclose%
\pgfusepath{fill}%
\end{pgfscope}%
\begin{pgfscope}%
\pgfpathrectangle{\pgfqpoint{1.020000in}{0.880000in}}{\pgfqpoint{6.160000in}{6.160000in}}%
\pgfusepath{clip}%
\pgfsetbuttcap%
\pgfsetroundjoin%
\definecolor{currentfill}{rgb}{0.630089,0.752516,0.998508}%
\pgfsetfillcolor{currentfill}%
\pgfsetlinewidth{0.000000pt}%
\definecolor{currentstroke}{rgb}{0.000000,0.000000,0.000000}%
\pgfsetstrokecolor{currentstroke}%
\pgfsetdash{}{0pt}%
\pgfpathmoveto{\pgfqpoint{4.395823in}{3.412385in}}%
\pgfpathlineto{\pgfqpoint{4.405631in}{3.425881in}}%
\pgfpathlineto{\pgfqpoint{4.415018in}{3.247858in}}%
\pgfpathlineto{\pgfqpoint{4.449213in}{3.273322in}}%
\pgfpathlineto{\pgfqpoint{4.483656in}{3.382570in}}%
\pgfpathlineto{\pgfqpoint{4.473467in}{3.271018in}}%
\pgfpathlineto{\pgfqpoint{4.463734in}{3.306501in}}%
\pgfpathlineto{\pgfqpoint{4.429410in}{3.198300in}}%
\pgfpathlineto{\pgfqpoint{4.395823in}{3.412385in}}%
\pgfpathclose%
\pgfusepath{fill}%
\end{pgfscope}%
\begin{pgfscope}%
\pgfpathrectangle{\pgfqpoint{1.020000in}{0.880000in}}{\pgfqpoint{6.160000in}{6.160000in}}%
\pgfusepath{clip}%
\pgfsetbuttcap%
\pgfsetroundjoin%
\definecolor{currentfill}{rgb}{0.576051,0.708780,0.997755}%
\pgfsetfillcolor{currentfill}%
\pgfsetlinewidth{0.000000pt}%
\definecolor{currentstroke}{rgb}{0.000000,0.000000,0.000000}%
\pgfsetstrokecolor{currentstroke}%
\pgfsetdash{}{0pt}%
\pgfpathmoveto{\pgfqpoint{4.551409in}{3.250042in}}%
\pgfpathlineto{\pgfqpoint{4.561637in}{3.325693in}}%
\pgfpathlineto{\pgfqpoint{4.571117in}{3.199486in}}%
\pgfpathlineto{\pgfqpoint{4.605153in}{3.185357in}}%
\pgfpathlineto{\pgfqpoint{4.638965in}{3.128792in}}%
\pgfpathlineto{\pgfqpoint{4.629092in}{3.161702in}}%
\pgfpathlineto{\pgfqpoint{4.619495in}{3.253915in}}%
\pgfpathlineto{\pgfqpoint{4.585481in}{3.256867in}}%
\pgfpathlineto{\pgfqpoint{4.551409in}{3.250042in}}%
\pgfpathclose%
\pgfusepath{fill}%
\end{pgfscope}%
\begin{pgfscope}%
\pgfpathrectangle{\pgfqpoint{1.020000in}{0.880000in}}{\pgfqpoint{6.160000in}{6.160000in}}%
\pgfusepath{clip}%
\pgfsetbuttcap%
\pgfsetroundjoin%
\definecolor{currentfill}{rgb}{0.724041,0.814910,0.975651}%
\pgfsetfillcolor{currentfill}%
\pgfsetlinewidth{0.000000pt}%
\definecolor{currentstroke}{rgb}{0.000000,0.000000,0.000000}%
\pgfsetstrokecolor{currentstroke}%
\pgfsetdash{}{0pt}%
\pgfpathmoveto{\pgfqpoint{3.276475in}{3.403092in}}%
\pgfpathlineto{\pgfqpoint{3.283507in}{3.571173in}}%
\pgfpathlineto{\pgfqpoint{3.294045in}{3.350616in}}%
\pgfpathlineto{\pgfqpoint{3.326314in}{3.604604in}}%
\pgfpathlineto{\pgfqpoint{3.361834in}{3.481381in}}%
\pgfpathlineto{\pgfqpoint{3.352451in}{3.571252in}}%
\pgfpathlineto{\pgfqpoint{3.344290in}{3.515683in}}%
\pgfpathlineto{\pgfqpoint{3.310122in}{3.488259in}}%
\pgfpathlineto{\pgfqpoint{3.276475in}{3.403092in}}%
\pgfpathclose%
\pgfusepath{fill}%
\end{pgfscope}%
\begin{pgfscope}%
\pgfpathrectangle{\pgfqpoint{1.020000in}{0.880000in}}{\pgfqpoint{6.160000in}{6.160000in}}%
\pgfusepath{clip}%
\pgfsetbuttcap%
\pgfsetroundjoin%
\definecolor{currentfill}{rgb}{0.782049,0.842864,0.942980}%
\pgfsetfillcolor{currentfill}%
\pgfsetlinewidth{0.000000pt}%
\definecolor{currentstroke}{rgb}{0.000000,0.000000,0.000000}%
\pgfsetstrokecolor{currentstroke}%
\pgfsetdash{}{0pt}%
\pgfpathmoveto{\pgfqpoint{3.928922in}{3.469601in}}%
\pgfpathlineto{\pgfqpoint{3.937497in}{3.744333in}}%
\pgfpathlineto{\pgfqpoint{3.946755in}{3.759490in}}%
\pgfpathlineto{\pgfqpoint{3.981351in}{3.635052in}}%
\pgfpathlineto{\pgfqpoint{4.015870in}{3.493338in}}%
\pgfpathlineto{\pgfqpoint{4.006357in}{3.591621in}}%
\pgfpathlineto{\pgfqpoint{3.997009in}{3.602737in}}%
\pgfpathlineto{\pgfqpoint{3.962724in}{3.640362in}}%
\pgfpathlineto{\pgfqpoint{3.928922in}{3.469601in}}%
\pgfpathclose%
\pgfusepath{fill}%
\end{pgfscope}%
\begin{pgfscope}%
\pgfpathrectangle{\pgfqpoint{1.020000in}{0.880000in}}{\pgfqpoint{6.160000in}{6.160000in}}%
\pgfusepath{clip}%
\pgfsetbuttcap%
\pgfsetroundjoin%
\definecolor{currentfill}{rgb}{0.299441,0.400248,0.839842}%
\pgfsetfillcolor{currentfill}%
\pgfsetlinewidth{0.000000pt}%
\definecolor{currentstroke}{rgb}{0.000000,0.000000,0.000000}%
\pgfsetstrokecolor{currentstroke}%
\pgfsetdash{}{0pt}%
\pgfpathmoveto{\pgfqpoint{5.260702in}{2.615855in}}%
\pgfpathlineto{\pgfqpoint{5.272529in}{2.717418in}}%
\pgfpathlineto{\pgfqpoint{5.283146in}{2.705717in}}%
\pgfpathlineto{\pgfqpoint{5.315665in}{2.585336in}}%
\pgfpathlineto{\pgfqpoint{5.351872in}{2.787223in}}%
\pgfpathlineto{\pgfqpoint{5.341258in}{2.805121in}}%
\pgfpathlineto{\pgfqpoint{5.329183in}{2.694793in}}%
\pgfpathlineto{\pgfqpoint{5.294814in}{2.644155in}}%
\pgfpathlineto{\pgfqpoint{5.260702in}{2.615855in}}%
\pgfpathclose%
\pgfusepath{fill}%
\end{pgfscope}%
\begin{pgfscope}%
\pgfpathrectangle{\pgfqpoint{1.020000in}{0.880000in}}{\pgfqpoint{6.160000in}{6.160000in}}%
\pgfusepath{clip}%
\pgfsetbuttcap%
\pgfsetroundjoin%
\definecolor{currentfill}{rgb}{0.313946,0.420052,0.854993}%
\pgfsetfillcolor{currentfill}%
\pgfsetlinewidth{0.000000pt}%
\definecolor{currentstroke}{rgb}{0.000000,0.000000,0.000000}%
\pgfsetstrokecolor{currentstroke}%
\pgfsetdash{}{0pt}%
\pgfpathmoveto{\pgfqpoint{5.553082in}{2.659747in}}%
\pgfpathlineto{\pgfqpoint{5.563282in}{2.599740in}}%
\pgfpathlineto{\pgfqpoint{5.575617in}{2.691243in}}%
\pgfpathlineto{\pgfqpoint{5.610018in}{2.733696in}}%
\pgfpathlineto{\pgfqpoint{5.643609in}{2.720510in}}%
\pgfpathlineto{\pgfqpoint{5.633056in}{2.760275in}}%
\pgfpathlineto{\pgfqpoint{5.622638in}{2.808291in}}%
\pgfpathlineto{\pgfqpoint{5.588636in}{2.789950in}}%
\pgfpathlineto{\pgfqpoint{5.553082in}{2.659747in}}%
\pgfpathclose%
\pgfusepath{fill}%
\end{pgfscope}%
\begin{pgfscope}%
\pgfpathrectangle{\pgfqpoint{1.020000in}{0.880000in}}{\pgfqpoint{6.160000in}{6.160000in}}%
\pgfusepath{clip}%
\pgfsetbuttcap%
\pgfsetroundjoin%
\definecolor{currentfill}{rgb}{0.328604,0.439712,0.869587}%
\pgfsetfillcolor{currentfill}%
\pgfsetlinewidth{0.000000pt}%
\definecolor{currentstroke}{rgb}{0.000000,0.000000,0.000000}%
\pgfsetstrokecolor{currentstroke}%
\pgfsetdash{}{0pt}%
\pgfpathmoveto{\pgfqpoint{5.845390in}{2.692762in}}%
\pgfpathlineto{\pgfqpoint{5.858365in}{2.788802in}}%
\pgfpathlineto{\pgfqpoint{5.868359in}{2.706502in}}%
\pgfpathlineto{\pgfqpoint{5.902584in}{2.736931in}}%
\pgfpathlineto{\pgfqpoint{5.937521in}{2.807703in}}%
\pgfpathlineto{\pgfqpoint{5.920847in}{2.510432in}}%
\pgfpathlineto{\pgfqpoint{5.917541in}{2.975054in}}%
\pgfpathlineto{\pgfqpoint{5.880103in}{2.756348in}}%
\pgfpathlineto{\pgfqpoint{5.845390in}{2.692762in}}%
\pgfpathclose%
\pgfusepath{fill}%
\end{pgfscope}%
\begin{pgfscope}%
\pgfpathrectangle{\pgfqpoint{1.020000in}{0.880000in}}{\pgfqpoint{6.160000in}{6.160000in}}%
\pgfusepath{clip}%
\pgfsetbuttcap%
\pgfsetroundjoin%
\definecolor{currentfill}{rgb}{0.791392,0.846750,0.936641}%
\pgfsetfillcolor{currentfill}%
\pgfsetlinewidth{0.000000pt}%
\definecolor{currentstroke}{rgb}{0.000000,0.000000,0.000000}%
\pgfsetstrokecolor{currentstroke}%
\pgfsetdash{}{0pt}%
\pgfpathmoveto{\pgfqpoint{3.411322in}{3.739449in}}%
\pgfpathlineto{\pgfqpoint{3.421746in}{3.514238in}}%
\pgfpathlineto{\pgfqpoint{3.429687in}{3.616344in}}%
\pgfpathlineto{\pgfqpoint{3.463576in}{3.694455in}}%
\pgfpathlineto{\pgfqpoint{3.498505in}{3.625172in}}%
\pgfpathlineto{\pgfqpoint{3.489503in}{3.656051in}}%
\pgfpathlineto{\pgfqpoint{3.481121in}{3.600255in}}%
\pgfpathlineto{\pgfqpoint{3.446542in}{3.630891in}}%
\pgfpathlineto{\pgfqpoint{3.411322in}{3.739449in}}%
\pgfpathclose%
\pgfusepath{fill}%
\end{pgfscope}%
\begin{pgfscope}%
\pgfpathrectangle{\pgfqpoint{1.020000in}{0.880000in}}{\pgfqpoint{6.160000in}{6.160000in}}%
\pgfusepath{clip}%
\pgfsetbuttcap%
\pgfsetroundjoin%
\definecolor{currentfill}{rgb}{0.708720,0.805721,0.981117}%
\pgfsetfillcolor{currentfill}%
\pgfsetlinewidth{0.000000pt}%
\definecolor{currentstroke}{rgb}{0.000000,0.000000,0.000000}%
\pgfsetstrokecolor{currentstroke}%
\pgfsetdash{}{0pt}%
\pgfpathmoveto{\pgfqpoint{3.206204in}{3.549402in}}%
\pgfpathlineto{\pgfqpoint{3.216283in}{3.384659in}}%
\pgfpathlineto{\pgfqpoint{3.224153in}{3.449223in}}%
\pgfpathlineto{\pgfqpoint{3.258591in}{3.457851in}}%
\pgfpathlineto{\pgfqpoint{3.294045in}{3.350616in}}%
\pgfpathlineto{\pgfqpoint{3.283507in}{3.571173in}}%
\pgfpathlineto{\pgfqpoint{3.276475in}{3.403092in}}%
\pgfpathlineto{\pgfqpoint{3.240703in}{3.547642in}}%
\pgfpathlineto{\pgfqpoint{3.206204in}{3.549402in}}%
\pgfpathclose%
\pgfusepath{fill}%
\end{pgfscope}%
\begin{pgfscope}%
\pgfpathrectangle{\pgfqpoint{1.020000in}{0.880000in}}{\pgfqpoint{6.160000in}{6.160000in}}%
\pgfusepath{clip}%
\pgfsetbuttcap%
\pgfsetroundjoin%
\definecolor{currentfill}{rgb}{0.358415,0.478426,0.896795}%
\pgfsetfillcolor{currentfill}%
\pgfsetlinewidth{0.000000pt}%
\definecolor{currentstroke}{rgb}{0.000000,0.000000,0.000000}%
\pgfsetstrokecolor{currentstroke}%
\pgfsetdash{}{0pt}%
\pgfpathmoveto{\pgfqpoint{5.105811in}{2.786347in}}%
\pgfpathlineto{\pgfqpoint{5.116772in}{2.830212in}}%
\pgfpathlineto{\pgfqpoint{5.128315in}{2.931878in}}%
\pgfpathlineto{\pgfqpoint{5.160409in}{2.739636in}}%
\pgfpathlineto{\pgfqpoint{5.194787in}{2.786620in}}%
\pgfpathlineto{\pgfqpoint{5.184700in}{2.841435in}}%
\pgfpathlineto{\pgfqpoint{5.172549in}{2.688119in}}%
\pgfpathlineto{\pgfqpoint{5.140477in}{2.869121in}}%
\pgfpathlineto{\pgfqpoint{5.105811in}{2.786347in}}%
\pgfpathclose%
\pgfusepath{fill}%
\end{pgfscope}%
\begin{pgfscope}%
\pgfpathrectangle{\pgfqpoint{1.020000in}{0.880000in}}{\pgfqpoint{6.160000in}{6.160000in}}%
\pgfusepath{clip}%
\pgfsetbuttcap%
\pgfsetroundjoin%
\definecolor{currentfill}{rgb}{0.378598,0.503856,0.913692}%
\pgfsetfillcolor{currentfill}%
\pgfsetlinewidth{0.000000pt}%
\definecolor{currentstroke}{rgb}{0.000000,0.000000,0.000000}%
\pgfsetstrokecolor{currentstroke}%
\pgfsetdash{}{0pt}%
\pgfpathmoveto{\pgfqpoint{4.882692in}{2.949890in}}%
\pgfpathlineto{\pgfqpoint{4.893042in}{2.954598in}}%
\pgfpathlineto{\pgfqpoint{4.903235in}{2.934282in}}%
\pgfpathlineto{\pgfqpoint{4.935947in}{2.758611in}}%
\pgfpathlineto{\pgfqpoint{4.970379in}{2.817793in}}%
\pgfpathlineto{\pgfqpoint{4.959234in}{2.721676in}}%
\pgfpathlineto{\pgfqpoint{4.948784in}{2.710017in}}%
\pgfpathlineto{\pgfqpoint{4.916430in}{2.916908in}}%
\pgfpathlineto{\pgfqpoint{4.882692in}{2.949890in}}%
\pgfpathclose%
\pgfusepath{fill}%
\end{pgfscope}%
\begin{pgfscope}%
\pgfpathrectangle{\pgfqpoint{1.020000in}{0.880000in}}{\pgfqpoint{6.160000in}{6.160000in}}%
\pgfusepath{clip}%
\pgfsetbuttcap%
\pgfsetroundjoin%
\definecolor{currentfill}{rgb}{0.603162,0.731527,0.999565}%
\pgfsetfillcolor{currentfill}%
\pgfsetlinewidth{0.000000pt}%
\definecolor{currentstroke}{rgb}{0.000000,0.000000,0.000000}%
\pgfsetstrokecolor{currentstroke}%
\pgfsetdash{}{0pt}%
\pgfpathmoveto{\pgfqpoint{2.709896in}{3.239405in}}%
\pgfpathlineto{\pgfqpoint{2.717713in}{3.247206in}}%
\pgfpathlineto{\pgfqpoint{2.722651in}{3.453420in}}%
\pgfpathlineto{\pgfqpoint{2.761741in}{3.148461in}}%
\pgfpathlineto{\pgfqpoint{2.794905in}{3.252935in}}%
\pgfpathlineto{\pgfqpoint{2.787075in}{3.239312in}}%
\pgfpathlineto{\pgfqpoint{2.777425in}{3.356406in}}%
\pgfpathlineto{\pgfqpoint{2.745381in}{3.177847in}}%
\pgfpathlineto{\pgfqpoint{2.709896in}{3.239405in}}%
\pgfpathclose%
\pgfusepath{fill}%
\end{pgfscope}%
\begin{pgfscope}%
\pgfpathrectangle{\pgfqpoint{1.020000in}{0.880000in}}{\pgfqpoint{6.160000in}{6.160000in}}%
\pgfusepath{clip}%
\pgfsetbuttcap%
\pgfsetroundjoin%
\definecolor{currentfill}{rgb}{0.299441,0.400248,0.839842}%
\pgfsetfillcolor{currentfill}%
\pgfsetlinewidth{0.000000pt}%
\definecolor{currentstroke}{rgb}{0.000000,0.000000,0.000000}%
\pgfsetstrokecolor{currentstroke}%
\pgfsetdash{}{0pt}%
\pgfpathmoveto{\pgfqpoint{5.485570in}{2.657871in}}%
\pgfpathlineto{\pgfqpoint{5.496540in}{2.658052in}}%
\pgfpathlineto{\pgfqpoint{5.508627in}{2.740176in}}%
\pgfpathlineto{\pgfqpoint{5.542757in}{2.760691in}}%
\pgfpathlineto{\pgfqpoint{5.575617in}{2.691243in}}%
\pgfpathlineto{\pgfqpoint{5.563282in}{2.599740in}}%
\pgfpathlineto{\pgfqpoint{5.553082in}{2.659747in}}%
\pgfpathlineto{\pgfqpoint{5.520143in}{2.718737in}}%
\pgfpathlineto{\pgfqpoint{5.485570in}{2.657871in}}%
\pgfpathclose%
\pgfusepath{fill}%
\end{pgfscope}%
\begin{pgfscope}%
\pgfpathrectangle{\pgfqpoint{1.020000in}{0.880000in}}{\pgfqpoint{6.160000in}{6.160000in}}%
\pgfusepath{clip}%
\pgfsetbuttcap%
\pgfsetroundjoin%
\definecolor{currentfill}{rgb}{0.266381,0.353304,0.801637}%
\pgfsetfillcolor{currentfill}%
\pgfsetlinewidth{0.000000pt}%
\definecolor{currentstroke}{rgb}{0.000000,0.000000,0.000000}%
\pgfsetstrokecolor{currentstroke}%
\pgfsetdash{}{0pt}%
\pgfpathmoveto{\pgfqpoint{5.937521in}{2.807703in}}%
\pgfpathlineto{\pgfqpoint{5.946334in}{2.657278in}}%
\pgfpathlineto{\pgfqpoint{5.955836in}{2.545960in}}%
\pgfpathlineto{\pgfqpoint{5.991667in}{2.664561in}}%
\pgfpathlineto{\pgfqpoint{6.023814in}{2.581536in}}%
\pgfpathlineto{\pgfqpoint{6.010603in}{2.491584in}}%
\pgfpathlineto{\pgfqpoint{6.001215in}{2.607619in}}%
\pgfpathlineto{\pgfqpoint{5.967323in}{2.591547in}}%
\pgfpathlineto{\pgfqpoint{5.937521in}{2.807703in}}%
\pgfpathclose%
\pgfusepath{fill}%
\end{pgfscope}%
\begin{pgfscope}%
\pgfpathrectangle{\pgfqpoint{1.020000in}{0.880000in}}{\pgfqpoint{6.160000in}{6.160000in}}%
\pgfusepath{clip}%
\pgfsetbuttcap%
\pgfsetroundjoin%
\definecolor{currentfill}{rgb}{0.708720,0.805721,0.981117}%
\pgfsetfillcolor{currentfill}%
\pgfsetlinewidth{0.000000pt}%
\definecolor{currentstroke}{rgb}{0.000000,0.000000,0.000000}%
\pgfsetstrokecolor{currentstroke}%
\pgfsetdash{}{0pt}%
\pgfpathmoveto{\pgfqpoint{4.239965in}{3.458446in}}%
\pgfpathlineto{\pgfqpoint{4.249626in}{3.545202in}}%
\pgfpathlineto{\pgfqpoint{4.259299in}{3.599020in}}%
\pgfpathlineto{\pgfqpoint{4.293398in}{3.451318in}}%
\pgfpathlineto{\pgfqpoint{4.327567in}{3.437178in}}%
\pgfpathlineto{\pgfqpoint{4.317810in}{3.387176in}}%
\pgfpathlineto{\pgfqpoint{4.308209in}{3.438795in}}%
\pgfpathlineto{\pgfqpoint{4.274027in}{3.367317in}}%
\pgfpathlineto{\pgfqpoint{4.239965in}{3.458446in}}%
\pgfpathclose%
\pgfusepath{fill}%
\end{pgfscope}%
\begin{pgfscope}%
\pgfpathrectangle{\pgfqpoint{1.020000in}{0.880000in}}{\pgfqpoint{6.160000in}{6.160000in}}%
\pgfusepath{clip}%
\pgfsetbuttcap%
\pgfsetroundjoin%
\definecolor{currentfill}{rgb}{0.313946,0.420052,0.854993}%
\pgfsetfillcolor{currentfill}%
\pgfsetlinewidth{0.000000pt}%
\definecolor{currentstroke}{rgb}{0.000000,0.000000,0.000000}%
\pgfsetstrokecolor{currentstroke}%
\pgfsetdash{}{0pt}%
\pgfpathmoveto{\pgfqpoint{5.780332in}{2.833142in}}%
\pgfpathlineto{\pgfqpoint{5.789473in}{2.699449in}}%
\pgfpathlineto{\pgfqpoint{5.797864in}{2.520093in}}%
\pgfpathlineto{\pgfqpoint{5.834213in}{2.680779in}}%
\pgfpathlineto{\pgfqpoint{5.868359in}{2.706502in}}%
\pgfpathlineto{\pgfqpoint{5.858365in}{2.788802in}}%
\pgfpathlineto{\pgfqpoint{5.845390in}{2.692762in}}%
\pgfpathlineto{\pgfqpoint{5.814005in}{2.830581in}}%
\pgfpathlineto{\pgfqpoint{5.780332in}{2.833142in}}%
\pgfpathclose%
\pgfusepath{fill}%
\end{pgfscope}%
\begin{pgfscope}%
\pgfpathrectangle{\pgfqpoint{1.020000in}{0.880000in}}{\pgfqpoint{6.160000in}{6.160000in}}%
\pgfusepath{clip}%
\pgfsetbuttcap%
\pgfsetroundjoin%
\definecolor{currentfill}{rgb}{0.818056,0.855590,0.914638}%
\pgfsetfillcolor{currentfill}%
\pgfsetlinewidth{0.000000pt}%
\definecolor{currentstroke}{rgb}{0.000000,0.000000,0.000000}%
\pgfsetstrokecolor{currentstroke}%
\pgfsetdash{}{0pt}%
\pgfpathmoveto{\pgfqpoint{3.704446in}{3.669435in}}%
\pgfpathlineto{\pgfqpoint{3.713293in}{3.706266in}}%
\pgfpathlineto{\pgfqpoint{3.721611in}{3.864042in}}%
\pgfpathlineto{\pgfqpoint{3.756875in}{3.669489in}}%
\pgfpathlineto{\pgfqpoint{3.791111in}{3.698644in}}%
\pgfpathlineto{\pgfqpoint{3.781785in}{3.752246in}}%
\pgfpathlineto{\pgfqpoint{3.773850in}{3.468521in}}%
\pgfpathlineto{\pgfqpoint{3.738543in}{3.715545in}}%
\pgfpathlineto{\pgfqpoint{3.704446in}{3.669435in}}%
\pgfpathclose%
\pgfusepath{fill}%
\end{pgfscope}%
\begin{pgfscope}%
\pgfpathrectangle{\pgfqpoint{1.020000in}{0.880000in}}{\pgfqpoint{6.160000in}{6.160000in}}%
\pgfusepath{clip}%
\pgfsetbuttcap%
\pgfsetroundjoin%
\definecolor{currentfill}{rgb}{0.441123,0.576532,0.954545}%
\pgfsetfillcolor{currentfill}%
\pgfsetlinewidth{0.000000pt}%
\definecolor{currentstroke}{rgb}{0.000000,0.000000,0.000000}%
\pgfsetstrokecolor{currentstroke}%
\pgfsetdash{}{0pt}%
\pgfpathmoveto{\pgfqpoint{4.726082in}{2.915714in}}%
\pgfpathlineto{\pgfqpoint{4.736630in}{2.990610in}}%
\pgfpathlineto{\pgfqpoint{4.745663in}{2.791137in}}%
\pgfpathlineto{\pgfqpoint{4.780523in}{2.932786in}}%
\pgfpathlineto{\pgfqpoint{4.814666in}{2.949554in}}%
\pgfpathlineto{\pgfqpoint{4.805536in}{3.125136in}}%
\pgfpathlineto{\pgfqpoint{4.794624in}{3.012638in}}%
\pgfpathlineto{\pgfqpoint{4.760437in}{2.979671in}}%
\pgfpathlineto{\pgfqpoint{4.726082in}{2.915714in}}%
\pgfpathclose%
\pgfusepath{fill}%
\end{pgfscope}%
\begin{pgfscope}%
\pgfpathrectangle{\pgfqpoint{1.020000in}{0.880000in}}{\pgfqpoint{6.160000in}{6.160000in}}%
\pgfusepath{clip}%
\pgfsetbuttcap%
\pgfsetroundjoin%
\definecolor{currentfill}{rgb}{0.661968,0.775491,0.993937}%
\pgfsetfillcolor{currentfill}%
\pgfsetlinewidth{0.000000pt}%
\definecolor{currentstroke}{rgb}{0.000000,0.000000,0.000000}%
\pgfsetstrokecolor{currentstroke}%
\pgfsetdash{}{0pt}%
\pgfpathmoveto{\pgfqpoint{3.068687in}{3.493721in}}%
\pgfpathlineto{\pgfqpoint{3.079154in}{3.301313in}}%
\pgfpathlineto{\pgfqpoint{3.088321in}{3.226098in}}%
\pgfpathlineto{\pgfqpoint{3.121585in}{3.344683in}}%
\pgfpathlineto{\pgfqpoint{3.154051in}{3.546950in}}%
\pgfpathlineto{\pgfqpoint{3.147191in}{3.395163in}}%
\pgfpathlineto{\pgfqpoint{3.138239in}{3.447742in}}%
\pgfpathlineto{\pgfqpoint{3.106338in}{3.206030in}}%
\pgfpathlineto{\pgfqpoint{3.068687in}{3.493721in}}%
\pgfpathclose%
\pgfusepath{fill}%
\end{pgfscope}%
\begin{pgfscope}%
\pgfpathrectangle{\pgfqpoint{1.020000in}{0.880000in}}{\pgfqpoint{6.160000in}{6.160000in}}%
\pgfusepath{clip}%
\pgfsetbuttcap%
\pgfsetroundjoin%
\definecolor{currentfill}{rgb}{0.768034,0.837035,0.952488}%
\pgfsetfillcolor{currentfill}%
\pgfsetlinewidth{0.000000pt}%
\definecolor{currentstroke}{rgb}{0.000000,0.000000,0.000000}%
\pgfsetstrokecolor{currentstroke}%
\pgfsetdash{}{0pt}%
\pgfpathmoveto{\pgfqpoint{4.084325in}{3.441710in}}%
\pgfpathlineto{\pgfqpoint{4.093711in}{3.482691in}}%
\pgfpathlineto{\pgfqpoint{4.103054in}{3.613739in}}%
\pgfpathlineto{\pgfqpoint{4.137351in}{3.577854in}}%
\pgfpathlineto{\pgfqpoint{4.171600in}{3.572925in}}%
\pgfpathlineto{\pgfqpoint{4.162071in}{3.590532in}}%
\pgfpathlineto{\pgfqpoint{4.152536in}{3.685756in}}%
\pgfpathlineto{\pgfqpoint{4.118364in}{3.618903in}}%
\pgfpathlineto{\pgfqpoint{4.084325in}{3.441710in}}%
\pgfpathclose%
\pgfusepath{fill}%
\end{pgfscope}%
\begin{pgfscope}%
\pgfpathrectangle{\pgfqpoint{1.020000in}{0.880000in}}{\pgfqpoint{6.160000in}{6.160000in}}%
\pgfusepath{clip}%
\pgfsetbuttcap%
\pgfsetroundjoin%
\definecolor{currentfill}{rgb}{0.603162,0.731527,0.999565}%
\pgfsetfillcolor{currentfill}%
\pgfsetlinewidth{0.000000pt}%
\definecolor{currentstroke}{rgb}{0.000000,0.000000,0.000000}%
\pgfsetstrokecolor{currentstroke}%
\pgfsetdash{}{0pt}%
\pgfpathmoveto{\pgfqpoint{2.933192in}{3.261934in}}%
\pgfpathlineto{\pgfqpoint{2.941197in}{3.275996in}}%
\pgfpathlineto{\pgfqpoint{2.950558in}{3.182002in}}%
\pgfpathlineto{\pgfqpoint{2.984664in}{3.222541in}}%
\pgfpathlineto{\pgfqpoint{3.018105in}{3.320769in}}%
\pgfpathlineto{\pgfqpoint{3.010296in}{3.281254in}}%
\pgfpathlineto{\pgfqpoint{3.002368in}{3.253861in}}%
\pgfpathlineto{\pgfqpoint{2.967856in}{3.252660in}}%
\pgfpathlineto{\pgfqpoint{2.933192in}{3.261934in}}%
\pgfpathclose%
\pgfusepath{fill}%
\end{pgfscope}%
\begin{pgfscope}%
\pgfpathrectangle{\pgfqpoint{1.020000in}{0.880000in}}{\pgfqpoint{6.160000in}{6.160000in}}%
\pgfusepath{clip}%
\pgfsetbuttcap%
\pgfsetroundjoin%
\definecolor{currentfill}{rgb}{0.478462,0.616564,0.972721}%
\pgfsetfillcolor{currentfill}%
\pgfsetlinewidth{0.000000pt}%
\definecolor{currentstroke}{rgb}{0.000000,0.000000,0.000000}%
\pgfsetstrokecolor{currentstroke}%
\pgfsetdash{}{0pt}%
\pgfpathmoveto{\pgfqpoint{4.794624in}{3.012638in}}%
\pgfpathlineto{\pgfqpoint{4.805536in}{3.125136in}}%
\pgfpathlineto{\pgfqpoint{4.814666in}{2.949554in}}%
\pgfpathlineto{\pgfqpoint{4.848979in}{2.992947in}}%
\pgfpathlineto{\pgfqpoint{4.882692in}{2.949890in}}%
\pgfpathlineto{\pgfqpoint{4.872966in}{3.031883in}}%
\pgfpathlineto{\pgfqpoint{4.863098in}{3.092214in}}%
\pgfpathlineto{\pgfqpoint{4.829224in}{3.108858in}}%
\pgfpathlineto{\pgfqpoint{4.794624in}{3.012638in}}%
\pgfpathclose%
\pgfusepath{fill}%
\end{pgfscope}%
\begin{pgfscope}%
\pgfpathrectangle{\pgfqpoint{1.020000in}{0.880000in}}{\pgfqpoint{6.160000in}{6.160000in}}%
\pgfusepath{clip}%
\pgfsetbuttcap%
\pgfsetroundjoin%
\definecolor{currentfill}{rgb}{0.635474,0.756714,0.998297}%
\pgfsetfillcolor{currentfill}%
\pgfsetlinewidth{0.000000pt}%
\definecolor{currentstroke}{rgb}{0.000000,0.000000,0.000000}%
\pgfsetstrokecolor{currentstroke}%
\pgfsetdash{}{0pt}%
\pgfpathmoveto{\pgfqpoint{3.002368in}{3.253861in}}%
\pgfpathlineto{\pgfqpoint{3.010296in}{3.281254in}}%
\pgfpathlineto{\pgfqpoint{3.018105in}{3.320769in}}%
\pgfpathlineto{\pgfqpoint{3.052393in}{3.348298in}}%
\pgfpathlineto{\pgfqpoint{3.088321in}{3.226098in}}%
\pgfpathlineto{\pgfqpoint{3.079154in}{3.301313in}}%
\pgfpathlineto{\pgfqpoint{3.068687in}{3.493721in}}%
\pgfpathlineto{\pgfqpoint{3.036181in}{3.314156in}}%
\pgfpathlineto{\pgfqpoint{3.002368in}{3.253861in}}%
\pgfpathclose%
\pgfusepath{fill}%
\end{pgfscope}%
\begin{pgfscope}%
\pgfpathrectangle{\pgfqpoint{1.020000in}{0.880000in}}{\pgfqpoint{6.160000in}{6.160000in}}%
\pgfusepath{clip}%
\pgfsetbuttcap%
\pgfsetroundjoin%
\definecolor{currentfill}{rgb}{0.266381,0.353304,0.801637}%
\pgfsetfillcolor{currentfill}%
\pgfsetlinewidth{0.000000pt}%
\definecolor{currentstroke}{rgb}{0.000000,0.000000,0.000000}%
\pgfsetstrokecolor{currentstroke}%
\pgfsetdash{}{0pt}%
\pgfpathmoveto{\pgfqpoint{6.158078in}{2.572921in}}%
\pgfpathlineto{\pgfqpoint{6.174279in}{2.797102in}}%
\pgfpathlineto{\pgfqpoint{6.181582in}{2.575871in}}%
\pgfpathlineto{\pgfqpoint{6.215631in}{2.596705in}}%
\pgfpathlineto{\pgfqpoint{6.246665in}{2.473093in}}%
\pgfpathlineto{\pgfqpoint{6.236943in}{2.572941in}}%
\pgfpathlineto{\pgfqpoint{6.226370in}{2.631355in}}%
\pgfpathlineto{\pgfqpoint{6.193343in}{2.657352in}}%
\pgfpathlineto{\pgfqpoint{6.158078in}{2.572921in}}%
\pgfpathclose%
\pgfusepath{fill}%
\end{pgfscope}%
\begin{pgfscope}%
\pgfpathrectangle{\pgfqpoint{1.020000in}{0.880000in}}{\pgfqpoint{6.160000in}{6.160000in}}%
\pgfusepath{clip}%
\pgfsetbuttcap%
\pgfsetroundjoin%
\definecolor{currentfill}{rgb}{0.804965,0.851666,0.926165}%
\pgfsetfillcolor{currentfill}%
\pgfsetlinewidth{0.000000pt}%
\definecolor{currentstroke}{rgb}{0.000000,0.000000,0.000000}%
\pgfsetstrokecolor{currentstroke}%
\pgfsetdash{}{0pt}%
\pgfpathmoveto{\pgfqpoint{3.635860in}{3.655060in}}%
\pgfpathlineto{\pgfqpoint{3.645305in}{3.562790in}}%
\pgfpathlineto{\pgfqpoint{3.654237in}{3.568178in}}%
\pgfpathlineto{\pgfqpoint{3.688010in}{3.690530in}}%
\pgfpathlineto{\pgfqpoint{3.721611in}{3.864042in}}%
\pgfpathlineto{\pgfqpoint{3.713293in}{3.706266in}}%
\pgfpathlineto{\pgfqpoint{3.704446in}{3.669435in}}%
\pgfpathlineto{\pgfqpoint{3.670847in}{3.528155in}}%
\pgfpathlineto{\pgfqpoint{3.635860in}{3.655060in}}%
\pgfpathclose%
\pgfusepath{fill}%
\end{pgfscope}%
\begin{pgfscope}%
\pgfpathrectangle{\pgfqpoint{1.020000in}{0.880000in}}{\pgfqpoint{6.160000in}{6.160000in}}%
\pgfusepath{clip}%
\pgfsetbuttcap%
\pgfsetroundjoin%
\definecolor{currentfill}{rgb}{0.597777,0.727330,0.999777}%
\pgfsetfillcolor{currentfill}%
\pgfsetlinewidth{0.000000pt}%
\definecolor{currentstroke}{rgb}{0.000000,0.000000,0.000000}%
\pgfsetstrokecolor{currentstroke}%
\pgfsetdash{}{0pt}%
\pgfpathmoveto{\pgfqpoint{4.483656in}{3.382570in}}%
\pgfpathlineto{\pgfqpoint{4.493156in}{3.258060in}}%
\pgfpathlineto{\pgfqpoint{4.502816in}{3.184801in}}%
\pgfpathlineto{\pgfqpoint{4.536725in}{3.121713in}}%
\pgfpathlineto{\pgfqpoint{4.571117in}{3.199486in}}%
\pgfpathlineto{\pgfqpoint{4.561637in}{3.325693in}}%
\pgfpathlineto{\pgfqpoint{4.551409in}{3.250042in}}%
\pgfpathlineto{\pgfqpoint{4.517272in}{3.228539in}}%
\pgfpathlineto{\pgfqpoint{4.483656in}{3.382570in}}%
\pgfpathclose%
\pgfusepath{fill}%
\end{pgfscope}%
\begin{pgfscope}%
\pgfpathrectangle{\pgfqpoint{1.020000in}{0.880000in}}{\pgfqpoint{6.160000in}{6.160000in}}%
\pgfusepath{clip}%
\pgfsetbuttcap%
\pgfsetroundjoin%
\definecolor{currentfill}{rgb}{0.708720,0.805721,0.981117}%
\pgfsetfillcolor{currentfill}%
\pgfsetlinewidth{0.000000pt}%
\definecolor{currentstroke}{rgb}{0.000000,0.000000,0.000000}%
\pgfsetstrokecolor{currentstroke}%
\pgfsetdash{}{0pt}%
\pgfpathmoveto{\pgfqpoint{3.138239in}{3.447742in}}%
\pgfpathlineto{\pgfqpoint{3.147191in}{3.395163in}}%
\pgfpathlineto{\pgfqpoint{3.154051in}{3.546950in}}%
\pgfpathlineto{\pgfqpoint{3.189353in}{3.475564in}}%
\pgfpathlineto{\pgfqpoint{3.224153in}{3.449223in}}%
\pgfpathlineto{\pgfqpoint{3.216283in}{3.384659in}}%
\pgfpathlineto{\pgfqpoint{3.206204in}{3.549402in}}%
\pgfpathlineto{\pgfqpoint{3.173117in}{3.409254in}}%
\pgfpathlineto{\pgfqpoint{3.138239in}{3.447742in}}%
\pgfpathclose%
\pgfusepath{fill}%
\end{pgfscope}%
\begin{pgfscope}%
\pgfpathrectangle{\pgfqpoint{1.020000in}{0.880000in}}{\pgfqpoint{6.160000in}{6.160000in}}%
\pgfusepath{clip}%
\pgfsetbuttcap%
\pgfsetroundjoin%
\definecolor{currentfill}{rgb}{0.309060,0.413498,0.850128}%
\pgfsetfillcolor{currentfill}%
\pgfsetlinewidth{0.000000pt}%
\definecolor{currentstroke}{rgb}{0.000000,0.000000,0.000000}%
\pgfsetstrokecolor{currentstroke}%
\pgfsetdash{}{0pt}%
\pgfpathmoveto{\pgfqpoint{5.194787in}{2.786620in}}%
\pgfpathlineto{\pgfqpoint{5.204556in}{2.699698in}}%
\pgfpathlineto{\pgfqpoint{5.213834in}{2.565087in}}%
\pgfpathlineto{\pgfqpoint{5.250657in}{2.838452in}}%
\pgfpathlineto{\pgfqpoint{5.283146in}{2.705717in}}%
\pgfpathlineto{\pgfqpoint{5.272529in}{2.717418in}}%
\pgfpathlineto{\pgfqpoint{5.260702in}{2.615855in}}%
\pgfpathlineto{\pgfqpoint{5.227891in}{2.711419in}}%
\pgfpathlineto{\pgfqpoint{5.194787in}{2.786620in}}%
\pgfpathclose%
\pgfusepath{fill}%
\end{pgfscope}%
\begin{pgfscope}%
\pgfpathrectangle{\pgfqpoint{1.020000in}{0.880000in}}{\pgfqpoint{6.160000in}{6.160000in}}%
\pgfusepath{clip}%
\pgfsetbuttcap%
\pgfsetroundjoin%
\definecolor{currentfill}{rgb}{0.309060,0.413498,0.850128}%
\pgfsetfillcolor{currentfill}%
\pgfsetlinewidth{0.000000pt}%
\definecolor{currentstroke}{rgb}{0.000000,0.000000,0.000000}%
\pgfsetstrokecolor{currentstroke}%
\pgfsetdash{}{0pt}%
\pgfpathmoveto{\pgfqpoint{5.711082in}{2.718939in}}%
\pgfpathlineto{\pgfqpoint{5.723588in}{2.801603in}}%
\pgfpathlineto{\pgfqpoint{5.732426in}{2.646594in}}%
\pgfpathlineto{\pgfqpoint{5.766673in}{2.677067in}}%
\pgfpathlineto{\pgfqpoint{5.797864in}{2.520093in}}%
\pgfpathlineto{\pgfqpoint{5.789473in}{2.699449in}}%
\pgfpathlineto{\pgfqpoint{5.780332in}{2.833142in}}%
\pgfpathlineto{\pgfqpoint{5.744800in}{2.718979in}}%
\pgfpathlineto{\pgfqpoint{5.711082in}{2.718939in}}%
\pgfpathclose%
\pgfusepath{fill}%
\end{pgfscope}%
\begin{pgfscope}%
\pgfpathrectangle{\pgfqpoint{1.020000in}{0.880000in}}{\pgfqpoint{6.160000in}{6.160000in}}%
\pgfusepath{clip}%
\pgfsetbuttcap%
\pgfsetroundjoin%
\definecolor{currentfill}{rgb}{0.252663,0.332837,0.783665}%
\pgfsetfillcolor{currentfill}%
\pgfsetlinewidth{0.000000pt}%
\definecolor{currentstroke}{rgb}{0.000000,0.000000,0.000000}%
\pgfsetstrokecolor{currentstroke}%
\pgfsetdash{}{0pt}%
\pgfpathmoveto{\pgfqpoint{6.092802in}{2.671376in}}%
\pgfpathlineto{\pgfqpoint{6.101356in}{2.511252in}}%
\pgfpathlineto{\pgfqpoint{6.110737in}{2.394168in}}%
\pgfpathlineto{\pgfqpoint{6.145972in}{2.476597in}}%
\pgfpathlineto{\pgfqpoint{6.181582in}{2.575871in}}%
\pgfpathlineto{\pgfqpoint{6.174279in}{2.797102in}}%
\pgfpathlineto{\pgfqpoint{6.158078in}{2.572921in}}%
\pgfpathlineto{\pgfqpoint{6.125116in}{2.604227in}}%
\pgfpathlineto{\pgfqpoint{6.092802in}{2.671376in}}%
\pgfpathclose%
\pgfusepath{fill}%
\end{pgfscope}%
\begin{pgfscope}%
\pgfpathrectangle{\pgfqpoint{1.020000in}{0.880000in}}{\pgfqpoint{6.160000in}{6.160000in}}%
\pgfusepath{clip}%
\pgfsetbuttcap%
\pgfsetroundjoin%
\definecolor{currentfill}{rgb}{0.661968,0.775491,0.993937}%
\pgfsetfillcolor{currentfill}%
\pgfsetlinewidth{0.000000pt}%
\definecolor{currentstroke}{rgb}{0.000000,0.000000,0.000000}%
\pgfsetstrokecolor{currentstroke}%
\pgfsetdash{}{0pt}%
\pgfpathmoveto{\pgfqpoint{4.327567in}{3.437178in}}%
\pgfpathlineto{\pgfqpoint{4.337136in}{3.343890in}}%
\pgfpathlineto{\pgfqpoint{4.346716in}{3.256213in}}%
\pgfpathlineto{\pgfqpoint{4.381142in}{3.384749in}}%
\pgfpathlineto{\pgfqpoint{4.415018in}{3.247858in}}%
\pgfpathlineto{\pgfqpoint{4.405631in}{3.425881in}}%
\pgfpathlineto{\pgfqpoint{4.395823in}{3.412385in}}%
\pgfpathlineto{\pgfqpoint{4.361639in}{3.385042in}}%
\pgfpathlineto{\pgfqpoint{4.327567in}{3.437178in}}%
\pgfpathclose%
\pgfusepath{fill}%
\end{pgfscope}%
\begin{pgfscope}%
\pgfpathrectangle{\pgfqpoint{1.020000in}{0.880000in}}{\pgfqpoint{6.160000in}{6.160000in}}%
\pgfusepath{clip}%
\pgfsetbuttcap%
\pgfsetroundjoin%
\definecolor{currentfill}{rgb}{0.613933,0.739923,0.999142}%
\pgfsetfillcolor{currentfill}%
\pgfsetlinewidth{0.000000pt}%
\definecolor{currentstroke}{rgb}{0.000000,0.000000,0.000000}%
\pgfsetstrokecolor{currentstroke}%
\pgfsetdash{}{0pt}%
\pgfpathmoveto{\pgfqpoint{2.643525in}{3.052434in}}%
\pgfpathlineto{\pgfqpoint{2.646615in}{3.361761in}}%
\pgfpathlineto{\pgfqpoint{2.655295in}{3.309968in}}%
\pgfpathlineto{\pgfqpoint{2.693255in}{3.094341in}}%
\pgfpathlineto{\pgfqpoint{2.722651in}{3.453420in}}%
\pgfpathlineto{\pgfqpoint{2.717713in}{3.247206in}}%
\pgfpathlineto{\pgfqpoint{2.709896in}{3.239405in}}%
\pgfpathlineto{\pgfqpoint{2.672348in}{3.433236in}}%
\pgfpathlineto{\pgfqpoint{2.643525in}{3.052434in}}%
\pgfpathclose%
\pgfusepath{fill}%
\end{pgfscope}%
\begin{pgfscope}%
\pgfpathrectangle{\pgfqpoint{1.020000in}{0.880000in}}{\pgfqpoint{6.160000in}{6.160000in}}%
\pgfusepath{clip}%
\pgfsetbuttcap%
\pgfsetroundjoin%
\definecolor{currentfill}{rgb}{0.597777,0.727330,0.999777}%
\pgfsetfillcolor{currentfill}%
\pgfsetlinewidth{0.000000pt}%
\definecolor{currentstroke}{rgb}{0.000000,0.000000,0.000000}%
\pgfsetstrokecolor{currentstroke}%
\pgfsetdash{}{0pt}%
\pgfpathmoveto{\pgfqpoint{2.865911in}{3.119422in}}%
\pgfpathlineto{\pgfqpoint{2.873289in}{3.174300in}}%
\pgfpathlineto{\pgfqpoint{2.880816in}{3.219682in}}%
\pgfpathlineto{\pgfqpoint{2.914895in}{3.264573in}}%
\pgfpathlineto{\pgfqpoint{2.950558in}{3.182002in}}%
\pgfpathlineto{\pgfqpoint{2.941197in}{3.275996in}}%
\pgfpathlineto{\pgfqpoint{2.933192in}{3.261934in}}%
\pgfpathlineto{\pgfqpoint{2.896541in}{3.423542in}}%
\pgfpathlineto{\pgfqpoint{2.865911in}{3.119422in}}%
\pgfpathclose%
\pgfusepath{fill}%
\end{pgfscope}%
\begin{pgfscope}%
\pgfpathrectangle{\pgfqpoint{1.020000in}{0.880000in}}{\pgfqpoint{6.160000in}{6.160000in}}%
\pgfusepath{clip}%
\pgfsetbuttcap%
\pgfsetroundjoin%
\definecolor{currentfill}{rgb}{0.304174,0.406945,0.845263}%
\pgfsetfillcolor{currentfill}%
\pgfsetlinewidth{0.000000pt}%
\definecolor{currentstroke}{rgb}{0.000000,0.000000,0.000000}%
\pgfsetstrokecolor{currentstroke}%
\pgfsetdash{}{0pt}%
\pgfpathmoveto{\pgfqpoint{5.419301in}{2.762330in}}%
\pgfpathlineto{\pgfqpoint{5.429864in}{2.734558in}}%
\pgfpathlineto{\pgfqpoint{5.438291in}{2.536722in}}%
\pgfpathlineto{\pgfqpoint{5.474384in}{2.711491in}}%
\pgfpathlineto{\pgfqpoint{5.508627in}{2.740176in}}%
\pgfpathlineto{\pgfqpoint{5.496540in}{2.658052in}}%
\pgfpathlineto{\pgfqpoint{5.485570in}{2.657871in}}%
\pgfpathlineto{\pgfqpoint{5.452883in}{2.742904in}}%
\pgfpathlineto{\pgfqpoint{5.419301in}{2.762330in}}%
\pgfpathclose%
\pgfusepath{fill}%
\end{pgfscope}%
\begin{pgfscope}%
\pgfpathrectangle{\pgfqpoint{1.020000in}{0.880000in}}{\pgfqpoint{6.160000in}{6.160000in}}%
\pgfusepath{clip}%
\pgfsetbuttcap%
\pgfsetroundjoin%
\definecolor{currentfill}{rgb}{0.763363,0.835092,0.955658}%
\pgfsetfillcolor{currentfill}%
\pgfsetlinewidth{0.000000pt}%
\definecolor{currentstroke}{rgb}{0.000000,0.000000,0.000000}%
\pgfsetstrokecolor{currentstroke}%
\pgfsetdash{}{0pt}%
\pgfpathmoveto{\pgfqpoint{4.015870in}{3.493338in}}%
\pgfpathlineto{\pgfqpoint{4.025027in}{3.621079in}}%
\pgfpathlineto{\pgfqpoint{4.034387in}{3.642754in}}%
\pgfpathlineto{\pgfqpoint{4.068765in}{3.602649in}}%
\pgfpathlineto{\pgfqpoint{4.103054in}{3.613739in}}%
\pgfpathlineto{\pgfqpoint{4.093711in}{3.482691in}}%
\pgfpathlineto{\pgfqpoint{4.084325in}{3.441710in}}%
\pgfpathlineto{\pgfqpoint{4.049944in}{3.596791in}}%
\pgfpathlineto{\pgfqpoint{4.015870in}{3.493338in}}%
\pgfpathclose%
\pgfusepath{fill}%
\end{pgfscope}%
\begin{pgfscope}%
\pgfpathrectangle{\pgfqpoint{1.020000in}{0.880000in}}{\pgfqpoint{6.160000in}{6.160000in}}%
\pgfusepath{clip}%
\pgfsetbuttcap%
\pgfsetroundjoin%
\definecolor{currentfill}{rgb}{0.538004,0.674902,0.991722}%
\pgfsetfillcolor{currentfill}%
\pgfsetlinewidth{0.000000pt}%
\definecolor{currentstroke}{rgb}{0.000000,0.000000,0.000000}%
\pgfsetstrokecolor{currentstroke}%
\pgfsetdash{}{0pt}%
\pgfpathmoveto{\pgfqpoint{4.571117in}{3.199486in}}%
\pgfpathlineto{\pgfqpoint{4.580396in}{3.023782in}}%
\pgfpathlineto{\pgfqpoint{4.590964in}{3.166427in}}%
\pgfpathlineto{\pgfqpoint{4.625054in}{3.152346in}}%
\pgfpathlineto{\pgfqpoint{4.659119in}{3.139382in}}%
\pgfpathlineto{\pgfqpoint{4.648591in}{3.039985in}}%
\pgfpathlineto{\pgfqpoint{4.638965in}{3.128792in}}%
\pgfpathlineto{\pgfqpoint{4.605153in}{3.185357in}}%
\pgfpathlineto{\pgfqpoint{4.571117in}{3.199486in}}%
\pgfpathclose%
\pgfusepath{fill}%
\end{pgfscope}%
\begin{pgfscope}%
\pgfpathrectangle{\pgfqpoint{1.020000in}{0.880000in}}{\pgfqpoint{6.160000in}{6.160000in}}%
\pgfusepath{clip}%
\pgfsetbuttcap%
\pgfsetroundjoin%
\definecolor{currentfill}{rgb}{0.383662,0.510183,0.917831}%
\pgfsetfillcolor{currentfill}%
\pgfsetlinewidth{0.000000pt}%
\definecolor{currentstroke}{rgb}{0.000000,0.000000,0.000000}%
\pgfsetstrokecolor{currentstroke}%
\pgfsetdash{}{0pt}%
\pgfpathmoveto{\pgfqpoint{5.038399in}{2.831348in}}%
\pgfpathlineto{\pgfqpoint{5.048400in}{2.775647in}}%
\pgfpathlineto{\pgfqpoint{5.060292in}{2.933291in}}%
\pgfpathlineto{\pgfqpoint{5.093730in}{2.868737in}}%
\pgfpathlineto{\pgfqpoint{5.128315in}{2.931878in}}%
\pgfpathlineto{\pgfqpoint{5.116772in}{2.830212in}}%
\pgfpathlineto{\pgfqpoint{5.105811in}{2.786347in}}%
\pgfpathlineto{\pgfqpoint{5.072078in}{2.803825in}}%
\pgfpathlineto{\pgfqpoint{5.038399in}{2.831348in}}%
\pgfpathclose%
\pgfusepath{fill}%
\end{pgfscope}%
\begin{pgfscope}%
\pgfpathrectangle{\pgfqpoint{1.020000in}{0.880000in}}{\pgfqpoint{6.160000in}{6.160000in}}%
\pgfusepath{clip}%
\pgfsetbuttcap%
\pgfsetroundjoin%
\definecolor{currentfill}{rgb}{0.581486,0.713451,0.998314}%
\pgfsetfillcolor{currentfill}%
\pgfsetlinewidth{0.000000pt}%
\definecolor{currentstroke}{rgb}{0.000000,0.000000,0.000000}%
\pgfsetstrokecolor{currentstroke}%
\pgfsetdash{}{0pt}%
\pgfpathmoveto{\pgfqpoint{2.794905in}{3.252935in}}%
\pgfpathlineto{\pgfqpoint{2.800920in}{3.399216in}}%
\pgfpathlineto{\pgfqpoint{2.812277in}{3.159949in}}%
\pgfpathlineto{\pgfqpoint{2.846883in}{3.164753in}}%
\pgfpathlineto{\pgfqpoint{2.880816in}{3.219682in}}%
\pgfpathlineto{\pgfqpoint{2.873289in}{3.174300in}}%
\pgfpathlineto{\pgfqpoint{2.865911in}{3.119422in}}%
\pgfpathlineto{\pgfqpoint{2.830346in}{3.193241in}}%
\pgfpathlineto{\pgfqpoint{2.794905in}{3.252935in}}%
\pgfpathclose%
\pgfusepath{fill}%
\end{pgfscope}%
\begin{pgfscope}%
\pgfpathrectangle{\pgfqpoint{1.020000in}{0.880000in}}{\pgfqpoint{6.160000in}{6.160000in}}%
\pgfusepath{clip}%
\pgfsetbuttcap%
\pgfsetroundjoin%
\definecolor{currentfill}{rgb}{0.289996,0.386836,0.828926}%
\pgfsetfillcolor{currentfill}%
\pgfsetlinewidth{0.000000pt}%
\definecolor{currentstroke}{rgb}{0.000000,0.000000,0.000000}%
\pgfsetstrokecolor{currentstroke}%
\pgfsetdash{}{0pt}%
\pgfpathmoveto{\pgfqpoint{5.868359in}{2.706502in}}%
\pgfpathlineto{\pgfqpoint{5.876868in}{2.536928in}}%
\pgfpathlineto{\pgfqpoint{5.889155in}{2.587731in}}%
\pgfpathlineto{\pgfqpoint{5.924177in}{2.661694in}}%
\pgfpathlineto{\pgfqpoint{5.955836in}{2.545960in}}%
\pgfpathlineto{\pgfqpoint{5.946334in}{2.657278in}}%
\pgfpathlineto{\pgfqpoint{5.937521in}{2.807703in}}%
\pgfpathlineto{\pgfqpoint{5.902584in}{2.736931in}}%
\pgfpathlineto{\pgfqpoint{5.868359in}{2.706502in}}%
\pgfpathclose%
\pgfusepath{fill}%
\end{pgfscope}%
\begin{pgfscope}%
\pgfpathrectangle{\pgfqpoint{1.020000in}{0.880000in}}{\pgfqpoint{6.160000in}{6.160000in}}%
\pgfusepath{clip}%
\pgfsetbuttcap%
\pgfsetroundjoin%
\definecolor{currentfill}{rgb}{0.597777,0.727330,0.999777}%
\pgfsetfillcolor{currentfill}%
\pgfsetlinewidth{0.000000pt}%
\definecolor{currentstroke}{rgb}{0.000000,0.000000,0.000000}%
\pgfsetstrokecolor{currentstroke}%
\pgfsetdash{}{0pt}%
\pgfpathmoveto{\pgfqpoint{4.415018in}{3.247858in}}%
\pgfpathlineto{\pgfqpoint{4.424404in}{3.074731in}}%
\pgfpathlineto{\pgfqpoint{4.434982in}{3.380400in}}%
\pgfpathlineto{\pgfqpoint{4.468586in}{3.157828in}}%
\pgfpathlineto{\pgfqpoint{4.502816in}{3.184801in}}%
\pgfpathlineto{\pgfqpoint{4.493156in}{3.258060in}}%
\pgfpathlineto{\pgfqpoint{4.483656in}{3.382570in}}%
\pgfpathlineto{\pgfqpoint{4.449213in}{3.273322in}}%
\pgfpathlineto{\pgfqpoint{4.415018in}{3.247858in}}%
\pgfpathclose%
\pgfusepath{fill}%
\end{pgfscope}%
\begin{pgfscope}%
\pgfpathrectangle{\pgfqpoint{1.020000in}{0.880000in}}{\pgfqpoint{6.160000in}{6.160000in}}%
\pgfusepath{clip}%
\pgfsetbuttcap%
\pgfsetroundjoin%
\definecolor{currentfill}{rgb}{0.603162,0.731527,0.999565}%
\pgfsetfillcolor{currentfill}%
\pgfsetlinewidth{0.000000pt}%
\definecolor{currentstroke}{rgb}{0.000000,0.000000,0.000000}%
\pgfsetstrokecolor{currentstroke}%
\pgfsetdash{}{0pt}%
\pgfpathmoveto{\pgfqpoint{2.570259in}{3.303360in}}%
\pgfpathlineto{\pgfqpoint{2.580539in}{3.148733in}}%
\pgfpathlineto{\pgfqpoint{2.584505in}{3.388401in}}%
\pgfpathlineto{\pgfqpoint{2.620938in}{3.284400in}}%
\pgfpathlineto{\pgfqpoint{2.655295in}{3.309968in}}%
\pgfpathlineto{\pgfqpoint{2.646615in}{3.361761in}}%
\pgfpathlineto{\pgfqpoint{2.643525in}{3.052434in}}%
\pgfpathlineto{\pgfqpoint{2.606860in}{3.183358in}}%
\pgfpathlineto{\pgfqpoint{2.570259in}{3.303360in}}%
\pgfpathclose%
\pgfusepath{fill}%
\end{pgfscope}%
\begin{pgfscope}%
\pgfpathrectangle{\pgfqpoint{1.020000in}{0.880000in}}{\pgfqpoint{6.160000in}{6.160000in}}%
\pgfusepath{clip}%
\pgfsetbuttcap%
\pgfsetroundjoin%
\definecolor{currentfill}{rgb}{0.826784,0.858205,0.906953}%
\pgfsetfillcolor{currentfill}%
\pgfsetlinewidth{0.000000pt}%
\definecolor{currentstroke}{rgb}{0.000000,0.000000,0.000000}%
\pgfsetstrokecolor{currentstroke}%
\pgfsetdash{}{0pt}%
\pgfpathmoveto{\pgfqpoint{3.859741in}{3.706788in}}%
\pgfpathlineto{\pgfqpoint{3.868750in}{3.767827in}}%
\pgfpathlineto{\pgfqpoint{3.878500in}{3.595949in}}%
\pgfpathlineto{\pgfqpoint{3.912299in}{3.792948in}}%
\pgfpathlineto{\pgfqpoint{3.946755in}{3.759490in}}%
\pgfpathlineto{\pgfqpoint{3.937497in}{3.744333in}}%
\pgfpathlineto{\pgfqpoint{3.928922in}{3.469601in}}%
\pgfpathlineto{\pgfqpoint{3.893940in}{3.741607in}}%
\pgfpathlineto{\pgfqpoint{3.859741in}{3.706788in}}%
\pgfpathclose%
\pgfusepath{fill}%
\end{pgfscope}%
\begin{pgfscope}%
\pgfpathrectangle{\pgfqpoint{1.020000in}{0.880000in}}{\pgfqpoint{6.160000in}{6.160000in}}%
\pgfusepath{clip}%
\pgfsetbuttcap%
\pgfsetroundjoin%
\definecolor{currentfill}{rgb}{0.266381,0.353304,0.801637}%
\pgfsetfillcolor{currentfill}%
\pgfsetlinewidth{0.000000pt}%
\definecolor{currentstroke}{rgb}{0.000000,0.000000,0.000000}%
\pgfsetstrokecolor{currentstroke}%
\pgfsetdash{}{0pt}%
\pgfpathmoveto{\pgfqpoint{5.797864in}{2.520093in}}%
\pgfpathlineto{\pgfqpoint{5.809119in}{2.516209in}}%
\pgfpathlineto{\pgfqpoint{5.821584in}{2.584097in}}%
\pgfpathlineto{\pgfqpoint{5.857504in}{2.711942in}}%
\pgfpathlineto{\pgfqpoint{5.889155in}{2.587731in}}%
\pgfpathlineto{\pgfqpoint{5.876868in}{2.536928in}}%
\pgfpathlineto{\pgfqpoint{5.868359in}{2.706502in}}%
\pgfpathlineto{\pgfqpoint{5.834213in}{2.680779in}}%
\pgfpathlineto{\pgfqpoint{5.797864in}{2.520093in}}%
\pgfpathclose%
\pgfusepath{fill}%
\end{pgfscope}%
\begin{pgfscope}%
\pgfpathrectangle{\pgfqpoint{1.020000in}{0.880000in}}{\pgfqpoint{6.160000in}{6.160000in}}%
\pgfusepath{clip}%
\pgfsetbuttcap%
\pgfsetroundjoin%
\definecolor{currentfill}{rgb}{0.313946,0.420052,0.854993}%
\pgfsetfillcolor{currentfill}%
\pgfsetlinewidth{0.000000pt}%
\definecolor{currentstroke}{rgb}{0.000000,0.000000,0.000000}%
\pgfsetstrokecolor{currentstroke}%
\pgfsetdash{}{0pt}%
\pgfpathmoveto{\pgfqpoint{5.643609in}{2.720510in}}%
\pgfpathlineto{\pgfqpoint{5.652647in}{2.577529in}}%
\pgfpathlineto{\pgfqpoint{5.665324in}{2.678707in}}%
\pgfpathlineto{\pgfqpoint{5.699785in}{2.721223in}}%
\pgfpathlineto{\pgfqpoint{5.732426in}{2.646594in}}%
\pgfpathlineto{\pgfqpoint{5.723588in}{2.801603in}}%
\pgfpathlineto{\pgfqpoint{5.711082in}{2.718939in}}%
\pgfpathlineto{\pgfqpoint{5.678167in}{2.773586in}}%
\pgfpathlineto{\pgfqpoint{5.643609in}{2.720510in}}%
\pgfpathclose%
\pgfusepath{fill}%
\end{pgfscope}%
\begin{pgfscope}%
\pgfpathrectangle{\pgfqpoint{1.020000in}{0.880000in}}{\pgfqpoint{6.160000in}{6.160000in}}%
\pgfusepath{clip}%
\pgfsetbuttcap%
\pgfsetroundjoin%
\definecolor{currentfill}{rgb}{0.473070,0.611077,0.970634}%
\pgfsetfillcolor{currentfill}%
\pgfsetlinewidth{0.000000pt}%
\definecolor{currentstroke}{rgb}{0.000000,0.000000,0.000000}%
\pgfsetstrokecolor{currentstroke}%
\pgfsetdash{}{0pt}%
\pgfpathmoveto{\pgfqpoint{4.659119in}{3.139382in}}%
\pgfpathlineto{\pgfqpoint{4.668841in}{3.064625in}}%
\pgfpathlineto{\pgfqpoint{4.678979in}{3.071976in}}%
\pgfpathlineto{\pgfqpoint{4.713485in}{3.139498in}}%
\pgfpathlineto{\pgfqpoint{4.745663in}{2.791137in}}%
\pgfpathlineto{\pgfqpoint{4.736630in}{2.990610in}}%
\pgfpathlineto{\pgfqpoint{4.726082in}{2.915714in}}%
\pgfpathlineto{\pgfqpoint{4.692194in}{2.938559in}}%
\pgfpathlineto{\pgfqpoint{4.659119in}{3.139382in}}%
\pgfpathclose%
\pgfusepath{fill}%
\end{pgfscope}%
\begin{pgfscope}%
\pgfpathrectangle{\pgfqpoint{1.020000in}{0.880000in}}{\pgfqpoint{6.160000in}{6.160000in}}%
\pgfusepath{clip}%
\pgfsetbuttcap%
\pgfsetroundjoin%
\definecolor{currentfill}{rgb}{0.252663,0.332837,0.783665}%
\pgfsetfillcolor{currentfill}%
\pgfsetlinewidth{0.000000pt}%
\definecolor{currentstroke}{rgb}{0.000000,0.000000,0.000000}%
\pgfsetstrokecolor{currentstroke}%
\pgfsetdash{}{0pt}%
\pgfpathmoveto{\pgfqpoint{6.023814in}{2.581536in}}%
\pgfpathlineto{\pgfqpoint{6.035377in}{2.580806in}}%
\pgfpathlineto{\pgfqpoint{6.049897in}{2.736005in}}%
\pgfpathlineto{\pgfqpoint{6.080395in}{2.565587in}}%
\pgfpathlineto{\pgfqpoint{6.110737in}{2.394168in}}%
\pgfpathlineto{\pgfqpoint{6.101356in}{2.511252in}}%
\pgfpathlineto{\pgfqpoint{6.092802in}{2.671376in}}%
\pgfpathlineto{\pgfqpoint{6.057159in}{2.565956in}}%
\pgfpathlineto{\pgfqpoint{6.023814in}{2.581536in}}%
\pgfpathclose%
\pgfusepath{fill}%
\end{pgfscope}%
\begin{pgfscope}%
\pgfpathrectangle{\pgfqpoint{1.020000in}{0.880000in}}{\pgfqpoint{6.160000in}{6.160000in}}%
\pgfusepath{clip}%
\pgfsetbuttcap%
\pgfsetroundjoin%
\definecolor{currentfill}{rgb}{0.294718,0.393542,0.834384}%
\pgfsetfillcolor{currentfill}%
\pgfsetlinewidth{0.000000pt}%
\definecolor{currentstroke}{rgb}{0.000000,0.000000,0.000000}%
\pgfsetstrokecolor{currentstroke}%
\pgfsetdash{}{0pt}%
\pgfpathmoveto{\pgfqpoint{5.575617in}{2.691243in}}%
\pgfpathlineto{\pgfqpoint{5.586350in}{2.666235in}}%
\pgfpathlineto{\pgfqpoint{5.595009in}{2.494309in}}%
\pgfpathlineto{\pgfqpoint{5.631608in}{2.686796in}}%
\pgfpathlineto{\pgfqpoint{5.665324in}{2.678707in}}%
\pgfpathlineto{\pgfqpoint{5.652647in}{2.577529in}}%
\pgfpathlineto{\pgfqpoint{5.643609in}{2.720510in}}%
\pgfpathlineto{\pgfqpoint{5.610018in}{2.733696in}}%
\pgfpathlineto{\pgfqpoint{5.575617in}{2.691243in}}%
\pgfpathclose%
\pgfusepath{fill}%
\end{pgfscope}%
\begin{pgfscope}%
\pgfpathrectangle{\pgfqpoint{1.020000in}{0.880000in}}{\pgfqpoint{6.160000in}{6.160000in}}%
\pgfusepath{clip}%
\pgfsetbuttcap%
\pgfsetroundjoin%
\definecolor{currentfill}{rgb}{0.748682,0.827679,0.963334}%
\pgfsetfillcolor{currentfill}%
\pgfsetlinewidth{0.000000pt}%
\definecolor{currentstroke}{rgb}{0.000000,0.000000,0.000000}%
\pgfsetstrokecolor{currentstroke}%
\pgfsetdash{}{0pt}%
\pgfpathmoveto{\pgfqpoint{4.171600in}{3.572925in}}%
\pgfpathlineto{\pgfqpoint{4.181146in}{3.544557in}}%
\pgfpathlineto{\pgfqpoint{4.190710in}{3.553749in}}%
\pgfpathlineto{\pgfqpoint{4.224966in}{3.477539in}}%
\pgfpathlineto{\pgfqpoint{4.259299in}{3.599020in}}%
\pgfpathlineto{\pgfqpoint{4.249626in}{3.545202in}}%
\pgfpathlineto{\pgfqpoint{4.239965in}{3.458446in}}%
\pgfpathlineto{\pgfqpoint{4.205797in}{3.448634in}}%
\pgfpathlineto{\pgfqpoint{4.171600in}{3.572925in}}%
\pgfpathclose%
\pgfusepath{fill}%
\end{pgfscope}%
\begin{pgfscope}%
\pgfpathrectangle{\pgfqpoint{1.020000in}{0.880000in}}{\pgfqpoint{6.160000in}{6.160000in}}%
\pgfusepath{clip}%
\pgfsetbuttcap%
\pgfsetroundjoin%
\definecolor{currentfill}{rgb}{0.383662,0.510183,0.917831}%
\pgfsetfillcolor{currentfill}%
\pgfsetlinewidth{0.000000pt}%
\definecolor{currentstroke}{rgb}{0.000000,0.000000,0.000000}%
\pgfsetstrokecolor{currentstroke}%
\pgfsetdash{}{0pt}%
\pgfpathmoveto{\pgfqpoint{4.970379in}{2.817793in}}%
\pgfpathlineto{\pgfqpoint{4.980543in}{2.786324in}}%
\pgfpathlineto{\pgfqpoint{4.990473in}{2.724090in}}%
\pgfpathlineto{\pgfqpoint{5.026716in}{2.989062in}}%
\pgfpathlineto{\pgfqpoint{5.060292in}{2.933291in}}%
\pgfpathlineto{\pgfqpoint{5.048400in}{2.775647in}}%
\pgfpathlineto{\pgfqpoint{5.038399in}{2.831348in}}%
\pgfpathlineto{\pgfqpoint{5.004443in}{2.830090in}}%
\pgfpathlineto{\pgfqpoint{4.970379in}{2.817793in}}%
\pgfpathclose%
\pgfusepath{fill}%
\end{pgfscope}%
\begin{pgfscope}%
\pgfpathrectangle{\pgfqpoint{1.020000in}{0.880000in}}{\pgfqpoint{6.160000in}{6.160000in}}%
\pgfusepath{clip}%
\pgfsetbuttcap%
\pgfsetroundjoin%
\definecolor{currentfill}{rgb}{0.718985,0.811993,0.977656}%
\pgfsetfillcolor{currentfill}%
\pgfsetlinewidth{0.000000pt}%
\definecolor{currentstroke}{rgb}{0.000000,0.000000,0.000000}%
\pgfsetstrokecolor{currentstroke}%
\pgfsetdash{}{0pt}%
\pgfpathmoveto{\pgfqpoint{3.294045in}{3.350616in}}%
\pgfpathlineto{\pgfqpoint{3.301891in}{3.433326in}}%
\pgfpathlineto{\pgfqpoint{3.311782in}{3.285511in}}%
\pgfpathlineto{\pgfqpoint{3.344815in}{3.459230in}}%
\pgfpathlineto{\pgfqpoint{3.378191in}{3.602511in}}%
\pgfpathlineto{\pgfqpoint{3.370577in}{3.470691in}}%
\pgfpathlineto{\pgfqpoint{3.361834in}{3.481381in}}%
\pgfpathlineto{\pgfqpoint{3.326314in}{3.604604in}}%
\pgfpathlineto{\pgfqpoint{3.294045in}{3.350616in}}%
\pgfpathclose%
\pgfusepath{fill}%
\end{pgfscope}%
\begin{pgfscope}%
\pgfpathrectangle{\pgfqpoint{1.020000in}{0.880000in}}{\pgfqpoint{6.160000in}{6.160000in}}%
\pgfusepath{clip}%
\pgfsetbuttcap%
\pgfsetroundjoin%
\definecolor{currentfill}{rgb}{0.831148,0.859513,0.903110}%
\pgfsetfillcolor{currentfill}%
\pgfsetlinewidth{0.000000pt}%
\definecolor{currentstroke}{rgb}{0.000000,0.000000,0.000000}%
\pgfsetstrokecolor{currentstroke}%
\pgfsetdash{}{0pt}%
\pgfpathmoveto{\pgfqpoint{3.566583in}{3.742788in}}%
\pgfpathlineto{\pgfqpoint{3.574861in}{3.843337in}}%
\pgfpathlineto{\pgfqpoint{3.584555in}{3.713175in}}%
\pgfpathlineto{\pgfqpoint{3.618599in}{3.788554in}}%
\pgfpathlineto{\pgfqpoint{3.654237in}{3.568178in}}%
\pgfpathlineto{\pgfqpoint{3.645305in}{3.562790in}}%
\pgfpathlineto{\pgfqpoint{3.635860in}{3.655060in}}%
\pgfpathlineto{\pgfqpoint{3.600853in}{3.766767in}}%
\pgfpathlineto{\pgfqpoint{3.566583in}{3.742788in}}%
\pgfpathclose%
\pgfusepath{fill}%
\end{pgfscope}%
\begin{pgfscope}%
\pgfpathrectangle{\pgfqpoint{1.020000in}{0.880000in}}{\pgfqpoint{6.160000in}{6.160000in}}%
\pgfusepath{clip}%
\pgfsetbuttcap%
\pgfsetroundjoin%
\definecolor{currentfill}{rgb}{0.435815,0.570707,0.951717}%
\pgfsetfillcolor{currentfill}%
\pgfsetlinewidth{0.000000pt}%
\definecolor{currentstroke}{rgb}{0.000000,0.000000,0.000000}%
\pgfsetstrokecolor{currentstroke}%
\pgfsetdash{}{0pt}%
\pgfpathmoveto{\pgfqpoint{4.814666in}{2.949554in}}%
\pgfpathlineto{\pgfqpoint{4.823969in}{2.802963in}}%
\pgfpathlineto{\pgfqpoint{4.834831in}{2.895533in}}%
\pgfpathlineto{\pgfqpoint{4.869593in}{2.995587in}}%
\pgfpathlineto{\pgfqpoint{4.903235in}{2.934282in}}%
\pgfpathlineto{\pgfqpoint{4.893042in}{2.954598in}}%
\pgfpathlineto{\pgfqpoint{4.882692in}{2.949890in}}%
\pgfpathlineto{\pgfqpoint{4.848979in}{2.992947in}}%
\pgfpathlineto{\pgfqpoint{4.814666in}{2.949554in}}%
\pgfpathclose%
\pgfusepath{fill}%
\end{pgfscope}%
\begin{pgfscope}%
\pgfpathrectangle{\pgfqpoint{1.020000in}{0.880000in}}{\pgfqpoint{6.160000in}{6.160000in}}%
\pgfusepath{clip}%
\pgfsetbuttcap%
\pgfsetroundjoin%
\definecolor{currentfill}{rgb}{0.243520,0.319189,0.771672}%
\pgfsetfillcolor{currentfill}%
\pgfsetlinewidth{0.000000pt}%
\definecolor{currentstroke}{rgb}{0.000000,0.000000,0.000000}%
\pgfsetstrokecolor{currentstroke}%
\pgfsetdash{}{0pt}%
\pgfpathmoveto{\pgfqpoint{6.246665in}{2.473093in}}%
\pgfpathlineto{\pgfqpoint{6.262933in}{2.685614in}}%
\pgfpathlineto{\pgfqpoint{6.271416in}{2.525013in}}%
\pgfpathlineto{\pgfqpoint{6.306225in}{2.580033in}}%
\pgfpathlineto{\pgfqpoint{6.293736in}{2.551809in}}%
\pgfpathlineto{\pgfqpoint{6.280494in}{2.486456in}}%
\pgfpathlineto{\pgfqpoint{6.246665in}{2.473093in}}%
\pgfpathclose%
\pgfusepath{fill}%
\end{pgfscope}%
\begin{pgfscope}%
\pgfpathrectangle{\pgfqpoint{1.020000in}{0.880000in}}{\pgfqpoint{6.160000in}{6.160000in}}%
\pgfusepath{clip}%
\pgfsetbuttcap%
\pgfsetroundjoin%
\definecolor{currentfill}{rgb}{0.323718,0.433158,0.864722}%
\pgfsetfillcolor{currentfill}%
\pgfsetlinewidth{0.000000pt}%
\definecolor{currentstroke}{rgb}{0.000000,0.000000,0.000000}%
\pgfsetstrokecolor{currentstroke}%
\pgfsetdash{}{0pt}%
\pgfpathmoveto{\pgfqpoint{5.351872in}{2.787223in}}%
\pgfpathlineto{\pgfqpoint{5.360937in}{2.637190in}}%
\pgfpathlineto{\pgfqpoint{5.375369in}{2.935884in}}%
\pgfpathlineto{\pgfqpoint{5.406499in}{2.703111in}}%
\pgfpathlineto{\pgfqpoint{5.438291in}{2.536722in}}%
\pgfpathlineto{\pgfqpoint{5.429864in}{2.734558in}}%
\pgfpathlineto{\pgfqpoint{5.419301in}{2.762330in}}%
\pgfpathlineto{\pgfqpoint{5.383597in}{2.609203in}}%
\pgfpathlineto{\pgfqpoint{5.351872in}{2.787223in}}%
\pgfpathclose%
\pgfusepath{fill}%
\end{pgfscope}%
\begin{pgfscope}%
\pgfpathrectangle{\pgfqpoint{1.020000in}{0.880000in}}{\pgfqpoint{6.160000in}{6.160000in}}%
\pgfusepath{clip}%
\pgfsetbuttcap%
\pgfsetroundjoin%
\definecolor{currentfill}{rgb}{0.348323,0.465711,0.888346}%
\pgfsetfillcolor{currentfill}%
\pgfsetlinewidth{0.000000pt}%
\definecolor{currentstroke}{rgb}{0.000000,0.000000,0.000000}%
\pgfsetstrokecolor{currentstroke}%
\pgfsetdash{}{0pt}%
\pgfpathmoveto{\pgfqpoint{5.128315in}{2.931878in}}%
\pgfpathlineto{\pgfqpoint{5.138850in}{2.923991in}}%
\pgfpathlineto{\pgfqpoint{5.148542in}{2.827116in}}%
\pgfpathlineto{\pgfqpoint{5.180892in}{2.661094in}}%
\pgfpathlineto{\pgfqpoint{5.213834in}{2.565087in}}%
\pgfpathlineto{\pgfqpoint{5.204556in}{2.699698in}}%
\pgfpathlineto{\pgfqpoint{5.194787in}{2.786620in}}%
\pgfpathlineto{\pgfqpoint{5.160409in}{2.739636in}}%
\pgfpathlineto{\pgfqpoint{5.128315in}{2.931878in}}%
\pgfpathclose%
\pgfusepath{fill}%
\end{pgfscope}%
\begin{pgfscope}%
\pgfpathrectangle{\pgfqpoint{1.020000in}{0.880000in}}{\pgfqpoint{6.160000in}{6.160000in}}%
\pgfusepath{clip}%
\pgfsetbuttcap%
\pgfsetroundjoin%
\definecolor{currentfill}{rgb}{0.831148,0.859513,0.903110}%
\pgfsetfillcolor{currentfill}%
\pgfsetlinewidth{0.000000pt}%
\definecolor{currentstroke}{rgb}{0.000000,0.000000,0.000000}%
\pgfsetstrokecolor{currentstroke}%
\pgfsetdash{}{0pt}%
\pgfpathmoveto{\pgfqpoint{3.791111in}{3.698644in}}%
\pgfpathlineto{\pgfqpoint{3.799991in}{3.764185in}}%
\pgfpathlineto{\pgfqpoint{3.809440in}{3.685853in}}%
\pgfpathlineto{\pgfqpoint{3.843825in}{3.691353in}}%
\pgfpathlineto{\pgfqpoint{3.878500in}{3.595949in}}%
\pgfpathlineto{\pgfqpoint{3.868750in}{3.767827in}}%
\pgfpathlineto{\pgfqpoint{3.859741in}{3.706788in}}%
\pgfpathlineto{\pgfqpoint{3.825545in}{3.671891in}}%
\pgfpathlineto{\pgfqpoint{3.791111in}{3.698644in}}%
\pgfpathclose%
\pgfusepath{fill}%
\end{pgfscope}%
\begin{pgfscope}%
\pgfpathrectangle{\pgfqpoint{1.020000in}{0.880000in}}{\pgfqpoint{6.160000in}{6.160000in}}%
\pgfusepath{clip}%
\pgfsetbuttcap%
\pgfsetroundjoin%
\definecolor{currentfill}{rgb}{0.425199,0.559058,0.946061}%
\pgfsetfillcolor{currentfill}%
\pgfsetlinewidth{0.000000pt}%
\definecolor{currentstroke}{rgb}{0.000000,0.000000,0.000000}%
\pgfsetstrokecolor{currentstroke}%
\pgfsetdash{}{0pt}%
\pgfpathmoveto{\pgfqpoint{4.745663in}{2.791137in}}%
\pgfpathlineto{\pgfqpoint{4.757054in}{3.004184in}}%
\pgfpathlineto{\pgfqpoint{4.766966in}{2.953663in}}%
\pgfpathlineto{\pgfqpoint{4.801085in}{2.951356in}}%
\pgfpathlineto{\pgfqpoint{4.834831in}{2.895533in}}%
\pgfpathlineto{\pgfqpoint{4.823969in}{2.802963in}}%
\pgfpathlineto{\pgfqpoint{4.814666in}{2.949554in}}%
\pgfpathlineto{\pgfqpoint{4.780523in}{2.932786in}}%
\pgfpathlineto{\pgfqpoint{4.745663in}{2.791137in}}%
\pgfpathclose%
\pgfusepath{fill}%
\end{pgfscope}%
\begin{pgfscope}%
\pgfpathrectangle{\pgfqpoint{1.020000in}{0.880000in}}{\pgfqpoint{6.160000in}{6.160000in}}%
\pgfusepath{clip}%
\pgfsetbuttcap%
\pgfsetroundjoin%
\definecolor{currentfill}{rgb}{0.713852,0.808857,0.979386}%
\pgfsetfillcolor{currentfill}%
\pgfsetlinewidth{0.000000pt}%
\definecolor{currentstroke}{rgb}{0.000000,0.000000,0.000000}%
\pgfsetstrokecolor{currentstroke}%
\pgfsetdash{}{0pt}%
\pgfpathmoveto{\pgfqpoint{3.224153in}{3.449223in}}%
\pgfpathlineto{\pgfqpoint{3.231447in}{3.577524in}}%
\pgfpathlineto{\pgfqpoint{3.240575in}{3.515421in}}%
\pgfpathlineto{\pgfqpoint{3.275200in}{3.513884in}}%
\pgfpathlineto{\pgfqpoint{3.311782in}{3.285511in}}%
\pgfpathlineto{\pgfqpoint{3.301891in}{3.433326in}}%
\pgfpathlineto{\pgfqpoint{3.294045in}{3.350616in}}%
\pgfpathlineto{\pgfqpoint{3.258591in}{3.457851in}}%
\pgfpathlineto{\pgfqpoint{3.224153in}{3.449223in}}%
\pgfpathclose%
\pgfusepath{fill}%
\end{pgfscope}%
\begin{pgfscope}%
\pgfpathrectangle{\pgfqpoint{1.020000in}{0.880000in}}{\pgfqpoint{6.160000in}{6.160000in}}%
\pgfusepath{clip}%
\pgfsetbuttcap%
\pgfsetroundjoin%
\definecolor{currentfill}{rgb}{0.280550,0.373423,0.818011}%
\pgfsetfillcolor{currentfill}%
\pgfsetlinewidth{0.000000pt}%
\definecolor{currentstroke}{rgb}{0.000000,0.000000,0.000000}%
\pgfsetstrokecolor{currentstroke}%
\pgfsetdash{}{0pt}%
\pgfpathmoveto{\pgfqpoint{5.732426in}{2.646594in}}%
\pgfpathlineto{\pgfqpoint{5.744574in}{2.702650in}}%
\pgfpathlineto{\pgfqpoint{5.755356in}{2.670157in}}%
\pgfpathlineto{\pgfqpoint{5.788828in}{2.647715in}}%
\pgfpathlineto{\pgfqpoint{5.821584in}{2.584097in}}%
\pgfpathlineto{\pgfqpoint{5.809119in}{2.516209in}}%
\pgfpathlineto{\pgfqpoint{5.797864in}{2.520093in}}%
\pgfpathlineto{\pgfqpoint{5.766673in}{2.677067in}}%
\pgfpathlineto{\pgfqpoint{5.732426in}{2.646594in}}%
\pgfpathclose%
\pgfusepath{fill}%
\end{pgfscope}%
\begin{pgfscope}%
\pgfpathrectangle{\pgfqpoint{1.020000in}{0.880000in}}{\pgfqpoint{6.160000in}{6.160000in}}%
\pgfusepath{clip}%
\pgfsetbuttcap%
\pgfsetroundjoin%
\definecolor{currentfill}{rgb}{0.796064,0.848693,0.933471}%
\pgfsetfillcolor{currentfill}%
\pgfsetlinewidth{0.000000pt}%
\definecolor{currentstroke}{rgb}{0.000000,0.000000,0.000000}%
\pgfsetstrokecolor{currentstroke}%
\pgfsetdash{}{0pt}%
\pgfpathmoveto{\pgfqpoint{3.946755in}{3.759490in}}%
\pgfpathlineto{\pgfqpoint{3.956480in}{3.578255in}}%
\pgfpathlineto{\pgfqpoint{3.965216in}{3.846775in}}%
\pgfpathlineto{\pgfqpoint{4.000492in}{3.391552in}}%
\pgfpathlineto{\pgfqpoint{4.034387in}{3.642754in}}%
\pgfpathlineto{\pgfqpoint{4.025027in}{3.621079in}}%
\pgfpathlineto{\pgfqpoint{4.015870in}{3.493338in}}%
\pgfpathlineto{\pgfqpoint{3.981351in}{3.635052in}}%
\pgfpathlineto{\pgfqpoint{3.946755in}{3.759490in}}%
\pgfpathclose%
\pgfusepath{fill}%
\end{pgfscope}%
\begin{pgfscope}%
\pgfpathrectangle{\pgfqpoint{1.020000in}{0.880000in}}{\pgfqpoint{6.160000in}{6.160000in}}%
\pgfusepath{clip}%
\pgfsetbuttcap%
\pgfsetroundjoin%
\definecolor{currentfill}{rgb}{0.570616,0.704109,0.997195}%
\pgfsetfillcolor{currentfill}%
\pgfsetlinewidth{0.000000pt}%
\definecolor{currentstroke}{rgb}{0.000000,0.000000,0.000000}%
\pgfsetstrokecolor{currentstroke}%
\pgfsetdash{}{0pt}%
\pgfpathmoveto{\pgfqpoint{4.502816in}{3.184801in}}%
\pgfpathlineto{\pgfqpoint{4.513198in}{3.327716in}}%
\pgfpathlineto{\pgfqpoint{4.522804in}{3.227684in}}%
\pgfpathlineto{\pgfqpoint{4.556761in}{3.158747in}}%
\pgfpathlineto{\pgfqpoint{4.590964in}{3.166427in}}%
\pgfpathlineto{\pgfqpoint{4.580396in}{3.023782in}}%
\pgfpathlineto{\pgfqpoint{4.571117in}{3.199486in}}%
\pgfpathlineto{\pgfqpoint{4.536725in}{3.121713in}}%
\pgfpathlineto{\pgfqpoint{4.502816in}{3.184801in}}%
\pgfpathclose%
\pgfusepath{fill}%
\end{pgfscope}%
\begin{pgfscope}%
\pgfpathrectangle{\pgfqpoint{1.020000in}{0.880000in}}{\pgfqpoint{6.160000in}{6.160000in}}%
\pgfusepath{clip}%
\pgfsetbuttcap%
\pgfsetroundjoin%
\definecolor{currentfill}{rgb}{0.804965,0.851666,0.926165}%
\pgfsetfillcolor{currentfill}%
\pgfsetlinewidth{0.000000pt}%
\definecolor{currentstroke}{rgb}{0.000000,0.000000,0.000000}%
\pgfsetstrokecolor{currentstroke}%
\pgfsetdash{}{0pt}%
\pgfpathmoveto{\pgfqpoint{3.429687in}{3.616344in}}%
\pgfpathlineto{\pgfqpoint{3.438109in}{3.658836in}}%
\pgfpathlineto{\pgfqpoint{3.446392in}{3.723788in}}%
\pgfpathlineto{\pgfqpoint{3.482416in}{3.514225in}}%
\pgfpathlineto{\pgfqpoint{3.516289in}{3.602812in}}%
\pgfpathlineto{\pgfqpoint{3.506955in}{3.678253in}}%
\pgfpathlineto{\pgfqpoint{3.498505in}{3.625172in}}%
\pgfpathlineto{\pgfqpoint{3.463576in}{3.694455in}}%
\pgfpathlineto{\pgfqpoint{3.429687in}{3.616344in}}%
\pgfpathclose%
\pgfusepath{fill}%
\end{pgfscope}%
\begin{pgfscope}%
\pgfpathrectangle{\pgfqpoint{1.020000in}{0.880000in}}{\pgfqpoint{6.160000in}{6.160000in}}%
\pgfusepath{clip}%
\pgfsetbuttcap%
\pgfsetroundjoin%
\definecolor{currentfill}{rgb}{0.304174,0.406945,0.845263}%
\pgfsetfillcolor{currentfill}%
\pgfsetlinewidth{0.000000pt}%
\definecolor{currentstroke}{rgb}{0.000000,0.000000,0.000000}%
\pgfsetstrokecolor{currentstroke}%
\pgfsetdash{}{0pt}%
\pgfpathmoveto{\pgfqpoint{5.508627in}{2.740176in}}%
\pgfpathlineto{\pgfqpoint{5.519551in}{2.732732in}}%
\pgfpathlineto{\pgfqpoint{5.528207in}{2.556431in}}%
\pgfpathlineto{\pgfqpoint{5.564613in}{2.739929in}}%
\pgfpathlineto{\pgfqpoint{5.595009in}{2.494309in}}%
\pgfpathlineto{\pgfqpoint{5.586350in}{2.666235in}}%
\pgfpathlineto{\pgfqpoint{5.575617in}{2.691243in}}%
\pgfpathlineto{\pgfqpoint{5.542757in}{2.760691in}}%
\pgfpathlineto{\pgfqpoint{5.508627in}{2.740176in}}%
\pgfpathclose%
\pgfusepath{fill}%
\end{pgfscope}%
\begin{pgfscope}%
\pgfpathrectangle{\pgfqpoint{1.020000in}{0.880000in}}{\pgfqpoint{6.160000in}{6.160000in}}%
\pgfusepath{clip}%
\pgfsetbuttcap%
\pgfsetroundjoin%
\definecolor{currentfill}{rgb}{0.388852,0.516298,0.921373}%
\pgfsetfillcolor{currentfill}%
\pgfsetlinewidth{0.000000pt}%
\definecolor{currentstroke}{rgb}{0.000000,0.000000,0.000000}%
\pgfsetstrokecolor{currentstroke}%
\pgfsetdash{}{0pt}%
\pgfpathmoveto{\pgfqpoint{4.903235in}{2.934282in}}%
\pgfpathlineto{\pgfqpoint{4.912077in}{2.727483in}}%
\pgfpathlineto{\pgfqpoint{4.924733in}{3.035673in}}%
\pgfpathlineto{\pgfqpoint{4.958146in}{2.941435in}}%
\pgfpathlineto{\pgfqpoint{4.990473in}{2.724090in}}%
\pgfpathlineto{\pgfqpoint{4.980543in}{2.786324in}}%
\pgfpathlineto{\pgfqpoint{4.970379in}{2.817793in}}%
\pgfpathlineto{\pgfqpoint{4.935947in}{2.758611in}}%
\pgfpathlineto{\pgfqpoint{4.903235in}{2.934282in}}%
\pgfpathclose%
\pgfusepath{fill}%
\end{pgfscope}%
\begin{pgfscope}%
\pgfpathrectangle{\pgfqpoint{1.020000in}{0.880000in}}{\pgfqpoint{6.160000in}{6.160000in}}%
\pgfusepath{clip}%
\pgfsetbuttcap%
\pgfsetroundjoin%
\definecolor{currentfill}{rgb}{0.835345,0.860514,0.898970}%
\pgfsetfillcolor{currentfill}%
\pgfsetlinewidth{0.000000pt}%
\definecolor{currentstroke}{rgb}{0.000000,0.000000,0.000000}%
\pgfsetstrokecolor{currentstroke}%
\pgfsetdash{}{0pt}%
\pgfpathmoveto{\pgfqpoint{3.498505in}{3.625172in}}%
\pgfpathlineto{\pgfqpoint{3.506955in}{3.678253in}}%
\pgfpathlineto{\pgfqpoint{3.516289in}{3.602812in}}%
\pgfpathlineto{\pgfqpoint{3.549986in}{3.725601in}}%
\pgfpathlineto{\pgfqpoint{3.584555in}{3.713175in}}%
\pgfpathlineto{\pgfqpoint{3.574861in}{3.843337in}}%
\pgfpathlineto{\pgfqpoint{3.566583in}{3.742788in}}%
\pgfpathlineto{\pgfqpoint{3.532122in}{3.747484in}}%
\pgfpathlineto{\pgfqpoint{3.498505in}{3.625172in}}%
\pgfpathclose%
\pgfusepath{fill}%
\end{pgfscope}%
\begin{pgfscope}%
\pgfpathrectangle{\pgfqpoint{1.020000in}{0.880000in}}{\pgfqpoint{6.160000in}{6.160000in}}%
\pgfusepath{clip}%
\pgfsetbuttcap%
\pgfsetroundjoin%
\definecolor{currentfill}{rgb}{0.271104,0.360011,0.807095}%
\pgfsetfillcolor{currentfill}%
\pgfsetlinewidth{0.000000pt}%
\definecolor{currentstroke}{rgb}{0.000000,0.000000,0.000000}%
\pgfsetstrokecolor{currentstroke}%
\pgfsetdash{}{0pt}%
\pgfpathmoveto{\pgfqpoint{5.955836in}{2.545960in}}%
\pgfpathlineto{\pgfqpoint{5.966393in}{2.493252in}}%
\pgfpathlineto{\pgfqpoint{5.978631in}{2.532598in}}%
\pgfpathlineto{\pgfqpoint{6.015688in}{2.712747in}}%
\pgfpathlineto{\pgfqpoint{6.049897in}{2.736005in}}%
\pgfpathlineto{\pgfqpoint{6.035377in}{2.580806in}}%
\pgfpathlineto{\pgfqpoint{6.023814in}{2.581536in}}%
\pgfpathlineto{\pgfqpoint{5.991667in}{2.664561in}}%
\pgfpathlineto{\pgfqpoint{5.955836in}{2.545960in}}%
\pgfpathclose%
\pgfusepath{fill}%
\end{pgfscope}%
\begin{pgfscope}%
\pgfpathrectangle{\pgfqpoint{1.020000in}{0.880000in}}{\pgfqpoint{6.160000in}{6.160000in}}%
\pgfusepath{clip}%
\pgfsetbuttcap%
\pgfsetroundjoin%
\definecolor{currentfill}{rgb}{0.243520,0.319189,0.771672}%
\pgfsetfillcolor{currentfill}%
\pgfsetlinewidth{0.000000pt}%
\definecolor{currentstroke}{rgb}{0.000000,0.000000,0.000000}%
\pgfsetstrokecolor{currentstroke}%
\pgfsetdash{}{0pt}%
\pgfpathmoveto{\pgfqpoint{5.889155in}{2.587731in}}%
\pgfpathlineto{\pgfqpoint{5.898110in}{2.444052in}}%
\pgfpathlineto{\pgfqpoint{5.912749in}{2.626837in}}%
\pgfpathlineto{\pgfqpoint{5.943828in}{2.473492in}}%
\pgfpathlineto{\pgfqpoint{5.978631in}{2.532598in}}%
\pgfpathlineto{\pgfqpoint{5.966393in}{2.493252in}}%
\pgfpathlineto{\pgfqpoint{5.955836in}{2.545960in}}%
\pgfpathlineto{\pgfqpoint{5.924177in}{2.661694in}}%
\pgfpathlineto{\pgfqpoint{5.889155in}{2.587731in}}%
\pgfpathclose%
\pgfusepath{fill}%
\end{pgfscope}%
\begin{pgfscope}%
\pgfpathrectangle{\pgfqpoint{1.020000in}{0.880000in}}{\pgfqpoint{6.160000in}{6.160000in}}%
\pgfusepath{clip}%
\pgfsetbuttcap%
\pgfsetroundjoin%
\definecolor{currentfill}{rgb}{0.713852,0.808857,0.979386}%
\pgfsetfillcolor{currentfill}%
\pgfsetlinewidth{0.000000pt}%
\definecolor{currentstroke}{rgb}{0.000000,0.000000,0.000000}%
\pgfsetstrokecolor{currentstroke}%
\pgfsetdash{}{0pt}%
\pgfpathmoveto{\pgfqpoint{4.259299in}{3.599020in}}%
\pgfpathlineto{\pgfqpoint{4.268904in}{3.536581in}}%
\pgfpathlineto{\pgfqpoint{4.278514in}{3.469213in}}%
\pgfpathlineto{\pgfqpoint{4.312853in}{3.510508in}}%
\pgfpathlineto{\pgfqpoint{4.346716in}{3.256213in}}%
\pgfpathlineto{\pgfqpoint{4.337136in}{3.343890in}}%
\pgfpathlineto{\pgfqpoint{4.327567in}{3.437178in}}%
\pgfpathlineto{\pgfqpoint{4.293398in}{3.451318in}}%
\pgfpathlineto{\pgfqpoint{4.259299in}{3.599020in}}%
\pgfpathclose%
\pgfusepath{fill}%
\end{pgfscope}%
\begin{pgfscope}%
\pgfpathrectangle{\pgfqpoint{1.020000in}{0.880000in}}{\pgfqpoint{6.160000in}{6.160000in}}%
\pgfusepath{clip}%
\pgfsetbuttcap%
\pgfsetroundjoin%
\definecolor{currentfill}{rgb}{0.554312,0.690097,0.995516}%
\pgfsetfillcolor{currentfill}%
\pgfsetlinewidth{0.000000pt}%
\definecolor{currentstroke}{rgb}{0.000000,0.000000,0.000000}%
\pgfsetstrokecolor{currentstroke}%
\pgfsetdash{}{0pt}%
\pgfpathmoveto{\pgfqpoint{2.584505in}{3.388401in}}%
\pgfpathlineto{\pgfqpoint{2.596508in}{3.126603in}}%
\pgfpathlineto{\pgfqpoint{2.607578in}{2.922177in}}%
\pgfpathlineto{\pgfqpoint{2.643283in}{2.858157in}}%
\pgfpathlineto{\pgfqpoint{2.673846in}{3.129500in}}%
\pgfpathlineto{\pgfqpoint{2.665234in}{3.176048in}}%
\pgfpathlineto{\pgfqpoint{2.655295in}{3.309968in}}%
\pgfpathlineto{\pgfqpoint{2.620938in}{3.284400in}}%
\pgfpathlineto{\pgfqpoint{2.584505in}{3.388401in}}%
\pgfpathclose%
\pgfusepath{fill}%
\end{pgfscope}%
\begin{pgfscope}%
\pgfpathrectangle{\pgfqpoint{1.020000in}{0.880000in}}{\pgfqpoint{6.160000in}{6.160000in}}%
\pgfusepath{clip}%
\pgfsetbuttcap%
\pgfsetroundjoin%
\definecolor{currentfill}{rgb}{0.624703,0.748318,0.998719}%
\pgfsetfillcolor{currentfill}%
\pgfsetlinewidth{0.000000pt}%
\definecolor{currentstroke}{rgb}{0.000000,0.000000,0.000000}%
\pgfsetstrokecolor{currentstroke}%
\pgfsetdash{}{0pt}%
\pgfpathmoveto{\pgfqpoint{2.722651in}{3.453420in}}%
\pgfpathlineto{\pgfqpoint{2.734192in}{3.208457in}}%
\pgfpathlineto{\pgfqpoint{2.740506in}{3.323846in}}%
\pgfpathlineto{\pgfqpoint{2.776526in}{3.235184in}}%
\pgfpathlineto{\pgfqpoint{2.812277in}{3.159949in}}%
\pgfpathlineto{\pgfqpoint{2.800920in}{3.399216in}}%
\pgfpathlineto{\pgfqpoint{2.794905in}{3.252935in}}%
\pgfpathlineto{\pgfqpoint{2.761741in}{3.148461in}}%
\pgfpathlineto{\pgfqpoint{2.722651in}{3.453420in}}%
\pgfpathclose%
\pgfusepath{fill}%
\end{pgfscope}%
\begin{pgfscope}%
\pgfpathrectangle{\pgfqpoint{1.020000in}{0.880000in}}{\pgfqpoint{6.160000in}{6.160000in}}%
\pgfusepath{clip}%
\pgfsetbuttcap%
\pgfsetroundjoin%
\definecolor{currentfill}{rgb}{0.640828,0.760752,0.997846}%
\pgfsetfillcolor{currentfill}%
\pgfsetlinewidth{0.000000pt}%
\definecolor{currentstroke}{rgb}{0.000000,0.000000,0.000000}%
\pgfsetstrokecolor{currentstroke}%
\pgfsetdash{}{0pt}%
\pgfpathmoveto{\pgfqpoint{4.346716in}{3.256213in}}%
\pgfpathlineto{\pgfqpoint{4.356674in}{3.383987in}}%
\pgfpathlineto{\pgfqpoint{4.366306in}{3.308420in}}%
\pgfpathlineto{\pgfqpoint{4.400741in}{3.389986in}}%
\pgfpathlineto{\pgfqpoint{4.434982in}{3.380400in}}%
\pgfpathlineto{\pgfqpoint{4.424404in}{3.074731in}}%
\pgfpathlineto{\pgfqpoint{4.415018in}{3.247858in}}%
\pgfpathlineto{\pgfqpoint{4.381142in}{3.384749in}}%
\pgfpathlineto{\pgfqpoint{4.346716in}{3.256213in}}%
\pgfpathclose%
\pgfusepath{fill}%
\end{pgfscope}%
\begin{pgfscope}%
\pgfpathrectangle{\pgfqpoint{1.020000in}{0.880000in}}{\pgfqpoint{6.160000in}{6.160000in}}%
\pgfusepath{clip}%
\pgfsetbuttcap%
\pgfsetroundjoin%
\definecolor{currentfill}{rgb}{0.791392,0.846750,0.936641}%
\pgfsetfillcolor{currentfill}%
\pgfsetlinewidth{0.000000pt}%
\definecolor{currentstroke}{rgb}{0.000000,0.000000,0.000000}%
\pgfsetstrokecolor{currentstroke}%
\pgfsetdash{}{0pt}%
\pgfpathmoveto{\pgfqpoint{3.361834in}{3.481381in}}%
\pgfpathlineto{\pgfqpoint{3.370577in}{3.470691in}}%
\pgfpathlineto{\pgfqpoint{3.378191in}{3.602511in}}%
\pgfpathlineto{\pgfqpoint{3.410903in}{3.842273in}}%
\pgfpathlineto{\pgfqpoint{3.446392in}{3.723788in}}%
\pgfpathlineto{\pgfqpoint{3.438109in}{3.658836in}}%
\pgfpathlineto{\pgfqpoint{3.429687in}{3.616344in}}%
\pgfpathlineto{\pgfqpoint{3.396601in}{3.440366in}}%
\pgfpathlineto{\pgfqpoint{3.361834in}{3.481381in}}%
\pgfpathclose%
\pgfusepath{fill}%
\end{pgfscope}%
\begin{pgfscope}%
\pgfpathrectangle{\pgfqpoint{1.020000in}{0.880000in}}{\pgfqpoint{6.160000in}{6.160000in}}%
\pgfusepath{clip}%
\pgfsetbuttcap%
\pgfsetroundjoin%
\definecolor{currentfill}{rgb}{0.597777,0.727330,0.999777}%
\pgfsetfillcolor{currentfill}%
\pgfsetlinewidth{0.000000pt}%
\definecolor{currentstroke}{rgb}{0.000000,0.000000,0.000000}%
\pgfsetstrokecolor{currentstroke}%
\pgfsetdash{}{0pt}%
\pgfpathmoveto{\pgfqpoint{2.655295in}{3.309968in}}%
\pgfpathlineto{\pgfqpoint{2.665234in}{3.176048in}}%
\pgfpathlineto{\pgfqpoint{2.673846in}{3.129500in}}%
\pgfpathlineto{\pgfqpoint{2.708972in}{3.103708in}}%
\pgfpathlineto{\pgfqpoint{2.740506in}{3.323846in}}%
\pgfpathlineto{\pgfqpoint{2.734192in}{3.208457in}}%
\pgfpathlineto{\pgfqpoint{2.722651in}{3.453420in}}%
\pgfpathlineto{\pgfqpoint{2.693255in}{3.094341in}}%
\pgfpathlineto{\pgfqpoint{2.655295in}{3.309968in}}%
\pgfpathclose%
\pgfusepath{fill}%
\end{pgfscope}%
\begin{pgfscope}%
\pgfpathrectangle{\pgfqpoint{1.020000in}{0.880000in}}{\pgfqpoint{6.160000in}{6.160000in}}%
\pgfusepath{clip}%
\pgfsetbuttcap%
\pgfsetroundjoin%
\definecolor{currentfill}{rgb}{0.338377,0.452819,0.879317}%
\pgfsetfillcolor{currentfill}%
\pgfsetlinewidth{0.000000pt}%
\definecolor{currentstroke}{rgb}{0.000000,0.000000,0.000000}%
\pgfsetstrokecolor{currentstroke}%
\pgfsetdash{}{0pt}%
\pgfpathmoveto{\pgfqpoint{5.283146in}{2.705717in}}%
\pgfpathlineto{\pgfqpoint{5.293786in}{2.694118in}}%
\pgfpathlineto{\pgfqpoint{5.305351in}{2.762534in}}%
\pgfpathlineto{\pgfqpoint{5.339646in}{2.789495in}}%
\pgfpathlineto{\pgfqpoint{5.375369in}{2.935884in}}%
\pgfpathlineto{\pgfqpoint{5.360937in}{2.637190in}}%
\pgfpathlineto{\pgfqpoint{5.351872in}{2.787223in}}%
\pgfpathlineto{\pgfqpoint{5.315665in}{2.585336in}}%
\pgfpathlineto{\pgfqpoint{5.283146in}{2.705717in}}%
\pgfpathclose%
\pgfusepath{fill}%
\end{pgfscope}%
\begin{pgfscope}%
\pgfpathrectangle{\pgfqpoint{1.020000in}{0.880000in}}{\pgfqpoint{6.160000in}{6.160000in}}%
\pgfusepath{clip}%
\pgfsetbuttcap%
\pgfsetroundjoin%
\definecolor{currentfill}{rgb}{0.271104,0.360011,0.807095}%
\pgfsetfillcolor{currentfill}%
\pgfsetlinewidth{0.000000pt}%
\definecolor{currentstroke}{rgb}{0.000000,0.000000,0.000000}%
\pgfsetstrokecolor{currentstroke}%
\pgfsetdash{}{0pt}%
\pgfpathmoveto{\pgfqpoint{6.181582in}{2.575871in}}%
\pgfpathlineto{\pgfqpoint{6.197659in}{2.788724in}}%
\pgfpathlineto{\pgfqpoint{6.205896in}{2.613415in}}%
\pgfpathlineto{\pgfqpoint{6.238109in}{2.541618in}}%
\pgfpathlineto{\pgfqpoint{6.271416in}{2.525013in}}%
\pgfpathlineto{\pgfqpoint{6.262933in}{2.685614in}}%
\pgfpathlineto{\pgfqpoint{6.246665in}{2.473093in}}%
\pgfpathlineto{\pgfqpoint{6.215631in}{2.596705in}}%
\pgfpathlineto{\pgfqpoint{6.181582in}{2.575871in}}%
\pgfpathclose%
\pgfusepath{fill}%
\end{pgfscope}%
\begin{pgfscope}%
\pgfpathrectangle{\pgfqpoint{1.020000in}{0.880000in}}{\pgfqpoint{6.160000in}{6.160000in}}%
\pgfusepath{clip}%
\pgfsetbuttcap%
\pgfsetroundjoin%
\definecolor{currentfill}{rgb}{0.782049,0.842864,0.942980}%
\pgfsetfillcolor{currentfill}%
\pgfsetlinewidth{0.000000pt}%
\definecolor{currentstroke}{rgb}{0.000000,0.000000,0.000000}%
\pgfsetstrokecolor{currentstroke}%
\pgfsetdash{}{0pt}%
\pgfpathmoveto{\pgfqpoint{4.103054in}{3.613739in}}%
\pgfpathlineto{\pgfqpoint{4.112532in}{3.617445in}}%
\pgfpathlineto{\pgfqpoint{4.122086in}{3.530542in}}%
\pgfpathlineto{\pgfqpoint{4.156387in}{3.607274in}}%
\pgfpathlineto{\pgfqpoint{4.190710in}{3.553749in}}%
\pgfpathlineto{\pgfqpoint{4.181146in}{3.544557in}}%
\pgfpathlineto{\pgfqpoint{4.171600in}{3.572925in}}%
\pgfpathlineto{\pgfqpoint{4.137351in}{3.577854in}}%
\pgfpathlineto{\pgfqpoint{4.103054in}{3.613739in}}%
\pgfpathclose%
\pgfusepath{fill}%
\end{pgfscope}%
\begin{pgfscope}%
\pgfpathrectangle{\pgfqpoint{1.020000in}{0.880000in}}{\pgfqpoint{6.160000in}{6.160000in}}%
\pgfusepath{clip}%
\pgfsetbuttcap%
\pgfsetroundjoin%
\definecolor{currentfill}{rgb}{0.839351,0.861167,0.894494}%
\pgfsetfillcolor{currentfill}%
\pgfsetlinewidth{0.000000pt}%
\definecolor{currentstroke}{rgb}{0.000000,0.000000,0.000000}%
\pgfsetstrokecolor{currentstroke}%
\pgfsetdash{}{0pt}%
\pgfpathmoveto{\pgfqpoint{3.721611in}{3.864042in}}%
\pgfpathlineto{\pgfqpoint{3.731634in}{3.654041in}}%
\pgfpathlineto{\pgfqpoint{3.740083in}{3.796278in}}%
\pgfpathlineto{\pgfqpoint{3.775542in}{3.556548in}}%
\pgfpathlineto{\pgfqpoint{3.809440in}{3.685853in}}%
\pgfpathlineto{\pgfqpoint{3.799991in}{3.764185in}}%
\pgfpathlineto{\pgfqpoint{3.791111in}{3.698644in}}%
\pgfpathlineto{\pgfqpoint{3.756875in}{3.669489in}}%
\pgfpathlineto{\pgfqpoint{3.721611in}{3.864042in}}%
\pgfpathclose%
\pgfusepath{fill}%
\end{pgfscope}%
\begin{pgfscope}%
\pgfpathrectangle{\pgfqpoint{1.020000in}{0.880000in}}{\pgfqpoint{6.160000in}{6.160000in}}%
\pgfusepath{clip}%
\pgfsetbuttcap%
\pgfsetroundjoin%
\definecolor{currentfill}{rgb}{0.651398,0.768121,0.995891}%
\pgfsetfillcolor{currentfill}%
\pgfsetlinewidth{0.000000pt}%
\definecolor{currentstroke}{rgb}{0.000000,0.000000,0.000000}%
\pgfsetstrokecolor{currentstroke}%
\pgfsetdash{}{0pt}%
\pgfpathmoveto{\pgfqpoint{3.018105in}{3.320769in}}%
\pgfpathlineto{\pgfqpoint{3.025820in}{3.370576in}}%
\pgfpathlineto{\pgfqpoint{3.034600in}{3.329968in}}%
\pgfpathlineto{\pgfqpoint{3.069535in}{3.306498in}}%
\pgfpathlineto{\pgfqpoint{3.104387in}{3.287205in}}%
\pgfpathlineto{\pgfqpoint{3.094575in}{3.419026in}}%
\pgfpathlineto{\pgfqpoint{3.088321in}{3.226098in}}%
\pgfpathlineto{\pgfqpoint{3.052393in}{3.348298in}}%
\pgfpathlineto{\pgfqpoint{3.018105in}{3.320769in}}%
\pgfpathclose%
\pgfusepath{fill}%
\end{pgfscope}%
\begin{pgfscope}%
\pgfpathrectangle{\pgfqpoint{1.020000in}{0.880000in}}{\pgfqpoint{6.160000in}{6.160000in}}%
\pgfusepath{clip}%
\pgfsetbuttcap%
\pgfsetroundjoin%
\definecolor{currentfill}{rgb}{0.299441,0.400248,0.839842}%
\pgfsetfillcolor{currentfill}%
\pgfsetlinewidth{0.000000pt}%
\definecolor{currentstroke}{rgb}{0.000000,0.000000,0.000000}%
\pgfsetstrokecolor{currentstroke}%
\pgfsetdash{}{0pt}%
\pgfpathmoveto{\pgfqpoint{5.438291in}{2.536722in}}%
\pgfpathlineto{\pgfqpoint{5.450065in}{2.602449in}}%
\pgfpathlineto{\pgfqpoint{5.461470in}{2.636604in}}%
\pgfpathlineto{\pgfqpoint{5.496598in}{2.727721in}}%
\pgfpathlineto{\pgfqpoint{5.528207in}{2.556431in}}%
\pgfpathlineto{\pgfqpoint{5.519551in}{2.732732in}}%
\pgfpathlineto{\pgfqpoint{5.508627in}{2.740176in}}%
\pgfpathlineto{\pgfqpoint{5.474384in}{2.711491in}}%
\pgfpathlineto{\pgfqpoint{5.438291in}{2.536722in}}%
\pgfpathclose%
\pgfusepath{fill}%
\end{pgfscope}%
\begin{pgfscope}%
\pgfpathrectangle{\pgfqpoint{1.020000in}{0.880000in}}{\pgfqpoint{6.160000in}{6.160000in}}%
\pgfusepath{clip}%
\pgfsetbuttcap%
\pgfsetroundjoin%
\definecolor{currentfill}{rgb}{0.624703,0.748318,0.998719}%
\pgfsetfillcolor{currentfill}%
\pgfsetlinewidth{0.000000pt}%
\definecolor{currentstroke}{rgb}{0.000000,0.000000,0.000000}%
\pgfsetstrokecolor{currentstroke}%
\pgfsetdash{}{0pt}%
\pgfpathmoveto{\pgfqpoint{2.950558in}{3.182002in}}%
\pgfpathlineto{\pgfqpoint{2.957921in}{3.250969in}}%
\pgfpathlineto{\pgfqpoint{2.965882in}{3.273272in}}%
\pgfpathlineto{\pgfqpoint{3.000866in}{3.248947in}}%
\pgfpathlineto{\pgfqpoint{3.034600in}{3.329968in}}%
\pgfpathlineto{\pgfqpoint{3.025820in}{3.370576in}}%
\pgfpathlineto{\pgfqpoint{3.018105in}{3.320769in}}%
\pgfpathlineto{\pgfqpoint{2.984664in}{3.222541in}}%
\pgfpathlineto{\pgfqpoint{2.950558in}{3.182002in}}%
\pgfpathclose%
\pgfusepath{fill}%
\end{pgfscope}%
\begin{pgfscope}%
\pgfpathrectangle{\pgfqpoint{1.020000in}{0.880000in}}{\pgfqpoint{6.160000in}{6.160000in}}%
\pgfusepath{clip}%
\pgfsetbuttcap%
\pgfsetroundjoin%
\definecolor{currentfill}{rgb}{0.252663,0.332837,0.783665}%
\pgfsetfillcolor{currentfill}%
\pgfsetlinewidth{0.000000pt}%
\definecolor{currentstroke}{rgb}{0.000000,0.000000,0.000000}%
\pgfsetstrokecolor{currentstroke}%
\pgfsetdash{}{0pt}%
\pgfpathmoveto{\pgfqpoint{6.110737in}{2.394168in}}%
\pgfpathlineto{\pgfqpoint{6.125224in}{2.537809in}}%
\pgfpathlineto{\pgfqpoint{6.138095in}{2.596432in}}%
\pgfpathlineto{\pgfqpoint{6.169776in}{2.493848in}}%
\pgfpathlineto{\pgfqpoint{6.205896in}{2.613415in}}%
\pgfpathlineto{\pgfqpoint{6.197659in}{2.788724in}}%
\pgfpathlineto{\pgfqpoint{6.181582in}{2.575871in}}%
\pgfpathlineto{\pgfqpoint{6.145972in}{2.476597in}}%
\pgfpathlineto{\pgfqpoint{6.110737in}{2.394168in}}%
\pgfpathclose%
\pgfusepath{fill}%
\end{pgfscope}%
\begin{pgfscope}%
\pgfpathrectangle{\pgfqpoint{1.020000in}{0.880000in}}{\pgfqpoint{6.160000in}{6.160000in}}%
\pgfusepath{clip}%
\pgfsetbuttcap%
\pgfsetroundjoin%
\definecolor{currentfill}{rgb}{0.693321,0.796314,0.986308}%
\pgfsetfillcolor{currentfill}%
\pgfsetlinewidth{0.000000pt}%
\definecolor{currentstroke}{rgb}{0.000000,0.000000,0.000000}%
\pgfsetstrokecolor{currentstroke}%
\pgfsetdash{}{0pt}%
\pgfpathmoveto{\pgfqpoint{3.088321in}{3.226098in}}%
\pgfpathlineto{\pgfqpoint{3.094575in}{3.419026in}}%
\pgfpathlineto{\pgfqpoint{3.104387in}{3.287205in}}%
\pgfpathlineto{\pgfqpoint{3.137908in}{3.391952in}}%
\pgfpathlineto{\pgfqpoint{3.171822in}{3.462824in}}%
\pgfpathlineto{\pgfqpoint{3.162629in}{3.534494in}}%
\pgfpathlineto{\pgfqpoint{3.154051in}{3.546950in}}%
\pgfpathlineto{\pgfqpoint{3.121585in}{3.344683in}}%
\pgfpathlineto{\pgfqpoint{3.088321in}{3.226098in}}%
\pgfpathclose%
\pgfusepath{fill}%
\end{pgfscope}%
\begin{pgfscope}%
\pgfpathrectangle{\pgfqpoint{1.020000in}{0.880000in}}{\pgfqpoint{6.160000in}{6.160000in}}%
\pgfusepath{clip}%
\pgfsetbuttcap%
\pgfsetroundjoin%
\definecolor{currentfill}{rgb}{0.299441,0.400248,0.839842}%
\pgfsetfillcolor{currentfill}%
\pgfsetlinewidth{0.000000pt}%
\definecolor{currentstroke}{rgb}{0.000000,0.000000,0.000000}%
\pgfsetstrokecolor{currentstroke}%
\pgfsetdash{}{0pt}%
\pgfpathmoveto{\pgfqpoint{5.665324in}{2.678707in}}%
\pgfpathlineto{\pgfqpoint{5.673910in}{2.504849in}}%
\pgfpathlineto{\pgfqpoint{5.688260in}{2.712220in}}%
\pgfpathlineto{\pgfqpoint{5.719904in}{2.567236in}}%
\pgfpathlineto{\pgfqpoint{5.755356in}{2.670157in}}%
\pgfpathlineto{\pgfqpoint{5.744574in}{2.702650in}}%
\pgfpathlineto{\pgfqpoint{5.732426in}{2.646594in}}%
\pgfpathlineto{\pgfqpoint{5.699785in}{2.721223in}}%
\pgfpathlineto{\pgfqpoint{5.665324in}{2.678707in}}%
\pgfpathclose%
\pgfusepath{fill}%
\end{pgfscope}%
\begin{pgfscope}%
\pgfpathrectangle{\pgfqpoint{1.020000in}{0.880000in}}{\pgfqpoint{6.160000in}{6.160000in}}%
\pgfusepath{clip}%
\pgfsetbuttcap%
\pgfsetroundjoin%
\definecolor{currentfill}{rgb}{0.271104,0.360011,0.807095}%
\pgfsetfillcolor{currentfill}%
\pgfsetlinewidth{0.000000pt}%
\definecolor{currentstroke}{rgb}{0.000000,0.000000,0.000000}%
\pgfsetstrokecolor{currentstroke}%
\pgfsetdash{}{0pt}%
\pgfpathmoveto{\pgfqpoint{5.595009in}{2.494309in}}%
\pgfpathlineto{\pgfqpoint{5.607164in}{2.566879in}}%
\pgfpathlineto{\pgfqpoint{5.618452in}{2.576669in}}%
\pgfpathlineto{\pgfqpoint{5.651157in}{2.496887in}}%
\pgfpathlineto{\pgfqpoint{5.688260in}{2.712220in}}%
\pgfpathlineto{\pgfqpoint{5.673910in}{2.504849in}}%
\pgfpathlineto{\pgfqpoint{5.665324in}{2.678707in}}%
\pgfpathlineto{\pgfqpoint{5.631608in}{2.686796in}}%
\pgfpathlineto{\pgfqpoint{5.595009in}{2.494309in}}%
\pgfpathclose%
\pgfusepath{fill}%
\end{pgfscope}%
\begin{pgfscope}%
\pgfpathrectangle{\pgfqpoint{1.020000in}{0.880000in}}{\pgfqpoint{6.160000in}{6.160000in}}%
\pgfusepath{clip}%
\pgfsetbuttcap%
\pgfsetroundjoin%
\definecolor{currentfill}{rgb}{0.333490,0.446265,0.874452}%
\pgfsetfillcolor{currentfill}%
\pgfsetlinewidth{0.000000pt}%
\definecolor{currentstroke}{rgb}{0.000000,0.000000,0.000000}%
\pgfsetstrokecolor{currentstroke}%
\pgfsetdash{}{0pt}%
\pgfpathmoveto{\pgfqpoint{5.213834in}{2.565087in}}%
\pgfpathlineto{\pgfqpoint{5.226079in}{2.713421in}}%
\pgfpathlineto{\pgfqpoint{5.237866in}{2.813491in}}%
\pgfpathlineto{\pgfqpoint{5.270642in}{2.698030in}}%
\pgfpathlineto{\pgfqpoint{5.305351in}{2.762534in}}%
\pgfpathlineto{\pgfqpoint{5.293786in}{2.694118in}}%
\pgfpathlineto{\pgfqpoint{5.283146in}{2.705717in}}%
\pgfpathlineto{\pgfqpoint{5.250657in}{2.838452in}}%
\pgfpathlineto{\pgfqpoint{5.213834in}{2.565087in}}%
\pgfpathclose%
\pgfusepath{fill}%
\end{pgfscope}%
\begin{pgfscope}%
\pgfpathrectangle{\pgfqpoint{1.020000in}{0.880000in}}{\pgfqpoint{6.160000in}{6.160000in}}%
\pgfusepath{clip}%
\pgfsetbuttcap%
\pgfsetroundjoin%
\definecolor{currentfill}{rgb}{0.543440,0.680003,0.993051}%
\pgfsetfillcolor{currentfill}%
\pgfsetlinewidth{0.000000pt}%
\definecolor{currentstroke}{rgb}{0.000000,0.000000,0.000000}%
\pgfsetstrokecolor{currentstroke}%
\pgfsetdash{}{0pt}%
\pgfpathmoveto{\pgfqpoint{4.590964in}{3.166427in}}%
\pgfpathlineto{\pgfqpoint{4.601015in}{3.173737in}}%
\pgfpathlineto{\pgfqpoint{4.610908in}{3.138775in}}%
\pgfpathlineto{\pgfqpoint{4.644534in}{3.012345in}}%
\pgfpathlineto{\pgfqpoint{4.678979in}{3.071976in}}%
\pgfpathlineto{\pgfqpoint{4.668841in}{3.064625in}}%
\pgfpathlineto{\pgfqpoint{4.659119in}{3.139382in}}%
\pgfpathlineto{\pgfqpoint{4.625054in}{3.152346in}}%
\pgfpathlineto{\pgfqpoint{4.590964in}{3.166427in}}%
\pgfpathclose%
\pgfusepath{fill}%
\end{pgfscope}%
\begin{pgfscope}%
\pgfpathrectangle{\pgfqpoint{1.020000in}{0.880000in}}{\pgfqpoint{6.160000in}{6.160000in}}%
\pgfusepath{clip}%
\pgfsetbuttcap%
\pgfsetroundjoin%
\definecolor{currentfill}{rgb}{0.613933,0.739923,0.999142}%
\pgfsetfillcolor{currentfill}%
\pgfsetlinewidth{0.000000pt}%
\definecolor{currentstroke}{rgb}{0.000000,0.000000,0.000000}%
\pgfsetstrokecolor{currentstroke}%
\pgfsetdash{}{0pt}%
\pgfpathmoveto{\pgfqpoint{2.880816in}{3.219682in}}%
\pgfpathlineto{\pgfqpoint{2.888736in}{3.236820in}}%
\pgfpathlineto{\pgfqpoint{2.897405in}{3.197407in}}%
\pgfpathlineto{\pgfqpoint{2.930438in}{3.331261in}}%
\pgfpathlineto{\pgfqpoint{2.965882in}{3.273272in}}%
\pgfpathlineto{\pgfqpoint{2.957921in}{3.250969in}}%
\pgfpathlineto{\pgfqpoint{2.950558in}{3.182002in}}%
\pgfpathlineto{\pgfqpoint{2.914895in}{3.264573in}}%
\pgfpathlineto{\pgfqpoint{2.880816in}{3.219682in}}%
\pgfpathclose%
\pgfusepath{fill}%
\end{pgfscope}%
\begin{pgfscope}%
\pgfpathrectangle{\pgfqpoint{1.020000in}{0.880000in}}{\pgfqpoint{6.160000in}{6.160000in}}%
\pgfusepath{clip}%
\pgfsetbuttcap%
\pgfsetroundjoin%
\definecolor{currentfill}{rgb}{0.581486,0.713451,0.998314}%
\pgfsetfillcolor{currentfill}%
\pgfsetlinewidth{0.000000pt}%
\definecolor{currentstroke}{rgb}{0.000000,0.000000,0.000000}%
\pgfsetstrokecolor{currentstroke}%
\pgfsetdash{}{0pt}%
\pgfpathmoveto{\pgfqpoint{2.812277in}{3.159949in}}%
\pgfpathlineto{\pgfqpoint{2.818369in}{3.304427in}}%
\pgfpathlineto{\pgfqpoint{2.829348in}{3.092125in}}%
\pgfpathlineto{\pgfqpoint{2.863805in}{3.111819in}}%
\pgfpathlineto{\pgfqpoint{2.897405in}{3.197407in}}%
\pgfpathlineto{\pgfqpoint{2.888736in}{3.236820in}}%
\pgfpathlineto{\pgfqpoint{2.880816in}{3.219682in}}%
\pgfpathlineto{\pgfqpoint{2.846883in}{3.164753in}}%
\pgfpathlineto{\pgfqpoint{2.812277in}{3.159949in}}%
\pgfpathclose%
\pgfusepath{fill}%
\end{pgfscope}%
\begin{pgfscope}%
\pgfpathrectangle{\pgfqpoint{1.020000in}{0.880000in}}{\pgfqpoint{6.160000in}{6.160000in}}%
\pgfusepath{clip}%
\pgfsetbuttcap%
\pgfsetroundjoin%
\definecolor{currentfill}{rgb}{0.839351,0.861167,0.894494}%
\pgfsetfillcolor{currentfill}%
\pgfsetlinewidth{0.000000pt}%
\definecolor{currentstroke}{rgb}{0.000000,0.000000,0.000000}%
\pgfsetstrokecolor{currentstroke}%
\pgfsetdash{}{0pt}%
\pgfpathmoveto{\pgfqpoint{3.654237in}{3.568178in}}%
\pgfpathlineto{\pgfqpoint{3.662032in}{3.796603in}}%
\pgfpathlineto{\pgfqpoint{3.672208in}{3.570208in}}%
\pgfpathlineto{\pgfqpoint{3.706104in}{3.685899in}}%
\pgfpathlineto{\pgfqpoint{3.740083in}{3.796278in}}%
\pgfpathlineto{\pgfqpoint{3.731634in}{3.654041in}}%
\pgfpathlineto{\pgfqpoint{3.721611in}{3.864042in}}%
\pgfpathlineto{\pgfqpoint{3.688010in}{3.690530in}}%
\pgfpathlineto{\pgfqpoint{3.654237in}{3.568178in}}%
\pgfpathclose%
\pgfusepath{fill}%
\end{pgfscope}%
\begin{pgfscope}%
\pgfpathrectangle{\pgfqpoint{1.020000in}{0.880000in}}{\pgfqpoint{6.160000in}{6.160000in}}%
\pgfusepath{clip}%
\pgfsetbuttcap%
\pgfsetroundjoin%
\definecolor{currentfill}{rgb}{0.748682,0.827679,0.963334}%
\pgfsetfillcolor{currentfill}%
\pgfsetlinewidth{0.000000pt}%
\definecolor{currentstroke}{rgb}{0.000000,0.000000,0.000000}%
\pgfsetstrokecolor{currentstroke}%
\pgfsetdash{}{0pt}%
\pgfpathmoveto{\pgfqpoint{3.154051in}{3.546950in}}%
\pgfpathlineto{\pgfqpoint{3.162629in}{3.534494in}}%
\pgfpathlineto{\pgfqpoint{3.171822in}{3.462824in}}%
\pgfpathlineto{\pgfqpoint{3.206599in}{3.448302in}}%
\pgfpathlineto{\pgfqpoint{3.240575in}{3.515421in}}%
\pgfpathlineto{\pgfqpoint{3.231447in}{3.577524in}}%
\pgfpathlineto{\pgfqpoint{3.224153in}{3.449223in}}%
\pgfpathlineto{\pgfqpoint{3.189353in}{3.475564in}}%
\pgfpathlineto{\pgfqpoint{3.154051in}{3.546950in}}%
\pgfpathclose%
\pgfusepath{fill}%
\end{pgfscope}%
\begin{pgfscope}%
\pgfpathrectangle{\pgfqpoint{1.020000in}{0.880000in}}{\pgfqpoint{6.160000in}{6.160000in}}%
\pgfusepath{clip}%
\pgfsetbuttcap%
\pgfsetroundjoin%
\definecolor{currentfill}{rgb}{0.743754,0.825125,0.965798}%
\pgfsetfillcolor{currentfill}%
\pgfsetlinewidth{0.000000pt}%
\definecolor{currentstroke}{rgb}{0.000000,0.000000,0.000000}%
\pgfsetstrokecolor{currentstroke}%
\pgfsetdash{}{0pt}%
\pgfpathmoveto{\pgfqpoint{4.190710in}{3.553749in}}%
\pgfpathlineto{\pgfqpoint{4.200272in}{3.414583in}}%
\pgfpathlineto{\pgfqpoint{4.209890in}{3.509267in}}%
\pgfpathlineto{\pgfqpoint{4.244167in}{3.404186in}}%
\pgfpathlineto{\pgfqpoint{4.278514in}{3.469213in}}%
\pgfpathlineto{\pgfqpoint{4.268904in}{3.536581in}}%
\pgfpathlineto{\pgfqpoint{4.259299in}{3.599020in}}%
\pgfpathlineto{\pgfqpoint{4.224966in}{3.477539in}}%
\pgfpathlineto{\pgfqpoint{4.190710in}{3.553749in}}%
\pgfpathclose%
\pgfusepath{fill}%
\end{pgfscope}%
\begin{pgfscope}%
\pgfpathrectangle{\pgfqpoint{1.020000in}{0.880000in}}{\pgfqpoint{6.160000in}{6.160000in}}%
\pgfusepath{clip}%
\pgfsetbuttcap%
\pgfsetroundjoin%
\definecolor{currentfill}{rgb}{0.309060,0.413498,0.850128}%
\pgfsetfillcolor{currentfill}%
\pgfsetlinewidth{0.000000pt}%
\definecolor{currentstroke}{rgb}{0.000000,0.000000,0.000000}%
\pgfsetstrokecolor{currentstroke}%
\pgfsetdash{}{0pt}%
\pgfpathmoveto{\pgfqpoint{5.375369in}{2.935884in}}%
\pgfpathlineto{\pgfqpoint{5.382539in}{2.626542in}}%
\pgfpathlineto{\pgfqpoint{5.394154in}{2.685130in}}%
\pgfpathlineto{\pgfqpoint{5.427306in}{2.619016in}}%
\pgfpathlineto{\pgfqpoint{5.461470in}{2.636604in}}%
\pgfpathlineto{\pgfqpoint{5.450065in}{2.602449in}}%
\pgfpathlineto{\pgfqpoint{5.438291in}{2.536722in}}%
\pgfpathlineto{\pgfqpoint{5.406499in}{2.703111in}}%
\pgfpathlineto{\pgfqpoint{5.375369in}{2.935884in}}%
\pgfpathclose%
\pgfusepath{fill}%
\end{pgfscope}%
\begin{pgfscope}%
\pgfpathrectangle{\pgfqpoint{1.020000in}{0.880000in}}{\pgfqpoint{6.160000in}{6.160000in}}%
\pgfusepath{clip}%
\pgfsetbuttcap%
\pgfsetroundjoin%
\definecolor{currentfill}{rgb}{0.409611,0.540759,0.935545}%
\pgfsetfillcolor{currentfill}%
\pgfsetlinewidth{0.000000pt}%
\definecolor{currentstroke}{rgb}{0.000000,0.000000,0.000000}%
\pgfsetstrokecolor{currentstroke}%
\pgfsetdash{}{0pt}%
\pgfpathmoveto{\pgfqpoint{5.060292in}{2.933291in}}%
\pgfpathlineto{\pgfqpoint{5.070659in}{2.913518in}}%
\pgfpathlineto{\pgfqpoint{5.079045in}{2.672267in}}%
\pgfpathlineto{\pgfqpoint{5.114668in}{2.844711in}}%
\pgfpathlineto{\pgfqpoint{5.148542in}{2.827116in}}%
\pgfpathlineto{\pgfqpoint{5.138850in}{2.923991in}}%
\pgfpathlineto{\pgfqpoint{5.128315in}{2.931878in}}%
\pgfpathlineto{\pgfqpoint{5.093730in}{2.868737in}}%
\pgfpathlineto{\pgfqpoint{5.060292in}{2.933291in}}%
\pgfpathclose%
\pgfusepath{fill}%
\end{pgfscope}%
\begin{pgfscope}%
\pgfpathrectangle{\pgfqpoint{1.020000in}{0.880000in}}{\pgfqpoint{6.160000in}{6.160000in}}%
\pgfusepath{clip}%
\pgfsetbuttcap%
\pgfsetroundjoin%
\definecolor{currentfill}{rgb}{0.266381,0.353304,0.801637}%
\pgfsetfillcolor{currentfill}%
\pgfsetlinewidth{0.000000pt}%
\definecolor{currentstroke}{rgb}{0.000000,0.000000,0.000000}%
\pgfsetstrokecolor{currentstroke}%
\pgfsetdash{}{0pt}%
\pgfpathmoveto{\pgfqpoint{5.528207in}{2.556431in}}%
\pgfpathlineto{\pgfqpoint{5.538939in}{2.532967in}}%
\pgfpathlineto{\pgfqpoint{5.550975in}{2.602708in}}%
\pgfpathlineto{\pgfqpoint{5.584267in}{2.557049in}}%
\pgfpathlineto{\pgfqpoint{5.618452in}{2.576669in}}%
\pgfpathlineto{\pgfqpoint{5.607164in}{2.566879in}}%
\pgfpathlineto{\pgfqpoint{5.595009in}{2.494309in}}%
\pgfpathlineto{\pgfqpoint{5.564613in}{2.739929in}}%
\pgfpathlineto{\pgfqpoint{5.528207in}{2.556431in}}%
\pgfpathclose%
\pgfusepath{fill}%
\end{pgfscope}%
\begin{pgfscope}%
\pgfpathrectangle{\pgfqpoint{1.020000in}{0.880000in}}{\pgfqpoint{6.160000in}{6.160000in}}%
\pgfusepath{clip}%
\pgfsetbuttcap%
\pgfsetroundjoin%
\definecolor{currentfill}{rgb}{0.570616,0.704109,0.997195}%
\pgfsetfillcolor{currentfill}%
\pgfsetlinewidth{0.000000pt}%
\definecolor{currentstroke}{rgb}{0.000000,0.000000,0.000000}%
\pgfsetstrokecolor{currentstroke}%
\pgfsetdash{}{0pt}%
\pgfpathmoveto{\pgfqpoint{2.740506in}{3.323846in}}%
\pgfpathlineto{\pgfqpoint{2.751022in}{3.149288in}}%
\pgfpathlineto{\pgfqpoint{2.759818in}{3.094334in}}%
\pgfpathlineto{\pgfqpoint{2.796503in}{2.956219in}}%
\pgfpathlineto{\pgfqpoint{2.829348in}{3.092125in}}%
\pgfpathlineto{\pgfqpoint{2.818369in}{3.304427in}}%
\pgfpathlineto{\pgfqpoint{2.812277in}{3.159949in}}%
\pgfpathlineto{\pgfqpoint{2.776526in}{3.235184in}}%
\pgfpathlineto{\pgfqpoint{2.740506in}{3.323846in}}%
\pgfpathclose%
\pgfusepath{fill}%
\end{pgfscope}%
\begin{pgfscope}%
\pgfpathrectangle{\pgfqpoint{1.020000in}{0.880000in}}{\pgfqpoint{6.160000in}{6.160000in}}%
\pgfusepath{clip}%
\pgfsetbuttcap%
\pgfsetroundjoin%
\definecolor{currentfill}{rgb}{0.280550,0.373423,0.818011}%
\pgfsetfillcolor{currentfill}%
\pgfsetlinewidth{0.000000pt}%
\definecolor{currentstroke}{rgb}{0.000000,0.000000,0.000000}%
\pgfsetstrokecolor{currentstroke}%
\pgfsetdash{}{0pt}%
\pgfpathmoveto{\pgfqpoint{5.821584in}{2.584097in}}%
\pgfpathlineto{\pgfqpoint{5.833570in}{2.620775in}}%
\pgfpathlineto{\pgfqpoint{5.845510in}{2.652698in}}%
\pgfpathlineto{\pgfqpoint{5.878623in}{2.609256in}}%
\pgfpathlineto{\pgfqpoint{5.912749in}{2.626837in}}%
\pgfpathlineto{\pgfqpoint{5.898110in}{2.444052in}}%
\pgfpathlineto{\pgfqpoint{5.889155in}{2.587731in}}%
\pgfpathlineto{\pgfqpoint{5.857504in}{2.711942in}}%
\pgfpathlineto{\pgfqpoint{5.821584in}{2.584097in}}%
\pgfpathclose%
\pgfusepath{fill}%
\end{pgfscope}%
\begin{pgfscope}%
\pgfpathrectangle{\pgfqpoint{1.020000in}{0.880000in}}{\pgfqpoint{6.160000in}{6.160000in}}%
\pgfusepath{clip}%
\pgfsetbuttcap%
\pgfsetroundjoin%
\definecolor{currentfill}{rgb}{0.430507,0.564883,0.948889}%
\pgfsetfillcolor{currentfill}%
\pgfsetlinewidth{0.000000pt}%
\definecolor{currentstroke}{rgb}{0.000000,0.000000,0.000000}%
\pgfsetstrokecolor{currentstroke}%
\pgfsetdash{}{0pt}%
\pgfpathmoveto{\pgfqpoint{4.834831in}{2.895533in}}%
\pgfpathlineto{\pgfqpoint{4.844926in}{2.866532in}}%
\pgfpathlineto{\pgfqpoint{4.855459in}{2.899860in}}%
\pgfpathlineto{\pgfqpoint{4.889655in}{2.907517in}}%
\pgfpathlineto{\pgfqpoint{4.924733in}{3.035673in}}%
\pgfpathlineto{\pgfqpoint{4.912077in}{2.727483in}}%
\pgfpathlineto{\pgfqpoint{4.903235in}{2.934282in}}%
\pgfpathlineto{\pgfqpoint{4.869593in}{2.995587in}}%
\pgfpathlineto{\pgfqpoint{4.834831in}{2.895533in}}%
\pgfpathclose%
\pgfusepath{fill}%
\end{pgfscope}%
\begin{pgfscope}%
\pgfpathrectangle{\pgfqpoint{1.020000in}{0.880000in}}{\pgfqpoint{6.160000in}{6.160000in}}%
\pgfusepath{clip}%
\pgfsetbuttcap%
\pgfsetroundjoin%
\definecolor{currentfill}{rgb}{0.831148,0.859513,0.903110}%
\pgfsetfillcolor{currentfill}%
\pgfsetlinewidth{0.000000pt}%
\definecolor{currentstroke}{rgb}{0.000000,0.000000,0.000000}%
\pgfsetstrokecolor{currentstroke}%
\pgfsetdash{}{0pt}%
\pgfpathmoveto{\pgfqpoint{3.584555in}{3.713175in}}%
\pgfpathlineto{\pgfqpoint{3.594337in}{3.567252in}}%
\pgfpathlineto{\pgfqpoint{3.602055in}{3.776551in}}%
\pgfpathlineto{\pgfqpoint{3.637209in}{3.667581in}}%
\pgfpathlineto{\pgfqpoint{3.672208in}{3.570208in}}%
\pgfpathlineto{\pgfqpoint{3.662032in}{3.796603in}}%
\pgfpathlineto{\pgfqpoint{3.654237in}{3.568178in}}%
\pgfpathlineto{\pgfqpoint{3.618599in}{3.788554in}}%
\pgfpathlineto{\pgfqpoint{3.584555in}{3.713175in}}%
\pgfpathclose%
\pgfusepath{fill}%
\end{pgfscope}%
\begin{pgfscope}%
\pgfpathrectangle{\pgfqpoint{1.020000in}{0.880000in}}{\pgfqpoint{6.160000in}{6.160000in}}%
\pgfusepath{clip}%
\pgfsetbuttcap%
\pgfsetroundjoin%
\definecolor{currentfill}{rgb}{0.624703,0.748318,0.998719}%
\pgfsetfillcolor{currentfill}%
\pgfsetlinewidth{0.000000pt}%
\definecolor{currentstroke}{rgb}{0.000000,0.000000,0.000000}%
\pgfsetstrokecolor{currentstroke}%
\pgfsetdash{}{0pt}%
\pgfpathmoveto{\pgfqpoint{4.434982in}{3.380400in}}%
\pgfpathlineto{\pgfqpoint{4.444447in}{3.229169in}}%
\pgfpathlineto{\pgfqpoint{4.454676in}{3.362482in}}%
\pgfpathlineto{\pgfqpoint{4.488589in}{3.234700in}}%
\pgfpathlineto{\pgfqpoint{4.522804in}{3.227684in}}%
\pgfpathlineto{\pgfqpoint{4.513198in}{3.327716in}}%
\pgfpathlineto{\pgfqpoint{4.502816in}{3.184801in}}%
\pgfpathlineto{\pgfqpoint{4.468586in}{3.157828in}}%
\pgfpathlineto{\pgfqpoint{4.434982in}{3.380400in}}%
\pgfpathclose%
\pgfusepath{fill}%
\end{pgfscope}%
\begin{pgfscope}%
\pgfpathrectangle{\pgfqpoint{1.020000in}{0.880000in}}{\pgfqpoint{6.160000in}{6.160000in}}%
\pgfusepath{clip}%
\pgfsetbuttcap%
\pgfsetroundjoin%
\definecolor{currentfill}{rgb}{0.338377,0.452819,0.879317}%
\pgfsetfillcolor{currentfill}%
\pgfsetlinewidth{0.000000pt}%
\definecolor{currentstroke}{rgb}{0.000000,0.000000,0.000000}%
\pgfsetstrokecolor{currentstroke}%
\pgfsetdash{}{0pt}%
\pgfpathmoveto{\pgfqpoint{5.148542in}{2.827116in}}%
\pgfpathlineto{\pgfqpoint{5.158351in}{2.741816in}}%
\pgfpathlineto{\pgfqpoint{5.168902in}{2.730820in}}%
\pgfpathlineto{\pgfqpoint{5.203020in}{2.737036in}}%
\pgfpathlineto{\pgfqpoint{5.237866in}{2.813491in}}%
\pgfpathlineto{\pgfqpoint{5.226079in}{2.713421in}}%
\pgfpathlineto{\pgfqpoint{5.213834in}{2.565087in}}%
\pgfpathlineto{\pgfqpoint{5.180892in}{2.661094in}}%
\pgfpathlineto{\pgfqpoint{5.148542in}{2.827116in}}%
\pgfpathclose%
\pgfusepath{fill}%
\end{pgfscope}%
\begin{pgfscope}%
\pgfpathrectangle{\pgfqpoint{1.020000in}{0.880000in}}{\pgfqpoint{6.160000in}{6.160000in}}%
\pgfusepath{clip}%
\pgfsetbuttcap%
\pgfsetroundjoin%
\definecolor{currentfill}{rgb}{0.489246,0.627536,0.976896}%
\pgfsetfillcolor{currentfill}%
\pgfsetlinewidth{0.000000pt}%
\definecolor{currentstroke}{rgb}{0.000000,0.000000,0.000000}%
\pgfsetstrokecolor{currentstroke}%
\pgfsetdash{}{0pt}%
\pgfpathmoveto{\pgfqpoint{4.678979in}{3.071976in}}%
\pgfpathlineto{\pgfqpoint{4.689962in}{3.241372in}}%
\pgfpathlineto{\pgfqpoint{4.699154in}{3.053703in}}%
\pgfpathlineto{\pgfqpoint{4.732349in}{2.870029in}}%
\pgfpathlineto{\pgfqpoint{4.766966in}{2.953663in}}%
\pgfpathlineto{\pgfqpoint{4.757054in}{3.004184in}}%
\pgfpathlineto{\pgfqpoint{4.745663in}{2.791137in}}%
\pgfpathlineto{\pgfqpoint{4.713485in}{3.139498in}}%
\pgfpathlineto{\pgfqpoint{4.678979in}{3.071976in}}%
\pgfpathclose%
\pgfusepath{fill}%
\end{pgfscope}%
\begin{pgfscope}%
\pgfpathrectangle{\pgfqpoint{1.020000in}{0.880000in}}{\pgfqpoint{6.160000in}{6.160000in}}%
\pgfusepath{clip}%
\pgfsetbuttcap%
\pgfsetroundjoin%
\definecolor{currentfill}{rgb}{0.516260,0.654498,0.986407}%
\pgfsetfillcolor{currentfill}%
\pgfsetlinewidth{0.000000pt}%
\definecolor{currentstroke}{rgb}{0.000000,0.000000,0.000000}%
\pgfsetstrokecolor{currentstroke}%
\pgfsetdash{}{0pt}%
\pgfpathmoveto{\pgfqpoint{2.607578in}{2.922177in}}%
\pgfpathlineto{\pgfqpoint{2.611853in}{3.147932in}}%
\pgfpathlineto{\pgfqpoint{2.620804in}{3.078903in}}%
\pgfpathlineto{\pgfqpoint{2.656745in}{3.007110in}}%
\pgfpathlineto{\pgfqpoint{2.690062in}{3.105725in}}%
\pgfpathlineto{\pgfqpoint{2.679718in}{3.265589in}}%
\pgfpathlineto{\pgfqpoint{2.673846in}{3.129500in}}%
\pgfpathlineto{\pgfqpoint{2.643283in}{2.858157in}}%
\pgfpathlineto{\pgfqpoint{2.607578in}{2.922177in}}%
\pgfpathclose%
\pgfusepath{fill}%
\end{pgfscope}%
\begin{pgfscope}%
\pgfpathrectangle{\pgfqpoint{1.020000in}{0.880000in}}{\pgfqpoint{6.160000in}{6.160000in}}%
\pgfusepath{clip}%
\pgfsetbuttcap%
\pgfsetroundjoin%
\definecolor{currentfill}{rgb}{0.394042,0.522413,0.924916}%
\pgfsetfillcolor{currentfill}%
\pgfsetlinewidth{0.000000pt}%
\definecolor{currentstroke}{rgb}{0.000000,0.000000,0.000000}%
\pgfsetstrokecolor{currentstroke}%
\pgfsetdash{}{0pt}%
\pgfpathmoveto{\pgfqpoint{4.990473in}{2.724090in}}%
\pgfpathlineto{\pgfqpoint{5.002785in}{2.950017in}}%
\pgfpathlineto{\pgfqpoint{5.011899in}{2.783848in}}%
\pgfpathlineto{\pgfqpoint{5.045204in}{2.693811in}}%
\pgfpathlineto{\pgfqpoint{5.079045in}{2.672267in}}%
\pgfpathlineto{\pgfqpoint{5.070659in}{2.913518in}}%
\pgfpathlineto{\pgfqpoint{5.060292in}{2.933291in}}%
\pgfpathlineto{\pgfqpoint{5.026716in}{2.989062in}}%
\pgfpathlineto{\pgfqpoint{4.990473in}{2.724090in}}%
\pgfpathclose%
\pgfusepath{fill}%
\end{pgfscope}%
\begin{pgfscope}%
\pgfpathrectangle{\pgfqpoint{1.020000in}{0.880000in}}{\pgfqpoint{6.160000in}{6.160000in}}%
\pgfusepath{clip}%
\pgfsetbuttcap%
\pgfsetroundjoin%
\definecolor{currentfill}{rgb}{0.855378,0.863778,0.876587}%
\pgfsetfillcolor{currentfill}%
\pgfsetlinewidth{0.000000pt}%
\definecolor{currentstroke}{rgb}{0.000000,0.000000,0.000000}%
\pgfsetstrokecolor{currentstroke}%
\pgfsetdash{}{0pt}%
\pgfpathmoveto{\pgfqpoint{3.878500in}{3.595949in}}%
\pgfpathlineto{\pgfqpoint{3.887479in}{3.681020in}}%
\pgfpathlineto{\pgfqpoint{3.896356in}{3.814568in}}%
\pgfpathlineto{\pgfqpoint{3.930726in}{3.856538in}}%
\pgfpathlineto{\pgfqpoint{3.965216in}{3.846775in}}%
\pgfpathlineto{\pgfqpoint{3.956480in}{3.578255in}}%
\pgfpathlineto{\pgfqpoint{3.946755in}{3.759490in}}%
\pgfpathlineto{\pgfqpoint{3.912299in}{3.792948in}}%
\pgfpathlineto{\pgfqpoint{3.878500in}{3.595949in}}%
\pgfpathclose%
\pgfusepath{fill}%
\end{pgfscope}%
\begin{pgfscope}%
\pgfpathrectangle{\pgfqpoint{1.020000in}{0.880000in}}{\pgfqpoint{6.160000in}{6.160000in}}%
\pgfusepath{clip}%
\pgfsetbuttcap%
\pgfsetroundjoin%
\definecolor{currentfill}{rgb}{0.818056,0.855590,0.914638}%
\pgfsetfillcolor{currentfill}%
\pgfsetlinewidth{0.000000pt}%
\definecolor{currentstroke}{rgb}{0.000000,0.000000,0.000000}%
\pgfsetstrokecolor{currentstroke}%
\pgfsetdash{}{0pt}%
\pgfpathmoveto{\pgfqpoint{4.034387in}{3.642754in}}%
\pgfpathlineto{\pgfqpoint{4.043639in}{3.758809in}}%
\pgfpathlineto{\pgfqpoint{4.053250in}{3.631317in}}%
\pgfpathlineto{\pgfqpoint{4.087517in}{3.768203in}}%
\pgfpathlineto{\pgfqpoint{4.122086in}{3.530542in}}%
\pgfpathlineto{\pgfqpoint{4.112532in}{3.617445in}}%
\pgfpathlineto{\pgfqpoint{4.103054in}{3.613739in}}%
\pgfpathlineto{\pgfqpoint{4.068765in}{3.602649in}}%
\pgfpathlineto{\pgfqpoint{4.034387in}{3.642754in}}%
\pgfpathclose%
\pgfusepath{fill}%
\end{pgfscope}%
\begin{pgfscope}%
\pgfpathrectangle{\pgfqpoint{1.020000in}{0.880000in}}{\pgfqpoint{6.160000in}{6.160000in}}%
\pgfusepath{clip}%
\pgfsetbuttcap%
\pgfsetroundjoin%
\definecolor{currentfill}{rgb}{0.738826,0.822572,0.968261}%
\pgfsetfillcolor{currentfill}%
\pgfsetlinewidth{0.000000pt}%
\definecolor{currentstroke}{rgb}{0.000000,0.000000,0.000000}%
\pgfsetstrokecolor{currentstroke}%
\pgfsetdash{}{0pt}%
\pgfpathmoveto{\pgfqpoint{3.311782in}{3.285511in}}%
\pgfpathlineto{\pgfqpoint{3.318402in}{3.515423in}}%
\pgfpathlineto{\pgfqpoint{3.326883in}{3.534630in}}%
\pgfpathlineto{\pgfqpoint{3.362507in}{3.415424in}}%
\pgfpathlineto{\pgfqpoint{3.396758in}{3.457596in}}%
\pgfpathlineto{\pgfqpoint{3.387115in}{3.574952in}}%
\pgfpathlineto{\pgfqpoint{3.378191in}{3.602511in}}%
\pgfpathlineto{\pgfqpoint{3.344815in}{3.459230in}}%
\pgfpathlineto{\pgfqpoint{3.311782in}{3.285511in}}%
\pgfpathclose%
\pgfusepath{fill}%
\end{pgfscope}%
\begin{pgfscope}%
\pgfpathrectangle{\pgfqpoint{1.020000in}{0.880000in}}{\pgfqpoint{6.160000in}{6.160000in}}%
\pgfusepath{clip}%
\pgfsetbuttcap%
\pgfsetroundjoin%
\definecolor{currentfill}{rgb}{0.289996,0.386836,0.828926}%
\pgfsetfillcolor{currentfill}%
\pgfsetlinewidth{0.000000pt}%
\definecolor{currentstroke}{rgb}{0.000000,0.000000,0.000000}%
\pgfsetstrokecolor{currentstroke}%
\pgfsetdash{}{0pt}%
\pgfpathmoveto{\pgfqpoint{6.049897in}{2.736005in}}%
\pgfpathlineto{\pgfqpoint{6.061405in}{2.728736in}}%
\pgfpathlineto{\pgfqpoint{6.075570in}{2.859788in}}%
\pgfpathlineto{\pgfqpoint{6.103038in}{2.529362in}}%
\pgfpathlineto{\pgfqpoint{6.138095in}{2.596432in}}%
\pgfpathlineto{\pgfqpoint{6.125224in}{2.537809in}}%
\pgfpathlineto{\pgfqpoint{6.110737in}{2.394168in}}%
\pgfpathlineto{\pgfqpoint{6.080395in}{2.565587in}}%
\pgfpathlineto{\pgfqpoint{6.049897in}{2.736005in}}%
\pgfpathclose%
\pgfusepath{fill}%
\end{pgfscope}%
\begin{pgfscope}%
\pgfpathrectangle{\pgfqpoint{1.020000in}{0.880000in}}{\pgfqpoint{6.160000in}{6.160000in}}%
\pgfusepath{clip}%
\pgfsetbuttcap%
\pgfsetroundjoin%
\definecolor{currentfill}{rgb}{0.280550,0.373423,0.818011}%
\pgfsetfillcolor{currentfill}%
\pgfsetlinewidth{0.000000pt}%
\definecolor{currentstroke}{rgb}{0.000000,0.000000,0.000000}%
\pgfsetstrokecolor{currentstroke}%
\pgfsetdash{}{0pt}%
\pgfpathmoveto{\pgfqpoint{5.755356in}{2.670157in}}%
\pgfpathlineto{\pgfqpoint{5.765266in}{2.581963in}}%
\pgfpathlineto{\pgfqpoint{5.775396in}{2.507059in}}%
\pgfpathlineto{\pgfqpoint{5.809055in}{2.495581in}}%
\pgfpathlineto{\pgfqpoint{5.845510in}{2.652698in}}%
\pgfpathlineto{\pgfqpoint{5.833570in}{2.620775in}}%
\pgfpathlineto{\pgfqpoint{5.821584in}{2.584097in}}%
\pgfpathlineto{\pgfqpoint{5.788828in}{2.647715in}}%
\pgfpathlineto{\pgfqpoint{5.755356in}{2.670157in}}%
\pgfpathclose%
\pgfusepath{fill}%
\end{pgfscope}%
\begin{pgfscope}%
\pgfpathrectangle{\pgfqpoint{1.020000in}{0.880000in}}{\pgfqpoint{6.160000in}{6.160000in}}%
\pgfusepath{clip}%
\pgfsetbuttcap%
\pgfsetroundjoin%
\definecolor{currentfill}{rgb}{0.826784,0.858205,0.906953}%
\pgfsetfillcolor{currentfill}%
\pgfsetlinewidth{0.000000pt}%
\definecolor{currentstroke}{rgb}{0.000000,0.000000,0.000000}%
\pgfsetstrokecolor{currentstroke}%
\pgfsetdash{}{0pt}%
\pgfpathmoveto{\pgfqpoint{3.516289in}{3.602812in}}%
\pgfpathlineto{\pgfqpoint{3.525141in}{3.601659in}}%
\pgfpathlineto{\pgfqpoint{3.533116in}{3.738769in}}%
\pgfpathlineto{\pgfqpoint{3.568352in}{3.633374in}}%
\pgfpathlineto{\pgfqpoint{3.602055in}{3.776551in}}%
\pgfpathlineto{\pgfqpoint{3.594337in}{3.567252in}}%
\pgfpathlineto{\pgfqpoint{3.584555in}{3.713175in}}%
\pgfpathlineto{\pgfqpoint{3.549986in}{3.725601in}}%
\pgfpathlineto{\pgfqpoint{3.516289in}{3.602812in}}%
\pgfpathclose%
\pgfusepath{fill}%
\end{pgfscope}%
\begin{pgfscope}%
\pgfpathrectangle{\pgfqpoint{1.020000in}{0.880000in}}{\pgfqpoint{6.160000in}{6.160000in}}%
\pgfusepath{clip}%
\pgfsetbuttcap%
\pgfsetroundjoin%
\definecolor{currentfill}{rgb}{0.271104,0.360011,0.807095}%
\pgfsetfillcolor{currentfill}%
\pgfsetlinewidth{0.000000pt}%
\definecolor{currentstroke}{rgb}{0.000000,0.000000,0.000000}%
\pgfsetstrokecolor{currentstroke}%
\pgfsetdash{}{0pt}%
\pgfpathmoveto{\pgfqpoint{6.271416in}{2.525013in}}%
\pgfpathlineto{\pgfqpoint{6.282877in}{2.505520in}}%
\pgfpathlineto{\pgfqpoint{6.297258in}{2.622035in}}%
\pgfpathlineto{\pgfqpoint{6.330869in}{2.616536in}}%
\pgfpathlineto{\pgfqpoint{6.318439in}{2.593968in}}%
\pgfpathlineto{\pgfqpoint{6.306225in}{2.580033in}}%
\pgfpathlineto{\pgfqpoint{6.271416in}{2.525013in}}%
\pgfpathclose%
\pgfusepath{fill}%
\end{pgfscope}%
\begin{pgfscope}%
\pgfpathrectangle{\pgfqpoint{1.020000in}{0.880000in}}{\pgfqpoint{6.160000in}{6.160000in}}%
\pgfusepath{clip}%
\pgfsetbuttcap%
\pgfsetroundjoin%
\definecolor{currentfill}{rgb}{0.843358,0.861820,0.890017}%
\pgfsetfillcolor{currentfill}%
\pgfsetlinewidth{0.000000pt}%
\definecolor{currentstroke}{rgb}{0.000000,0.000000,0.000000}%
\pgfsetstrokecolor{currentstroke}%
\pgfsetdash{}{0pt}%
\pgfpathmoveto{\pgfqpoint{3.809440in}{3.685853in}}%
\pgfpathlineto{\pgfqpoint{3.818416in}{3.739107in}}%
\pgfpathlineto{\pgfqpoint{3.827478in}{3.776658in}}%
\pgfpathlineto{\pgfqpoint{3.862475in}{3.622596in}}%
\pgfpathlineto{\pgfqpoint{3.896356in}{3.814568in}}%
\pgfpathlineto{\pgfqpoint{3.887479in}{3.681020in}}%
\pgfpathlineto{\pgfqpoint{3.878500in}{3.595949in}}%
\pgfpathlineto{\pgfqpoint{3.843825in}{3.691353in}}%
\pgfpathlineto{\pgfqpoint{3.809440in}{3.685853in}}%
\pgfpathclose%
\pgfusepath{fill}%
\end{pgfscope}%
\begin{pgfscope}%
\pgfpathrectangle{\pgfqpoint{1.020000in}{0.880000in}}{\pgfqpoint{6.160000in}{6.160000in}}%
\pgfusepath{clip}%
\pgfsetbuttcap%
\pgfsetroundjoin%
\definecolor{currentfill}{rgb}{0.576051,0.708780,0.997755}%
\pgfsetfillcolor{currentfill}%
\pgfsetlinewidth{0.000000pt}%
\definecolor{currentstroke}{rgb}{0.000000,0.000000,0.000000}%
\pgfsetstrokecolor{currentstroke}%
\pgfsetdash{}{0pt}%
\pgfpathmoveto{\pgfqpoint{2.673846in}{3.129500in}}%
\pgfpathlineto{\pgfqpoint{2.679718in}{3.265589in}}%
\pgfpathlineto{\pgfqpoint{2.690062in}{3.105725in}}%
\pgfpathlineto{\pgfqpoint{2.724328in}{3.142707in}}%
\pgfpathlineto{\pgfqpoint{2.759818in}{3.094334in}}%
\pgfpathlineto{\pgfqpoint{2.751022in}{3.149288in}}%
\pgfpathlineto{\pgfqpoint{2.740506in}{3.323846in}}%
\pgfpathlineto{\pgfqpoint{2.708972in}{3.103708in}}%
\pgfpathlineto{\pgfqpoint{2.673846in}{3.129500in}}%
\pgfpathclose%
\pgfusepath{fill}%
\end{pgfscope}%
\begin{pgfscope}%
\pgfpathrectangle{\pgfqpoint{1.020000in}{0.880000in}}{\pgfqpoint{6.160000in}{6.160000in}}%
\pgfusepath{clip}%
\pgfsetbuttcap%
\pgfsetroundjoin%
\definecolor{currentfill}{rgb}{0.576051,0.708780,0.997755}%
\pgfsetfillcolor{currentfill}%
\pgfsetlinewidth{0.000000pt}%
\definecolor{currentstroke}{rgb}{0.000000,0.000000,0.000000}%
\pgfsetstrokecolor{currentstroke}%
\pgfsetdash{}{0pt}%
\pgfpathmoveto{\pgfqpoint{4.522804in}{3.227684in}}%
\pgfpathlineto{\pgfqpoint{4.532787in}{3.234917in}}%
\pgfpathlineto{\pgfqpoint{4.541894in}{2.992506in}}%
\pgfpathlineto{\pgfqpoint{4.576928in}{3.203043in}}%
\pgfpathlineto{\pgfqpoint{4.610908in}{3.138775in}}%
\pgfpathlineto{\pgfqpoint{4.601015in}{3.173737in}}%
\pgfpathlineto{\pgfqpoint{4.590964in}{3.166427in}}%
\pgfpathlineto{\pgfqpoint{4.556761in}{3.158747in}}%
\pgfpathlineto{\pgfqpoint{4.522804in}{3.227684in}}%
\pgfpathclose%
\pgfusepath{fill}%
\end{pgfscope}%
\begin{pgfscope}%
\pgfpathrectangle{\pgfqpoint{1.020000in}{0.880000in}}{\pgfqpoint{6.160000in}{6.160000in}}%
\pgfusepath{clip}%
\pgfsetbuttcap%
\pgfsetroundjoin%
\definecolor{currentfill}{rgb}{0.441123,0.576532,0.954545}%
\pgfsetfillcolor{currentfill}%
\pgfsetlinewidth{0.000000pt}%
\definecolor{currentstroke}{rgb}{0.000000,0.000000,0.000000}%
\pgfsetstrokecolor{currentstroke}%
\pgfsetdash{}{0pt}%
\pgfpathmoveto{\pgfqpoint{4.766966in}{2.953663in}}%
\pgfpathlineto{\pgfqpoint{4.776907in}{2.906211in}}%
\pgfpathlineto{\pgfqpoint{4.787100in}{2.898013in}}%
\pgfpathlineto{\pgfqpoint{4.821563in}{2.941583in}}%
\pgfpathlineto{\pgfqpoint{4.855459in}{2.899860in}}%
\pgfpathlineto{\pgfqpoint{4.844926in}{2.866532in}}%
\pgfpathlineto{\pgfqpoint{4.834831in}{2.895533in}}%
\pgfpathlineto{\pgfqpoint{4.801085in}{2.951356in}}%
\pgfpathlineto{\pgfqpoint{4.766966in}{2.953663in}}%
\pgfpathclose%
\pgfusepath{fill}%
\end{pgfscope}%
\begin{pgfscope}%
\pgfpathrectangle{\pgfqpoint{1.020000in}{0.880000in}}{\pgfqpoint{6.160000in}{6.160000in}}%
\pgfusepath{clip}%
\pgfsetbuttcap%
\pgfsetroundjoin%
\definecolor{currentfill}{rgb}{0.257234,0.339661,0.789661}%
\pgfsetfillcolor{currentfill}%
\pgfsetlinewidth{0.000000pt}%
\definecolor{currentstroke}{rgb}{0.000000,0.000000,0.000000}%
\pgfsetstrokecolor{currentstroke}%
\pgfsetdash{}{0pt}%
\pgfpathmoveto{\pgfqpoint{6.205896in}{2.613415in}}%
\pgfpathlineto{\pgfqpoint{6.214895in}{2.475993in}}%
\pgfpathlineto{\pgfqpoint{6.226821in}{2.480787in}}%
\pgfpathlineto{\pgfqpoint{6.263157in}{2.605344in}}%
\pgfpathlineto{\pgfqpoint{6.297258in}{2.622035in}}%
\pgfpathlineto{\pgfqpoint{6.282877in}{2.505520in}}%
\pgfpathlineto{\pgfqpoint{6.271416in}{2.525013in}}%
\pgfpathlineto{\pgfqpoint{6.238109in}{2.541618in}}%
\pgfpathlineto{\pgfqpoint{6.205896in}{2.613415in}}%
\pgfpathclose%
\pgfusepath{fill}%
\end{pgfscope}%
\begin{pgfscope}%
\pgfpathrectangle{\pgfqpoint{1.020000in}{0.880000in}}{\pgfqpoint{6.160000in}{6.160000in}}%
\pgfusepath{clip}%
\pgfsetbuttcap%
\pgfsetroundjoin%
\definecolor{currentfill}{rgb}{0.280550,0.373423,0.818011}%
\pgfsetfillcolor{currentfill}%
\pgfsetlinewidth{0.000000pt}%
\definecolor{currentstroke}{rgb}{0.000000,0.000000,0.000000}%
\pgfsetstrokecolor{currentstroke}%
\pgfsetdash{}{0pt}%
\pgfpathmoveto{\pgfqpoint{5.461470in}{2.636604in}}%
\pgfpathlineto{\pgfqpoint{5.472809in}{2.663129in}}%
\pgfpathlineto{\pgfqpoint{5.482630in}{2.571592in}}%
\pgfpathlineto{\pgfqpoint{5.515689in}{2.503959in}}%
\pgfpathlineto{\pgfqpoint{5.550975in}{2.602708in}}%
\pgfpathlineto{\pgfqpoint{5.538939in}{2.532967in}}%
\pgfpathlineto{\pgfqpoint{5.528207in}{2.556431in}}%
\pgfpathlineto{\pgfqpoint{5.496598in}{2.727721in}}%
\pgfpathlineto{\pgfqpoint{5.461470in}{2.636604in}}%
\pgfpathclose%
\pgfusepath{fill}%
\end{pgfscope}%
\begin{pgfscope}%
\pgfpathrectangle{\pgfqpoint{1.020000in}{0.880000in}}{\pgfqpoint{6.160000in}{6.160000in}}%
\pgfusepath{clip}%
\pgfsetbuttcap%
\pgfsetroundjoin%
\definecolor{currentfill}{rgb}{0.635474,0.756714,0.998297}%
\pgfsetfillcolor{currentfill}%
\pgfsetlinewidth{0.000000pt}%
\definecolor{currentstroke}{rgb}{0.000000,0.000000,0.000000}%
\pgfsetstrokecolor{currentstroke}%
\pgfsetdash{}{0pt}%
\pgfpathmoveto{\pgfqpoint{3.034600in}{3.329968in}}%
\pgfpathlineto{\pgfqpoint{3.044559in}{3.187180in}}%
\pgfpathlineto{\pgfqpoint{3.052554in}{3.217172in}}%
\pgfpathlineto{\pgfqpoint{3.086672in}{3.270601in}}%
\pgfpathlineto{\pgfqpoint{3.121880in}{3.222974in}}%
\pgfpathlineto{\pgfqpoint{3.111913in}{3.368365in}}%
\pgfpathlineto{\pgfqpoint{3.104387in}{3.287205in}}%
\pgfpathlineto{\pgfqpoint{3.069535in}{3.306498in}}%
\pgfpathlineto{\pgfqpoint{3.034600in}{3.329968in}}%
\pgfpathclose%
\pgfusepath{fill}%
\end{pgfscope}%
\begin{pgfscope}%
\pgfpathrectangle{\pgfqpoint{1.020000in}{0.880000in}}{\pgfqpoint{6.160000in}{6.160000in}}%
\pgfusepath{clip}%
\pgfsetbuttcap%
\pgfsetroundjoin%
\definecolor{currentfill}{rgb}{0.813693,0.854282,0.918480}%
\pgfsetfillcolor{currentfill}%
\pgfsetlinewidth{0.000000pt}%
\definecolor{currentstroke}{rgb}{0.000000,0.000000,0.000000}%
\pgfsetstrokecolor{currentstroke}%
\pgfsetdash{}{0pt}%
\pgfpathmoveto{\pgfqpoint{3.965216in}{3.846775in}}%
\pgfpathlineto{\pgfqpoint{3.974961in}{3.663334in}}%
\pgfpathlineto{\pgfqpoint{3.984553in}{3.549871in}}%
\pgfpathlineto{\pgfqpoint{4.018953in}{3.557250in}}%
\pgfpathlineto{\pgfqpoint{4.053250in}{3.631317in}}%
\pgfpathlineto{\pgfqpoint{4.043639in}{3.758809in}}%
\pgfpathlineto{\pgfqpoint{4.034387in}{3.642754in}}%
\pgfpathlineto{\pgfqpoint{4.000492in}{3.391552in}}%
\pgfpathlineto{\pgfqpoint{3.965216in}{3.846775in}}%
\pgfpathclose%
\pgfusepath{fill}%
\end{pgfscope}%
\begin{pgfscope}%
\pgfpathrectangle{\pgfqpoint{1.020000in}{0.880000in}}{\pgfqpoint{6.160000in}{6.160000in}}%
\pgfusepath{clip}%
\pgfsetbuttcap%
\pgfsetroundjoin%
\definecolor{currentfill}{rgb}{0.818056,0.855590,0.914638}%
\pgfsetfillcolor{currentfill}%
\pgfsetlinewidth{0.000000pt}%
\definecolor{currentstroke}{rgb}{0.000000,0.000000,0.000000}%
\pgfsetstrokecolor{currentstroke}%
\pgfsetdash{}{0pt}%
\pgfpathmoveto{\pgfqpoint{3.446392in}{3.723788in}}%
\pgfpathlineto{\pgfqpoint{3.455403in}{3.692671in}}%
\pgfpathlineto{\pgfqpoint{3.463913in}{3.733312in}}%
\pgfpathlineto{\pgfqpoint{3.499713in}{3.561831in}}%
\pgfpathlineto{\pgfqpoint{3.533116in}{3.738769in}}%
\pgfpathlineto{\pgfqpoint{3.525141in}{3.601659in}}%
\pgfpathlineto{\pgfqpoint{3.516289in}{3.602812in}}%
\pgfpathlineto{\pgfqpoint{3.482416in}{3.514225in}}%
\pgfpathlineto{\pgfqpoint{3.446392in}{3.723788in}}%
\pgfpathclose%
\pgfusepath{fill}%
\end{pgfscope}%
\begin{pgfscope}%
\pgfpathrectangle{\pgfqpoint{1.020000in}{0.880000in}}{\pgfqpoint{6.160000in}{6.160000in}}%
\pgfusepath{clip}%
\pgfsetbuttcap%
\pgfsetroundjoin%
\definecolor{currentfill}{rgb}{0.724041,0.814910,0.975651}%
\pgfsetfillcolor{currentfill}%
\pgfsetlinewidth{0.000000pt}%
\definecolor{currentstroke}{rgb}{0.000000,0.000000,0.000000}%
\pgfsetstrokecolor{currentstroke}%
\pgfsetdash{}{0pt}%
\pgfpathmoveto{\pgfqpoint{4.278514in}{3.469213in}}%
\pgfpathlineto{\pgfqpoint{4.288284in}{3.547156in}}%
\pgfpathlineto{\pgfqpoint{4.297932in}{3.486989in}}%
\pgfpathlineto{\pgfqpoint{4.332432in}{3.579992in}}%
\pgfpathlineto{\pgfqpoint{4.366306in}{3.308420in}}%
\pgfpathlineto{\pgfqpoint{4.356674in}{3.383987in}}%
\pgfpathlineto{\pgfqpoint{4.346716in}{3.256213in}}%
\pgfpathlineto{\pgfqpoint{4.312853in}{3.510508in}}%
\pgfpathlineto{\pgfqpoint{4.278514in}{3.469213in}}%
\pgfpathclose%
\pgfusepath{fill}%
\end{pgfscope}%
\begin{pgfscope}%
\pgfpathrectangle{\pgfqpoint{1.020000in}{0.880000in}}{\pgfqpoint{6.160000in}{6.160000in}}%
\pgfusepath{clip}%
\pgfsetbuttcap%
\pgfsetroundjoin%
\definecolor{currentfill}{rgb}{0.738826,0.822572,0.968261}%
\pgfsetfillcolor{currentfill}%
\pgfsetlinewidth{0.000000pt}%
\definecolor{currentstroke}{rgb}{0.000000,0.000000,0.000000}%
\pgfsetstrokecolor{currentstroke}%
\pgfsetdash{}{0pt}%
\pgfpathmoveto{\pgfqpoint{3.240575in}{3.515421in}}%
\pgfpathlineto{\pgfqpoint{3.249061in}{3.522902in}}%
\pgfpathlineto{\pgfqpoint{3.258065in}{3.476660in}}%
\pgfpathlineto{\pgfqpoint{3.293464in}{3.394492in}}%
\pgfpathlineto{\pgfqpoint{3.326883in}{3.534630in}}%
\pgfpathlineto{\pgfqpoint{3.318402in}{3.515423in}}%
\pgfpathlineto{\pgfqpoint{3.311782in}{3.285511in}}%
\pgfpathlineto{\pgfqpoint{3.275200in}{3.513884in}}%
\pgfpathlineto{\pgfqpoint{3.240575in}{3.515421in}}%
\pgfpathclose%
\pgfusepath{fill}%
\end{pgfscope}%
\begin{pgfscope}%
\pgfpathrectangle{\pgfqpoint{1.020000in}{0.880000in}}{\pgfqpoint{6.160000in}{6.160000in}}%
\pgfusepath{clip}%
\pgfsetbuttcap%
\pgfsetroundjoin%
\definecolor{currentfill}{rgb}{0.363461,0.484784,0.901019}%
\pgfsetfillcolor{currentfill}%
\pgfsetlinewidth{0.000000pt}%
\definecolor{currentstroke}{rgb}{0.000000,0.000000,0.000000}%
\pgfsetstrokecolor{currentstroke}%
\pgfsetdash{}{0pt}%
\pgfpathmoveto{\pgfqpoint{5.079045in}{2.672267in}}%
\pgfpathlineto{\pgfqpoint{5.089553in}{2.666462in}}%
\pgfpathlineto{\pgfqpoint{5.101010in}{2.760963in}}%
\pgfpathlineto{\pgfqpoint{5.136030in}{2.856333in}}%
\pgfpathlineto{\pgfqpoint{5.168902in}{2.730820in}}%
\pgfpathlineto{\pgfqpoint{5.158351in}{2.741816in}}%
\pgfpathlineto{\pgfqpoint{5.148542in}{2.827116in}}%
\pgfpathlineto{\pgfqpoint{5.114668in}{2.844711in}}%
\pgfpathlineto{\pgfqpoint{5.079045in}{2.672267in}}%
\pgfpathclose%
\pgfusepath{fill}%
\end{pgfscope}%
\begin{pgfscope}%
\pgfpathrectangle{\pgfqpoint{1.020000in}{0.880000in}}{\pgfqpoint{6.160000in}{6.160000in}}%
\pgfusepath{clip}%
\pgfsetbuttcap%
\pgfsetroundjoin%
\definecolor{currentfill}{rgb}{0.266381,0.353304,0.801637}%
\pgfsetfillcolor{currentfill}%
\pgfsetlinewidth{0.000000pt}%
\definecolor{currentstroke}{rgb}{0.000000,0.000000,0.000000}%
\pgfsetstrokecolor{currentstroke}%
\pgfsetdash{}{0pt}%
\pgfpathmoveto{\pgfqpoint{5.912749in}{2.626837in}}%
\pgfpathlineto{\pgfqpoint{5.924057in}{2.616008in}}%
\pgfpathlineto{\pgfqpoint{5.937992in}{2.752555in}}%
\pgfpathlineto{\pgfqpoint{5.968345in}{2.554704in}}%
\pgfpathlineto{\pgfqpoint{5.999001in}{2.381886in}}%
\pgfpathlineto{\pgfqpoint{5.990176in}{2.532049in}}%
\pgfpathlineto{\pgfqpoint{5.978631in}{2.532598in}}%
\pgfpathlineto{\pgfqpoint{5.943828in}{2.473492in}}%
\pgfpathlineto{\pgfqpoint{5.912749in}{2.626837in}}%
\pgfpathclose%
\pgfusepath{fill}%
\end{pgfscope}%
\begin{pgfscope}%
\pgfpathrectangle{\pgfqpoint{1.020000in}{0.880000in}}{\pgfqpoint{6.160000in}{6.160000in}}%
\pgfusepath{clip}%
\pgfsetbuttcap%
\pgfsetroundjoin%
\definecolor{currentfill}{rgb}{0.782049,0.842864,0.942980}%
\pgfsetfillcolor{currentfill}%
\pgfsetlinewidth{0.000000pt}%
\definecolor{currentstroke}{rgb}{0.000000,0.000000,0.000000}%
\pgfsetstrokecolor{currentstroke}%
\pgfsetdash{}{0pt}%
\pgfpathmoveto{\pgfqpoint{4.122086in}{3.530542in}}%
\pgfpathlineto{\pgfqpoint{4.131566in}{3.597441in}}%
\pgfpathlineto{\pgfqpoint{4.141050in}{3.732840in}}%
\pgfpathlineto{\pgfqpoint{4.175525in}{3.547476in}}%
\pgfpathlineto{\pgfqpoint{4.209890in}{3.509267in}}%
\pgfpathlineto{\pgfqpoint{4.200272in}{3.414583in}}%
\pgfpathlineto{\pgfqpoint{4.190710in}{3.553749in}}%
\pgfpathlineto{\pgfqpoint{4.156387in}{3.607274in}}%
\pgfpathlineto{\pgfqpoint{4.122086in}{3.530542in}}%
\pgfpathclose%
\pgfusepath{fill}%
\end{pgfscope}%
\begin{pgfscope}%
\pgfpathrectangle{\pgfqpoint{1.020000in}{0.880000in}}{\pgfqpoint{6.160000in}{6.160000in}}%
\pgfusepath{clip}%
\pgfsetbuttcap%
\pgfsetroundjoin%
\definecolor{currentfill}{rgb}{0.414801,0.546874,0.939088}%
\pgfsetfillcolor{currentfill}%
\pgfsetlinewidth{0.000000pt}%
\definecolor{currentstroke}{rgb}{0.000000,0.000000,0.000000}%
\pgfsetstrokecolor{currentstroke}%
\pgfsetdash{}{0pt}%
\pgfpathmoveto{\pgfqpoint{4.924733in}{3.035673in}}%
\pgfpathlineto{\pgfqpoint{4.933015in}{2.752902in}}%
\pgfpathlineto{\pgfqpoint{4.944404in}{2.880937in}}%
\pgfpathlineto{\pgfqpoint{4.978129in}{2.826135in}}%
\pgfpathlineto{\pgfqpoint{5.011899in}{2.783848in}}%
\pgfpathlineto{\pgfqpoint{5.002785in}{2.950017in}}%
\pgfpathlineto{\pgfqpoint{4.990473in}{2.724090in}}%
\pgfpathlineto{\pgfqpoint{4.958146in}{2.941435in}}%
\pgfpathlineto{\pgfqpoint{4.924733in}{3.035673in}}%
\pgfpathclose%
\pgfusepath{fill}%
\end{pgfscope}%
\begin{pgfscope}%
\pgfpathrectangle{\pgfqpoint{1.020000in}{0.880000in}}{\pgfqpoint{6.160000in}{6.160000in}}%
\pgfusepath{clip}%
\pgfsetbuttcap%
\pgfsetroundjoin%
\definecolor{currentfill}{rgb}{0.285273,0.380129,0.823469}%
\pgfsetfillcolor{currentfill}%
\pgfsetlinewidth{0.000000pt}%
\definecolor{currentstroke}{rgb}{0.000000,0.000000,0.000000}%
\pgfsetstrokecolor{currentstroke}%
\pgfsetdash{}{0pt}%
\pgfpathmoveto{\pgfqpoint{5.688260in}{2.712220in}}%
\pgfpathlineto{\pgfqpoint{5.696618in}{2.522513in}}%
\pgfpathlineto{\pgfqpoint{5.708248in}{2.546683in}}%
\pgfpathlineto{\pgfqpoint{5.744101in}{2.670721in}}%
\pgfpathlineto{\pgfqpoint{5.775396in}{2.507059in}}%
\pgfpathlineto{\pgfqpoint{5.765266in}{2.581963in}}%
\pgfpathlineto{\pgfqpoint{5.755356in}{2.670157in}}%
\pgfpathlineto{\pgfqpoint{5.719904in}{2.567236in}}%
\pgfpathlineto{\pgfqpoint{5.688260in}{2.712220in}}%
\pgfpathclose%
\pgfusepath{fill}%
\end{pgfscope}%
\begin{pgfscope}%
\pgfpathrectangle{\pgfqpoint{1.020000in}{0.880000in}}{\pgfqpoint{6.160000in}{6.160000in}}%
\pgfusepath{clip}%
\pgfsetbuttcap%
\pgfsetroundjoin%
\definecolor{currentfill}{rgb}{0.688188,0.793178,0.988038}%
\pgfsetfillcolor{currentfill}%
\pgfsetlinewidth{0.000000pt}%
\definecolor{currentstroke}{rgb}{0.000000,0.000000,0.000000}%
\pgfsetstrokecolor{currentstroke}%
\pgfsetdash{}{0pt}%
\pgfpathmoveto{\pgfqpoint{3.104387in}{3.287205in}}%
\pgfpathlineto{\pgfqpoint{3.111913in}{3.368365in}}%
\pgfpathlineto{\pgfqpoint{3.121880in}{3.222974in}}%
\pgfpathlineto{\pgfqpoint{3.155372in}{3.336950in}}%
\pgfpathlineto{\pgfqpoint{3.188090in}{3.534624in}}%
\pgfpathlineto{\pgfqpoint{3.181532in}{3.340323in}}%
\pgfpathlineto{\pgfqpoint{3.171822in}{3.462824in}}%
\pgfpathlineto{\pgfqpoint{3.137908in}{3.391952in}}%
\pgfpathlineto{\pgfqpoint{3.104387in}{3.287205in}}%
\pgfpathclose%
\pgfusepath{fill}%
\end{pgfscope}%
\begin{pgfscope}%
\pgfpathrectangle{\pgfqpoint{1.020000in}{0.880000in}}{\pgfqpoint{6.160000in}{6.160000in}}%
\pgfusepath{clip}%
\pgfsetbuttcap%
\pgfsetroundjoin%
\definecolor{currentfill}{rgb}{0.353369,0.472069,0.892570}%
\pgfsetfillcolor{currentfill}%
\pgfsetlinewidth{0.000000pt}%
\definecolor{currentstroke}{rgb}{0.000000,0.000000,0.000000}%
\pgfsetstrokecolor{currentstroke}%
\pgfsetdash{}{0pt}%
\pgfpathmoveto{\pgfqpoint{5.305351in}{2.762534in}}%
\pgfpathlineto{\pgfqpoint{5.315736in}{2.724191in}}%
\pgfpathlineto{\pgfqpoint{5.326081in}{2.680901in}}%
\pgfpathlineto{\pgfqpoint{5.360136in}{2.683741in}}%
\pgfpathlineto{\pgfqpoint{5.394154in}{2.685130in}}%
\pgfpathlineto{\pgfqpoint{5.382539in}{2.626542in}}%
\pgfpathlineto{\pgfqpoint{5.375369in}{2.935884in}}%
\pgfpathlineto{\pgfqpoint{5.339646in}{2.789495in}}%
\pgfpathlineto{\pgfqpoint{5.305351in}{2.762534in}}%
\pgfpathclose%
\pgfusepath{fill}%
\end{pgfscope}%
\begin{pgfscope}%
\pgfpathrectangle{\pgfqpoint{1.020000in}{0.880000in}}{\pgfqpoint{6.160000in}{6.160000in}}%
\pgfusepath{clip}%
\pgfsetbuttcap%
\pgfsetroundjoin%
\definecolor{currentfill}{rgb}{0.559747,0.694768,0.996075}%
\pgfsetfillcolor{currentfill}%
\pgfsetlinewidth{0.000000pt}%
\definecolor{currentstroke}{rgb}{0.000000,0.000000,0.000000}%
\pgfsetstrokecolor{currentstroke}%
\pgfsetdash{}{0pt}%
\pgfpathmoveto{\pgfqpoint{2.829348in}{3.092125in}}%
\pgfpathlineto{\pgfqpoint{2.837748in}{3.069570in}}%
\pgfpathlineto{\pgfqpoint{2.844575in}{3.165332in}}%
\pgfpathlineto{\pgfqpoint{2.878719in}{3.216596in}}%
\pgfpathlineto{\pgfqpoint{2.915402in}{3.069469in}}%
\pgfpathlineto{\pgfqpoint{2.906485in}{3.126748in}}%
\pgfpathlineto{\pgfqpoint{2.897405in}{3.197407in}}%
\pgfpathlineto{\pgfqpoint{2.863805in}{3.111819in}}%
\pgfpathlineto{\pgfqpoint{2.829348in}{3.092125in}}%
\pgfpathclose%
\pgfusepath{fill}%
\end{pgfscope}%
\begin{pgfscope}%
\pgfpathrectangle{\pgfqpoint{1.020000in}{0.880000in}}{\pgfqpoint{6.160000in}{6.160000in}}%
\pgfusepath{clip}%
\pgfsetbuttcap%
\pgfsetroundjoin%
\definecolor{currentfill}{rgb}{0.527132,0.664700,0.989065}%
\pgfsetfillcolor{currentfill}%
\pgfsetlinewidth{0.000000pt}%
\definecolor{currentstroke}{rgb}{0.000000,0.000000,0.000000}%
\pgfsetstrokecolor{currentstroke}%
\pgfsetdash{}{0pt}%
\pgfpathmoveto{\pgfqpoint{2.759818in}{3.094334in}}%
\pgfpathlineto{\pgfqpoint{2.767481in}{3.119771in}}%
\pgfpathlineto{\pgfqpoint{2.777192in}{3.001786in}}%
\pgfpathlineto{\pgfqpoint{2.811140in}{3.063584in}}%
\pgfpathlineto{\pgfqpoint{2.844575in}{3.165332in}}%
\pgfpathlineto{\pgfqpoint{2.837748in}{3.069570in}}%
\pgfpathlineto{\pgfqpoint{2.829348in}{3.092125in}}%
\pgfpathlineto{\pgfqpoint{2.796503in}{2.956219in}}%
\pgfpathlineto{\pgfqpoint{2.759818in}{3.094334in}}%
\pgfpathclose%
\pgfusepath{fill}%
\end{pgfscope}%
\begin{pgfscope}%
\pgfpathrectangle{\pgfqpoint{1.020000in}{0.880000in}}{\pgfqpoint{6.160000in}{6.160000in}}%
\pgfusepath{clip}%
\pgfsetbuttcap%
\pgfsetroundjoin%
\definecolor{currentfill}{rgb}{0.527132,0.664700,0.989065}%
\pgfsetfillcolor{currentfill}%
\pgfsetlinewidth{0.000000pt}%
\definecolor{currentstroke}{rgb}{0.000000,0.000000,0.000000}%
\pgfsetstrokecolor{currentstroke}%
\pgfsetdash{}{0pt}%
\pgfpathmoveto{\pgfqpoint{4.610908in}{3.138775in}}%
\pgfpathlineto{\pgfqpoint{4.620881in}{3.118728in}}%
\pgfpathlineto{\pgfqpoint{4.629610in}{2.817530in}}%
\pgfpathlineto{\pgfqpoint{4.665018in}{3.073427in}}%
\pgfpathlineto{\pgfqpoint{4.699154in}{3.053703in}}%
\pgfpathlineto{\pgfqpoint{4.689962in}{3.241372in}}%
\pgfpathlineto{\pgfqpoint{4.678979in}{3.071976in}}%
\pgfpathlineto{\pgfqpoint{4.644534in}{3.012345in}}%
\pgfpathlineto{\pgfqpoint{4.610908in}{3.138775in}}%
\pgfpathclose%
\pgfusepath{fill}%
\end{pgfscope}%
\begin{pgfscope}%
\pgfpathrectangle{\pgfqpoint{1.020000in}{0.880000in}}{\pgfqpoint{6.160000in}{6.160000in}}%
\pgfusepath{clip}%
\pgfsetbuttcap%
\pgfsetroundjoin%
\definecolor{currentfill}{rgb}{0.304174,0.406945,0.845263}%
\pgfsetfillcolor{currentfill}%
\pgfsetlinewidth{0.000000pt}%
\definecolor{currentstroke}{rgb}{0.000000,0.000000,0.000000}%
\pgfsetstrokecolor{currentstroke}%
\pgfsetdash{}{0pt}%
\pgfpathmoveto{\pgfqpoint{5.978631in}{2.532598in}}%
\pgfpathlineto{\pgfqpoint{5.990176in}{2.532049in}}%
\pgfpathlineto{\pgfqpoint{5.999001in}{2.381886in}}%
\pgfpathlineto{\pgfqpoint{6.036737in}{2.595282in}}%
\pgfpathlineto{\pgfqpoint{6.075570in}{2.859788in}}%
\pgfpathlineto{\pgfqpoint{6.061405in}{2.728736in}}%
\pgfpathlineto{\pgfqpoint{6.049897in}{2.736005in}}%
\pgfpathlineto{\pgfqpoint{6.015688in}{2.712747in}}%
\pgfpathlineto{\pgfqpoint{5.978631in}{2.532598in}}%
\pgfpathclose%
\pgfusepath{fill}%
\end{pgfscope}%
\begin{pgfscope}%
\pgfpathrectangle{\pgfqpoint{1.020000in}{0.880000in}}{\pgfqpoint{6.160000in}{6.160000in}}%
\pgfusepath{clip}%
\pgfsetbuttcap%
\pgfsetroundjoin%
\definecolor{currentfill}{rgb}{0.630089,0.752516,0.998508}%
\pgfsetfillcolor{currentfill}%
\pgfsetlinewidth{0.000000pt}%
\definecolor{currentstroke}{rgb}{0.000000,0.000000,0.000000}%
\pgfsetstrokecolor{currentstroke}%
\pgfsetdash{}{0pt}%
\pgfpathmoveto{\pgfqpoint{2.965882in}{3.273272in}}%
\pgfpathlineto{\pgfqpoint{2.974369in}{3.253942in}}%
\pgfpathlineto{\pgfqpoint{2.982261in}{3.285359in}}%
\pgfpathlineto{\pgfqpoint{3.016850in}{3.300931in}}%
\pgfpathlineto{\pgfqpoint{3.052554in}{3.217172in}}%
\pgfpathlineto{\pgfqpoint{3.044559in}{3.187180in}}%
\pgfpathlineto{\pgfqpoint{3.034600in}{3.329968in}}%
\pgfpathlineto{\pgfqpoint{3.000866in}{3.248947in}}%
\pgfpathlineto{\pgfqpoint{2.965882in}{3.273272in}}%
\pgfpathclose%
\pgfusepath{fill}%
\end{pgfscope}%
\begin{pgfscope}%
\pgfpathrectangle{\pgfqpoint{1.020000in}{0.880000in}}{\pgfqpoint{6.160000in}{6.160000in}}%
\pgfusepath{clip}%
\pgfsetbuttcap%
\pgfsetroundjoin%
\definecolor{currentfill}{rgb}{0.261805,0.346484,0.795658}%
\pgfsetfillcolor{currentfill}%
\pgfsetlinewidth{0.000000pt}%
\definecolor{currentstroke}{rgb}{0.000000,0.000000,0.000000}%
\pgfsetstrokecolor{currentstroke}%
\pgfsetdash{}{0pt}%
\pgfpathmoveto{\pgfqpoint{6.138095in}{2.596432in}}%
\pgfpathlineto{\pgfqpoint{6.149257in}{2.566848in}}%
\pgfpathlineto{\pgfqpoint{6.160868in}{2.558821in}}%
\pgfpathlineto{\pgfqpoint{6.195272in}{2.588984in}}%
\pgfpathlineto{\pgfqpoint{6.226821in}{2.480787in}}%
\pgfpathlineto{\pgfqpoint{6.214895in}{2.475993in}}%
\pgfpathlineto{\pgfqpoint{6.205896in}{2.613415in}}%
\pgfpathlineto{\pgfqpoint{6.169776in}{2.493848in}}%
\pgfpathlineto{\pgfqpoint{6.138095in}{2.596432in}}%
\pgfpathclose%
\pgfusepath{fill}%
\end{pgfscope}%
\begin{pgfscope}%
\pgfpathrectangle{\pgfqpoint{1.020000in}{0.880000in}}{\pgfqpoint{6.160000in}{6.160000in}}%
\pgfusepath{clip}%
\pgfsetbuttcap%
\pgfsetroundjoin%
\definecolor{currentfill}{rgb}{0.608547,0.735725,0.999354}%
\pgfsetfillcolor{currentfill}%
\pgfsetlinewidth{0.000000pt}%
\definecolor{currentstroke}{rgb}{0.000000,0.000000,0.000000}%
\pgfsetstrokecolor{currentstroke}%
\pgfsetdash{}{0pt}%
\pgfpathmoveto{\pgfqpoint{2.897405in}{3.197407in}}%
\pgfpathlineto{\pgfqpoint{2.906485in}{3.126748in}}%
\pgfpathlineto{\pgfqpoint{2.915402in}{3.069469in}}%
\pgfpathlineto{\pgfqpoint{2.948768in}{3.180329in}}%
\pgfpathlineto{\pgfqpoint{2.982261in}{3.285359in}}%
\pgfpathlineto{\pgfqpoint{2.974369in}{3.253942in}}%
\pgfpathlineto{\pgfqpoint{2.965882in}{3.273272in}}%
\pgfpathlineto{\pgfqpoint{2.930438in}{3.331261in}}%
\pgfpathlineto{\pgfqpoint{2.897405in}{3.197407in}}%
\pgfpathclose%
\pgfusepath{fill}%
\end{pgfscope}%
\begin{pgfscope}%
\pgfpathrectangle{\pgfqpoint{1.020000in}{0.880000in}}{\pgfqpoint{6.160000in}{6.160000in}}%
\pgfusepath{clip}%
\pgfsetbuttcap%
\pgfsetroundjoin%
\definecolor{currentfill}{rgb}{0.353369,0.472069,0.892570}%
\pgfsetfillcolor{currentfill}%
\pgfsetlinewidth{0.000000pt}%
\definecolor{currentstroke}{rgb}{0.000000,0.000000,0.000000}%
\pgfsetstrokecolor{currentstroke}%
\pgfsetdash{}{0pt}%
\pgfpathmoveto{\pgfqpoint{5.011899in}{2.783848in}}%
\pgfpathlineto{\pgfqpoint{5.022025in}{2.739079in}}%
\pgfpathlineto{\pgfqpoint{5.033860in}{2.892454in}}%
\pgfpathlineto{\pgfqpoint{5.066483in}{2.715925in}}%
\pgfpathlineto{\pgfqpoint{5.101010in}{2.760963in}}%
\pgfpathlineto{\pgfqpoint{5.089553in}{2.666462in}}%
\pgfpathlineto{\pgfqpoint{5.079045in}{2.672267in}}%
\pgfpathlineto{\pgfqpoint{5.045204in}{2.693811in}}%
\pgfpathlineto{\pgfqpoint{5.011899in}{2.783848in}}%
\pgfpathclose%
\pgfusepath{fill}%
\end{pgfscope}%
\begin{pgfscope}%
\pgfpathrectangle{\pgfqpoint{1.020000in}{0.880000in}}{\pgfqpoint{6.160000in}{6.160000in}}%
\pgfusepath{clip}%
\pgfsetbuttcap%
\pgfsetroundjoin%
\definecolor{currentfill}{rgb}{0.822420,0.856898,0.910795}%
\pgfsetfillcolor{currentfill}%
\pgfsetlinewidth{0.000000pt}%
\definecolor{currentstroke}{rgb}{0.000000,0.000000,0.000000}%
\pgfsetstrokecolor{currentstroke}%
\pgfsetdash{}{0pt}%
\pgfpathmoveto{\pgfqpoint{3.378191in}{3.602511in}}%
\pgfpathlineto{\pgfqpoint{3.387115in}{3.574952in}}%
\pgfpathlineto{\pgfqpoint{3.396758in}{3.457596in}}%
\pgfpathlineto{\pgfqpoint{3.430617in}{3.553145in}}%
\pgfpathlineto{\pgfqpoint{3.463913in}{3.733312in}}%
\pgfpathlineto{\pgfqpoint{3.455403in}{3.692671in}}%
\pgfpathlineto{\pgfqpoint{3.446392in}{3.723788in}}%
\pgfpathlineto{\pgfqpoint{3.410903in}{3.842273in}}%
\pgfpathlineto{\pgfqpoint{3.378191in}{3.602511in}}%
\pgfpathclose%
\pgfusepath{fill}%
\end{pgfscope}%
\begin{pgfscope}%
\pgfpathrectangle{\pgfqpoint{1.020000in}{0.880000in}}{\pgfqpoint{6.160000in}{6.160000in}}%
\pgfusepath{clip}%
\pgfsetbuttcap%
\pgfsetroundjoin%
\definecolor{currentfill}{rgb}{0.859385,0.864431,0.872111}%
\pgfsetfillcolor{currentfill}%
\pgfsetlinewidth{0.000000pt}%
\definecolor{currentstroke}{rgb}{0.000000,0.000000,0.000000}%
\pgfsetstrokecolor{currentstroke}%
\pgfsetdash{}{0pt}%
\pgfpathmoveto{\pgfqpoint{3.740083in}{3.796278in}}%
\pgfpathlineto{\pgfqpoint{3.748765in}{3.895252in}}%
\pgfpathlineto{\pgfqpoint{3.758753in}{3.694796in}}%
\pgfpathlineto{\pgfqpoint{3.793088in}{3.741707in}}%
\pgfpathlineto{\pgfqpoint{3.827478in}{3.776658in}}%
\pgfpathlineto{\pgfqpoint{3.818416in}{3.739107in}}%
\pgfpathlineto{\pgfqpoint{3.809440in}{3.685853in}}%
\pgfpathlineto{\pgfqpoint{3.775542in}{3.556548in}}%
\pgfpathlineto{\pgfqpoint{3.740083in}{3.796278in}}%
\pgfpathclose%
\pgfusepath{fill}%
\end{pgfscope}%
\begin{pgfscope}%
\pgfpathrectangle{\pgfqpoint{1.020000in}{0.880000in}}{\pgfqpoint{6.160000in}{6.160000in}}%
\pgfusepath{clip}%
\pgfsetbuttcap%
\pgfsetroundjoin%
\definecolor{currentfill}{rgb}{0.338377,0.452819,0.879317}%
\pgfsetfillcolor{currentfill}%
\pgfsetlinewidth{0.000000pt}%
\definecolor{currentstroke}{rgb}{0.000000,0.000000,0.000000}%
\pgfsetstrokecolor{currentstroke}%
\pgfsetdash{}{0pt}%
\pgfpathmoveto{\pgfqpoint{5.237866in}{2.813491in}}%
\pgfpathlineto{\pgfqpoint{5.247020in}{2.663593in}}%
\pgfpathlineto{\pgfqpoint{5.257584in}{2.644775in}}%
\pgfpathlineto{\pgfqpoint{5.292098in}{2.686255in}}%
\pgfpathlineto{\pgfqpoint{5.326081in}{2.680901in}}%
\pgfpathlineto{\pgfqpoint{5.315736in}{2.724191in}}%
\pgfpathlineto{\pgfqpoint{5.305351in}{2.762534in}}%
\pgfpathlineto{\pgfqpoint{5.270642in}{2.698030in}}%
\pgfpathlineto{\pgfqpoint{5.237866in}{2.813491in}}%
\pgfpathclose%
\pgfusepath{fill}%
\end{pgfscope}%
\begin{pgfscope}%
\pgfpathrectangle{\pgfqpoint{1.020000in}{0.880000in}}{\pgfqpoint{6.160000in}{6.160000in}}%
\pgfusepath{clip}%
\pgfsetbuttcap%
\pgfsetroundjoin%
\definecolor{currentfill}{rgb}{0.683056,0.790043,0.989768}%
\pgfsetfillcolor{currentfill}%
\pgfsetlinewidth{0.000000pt}%
\definecolor{currentstroke}{rgb}{0.000000,0.000000,0.000000}%
\pgfsetstrokecolor{currentstroke}%
\pgfsetdash{}{0pt}%
\pgfpathmoveto{\pgfqpoint{4.366306in}{3.308420in}}%
\pgfpathlineto{\pgfqpoint{4.376590in}{3.567098in}}%
\pgfpathlineto{\pgfqpoint{4.385789in}{3.255428in}}%
\pgfpathlineto{\pgfqpoint{4.420314in}{3.346760in}}%
\pgfpathlineto{\pgfqpoint{4.454676in}{3.362482in}}%
\pgfpathlineto{\pgfqpoint{4.444447in}{3.229169in}}%
\pgfpathlineto{\pgfqpoint{4.434982in}{3.380400in}}%
\pgfpathlineto{\pgfqpoint{4.400741in}{3.389986in}}%
\pgfpathlineto{\pgfqpoint{4.366306in}{3.308420in}}%
\pgfpathclose%
\pgfusepath{fill}%
\end{pgfscope}%
\begin{pgfscope}%
\pgfpathrectangle{\pgfqpoint{1.020000in}{0.880000in}}{\pgfqpoint{6.160000in}{6.160000in}}%
\pgfusepath{clip}%
\pgfsetbuttcap%
\pgfsetroundjoin%
\definecolor{currentfill}{rgb}{0.280550,0.373423,0.818011}%
\pgfsetfillcolor{currentfill}%
\pgfsetlinewidth{0.000000pt}%
\definecolor{currentstroke}{rgb}{0.000000,0.000000,0.000000}%
\pgfsetstrokecolor{currentstroke}%
\pgfsetdash{}{0pt}%
\pgfpathmoveto{\pgfqpoint{5.618452in}{2.576669in}}%
\pgfpathlineto{\pgfqpoint{5.631338in}{2.694149in}}%
\pgfpathlineto{\pgfqpoint{5.640237in}{2.536962in}}%
\pgfpathlineto{\pgfqpoint{5.675097in}{2.598071in}}%
\pgfpathlineto{\pgfqpoint{5.708248in}{2.546683in}}%
\pgfpathlineto{\pgfqpoint{5.696618in}{2.522513in}}%
\pgfpathlineto{\pgfqpoint{5.688260in}{2.712220in}}%
\pgfpathlineto{\pgfqpoint{5.651157in}{2.496887in}}%
\pgfpathlineto{\pgfqpoint{5.618452in}{2.576669in}}%
\pgfpathclose%
\pgfusepath{fill}%
\end{pgfscope}%
\begin{pgfscope}%
\pgfpathrectangle{\pgfqpoint{1.020000in}{0.880000in}}{\pgfqpoint{6.160000in}{6.160000in}}%
\pgfusepath{clip}%
\pgfsetbuttcap%
\pgfsetroundjoin%
\definecolor{currentfill}{rgb}{0.299441,0.400248,0.839842}%
\pgfsetfillcolor{currentfill}%
\pgfsetlinewidth{0.000000pt}%
\definecolor{currentstroke}{rgb}{0.000000,0.000000,0.000000}%
\pgfsetstrokecolor{currentstroke}%
\pgfsetdash{}{0pt}%
\pgfpathmoveto{\pgfqpoint{5.394154in}{2.685130in}}%
\pgfpathlineto{\pgfqpoint{5.404261in}{2.618197in}}%
\pgfpathlineto{\pgfqpoint{5.415723in}{2.659590in}}%
\pgfpathlineto{\pgfqpoint{5.448033in}{2.524040in}}%
\pgfpathlineto{\pgfqpoint{5.482630in}{2.571592in}}%
\pgfpathlineto{\pgfqpoint{5.472809in}{2.663129in}}%
\pgfpathlineto{\pgfqpoint{5.461470in}{2.636604in}}%
\pgfpathlineto{\pgfqpoint{5.427306in}{2.619016in}}%
\pgfpathlineto{\pgfqpoint{5.394154in}{2.685130in}}%
\pgfpathclose%
\pgfusepath{fill}%
\end{pgfscope}%
\begin{pgfscope}%
\pgfpathrectangle{\pgfqpoint{1.020000in}{0.880000in}}{\pgfqpoint{6.160000in}{6.160000in}}%
\pgfusepath{clip}%
\pgfsetbuttcap%
\pgfsetroundjoin%
\definecolor{currentfill}{rgb}{0.738826,0.822572,0.968261}%
\pgfsetfillcolor{currentfill}%
\pgfsetlinewidth{0.000000pt}%
\definecolor{currentstroke}{rgb}{0.000000,0.000000,0.000000}%
\pgfsetstrokecolor{currentstroke}%
\pgfsetdash{}{0pt}%
\pgfpathmoveto{\pgfqpoint{4.209890in}{3.509267in}}%
\pgfpathlineto{\pgfqpoint{4.219477in}{3.409732in}}%
\pgfpathlineto{\pgfqpoint{4.229169in}{3.534047in}}%
\pgfpathlineto{\pgfqpoint{4.263413in}{3.315786in}}%
\pgfpathlineto{\pgfqpoint{4.297932in}{3.486989in}}%
\pgfpathlineto{\pgfqpoint{4.288284in}{3.547156in}}%
\pgfpathlineto{\pgfqpoint{4.278514in}{3.469213in}}%
\pgfpathlineto{\pgfqpoint{4.244167in}{3.404186in}}%
\pgfpathlineto{\pgfqpoint{4.209890in}{3.509267in}}%
\pgfpathclose%
\pgfusepath{fill}%
\end{pgfscope}%
\begin{pgfscope}%
\pgfpathrectangle{\pgfqpoint{1.020000in}{0.880000in}}{\pgfqpoint{6.160000in}{6.160000in}}%
\pgfusepath{clip}%
\pgfsetbuttcap%
\pgfsetroundjoin%
\definecolor{currentfill}{rgb}{0.521696,0.659599,0.987736}%
\pgfsetfillcolor{currentfill}%
\pgfsetlinewidth{0.000000pt}%
\definecolor{currentstroke}{rgb}{0.000000,0.000000,0.000000}%
\pgfsetstrokecolor{currentstroke}%
\pgfsetdash{}{0pt}%
\pgfpathmoveto{\pgfqpoint{4.541894in}{2.992506in}}%
\pgfpathlineto{\pgfqpoint{4.552173in}{3.074010in}}%
\pgfpathlineto{\pgfqpoint{4.562176in}{3.072546in}}%
\pgfpathlineto{\pgfqpoint{4.596181in}{3.001486in}}%
\pgfpathlineto{\pgfqpoint{4.629610in}{2.817530in}}%
\pgfpathlineto{\pgfqpoint{4.620881in}{3.118728in}}%
\pgfpathlineto{\pgfqpoint{4.610908in}{3.138775in}}%
\pgfpathlineto{\pgfqpoint{4.576928in}{3.203043in}}%
\pgfpathlineto{\pgfqpoint{4.541894in}{2.992506in}}%
\pgfpathclose%
\pgfusepath{fill}%
\end{pgfscope}%
\begin{pgfscope}%
\pgfpathrectangle{\pgfqpoint{1.020000in}{0.880000in}}{\pgfqpoint{6.160000in}{6.160000in}}%
\pgfusepath{clip}%
\pgfsetbuttcap%
\pgfsetroundjoin%
\definecolor{currentfill}{rgb}{0.758539,0.832787,0.958408}%
\pgfsetfillcolor{currentfill}%
\pgfsetlinewidth{0.000000pt}%
\definecolor{currentstroke}{rgb}{0.000000,0.000000,0.000000}%
\pgfsetstrokecolor{currentstroke}%
\pgfsetdash{}{0pt}%
\pgfpathmoveto{\pgfqpoint{3.171822in}{3.462824in}}%
\pgfpathlineto{\pgfqpoint{3.181532in}{3.340323in}}%
\pgfpathlineto{\pgfqpoint{3.188090in}{3.534624in}}%
\pgfpathlineto{\pgfqpoint{3.221241in}{3.699174in}}%
\pgfpathlineto{\pgfqpoint{3.258065in}{3.476660in}}%
\pgfpathlineto{\pgfqpoint{3.249061in}{3.522902in}}%
\pgfpathlineto{\pgfqpoint{3.240575in}{3.515421in}}%
\pgfpathlineto{\pgfqpoint{3.206599in}{3.448302in}}%
\pgfpathlineto{\pgfqpoint{3.171822in}{3.462824in}}%
\pgfpathclose%
\pgfusepath{fill}%
\end{pgfscope}%
\begin{pgfscope}%
\pgfpathrectangle{\pgfqpoint{1.020000in}{0.880000in}}{\pgfqpoint{6.160000in}{6.160000in}}%
\pgfusepath{clip}%
\pgfsetbuttcap%
\pgfsetroundjoin%
\definecolor{currentfill}{rgb}{0.619318,0.744121,0.998931}%
\pgfsetfillcolor{currentfill}%
\pgfsetlinewidth{0.000000pt}%
\definecolor{currentstroke}{rgb}{0.000000,0.000000,0.000000}%
\pgfsetstrokecolor{currentstroke}%
\pgfsetdash{}{0pt}%
\pgfpathmoveto{\pgfqpoint{4.454676in}{3.362482in}}%
\pgfpathlineto{\pgfqpoint{4.464216in}{3.233717in}}%
\pgfpathlineto{\pgfqpoint{4.474447in}{3.343766in}}%
\pgfpathlineto{\pgfqpoint{4.508283in}{3.180986in}}%
\pgfpathlineto{\pgfqpoint{4.541894in}{2.992506in}}%
\pgfpathlineto{\pgfqpoint{4.532787in}{3.234917in}}%
\pgfpathlineto{\pgfqpoint{4.522804in}{3.227684in}}%
\pgfpathlineto{\pgfqpoint{4.488589in}{3.234700in}}%
\pgfpathlineto{\pgfqpoint{4.454676in}{3.362482in}}%
\pgfpathclose%
\pgfusepath{fill}%
\end{pgfscope}%
\begin{pgfscope}%
\pgfpathrectangle{\pgfqpoint{1.020000in}{0.880000in}}{\pgfqpoint{6.160000in}{6.160000in}}%
\pgfusepath{clip}%
\pgfsetbuttcap%
\pgfsetroundjoin%
\definecolor{currentfill}{rgb}{0.478462,0.616564,0.972721}%
\pgfsetfillcolor{currentfill}%
\pgfsetlinewidth{0.000000pt}%
\definecolor{currentstroke}{rgb}{0.000000,0.000000,0.000000}%
\pgfsetstrokecolor{currentstroke}%
\pgfsetdash{}{0pt}%
\pgfpathmoveto{\pgfqpoint{4.699154in}{3.053703in}}%
\pgfpathlineto{\pgfqpoint{4.709640in}{3.114370in}}%
\pgfpathlineto{\pgfqpoint{4.719210in}{2.998207in}}%
\pgfpathlineto{\pgfqpoint{4.753375in}{2.981591in}}%
\pgfpathlineto{\pgfqpoint{4.787100in}{2.898013in}}%
\pgfpathlineto{\pgfqpoint{4.776907in}{2.906211in}}%
\pgfpathlineto{\pgfqpoint{4.766966in}{2.953663in}}%
\pgfpathlineto{\pgfqpoint{4.732349in}{2.870029in}}%
\pgfpathlineto{\pgfqpoint{4.699154in}{3.053703in}}%
\pgfpathclose%
\pgfusepath{fill}%
\end{pgfscope}%
\begin{pgfscope}%
\pgfpathrectangle{\pgfqpoint{1.020000in}{0.880000in}}{\pgfqpoint{6.160000in}{6.160000in}}%
\pgfusepath{clip}%
\pgfsetbuttcap%
\pgfsetroundjoin%
\definecolor{currentfill}{rgb}{0.855378,0.863778,0.876587}%
\pgfsetfillcolor{currentfill}%
\pgfsetlinewidth{0.000000pt}%
\definecolor{currentstroke}{rgb}{0.000000,0.000000,0.000000}%
\pgfsetstrokecolor{currentstroke}%
\pgfsetdash{}{0pt}%
\pgfpathmoveto{\pgfqpoint{3.672208in}{3.570208in}}%
\pgfpathlineto{\pgfqpoint{3.680795in}{3.656437in}}%
\pgfpathlineto{\pgfqpoint{3.689698in}{3.685791in}}%
\pgfpathlineto{\pgfqpoint{3.723988in}{3.744218in}}%
\pgfpathlineto{\pgfqpoint{3.758753in}{3.694796in}}%
\pgfpathlineto{\pgfqpoint{3.748765in}{3.895252in}}%
\pgfpathlineto{\pgfqpoint{3.740083in}{3.796278in}}%
\pgfpathlineto{\pgfqpoint{3.706104in}{3.685899in}}%
\pgfpathlineto{\pgfqpoint{3.672208in}{3.570208in}}%
\pgfpathclose%
\pgfusepath{fill}%
\end{pgfscope}%
\begin{pgfscope}%
\pgfpathrectangle{\pgfqpoint{1.020000in}{0.880000in}}{\pgfqpoint{6.160000in}{6.160000in}}%
\pgfusepath{clip}%
\pgfsetbuttcap%
\pgfsetroundjoin%
\definecolor{currentfill}{rgb}{0.441123,0.576532,0.954545}%
\pgfsetfillcolor{currentfill}%
\pgfsetlinewidth{0.000000pt}%
\definecolor{currentstroke}{rgb}{0.000000,0.000000,0.000000}%
\pgfsetstrokecolor{currentstroke}%
\pgfsetdash{}{0pt}%
\pgfpathmoveto{\pgfqpoint{4.855459in}{2.899860in}}%
\pgfpathlineto{\pgfqpoint{4.865741in}{2.892413in}}%
\pgfpathlineto{\pgfqpoint{4.876690in}{2.977642in}}%
\pgfpathlineto{\pgfqpoint{4.909927in}{2.840742in}}%
\pgfpathlineto{\pgfqpoint{4.944404in}{2.880937in}}%
\pgfpathlineto{\pgfqpoint{4.933015in}{2.752902in}}%
\pgfpathlineto{\pgfqpoint{4.924733in}{3.035673in}}%
\pgfpathlineto{\pgfqpoint{4.889655in}{2.907517in}}%
\pgfpathlineto{\pgfqpoint{4.855459in}{2.899860in}}%
\pgfpathclose%
\pgfusepath{fill}%
\end{pgfscope}%
\begin{pgfscope}%
\pgfpathrectangle{\pgfqpoint{1.020000in}{0.880000in}}{\pgfqpoint{6.160000in}{6.160000in}}%
\pgfusepath{clip}%
\pgfsetbuttcap%
\pgfsetroundjoin%
\definecolor{currentfill}{rgb}{0.548876,0.685104,0.994379}%
\pgfsetfillcolor{currentfill}%
\pgfsetlinewidth{0.000000pt}%
\definecolor{currentstroke}{rgb}{0.000000,0.000000,0.000000}%
\pgfsetstrokecolor{currentstroke}%
\pgfsetdash{}{0pt}%
\pgfpathmoveto{\pgfqpoint{2.690062in}{3.105725in}}%
\pgfpathlineto{\pgfqpoint{2.698829in}{3.051171in}}%
\pgfpathlineto{\pgfqpoint{2.704244in}{3.223347in}}%
\pgfpathlineto{\pgfqpoint{2.741220in}{3.081315in}}%
\pgfpathlineto{\pgfqpoint{2.777192in}{3.001786in}}%
\pgfpathlineto{\pgfqpoint{2.767481in}{3.119771in}}%
\pgfpathlineto{\pgfqpoint{2.759818in}{3.094334in}}%
\pgfpathlineto{\pgfqpoint{2.724328in}{3.142707in}}%
\pgfpathlineto{\pgfqpoint{2.690062in}{3.105725in}}%
\pgfpathclose%
\pgfusepath{fill}%
\end{pgfscope}%
\begin{pgfscope}%
\pgfpathrectangle{\pgfqpoint{1.020000in}{0.880000in}}{\pgfqpoint{6.160000in}{6.160000in}}%
\pgfusepath{clip}%
\pgfsetbuttcap%
\pgfsetroundjoin%
\definecolor{currentfill}{rgb}{0.348323,0.465711,0.888346}%
\pgfsetfillcolor{currentfill}%
\pgfsetlinewidth{0.000000pt}%
\definecolor{currentstroke}{rgb}{0.000000,0.000000,0.000000}%
\pgfsetstrokecolor{currentstroke}%
\pgfsetdash{}{0pt}%
\pgfpathmoveto{\pgfqpoint{5.168902in}{2.730820in}}%
\pgfpathlineto{\pgfqpoint{5.178644in}{2.637019in}}%
\pgfpathlineto{\pgfqpoint{5.190890in}{2.789757in}}%
\pgfpathlineto{\pgfqpoint{5.224065in}{2.697497in}}%
\pgfpathlineto{\pgfqpoint{5.257584in}{2.644775in}}%
\pgfpathlineto{\pgfqpoint{5.247020in}{2.663593in}}%
\pgfpathlineto{\pgfqpoint{5.237866in}{2.813491in}}%
\pgfpathlineto{\pgfqpoint{5.203020in}{2.737036in}}%
\pgfpathlineto{\pgfqpoint{5.168902in}{2.730820in}}%
\pgfpathclose%
\pgfusepath{fill}%
\end{pgfscope}%
\begin{pgfscope}%
\pgfpathrectangle{\pgfqpoint{1.020000in}{0.880000in}}{\pgfqpoint{6.160000in}{6.160000in}}%
\pgfusepath{clip}%
\pgfsetbuttcap%
\pgfsetroundjoin%
\definecolor{currentfill}{rgb}{0.847365,0.862472,0.885540}%
\pgfsetfillcolor{currentfill}%
\pgfsetlinewidth{0.000000pt}%
\definecolor{currentstroke}{rgb}{0.000000,0.000000,0.000000}%
\pgfsetstrokecolor{currentstroke}%
\pgfsetdash{}{0pt}%
\pgfpathmoveto{\pgfqpoint{3.602055in}{3.776551in}}%
\pgfpathlineto{\pgfqpoint{3.611705in}{3.657077in}}%
\pgfpathlineto{\pgfqpoint{3.620442in}{3.699839in}}%
\pgfpathlineto{\pgfqpoint{3.654471in}{3.809103in}}%
\pgfpathlineto{\pgfqpoint{3.689698in}{3.685791in}}%
\pgfpathlineto{\pgfqpoint{3.680795in}{3.656437in}}%
\pgfpathlineto{\pgfqpoint{3.672208in}{3.570208in}}%
\pgfpathlineto{\pgfqpoint{3.637209in}{3.667581in}}%
\pgfpathlineto{\pgfqpoint{3.602055in}{3.776551in}}%
\pgfpathclose%
\pgfusepath{fill}%
\end{pgfscope}%
\begin{pgfscope}%
\pgfpathrectangle{\pgfqpoint{1.020000in}{0.880000in}}{\pgfqpoint{6.160000in}{6.160000in}}%
\pgfusepath{clip}%
\pgfsetbuttcap%
\pgfsetroundjoin%
\definecolor{currentfill}{rgb}{0.285273,0.380129,0.823469}%
\pgfsetfillcolor{currentfill}%
\pgfsetlinewidth{0.000000pt}%
\definecolor{currentstroke}{rgb}{0.000000,0.000000,0.000000}%
\pgfsetstrokecolor{currentstroke}%
\pgfsetdash{}{0pt}%
\pgfpathmoveto{\pgfqpoint{6.075570in}{2.859788in}}%
\pgfpathlineto{\pgfqpoint{6.078927in}{2.423458in}}%
\pgfpathlineto{\pgfqpoint{6.093851in}{2.591497in}}%
\pgfpathlineto{\pgfqpoint{6.126808in}{2.546205in}}%
\pgfpathlineto{\pgfqpoint{6.160868in}{2.558821in}}%
\pgfpathlineto{\pgfqpoint{6.149257in}{2.566848in}}%
\pgfpathlineto{\pgfqpoint{6.138095in}{2.596432in}}%
\pgfpathlineto{\pgfqpoint{6.103038in}{2.529362in}}%
\pgfpathlineto{\pgfqpoint{6.075570in}{2.859788in}}%
\pgfpathclose%
\pgfusepath{fill}%
\end{pgfscope}%
\begin{pgfscope}%
\pgfpathrectangle{\pgfqpoint{1.020000in}{0.880000in}}{\pgfqpoint{6.160000in}{6.160000in}}%
\pgfusepath{clip}%
\pgfsetbuttcap%
\pgfsetroundjoin%
\definecolor{currentfill}{rgb}{0.859385,0.864431,0.872111}%
\pgfsetfillcolor{currentfill}%
\pgfsetlinewidth{0.000000pt}%
\definecolor{currentstroke}{rgb}{0.000000,0.000000,0.000000}%
\pgfsetstrokecolor{currentstroke}%
\pgfsetdash{}{0pt}%
\pgfpathmoveto{\pgfqpoint{3.896356in}{3.814568in}}%
\pgfpathlineto{\pgfqpoint{3.905863in}{3.736866in}}%
\pgfpathlineto{\pgfqpoint{3.915590in}{3.578735in}}%
\pgfpathlineto{\pgfqpoint{3.949721in}{3.721634in}}%
\pgfpathlineto{\pgfqpoint{3.984553in}{3.549871in}}%
\pgfpathlineto{\pgfqpoint{3.974961in}{3.663334in}}%
\pgfpathlineto{\pgfqpoint{3.965216in}{3.846775in}}%
\pgfpathlineto{\pgfqpoint{3.930726in}{3.856538in}}%
\pgfpathlineto{\pgfqpoint{3.896356in}{3.814568in}}%
\pgfpathclose%
\pgfusepath{fill}%
\end{pgfscope}%
\begin{pgfscope}%
\pgfpathrectangle{\pgfqpoint{1.020000in}{0.880000in}}{\pgfqpoint{6.160000in}{6.160000in}}%
\pgfusepath{clip}%
\pgfsetbuttcap%
\pgfsetroundjoin%
\definecolor{currentfill}{rgb}{0.289996,0.386836,0.828926}%
\pgfsetfillcolor{currentfill}%
\pgfsetlinewidth{0.000000pt}%
\definecolor{currentstroke}{rgb}{0.000000,0.000000,0.000000}%
\pgfsetstrokecolor{currentstroke}%
\pgfsetdash{}{0pt}%
\pgfpathmoveto{\pgfqpoint{5.550975in}{2.602708in}}%
\pgfpathlineto{\pgfqpoint{5.561940in}{2.592516in}}%
\pgfpathlineto{\pgfqpoint{5.573818in}{2.645865in}}%
\pgfpathlineto{\pgfqpoint{5.606623in}{2.561184in}}%
\pgfpathlineto{\pgfqpoint{5.640237in}{2.536962in}}%
\pgfpathlineto{\pgfqpoint{5.631338in}{2.694149in}}%
\pgfpathlineto{\pgfqpoint{5.618452in}{2.576669in}}%
\pgfpathlineto{\pgfqpoint{5.584267in}{2.557049in}}%
\pgfpathlineto{\pgfqpoint{5.550975in}{2.602708in}}%
\pgfpathclose%
\pgfusepath{fill}%
\end{pgfscope}%
\begin{pgfscope}%
\pgfpathrectangle{\pgfqpoint{1.020000in}{0.880000in}}{\pgfqpoint{6.160000in}{6.160000in}}%
\pgfusepath{clip}%
\pgfsetbuttcap%
\pgfsetroundjoin%
\definecolor{currentfill}{rgb}{0.313946,0.420052,0.854993}%
\pgfsetfillcolor{currentfill}%
\pgfsetlinewidth{0.000000pt}%
\definecolor{currentstroke}{rgb}{0.000000,0.000000,0.000000}%
\pgfsetstrokecolor{currentstroke}%
\pgfsetdash{}{0pt}%
\pgfpathmoveto{\pgfqpoint{5.326081in}{2.680901in}}%
\pgfpathlineto{\pgfqpoint{5.336636in}{2.654526in}}%
\pgfpathlineto{\pgfqpoint{5.346157in}{2.538436in}}%
\pgfpathlineto{\pgfqpoint{5.381144in}{2.616751in}}%
\pgfpathlineto{\pgfqpoint{5.415723in}{2.659590in}}%
\pgfpathlineto{\pgfqpoint{5.404261in}{2.618197in}}%
\pgfpathlineto{\pgfqpoint{5.394154in}{2.685130in}}%
\pgfpathlineto{\pgfqpoint{5.360136in}{2.683741in}}%
\pgfpathlineto{\pgfqpoint{5.326081in}{2.680901in}}%
\pgfpathclose%
\pgfusepath{fill}%
\end{pgfscope}%
\begin{pgfscope}%
\pgfpathrectangle{\pgfqpoint{1.020000in}{0.880000in}}{\pgfqpoint{6.160000in}{6.160000in}}%
\pgfusepath{clip}%
\pgfsetbuttcap%
\pgfsetroundjoin%
\definecolor{currentfill}{rgb}{0.378598,0.503856,0.913692}%
\pgfsetfillcolor{currentfill}%
\pgfsetlinewidth{0.000000pt}%
\definecolor{currentstroke}{rgb}{0.000000,0.000000,0.000000}%
\pgfsetstrokecolor{currentstroke}%
\pgfsetdash{}{0pt}%
\pgfpathmoveto{\pgfqpoint{4.944404in}{2.880937in}}%
\pgfpathlineto{\pgfqpoint{4.953160in}{2.663074in}}%
\pgfpathlineto{\pgfqpoint{4.964478in}{2.773933in}}%
\pgfpathlineto{\pgfqpoint{4.997834in}{2.671723in}}%
\pgfpathlineto{\pgfqpoint{5.033860in}{2.892454in}}%
\pgfpathlineto{\pgfqpoint{5.022025in}{2.739079in}}%
\pgfpathlineto{\pgfqpoint{5.011899in}{2.783848in}}%
\pgfpathlineto{\pgfqpoint{4.978129in}{2.826135in}}%
\pgfpathlineto{\pgfqpoint{4.944404in}{2.880937in}}%
\pgfpathclose%
\pgfusepath{fill}%
\end{pgfscope}%
\begin{pgfscope}%
\pgfpathrectangle{\pgfqpoint{1.020000in}{0.880000in}}{\pgfqpoint{6.160000in}{6.160000in}}%
\pgfusepath{clip}%
\pgfsetbuttcap%
\pgfsetroundjoin%
\definecolor{currentfill}{rgb}{0.809329,0.852974,0.922323}%
\pgfsetfillcolor{currentfill}%
\pgfsetlinewidth{0.000000pt}%
\definecolor{currentstroke}{rgb}{0.000000,0.000000,0.000000}%
\pgfsetstrokecolor{currentstroke}%
\pgfsetdash{}{0pt}%
\pgfpathmoveto{\pgfqpoint{3.463913in}{3.733312in}}%
\pgfpathlineto{\pgfqpoint{3.474279in}{3.515760in}}%
\pgfpathlineto{\pgfqpoint{3.482867in}{3.549586in}}%
\pgfpathlineto{\pgfqpoint{3.517995in}{3.475523in}}%
\pgfpathlineto{\pgfqpoint{3.552375in}{3.508815in}}%
\pgfpathlineto{\pgfqpoint{3.542182in}{3.711972in}}%
\pgfpathlineto{\pgfqpoint{3.533116in}{3.738769in}}%
\pgfpathlineto{\pgfqpoint{3.499713in}{3.561831in}}%
\pgfpathlineto{\pgfqpoint{3.463913in}{3.733312in}}%
\pgfpathclose%
\pgfusepath{fill}%
\end{pgfscope}%
\begin{pgfscope}%
\pgfpathrectangle{\pgfqpoint{1.020000in}{0.880000in}}{\pgfqpoint{6.160000in}{6.160000in}}%
\pgfusepath{clip}%
\pgfsetbuttcap%
\pgfsetroundjoin%
\definecolor{currentfill}{rgb}{0.543440,0.680003,0.993051}%
\pgfsetfillcolor{currentfill}%
\pgfsetlinewidth{0.000000pt}%
\definecolor{currentstroke}{rgb}{0.000000,0.000000,0.000000}%
\pgfsetstrokecolor{currentstroke}%
\pgfsetdash{}{0pt}%
\pgfpathmoveto{\pgfqpoint{2.620804in}{3.078903in}}%
\pgfpathlineto{\pgfqpoint{2.628807in}{3.071049in}}%
\pgfpathlineto{\pgfqpoint{2.635460in}{3.151216in}}%
\pgfpathlineto{\pgfqpoint{2.672733in}{2.997248in}}%
\pgfpathlineto{\pgfqpoint{2.704244in}{3.223347in}}%
\pgfpathlineto{\pgfqpoint{2.698829in}{3.051171in}}%
\pgfpathlineto{\pgfqpoint{2.690062in}{3.105725in}}%
\pgfpathlineto{\pgfqpoint{2.656745in}{3.007110in}}%
\pgfpathlineto{\pgfqpoint{2.620804in}{3.078903in}}%
\pgfpathclose%
\pgfusepath{fill}%
\end{pgfscope}%
\begin{pgfscope}%
\pgfpathrectangle{\pgfqpoint{1.020000in}{0.880000in}}{\pgfqpoint{6.160000in}{6.160000in}}%
\pgfusepath{clip}%
\pgfsetbuttcap%
\pgfsetroundjoin%
\definecolor{currentfill}{rgb}{0.318832,0.426605,0.859857}%
\pgfsetfillcolor{currentfill}%
\pgfsetlinewidth{0.000000pt}%
\definecolor{currentstroke}{rgb}{0.000000,0.000000,0.000000}%
\pgfsetstrokecolor{currentstroke}%
\pgfsetdash{}{0pt}%
\pgfpathmoveto{\pgfqpoint{5.845510in}{2.652698in}}%
\pgfpathlineto{\pgfqpoint{5.856432in}{2.622399in}}%
\pgfpathlineto{\pgfqpoint{5.868308in}{2.647149in}}%
\pgfpathlineto{\pgfqpoint{5.902439in}{2.659242in}}%
\pgfpathlineto{\pgfqpoint{5.937992in}{2.752555in}}%
\pgfpathlineto{\pgfqpoint{5.924057in}{2.616008in}}%
\pgfpathlineto{\pgfqpoint{5.912749in}{2.626837in}}%
\pgfpathlineto{\pgfqpoint{5.878623in}{2.609256in}}%
\pgfpathlineto{\pgfqpoint{5.845510in}{2.652698in}}%
\pgfpathclose%
\pgfusepath{fill}%
\end{pgfscope}%
\begin{pgfscope}%
\pgfpathrectangle{\pgfqpoint{1.020000in}{0.880000in}}{\pgfqpoint{6.160000in}{6.160000in}}%
\pgfusepath{clip}%
\pgfsetbuttcap%
\pgfsetroundjoin%
\definecolor{currentfill}{rgb}{0.289996,0.386836,0.828926}%
\pgfsetfillcolor{currentfill}%
\pgfsetlinewidth{0.000000pt}%
\definecolor{currentstroke}{rgb}{0.000000,0.000000,0.000000}%
\pgfsetstrokecolor{currentstroke}%
\pgfsetdash{}{0pt}%
\pgfpathmoveto{\pgfqpoint{5.775396in}{2.507059in}}%
\pgfpathlineto{\pgfqpoint{5.787561in}{2.557479in}}%
\pgfpathlineto{\pgfqpoint{5.801420in}{2.709541in}}%
\pgfpathlineto{\pgfqpoint{5.832492in}{2.534560in}}%
\pgfpathlineto{\pgfqpoint{5.868308in}{2.647149in}}%
\pgfpathlineto{\pgfqpoint{5.856432in}{2.622399in}}%
\pgfpathlineto{\pgfqpoint{5.845510in}{2.652698in}}%
\pgfpathlineto{\pgfqpoint{5.809055in}{2.495581in}}%
\pgfpathlineto{\pgfqpoint{5.775396in}{2.507059in}}%
\pgfpathclose%
\pgfusepath{fill}%
\end{pgfscope}%
\begin{pgfscope}%
\pgfpathrectangle{\pgfqpoint{1.020000in}{0.880000in}}{\pgfqpoint{6.160000in}{6.160000in}}%
\pgfusepath{clip}%
\pgfsetbuttcap%
\pgfsetroundjoin%
\definecolor{currentfill}{rgb}{0.343278,0.459354,0.884122}%
\pgfsetfillcolor{currentfill}%
\pgfsetlinewidth{0.000000pt}%
\definecolor{currentstroke}{rgb}{0.000000,0.000000,0.000000}%
\pgfsetstrokecolor{currentstroke}%
\pgfsetdash{}{0pt}%
\pgfpathmoveto{\pgfqpoint{5.101010in}{2.760963in}}%
\pgfpathlineto{\pgfqpoint{5.110191in}{2.607180in}}%
\pgfpathlineto{\pgfqpoint{5.119002in}{2.415374in}}%
\pgfpathlineto{\pgfqpoint{5.156789in}{2.796211in}}%
\pgfpathlineto{\pgfqpoint{5.190890in}{2.789757in}}%
\pgfpathlineto{\pgfqpoint{5.178644in}{2.637019in}}%
\pgfpathlineto{\pgfqpoint{5.168902in}{2.730820in}}%
\pgfpathlineto{\pgfqpoint{5.136030in}{2.856333in}}%
\pgfpathlineto{\pgfqpoint{5.101010in}{2.760963in}}%
\pgfpathclose%
\pgfusepath{fill}%
\end{pgfscope}%
\begin{pgfscope}%
\pgfpathrectangle{\pgfqpoint{1.020000in}{0.880000in}}{\pgfqpoint{6.160000in}{6.160000in}}%
\pgfusepath{clip}%
\pgfsetbuttcap%
\pgfsetroundjoin%
\definecolor{currentfill}{rgb}{0.758539,0.832787,0.958408}%
\pgfsetfillcolor{currentfill}%
\pgfsetlinewidth{0.000000pt}%
\definecolor{currentstroke}{rgb}{0.000000,0.000000,0.000000}%
\pgfsetstrokecolor{currentstroke}%
\pgfsetdash{}{0pt}%
\pgfpathmoveto{\pgfqpoint{3.326883in}{3.534630in}}%
\pgfpathlineto{\pgfqpoint{3.336623in}{3.407781in}}%
\pgfpathlineto{\pgfqpoint{3.344573in}{3.494250in}}%
\pgfpathlineto{\pgfqpoint{3.379530in}{3.462512in}}%
\pgfpathlineto{\pgfqpoint{3.414671in}{3.401185in}}%
\pgfpathlineto{\pgfqpoint{3.403429in}{3.722452in}}%
\pgfpathlineto{\pgfqpoint{3.396758in}{3.457596in}}%
\pgfpathlineto{\pgfqpoint{3.362507in}{3.415424in}}%
\pgfpathlineto{\pgfqpoint{3.326883in}{3.534630in}}%
\pgfpathclose%
\pgfusepath{fill}%
\end{pgfscope}%
\begin{pgfscope}%
\pgfpathrectangle{\pgfqpoint{1.020000in}{0.880000in}}{\pgfqpoint{6.160000in}{6.160000in}}%
\pgfusepath{clip}%
\pgfsetbuttcap%
\pgfsetroundjoin%
\definecolor{currentfill}{rgb}{0.835345,0.860514,0.898970}%
\pgfsetfillcolor{currentfill}%
\pgfsetlinewidth{0.000000pt}%
\definecolor{currentstroke}{rgb}{0.000000,0.000000,0.000000}%
\pgfsetstrokecolor{currentstroke}%
\pgfsetdash{}{0pt}%
\pgfpathmoveto{\pgfqpoint{4.053250in}{3.631317in}}%
\pgfpathlineto{\pgfqpoint{4.062552in}{3.758082in}}%
\pgfpathlineto{\pgfqpoint{4.072125in}{3.680039in}}%
\pgfpathlineto{\pgfqpoint{4.106681in}{3.585954in}}%
\pgfpathlineto{\pgfqpoint{4.141050in}{3.732840in}}%
\pgfpathlineto{\pgfqpoint{4.131566in}{3.597441in}}%
\pgfpathlineto{\pgfqpoint{4.122086in}{3.530542in}}%
\pgfpathlineto{\pgfqpoint{4.087517in}{3.768203in}}%
\pgfpathlineto{\pgfqpoint{4.053250in}{3.631317in}}%
\pgfpathclose%
\pgfusepath{fill}%
\end{pgfscope}%
\begin{pgfscope}%
\pgfpathrectangle{\pgfqpoint{1.020000in}{0.880000in}}{\pgfqpoint{6.160000in}{6.160000in}}%
\pgfusepath{clip}%
\pgfsetbuttcap%
\pgfsetroundjoin%
\definecolor{currentfill}{rgb}{0.718985,0.811993,0.977656}%
\pgfsetfillcolor{currentfill}%
\pgfsetlinewidth{0.000000pt}%
\definecolor{currentstroke}{rgb}{0.000000,0.000000,0.000000}%
\pgfsetstrokecolor{currentstroke}%
\pgfsetdash{}{0pt}%
\pgfpathmoveto{\pgfqpoint{4.297932in}{3.486989in}}%
\pgfpathlineto{\pgfqpoint{4.307457in}{3.319918in}}%
\pgfpathlineto{\pgfqpoint{4.317230in}{3.346006in}}%
\pgfpathlineto{\pgfqpoint{4.351699in}{3.397767in}}%
\pgfpathlineto{\pgfqpoint{4.385789in}{3.255428in}}%
\pgfpathlineto{\pgfqpoint{4.376590in}{3.567098in}}%
\pgfpathlineto{\pgfqpoint{4.366306in}{3.308420in}}%
\pgfpathlineto{\pgfqpoint{4.332432in}{3.579992in}}%
\pgfpathlineto{\pgfqpoint{4.297932in}{3.486989in}}%
\pgfpathclose%
\pgfusepath{fill}%
\end{pgfscope}%
\begin{pgfscope}%
\pgfpathrectangle{\pgfqpoint{1.020000in}{0.880000in}}{\pgfqpoint{6.160000in}{6.160000in}}%
\pgfusepath{clip}%
\pgfsetbuttcap%
\pgfsetroundjoin%
\definecolor{currentfill}{rgb}{0.839351,0.861167,0.894494}%
\pgfsetfillcolor{currentfill}%
\pgfsetlinewidth{0.000000pt}%
\definecolor{currentstroke}{rgb}{0.000000,0.000000,0.000000}%
\pgfsetstrokecolor{currentstroke}%
\pgfsetdash{}{0pt}%
\pgfpathmoveto{\pgfqpoint{3.533116in}{3.738769in}}%
\pgfpathlineto{\pgfqpoint{3.542182in}{3.711972in}}%
\pgfpathlineto{\pgfqpoint{3.552375in}{3.508815in}}%
\pgfpathlineto{\pgfqpoint{3.586055in}{3.659985in}}%
\pgfpathlineto{\pgfqpoint{3.620442in}{3.699839in}}%
\pgfpathlineto{\pgfqpoint{3.611705in}{3.657077in}}%
\pgfpathlineto{\pgfqpoint{3.602055in}{3.776551in}}%
\pgfpathlineto{\pgfqpoint{3.568352in}{3.633374in}}%
\pgfpathlineto{\pgfqpoint{3.533116in}{3.738769in}}%
\pgfpathclose%
\pgfusepath{fill}%
\end{pgfscope}%
\begin{pgfscope}%
\pgfpathrectangle{\pgfqpoint{1.020000in}{0.880000in}}{\pgfqpoint{6.160000in}{6.160000in}}%
\pgfusepath{clip}%
\pgfsetbuttcap%
\pgfsetroundjoin%
\definecolor{currentfill}{rgb}{0.791392,0.846750,0.936641}%
\pgfsetfillcolor{currentfill}%
\pgfsetlinewidth{0.000000pt}%
\definecolor{currentstroke}{rgb}{0.000000,0.000000,0.000000}%
\pgfsetstrokecolor{currentstroke}%
\pgfsetdash{}{0pt}%
\pgfpathmoveto{\pgfqpoint{3.396758in}{3.457596in}}%
\pgfpathlineto{\pgfqpoint{3.403429in}{3.722452in}}%
\pgfpathlineto{\pgfqpoint{3.414671in}{3.401185in}}%
\pgfpathlineto{\pgfqpoint{3.448190in}{3.552118in}}%
\pgfpathlineto{\pgfqpoint{3.482867in}{3.549586in}}%
\pgfpathlineto{\pgfqpoint{3.474279in}{3.515760in}}%
\pgfpathlineto{\pgfqpoint{3.463913in}{3.733312in}}%
\pgfpathlineto{\pgfqpoint{3.430617in}{3.553145in}}%
\pgfpathlineto{\pgfqpoint{3.396758in}{3.457596in}}%
\pgfpathclose%
\pgfusepath{fill}%
\end{pgfscope}%
\begin{pgfscope}%
\pgfpathrectangle{\pgfqpoint{1.020000in}{0.880000in}}{\pgfqpoint{6.160000in}{6.160000in}}%
\pgfusepath{clip}%
\pgfsetbuttcap%
\pgfsetroundjoin%
\definecolor{currentfill}{rgb}{0.280550,0.373423,0.818011}%
\pgfsetfillcolor{currentfill}%
\pgfsetlinewidth{0.000000pt}%
\definecolor{currentstroke}{rgb}{0.000000,0.000000,0.000000}%
\pgfsetstrokecolor{currentstroke}%
\pgfsetdash{}{0pt}%
\pgfpathmoveto{\pgfqpoint{5.999001in}{2.381886in}}%
\pgfpathlineto{\pgfqpoint{6.013188in}{2.522627in}}%
\pgfpathlineto{\pgfqpoint{6.027249in}{2.653380in}}%
\pgfpathlineto{\pgfqpoint{6.059054in}{2.542276in}}%
\pgfpathlineto{\pgfqpoint{6.093851in}{2.591497in}}%
\pgfpathlineto{\pgfqpoint{6.078927in}{2.423458in}}%
\pgfpathlineto{\pgfqpoint{6.075570in}{2.859788in}}%
\pgfpathlineto{\pgfqpoint{6.036737in}{2.595282in}}%
\pgfpathlineto{\pgfqpoint{5.999001in}{2.381886in}}%
\pgfpathclose%
\pgfusepath{fill}%
\end{pgfscope}%
\begin{pgfscope}%
\pgfpathrectangle{\pgfqpoint{1.020000in}{0.880000in}}{\pgfqpoint{6.160000in}{6.160000in}}%
\pgfusepath{clip}%
\pgfsetbuttcap%
\pgfsetroundjoin%
\definecolor{currentfill}{rgb}{0.257234,0.339661,0.789661}%
\pgfsetfillcolor{currentfill}%
\pgfsetlinewidth{0.000000pt}%
\definecolor{currentstroke}{rgb}{0.000000,0.000000,0.000000}%
\pgfsetstrokecolor{currentstroke}%
\pgfsetdash{}{0pt}%
\pgfpathmoveto{\pgfqpoint{5.937992in}{2.752555in}}%
\pgfpathlineto{\pgfqpoint{5.941687in}{2.310044in}}%
\pgfpathlineto{\pgfqpoint{5.954997in}{2.409009in}}%
\pgfpathlineto{\pgfqpoint{5.991192in}{2.536731in}}%
\pgfpathlineto{\pgfqpoint{6.027249in}{2.653380in}}%
\pgfpathlineto{\pgfqpoint{6.013188in}{2.522627in}}%
\pgfpathlineto{\pgfqpoint{5.999001in}{2.381886in}}%
\pgfpathlineto{\pgfqpoint{5.968345in}{2.554704in}}%
\pgfpathlineto{\pgfqpoint{5.937992in}{2.752555in}}%
\pgfpathclose%
\pgfusepath{fill}%
\end{pgfscope}%
\begin{pgfscope}%
\pgfpathrectangle{\pgfqpoint{1.020000in}{0.880000in}}{\pgfqpoint{6.160000in}{6.160000in}}%
\pgfusepath{clip}%
\pgfsetbuttcap%
\pgfsetroundjoin%
\definecolor{currentfill}{rgb}{0.457046,0.594006,0.963029}%
\pgfsetfillcolor{currentfill}%
\pgfsetlinewidth{0.000000pt}%
\definecolor{currentstroke}{rgb}{0.000000,0.000000,0.000000}%
\pgfsetstrokecolor{currentstroke}%
\pgfsetdash{}{0pt}%
\pgfpathmoveto{\pgfqpoint{4.787100in}{2.898013in}}%
\pgfpathlineto{\pgfqpoint{4.798200in}{3.033257in}}%
\pgfpathlineto{\pgfqpoint{4.807841in}{2.928202in}}%
\pgfpathlineto{\pgfqpoint{4.841029in}{2.765844in}}%
\pgfpathlineto{\pgfqpoint{4.876690in}{2.977642in}}%
\pgfpathlineto{\pgfqpoint{4.865741in}{2.892413in}}%
\pgfpathlineto{\pgfqpoint{4.855459in}{2.899860in}}%
\pgfpathlineto{\pgfqpoint{4.821563in}{2.941583in}}%
\pgfpathlineto{\pgfqpoint{4.787100in}{2.898013in}}%
\pgfpathclose%
\pgfusepath{fill}%
\end{pgfscope}%
\begin{pgfscope}%
\pgfpathrectangle{\pgfqpoint{1.020000in}{0.880000in}}{\pgfqpoint{6.160000in}{6.160000in}}%
\pgfusepath{clip}%
\pgfsetbuttcap%
\pgfsetroundjoin%
\definecolor{currentfill}{rgb}{0.289996,0.386836,0.828926}%
\pgfsetfillcolor{currentfill}%
\pgfsetlinewidth{0.000000pt}%
\definecolor{currentstroke}{rgb}{0.000000,0.000000,0.000000}%
\pgfsetstrokecolor{currentstroke}%
\pgfsetdash{}{0pt}%
\pgfpathmoveto{\pgfqpoint{6.297258in}{2.622035in}}%
\pgfpathlineto{\pgfqpoint{6.309131in}{2.618988in}}%
\pgfpathlineto{\pgfqpoint{6.317917in}{2.471524in}}%
\pgfpathlineto{\pgfqpoint{6.353573in}{2.558755in}}%
\pgfpathlineto{\pgfqpoint{6.343768in}{2.659155in}}%
\pgfpathlineto{\pgfqpoint{6.330869in}{2.616536in}}%
\pgfpathlineto{\pgfqpoint{6.297258in}{2.622035in}}%
\pgfpathclose%
\pgfusepath{fill}%
\end{pgfscope}%
\begin{pgfscope}%
\pgfpathrectangle{\pgfqpoint{1.020000in}{0.880000in}}{\pgfqpoint{6.160000in}{6.160000in}}%
\pgfusepath{clip}%
\pgfsetbuttcap%
\pgfsetroundjoin%
\definecolor{currentfill}{rgb}{0.646113,0.764436,0.996868}%
\pgfsetfillcolor{currentfill}%
\pgfsetlinewidth{0.000000pt}%
\definecolor{currentstroke}{rgb}{0.000000,0.000000,0.000000}%
\pgfsetstrokecolor{currentstroke}%
\pgfsetdash{}{0pt}%
\pgfpathmoveto{\pgfqpoint{3.052554in}{3.217172in}}%
\pgfpathlineto{\pgfqpoint{3.059881in}{3.308594in}}%
\pgfpathlineto{\pgfqpoint{3.070143in}{3.140146in}}%
\pgfpathlineto{\pgfqpoint{3.103049in}{3.310581in}}%
\pgfpathlineto{\pgfqpoint{3.137028in}{3.387811in}}%
\pgfpathlineto{\pgfqpoint{3.129780in}{3.272959in}}%
\pgfpathlineto{\pgfqpoint{3.121880in}{3.222974in}}%
\pgfpathlineto{\pgfqpoint{3.086672in}{3.270601in}}%
\pgfpathlineto{\pgfqpoint{3.052554in}{3.217172in}}%
\pgfpathclose%
\pgfusepath{fill}%
\end{pgfscope}%
\begin{pgfscope}%
\pgfpathrectangle{\pgfqpoint{1.020000in}{0.880000in}}{\pgfqpoint{6.160000in}{6.160000in}}%
\pgfusepath{clip}%
\pgfsetbuttcap%
\pgfsetroundjoin%
\definecolor{currentfill}{rgb}{0.294718,0.393542,0.834384}%
\pgfsetfillcolor{currentfill}%
\pgfsetlinewidth{0.000000pt}%
\definecolor{currentstroke}{rgb}{0.000000,0.000000,0.000000}%
\pgfsetstrokecolor{currentstroke}%
\pgfsetdash{}{0pt}%
\pgfpathmoveto{\pgfqpoint{5.482630in}{2.571592in}}%
\pgfpathlineto{\pgfqpoint{5.494828in}{2.659586in}}%
\pgfpathlineto{\pgfqpoint{5.504726in}{2.571744in}}%
\pgfpathlineto{\pgfqpoint{5.539241in}{2.606673in}}%
\pgfpathlineto{\pgfqpoint{5.573818in}{2.645865in}}%
\pgfpathlineto{\pgfqpoint{5.561940in}{2.592516in}}%
\pgfpathlineto{\pgfqpoint{5.550975in}{2.602708in}}%
\pgfpathlineto{\pgfqpoint{5.515689in}{2.503959in}}%
\pgfpathlineto{\pgfqpoint{5.482630in}{2.571592in}}%
\pgfpathclose%
\pgfusepath{fill}%
\end{pgfscope}%
\begin{pgfscope}%
\pgfpathrectangle{\pgfqpoint{1.020000in}{0.880000in}}{\pgfqpoint{6.160000in}{6.160000in}}%
\pgfusepath{clip}%
\pgfsetbuttcap%
\pgfsetroundjoin%
\definecolor{currentfill}{rgb}{0.863392,0.865084,0.867634}%
\pgfsetfillcolor{currentfill}%
\pgfsetlinewidth{0.000000pt}%
\definecolor{currentstroke}{rgb}{0.000000,0.000000,0.000000}%
\pgfsetstrokecolor{currentstroke}%
\pgfsetdash{}{0pt}%
\pgfpathmoveto{\pgfqpoint{3.827478in}{3.776658in}}%
\pgfpathlineto{\pgfqpoint{3.837157in}{3.643177in}}%
\pgfpathlineto{\pgfqpoint{3.846130in}{3.718347in}}%
\pgfpathlineto{\pgfqpoint{3.880145in}{3.897008in}}%
\pgfpathlineto{\pgfqpoint{3.915590in}{3.578735in}}%
\pgfpathlineto{\pgfqpoint{3.905863in}{3.736866in}}%
\pgfpathlineto{\pgfqpoint{3.896356in}{3.814568in}}%
\pgfpathlineto{\pgfqpoint{3.862475in}{3.622596in}}%
\pgfpathlineto{\pgfqpoint{3.827478in}{3.776658in}}%
\pgfpathclose%
\pgfusepath{fill}%
\end{pgfscope}%
\begin{pgfscope}%
\pgfpathrectangle{\pgfqpoint{1.020000in}{0.880000in}}{\pgfqpoint{6.160000in}{6.160000in}}%
\pgfusepath{clip}%
\pgfsetbuttcap%
\pgfsetroundjoin%
\definecolor{currentfill}{rgb}{0.516260,0.654498,0.986407}%
\pgfsetfillcolor{currentfill}%
\pgfsetlinewidth{0.000000pt}%
\definecolor{currentstroke}{rgb}{0.000000,0.000000,0.000000}%
\pgfsetstrokecolor{currentstroke}%
\pgfsetdash{}{0pt}%
\pgfpathmoveto{\pgfqpoint{4.629610in}{2.817530in}}%
\pgfpathlineto{\pgfqpoint{4.641474in}{3.207448in}}%
\pgfpathlineto{\pgfqpoint{4.650673in}{3.007972in}}%
\pgfpathlineto{\pgfqpoint{4.684565in}{2.925941in}}%
\pgfpathlineto{\pgfqpoint{4.719210in}{2.998207in}}%
\pgfpathlineto{\pgfqpoint{4.709640in}{3.114370in}}%
\pgfpathlineto{\pgfqpoint{4.699154in}{3.053703in}}%
\pgfpathlineto{\pgfqpoint{4.665018in}{3.073427in}}%
\pgfpathlineto{\pgfqpoint{4.629610in}{2.817530in}}%
\pgfpathclose%
\pgfusepath{fill}%
\end{pgfscope}%
\begin{pgfscope}%
\pgfpathrectangle{\pgfqpoint{1.020000in}{0.880000in}}{\pgfqpoint{6.160000in}{6.160000in}}%
\pgfusepath{clip}%
\pgfsetbuttcap%
\pgfsetroundjoin%
\definecolor{currentfill}{rgb}{0.275827,0.366717,0.812553}%
\pgfsetfillcolor{currentfill}%
\pgfsetlinewidth{0.000000pt}%
\definecolor{currentstroke}{rgb}{0.000000,0.000000,0.000000}%
\pgfsetstrokecolor{currentstroke}%
\pgfsetdash{}{0pt}%
\pgfpathmoveto{\pgfqpoint{6.226821in}{2.480787in}}%
\pgfpathlineto{\pgfqpoint{6.238894in}{2.491340in}}%
\pgfpathlineto{\pgfqpoint{6.255286in}{2.707426in}}%
\pgfpathlineto{\pgfqpoint{6.284575in}{2.491575in}}%
\pgfpathlineto{\pgfqpoint{6.317917in}{2.471524in}}%
\pgfpathlineto{\pgfqpoint{6.309131in}{2.618988in}}%
\pgfpathlineto{\pgfqpoint{6.297258in}{2.622035in}}%
\pgfpathlineto{\pgfqpoint{6.263157in}{2.605344in}}%
\pgfpathlineto{\pgfqpoint{6.226821in}{2.480787in}}%
\pgfpathclose%
\pgfusepath{fill}%
\end{pgfscope}%
\begin{pgfscope}%
\pgfpathrectangle{\pgfqpoint{1.020000in}{0.880000in}}{\pgfqpoint{6.160000in}{6.160000in}}%
\pgfusepath{clip}%
\pgfsetbuttcap%
\pgfsetroundjoin%
\definecolor{currentfill}{rgb}{0.318832,0.426605,0.859857}%
\pgfsetfillcolor{currentfill}%
\pgfsetlinewidth{0.000000pt}%
\definecolor{currentstroke}{rgb}{0.000000,0.000000,0.000000}%
\pgfsetstrokecolor{currentstroke}%
\pgfsetdash{}{0pt}%
\pgfpathmoveto{\pgfqpoint{5.257584in}{2.644775in}}%
\pgfpathlineto{\pgfqpoint{5.268504in}{2.656729in}}%
\pgfpathlineto{\pgfqpoint{5.279085in}{2.635585in}}%
\pgfpathlineto{\pgfqpoint{5.313535in}{2.664833in}}%
\pgfpathlineto{\pgfqpoint{5.346157in}{2.538436in}}%
\pgfpathlineto{\pgfqpoint{5.336636in}{2.654526in}}%
\pgfpathlineto{\pgfqpoint{5.326081in}{2.680901in}}%
\pgfpathlineto{\pgfqpoint{5.292098in}{2.686255in}}%
\pgfpathlineto{\pgfqpoint{5.257584in}{2.644775in}}%
\pgfpathclose%
\pgfusepath{fill}%
\end{pgfscope}%
\begin{pgfscope}%
\pgfpathrectangle{\pgfqpoint{1.020000in}{0.880000in}}{\pgfqpoint{6.160000in}{6.160000in}}%
\pgfusepath{clip}%
\pgfsetbuttcap%
\pgfsetroundjoin%
\definecolor{currentfill}{rgb}{0.338377,0.452819,0.879317}%
\pgfsetfillcolor{currentfill}%
\pgfsetlinewidth{0.000000pt}%
\definecolor{currentstroke}{rgb}{0.000000,0.000000,0.000000}%
\pgfsetstrokecolor{currentstroke}%
\pgfsetdash{}{0pt}%
\pgfpathmoveto{\pgfqpoint{5.033860in}{2.892454in}}%
\pgfpathlineto{\pgfqpoint{5.042424in}{2.661649in}}%
\pgfpathlineto{\pgfqpoint{5.053007in}{2.664153in}}%
\pgfpathlineto{\pgfqpoint{5.087862in}{2.737774in}}%
\pgfpathlineto{\pgfqpoint{5.119002in}{2.415374in}}%
\pgfpathlineto{\pgfqpoint{5.110191in}{2.607180in}}%
\pgfpathlineto{\pgfqpoint{5.101010in}{2.760963in}}%
\pgfpathlineto{\pgfqpoint{5.066483in}{2.715925in}}%
\pgfpathlineto{\pgfqpoint{5.033860in}{2.892454in}}%
\pgfpathclose%
\pgfusepath{fill}%
\end{pgfscope}%
\begin{pgfscope}%
\pgfpathrectangle{\pgfqpoint{1.020000in}{0.880000in}}{\pgfqpoint{6.160000in}{6.160000in}}%
\pgfusepath{clip}%
\pgfsetbuttcap%
\pgfsetroundjoin%
\definecolor{currentfill}{rgb}{0.843358,0.861820,0.890017}%
\pgfsetfillcolor{currentfill}%
\pgfsetlinewidth{0.000000pt}%
\definecolor{currentstroke}{rgb}{0.000000,0.000000,0.000000}%
\pgfsetstrokecolor{currentstroke}%
\pgfsetdash{}{0pt}%
\pgfpathmoveto{\pgfqpoint{3.758753in}{3.694796in}}%
\pgfpathlineto{\pgfqpoint{3.767900in}{3.693313in}}%
\pgfpathlineto{\pgfqpoint{3.777088in}{3.686577in}}%
\pgfpathlineto{\pgfqpoint{3.812419in}{3.485325in}}%
\pgfpathlineto{\pgfqpoint{3.846130in}{3.718347in}}%
\pgfpathlineto{\pgfqpoint{3.837157in}{3.643177in}}%
\pgfpathlineto{\pgfqpoint{3.827478in}{3.776658in}}%
\pgfpathlineto{\pgfqpoint{3.793088in}{3.741707in}}%
\pgfpathlineto{\pgfqpoint{3.758753in}{3.694796in}}%
\pgfpathclose%
\pgfusepath{fill}%
\end{pgfscope}%
\begin{pgfscope}%
\pgfpathrectangle{\pgfqpoint{1.020000in}{0.880000in}}{\pgfqpoint{6.160000in}{6.160000in}}%
\pgfusepath{clip}%
\pgfsetbuttcap%
\pgfsetroundjoin%
\definecolor{currentfill}{rgb}{0.753611,0.830233,0.960871}%
\pgfsetfillcolor{currentfill}%
\pgfsetlinewidth{0.000000pt}%
\definecolor{currentstroke}{rgb}{0.000000,0.000000,0.000000}%
\pgfsetstrokecolor{currentstroke}%
\pgfsetdash{}{0pt}%
\pgfpathmoveto{\pgfqpoint{3.258065in}{3.476660in}}%
\pgfpathlineto{\pgfqpoint{3.266490in}{3.494712in}}%
\pgfpathlineto{\pgfqpoint{3.274745in}{3.533844in}}%
\pgfpathlineto{\pgfqpoint{3.310295in}{3.443173in}}%
\pgfpathlineto{\pgfqpoint{3.344573in}{3.494250in}}%
\pgfpathlineto{\pgfqpoint{3.336623in}{3.407781in}}%
\pgfpathlineto{\pgfqpoint{3.326883in}{3.534630in}}%
\pgfpathlineto{\pgfqpoint{3.293464in}{3.394492in}}%
\pgfpathlineto{\pgfqpoint{3.258065in}{3.476660in}}%
\pgfpathclose%
\pgfusepath{fill}%
\end{pgfscope}%
\begin{pgfscope}%
\pgfpathrectangle{\pgfqpoint{1.020000in}{0.880000in}}{\pgfqpoint{6.160000in}{6.160000in}}%
\pgfusepath{clip}%
\pgfsetbuttcap%
\pgfsetroundjoin%
\definecolor{currentfill}{rgb}{0.576051,0.708780,0.997755}%
\pgfsetfillcolor{currentfill}%
\pgfsetlinewidth{0.000000pt}%
\definecolor{currentstroke}{rgb}{0.000000,0.000000,0.000000}%
\pgfsetstrokecolor{currentstroke}%
\pgfsetdash{}{0pt}%
\pgfpathmoveto{\pgfqpoint{4.474447in}{3.343766in}}%
\pgfpathlineto{\pgfqpoint{4.484285in}{3.308692in}}%
\pgfpathlineto{\pgfqpoint{4.493401in}{3.037049in}}%
\pgfpathlineto{\pgfqpoint{4.527910in}{3.088502in}}%
\pgfpathlineto{\pgfqpoint{4.562176in}{3.072546in}}%
\pgfpathlineto{\pgfqpoint{4.552173in}{3.074010in}}%
\pgfpathlineto{\pgfqpoint{4.541894in}{2.992506in}}%
\pgfpathlineto{\pgfqpoint{4.508283in}{3.180986in}}%
\pgfpathlineto{\pgfqpoint{4.474447in}{3.343766in}}%
\pgfpathclose%
\pgfusepath{fill}%
\end{pgfscope}%
\begin{pgfscope}%
\pgfpathrectangle{\pgfqpoint{1.020000in}{0.880000in}}{\pgfqpoint{6.160000in}{6.160000in}}%
\pgfusepath{clip}%
\pgfsetbuttcap%
\pgfsetroundjoin%
\definecolor{currentfill}{rgb}{0.831148,0.859513,0.903110}%
\pgfsetfillcolor{currentfill}%
\pgfsetlinewidth{0.000000pt}%
\definecolor{currentstroke}{rgb}{0.000000,0.000000,0.000000}%
\pgfsetstrokecolor{currentstroke}%
\pgfsetdash{}{0pt}%
\pgfpathmoveto{\pgfqpoint{3.984553in}{3.549871in}}%
\pgfpathlineto{\pgfqpoint{3.993727in}{3.657700in}}%
\pgfpathlineto{\pgfqpoint{4.003023in}{3.719574in}}%
\pgfpathlineto{\pgfqpoint{4.037699in}{3.625751in}}%
\pgfpathlineto{\pgfqpoint{4.072125in}{3.680039in}}%
\pgfpathlineto{\pgfqpoint{4.062552in}{3.758082in}}%
\pgfpathlineto{\pgfqpoint{4.053250in}{3.631317in}}%
\pgfpathlineto{\pgfqpoint{4.018953in}{3.557250in}}%
\pgfpathlineto{\pgfqpoint{3.984553in}{3.549871in}}%
\pgfpathclose%
\pgfusepath{fill}%
\end{pgfscope}%
\begin{pgfscope}%
\pgfpathrectangle{\pgfqpoint{1.020000in}{0.880000in}}{\pgfqpoint{6.160000in}{6.160000in}}%
\pgfusepath{clip}%
\pgfsetbuttcap%
\pgfsetroundjoin%
\definecolor{currentfill}{rgb}{0.693321,0.796314,0.986308}%
\pgfsetfillcolor{currentfill}%
\pgfsetlinewidth{0.000000pt}%
\definecolor{currentstroke}{rgb}{0.000000,0.000000,0.000000}%
\pgfsetstrokecolor{currentstroke}%
\pgfsetdash{}{0pt}%
\pgfpathmoveto{\pgfqpoint{3.121880in}{3.222974in}}%
\pgfpathlineto{\pgfqpoint{3.129780in}{3.272959in}}%
\pgfpathlineto{\pgfqpoint{3.137028in}{3.387811in}}%
\pgfpathlineto{\pgfqpoint{3.173276in}{3.243954in}}%
\pgfpathlineto{\pgfqpoint{3.206822in}{3.364273in}}%
\pgfpathlineto{\pgfqpoint{3.197307in}{3.464314in}}%
\pgfpathlineto{\pgfqpoint{3.188090in}{3.534624in}}%
\pgfpathlineto{\pgfqpoint{3.155372in}{3.336950in}}%
\pgfpathlineto{\pgfqpoint{3.121880in}{3.222974in}}%
\pgfpathclose%
\pgfusepath{fill}%
\end{pgfscope}%
\begin{pgfscope}%
\pgfpathrectangle{\pgfqpoint{1.020000in}{0.880000in}}{\pgfqpoint{6.160000in}{6.160000in}}%
\pgfusepath{clip}%
\pgfsetbuttcap%
\pgfsetroundjoin%
\definecolor{currentfill}{rgb}{0.304174,0.406945,0.845263}%
\pgfsetfillcolor{currentfill}%
\pgfsetlinewidth{0.000000pt}%
\definecolor{currentstroke}{rgb}{0.000000,0.000000,0.000000}%
\pgfsetstrokecolor{currentstroke}%
\pgfsetdash{}{0pt}%
\pgfpathmoveto{\pgfqpoint{5.708248in}{2.546683in}}%
\pgfpathlineto{\pgfqpoint{5.719709in}{2.557964in}}%
\pgfpathlineto{\pgfqpoint{5.730539in}{2.527126in}}%
\pgfpathlineto{\pgfqpoint{5.768948in}{2.805662in}}%
\pgfpathlineto{\pgfqpoint{5.801420in}{2.709541in}}%
\pgfpathlineto{\pgfqpoint{5.787561in}{2.557479in}}%
\pgfpathlineto{\pgfqpoint{5.775396in}{2.507059in}}%
\pgfpathlineto{\pgfqpoint{5.744101in}{2.670721in}}%
\pgfpathlineto{\pgfqpoint{5.708248in}{2.546683in}}%
\pgfpathclose%
\pgfusepath{fill}%
\end{pgfscope}%
\begin{pgfscope}%
\pgfpathrectangle{\pgfqpoint{1.020000in}{0.880000in}}{\pgfqpoint{6.160000in}{6.160000in}}%
\pgfusepath{clip}%
\pgfsetbuttcap%
\pgfsetroundjoin%
\definecolor{currentfill}{rgb}{0.548876,0.685104,0.994379}%
\pgfsetfillcolor{currentfill}%
\pgfsetlinewidth{0.000000pt}%
\definecolor{currentstroke}{rgb}{0.000000,0.000000,0.000000}%
\pgfsetstrokecolor{currentstroke}%
\pgfsetdash{}{0pt}%
\pgfpathmoveto{\pgfqpoint{2.777192in}{3.001786in}}%
\pgfpathlineto{\pgfqpoint{2.787489in}{2.841888in}}%
\pgfpathlineto{\pgfqpoint{2.791824in}{3.108529in}}%
\pgfpathlineto{\pgfqpoint{2.826598in}{3.118452in}}%
\pgfpathlineto{\pgfqpoint{2.859673in}{3.255405in}}%
\pgfpathlineto{\pgfqpoint{2.853985in}{3.069912in}}%
\pgfpathlineto{\pgfqpoint{2.844575in}{3.165332in}}%
\pgfpathlineto{\pgfqpoint{2.811140in}{3.063584in}}%
\pgfpathlineto{\pgfqpoint{2.777192in}{3.001786in}}%
\pgfpathclose%
\pgfusepath{fill}%
\end{pgfscope}%
\begin{pgfscope}%
\pgfpathrectangle{\pgfqpoint{1.020000in}{0.880000in}}{\pgfqpoint{6.160000in}{6.160000in}}%
\pgfusepath{clip}%
\pgfsetbuttcap%
\pgfsetroundjoin%
\definecolor{currentfill}{rgb}{0.271104,0.360011,0.807095}%
\pgfsetfillcolor{currentfill}%
\pgfsetlinewidth{0.000000pt}%
\definecolor{currentstroke}{rgb}{0.000000,0.000000,0.000000}%
\pgfsetstrokecolor{currentstroke}%
\pgfsetdash{}{0pt}%
\pgfpathmoveto{\pgfqpoint{6.160868in}{2.558821in}}%
\pgfpathlineto{\pgfqpoint{6.171513in}{2.501484in}}%
\pgfpathlineto{\pgfqpoint{6.184350in}{2.552065in}}%
\pgfpathlineto{\pgfqpoint{6.216404in}{2.464269in}}%
\pgfpathlineto{\pgfqpoint{6.255286in}{2.707426in}}%
\pgfpathlineto{\pgfqpoint{6.238894in}{2.491340in}}%
\pgfpathlineto{\pgfqpoint{6.226821in}{2.480787in}}%
\pgfpathlineto{\pgfqpoint{6.195272in}{2.588984in}}%
\pgfpathlineto{\pgfqpoint{6.160868in}{2.558821in}}%
\pgfpathclose%
\pgfusepath{fill}%
\end{pgfscope}%
\begin{pgfscope}%
\pgfpathrectangle{\pgfqpoint{1.020000in}{0.880000in}}{\pgfqpoint{6.160000in}{6.160000in}}%
\pgfusepath{clip}%
\pgfsetbuttcap%
\pgfsetroundjoin%
\definecolor{currentfill}{rgb}{0.683056,0.790043,0.989768}%
\pgfsetfillcolor{currentfill}%
\pgfsetlinewidth{0.000000pt}%
\definecolor{currentstroke}{rgb}{0.000000,0.000000,0.000000}%
\pgfsetstrokecolor{currentstroke}%
\pgfsetdash{}{0pt}%
\pgfpathmoveto{\pgfqpoint{4.385789in}{3.255428in}}%
\pgfpathlineto{\pgfqpoint{4.395647in}{3.272194in}}%
\pgfpathlineto{\pgfqpoint{4.405746in}{3.386266in}}%
\pgfpathlineto{\pgfqpoint{4.440302in}{3.438106in}}%
\pgfpathlineto{\pgfqpoint{4.474447in}{3.343766in}}%
\pgfpathlineto{\pgfqpoint{4.464216in}{3.233717in}}%
\pgfpathlineto{\pgfqpoint{4.454676in}{3.362482in}}%
\pgfpathlineto{\pgfqpoint{4.420314in}{3.346760in}}%
\pgfpathlineto{\pgfqpoint{4.385789in}{3.255428in}}%
\pgfpathclose%
\pgfusepath{fill}%
\end{pgfscope}%
\begin{pgfscope}%
\pgfpathrectangle{\pgfqpoint{1.020000in}{0.880000in}}{\pgfqpoint{6.160000in}{6.160000in}}%
\pgfusepath{clip}%
\pgfsetbuttcap%
\pgfsetroundjoin%
\definecolor{currentfill}{rgb}{0.294718,0.393542,0.834384}%
\pgfsetfillcolor{currentfill}%
\pgfsetlinewidth{0.000000pt}%
\definecolor{currentstroke}{rgb}{0.000000,0.000000,0.000000}%
\pgfsetstrokecolor{currentstroke}%
\pgfsetdash{}{0pt}%
\pgfpathmoveto{\pgfqpoint{5.415723in}{2.659590in}}%
\pgfpathlineto{\pgfqpoint{5.425899in}{2.595684in}}%
\pgfpathlineto{\pgfqpoint{5.436898in}{2.596169in}}%
\pgfpathlineto{\pgfqpoint{5.470561in}{2.563457in}}%
\pgfpathlineto{\pgfqpoint{5.504726in}{2.571744in}}%
\pgfpathlineto{\pgfqpoint{5.494828in}{2.659586in}}%
\pgfpathlineto{\pgfqpoint{5.482630in}{2.571592in}}%
\pgfpathlineto{\pgfqpoint{5.448033in}{2.524040in}}%
\pgfpathlineto{\pgfqpoint{5.415723in}{2.659590in}}%
\pgfpathclose%
\pgfusepath{fill}%
\end{pgfscope}%
\begin{pgfscope}%
\pgfpathrectangle{\pgfqpoint{1.020000in}{0.880000in}}{\pgfqpoint{6.160000in}{6.160000in}}%
\pgfusepath{clip}%
\pgfsetbuttcap%
\pgfsetroundjoin%
\definecolor{currentfill}{rgb}{0.733898,0.820018,0.970724}%
\pgfsetfillcolor{currentfill}%
\pgfsetlinewidth{0.000000pt}%
\definecolor{currentstroke}{rgb}{0.000000,0.000000,0.000000}%
\pgfsetstrokecolor{currentstroke}%
\pgfsetdash{}{0pt}%
\pgfpathmoveto{\pgfqpoint{4.229169in}{3.534047in}}%
\pgfpathlineto{\pgfqpoint{4.238793in}{3.455914in}}%
\pgfpathlineto{\pgfqpoint{4.248522in}{3.529277in}}%
\pgfpathlineto{\pgfqpoint{4.282941in}{3.460674in}}%
\pgfpathlineto{\pgfqpoint{4.317230in}{3.346006in}}%
\pgfpathlineto{\pgfqpoint{4.307457in}{3.319918in}}%
\pgfpathlineto{\pgfqpoint{4.297932in}{3.486989in}}%
\pgfpathlineto{\pgfqpoint{4.263413in}{3.315786in}}%
\pgfpathlineto{\pgfqpoint{4.229169in}{3.534047in}}%
\pgfpathclose%
\pgfusepath{fill}%
\end{pgfscope}%
\begin{pgfscope}%
\pgfpathrectangle{\pgfqpoint{1.020000in}{0.880000in}}{\pgfqpoint{6.160000in}{6.160000in}}%
\pgfusepath{clip}%
\pgfsetbuttcap%
\pgfsetroundjoin%
\definecolor{currentfill}{rgb}{0.656683,0.771806,0.994914}%
\pgfsetfillcolor{currentfill}%
\pgfsetlinewidth{0.000000pt}%
\definecolor{currentstroke}{rgb}{0.000000,0.000000,0.000000}%
\pgfsetstrokecolor{currentstroke}%
\pgfsetdash{}{0pt}%
\pgfpathmoveto{\pgfqpoint{2.982261in}{3.285359in}}%
\pgfpathlineto{\pgfqpoint{2.989758in}{3.351763in}}%
\pgfpathlineto{\pgfqpoint{2.997907in}{3.365638in}}%
\pgfpathlineto{\pgfqpoint{3.034150in}{3.246373in}}%
\pgfpathlineto{\pgfqpoint{3.070143in}{3.140146in}}%
\pgfpathlineto{\pgfqpoint{3.059881in}{3.308594in}}%
\pgfpathlineto{\pgfqpoint{3.052554in}{3.217172in}}%
\pgfpathlineto{\pgfqpoint{3.016850in}{3.300931in}}%
\pgfpathlineto{\pgfqpoint{2.982261in}{3.285359in}}%
\pgfpathclose%
\pgfusepath{fill}%
\end{pgfscope}%
\begin{pgfscope}%
\pgfpathrectangle{\pgfqpoint{1.020000in}{0.880000in}}{\pgfqpoint{6.160000in}{6.160000in}}%
\pgfusepath{clip}%
\pgfsetbuttcap%
\pgfsetroundjoin%
\definecolor{currentfill}{rgb}{0.285273,0.380129,0.823469}%
\pgfsetfillcolor{currentfill}%
\pgfsetlinewidth{0.000000pt}%
\definecolor{currentstroke}{rgb}{0.000000,0.000000,0.000000}%
\pgfsetstrokecolor{currentstroke}%
\pgfsetdash{}{0pt}%
\pgfpathmoveto{\pgfqpoint{5.640237in}{2.536962in}}%
\pgfpathlineto{\pgfqpoint{5.650893in}{2.499115in}}%
\pgfpathlineto{\pgfqpoint{5.662529in}{2.525796in}}%
\pgfpathlineto{\pgfqpoint{5.699980in}{2.751425in}}%
\pgfpathlineto{\pgfqpoint{5.730539in}{2.527126in}}%
\pgfpathlineto{\pgfqpoint{5.719709in}{2.557964in}}%
\pgfpathlineto{\pgfqpoint{5.708248in}{2.546683in}}%
\pgfpathlineto{\pgfqpoint{5.675097in}{2.598071in}}%
\pgfpathlineto{\pgfqpoint{5.640237in}{2.536962in}}%
\pgfpathclose%
\pgfusepath{fill}%
\end{pgfscope}%
\begin{pgfscope}%
\pgfpathrectangle{\pgfqpoint{1.020000in}{0.880000in}}{\pgfqpoint{6.160000in}{6.160000in}}%
\pgfusepath{clip}%
\pgfsetbuttcap%
\pgfsetroundjoin%
\definecolor{currentfill}{rgb}{0.521696,0.659599,0.987736}%
\pgfsetfillcolor{currentfill}%
\pgfsetlinewidth{0.000000pt}%
\definecolor{currentstroke}{rgb}{0.000000,0.000000,0.000000}%
\pgfsetstrokecolor{currentstroke}%
\pgfsetdash{}{0pt}%
\pgfpathmoveto{\pgfqpoint{4.562176in}{3.072546in}}%
\pgfpathlineto{\pgfqpoint{4.572294in}{3.095199in}}%
\pgfpathlineto{\pgfqpoint{4.581620in}{2.913211in}}%
\pgfpathlineto{\pgfqpoint{4.616872in}{3.128829in}}%
\pgfpathlineto{\pgfqpoint{4.650673in}{3.007972in}}%
\pgfpathlineto{\pgfqpoint{4.641474in}{3.207448in}}%
\pgfpathlineto{\pgfqpoint{4.629610in}{2.817530in}}%
\pgfpathlineto{\pgfqpoint{4.596181in}{3.001486in}}%
\pgfpathlineto{\pgfqpoint{4.562176in}{3.072546in}}%
\pgfpathclose%
\pgfusepath{fill}%
\end{pgfscope}%
\begin{pgfscope}%
\pgfpathrectangle{\pgfqpoint{1.020000in}{0.880000in}}{\pgfqpoint{6.160000in}{6.160000in}}%
\pgfusepath{clip}%
\pgfsetbuttcap%
\pgfsetroundjoin%
\definecolor{currentfill}{rgb}{0.261805,0.346484,0.795658}%
\pgfsetfillcolor{currentfill}%
\pgfsetlinewidth{0.000000pt}%
\definecolor{currentstroke}{rgb}{0.000000,0.000000,0.000000}%
\pgfsetstrokecolor{currentstroke}%
\pgfsetdash{}{0pt}%
\pgfpathmoveto{\pgfqpoint{5.573818in}{2.645865in}}%
\pgfpathlineto{\pgfqpoint{5.582135in}{2.444607in}}%
\pgfpathlineto{\pgfqpoint{5.594837in}{2.552640in}}%
\pgfpathlineto{\pgfqpoint{5.625870in}{2.345353in}}%
\pgfpathlineto{\pgfqpoint{5.662529in}{2.525796in}}%
\pgfpathlineto{\pgfqpoint{5.650893in}{2.499115in}}%
\pgfpathlineto{\pgfqpoint{5.640237in}{2.536962in}}%
\pgfpathlineto{\pgfqpoint{5.606623in}{2.561184in}}%
\pgfpathlineto{\pgfqpoint{5.573818in}{2.645865in}}%
\pgfpathclose%
\pgfusepath{fill}%
\end{pgfscope}%
\begin{pgfscope}%
\pgfpathrectangle{\pgfqpoint{1.020000in}{0.880000in}}{\pgfqpoint{6.160000in}{6.160000in}}%
\pgfusepath{clip}%
\pgfsetbuttcap%
\pgfsetroundjoin%
\definecolor{currentfill}{rgb}{0.419991,0.552989,0.942630}%
\pgfsetfillcolor{currentfill}%
\pgfsetlinewidth{0.000000pt}%
\definecolor{currentstroke}{rgb}{0.000000,0.000000,0.000000}%
\pgfsetstrokecolor{currentstroke}%
\pgfsetdash{}{0pt}%
\pgfpathmoveto{\pgfqpoint{4.876690in}{2.977642in}}%
\pgfpathlineto{\pgfqpoint{4.886378in}{2.878722in}}%
\pgfpathlineto{\pgfqpoint{4.896961in}{2.904247in}}%
\pgfpathlineto{\pgfqpoint{4.930586in}{2.816978in}}%
\pgfpathlineto{\pgfqpoint{4.964478in}{2.773933in}}%
\pgfpathlineto{\pgfqpoint{4.953160in}{2.663074in}}%
\pgfpathlineto{\pgfqpoint{4.944404in}{2.880937in}}%
\pgfpathlineto{\pgfqpoint{4.909927in}{2.840742in}}%
\pgfpathlineto{\pgfqpoint{4.876690in}{2.977642in}}%
\pgfpathclose%
\pgfusepath{fill}%
\end{pgfscope}%
\begin{pgfscope}%
\pgfpathrectangle{\pgfqpoint{1.020000in}{0.880000in}}{\pgfqpoint{6.160000in}{6.160000in}}%
\pgfusepath{clip}%
\pgfsetbuttcap%
\pgfsetroundjoin%
\definecolor{currentfill}{rgb}{0.630089,0.752516,0.998508}%
\pgfsetfillcolor{currentfill}%
\pgfsetlinewidth{0.000000pt}%
\definecolor{currentstroke}{rgb}{0.000000,0.000000,0.000000}%
\pgfsetstrokecolor{currentstroke}%
\pgfsetdash{}{0pt}%
\pgfpathmoveto{\pgfqpoint{2.915402in}{3.069469in}}%
\pgfpathlineto{\pgfqpoint{2.921919in}{3.202656in}}%
\pgfpathlineto{\pgfqpoint{2.931504in}{3.094829in}}%
\pgfpathlineto{\pgfqpoint{2.964331in}{3.257851in}}%
\pgfpathlineto{\pgfqpoint{2.997907in}{3.365638in}}%
\pgfpathlineto{\pgfqpoint{2.989758in}{3.351763in}}%
\pgfpathlineto{\pgfqpoint{2.982261in}{3.285359in}}%
\pgfpathlineto{\pgfqpoint{2.948768in}{3.180329in}}%
\pgfpathlineto{\pgfqpoint{2.915402in}{3.069469in}}%
\pgfpathclose%
\pgfusepath{fill}%
\end{pgfscope}%
\begin{pgfscope}%
\pgfpathrectangle{\pgfqpoint{1.020000in}{0.880000in}}{\pgfqpoint{6.160000in}{6.160000in}}%
\pgfusepath{clip}%
\pgfsetbuttcap%
\pgfsetroundjoin%
\definecolor{currentfill}{rgb}{0.818056,0.855590,0.914638}%
\pgfsetfillcolor{currentfill}%
\pgfsetlinewidth{0.000000pt}%
\definecolor{currentstroke}{rgb}{0.000000,0.000000,0.000000}%
\pgfsetstrokecolor{currentstroke}%
\pgfsetdash{}{0pt}%
\pgfpathmoveto{\pgfqpoint{4.141050in}{3.732840in}}%
\pgfpathlineto{\pgfqpoint{4.150616in}{3.761576in}}%
\pgfpathlineto{\pgfqpoint{4.160252in}{3.597304in}}%
\pgfpathlineto{\pgfqpoint{4.194759in}{3.799553in}}%
\pgfpathlineto{\pgfqpoint{4.229169in}{3.534047in}}%
\pgfpathlineto{\pgfqpoint{4.219477in}{3.409732in}}%
\pgfpathlineto{\pgfqpoint{4.209890in}{3.509267in}}%
\pgfpathlineto{\pgfqpoint{4.175525in}{3.547476in}}%
\pgfpathlineto{\pgfqpoint{4.141050in}{3.732840in}}%
\pgfpathclose%
\pgfusepath{fill}%
\end{pgfscope}%
\begin{pgfscope}%
\pgfpathrectangle{\pgfqpoint{1.020000in}{0.880000in}}{\pgfqpoint{6.160000in}{6.160000in}}%
\pgfusepath{clip}%
\pgfsetbuttcap%
\pgfsetroundjoin%
\definecolor{currentfill}{rgb}{0.478462,0.616564,0.972721}%
\pgfsetfillcolor{currentfill}%
\pgfsetlinewidth{0.000000pt}%
\definecolor{currentstroke}{rgb}{0.000000,0.000000,0.000000}%
\pgfsetstrokecolor{currentstroke}%
\pgfsetdash{}{0pt}%
\pgfpathmoveto{\pgfqpoint{4.719210in}{2.998207in}}%
\pgfpathlineto{\pgfqpoint{4.729561in}{3.024756in}}%
\pgfpathlineto{\pgfqpoint{4.738884in}{2.862287in}}%
\pgfpathlineto{\pgfqpoint{4.773272in}{2.880081in}}%
\pgfpathlineto{\pgfqpoint{4.807841in}{2.928202in}}%
\pgfpathlineto{\pgfqpoint{4.798200in}{3.033257in}}%
\pgfpathlineto{\pgfqpoint{4.787100in}{2.898013in}}%
\pgfpathlineto{\pgfqpoint{4.753375in}{2.981591in}}%
\pgfpathlineto{\pgfqpoint{4.719210in}{2.998207in}}%
\pgfpathclose%
\pgfusepath{fill}%
\end{pgfscope}%
\begin{pgfscope}%
\pgfpathrectangle{\pgfqpoint{1.020000in}{0.880000in}}{\pgfqpoint{6.160000in}{6.160000in}}%
\pgfusepath{clip}%
\pgfsetbuttcap%
\pgfsetroundjoin%
\definecolor{currentfill}{rgb}{0.280550,0.373423,0.818011}%
\pgfsetfillcolor{currentfill}%
\pgfsetlinewidth{0.000000pt}%
\definecolor{currentstroke}{rgb}{0.000000,0.000000,0.000000}%
\pgfsetstrokecolor{currentstroke}%
\pgfsetdash{}{0pt}%
\pgfpathmoveto{\pgfqpoint{5.868308in}{2.647149in}}%
\pgfpathlineto{\pgfqpoint{5.878240in}{2.556365in}}%
\pgfpathlineto{\pgfqpoint{5.889633in}{2.550252in}}%
\pgfpathlineto{\pgfqpoint{5.923584in}{2.549956in}}%
\pgfpathlineto{\pgfqpoint{5.954997in}{2.409009in}}%
\pgfpathlineto{\pgfqpoint{5.941687in}{2.310044in}}%
\pgfpathlineto{\pgfqpoint{5.937992in}{2.752555in}}%
\pgfpathlineto{\pgfqpoint{5.902439in}{2.659242in}}%
\pgfpathlineto{\pgfqpoint{5.868308in}{2.647149in}}%
\pgfpathclose%
\pgfusepath{fill}%
\end{pgfscope}%
\begin{pgfscope}%
\pgfpathrectangle{\pgfqpoint{1.020000in}{0.880000in}}{\pgfqpoint{6.160000in}{6.160000in}}%
\pgfusepath{clip}%
\pgfsetbuttcap%
\pgfsetroundjoin%
\definecolor{currentfill}{rgb}{0.603162,0.731527,0.999565}%
\pgfsetfillcolor{currentfill}%
\pgfsetlinewidth{0.000000pt}%
\definecolor{currentstroke}{rgb}{0.000000,0.000000,0.000000}%
\pgfsetstrokecolor{currentstroke}%
\pgfsetdash{}{0pt}%
\pgfpathmoveto{\pgfqpoint{2.844575in}{3.165332in}}%
\pgfpathlineto{\pgfqpoint{2.853985in}{3.069912in}}%
\pgfpathlineto{\pgfqpoint{2.859673in}{3.255405in}}%
\pgfpathlineto{\pgfqpoint{2.893362in}{3.350561in}}%
\pgfpathlineto{\pgfqpoint{2.931504in}{3.094829in}}%
\pgfpathlineto{\pgfqpoint{2.921919in}{3.202656in}}%
\pgfpathlineto{\pgfqpoint{2.915402in}{3.069469in}}%
\pgfpathlineto{\pgfqpoint{2.878719in}{3.216596in}}%
\pgfpathlineto{\pgfqpoint{2.844575in}{3.165332in}}%
\pgfpathclose%
\pgfusepath{fill}%
\end{pgfscope}%
\begin{pgfscope}%
\pgfpathrectangle{\pgfqpoint{1.020000in}{0.880000in}}{\pgfqpoint{6.160000in}{6.160000in}}%
\pgfusepath{clip}%
\pgfsetbuttcap%
\pgfsetroundjoin%
\definecolor{currentfill}{rgb}{0.855378,0.863778,0.876587}%
\pgfsetfillcolor{currentfill}%
\pgfsetlinewidth{0.000000pt}%
\definecolor{currentstroke}{rgb}{0.000000,0.000000,0.000000}%
\pgfsetstrokecolor{currentstroke}%
\pgfsetdash{}{0pt}%
\pgfpathmoveto{\pgfqpoint{3.689698in}{3.685791in}}%
\pgfpathlineto{\pgfqpoint{3.698676in}{3.704896in}}%
\pgfpathlineto{\pgfqpoint{3.707884in}{3.680396in}}%
\pgfpathlineto{\pgfqpoint{3.742511in}{3.680003in}}%
\pgfpathlineto{\pgfqpoint{3.777088in}{3.686577in}}%
\pgfpathlineto{\pgfqpoint{3.767900in}{3.693313in}}%
\pgfpathlineto{\pgfqpoint{3.758753in}{3.694796in}}%
\pgfpathlineto{\pgfqpoint{3.723988in}{3.744218in}}%
\pgfpathlineto{\pgfqpoint{3.689698in}{3.685791in}}%
\pgfpathclose%
\pgfusepath{fill}%
\end{pgfscope}%
\begin{pgfscope}%
\pgfpathrectangle{\pgfqpoint{1.020000in}{0.880000in}}{\pgfqpoint{6.160000in}{6.160000in}}%
\pgfusepath{clip}%
\pgfsetbuttcap%
\pgfsetroundjoin%
\definecolor{currentfill}{rgb}{0.538004,0.674902,0.991722}%
\pgfsetfillcolor{currentfill}%
\pgfsetlinewidth{0.000000pt}%
\definecolor{currentstroke}{rgb}{0.000000,0.000000,0.000000}%
\pgfsetstrokecolor{currentstroke}%
\pgfsetdash{}{0pt}%
\pgfpathmoveto{\pgfqpoint{2.704244in}{3.223347in}}%
\pgfpathlineto{\pgfqpoint{2.713954in}{3.107252in}}%
\pgfpathlineto{\pgfqpoint{2.721414in}{3.144549in}}%
\pgfpathlineto{\pgfqpoint{2.758754in}{2.978544in}}%
\pgfpathlineto{\pgfqpoint{2.791824in}{3.108529in}}%
\pgfpathlineto{\pgfqpoint{2.787489in}{2.841888in}}%
\pgfpathlineto{\pgfqpoint{2.777192in}{3.001786in}}%
\pgfpathlineto{\pgfqpoint{2.741220in}{3.081315in}}%
\pgfpathlineto{\pgfqpoint{2.704244in}{3.223347in}}%
\pgfpathclose%
\pgfusepath{fill}%
\end{pgfscope}%
\begin{pgfscope}%
\pgfpathrectangle{\pgfqpoint{1.020000in}{0.880000in}}{\pgfqpoint{6.160000in}{6.160000in}}%
\pgfusepath{clip}%
\pgfsetbuttcap%
\pgfsetroundjoin%
\definecolor{currentfill}{rgb}{0.843358,0.861820,0.890017}%
\pgfsetfillcolor{currentfill}%
\pgfsetlinewidth{0.000000pt}%
\definecolor{currentstroke}{rgb}{0.000000,0.000000,0.000000}%
\pgfsetstrokecolor{currentstroke}%
\pgfsetdash{}{0pt}%
\pgfpathmoveto{\pgfqpoint{3.915590in}{3.578735in}}%
\pgfpathlineto{\pgfqpoint{3.924704in}{3.654832in}}%
\pgfpathlineto{\pgfqpoint{3.933801in}{3.751441in}}%
\pgfpathlineto{\pgfqpoint{3.968577in}{3.666114in}}%
\pgfpathlineto{\pgfqpoint{4.003023in}{3.719574in}}%
\pgfpathlineto{\pgfqpoint{3.993727in}{3.657700in}}%
\pgfpathlineto{\pgfqpoint{3.984553in}{3.549871in}}%
\pgfpathlineto{\pgfqpoint{3.949721in}{3.721634in}}%
\pgfpathlineto{\pgfqpoint{3.915590in}{3.578735in}}%
\pgfpathclose%
\pgfusepath{fill}%
\end{pgfscope}%
\begin{pgfscope}%
\pgfpathrectangle{\pgfqpoint{1.020000in}{0.880000in}}{\pgfqpoint{6.160000in}{6.160000in}}%
\pgfusepath{clip}%
\pgfsetbuttcap%
\pgfsetroundjoin%
\definecolor{currentfill}{rgb}{0.266381,0.353304,0.801637}%
\pgfsetfillcolor{currentfill}%
\pgfsetlinewidth{0.000000pt}%
\definecolor{currentstroke}{rgb}{0.000000,0.000000,0.000000}%
\pgfsetstrokecolor{currentstroke}%
\pgfsetdash{}{0pt}%
\pgfpathmoveto{\pgfqpoint{6.093851in}{2.591497in}}%
\pgfpathlineto{\pgfqpoint{6.105119in}{2.567530in}}%
\pgfpathlineto{\pgfqpoint{6.114660in}{2.453994in}}%
\pgfpathlineto{\pgfqpoint{6.148466in}{2.451021in}}%
\pgfpathlineto{\pgfqpoint{6.184350in}{2.552065in}}%
\pgfpathlineto{\pgfqpoint{6.171513in}{2.501484in}}%
\pgfpathlineto{\pgfqpoint{6.160868in}{2.558821in}}%
\pgfpathlineto{\pgfqpoint{6.126808in}{2.546205in}}%
\pgfpathlineto{\pgfqpoint{6.093851in}{2.591497in}}%
\pgfpathclose%
\pgfusepath{fill}%
\end{pgfscope}%
\begin{pgfscope}%
\pgfpathrectangle{\pgfqpoint{1.020000in}{0.880000in}}{\pgfqpoint{6.160000in}{6.160000in}}%
\pgfusepath{clip}%
\pgfsetbuttcap%
\pgfsetroundjoin%
\definecolor{currentfill}{rgb}{0.338377,0.452819,0.879317}%
\pgfsetfillcolor{currentfill}%
\pgfsetlinewidth{0.000000pt}%
\definecolor{currentstroke}{rgb}{0.000000,0.000000,0.000000}%
\pgfsetstrokecolor{currentstroke}%
\pgfsetdash{}{0pt}%
\pgfpathmoveto{\pgfqpoint{5.190890in}{2.789757in}}%
\pgfpathlineto{\pgfqpoint{5.200403in}{2.670460in}}%
\pgfpathlineto{\pgfqpoint{5.211521in}{2.706199in}}%
\pgfpathlineto{\pgfqpoint{5.244938in}{2.634668in}}%
\pgfpathlineto{\pgfqpoint{5.279085in}{2.635585in}}%
\pgfpathlineto{\pgfqpoint{5.268504in}{2.656729in}}%
\pgfpathlineto{\pgfqpoint{5.257584in}{2.644775in}}%
\pgfpathlineto{\pgfqpoint{5.224065in}{2.697497in}}%
\pgfpathlineto{\pgfqpoint{5.190890in}{2.789757in}}%
\pgfpathclose%
\pgfusepath{fill}%
\end{pgfscope}%
\begin{pgfscope}%
\pgfpathrectangle{\pgfqpoint{1.020000in}{0.880000in}}{\pgfqpoint{6.160000in}{6.160000in}}%
\pgfusepath{clip}%
\pgfsetbuttcap%
\pgfsetroundjoin%
\definecolor{currentfill}{rgb}{0.275827,0.366717,0.812553}%
\pgfsetfillcolor{currentfill}%
\pgfsetlinewidth{0.000000pt}%
\definecolor{currentstroke}{rgb}{0.000000,0.000000,0.000000}%
\pgfsetstrokecolor{currentstroke}%
\pgfsetdash{}{0pt}%
\pgfpathmoveto{\pgfqpoint{6.317917in}{2.471524in}}%
\pgfpathlineto{\pgfqpoint{6.332112in}{2.574059in}}%
\pgfpathlineto{\pgfqpoint{6.365539in}{2.556435in}}%
\pgfpathlineto{\pgfqpoint{6.353573in}{2.558755in}}%
\pgfpathlineto{\pgfqpoint{6.317917in}{2.471524in}}%
\pgfpathclose%
\pgfusepath{fill}%
\end{pgfscope}%
\begin{pgfscope}%
\pgfpathrectangle{\pgfqpoint{1.020000in}{0.880000in}}{\pgfqpoint{6.160000in}{6.160000in}}%
\pgfusepath{clip}%
\pgfsetbuttcap%
\pgfsetroundjoin%
\definecolor{currentfill}{rgb}{0.294718,0.393542,0.834384}%
\pgfsetfillcolor{currentfill}%
\pgfsetlinewidth{0.000000pt}%
\definecolor{currentstroke}{rgb}{0.000000,0.000000,0.000000}%
\pgfsetstrokecolor{currentstroke}%
\pgfsetdash{}{0pt}%
\pgfpathmoveto{\pgfqpoint{5.346157in}{2.538436in}}%
\pgfpathlineto{\pgfqpoint{5.357530in}{2.578831in}}%
\pgfpathlineto{\pgfqpoint{5.368322in}{2.567824in}}%
\pgfpathlineto{\pgfqpoint{5.402138in}{2.543199in}}%
\pgfpathlineto{\pgfqpoint{5.436898in}{2.596169in}}%
\pgfpathlineto{\pgfqpoint{5.425899in}{2.595684in}}%
\pgfpathlineto{\pgfqpoint{5.415723in}{2.659590in}}%
\pgfpathlineto{\pgfqpoint{5.381144in}{2.616751in}}%
\pgfpathlineto{\pgfqpoint{5.346157in}{2.538436in}}%
\pgfpathclose%
\pgfusepath{fill}%
\end{pgfscope}%
\begin{pgfscope}%
\pgfpathrectangle{\pgfqpoint{1.020000in}{0.880000in}}{\pgfqpoint{6.160000in}{6.160000in}}%
\pgfusepath{clip}%
\pgfsetbuttcap%
\pgfsetroundjoin%
\definecolor{currentfill}{rgb}{0.777378,0.840921,0.946149}%
\pgfsetfillcolor{currentfill}%
\pgfsetlinewidth{0.000000pt}%
\definecolor{currentstroke}{rgb}{0.000000,0.000000,0.000000}%
\pgfsetstrokecolor{currentstroke}%
\pgfsetdash{}{0pt}%
\pgfpathmoveto{\pgfqpoint{3.188090in}{3.534624in}}%
\pgfpathlineto{\pgfqpoint{3.197307in}{3.464314in}}%
\pgfpathlineto{\pgfqpoint{3.206822in}{3.364273in}}%
\pgfpathlineto{\pgfqpoint{3.240157in}{3.513608in}}%
\pgfpathlineto{\pgfqpoint{3.274745in}{3.533844in}}%
\pgfpathlineto{\pgfqpoint{3.266490in}{3.494712in}}%
\pgfpathlineto{\pgfqpoint{3.258065in}{3.476660in}}%
\pgfpathlineto{\pgfqpoint{3.221241in}{3.699174in}}%
\pgfpathlineto{\pgfqpoint{3.188090in}{3.534624in}}%
\pgfpathclose%
\pgfusepath{fill}%
\end{pgfscope}%
\begin{pgfscope}%
\pgfpathrectangle{\pgfqpoint{1.020000in}{0.880000in}}{\pgfqpoint{6.160000in}{6.160000in}}%
\pgfusepath{clip}%
\pgfsetbuttcap%
\pgfsetroundjoin%
\definecolor{currentfill}{rgb}{0.368507,0.491141,0.905243}%
\pgfsetfillcolor{currentfill}%
\pgfsetlinewidth{0.000000pt}%
\definecolor{currentstroke}{rgb}{0.000000,0.000000,0.000000}%
\pgfsetstrokecolor{currentstroke}%
\pgfsetdash{}{0pt}%
\pgfpathmoveto{\pgfqpoint{4.964478in}{2.773933in}}%
\pgfpathlineto{\pgfqpoint{4.974683in}{2.739314in}}%
\pgfpathlineto{\pgfqpoint{4.985017in}{2.718650in}}%
\pgfpathlineto{\pgfqpoint{5.019290in}{2.722069in}}%
\pgfpathlineto{\pgfqpoint{5.053007in}{2.664153in}}%
\pgfpathlineto{\pgfqpoint{5.042424in}{2.661649in}}%
\pgfpathlineto{\pgfqpoint{5.033860in}{2.892454in}}%
\pgfpathlineto{\pgfqpoint{4.997834in}{2.671723in}}%
\pgfpathlineto{\pgfqpoint{4.964478in}{2.773933in}}%
\pgfpathclose%
\pgfusepath{fill}%
\end{pgfscope}%
\begin{pgfscope}%
\pgfpathrectangle{\pgfqpoint{1.020000in}{0.880000in}}{\pgfqpoint{6.160000in}{6.160000in}}%
\pgfusepath{clip}%
\pgfsetbuttcap%
\pgfsetroundjoin%
\definecolor{currentfill}{rgb}{0.309060,0.413498,0.850128}%
\pgfsetfillcolor{currentfill}%
\pgfsetlinewidth{0.000000pt}%
\definecolor{currentstroke}{rgb}{0.000000,0.000000,0.000000}%
\pgfsetstrokecolor{currentstroke}%
\pgfsetdash{}{0pt}%
\pgfpathmoveto{\pgfqpoint{5.053007in}{2.664153in}}%
\pgfpathlineto{\pgfqpoint{5.063518in}{2.655766in}}%
\pgfpathlineto{\pgfqpoint{5.072840in}{2.512608in}}%
\pgfpathlineto{\pgfqpoint{5.108527in}{2.671956in}}%
\pgfpathlineto{\pgfqpoint{5.141939in}{2.586975in}}%
\pgfpathlineto{\pgfqpoint{5.131415in}{2.602190in}}%
\pgfpathlineto{\pgfqpoint{5.119002in}{2.415374in}}%
\pgfpathlineto{\pgfqpoint{5.087862in}{2.737774in}}%
\pgfpathlineto{\pgfqpoint{5.053007in}{2.664153in}}%
\pgfpathclose%
\pgfusepath{fill}%
\end{pgfscope}%
\begin{pgfscope}%
\pgfpathrectangle{\pgfqpoint{1.020000in}{0.880000in}}{\pgfqpoint{6.160000in}{6.160000in}}%
\pgfusepath{clip}%
\pgfsetbuttcap%
\pgfsetroundjoin%
\definecolor{currentfill}{rgb}{0.338377,0.452819,0.879317}%
\pgfsetfillcolor{currentfill}%
\pgfsetlinewidth{0.000000pt}%
\definecolor{currentstroke}{rgb}{0.000000,0.000000,0.000000}%
\pgfsetstrokecolor{currentstroke}%
\pgfsetdash{}{0pt}%
\pgfpathmoveto{\pgfqpoint{5.119002in}{2.415374in}}%
\pgfpathlineto{\pgfqpoint{5.131415in}{2.602190in}}%
\pgfpathlineto{\pgfqpoint{5.141939in}{2.586975in}}%
\pgfpathlineto{\pgfqpoint{5.178234in}{2.797765in}}%
\pgfpathlineto{\pgfqpoint{5.211521in}{2.706199in}}%
\pgfpathlineto{\pgfqpoint{5.200403in}{2.670460in}}%
\pgfpathlineto{\pgfqpoint{5.190890in}{2.789757in}}%
\pgfpathlineto{\pgfqpoint{5.156789in}{2.796211in}}%
\pgfpathlineto{\pgfqpoint{5.119002in}{2.415374in}}%
\pgfpathclose%
\pgfusepath{fill}%
\end{pgfscope}%
\begin{pgfscope}%
\pgfpathrectangle{\pgfqpoint{1.020000in}{0.880000in}}{\pgfqpoint{6.160000in}{6.160000in}}%
\pgfusepath{clip}%
\pgfsetbuttcap%
\pgfsetroundjoin%
\definecolor{currentfill}{rgb}{0.280550,0.373423,0.818011}%
\pgfsetfillcolor{currentfill}%
\pgfsetlinewidth{0.000000pt}%
\definecolor{currentstroke}{rgb}{0.000000,0.000000,0.000000}%
\pgfsetstrokecolor{currentstroke}%
\pgfsetdash{}{0pt}%
\pgfpathmoveto{\pgfqpoint{5.504726in}{2.571744in}}%
\pgfpathlineto{\pgfqpoint{5.513385in}{2.391282in}}%
\pgfpathlineto{\pgfqpoint{5.528734in}{2.705655in}}%
\pgfpathlineto{\pgfqpoint{5.559406in}{2.455289in}}%
\pgfpathlineto{\pgfqpoint{5.594837in}{2.552640in}}%
\pgfpathlineto{\pgfqpoint{5.582135in}{2.444607in}}%
\pgfpathlineto{\pgfqpoint{5.573818in}{2.645865in}}%
\pgfpathlineto{\pgfqpoint{5.539241in}{2.606673in}}%
\pgfpathlineto{\pgfqpoint{5.504726in}{2.571744in}}%
\pgfpathclose%
\pgfusepath{fill}%
\end{pgfscope}%
\begin{pgfscope}%
\pgfpathrectangle{\pgfqpoint{1.020000in}{0.880000in}}{\pgfqpoint{6.160000in}{6.160000in}}%
\pgfusepath{clip}%
\pgfsetbuttcap%
\pgfsetroundjoin%
\definecolor{currentfill}{rgb}{0.446431,0.582356,0.957373}%
\pgfsetfillcolor{currentfill}%
\pgfsetlinewidth{0.000000pt}%
\definecolor{currentstroke}{rgb}{0.000000,0.000000,0.000000}%
\pgfsetstrokecolor{currentstroke}%
\pgfsetdash{}{0pt}%
\pgfpathmoveto{\pgfqpoint{4.807841in}{2.928202in}}%
\pgfpathlineto{\pgfqpoint{4.817517in}{2.828423in}}%
\pgfpathlineto{\pgfqpoint{4.827837in}{2.827633in}}%
\pgfpathlineto{\pgfqpoint{4.862518in}{2.883711in}}%
\pgfpathlineto{\pgfqpoint{4.896961in}{2.904247in}}%
\pgfpathlineto{\pgfqpoint{4.886378in}{2.878722in}}%
\pgfpathlineto{\pgfqpoint{4.876690in}{2.977642in}}%
\pgfpathlineto{\pgfqpoint{4.841029in}{2.765844in}}%
\pgfpathlineto{\pgfqpoint{4.807841in}{2.928202in}}%
\pgfpathclose%
\pgfusepath{fill}%
\end{pgfscope}%
\begin{pgfscope}%
\pgfpathrectangle{\pgfqpoint{1.020000in}{0.880000in}}{\pgfqpoint{6.160000in}{6.160000in}}%
\pgfusepath{clip}%
\pgfsetbuttcap%
\pgfsetroundjoin%
\definecolor{currentfill}{rgb}{0.285273,0.380129,0.823469}%
\pgfsetfillcolor{currentfill}%
\pgfsetlinewidth{0.000000pt}%
\definecolor{currentstroke}{rgb}{0.000000,0.000000,0.000000}%
\pgfsetstrokecolor{currentstroke}%
\pgfsetdash{}{0pt}%
\pgfpathmoveto{\pgfqpoint{6.255286in}{2.707426in}}%
\pgfpathlineto{\pgfqpoint{6.263996in}{2.553863in}}%
\pgfpathlineto{\pgfqpoint{6.297882in}{2.555574in}}%
\pgfpathlineto{\pgfqpoint{6.332112in}{2.574059in}}%
\pgfpathlineto{\pgfqpoint{6.317917in}{2.471524in}}%
\pgfpathlineto{\pgfqpoint{6.284575in}{2.491575in}}%
\pgfpathlineto{\pgfqpoint{6.255286in}{2.707426in}}%
\pgfpathclose%
\pgfusepath{fill}%
\end{pgfscope}%
\begin{pgfscope}%
\pgfpathrectangle{\pgfqpoint{1.020000in}{0.880000in}}{\pgfqpoint{6.160000in}{6.160000in}}%
\pgfusepath{clip}%
\pgfsetbuttcap%
\pgfsetroundjoin%
\definecolor{currentfill}{rgb}{0.818056,0.855590,0.914638}%
\pgfsetfillcolor{currentfill}%
\pgfsetlinewidth{0.000000pt}%
\definecolor{currentstroke}{rgb}{0.000000,0.000000,0.000000}%
\pgfsetstrokecolor{currentstroke}%
\pgfsetdash{}{0pt}%
\pgfpathmoveto{\pgfqpoint{3.482867in}{3.549586in}}%
\pgfpathlineto{\pgfqpoint{3.490745in}{3.689842in}}%
\pgfpathlineto{\pgfqpoint{3.500024in}{3.631147in}}%
\pgfpathlineto{\pgfqpoint{3.534859in}{3.615733in}}%
\pgfpathlineto{\pgfqpoint{3.569376in}{3.646927in}}%
\pgfpathlineto{\pgfqpoint{3.560533in}{3.630393in}}%
\pgfpathlineto{\pgfqpoint{3.552375in}{3.508815in}}%
\pgfpathlineto{\pgfqpoint{3.517995in}{3.475523in}}%
\pgfpathlineto{\pgfqpoint{3.482867in}{3.549586in}}%
\pgfpathclose%
\pgfusepath{fill}%
\end{pgfscope}%
\begin{pgfscope}%
\pgfpathrectangle{\pgfqpoint{1.020000in}{0.880000in}}{\pgfqpoint{6.160000in}{6.160000in}}%
\pgfusepath{clip}%
\pgfsetbuttcap%
\pgfsetroundjoin%
\definecolor{currentfill}{rgb}{0.724041,0.814910,0.975651}%
\pgfsetfillcolor{currentfill}%
\pgfsetlinewidth{0.000000pt}%
\definecolor{currentstroke}{rgb}{0.000000,0.000000,0.000000}%
\pgfsetstrokecolor{currentstroke}%
\pgfsetdash{}{0pt}%
\pgfpathmoveto{\pgfqpoint{4.317230in}{3.346006in}}%
\pgfpathlineto{\pgfqpoint{4.327002in}{3.353813in}}%
\pgfpathlineto{\pgfqpoint{4.337043in}{3.521100in}}%
\pgfpathlineto{\pgfqpoint{4.371725in}{3.611431in}}%
\pgfpathlineto{\pgfqpoint{4.405746in}{3.386266in}}%
\pgfpathlineto{\pgfqpoint{4.395647in}{3.272194in}}%
\pgfpathlineto{\pgfqpoint{4.385789in}{3.255428in}}%
\pgfpathlineto{\pgfqpoint{4.351699in}{3.397767in}}%
\pgfpathlineto{\pgfqpoint{4.317230in}{3.346006in}}%
\pgfpathclose%
\pgfusepath{fill}%
\end{pgfscope}%
\begin{pgfscope}%
\pgfpathrectangle{\pgfqpoint{1.020000in}{0.880000in}}{\pgfqpoint{6.160000in}{6.160000in}}%
\pgfusepath{clip}%
\pgfsetbuttcap%
\pgfsetroundjoin%
\definecolor{currentfill}{rgb}{0.500031,0.638508,0.981070}%
\pgfsetfillcolor{currentfill}%
\pgfsetlinewidth{0.000000pt}%
\definecolor{currentstroke}{rgb}{0.000000,0.000000,0.000000}%
\pgfsetstrokecolor{currentstroke}%
\pgfsetdash{}{0pt}%
\pgfpathmoveto{\pgfqpoint{4.650673in}{3.007972in}}%
\pgfpathlineto{\pgfqpoint{4.660964in}{3.039994in}}%
\pgfpathlineto{\pgfqpoint{4.670809in}{2.975099in}}%
\pgfpathlineto{\pgfqpoint{4.705350in}{3.009011in}}%
\pgfpathlineto{\pgfqpoint{4.738884in}{2.862287in}}%
\pgfpathlineto{\pgfqpoint{4.729561in}{3.024756in}}%
\pgfpathlineto{\pgfqpoint{4.719210in}{2.998207in}}%
\pgfpathlineto{\pgfqpoint{4.684565in}{2.925941in}}%
\pgfpathlineto{\pgfqpoint{4.650673in}{3.007972in}}%
\pgfpathclose%
\pgfusepath{fill}%
\end{pgfscope}%
\begin{pgfscope}%
\pgfpathrectangle{\pgfqpoint{1.020000in}{0.880000in}}{\pgfqpoint{6.160000in}{6.160000in}}%
\pgfusepath{clip}%
\pgfsetbuttcap%
\pgfsetroundjoin%
\definecolor{currentfill}{rgb}{0.285273,0.380129,0.823469}%
\pgfsetfillcolor{currentfill}%
\pgfsetlinewidth{0.000000pt}%
\definecolor{currentstroke}{rgb}{0.000000,0.000000,0.000000}%
\pgfsetstrokecolor{currentstroke}%
\pgfsetdash{}{0pt}%
\pgfpathmoveto{\pgfqpoint{6.027249in}{2.653380in}}%
\pgfpathlineto{\pgfqpoint{6.037436in}{2.574014in}}%
\pgfpathlineto{\pgfqpoint{6.046019in}{2.408976in}}%
\pgfpathlineto{\pgfqpoint{6.084125in}{2.628917in}}%
\pgfpathlineto{\pgfqpoint{6.114660in}{2.453994in}}%
\pgfpathlineto{\pgfqpoint{6.105119in}{2.567530in}}%
\pgfpathlineto{\pgfqpoint{6.093851in}{2.591497in}}%
\pgfpathlineto{\pgfqpoint{6.059054in}{2.542276in}}%
\pgfpathlineto{\pgfqpoint{6.027249in}{2.653380in}}%
\pgfpathclose%
\pgfusepath{fill}%
\end{pgfscope}%
\begin{pgfscope}%
\pgfpathrectangle{\pgfqpoint{1.020000in}{0.880000in}}{\pgfqpoint{6.160000in}{6.160000in}}%
\pgfusepath{clip}%
\pgfsetbuttcap%
\pgfsetroundjoin%
\definecolor{currentfill}{rgb}{0.299441,0.400248,0.839842}%
\pgfsetfillcolor{currentfill}%
\pgfsetlinewidth{0.000000pt}%
\definecolor{currentstroke}{rgb}{0.000000,0.000000,0.000000}%
\pgfsetstrokecolor{currentstroke}%
\pgfsetdash{}{0pt}%
\pgfpathmoveto{\pgfqpoint{5.801420in}{2.709541in}}%
\pgfpathlineto{\pgfqpoint{5.810043in}{2.538989in}}%
\pgfpathlineto{\pgfqpoint{5.823572in}{2.666048in}}%
\pgfpathlineto{\pgfqpoint{5.854926in}{2.506987in}}%
\pgfpathlineto{\pgfqpoint{5.889633in}{2.550252in}}%
\pgfpathlineto{\pgfqpoint{5.878240in}{2.556365in}}%
\pgfpathlineto{\pgfqpoint{5.868308in}{2.647149in}}%
\pgfpathlineto{\pgfqpoint{5.832492in}{2.534560in}}%
\pgfpathlineto{\pgfqpoint{5.801420in}{2.709541in}}%
\pgfpathclose%
\pgfusepath{fill}%
\end{pgfscope}%
\begin{pgfscope}%
\pgfpathrectangle{\pgfqpoint{1.020000in}{0.880000in}}{\pgfqpoint{6.160000in}{6.160000in}}%
\pgfusepath{clip}%
\pgfsetbuttcap%
\pgfsetroundjoin%
\definecolor{currentfill}{rgb}{0.875557,0.860242,0.851430}%
\pgfsetfillcolor{currentfill}%
\pgfsetlinewidth{0.000000pt}%
\definecolor{currentstroke}{rgb}{0.000000,0.000000,0.000000}%
\pgfsetstrokecolor{currentstroke}%
\pgfsetdash{}{0pt}%
\pgfpathmoveto{\pgfqpoint{3.620442in}{3.699839in}}%
\pgfpathlineto{\pgfqpoint{3.628533in}{3.864669in}}%
\pgfpathlineto{\pgfqpoint{3.638362in}{3.718664in}}%
\pgfpathlineto{\pgfqpoint{3.673077in}{3.711900in}}%
\pgfpathlineto{\pgfqpoint{3.707884in}{3.680396in}}%
\pgfpathlineto{\pgfqpoint{3.698676in}{3.704896in}}%
\pgfpathlineto{\pgfqpoint{3.689698in}{3.685791in}}%
\pgfpathlineto{\pgfqpoint{3.654471in}{3.809103in}}%
\pgfpathlineto{\pgfqpoint{3.620442in}{3.699839in}}%
\pgfpathclose%
\pgfusepath{fill}%
\end{pgfscope}%
\begin{pgfscope}%
\pgfpathrectangle{\pgfqpoint{1.020000in}{0.880000in}}{\pgfqpoint{6.160000in}{6.160000in}}%
\pgfusepath{clip}%
\pgfsetbuttcap%
\pgfsetroundjoin%
\definecolor{currentfill}{rgb}{0.855378,0.863778,0.876587}%
\pgfsetfillcolor{currentfill}%
\pgfsetlinewidth{0.000000pt}%
\definecolor{currentstroke}{rgb}{0.000000,0.000000,0.000000}%
\pgfsetstrokecolor{currentstroke}%
\pgfsetdash{}{0pt}%
\pgfpathmoveto{\pgfqpoint{3.552375in}{3.508815in}}%
\pgfpathlineto{\pgfqpoint{3.560533in}{3.630393in}}%
\pgfpathlineto{\pgfqpoint{3.569376in}{3.646927in}}%
\pgfpathlineto{\pgfqpoint{3.603695in}{3.712657in}}%
\pgfpathlineto{\pgfqpoint{3.638362in}{3.718664in}}%
\pgfpathlineto{\pgfqpoint{3.628533in}{3.864669in}}%
\pgfpathlineto{\pgfqpoint{3.620442in}{3.699839in}}%
\pgfpathlineto{\pgfqpoint{3.586055in}{3.659985in}}%
\pgfpathlineto{\pgfqpoint{3.552375in}{3.508815in}}%
\pgfpathclose%
\pgfusepath{fill}%
\end{pgfscope}%
\begin{pgfscope}%
\pgfpathrectangle{\pgfqpoint{1.020000in}{0.880000in}}{\pgfqpoint{6.160000in}{6.160000in}}%
\pgfusepath{clip}%
\pgfsetbuttcap%
\pgfsetroundjoin%
\definecolor{currentfill}{rgb}{0.261805,0.346484,0.795658}%
\pgfsetfillcolor{currentfill}%
\pgfsetlinewidth{0.000000pt}%
\definecolor{currentstroke}{rgb}{0.000000,0.000000,0.000000}%
\pgfsetstrokecolor{currentstroke}%
\pgfsetdash{}{0pt}%
\pgfpathmoveto{\pgfqpoint{5.954997in}{2.409009in}}%
\pgfpathlineto{\pgfqpoint{5.968290in}{2.504360in}}%
\pgfpathlineto{\pgfqpoint{5.979038in}{2.456762in}}%
\pgfpathlineto{\pgfqpoint{6.013857in}{2.503745in}}%
\pgfpathlineto{\pgfqpoint{6.046019in}{2.408976in}}%
\pgfpathlineto{\pgfqpoint{6.037436in}{2.574014in}}%
\pgfpathlineto{\pgfqpoint{6.027249in}{2.653380in}}%
\pgfpathlineto{\pgfqpoint{5.991192in}{2.536731in}}%
\pgfpathlineto{\pgfqpoint{5.954997in}{2.409009in}}%
\pgfpathclose%
\pgfusepath{fill}%
\end{pgfscope}%
\begin{pgfscope}%
\pgfpathrectangle{\pgfqpoint{1.020000in}{0.880000in}}{\pgfqpoint{6.160000in}{6.160000in}}%
\pgfusepath{clip}%
\pgfsetbuttcap%
\pgfsetroundjoin%
\definecolor{currentfill}{rgb}{0.651398,0.768121,0.995891}%
\pgfsetfillcolor{currentfill}%
\pgfsetlinewidth{0.000000pt}%
\definecolor{currentstroke}{rgb}{0.000000,0.000000,0.000000}%
\pgfsetstrokecolor{currentstroke}%
\pgfsetdash{}{0pt}%
\pgfpathmoveto{\pgfqpoint{4.405746in}{3.386266in}}%
\pgfpathlineto{\pgfqpoint{4.415278in}{3.237526in}}%
\pgfpathlineto{\pgfqpoint{4.424829in}{3.099330in}}%
\pgfpathlineto{\pgfqpoint{4.459525in}{3.208400in}}%
\pgfpathlineto{\pgfqpoint{4.493401in}{3.037049in}}%
\pgfpathlineto{\pgfqpoint{4.484285in}{3.308692in}}%
\pgfpathlineto{\pgfqpoint{4.474447in}{3.343766in}}%
\pgfpathlineto{\pgfqpoint{4.440302in}{3.438106in}}%
\pgfpathlineto{\pgfqpoint{4.405746in}{3.386266in}}%
\pgfpathclose%
\pgfusepath{fill}%
\end{pgfscope}%
\begin{pgfscope}%
\pgfpathrectangle{\pgfqpoint{1.020000in}{0.880000in}}{\pgfqpoint{6.160000in}{6.160000in}}%
\pgfusepath{clip}%
\pgfsetbuttcap%
\pgfsetroundjoin%
\definecolor{currentfill}{rgb}{0.851372,0.863125,0.881064}%
\pgfsetfillcolor{currentfill}%
\pgfsetlinewidth{0.000000pt}%
\definecolor{currentstroke}{rgb}{0.000000,0.000000,0.000000}%
\pgfsetstrokecolor{currentstroke}%
\pgfsetdash{}{0pt}%
\pgfpathmoveto{\pgfqpoint{4.072125in}{3.680039in}}%
\pgfpathlineto{\pgfqpoint{4.081651in}{3.655012in}}%
\pgfpathlineto{\pgfqpoint{4.091193in}{3.629472in}}%
\pgfpathlineto{\pgfqpoint{4.125663in}{3.745301in}}%
\pgfpathlineto{\pgfqpoint{4.160252in}{3.597304in}}%
\pgfpathlineto{\pgfqpoint{4.150616in}{3.761576in}}%
\pgfpathlineto{\pgfqpoint{4.141050in}{3.732840in}}%
\pgfpathlineto{\pgfqpoint{4.106681in}{3.585954in}}%
\pgfpathlineto{\pgfqpoint{4.072125in}{3.680039in}}%
\pgfpathclose%
\pgfusepath{fill}%
\end{pgfscope}%
\begin{pgfscope}%
\pgfpathrectangle{\pgfqpoint{1.020000in}{0.880000in}}{\pgfqpoint{6.160000in}{6.160000in}}%
\pgfusepath{clip}%
\pgfsetbuttcap%
\pgfsetroundjoin%
\definecolor{currentfill}{rgb}{0.318832,0.426605,0.859857}%
\pgfsetfillcolor{currentfill}%
\pgfsetlinewidth{0.000000pt}%
\definecolor{currentstroke}{rgb}{0.000000,0.000000,0.000000}%
\pgfsetstrokecolor{currentstroke}%
\pgfsetdash{}{0pt}%
\pgfpathmoveto{\pgfqpoint{5.279085in}{2.635585in}}%
\pgfpathlineto{\pgfqpoint{5.290754in}{2.710198in}}%
\pgfpathlineto{\pgfqpoint{5.300239in}{2.587354in}}%
\pgfpathlineto{\pgfqpoint{5.335262in}{2.660993in}}%
\pgfpathlineto{\pgfqpoint{5.368322in}{2.567824in}}%
\pgfpathlineto{\pgfqpoint{5.357530in}{2.578831in}}%
\pgfpathlineto{\pgfqpoint{5.346157in}{2.538436in}}%
\pgfpathlineto{\pgfqpoint{5.313535in}{2.664833in}}%
\pgfpathlineto{\pgfqpoint{5.279085in}{2.635585in}}%
\pgfpathclose%
\pgfusepath{fill}%
\end{pgfscope}%
\begin{pgfscope}%
\pgfpathrectangle{\pgfqpoint{1.020000in}{0.880000in}}{\pgfqpoint{6.160000in}{6.160000in}}%
\pgfusepath{clip}%
\pgfsetbuttcap%
\pgfsetroundjoin%
\definecolor{currentfill}{rgb}{0.275827,0.366717,0.812553}%
\pgfsetfillcolor{currentfill}%
\pgfsetlinewidth{0.000000pt}%
\definecolor{currentstroke}{rgb}{0.000000,0.000000,0.000000}%
\pgfsetstrokecolor{currentstroke}%
\pgfsetdash{}{0pt}%
\pgfpathmoveto{\pgfqpoint{6.184350in}{2.552065in}}%
\pgfpathlineto{\pgfqpoint{6.195180in}{2.501998in}}%
\pgfpathlineto{\pgfqpoint{6.227218in}{2.413085in}}%
\pgfpathlineto{\pgfqpoint{6.263996in}{2.553863in}}%
\pgfpathlineto{\pgfqpoint{6.255286in}{2.707426in}}%
\pgfpathlineto{\pgfqpoint{6.216404in}{2.464269in}}%
\pgfpathlineto{\pgfqpoint{6.184350in}{2.552065in}}%
\pgfpathclose%
\pgfusepath{fill}%
\end{pgfscope}%
\begin{pgfscope}%
\pgfpathrectangle{\pgfqpoint{1.020000in}{0.880000in}}{\pgfqpoint{6.160000in}{6.160000in}}%
\pgfusepath{clip}%
\pgfsetbuttcap%
\pgfsetroundjoin%
\definecolor{currentfill}{rgb}{0.656683,0.771806,0.994914}%
\pgfsetfillcolor{currentfill}%
\pgfsetlinewidth{0.000000pt}%
\definecolor{currentstroke}{rgb}{0.000000,0.000000,0.000000}%
\pgfsetstrokecolor{currentstroke}%
\pgfsetdash{}{0pt}%
\pgfpathmoveto{\pgfqpoint{3.070143in}{3.140146in}}%
\pgfpathlineto{\pgfqpoint{3.077550in}{3.228579in}}%
\pgfpathlineto{\pgfqpoint{3.084923in}{3.323011in}}%
\pgfpathlineto{\pgfqpoint{3.121712in}{3.144094in}}%
\pgfpathlineto{\pgfqpoint{3.154917in}{3.299094in}}%
\pgfpathlineto{\pgfqpoint{3.146693in}{3.273450in}}%
\pgfpathlineto{\pgfqpoint{3.137028in}{3.387811in}}%
\pgfpathlineto{\pgfqpoint{3.103049in}{3.310581in}}%
\pgfpathlineto{\pgfqpoint{3.070143in}{3.140146in}}%
\pgfpathclose%
\pgfusepath{fill}%
\end{pgfscope}%
\begin{pgfscope}%
\pgfpathrectangle{\pgfqpoint{1.020000in}{0.880000in}}{\pgfqpoint{6.160000in}{6.160000in}}%
\pgfusepath{clip}%
\pgfsetbuttcap%
\pgfsetroundjoin%
\definecolor{currentfill}{rgb}{0.693321,0.796314,0.986308}%
\pgfsetfillcolor{currentfill}%
\pgfsetlinewidth{0.000000pt}%
\definecolor{currentstroke}{rgb}{0.000000,0.000000,0.000000}%
\pgfsetstrokecolor{currentstroke}%
\pgfsetdash{}{0pt}%
\pgfpathmoveto{\pgfqpoint{3.137028in}{3.387811in}}%
\pgfpathlineto{\pgfqpoint{3.146693in}{3.273450in}}%
\pgfpathlineto{\pgfqpoint{3.154917in}{3.299094in}}%
\pgfpathlineto{\pgfqpoint{3.188929in}{3.381613in}}%
\pgfpathlineto{\pgfqpoint{3.225024in}{3.250873in}}%
\pgfpathlineto{\pgfqpoint{3.214548in}{3.449029in}}%
\pgfpathlineto{\pgfqpoint{3.206822in}{3.364273in}}%
\pgfpathlineto{\pgfqpoint{3.173276in}{3.243954in}}%
\pgfpathlineto{\pgfqpoint{3.137028in}{3.387811in}}%
\pgfpathclose%
\pgfusepath{fill}%
\end{pgfscope}%
\begin{pgfscope}%
\pgfpathrectangle{\pgfqpoint{1.020000in}{0.880000in}}{\pgfqpoint{6.160000in}{6.160000in}}%
\pgfusepath{clip}%
\pgfsetbuttcap%
\pgfsetroundjoin%
\definecolor{currentfill}{rgb}{0.576051,0.708780,0.997755}%
\pgfsetfillcolor{currentfill}%
\pgfsetlinewidth{0.000000pt}%
\definecolor{currentstroke}{rgb}{0.000000,0.000000,0.000000}%
\pgfsetstrokecolor{currentstroke}%
\pgfsetdash{}{0pt}%
\pgfpathmoveto{\pgfqpoint{4.493401in}{3.037049in}}%
\pgfpathlineto{\pgfqpoint{4.504125in}{3.275700in}}%
\pgfpathlineto{\pgfqpoint{4.514258in}{3.314925in}}%
\pgfpathlineto{\pgfqpoint{4.548096in}{3.135213in}}%
\pgfpathlineto{\pgfqpoint{4.581620in}{2.913211in}}%
\pgfpathlineto{\pgfqpoint{4.572294in}{3.095199in}}%
\pgfpathlineto{\pgfqpoint{4.562176in}{3.072546in}}%
\pgfpathlineto{\pgfqpoint{4.527910in}{3.088502in}}%
\pgfpathlineto{\pgfqpoint{4.493401in}{3.037049in}}%
\pgfpathclose%
\pgfusepath{fill}%
\end{pgfscope}%
\begin{pgfscope}%
\pgfpathrectangle{\pgfqpoint{1.020000in}{0.880000in}}{\pgfqpoint{6.160000in}{6.160000in}}%
\pgfusepath{clip}%
\pgfsetbuttcap%
\pgfsetroundjoin%
\definecolor{currentfill}{rgb}{0.388852,0.516298,0.921373}%
\pgfsetfillcolor{currentfill}%
\pgfsetlinewidth{0.000000pt}%
\definecolor{currentstroke}{rgb}{0.000000,0.000000,0.000000}%
\pgfsetstrokecolor{currentstroke}%
\pgfsetdash{}{0pt}%
\pgfpathmoveto{\pgfqpoint{4.896961in}{2.904247in}}%
\pgfpathlineto{\pgfqpoint{4.906678in}{2.806898in}}%
\pgfpathlineto{\pgfqpoint{4.916555in}{2.730979in}}%
\pgfpathlineto{\pgfqpoint{4.950029in}{2.625262in}}%
\pgfpathlineto{\pgfqpoint{4.985017in}{2.718650in}}%
\pgfpathlineto{\pgfqpoint{4.974683in}{2.739314in}}%
\pgfpathlineto{\pgfqpoint{4.964478in}{2.773933in}}%
\pgfpathlineto{\pgfqpoint{4.930586in}{2.816978in}}%
\pgfpathlineto{\pgfqpoint{4.896961in}{2.904247in}}%
\pgfpathclose%
\pgfusepath{fill}%
\end{pgfscope}%
\begin{pgfscope}%
\pgfpathrectangle{\pgfqpoint{1.020000in}{0.880000in}}{\pgfqpoint{6.160000in}{6.160000in}}%
\pgfusepath{clip}%
\pgfsetbuttcap%
\pgfsetroundjoin%
\definecolor{currentfill}{rgb}{0.581486,0.713451,0.998314}%
\pgfsetfillcolor{currentfill}%
\pgfsetlinewidth{0.000000pt}%
\definecolor{currentstroke}{rgb}{0.000000,0.000000,0.000000}%
\pgfsetstrokecolor{currentstroke}%
\pgfsetdash{}{0pt}%
\pgfpathmoveto{\pgfqpoint{2.635460in}{3.151216in}}%
\pgfpathlineto{\pgfqpoint{2.645708in}{3.000459in}}%
\pgfpathlineto{\pgfqpoint{2.650848in}{3.181872in}}%
\pgfpathlineto{\pgfqpoint{2.685693in}{3.193532in}}%
\pgfpathlineto{\pgfqpoint{2.721414in}{3.144549in}}%
\pgfpathlineto{\pgfqpoint{2.713954in}{3.107252in}}%
\pgfpathlineto{\pgfqpoint{2.704244in}{3.223347in}}%
\pgfpathlineto{\pgfqpoint{2.672733in}{2.997248in}}%
\pgfpathlineto{\pgfqpoint{2.635460in}{3.151216in}}%
\pgfpathclose%
\pgfusepath{fill}%
\end{pgfscope}%
\begin{pgfscope}%
\pgfpathrectangle{\pgfqpoint{1.020000in}{0.880000in}}{\pgfqpoint{6.160000in}{6.160000in}}%
\pgfusepath{clip}%
\pgfsetbuttcap%
\pgfsetroundjoin%
\definecolor{currentfill}{rgb}{0.285273,0.380129,0.823469}%
\pgfsetfillcolor{currentfill}%
\pgfsetlinewidth{0.000000pt}%
\definecolor{currentstroke}{rgb}{0.000000,0.000000,0.000000}%
\pgfsetstrokecolor{currentstroke}%
\pgfsetdash{}{0pt}%
\pgfpathmoveto{\pgfqpoint{5.436898in}{2.596169in}}%
\pgfpathlineto{\pgfqpoint{5.447140in}{2.535317in}}%
\pgfpathlineto{\pgfqpoint{5.458458in}{2.557235in}}%
\pgfpathlineto{\pgfqpoint{5.491716in}{2.490251in}}%
\pgfpathlineto{\pgfqpoint{5.528734in}{2.705655in}}%
\pgfpathlineto{\pgfqpoint{5.513385in}{2.391282in}}%
\pgfpathlineto{\pgfqpoint{5.504726in}{2.571744in}}%
\pgfpathlineto{\pgfqpoint{5.470561in}{2.563457in}}%
\pgfpathlineto{\pgfqpoint{5.436898in}{2.596169in}}%
\pgfpathclose%
\pgfusepath{fill}%
\end{pgfscope}%
\begin{pgfscope}%
\pgfpathrectangle{\pgfqpoint{1.020000in}{0.880000in}}{\pgfqpoint{6.160000in}{6.160000in}}%
\pgfusepath{clip}%
\pgfsetbuttcap%
\pgfsetroundjoin%
\definecolor{currentfill}{rgb}{0.451739,0.588181,0.960201}%
\pgfsetfillcolor{currentfill}%
\pgfsetlinewidth{0.000000pt}%
\definecolor{currentstroke}{rgb}{0.000000,0.000000,0.000000}%
\pgfsetstrokecolor{currentstroke}%
\pgfsetdash{}{0pt}%
\pgfpathmoveto{\pgfqpoint{4.738884in}{2.862287in}}%
\pgfpathlineto{\pgfqpoint{4.749935in}{3.005464in}}%
\pgfpathlineto{\pgfqpoint{4.758896in}{2.778715in}}%
\pgfpathlineto{\pgfqpoint{4.794121in}{2.925666in}}%
\pgfpathlineto{\pgfqpoint{4.827837in}{2.827633in}}%
\pgfpathlineto{\pgfqpoint{4.817517in}{2.828423in}}%
\pgfpathlineto{\pgfqpoint{4.807841in}{2.928202in}}%
\pgfpathlineto{\pgfqpoint{4.773272in}{2.880081in}}%
\pgfpathlineto{\pgfqpoint{4.738884in}{2.862287in}}%
\pgfpathclose%
\pgfusepath{fill}%
\end{pgfscope}%
\begin{pgfscope}%
\pgfpathrectangle{\pgfqpoint{1.020000in}{0.880000in}}{\pgfqpoint{6.160000in}{6.160000in}}%
\pgfusepath{clip}%
\pgfsetbuttcap%
\pgfsetroundjoin%
\definecolor{currentfill}{rgb}{0.843358,0.861820,0.890017}%
\pgfsetfillcolor{currentfill}%
\pgfsetlinewidth{0.000000pt}%
\definecolor{currentstroke}{rgb}{0.000000,0.000000,0.000000}%
\pgfsetstrokecolor{currentstroke}%
\pgfsetdash{}{0pt}%
\pgfpathmoveto{\pgfqpoint{4.003023in}{3.719574in}}%
\pgfpathlineto{\pgfqpoint{4.012701in}{3.576669in}}%
\pgfpathlineto{\pgfqpoint{4.021997in}{3.666830in}}%
\pgfpathlineto{\pgfqpoint{4.056652in}{3.616303in}}%
\pgfpathlineto{\pgfqpoint{4.091193in}{3.629472in}}%
\pgfpathlineto{\pgfqpoint{4.081651in}{3.655012in}}%
\pgfpathlineto{\pgfqpoint{4.072125in}{3.680039in}}%
\pgfpathlineto{\pgfqpoint{4.037699in}{3.625751in}}%
\pgfpathlineto{\pgfqpoint{4.003023in}{3.719574in}}%
\pgfpathclose%
\pgfusepath{fill}%
\end{pgfscope}%
\begin{pgfscope}%
\pgfpathrectangle{\pgfqpoint{1.020000in}{0.880000in}}{\pgfqpoint{6.160000in}{6.160000in}}%
\pgfusepath{clip}%
\pgfsetbuttcap%
\pgfsetroundjoin%
\definecolor{currentfill}{rgb}{0.733898,0.820018,0.970724}%
\pgfsetfillcolor{currentfill}%
\pgfsetlinewidth{0.000000pt}%
\definecolor{currentstroke}{rgb}{0.000000,0.000000,0.000000}%
\pgfsetstrokecolor{currentstroke}%
\pgfsetdash{}{0pt}%
\pgfpathmoveto{\pgfqpoint{3.206822in}{3.364273in}}%
\pgfpathlineto{\pgfqpoint{3.214548in}{3.449029in}}%
\pgfpathlineto{\pgfqpoint{3.225024in}{3.250873in}}%
\pgfpathlineto{\pgfqpoint{3.260152in}{3.212546in}}%
\pgfpathlineto{\pgfqpoint{3.291945in}{3.543493in}}%
\pgfpathlineto{\pgfqpoint{3.284320in}{3.429072in}}%
\pgfpathlineto{\pgfqpoint{3.274745in}{3.533844in}}%
\pgfpathlineto{\pgfqpoint{3.240157in}{3.513608in}}%
\pgfpathlineto{\pgfqpoint{3.206822in}{3.364273in}}%
\pgfpathclose%
\pgfusepath{fill}%
\end{pgfscope}%
\begin{pgfscope}%
\pgfpathrectangle{\pgfqpoint{1.020000in}{0.880000in}}{\pgfqpoint{6.160000in}{6.160000in}}%
\pgfusepath{clip}%
\pgfsetbuttcap%
\pgfsetroundjoin%
\definecolor{currentfill}{rgb}{0.796064,0.848693,0.933471}%
\pgfsetfillcolor{currentfill}%
\pgfsetlinewidth{0.000000pt}%
\definecolor{currentstroke}{rgb}{0.000000,0.000000,0.000000}%
\pgfsetstrokecolor{currentstroke}%
\pgfsetdash{}{0pt}%
\pgfpathmoveto{\pgfqpoint{3.344573in}{3.494250in}}%
\pgfpathlineto{\pgfqpoint{3.352235in}{3.619429in}}%
\pgfpathlineto{\pgfqpoint{3.362504in}{3.431952in}}%
\pgfpathlineto{\pgfqpoint{3.396177in}{3.569119in}}%
\pgfpathlineto{\pgfqpoint{3.430951in}{3.569688in}}%
\pgfpathlineto{\pgfqpoint{3.420902in}{3.734300in}}%
\pgfpathlineto{\pgfqpoint{3.414671in}{3.401185in}}%
\pgfpathlineto{\pgfqpoint{3.379530in}{3.462512in}}%
\pgfpathlineto{\pgfqpoint{3.344573in}{3.494250in}}%
\pgfpathclose%
\pgfusepath{fill}%
\end{pgfscope}%
\begin{pgfscope}%
\pgfpathrectangle{\pgfqpoint{1.020000in}{0.880000in}}{\pgfqpoint{6.160000in}{6.160000in}}%
\pgfusepath{clip}%
\pgfsetbuttcap%
\pgfsetroundjoin%
\definecolor{currentfill}{rgb}{0.879622,0.858175,0.845844}%
\pgfsetfillcolor{currentfill}%
\pgfsetlinewidth{0.000000pt}%
\definecolor{currentstroke}{rgb}{0.000000,0.000000,0.000000}%
\pgfsetstrokecolor{currentstroke}%
\pgfsetdash{}{0pt}%
\pgfpathmoveto{\pgfqpoint{3.846130in}{3.718347in}}%
\pgfpathlineto{\pgfqpoint{3.854807in}{3.894715in}}%
\pgfpathlineto{\pgfqpoint{3.864514in}{3.763182in}}%
\pgfpathlineto{\pgfqpoint{3.899333in}{3.699907in}}%
\pgfpathlineto{\pgfqpoint{3.933801in}{3.751441in}}%
\pgfpathlineto{\pgfqpoint{3.924704in}{3.654832in}}%
\pgfpathlineto{\pgfqpoint{3.915590in}{3.578735in}}%
\pgfpathlineto{\pgfqpoint{3.880145in}{3.897008in}}%
\pgfpathlineto{\pgfqpoint{3.846130in}{3.718347in}}%
\pgfpathclose%
\pgfusepath{fill}%
\end{pgfscope}%
\begin{pgfscope}%
\pgfpathrectangle{\pgfqpoint{1.020000in}{0.880000in}}{\pgfqpoint{6.160000in}{6.160000in}}%
\pgfusepath{clip}%
\pgfsetbuttcap%
\pgfsetroundjoin%
\definecolor{currentfill}{rgb}{0.822420,0.856898,0.910795}%
\pgfsetfillcolor{currentfill}%
\pgfsetlinewidth{0.000000pt}%
\definecolor{currentstroke}{rgb}{0.000000,0.000000,0.000000}%
\pgfsetstrokecolor{currentstroke}%
\pgfsetdash{}{0pt}%
\pgfpathmoveto{\pgfqpoint{3.414671in}{3.401185in}}%
\pgfpathlineto{\pgfqpoint{3.420902in}{3.734300in}}%
\pgfpathlineto{\pgfqpoint{3.430951in}{3.569688in}}%
\pgfpathlineto{\pgfqpoint{3.465114in}{3.652431in}}%
\pgfpathlineto{\pgfqpoint{3.500024in}{3.631147in}}%
\pgfpathlineto{\pgfqpoint{3.490745in}{3.689842in}}%
\pgfpathlineto{\pgfqpoint{3.482867in}{3.549586in}}%
\pgfpathlineto{\pgfqpoint{3.448190in}{3.552118in}}%
\pgfpathlineto{\pgfqpoint{3.414671in}{3.401185in}}%
\pgfpathclose%
\pgfusepath{fill}%
\end{pgfscope}%
\begin{pgfscope}%
\pgfpathrectangle{\pgfqpoint{1.020000in}{0.880000in}}{\pgfqpoint{6.160000in}{6.160000in}}%
\pgfusepath{clip}%
\pgfsetbuttcap%
\pgfsetroundjoin%
\definecolor{currentfill}{rgb}{0.804965,0.851666,0.926165}%
\pgfsetfillcolor{currentfill}%
\pgfsetlinewidth{0.000000pt}%
\definecolor{currentstroke}{rgb}{0.000000,0.000000,0.000000}%
\pgfsetstrokecolor{currentstroke}%
\pgfsetdash{}{0pt}%
\pgfpathmoveto{\pgfqpoint{4.160252in}{3.597304in}}%
\pgfpathlineto{\pgfqpoint{4.169875in}{3.456927in}}%
\pgfpathlineto{\pgfqpoint{4.179485in}{3.511098in}}%
\pgfpathlineto{\pgfqpoint{4.214035in}{3.598391in}}%
\pgfpathlineto{\pgfqpoint{4.248522in}{3.529277in}}%
\pgfpathlineto{\pgfqpoint{4.238793in}{3.455914in}}%
\pgfpathlineto{\pgfqpoint{4.229169in}{3.534047in}}%
\pgfpathlineto{\pgfqpoint{4.194759in}{3.799553in}}%
\pgfpathlineto{\pgfqpoint{4.160252in}{3.597304in}}%
\pgfpathclose%
\pgfusepath{fill}%
\end{pgfscope}%
\begin{pgfscope}%
\pgfpathrectangle{\pgfqpoint{1.020000in}{0.880000in}}{\pgfqpoint{6.160000in}{6.160000in}}%
\pgfusepath{clip}%
\pgfsetbuttcap%
\pgfsetroundjoin%
\definecolor{currentfill}{rgb}{0.261805,0.346484,0.795658}%
\pgfsetfillcolor{currentfill}%
\pgfsetlinewidth{0.000000pt}%
\definecolor{currentstroke}{rgb}{0.000000,0.000000,0.000000}%
\pgfsetstrokecolor{currentstroke}%
\pgfsetdash{}{0pt}%
\pgfpathmoveto{\pgfqpoint{5.889633in}{2.550252in}}%
\pgfpathlineto{\pgfqpoint{5.897888in}{2.361644in}}%
\pgfpathlineto{\pgfqpoint{5.911145in}{2.460771in}}%
\pgfpathlineto{\pgfqpoint{5.948405in}{2.644359in}}%
\pgfpathlineto{\pgfqpoint{5.979038in}{2.456762in}}%
\pgfpathlineto{\pgfqpoint{5.968290in}{2.504360in}}%
\pgfpathlineto{\pgfqpoint{5.954997in}{2.409009in}}%
\pgfpathlineto{\pgfqpoint{5.923584in}{2.549956in}}%
\pgfpathlineto{\pgfqpoint{5.889633in}{2.550252in}}%
\pgfpathclose%
\pgfusepath{fill}%
\end{pgfscope}%
\begin{pgfscope}%
\pgfpathrectangle{\pgfqpoint{1.020000in}{0.880000in}}{\pgfqpoint{6.160000in}{6.160000in}}%
\pgfusepath{clip}%
\pgfsetbuttcap%
\pgfsetroundjoin%
\definecolor{currentfill}{rgb}{0.510824,0.649397,0.985079}%
\pgfsetfillcolor{currentfill}%
\pgfsetlinewidth{0.000000pt}%
\definecolor{currentstroke}{rgb}{0.000000,0.000000,0.000000}%
\pgfsetstrokecolor{currentstroke}%
\pgfsetdash{}{0pt}%
\pgfpathmoveto{\pgfqpoint{4.581620in}{2.913211in}}%
\pgfpathlineto{\pgfqpoint{4.591741in}{2.928608in}}%
\pgfpathlineto{\pgfqpoint{4.602478in}{3.084331in}}%
\pgfpathlineto{\pgfqpoint{4.635821in}{2.842683in}}%
\pgfpathlineto{\pgfqpoint{4.670809in}{2.975099in}}%
\pgfpathlineto{\pgfqpoint{4.660964in}{3.039994in}}%
\pgfpathlineto{\pgfqpoint{4.650673in}{3.007972in}}%
\pgfpathlineto{\pgfqpoint{4.616872in}{3.128829in}}%
\pgfpathlineto{\pgfqpoint{4.581620in}{2.913211in}}%
\pgfpathclose%
\pgfusepath{fill}%
\end{pgfscope}%
\begin{pgfscope}%
\pgfpathrectangle{\pgfqpoint{1.020000in}{0.880000in}}{\pgfqpoint{6.160000in}{6.160000in}}%
\pgfusepath{clip}%
\pgfsetbuttcap%
\pgfsetroundjoin%
\definecolor{currentfill}{rgb}{0.271104,0.360011,0.807095}%
\pgfsetfillcolor{currentfill}%
\pgfsetlinewidth{0.000000pt}%
\definecolor{currentstroke}{rgb}{0.000000,0.000000,0.000000}%
\pgfsetstrokecolor{currentstroke}%
\pgfsetdash{}{0pt}%
\pgfpathmoveto{\pgfqpoint{6.114660in}{2.453994in}}%
\pgfpathlineto{\pgfqpoint{6.131046in}{2.689327in}}%
\pgfpathlineto{\pgfqpoint{6.160280in}{2.451623in}}%
\pgfpathlineto{\pgfqpoint{6.195180in}{2.501998in}}%
\pgfpathlineto{\pgfqpoint{6.184350in}{2.552065in}}%
\pgfpathlineto{\pgfqpoint{6.148466in}{2.451021in}}%
\pgfpathlineto{\pgfqpoint{6.114660in}{2.453994in}}%
\pgfpathclose%
\pgfusepath{fill}%
\end{pgfscope}%
\begin{pgfscope}%
\pgfpathrectangle{\pgfqpoint{1.020000in}{0.880000in}}{\pgfqpoint{6.160000in}{6.160000in}}%
\pgfusepath{clip}%
\pgfsetbuttcap%
\pgfsetroundjoin%
\definecolor{currentfill}{rgb}{0.619318,0.744121,0.998931}%
\pgfsetfillcolor{currentfill}%
\pgfsetlinewidth{0.000000pt}%
\definecolor{currentstroke}{rgb}{0.000000,0.000000,0.000000}%
\pgfsetstrokecolor{currentstroke}%
\pgfsetdash{}{0pt}%
\pgfpathmoveto{\pgfqpoint{2.931504in}{3.094829in}}%
\pgfpathlineto{\pgfqpoint{2.939044in}{3.150817in}}%
\pgfpathlineto{\pgfqpoint{2.947567in}{3.129778in}}%
\pgfpathlineto{\pgfqpoint{2.984470in}{2.963713in}}%
\pgfpathlineto{\pgfqpoint{3.015223in}{3.311677in}}%
\pgfpathlineto{\pgfqpoint{3.007944in}{3.220955in}}%
\pgfpathlineto{\pgfqpoint{2.997907in}{3.365638in}}%
\pgfpathlineto{\pgfqpoint{2.964331in}{3.257851in}}%
\pgfpathlineto{\pgfqpoint{2.931504in}{3.094829in}}%
\pgfpathclose%
\pgfusepath{fill}%
\end{pgfscope}%
\begin{pgfscope}%
\pgfpathrectangle{\pgfqpoint{1.020000in}{0.880000in}}{\pgfqpoint{6.160000in}{6.160000in}}%
\pgfusepath{clip}%
\pgfsetbuttcap%
\pgfsetroundjoin%
\definecolor{currentfill}{rgb}{0.867428,0.864377,0.862602}%
\pgfsetfillcolor{currentfill}%
\pgfsetlinewidth{0.000000pt}%
\definecolor{currentstroke}{rgb}{0.000000,0.000000,0.000000}%
\pgfsetstrokecolor{currentstroke}%
\pgfsetdash{}{0pt}%
\pgfpathmoveto{\pgfqpoint{3.777088in}{3.686577in}}%
\pgfpathlineto{\pgfqpoint{3.786346in}{3.666844in}}%
\pgfpathlineto{\pgfqpoint{3.795491in}{3.680585in}}%
\pgfpathlineto{\pgfqpoint{3.829947in}{3.736632in}}%
\pgfpathlineto{\pgfqpoint{3.864514in}{3.763182in}}%
\pgfpathlineto{\pgfqpoint{3.854807in}{3.894715in}}%
\pgfpathlineto{\pgfqpoint{3.846130in}{3.718347in}}%
\pgfpathlineto{\pgfqpoint{3.812419in}{3.485325in}}%
\pgfpathlineto{\pgfqpoint{3.777088in}{3.686577in}}%
\pgfpathclose%
\pgfusepath{fill}%
\end{pgfscope}%
\begin{pgfscope}%
\pgfpathrectangle{\pgfqpoint{1.020000in}{0.880000in}}{\pgfqpoint{6.160000in}{6.160000in}}%
\pgfusepath{clip}%
\pgfsetbuttcap%
\pgfsetroundjoin%
\definecolor{currentfill}{rgb}{0.586921,0.718121,0.998874}%
\pgfsetfillcolor{currentfill}%
\pgfsetlinewidth{0.000000pt}%
\definecolor{currentstroke}{rgb}{0.000000,0.000000,0.000000}%
\pgfsetstrokecolor{currentstroke}%
\pgfsetdash{}{0pt}%
\pgfpathmoveto{\pgfqpoint{2.859673in}{3.255405in}}%
\pgfpathlineto{\pgfqpoint{2.869053in}{3.164438in}}%
\pgfpathlineto{\pgfqpoint{2.879495in}{2.992523in}}%
\pgfpathlineto{\pgfqpoint{2.915518in}{2.904215in}}%
\pgfpathlineto{\pgfqpoint{2.947567in}{3.129778in}}%
\pgfpathlineto{\pgfqpoint{2.939044in}{3.150817in}}%
\pgfpathlineto{\pgfqpoint{2.931504in}{3.094829in}}%
\pgfpathlineto{\pgfqpoint{2.893362in}{3.350561in}}%
\pgfpathlineto{\pgfqpoint{2.859673in}{3.255405in}}%
\pgfpathclose%
\pgfusepath{fill}%
\end{pgfscope}%
\begin{pgfscope}%
\pgfpathrectangle{\pgfqpoint{1.020000in}{0.880000in}}{\pgfqpoint{6.160000in}{6.160000in}}%
\pgfusepath{clip}%
\pgfsetbuttcap%
\pgfsetroundjoin%
\definecolor{currentfill}{rgb}{0.313946,0.420052,0.854993}%
\pgfsetfillcolor{currentfill}%
\pgfsetlinewidth{0.000000pt}%
\definecolor{currentstroke}{rgb}{0.000000,0.000000,0.000000}%
\pgfsetstrokecolor{currentstroke}%
\pgfsetdash{}{0pt}%
\pgfpathmoveto{\pgfqpoint{5.730539in}{2.527126in}}%
\pgfpathlineto{\pgfqpoint{5.743141in}{2.607731in}}%
\pgfpathlineto{\pgfqpoint{5.754040in}{2.578125in}}%
\pgfpathlineto{\pgfqpoint{5.785240in}{2.402044in}}%
\pgfpathlineto{\pgfqpoint{5.823572in}{2.666048in}}%
\pgfpathlineto{\pgfqpoint{5.810043in}{2.538989in}}%
\pgfpathlineto{\pgfqpoint{5.801420in}{2.709541in}}%
\pgfpathlineto{\pgfqpoint{5.768948in}{2.805662in}}%
\pgfpathlineto{\pgfqpoint{5.730539in}{2.527126in}}%
\pgfpathclose%
\pgfusepath{fill}%
\end{pgfscope}%
\begin{pgfscope}%
\pgfpathrectangle{\pgfqpoint{1.020000in}{0.880000in}}{\pgfqpoint{6.160000in}{6.160000in}}%
\pgfusepath{clip}%
\pgfsetbuttcap%
\pgfsetroundjoin%
\definecolor{currentfill}{rgb}{0.661968,0.775491,0.993937}%
\pgfsetfillcolor{currentfill}%
\pgfsetlinewidth{0.000000pt}%
\definecolor{currentstroke}{rgb}{0.000000,0.000000,0.000000}%
\pgfsetstrokecolor{currentstroke}%
\pgfsetdash{}{0pt}%
\pgfpathmoveto{\pgfqpoint{2.997907in}{3.365638in}}%
\pgfpathlineto{\pgfqpoint{3.007944in}{3.220955in}}%
\pgfpathlineto{\pgfqpoint{3.015223in}{3.311677in}}%
\pgfpathlineto{\pgfqpoint{3.050727in}{3.260527in}}%
\pgfpathlineto{\pgfqpoint{3.084923in}{3.323011in}}%
\pgfpathlineto{\pgfqpoint{3.077550in}{3.228579in}}%
\pgfpathlineto{\pgfqpoint{3.070143in}{3.140146in}}%
\pgfpathlineto{\pgfqpoint{3.034150in}{3.246373in}}%
\pgfpathlineto{\pgfqpoint{2.997907in}{3.365638in}}%
\pgfpathclose%
\pgfusepath{fill}%
\end{pgfscope}%
\begin{pgfscope}%
\pgfpathrectangle{\pgfqpoint{1.020000in}{0.880000in}}{\pgfqpoint{6.160000in}{6.160000in}}%
\pgfusepath{clip}%
\pgfsetbuttcap%
\pgfsetroundjoin%
\definecolor{currentfill}{rgb}{0.323718,0.433158,0.864722}%
\pgfsetfillcolor{currentfill}%
\pgfsetlinewidth{0.000000pt}%
\definecolor{currentstroke}{rgb}{0.000000,0.000000,0.000000}%
\pgfsetstrokecolor{currentstroke}%
\pgfsetdash{}{0pt}%
\pgfpathmoveto{\pgfqpoint{5.211521in}{2.706199in}}%
\pgfpathlineto{\pgfqpoint{5.221038in}{2.585961in}}%
\pgfpathlineto{\pgfqpoint{5.231086in}{2.516232in}}%
\pgfpathlineto{\pgfqpoint{5.266430in}{2.622258in}}%
\pgfpathlineto{\pgfqpoint{5.300239in}{2.587354in}}%
\pgfpathlineto{\pgfqpoint{5.290754in}{2.710198in}}%
\pgfpathlineto{\pgfqpoint{5.279085in}{2.635585in}}%
\pgfpathlineto{\pgfqpoint{5.244938in}{2.634668in}}%
\pgfpathlineto{\pgfqpoint{5.211521in}{2.706199in}}%
\pgfpathclose%
\pgfusepath{fill}%
\end{pgfscope}%
\begin{pgfscope}%
\pgfpathrectangle{\pgfqpoint{1.020000in}{0.880000in}}{\pgfqpoint{6.160000in}{6.160000in}}%
\pgfusepath{clip}%
\pgfsetbuttcap%
\pgfsetroundjoin%
\definecolor{currentfill}{rgb}{0.343278,0.459354,0.884122}%
\pgfsetfillcolor{currentfill}%
\pgfsetlinewidth{0.000000pt}%
\definecolor{currentstroke}{rgb}{0.000000,0.000000,0.000000}%
\pgfsetstrokecolor{currentstroke}%
\pgfsetdash{}{0pt}%
\pgfpathmoveto{\pgfqpoint{4.985017in}{2.718650in}}%
\pgfpathlineto{\pgfqpoint{4.995739in}{2.743175in}}%
\pgfpathlineto{\pgfqpoint{5.005542in}{2.653309in}}%
\pgfpathlineto{\pgfqpoint{5.039530in}{2.618222in}}%
\pgfpathlineto{\pgfqpoint{5.072840in}{2.512608in}}%
\pgfpathlineto{\pgfqpoint{5.063518in}{2.655766in}}%
\pgfpathlineto{\pgfqpoint{5.053007in}{2.664153in}}%
\pgfpathlineto{\pgfqpoint{5.019290in}{2.722069in}}%
\pgfpathlineto{\pgfqpoint{4.985017in}{2.718650in}}%
\pgfpathclose%
\pgfusepath{fill}%
\end{pgfscope}%
\begin{pgfscope}%
\pgfpathrectangle{\pgfqpoint{1.020000in}{0.880000in}}{\pgfqpoint{6.160000in}{6.160000in}}%
\pgfusepath{clip}%
\pgfsetbuttcap%
\pgfsetroundjoin%
\definecolor{currentfill}{rgb}{0.786721,0.844807,0.939810}%
\pgfsetfillcolor{currentfill}%
\pgfsetlinewidth{0.000000pt}%
\definecolor{currentstroke}{rgb}{0.000000,0.000000,0.000000}%
\pgfsetstrokecolor{currentstroke}%
\pgfsetdash{}{0pt}%
\pgfpathmoveto{\pgfqpoint{3.274745in}{3.533844in}}%
\pgfpathlineto{\pgfqpoint{3.284320in}{3.429072in}}%
\pgfpathlineto{\pgfqpoint{3.291945in}{3.543493in}}%
\pgfpathlineto{\pgfqpoint{3.326175in}{3.613901in}}%
\pgfpathlineto{\pgfqpoint{3.362504in}{3.431952in}}%
\pgfpathlineto{\pgfqpoint{3.352235in}{3.619429in}}%
\pgfpathlineto{\pgfqpoint{3.344573in}{3.494250in}}%
\pgfpathlineto{\pgfqpoint{3.310295in}{3.443173in}}%
\pgfpathlineto{\pgfqpoint{3.274745in}{3.533844in}}%
\pgfpathclose%
\pgfusepath{fill}%
\end{pgfscope}%
\begin{pgfscope}%
\pgfpathrectangle{\pgfqpoint{1.020000in}{0.880000in}}{\pgfqpoint{6.160000in}{6.160000in}}%
\pgfusepath{clip}%
\pgfsetbuttcap%
\pgfsetroundjoin%
\definecolor{currentfill}{rgb}{0.748682,0.827679,0.963334}%
\pgfsetfillcolor{currentfill}%
\pgfsetlinewidth{0.000000pt}%
\definecolor{currentstroke}{rgb}{0.000000,0.000000,0.000000}%
\pgfsetstrokecolor{currentstroke}%
\pgfsetdash{}{0pt}%
\pgfpathmoveto{\pgfqpoint{4.248522in}{3.529277in}}%
\pgfpathlineto{\pgfqpoint{4.258215in}{3.507355in}}%
\pgfpathlineto{\pgfqpoint{4.267830in}{3.372666in}}%
\pgfpathlineto{\pgfqpoint{4.302342in}{3.380286in}}%
\pgfpathlineto{\pgfqpoint{4.337043in}{3.521100in}}%
\pgfpathlineto{\pgfqpoint{4.327002in}{3.353813in}}%
\pgfpathlineto{\pgfqpoint{4.317230in}{3.346006in}}%
\pgfpathlineto{\pgfqpoint{4.282941in}{3.460674in}}%
\pgfpathlineto{\pgfqpoint{4.248522in}{3.529277in}}%
\pgfpathclose%
\pgfusepath{fill}%
\end{pgfscope}%
\begin{pgfscope}%
\pgfpathrectangle{\pgfqpoint{1.020000in}{0.880000in}}{\pgfqpoint{6.160000in}{6.160000in}}%
\pgfusepath{clip}%
\pgfsetbuttcap%
\pgfsetroundjoin%
\definecolor{currentfill}{rgb}{0.548876,0.685104,0.994379}%
\pgfsetfillcolor{currentfill}%
\pgfsetlinewidth{0.000000pt}%
\definecolor{currentstroke}{rgb}{0.000000,0.000000,0.000000}%
\pgfsetstrokecolor{currentstroke}%
\pgfsetdash{}{0pt}%
\pgfpathmoveto{\pgfqpoint{2.721414in}{3.144549in}}%
\pgfpathlineto{\pgfqpoint{2.730705in}{3.057904in}}%
\pgfpathlineto{\pgfqpoint{2.739732in}{2.989683in}}%
\pgfpathlineto{\pgfqpoint{2.775406in}{2.940851in}}%
\pgfpathlineto{\pgfqpoint{2.807595in}{3.140454in}}%
\pgfpathlineto{\pgfqpoint{2.801117in}{3.022399in}}%
\pgfpathlineto{\pgfqpoint{2.791824in}{3.108529in}}%
\pgfpathlineto{\pgfqpoint{2.758754in}{2.978544in}}%
\pgfpathlineto{\pgfqpoint{2.721414in}{3.144549in}}%
\pgfpathclose%
\pgfusepath{fill}%
\end{pgfscope}%
\begin{pgfscope}%
\pgfpathrectangle{\pgfqpoint{1.020000in}{0.880000in}}{\pgfqpoint{6.160000in}{6.160000in}}%
\pgfusepath{clip}%
\pgfsetbuttcap%
\pgfsetroundjoin%
\definecolor{currentfill}{rgb}{0.261805,0.346484,0.795658}%
\pgfsetfillcolor{currentfill}%
\pgfsetlinewidth{0.000000pt}%
\definecolor{currentstroke}{rgb}{0.000000,0.000000,0.000000}%
\pgfsetstrokecolor{currentstroke}%
\pgfsetdash{}{0pt}%
\pgfpathmoveto{\pgfqpoint{5.823572in}{2.666048in}}%
\pgfpathlineto{\pgfqpoint{5.830569in}{2.397409in}}%
\pgfpathlineto{\pgfqpoint{5.844054in}{2.518024in}}%
\pgfpathlineto{\pgfqpoint{5.876489in}{2.423290in}}%
\pgfpathlineto{\pgfqpoint{5.911145in}{2.460771in}}%
\pgfpathlineto{\pgfqpoint{5.897888in}{2.361644in}}%
\pgfpathlineto{\pgfqpoint{5.889633in}{2.550252in}}%
\pgfpathlineto{\pgfqpoint{5.854926in}{2.506987in}}%
\pgfpathlineto{\pgfqpoint{5.823572in}{2.666048in}}%
\pgfpathclose%
\pgfusepath{fill}%
\end{pgfscope}%
\begin{pgfscope}%
\pgfpathrectangle{\pgfqpoint{1.020000in}{0.880000in}}{\pgfqpoint{6.160000in}{6.160000in}}%
\pgfusepath{clip}%
\pgfsetbuttcap%
\pgfsetroundjoin%
\definecolor{currentfill}{rgb}{0.309060,0.413498,0.850128}%
\pgfsetfillcolor{currentfill}%
\pgfsetlinewidth{0.000000pt}%
\definecolor{currentstroke}{rgb}{0.000000,0.000000,0.000000}%
\pgfsetstrokecolor{currentstroke}%
\pgfsetdash{}{0pt}%
\pgfpathmoveto{\pgfqpoint{5.662529in}{2.525796in}}%
\pgfpathlineto{\pgfqpoint{5.674678in}{2.584557in}}%
\pgfpathlineto{\pgfqpoint{5.686931in}{2.647689in}}%
\pgfpathlineto{\pgfqpoint{5.718561in}{2.487280in}}%
\pgfpathlineto{\pgfqpoint{5.754040in}{2.578125in}}%
\pgfpathlineto{\pgfqpoint{5.743141in}{2.607731in}}%
\pgfpathlineto{\pgfqpoint{5.730539in}{2.527126in}}%
\pgfpathlineto{\pgfqpoint{5.699980in}{2.751425in}}%
\pgfpathlineto{\pgfqpoint{5.662529in}{2.525796in}}%
\pgfpathclose%
\pgfusepath{fill}%
\end{pgfscope}%
\begin{pgfscope}%
\pgfpathrectangle{\pgfqpoint{1.020000in}{0.880000in}}{\pgfqpoint{6.160000in}{6.160000in}}%
\pgfusepath{clip}%
\pgfsetbuttcap%
\pgfsetroundjoin%
\definecolor{currentfill}{rgb}{0.586921,0.718121,0.998874}%
\pgfsetfillcolor{currentfill}%
\pgfsetlinewidth{0.000000pt}%
\definecolor{currentstroke}{rgb}{0.000000,0.000000,0.000000}%
\pgfsetstrokecolor{currentstroke}%
\pgfsetdash{}{0pt}%
\pgfpathmoveto{\pgfqpoint{2.791824in}{3.108529in}}%
\pgfpathlineto{\pgfqpoint{2.801117in}{3.022399in}}%
\pgfpathlineto{\pgfqpoint{2.807595in}{3.140454in}}%
\pgfpathlineto{\pgfqpoint{2.842696in}{3.132365in}}%
\pgfpathlineto{\pgfqpoint{2.879495in}{2.992523in}}%
\pgfpathlineto{\pgfqpoint{2.869053in}{3.164438in}}%
\pgfpathlineto{\pgfqpoint{2.859673in}{3.255405in}}%
\pgfpathlineto{\pgfqpoint{2.826598in}{3.118452in}}%
\pgfpathlineto{\pgfqpoint{2.791824in}{3.108529in}}%
\pgfpathclose%
\pgfusepath{fill}%
\end{pgfscope}%
\begin{pgfscope}%
\pgfpathrectangle{\pgfqpoint{1.020000in}{0.880000in}}{\pgfqpoint{6.160000in}{6.160000in}}%
\pgfusepath{clip}%
\pgfsetbuttcap%
\pgfsetroundjoin%
\definecolor{currentfill}{rgb}{0.299441,0.400248,0.839842}%
\pgfsetfillcolor{currentfill}%
\pgfsetlinewidth{0.000000pt}%
\definecolor{currentstroke}{rgb}{0.000000,0.000000,0.000000}%
\pgfsetstrokecolor{currentstroke}%
\pgfsetdash{}{0pt}%
\pgfpathmoveto{\pgfqpoint{5.368322in}{2.567824in}}%
\pgfpathlineto{\pgfqpoint{5.380050in}{2.632675in}}%
\pgfpathlineto{\pgfqpoint{5.388939in}{2.461126in}}%
\pgfpathlineto{\pgfqpoint{5.425403in}{2.645794in}}%
\pgfpathlineto{\pgfqpoint{5.458458in}{2.557235in}}%
\pgfpathlineto{\pgfqpoint{5.447140in}{2.535317in}}%
\pgfpathlineto{\pgfqpoint{5.436898in}{2.596169in}}%
\pgfpathlineto{\pgfqpoint{5.402138in}{2.543199in}}%
\pgfpathlineto{\pgfqpoint{5.368322in}{2.567824in}}%
\pgfpathclose%
\pgfusepath{fill}%
\end{pgfscope}%
\begin{pgfscope}%
\pgfpathrectangle{\pgfqpoint{1.020000in}{0.880000in}}{\pgfqpoint{6.160000in}{6.160000in}}%
\pgfusepath{clip}%
\pgfsetbuttcap%
\pgfsetroundjoin%
\definecolor{currentfill}{rgb}{0.248091,0.326013,0.777669}%
\pgfsetfillcolor{currentfill}%
\pgfsetlinewidth{0.000000pt}%
\definecolor{currentstroke}{rgb}{0.000000,0.000000,0.000000}%
\pgfsetstrokecolor{currentstroke}%
\pgfsetdash{}{0pt}%
\pgfpathmoveto{\pgfqpoint{5.979038in}{2.456762in}}%
\pgfpathlineto{\pgfqpoint{5.990162in}{2.428885in}}%
\pgfpathlineto{\pgfqpoint{6.023524in}{2.395960in}}%
\pgfpathlineto{\pgfqpoint{6.060050in}{2.532347in}}%
\pgfpathlineto{\pgfqpoint{6.046019in}{2.408976in}}%
\pgfpathlineto{\pgfqpoint{6.013857in}{2.503745in}}%
\pgfpathlineto{\pgfqpoint{5.979038in}{2.456762in}}%
\pgfpathclose%
\pgfusepath{fill}%
\end{pgfscope}%
\begin{pgfscope}%
\pgfpathrectangle{\pgfqpoint{1.020000in}{0.880000in}}{\pgfqpoint{6.160000in}{6.160000in}}%
\pgfusepath{clip}%
\pgfsetbuttcap%
\pgfsetroundjoin%
\definecolor{currentfill}{rgb}{0.867428,0.864377,0.862602}%
\pgfsetfillcolor{currentfill}%
\pgfsetlinewidth{0.000000pt}%
\definecolor{currentstroke}{rgb}{0.000000,0.000000,0.000000}%
\pgfsetstrokecolor{currentstroke}%
\pgfsetdash{}{0pt}%
\pgfpathmoveto{\pgfqpoint{3.707884in}{3.680396in}}%
\pgfpathlineto{\pgfqpoint{3.716556in}{3.774256in}}%
\pgfpathlineto{\pgfqpoint{3.725986in}{3.710340in}}%
\pgfpathlineto{\pgfqpoint{3.760531in}{3.748524in}}%
\pgfpathlineto{\pgfqpoint{3.795491in}{3.680585in}}%
\pgfpathlineto{\pgfqpoint{3.786346in}{3.666844in}}%
\pgfpathlineto{\pgfqpoint{3.777088in}{3.686577in}}%
\pgfpathlineto{\pgfqpoint{3.742511in}{3.680003in}}%
\pgfpathlineto{\pgfqpoint{3.707884in}{3.680396in}}%
\pgfpathclose%
\pgfusepath{fill}%
\end{pgfscope}%
\begin{pgfscope}%
\pgfpathrectangle{\pgfqpoint{1.020000in}{0.880000in}}{\pgfqpoint{6.160000in}{6.160000in}}%
\pgfusepath{clip}%
\pgfsetbuttcap%
\pgfsetroundjoin%
\definecolor{currentfill}{rgb}{0.289996,0.386836,0.828926}%
\pgfsetfillcolor{currentfill}%
\pgfsetlinewidth{0.000000pt}%
\definecolor{currentstroke}{rgb}{0.000000,0.000000,0.000000}%
\pgfsetstrokecolor{currentstroke}%
\pgfsetdash{}{0pt}%
\pgfpathmoveto{\pgfqpoint{5.594837in}{2.552640in}}%
\pgfpathlineto{\pgfqpoint{5.607169in}{2.631516in}}%
\pgfpathlineto{\pgfqpoint{5.617134in}{2.543831in}}%
\pgfpathlineto{\pgfqpoint{5.650758in}{2.510207in}}%
\pgfpathlineto{\pgfqpoint{5.686931in}{2.647689in}}%
\pgfpathlineto{\pgfqpoint{5.674678in}{2.584557in}}%
\pgfpathlineto{\pgfqpoint{5.662529in}{2.525796in}}%
\pgfpathlineto{\pgfqpoint{5.625870in}{2.345353in}}%
\pgfpathlineto{\pgfqpoint{5.594837in}{2.552640in}}%
\pgfpathclose%
\pgfusepath{fill}%
\end{pgfscope}%
\begin{pgfscope}%
\pgfpathrectangle{\pgfqpoint{1.020000in}{0.880000in}}{\pgfqpoint{6.160000in}{6.160000in}}%
\pgfusepath{clip}%
\pgfsetbuttcap%
\pgfsetroundjoin%
\definecolor{currentfill}{rgb}{0.859385,0.864431,0.872111}%
\pgfsetfillcolor{currentfill}%
\pgfsetlinewidth{0.000000pt}%
\definecolor{currentstroke}{rgb}{0.000000,0.000000,0.000000}%
\pgfsetstrokecolor{currentstroke}%
\pgfsetdash{}{0pt}%
\pgfpathmoveto{\pgfqpoint{3.933801in}{3.751441in}}%
\pgfpathlineto{\pgfqpoint{3.943304in}{3.694635in}}%
\pgfpathlineto{\pgfqpoint{3.952495in}{3.777824in}}%
\pgfpathlineto{\pgfqpoint{3.987502in}{3.606565in}}%
\pgfpathlineto{\pgfqpoint{4.021997in}{3.666830in}}%
\pgfpathlineto{\pgfqpoint{4.012701in}{3.576669in}}%
\pgfpathlineto{\pgfqpoint{4.003023in}{3.719574in}}%
\pgfpathlineto{\pgfqpoint{3.968577in}{3.666114in}}%
\pgfpathlineto{\pgfqpoint{3.933801in}{3.751441in}}%
\pgfpathclose%
\pgfusepath{fill}%
\end{pgfscope}%
\begin{pgfscope}%
\pgfpathrectangle{\pgfqpoint{1.020000in}{0.880000in}}{\pgfqpoint{6.160000in}{6.160000in}}%
\pgfusepath{clip}%
\pgfsetbuttcap%
\pgfsetroundjoin%
\definecolor{currentfill}{rgb}{0.478462,0.616564,0.972721}%
\pgfsetfillcolor{currentfill}%
\pgfsetlinewidth{0.000000pt}%
\definecolor{currentstroke}{rgb}{0.000000,0.000000,0.000000}%
\pgfsetstrokecolor{currentstroke}%
\pgfsetdash{}{0pt}%
\pgfpathmoveto{\pgfqpoint{4.670809in}{2.975099in}}%
\pgfpathlineto{\pgfqpoint{4.680682in}{2.914026in}}%
\pgfpathlineto{\pgfqpoint{4.690774in}{2.893942in}}%
\pgfpathlineto{\pgfqpoint{4.725143in}{2.887335in}}%
\pgfpathlineto{\pgfqpoint{4.758896in}{2.778715in}}%
\pgfpathlineto{\pgfqpoint{4.749935in}{3.005464in}}%
\pgfpathlineto{\pgfqpoint{4.738884in}{2.862287in}}%
\pgfpathlineto{\pgfqpoint{4.705350in}{3.009011in}}%
\pgfpathlineto{\pgfqpoint{4.670809in}{2.975099in}}%
\pgfpathclose%
\pgfusepath{fill}%
\end{pgfscope}%
\begin{pgfscope}%
\pgfpathrectangle{\pgfqpoint{1.020000in}{0.880000in}}{\pgfqpoint{6.160000in}{6.160000in}}%
\pgfusepath{clip}%
\pgfsetbuttcap%
\pgfsetroundjoin%
\definecolor{currentfill}{rgb}{0.619318,0.744121,0.998931}%
\pgfsetfillcolor{currentfill}%
\pgfsetlinewidth{0.000000pt}%
\definecolor{currentstroke}{rgb}{0.000000,0.000000,0.000000}%
\pgfsetstrokecolor{currentstroke}%
\pgfsetdash{}{0pt}%
\pgfpathmoveto{\pgfqpoint{4.424829in}{3.099330in}}%
\pgfpathlineto{\pgfqpoint{4.434880in}{3.161796in}}%
\pgfpathlineto{\pgfqpoint{4.444901in}{3.199872in}}%
\pgfpathlineto{\pgfqpoint{4.479204in}{3.134024in}}%
\pgfpathlineto{\pgfqpoint{4.514258in}{3.314925in}}%
\pgfpathlineto{\pgfqpoint{4.504125in}{3.275700in}}%
\pgfpathlineto{\pgfqpoint{4.493401in}{3.037049in}}%
\pgfpathlineto{\pgfqpoint{4.459525in}{3.208400in}}%
\pgfpathlineto{\pgfqpoint{4.424829in}{3.099330in}}%
\pgfpathclose%
\pgfusepath{fill}%
\end{pgfscope}%
\begin{pgfscope}%
\pgfpathrectangle{\pgfqpoint{1.020000in}{0.880000in}}{\pgfqpoint{6.160000in}{6.160000in}}%
\pgfusepath{clip}%
\pgfsetbuttcap%
\pgfsetroundjoin%
\definecolor{currentfill}{rgb}{0.826784,0.858205,0.906953}%
\pgfsetfillcolor{currentfill}%
\pgfsetlinewidth{0.000000pt}%
\definecolor{currentstroke}{rgb}{0.000000,0.000000,0.000000}%
\pgfsetstrokecolor{currentstroke}%
\pgfsetdash{}{0pt}%
\pgfpathmoveto{\pgfqpoint{4.091193in}{3.629472in}}%
\pgfpathlineto{\pgfqpoint{4.100781in}{3.567982in}}%
\pgfpathlineto{\pgfqpoint{4.110298in}{3.616914in}}%
\pgfpathlineto{\pgfqpoint{4.144899in}{3.620503in}}%
\pgfpathlineto{\pgfqpoint{4.179485in}{3.511098in}}%
\pgfpathlineto{\pgfqpoint{4.169875in}{3.456927in}}%
\pgfpathlineto{\pgfqpoint{4.160252in}{3.597304in}}%
\pgfpathlineto{\pgfqpoint{4.125663in}{3.745301in}}%
\pgfpathlineto{\pgfqpoint{4.091193in}{3.629472in}}%
\pgfpathclose%
\pgfusepath{fill}%
\end{pgfscope}%
\begin{pgfscope}%
\pgfpathrectangle{\pgfqpoint{1.020000in}{0.880000in}}{\pgfqpoint{6.160000in}{6.160000in}}%
\pgfusepath{clip}%
\pgfsetbuttcap%
\pgfsetroundjoin%
\definecolor{currentfill}{rgb}{0.333490,0.446265,0.874452}%
\pgfsetfillcolor{currentfill}%
\pgfsetlinewidth{0.000000pt}%
\definecolor{currentstroke}{rgb}{0.000000,0.000000,0.000000}%
\pgfsetstrokecolor{currentstroke}%
\pgfsetdash{}{0pt}%
\pgfpathmoveto{\pgfqpoint{5.141939in}{2.586975in}}%
\pgfpathlineto{\pgfqpoint{5.152831in}{2.607324in}}%
\pgfpathlineto{\pgfqpoint{5.163078in}{2.559588in}}%
\pgfpathlineto{\pgfqpoint{5.199043in}{2.727822in}}%
\pgfpathlineto{\pgfqpoint{5.231086in}{2.516232in}}%
\pgfpathlineto{\pgfqpoint{5.221038in}{2.585961in}}%
\pgfpathlineto{\pgfqpoint{5.211521in}{2.706199in}}%
\pgfpathlineto{\pgfqpoint{5.178234in}{2.797765in}}%
\pgfpathlineto{\pgfqpoint{5.141939in}{2.586975in}}%
\pgfpathclose%
\pgfusepath{fill}%
\end{pgfscope}%
\begin{pgfscope}%
\pgfpathrectangle{\pgfqpoint{1.020000in}{0.880000in}}{\pgfqpoint{6.160000in}{6.160000in}}%
\pgfusepath{clip}%
\pgfsetbuttcap%
\pgfsetroundjoin%
\definecolor{currentfill}{rgb}{0.289996,0.386836,0.828926}%
\pgfsetfillcolor{currentfill}%
\pgfsetlinewidth{0.000000pt}%
\definecolor{currentstroke}{rgb}{0.000000,0.000000,0.000000}%
\pgfsetstrokecolor{currentstroke}%
\pgfsetdash{}{0pt}%
\pgfpathmoveto{\pgfqpoint{6.046019in}{2.408976in}}%
\pgfpathlineto{\pgfqpoint{6.060050in}{2.532347in}}%
\pgfpathlineto{\pgfqpoint{6.094777in}{2.571724in}}%
\pgfpathlineto{\pgfqpoint{6.131046in}{2.689327in}}%
\pgfpathlineto{\pgfqpoint{6.114660in}{2.453994in}}%
\pgfpathlineto{\pgfqpoint{6.084125in}{2.628917in}}%
\pgfpathlineto{\pgfqpoint{6.046019in}{2.408976in}}%
\pgfpathclose%
\pgfusepath{fill}%
\end{pgfscope}%
\begin{pgfscope}%
\pgfpathrectangle{\pgfqpoint{1.020000in}{0.880000in}}{\pgfqpoint{6.160000in}{6.160000in}}%
\pgfusepath{clip}%
\pgfsetbuttcap%
\pgfsetroundjoin%
\definecolor{currentfill}{rgb}{0.289996,0.386836,0.828926}%
\pgfsetfillcolor{currentfill}%
\pgfsetlinewidth{0.000000pt}%
\definecolor{currentstroke}{rgb}{0.000000,0.000000,0.000000}%
\pgfsetstrokecolor{currentstroke}%
\pgfsetdash{}{0pt}%
\pgfpathmoveto{\pgfqpoint{5.300239in}{2.587354in}}%
\pgfpathlineto{\pgfqpoint{5.310771in}{2.556803in}}%
\pgfpathlineto{\pgfqpoint{5.322245in}{2.606855in}}%
\pgfpathlineto{\pgfqpoint{5.352856in}{2.299721in}}%
\pgfpathlineto{\pgfqpoint{5.388939in}{2.461126in}}%
\pgfpathlineto{\pgfqpoint{5.380050in}{2.632675in}}%
\pgfpathlineto{\pgfqpoint{5.368322in}{2.567824in}}%
\pgfpathlineto{\pgfqpoint{5.335262in}{2.660993in}}%
\pgfpathlineto{\pgfqpoint{5.300239in}{2.587354in}}%
\pgfpathclose%
\pgfusepath{fill}%
\end{pgfscope}%
\begin{pgfscope}%
\pgfpathrectangle{\pgfqpoint{1.020000in}{0.880000in}}{\pgfqpoint{6.160000in}{6.160000in}}%
\pgfusepath{clip}%
\pgfsetbuttcap%
\pgfsetroundjoin%
\definecolor{currentfill}{rgb}{0.358415,0.478426,0.896795}%
\pgfsetfillcolor{currentfill}%
\pgfsetlinewidth{0.000000pt}%
\definecolor{currentstroke}{rgb}{0.000000,0.000000,0.000000}%
\pgfsetstrokecolor{currentstroke}%
\pgfsetdash{}{0pt}%
\pgfpathmoveto{\pgfqpoint{4.916555in}{2.730979in}}%
\pgfpathlineto{\pgfqpoint{4.927168in}{2.752291in}}%
\pgfpathlineto{\pgfqpoint{4.936730in}{2.631511in}}%
\pgfpathlineto{\pgfqpoint{4.970749in}{2.592883in}}%
\pgfpathlineto{\pgfqpoint{5.005542in}{2.653309in}}%
\pgfpathlineto{\pgfqpoint{4.995739in}{2.743175in}}%
\pgfpathlineto{\pgfqpoint{4.985017in}{2.718650in}}%
\pgfpathlineto{\pgfqpoint{4.950029in}{2.625262in}}%
\pgfpathlineto{\pgfqpoint{4.916555in}{2.730979in}}%
\pgfpathclose%
\pgfusepath{fill}%
\end{pgfscope}%
\begin{pgfscope}%
\pgfpathrectangle{\pgfqpoint{1.020000in}{0.880000in}}{\pgfqpoint{6.160000in}{6.160000in}}%
\pgfusepath{clip}%
\pgfsetbuttcap%
\pgfsetroundjoin%
\definecolor{currentfill}{rgb}{0.441123,0.576532,0.954545}%
\pgfsetfillcolor{currentfill}%
\pgfsetlinewidth{0.000000pt}%
\definecolor{currentstroke}{rgb}{0.000000,0.000000,0.000000}%
\pgfsetstrokecolor{currentstroke}%
\pgfsetdash{}{0pt}%
\pgfpathmoveto{\pgfqpoint{4.827837in}{2.827633in}}%
\pgfpathlineto{\pgfqpoint{4.838406in}{2.861156in}}%
\pgfpathlineto{\pgfqpoint{4.848570in}{2.829624in}}%
\pgfpathlineto{\pgfqpoint{4.883274in}{2.877433in}}%
\pgfpathlineto{\pgfqpoint{4.916555in}{2.730979in}}%
\pgfpathlineto{\pgfqpoint{4.906678in}{2.806898in}}%
\pgfpathlineto{\pgfqpoint{4.896961in}{2.904247in}}%
\pgfpathlineto{\pgfqpoint{4.862518in}{2.883711in}}%
\pgfpathlineto{\pgfqpoint{4.827837in}{2.827633in}}%
\pgfpathclose%
\pgfusepath{fill}%
\end{pgfscope}%
\begin{pgfscope}%
\pgfpathrectangle{\pgfqpoint{1.020000in}{0.880000in}}{\pgfqpoint{6.160000in}{6.160000in}}%
\pgfusepath{clip}%
\pgfsetbuttcap%
\pgfsetroundjoin%
\definecolor{currentfill}{rgb}{0.713852,0.808857,0.979386}%
\pgfsetfillcolor{currentfill}%
\pgfsetlinewidth{0.000000pt}%
\definecolor{currentstroke}{rgb}{0.000000,0.000000,0.000000}%
\pgfsetstrokecolor{currentstroke}%
\pgfsetdash{}{0pt}%
\pgfpathmoveto{\pgfqpoint{4.337043in}{3.521100in}}%
\pgfpathlineto{\pgfqpoint{4.346536in}{3.319825in}}%
\pgfpathlineto{\pgfqpoint{4.356608in}{3.463140in}}%
\pgfpathlineto{\pgfqpoint{4.390707in}{3.240679in}}%
\pgfpathlineto{\pgfqpoint{4.424829in}{3.099330in}}%
\pgfpathlineto{\pgfqpoint{4.415278in}{3.237526in}}%
\pgfpathlineto{\pgfqpoint{4.405746in}{3.386266in}}%
\pgfpathlineto{\pgfqpoint{4.371725in}{3.611431in}}%
\pgfpathlineto{\pgfqpoint{4.337043in}{3.521100in}}%
\pgfpathclose%
\pgfusepath{fill}%
\end{pgfscope}%
\begin{pgfscope}%
\pgfpathrectangle{\pgfqpoint{1.020000in}{0.880000in}}{\pgfqpoint{6.160000in}{6.160000in}}%
\pgfusepath{clip}%
\pgfsetbuttcap%
\pgfsetroundjoin%
\definecolor{currentfill}{rgb}{0.532568,0.669801,0.990393}%
\pgfsetfillcolor{currentfill}%
\pgfsetlinewidth{0.000000pt}%
\definecolor{currentstroke}{rgb}{0.000000,0.000000,0.000000}%
\pgfsetstrokecolor{currentstroke}%
\pgfsetdash{}{0pt}%
\pgfpathmoveto{\pgfqpoint{2.879495in}{2.992523in}}%
\pgfpathlineto{\pgfqpoint{2.888233in}{2.950801in}}%
\pgfpathlineto{\pgfqpoint{2.895956in}{2.988508in}}%
\pgfpathlineto{\pgfqpoint{2.930493in}{3.022772in}}%
\pgfpathlineto{\pgfqpoint{2.964973in}{3.061832in}}%
\pgfpathlineto{\pgfqpoint{2.956689in}{3.061259in}}%
\pgfpathlineto{\pgfqpoint{2.947567in}{3.129778in}}%
\pgfpathlineto{\pgfqpoint{2.915518in}{2.904215in}}%
\pgfpathlineto{\pgfqpoint{2.879495in}{2.992523in}}%
\pgfpathclose%
\pgfusepath{fill}%
\end{pgfscope}%
\begin{pgfscope}%
\pgfpathrectangle{\pgfqpoint{1.020000in}{0.880000in}}{\pgfqpoint{6.160000in}{6.160000in}}%
\pgfusepath{clip}%
\pgfsetbuttcap%
\pgfsetroundjoin%
\definecolor{currentfill}{rgb}{0.313946,0.420052,0.854993}%
\pgfsetfillcolor{currentfill}%
\pgfsetlinewidth{0.000000pt}%
\definecolor{currentstroke}{rgb}{0.000000,0.000000,0.000000}%
\pgfsetstrokecolor{currentstroke}%
\pgfsetdash{}{0pt}%
\pgfpathmoveto{\pgfqpoint{5.072840in}{2.512608in}}%
\pgfpathlineto{\pgfqpoint{5.085793in}{2.769829in}}%
\pgfpathlineto{\pgfqpoint{5.094471in}{2.553421in}}%
\pgfpathlineto{\pgfqpoint{5.128234in}{2.498858in}}%
\pgfpathlineto{\pgfqpoint{5.163078in}{2.559588in}}%
\pgfpathlineto{\pgfqpoint{5.152831in}{2.607324in}}%
\pgfpathlineto{\pgfqpoint{5.141939in}{2.586975in}}%
\pgfpathlineto{\pgfqpoint{5.108527in}{2.671956in}}%
\pgfpathlineto{\pgfqpoint{5.072840in}{2.512608in}}%
\pgfpathclose%
\pgfusepath{fill}%
\end{pgfscope}%
\begin{pgfscope}%
\pgfpathrectangle{\pgfqpoint{1.020000in}{0.880000in}}{\pgfqpoint{6.160000in}{6.160000in}}%
\pgfusepath{clip}%
\pgfsetbuttcap%
\pgfsetroundjoin%
\definecolor{currentfill}{rgb}{0.554312,0.690097,0.995516}%
\pgfsetfillcolor{currentfill}%
\pgfsetlinewidth{0.000000pt}%
\definecolor{currentstroke}{rgb}{0.000000,0.000000,0.000000}%
\pgfsetstrokecolor{currentstroke}%
\pgfsetdash{}{0pt}%
\pgfpathmoveto{\pgfqpoint{2.650848in}{3.181872in}}%
\pgfpathlineto{\pgfqpoint{2.662678in}{2.928462in}}%
\pgfpathlineto{\pgfqpoint{2.670472in}{2.939091in}}%
\pgfpathlineto{\pgfqpoint{2.704873in}{2.979955in}}%
\pgfpathlineto{\pgfqpoint{2.739732in}{2.989683in}}%
\pgfpathlineto{\pgfqpoint{2.730705in}{3.057904in}}%
\pgfpathlineto{\pgfqpoint{2.721414in}{3.144549in}}%
\pgfpathlineto{\pgfqpoint{2.685693in}{3.193532in}}%
\pgfpathlineto{\pgfqpoint{2.650848in}{3.181872in}}%
\pgfpathclose%
\pgfusepath{fill}%
\end{pgfscope}%
\begin{pgfscope}%
\pgfpathrectangle{\pgfqpoint{1.020000in}{0.880000in}}{\pgfqpoint{6.160000in}{6.160000in}}%
\pgfusepath{clip}%
\pgfsetbuttcap%
\pgfsetroundjoin%
\definecolor{currentfill}{rgb}{0.843358,0.861820,0.890017}%
\pgfsetfillcolor{currentfill}%
\pgfsetlinewidth{0.000000pt}%
\definecolor{currentstroke}{rgb}{0.000000,0.000000,0.000000}%
\pgfsetstrokecolor{currentstroke}%
\pgfsetdash{}{0pt}%
\pgfpathmoveto{\pgfqpoint{3.569376in}{3.646927in}}%
\pgfpathlineto{\pgfqpoint{3.579016in}{3.535253in}}%
\pgfpathlineto{\pgfqpoint{3.587999in}{3.533740in}}%
\pgfpathlineto{\pgfqpoint{3.622489in}{3.576678in}}%
\pgfpathlineto{\pgfqpoint{3.656380in}{3.735065in}}%
\pgfpathlineto{\pgfqpoint{3.647785in}{3.647419in}}%
\pgfpathlineto{\pgfqpoint{3.638362in}{3.718664in}}%
\pgfpathlineto{\pgfqpoint{3.603695in}{3.712657in}}%
\pgfpathlineto{\pgfqpoint{3.569376in}{3.646927in}}%
\pgfpathclose%
\pgfusepath{fill}%
\end{pgfscope}%
\begin{pgfscope}%
\pgfpathrectangle{\pgfqpoint{1.020000in}{0.880000in}}{\pgfqpoint{6.160000in}{6.160000in}}%
\pgfusepath{clip}%
\pgfsetbuttcap%
\pgfsetroundjoin%
\definecolor{currentfill}{rgb}{0.871493,0.862309,0.857016}%
\pgfsetfillcolor{currentfill}%
\pgfsetlinewidth{0.000000pt}%
\definecolor{currentstroke}{rgb}{0.000000,0.000000,0.000000}%
\pgfsetstrokecolor{currentstroke}%
\pgfsetdash{}{0pt}%
\pgfpathmoveto{\pgfqpoint{3.638362in}{3.718664in}}%
\pgfpathlineto{\pgfqpoint{3.647785in}{3.647419in}}%
\pgfpathlineto{\pgfqpoint{3.656380in}{3.735065in}}%
\pgfpathlineto{\pgfqpoint{3.691573in}{3.646273in}}%
\pgfpathlineto{\pgfqpoint{3.725986in}{3.710340in}}%
\pgfpathlineto{\pgfqpoint{3.716556in}{3.774256in}}%
\pgfpathlineto{\pgfqpoint{3.707884in}{3.680396in}}%
\pgfpathlineto{\pgfqpoint{3.673077in}{3.711900in}}%
\pgfpathlineto{\pgfqpoint{3.638362in}{3.718664in}}%
\pgfpathclose%
\pgfusepath{fill}%
\end{pgfscope}%
\begin{pgfscope}%
\pgfpathrectangle{\pgfqpoint{1.020000in}{0.880000in}}{\pgfqpoint{6.160000in}{6.160000in}}%
\pgfusepath{clip}%
\pgfsetbuttcap%
\pgfsetroundjoin%
\definecolor{currentfill}{rgb}{0.576051,0.708780,0.997755}%
\pgfsetfillcolor{currentfill}%
\pgfsetlinewidth{0.000000pt}%
\definecolor{currentstroke}{rgb}{0.000000,0.000000,0.000000}%
\pgfsetstrokecolor{currentstroke}%
\pgfsetdash{}{0pt}%
\pgfpathmoveto{\pgfqpoint{2.947567in}{3.129778in}}%
\pgfpathlineto{\pgfqpoint{2.956689in}{3.061259in}}%
\pgfpathlineto{\pgfqpoint{2.964973in}{3.061832in}}%
\pgfpathlineto{\pgfqpoint{2.999595in}{3.089483in}}%
\pgfpathlineto{\pgfqpoint{3.034051in}{3.131396in}}%
\pgfpathlineto{\pgfqpoint{3.026992in}{3.018606in}}%
\pgfpathlineto{\pgfqpoint{3.015223in}{3.311677in}}%
\pgfpathlineto{\pgfqpoint{2.984470in}{2.963713in}}%
\pgfpathlineto{\pgfqpoint{2.947567in}{3.129778in}}%
\pgfpathclose%
\pgfusepath{fill}%
\end{pgfscope}%
\begin{pgfscope}%
\pgfpathrectangle{\pgfqpoint{1.020000in}{0.880000in}}{\pgfqpoint{6.160000in}{6.160000in}}%
\pgfusepath{clip}%
\pgfsetbuttcap%
\pgfsetroundjoin%
\definecolor{currentfill}{rgb}{0.261805,0.346484,0.795658}%
\pgfsetfillcolor{currentfill}%
\pgfsetlinewidth{0.000000pt}%
\definecolor{currentstroke}{rgb}{0.000000,0.000000,0.000000}%
\pgfsetstrokecolor{currentstroke}%
\pgfsetdash{}{0pt}%
\pgfpathmoveto{\pgfqpoint{5.911145in}{2.460771in}}%
\pgfpathlineto{\pgfqpoint{5.922388in}{2.442468in}}%
\pgfpathlineto{\pgfqpoint{5.955979in}{2.418435in}}%
\pgfpathlineto{\pgfqpoint{5.990162in}{2.428885in}}%
\pgfpathlineto{\pgfqpoint{5.979038in}{2.456762in}}%
\pgfpathlineto{\pgfqpoint{5.948405in}{2.644359in}}%
\pgfpathlineto{\pgfqpoint{5.911145in}{2.460771in}}%
\pgfpathclose%
\pgfusepath{fill}%
\end{pgfscope}%
\begin{pgfscope}%
\pgfpathrectangle{\pgfqpoint{1.020000in}{0.880000in}}{\pgfqpoint{6.160000in}{6.160000in}}%
\pgfusepath{clip}%
\pgfsetbuttcap%
\pgfsetroundjoin%
\definecolor{currentfill}{rgb}{0.570616,0.704109,0.997195}%
\pgfsetfillcolor{currentfill}%
\pgfsetlinewidth{0.000000pt}%
\definecolor{currentstroke}{rgb}{0.000000,0.000000,0.000000}%
\pgfsetstrokecolor{currentstroke}%
\pgfsetdash{}{0pt}%
\pgfpathmoveto{\pgfqpoint{4.514258in}{3.314925in}}%
\pgfpathlineto{\pgfqpoint{4.523808in}{3.174835in}}%
\pgfpathlineto{\pgfqpoint{4.533322in}{3.026088in}}%
\pgfpathlineto{\pgfqpoint{4.567806in}{3.030834in}}%
\pgfpathlineto{\pgfqpoint{4.602478in}{3.084331in}}%
\pgfpathlineto{\pgfqpoint{4.591741in}{2.928608in}}%
\pgfpathlineto{\pgfqpoint{4.581620in}{2.913211in}}%
\pgfpathlineto{\pgfqpoint{4.548096in}{3.135213in}}%
\pgfpathlineto{\pgfqpoint{4.514258in}{3.314925in}}%
\pgfpathclose%
\pgfusepath{fill}%
\end{pgfscope}%
\begin{pgfscope}%
\pgfpathrectangle{\pgfqpoint{1.020000in}{0.880000in}}{\pgfqpoint{6.160000in}{6.160000in}}%
\pgfusepath{clip}%
\pgfsetbuttcap%
\pgfsetroundjoin%
\definecolor{currentfill}{rgb}{0.826784,0.858205,0.906953}%
\pgfsetfillcolor{currentfill}%
\pgfsetlinewidth{0.000000pt}%
\definecolor{currentstroke}{rgb}{0.000000,0.000000,0.000000}%
\pgfsetstrokecolor{currentstroke}%
\pgfsetdash{}{0pt}%
\pgfpathmoveto{\pgfqpoint{4.021997in}{3.666830in}}%
\pgfpathlineto{\pgfqpoint{4.031486in}{3.654658in}}%
\pgfpathlineto{\pgfqpoint{4.041130in}{3.546807in}}%
\pgfpathlineto{\pgfqpoint{4.075876in}{3.429031in}}%
\pgfpathlineto{\pgfqpoint{4.110298in}{3.616914in}}%
\pgfpathlineto{\pgfqpoint{4.100781in}{3.567982in}}%
\pgfpathlineto{\pgfqpoint{4.091193in}{3.629472in}}%
\pgfpathlineto{\pgfqpoint{4.056652in}{3.616303in}}%
\pgfpathlineto{\pgfqpoint{4.021997in}{3.666830in}}%
\pgfpathclose%
\pgfusepath{fill}%
\end{pgfscope}%
\begin{pgfscope}%
\pgfpathrectangle{\pgfqpoint{1.020000in}{0.880000in}}{\pgfqpoint{6.160000in}{6.160000in}}%
\pgfusepath{clip}%
\pgfsetbuttcap%
\pgfsetroundjoin%
\definecolor{currentfill}{rgb}{0.718985,0.811993,0.977656}%
\pgfsetfillcolor{currentfill}%
\pgfsetlinewidth{0.000000pt}%
\definecolor{currentstroke}{rgb}{0.000000,0.000000,0.000000}%
\pgfsetstrokecolor{currentstroke}%
\pgfsetdash{}{0pt}%
\pgfpathmoveto{\pgfqpoint{3.225024in}{3.250873in}}%
\pgfpathlineto{\pgfqpoint{3.232579in}{3.358114in}}%
\pgfpathlineto{\pgfqpoint{3.241782in}{3.294586in}}%
\pgfpathlineto{\pgfqpoint{3.275198in}{3.453599in}}%
\pgfpathlineto{\pgfqpoint{3.311653in}{3.274822in}}%
\pgfpathlineto{\pgfqpoint{3.301257in}{3.470806in}}%
\pgfpathlineto{\pgfqpoint{3.291945in}{3.543493in}}%
\pgfpathlineto{\pgfqpoint{3.260152in}{3.212546in}}%
\pgfpathlineto{\pgfqpoint{3.225024in}{3.250873in}}%
\pgfpathclose%
\pgfusepath{fill}%
\end{pgfscope}%
\begin{pgfscope}%
\pgfpathrectangle{\pgfqpoint{1.020000in}{0.880000in}}{\pgfqpoint{6.160000in}{6.160000in}}%
\pgfusepath{clip}%
\pgfsetbuttcap%
\pgfsetroundjoin%
\definecolor{currentfill}{rgb}{0.831148,0.859513,0.903110}%
\pgfsetfillcolor{currentfill}%
\pgfsetlinewidth{0.000000pt}%
\definecolor{currentstroke}{rgb}{0.000000,0.000000,0.000000}%
\pgfsetstrokecolor{currentstroke}%
\pgfsetdash{}{0pt}%
\pgfpathmoveto{\pgfqpoint{3.500024in}{3.631147in}}%
\pgfpathlineto{\pgfqpoint{3.508968in}{3.623507in}}%
\pgfpathlineto{\pgfqpoint{3.517488in}{3.682501in}}%
\pgfpathlineto{\pgfqpoint{3.552982in}{3.576053in}}%
\pgfpathlineto{\pgfqpoint{3.587999in}{3.533740in}}%
\pgfpathlineto{\pgfqpoint{3.579016in}{3.535253in}}%
\pgfpathlineto{\pgfqpoint{3.569376in}{3.646927in}}%
\pgfpathlineto{\pgfqpoint{3.534859in}{3.615733in}}%
\pgfpathlineto{\pgfqpoint{3.500024in}{3.631147in}}%
\pgfpathclose%
\pgfusepath{fill}%
\end{pgfscope}%
\begin{pgfscope}%
\pgfpathrectangle{\pgfqpoint{1.020000in}{0.880000in}}{\pgfqpoint{6.160000in}{6.160000in}}%
\pgfusepath{clip}%
\pgfsetbuttcap%
\pgfsetroundjoin%
\definecolor{currentfill}{rgb}{0.285273,0.380129,0.823469}%
\pgfsetfillcolor{currentfill}%
\pgfsetlinewidth{0.000000pt}%
\definecolor{currentstroke}{rgb}{0.000000,0.000000,0.000000}%
\pgfsetstrokecolor{currentstroke}%
\pgfsetdash{}{0pt}%
\pgfpathmoveto{\pgfqpoint{5.754040in}{2.578125in}}%
\pgfpathlineto{\pgfqpoint{5.766369in}{2.636945in}}%
\pgfpathlineto{\pgfqpoint{5.773670in}{2.380554in}}%
\pgfpathlineto{\pgfqpoint{5.812226in}{2.654992in}}%
\pgfpathlineto{\pgfqpoint{5.844054in}{2.518024in}}%
\pgfpathlineto{\pgfqpoint{5.830569in}{2.397409in}}%
\pgfpathlineto{\pgfqpoint{5.823572in}{2.666048in}}%
\pgfpathlineto{\pgfqpoint{5.785240in}{2.402044in}}%
\pgfpathlineto{\pgfqpoint{5.754040in}{2.578125in}}%
\pgfpathclose%
\pgfusepath{fill}%
\end{pgfscope}%
\begin{pgfscope}%
\pgfpathrectangle{\pgfqpoint{1.020000in}{0.880000in}}{\pgfqpoint{6.160000in}{6.160000in}}%
\pgfusepath{clip}%
\pgfsetbuttcap%
\pgfsetroundjoin%
\definecolor{currentfill}{rgb}{0.304174,0.406945,0.845263}%
\pgfsetfillcolor{currentfill}%
\pgfsetlinewidth{0.000000pt}%
\definecolor{currentstroke}{rgb}{0.000000,0.000000,0.000000}%
\pgfsetstrokecolor{currentstroke}%
\pgfsetdash{}{0pt}%
\pgfpathmoveto{\pgfqpoint{5.528734in}{2.705655in}}%
\pgfpathlineto{\pgfqpoint{5.539173in}{2.653802in}}%
\pgfpathlineto{\pgfqpoint{5.548477in}{2.518512in}}%
\pgfpathlineto{\pgfqpoint{5.582228in}{2.489849in}}%
\pgfpathlineto{\pgfqpoint{5.617134in}{2.543831in}}%
\pgfpathlineto{\pgfqpoint{5.607169in}{2.631516in}}%
\pgfpathlineto{\pgfqpoint{5.594837in}{2.552640in}}%
\pgfpathlineto{\pgfqpoint{5.559406in}{2.455289in}}%
\pgfpathlineto{\pgfqpoint{5.528734in}{2.705655in}}%
\pgfpathclose%
\pgfusepath{fill}%
\end{pgfscope}%
\begin{pgfscope}%
\pgfpathrectangle{\pgfqpoint{1.020000in}{0.880000in}}{\pgfqpoint{6.160000in}{6.160000in}}%
\pgfusepath{clip}%
\pgfsetbuttcap%
\pgfsetroundjoin%
\definecolor{currentfill}{rgb}{0.299441,0.400248,0.839842}%
\pgfsetfillcolor{currentfill}%
\pgfsetlinewidth{0.000000pt}%
\definecolor{currentstroke}{rgb}{0.000000,0.000000,0.000000}%
\pgfsetstrokecolor{currentstroke}%
\pgfsetdash{}{0pt}%
\pgfpathmoveto{\pgfqpoint{5.231086in}{2.516232in}}%
\pgfpathlineto{\pgfqpoint{5.243888in}{2.703392in}}%
\pgfpathlineto{\pgfqpoint{5.251593in}{2.412282in}}%
\pgfpathlineto{\pgfqpoint{5.286357in}{2.461173in}}%
\pgfpathlineto{\pgfqpoint{5.322245in}{2.606855in}}%
\pgfpathlineto{\pgfqpoint{5.310771in}{2.556803in}}%
\pgfpathlineto{\pgfqpoint{5.300239in}{2.587354in}}%
\pgfpathlineto{\pgfqpoint{5.266430in}{2.622258in}}%
\pgfpathlineto{\pgfqpoint{5.231086in}{2.516232in}}%
\pgfpathclose%
\pgfusepath{fill}%
\end{pgfscope}%
\begin{pgfscope}%
\pgfpathrectangle{\pgfqpoint{1.020000in}{0.880000in}}{\pgfqpoint{6.160000in}{6.160000in}}%
\pgfusepath{clip}%
\pgfsetbuttcap%
\pgfsetroundjoin%
\definecolor{currentfill}{rgb}{0.703587,0.802586,0.982847}%
\pgfsetfillcolor{currentfill}%
\pgfsetlinewidth{0.000000pt}%
\definecolor{currentstroke}{rgb}{0.000000,0.000000,0.000000}%
\pgfsetstrokecolor{currentstroke}%
\pgfsetdash{}{0pt}%
\pgfpathmoveto{\pgfqpoint{3.154917in}{3.299094in}}%
\pgfpathlineto{\pgfqpoint{3.162454in}{3.394108in}}%
\pgfpathlineto{\pgfqpoint{3.171721in}{3.321831in}}%
\pgfpathlineto{\pgfqpoint{3.206984in}{3.285964in}}%
\pgfpathlineto{\pgfqpoint{3.241782in}{3.294586in}}%
\pgfpathlineto{\pgfqpoint{3.232579in}{3.358114in}}%
\pgfpathlineto{\pgfqpoint{3.225024in}{3.250873in}}%
\pgfpathlineto{\pgfqpoint{3.188929in}{3.381613in}}%
\pgfpathlineto{\pgfqpoint{3.154917in}{3.299094in}}%
\pgfpathclose%
\pgfusepath{fill}%
\end{pgfscope}%
\begin{pgfscope}%
\pgfpathrectangle{\pgfqpoint{1.020000in}{0.880000in}}{\pgfqpoint{6.160000in}{6.160000in}}%
\pgfusepath{clip}%
\pgfsetbuttcap%
\pgfsetroundjoin%
\definecolor{currentfill}{rgb}{0.451739,0.588181,0.960201}%
\pgfsetfillcolor{currentfill}%
\pgfsetlinewidth{0.000000pt}%
\definecolor{currentstroke}{rgb}{0.000000,0.000000,0.000000}%
\pgfsetstrokecolor{currentstroke}%
\pgfsetdash{}{0pt}%
\pgfpathmoveto{\pgfqpoint{4.758896in}{2.778715in}}%
\pgfpathlineto{\pgfqpoint{4.769822in}{2.889444in}}%
\pgfpathlineto{\pgfqpoint{4.780394in}{2.933469in}}%
\pgfpathlineto{\pgfqpoint{4.814153in}{2.825337in}}%
\pgfpathlineto{\pgfqpoint{4.848570in}{2.829624in}}%
\pgfpathlineto{\pgfqpoint{4.838406in}{2.861156in}}%
\pgfpathlineto{\pgfqpoint{4.827837in}{2.827633in}}%
\pgfpathlineto{\pgfqpoint{4.794121in}{2.925666in}}%
\pgfpathlineto{\pgfqpoint{4.758896in}{2.778715in}}%
\pgfpathclose%
\pgfusepath{fill}%
\end{pgfscope}%
\begin{pgfscope}%
\pgfpathrectangle{\pgfqpoint{1.020000in}{0.880000in}}{\pgfqpoint{6.160000in}{6.160000in}}%
\pgfusepath{clip}%
\pgfsetbuttcap%
\pgfsetroundjoin%
\definecolor{currentfill}{rgb}{0.640828,0.760752,0.997846}%
\pgfsetfillcolor{currentfill}%
\pgfsetlinewidth{0.000000pt}%
\definecolor{currentstroke}{rgb}{0.000000,0.000000,0.000000}%
\pgfsetstrokecolor{currentstroke}%
\pgfsetdash{}{0pt}%
\pgfpathmoveto{\pgfqpoint{3.015223in}{3.311677in}}%
\pgfpathlineto{\pgfqpoint{3.026992in}{3.018606in}}%
\pgfpathlineto{\pgfqpoint{3.034051in}{3.131396in}}%
\pgfpathlineto{\pgfqpoint{3.068208in}{3.200756in}}%
\pgfpathlineto{\pgfqpoint{3.103964in}{3.124885in}}%
\pgfpathlineto{\pgfqpoint{3.093624in}{3.298986in}}%
\pgfpathlineto{\pgfqpoint{3.084923in}{3.323011in}}%
\pgfpathlineto{\pgfqpoint{3.050727in}{3.260527in}}%
\pgfpathlineto{\pgfqpoint{3.015223in}{3.311677in}}%
\pgfpathclose%
\pgfusepath{fill}%
\end{pgfscope}%
\begin{pgfscope}%
\pgfpathrectangle{\pgfqpoint{1.020000in}{0.880000in}}{\pgfqpoint{6.160000in}{6.160000in}}%
\pgfusepath{clip}%
\pgfsetbuttcap%
\pgfsetroundjoin%
\definecolor{currentfill}{rgb}{0.271104,0.360011,0.807095}%
\pgfsetfillcolor{currentfill}%
\pgfsetlinewidth{0.000000pt}%
\definecolor{currentstroke}{rgb}{0.000000,0.000000,0.000000}%
\pgfsetstrokecolor{currentstroke}%
\pgfsetdash{}{0pt}%
\pgfpathmoveto{\pgfqpoint{5.686931in}{2.647689in}}%
\pgfpathlineto{\pgfqpoint{5.693460in}{2.333324in}}%
\pgfpathlineto{\pgfqpoint{5.704853in}{2.338120in}}%
\pgfpathlineto{\pgfqpoint{5.742726in}{2.579437in}}%
\pgfpathlineto{\pgfqpoint{5.773670in}{2.380554in}}%
\pgfpathlineto{\pgfqpoint{5.766369in}{2.636945in}}%
\pgfpathlineto{\pgfqpoint{5.754040in}{2.578125in}}%
\pgfpathlineto{\pgfqpoint{5.718561in}{2.487280in}}%
\pgfpathlineto{\pgfqpoint{5.686931in}{2.647689in}}%
\pgfpathclose%
\pgfusepath{fill}%
\end{pgfscope}%
\begin{pgfscope}%
\pgfpathrectangle{\pgfqpoint{1.020000in}{0.880000in}}{\pgfqpoint{6.160000in}{6.160000in}}%
\pgfusepath{clip}%
\pgfsetbuttcap%
\pgfsetroundjoin%
\definecolor{currentfill}{rgb}{0.677823,0.786546,0.991005}%
\pgfsetfillcolor{currentfill}%
\pgfsetlinewidth{0.000000pt}%
\definecolor{currentstroke}{rgb}{0.000000,0.000000,0.000000}%
\pgfsetstrokecolor{currentstroke}%
\pgfsetdash{}{0pt}%
\pgfpathmoveto{\pgfqpoint{3.084923in}{3.323011in}}%
\pgfpathlineto{\pgfqpoint{3.093624in}{3.298986in}}%
\pgfpathlineto{\pgfqpoint{3.103964in}{3.124885in}}%
\pgfpathlineto{\pgfqpoint{3.137355in}{3.267601in}}%
\pgfpathlineto{\pgfqpoint{3.171721in}{3.321831in}}%
\pgfpathlineto{\pgfqpoint{3.162454in}{3.394108in}}%
\pgfpathlineto{\pgfqpoint{3.154917in}{3.299094in}}%
\pgfpathlineto{\pgfqpoint{3.121712in}{3.144094in}}%
\pgfpathlineto{\pgfqpoint{3.084923in}{3.323011in}}%
\pgfpathclose%
\pgfusepath{fill}%
\end{pgfscope}%
\begin{pgfscope}%
\pgfpathrectangle{\pgfqpoint{1.020000in}{0.880000in}}{\pgfqpoint{6.160000in}{6.160000in}}%
\pgfusepath{clip}%
\pgfsetbuttcap%
\pgfsetroundjoin%
\definecolor{currentfill}{rgb}{0.796064,0.848693,0.933471}%
\pgfsetfillcolor{currentfill}%
\pgfsetlinewidth{0.000000pt}%
\definecolor{currentstroke}{rgb}{0.000000,0.000000,0.000000}%
\pgfsetstrokecolor{currentstroke}%
\pgfsetdash{}{0pt}%
\pgfpathmoveto{\pgfqpoint{3.362504in}{3.431952in}}%
\pgfpathlineto{\pgfqpoint{3.370505in}{3.522166in}}%
\pgfpathlineto{\pgfqpoint{3.379297in}{3.518266in}}%
\pgfpathlineto{\pgfqpoint{3.414892in}{3.421601in}}%
\pgfpathlineto{\pgfqpoint{3.449085in}{3.504092in}}%
\pgfpathlineto{\pgfqpoint{3.439495in}{3.606354in}}%
\pgfpathlineto{\pgfqpoint{3.430951in}{3.569688in}}%
\pgfpathlineto{\pgfqpoint{3.396177in}{3.569119in}}%
\pgfpathlineto{\pgfqpoint{3.362504in}{3.431952in}}%
\pgfpathclose%
\pgfusepath{fill}%
\end{pgfscope}%
\begin{pgfscope}%
\pgfpathrectangle{\pgfqpoint{1.020000in}{0.880000in}}{\pgfqpoint{6.160000in}{6.160000in}}%
\pgfusepath{clip}%
\pgfsetbuttcap%
\pgfsetroundjoin%
\definecolor{currentfill}{rgb}{0.294718,0.393542,0.834384}%
\pgfsetfillcolor{currentfill}%
\pgfsetlinewidth{0.000000pt}%
\definecolor{currentstroke}{rgb}{0.000000,0.000000,0.000000}%
\pgfsetstrokecolor{currentstroke}%
\pgfsetdash{}{0pt}%
\pgfpathmoveto{\pgfqpoint{5.458458in}{2.557235in}}%
\pgfpathlineto{\pgfqpoint{5.468296in}{2.462836in}}%
\pgfpathlineto{\pgfqpoint{5.479064in}{2.439308in}}%
\pgfpathlineto{\pgfqpoint{5.513729in}{2.476019in}}%
\pgfpathlineto{\pgfqpoint{5.548477in}{2.518512in}}%
\pgfpathlineto{\pgfqpoint{5.539173in}{2.653802in}}%
\pgfpathlineto{\pgfqpoint{5.528734in}{2.705655in}}%
\pgfpathlineto{\pgfqpoint{5.491716in}{2.490251in}}%
\pgfpathlineto{\pgfqpoint{5.458458in}{2.557235in}}%
\pgfpathclose%
\pgfusepath{fill}%
\end{pgfscope}%
\begin{pgfscope}%
\pgfpathrectangle{\pgfqpoint{1.020000in}{0.880000in}}{\pgfqpoint{6.160000in}{6.160000in}}%
\pgfusepath{clip}%
\pgfsetbuttcap%
\pgfsetroundjoin%
\definecolor{currentfill}{rgb}{0.804965,0.851666,0.926165}%
\pgfsetfillcolor{currentfill}%
\pgfsetlinewidth{0.000000pt}%
\definecolor{currentstroke}{rgb}{0.000000,0.000000,0.000000}%
\pgfsetstrokecolor{currentstroke}%
\pgfsetdash{}{0pt}%
\pgfpathmoveto{\pgfqpoint{4.179485in}{3.511098in}}%
\pgfpathlineto{\pgfqpoint{4.189131in}{3.646147in}}%
\pgfpathlineto{\pgfqpoint{4.198791in}{3.566897in}}%
\pgfpathlineto{\pgfqpoint{4.233412in}{3.583728in}}%
\pgfpathlineto{\pgfqpoint{4.267830in}{3.372666in}}%
\pgfpathlineto{\pgfqpoint{4.258215in}{3.507355in}}%
\pgfpathlineto{\pgfqpoint{4.248522in}{3.529277in}}%
\pgfpathlineto{\pgfqpoint{4.214035in}{3.598391in}}%
\pgfpathlineto{\pgfqpoint{4.179485in}{3.511098in}}%
\pgfpathclose%
\pgfusepath{fill}%
\end{pgfscope}%
\begin{pgfscope}%
\pgfpathrectangle{\pgfqpoint{1.020000in}{0.880000in}}{\pgfqpoint{6.160000in}{6.160000in}}%
\pgfusepath{clip}%
\pgfsetbuttcap%
\pgfsetroundjoin%
\definecolor{currentfill}{rgb}{0.510824,0.649397,0.985079}%
\pgfsetfillcolor{currentfill}%
\pgfsetlinewidth{0.000000pt}%
\definecolor{currentstroke}{rgb}{0.000000,0.000000,0.000000}%
\pgfsetstrokecolor{currentstroke}%
\pgfsetdash{}{0pt}%
\pgfpathmoveto{\pgfqpoint{4.602478in}{3.084331in}}%
\pgfpathlineto{\pgfqpoint{4.612450in}{3.050146in}}%
\pgfpathlineto{\pgfqpoint{4.622370in}{3.000512in}}%
\pgfpathlineto{\pgfqpoint{4.656766in}{2.981839in}}%
\pgfpathlineto{\pgfqpoint{4.690774in}{2.893942in}}%
\pgfpathlineto{\pgfqpoint{4.680682in}{2.914026in}}%
\pgfpathlineto{\pgfqpoint{4.670809in}{2.975099in}}%
\pgfpathlineto{\pgfqpoint{4.635821in}{2.842683in}}%
\pgfpathlineto{\pgfqpoint{4.602478in}{3.084331in}}%
\pgfpathclose%
\pgfusepath{fill}%
\end{pgfscope}%
\begin{pgfscope}%
\pgfpathrectangle{\pgfqpoint{1.020000in}{0.880000in}}{\pgfqpoint{6.160000in}{6.160000in}}%
\pgfusepath{clip}%
\pgfsetbuttcap%
\pgfsetroundjoin%
\definecolor{currentfill}{rgb}{0.378598,0.503856,0.913692}%
\pgfsetfillcolor{currentfill}%
\pgfsetlinewidth{0.000000pt}%
\definecolor{currentstroke}{rgb}{0.000000,0.000000,0.000000}%
\pgfsetstrokecolor{currentstroke}%
\pgfsetdash{}{0pt}%
\pgfpathmoveto{\pgfqpoint{4.848570in}{2.829624in}}%
\pgfpathlineto{\pgfqpoint{4.857684in}{2.641391in}}%
\pgfpathlineto{\pgfqpoint{4.867260in}{2.522374in}}%
\pgfpathlineto{\pgfqpoint{4.903054in}{2.724168in}}%
\pgfpathlineto{\pgfqpoint{4.936730in}{2.631511in}}%
\pgfpathlineto{\pgfqpoint{4.927168in}{2.752291in}}%
\pgfpathlineto{\pgfqpoint{4.916555in}{2.730979in}}%
\pgfpathlineto{\pgfqpoint{4.883274in}{2.877433in}}%
\pgfpathlineto{\pgfqpoint{4.848570in}{2.829624in}}%
\pgfpathclose%
\pgfusepath{fill}%
\end{pgfscope}%
\begin{pgfscope}%
\pgfpathrectangle{\pgfqpoint{1.020000in}{0.880000in}}{\pgfqpoint{6.160000in}{6.160000in}}%
\pgfusepath{clip}%
\pgfsetbuttcap%
\pgfsetroundjoin%
\definecolor{currentfill}{rgb}{0.772706,0.838978,0.949319}%
\pgfsetfillcolor{currentfill}%
\pgfsetlinewidth{0.000000pt}%
\definecolor{currentstroke}{rgb}{0.000000,0.000000,0.000000}%
\pgfsetstrokecolor{currentstroke}%
\pgfsetdash{}{0pt}%
\pgfpathmoveto{\pgfqpoint{3.291945in}{3.543493in}}%
\pgfpathlineto{\pgfqpoint{3.301257in}{3.470806in}}%
\pgfpathlineto{\pgfqpoint{3.311653in}{3.274822in}}%
\pgfpathlineto{\pgfqpoint{3.345758in}{3.358948in}}%
\pgfpathlineto{\pgfqpoint{3.379297in}{3.518266in}}%
\pgfpathlineto{\pgfqpoint{3.370505in}{3.522166in}}%
\pgfpathlineto{\pgfqpoint{3.362504in}{3.431952in}}%
\pgfpathlineto{\pgfqpoint{3.326175in}{3.613901in}}%
\pgfpathlineto{\pgfqpoint{3.291945in}{3.543493in}}%
\pgfpathclose%
\pgfusepath{fill}%
\end{pgfscope}%
\begin{pgfscope}%
\pgfpathrectangle{\pgfqpoint{1.020000in}{0.880000in}}{\pgfqpoint{6.160000in}{6.160000in}}%
\pgfusepath{clip}%
\pgfsetbuttcap%
\pgfsetroundjoin%
\definecolor{currentfill}{rgb}{0.548876,0.685104,0.994379}%
\pgfsetfillcolor{currentfill}%
\pgfsetlinewidth{0.000000pt}%
\definecolor{currentstroke}{rgb}{0.000000,0.000000,0.000000}%
\pgfsetstrokecolor{currentstroke}%
\pgfsetdash{}{0pt}%
\pgfpathmoveto{\pgfqpoint{2.807595in}{3.140454in}}%
\pgfpathlineto{\pgfqpoint{2.817573in}{3.006125in}}%
\pgfpathlineto{\pgfqpoint{2.824871in}{3.068026in}}%
\pgfpathlineto{\pgfqpoint{2.860493in}{3.024169in}}%
\pgfpathlineto{\pgfqpoint{2.895956in}{2.988508in}}%
\pgfpathlineto{\pgfqpoint{2.888233in}{2.950801in}}%
\pgfpathlineto{\pgfqpoint{2.879495in}{2.992523in}}%
\pgfpathlineto{\pgfqpoint{2.842696in}{3.132365in}}%
\pgfpathlineto{\pgfqpoint{2.807595in}{3.140454in}}%
\pgfpathclose%
\pgfusepath{fill}%
\end{pgfscope}%
\begin{pgfscope}%
\pgfpathrectangle{\pgfqpoint{1.020000in}{0.880000in}}{\pgfqpoint{6.160000in}{6.160000in}}%
\pgfusepath{clip}%
\pgfsetbuttcap%
\pgfsetroundjoin%
\definecolor{currentfill}{rgb}{0.887752,0.854040,0.834671}%
\pgfsetfillcolor{currentfill}%
\pgfsetlinewidth{0.000000pt}%
\definecolor{currentstroke}{rgb}{0.000000,0.000000,0.000000}%
\pgfsetstrokecolor{currentstroke}%
\pgfsetdash{}{0pt}%
\pgfpathmoveto{\pgfqpoint{3.864514in}{3.763182in}}%
\pgfpathlineto{\pgfqpoint{3.873475in}{3.872581in}}%
\pgfpathlineto{\pgfqpoint{3.883518in}{3.633610in}}%
\pgfpathlineto{\pgfqpoint{3.917942in}{3.724604in}}%
\pgfpathlineto{\pgfqpoint{3.952495in}{3.777824in}}%
\pgfpathlineto{\pgfqpoint{3.943304in}{3.694635in}}%
\pgfpathlineto{\pgfqpoint{3.933801in}{3.751441in}}%
\pgfpathlineto{\pgfqpoint{3.899333in}{3.699907in}}%
\pgfpathlineto{\pgfqpoint{3.864514in}{3.763182in}}%
\pgfpathclose%
\pgfusepath{fill}%
\end{pgfscope}%
\begin{pgfscope}%
\pgfpathrectangle{\pgfqpoint{1.020000in}{0.880000in}}{\pgfqpoint{6.160000in}{6.160000in}}%
\pgfusepath{clip}%
\pgfsetbuttcap%
\pgfsetroundjoin%
\definecolor{currentfill}{rgb}{0.261805,0.346484,0.795658}%
\pgfsetfillcolor{currentfill}%
\pgfsetlinewidth{0.000000pt}%
\definecolor{currentstroke}{rgb}{0.000000,0.000000,0.000000}%
\pgfsetstrokecolor{currentstroke}%
\pgfsetdash{}{0pt}%
\pgfpathmoveto{\pgfqpoint{5.844054in}{2.518024in}}%
\pgfpathlineto{\pgfqpoint{5.855915in}{2.539286in}}%
\pgfpathlineto{\pgfqpoint{5.888124in}{2.429664in}}%
\pgfpathlineto{\pgfqpoint{5.922388in}{2.442468in}}%
\pgfpathlineto{\pgfqpoint{5.911145in}{2.460771in}}%
\pgfpathlineto{\pgfqpoint{5.876489in}{2.423290in}}%
\pgfpathlineto{\pgfqpoint{5.844054in}{2.518024in}}%
\pgfpathclose%
\pgfusepath{fill}%
\end{pgfscope}%
\begin{pgfscope}%
\pgfpathrectangle{\pgfqpoint{1.020000in}{0.880000in}}{\pgfqpoint{6.160000in}{6.160000in}}%
\pgfusepath{clip}%
\pgfsetbuttcap%
\pgfsetroundjoin%
\definecolor{currentfill}{rgb}{0.835345,0.860514,0.898970}%
\pgfsetfillcolor{currentfill}%
\pgfsetlinewidth{0.000000pt}%
\definecolor{currentstroke}{rgb}{0.000000,0.000000,0.000000}%
\pgfsetstrokecolor{currentstroke}%
\pgfsetdash{}{0pt}%
\pgfpathmoveto{\pgfqpoint{3.430951in}{3.569688in}}%
\pgfpathlineto{\pgfqpoint{3.439495in}{3.606354in}}%
\pgfpathlineto{\pgfqpoint{3.449085in}{3.504092in}}%
\pgfpathlineto{\pgfqpoint{3.483720in}{3.528279in}}%
\pgfpathlineto{\pgfqpoint{3.517488in}{3.682501in}}%
\pgfpathlineto{\pgfqpoint{3.508968in}{3.623507in}}%
\pgfpathlineto{\pgfqpoint{3.500024in}{3.631147in}}%
\pgfpathlineto{\pgfqpoint{3.465114in}{3.652431in}}%
\pgfpathlineto{\pgfqpoint{3.430951in}{3.569688in}}%
\pgfpathclose%
\pgfusepath{fill}%
\end{pgfscope}%
\begin{pgfscope}%
\pgfpathrectangle{\pgfqpoint{1.020000in}{0.880000in}}{\pgfqpoint{6.160000in}{6.160000in}}%
\pgfusepath{clip}%
\pgfsetbuttcap%
\pgfsetroundjoin%
\definecolor{currentfill}{rgb}{0.871493,0.862309,0.857016}%
\pgfsetfillcolor{currentfill}%
\pgfsetlinewidth{0.000000pt}%
\definecolor{currentstroke}{rgb}{0.000000,0.000000,0.000000}%
\pgfsetstrokecolor{currentstroke}%
\pgfsetdash{}{0pt}%
\pgfpathmoveto{\pgfqpoint{3.795491in}{3.680585in}}%
\pgfpathlineto{\pgfqpoint{3.805150in}{3.564429in}}%
\pgfpathlineto{\pgfqpoint{3.814118in}{3.635353in}}%
\pgfpathlineto{\pgfqpoint{3.848810in}{3.640315in}}%
\pgfpathlineto{\pgfqpoint{3.883518in}{3.633610in}}%
\pgfpathlineto{\pgfqpoint{3.873475in}{3.872581in}}%
\pgfpathlineto{\pgfqpoint{3.864514in}{3.763182in}}%
\pgfpathlineto{\pgfqpoint{3.829947in}{3.736632in}}%
\pgfpathlineto{\pgfqpoint{3.795491in}{3.680585in}}%
\pgfpathclose%
\pgfusepath{fill}%
\end{pgfscope}%
\begin{pgfscope}%
\pgfpathrectangle{\pgfqpoint{1.020000in}{0.880000in}}{\pgfqpoint{6.160000in}{6.160000in}}%
\pgfusepath{clip}%
\pgfsetbuttcap%
\pgfsetroundjoin%
\definecolor{currentfill}{rgb}{0.343278,0.459354,0.884122}%
\pgfsetfillcolor{currentfill}%
\pgfsetlinewidth{0.000000pt}%
\definecolor{currentstroke}{rgb}{0.000000,0.000000,0.000000}%
\pgfsetstrokecolor{currentstroke}%
\pgfsetdash{}{0pt}%
\pgfpathmoveto{\pgfqpoint{5.005542in}{2.653309in}}%
\pgfpathlineto{\pgfqpoint{5.016596in}{2.712057in}}%
\pgfpathlineto{\pgfqpoint{5.026267in}{2.603727in}}%
\pgfpathlineto{\pgfqpoint{5.061528in}{2.707393in}}%
\pgfpathlineto{\pgfqpoint{5.094471in}{2.553421in}}%
\pgfpathlineto{\pgfqpoint{5.085793in}{2.769829in}}%
\pgfpathlineto{\pgfqpoint{5.072840in}{2.512608in}}%
\pgfpathlineto{\pgfqpoint{5.039530in}{2.618222in}}%
\pgfpathlineto{\pgfqpoint{5.005542in}{2.653309in}}%
\pgfpathclose%
\pgfusepath{fill}%
\end{pgfscope}%
\begin{pgfscope}%
\pgfpathrectangle{\pgfqpoint{1.020000in}{0.880000in}}{\pgfqpoint{6.160000in}{6.160000in}}%
\pgfusepath{clip}%
\pgfsetbuttcap%
\pgfsetroundjoin%
\definecolor{currentfill}{rgb}{0.266381,0.353304,0.801637}%
\pgfsetfillcolor{currentfill}%
\pgfsetlinewidth{0.000000pt}%
\definecolor{currentstroke}{rgb}{0.000000,0.000000,0.000000}%
\pgfsetstrokecolor{currentstroke}%
\pgfsetdash{}{0pt}%
\pgfpathmoveto{\pgfqpoint{5.617134in}{2.543831in}}%
\pgfpathlineto{\pgfqpoint{5.627623in}{2.491746in}}%
\pgfpathlineto{\pgfqpoint{5.638732in}{2.480989in}}%
\pgfpathlineto{\pgfqpoint{5.672338in}{2.443509in}}%
\pgfpathlineto{\pgfqpoint{5.704853in}{2.338120in}}%
\pgfpathlineto{\pgfqpoint{5.693460in}{2.333324in}}%
\pgfpathlineto{\pgfqpoint{5.686931in}{2.647689in}}%
\pgfpathlineto{\pgfqpoint{5.650758in}{2.510207in}}%
\pgfpathlineto{\pgfqpoint{5.617134in}{2.543831in}}%
\pgfpathclose%
\pgfusepath{fill}%
\end{pgfscope}%
\begin{pgfscope}%
\pgfpathrectangle{\pgfqpoint{1.020000in}{0.880000in}}{\pgfqpoint{6.160000in}{6.160000in}}%
\pgfusepath{clip}%
\pgfsetbuttcap%
\pgfsetroundjoin%
\definecolor{currentfill}{rgb}{0.672538,0.782861,0.991982}%
\pgfsetfillcolor{currentfill}%
\pgfsetlinewidth{0.000000pt}%
\definecolor{currentstroke}{rgb}{0.000000,0.000000,0.000000}%
\pgfsetstrokecolor{currentstroke}%
\pgfsetdash{}{0pt}%
\pgfpathmoveto{\pgfqpoint{4.356608in}{3.463140in}}%
\pgfpathlineto{\pgfqpoint{4.366326in}{3.385960in}}%
\pgfpathlineto{\pgfqpoint{4.376034in}{3.300374in}}%
\pgfpathlineto{\pgfqpoint{4.410387in}{3.202382in}}%
\pgfpathlineto{\pgfqpoint{4.444901in}{3.199872in}}%
\pgfpathlineto{\pgfqpoint{4.434880in}{3.161796in}}%
\pgfpathlineto{\pgfqpoint{4.424829in}{3.099330in}}%
\pgfpathlineto{\pgfqpoint{4.390707in}{3.240679in}}%
\pgfpathlineto{\pgfqpoint{4.356608in}{3.463140in}}%
\pgfpathclose%
\pgfusepath{fill}%
\end{pgfscope}%
\begin{pgfscope}%
\pgfpathrectangle{\pgfqpoint{1.020000in}{0.880000in}}{\pgfqpoint{6.160000in}{6.160000in}}%
\pgfusepath{clip}%
\pgfsetbuttcap%
\pgfsetroundjoin%
\definecolor{currentfill}{rgb}{0.543440,0.680003,0.993051}%
\pgfsetfillcolor{currentfill}%
\pgfsetlinewidth{0.000000pt}%
\definecolor{currentstroke}{rgb}{0.000000,0.000000,0.000000}%
\pgfsetstrokecolor{currentstroke}%
\pgfsetdash{}{0pt}%
\pgfpathmoveto{\pgfqpoint{2.739732in}{2.989683in}}%
\pgfpathlineto{\pgfqpoint{2.747883in}{2.982738in}}%
\pgfpathlineto{\pgfqpoint{2.755842in}{2.990349in}}%
\pgfpathlineto{\pgfqpoint{2.790155in}{3.043374in}}%
\pgfpathlineto{\pgfqpoint{2.824871in}{3.068026in}}%
\pgfpathlineto{\pgfqpoint{2.817573in}{3.006125in}}%
\pgfpathlineto{\pgfqpoint{2.807595in}{3.140454in}}%
\pgfpathlineto{\pgfqpoint{2.775406in}{2.940851in}}%
\pgfpathlineto{\pgfqpoint{2.739732in}{2.989683in}}%
\pgfpathclose%
\pgfusepath{fill}%
\end{pgfscope}%
\begin{pgfscope}%
\pgfpathrectangle{\pgfqpoint{1.020000in}{0.880000in}}{\pgfqpoint{6.160000in}{6.160000in}}%
\pgfusepath{clip}%
\pgfsetbuttcap%
\pgfsetroundjoin%
\definecolor{currentfill}{rgb}{0.763363,0.835092,0.955658}%
\pgfsetfillcolor{currentfill}%
\pgfsetlinewidth{0.000000pt}%
\definecolor{currentstroke}{rgb}{0.000000,0.000000,0.000000}%
\pgfsetstrokecolor{currentstroke}%
\pgfsetdash{}{0pt}%
\pgfpathmoveto{\pgfqpoint{4.267830in}{3.372666in}}%
\pgfpathlineto{\pgfqpoint{4.277630in}{3.438920in}}%
\pgfpathlineto{\pgfqpoint{4.287549in}{3.586782in}}%
\pgfpathlineto{\pgfqpoint{4.321954in}{3.411798in}}%
\pgfpathlineto{\pgfqpoint{4.356608in}{3.463140in}}%
\pgfpathlineto{\pgfqpoint{4.346536in}{3.319825in}}%
\pgfpathlineto{\pgfqpoint{4.337043in}{3.521100in}}%
\pgfpathlineto{\pgfqpoint{4.302342in}{3.380286in}}%
\pgfpathlineto{\pgfqpoint{4.267830in}{3.372666in}}%
\pgfpathclose%
\pgfusepath{fill}%
\end{pgfscope}%
\begin{pgfscope}%
\pgfpathrectangle{\pgfqpoint{1.020000in}{0.880000in}}{\pgfqpoint{6.160000in}{6.160000in}}%
\pgfusepath{clip}%
\pgfsetbuttcap%
\pgfsetroundjoin%
\definecolor{currentfill}{rgb}{0.294718,0.393542,0.834384}%
\pgfsetfillcolor{currentfill}%
\pgfsetlinewidth{0.000000pt}%
\definecolor{currentstroke}{rgb}{0.000000,0.000000,0.000000}%
\pgfsetstrokecolor{currentstroke}%
\pgfsetdash{}{0pt}%
\pgfpathmoveto{\pgfqpoint{5.388939in}{2.461126in}}%
\pgfpathlineto{\pgfqpoint{5.402951in}{2.707664in}}%
\pgfpathlineto{\pgfqpoint{5.411936in}{2.542335in}}%
\pgfpathlineto{\pgfqpoint{5.445307in}{2.473427in}}%
\pgfpathlineto{\pgfqpoint{5.479064in}{2.439308in}}%
\pgfpathlineto{\pgfqpoint{5.468296in}{2.462836in}}%
\pgfpathlineto{\pgfqpoint{5.458458in}{2.557235in}}%
\pgfpathlineto{\pgfqpoint{5.425403in}{2.645794in}}%
\pgfpathlineto{\pgfqpoint{5.388939in}{2.461126in}}%
\pgfpathclose%
\pgfusepath{fill}%
\end{pgfscope}%
\begin{pgfscope}%
\pgfpathrectangle{\pgfqpoint{1.020000in}{0.880000in}}{\pgfqpoint{6.160000in}{6.160000in}}%
\pgfusepath{clip}%
\pgfsetbuttcap%
\pgfsetroundjoin%
\definecolor{currentfill}{rgb}{0.851372,0.863125,0.881064}%
\pgfsetfillcolor{currentfill}%
\pgfsetlinewidth{0.000000pt}%
\definecolor{currentstroke}{rgb}{0.000000,0.000000,0.000000}%
\pgfsetstrokecolor{currentstroke}%
\pgfsetdash{}{0pt}%
\pgfpathmoveto{\pgfqpoint{3.952495in}{3.777824in}}%
\pgfpathlineto{\pgfqpoint{3.962244in}{3.624508in}}%
\pgfpathlineto{\pgfqpoint{3.971625in}{3.641696in}}%
\pgfpathlineto{\pgfqpoint{4.006466in}{3.559437in}}%
\pgfpathlineto{\pgfqpoint{4.041130in}{3.546807in}}%
\pgfpathlineto{\pgfqpoint{4.031486in}{3.654658in}}%
\pgfpathlineto{\pgfqpoint{4.021997in}{3.666830in}}%
\pgfpathlineto{\pgfqpoint{3.987502in}{3.606565in}}%
\pgfpathlineto{\pgfqpoint{3.952495in}{3.777824in}}%
\pgfpathclose%
\pgfusepath{fill}%
\end{pgfscope}%
\begin{pgfscope}%
\pgfpathrectangle{\pgfqpoint{1.020000in}{0.880000in}}{\pgfqpoint{6.160000in}{6.160000in}}%
\pgfusepath{clip}%
\pgfsetbuttcap%
\pgfsetroundjoin%
\definecolor{currentfill}{rgb}{0.467678,0.605591,0.968546}%
\pgfsetfillcolor{currentfill}%
\pgfsetlinewidth{0.000000pt}%
\definecolor{currentstroke}{rgb}{0.000000,0.000000,0.000000}%
\pgfsetstrokecolor{currentstroke}%
\pgfsetdash{}{0pt}%
\pgfpathmoveto{\pgfqpoint{4.690774in}{2.893942in}}%
\pgfpathlineto{\pgfqpoint{4.700870in}{2.871011in}}%
\pgfpathlineto{\pgfqpoint{4.711758in}{2.993807in}}%
\pgfpathlineto{\pgfqpoint{4.744739in}{2.723418in}}%
\pgfpathlineto{\pgfqpoint{4.780394in}{2.933469in}}%
\pgfpathlineto{\pgfqpoint{4.769822in}{2.889444in}}%
\pgfpathlineto{\pgfqpoint{4.758896in}{2.778715in}}%
\pgfpathlineto{\pgfqpoint{4.725143in}{2.887335in}}%
\pgfpathlineto{\pgfqpoint{4.690774in}{2.893942in}}%
\pgfpathclose%
\pgfusepath{fill}%
\end{pgfscope}%
\begin{pgfscope}%
\pgfpathrectangle{\pgfqpoint{1.020000in}{0.880000in}}{\pgfqpoint{6.160000in}{6.160000in}}%
\pgfusepath{clip}%
\pgfsetbuttcap%
\pgfsetroundjoin%
\definecolor{currentfill}{rgb}{0.871493,0.862309,0.857016}%
\pgfsetfillcolor{currentfill}%
\pgfsetlinewidth{0.000000pt}%
\definecolor{currentstroke}{rgb}{0.000000,0.000000,0.000000}%
\pgfsetstrokecolor{currentstroke}%
\pgfsetdash{}{0pt}%
\pgfpathmoveto{\pgfqpoint{3.725986in}{3.710340in}}%
\pgfpathlineto{\pgfqpoint{3.735129in}{3.712324in}}%
\pgfpathlineto{\pgfqpoint{3.744124in}{3.752891in}}%
\pgfpathlineto{\pgfqpoint{3.779268in}{3.665196in}}%
\pgfpathlineto{\pgfqpoint{3.814118in}{3.635353in}}%
\pgfpathlineto{\pgfqpoint{3.805150in}{3.564429in}}%
\pgfpathlineto{\pgfqpoint{3.795491in}{3.680585in}}%
\pgfpathlineto{\pgfqpoint{3.760531in}{3.748524in}}%
\pgfpathlineto{\pgfqpoint{3.725986in}{3.710340in}}%
\pgfpathclose%
\pgfusepath{fill}%
\end{pgfscope}%
\begin{pgfscope}%
\pgfpathrectangle{\pgfqpoint{1.020000in}{0.880000in}}{\pgfqpoint{6.160000in}{6.160000in}}%
\pgfusepath{clip}%
\pgfsetbuttcap%
\pgfsetroundjoin%
\definecolor{currentfill}{rgb}{0.323718,0.433158,0.864722}%
\pgfsetfillcolor{currentfill}%
\pgfsetlinewidth{0.000000pt}%
\definecolor{currentstroke}{rgb}{0.000000,0.000000,0.000000}%
\pgfsetstrokecolor{currentstroke}%
\pgfsetdash{}{0pt}%
\pgfpathmoveto{\pgfqpoint{5.163078in}{2.559588in}}%
\pgfpathlineto{\pgfqpoint{5.174325in}{2.610538in}}%
\pgfpathlineto{\pgfqpoint{5.184958in}{2.597253in}}%
\pgfpathlineto{\pgfqpoint{5.219115in}{2.581069in}}%
\pgfpathlineto{\pgfqpoint{5.251593in}{2.412282in}}%
\pgfpathlineto{\pgfqpoint{5.243888in}{2.703392in}}%
\pgfpathlineto{\pgfqpoint{5.231086in}{2.516232in}}%
\pgfpathlineto{\pgfqpoint{5.199043in}{2.727822in}}%
\pgfpathlineto{\pgfqpoint{5.163078in}{2.559588in}}%
\pgfpathclose%
\pgfusepath{fill}%
\end{pgfscope}%
\begin{pgfscope}%
\pgfpathrectangle{\pgfqpoint{1.020000in}{0.880000in}}{\pgfqpoint{6.160000in}{6.160000in}}%
\pgfusepath{clip}%
\pgfsetbuttcap%
\pgfsetroundjoin%
\definecolor{currentfill}{rgb}{0.261805,0.346484,0.795658}%
\pgfsetfillcolor{currentfill}%
\pgfsetlinewidth{0.000000pt}%
\definecolor{currentstroke}{rgb}{0.000000,0.000000,0.000000}%
\pgfsetstrokecolor{currentstroke}%
\pgfsetdash{}{0pt}%
\pgfpathmoveto{\pgfqpoint{5.548477in}{2.518512in}}%
\pgfpathlineto{\pgfqpoint{5.558539in}{2.438319in}}%
\pgfpathlineto{\pgfqpoint{5.569370in}{2.412504in}}%
\pgfpathlineto{\pgfqpoint{5.602091in}{2.310231in}}%
\pgfpathlineto{\pgfqpoint{5.638732in}{2.480989in}}%
\pgfpathlineto{\pgfqpoint{5.627623in}{2.491746in}}%
\pgfpathlineto{\pgfqpoint{5.617134in}{2.543831in}}%
\pgfpathlineto{\pgfqpoint{5.582228in}{2.489849in}}%
\pgfpathlineto{\pgfqpoint{5.548477in}{2.518512in}}%
\pgfpathclose%
\pgfusepath{fill}%
\end{pgfscope}%
\begin{pgfscope}%
\pgfpathrectangle{\pgfqpoint{1.020000in}{0.880000in}}{\pgfqpoint{6.160000in}{6.160000in}}%
\pgfusepath{clip}%
\pgfsetbuttcap%
\pgfsetroundjoin%
\definecolor{currentfill}{rgb}{0.532568,0.669801,0.990393}%
\pgfsetfillcolor{currentfill}%
\pgfsetlinewidth{0.000000pt}%
\definecolor{currentstroke}{rgb}{0.000000,0.000000,0.000000}%
\pgfsetstrokecolor{currentstroke}%
\pgfsetdash{}{0pt}%
\pgfpathmoveto{\pgfqpoint{2.670472in}{2.939091in}}%
\pgfpathlineto{\pgfqpoint{2.676409in}{3.073797in}}%
\pgfpathlineto{\pgfqpoint{2.684773in}{3.049792in}}%
\pgfpathlineto{\pgfqpoint{2.721217in}{2.959603in}}%
\pgfpathlineto{\pgfqpoint{2.755842in}{2.990349in}}%
\pgfpathlineto{\pgfqpoint{2.747883in}{2.982738in}}%
\pgfpathlineto{\pgfqpoint{2.739732in}{2.989683in}}%
\pgfpathlineto{\pgfqpoint{2.704873in}{2.979955in}}%
\pgfpathlineto{\pgfqpoint{2.670472in}{2.939091in}}%
\pgfpathclose%
\pgfusepath{fill}%
\end{pgfscope}%
\begin{pgfscope}%
\pgfpathrectangle{\pgfqpoint{1.020000in}{0.880000in}}{\pgfqpoint{6.160000in}{6.160000in}}%
\pgfusepath{clip}%
\pgfsetbuttcap%
\pgfsetroundjoin%
\definecolor{currentfill}{rgb}{0.619318,0.744121,0.998931}%
\pgfsetfillcolor{currentfill}%
\pgfsetlinewidth{0.000000pt}%
\definecolor{currentstroke}{rgb}{0.000000,0.000000,0.000000}%
\pgfsetstrokecolor{currentstroke}%
\pgfsetdash{}{0pt}%
\pgfpathmoveto{\pgfqpoint{4.444901in}{3.199872in}}%
\pgfpathlineto{\pgfqpoint{4.454586in}{3.103528in}}%
\pgfpathlineto{\pgfqpoint{4.464481in}{3.080740in}}%
\pgfpathlineto{\pgfqpoint{4.499295in}{3.171138in}}%
\pgfpathlineto{\pgfqpoint{4.533322in}{3.026088in}}%
\pgfpathlineto{\pgfqpoint{4.523808in}{3.174835in}}%
\pgfpathlineto{\pgfqpoint{4.514258in}{3.314925in}}%
\pgfpathlineto{\pgfqpoint{4.479204in}{3.134024in}}%
\pgfpathlineto{\pgfqpoint{4.444901in}{3.199872in}}%
\pgfpathclose%
\pgfusepath{fill}%
\end{pgfscope}%
\begin{pgfscope}%
\pgfpathrectangle{\pgfqpoint{1.020000in}{0.880000in}}{\pgfqpoint{6.160000in}{6.160000in}}%
\pgfusepath{clip}%
\pgfsetbuttcap%
\pgfsetroundjoin%
\definecolor{currentfill}{rgb}{0.275827,0.366717,0.812553}%
\pgfsetfillcolor{currentfill}%
\pgfsetlinewidth{0.000000pt}%
\definecolor{currentstroke}{rgb}{0.000000,0.000000,0.000000}%
\pgfsetstrokecolor{currentstroke}%
\pgfsetdash{}{0pt}%
\pgfpathmoveto{\pgfqpoint{5.773670in}{2.380554in}}%
\pgfpathlineto{\pgfqpoint{5.786381in}{2.460438in}}%
\pgfpathlineto{\pgfqpoint{5.819141in}{2.378620in}}%
\pgfpathlineto{\pgfqpoint{5.855915in}{2.539286in}}%
\pgfpathlineto{\pgfqpoint{5.844054in}{2.518024in}}%
\pgfpathlineto{\pgfqpoint{5.812226in}{2.654992in}}%
\pgfpathlineto{\pgfqpoint{5.773670in}{2.380554in}}%
\pgfpathclose%
\pgfusepath{fill}%
\end{pgfscope}%
\begin{pgfscope}%
\pgfpathrectangle{\pgfqpoint{1.020000in}{0.880000in}}{\pgfqpoint{6.160000in}{6.160000in}}%
\pgfusepath{clip}%
\pgfsetbuttcap%
\pgfsetroundjoin%
\definecolor{currentfill}{rgb}{0.248091,0.326013,0.777669}%
\pgfsetfillcolor{currentfill}%
\pgfsetlinewidth{0.000000pt}%
\definecolor{currentstroke}{rgb}{0.000000,0.000000,0.000000}%
\pgfsetstrokecolor{currentstroke}%
\pgfsetdash{}{0pt}%
\pgfpathmoveto{\pgfqpoint{5.704853in}{2.338120in}}%
\pgfpathlineto{\pgfqpoint{5.716138in}{2.334320in}}%
\pgfpathlineto{\pgfqpoint{5.752599in}{2.482650in}}%
\pgfpathlineto{\pgfqpoint{5.786381in}{2.460438in}}%
\pgfpathlineto{\pgfqpoint{5.773670in}{2.380554in}}%
\pgfpathlineto{\pgfqpoint{5.742726in}{2.579437in}}%
\pgfpathlineto{\pgfqpoint{5.704853in}{2.338120in}}%
\pgfpathclose%
\pgfusepath{fill}%
\end{pgfscope}%
\begin{pgfscope}%
\pgfpathrectangle{\pgfqpoint{1.020000in}{0.880000in}}{\pgfqpoint{6.160000in}{6.160000in}}%
\pgfusepath{clip}%
\pgfsetbuttcap%
\pgfsetroundjoin%
\definecolor{currentfill}{rgb}{0.299441,0.400248,0.839842}%
\pgfsetfillcolor{currentfill}%
\pgfsetlinewidth{0.000000pt}%
\definecolor{currentstroke}{rgb}{0.000000,0.000000,0.000000}%
\pgfsetstrokecolor{currentstroke}%
\pgfsetdash{}{0pt}%
\pgfpathmoveto{\pgfqpoint{5.094471in}{2.553421in}}%
\pgfpathlineto{\pgfqpoint{5.105386in}{2.580091in}}%
\pgfpathlineto{\pgfqpoint{5.114770in}{2.440798in}}%
\pgfpathlineto{\pgfqpoint{5.149750in}{2.509037in}}%
\pgfpathlineto{\pgfqpoint{5.184958in}{2.597253in}}%
\pgfpathlineto{\pgfqpoint{5.174325in}{2.610538in}}%
\pgfpathlineto{\pgfqpoint{5.163078in}{2.559588in}}%
\pgfpathlineto{\pgfqpoint{5.128234in}{2.498858in}}%
\pgfpathlineto{\pgfqpoint{5.094471in}{2.553421in}}%
\pgfpathclose%
\pgfusepath{fill}%
\end{pgfscope}%
\begin{pgfscope}%
\pgfpathrectangle{\pgfqpoint{1.020000in}{0.880000in}}{\pgfqpoint{6.160000in}{6.160000in}}%
\pgfusepath{clip}%
\pgfsetbuttcap%
\pgfsetroundjoin%
\definecolor{currentfill}{rgb}{0.822420,0.856898,0.910795}%
\pgfsetfillcolor{currentfill}%
\pgfsetlinewidth{0.000000pt}%
\definecolor{currentstroke}{rgb}{0.000000,0.000000,0.000000}%
\pgfsetstrokecolor{currentstroke}%
\pgfsetdash{}{0pt}%
\pgfpathmoveto{\pgfqpoint{4.110298in}{3.616914in}}%
\pgfpathlineto{\pgfqpoint{4.119970in}{3.468235in}}%
\pgfpathlineto{\pgfqpoint{4.129456in}{3.651980in}}%
\pgfpathlineto{\pgfqpoint{4.164176in}{3.469071in}}%
\pgfpathlineto{\pgfqpoint{4.198791in}{3.566897in}}%
\pgfpathlineto{\pgfqpoint{4.189131in}{3.646147in}}%
\pgfpathlineto{\pgfqpoint{4.179485in}{3.511098in}}%
\pgfpathlineto{\pgfqpoint{4.144899in}{3.620503in}}%
\pgfpathlineto{\pgfqpoint{4.110298in}{3.616914in}}%
\pgfpathclose%
\pgfusepath{fill}%
\end{pgfscope}%
\begin{pgfscope}%
\pgfpathrectangle{\pgfqpoint{1.020000in}{0.880000in}}{\pgfqpoint{6.160000in}{6.160000in}}%
\pgfusepath{clip}%
\pgfsetbuttcap%
\pgfsetroundjoin%
\definecolor{currentfill}{rgb}{0.353369,0.472069,0.892570}%
\pgfsetfillcolor{currentfill}%
\pgfsetlinewidth{0.000000pt}%
\definecolor{currentstroke}{rgb}{0.000000,0.000000,0.000000}%
\pgfsetstrokecolor{currentstroke}%
\pgfsetdash{}{0pt}%
\pgfpathmoveto{\pgfqpoint{4.936730in}{2.631511in}}%
\pgfpathlineto{\pgfqpoint{4.948166in}{2.754515in}}%
\pgfpathlineto{\pgfqpoint{4.958149in}{2.684997in}}%
\pgfpathlineto{\pgfqpoint{4.991655in}{2.573529in}}%
\pgfpathlineto{\pgfqpoint{5.026267in}{2.603727in}}%
\pgfpathlineto{\pgfqpoint{5.016596in}{2.712057in}}%
\pgfpathlineto{\pgfqpoint{5.005542in}{2.653309in}}%
\pgfpathlineto{\pgfqpoint{4.970749in}{2.592883in}}%
\pgfpathlineto{\pgfqpoint{4.936730in}{2.631511in}}%
\pgfpathclose%
\pgfusepath{fill}%
\end{pgfscope}%
\begin{pgfscope}%
\pgfpathrectangle{\pgfqpoint{1.020000in}{0.880000in}}{\pgfqpoint{6.160000in}{6.160000in}}%
\pgfusepath{clip}%
\pgfsetbuttcap%
\pgfsetroundjoin%
\definecolor{currentfill}{rgb}{0.527132,0.664700,0.989065}%
\pgfsetfillcolor{currentfill}%
\pgfsetlinewidth{0.000000pt}%
\definecolor{currentstroke}{rgb}{0.000000,0.000000,0.000000}%
\pgfsetstrokecolor{currentstroke}%
\pgfsetdash{}{0pt}%
\pgfpathmoveto{\pgfqpoint{2.895956in}{2.988508in}}%
\pgfpathlineto{\pgfqpoint{2.904779in}{2.942436in}}%
\pgfpathlineto{\pgfqpoint{2.914049in}{2.862125in}}%
\pgfpathlineto{\pgfqpoint{2.948515in}{2.904843in}}%
\pgfpathlineto{\pgfqpoint{2.982372in}{2.998720in}}%
\pgfpathlineto{\pgfqpoint{2.973238in}{3.065533in}}%
\pgfpathlineto{\pgfqpoint{2.964973in}{3.061832in}}%
\pgfpathlineto{\pgfqpoint{2.930493in}{3.022772in}}%
\pgfpathlineto{\pgfqpoint{2.895956in}{2.988508in}}%
\pgfpathclose%
\pgfusepath{fill}%
\end{pgfscope}%
\begin{pgfscope}%
\pgfpathrectangle{\pgfqpoint{1.020000in}{0.880000in}}{\pgfqpoint{6.160000in}{6.160000in}}%
\pgfusepath{clip}%
\pgfsetbuttcap%
\pgfsetroundjoin%
\definecolor{currentfill}{rgb}{0.294718,0.393542,0.834384}%
\pgfsetfillcolor{currentfill}%
\pgfsetlinewidth{0.000000pt}%
\definecolor{currentstroke}{rgb}{0.000000,0.000000,0.000000}%
\pgfsetstrokecolor{currentstroke}%
\pgfsetdash{}{0pt}%
\pgfpathmoveto{\pgfqpoint{5.322245in}{2.606855in}}%
\pgfpathlineto{\pgfqpoint{5.331798in}{2.488084in}}%
\pgfpathlineto{\pgfqpoint{5.343738in}{2.573441in}}%
\pgfpathlineto{\pgfqpoint{5.377692in}{2.544550in}}%
\pgfpathlineto{\pgfqpoint{5.411936in}{2.542335in}}%
\pgfpathlineto{\pgfqpoint{5.402951in}{2.707664in}}%
\pgfpathlineto{\pgfqpoint{5.388939in}{2.461126in}}%
\pgfpathlineto{\pgfqpoint{5.352856in}{2.299721in}}%
\pgfpathlineto{\pgfqpoint{5.322245in}{2.606855in}}%
\pgfpathclose%
\pgfusepath{fill}%
\end{pgfscope}%
\begin{pgfscope}%
\pgfpathrectangle{\pgfqpoint{1.020000in}{0.880000in}}{\pgfqpoint{6.160000in}{6.160000in}}%
\pgfusepath{clip}%
\pgfsetbuttcap%
\pgfsetroundjoin%
\definecolor{currentfill}{rgb}{0.839351,0.861167,0.894494}%
\pgfsetfillcolor{currentfill}%
\pgfsetlinewidth{0.000000pt}%
\definecolor{currentstroke}{rgb}{0.000000,0.000000,0.000000}%
\pgfsetstrokecolor{currentstroke}%
\pgfsetdash{}{0pt}%
\pgfpathmoveto{\pgfqpoint{3.587999in}{3.533740in}}%
\pgfpathlineto{\pgfqpoint{3.596077in}{3.690345in}}%
\pgfpathlineto{\pgfqpoint{3.606032in}{3.529651in}}%
\pgfpathlineto{\pgfqpoint{3.640532in}{3.584790in}}%
\pgfpathlineto{\pgfqpoint{3.675331in}{3.584412in}}%
\pgfpathlineto{\pgfqpoint{3.666030in}{3.625359in}}%
\pgfpathlineto{\pgfqpoint{3.656380in}{3.735065in}}%
\pgfpathlineto{\pgfqpoint{3.622489in}{3.576678in}}%
\pgfpathlineto{\pgfqpoint{3.587999in}{3.533740in}}%
\pgfpathclose%
\pgfusepath{fill}%
\end{pgfscope}%
\begin{pgfscope}%
\pgfpathrectangle{\pgfqpoint{1.020000in}{0.880000in}}{\pgfqpoint{6.160000in}{6.160000in}}%
\pgfusepath{clip}%
\pgfsetbuttcap%
\pgfsetroundjoin%
\definecolor{currentfill}{rgb}{0.548876,0.685104,0.994379}%
\pgfsetfillcolor{currentfill}%
\pgfsetlinewidth{0.000000pt}%
\definecolor{currentstroke}{rgb}{0.000000,0.000000,0.000000}%
\pgfsetstrokecolor{currentstroke}%
\pgfsetdash{}{0pt}%
\pgfpathmoveto{\pgfqpoint{4.533322in}{3.026088in}}%
\pgfpathlineto{\pgfqpoint{4.543748in}{3.131515in}}%
\pgfpathlineto{\pgfqpoint{4.553068in}{2.927929in}}%
\pgfpathlineto{\pgfqpoint{4.587636in}{2.944181in}}%
\pgfpathlineto{\pgfqpoint{4.622370in}{3.000512in}}%
\pgfpathlineto{\pgfqpoint{4.612450in}{3.050146in}}%
\pgfpathlineto{\pgfqpoint{4.602478in}{3.084331in}}%
\pgfpathlineto{\pgfqpoint{4.567806in}{3.030834in}}%
\pgfpathlineto{\pgfqpoint{4.533322in}{3.026088in}}%
\pgfpathclose%
\pgfusepath{fill}%
\end{pgfscope}%
\begin{pgfscope}%
\pgfpathrectangle{\pgfqpoint{1.020000in}{0.880000in}}{\pgfqpoint{6.160000in}{6.160000in}}%
\pgfusepath{clip}%
\pgfsetbuttcap%
\pgfsetroundjoin%
\definecolor{currentfill}{rgb}{0.234377,0.305542,0.759680}%
\pgfsetfillcolor{currentfill}%
\pgfsetlinewidth{0.000000pt}%
\definecolor{currentstroke}{rgb}{0.000000,0.000000,0.000000}%
\pgfsetstrokecolor{currentstroke}%
\pgfsetdash{}{0pt}%
\pgfpathmoveto{\pgfqpoint{5.638732in}{2.480989in}}%
\pgfpathlineto{\pgfqpoint{5.650717in}{2.527758in}}%
\pgfpathlineto{\pgfqpoint{5.680999in}{2.267907in}}%
\pgfpathlineto{\pgfqpoint{5.716138in}{2.334320in}}%
\pgfpathlineto{\pgfqpoint{5.704853in}{2.338120in}}%
\pgfpathlineto{\pgfqpoint{5.672338in}{2.443509in}}%
\pgfpathlineto{\pgfqpoint{5.638732in}{2.480989in}}%
\pgfpathclose%
\pgfusepath{fill}%
\end{pgfscope}%
\begin{pgfscope}%
\pgfpathrectangle{\pgfqpoint{1.020000in}{0.880000in}}{\pgfqpoint{6.160000in}{6.160000in}}%
\pgfusepath{clip}%
\pgfsetbuttcap%
\pgfsetroundjoin%
\definecolor{currentfill}{rgb}{0.419991,0.552989,0.942630}%
\pgfsetfillcolor{currentfill}%
\pgfsetlinewidth{0.000000pt}%
\definecolor{currentstroke}{rgb}{0.000000,0.000000,0.000000}%
\pgfsetstrokecolor{currentstroke}%
\pgfsetdash{}{0pt}%
\pgfpathmoveto{\pgfqpoint{4.780394in}{2.933469in}}%
\pgfpathlineto{\pgfqpoint{4.790819in}{2.948184in}}%
\pgfpathlineto{\pgfqpoint{4.800029in}{2.763082in}}%
\pgfpathlineto{\pgfqpoint{4.834670in}{2.791158in}}%
\pgfpathlineto{\pgfqpoint{4.867260in}{2.522374in}}%
\pgfpathlineto{\pgfqpoint{4.857684in}{2.641391in}}%
\pgfpathlineto{\pgfqpoint{4.848570in}{2.829624in}}%
\pgfpathlineto{\pgfqpoint{4.814153in}{2.825337in}}%
\pgfpathlineto{\pgfqpoint{4.780394in}{2.933469in}}%
\pgfpathclose%
\pgfusepath{fill}%
\end{pgfscope}%
\begin{pgfscope}%
\pgfpathrectangle{\pgfqpoint{1.020000in}{0.880000in}}{\pgfqpoint{6.160000in}{6.160000in}}%
\pgfusepath{clip}%
\pgfsetbuttcap%
\pgfsetroundjoin%
\definecolor{currentfill}{rgb}{0.871493,0.862309,0.857016}%
\pgfsetfillcolor{currentfill}%
\pgfsetlinewidth{0.000000pt}%
\definecolor{currentstroke}{rgb}{0.000000,0.000000,0.000000}%
\pgfsetstrokecolor{currentstroke}%
\pgfsetdash{}{0pt}%
\pgfpathmoveto{\pgfqpoint{3.656380in}{3.735065in}}%
\pgfpathlineto{\pgfqpoint{3.666030in}{3.625359in}}%
\pgfpathlineto{\pgfqpoint{3.675331in}{3.584412in}}%
\pgfpathlineto{\pgfqpoint{3.709526in}{3.707339in}}%
\pgfpathlineto{\pgfqpoint{3.744124in}{3.752891in}}%
\pgfpathlineto{\pgfqpoint{3.735129in}{3.712324in}}%
\pgfpathlineto{\pgfqpoint{3.725986in}{3.710340in}}%
\pgfpathlineto{\pgfqpoint{3.691573in}{3.646273in}}%
\pgfpathlineto{\pgfqpoint{3.656380in}{3.735065in}}%
\pgfpathclose%
\pgfusepath{fill}%
\end{pgfscope}%
\begin{pgfscope}%
\pgfpathrectangle{\pgfqpoint{1.020000in}{0.880000in}}{\pgfqpoint{6.160000in}{6.160000in}}%
\pgfusepath{clip}%
\pgfsetbuttcap%
\pgfsetroundjoin%
\definecolor{currentfill}{rgb}{0.822420,0.856898,0.910795}%
\pgfsetfillcolor{currentfill}%
\pgfsetlinewidth{0.000000pt}%
\definecolor{currentstroke}{rgb}{0.000000,0.000000,0.000000}%
\pgfsetstrokecolor{currentstroke}%
\pgfsetdash{}{0pt}%
\pgfpathmoveto{\pgfqpoint{4.041130in}{3.546807in}}%
\pgfpathlineto{\pgfqpoint{4.050584in}{3.581839in}}%
\pgfpathlineto{\pgfqpoint{4.060051in}{3.629024in}}%
\pgfpathlineto{\pgfqpoint{4.094860in}{3.528161in}}%
\pgfpathlineto{\pgfqpoint{4.129456in}{3.651980in}}%
\pgfpathlineto{\pgfqpoint{4.119970in}{3.468235in}}%
\pgfpathlineto{\pgfqpoint{4.110298in}{3.616914in}}%
\pgfpathlineto{\pgfqpoint{4.075876in}{3.429031in}}%
\pgfpathlineto{\pgfqpoint{4.041130in}{3.546807in}}%
\pgfpathclose%
\pgfusepath{fill}%
\end{pgfscope}%
\begin{pgfscope}%
\pgfpathrectangle{\pgfqpoint{1.020000in}{0.880000in}}{\pgfqpoint{6.160000in}{6.160000in}}%
\pgfusepath{clip}%
\pgfsetbuttcap%
\pgfsetroundjoin%
\definecolor{currentfill}{rgb}{0.280550,0.373423,0.818011}%
\pgfsetfillcolor{currentfill}%
\pgfsetlinewidth{0.000000pt}%
\definecolor{currentstroke}{rgb}{0.000000,0.000000,0.000000}%
\pgfsetstrokecolor{currentstroke}%
\pgfsetdash{}{0pt}%
\pgfpathmoveto{\pgfqpoint{5.251593in}{2.412282in}}%
\pgfpathlineto{\pgfqpoint{5.263319in}{2.493660in}}%
\pgfpathlineto{\pgfqpoint{5.272900in}{2.376477in}}%
\pgfpathlineto{\pgfqpoint{5.309347in}{2.567539in}}%
\pgfpathlineto{\pgfqpoint{5.343738in}{2.573441in}}%
\pgfpathlineto{\pgfqpoint{5.331798in}{2.488084in}}%
\pgfpathlineto{\pgfqpoint{5.322245in}{2.606855in}}%
\pgfpathlineto{\pgfqpoint{5.286357in}{2.461173in}}%
\pgfpathlineto{\pgfqpoint{5.251593in}{2.412282in}}%
\pgfpathclose%
\pgfusepath{fill}%
\end{pgfscope}%
\begin{pgfscope}%
\pgfpathrectangle{\pgfqpoint{1.020000in}{0.880000in}}{\pgfqpoint{6.160000in}{6.160000in}}%
\pgfusepath{clip}%
\pgfsetbuttcap%
\pgfsetroundjoin%
\definecolor{currentfill}{rgb}{0.266381,0.353304,0.801637}%
\pgfsetfillcolor{currentfill}%
\pgfsetlinewidth{0.000000pt}%
\definecolor{currentstroke}{rgb}{0.000000,0.000000,0.000000}%
\pgfsetstrokecolor{currentstroke}%
\pgfsetdash{}{0pt}%
\pgfpathmoveto{\pgfqpoint{5.479064in}{2.439308in}}%
\pgfpathlineto{\pgfqpoint{5.490851in}{2.491398in}}%
\pgfpathlineto{\pgfqpoint{5.502203in}{2.508154in}}%
\pgfpathlineto{\pgfqpoint{5.535554in}{2.441322in}}%
\pgfpathlineto{\pgfqpoint{5.569370in}{2.412504in}}%
\pgfpathlineto{\pgfqpoint{5.558539in}{2.438319in}}%
\pgfpathlineto{\pgfqpoint{5.548477in}{2.518512in}}%
\pgfpathlineto{\pgfqpoint{5.513729in}{2.476019in}}%
\pgfpathlineto{\pgfqpoint{5.479064in}{2.439308in}}%
\pgfpathclose%
\pgfusepath{fill}%
\end{pgfscope}%
\begin{pgfscope}%
\pgfpathrectangle{\pgfqpoint{1.020000in}{0.880000in}}{\pgfqpoint{6.160000in}{6.160000in}}%
\pgfusepath{clip}%
\pgfsetbuttcap%
\pgfsetroundjoin%
\definecolor{currentfill}{rgb}{0.831148,0.859513,0.903110}%
\pgfsetfillcolor{currentfill}%
\pgfsetlinewidth{0.000000pt}%
\definecolor{currentstroke}{rgb}{0.000000,0.000000,0.000000}%
\pgfsetstrokecolor{currentstroke}%
\pgfsetdash{}{0pt}%
\pgfpathmoveto{\pgfqpoint{3.517488in}{3.682501in}}%
\pgfpathlineto{\pgfqpoint{3.527152in}{3.571547in}}%
\pgfpathlineto{\pgfqpoint{3.536393in}{3.525892in}}%
\pgfpathlineto{\pgfqpoint{3.571235in}{3.525722in}}%
\pgfpathlineto{\pgfqpoint{3.606032in}{3.529651in}}%
\pgfpathlineto{\pgfqpoint{3.596077in}{3.690345in}}%
\pgfpathlineto{\pgfqpoint{3.587999in}{3.533740in}}%
\pgfpathlineto{\pgfqpoint{3.552982in}{3.576053in}}%
\pgfpathlineto{\pgfqpoint{3.517488in}{3.682501in}}%
\pgfpathclose%
\pgfusepath{fill}%
\end{pgfscope}%
\begin{pgfscope}%
\pgfpathrectangle{\pgfqpoint{1.020000in}{0.880000in}}{\pgfqpoint{6.160000in}{6.160000in}}%
\pgfusepath{clip}%
\pgfsetbuttcap%
\pgfsetroundjoin%
\definecolor{currentfill}{rgb}{0.592356,0.722792,0.999434}%
\pgfsetfillcolor{currentfill}%
\pgfsetlinewidth{0.000000pt}%
\definecolor{currentstroke}{rgb}{0.000000,0.000000,0.000000}%
\pgfsetstrokecolor{currentstroke}%
\pgfsetdash{}{0pt}%
\pgfpathmoveto{\pgfqpoint{2.964973in}{3.061832in}}%
\pgfpathlineto{\pgfqpoint{2.973238in}{3.065533in}}%
\pgfpathlineto{\pgfqpoint{2.982372in}{2.998720in}}%
\pgfpathlineto{\pgfqpoint{3.016023in}{3.113955in}}%
\pgfpathlineto{\pgfqpoint{3.050751in}{3.139652in}}%
\pgfpathlineto{\pgfqpoint{3.042308in}{3.142767in}}%
\pgfpathlineto{\pgfqpoint{3.034051in}{3.131396in}}%
\pgfpathlineto{\pgfqpoint{2.999595in}{3.089483in}}%
\pgfpathlineto{\pgfqpoint{2.964973in}{3.061832in}}%
\pgfpathclose%
\pgfusepath{fill}%
\end{pgfscope}%
\begin{pgfscope}%
\pgfpathrectangle{\pgfqpoint{1.020000in}{0.880000in}}{\pgfqpoint{6.160000in}{6.160000in}}%
\pgfusepath{clip}%
\pgfsetbuttcap%
\pgfsetroundjoin%
\definecolor{currentfill}{rgb}{0.782049,0.842864,0.942980}%
\pgfsetfillcolor{currentfill}%
\pgfsetlinewidth{0.000000pt}%
\definecolor{currentstroke}{rgb}{0.000000,0.000000,0.000000}%
\pgfsetstrokecolor{currentstroke}%
\pgfsetdash{}{0pt}%
\pgfpathmoveto{\pgfqpoint{3.379297in}{3.518266in}}%
\pgfpathlineto{\pgfqpoint{3.388592in}{3.453431in}}%
\pgfpathlineto{\pgfqpoint{3.398462in}{3.316314in}}%
\pgfpathlineto{\pgfqpoint{3.431821in}{3.515455in}}%
\pgfpathlineto{\pgfqpoint{3.466123in}{3.599185in}}%
\pgfpathlineto{\pgfqpoint{3.458677in}{3.401597in}}%
\pgfpathlineto{\pgfqpoint{3.449085in}{3.504092in}}%
\pgfpathlineto{\pgfqpoint{3.414892in}{3.421601in}}%
\pgfpathlineto{\pgfqpoint{3.379297in}{3.518266in}}%
\pgfpathclose%
\pgfusepath{fill}%
\end{pgfscope}%
\begin{pgfscope}%
\pgfpathrectangle{\pgfqpoint{1.020000in}{0.880000in}}{\pgfqpoint{6.160000in}{6.160000in}}%
\pgfusepath{clip}%
\pgfsetbuttcap%
\pgfsetroundjoin%
\definecolor{currentfill}{rgb}{0.527132,0.664700,0.989065}%
\pgfsetfillcolor{currentfill}%
\pgfsetlinewidth{0.000000pt}%
\definecolor{currentstroke}{rgb}{0.000000,0.000000,0.000000}%
\pgfsetstrokecolor{currentstroke}%
\pgfsetdash{}{0pt}%
\pgfpathmoveto{\pgfqpoint{2.824871in}{3.068026in}}%
\pgfpathlineto{\pgfqpoint{2.833284in}{3.049742in}}%
\pgfpathlineto{\pgfqpoint{2.843195in}{2.921368in}}%
\pgfpathlineto{\pgfqpoint{2.877847in}{2.952457in}}%
\pgfpathlineto{\pgfqpoint{2.914049in}{2.862125in}}%
\pgfpathlineto{\pgfqpoint{2.904779in}{2.942436in}}%
\pgfpathlineto{\pgfqpoint{2.895956in}{2.988508in}}%
\pgfpathlineto{\pgfqpoint{2.860493in}{3.024169in}}%
\pgfpathlineto{\pgfqpoint{2.824871in}{3.068026in}}%
\pgfpathclose%
\pgfusepath{fill}%
\end{pgfscope}%
\begin{pgfscope}%
\pgfpathrectangle{\pgfqpoint{1.020000in}{0.880000in}}{\pgfqpoint{6.160000in}{6.160000in}}%
\pgfusepath{clip}%
\pgfsetbuttcap%
\pgfsetroundjoin%
\definecolor{currentfill}{rgb}{0.693321,0.796314,0.986308}%
\pgfsetfillcolor{currentfill}%
\pgfsetlinewidth{0.000000pt}%
\definecolor{currentstroke}{rgb}{0.000000,0.000000,0.000000}%
\pgfsetstrokecolor{currentstroke}%
\pgfsetdash{}{0pt}%
\pgfpathmoveto{\pgfqpoint{3.171721in}{3.321831in}}%
\pgfpathlineto{\pgfqpoint{3.180740in}{3.274919in}}%
\pgfpathlineto{\pgfqpoint{3.190024in}{3.202456in}}%
\pgfpathlineto{\pgfqpoint{3.225304in}{3.167284in}}%
\pgfpathlineto{\pgfqpoint{3.258214in}{3.383313in}}%
\pgfpathlineto{\pgfqpoint{3.249932in}{3.344503in}}%
\pgfpathlineto{\pgfqpoint{3.241782in}{3.294586in}}%
\pgfpathlineto{\pgfqpoint{3.206984in}{3.285964in}}%
\pgfpathlineto{\pgfqpoint{3.171721in}{3.321831in}}%
\pgfpathclose%
\pgfusepath{fill}%
\end{pgfscope}%
\begin{pgfscope}%
\pgfpathrectangle{\pgfqpoint{1.020000in}{0.880000in}}{\pgfqpoint{6.160000in}{6.160000in}}%
\pgfusepath{clip}%
\pgfsetbuttcap%
\pgfsetroundjoin%
\definecolor{currentfill}{rgb}{0.358415,0.478426,0.896795}%
\pgfsetfillcolor{currentfill}%
\pgfsetlinewidth{0.000000pt}%
\definecolor{currentstroke}{rgb}{0.000000,0.000000,0.000000}%
\pgfsetstrokecolor{currentstroke}%
\pgfsetdash{}{0pt}%
\pgfpathmoveto{\pgfqpoint{4.867260in}{2.522374in}}%
\pgfpathlineto{\pgfqpoint{4.879996in}{2.854630in}}%
\pgfpathlineto{\pgfqpoint{4.888280in}{2.547731in}}%
\pgfpathlineto{\pgfqpoint{4.922573in}{2.532074in}}%
\pgfpathlineto{\pgfqpoint{4.958149in}{2.684997in}}%
\pgfpathlineto{\pgfqpoint{4.948166in}{2.754515in}}%
\pgfpathlineto{\pgfqpoint{4.936730in}{2.631511in}}%
\pgfpathlineto{\pgfqpoint{4.903054in}{2.724168in}}%
\pgfpathlineto{\pgfqpoint{4.867260in}{2.522374in}}%
\pgfpathclose%
\pgfusepath{fill}%
\end{pgfscope}%
\begin{pgfscope}%
\pgfpathrectangle{\pgfqpoint{1.020000in}{0.880000in}}{\pgfqpoint{6.160000in}{6.160000in}}%
\pgfusepath{clip}%
\pgfsetbuttcap%
\pgfsetroundjoin%
\definecolor{currentfill}{rgb}{0.661968,0.775491,0.993937}%
\pgfsetfillcolor{currentfill}%
\pgfsetlinewidth{0.000000pt}%
\definecolor{currentstroke}{rgb}{0.000000,0.000000,0.000000}%
\pgfsetstrokecolor{currentstroke}%
\pgfsetdash{}{0pt}%
\pgfpathmoveto{\pgfqpoint{3.103964in}{3.124885in}}%
\pgfpathlineto{\pgfqpoint{3.111518in}{3.209158in}}%
\pgfpathlineto{\pgfqpoint{3.119944in}{3.214607in}}%
\pgfpathlineto{\pgfqpoint{3.154898in}{3.218135in}}%
\pgfpathlineto{\pgfqpoint{3.190024in}{3.202456in}}%
\pgfpathlineto{\pgfqpoint{3.180740in}{3.274919in}}%
\pgfpathlineto{\pgfqpoint{3.171721in}{3.321831in}}%
\pgfpathlineto{\pgfqpoint{3.137355in}{3.267601in}}%
\pgfpathlineto{\pgfqpoint{3.103964in}{3.124885in}}%
\pgfpathclose%
\pgfusepath{fill}%
\end{pgfscope}%
\begin{pgfscope}%
\pgfpathrectangle{\pgfqpoint{1.020000in}{0.880000in}}{\pgfqpoint{6.160000in}{6.160000in}}%
\pgfusepath{clip}%
\pgfsetbuttcap%
\pgfsetroundjoin%
\definecolor{currentfill}{rgb}{0.796064,0.848693,0.933471}%
\pgfsetfillcolor{currentfill}%
\pgfsetlinewidth{0.000000pt}%
\definecolor{currentstroke}{rgb}{0.000000,0.000000,0.000000}%
\pgfsetstrokecolor{currentstroke}%
\pgfsetdash{}{0pt}%
\pgfpathmoveto{\pgfqpoint{4.198791in}{3.566897in}}%
\pgfpathlineto{\pgfqpoint{4.208438in}{3.413923in}}%
\pgfpathlineto{\pgfqpoint{4.218194in}{3.595114in}}%
\pgfpathlineto{\pgfqpoint{4.252744in}{3.390116in}}%
\pgfpathlineto{\pgfqpoint{4.287549in}{3.586782in}}%
\pgfpathlineto{\pgfqpoint{4.277630in}{3.438920in}}%
\pgfpathlineto{\pgfqpoint{4.267830in}{3.372666in}}%
\pgfpathlineto{\pgfqpoint{4.233412in}{3.583728in}}%
\pgfpathlineto{\pgfqpoint{4.198791in}{3.566897in}}%
\pgfpathclose%
\pgfusepath{fill}%
\end{pgfscope}%
\begin{pgfscope}%
\pgfpathrectangle{\pgfqpoint{1.020000in}{0.880000in}}{\pgfqpoint{6.160000in}{6.160000in}}%
\pgfusepath{clip}%
\pgfsetbuttcap%
\pgfsetroundjoin%
\definecolor{currentfill}{rgb}{0.635474,0.756714,0.998297}%
\pgfsetfillcolor{currentfill}%
\pgfsetlinewidth{0.000000pt}%
\definecolor{currentstroke}{rgb}{0.000000,0.000000,0.000000}%
\pgfsetstrokecolor{currentstroke}%
\pgfsetdash{}{0pt}%
\pgfpathmoveto{\pgfqpoint{3.034051in}{3.131396in}}%
\pgfpathlineto{\pgfqpoint{3.042308in}{3.142767in}}%
\pgfpathlineto{\pgfqpoint{3.050751in}{3.139652in}}%
\pgfpathlineto{\pgfqpoint{3.085006in}{3.207981in}}%
\pgfpathlineto{\pgfqpoint{3.119944in}{3.214607in}}%
\pgfpathlineto{\pgfqpoint{3.111518in}{3.209158in}}%
\pgfpathlineto{\pgfqpoint{3.103964in}{3.124885in}}%
\pgfpathlineto{\pgfqpoint{3.068208in}{3.200756in}}%
\pgfpathlineto{\pgfqpoint{3.034051in}{3.131396in}}%
\pgfpathclose%
\pgfusepath{fill}%
\end{pgfscope}%
\begin{pgfscope}%
\pgfpathrectangle{\pgfqpoint{1.020000in}{0.880000in}}{\pgfqpoint{6.160000in}{6.160000in}}%
\pgfusepath{clip}%
\pgfsetbuttcap%
\pgfsetroundjoin%
\definecolor{currentfill}{rgb}{0.500031,0.638508,0.981070}%
\pgfsetfillcolor{currentfill}%
\pgfsetlinewidth{0.000000pt}%
\definecolor{currentstroke}{rgb}{0.000000,0.000000,0.000000}%
\pgfsetstrokecolor{currentstroke}%
\pgfsetdash{}{0pt}%
\pgfpathmoveto{\pgfqpoint{4.622370in}{3.000512in}}%
\pgfpathlineto{\pgfqpoint{4.631989in}{2.881329in}}%
\pgfpathlineto{\pgfqpoint{4.642734in}{3.006721in}}%
\pgfpathlineto{\pgfqpoint{4.676030in}{2.752634in}}%
\pgfpathlineto{\pgfqpoint{4.711758in}{2.993807in}}%
\pgfpathlineto{\pgfqpoint{4.700870in}{2.871011in}}%
\pgfpathlineto{\pgfqpoint{4.690774in}{2.893942in}}%
\pgfpathlineto{\pgfqpoint{4.656766in}{2.981839in}}%
\pgfpathlineto{\pgfqpoint{4.622370in}{3.000512in}}%
\pgfpathclose%
\pgfusepath{fill}%
\end{pgfscope}%
\begin{pgfscope}%
\pgfpathrectangle{\pgfqpoint{1.020000in}{0.880000in}}{\pgfqpoint{6.160000in}{6.160000in}}%
\pgfusepath{clip}%
\pgfsetbuttcap%
\pgfsetroundjoin%
\definecolor{currentfill}{rgb}{0.818056,0.855590,0.914638}%
\pgfsetfillcolor{currentfill}%
\pgfsetlinewidth{0.000000pt}%
\definecolor{currentstroke}{rgb}{0.000000,0.000000,0.000000}%
\pgfsetstrokecolor{currentstroke}%
\pgfsetdash{}{0pt}%
\pgfpathmoveto{\pgfqpoint{3.449085in}{3.504092in}}%
\pgfpathlineto{\pgfqpoint{3.458677in}{3.401597in}}%
\pgfpathlineto{\pgfqpoint{3.466123in}{3.599185in}}%
\pgfpathlineto{\pgfqpoint{3.501362in}{3.550731in}}%
\pgfpathlineto{\pgfqpoint{3.536393in}{3.525892in}}%
\pgfpathlineto{\pgfqpoint{3.527152in}{3.571547in}}%
\pgfpathlineto{\pgfqpoint{3.517488in}{3.682501in}}%
\pgfpathlineto{\pgfqpoint{3.483720in}{3.528279in}}%
\pgfpathlineto{\pgfqpoint{3.449085in}{3.504092in}}%
\pgfpathclose%
\pgfusepath{fill}%
\end{pgfscope}%
\begin{pgfscope}%
\pgfpathrectangle{\pgfqpoint{1.020000in}{0.880000in}}{\pgfqpoint{6.160000in}{6.160000in}}%
\pgfusepath{clip}%
\pgfsetbuttcap%
\pgfsetroundjoin%
\definecolor{currentfill}{rgb}{0.763363,0.835092,0.955658}%
\pgfsetfillcolor{currentfill}%
\pgfsetlinewidth{0.000000pt}%
\definecolor{currentstroke}{rgb}{0.000000,0.000000,0.000000}%
\pgfsetstrokecolor{currentstroke}%
\pgfsetdash{}{0pt}%
\pgfpathmoveto{\pgfqpoint{3.311653in}{3.274822in}}%
\pgfpathlineto{\pgfqpoint{3.318693in}{3.465341in}}%
\pgfpathlineto{\pgfqpoint{3.326519in}{3.570229in}}%
\pgfpathlineto{\pgfqpoint{3.362551in}{3.442559in}}%
\pgfpathlineto{\pgfqpoint{3.398462in}{3.316314in}}%
\pgfpathlineto{\pgfqpoint{3.388592in}{3.453431in}}%
\pgfpathlineto{\pgfqpoint{3.379297in}{3.518266in}}%
\pgfpathlineto{\pgfqpoint{3.345758in}{3.358948in}}%
\pgfpathlineto{\pgfqpoint{3.311653in}{3.274822in}}%
\pgfpathclose%
\pgfusepath{fill}%
\end{pgfscope}%
\begin{pgfscope}%
\pgfpathrectangle{\pgfqpoint{1.020000in}{0.880000in}}{\pgfqpoint{6.160000in}{6.160000in}}%
\pgfusepath{clip}%
\pgfsetbuttcap%
\pgfsetroundjoin%
\definecolor{currentfill}{rgb}{0.323718,0.433158,0.864722}%
\pgfsetfillcolor{currentfill}%
\pgfsetlinewidth{0.000000pt}%
\definecolor{currentstroke}{rgb}{0.000000,0.000000,0.000000}%
\pgfsetstrokecolor{currentstroke}%
\pgfsetdash{}{0pt}%
\pgfpathmoveto{\pgfqpoint{5.026267in}{2.603727in}}%
\pgfpathlineto{\pgfqpoint{5.036326in}{2.540521in}}%
\pgfpathlineto{\pgfqpoint{5.047362in}{2.588359in}}%
\pgfpathlineto{\pgfqpoint{5.081796in}{2.591102in}}%
\pgfpathlineto{\pgfqpoint{5.114770in}{2.440798in}}%
\pgfpathlineto{\pgfqpoint{5.105386in}{2.580091in}}%
\pgfpathlineto{\pgfqpoint{5.094471in}{2.553421in}}%
\pgfpathlineto{\pgfqpoint{5.061528in}{2.707393in}}%
\pgfpathlineto{\pgfqpoint{5.026267in}{2.603727in}}%
\pgfpathclose%
\pgfusepath{fill}%
\end{pgfscope}%
\begin{pgfscope}%
\pgfpathrectangle{\pgfqpoint{1.020000in}{0.880000in}}{\pgfqpoint{6.160000in}{6.160000in}}%
\pgfusepath{clip}%
\pgfsetbuttcap%
\pgfsetroundjoin%
\definecolor{currentfill}{rgb}{0.738826,0.822572,0.968261}%
\pgfsetfillcolor{currentfill}%
\pgfsetlinewidth{0.000000pt}%
\definecolor{currentstroke}{rgb}{0.000000,0.000000,0.000000}%
\pgfsetstrokecolor{currentstroke}%
\pgfsetdash{}{0pt}%
\pgfpathmoveto{\pgfqpoint{3.241782in}{3.294586in}}%
\pgfpathlineto{\pgfqpoint{3.249932in}{3.344503in}}%
\pgfpathlineto{\pgfqpoint{3.258214in}{3.383313in}}%
\pgfpathlineto{\pgfqpoint{3.294569in}{3.226938in}}%
\pgfpathlineto{\pgfqpoint{3.326519in}{3.570229in}}%
\pgfpathlineto{\pgfqpoint{3.318693in}{3.465341in}}%
\pgfpathlineto{\pgfqpoint{3.311653in}{3.274822in}}%
\pgfpathlineto{\pgfqpoint{3.275198in}{3.453599in}}%
\pgfpathlineto{\pgfqpoint{3.241782in}{3.294586in}}%
\pgfpathclose%
\pgfusepath{fill}%
\end{pgfscope}%
\begin{pgfscope}%
\pgfpathrectangle{\pgfqpoint{1.020000in}{0.880000in}}{\pgfqpoint{6.160000in}{6.160000in}}%
\pgfusepath{clip}%
\pgfsetbuttcap%
\pgfsetroundjoin%
\definecolor{currentfill}{rgb}{0.651398,0.768121,0.995891}%
\pgfsetfillcolor{currentfill}%
\pgfsetlinewidth{0.000000pt}%
\definecolor{currentstroke}{rgb}{0.000000,0.000000,0.000000}%
\pgfsetstrokecolor{currentstroke}%
\pgfsetdash{}{0pt}%
\pgfpathmoveto{\pgfqpoint{4.376034in}{3.300374in}}%
\pgfpathlineto{\pgfqpoint{4.385764in}{3.222776in}}%
\pgfpathlineto{\pgfqpoint{4.395747in}{3.261160in}}%
\pgfpathlineto{\pgfqpoint{4.430258in}{3.212388in}}%
\pgfpathlineto{\pgfqpoint{4.464481in}{3.080740in}}%
\pgfpathlineto{\pgfqpoint{4.454586in}{3.103528in}}%
\pgfpathlineto{\pgfqpoint{4.444901in}{3.199872in}}%
\pgfpathlineto{\pgfqpoint{4.410387in}{3.202382in}}%
\pgfpathlineto{\pgfqpoint{4.376034in}{3.300374in}}%
\pgfpathclose%
\pgfusepath{fill}%
\end{pgfscope}%
\begin{pgfscope}%
\pgfpathrectangle{\pgfqpoint{1.020000in}{0.880000in}}{\pgfqpoint{6.160000in}{6.160000in}}%
\pgfusepath{clip}%
\pgfsetbuttcap%
\pgfsetroundjoin%
\definecolor{currentfill}{rgb}{0.887752,0.854040,0.834671}%
\pgfsetfillcolor{currentfill}%
\pgfsetlinewidth{0.000000pt}%
\definecolor{currentstroke}{rgb}{0.000000,0.000000,0.000000}%
\pgfsetstrokecolor{currentstroke}%
\pgfsetdash{}{0pt}%
\pgfpathmoveto{\pgfqpoint{3.883518in}{3.633610in}}%
\pgfpathlineto{\pgfqpoint{3.892325in}{3.812918in}}%
\pgfpathlineto{\pgfqpoint{3.901696in}{3.810614in}}%
\pgfpathlineto{\pgfqpoint{3.936854in}{3.663458in}}%
\pgfpathlineto{\pgfqpoint{3.971625in}{3.641696in}}%
\pgfpathlineto{\pgfqpoint{3.962244in}{3.624508in}}%
\pgfpathlineto{\pgfqpoint{3.952495in}{3.777824in}}%
\pgfpathlineto{\pgfqpoint{3.917942in}{3.724604in}}%
\pgfpathlineto{\pgfqpoint{3.883518in}{3.633610in}}%
\pgfpathclose%
\pgfusepath{fill}%
\end{pgfscope}%
\begin{pgfscope}%
\pgfpathrectangle{\pgfqpoint{1.020000in}{0.880000in}}{\pgfqpoint{6.160000in}{6.160000in}}%
\pgfusepath{clip}%
\pgfsetbuttcap%
\pgfsetroundjoin%
\definecolor{currentfill}{rgb}{0.252663,0.332837,0.783665}%
\pgfsetfillcolor{currentfill}%
\pgfsetlinewidth{0.000000pt}%
\definecolor{currentstroke}{rgb}{0.000000,0.000000,0.000000}%
\pgfsetstrokecolor{currentstroke}%
\pgfsetdash{}{0pt}%
\pgfpathmoveto{\pgfqpoint{5.569370in}{2.412504in}}%
\pgfpathlineto{\pgfqpoint{5.579461in}{2.332901in}}%
\pgfpathlineto{\pgfqpoint{5.615777in}{2.479518in}}%
\pgfpathlineto{\pgfqpoint{5.650717in}{2.527758in}}%
\pgfpathlineto{\pgfqpoint{5.638732in}{2.480989in}}%
\pgfpathlineto{\pgfqpoint{5.602091in}{2.310231in}}%
\pgfpathlineto{\pgfqpoint{5.569370in}{2.412504in}}%
\pgfpathclose%
\pgfusepath{fill}%
\end{pgfscope}%
\begin{pgfscope}%
\pgfpathrectangle{\pgfqpoint{1.020000in}{0.880000in}}{\pgfqpoint{6.160000in}{6.160000in}}%
\pgfusepath{clip}%
\pgfsetbuttcap%
\pgfsetroundjoin%
\definecolor{currentfill}{rgb}{0.289996,0.386836,0.828926}%
\pgfsetfillcolor{currentfill}%
\pgfsetlinewidth{0.000000pt}%
\definecolor{currentstroke}{rgb}{0.000000,0.000000,0.000000}%
\pgfsetstrokecolor{currentstroke}%
\pgfsetdash{}{0pt}%
\pgfpathmoveto{\pgfqpoint{5.184958in}{2.597253in}}%
\pgfpathlineto{\pgfqpoint{5.194715in}{2.496084in}}%
\pgfpathlineto{\pgfqpoint{5.205461in}{2.490645in}}%
\pgfpathlineto{\pgfqpoint{5.240798in}{2.582330in}}%
\pgfpathlineto{\pgfqpoint{5.272900in}{2.376477in}}%
\pgfpathlineto{\pgfqpoint{5.263319in}{2.493660in}}%
\pgfpathlineto{\pgfqpoint{5.251593in}{2.412282in}}%
\pgfpathlineto{\pgfqpoint{5.219115in}{2.581069in}}%
\pgfpathlineto{\pgfqpoint{5.184958in}{2.597253in}}%
\pgfpathclose%
\pgfusepath{fill}%
\end{pgfscope}%
\begin{pgfscope}%
\pgfpathrectangle{\pgfqpoint{1.020000in}{0.880000in}}{\pgfqpoint{6.160000in}{6.160000in}}%
\pgfusepath{clip}%
\pgfsetbuttcap%
\pgfsetroundjoin%
\definecolor{currentfill}{rgb}{0.280550,0.373423,0.818011}%
\pgfsetfillcolor{currentfill}%
\pgfsetlinewidth{0.000000pt}%
\definecolor{currentstroke}{rgb}{0.000000,0.000000,0.000000}%
\pgfsetstrokecolor{currentstroke}%
\pgfsetdash{}{0pt}%
\pgfpathmoveto{\pgfqpoint{5.411936in}{2.542335in}}%
\pgfpathlineto{\pgfqpoint{5.421281in}{2.406744in}}%
\pgfpathlineto{\pgfqpoint{5.432998in}{2.459576in}}%
\pgfpathlineto{\pgfqpoint{5.468554in}{2.557244in}}%
\pgfpathlineto{\pgfqpoint{5.502203in}{2.508154in}}%
\pgfpathlineto{\pgfqpoint{5.490851in}{2.491398in}}%
\pgfpathlineto{\pgfqpoint{5.479064in}{2.439308in}}%
\pgfpathlineto{\pgfqpoint{5.445307in}{2.473427in}}%
\pgfpathlineto{\pgfqpoint{5.411936in}{2.542335in}}%
\pgfpathclose%
\pgfusepath{fill}%
\end{pgfscope}%
\begin{pgfscope}%
\pgfpathrectangle{\pgfqpoint{1.020000in}{0.880000in}}{\pgfqpoint{6.160000in}{6.160000in}}%
\pgfusepath{clip}%
\pgfsetbuttcap%
\pgfsetroundjoin%
\definecolor{currentfill}{rgb}{0.368507,0.491141,0.905243}%
\pgfsetfillcolor{currentfill}%
\pgfsetlinewidth{0.000000pt}%
\definecolor{currentstroke}{rgb}{0.000000,0.000000,0.000000}%
\pgfsetstrokecolor{currentstroke}%
\pgfsetdash{}{0pt}%
\pgfpathmoveto{\pgfqpoint{4.800029in}{2.763082in}}%
\pgfpathlineto{\pgfqpoint{4.808812in}{2.512859in}}%
\pgfpathlineto{\pgfqpoint{4.820672in}{2.746502in}}%
\pgfpathlineto{\pgfqpoint{4.853817in}{2.543127in}}%
\pgfpathlineto{\pgfqpoint{4.888280in}{2.547731in}}%
\pgfpathlineto{\pgfqpoint{4.879996in}{2.854630in}}%
\pgfpathlineto{\pgfqpoint{4.867260in}{2.522374in}}%
\pgfpathlineto{\pgfqpoint{4.834670in}{2.791158in}}%
\pgfpathlineto{\pgfqpoint{4.800029in}{2.763082in}}%
\pgfpathclose%
\pgfusepath{fill}%
\end{pgfscope}%
\begin{pgfscope}%
\pgfpathrectangle{\pgfqpoint{1.020000in}{0.880000in}}{\pgfqpoint{6.160000in}{6.160000in}}%
\pgfusepath{clip}%
\pgfsetbuttcap%
\pgfsetroundjoin%
\definecolor{currentfill}{rgb}{0.839351,0.861167,0.894494}%
\pgfsetfillcolor{currentfill}%
\pgfsetlinewidth{0.000000pt}%
\definecolor{currentstroke}{rgb}{0.000000,0.000000,0.000000}%
\pgfsetstrokecolor{currentstroke}%
\pgfsetdash{}{0pt}%
\pgfpathmoveto{\pgfqpoint{3.971625in}{3.641696in}}%
\pgfpathlineto{\pgfqpoint{3.981330in}{3.510036in}}%
\pgfpathlineto{\pgfqpoint{3.990302in}{3.757288in}}%
\pgfpathlineto{\pgfqpoint{4.025527in}{3.491570in}}%
\pgfpathlineto{\pgfqpoint{4.060051in}{3.629024in}}%
\pgfpathlineto{\pgfqpoint{4.050584in}{3.581839in}}%
\pgfpathlineto{\pgfqpoint{4.041130in}{3.546807in}}%
\pgfpathlineto{\pgfqpoint{4.006466in}{3.559437in}}%
\pgfpathlineto{\pgfqpoint{3.971625in}{3.641696in}}%
\pgfpathclose%
\pgfusepath{fill}%
\end{pgfscope}%
\begin{pgfscope}%
\pgfpathrectangle{\pgfqpoint{1.020000in}{0.880000in}}{\pgfqpoint{6.160000in}{6.160000in}}%
\pgfusepath{clip}%
\pgfsetbuttcap%
\pgfsetroundjoin%
\definecolor{currentfill}{rgb}{0.879622,0.858175,0.845844}%
\pgfsetfillcolor{currentfill}%
\pgfsetlinewidth{0.000000pt}%
\definecolor{currentstroke}{rgb}{0.000000,0.000000,0.000000}%
\pgfsetstrokecolor{currentstroke}%
\pgfsetdash{}{0pt}%
\pgfpathmoveto{\pgfqpoint{3.814118in}{3.635353in}}%
\pgfpathlineto{\pgfqpoint{3.823101in}{3.711575in}}%
\pgfpathlineto{\pgfqpoint{3.832519in}{3.672995in}}%
\pgfpathlineto{\pgfqpoint{3.867520in}{3.606694in}}%
\pgfpathlineto{\pgfqpoint{3.901696in}{3.810614in}}%
\pgfpathlineto{\pgfqpoint{3.892325in}{3.812918in}}%
\pgfpathlineto{\pgfqpoint{3.883518in}{3.633610in}}%
\pgfpathlineto{\pgfqpoint{3.848810in}{3.640315in}}%
\pgfpathlineto{\pgfqpoint{3.814118in}{3.635353in}}%
\pgfpathclose%
\pgfusepath{fill}%
\end{pgfscope}%
\begin{pgfscope}%
\pgfpathrectangle{\pgfqpoint{1.020000in}{0.880000in}}{\pgfqpoint{6.160000in}{6.160000in}}%
\pgfusepath{clip}%
\pgfsetbuttcap%
\pgfsetroundjoin%
\definecolor{currentfill}{rgb}{0.758539,0.832787,0.958408}%
\pgfsetfillcolor{currentfill}%
\pgfsetlinewidth{0.000000pt}%
\definecolor{currentstroke}{rgb}{0.000000,0.000000,0.000000}%
\pgfsetstrokecolor{currentstroke}%
\pgfsetdash{}{0pt}%
\pgfpathmoveto{\pgfqpoint{4.287549in}{3.586782in}}%
\pgfpathlineto{\pgfqpoint{4.297199in}{3.454203in}}%
\pgfpathlineto{\pgfqpoint{4.306852in}{3.326881in}}%
\pgfpathlineto{\pgfqpoint{4.341464in}{3.317254in}}%
\pgfpathlineto{\pgfqpoint{4.376034in}{3.300374in}}%
\pgfpathlineto{\pgfqpoint{4.366326in}{3.385960in}}%
\pgfpathlineto{\pgfqpoint{4.356608in}{3.463140in}}%
\pgfpathlineto{\pgfqpoint{4.321954in}{3.411798in}}%
\pgfpathlineto{\pgfqpoint{4.287549in}{3.586782in}}%
\pgfpathclose%
\pgfusepath{fill}%
\end{pgfscope}%
\begin{pgfscope}%
\pgfpathrectangle{\pgfqpoint{1.020000in}{0.880000in}}{\pgfqpoint{6.160000in}{6.160000in}}%
\pgfusepath{clip}%
\pgfsetbuttcap%
\pgfsetroundjoin%
\definecolor{currentfill}{rgb}{0.548876,0.685104,0.994379}%
\pgfsetfillcolor{currentfill}%
\pgfsetlinewidth{0.000000pt}%
\definecolor{currentstroke}{rgb}{0.000000,0.000000,0.000000}%
\pgfsetstrokecolor{currentstroke}%
\pgfsetdash{}{0pt}%
\pgfpathmoveto{\pgfqpoint{2.755842in}{2.990349in}}%
\pgfpathlineto{\pgfqpoint{2.763035in}{3.053008in}}%
\pgfpathlineto{\pgfqpoint{2.774955in}{2.784123in}}%
\pgfpathlineto{\pgfqpoint{2.805829in}{3.086666in}}%
\pgfpathlineto{\pgfqpoint{2.843195in}{2.921368in}}%
\pgfpathlineto{\pgfqpoint{2.833284in}{3.049742in}}%
\pgfpathlineto{\pgfqpoint{2.824871in}{3.068026in}}%
\pgfpathlineto{\pgfqpoint{2.790155in}{3.043374in}}%
\pgfpathlineto{\pgfqpoint{2.755842in}{2.990349in}}%
\pgfpathclose%
\pgfusepath{fill}%
\end{pgfscope}%
\begin{pgfscope}%
\pgfpathrectangle{\pgfqpoint{1.020000in}{0.880000in}}{\pgfqpoint{6.160000in}{6.160000in}}%
\pgfusepath{clip}%
\pgfsetbuttcap%
\pgfsetroundjoin%
\definecolor{currentfill}{rgb}{0.328604,0.439712,0.869587}%
\pgfsetfillcolor{currentfill}%
\pgfsetlinewidth{0.000000pt}%
\definecolor{currentstroke}{rgb}{0.000000,0.000000,0.000000}%
\pgfsetstrokecolor{currentstroke}%
\pgfsetdash{}{0pt}%
\pgfpathmoveto{\pgfqpoint{4.958149in}{2.684997in}}%
\pgfpathlineto{\pgfqpoint{4.967329in}{2.512911in}}%
\pgfpathlineto{\pgfqpoint{4.977843in}{2.508734in}}%
\pgfpathlineto{\pgfqpoint{5.013388in}{2.642827in}}%
\pgfpathlineto{\pgfqpoint{5.047362in}{2.588359in}}%
\pgfpathlineto{\pgfqpoint{5.036326in}{2.540521in}}%
\pgfpathlineto{\pgfqpoint{5.026267in}{2.603727in}}%
\pgfpathlineto{\pgfqpoint{4.991655in}{2.573529in}}%
\pgfpathlineto{\pgfqpoint{4.958149in}{2.684997in}}%
\pgfpathclose%
\pgfusepath{fill}%
\end{pgfscope}%
\begin{pgfscope}%
\pgfpathrectangle{\pgfqpoint{1.020000in}{0.880000in}}{\pgfqpoint{6.160000in}{6.160000in}}%
\pgfusepath{clip}%
\pgfsetbuttcap%
\pgfsetroundjoin%
\definecolor{currentfill}{rgb}{0.521696,0.659599,0.987736}%
\pgfsetfillcolor{currentfill}%
\pgfsetlinewidth{0.000000pt}%
\definecolor{currentstroke}{rgb}{0.000000,0.000000,0.000000}%
\pgfsetstrokecolor{currentstroke}%
\pgfsetdash{}{0pt}%
\pgfpathmoveto{\pgfqpoint{4.553068in}{2.927929in}}%
\pgfpathlineto{\pgfqpoint{4.563645in}{3.058435in}}%
\pgfpathlineto{\pgfqpoint{4.573423in}{2.972801in}}%
\pgfpathlineto{\pgfqpoint{4.607426in}{2.835657in}}%
\pgfpathlineto{\pgfqpoint{4.642734in}{3.006721in}}%
\pgfpathlineto{\pgfqpoint{4.631989in}{2.881329in}}%
\pgfpathlineto{\pgfqpoint{4.622370in}{3.000512in}}%
\pgfpathlineto{\pgfqpoint{4.587636in}{2.944181in}}%
\pgfpathlineto{\pgfqpoint{4.553068in}{2.927929in}}%
\pgfpathclose%
\pgfusepath{fill}%
\end{pgfscope}%
\begin{pgfscope}%
\pgfpathrectangle{\pgfqpoint{1.020000in}{0.880000in}}{\pgfqpoint{6.160000in}{6.160000in}}%
\pgfusepath{clip}%
\pgfsetbuttcap%
\pgfsetroundjoin%
\definecolor{currentfill}{rgb}{0.608547,0.735725,0.999354}%
\pgfsetfillcolor{currentfill}%
\pgfsetlinewidth{0.000000pt}%
\definecolor{currentstroke}{rgb}{0.000000,0.000000,0.000000}%
\pgfsetstrokecolor{currentstroke}%
\pgfsetdash{}{0pt}%
\pgfpathmoveto{\pgfqpoint{4.464481in}{3.080740in}}%
\pgfpathlineto{\pgfqpoint{4.475123in}{3.307574in}}%
\pgfpathlineto{\pgfqpoint{4.484530in}{3.103195in}}%
\pgfpathlineto{\pgfqpoint{4.519301in}{3.152392in}}%
\pgfpathlineto{\pgfqpoint{4.553068in}{2.927929in}}%
\pgfpathlineto{\pgfqpoint{4.543748in}{3.131515in}}%
\pgfpathlineto{\pgfqpoint{4.533322in}{3.026088in}}%
\pgfpathlineto{\pgfqpoint{4.499295in}{3.171138in}}%
\pgfpathlineto{\pgfqpoint{4.464481in}{3.080740in}}%
\pgfpathclose%
\pgfusepath{fill}%
\end{pgfscope}%
\begin{pgfscope}%
\pgfpathrectangle{\pgfqpoint{1.020000in}{0.880000in}}{\pgfqpoint{6.160000in}{6.160000in}}%
\pgfusepath{clip}%
\pgfsetbuttcap%
\pgfsetroundjoin%
\definecolor{currentfill}{rgb}{0.473070,0.611077,0.970634}%
\pgfsetfillcolor{currentfill}%
\pgfsetlinewidth{0.000000pt}%
\definecolor{currentstroke}{rgb}{0.000000,0.000000,0.000000}%
\pgfsetstrokecolor{currentstroke}%
\pgfsetdash{}{0pt}%
\pgfpathmoveto{\pgfqpoint{4.711758in}{2.993807in}}%
\pgfpathlineto{\pgfqpoint{4.721408in}{2.879081in}}%
\pgfpathlineto{\pgfqpoint{4.731278in}{2.804245in}}%
\pgfpathlineto{\pgfqpoint{4.765890in}{2.821038in}}%
\pgfpathlineto{\pgfqpoint{4.800029in}{2.763082in}}%
\pgfpathlineto{\pgfqpoint{4.790819in}{2.948184in}}%
\pgfpathlineto{\pgfqpoint{4.780394in}{2.933469in}}%
\pgfpathlineto{\pgfqpoint{4.744739in}{2.723418in}}%
\pgfpathlineto{\pgfqpoint{4.711758in}{2.993807in}}%
\pgfpathclose%
\pgfusepath{fill}%
\end{pgfscope}%
\begin{pgfscope}%
\pgfpathrectangle{\pgfqpoint{1.020000in}{0.880000in}}{\pgfqpoint{6.160000in}{6.160000in}}%
\pgfusepath{clip}%
\pgfsetbuttcap%
\pgfsetroundjoin%
\definecolor{currentfill}{rgb}{0.532568,0.669801,0.990393}%
\pgfsetfillcolor{currentfill}%
\pgfsetlinewidth{0.000000pt}%
\definecolor{currentstroke}{rgb}{0.000000,0.000000,0.000000}%
\pgfsetstrokecolor{currentstroke}%
\pgfsetdash{}{0pt}%
\pgfpathmoveto{\pgfqpoint{2.684773in}{3.049792in}}%
\pgfpathlineto{\pgfqpoint{2.696020in}{2.833990in}}%
\pgfpathlineto{\pgfqpoint{2.700210in}{3.091225in}}%
\pgfpathlineto{\pgfqpoint{2.736780in}{2.997419in}}%
\pgfpathlineto{\pgfqpoint{2.774955in}{2.784123in}}%
\pgfpathlineto{\pgfqpoint{2.763035in}{3.053008in}}%
\pgfpathlineto{\pgfqpoint{2.755842in}{2.990349in}}%
\pgfpathlineto{\pgfqpoint{2.721217in}{2.959603in}}%
\pgfpathlineto{\pgfqpoint{2.684773in}{3.049792in}}%
\pgfpathclose%
\pgfusepath{fill}%
\end{pgfscope}%
\begin{pgfscope}%
\pgfpathrectangle{\pgfqpoint{1.020000in}{0.880000in}}{\pgfqpoint{6.160000in}{6.160000in}}%
\pgfusepath{clip}%
\pgfsetbuttcap%
\pgfsetroundjoin%
\definecolor{currentfill}{rgb}{0.822420,0.856898,0.910795}%
\pgfsetfillcolor{currentfill}%
\pgfsetlinewidth{0.000000pt}%
\definecolor{currentstroke}{rgb}{0.000000,0.000000,0.000000}%
\pgfsetstrokecolor{currentstroke}%
\pgfsetdash{}{0pt}%
\pgfpathmoveto{\pgfqpoint{4.129456in}{3.651980in}}%
\pgfpathlineto{\pgfqpoint{4.139118in}{3.566234in}}%
\pgfpathlineto{\pgfqpoint{4.148796in}{3.445268in}}%
\pgfpathlineto{\pgfqpoint{4.183480in}{3.620099in}}%
\pgfpathlineto{\pgfqpoint{4.218194in}{3.595114in}}%
\pgfpathlineto{\pgfqpoint{4.208438in}{3.413923in}}%
\pgfpathlineto{\pgfqpoint{4.198791in}{3.566897in}}%
\pgfpathlineto{\pgfqpoint{4.164176in}{3.469071in}}%
\pgfpathlineto{\pgfqpoint{4.129456in}{3.651980in}}%
\pgfpathclose%
\pgfusepath{fill}%
\end{pgfscope}%
\begin{pgfscope}%
\pgfpathrectangle{\pgfqpoint{1.020000in}{0.880000in}}{\pgfqpoint{6.160000in}{6.160000in}}%
\pgfusepath{clip}%
\pgfsetbuttcap%
\pgfsetroundjoin%
\definecolor{currentfill}{rgb}{0.875557,0.860242,0.851430}%
\pgfsetfillcolor{currentfill}%
\pgfsetlinewidth{0.000000pt}%
\definecolor{currentstroke}{rgb}{0.000000,0.000000,0.000000}%
\pgfsetstrokecolor{currentstroke}%
\pgfsetdash{}{0pt}%
\pgfpathmoveto{\pgfqpoint{3.744124in}{3.752891in}}%
\pgfpathlineto{\pgfqpoint{3.753588in}{3.690368in}}%
\pgfpathlineto{\pgfqpoint{3.763002in}{3.641771in}}%
\pgfpathlineto{\pgfqpoint{3.797871in}{3.630214in}}%
\pgfpathlineto{\pgfqpoint{3.832519in}{3.672995in}}%
\pgfpathlineto{\pgfqpoint{3.823101in}{3.711575in}}%
\pgfpathlineto{\pgfqpoint{3.814118in}{3.635353in}}%
\pgfpathlineto{\pgfqpoint{3.779268in}{3.665196in}}%
\pgfpathlineto{\pgfqpoint{3.744124in}{3.752891in}}%
\pgfpathclose%
\pgfusepath{fill}%
\end{pgfscope}%
\begin{pgfscope}%
\pgfpathrectangle{\pgfqpoint{1.020000in}{0.880000in}}{\pgfqpoint{6.160000in}{6.160000in}}%
\pgfusepath{clip}%
\pgfsetbuttcap%
\pgfsetroundjoin%
\definecolor{currentfill}{rgb}{0.266381,0.353304,0.801637}%
\pgfsetfillcolor{currentfill}%
\pgfsetlinewidth{0.000000pt}%
\definecolor{currentstroke}{rgb}{0.000000,0.000000,0.000000}%
\pgfsetstrokecolor{currentstroke}%
\pgfsetdash{}{0pt}%
\pgfpathmoveto{\pgfqpoint{5.502203in}{2.508154in}}%
\pgfpathlineto{\pgfqpoint{5.513569in}{2.523799in}}%
\pgfpathlineto{\pgfqpoint{5.546856in}{2.449950in}}%
\pgfpathlineto{\pgfqpoint{5.579461in}{2.332901in}}%
\pgfpathlineto{\pgfqpoint{5.569370in}{2.412504in}}%
\pgfpathlineto{\pgfqpoint{5.535554in}{2.441322in}}%
\pgfpathlineto{\pgfqpoint{5.502203in}{2.508154in}}%
\pgfpathclose%
\pgfusepath{fill}%
\end{pgfscope}%
\begin{pgfscope}%
\pgfpathrectangle{\pgfqpoint{1.020000in}{0.880000in}}{\pgfqpoint{6.160000in}{6.160000in}}%
\pgfusepath{clip}%
\pgfsetbuttcap%
\pgfsetroundjoin%
\definecolor{currentfill}{rgb}{0.289996,0.386836,0.828926}%
\pgfsetfillcolor{currentfill}%
\pgfsetlinewidth{0.000000pt}%
\definecolor{currentstroke}{rgb}{0.000000,0.000000,0.000000}%
\pgfsetstrokecolor{currentstroke}%
\pgfsetdash{}{0pt}%
\pgfpathmoveto{\pgfqpoint{5.343738in}{2.573441in}}%
\pgfpathlineto{\pgfqpoint{5.352975in}{2.426350in}}%
\pgfpathlineto{\pgfqpoint{5.364389in}{2.462513in}}%
\pgfpathlineto{\pgfqpoint{5.399883in}{2.557069in}}%
\pgfpathlineto{\pgfqpoint{5.432998in}{2.459576in}}%
\pgfpathlineto{\pgfqpoint{5.421281in}{2.406744in}}%
\pgfpathlineto{\pgfqpoint{5.411936in}{2.542335in}}%
\pgfpathlineto{\pgfqpoint{5.377692in}{2.544550in}}%
\pgfpathlineto{\pgfqpoint{5.343738in}{2.573441in}}%
\pgfpathclose%
\pgfusepath{fill}%
\end{pgfscope}%
\begin{pgfscope}%
\pgfpathrectangle{\pgfqpoint{1.020000in}{0.880000in}}{\pgfqpoint{6.160000in}{6.160000in}}%
\pgfusepath{clip}%
\pgfsetbuttcap%
\pgfsetroundjoin%
\definecolor{currentfill}{rgb}{0.457046,0.594006,0.963029}%
\pgfsetfillcolor{currentfill}%
\pgfsetlinewidth{0.000000pt}%
\definecolor{currentstroke}{rgb}{0.000000,0.000000,0.000000}%
\pgfsetstrokecolor{currentstroke}%
\pgfsetdash{}{0pt}%
\pgfpathmoveto{\pgfqpoint{4.642734in}{3.006721in}}%
\pgfpathlineto{\pgfqpoint{4.651399in}{2.680866in}}%
\pgfpathlineto{\pgfqpoint{4.661688in}{2.701120in}}%
\pgfpathlineto{\pgfqpoint{4.696708in}{2.797703in}}%
\pgfpathlineto{\pgfqpoint{4.731278in}{2.804245in}}%
\pgfpathlineto{\pgfqpoint{4.721408in}{2.879081in}}%
\pgfpathlineto{\pgfqpoint{4.711758in}{2.993807in}}%
\pgfpathlineto{\pgfqpoint{4.676030in}{2.752634in}}%
\pgfpathlineto{\pgfqpoint{4.642734in}{3.006721in}}%
\pgfpathclose%
\pgfusepath{fill}%
\end{pgfscope}%
\begin{pgfscope}%
\pgfpathrectangle{\pgfqpoint{1.020000in}{0.880000in}}{\pgfqpoint{6.160000in}{6.160000in}}%
\pgfusepath{clip}%
\pgfsetbuttcap%
\pgfsetroundjoin%
\definecolor{currentfill}{rgb}{0.527132,0.664700,0.989065}%
\pgfsetfillcolor{currentfill}%
\pgfsetlinewidth{0.000000pt}%
\definecolor{currentstroke}{rgb}{0.000000,0.000000,0.000000}%
\pgfsetstrokecolor{currentstroke}%
\pgfsetdash{}{0pt}%
\pgfpathmoveto{\pgfqpoint{2.914049in}{2.862125in}}%
\pgfpathlineto{\pgfqpoint{2.920366in}{3.015132in}}%
\pgfpathlineto{\pgfqpoint{2.928789in}{3.004506in}}%
\pgfpathlineto{\pgfqpoint{2.966100in}{2.825394in}}%
\pgfpathlineto{\pgfqpoint{2.999184in}{2.989347in}}%
\pgfpathlineto{\pgfqpoint{2.989985in}{3.059539in}}%
\pgfpathlineto{\pgfqpoint{2.982372in}{2.998720in}}%
\pgfpathlineto{\pgfqpoint{2.948515in}{2.904843in}}%
\pgfpathlineto{\pgfqpoint{2.914049in}{2.862125in}}%
\pgfpathclose%
\pgfusepath{fill}%
\end{pgfscope}%
\begin{pgfscope}%
\pgfpathrectangle{\pgfqpoint{1.020000in}{0.880000in}}{\pgfqpoint{6.160000in}{6.160000in}}%
\pgfusepath{clip}%
\pgfsetbuttcap%
\pgfsetroundjoin%
\definecolor{currentfill}{rgb}{0.683056,0.790043,0.989768}%
\pgfsetfillcolor{currentfill}%
\pgfsetlinewidth{0.000000pt}%
\definecolor{currentstroke}{rgb}{0.000000,0.000000,0.000000}%
\pgfsetstrokecolor{currentstroke}%
\pgfsetdash{}{0pt}%
\pgfpathmoveto{\pgfqpoint{4.306852in}{3.326881in}}%
\pgfpathlineto{\pgfqpoint{4.316551in}{3.237130in}}%
\pgfpathlineto{\pgfqpoint{4.326358in}{3.219801in}}%
\pgfpathlineto{\pgfqpoint{4.360859in}{3.129865in}}%
\pgfpathlineto{\pgfqpoint{4.395747in}{3.261160in}}%
\pgfpathlineto{\pgfqpoint{4.385764in}{3.222776in}}%
\pgfpathlineto{\pgfqpoint{4.376034in}{3.300374in}}%
\pgfpathlineto{\pgfqpoint{4.341464in}{3.317254in}}%
\pgfpathlineto{\pgfqpoint{4.306852in}{3.326881in}}%
\pgfpathclose%
\pgfusepath{fill}%
\end{pgfscope}%
\begin{pgfscope}%
\pgfpathrectangle{\pgfqpoint{1.020000in}{0.880000in}}{\pgfqpoint{6.160000in}{6.160000in}}%
\pgfusepath{clip}%
\pgfsetbuttcap%
\pgfsetroundjoin%
\definecolor{currentfill}{rgb}{0.333490,0.446265,0.874452}%
\pgfsetfillcolor{currentfill}%
\pgfsetlinewidth{0.000000pt}%
\definecolor{currentstroke}{rgb}{0.000000,0.000000,0.000000}%
\pgfsetstrokecolor{currentstroke}%
\pgfsetdash{}{0pt}%
\pgfpathmoveto{\pgfqpoint{4.888280in}{2.547731in}}%
\pgfpathlineto{\pgfqpoint{4.899257in}{2.619757in}}%
\pgfpathlineto{\pgfqpoint{4.910138in}{2.673502in}}%
\pgfpathlineto{\pgfqpoint{4.943899in}{2.574033in}}%
\pgfpathlineto{\pgfqpoint{4.977843in}{2.508734in}}%
\pgfpathlineto{\pgfqpoint{4.967329in}{2.512911in}}%
\pgfpathlineto{\pgfqpoint{4.958149in}{2.684997in}}%
\pgfpathlineto{\pgfqpoint{4.922573in}{2.532074in}}%
\pgfpathlineto{\pgfqpoint{4.888280in}{2.547731in}}%
\pgfpathclose%
\pgfusepath{fill}%
\end{pgfscope}%
\begin{pgfscope}%
\pgfpathrectangle{\pgfqpoint{1.020000in}{0.880000in}}{\pgfqpoint{6.160000in}{6.160000in}}%
\pgfusepath{clip}%
\pgfsetbuttcap%
\pgfsetroundjoin%
\definecolor{currentfill}{rgb}{0.318832,0.426605,0.859857}%
\pgfsetfillcolor{currentfill}%
\pgfsetlinewidth{0.000000pt}%
\definecolor{currentstroke}{rgb}{0.000000,0.000000,0.000000}%
\pgfsetstrokecolor{currentstroke}%
\pgfsetdash{}{0pt}%
\pgfpathmoveto{\pgfqpoint{5.114770in}{2.440798in}}%
\pgfpathlineto{\pgfqpoint{5.126109in}{2.507655in}}%
\pgfpathlineto{\pgfqpoint{5.139284in}{2.761278in}}%
\pgfpathlineto{\pgfqpoint{5.172607in}{2.643736in}}%
\pgfpathlineto{\pgfqpoint{5.205461in}{2.490645in}}%
\pgfpathlineto{\pgfqpoint{5.194715in}{2.496084in}}%
\pgfpathlineto{\pgfqpoint{5.184958in}{2.597253in}}%
\pgfpathlineto{\pgfqpoint{5.149750in}{2.509037in}}%
\pgfpathlineto{\pgfqpoint{5.114770in}{2.440798in}}%
\pgfpathclose%
\pgfusepath{fill}%
\end{pgfscope}%
\begin{pgfscope}%
\pgfpathrectangle{\pgfqpoint{1.020000in}{0.880000in}}{\pgfqpoint{6.160000in}{6.160000in}}%
\pgfusepath{clip}%
\pgfsetbuttcap%
\pgfsetroundjoin%
\definecolor{currentfill}{rgb}{0.875557,0.860242,0.851430}%
\pgfsetfillcolor{currentfill}%
\pgfsetlinewidth{0.000000pt}%
\definecolor{currentstroke}{rgb}{0.000000,0.000000,0.000000}%
\pgfsetstrokecolor{currentstroke}%
\pgfsetdash{}{0pt}%
\pgfpathmoveto{\pgfqpoint{3.675331in}{3.584412in}}%
\pgfpathlineto{\pgfqpoint{3.684349in}{3.602857in}}%
\pgfpathlineto{\pgfqpoint{3.692540in}{3.795249in}}%
\pgfpathlineto{\pgfqpoint{3.728580in}{3.548789in}}%
\pgfpathlineto{\pgfqpoint{3.763002in}{3.641771in}}%
\pgfpathlineto{\pgfqpoint{3.753588in}{3.690368in}}%
\pgfpathlineto{\pgfqpoint{3.744124in}{3.752891in}}%
\pgfpathlineto{\pgfqpoint{3.709526in}{3.707339in}}%
\pgfpathlineto{\pgfqpoint{3.675331in}{3.584412in}}%
\pgfpathclose%
\pgfusepath{fill}%
\end{pgfscope}%
\begin{pgfscope}%
\pgfpathrectangle{\pgfqpoint{1.020000in}{0.880000in}}{\pgfqpoint{6.160000in}{6.160000in}}%
\pgfusepath{clip}%
\pgfsetbuttcap%
\pgfsetroundjoin%
\definecolor{currentfill}{rgb}{0.646113,0.764436,0.996868}%
\pgfsetfillcolor{currentfill}%
\pgfsetlinewidth{0.000000pt}%
\definecolor{currentstroke}{rgb}{0.000000,0.000000,0.000000}%
\pgfsetstrokecolor{currentstroke}%
\pgfsetdash{}{0pt}%
\pgfpathmoveto{\pgfqpoint{3.119944in}{3.214607in}}%
\pgfpathlineto{\pgfqpoint{3.128152in}{3.242674in}}%
\pgfpathlineto{\pgfqpoint{3.139148in}{3.006978in}}%
\pgfpathlineto{\pgfqpoint{3.173369in}{3.082822in}}%
\pgfpathlineto{\pgfqpoint{3.207132in}{3.208694in}}%
\pgfpathlineto{\pgfqpoint{3.198479in}{3.214603in}}%
\pgfpathlineto{\pgfqpoint{3.190024in}{3.202456in}}%
\pgfpathlineto{\pgfqpoint{3.154898in}{3.218135in}}%
\pgfpathlineto{\pgfqpoint{3.119944in}{3.214607in}}%
\pgfpathclose%
\pgfusepath{fill}%
\end{pgfscope}%
\begin{pgfscope}%
\pgfpathrectangle{\pgfqpoint{1.020000in}{0.880000in}}{\pgfqpoint{6.160000in}{6.160000in}}%
\pgfusepath{clip}%
\pgfsetbuttcap%
\pgfsetroundjoin%
\definecolor{currentfill}{rgb}{0.800601,0.850358,0.930008}%
\pgfsetfillcolor{currentfill}%
\pgfsetlinewidth{0.000000pt}%
\definecolor{currentstroke}{rgb}{0.000000,0.000000,0.000000}%
\pgfsetstrokecolor{currentstroke}%
\pgfsetdash{}{0pt}%
\pgfpathmoveto{\pgfqpoint{3.466123in}{3.599185in}}%
\pgfpathlineto{\pgfqpoint{3.476150in}{3.439342in}}%
\pgfpathlineto{\pgfqpoint{3.485419in}{3.386848in}}%
\pgfpathlineto{\pgfqpoint{3.520241in}{3.399616in}}%
\pgfpathlineto{\pgfqpoint{3.555183in}{3.391249in}}%
\pgfpathlineto{\pgfqpoint{3.545006in}{3.580723in}}%
\pgfpathlineto{\pgfqpoint{3.536393in}{3.525892in}}%
\pgfpathlineto{\pgfqpoint{3.501362in}{3.550731in}}%
\pgfpathlineto{\pgfqpoint{3.466123in}{3.599185in}}%
\pgfpathclose%
\pgfusepath{fill}%
\end{pgfscope}%
\begin{pgfscope}%
\pgfpathrectangle{\pgfqpoint{1.020000in}{0.880000in}}{\pgfqpoint{6.160000in}{6.160000in}}%
\pgfusepath{clip}%
\pgfsetbuttcap%
\pgfsetroundjoin%
\definecolor{currentfill}{rgb}{0.677823,0.786546,0.991005}%
\pgfsetfillcolor{currentfill}%
\pgfsetlinewidth{0.000000pt}%
\definecolor{currentstroke}{rgb}{0.000000,0.000000,0.000000}%
\pgfsetstrokecolor{currentstroke}%
\pgfsetdash{}{0pt}%
\pgfpathmoveto{\pgfqpoint{3.190024in}{3.202456in}}%
\pgfpathlineto{\pgfqpoint{3.198479in}{3.214603in}}%
\pgfpathlineto{\pgfqpoint{3.207132in}{3.208694in}}%
\pgfpathlineto{\pgfqpoint{3.241543in}{3.272980in}}%
\pgfpathlineto{\pgfqpoint{3.277200in}{3.202566in}}%
\pgfpathlineto{\pgfqpoint{3.268070in}{3.253142in}}%
\pgfpathlineto{\pgfqpoint{3.258214in}{3.383313in}}%
\pgfpathlineto{\pgfqpoint{3.225304in}{3.167284in}}%
\pgfpathlineto{\pgfqpoint{3.190024in}{3.202456in}}%
\pgfpathclose%
\pgfusepath{fill}%
\end{pgfscope}%
\begin{pgfscope}%
\pgfpathrectangle{\pgfqpoint{1.020000in}{0.880000in}}{\pgfqpoint{6.160000in}{6.160000in}}%
\pgfusepath{clip}%
\pgfsetbuttcap%
\pgfsetroundjoin%
\definecolor{currentfill}{rgb}{0.826784,0.858205,0.906953}%
\pgfsetfillcolor{currentfill}%
\pgfsetlinewidth{0.000000pt}%
\definecolor{currentstroke}{rgb}{0.000000,0.000000,0.000000}%
\pgfsetstrokecolor{currentstroke}%
\pgfsetdash{}{0pt}%
\pgfpathmoveto{\pgfqpoint{3.536393in}{3.525892in}}%
\pgfpathlineto{\pgfqpoint{3.545006in}{3.580723in}}%
\pgfpathlineto{\pgfqpoint{3.555183in}{3.391249in}}%
\pgfpathlineto{\pgfqpoint{3.588781in}{3.604861in}}%
\pgfpathlineto{\pgfqpoint{3.624085in}{3.535737in}}%
\pgfpathlineto{\pgfqpoint{3.614147in}{3.691472in}}%
\pgfpathlineto{\pgfqpoint{3.606032in}{3.529651in}}%
\pgfpathlineto{\pgfqpoint{3.571235in}{3.525722in}}%
\pgfpathlineto{\pgfqpoint{3.536393in}{3.525892in}}%
\pgfpathclose%
\pgfusepath{fill}%
\end{pgfscope}%
\begin{pgfscope}%
\pgfpathrectangle{\pgfqpoint{1.020000in}{0.880000in}}{\pgfqpoint{6.160000in}{6.160000in}}%
\pgfusepath{clip}%
\pgfsetbuttcap%
\pgfsetroundjoin%
\definecolor{currentfill}{rgb}{0.586921,0.718121,0.998874}%
\pgfsetfillcolor{currentfill}%
\pgfsetlinewidth{0.000000pt}%
\definecolor{currentstroke}{rgb}{0.000000,0.000000,0.000000}%
\pgfsetstrokecolor{currentstroke}%
\pgfsetdash{}{0pt}%
\pgfpathmoveto{\pgfqpoint{2.982372in}{2.998720in}}%
\pgfpathlineto{\pgfqpoint{2.989985in}{3.059539in}}%
\pgfpathlineto{\pgfqpoint{2.999184in}{2.989347in}}%
\pgfpathlineto{\pgfqpoint{3.032487in}{3.142418in}}%
\pgfpathlineto{\pgfqpoint{3.067804in}{3.123261in}}%
\pgfpathlineto{\pgfqpoint{3.061304in}{2.950889in}}%
\pgfpathlineto{\pgfqpoint{3.050751in}{3.139652in}}%
\pgfpathlineto{\pgfqpoint{3.016023in}{3.113955in}}%
\pgfpathlineto{\pgfqpoint{2.982372in}{2.998720in}}%
\pgfpathclose%
\pgfusepath{fill}%
\end{pgfscope}%
\begin{pgfscope}%
\pgfpathrectangle{\pgfqpoint{1.020000in}{0.880000in}}{\pgfqpoint{6.160000in}{6.160000in}}%
\pgfusepath{clip}%
\pgfsetbuttcap%
\pgfsetroundjoin%
\definecolor{currentfill}{rgb}{0.289996,0.386836,0.828926}%
\pgfsetfillcolor{currentfill}%
\pgfsetlinewidth{0.000000pt}%
\definecolor{currentstroke}{rgb}{0.000000,0.000000,0.000000}%
\pgfsetstrokecolor{currentstroke}%
\pgfsetdash{}{0pt}%
\pgfpathmoveto{\pgfqpoint{5.272900in}{2.376477in}}%
\pgfpathlineto{\pgfqpoint{5.284212in}{2.414983in}}%
\pgfpathlineto{\pgfqpoint{5.295947in}{2.488218in}}%
\pgfpathlineto{\pgfqpoint{5.331710in}{2.607197in}}%
\pgfpathlineto{\pgfqpoint{5.364389in}{2.462513in}}%
\pgfpathlineto{\pgfqpoint{5.352975in}{2.426350in}}%
\pgfpathlineto{\pgfqpoint{5.343738in}{2.573441in}}%
\pgfpathlineto{\pgfqpoint{5.309347in}{2.567539in}}%
\pgfpathlineto{\pgfqpoint{5.272900in}{2.376477in}}%
\pgfpathclose%
\pgfusepath{fill}%
\end{pgfscope}%
\begin{pgfscope}%
\pgfpathrectangle{\pgfqpoint{1.020000in}{0.880000in}}{\pgfqpoint{6.160000in}{6.160000in}}%
\pgfusepath{clip}%
\pgfsetbuttcap%
\pgfsetroundjoin%
\definecolor{currentfill}{rgb}{0.786721,0.844807,0.939810}%
\pgfsetfillcolor{currentfill}%
\pgfsetlinewidth{0.000000pt}%
\definecolor{currentstroke}{rgb}{0.000000,0.000000,0.000000}%
\pgfsetstrokecolor{currentstroke}%
\pgfsetdash{}{0pt}%
\pgfpathmoveto{\pgfqpoint{4.218194in}{3.595114in}}%
\pgfpathlineto{\pgfqpoint{4.227875in}{3.483137in}}%
\pgfpathlineto{\pgfqpoint{4.237616in}{3.479979in}}%
\pgfpathlineto{\pgfqpoint{4.272209in}{3.333895in}}%
\pgfpathlineto{\pgfqpoint{4.306852in}{3.326881in}}%
\pgfpathlineto{\pgfqpoint{4.297199in}{3.454203in}}%
\pgfpathlineto{\pgfqpoint{4.287549in}{3.586782in}}%
\pgfpathlineto{\pgfqpoint{4.252744in}{3.390116in}}%
\pgfpathlineto{\pgfqpoint{4.218194in}{3.595114in}}%
\pgfpathclose%
\pgfusepath{fill}%
\end{pgfscope}%
\begin{pgfscope}%
\pgfpathrectangle{\pgfqpoint{1.020000in}{0.880000in}}{\pgfqpoint{6.160000in}{6.160000in}}%
\pgfusepath{clip}%
\pgfsetbuttcap%
\pgfsetroundjoin%
\definecolor{currentfill}{rgb}{0.624703,0.748318,0.998719}%
\pgfsetfillcolor{currentfill}%
\pgfsetlinewidth{0.000000pt}%
\definecolor{currentstroke}{rgb}{0.000000,0.000000,0.000000}%
\pgfsetstrokecolor{currentstroke}%
\pgfsetdash{}{0pt}%
\pgfpathmoveto{\pgfqpoint{3.050751in}{3.139652in}}%
\pgfpathlineto{\pgfqpoint{3.061304in}{2.950889in}}%
\pgfpathlineto{\pgfqpoint{3.067804in}{3.123261in}}%
\pgfpathlineto{\pgfqpoint{3.102216in}{3.184288in}}%
\pgfpathlineto{\pgfqpoint{3.139148in}{3.006978in}}%
\pgfpathlineto{\pgfqpoint{3.128152in}{3.242674in}}%
\pgfpathlineto{\pgfqpoint{3.119944in}{3.214607in}}%
\pgfpathlineto{\pgfqpoint{3.085006in}{3.207981in}}%
\pgfpathlineto{\pgfqpoint{3.050751in}{3.139652in}}%
\pgfpathclose%
\pgfusepath{fill}%
\end{pgfscope}%
\begin{pgfscope}%
\pgfpathrectangle{\pgfqpoint{1.020000in}{0.880000in}}{\pgfqpoint{6.160000in}{6.160000in}}%
\pgfusepath{clip}%
\pgfsetbuttcap%
\pgfsetroundjoin%
\definecolor{currentfill}{rgb}{0.859385,0.864431,0.872111}%
\pgfsetfillcolor{currentfill}%
\pgfsetlinewidth{0.000000pt}%
\definecolor{currentstroke}{rgb}{0.000000,0.000000,0.000000}%
\pgfsetstrokecolor{currentstroke}%
\pgfsetdash{}{0pt}%
\pgfpathmoveto{\pgfqpoint{3.606032in}{3.529651in}}%
\pgfpathlineto{\pgfqpoint{3.614147in}{3.691472in}}%
\pgfpathlineto{\pgfqpoint{3.624085in}{3.535737in}}%
\pgfpathlineto{\pgfqpoint{3.658050in}{3.709077in}}%
\pgfpathlineto{\pgfqpoint{3.692540in}{3.795249in}}%
\pgfpathlineto{\pgfqpoint{3.684349in}{3.602857in}}%
\pgfpathlineto{\pgfqpoint{3.675331in}{3.584412in}}%
\pgfpathlineto{\pgfqpoint{3.640532in}{3.584790in}}%
\pgfpathlineto{\pgfqpoint{3.606032in}{3.529651in}}%
\pgfpathclose%
\pgfusepath{fill}%
\end{pgfscope}%
\begin{pgfscope}%
\pgfpathrectangle{\pgfqpoint{1.020000in}{0.880000in}}{\pgfqpoint{6.160000in}{6.160000in}}%
\pgfusepath{clip}%
\pgfsetbuttcap%
\pgfsetroundjoin%
\definecolor{currentfill}{rgb}{0.835345,0.860514,0.898970}%
\pgfsetfillcolor{currentfill}%
\pgfsetlinewidth{0.000000pt}%
\definecolor{currentstroke}{rgb}{0.000000,0.000000,0.000000}%
\pgfsetstrokecolor{currentstroke}%
\pgfsetdash{}{0pt}%
\pgfpathmoveto{\pgfqpoint{4.060051in}{3.629024in}}%
\pgfpathlineto{\pgfqpoint{4.069599in}{3.631639in}}%
\pgfpathlineto{\pgfqpoint{4.079333in}{3.476342in}}%
\pgfpathlineto{\pgfqpoint{4.114009in}{3.564051in}}%
\pgfpathlineto{\pgfqpoint{4.148796in}{3.445268in}}%
\pgfpathlineto{\pgfqpoint{4.139118in}{3.566234in}}%
\pgfpathlineto{\pgfqpoint{4.129456in}{3.651980in}}%
\pgfpathlineto{\pgfqpoint{4.094860in}{3.528161in}}%
\pgfpathlineto{\pgfqpoint{4.060051in}{3.629024in}}%
\pgfpathclose%
\pgfusepath{fill}%
\end{pgfscope}%
\begin{pgfscope}%
\pgfpathrectangle{\pgfqpoint{1.020000in}{0.880000in}}{\pgfqpoint{6.160000in}{6.160000in}}%
\pgfusepath{clip}%
\pgfsetbuttcap%
\pgfsetroundjoin%
\definecolor{currentfill}{rgb}{0.532568,0.669801,0.990393}%
\pgfsetfillcolor{currentfill}%
\pgfsetlinewidth{0.000000pt}%
\definecolor{currentstroke}{rgb}{0.000000,0.000000,0.000000}%
\pgfsetstrokecolor{currentstroke}%
\pgfsetdash{}{0pt}%
\pgfpathmoveto{\pgfqpoint{2.843195in}{2.921368in}}%
\pgfpathlineto{\pgfqpoint{2.850531in}{2.985096in}}%
\pgfpathlineto{\pgfqpoint{2.858684in}{2.989651in}}%
\pgfpathlineto{\pgfqpoint{2.894189in}{2.962742in}}%
\pgfpathlineto{\pgfqpoint{2.928789in}{3.004506in}}%
\pgfpathlineto{\pgfqpoint{2.920366in}{3.015132in}}%
\pgfpathlineto{\pgfqpoint{2.914049in}{2.862125in}}%
\pgfpathlineto{\pgfqpoint{2.877847in}{2.952457in}}%
\pgfpathlineto{\pgfqpoint{2.843195in}{2.921368in}}%
\pgfpathclose%
\pgfusepath{fill}%
\end{pgfscope}%
\begin{pgfscope}%
\pgfpathrectangle{\pgfqpoint{1.020000in}{0.880000in}}{\pgfqpoint{6.160000in}{6.160000in}}%
\pgfusepath{clip}%
\pgfsetbuttcap%
\pgfsetroundjoin%
\definecolor{currentfill}{rgb}{0.500031,0.638508,0.981070}%
\pgfsetfillcolor{currentfill}%
\pgfsetlinewidth{0.000000pt}%
\definecolor{currentstroke}{rgb}{0.000000,0.000000,0.000000}%
\pgfsetstrokecolor{currentstroke}%
\pgfsetdash{}{0pt}%
\pgfpathmoveto{\pgfqpoint{2.774955in}{2.784123in}}%
\pgfpathlineto{\pgfqpoint{2.783374in}{2.761902in}}%
\pgfpathlineto{\pgfqpoint{2.790548in}{2.829525in}}%
\pgfpathlineto{\pgfqpoint{2.825331in}{2.855961in}}%
\pgfpathlineto{\pgfqpoint{2.858684in}{2.989651in}}%
\pgfpathlineto{\pgfqpoint{2.850531in}{2.985096in}}%
\pgfpathlineto{\pgfqpoint{2.843195in}{2.921368in}}%
\pgfpathlineto{\pgfqpoint{2.805829in}{3.086666in}}%
\pgfpathlineto{\pgfqpoint{2.774955in}{2.784123in}}%
\pgfpathclose%
\pgfusepath{fill}%
\end{pgfscope}%
\begin{pgfscope}%
\pgfpathrectangle{\pgfqpoint{1.020000in}{0.880000in}}{\pgfqpoint{6.160000in}{6.160000in}}%
\pgfusepath{clip}%
\pgfsetbuttcap%
\pgfsetroundjoin%
\definecolor{currentfill}{rgb}{0.786721,0.844807,0.939810}%
\pgfsetfillcolor{currentfill}%
\pgfsetlinewidth{0.000000pt}%
\definecolor{currentstroke}{rgb}{0.000000,0.000000,0.000000}%
\pgfsetstrokecolor{currentstroke}%
\pgfsetdash{}{0pt}%
\pgfpathmoveto{\pgfqpoint{3.398462in}{3.316314in}}%
\pgfpathlineto{\pgfqpoint{3.405987in}{3.480594in}}%
\pgfpathlineto{\pgfqpoint{3.415158in}{3.436690in}}%
\pgfpathlineto{\pgfqpoint{3.449831in}{3.476876in}}%
\pgfpathlineto{\pgfqpoint{3.485419in}{3.386848in}}%
\pgfpathlineto{\pgfqpoint{3.476150in}{3.439342in}}%
\pgfpathlineto{\pgfqpoint{3.466123in}{3.599185in}}%
\pgfpathlineto{\pgfqpoint{3.431821in}{3.515455in}}%
\pgfpathlineto{\pgfqpoint{3.398462in}{3.316314in}}%
\pgfpathclose%
\pgfusepath{fill}%
\end{pgfscope}%
\begin{pgfscope}%
\pgfpathrectangle{\pgfqpoint{1.020000in}{0.880000in}}{\pgfqpoint{6.160000in}{6.160000in}}%
\pgfusepath{clip}%
\pgfsetbuttcap%
\pgfsetroundjoin%
\definecolor{currentfill}{rgb}{0.879622,0.858175,0.845844}%
\pgfsetfillcolor{currentfill}%
\pgfsetlinewidth{0.000000pt}%
\definecolor{currentstroke}{rgb}{0.000000,0.000000,0.000000}%
\pgfsetstrokecolor{currentstroke}%
\pgfsetdash{}{0pt}%
\pgfpathmoveto{\pgfqpoint{3.901696in}{3.810614in}}%
\pgfpathlineto{\pgfqpoint{3.911294in}{3.731668in}}%
\pgfpathlineto{\pgfqpoint{3.920833in}{3.677712in}}%
\pgfpathlineto{\pgfqpoint{3.955792in}{3.617400in}}%
\pgfpathlineto{\pgfqpoint{3.990302in}{3.757288in}}%
\pgfpathlineto{\pgfqpoint{3.981330in}{3.510036in}}%
\pgfpathlineto{\pgfqpoint{3.971625in}{3.641696in}}%
\pgfpathlineto{\pgfqpoint{3.936854in}{3.663458in}}%
\pgfpathlineto{\pgfqpoint{3.901696in}{3.810614in}}%
\pgfpathclose%
\pgfusepath{fill}%
\end{pgfscope}%
\begin{pgfscope}%
\pgfpathrectangle{\pgfqpoint{1.020000in}{0.880000in}}{\pgfqpoint{6.160000in}{6.160000in}}%
\pgfusepath{clip}%
\pgfsetbuttcap%
\pgfsetroundjoin%
\definecolor{currentfill}{rgb}{0.419991,0.552989,0.942630}%
\pgfsetfillcolor{currentfill}%
\pgfsetlinewidth{0.000000pt}%
\definecolor{currentstroke}{rgb}{0.000000,0.000000,0.000000}%
\pgfsetstrokecolor{currentstroke}%
\pgfsetdash{}{0pt}%
\pgfpathmoveto{\pgfqpoint{4.731278in}{2.804245in}}%
\pgfpathlineto{\pgfqpoint{4.741572in}{2.803705in}}%
\pgfpathlineto{\pgfqpoint{4.752182in}{2.854374in}}%
\pgfpathlineto{\pgfqpoint{4.785718in}{2.678859in}}%
\pgfpathlineto{\pgfqpoint{4.820672in}{2.746502in}}%
\pgfpathlineto{\pgfqpoint{4.808812in}{2.512859in}}%
\pgfpathlineto{\pgfqpoint{4.800029in}{2.763082in}}%
\pgfpathlineto{\pgfqpoint{4.765890in}{2.821038in}}%
\pgfpathlineto{\pgfqpoint{4.731278in}{2.804245in}}%
\pgfpathclose%
\pgfusepath{fill}%
\end{pgfscope}%
\begin{pgfscope}%
\pgfpathrectangle{\pgfqpoint{1.020000in}{0.880000in}}{\pgfqpoint{6.160000in}{6.160000in}}%
\pgfusepath{clip}%
\pgfsetbuttcap%
\pgfsetroundjoin%
\definecolor{currentfill}{rgb}{0.565182,0.699438,0.996635}%
\pgfsetfillcolor{currentfill}%
\pgfsetlinewidth{0.000000pt}%
\definecolor{currentstroke}{rgb}{0.000000,0.000000,0.000000}%
\pgfsetstrokecolor{currentstroke}%
\pgfsetdash{}{0pt}%
\pgfpathmoveto{\pgfqpoint{4.484530in}{3.103195in}}%
\pgfpathlineto{\pgfqpoint{4.494053in}{2.942273in}}%
\pgfpathlineto{\pgfqpoint{4.504186in}{2.973772in}}%
\pgfpathlineto{\pgfqpoint{4.538858in}{2.984983in}}%
\pgfpathlineto{\pgfqpoint{4.573423in}{2.972801in}}%
\pgfpathlineto{\pgfqpoint{4.563645in}{3.058435in}}%
\pgfpathlineto{\pgfqpoint{4.553068in}{2.927929in}}%
\pgfpathlineto{\pgfqpoint{4.519301in}{3.152392in}}%
\pgfpathlineto{\pgfqpoint{4.484530in}{3.103195in}}%
\pgfpathclose%
\pgfusepath{fill}%
\end{pgfscope}%
\begin{pgfscope}%
\pgfpathrectangle{\pgfqpoint{1.020000in}{0.880000in}}{\pgfqpoint{6.160000in}{6.160000in}}%
\pgfusepath{clip}%
\pgfsetbuttcap%
\pgfsetroundjoin%
\definecolor{currentfill}{rgb}{0.651398,0.768121,0.995891}%
\pgfsetfillcolor{currentfill}%
\pgfsetlinewidth{0.000000pt}%
\definecolor{currentstroke}{rgb}{0.000000,0.000000,0.000000}%
\pgfsetstrokecolor{currentstroke}%
\pgfsetdash{}{0pt}%
\pgfpathmoveto{\pgfqpoint{4.395747in}{3.261160in}}%
\pgfpathlineto{\pgfqpoint{4.405518in}{3.191489in}}%
\pgfpathlineto{\pgfqpoint{4.415458in}{3.190636in}}%
\pgfpathlineto{\pgfqpoint{4.449845in}{3.081649in}}%
\pgfpathlineto{\pgfqpoint{4.484530in}{3.103195in}}%
\pgfpathlineto{\pgfqpoint{4.475123in}{3.307574in}}%
\pgfpathlineto{\pgfqpoint{4.464481in}{3.080740in}}%
\pgfpathlineto{\pgfqpoint{4.430258in}{3.212388in}}%
\pgfpathlineto{\pgfqpoint{4.395747in}{3.261160in}}%
\pgfpathclose%
\pgfusepath{fill}%
\end{pgfscope}%
\begin{pgfscope}%
\pgfpathrectangle{\pgfqpoint{1.020000in}{0.880000in}}{\pgfqpoint{6.160000in}{6.160000in}}%
\pgfusepath{clip}%
\pgfsetbuttcap%
\pgfsetroundjoin%
\definecolor{currentfill}{rgb}{0.483854,0.622050,0.974808}%
\pgfsetfillcolor{currentfill}%
\pgfsetlinewidth{0.000000pt}%
\definecolor{currentstroke}{rgb}{0.000000,0.000000,0.000000}%
\pgfsetstrokecolor{currentstroke}%
\pgfsetdash{}{0pt}%
\pgfpathmoveto{\pgfqpoint{2.700210in}{3.091225in}}%
\pgfpathlineto{\pgfqpoint{2.714143in}{2.694252in}}%
\pgfpathlineto{\pgfqpoint{2.719642in}{2.867105in}}%
\pgfpathlineto{\pgfqpoint{2.754310in}{2.904143in}}%
\pgfpathlineto{\pgfqpoint{2.790548in}{2.829525in}}%
\pgfpathlineto{\pgfqpoint{2.783374in}{2.761902in}}%
\pgfpathlineto{\pgfqpoint{2.774955in}{2.784123in}}%
\pgfpathlineto{\pgfqpoint{2.736780in}{2.997419in}}%
\pgfpathlineto{\pgfqpoint{2.700210in}{3.091225in}}%
\pgfpathclose%
\pgfusepath{fill}%
\end{pgfscope}%
\begin{pgfscope}%
\pgfpathrectangle{\pgfqpoint{1.020000in}{0.880000in}}{\pgfqpoint{6.160000in}{6.160000in}}%
\pgfusepath{clip}%
\pgfsetbuttcap%
\pgfsetroundjoin%
\definecolor{currentfill}{rgb}{0.333490,0.446265,0.874452}%
\pgfsetfillcolor{currentfill}%
\pgfsetlinewidth{0.000000pt}%
\definecolor{currentstroke}{rgb}{0.000000,0.000000,0.000000}%
\pgfsetstrokecolor{currentstroke}%
\pgfsetdash{}{0pt}%
\pgfpathmoveto{\pgfqpoint{5.047362in}{2.588359in}}%
\pgfpathlineto{\pgfqpoint{5.058473in}{2.640655in}}%
\pgfpathlineto{\pgfqpoint{5.067826in}{2.492367in}}%
\pgfpathlineto{\pgfqpoint{5.102628in}{2.530795in}}%
\pgfpathlineto{\pgfqpoint{5.139284in}{2.761278in}}%
\pgfpathlineto{\pgfqpoint{5.126109in}{2.507655in}}%
\pgfpathlineto{\pgfqpoint{5.114770in}{2.440798in}}%
\pgfpathlineto{\pgfqpoint{5.081796in}{2.591102in}}%
\pgfpathlineto{\pgfqpoint{5.047362in}{2.588359in}}%
\pgfpathclose%
\pgfusepath{fill}%
\end{pgfscope}%
\begin{pgfscope}%
\pgfpathrectangle{\pgfqpoint{1.020000in}{0.880000in}}{\pgfqpoint{6.160000in}{6.160000in}}%
\pgfusepath{clip}%
\pgfsetbuttcap%
\pgfsetroundjoin%
\definecolor{currentfill}{rgb}{0.289996,0.386836,0.828926}%
\pgfsetfillcolor{currentfill}%
\pgfsetlinewidth{0.000000pt}%
\definecolor{currentstroke}{rgb}{0.000000,0.000000,0.000000}%
\pgfsetstrokecolor{currentstroke}%
\pgfsetdash{}{0pt}%
\pgfpathmoveto{\pgfqpoint{5.205461in}{2.490645in}}%
\pgfpathlineto{\pgfqpoint{5.216400in}{2.501467in}}%
\pgfpathlineto{\pgfqpoint{5.226464in}{2.426790in}}%
\pgfpathlineto{\pgfqpoint{5.262074in}{2.537301in}}%
\pgfpathlineto{\pgfqpoint{5.295947in}{2.488218in}}%
\pgfpathlineto{\pgfqpoint{5.284212in}{2.414983in}}%
\pgfpathlineto{\pgfqpoint{5.272900in}{2.376477in}}%
\pgfpathlineto{\pgfqpoint{5.240798in}{2.582330in}}%
\pgfpathlineto{\pgfqpoint{5.205461in}{2.490645in}}%
\pgfpathclose%
\pgfusepath{fill}%
\end{pgfscope}%
\begin{pgfscope}%
\pgfpathrectangle{\pgfqpoint{1.020000in}{0.880000in}}{\pgfqpoint{6.160000in}{6.160000in}}%
\pgfusepath{clip}%
\pgfsetbuttcap%
\pgfsetroundjoin%
\definecolor{currentfill}{rgb}{0.748682,0.827679,0.963334}%
\pgfsetfillcolor{currentfill}%
\pgfsetlinewidth{0.000000pt}%
\definecolor{currentstroke}{rgb}{0.000000,0.000000,0.000000}%
\pgfsetstrokecolor{currentstroke}%
\pgfsetdash{}{0pt}%
\pgfpathmoveto{\pgfqpoint{3.258214in}{3.383313in}}%
\pgfpathlineto{\pgfqpoint{3.268070in}{3.253142in}}%
\pgfpathlineto{\pgfqpoint{3.277200in}{3.202566in}}%
\pgfpathlineto{\pgfqpoint{3.310299in}{3.417090in}}%
\pgfpathlineto{\pgfqpoint{3.344903in}{3.467232in}}%
\pgfpathlineto{\pgfqpoint{3.336489in}{3.426218in}}%
\pgfpathlineto{\pgfqpoint{3.326519in}{3.570229in}}%
\pgfpathlineto{\pgfqpoint{3.294569in}{3.226938in}}%
\pgfpathlineto{\pgfqpoint{3.258214in}{3.383313in}}%
\pgfpathclose%
\pgfusepath{fill}%
\end{pgfscope}%
\begin{pgfscope}%
\pgfpathrectangle{\pgfqpoint{1.020000in}{0.880000in}}{\pgfqpoint{6.160000in}{6.160000in}}%
\pgfusepath{clip}%
\pgfsetbuttcap%
\pgfsetroundjoin%
\definecolor{currentfill}{rgb}{0.489246,0.627536,0.976896}%
\pgfsetfillcolor{currentfill}%
\pgfsetlinewidth{0.000000pt}%
\definecolor{currentstroke}{rgb}{0.000000,0.000000,0.000000}%
\pgfsetstrokecolor{currentstroke}%
\pgfsetdash{}{0pt}%
\pgfpathmoveto{\pgfqpoint{4.573423in}{2.972801in}}%
\pgfpathlineto{\pgfqpoint{4.583387in}{2.932037in}}%
\pgfpathlineto{\pgfqpoint{4.592843in}{2.764541in}}%
\pgfpathlineto{\pgfqpoint{4.628775in}{3.065488in}}%
\pgfpathlineto{\pgfqpoint{4.661688in}{2.701120in}}%
\pgfpathlineto{\pgfqpoint{4.651399in}{2.680866in}}%
\pgfpathlineto{\pgfqpoint{4.642734in}{3.006721in}}%
\pgfpathlineto{\pgfqpoint{4.607426in}{2.835657in}}%
\pgfpathlineto{\pgfqpoint{4.573423in}{2.972801in}}%
\pgfpathclose%
\pgfusepath{fill}%
\end{pgfscope}%
\begin{pgfscope}%
\pgfpathrectangle{\pgfqpoint{1.020000in}{0.880000in}}{\pgfqpoint{6.160000in}{6.160000in}}%
\pgfusepath{clip}%
\pgfsetbuttcap%
\pgfsetroundjoin%
\definecolor{currentfill}{rgb}{0.786721,0.844807,0.939810}%
\pgfsetfillcolor{currentfill}%
\pgfsetlinewidth{0.000000pt}%
\definecolor{currentstroke}{rgb}{0.000000,0.000000,0.000000}%
\pgfsetstrokecolor{currentstroke}%
\pgfsetdash{}{0pt}%
\pgfpathmoveto{\pgfqpoint{3.326519in}{3.570229in}}%
\pgfpathlineto{\pgfqpoint{3.336489in}{3.426218in}}%
\pgfpathlineto{\pgfqpoint{3.344903in}{3.467232in}}%
\pgfpathlineto{\pgfqpoint{3.380250in}{3.426737in}}%
\pgfpathlineto{\pgfqpoint{3.415158in}{3.436690in}}%
\pgfpathlineto{\pgfqpoint{3.405987in}{3.480594in}}%
\pgfpathlineto{\pgfqpoint{3.398462in}{3.316314in}}%
\pgfpathlineto{\pgfqpoint{3.362551in}{3.442559in}}%
\pgfpathlineto{\pgfqpoint{3.326519in}{3.570229in}}%
\pgfpathclose%
\pgfusepath{fill}%
\end{pgfscope}%
\begin{pgfscope}%
\pgfpathrectangle{\pgfqpoint{1.020000in}{0.880000in}}{\pgfqpoint{6.160000in}{6.160000in}}%
\pgfusepath{clip}%
\pgfsetbuttcap%
\pgfsetroundjoin%
\definecolor{currentfill}{rgb}{0.728970,0.817464,0.973188}%
\pgfsetfillcolor{currentfill}%
\pgfsetlinewidth{0.000000pt}%
\definecolor{currentstroke}{rgb}{0.000000,0.000000,0.000000}%
\pgfsetstrokecolor{currentstroke}%
\pgfsetdash{}{0pt}%
\pgfpathmoveto{\pgfqpoint{4.237616in}{3.479979in}}%
\pgfpathlineto{\pgfqpoint{4.247263in}{3.299281in}}%
\pgfpathlineto{\pgfqpoint{4.257031in}{3.300303in}}%
\pgfpathlineto{\pgfqpoint{4.291826in}{3.358486in}}%
\pgfpathlineto{\pgfqpoint{4.326358in}{3.219801in}}%
\pgfpathlineto{\pgfqpoint{4.316551in}{3.237130in}}%
\pgfpathlineto{\pgfqpoint{4.306852in}{3.326881in}}%
\pgfpathlineto{\pgfqpoint{4.272209in}{3.333895in}}%
\pgfpathlineto{\pgfqpoint{4.237616in}{3.479979in}}%
\pgfpathclose%
\pgfusepath{fill}%
\end{pgfscope}%
\begin{pgfscope}%
\pgfpathrectangle{\pgfqpoint{1.020000in}{0.880000in}}{\pgfqpoint{6.160000in}{6.160000in}}%
\pgfusepath{clip}%
\pgfsetbuttcap%
\pgfsetroundjoin%
\definecolor{currentfill}{rgb}{0.318832,0.426605,0.859857}%
\pgfsetfillcolor{currentfill}%
\pgfsetlinewidth{0.000000pt}%
\definecolor{currentstroke}{rgb}{0.000000,0.000000,0.000000}%
\pgfsetstrokecolor{currentstroke}%
\pgfsetdash{}{0pt}%
\pgfpathmoveto{\pgfqpoint{5.432998in}{2.459576in}}%
\pgfpathlineto{\pgfqpoint{5.446266in}{2.631367in}}%
\pgfpathlineto{\pgfqpoint{5.479205in}{2.520841in}}%
\pgfpathlineto{\pgfqpoint{5.513569in}{2.523799in}}%
\pgfpathlineto{\pgfqpoint{5.502203in}{2.508154in}}%
\pgfpathlineto{\pgfqpoint{5.468554in}{2.557244in}}%
\pgfpathlineto{\pgfqpoint{5.432998in}{2.459576in}}%
\pgfpathclose%
\pgfusepath{fill}%
\end{pgfscope}%
\begin{pgfscope}%
\pgfpathrectangle{\pgfqpoint{1.020000in}{0.880000in}}{\pgfqpoint{6.160000in}{6.160000in}}%
\pgfusepath{clip}%
\pgfsetbuttcap%
\pgfsetroundjoin%
\definecolor{currentfill}{rgb}{0.851372,0.863125,0.881064}%
\pgfsetfillcolor{currentfill}%
\pgfsetlinewidth{0.000000pt}%
\definecolor{currentstroke}{rgb}{0.000000,0.000000,0.000000}%
\pgfsetstrokecolor{currentstroke}%
\pgfsetdash{}{0pt}%
\pgfpathmoveto{\pgfqpoint{3.990302in}{3.757288in}}%
\pgfpathlineto{\pgfqpoint{4.000185in}{3.543558in}}%
\pgfpathlineto{\pgfqpoint{4.009554in}{3.615034in}}%
\pgfpathlineto{\pgfqpoint{4.044418in}{3.588739in}}%
\pgfpathlineto{\pgfqpoint{4.079333in}{3.476342in}}%
\pgfpathlineto{\pgfqpoint{4.069599in}{3.631639in}}%
\pgfpathlineto{\pgfqpoint{4.060051in}{3.629024in}}%
\pgfpathlineto{\pgfqpoint{4.025527in}{3.491570in}}%
\pgfpathlineto{\pgfqpoint{3.990302in}{3.757288in}}%
\pgfpathclose%
\pgfusepath{fill}%
\end{pgfscope}%
\begin{pgfscope}%
\pgfpathrectangle{\pgfqpoint{1.020000in}{0.880000in}}{\pgfqpoint{6.160000in}{6.160000in}}%
\pgfusepath{clip}%
\pgfsetbuttcap%
\pgfsetroundjoin%
\definecolor{currentfill}{rgb}{0.891817,0.851973,0.829085}%
\pgfsetfillcolor{currentfill}%
\pgfsetlinewidth{0.000000pt}%
\definecolor{currentstroke}{rgb}{0.000000,0.000000,0.000000}%
\pgfsetstrokecolor{currentstroke}%
\pgfsetdash{}{0pt}%
\pgfpathmoveto{\pgfqpoint{3.832519in}{3.672995in}}%
\pgfpathlineto{\pgfqpoint{3.841698in}{3.708828in}}%
\pgfpathlineto{\pgfqpoint{3.851262in}{3.636668in}}%
\pgfpathlineto{\pgfqpoint{3.885985in}{3.679454in}}%
\pgfpathlineto{\pgfqpoint{3.920833in}{3.677712in}}%
\pgfpathlineto{\pgfqpoint{3.911294in}{3.731668in}}%
\pgfpathlineto{\pgfqpoint{3.901696in}{3.810614in}}%
\pgfpathlineto{\pgfqpoint{3.867520in}{3.606694in}}%
\pgfpathlineto{\pgfqpoint{3.832519in}{3.672995in}}%
\pgfpathclose%
\pgfusepath{fill}%
\end{pgfscope}%
\begin{pgfscope}%
\pgfpathrectangle{\pgfqpoint{1.020000in}{0.880000in}}{\pgfqpoint{6.160000in}{6.160000in}}%
\pgfusepath{clip}%
\pgfsetbuttcap%
\pgfsetroundjoin%
\definecolor{currentfill}{rgb}{0.809329,0.852974,0.922323}%
\pgfsetfillcolor{currentfill}%
\pgfsetlinewidth{0.000000pt}%
\definecolor{currentstroke}{rgb}{0.000000,0.000000,0.000000}%
\pgfsetstrokecolor{currentstroke}%
\pgfsetdash{}{0pt}%
\pgfpathmoveto{\pgfqpoint{4.148796in}{3.445268in}}%
\pgfpathlineto{\pgfqpoint{4.158420in}{3.536214in}}%
\pgfpathlineto{\pgfqpoint{4.168134in}{3.283663in}}%
\pgfpathlineto{\pgfqpoint{4.202881in}{3.533179in}}%
\pgfpathlineto{\pgfqpoint{4.237616in}{3.479979in}}%
\pgfpathlineto{\pgfqpoint{4.227875in}{3.483137in}}%
\pgfpathlineto{\pgfqpoint{4.218194in}{3.595114in}}%
\pgfpathlineto{\pgfqpoint{4.183480in}{3.620099in}}%
\pgfpathlineto{\pgfqpoint{4.148796in}{3.445268in}}%
\pgfpathclose%
\pgfusepath{fill}%
\end{pgfscope}%
\begin{pgfscope}%
\pgfpathrectangle{\pgfqpoint{1.020000in}{0.880000in}}{\pgfqpoint{6.160000in}{6.160000in}}%
\pgfusepath{clip}%
\pgfsetbuttcap%
\pgfsetroundjoin%
\definecolor{currentfill}{rgb}{0.786721,0.844807,0.939810}%
\pgfsetfillcolor{currentfill}%
\pgfsetlinewidth{0.000000pt}%
\definecolor{currentstroke}{rgb}{0.000000,0.000000,0.000000}%
\pgfsetstrokecolor{currentstroke}%
\pgfsetdash{}{0pt}%
\pgfpathmoveto{\pgfqpoint{4.079333in}{3.476342in}}%
\pgfpathlineto{\pgfqpoint{4.088893in}{3.495483in}}%
\pgfpathlineto{\pgfqpoint{4.098623in}{3.343674in}}%
\pgfpathlineto{\pgfqpoint{4.133377in}{3.359626in}}%
\pgfpathlineto{\pgfqpoint{4.168134in}{3.283663in}}%
\pgfpathlineto{\pgfqpoint{4.158420in}{3.536214in}}%
\pgfpathlineto{\pgfqpoint{4.148796in}{3.445268in}}%
\pgfpathlineto{\pgfqpoint{4.114009in}{3.564051in}}%
\pgfpathlineto{\pgfqpoint{4.079333in}{3.476342in}}%
\pgfpathclose%
\pgfusepath{fill}%
\end{pgfscope}%
\begin{pgfscope}%
\pgfpathrectangle{\pgfqpoint{1.020000in}{0.880000in}}{\pgfqpoint{6.160000in}{6.160000in}}%
\pgfusepath{clip}%
\pgfsetbuttcap%
\pgfsetroundjoin%
\definecolor{currentfill}{rgb}{0.343278,0.459354,0.884122}%
\pgfsetfillcolor{currentfill}%
\pgfsetlinewidth{0.000000pt}%
\definecolor{currentstroke}{rgb}{0.000000,0.000000,0.000000}%
\pgfsetstrokecolor{currentstroke}%
\pgfsetdash{}{0pt}%
\pgfpathmoveto{\pgfqpoint{4.977843in}{2.508734in}}%
\pgfpathlineto{\pgfqpoint{4.989422in}{2.632899in}}%
\pgfpathlineto{\pgfqpoint{5.000211in}{2.655019in}}%
\pgfpathlineto{\pgfqpoint{5.033210in}{2.474901in}}%
\pgfpathlineto{\pgfqpoint{5.067826in}{2.492367in}}%
\pgfpathlineto{\pgfqpoint{5.058473in}{2.640655in}}%
\pgfpathlineto{\pgfqpoint{5.047362in}{2.588359in}}%
\pgfpathlineto{\pgfqpoint{5.013388in}{2.642827in}}%
\pgfpathlineto{\pgfqpoint{4.977843in}{2.508734in}}%
\pgfpathclose%
\pgfusepath{fill}%
\end{pgfscope}%
\begin{pgfscope}%
\pgfpathrectangle{\pgfqpoint{1.020000in}{0.880000in}}{\pgfqpoint{6.160000in}{6.160000in}}%
\pgfusepath{clip}%
\pgfsetbuttcap%
\pgfsetroundjoin%
\definecolor{currentfill}{rgb}{0.441123,0.576532,0.954545}%
\pgfsetfillcolor{currentfill}%
\pgfsetlinewidth{0.000000pt}%
\definecolor{currentstroke}{rgb}{0.000000,0.000000,0.000000}%
\pgfsetstrokecolor{currentstroke}%
\pgfsetdash{}{0pt}%
\pgfpathmoveto{\pgfqpoint{4.661688in}{2.701120in}}%
\pgfpathlineto{\pgfqpoint{4.672429in}{2.807675in}}%
\pgfpathlineto{\pgfqpoint{4.682020in}{2.676647in}}%
\pgfpathlineto{\pgfqpoint{4.717045in}{2.758849in}}%
\pgfpathlineto{\pgfqpoint{4.752182in}{2.854374in}}%
\pgfpathlineto{\pgfqpoint{4.741572in}{2.803705in}}%
\pgfpathlineto{\pgfqpoint{4.731278in}{2.804245in}}%
\pgfpathlineto{\pgfqpoint{4.696708in}{2.797703in}}%
\pgfpathlineto{\pgfqpoint{4.661688in}{2.701120in}}%
\pgfpathclose%
\pgfusepath{fill}%
\end{pgfscope}%
\begin{pgfscope}%
\pgfpathrectangle{\pgfqpoint{1.020000in}{0.880000in}}{\pgfqpoint{6.160000in}{6.160000in}}%
\pgfusepath{clip}%
\pgfsetbuttcap%
\pgfsetroundjoin%
\definecolor{currentfill}{rgb}{0.394042,0.522413,0.924916}%
\pgfsetfillcolor{currentfill}%
\pgfsetlinewidth{0.000000pt}%
\definecolor{currentstroke}{rgb}{0.000000,0.000000,0.000000}%
\pgfsetstrokecolor{currentstroke}%
\pgfsetdash{}{0pt}%
\pgfpathmoveto{\pgfqpoint{4.820672in}{2.746502in}}%
\pgfpathlineto{\pgfqpoint{4.832635in}{2.984887in}}%
\pgfpathlineto{\pgfqpoint{4.841628in}{2.764207in}}%
\pgfpathlineto{\pgfqpoint{4.874979in}{2.585394in}}%
\pgfpathlineto{\pgfqpoint{4.910138in}{2.673502in}}%
\pgfpathlineto{\pgfqpoint{4.899257in}{2.619757in}}%
\pgfpathlineto{\pgfqpoint{4.888280in}{2.547731in}}%
\pgfpathlineto{\pgfqpoint{4.853817in}{2.543127in}}%
\pgfpathlineto{\pgfqpoint{4.820672in}{2.746502in}}%
\pgfpathclose%
\pgfusepath{fill}%
\end{pgfscope}%
\begin{pgfscope}%
\pgfpathrectangle{\pgfqpoint{1.020000in}{0.880000in}}{\pgfqpoint{6.160000in}{6.160000in}}%
\pgfusepath{clip}%
\pgfsetbuttcap%
\pgfsetroundjoin%
\definecolor{currentfill}{rgb}{0.624703,0.748318,0.998719}%
\pgfsetfillcolor{currentfill}%
\pgfsetlinewidth{0.000000pt}%
\definecolor{currentstroke}{rgb}{0.000000,0.000000,0.000000}%
\pgfsetstrokecolor{currentstroke}%
\pgfsetdash{}{0pt}%
\pgfpathmoveto{\pgfqpoint{3.139148in}{3.006978in}}%
\pgfpathlineto{\pgfqpoint{3.146662in}{3.104174in}}%
\pgfpathlineto{\pgfqpoint{3.154496in}{3.173712in}}%
\pgfpathlineto{\pgfqpoint{3.190541in}{3.080877in}}%
\pgfpathlineto{\pgfqpoint{3.226430in}{2.995254in}}%
\pgfpathlineto{\pgfqpoint{3.215501in}{3.233984in}}%
\pgfpathlineto{\pgfqpoint{3.207132in}{3.208694in}}%
\pgfpathlineto{\pgfqpoint{3.173369in}{3.082822in}}%
\pgfpathlineto{\pgfqpoint{3.139148in}{3.006978in}}%
\pgfpathclose%
\pgfusepath{fill}%
\end{pgfscope}%
\begin{pgfscope}%
\pgfpathrectangle{\pgfqpoint{1.020000in}{0.880000in}}{\pgfqpoint{6.160000in}{6.160000in}}%
\pgfusepath{clip}%
\pgfsetbuttcap%
\pgfsetroundjoin%
\definecolor{currentfill}{rgb}{0.677823,0.786546,0.991005}%
\pgfsetfillcolor{currentfill}%
\pgfsetlinewidth{0.000000pt}%
\definecolor{currentstroke}{rgb}{0.000000,0.000000,0.000000}%
\pgfsetstrokecolor{currentstroke}%
\pgfsetdash{}{0pt}%
\pgfpathmoveto{\pgfqpoint{4.326358in}{3.219801in}}%
\pgfpathlineto{\pgfqpoint{4.336265in}{3.255596in}}%
\pgfpathlineto{\pgfqpoint{4.346296in}{3.350000in}}%
\pgfpathlineto{\pgfqpoint{4.380640in}{3.131107in}}%
\pgfpathlineto{\pgfqpoint{4.415458in}{3.190636in}}%
\pgfpathlineto{\pgfqpoint{4.405518in}{3.191489in}}%
\pgfpathlineto{\pgfqpoint{4.395747in}{3.261160in}}%
\pgfpathlineto{\pgfqpoint{4.360859in}{3.129865in}}%
\pgfpathlineto{\pgfqpoint{4.326358in}{3.219801in}}%
\pgfpathclose%
\pgfusepath{fill}%
\end{pgfscope}%
\begin{pgfscope}%
\pgfpathrectangle{\pgfqpoint{1.020000in}{0.880000in}}{\pgfqpoint{6.160000in}{6.160000in}}%
\pgfusepath{clip}%
\pgfsetbuttcap%
\pgfsetroundjoin%
\definecolor{currentfill}{rgb}{0.559747,0.694768,0.996075}%
\pgfsetfillcolor{currentfill}%
\pgfsetlinewidth{0.000000pt}%
\definecolor{currentstroke}{rgb}{0.000000,0.000000,0.000000}%
\pgfsetstrokecolor{currentstroke}%
\pgfsetdash{}{0pt}%
\pgfpathmoveto{\pgfqpoint{2.999184in}{2.989347in}}%
\pgfpathlineto{\pgfqpoint{3.008869in}{2.878673in}}%
\pgfpathlineto{\pgfqpoint{3.017824in}{2.830205in}}%
\pgfpathlineto{\pgfqpoint{3.050679in}{3.026359in}}%
\pgfpathlineto{\pgfqpoint{3.085873in}{3.020943in}}%
\pgfpathlineto{\pgfqpoint{3.078173in}{2.951456in}}%
\pgfpathlineto{\pgfqpoint{3.067804in}{3.123261in}}%
\pgfpathlineto{\pgfqpoint{3.032487in}{3.142418in}}%
\pgfpathlineto{\pgfqpoint{2.999184in}{2.989347in}}%
\pgfpathclose%
\pgfusepath{fill}%
\end{pgfscope}%
\begin{pgfscope}%
\pgfpathrectangle{\pgfqpoint{1.020000in}{0.880000in}}{\pgfqpoint{6.160000in}{6.160000in}}%
\pgfusepath{clip}%
\pgfsetbuttcap%
\pgfsetroundjoin%
\definecolor{currentfill}{rgb}{0.887752,0.854040,0.834671}%
\pgfsetfillcolor{currentfill}%
\pgfsetlinewidth{0.000000pt}%
\definecolor{currentstroke}{rgb}{0.000000,0.000000,0.000000}%
\pgfsetstrokecolor{currentstroke}%
\pgfsetdash{}{0pt}%
\pgfpathmoveto{\pgfqpoint{3.763002in}{3.641771in}}%
\pgfpathlineto{\pgfqpoint{3.771748in}{3.759113in}}%
\pgfpathlineto{\pgfqpoint{3.781271in}{3.693056in}}%
\pgfpathlineto{\pgfqpoint{3.816316in}{3.656957in}}%
\pgfpathlineto{\pgfqpoint{3.851262in}{3.636668in}}%
\pgfpathlineto{\pgfqpoint{3.841698in}{3.708828in}}%
\pgfpathlineto{\pgfqpoint{3.832519in}{3.672995in}}%
\pgfpathlineto{\pgfqpoint{3.797871in}{3.630214in}}%
\pgfpathlineto{\pgfqpoint{3.763002in}{3.641771in}}%
\pgfpathclose%
\pgfusepath{fill}%
\end{pgfscope}%
\begin{pgfscope}%
\pgfpathrectangle{\pgfqpoint{1.020000in}{0.880000in}}{\pgfqpoint{6.160000in}{6.160000in}}%
\pgfusepath{clip}%
\pgfsetbuttcap%
\pgfsetroundjoin%
\definecolor{currentfill}{rgb}{0.527132,0.664700,0.989065}%
\pgfsetfillcolor{currentfill}%
\pgfsetlinewidth{0.000000pt}%
\definecolor{currentstroke}{rgb}{0.000000,0.000000,0.000000}%
\pgfsetstrokecolor{currentstroke}%
\pgfsetdash{}{0pt}%
\pgfpathmoveto{\pgfqpoint{2.928789in}{3.004506in}}%
\pgfpathlineto{\pgfqpoint{2.938174in}{2.918228in}}%
\pgfpathlineto{\pgfqpoint{2.945712in}{2.980904in}}%
\pgfpathlineto{\pgfqpoint{2.979945in}{3.059582in}}%
\pgfpathlineto{\pgfqpoint{3.017824in}{2.830205in}}%
\pgfpathlineto{\pgfqpoint{3.008869in}{2.878673in}}%
\pgfpathlineto{\pgfqpoint{2.999184in}{2.989347in}}%
\pgfpathlineto{\pgfqpoint{2.966100in}{2.825394in}}%
\pgfpathlineto{\pgfqpoint{2.928789in}{3.004506in}}%
\pgfpathclose%
\pgfusepath{fill}%
\end{pgfscope}%
\begin{pgfscope}%
\pgfpathrectangle{\pgfqpoint{1.020000in}{0.880000in}}{\pgfqpoint{6.160000in}{6.160000in}}%
\pgfusepath{clip}%
\pgfsetbuttcap%
\pgfsetroundjoin%
\definecolor{currentfill}{rgb}{0.661968,0.775491,0.993937}%
\pgfsetfillcolor{currentfill}%
\pgfsetlinewidth{0.000000pt}%
\definecolor{currentstroke}{rgb}{0.000000,0.000000,0.000000}%
\pgfsetstrokecolor{currentstroke}%
\pgfsetdash{}{0pt}%
\pgfpathmoveto{\pgfqpoint{3.207132in}{3.208694in}}%
\pgfpathlineto{\pgfqpoint{3.215501in}{3.233984in}}%
\pgfpathlineto{\pgfqpoint{3.226430in}{2.995254in}}%
\pgfpathlineto{\pgfqpoint{3.259777in}{3.175319in}}%
\pgfpathlineto{\pgfqpoint{3.293977in}{3.271459in}}%
\pgfpathlineto{\pgfqpoint{3.286558in}{3.127815in}}%
\pgfpathlineto{\pgfqpoint{3.277200in}{3.202566in}}%
\pgfpathlineto{\pgfqpoint{3.241543in}{3.272980in}}%
\pgfpathlineto{\pgfqpoint{3.207132in}{3.208694in}}%
\pgfpathclose%
\pgfusepath{fill}%
\end{pgfscope}%
\begin{pgfscope}%
\pgfpathrectangle{\pgfqpoint{1.020000in}{0.880000in}}{\pgfqpoint{6.160000in}{6.160000in}}%
\pgfusepath{clip}%
\pgfsetbuttcap%
\pgfsetroundjoin%
\definecolor{currentfill}{rgb}{0.328604,0.439712,0.869587}%
\pgfsetfillcolor{currentfill}%
\pgfsetlinewidth{0.000000pt}%
\definecolor{currentstroke}{rgb}{0.000000,0.000000,0.000000}%
\pgfsetstrokecolor{currentstroke}%
\pgfsetdash{}{0pt}%
\pgfpathmoveto{\pgfqpoint{5.364389in}{2.462513in}}%
\pgfpathlineto{\pgfqpoint{5.377169in}{2.609435in}}%
\pgfpathlineto{\pgfqpoint{5.410783in}{2.544371in}}%
\pgfpathlineto{\pgfqpoint{5.446266in}{2.631367in}}%
\pgfpathlineto{\pgfqpoint{5.432998in}{2.459576in}}%
\pgfpathlineto{\pgfqpoint{5.399883in}{2.557069in}}%
\pgfpathlineto{\pgfqpoint{5.364389in}{2.462513in}}%
\pgfpathclose%
\pgfusepath{fill}%
\end{pgfscope}%
\begin{pgfscope}%
\pgfpathrectangle{\pgfqpoint{1.020000in}{0.880000in}}{\pgfqpoint{6.160000in}{6.160000in}}%
\pgfusepath{clip}%
\pgfsetbuttcap%
\pgfsetroundjoin%
\definecolor{currentfill}{rgb}{0.608547,0.735725,0.999354}%
\pgfsetfillcolor{currentfill}%
\pgfsetlinewidth{0.000000pt}%
\definecolor{currentstroke}{rgb}{0.000000,0.000000,0.000000}%
\pgfsetstrokecolor{currentstroke}%
\pgfsetdash{}{0pt}%
\pgfpathmoveto{\pgfqpoint{4.415458in}{3.190636in}}%
\pgfpathlineto{\pgfqpoint{4.425629in}{3.275748in}}%
\pgfpathlineto{\pgfqpoint{4.435003in}{3.030499in}}%
\pgfpathlineto{\pgfqpoint{4.469719in}{3.037981in}}%
\pgfpathlineto{\pgfqpoint{4.504186in}{2.973772in}}%
\pgfpathlineto{\pgfqpoint{4.494053in}{2.942273in}}%
\pgfpathlineto{\pgfqpoint{4.484530in}{3.103195in}}%
\pgfpathlineto{\pgfqpoint{4.449845in}{3.081649in}}%
\pgfpathlineto{\pgfqpoint{4.415458in}{3.190636in}}%
\pgfpathclose%
\pgfusepath{fill}%
\end{pgfscope}%
\begin{pgfscope}%
\pgfpathrectangle{\pgfqpoint{1.020000in}{0.880000in}}{\pgfqpoint{6.160000in}{6.160000in}}%
\pgfusepath{clip}%
\pgfsetbuttcap%
\pgfsetroundjoin%
\definecolor{currentfill}{rgb}{0.608547,0.735725,0.999354}%
\pgfsetfillcolor{currentfill}%
\pgfsetlinewidth{0.000000pt}%
\definecolor{currentstroke}{rgb}{0.000000,0.000000,0.000000}%
\pgfsetstrokecolor{currentstroke}%
\pgfsetdash{}{0pt}%
\pgfpathmoveto{\pgfqpoint{3.067804in}{3.123261in}}%
\pgfpathlineto{\pgfqpoint{3.078173in}{2.951456in}}%
\pgfpathlineto{\pgfqpoint{3.085873in}{3.020943in}}%
\pgfpathlineto{\pgfqpoint{3.120148in}{3.099267in}}%
\pgfpathlineto{\pgfqpoint{3.154496in}{3.173712in}}%
\pgfpathlineto{\pgfqpoint{3.146662in}{3.104174in}}%
\pgfpathlineto{\pgfqpoint{3.139148in}{3.006978in}}%
\pgfpathlineto{\pgfqpoint{3.102216in}{3.184288in}}%
\pgfpathlineto{\pgfqpoint{3.067804in}{3.123261in}}%
\pgfpathclose%
\pgfusepath{fill}%
\end{pgfscope}%
\begin{pgfscope}%
\pgfpathrectangle{\pgfqpoint{1.020000in}{0.880000in}}{\pgfqpoint{6.160000in}{6.160000in}}%
\pgfusepath{clip}%
\pgfsetbuttcap%
\pgfsetroundjoin%
\definecolor{currentfill}{rgb}{0.275827,0.366717,0.812553}%
\pgfsetfillcolor{currentfill}%
\pgfsetlinewidth{0.000000pt}%
\definecolor{currentstroke}{rgb}{0.000000,0.000000,0.000000}%
\pgfsetstrokecolor{currentstroke}%
\pgfsetdash{}{0pt}%
\pgfpathmoveto{\pgfqpoint{5.226464in}{2.426790in}}%
\pgfpathlineto{\pgfqpoint{5.237996in}{2.489136in}}%
\pgfpathlineto{\pgfqpoint{5.270261in}{2.289181in}}%
\pgfpathlineto{\pgfqpoint{5.305602in}{2.374573in}}%
\pgfpathlineto{\pgfqpoint{5.295947in}{2.488218in}}%
\pgfpathlineto{\pgfqpoint{5.262074in}{2.537301in}}%
\pgfpathlineto{\pgfqpoint{5.226464in}{2.426790in}}%
\pgfpathclose%
\pgfusepath{fill}%
\end{pgfscope}%
\begin{pgfscope}%
\pgfpathrectangle{\pgfqpoint{1.020000in}{0.880000in}}{\pgfqpoint{6.160000in}{6.160000in}}%
\pgfusepath{clip}%
\pgfsetbuttcap%
\pgfsetroundjoin%
\definecolor{currentfill}{rgb}{0.867428,0.864377,0.862602}%
\pgfsetfillcolor{currentfill}%
\pgfsetlinewidth{0.000000pt}%
\definecolor{currentstroke}{rgb}{0.000000,0.000000,0.000000}%
\pgfsetstrokecolor{currentstroke}%
\pgfsetdash{}{0pt}%
\pgfpathmoveto{\pgfqpoint{3.920833in}{3.677712in}}%
\pgfpathlineto{\pgfqpoint{3.930189in}{3.700607in}}%
\pgfpathlineto{\pgfqpoint{3.940161in}{3.480026in}}%
\pgfpathlineto{\pgfqpoint{3.974798in}{3.570043in}}%
\pgfpathlineto{\pgfqpoint{4.009554in}{3.615034in}}%
\pgfpathlineto{\pgfqpoint{4.000185in}{3.543558in}}%
\pgfpathlineto{\pgfqpoint{3.990302in}{3.757288in}}%
\pgfpathlineto{\pgfqpoint{3.955792in}{3.617400in}}%
\pgfpathlineto{\pgfqpoint{3.920833in}{3.677712in}}%
\pgfpathclose%
\pgfusepath{fill}%
\end{pgfscope}%
\begin{pgfscope}%
\pgfpathrectangle{\pgfqpoint{1.020000in}{0.880000in}}{\pgfqpoint{6.160000in}{6.160000in}}%
\pgfusepath{clip}%
\pgfsetbuttcap%
\pgfsetroundjoin%
\definecolor{currentfill}{rgb}{0.748682,0.827679,0.963334}%
\pgfsetfillcolor{currentfill}%
\pgfsetlinewidth{0.000000pt}%
\definecolor{currentstroke}{rgb}{0.000000,0.000000,0.000000}%
\pgfsetstrokecolor{currentstroke}%
\pgfsetdash{}{0pt}%
\pgfpathmoveto{\pgfqpoint{3.344903in}{3.467232in}}%
\pgfpathlineto{\pgfqpoint{3.354825in}{3.329636in}}%
\pgfpathlineto{\pgfqpoint{3.363607in}{3.330438in}}%
\pgfpathlineto{\pgfqpoint{3.399885in}{3.173401in}}%
\pgfpathlineto{\pgfqpoint{3.432979in}{3.423307in}}%
\pgfpathlineto{\pgfqpoint{3.425477in}{3.243291in}}%
\pgfpathlineto{\pgfqpoint{3.415158in}{3.436690in}}%
\pgfpathlineto{\pgfqpoint{3.380250in}{3.426737in}}%
\pgfpathlineto{\pgfqpoint{3.344903in}{3.467232in}}%
\pgfpathclose%
\pgfusepath{fill}%
\end{pgfscope}%
\begin{pgfscope}%
\pgfpathrectangle{\pgfqpoint{1.020000in}{0.880000in}}{\pgfqpoint{6.160000in}{6.160000in}}%
\pgfusepath{clip}%
\pgfsetbuttcap%
\pgfsetroundjoin%
\definecolor{currentfill}{rgb}{0.338377,0.452819,0.879317}%
\pgfsetfillcolor{currentfill}%
\pgfsetlinewidth{0.000000pt}%
\definecolor{currentstroke}{rgb}{0.000000,0.000000,0.000000}%
\pgfsetstrokecolor{currentstroke}%
\pgfsetdash{}{0pt}%
\pgfpathmoveto{\pgfqpoint{5.139284in}{2.761278in}}%
\pgfpathlineto{\pgfqpoint{5.148140in}{2.563099in}}%
\pgfpathlineto{\pgfqpoint{5.159079in}{2.578483in}}%
\pgfpathlineto{\pgfqpoint{5.192986in}{2.520033in}}%
\pgfpathlineto{\pgfqpoint{5.226464in}{2.426790in}}%
\pgfpathlineto{\pgfqpoint{5.216400in}{2.501467in}}%
\pgfpathlineto{\pgfqpoint{5.205461in}{2.490645in}}%
\pgfpathlineto{\pgfqpoint{5.172607in}{2.643736in}}%
\pgfpathlineto{\pgfqpoint{5.139284in}{2.761278in}}%
\pgfpathclose%
\pgfusepath{fill}%
\end{pgfscope}%
\begin{pgfscope}%
\pgfpathrectangle{\pgfqpoint{1.020000in}{0.880000in}}{\pgfqpoint{6.160000in}{6.160000in}}%
\pgfusepath{clip}%
\pgfsetbuttcap%
\pgfsetroundjoin%
\definecolor{currentfill}{rgb}{0.777378,0.840921,0.946149}%
\pgfsetfillcolor{currentfill}%
\pgfsetlinewidth{0.000000pt}%
\definecolor{currentstroke}{rgb}{0.000000,0.000000,0.000000}%
\pgfsetstrokecolor{currentstroke}%
\pgfsetdash{}{0pt}%
\pgfpathmoveto{\pgfqpoint{3.415158in}{3.436690in}}%
\pgfpathlineto{\pgfqpoint{3.425477in}{3.243291in}}%
\pgfpathlineto{\pgfqpoint{3.432979in}{3.423307in}}%
\pgfpathlineto{\pgfqpoint{3.468260in}{3.388832in}}%
\pgfpathlineto{\pgfqpoint{3.503321in}{3.379931in}}%
\pgfpathlineto{\pgfqpoint{3.492915in}{3.593184in}}%
\pgfpathlineto{\pgfqpoint{3.485419in}{3.386848in}}%
\pgfpathlineto{\pgfqpoint{3.449831in}{3.476876in}}%
\pgfpathlineto{\pgfqpoint{3.415158in}{3.436690in}}%
\pgfpathclose%
\pgfusepath{fill}%
\end{pgfscope}%
\begin{pgfscope}%
\pgfpathrectangle{\pgfqpoint{1.020000in}{0.880000in}}{\pgfqpoint{6.160000in}{6.160000in}}%
\pgfusepath{clip}%
\pgfsetbuttcap%
\pgfsetroundjoin%
\definecolor{currentfill}{rgb}{0.457046,0.594006,0.963029}%
\pgfsetfillcolor{currentfill}%
\pgfsetlinewidth{0.000000pt}%
\definecolor{currentstroke}{rgb}{0.000000,0.000000,0.000000}%
\pgfsetstrokecolor{currentstroke}%
\pgfsetdash{}{0pt}%
\pgfpathmoveto{\pgfqpoint{4.592843in}{2.764541in}}%
\pgfpathlineto{\pgfqpoint{4.603080in}{2.784717in}}%
\pgfpathlineto{\pgfqpoint{4.613337in}{2.803096in}}%
\pgfpathlineto{\pgfqpoint{4.647933in}{2.787367in}}%
\pgfpathlineto{\pgfqpoint{4.682020in}{2.676647in}}%
\pgfpathlineto{\pgfqpoint{4.672429in}{2.807675in}}%
\pgfpathlineto{\pgfqpoint{4.661688in}{2.701120in}}%
\pgfpathlineto{\pgfqpoint{4.628775in}{3.065488in}}%
\pgfpathlineto{\pgfqpoint{4.592843in}{2.764541in}}%
\pgfpathclose%
\pgfusepath{fill}%
\end{pgfscope}%
\begin{pgfscope}%
\pgfpathrectangle{\pgfqpoint{1.020000in}{0.880000in}}{\pgfqpoint{6.160000in}{6.160000in}}%
\pgfusepath{clip}%
\pgfsetbuttcap%
\pgfsetroundjoin%
\definecolor{currentfill}{rgb}{0.318832,0.426605,0.859857}%
\pgfsetfillcolor{currentfill}%
\pgfsetlinewidth{0.000000pt}%
\definecolor{currentstroke}{rgb}{0.000000,0.000000,0.000000}%
\pgfsetstrokecolor{currentstroke}%
\pgfsetdash{}{0pt}%
\pgfpathmoveto{\pgfqpoint{5.295947in}{2.488218in}}%
\pgfpathlineto{\pgfqpoint{5.305602in}{2.374573in}}%
\pgfpathlineto{\pgfqpoint{5.342732in}{2.609768in}}%
\pgfpathlineto{\pgfqpoint{5.377169in}{2.609435in}}%
\pgfpathlineto{\pgfqpoint{5.364389in}{2.462513in}}%
\pgfpathlineto{\pgfqpoint{5.331710in}{2.607197in}}%
\pgfpathlineto{\pgfqpoint{5.295947in}{2.488218in}}%
\pgfpathclose%
\pgfusepath{fill}%
\end{pgfscope}%
\begin{pgfscope}%
\pgfpathrectangle{\pgfqpoint{1.020000in}{0.880000in}}{\pgfqpoint{6.160000in}{6.160000in}}%
\pgfusepath{clip}%
\pgfsetbuttcap%
\pgfsetroundjoin%
\definecolor{currentfill}{rgb}{0.494638,0.633022,0.978983}%
\pgfsetfillcolor{currentfill}%
\pgfsetlinewidth{0.000000pt}%
\definecolor{currentstroke}{rgb}{0.000000,0.000000,0.000000}%
\pgfsetstrokecolor{currentstroke}%
\pgfsetdash{}{0pt}%
\pgfpathmoveto{\pgfqpoint{2.790548in}{2.829525in}}%
\pgfpathlineto{\pgfqpoint{2.797862in}{2.889168in}}%
\pgfpathlineto{\pgfqpoint{2.808856in}{2.685163in}}%
\pgfpathlineto{\pgfqpoint{2.840373in}{2.955563in}}%
\pgfpathlineto{\pgfqpoint{2.877422in}{2.817843in}}%
\pgfpathlineto{\pgfqpoint{2.867736in}{2.927611in}}%
\pgfpathlineto{\pgfqpoint{2.858684in}{2.989651in}}%
\pgfpathlineto{\pgfqpoint{2.825331in}{2.855961in}}%
\pgfpathlineto{\pgfqpoint{2.790548in}{2.829525in}}%
\pgfpathclose%
\pgfusepath{fill}%
\end{pgfscope}%
\begin{pgfscope}%
\pgfpathrectangle{\pgfqpoint{1.020000in}{0.880000in}}{\pgfqpoint{6.160000in}{6.160000in}}%
\pgfusepath{clip}%
\pgfsetbuttcap%
\pgfsetroundjoin%
\definecolor{currentfill}{rgb}{0.532568,0.669801,0.990393}%
\pgfsetfillcolor{currentfill}%
\pgfsetlinewidth{0.000000pt}%
\definecolor{currentstroke}{rgb}{0.000000,0.000000,0.000000}%
\pgfsetstrokecolor{currentstroke}%
\pgfsetdash{}{0pt}%
\pgfpathmoveto{\pgfqpoint{2.858684in}{2.989651in}}%
\pgfpathlineto{\pgfqpoint{2.867736in}{2.927611in}}%
\pgfpathlineto{\pgfqpoint{2.877422in}{2.817843in}}%
\pgfpathlineto{\pgfqpoint{2.912138in}{2.853346in}}%
\pgfpathlineto{\pgfqpoint{2.945712in}{2.980904in}}%
\pgfpathlineto{\pgfqpoint{2.938174in}{2.918228in}}%
\pgfpathlineto{\pgfqpoint{2.928789in}{3.004506in}}%
\pgfpathlineto{\pgfqpoint{2.894189in}{2.962742in}}%
\pgfpathlineto{\pgfqpoint{2.858684in}{2.989651in}}%
\pgfpathclose%
\pgfusepath{fill}%
\end{pgfscope}%
\begin{pgfscope}%
\pgfpathrectangle{\pgfqpoint{1.020000in}{0.880000in}}{\pgfqpoint{6.160000in}{6.160000in}}%
\pgfusepath{clip}%
\pgfsetbuttcap%
\pgfsetroundjoin%
\definecolor{currentfill}{rgb}{0.467678,0.605591,0.968546}%
\pgfsetfillcolor{currentfill}%
\pgfsetlinewidth{0.000000pt}%
\definecolor{currentstroke}{rgb}{0.000000,0.000000,0.000000}%
\pgfsetstrokecolor{currentstroke}%
\pgfsetdash{}{0pt}%
\pgfpathmoveto{\pgfqpoint{2.719642in}{2.867105in}}%
\pgfpathlineto{\pgfqpoint{2.727314in}{2.894593in}}%
\pgfpathlineto{\pgfqpoint{2.739818in}{2.591770in}}%
\pgfpathlineto{\pgfqpoint{2.771358in}{2.847838in}}%
\pgfpathlineto{\pgfqpoint{2.808856in}{2.685163in}}%
\pgfpathlineto{\pgfqpoint{2.797862in}{2.889168in}}%
\pgfpathlineto{\pgfqpoint{2.790548in}{2.829525in}}%
\pgfpathlineto{\pgfqpoint{2.754310in}{2.904143in}}%
\pgfpathlineto{\pgfqpoint{2.719642in}{2.867105in}}%
\pgfpathclose%
\pgfusepath{fill}%
\end{pgfscope}%
\begin{pgfscope}%
\pgfpathrectangle{\pgfqpoint{1.020000in}{0.880000in}}{\pgfqpoint{6.160000in}{6.160000in}}%
\pgfusepath{clip}%
\pgfsetbuttcap%
\pgfsetroundjoin%
\definecolor{currentfill}{rgb}{0.813693,0.854282,0.918480}%
\pgfsetfillcolor{currentfill}%
\pgfsetlinewidth{0.000000pt}%
\definecolor{currentstroke}{rgb}{0.000000,0.000000,0.000000}%
\pgfsetstrokecolor{currentstroke}%
\pgfsetdash{}{0pt}%
\pgfpathmoveto{\pgfqpoint{3.485419in}{3.386848in}}%
\pgfpathlineto{\pgfqpoint{3.492915in}{3.593184in}}%
\pgfpathlineto{\pgfqpoint{3.503321in}{3.379931in}}%
\pgfpathlineto{\pgfqpoint{3.537553in}{3.495297in}}%
\pgfpathlineto{\pgfqpoint{3.571905in}{3.600522in}}%
\pgfpathlineto{\pgfqpoint{3.562179in}{3.713262in}}%
\pgfpathlineto{\pgfqpoint{3.555183in}{3.391249in}}%
\pgfpathlineto{\pgfqpoint{3.520241in}{3.399616in}}%
\pgfpathlineto{\pgfqpoint{3.485419in}{3.386848in}}%
\pgfpathclose%
\pgfusepath{fill}%
\end{pgfscope}%
\begin{pgfscope}%
\pgfpathrectangle{\pgfqpoint{1.020000in}{0.880000in}}{\pgfqpoint{6.160000in}{6.160000in}}%
\pgfusepath{clip}%
\pgfsetbuttcap%
\pgfsetroundjoin%
\definecolor{currentfill}{rgb}{0.891817,0.851973,0.829085}%
\pgfsetfillcolor{currentfill}%
\pgfsetlinewidth{0.000000pt}%
\definecolor{currentstroke}{rgb}{0.000000,0.000000,0.000000}%
\pgfsetstrokecolor{currentstroke}%
\pgfsetdash{}{0pt}%
\pgfpathmoveto{\pgfqpoint{3.692540in}{3.795249in}}%
\pgfpathlineto{\pgfqpoint{3.702157in}{3.702022in}}%
\pgfpathlineto{\pgfqpoint{3.711868in}{3.589343in}}%
\pgfpathlineto{\pgfqpoint{3.746043in}{3.760233in}}%
\pgfpathlineto{\pgfqpoint{3.781271in}{3.693056in}}%
\pgfpathlineto{\pgfqpoint{3.771748in}{3.759113in}}%
\pgfpathlineto{\pgfqpoint{3.763002in}{3.641771in}}%
\pgfpathlineto{\pgfqpoint{3.728580in}{3.548789in}}%
\pgfpathlineto{\pgfqpoint{3.692540in}{3.795249in}}%
\pgfpathclose%
\pgfusepath{fill}%
\end{pgfscope}%
\begin{pgfscope}%
\pgfpathrectangle{\pgfqpoint{1.020000in}{0.880000in}}{\pgfqpoint{6.160000in}{6.160000in}}%
\pgfusepath{clip}%
\pgfsetbuttcap%
\pgfsetroundjoin%
\definecolor{currentfill}{rgb}{0.554312,0.690097,0.995516}%
\pgfsetfillcolor{currentfill}%
\pgfsetlinewidth{0.000000pt}%
\definecolor{currentstroke}{rgb}{0.000000,0.000000,0.000000}%
\pgfsetstrokecolor{currentstroke}%
\pgfsetdash{}{0pt}%
\pgfpathmoveto{\pgfqpoint{4.504186in}{2.973772in}}%
\pgfpathlineto{\pgfqpoint{4.514765in}{3.130630in}}%
\pgfpathlineto{\pgfqpoint{4.524518in}{3.030609in}}%
\pgfpathlineto{\pgfqpoint{4.559218in}{3.024823in}}%
\pgfpathlineto{\pgfqpoint{4.592843in}{2.764541in}}%
\pgfpathlineto{\pgfqpoint{4.583387in}{2.932037in}}%
\pgfpathlineto{\pgfqpoint{4.573423in}{2.972801in}}%
\pgfpathlineto{\pgfqpoint{4.538858in}{2.984983in}}%
\pgfpathlineto{\pgfqpoint{4.504186in}{2.973772in}}%
\pgfpathclose%
\pgfusepath{fill}%
\end{pgfscope}%
\begin{pgfscope}%
\pgfpathrectangle{\pgfqpoint{1.020000in}{0.880000in}}{\pgfqpoint{6.160000in}{6.160000in}}%
\pgfusepath{clip}%
\pgfsetbuttcap%
\pgfsetroundjoin%
\definecolor{currentfill}{rgb}{0.368507,0.491141,0.905243}%
\pgfsetfillcolor{currentfill}%
\pgfsetlinewidth{0.000000pt}%
\definecolor{currentstroke}{rgb}{0.000000,0.000000,0.000000}%
\pgfsetstrokecolor{currentstroke}%
\pgfsetdash{}{0pt}%
\pgfpathmoveto{\pgfqpoint{4.910138in}{2.673502in}}%
\pgfpathlineto{\pgfqpoint{4.920230in}{2.616166in}}%
\pgfpathlineto{\pgfqpoint{4.931892in}{2.765782in}}%
\pgfpathlineto{\pgfqpoint{4.964930in}{2.564407in}}%
\pgfpathlineto{\pgfqpoint{5.000211in}{2.655019in}}%
\pgfpathlineto{\pgfqpoint{4.989422in}{2.632899in}}%
\pgfpathlineto{\pgfqpoint{4.977843in}{2.508734in}}%
\pgfpathlineto{\pgfqpoint{4.943899in}{2.574033in}}%
\pgfpathlineto{\pgfqpoint{4.910138in}{2.673502in}}%
\pgfpathclose%
\pgfusepath{fill}%
\end{pgfscope}%
\begin{pgfscope}%
\pgfpathrectangle{\pgfqpoint{1.020000in}{0.880000in}}{\pgfqpoint{6.160000in}{6.160000in}}%
\pgfusepath{clip}%
\pgfsetbuttcap%
\pgfsetroundjoin%
\definecolor{currentfill}{rgb}{0.733898,0.820018,0.970724}%
\pgfsetfillcolor{currentfill}%
\pgfsetlinewidth{0.000000pt}%
\definecolor{currentstroke}{rgb}{0.000000,0.000000,0.000000}%
\pgfsetstrokecolor{currentstroke}%
\pgfsetdash{}{0pt}%
\pgfpathmoveto{\pgfqpoint{3.277200in}{3.202566in}}%
\pgfpathlineto{\pgfqpoint{3.286558in}{3.127815in}}%
\pgfpathlineto{\pgfqpoint{3.293977in}{3.271459in}}%
\pgfpathlineto{\pgfqpoint{3.327787in}{3.418462in}}%
\pgfpathlineto{\pgfqpoint{3.363607in}{3.330438in}}%
\pgfpathlineto{\pgfqpoint{3.354825in}{3.329636in}}%
\pgfpathlineto{\pgfqpoint{3.344903in}{3.467232in}}%
\pgfpathlineto{\pgfqpoint{3.310299in}{3.417090in}}%
\pgfpathlineto{\pgfqpoint{3.277200in}{3.202566in}}%
\pgfpathclose%
\pgfusepath{fill}%
\end{pgfscope}%
\begin{pgfscope}%
\pgfpathrectangle{\pgfqpoint{1.020000in}{0.880000in}}{\pgfqpoint{6.160000in}{6.160000in}}%
\pgfusepath{clip}%
\pgfsetbuttcap%
\pgfsetroundjoin%
\definecolor{currentfill}{rgb}{0.419991,0.552989,0.942630}%
\pgfsetfillcolor{currentfill}%
\pgfsetlinewidth{0.000000pt}%
\definecolor{currentstroke}{rgb}{0.000000,0.000000,0.000000}%
\pgfsetstrokecolor{currentstroke}%
\pgfsetdash{}{0pt}%
\pgfpathmoveto{\pgfqpoint{4.752182in}{2.854374in}}%
\pgfpathlineto{\pgfqpoint{4.761062in}{2.602336in}}%
\pgfpathlineto{\pgfqpoint{4.771284in}{2.581656in}}%
\pgfpathlineto{\pgfqpoint{4.805852in}{2.580073in}}%
\pgfpathlineto{\pgfqpoint{4.841628in}{2.764207in}}%
\pgfpathlineto{\pgfqpoint{4.832635in}{2.984887in}}%
\pgfpathlineto{\pgfqpoint{4.820672in}{2.746502in}}%
\pgfpathlineto{\pgfqpoint{4.785718in}{2.678859in}}%
\pgfpathlineto{\pgfqpoint{4.752182in}{2.854374in}}%
\pgfpathclose%
\pgfusepath{fill}%
\end{pgfscope}%
\begin{pgfscope}%
\pgfpathrectangle{\pgfqpoint{1.020000in}{0.880000in}}{\pgfqpoint{6.160000in}{6.160000in}}%
\pgfusepath{clip}%
\pgfsetbuttcap%
\pgfsetroundjoin%
\definecolor{currentfill}{rgb}{0.851372,0.863125,0.881064}%
\pgfsetfillcolor{currentfill}%
\pgfsetlinewidth{0.000000pt}%
\definecolor{currentstroke}{rgb}{0.000000,0.000000,0.000000}%
\pgfsetstrokecolor{currentstroke}%
\pgfsetdash{}{0pt}%
\pgfpathmoveto{\pgfqpoint{3.555183in}{3.391249in}}%
\pgfpathlineto{\pgfqpoint{3.562179in}{3.713262in}}%
\pgfpathlineto{\pgfqpoint{3.571905in}{3.600522in}}%
\pgfpathlineto{\pgfqpoint{3.607756in}{3.454437in}}%
\pgfpathlineto{\pgfqpoint{3.641658in}{3.646049in}}%
\pgfpathlineto{\pgfqpoint{3.632517in}{3.652735in}}%
\pgfpathlineto{\pgfqpoint{3.624085in}{3.535737in}}%
\pgfpathlineto{\pgfqpoint{3.588781in}{3.604861in}}%
\pgfpathlineto{\pgfqpoint{3.555183in}{3.391249in}}%
\pgfpathclose%
\pgfusepath{fill}%
\end{pgfscope}%
\begin{pgfscope}%
\pgfpathrectangle{\pgfqpoint{1.020000in}{0.880000in}}{\pgfqpoint{6.160000in}{6.160000in}}%
\pgfusepath{clip}%
\pgfsetbuttcap%
\pgfsetroundjoin%
\definecolor{currentfill}{rgb}{0.777378,0.840921,0.946149}%
\pgfsetfillcolor{currentfill}%
\pgfsetlinewidth{0.000000pt}%
\definecolor{currentstroke}{rgb}{0.000000,0.000000,0.000000}%
\pgfsetstrokecolor{currentstroke}%
\pgfsetdash{}{0pt}%
\pgfpathmoveto{\pgfqpoint{4.168134in}{3.283663in}}%
\pgfpathlineto{\pgfqpoint{4.177797in}{3.481643in}}%
\pgfpathlineto{\pgfqpoint{4.187510in}{3.468279in}}%
\pgfpathlineto{\pgfqpoint{4.222332in}{3.437311in}}%
\pgfpathlineto{\pgfqpoint{4.257031in}{3.300303in}}%
\pgfpathlineto{\pgfqpoint{4.247263in}{3.299281in}}%
\pgfpathlineto{\pgfqpoint{4.237616in}{3.479979in}}%
\pgfpathlineto{\pgfqpoint{4.202881in}{3.533179in}}%
\pgfpathlineto{\pgfqpoint{4.168134in}{3.283663in}}%
\pgfpathclose%
\pgfusepath{fill}%
\end{pgfscope}%
\begin{pgfscope}%
\pgfpathrectangle{\pgfqpoint{1.020000in}{0.880000in}}{\pgfqpoint{6.160000in}{6.160000in}}%
\pgfusepath{clip}%
\pgfsetbuttcap%
\pgfsetroundjoin%
\definecolor{currentfill}{rgb}{0.328604,0.439712,0.869587}%
\pgfsetfillcolor{currentfill}%
\pgfsetlinewidth{0.000000pt}%
\definecolor{currentstroke}{rgb}{0.000000,0.000000,0.000000}%
\pgfsetstrokecolor{currentstroke}%
\pgfsetdash{}{0pt}%
\pgfpathmoveto{\pgfqpoint{5.067826in}{2.492367in}}%
\pgfpathlineto{\pgfqpoint{5.078303in}{2.469210in}}%
\pgfpathlineto{\pgfqpoint{5.089402in}{2.512036in}}%
\pgfpathlineto{\pgfqpoint{5.123013in}{2.415979in}}%
\pgfpathlineto{\pgfqpoint{5.159079in}{2.578483in}}%
\pgfpathlineto{\pgfqpoint{5.148140in}{2.563099in}}%
\pgfpathlineto{\pgfqpoint{5.139284in}{2.761278in}}%
\pgfpathlineto{\pgfqpoint{5.102628in}{2.530795in}}%
\pgfpathlineto{\pgfqpoint{5.067826in}{2.492367in}}%
\pgfpathclose%
\pgfusepath{fill}%
\end{pgfscope}%
\begin{pgfscope}%
\pgfpathrectangle{\pgfqpoint{1.020000in}{0.880000in}}{\pgfqpoint{6.160000in}{6.160000in}}%
\pgfusepath{clip}%
\pgfsetbuttcap%
\pgfsetroundjoin%
\definecolor{currentfill}{rgb}{0.822420,0.856898,0.910795}%
\pgfsetfillcolor{currentfill}%
\pgfsetlinewidth{0.000000pt}%
\definecolor{currentstroke}{rgb}{0.000000,0.000000,0.000000}%
\pgfsetstrokecolor{currentstroke}%
\pgfsetdash{}{0pt}%
\pgfpathmoveto{\pgfqpoint{4.009554in}{3.615034in}}%
\pgfpathlineto{\pgfqpoint{4.019187in}{3.546519in}}%
\pgfpathlineto{\pgfqpoint{4.028784in}{3.506140in}}%
\pgfpathlineto{\pgfqpoint{4.063640in}{3.511272in}}%
\pgfpathlineto{\pgfqpoint{4.098623in}{3.343674in}}%
\pgfpathlineto{\pgfqpoint{4.088893in}{3.495483in}}%
\pgfpathlineto{\pgfqpoint{4.079333in}{3.476342in}}%
\pgfpathlineto{\pgfqpoint{4.044418in}{3.588739in}}%
\pgfpathlineto{\pgfqpoint{4.009554in}{3.615034in}}%
\pgfpathclose%
\pgfusepath{fill}%
\end{pgfscope}%
\begin{pgfscope}%
\pgfpathrectangle{\pgfqpoint{1.020000in}{0.880000in}}{\pgfqpoint{6.160000in}{6.160000in}}%
\pgfusepath{clip}%
\pgfsetbuttcap%
\pgfsetroundjoin%
\definecolor{currentfill}{rgb}{0.867428,0.864377,0.862602}%
\pgfsetfillcolor{currentfill}%
\pgfsetlinewidth{0.000000pt}%
\definecolor{currentstroke}{rgb}{0.000000,0.000000,0.000000}%
\pgfsetstrokecolor{currentstroke}%
\pgfsetdash{}{0pt}%
\pgfpathmoveto{\pgfqpoint{3.851262in}{3.636668in}}%
\pgfpathlineto{\pgfqpoint{3.860923in}{3.536426in}}%
\pgfpathlineto{\pgfqpoint{3.869721in}{3.712979in}}%
\pgfpathlineto{\pgfqpoint{3.905249in}{3.504227in}}%
\pgfpathlineto{\pgfqpoint{3.940161in}{3.480026in}}%
\pgfpathlineto{\pgfqpoint{3.930189in}{3.700607in}}%
\pgfpathlineto{\pgfqpoint{3.920833in}{3.677712in}}%
\pgfpathlineto{\pgfqpoint{3.885985in}{3.679454in}}%
\pgfpathlineto{\pgfqpoint{3.851262in}{3.636668in}}%
\pgfpathclose%
\pgfusepath{fill}%
\end{pgfscope}%
\begin{pgfscope}%
\pgfpathrectangle{\pgfqpoint{1.020000in}{0.880000in}}{\pgfqpoint{6.160000in}{6.160000in}}%
\pgfusepath{clip}%
\pgfsetbuttcap%
\pgfsetroundjoin%
\definecolor{currentfill}{rgb}{0.708720,0.805721,0.981117}%
\pgfsetfillcolor{currentfill}%
\pgfsetlinewidth{0.000000pt}%
\definecolor{currentstroke}{rgb}{0.000000,0.000000,0.000000}%
\pgfsetstrokecolor{currentstroke}%
\pgfsetdash{}{0pt}%
\pgfpathmoveto{\pgfqpoint{4.257031in}{3.300303in}}%
\pgfpathlineto{\pgfqpoint{4.266864in}{3.352388in}}%
\pgfpathlineto{\pgfqpoint{4.276429in}{3.084375in}}%
\pgfpathlineto{\pgfqpoint{4.311272in}{3.173862in}}%
\pgfpathlineto{\pgfqpoint{4.346296in}{3.350000in}}%
\pgfpathlineto{\pgfqpoint{4.336265in}{3.255596in}}%
\pgfpathlineto{\pgfqpoint{4.326358in}{3.219801in}}%
\pgfpathlineto{\pgfqpoint{4.291826in}{3.358486in}}%
\pgfpathlineto{\pgfqpoint{4.257031in}{3.300303in}}%
\pgfpathclose%
\pgfusepath{fill}%
\end{pgfscope}%
\begin{pgfscope}%
\pgfpathrectangle{\pgfqpoint{1.020000in}{0.880000in}}{\pgfqpoint{6.160000in}{6.160000in}}%
\pgfusepath{clip}%
\pgfsetbuttcap%
\pgfsetroundjoin%
\definecolor{currentfill}{rgb}{0.887752,0.854040,0.834671}%
\pgfsetfillcolor{currentfill}%
\pgfsetlinewidth{0.000000pt}%
\definecolor{currentstroke}{rgb}{0.000000,0.000000,0.000000}%
\pgfsetstrokecolor{currentstroke}%
\pgfsetdash{}{0pt}%
\pgfpathmoveto{\pgfqpoint{3.624085in}{3.535737in}}%
\pgfpathlineto{\pgfqpoint{3.632517in}{3.652735in}}%
\pgfpathlineto{\pgfqpoint{3.641658in}{3.646049in}}%
\pgfpathlineto{\pgfqpoint{3.676548in}{3.664346in}}%
\pgfpathlineto{\pgfqpoint{3.711868in}{3.589343in}}%
\pgfpathlineto{\pgfqpoint{3.702157in}{3.702022in}}%
\pgfpathlineto{\pgfqpoint{3.692540in}{3.795249in}}%
\pgfpathlineto{\pgfqpoint{3.658050in}{3.709077in}}%
\pgfpathlineto{\pgfqpoint{3.624085in}{3.535737in}}%
\pgfpathclose%
\pgfusepath{fill}%
\end{pgfscope}%
\begin{pgfscope}%
\pgfpathrectangle{\pgfqpoint{1.020000in}{0.880000in}}{\pgfqpoint{6.160000in}{6.160000in}}%
\pgfusepath{clip}%
\pgfsetbuttcap%
\pgfsetroundjoin%
\definecolor{currentfill}{rgb}{0.419991,0.552989,0.942630}%
\pgfsetfillcolor{currentfill}%
\pgfsetlinewidth{0.000000pt}%
\definecolor{currentstroke}{rgb}{0.000000,0.000000,0.000000}%
\pgfsetstrokecolor{currentstroke}%
\pgfsetdash{}{0pt}%
\pgfpathmoveto{\pgfqpoint{4.682020in}{2.676647in}}%
\pgfpathlineto{\pgfqpoint{4.693010in}{2.819133in}}%
\pgfpathlineto{\pgfqpoint{4.702564in}{2.677267in}}%
\pgfpathlineto{\pgfqpoint{4.737758in}{2.773868in}}%
\pgfpathlineto{\pgfqpoint{4.771284in}{2.581656in}}%
\pgfpathlineto{\pgfqpoint{4.761062in}{2.602336in}}%
\pgfpathlineto{\pgfqpoint{4.752182in}{2.854374in}}%
\pgfpathlineto{\pgfqpoint{4.717045in}{2.758849in}}%
\pgfpathlineto{\pgfqpoint{4.682020in}{2.676647in}}%
\pgfpathclose%
\pgfusepath{fill}%
\end{pgfscope}%
\begin{pgfscope}%
\pgfpathrectangle{\pgfqpoint{1.020000in}{0.880000in}}{\pgfqpoint{6.160000in}{6.160000in}}%
\pgfusepath{clip}%
\pgfsetbuttcap%
\pgfsetroundjoin%
\definecolor{currentfill}{rgb}{0.304174,0.406945,0.845263}%
\pgfsetfillcolor{currentfill}%
\pgfsetlinewidth{0.000000pt}%
\definecolor{currentstroke}{rgb}{0.000000,0.000000,0.000000}%
\pgfsetstrokecolor{currentstroke}%
\pgfsetdash{}{0pt}%
\pgfpathmoveto{\pgfqpoint{5.159079in}{2.578483in}}%
\pgfpathlineto{\pgfqpoint{5.168708in}{2.459418in}}%
\pgfpathlineto{\pgfqpoint{5.202810in}{2.421108in}}%
\pgfpathlineto{\pgfqpoint{5.237996in}{2.489136in}}%
\pgfpathlineto{\pgfqpoint{5.226464in}{2.426790in}}%
\pgfpathlineto{\pgfqpoint{5.192986in}{2.520033in}}%
\pgfpathlineto{\pgfqpoint{5.159079in}{2.578483in}}%
\pgfpathclose%
\pgfusepath{fill}%
\end{pgfscope}%
\begin{pgfscope}%
\pgfpathrectangle{\pgfqpoint{1.020000in}{0.880000in}}{\pgfqpoint{6.160000in}{6.160000in}}%
\pgfusepath{clip}%
\pgfsetbuttcap%
\pgfsetroundjoin%
\definecolor{currentfill}{rgb}{0.565182,0.699438,0.996635}%
\pgfsetfillcolor{currentfill}%
\pgfsetlinewidth{0.000000pt}%
\definecolor{currentstroke}{rgb}{0.000000,0.000000,0.000000}%
\pgfsetstrokecolor{currentstroke}%
\pgfsetdash{}{0pt}%
\pgfpathmoveto{\pgfqpoint{4.435003in}{3.030499in}}%
\pgfpathlineto{\pgfqpoint{4.445213in}{3.113274in}}%
\pgfpathlineto{\pgfqpoint{4.454333in}{2.782300in}}%
\pgfpathlineto{\pgfqpoint{4.489078in}{2.804047in}}%
\pgfpathlineto{\pgfqpoint{4.524518in}{3.030609in}}%
\pgfpathlineto{\pgfqpoint{4.514765in}{3.130630in}}%
\pgfpathlineto{\pgfqpoint{4.504186in}{2.973772in}}%
\pgfpathlineto{\pgfqpoint{4.469719in}{3.037981in}}%
\pgfpathlineto{\pgfqpoint{4.435003in}{3.030499in}}%
\pgfpathclose%
\pgfusepath{fill}%
\end{pgfscope}%
\begin{pgfscope}%
\pgfpathrectangle{\pgfqpoint{1.020000in}{0.880000in}}{\pgfqpoint{6.160000in}{6.160000in}}%
\pgfusepath{clip}%
\pgfsetbuttcap%
\pgfsetroundjoin%
\definecolor{currentfill}{rgb}{0.338377,0.452819,0.879317}%
\pgfsetfillcolor{currentfill}%
\pgfsetlinewidth{0.000000pt}%
\definecolor{currentstroke}{rgb}{0.000000,0.000000,0.000000}%
\pgfsetstrokecolor{currentstroke}%
\pgfsetdash{}{0pt}%
\pgfpathmoveto{\pgfqpoint{5.000211in}{2.655019in}}%
\pgfpathlineto{\pgfqpoint{5.009911in}{2.543123in}}%
\pgfpathlineto{\pgfqpoint{5.020932in}{2.587748in}}%
\pgfpathlineto{\pgfqpoint{5.056186in}{2.662988in}}%
\pgfpathlineto{\pgfqpoint{5.089402in}{2.512036in}}%
\pgfpathlineto{\pgfqpoint{5.078303in}{2.469210in}}%
\pgfpathlineto{\pgfqpoint{5.067826in}{2.492367in}}%
\pgfpathlineto{\pgfqpoint{5.033210in}{2.474901in}}%
\pgfpathlineto{\pgfqpoint{5.000211in}{2.655019in}}%
\pgfpathclose%
\pgfusepath{fill}%
\end{pgfscope}%
\begin{pgfscope}%
\pgfpathrectangle{\pgfqpoint{1.020000in}{0.880000in}}{\pgfqpoint{6.160000in}{6.160000in}}%
\pgfusepath{clip}%
\pgfsetbuttcap%
\pgfsetroundjoin%
\definecolor{currentfill}{rgb}{0.608547,0.735725,0.999354}%
\pgfsetfillcolor{currentfill}%
\pgfsetlinewidth{0.000000pt}%
\definecolor{currentstroke}{rgb}{0.000000,0.000000,0.000000}%
\pgfsetstrokecolor{currentstroke}%
\pgfsetdash{}{0pt}%
\pgfpathmoveto{\pgfqpoint{3.154496in}{3.173712in}}%
\pgfpathlineto{\pgfqpoint{3.164408in}{3.043157in}}%
\pgfpathlineto{\pgfqpoint{3.173064in}{3.036010in}}%
\pgfpathlineto{\pgfqpoint{3.208207in}{3.036214in}}%
\pgfpathlineto{\pgfqpoint{3.243158in}{3.054448in}}%
\pgfpathlineto{\pgfqpoint{3.233821in}{3.125704in}}%
\pgfpathlineto{\pgfqpoint{3.226430in}{2.995254in}}%
\pgfpathlineto{\pgfqpoint{3.190541in}{3.080877in}}%
\pgfpathlineto{\pgfqpoint{3.154496in}{3.173712in}}%
\pgfpathclose%
\pgfusepath{fill}%
\end{pgfscope}%
\begin{pgfscope}%
\pgfpathrectangle{\pgfqpoint{1.020000in}{0.880000in}}{\pgfqpoint{6.160000in}{6.160000in}}%
\pgfusepath{clip}%
\pgfsetbuttcap%
\pgfsetroundjoin%
\definecolor{currentfill}{rgb}{0.796064,0.848693,0.933471}%
\pgfsetfillcolor{currentfill}%
\pgfsetlinewidth{0.000000pt}%
\definecolor{currentstroke}{rgb}{0.000000,0.000000,0.000000}%
\pgfsetstrokecolor{currentstroke}%
\pgfsetdash{}{0pt}%
\pgfpathmoveto{\pgfqpoint{4.098623in}{3.343674in}}%
\pgfpathlineto{\pgfqpoint{4.108058in}{3.574458in}}%
\pgfpathlineto{\pgfqpoint{4.117846in}{3.366789in}}%
\pgfpathlineto{\pgfqpoint{4.152621in}{3.580313in}}%
\pgfpathlineto{\pgfqpoint{4.187510in}{3.468279in}}%
\pgfpathlineto{\pgfqpoint{4.177797in}{3.481643in}}%
\pgfpathlineto{\pgfqpoint{4.168134in}{3.283663in}}%
\pgfpathlineto{\pgfqpoint{4.133377in}{3.359626in}}%
\pgfpathlineto{\pgfqpoint{4.098623in}{3.343674in}}%
\pgfpathclose%
\pgfusepath{fill}%
\end{pgfscope}%
\begin{pgfscope}%
\pgfpathrectangle{\pgfqpoint{1.020000in}{0.880000in}}{\pgfqpoint{6.160000in}{6.160000in}}%
\pgfusepath{clip}%
\pgfsetbuttcap%
\pgfsetroundjoin%
\definecolor{currentfill}{rgb}{0.500031,0.638508,0.981070}%
\pgfsetfillcolor{currentfill}%
\pgfsetlinewidth{0.000000pt}%
\definecolor{currentstroke}{rgb}{0.000000,0.000000,0.000000}%
\pgfsetstrokecolor{currentstroke}%
\pgfsetdash{}{0pt}%
\pgfpathmoveto{\pgfqpoint{4.524518in}{3.030609in}}%
\pgfpathlineto{\pgfqpoint{4.533588in}{2.736166in}}%
\pgfpathlineto{\pgfqpoint{4.544351in}{2.918839in}}%
\pgfpathlineto{\pgfqpoint{4.578620in}{2.796894in}}%
\pgfpathlineto{\pgfqpoint{4.613337in}{2.803096in}}%
\pgfpathlineto{\pgfqpoint{4.603080in}{2.784717in}}%
\pgfpathlineto{\pgfqpoint{4.592843in}{2.764541in}}%
\pgfpathlineto{\pgfqpoint{4.559218in}{3.024823in}}%
\pgfpathlineto{\pgfqpoint{4.524518in}{3.030609in}}%
\pgfpathclose%
\pgfusepath{fill}%
\end{pgfscope}%
\begin{pgfscope}%
\pgfpathrectangle{\pgfqpoint{1.020000in}{0.880000in}}{\pgfqpoint{6.160000in}{6.160000in}}%
\pgfusepath{clip}%
\pgfsetbuttcap%
\pgfsetroundjoin%
\definecolor{currentfill}{rgb}{0.839351,0.861167,0.894494}%
\pgfsetfillcolor{currentfill}%
\pgfsetlinewidth{0.000000pt}%
\definecolor{currentstroke}{rgb}{0.000000,0.000000,0.000000}%
\pgfsetstrokecolor{currentstroke}%
\pgfsetdash{}{0pt}%
\pgfpathmoveto{\pgfqpoint{3.940161in}{3.480026in}}%
\pgfpathlineto{\pgfqpoint{3.949204in}{3.648845in}}%
\pgfpathlineto{\pgfqpoint{3.958978in}{3.514746in}}%
\pgfpathlineto{\pgfqpoint{3.994122in}{3.389808in}}%
\pgfpathlineto{\pgfqpoint{4.028784in}{3.506140in}}%
\pgfpathlineto{\pgfqpoint{4.019187in}{3.546519in}}%
\pgfpathlineto{\pgfqpoint{4.009554in}{3.615034in}}%
\pgfpathlineto{\pgfqpoint{3.974798in}{3.570043in}}%
\pgfpathlineto{\pgfqpoint{3.940161in}{3.480026in}}%
\pgfpathclose%
\pgfusepath{fill}%
\end{pgfscope}%
\begin{pgfscope}%
\pgfpathrectangle{\pgfqpoint{1.020000in}{0.880000in}}{\pgfqpoint{6.160000in}{6.160000in}}%
\pgfusepath{clip}%
\pgfsetbuttcap%
\pgfsetroundjoin%
\definecolor{currentfill}{rgb}{0.394042,0.522413,0.924916}%
\pgfsetfillcolor{currentfill}%
\pgfsetlinewidth{0.000000pt}%
\definecolor{currentstroke}{rgb}{0.000000,0.000000,0.000000}%
\pgfsetstrokecolor{currentstroke}%
\pgfsetdash{}{0pt}%
\pgfpathmoveto{\pgfqpoint{4.841628in}{2.764207in}}%
\pgfpathlineto{\pgfqpoint{4.851420in}{2.665618in}}%
\pgfpathlineto{\pgfqpoint{4.861384in}{2.591811in}}%
\pgfpathlineto{\pgfqpoint{4.896347in}{2.640980in}}%
\pgfpathlineto{\pgfqpoint{4.931892in}{2.765782in}}%
\pgfpathlineto{\pgfqpoint{4.920230in}{2.616166in}}%
\pgfpathlineto{\pgfqpoint{4.910138in}{2.673502in}}%
\pgfpathlineto{\pgfqpoint{4.874979in}{2.585394in}}%
\pgfpathlineto{\pgfqpoint{4.841628in}{2.764207in}}%
\pgfpathclose%
\pgfusepath{fill}%
\end{pgfscope}%
\begin{pgfscope}%
\pgfpathrectangle{\pgfqpoint{1.020000in}{0.880000in}}{\pgfqpoint{6.160000in}{6.160000in}}%
\pgfusepath{clip}%
\pgfsetbuttcap%
\pgfsetroundjoin%
\definecolor{currentfill}{rgb}{0.672538,0.782861,0.991982}%
\pgfsetfillcolor{currentfill}%
\pgfsetlinewidth{0.000000pt}%
\definecolor{currentstroke}{rgb}{0.000000,0.000000,0.000000}%
\pgfsetstrokecolor{currentstroke}%
\pgfsetdash{}{0pt}%
\pgfpathmoveto{\pgfqpoint{4.346296in}{3.350000in}}%
\pgfpathlineto{\pgfqpoint{4.355847in}{3.142956in}}%
\pgfpathlineto{\pgfqpoint{4.365858in}{3.198123in}}%
\pgfpathlineto{\pgfqpoint{4.400486in}{3.121431in}}%
\pgfpathlineto{\pgfqpoint{4.435003in}{3.030499in}}%
\pgfpathlineto{\pgfqpoint{4.425629in}{3.275748in}}%
\pgfpathlineto{\pgfqpoint{4.415458in}{3.190636in}}%
\pgfpathlineto{\pgfqpoint{4.380640in}{3.131107in}}%
\pgfpathlineto{\pgfqpoint{4.346296in}{3.350000in}}%
\pgfpathclose%
\pgfusepath{fill}%
\end{pgfscope}%
\begin{pgfscope}%
\pgfpathrectangle{\pgfqpoint{1.020000in}{0.880000in}}{\pgfqpoint{6.160000in}{6.160000in}}%
\pgfusepath{clip}%
\pgfsetbuttcap%
\pgfsetroundjoin%
\definecolor{currentfill}{rgb}{0.651398,0.768121,0.995891}%
\pgfsetfillcolor{currentfill}%
\pgfsetlinewidth{0.000000pt}%
\definecolor{currentstroke}{rgb}{0.000000,0.000000,0.000000}%
\pgfsetstrokecolor{currentstroke}%
\pgfsetdash{}{0pt}%
\pgfpathmoveto{\pgfqpoint{3.226430in}{2.995254in}}%
\pgfpathlineto{\pgfqpoint{3.233821in}{3.125704in}}%
\pgfpathlineto{\pgfqpoint{3.243158in}{3.054448in}}%
\pgfpathlineto{\pgfqpoint{3.277621in}{3.125382in}}%
\pgfpathlineto{\pgfqpoint{3.312756in}{3.122611in}}%
\pgfpathlineto{\pgfqpoint{3.302897in}{3.249893in}}%
\pgfpathlineto{\pgfqpoint{3.293977in}{3.271459in}}%
\pgfpathlineto{\pgfqpoint{3.259777in}{3.175319in}}%
\pgfpathlineto{\pgfqpoint{3.226430in}{2.995254in}}%
\pgfpathclose%
\pgfusepath{fill}%
\end{pgfscope}%
\begin{pgfscope}%
\pgfpathrectangle{\pgfqpoint{1.020000in}{0.880000in}}{\pgfqpoint{6.160000in}{6.160000in}}%
\pgfusepath{clip}%
\pgfsetbuttcap%
\pgfsetroundjoin%
\definecolor{currentfill}{rgb}{0.883687,0.856108,0.840258}%
\pgfsetfillcolor{currentfill}%
\pgfsetlinewidth{0.000000pt}%
\definecolor{currentstroke}{rgb}{0.000000,0.000000,0.000000}%
\pgfsetstrokecolor{currentstroke}%
\pgfsetdash{}{0pt}%
\pgfpathmoveto{\pgfqpoint{3.781271in}{3.693056in}}%
\pgfpathlineto{\pgfqpoint{3.790704in}{3.651863in}}%
\pgfpathlineto{\pgfqpoint{3.799922in}{3.670381in}}%
\pgfpathlineto{\pgfqpoint{3.835043in}{3.628784in}}%
\pgfpathlineto{\pgfqpoint{3.869721in}{3.712979in}}%
\pgfpathlineto{\pgfqpoint{3.860923in}{3.536426in}}%
\pgfpathlineto{\pgfqpoint{3.851262in}{3.636668in}}%
\pgfpathlineto{\pgfqpoint{3.816316in}{3.656957in}}%
\pgfpathlineto{\pgfqpoint{3.781271in}{3.693056in}}%
\pgfpathclose%
\pgfusepath{fill}%
\end{pgfscope}%
\begin{pgfscope}%
\pgfpathrectangle{\pgfqpoint{1.020000in}{0.880000in}}{\pgfqpoint{6.160000in}{6.160000in}}%
\pgfusepath{clip}%
\pgfsetbuttcap%
\pgfsetroundjoin%
\definecolor{currentfill}{rgb}{0.378598,0.503856,0.913692}%
\pgfsetfillcolor{currentfill}%
\pgfsetlinewidth{0.000000pt}%
\definecolor{currentstroke}{rgb}{0.000000,0.000000,0.000000}%
\pgfsetstrokecolor{currentstroke}%
\pgfsetdash{}{0pt}%
\pgfpathmoveto{\pgfqpoint{4.771284in}{2.581656in}}%
\pgfpathlineto{\pgfqpoint{4.782511in}{2.724347in}}%
\pgfpathlineto{\pgfqpoint{4.791727in}{2.530701in}}%
\pgfpathlineto{\pgfqpoint{4.826623in}{2.571641in}}%
\pgfpathlineto{\pgfqpoint{4.861384in}{2.591811in}}%
\pgfpathlineto{\pgfqpoint{4.851420in}{2.665618in}}%
\pgfpathlineto{\pgfqpoint{4.841628in}{2.764207in}}%
\pgfpathlineto{\pgfqpoint{4.805852in}{2.580073in}}%
\pgfpathlineto{\pgfqpoint{4.771284in}{2.581656in}}%
\pgfpathclose%
\pgfusepath{fill}%
\end{pgfscope}%
\begin{pgfscope}%
\pgfpathrectangle{\pgfqpoint{1.020000in}{0.880000in}}{\pgfqpoint{6.160000in}{6.160000in}}%
\pgfusepath{clip}%
\pgfsetbuttcap%
\pgfsetroundjoin%
\definecolor{currentfill}{rgb}{0.597777,0.727330,0.999777}%
\pgfsetfillcolor{currentfill}%
\pgfsetlinewidth{0.000000pt}%
\definecolor{currentstroke}{rgb}{0.000000,0.000000,0.000000}%
\pgfsetstrokecolor{currentstroke}%
\pgfsetdash{}{0pt}%
\pgfpathmoveto{\pgfqpoint{3.085873in}{3.020943in}}%
\pgfpathlineto{\pgfqpoint{3.094385in}{3.018974in}}%
\pgfpathlineto{\pgfqpoint{3.103091in}{3.000819in}}%
\pgfpathlineto{\pgfqpoint{3.139082in}{2.923202in}}%
\pgfpathlineto{\pgfqpoint{3.173064in}{3.036010in}}%
\pgfpathlineto{\pgfqpoint{3.164408in}{3.043157in}}%
\pgfpathlineto{\pgfqpoint{3.154496in}{3.173712in}}%
\pgfpathlineto{\pgfqpoint{3.120148in}{3.099267in}}%
\pgfpathlineto{\pgfqpoint{3.085873in}{3.020943in}}%
\pgfpathclose%
\pgfusepath{fill}%
\end{pgfscope}%
\begin{pgfscope}%
\pgfpathrectangle{\pgfqpoint{1.020000in}{0.880000in}}{\pgfqpoint{6.160000in}{6.160000in}}%
\pgfusepath{clip}%
\pgfsetbuttcap%
\pgfsetroundjoin%
\definecolor{currentfill}{rgb}{0.733898,0.820018,0.970724}%
\pgfsetfillcolor{currentfill}%
\pgfsetlinewidth{0.000000pt}%
\definecolor{currentstroke}{rgb}{0.000000,0.000000,0.000000}%
\pgfsetstrokecolor{currentstroke}%
\pgfsetdash{}{0pt}%
\pgfpathmoveto{\pgfqpoint{4.187510in}{3.468279in}}%
\pgfpathlineto{\pgfqpoint{4.197222in}{3.309139in}}%
\pgfpathlineto{\pgfqpoint{4.206940in}{3.212523in}}%
\pgfpathlineto{\pgfqpoint{4.241788in}{3.267608in}}%
\pgfpathlineto{\pgfqpoint{4.276429in}{3.084375in}}%
\pgfpathlineto{\pgfqpoint{4.266864in}{3.352388in}}%
\pgfpathlineto{\pgfqpoint{4.257031in}{3.300303in}}%
\pgfpathlineto{\pgfqpoint{4.222332in}{3.437311in}}%
\pgfpathlineto{\pgfqpoint{4.187510in}{3.468279in}}%
\pgfpathclose%
\pgfusepath{fill}%
\end{pgfscope}%
\begin{pgfscope}%
\pgfpathrectangle{\pgfqpoint{1.020000in}{0.880000in}}{\pgfqpoint{6.160000in}{6.160000in}}%
\pgfusepath{clip}%
\pgfsetbuttcap%
\pgfsetroundjoin%
\definecolor{currentfill}{rgb}{0.313946,0.420052,0.854993}%
\pgfsetfillcolor{currentfill}%
\pgfsetlinewidth{0.000000pt}%
\definecolor{currentstroke}{rgb}{0.000000,0.000000,0.000000}%
\pgfsetstrokecolor{currentstroke}%
\pgfsetdash{}{0pt}%
\pgfpathmoveto{\pgfqpoint{5.089402in}{2.512036in}}%
\pgfpathlineto{\pgfqpoint{5.099300in}{2.421413in}}%
\pgfpathlineto{\pgfqpoint{5.135049in}{2.548803in}}%
\pgfpathlineto{\pgfqpoint{5.168708in}{2.459418in}}%
\pgfpathlineto{\pgfqpoint{5.159079in}{2.578483in}}%
\pgfpathlineto{\pgfqpoint{5.123013in}{2.415979in}}%
\pgfpathlineto{\pgfqpoint{5.089402in}{2.512036in}}%
\pgfpathclose%
\pgfusepath{fill}%
\end{pgfscope}%
\begin{pgfscope}%
\pgfpathrectangle{\pgfqpoint{1.020000in}{0.880000in}}{\pgfqpoint{6.160000in}{6.160000in}}%
\pgfusepath{clip}%
\pgfsetbuttcap%
\pgfsetroundjoin%
\definecolor{currentfill}{rgb}{0.570616,0.704109,0.997195}%
\pgfsetfillcolor{currentfill}%
\pgfsetlinewidth{0.000000pt}%
\definecolor{currentstroke}{rgb}{0.000000,0.000000,0.000000}%
\pgfsetstrokecolor{currentstroke}%
\pgfsetdash{}{0pt}%
\pgfpathmoveto{\pgfqpoint{3.017824in}{2.830205in}}%
\pgfpathlineto{\pgfqpoint{3.025109in}{2.926065in}}%
\pgfpathlineto{\pgfqpoint{3.033204in}{2.954911in}}%
\pgfpathlineto{\pgfqpoint{3.066826in}{3.096056in}}%
\pgfpathlineto{\pgfqpoint{3.103091in}{3.000819in}}%
\pgfpathlineto{\pgfqpoint{3.094385in}{3.018974in}}%
\pgfpathlineto{\pgfqpoint{3.085873in}{3.020943in}}%
\pgfpathlineto{\pgfqpoint{3.050679in}{3.026359in}}%
\pgfpathlineto{\pgfqpoint{3.017824in}{2.830205in}}%
\pgfpathclose%
\pgfusepath{fill}%
\end{pgfscope}%
\begin{pgfscope}%
\pgfpathrectangle{\pgfqpoint{1.020000in}{0.880000in}}{\pgfqpoint{6.160000in}{6.160000in}}%
\pgfusepath{clip}%
\pgfsetbuttcap%
\pgfsetroundjoin%
\definecolor{currentfill}{rgb}{0.441123,0.576532,0.954545}%
\pgfsetfillcolor{currentfill}%
\pgfsetlinewidth{0.000000pt}%
\definecolor{currentstroke}{rgb}{0.000000,0.000000,0.000000}%
\pgfsetstrokecolor{currentstroke}%
\pgfsetdash{}{0pt}%
\pgfpathmoveto{\pgfqpoint{4.613337in}{2.803096in}}%
\pgfpathlineto{\pgfqpoint{4.623655in}{2.829449in}}%
\pgfpathlineto{\pgfqpoint{4.633282in}{2.697311in}}%
\pgfpathlineto{\pgfqpoint{4.667692in}{2.637004in}}%
\pgfpathlineto{\pgfqpoint{4.702564in}{2.677267in}}%
\pgfpathlineto{\pgfqpoint{4.693010in}{2.819133in}}%
\pgfpathlineto{\pgfqpoint{4.682020in}{2.676647in}}%
\pgfpathlineto{\pgfqpoint{4.647933in}{2.787367in}}%
\pgfpathlineto{\pgfqpoint{4.613337in}{2.803096in}}%
\pgfpathclose%
\pgfusepath{fill}%
\end{pgfscope}%
\begin{pgfscope}%
\pgfpathrectangle{\pgfqpoint{1.020000in}{0.880000in}}{\pgfqpoint{6.160000in}{6.160000in}}%
\pgfusepath{clip}%
\pgfsetbuttcap%
\pgfsetroundjoin%
\definecolor{currentfill}{rgb}{0.543440,0.680003,0.993051}%
\pgfsetfillcolor{currentfill}%
\pgfsetlinewidth{0.000000pt}%
\definecolor{currentstroke}{rgb}{0.000000,0.000000,0.000000}%
\pgfsetstrokecolor{currentstroke}%
\pgfsetdash{}{0pt}%
\pgfpathmoveto{\pgfqpoint{2.945712in}{2.980904in}}%
\pgfpathlineto{\pgfqpoint{2.955041in}{2.900876in}}%
\pgfpathlineto{\pgfqpoint{2.965485in}{2.730401in}}%
\pgfpathlineto{\pgfqpoint{2.996078in}{3.114538in}}%
\pgfpathlineto{\pgfqpoint{3.033204in}{2.954911in}}%
\pgfpathlineto{\pgfqpoint{3.025109in}{2.926065in}}%
\pgfpathlineto{\pgfqpoint{3.017824in}{2.830205in}}%
\pgfpathlineto{\pgfqpoint{2.979945in}{3.059582in}}%
\pgfpathlineto{\pgfqpoint{2.945712in}{2.980904in}}%
\pgfpathclose%
\pgfusepath{fill}%
\end{pgfscope}%
\begin{pgfscope}%
\pgfpathrectangle{\pgfqpoint{1.020000in}{0.880000in}}{\pgfqpoint{6.160000in}{6.160000in}}%
\pgfusepath{clip}%
\pgfsetbuttcap%
\pgfsetroundjoin%
\definecolor{currentfill}{rgb}{0.847365,0.862472,0.885540}%
\pgfsetfillcolor{currentfill}%
\pgfsetlinewidth{0.000000pt}%
\definecolor{currentstroke}{rgb}{0.000000,0.000000,0.000000}%
\pgfsetstrokecolor{currentstroke}%
\pgfsetdash{}{0pt}%
\pgfpathmoveto{\pgfqpoint{3.641658in}{3.646049in}}%
\pgfpathlineto{\pgfqpoint{3.651573in}{3.497143in}}%
\pgfpathlineto{\pgfqpoint{3.661413in}{3.361567in}}%
\pgfpathlineto{\pgfqpoint{3.695711in}{3.496429in}}%
\pgfpathlineto{\pgfqpoint{3.730008in}{3.647548in}}%
\pgfpathlineto{\pgfqpoint{3.721210in}{3.556622in}}%
\pgfpathlineto{\pgfqpoint{3.711868in}{3.589343in}}%
\pgfpathlineto{\pgfqpoint{3.676548in}{3.664346in}}%
\pgfpathlineto{\pgfqpoint{3.641658in}{3.646049in}}%
\pgfpathclose%
\pgfusepath{fill}%
\end{pgfscope}%
\begin{pgfscope}%
\pgfpathrectangle{\pgfqpoint{1.020000in}{0.880000in}}{\pgfqpoint{6.160000in}{6.160000in}}%
\pgfusepath{clip}%
\pgfsetbuttcap%
\pgfsetroundjoin%
\definecolor{currentfill}{rgb}{0.343278,0.459354,0.884122}%
\pgfsetfillcolor{currentfill}%
\pgfsetlinewidth{0.000000pt}%
\definecolor{currentstroke}{rgb}{0.000000,0.000000,0.000000}%
\pgfsetstrokecolor{currentstroke}%
\pgfsetdash{}{0pt}%
\pgfpathmoveto{\pgfqpoint{4.931892in}{2.765782in}}%
\pgfpathlineto{\pgfqpoint{4.940051in}{2.449829in}}%
\pgfpathlineto{\pgfqpoint{4.949978in}{2.367641in}}%
\pgfpathlineto{\pgfqpoint{4.985457in}{2.481327in}}%
\pgfpathlineto{\pgfqpoint{5.020932in}{2.587748in}}%
\pgfpathlineto{\pgfqpoint{5.009911in}{2.543123in}}%
\pgfpathlineto{\pgfqpoint{5.000211in}{2.655019in}}%
\pgfpathlineto{\pgfqpoint{4.964930in}{2.564407in}}%
\pgfpathlineto{\pgfqpoint{4.931892in}{2.765782in}}%
\pgfpathclose%
\pgfusepath{fill}%
\end{pgfscope}%
\begin{pgfscope}%
\pgfpathrectangle{\pgfqpoint{1.020000in}{0.880000in}}{\pgfqpoint{6.160000in}{6.160000in}}%
\pgfusepath{clip}%
\pgfsetbuttcap%
\pgfsetroundjoin%
\definecolor{currentfill}{rgb}{0.516260,0.654498,0.986407}%
\pgfsetfillcolor{currentfill}%
\pgfsetlinewidth{0.000000pt}%
\definecolor{currentstroke}{rgb}{0.000000,0.000000,0.000000}%
\pgfsetstrokecolor{currentstroke}%
\pgfsetdash{}{0pt}%
\pgfpathmoveto{\pgfqpoint{2.877422in}{2.817843in}}%
\pgfpathlineto{\pgfqpoint{2.884082in}{2.939962in}}%
\pgfpathlineto{\pgfqpoint{2.891863in}{2.978625in}}%
\pgfpathlineto{\pgfqpoint{2.928856in}{2.844424in}}%
\pgfpathlineto{\pgfqpoint{2.965485in}{2.730401in}}%
\pgfpathlineto{\pgfqpoint{2.955041in}{2.900876in}}%
\pgfpathlineto{\pgfqpoint{2.945712in}{2.980904in}}%
\pgfpathlineto{\pgfqpoint{2.912138in}{2.853346in}}%
\pgfpathlineto{\pgfqpoint{2.877422in}{2.817843in}}%
\pgfpathclose%
\pgfusepath{fill}%
\end{pgfscope}%
\begin{pgfscope}%
\pgfpathrectangle{\pgfqpoint{1.020000in}{0.880000in}}{\pgfqpoint{6.160000in}{6.160000in}}%
\pgfusepath{clip}%
\pgfsetbuttcap%
\pgfsetroundjoin%
\definecolor{currentfill}{rgb}{0.875557,0.860242,0.851430}%
\pgfsetfillcolor{currentfill}%
\pgfsetlinewidth{0.000000pt}%
\definecolor{currentstroke}{rgb}{0.000000,0.000000,0.000000}%
\pgfsetstrokecolor{currentstroke}%
\pgfsetdash{}{0pt}%
\pgfpathmoveto{\pgfqpoint{3.711868in}{3.589343in}}%
\pgfpathlineto{\pgfqpoint{3.721210in}{3.556622in}}%
\pgfpathlineto{\pgfqpoint{3.730008in}{3.647548in}}%
\pgfpathlineto{\pgfqpoint{3.765908in}{3.434747in}}%
\pgfpathlineto{\pgfqpoint{3.799922in}{3.670381in}}%
\pgfpathlineto{\pgfqpoint{3.790704in}{3.651863in}}%
\pgfpathlineto{\pgfqpoint{3.781271in}{3.693056in}}%
\pgfpathlineto{\pgfqpoint{3.746043in}{3.760233in}}%
\pgfpathlineto{\pgfqpoint{3.711868in}{3.589343in}}%
\pgfpathclose%
\pgfusepath{fill}%
\end{pgfscope}%
\begin{pgfscope}%
\pgfpathrectangle{\pgfqpoint{1.020000in}{0.880000in}}{\pgfqpoint{6.160000in}{6.160000in}}%
\pgfusepath{clip}%
\pgfsetbuttcap%
\pgfsetroundjoin%
\definecolor{currentfill}{rgb}{0.667253,0.779176,0.992959}%
\pgfsetfillcolor{currentfill}%
\pgfsetlinewidth{0.000000pt}%
\definecolor{currentstroke}{rgb}{0.000000,0.000000,0.000000}%
\pgfsetstrokecolor{currentstroke}%
\pgfsetdash{}{0pt}%
\pgfpathmoveto{\pgfqpoint{4.276429in}{3.084375in}}%
\pgfpathlineto{\pgfqpoint{4.286380in}{3.219380in}}%
\pgfpathlineto{\pgfqpoint{4.295981in}{3.002162in}}%
\pgfpathlineto{\pgfqpoint{4.331052in}{3.205291in}}%
\pgfpathlineto{\pgfqpoint{4.365858in}{3.198123in}}%
\pgfpathlineto{\pgfqpoint{4.355847in}{3.142956in}}%
\pgfpathlineto{\pgfqpoint{4.346296in}{3.350000in}}%
\pgfpathlineto{\pgfqpoint{4.311272in}{3.173862in}}%
\pgfpathlineto{\pgfqpoint{4.276429in}{3.084375in}}%
\pgfpathclose%
\pgfusepath{fill}%
\end{pgfscope}%
\begin{pgfscope}%
\pgfpathrectangle{\pgfqpoint{1.020000in}{0.880000in}}{\pgfqpoint{6.160000in}{6.160000in}}%
\pgfusepath{clip}%
\pgfsetbuttcap%
\pgfsetroundjoin%
\definecolor{currentfill}{rgb}{0.826784,0.858205,0.906953}%
\pgfsetfillcolor{currentfill}%
\pgfsetlinewidth{0.000000pt}%
\definecolor{currentstroke}{rgb}{0.000000,0.000000,0.000000}%
\pgfsetstrokecolor{currentstroke}%
\pgfsetdash{}{0pt}%
\pgfpathmoveto{\pgfqpoint{3.571905in}{3.600522in}}%
\pgfpathlineto{\pgfqpoint{3.582039in}{3.420004in}}%
\pgfpathlineto{\pgfqpoint{3.591007in}{3.434318in}}%
\pgfpathlineto{\pgfqpoint{3.625033in}{3.613337in}}%
\pgfpathlineto{\pgfqpoint{3.661413in}{3.361567in}}%
\pgfpathlineto{\pgfqpoint{3.651573in}{3.497143in}}%
\pgfpathlineto{\pgfqpoint{3.641658in}{3.646049in}}%
\pgfpathlineto{\pgfqpoint{3.607756in}{3.454437in}}%
\pgfpathlineto{\pgfqpoint{3.571905in}{3.600522in}}%
\pgfpathclose%
\pgfusepath{fill}%
\end{pgfscope}%
\begin{pgfscope}%
\pgfpathrectangle{\pgfqpoint{1.020000in}{0.880000in}}{\pgfqpoint{6.160000in}{6.160000in}}%
\pgfusepath{clip}%
\pgfsetbuttcap%
\pgfsetroundjoin%
\definecolor{currentfill}{rgb}{0.713852,0.808857,0.979386}%
\pgfsetfillcolor{currentfill}%
\pgfsetlinewidth{0.000000pt}%
\definecolor{currentstroke}{rgb}{0.000000,0.000000,0.000000}%
\pgfsetstrokecolor{currentstroke}%
\pgfsetdash{}{0pt}%
\pgfpathmoveto{\pgfqpoint{3.293977in}{3.271459in}}%
\pgfpathlineto{\pgfqpoint{3.302897in}{3.249893in}}%
\pgfpathlineto{\pgfqpoint{3.312756in}{3.122611in}}%
\pgfpathlineto{\pgfqpoint{3.348593in}{3.033291in}}%
\pgfpathlineto{\pgfqpoint{3.382452in}{3.179687in}}%
\pgfpathlineto{\pgfqpoint{3.371487in}{3.444677in}}%
\pgfpathlineto{\pgfqpoint{3.363607in}{3.330438in}}%
\pgfpathlineto{\pgfqpoint{3.327787in}{3.418462in}}%
\pgfpathlineto{\pgfqpoint{3.293977in}{3.271459in}}%
\pgfpathclose%
\pgfusepath{fill}%
\end{pgfscope}%
\begin{pgfscope}%
\pgfpathrectangle{\pgfqpoint{1.020000in}{0.880000in}}{\pgfqpoint{6.160000in}{6.160000in}}%
\pgfusepath{clip}%
\pgfsetbuttcap%
\pgfsetroundjoin%
\definecolor{currentfill}{rgb}{0.451739,0.588181,0.960201}%
\pgfsetfillcolor{currentfill}%
\pgfsetlinewidth{0.000000pt}%
\definecolor{currentstroke}{rgb}{0.000000,0.000000,0.000000}%
\pgfsetstrokecolor{currentstroke}%
\pgfsetdash{}{0pt}%
\pgfpathmoveto{\pgfqpoint{2.739818in}{2.591770in}}%
\pgfpathlineto{\pgfqpoint{2.744288in}{2.841733in}}%
\pgfpathlineto{\pgfqpoint{2.754803in}{2.675838in}}%
\pgfpathlineto{\pgfqpoint{2.786768in}{2.911962in}}%
\pgfpathlineto{\pgfqpoint{2.822704in}{2.868576in}}%
\pgfpathlineto{\pgfqpoint{2.817229in}{2.670349in}}%
\pgfpathlineto{\pgfqpoint{2.808856in}{2.685163in}}%
\pgfpathlineto{\pgfqpoint{2.771358in}{2.847838in}}%
\pgfpathlineto{\pgfqpoint{2.739818in}{2.591770in}}%
\pgfpathclose%
\pgfusepath{fill}%
\end{pgfscope}%
\begin{pgfscope}%
\pgfpathrectangle{\pgfqpoint{1.020000in}{0.880000in}}{\pgfqpoint{6.160000in}{6.160000in}}%
\pgfusepath{clip}%
\pgfsetbuttcap%
\pgfsetroundjoin%
\definecolor{currentfill}{rgb}{0.743754,0.825125,0.965798}%
\pgfsetfillcolor{currentfill}%
\pgfsetlinewidth{0.000000pt}%
\definecolor{currentstroke}{rgb}{0.000000,0.000000,0.000000}%
\pgfsetstrokecolor{currentstroke}%
\pgfsetdash{}{0pt}%
\pgfpathmoveto{\pgfqpoint{3.363607in}{3.330438in}}%
\pgfpathlineto{\pgfqpoint{3.371487in}{3.444677in}}%
\pgfpathlineto{\pgfqpoint{3.382452in}{3.179687in}}%
\pgfpathlineto{\pgfqpoint{3.415244in}{3.475767in}}%
\pgfpathlineto{\pgfqpoint{3.452209in}{3.227579in}}%
\pgfpathlineto{\pgfqpoint{3.442568in}{3.328837in}}%
\pgfpathlineto{\pgfqpoint{3.432979in}{3.423307in}}%
\pgfpathlineto{\pgfqpoint{3.399885in}{3.173401in}}%
\pgfpathlineto{\pgfqpoint{3.363607in}{3.330438in}}%
\pgfpathclose%
\pgfusepath{fill}%
\end{pgfscope}%
\begin{pgfscope}%
\pgfpathrectangle{\pgfqpoint{1.020000in}{0.880000in}}{\pgfqpoint{6.160000in}{6.160000in}}%
\pgfusepath{clip}%
\pgfsetbuttcap%
\pgfsetroundjoin%
\definecolor{currentfill}{rgb}{0.782049,0.842864,0.942980}%
\pgfsetfillcolor{currentfill}%
\pgfsetlinewidth{0.000000pt}%
\definecolor{currentstroke}{rgb}{0.000000,0.000000,0.000000}%
\pgfsetstrokecolor{currentstroke}%
\pgfsetdash{}{0pt}%
\pgfpathmoveto{\pgfqpoint{3.432979in}{3.423307in}}%
\pgfpathlineto{\pgfqpoint{3.442568in}{3.328837in}}%
\pgfpathlineto{\pgfqpoint{3.452209in}{3.227579in}}%
\pgfpathlineto{\pgfqpoint{3.485828in}{3.431122in}}%
\pgfpathlineto{\pgfqpoint{3.520442in}{3.502577in}}%
\pgfpathlineto{\pgfqpoint{3.511673in}{3.470017in}}%
\pgfpathlineto{\pgfqpoint{3.503321in}{3.379931in}}%
\pgfpathlineto{\pgfqpoint{3.468260in}{3.388832in}}%
\pgfpathlineto{\pgfqpoint{3.432979in}{3.423307in}}%
\pgfpathclose%
\pgfusepath{fill}%
\end{pgfscope}%
\begin{pgfscope}%
\pgfpathrectangle{\pgfqpoint{1.020000in}{0.880000in}}{\pgfqpoint{6.160000in}{6.160000in}}%
\pgfusepath{clip}%
\pgfsetbuttcap%
\pgfsetroundjoin%
\definecolor{currentfill}{rgb}{0.804965,0.851666,0.926165}%
\pgfsetfillcolor{currentfill}%
\pgfsetlinewidth{0.000000pt}%
\definecolor{currentstroke}{rgb}{0.000000,0.000000,0.000000}%
\pgfsetstrokecolor{currentstroke}%
\pgfsetdash{}{0pt}%
\pgfpathmoveto{\pgfqpoint{4.028784in}{3.506140in}}%
\pgfpathlineto{\pgfqpoint{4.038497in}{3.395714in}}%
\pgfpathlineto{\pgfqpoint{4.048085in}{3.377148in}}%
\pgfpathlineto{\pgfqpoint{4.082840in}{3.508899in}}%
\pgfpathlineto{\pgfqpoint{4.117846in}{3.366789in}}%
\pgfpathlineto{\pgfqpoint{4.108058in}{3.574458in}}%
\pgfpathlineto{\pgfqpoint{4.098623in}{3.343674in}}%
\pgfpathlineto{\pgfqpoint{4.063640in}{3.511272in}}%
\pgfpathlineto{\pgfqpoint{4.028784in}{3.506140in}}%
\pgfpathclose%
\pgfusepath{fill}%
\end{pgfscope}%
\begin{pgfscope}%
\pgfpathrectangle{\pgfqpoint{1.020000in}{0.880000in}}{\pgfqpoint{6.160000in}{6.160000in}}%
\pgfusepath{clip}%
\pgfsetbuttcap%
\pgfsetroundjoin%
\definecolor{currentfill}{rgb}{0.813693,0.854282,0.918480}%
\pgfsetfillcolor{currentfill}%
\pgfsetlinewidth{0.000000pt}%
\definecolor{currentstroke}{rgb}{0.000000,0.000000,0.000000}%
\pgfsetstrokecolor{currentstroke}%
\pgfsetdash{}{0pt}%
\pgfpathmoveto{\pgfqpoint{3.503321in}{3.379931in}}%
\pgfpathlineto{\pgfqpoint{3.511673in}{3.470017in}}%
\pgfpathlineto{\pgfqpoint{3.520442in}{3.502577in}}%
\pgfpathlineto{\pgfqpoint{3.556084in}{3.414829in}}%
\pgfpathlineto{\pgfqpoint{3.591007in}{3.434318in}}%
\pgfpathlineto{\pgfqpoint{3.582039in}{3.420004in}}%
\pgfpathlineto{\pgfqpoint{3.571905in}{3.600522in}}%
\pgfpathlineto{\pgfqpoint{3.537553in}{3.495297in}}%
\pgfpathlineto{\pgfqpoint{3.503321in}{3.379931in}}%
\pgfpathclose%
\pgfusepath{fill}%
\end{pgfscope}%
\begin{pgfscope}%
\pgfpathrectangle{\pgfqpoint{1.020000in}{0.880000in}}{\pgfqpoint{6.160000in}{6.160000in}}%
\pgfusepath{clip}%
\pgfsetbuttcap%
\pgfsetroundjoin%
\definecolor{currentfill}{rgb}{0.500031,0.638508,0.981070}%
\pgfsetfillcolor{currentfill}%
\pgfsetlinewidth{0.000000pt}%
\definecolor{currentstroke}{rgb}{0.000000,0.000000,0.000000}%
\pgfsetstrokecolor{currentstroke}%
\pgfsetdash{}{0pt}%
\pgfpathmoveto{\pgfqpoint{2.808856in}{2.685163in}}%
\pgfpathlineto{\pgfqpoint{2.817229in}{2.670349in}}%
\pgfpathlineto{\pgfqpoint{2.822704in}{2.868576in}}%
\pgfpathlineto{\pgfqpoint{2.857649in}{2.895534in}}%
\pgfpathlineto{\pgfqpoint{2.891863in}{2.978625in}}%
\pgfpathlineto{\pgfqpoint{2.884082in}{2.939962in}}%
\pgfpathlineto{\pgfqpoint{2.877422in}{2.817843in}}%
\pgfpathlineto{\pgfqpoint{2.840373in}{2.955563in}}%
\pgfpathlineto{\pgfqpoint{2.808856in}{2.685163in}}%
\pgfpathclose%
\pgfusepath{fill}%
\end{pgfscope}%
\begin{pgfscope}%
\pgfpathrectangle{\pgfqpoint{1.020000in}{0.880000in}}{\pgfqpoint{6.160000in}{6.160000in}}%
\pgfusepath{clip}%
\pgfsetbuttcap%
\pgfsetroundjoin%
\definecolor{currentfill}{rgb}{0.516260,0.654498,0.986407}%
\pgfsetfillcolor{currentfill}%
\pgfsetlinewidth{0.000000pt}%
\definecolor{currentstroke}{rgb}{0.000000,0.000000,0.000000}%
\pgfsetstrokecolor{currentstroke}%
\pgfsetdash{}{0pt}%
\pgfpathmoveto{\pgfqpoint{4.454333in}{2.782300in}}%
\pgfpathlineto{\pgfqpoint{4.465003in}{3.015127in}}%
\pgfpathlineto{\pgfqpoint{4.474671in}{2.881965in}}%
\pgfpathlineto{\pgfqpoint{4.509474in}{2.887769in}}%
\pgfpathlineto{\pgfqpoint{4.544351in}{2.918839in}}%
\pgfpathlineto{\pgfqpoint{4.533588in}{2.736166in}}%
\pgfpathlineto{\pgfqpoint{4.524518in}{3.030609in}}%
\pgfpathlineto{\pgfqpoint{4.489078in}{2.804047in}}%
\pgfpathlineto{\pgfqpoint{4.454333in}{2.782300in}}%
\pgfpathclose%
\pgfusepath{fill}%
\end{pgfscope}%
\begin{pgfscope}%
\pgfpathrectangle{\pgfqpoint{1.020000in}{0.880000in}}{\pgfqpoint{6.160000in}{6.160000in}}%
\pgfusepath{clip}%
\pgfsetbuttcap%
\pgfsetroundjoin%
\definecolor{currentfill}{rgb}{0.603162,0.731527,0.999565}%
\pgfsetfillcolor{currentfill}%
\pgfsetlinewidth{0.000000pt}%
\definecolor{currentstroke}{rgb}{0.000000,0.000000,0.000000}%
\pgfsetstrokecolor{currentstroke}%
\pgfsetdash{}{0pt}%
\pgfpathmoveto{\pgfqpoint{4.365858in}{3.198123in}}%
\pgfpathlineto{\pgfqpoint{4.375603in}{3.098312in}}%
\pgfpathlineto{\pgfqpoint{4.385434in}{3.039697in}}%
\pgfpathlineto{\pgfqpoint{4.420099in}{2.978062in}}%
\pgfpathlineto{\pgfqpoint{4.454333in}{2.782300in}}%
\pgfpathlineto{\pgfqpoint{4.445213in}{3.113274in}}%
\pgfpathlineto{\pgfqpoint{4.435003in}{3.030499in}}%
\pgfpathlineto{\pgfqpoint{4.400486in}{3.121431in}}%
\pgfpathlineto{\pgfqpoint{4.365858in}{3.198123in}}%
\pgfpathclose%
\pgfusepath{fill}%
\end{pgfscope}%
\begin{pgfscope}%
\pgfpathrectangle{\pgfqpoint{1.020000in}{0.880000in}}{\pgfqpoint{6.160000in}{6.160000in}}%
\pgfusepath{clip}%
\pgfsetbuttcap%
\pgfsetroundjoin%
\definecolor{currentfill}{rgb}{0.855378,0.863778,0.876587}%
\pgfsetfillcolor{currentfill}%
\pgfsetlinewidth{0.000000pt}%
\definecolor{currentstroke}{rgb}{0.000000,0.000000,0.000000}%
\pgfsetstrokecolor{currentstroke}%
\pgfsetdash{}{0pt}%
\pgfpathmoveto{\pgfqpoint{3.869721in}{3.712979in}}%
\pgfpathlineto{\pgfqpoint{3.879204in}{3.681155in}}%
\pgfpathlineto{\pgfqpoint{3.889117in}{3.508214in}}%
\pgfpathlineto{\pgfqpoint{3.924096in}{3.496258in}}%
\pgfpathlineto{\pgfqpoint{3.958978in}{3.514746in}}%
\pgfpathlineto{\pgfqpoint{3.949204in}{3.648845in}}%
\pgfpathlineto{\pgfqpoint{3.940161in}{3.480026in}}%
\pgfpathlineto{\pgfqpoint{3.905249in}{3.504227in}}%
\pgfpathlineto{\pgfqpoint{3.869721in}{3.712979in}}%
\pgfpathclose%
\pgfusepath{fill}%
\end{pgfscope}%
\begin{pgfscope}%
\pgfpathrectangle{\pgfqpoint{1.020000in}{0.880000in}}{\pgfqpoint{6.160000in}{6.160000in}}%
\pgfusepath{clip}%
\pgfsetbuttcap%
\pgfsetroundjoin%
\definecolor{currentfill}{rgb}{0.796064,0.848693,0.933471}%
\pgfsetfillcolor{currentfill}%
\pgfsetlinewidth{0.000000pt}%
\definecolor{currentstroke}{rgb}{0.000000,0.000000,0.000000}%
\pgfsetstrokecolor{currentstroke}%
\pgfsetdash{}{0pt}%
\pgfpathmoveto{\pgfqpoint{3.958978in}{3.514746in}}%
\pgfpathlineto{\pgfqpoint{3.968676in}{3.413663in}}%
\pgfpathlineto{\pgfqpoint{3.978249in}{3.374967in}}%
\pgfpathlineto{\pgfqpoint{4.013145in}{3.395234in}}%
\pgfpathlineto{\pgfqpoint{4.048085in}{3.377148in}}%
\pgfpathlineto{\pgfqpoint{4.038497in}{3.395714in}}%
\pgfpathlineto{\pgfqpoint{4.028784in}{3.506140in}}%
\pgfpathlineto{\pgfqpoint{3.994122in}{3.389808in}}%
\pgfpathlineto{\pgfqpoint{3.958978in}{3.514746in}}%
\pgfpathclose%
\pgfusepath{fill}%
\end{pgfscope}%
\begin{pgfscope}%
\pgfpathrectangle{\pgfqpoint{1.020000in}{0.880000in}}{\pgfqpoint{6.160000in}{6.160000in}}%
\pgfusepath{clip}%
\pgfsetbuttcap%
\pgfsetroundjoin%
\definecolor{currentfill}{rgb}{0.348323,0.465711,0.888346}%
\pgfsetfillcolor{currentfill}%
\pgfsetlinewidth{0.000000pt}%
\definecolor{currentstroke}{rgb}{0.000000,0.000000,0.000000}%
\pgfsetstrokecolor{currentstroke}%
\pgfsetdash{}{0pt}%
\pgfpathmoveto{\pgfqpoint{4.861384in}{2.591811in}}%
\pgfpathlineto{\pgfqpoint{4.872336in}{2.659041in}}%
\pgfpathlineto{\pgfqpoint{4.880962in}{2.391038in}}%
\pgfpathlineto{\pgfqpoint{4.916702in}{2.543858in}}%
\pgfpathlineto{\pgfqpoint{4.949978in}{2.367641in}}%
\pgfpathlineto{\pgfqpoint{4.940051in}{2.449829in}}%
\pgfpathlineto{\pgfqpoint{4.931892in}{2.765782in}}%
\pgfpathlineto{\pgfqpoint{4.896347in}{2.640980in}}%
\pgfpathlineto{\pgfqpoint{4.861384in}{2.591811in}}%
\pgfpathclose%
\pgfusepath{fill}%
\end{pgfscope}%
\begin{pgfscope}%
\pgfpathrectangle{\pgfqpoint{1.020000in}{0.880000in}}{\pgfqpoint{6.160000in}{6.160000in}}%
\pgfusepath{clip}%
\pgfsetbuttcap%
\pgfsetroundjoin%
\definecolor{currentfill}{rgb}{0.677823,0.786546,0.991005}%
\pgfsetfillcolor{currentfill}%
\pgfsetlinewidth{0.000000pt}%
\definecolor{currentstroke}{rgb}{0.000000,0.000000,0.000000}%
\pgfsetstrokecolor{currentstroke}%
\pgfsetdash{}{0pt}%
\pgfpathmoveto{\pgfqpoint{4.206940in}{3.212523in}}%
\pgfpathlineto{\pgfqpoint{4.216716in}{3.283692in}}%
\pgfpathlineto{\pgfqpoint{4.226489in}{3.264426in}}%
\pgfpathlineto{\pgfqpoint{4.261260in}{3.092401in}}%
\pgfpathlineto{\pgfqpoint{4.295981in}{3.002162in}}%
\pgfpathlineto{\pgfqpoint{4.286380in}{3.219380in}}%
\pgfpathlineto{\pgfqpoint{4.276429in}{3.084375in}}%
\pgfpathlineto{\pgfqpoint{4.241788in}{3.267608in}}%
\pgfpathlineto{\pgfqpoint{4.206940in}{3.212523in}}%
\pgfpathclose%
\pgfusepath{fill}%
\end{pgfscope}%
\begin{pgfscope}%
\pgfpathrectangle{\pgfqpoint{1.020000in}{0.880000in}}{\pgfqpoint{6.160000in}{6.160000in}}%
\pgfusepath{clip}%
\pgfsetbuttcap%
\pgfsetroundjoin%
\definecolor{currentfill}{rgb}{0.409611,0.540759,0.935545}%
\pgfsetfillcolor{currentfill}%
\pgfsetlinewidth{0.000000pt}%
\definecolor{currentstroke}{rgb}{0.000000,0.000000,0.000000}%
\pgfsetstrokecolor{currentstroke}%
\pgfsetdash{}{0pt}%
\pgfpathmoveto{\pgfqpoint{4.702564in}{2.677267in}}%
\pgfpathlineto{\pgfqpoint{4.712998in}{2.701411in}}%
\pgfpathlineto{\pgfqpoint{4.723007in}{2.642175in}}%
\pgfpathlineto{\pgfqpoint{4.758483in}{2.773933in}}%
\pgfpathlineto{\pgfqpoint{4.791727in}{2.530701in}}%
\pgfpathlineto{\pgfqpoint{4.782511in}{2.724347in}}%
\pgfpathlineto{\pgfqpoint{4.771284in}{2.581656in}}%
\pgfpathlineto{\pgfqpoint{4.737758in}{2.773868in}}%
\pgfpathlineto{\pgfqpoint{4.702564in}{2.677267in}}%
\pgfpathclose%
\pgfusepath{fill}%
\end{pgfscope}%
\begin{pgfscope}%
\pgfpathrectangle{\pgfqpoint{1.020000in}{0.880000in}}{\pgfqpoint{6.160000in}{6.160000in}}%
\pgfusepath{clip}%
\pgfsetbuttcap%
\pgfsetroundjoin%
\definecolor{currentfill}{rgb}{0.318832,0.426605,0.859857}%
\pgfsetfillcolor{currentfill}%
\pgfsetlinewidth{0.000000pt}%
\definecolor{currentstroke}{rgb}{0.000000,0.000000,0.000000}%
\pgfsetstrokecolor{currentstroke}%
\pgfsetdash{}{0pt}%
\pgfpathmoveto{\pgfqpoint{4.949978in}{2.367641in}}%
\pgfpathlineto{\pgfqpoint{4.961538in}{2.492492in}}%
\pgfpathlineto{\pgfqpoint{4.995852in}{2.453167in}}%
\pgfpathlineto{\pgfqpoint{5.031051in}{2.522934in}}%
\pgfpathlineto{\pgfqpoint{5.020932in}{2.587748in}}%
\pgfpathlineto{\pgfqpoint{4.985457in}{2.481327in}}%
\pgfpathlineto{\pgfqpoint{4.949978in}{2.367641in}}%
\pgfpathclose%
\pgfusepath{fill}%
\end{pgfscope}%
\begin{pgfscope}%
\pgfpathrectangle{\pgfqpoint{1.020000in}{0.880000in}}{\pgfqpoint{6.160000in}{6.160000in}}%
\pgfusepath{clip}%
\pgfsetbuttcap%
\pgfsetroundjoin%
\definecolor{currentfill}{rgb}{0.772706,0.838978,0.949319}%
\pgfsetfillcolor{currentfill}%
\pgfsetlinewidth{0.000000pt}%
\definecolor{currentstroke}{rgb}{0.000000,0.000000,0.000000}%
\pgfsetstrokecolor{currentstroke}%
\pgfsetdash{}{0pt}%
\pgfpathmoveto{\pgfqpoint{4.117846in}{3.366789in}}%
\pgfpathlineto{\pgfqpoint{4.127500in}{3.363763in}}%
\pgfpathlineto{\pgfqpoint{4.137195in}{3.319595in}}%
\pgfpathlineto{\pgfqpoint{4.172092in}{3.312888in}}%
\pgfpathlineto{\pgfqpoint{4.206940in}{3.212523in}}%
\pgfpathlineto{\pgfqpoint{4.197222in}{3.309139in}}%
\pgfpathlineto{\pgfqpoint{4.187510in}{3.468279in}}%
\pgfpathlineto{\pgfqpoint{4.152621in}{3.580313in}}%
\pgfpathlineto{\pgfqpoint{4.117846in}{3.366789in}}%
\pgfpathclose%
\pgfusepath{fill}%
\end{pgfscope}%
\begin{pgfscope}%
\pgfpathrectangle{\pgfqpoint{1.020000in}{0.880000in}}{\pgfqpoint{6.160000in}{6.160000in}}%
\pgfusepath{clip}%
\pgfsetbuttcap%
\pgfsetroundjoin%
\definecolor{currentfill}{rgb}{0.831148,0.859513,0.903110}%
\pgfsetfillcolor{currentfill}%
\pgfsetlinewidth{0.000000pt}%
\definecolor{currentstroke}{rgb}{0.000000,0.000000,0.000000}%
\pgfsetstrokecolor{currentstroke}%
\pgfsetdash{}{0pt}%
\pgfpathmoveto{\pgfqpoint{3.730008in}{3.647548in}}%
\pgfpathlineto{\pgfqpoint{3.740255in}{3.419476in}}%
\pgfpathlineto{\pgfqpoint{3.749691in}{3.373578in}}%
\pgfpathlineto{\pgfqpoint{3.784111in}{3.516009in}}%
\pgfpathlineto{\pgfqpoint{3.819233in}{3.486813in}}%
\pgfpathlineto{\pgfqpoint{3.810010in}{3.462301in}}%
\pgfpathlineto{\pgfqpoint{3.799922in}{3.670381in}}%
\pgfpathlineto{\pgfqpoint{3.765908in}{3.434747in}}%
\pgfpathlineto{\pgfqpoint{3.730008in}{3.647548in}}%
\pgfpathclose%
\pgfusepath{fill}%
\end{pgfscope}%
\begin{pgfscope}%
\pgfpathrectangle{\pgfqpoint{1.020000in}{0.880000in}}{\pgfqpoint{6.160000in}{6.160000in}}%
\pgfusepath{clip}%
\pgfsetbuttcap%
\pgfsetroundjoin%
\definecolor{currentfill}{rgb}{0.473070,0.611077,0.970634}%
\pgfsetfillcolor{currentfill}%
\pgfsetlinewidth{0.000000pt}%
\definecolor{currentstroke}{rgb}{0.000000,0.000000,0.000000}%
\pgfsetstrokecolor{currentstroke}%
\pgfsetdash{}{0pt}%
\pgfpathmoveto{\pgfqpoint{4.544351in}{2.918839in}}%
\pgfpathlineto{\pgfqpoint{4.554031in}{2.795062in}}%
\pgfpathlineto{\pgfqpoint{4.563999in}{2.747020in}}%
\pgfpathlineto{\pgfqpoint{4.598767in}{2.747806in}}%
\pgfpathlineto{\pgfqpoint{4.633282in}{2.697311in}}%
\pgfpathlineto{\pgfqpoint{4.623655in}{2.829449in}}%
\pgfpathlineto{\pgfqpoint{4.613337in}{2.803096in}}%
\pgfpathlineto{\pgfqpoint{4.578620in}{2.796894in}}%
\pgfpathlineto{\pgfqpoint{4.544351in}{2.918839in}}%
\pgfpathclose%
\pgfusepath{fill}%
\end{pgfscope}%
\begin{pgfscope}%
\pgfpathrectangle{\pgfqpoint{1.020000in}{0.880000in}}{\pgfqpoint{6.160000in}{6.160000in}}%
\pgfusepath{clip}%
\pgfsetbuttcap%
\pgfsetroundjoin%
\definecolor{currentfill}{rgb}{0.343278,0.459354,0.884122}%
\pgfsetfillcolor{currentfill}%
\pgfsetlinewidth{0.000000pt}%
\definecolor{currentstroke}{rgb}{0.000000,0.000000,0.000000}%
\pgfsetstrokecolor{currentstroke}%
\pgfsetdash{}{0pt}%
\pgfpathmoveto{\pgfqpoint{4.791727in}{2.530701in}}%
\pgfpathlineto{\pgfqpoint{4.802179in}{2.538005in}}%
\pgfpathlineto{\pgfqpoint{4.812117in}{2.460316in}}%
\pgfpathlineto{\pgfqpoint{4.847656in}{2.588850in}}%
\pgfpathlineto{\pgfqpoint{4.880962in}{2.391038in}}%
\pgfpathlineto{\pgfqpoint{4.872336in}{2.659041in}}%
\pgfpathlineto{\pgfqpoint{4.861384in}{2.591811in}}%
\pgfpathlineto{\pgfqpoint{4.826623in}{2.571641in}}%
\pgfpathlineto{\pgfqpoint{4.791727in}{2.530701in}}%
\pgfpathclose%
\pgfusepath{fill}%
\end{pgfscope}%
\begin{pgfscope}%
\pgfpathrectangle{\pgfqpoint{1.020000in}{0.880000in}}{\pgfqpoint{6.160000in}{6.160000in}}%
\pgfusepath{clip}%
\pgfsetbuttcap%
\pgfsetroundjoin%
\definecolor{currentfill}{rgb}{0.353369,0.472069,0.892570}%
\pgfsetfillcolor{currentfill}%
\pgfsetlinewidth{0.000000pt}%
\definecolor{currentstroke}{rgb}{0.000000,0.000000,0.000000}%
\pgfsetstrokecolor{currentstroke}%
\pgfsetdash{}{0pt}%
\pgfpathmoveto{\pgfqpoint{5.020932in}{2.587748in}}%
\pgfpathlineto{\pgfqpoint{5.031051in}{2.522934in}}%
\pgfpathlineto{\pgfqpoint{5.066895in}{2.662032in}}%
\pgfpathlineto{\pgfqpoint{5.099300in}{2.421413in}}%
\pgfpathlineto{\pgfqpoint{5.089402in}{2.512036in}}%
\pgfpathlineto{\pgfqpoint{5.056186in}{2.662988in}}%
\pgfpathlineto{\pgfqpoint{5.020932in}{2.587748in}}%
\pgfpathclose%
\pgfusepath{fill}%
\end{pgfscope}%
\begin{pgfscope}%
\pgfpathrectangle{\pgfqpoint{1.020000in}{0.880000in}}{\pgfqpoint{6.160000in}{6.160000in}}%
\pgfusepath{clip}%
\pgfsetbuttcap%
\pgfsetroundjoin%
\definecolor{currentfill}{rgb}{0.619318,0.744121,0.998931}%
\pgfsetfillcolor{currentfill}%
\pgfsetlinewidth{0.000000pt}%
\definecolor{currentstroke}{rgb}{0.000000,0.000000,0.000000}%
\pgfsetstrokecolor{currentstroke}%
\pgfsetdash{}{0pt}%
\pgfpathmoveto{\pgfqpoint{4.295981in}{3.002162in}}%
\pgfpathlineto{\pgfqpoint{4.305923in}{3.079404in}}%
\pgfpathlineto{\pgfqpoint{4.315737in}{3.032783in}}%
\pgfpathlineto{\pgfqpoint{4.350272in}{2.843750in}}%
\pgfpathlineto{\pgfqpoint{4.385434in}{3.039697in}}%
\pgfpathlineto{\pgfqpoint{4.375603in}{3.098312in}}%
\pgfpathlineto{\pgfqpoint{4.365858in}{3.198123in}}%
\pgfpathlineto{\pgfqpoint{4.331052in}{3.205291in}}%
\pgfpathlineto{\pgfqpoint{4.295981in}{3.002162in}}%
\pgfpathclose%
\pgfusepath{fill}%
\end{pgfscope}%
\begin{pgfscope}%
\pgfpathrectangle{\pgfqpoint{1.020000in}{0.880000in}}{\pgfqpoint{6.160000in}{6.160000in}}%
\pgfusepath{clip}%
\pgfsetbuttcap%
\pgfsetroundjoin%
\definecolor{currentfill}{rgb}{0.863392,0.865084,0.867634}%
\pgfsetfillcolor{currentfill}%
\pgfsetlinewidth{0.000000pt}%
\definecolor{currentstroke}{rgb}{0.000000,0.000000,0.000000}%
\pgfsetstrokecolor{currentstroke}%
\pgfsetdash{}{0pt}%
\pgfpathmoveto{\pgfqpoint{3.799922in}{3.670381in}}%
\pgfpathlineto{\pgfqpoint{3.810010in}{3.462301in}}%
\pgfpathlineto{\pgfqpoint{3.819233in}{3.486813in}}%
\pgfpathlineto{\pgfqpoint{3.854124in}{3.515223in}}%
\pgfpathlineto{\pgfqpoint{3.889117in}{3.508214in}}%
\pgfpathlineto{\pgfqpoint{3.879204in}{3.681155in}}%
\pgfpathlineto{\pgfqpoint{3.869721in}{3.712979in}}%
\pgfpathlineto{\pgfqpoint{3.835043in}{3.628784in}}%
\pgfpathlineto{\pgfqpoint{3.799922in}{3.670381in}}%
\pgfpathclose%
\pgfusepath{fill}%
\end{pgfscope}%
\begin{pgfscope}%
\pgfpathrectangle{\pgfqpoint{1.020000in}{0.880000in}}{\pgfqpoint{6.160000in}{6.160000in}}%
\pgfusepath{clip}%
\pgfsetbuttcap%
\pgfsetroundjoin%
\definecolor{currentfill}{rgb}{0.818056,0.855590,0.914638}%
\pgfsetfillcolor{currentfill}%
\pgfsetlinewidth{0.000000pt}%
\definecolor{currentstroke}{rgb}{0.000000,0.000000,0.000000}%
\pgfsetstrokecolor{currentstroke}%
\pgfsetdash{}{0pt}%
\pgfpathmoveto{\pgfqpoint{3.661413in}{3.361567in}}%
\pgfpathlineto{\pgfqpoint{3.670127in}{3.444326in}}%
\pgfpathlineto{\pgfqpoint{3.679433in}{3.416656in}}%
\pgfpathlineto{\pgfqpoint{3.713812in}{3.559159in}}%
\pgfpathlineto{\pgfqpoint{3.749691in}{3.373578in}}%
\pgfpathlineto{\pgfqpoint{3.740255in}{3.419476in}}%
\pgfpathlineto{\pgfqpoint{3.730008in}{3.647548in}}%
\pgfpathlineto{\pgfqpoint{3.695711in}{3.496429in}}%
\pgfpathlineto{\pgfqpoint{3.661413in}{3.361567in}}%
\pgfpathclose%
\pgfusepath{fill}%
\end{pgfscope}%
\begin{pgfscope}%
\pgfpathrectangle{\pgfqpoint{1.020000in}{0.880000in}}{\pgfqpoint{6.160000in}{6.160000in}}%
\pgfusepath{clip}%
\pgfsetbuttcap%
\pgfsetroundjoin%
\definecolor{currentfill}{rgb}{0.804965,0.851666,0.926165}%
\pgfsetfillcolor{currentfill}%
\pgfsetlinewidth{0.000000pt}%
\definecolor{currentstroke}{rgb}{0.000000,0.000000,0.000000}%
\pgfsetstrokecolor{currentstroke}%
\pgfsetdash{}{0pt}%
\pgfpathmoveto{\pgfqpoint{3.889117in}{3.508214in}}%
\pgfpathlineto{\pgfqpoint{3.898434in}{3.540678in}}%
\pgfpathlineto{\pgfqpoint{3.908705in}{3.236251in}}%
\pgfpathlineto{\pgfqpoint{3.943247in}{3.395867in}}%
\pgfpathlineto{\pgfqpoint{3.978249in}{3.374967in}}%
\pgfpathlineto{\pgfqpoint{3.968676in}{3.413663in}}%
\pgfpathlineto{\pgfqpoint{3.958978in}{3.514746in}}%
\pgfpathlineto{\pgfqpoint{3.924096in}{3.496258in}}%
\pgfpathlineto{\pgfqpoint{3.889117in}{3.508214in}}%
\pgfpathclose%
\pgfusepath{fill}%
\end{pgfscope}%
\begin{pgfscope}%
\pgfpathrectangle{\pgfqpoint{1.020000in}{0.880000in}}{\pgfqpoint{6.160000in}{6.160000in}}%
\pgfusepath{clip}%
\pgfsetbuttcap%
\pgfsetroundjoin%
\definecolor{currentfill}{rgb}{0.677823,0.786546,0.991005}%
\pgfsetfillcolor{currentfill}%
\pgfsetlinewidth{0.000000pt}%
\definecolor{currentstroke}{rgb}{0.000000,0.000000,0.000000}%
\pgfsetstrokecolor{currentstroke}%
\pgfsetdash{}{0pt}%
\pgfpathmoveto{\pgfqpoint{3.312756in}{3.122611in}}%
\pgfpathlineto{\pgfqpoint{3.321563in}{3.116681in}}%
\pgfpathlineto{\pgfqpoint{3.329370in}{3.229879in}}%
\pgfpathlineto{\pgfqpoint{3.365609in}{3.102979in}}%
\pgfpathlineto{\pgfqpoint{3.399409in}{3.272704in}}%
\pgfpathlineto{\pgfqpoint{3.390332in}{3.299796in}}%
\pgfpathlineto{\pgfqpoint{3.382452in}{3.179687in}}%
\pgfpathlineto{\pgfqpoint{3.348593in}{3.033291in}}%
\pgfpathlineto{\pgfqpoint{3.312756in}{3.122611in}}%
\pgfpathclose%
\pgfusepath{fill}%
\end{pgfscope}%
\begin{pgfscope}%
\pgfpathrectangle{\pgfqpoint{1.020000in}{0.880000in}}{\pgfqpoint{6.160000in}{6.160000in}}%
\pgfusepath{clip}%
\pgfsetbuttcap%
\pgfsetroundjoin%
\definecolor{currentfill}{rgb}{0.548876,0.685104,0.994379}%
\pgfsetfillcolor{currentfill}%
\pgfsetlinewidth{0.000000pt}%
\definecolor{currentstroke}{rgb}{0.000000,0.000000,0.000000}%
\pgfsetstrokecolor{currentstroke}%
\pgfsetdash{}{0pt}%
\pgfpathmoveto{\pgfqpoint{4.385434in}{3.039697in}}%
\pgfpathlineto{\pgfqpoint{4.395285in}{2.985478in}}%
\pgfpathlineto{\pgfqpoint{4.405201in}{2.955313in}}%
\pgfpathlineto{\pgfqpoint{4.439605in}{2.781154in}}%
\pgfpathlineto{\pgfqpoint{4.474671in}{2.881965in}}%
\pgfpathlineto{\pgfqpoint{4.465003in}{3.015127in}}%
\pgfpathlineto{\pgfqpoint{4.454333in}{2.782300in}}%
\pgfpathlineto{\pgfqpoint{4.420099in}{2.978062in}}%
\pgfpathlineto{\pgfqpoint{4.385434in}{3.039697in}}%
\pgfpathclose%
\pgfusepath{fill}%
\end{pgfscope}%
\begin{pgfscope}%
\pgfpathrectangle{\pgfqpoint{1.020000in}{0.880000in}}{\pgfqpoint{6.160000in}{6.160000in}}%
\pgfusepath{clip}%
\pgfsetbuttcap%
\pgfsetroundjoin%
\definecolor{currentfill}{rgb}{0.409611,0.540759,0.935545}%
\pgfsetfillcolor{currentfill}%
\pgfsetlinewidth{0.000000pt}%
\definecolor{currentstroke}{rgb}{0.000000,0.000000,0.000000}%
\pgfsetstrokecolor{currentstroke}%
\pgfsetdash{}{0pt}%
\pgfpathmoveto{\pgfqpoint{4.633282in}{2.697311in}}%
\pgfpathlineto{\pgfqpoint{4.643783in}{2.754418in}}%
\pgfpathlineto{\pgfqpoint{4.653371in}{2.611729in}}%
\pgfpathlineto{\pgfqpoint{4.688249in}{2.637686in}}%
\pgfpathlineto{\pgfqpoint{4.723007in}{2.642175in}}%
\pgfpathlineto{\pgfqpoint{4.712998in}{2.701411in}}%
\pgfpathlineto{\pgfqpoint{4.702564in}{2.677267in}}%
\pgfpathlineto{\pgfqpoint{4.667692in}{2.637004in}}%
\pgfpathlineto{\pgfqpoint{4.633282in}{2.697311in}}%
\pgfpathclose%
\pgfusepath{fill}%
\end{pgfscope}%
\begin{pgfscope}%
\pgfpathrectangle{\pgfqpoint{1.020000in}{0.880000in}}{\pgfqpoint{6.160000in}{6.160000in}}%
\pgfusepath{clip}%
\pgfsetbuttcap%
\pgfsetroundjoin%
\definecolor{currentfill}{rgb}{0.483854,0.622050,0.974808}%
\pgfsetfillcolor{currentfill}%
\pgfsetlinewidth{0.000000pt}%
\definecolor{currentstroke}{rgb}{0.000000,0.000000,0.000000}%
\pgfsetstrokecolor{currentstroke}%
\pgfsetdash{}{0pt}%
\pgfpathmoveto{\pgfqpoint{2.891863in}{2.978625in}}%
\pgfpathlineto{\pgfqpoint{2.903821in}{2.695140in}}%
\pgfpathlineto{\pgfqpoint{2.910871in}{2.792702in}}%
\pgfpathlineto{\pgfqpoint{2.947689in}{2.671347in}}%
\pgfpathlineto{\pgfqpoint{2.980762in}{2.850741in}}%
\pgfpathlineto{\pgfqpoint{2.972329in}{2.854579in}}%
\pgfpathlineto{\pgfqpoint{2.965485in}{2.730401in}}%
\pgfpathlineto{\pgfqpoint{2.928856in}{2.844424in}}%
\pgfpathlineto{\pgfqpoint{2.891863in}{2.978625in}}%
\pgfpathclose%
\pgfusepath{fill}%
\end{pgfscope}%
\begin{pgfscope}%
\pgfpathrectangle{\pgfqpoint{1.020000in}{0.880000in}}{\pgfqpoint{6.160000in}{6.160000in}}%
\pgfusepath{clip}%
\pgfsetbuttcap%
\pgfsetroundjoin%
\definecolor{currentfill}{rgb}{0.713852,0.808857,0.979386}%
\pgfsetfillcolor{currentfill}%
\pgfsetlinewidth{0.000000pt}%
\definecolor{currentstroke}{rgb}{0.000000,0.000000,0.000000}%
\pgfsetstrokecolor{currentstroke}%
\pgfsetdash{}{0pt}%
\pgfpathmoveto{\pgfqpoint{3.382452in}{3.179687in}}%
\pgfpathlineto{\pgfqpoint{3.390332in}{3.299796in}}%
\pgfpathlineto{\pgfqpoint{3.399409in}{3.272704in}}%
\pgfpathlineto{\pgfqpoint{3.435910in}{3.095263in}}%
\pgfpathlineto{\pgfqpoint{3.470378in}{3.181571in}}%
\pgfpathlineto{\pgfqpoint{3.461566in}{3.165922in}}%
\pgfpathlineto{\pgfqpoint{3.452209in}{3.227579in}}%
\pgfpathlineto{\pgfqpoint{3.415244in}{3.475767in}}%
\pgfpathlineto{\pgfqpoint{3.382452in}{3.179687in}}%
\pgfpathclose%
\pgfusepath{fill}%
\end{pgfscope}%
\begin{pgfscope}%
\pgfpathrectangle{\pgfqpoint{1.020000in}{0.880000in}}{\pgfqpoint{6.160000in}{6.160000in}}%
\pgfusepath{clip}%
\pgfsetbuttcap%
\pgfsetroundjoin%
\definecolor{currentfill}{rgb}{0.613933,0.739923,0.999142}%
\pgfsetfillcolor{currentfill}%
\pgfsetlinewidth{0.000000pt}%
\definecolor{currentstroke}{rgb}{0.000000,0.000000,0.000000}%
\pgfsetstrokecolor{currentstroke}%
\pgfsetdash{}{0pt}%
\pgfpathmoveto{\pgfqpoint{3.173064in}{3.036010in}}%
\pgfpathlineto{\pgfqpoint{3.181696in}{3.033112in}}%
\pgfpathlineto{\pgfqpoint{3.189930in}{3.071973in}}%
\pgfpathlineto{\pgfqpoint{3.227221in}{2.857722in}}%
\pgfpathlineto{\pgfqpoint{3.259053in}{3.214090in}}%
\pgfpathlineto{\pgfqpoint{3.251277in}{3.114110in}}%
\pgfpathlineto{\pgfqpoint{3.243158in}{3.054448in}}%
\pgfpathlineto{\pgfqpoint{3.208207in}{3.036214in}}%
\pgfpathlineto{\pgfqpoint{3.173064in}{3.036010in}}%
\pgfpathclose%
\pgfusepath{fill}%
\end{pgfscope}%
\begin{pgfscope}%
\pgfpathrectangle{\pgfqpoint{1.020000in}{0.880000in}}{\pgfqpoint{6.160000in}{6.160000in}}%
\pgfusepath{clip}%
\pgfsetbuttcap%
\pgfsetroundjoin%
\definecolor{currentfill}{rgb}{0.586921,0.718121,0.998874}%
\pgfsetfillcolor{currentfill}%
\pgfsetlinewidth{0.000000pt}%
\definecolor{currentstroke}{rgb}{0.000000,0.000000,0.000000}%
\pgfsetstrokecolor{currentstroke}%
\pgfsetdash{}{0pt}%
\pgfpathmoveto{\pgfqpoint{3.103091in}{3.000819in}}%
\pgfpathlineto{\pgfqpoint{3.112222in}{2.944685in}}%
\pgfpathlineto{\pgfqpoint{3.120548in}{2.964825in}}%
\pgfpathlineto{\pgfqpoint{3.155638in}{2.978972in}}%
\pgfpathlineto{\pgfqpoint{3.189930in}{3.071973in}}%
\pgfpathlineto{\pgfqpoint{3.181696in}{3.033112in}}%
\pgfpathlineto{\pgfqpoint{3.173064in}{3.036010in}}%
\pgfpathlineto{\pgfqpoint{3.139082in}{2.923202in}}%
\pgfpathlineto{\pgfqpoint{3.103091in}{3.000819in}}%
\pgfpathclose%
\pgfusepath{fill}%
\end{pgfscope}%
\begin{pgfscope}%
\pgfpathrectangle{\pgfqpoint{1.020000in}{0.880000in}}{\pgfqpoint{6.160000in}{6.160000in}}%
\pgfusepath{clip}%
\pgfsetbuttcap%
\pgfsetroundjoin%
\definecolor{currentfill}{rgb}{0.804965,0.851666,0.926165}%
\pgfsetfillcolor{currentfill}%
\pgfsetlinewidth{0.000000pt}%
\definecolor{currentstroke}{rgb}{0.000000,0.000000,0.000000}%
\pgfsetstrokecolor{currentstroke}%
\pgfsetdash{}{0pt}%
\pgfpathmoveto{\pgfqpoint{3.591007in}{3.434318in}}%
\pgfpathlineto{\pgfqpoint{3.600825in}{3.306660in}}%
\pgfpathlineto{\pgfqpoint{3.609281in}{3.415724in}}%
\pgfpathlineto{\pgfqpoint{3.644050in}{3.475218in}}%
\pgfpathlineto{\pgfqpoint{3.679433in}{3.416656in}}%
\pgfpathlineto{\pgfqpoint{3.670127in}{3.444326in}}%
\pgfpathlineto{\pgfqpoint{3.661413in}{3.361567in}}%
\pgfpathlineto{\pgfqpoint{3.625033in}{3.613337in}}%
\pgfpathlineto{\pgfqpoint{3.591007in}{3.434318in}}%
\pgfpathclose%
\pgfusepath{fill}%
\end{pgfscope}%
\begin{pgfscope}%
\pgfpathrectangle{\pgfqpoint{1.020000in}{0.880000in}}{\pgfqpoint{6.160000in}{6.160000in}}%
\pgfusepath{clip}%
\pgfsetbuttcap%
\pgfsetroundjoin%
\definecolor{currentfill}{rgb}{0.323718,0.433158,0.864722}%
\pgfsetfillcolor{currentfill}%
\pgfsetlinewidth{0.000000pt}%
\definecolor{currentstroke}{rgb}{0.000000,0.000000,0.000000}%
\pgfsetstrokecolor{currentstroke}%
\pgfsetdash{}{0pt}%
\pgfpathmoveto{\pgfqpoint{4.880962in}{2.391038in}}%
\pgfpathlineto{\pgfqpoint{4.892338in}{2.512056in}}%
\pgfpathlineto{\pgfqpoint{4.927917in}{2.630741in}}%
\pgfpathlineto{\pgfqpoint{4.961538in}{2.492492in}}%
\pgfpathlineto{\pgfqpoint{4.949978in}{2.367641in}}%
\pgfpathlineto{\pgfqpoint{4.916702in}{2.543858in}}%
\pgfpathlineto{\pgfqpoint{4.880962in}{2.391038in}}%
\pgfpathclose%
\pgfusepath{fill}%
\end{pgfscope}%
\begin{pgfscope}%
\pgfpathrectangle{\pgfqpoint{1.020000in}{0.880000in}}{\pgfqpoint{6.160000in}{6.160000in}}%
\pgfusepath{clip}%
\pgfsetbuttcap%
\pgfsetroundjoin%
\definecolor{currentfill}{rgb}{0.661968,0.775491,0.993937}%
\pgfsetfillcolor{currentfill}%
\pgfsetlinewidth{0.000000pt}%
\definecolor{currentstroke}{rgb}{0.000000,0.000000,0.000000}%
\pgfsetstrokecolor{currentstroke}%
\pgfsetdash{}{0pt}%
\pgfpathmoveto{\pgfqpoint{3.243158in}{3.054448in}}%
\pgfpathlineto{\pgfqpoint{3.251277in}{3.114110in}}%
\pgfpathlineto{\pgfqpoint{3.259053in}{3.214090in}}%
\pgfpathlineto{\pgfqpoint{3.295254in}{3.105926in}}%
\pgfpathlineto{\pgfqpoint{3.329370in}{3.229879in}}%
\pgfpathlineto{\pgfqpoint{3.321563in}{3.116681in}}%
\pgfpathlineto{\pgfqpoint{3.312756in}{3.122611in}}%
\pgfpathlineto{\pgfqpoint{3.277621in}{3.125382in}}%
\pgfpathlineto{\pgfqpoint{3.243158in}{3.054448in}}%
\pgfpathclose%
\pgfusepath{fill}%
\end{pgfscope}%
\begin{pgfscope}%
\pgfpathrectangle{\pgfqpoint{1.020000in}{0.880000in}}{\pgfqpoint{6.160000in}{6.160000in}}%
\pgfusepath{clip}%
\pgfsetbuttcap%
\pgfsetroundjoin%
\definecolor{currentfill}{rgb}{0.753611,0.830233,0.960871}%
\pgfsetfillcolor{currentfill}%
\pgfsetlinewidth{0.000000pt}%
\definecolor{currentstroke}{rgb}{0.000000,0.000000,0.000000}%
\pgfsetstrokecolor{currentstroke}%
\pgfsetdash{}{0pt}%
\pgfpathmoveto{\pgfqpoint{3.452209in}{3.227579in}}%
\pgfpathlineto{\pgfqpoint{3.461566in}{3.165922in}}%
\pgfpathlineto{\pgfqpoint{3.470378in}{3.181571in}}%
\pgfpathlineto{\pgfqpoint{3.504853in}{3.272855in}}%
\pgfpathlineto{\pgfqpoint{3.538837in}{3.446652in}}%
\pgfpathlineto{\pgfqpoint{3.530408in}{3.355853in}}%
\pgfpathlineto{\pgfqpoint{3.520442in}{3.502577in}}%
\pgfpathlineto{\pgfqpoint{3.485828in}{3.431122in}}%
\pgfpathlineto{\pgfqpoint{3.452209in}{3.227579in}}%
\pgfpathclose%
\pgfusepath{fill}%
\end{pgfscope}%
\begin{pgfscope}%
\pgfpathrectangle{\pgfqpoint{1.020000in}{0.880000in}}{\pgfqpoint{6.160000in}{6.160000in}}%
\pgfusepath{clip}%
\pgfsetbuttcap%
\pgfsetroundjoin%
\definecolor{currentfill}{rgb}{0.786721,0.844807,0.939810}%
\pgfsetfillcolor{currentfill}%
\pgfsetlinewidth{0.000000pt}%
\definecolor{currentstroke}{rgb}{0.000000,0.000000,0.000000}%
\pgfsetstrokecolor{currentstroke}%
\pgfsetdash{}{0pt}%
\pgfpathmoveto{\pgfqpoint{4.048085in}{3.377148in}}%
\pgfpathlineto{\pgfqpoint{4.057621in}{3.413676in}}%
\pgfpathlineto{\pgfqpoint{4.067254in}{3.391660in}}%
\pgfpathlineto{\pgfqpoint{4.102237in}{3.369536in}}%
\pgfpathlineto{\pgfqpoint{4.137195in}{3.319595in}}%
\pgfpathlineto{\pgfqpoint{4.127500in}{3.363763in}}%
\pgfpathlineto{\pgfqpoint{4.117846in}{3.366789in}}%
\pgfpathlineto{\pgfqpoint{4.082840in}{3.508899in}}%
\pgfpathlineto{\pgfqpoint{4.048085in}{3.377148in}}%
\pgfpathclose%
\pgfusepath{fill}%
\end{pgfscope}%
\begin{pgfscope}%
\pgfpathrectangle{\pgfqpoint{1.020000in}{0.880000in}}{\pgfqpoint{6.160000in}{6.160000in}}%
\pgfusepath{clip}%
\pgfsetbuttcap%
\pgfsetroundjoin%
\definecolor{currentfill}{rgb}{0.576051,0.708780,0.997755}%
\pgfsetfillcolor{currentfill}%
\pgfsetlinewidth{0.000000pt}%
\definecolor{currentstroke}{rgb}{0.000000,0.000000,0.000000}%
\pgfsetstrokecolor{currentstroke}%
\pgfsetdash{}{0pt}%
\pgfpathmoveto{\pgfqpoint{3.033204in}{2.954911in}}%
\pgfpathlineto{\pgfqpoint{3.040923in}{3.018620in}}%
\pgfpathlineto{\pgfqpoint{3.050412in}{2.929370in}}%
\pgfpathlineto{\pgfqpoint{3.086124in}{2.889159in}}%
\pgfpathlineto{\pgfqpoint{3.120548in}{2.964825in}}%
\pgfpathlineto{\pgfqpoint{3.112222in}{2.944685in}}%
\pgfpathlineto{\pgfqpoint{3.103091in}{3.000819in}}%
\pgfpathlineto{\pgfqpoint{3.066826in}{3.096056in}}%
\pgfpathlineto{\pgfqpoint{3.033204in}{2.954911in}}%
\pgfpathclose%
\pgfusepath{fill}%
\end{pgfscope}%
\begin{pgfscope}%
\pgfpathrectangle{\pgfqpoint{1.020000in}{0.880000in}}{\pgfqpoint{6.160000in}{6.160000in}}%
\pgfusepath{clip}%
\pgfsetbuttcap%
\pgfsetroundjoin%
\definecolor{currentfill}{rgb}{0.796064,0.848693,0.933471}%
\pgfsetfillcolor{currentfill}%
\pgfsetlinewidth{0.000000pt}%
\definecolor{currentstroke}{rgb}{0.000000,0.000000,0.000000}%
\pgfsetstrokecolor{currentstroke}%
\pgfsetdash{}{0pt}%
\pgfpathmoveto{\pgfqpoint{3.520442in}{3.502577in}}%
\pgfpathlineto{\pgfqpoint{3.530408in}{3.355853in}}%
\pgfpathlineto{\pgfqpoint{3.538837in}{3.446652in}}%
\pgfpathlineto{\pgfqpoint{3.574215in}{3.408324in}}%
\pgfpathlineto{\pgfqpoint{3.609281in}{3.415724in}}%
\pgfpathlineto{\pgfqpoint{3.600825in}{3.306660in}}%
\pgfpathlineto{\pgfqpoint{3.591007in}{3.434318in}}%
\pgfpathlineto{\pgfqpoint{3.556084in}{3.414829in}}%
\pgfpathlineto{\pgfqpoint{3.520442in}{3.502577in}}%
\pgfpathclose%
\pgfusepath{fill}%
\end{pgfscope}%
\begin{pgfscope}%
\pgfpathrectangle{\pgfqpoint{1.020000in}{0.880000in}}{\pgfqpoint{6.160000in}{6.160000in}}%
\pgfusepath{clip}%
\pgfsetbuttcap%
\pgfsetroundjoin%
\definecolor{currentfill}{rgb}{0.713852,0.808857,0.979386}%
\pgfsetfillcolor{currentfill}%
\pgfsetlinewidth{0.000000pt}%
\definecolor{currentstroke}{rgb}{0.000000,0.000000,0.000000}%
\pgfsetstrokecolor{currentstroke}%
\pgfsetdash{}{0pt}%
\pgfpathmoveto{\pgfqpoint{4.137195in}{3.319595in}}%
\pgfpathlineto{\pgfqpoint{4.146937in}{3.181236in}}%
\pgfpathlineto{\pgfqpoint{4.156634in}{3.189722in}}%
\pgfpathlineto{\pgfqpoint{4.191554in}{3.157030in}}%
\pgfpathlineto{\pgfqpoint{4.226489in}{3.264426in}}%
\pgfpathlineto{\pgfqpoint{4.216716in}{3.283692in}}%
\pgfpathlineto{\pgfqpoint{4.206940in}{3.212523in}}%
\pgfpathlineto{\pgfqpoint{4.172092in}{3.312888in}}%
\pgfpathlineto{\pgfqpoint{4.137195in}{3.319595in}}%
\pgfpathclose%
\pgfusepath{fill}%
\end{pgfscope}%
\begin{pgfscope}%
\pgfpathrectangle{\pgfqpoint{1.020000in}{0.880000in}}{\pgfqpoint{6.160000in}{6.160000in}}%
\pgfusepath{clip}%
\pgfsetbuttcap%
\pgfsetroundjoin%
\definecolor{currentfill}{rgb}{0.478462,0.616564,0.972721}%
\pgfsetfillcolor{currentfill}%
\pgfsetlinewidth{0.000000pt}%
\definecolor{currentstroke}{rgb}{0.000000,0.000000,0.000000}%
\pgfsetstrokecolor{currentstroke}%
\pgfsetdash{}{0pt}%
\pgfpathmoveto{\pgfqpoint{2.822704in}{2.868576in}}%
\pgfpathlineto{\pgfqpoint{2.832823in}{2.729057in}}%
\pgfpathlineto{\pgfqpoint{2.841549in}{2.692174in}}%
\pgfpathlineto{\pgfqpoint{2.876725in}{2.702799in}}%
\pgfpathlineto{\pgfqpoint{2.910871in}{2.792702in}}%
\pgfpathlineto{\pgfqpoint{2.903821in}{2.695140in}}%
\pgfpathlineto{\pgfqpoint{2.891863in}{2.978625in}}%
\pgfpathlineto{\pgfqpoint{2.857649in}{2.895534in}}%
\pgfpathlineto{\pgfqpoint{2.822704in}{2.868576in}}%
\pgfpathclose%
\pgfusepath{fill}%
\end{pgfscope}%
\begin{pgfscope}%
\pgfpathrectangle{\pgfqpoint{1.020000in}{0.880000in}}{\pgfqpoint{6.160000in}{6.160000in}}%
\pgfusepath{clip}%
\pgfsetbuttcap%
\pgfsetroundjoin%
\definecolor{currentfill}{rgb}{0.548876,0.685104,0.994379}%
\pgfsetfillcolor{currentfill}%
\pgfsetlinewidth{0.000000pt}%
\definecolor{currentstroke}{rgb}{0.000000,0.000000,0.000000}%
\pgfsetstrokecolor{currentstroke}%
\pgfsetdash{}{0pt}%
\pgfpathmoveto{\pgfqpoint{2.965485in}{2.730401in}}%
\pgfpathlineto{\pgfqpoint{2.972329in}{2.854579in}}%
\pgfpathlineto{\pgfqpoint{2.980762in}{2.850741in}}%
\pgfpathlineto{\pgfqpoint{3.015266in}{2.917181in}}%
\pgfpathlineto{\pgfqpoint{3.050412in}{2.929370in}}%
\pgfpathlineto{\pgfqpoint{3.040923in}{3.018620in}}%
\pgfpathlineto{\pgfqpoint{3.033204in}{2.954911in}}%
\pgfpathlineto{\pgfqpoint{2.996078in}{3.114538in}}%
\pgfpathlineto{\pgfqpoint{2.965485in}{2.730401in}}%
\pgfpathclose%
\pgfusepath{fill}%
\end{pgfscope}%
\begin{pgfscope}%
\pgfpathrectangle{\pgfqpoint{1.020000in}{0.880000in}}{\pgfqpoint{6.160000in}{6.160000in}}%
\pgfusepath{clip}%
\pgfsetbuttcap%
\pgfsetroundjoin%
\definecolor{currentfill}{rgb}{0.505423,0.643995,0.983157}%
\pgfsetfillcolor{currentfill}%
\pgfsetlinewidth{0.000000pt}%
\definecolor{currentstroke}{rgb}{0.000000,0.000000,0.000000}%
\pgfsetstrokecolor{currentstroke}%
\pgfsetdash{}{0pt}%
\pgfpathmoveto{\pgfqpoint{4.474671in}{2.881965in}}%
\pgfpathlineto{\pgfqpoint{4.484501in}{2.804101in}}%
\pgfpathlineto{\pgfqpoint{4.494616in}{2.814763in}}%
\pgfpathlineto{\pgfqpoint{4.529585in}{2.853862in}}%
\pgfpathlineto{\pgfqpoint{4.563999in}{2.747020in}}%
\pgfpathlineto{\pgfqpoint{4.554031in}{2.795062in}}%
\pgfpathlineto{\pgfqpoint{4.544351in}{2.918839in}}%
\pgfpathlineto{\pgfqpoint{4.509474in}{2.887769in}}%
\pgfpathlineto{\pgfqpoint{4.474671in}{2.881965in}}%
\pgfpathclose%
\pgfusepath{fill}%
\end{pgfscope}%
\begin{pgfscope}%
\pgfpathrectangle{\pgfqpoint{1.020000in}{0.880000in}}{\pgfqpoint{6.160000in}{6.160000in}}%
\pgfusepath{clip}%
\pgfsetbuttcap%
\pgfsetroundjoin%
\definecolor{currentfill}{rgb}{0.419991,0.552989,0.942630}%
\pgfsetfillcolor{currentfill}%
\pgfsetlinewidth{0.000000pt}%
\definecolor{currentstroke}{rgb}{0.000000,0.000000,0.000000}%
\pgfsetstrokecolor{currentstroke}%
\pgfsetdash{}{0pt}%
\pgfpathmoveto{\pgfqpoint{4.563999in}{2.747020in}}%
\pgfpathlineto{\pgfqpoint{4.573865in}{2.669957in}}%
\pgfpathlineto{\pgfqpoint{4.583907in}{2.634563in}}%
\pgfpathlineto{\pgfqpoint{4.618401in}{2.565518in}}%
\pgfpathlineto{\pgfqpoint{4.653371in}{2.611729in}}%
\pgfpathlineto{\pgfqpoint{4.643783in}{2.754418in}}%
\pgfpathlineto{\pgfqpoint{4.633282in}{2.697311in}}%
\pgfpathlineto{\pgfqpoint{4.598767in}{2.747806in}}%
\pgfpathlineto{\pgfqpoint{4.563999in}{2.747020in}}%
\pgfpathclose%
\pgfusepath{fill}%
\end{pgfscope}%
\begin{pgfscope}%
\pgfpathrectangle{\pgfqpoint{1.020000in}{0.880000in}}{\pgfqpoint{6.160000in}{6.160000in}}%
\pgfusepath{clip}%
\pgfsetbuttcap%
\pgfsetroundjoin%
\definecolor{currentfill}{rgb}{0.457046,0.594006,0.963029}%
\pgfsetfillcolor{currentfill}%
\pgfsetlinewidth{0.000000pt}%
\definecolor{currentstroke}{rgb}{0.000000,0.000000,0.000000}%
\pgfsetstrokecolor{currentstroke}%
\pgfsetdash{}{0pt}%
\pgfpathmoveto{\pgfqpoint{2.754803in}{2.675838in}}%
\pgfpathlineto{\pgfqpoint{2.763837in}{2.612784in}}%
\pgfpathlineto{\pgfqpoint{2.769487in}{2.788120in}}%
\pgfpathlineto{\pgfqpoint{2.805684in}{2.730208in}}%
\pgfpathlineto{\pgfqpoint{2.841549in}{2.692174in}}%
\pgfpathlineto{\pgfqpoint{2.832823in}{2.729057in}}%
\pgfpathlineto{\pgfqpoint{2.822704in}{2.868576in}}%
\pgfpathlineto{\pgfqpoint{2.786768in}{2.911962in}}%
\pgfpathlineto{\pgfqpoint{2.754803in}{2.675838in}}%
\pgfpathclose%
\pgfusepath{fill}%
\end{pgfscope}%
\begin{pgfscope}%
\pgfpathrectangle{\pgfqpoint{1.020000in}{0.880000in}}{\pgfqpoint{6.160000in}{6.160000in}}%
\pgfusepath{clip}%
\pgfsetbuttcap%
\pgfsetroundjoin%
\definecolor{currentfill}{rgb}{0.565182,0.699438,0.996635}%
\pgfsetfillcolor{currentfill}%
\pgfsetlinewidth{0.000000pt}%
\definecolor{currentstroke}{rgb}{0.000000,0.000000,0.000000}%
\pgfsetstrokecolor{currentstroke}%
\pgfsetdash{}{0pt}%
\pgfpathmoveto{\pgfqpoint{4.315737in}{3.032783in}}%
\pgfpathlineto{\pgfqpoint{4.325420in}{2.885016in}}%
\pgfpathlineto{\pgfqpoint{4.335374in}{2.919414in}}%
\pgfpathlineto{\pgfqpoint{4.370230in}{2.904425in}}%
\pgfpathlineto{\pgfqpoint{4.405201in}{2.955313in}}%
\pgfpathlineto{\pgfqpoint{4.395285in}{2.985478in}}%
\pgfpathlineto{\pgfqpoint{4.385434in}{3.039697in}}%
\pgfpathlineto{\pgfqpoint{4.350272in}{2.843750in}}%
\pgfpathlineto{\pgfqpoint{4.315737in}{3.032783in}}%
\pgfpathclose%
\pgfusepath{fill}%
\end{pgfscope}%
\begin{pgfscope}%
\pgfpathrectangle{\pgfqpoint{1.020000in}{0.880000in}}{\pgfqpoint{6.160000in}{6.160000in}}%
\pgfusepath{clip}%
\pgfsetbuttcap%
\pgfsetroundjoin%
\definecolor{currentfill}{rgb}{0.378598,0.503856,0.913692}%
\pgfsetfillcolor{currentfill}%
\pgfsetlinewidth{0.000000pt}%
\definecolor{currentstroke}{rgb}{0.000000,0.000000,0.000000}%
\pgfsetstrokecolor{currentstroke}%
\pgfsetdash{}{0pt}%
\pgfpathmoveto{\pgfqpoint{4.723007in}{2.642175in}}%
\pgfpathlineto{\pgfqpoint{4.732782in}{2.538861in}}%
\pgfpathlineto{\pgfqpoint{4.744239in}{2.733040in}}%
\pgfpathlineto{\pgfqpoint{4.777688in}{2.505291in}}%
\pgfpathlineto{\pgfqpoint{4.812117in}{2.460316in}}%
\pgfpathlineto{\pgfqpoint{4.802179in}{2.538005in}}%
\pgfpathlineto{\pgfqpoint{4.791727in}{2.530701in}}%
\pgfpathlineto{\pgfqpoint{4.758483in}{2.773933in}}%
\pgfpathlineto{\pgfqpoint{4.723007in}{2.642175in}}%
\pgfpathclose%
\pgfusepath{fill}%
\end{pgfscope}%
\begin{pgfscope}%
\pgfpathrectangle{\pgfqpoint{1.020000in}{0.880000in}}{\pgfqpoint{6.160000in}{6.160000in}}%
\pgfusepath{clip}%
\pgfsetbuttcap%
\pgfsetroundjoin%
\definecolor{currentfill}{rgb}{0.328604,0.439712,0.869587}%
\pgfsetfillcolor{currentfill}%
\pgfsetlinewidth{0.000000pt}%
\definecolor{currentstroke}{rgb}{0.000000,0.000000,0.000000}%
\pgfsetstrokecolor{currentstroke}%
\pgfsetdash{}{0pt}%
\pgfpathmoveto{\pgfqpoint{4.812117in}{2.460316in}}%
\pgfpathlineto{\pgfqpoint{4.822699in}{2.481199in}}%
\pgfpathlineto{\pgfqpoint{4.857562in}{2.502261in}}%
\pgfpathlineto{\pgfqpoint{4.892338in}{2.512056in}}%
\pgfpathlineto{\pgfqpoint{4.880962in}{2.391038in}}%
\pgfpathlineto{\pgfqpoint{4.847656in}{2.588850in}}%
\pgfpathlineto{\pgfqpoint{4.812117in}{2.460316in}}%
\pgfpathclose%
\pgfusepath{fill}%
\end{pgfscope}%
\begin{pgfscope}%
\pgfpathrectangle{\pgfqpoint{1.020000in}{0.880000in}}{\pgfqpoint{6.160000in}{6.160000in}}%
\pgfusepath{clip}%
\pgfsetbuttcap%
\pgfsetroundjoin%
\definecolor{currentfill}{rgb}{0.826784,0.858205,0.906953}%
\pgfsetfillcolor{currentfill}%
\pgfsetlinewidth{0.000000pt}%
\definecolor{currentstroke}{rgb}{0.000000,0.000000,0.000000}%
\pgfsetstrokecolor{currentstroke}%
\pgfsetdash{}{0pt}%
\pgfpathmoveto{\pgfqpoint{3.819233in}{3.486813in}}%
\pgfpathlineto{\pgfqpoint{3.828333in}{3.553391in}}%
\pgfpathlineto{\pgfqpoint{3.838327in}{3.370465in}}%
\pgfpathlineto{\pgfqpoint{3.872710in}{3.574398in}}%
\pgfpathlineto{\pgfqpoint{3.908705in}{3.236251in}}%
\pgfpathlineto{\pgfqpoint{3.898434in}{3.540678in}}%
\pgfpathlineto{\pgfqpoint{3.889117in}{3.508214in}}%
\pgfpathlineto{\pgfqpoint{3.854124in}{3.515223in}}%
\pgfpathlineto{\pgfqpoint{3.819233in}{3.486813in}}%
\pgfpathclose%
\pgfusepath{fill}%
\end{pgfscope}%
\begin{pgfscope}%
\pgfpathrectangle{\pgfqpoint{1.020000in}{0.880000in}}{\pgfqpoint{6.160000in}{6.160000in}}%
\pgfusepath{clip}%
\pgfsetbuttcap%
\pgfsetroundjoin%
\definecolor{currentfill}{rgb}{0.661968,0.775491,0.993937}%
\pgfsetfillcolor{currentfill}%
\pgfsetlinewidth{0.000000pt}%
\definecolor{currentstroke}{rgb}{0.000000,0.000000,0.000000}%
\pgfsetstrokecolor{currentstroke}%
\pgfsetdash{}{0pt}%
\pgfpathmoveto{\pgfqpoint{4.226489in}{3.264426in}}%
\pgfpathlineto{\pgfqpoint{4.236310in}{3.301131in}}%
\pgfpathlineto{\pgfqpoint{4.246041in}{3.148837in}}%
\pgfpathlineto{\pgfqpoint{4.280918in}{3.088719in}}%
\pgfpathlineto{\pgfqpoint{4.315737in}{3.032783in}}%
\pgfpathlineto{\pgfqpoint{4.305923in}{3.079404in}}%
\pgfpathlineto{\pgfqpoint{4.295981in}{3.002162in}}%
\pgfpathlineto{\pgfqpoint{4.261260in}{3.092401in}}%
\pgfpathlineto{\pgfqpoint{4.226489in}{3.264426in}}%
\pgfpathclose%
\pgfusepath{fill}%
\end{pgfscope}%
\begin{pgfscope}%
\pgfpathrectangle{\pgfqpoint{1.020000in}{0.880000in}}{\pgfqpoint{6.160000in}{6.160000in}}%
\pgfusepath{clip}%
\pgfsetbuttcap%
\pgfsetroundjoin%
\definecolor{currentfill}{rgb}{0.791392,0.846750,0.936641}%
\pgfsetfillcolor{currentfill}%
\pgfsetlinewidth{0.000000pt}%
\definecolor{currentstroke}{rgb}{0.000000,0.000000,0.000000}%
\pgfsetstrokecolor{currentstroke}%
\pgfsetdash{}{0pt}%
\pgfpathmoveto{\pgfqpoint{3.978249in}{3.374967in}}%
\pgfpathlineto{\pgfqpoint{3.987361in}{3.578622in}}%
\pgfpathlineto{\pgfqpoint{3.997389in}{3.322409in}}%
\pgfpathlineto{\pgfqpoint{4.032402in}{3.303323in}}%
\pgfpathlineto{\pgfqpoint{4.067254in}{3.391660in}}%
\pgfpathlineto{\pgfqpoint{4.057621in}{3.413676in}}%
\pgfpathlineto{\pgfqpoint{4.048085in}{3.377148in}}%
\pgfpathlineto{\pgfqpoint{4.013145in}{3.395234in}}%
\pgfpathlineto{\pgfqpoint{3.978249in}{3.374967in}}%
\pgfpathclose%
\pgfusepath{fill}%
\end{pgfscope}%
\begin{pgfscope}%
\pgfpathrectangle{\pgfqpoint{1.020000in}{0.880000in}}{\pgfqpoint{6.160000in}{6.160000in}}%
\pgfusepath{clip}%
\pgfsetbuttcap%
\pgfsetroundjoin%
\definecolor{currentfill}{rgb}{0.813693,0.854282,0.918480}%
\pgfsetfillcolor{currentfill}%
\pgfsetlinewidth{0.000000pt}%
\definecolor{currentstroke}{rgb}{0.000000,0.000000,0.000000}%
\pgfsetstrokecolor{currentstroke}%
\pgfsetdash{}{0pt}%
\pgfpathmoveto{\pgfqpoint{3.749691in}{3.373578in}}%
\pgfpathlineto{\pgfqpoint{3.759115in}{3.333256in}}%
\pgfpathlineto{\pgfqpoint{3.767524in}{3.540610in}}%
\pgfpathlineto{\pgfqpoint{3.803385in}{3.345488in}}%
\pgfpathlineto{\pgfqpoint{3.838327in}{3.370465in}}%
\pgfpathlineto{\pgfqpoint{3.828333in}{3.553391in}}%
\pgfpathlineto{\pgfqpoint{3.819233in}{3.486813in}}%
\pgfpathlineto{\pgfqpoint{3.784111in}{3.516009in}}%
\pgfpathlineto{\pgfqpoint{3.749691in}{3.373578in}}%
\pgfpathclose%
\pgfusepath{fill}%
\end{pgfscope}%
\begin{pgfscope}%
\pgfpathrectangle{\pgfqpoint{1.020000in}{0.880000in}}{\pgfqpoint{6.160000in}{6.160000in}}%
\pgfusepath{clip}%
\pgfsetbuttcap%
\pgfsetroundjoin%
\definecolor{currentfill}{rgb}{0.728970,0.817464,0.973188}%
\pgfsetfillcolor{currentfill}%
\pgfsetlinewidth{0.000000pt}%
\definecolor{currentstroke}{rgb}{0.000000,0.000000,0.000000}%
\pgfsetstrokecolor{currentstroke}%
\pgfsetdash{}{0pt}%
\pgfpathmoveto{\pgfqpoint{4.067254in}{3.391660in}}%
\pgfpathlineto{\pgfqpoint{4.076957in}{3.320138in}}%
\pgfpathlineto{\pgfqpoint{4.086833in}{3.077301in}}%
\pgfpathlineto{\pgfqpoint{4.121674in}{3.216575in}}%
\pgfpathlineto{\pgfqpoint{4.156634in}{3.189722in}}%
\pgfpathlineto{\pgfqpoint{4.146937in}{3.181236in}}%
\pgfpathlineto{\pgfqpoint{4.137195in}{3.319595in}}%
\pgfpathlineto{\pgfqpoint{4.102237in}{3.369536in}}%
\pgfpathlineto{\pgfqpoint{4.067254in}{3.391660in}}%
\pgfpathclose%
\pgfusepath{fill}%
\end{pgfscope}%
\begin{pgfscope}%
\pgfpathrectangle{\pgfqpoint{1.020000in}{0.880000in}}{\pgfqpoint{6.160000in}{6.160000in}}%
\pgfusepath{clip}%
\pgfsetbuttcap%
\pgfsetroundjoin%
\definecolor{currentfill}{rgb}{0.414801,0.546874,0.939088}%
\pgfsetfillcolor{currentfill}%
\pgfsetlinewidth{0.000000pt}%
\definecolor{currentstroke}{rgb}{0.000000,0.000000,0.000000}%
\pgfsetstrokecolor{currentstroke}%
\pgfsetdash{}{0pt}%
\pgfpathmoveto{\pgfqpoint{2.841549in}{2.692174in}}%
\pgfpathlineto{\pgfqpoint{2.850956in}{2.605614in}}%
\pgfpathlineto{\pgfqpoint{2.861767in}{2.414118in}}%
\pgfpathlineto{\pgfqpoint{2.894721in}{2.593931in}}%
\pgfpathlineto{\pgfqpoint{2.930021in}{2.596500in}}%
\pgfpathlineto{\pgfqpoint{2.917990in}{2.887577in}}%
\pgfpathlineto{\pgfqpoint{2.910871in}{2.792702in}}%
\pgfpathlineto{\pgfqpoint{2.876725in}{2.702799in}}%
\pgfpathlineto{\pgfqpoint{2.841549in}{2.692174in}}%
\pgfpathclose%
\pgfusepath{fill}%
\end{pgfscope}%
\begin{pgfscope}%
\pgfpathrectangle{\pgfqpoint{1.020000in}{0.880000in}}{\pgfqpoint{6.160000in}{6.160000in}}%
\pgfusepath{clip}%
\pgfsetbuttcap%
\pgfsetroundjoin%
\definecolor{currentfill}{rgb}{0.683056,0.790043,0.989768}%
\pgfsetfillcolor{currentfill}%
\pgfsetlinewidth{0.000000pt}%
\definecolor{currentstroke}{rgb}{0.000000,0.000000,0.000000}%
\pgfsetstrokecolor{currentstroke}%
\pgfsetdash{}{0pt}%
\pgfpathmoveto{\pgfqpoint{3.399409in}{3.272704in}}%
\pgfpathlineto{\pgfqpoint{3.408391in}{3.259752in}}%
\pgfpathlineto{\pgfqpoint{3.418429in}{3.111412in}}%
\pgfpathlineto{\pgfqpoint{3.453572in}{3.117257in}}%
\pgfpathlineto{\pgfqpoint{3.489106in}{3.064501in}}%
\pgfpathlineto{\pgfqpoint{3.479039in}{3.221964in}}%
\pgfpathlineto{\pgfqpoint{3.470378in}{3.181571in}}%
\pgfpathlineto{\pgfqpoint{3.435910in}{3.095263in}}%
\pgfpathlineto{\pgfqpoint{3.399409in}{3.272704in}}%
\pgfpathclose%
\pgfusepath{fill}%
\end{pgfscope}%
\begin{pgfscope}%
\pgfpathrectangle{\pgfqpoint{1.020000in}{0.880000in}}{\pgfqpoint{6.160000in}{6.160000in}}%
\pgfusepath{clip}%
\pgfsetbuttcap%
\pgfsetroundjoin%
\definecolor{currentfill}{rgb}{0.521696,0.659599,0.987736}%
\pgfsetfillcolor{currentfill}%
\pgfsetlinewidth{0.000000pt}%
\definecolor{currentstroke}{rgb}{0.000000,0.000000,0.000000}%
\pgfsetstrokecolor{currentstroke}%
\pgfsetdash{}{0pt}%
\pgfpathmoveto{\pgfqpoint{4.405201in}{2.955313in}}%
\pgfpathlineto{\pgfqpoint{4.414854in}{2.805326in}}%
\pgfpathlineto{\pgfqpoint{4.425295in}{2.982865in}}%
\pgfpathlineto{\pgfqpoint{4.459917in}{2.870297in}}%
\pgfpathlineto{\pgfqpoint{4.494616in}{2.814763in}}%
\pgfpathlineto{\pgfqpoint{4.484501in}{2.804101in}}%
\pgfpathlineto{\pgfqpoint{4.474671in}{2.881965in}}%
\pgfpathlineto{\pgfqpoint{4.439605in}{2.781154in}}%
\pgfpathlineto{\pgfqpoint{4.405201in}{2.955313in}}%
\pgfpathclose%
\pgfusepath{fill}%
\end{pgfscope}%
\begin{pgfscope}%
\pgfpathrectangle{\pgfqpoint{1.020000in}{0.880000in}}{\pgfqpoint{6.160000in}{6.160000in}}%
\pgfusepath{clip}%
\pgfsetbuttcap%
\pgfsetroundjoin%
\definecolor{currentfill}{rgb}{0.791392,0.846750,0.936641}%
\pgfsetfillcolor{currentfill}%
\pgfsetlinewidth{0.000000pt}%
\definecolor{currentstroke}{rgb}{0.000000,0.000000,0.000000}%
\pgfsetstrokecolor{currentstroke}%
\pgfsetdash{}{0pt}%
\pgfpathmoveto{\pgfqpoint{3.908705in}{3.236251in}}%
\pgfpathlineto{\pgfqpoint{3.917360in}{3.526291in}}%
\pgfpathlineto{\pgfqpoint{3.927242in}{3.367098in}}%
\pgfpathlineto{\pgfqpoint{3.962463in}{3.286343in}}%
\pgfpathlineto{\pgfqpoint{3.997389in}{3.322409in}}%
\pgfpathlineto{\pgfqpoint{3.987361in}{3.578622in}}%
\pgfpathlineto{\pgfqpoint{3.978249in}{3.374967in}}%
\pgfpathlineto{\pgfqpoint{3.943247in}{3.395867in}}%
\pgfpathlineto{\pgfqpoint{3.908705in}{3.236251in}}%
\pgfpathclose%
\pgfusepath{fill}%
\end{pgfscope}%
\begin{pgfscope}%
\pgfpathrectangle{\pgfqpoint{1.020000in}{0.880000in}}{\pgfqpoint{6.160000in}{6.160000in}}%
\pgfusepath{clip}%
\pgfsetbuttcap%
\pgfsetroundjoin%
\definecolor{currentfill}{rgb}{0.724041,0.814910,0.975651}%
\pgfsetfillcolor{currentfill}%
\pgfsetlinewidth{0.000000pt}%
\definecolor{currentstroke}{rgb}{0.000000,0.000000,0.000000}%
\pgfsetstrokecolor{currentstroke}%
\pgfsetdash{}{0pt}%
\pgfpathmoveto{\pgfqpoint{3.470378in}{3.181571in}}%
\pgfpathlineto{\pgfqpoint{3.479039in}{3.221964in}}%
\pgfpathlineto{\pgfqpoint{3.489106in}{3.064501in}}%
\pgfpathlineto{\pgfqpoint{3.523379in}{3.193391in}}%
\pgfpathlineto{\pgfqpoint{3.558463in}{3.203828in}}%
\pgfpathlineto{\pgfqpoint{3.548431in}{3.359841in}}%
\pgfpathlineto{\pgfqpoint{3.538837in}{3.446652in}}%
\pgfpathlineto{\pgfqpoint{3.504853in}{3.272855in}}%
\pgfpathlineto{\pgfqpoint{3.470378in}{3.181571in}}%
\pgfpathclose%
\pgfusepath{fill}%
\end{pgfscope}%
\begin{pgfscope}%
\pgfpathrectangle{\pgfqpoint{1.020000in}{0.880000in}}{\pgfqpoint{6.160000in}{6.160000in}}%
\pgfusepath{clip}%
\pgfsetbuttcap%
\pgfsetroundjoin%
\definecolor{currentfill}{rgb}{0.822420,0.856898,0.910795}%
\pgfsetfillcolor{currentfill}%
\pgfsetlinewidth{0.000000pt}%
\definecolor{currentstroke}{rgb}{0.000000,0.000000,0.000000}%
\pgfsetstrokecolor{currentstroke}%
\pgfsetdash{}{0pt}%
\pgfpathmoveto{\pgfqpoint{3.679433in}{3.416656in}}%
\pgfpathlineto{\pgfqpoint{3.688084in}{3.524694in}}%
\pgfpathlineto{\pgfqpoint{3.698228in}{3.333429in}}%
\pgfpathlineto{\pgfqpoint{3.732536in}{3.506762in}}%
\pgfpathlineto{\pgfqpoint{3.767524in}{3.540610in}}%
\pgfpathlineto{\pgfqpoint{3.759115in}{3.333256in}}%
\pgfpathlineto{\pgfqpoint{3.749691in}{3.373578in}}%
\pgfpathlineto{\pgfqpoint{3.713812in}{3.559159in}}%
\pgfpathlineto{\pgfqpoint{3.679433in}{3.416656in}}%
\pgfpathclose%
\pgfusepath{fill}%
\end{pgfscope}%
\begin{pgfscope}%
\pgfpathrectangle{\pgfqpoint{1.020000in}{0.880000in}}{\pgfqpoint{6.160000in}{6.160000in}}%
\pgfusepath{clip}%
\pgfsetbuttcap%
\pgfsetroundjoin%
\definecolor{currentfill}{rgb}{0.796064,0.848693,0.933471}%
\pgfsetfillcolor{currentfill}%
\pgfsetlinewidth{0.000000pt}%
\definecolor{currentstroke}{rgb}{0.000000,0.000000,0.000000}%
\pgfsetstrokecolor{currentstroke}%
\pgfsetdash{}{0pt}%
\pgfpathmoveto{\pgfqpoint{3.609281in}{3.415724in}}%
\pgfpathlineto{\pgfqpoint{3.619175in}{3.277288in}}%
\pgfpathlineto{\pgfqpoint{3.628484in}{3.243302in}}%
\pgfpathlineto{\pgfqpoint{3.662708in}{3.411921in}}%
\pgfpathlineto{\pgfqpoint{3.698228in}{3.333429in}}%
\pgfpathlineto{\pgfqpoint{3.688084in}{3.524694in}}%
\pgfpathlineto{\pgfqpoint{3.679433in}{3.416656in}}%
\pgfpathlineto{\pgfqpoint{3.644050in}{3.475218in}}%
\pgfpathlineto{\pgfqpoint{3.609281in}{3.415724in}}%
\pgfpathclose%
\pgfusepath{fill}%
\end{pgfscope}%
\begin{pgfscope}%
\pgfpathrectangle{\pgfqpoint{1.020000in}{0.880000in}}{\pgfqpoint{6.160000in}{6.160000in}}%
\pgfusepath{clip}%
\pgfsetbuttcap%
\pgfsetroundjoin%
\definecolor{currentfill}{rgb}{0.768034,0.837035,0.952488}%
\pgfsetfillcolor{currentfill}%
\pgfsetlinewidth{0.000000pt}%
\definecolor{currentstroke}{rgb}{0.000000,0.000000,0.000000}%
\pgfsetstrokecolor{currentstroke}%
\pgfsetdash{}{0pt}%
\pgfpathmoveto{\pgfqpoint{3.538837in}{3.446652in}}%
\pgfpathlineto{\pgfqpoint{3.548431in}{3.359841in}}%
\pgfpathlineto{\pgfqpoint{3.558463in}{3.203828in}}%
\pgfpathlineto{\pgfqpoint{3.592993in}{3.305631in}}%
\pgfpathlineto{\pgfqpoint{3.628484in}{3.243302in}}%
\pgfpathlineto{\pgfqpoint{3.619175in}{3.277288in}}%
\pgfpathlineto{\pgfqpoint{3.609281in}{3.415724in}}%
\pgfpathlineto{\pgfqpoint{3.574215in}{3.408324in}}%
\pgfpathlineto{\pgfqpoint{3.538837in}{3.446652in}}%
\pgfpathclose%
\pgfusepath{fill}%
\end{pgfscope}%
\begin{pgfscope}%
\pgfpathrectangle{\pgfqpoint{1.020000in}{0.880000in}}{\pgfqpoint{6.160000in}{6.160000in}}%
\pgfusepath{clip}%
\pgfsetbuttcap%
\pgfsetroundjoin%
\definecolor{currentfill}{rgb}{0.683056,0.790043,0.989768}%
\pgfsetfillcolor{currentfill}%
\pgfsetlinewidth{0.000000pt}%
\definecolor{currentstroke}{rgb}{0.000000,0.000000,0.000000}%
\pgfsetstrokecolor{currentstroke}%
\pgfsetdash{}{0pt}%
\pgfpathmoveto{\pgfqpoint{3.329370in}{3.229879in}}%
\pgfpathlineto{\pgfqpoint{3.338949in}{3.138419in}}%
\pgfpathlineto{\pgfqpoint{3.349044in}{2.985747in}}%
\pgfpathlineto{\pgfqpoint{3.383075in}{3.129678in}}%
\pgfpathlineto{\pgfqpoint{3.418429in}{3.111412in}}%
\pgfpathlineto{\pgfqpoint{3.408391in}{3.259752in}}%
\pgfpathlineto{\pgfqpoint{3.399409in}{3.272704in}}%
\pgfpathlineto{\pgfqpoint{3.365609in}{3.102979in}}%
\pgfpathlineto{\pgfqpoint{3.329370in}{3.229879in}}%
\pgfpathclose%
\pgfusepath{fill}%
\end{pgfscope}%
\begin{pgfscope}%
\pgfpathrectangle{\pgfqpoint{1.020000in}{0.880000in}}{\pgfqpoint{6.160000in}{6.160000in}}%
\pgfusepath{clip}%
\pgfsetbuttcap%
\pgfsetroundjoin%
\definecolor{currentfill}{rgb}{0.419991,0.552989,0.942630}%
\pgfsetfillcolor{currentfill}%
\pgfsetlinewidth{0.000000pt}%
\definecolor{currentstroke}{rgb}{0.000000,0.000000,0.000000}%
\pgfsetstrokecolor{currentstroke}%
\pgfsetdash{}{0pt}%
\pgfpathmoveto{\pgfqpoint{2.769487in}{2.788120in}}%
\pgfpathlineto{\pgfqpoint{2.779608in}{2.650713in}}%
\pgfpathlineto{\pgfqpoint{2.786961in}{2.709822in}}%
\pgfpathlineto{\pgfqpoint{2.822910in}{2.672689in}}%
\pgfpathlineto{\pgfqpoint{2.861767in}{2.414118in}}%
\pgfpathlineto{\pgfqpoint{2.850956in}{2.605614in}}%
\pgfpathlineto{\pgfqpoint{2.841549in}{2.692174in}}%
\pgfpathlineto{\pgfqpoint{2.805684in}{2.730208in}}%
\pgfpathlineto{\pgfqpoint{2.769487in}{2.788120in}}%
\pgfpathclose%
\pgfusepath{fill}%
\end{pgfscope}%
\begin{pgfscope}%
\pgfpathrectangle{\pgfqpoint{1.020000in}{0.880000in}}{\pgfqpoint{6.160000in}{6.160000in}}%
\pgfusepath{clip}%
\pgfsetbuttcap%
\pgfsetroundjoin%
\definecolor{currentfill}{rgb}{0.419991,0.552989,0.942630}%
\pgfsetfillcolor{currentfill}%
\pgfsetlinewidth{0.000000pt}%
\definecolor{currentstroke}{rgb}{0.000000,0.000000,0.000000}%
\pgfsetstrokecolor{currentstroke}%
\pgfsetdash{}{0pt}%
\pgfpathmoveto{\pgfqpoint{4.653371in}{2.611729in}}%
\pgfpathlineto{\pgfqpoint{4.664219in}{2.730300in}}%
\pgfpathlineto{\pgfqpoint{4.674377in}{2.699875in}}%
\pgfpathlineto{\pgfqpoint{4.708988in}{2.655106in}}%
\pgfpathlineto{\pgfqpoint{4.744239in}{2.733040in}}%
\pgfpathlineto{\pgfqpoint{4.732782in}{2.538861in}}%
\pgfpathlineto{\pgfqpoint{4.723007in}{2.642175in}}%
\pgfpathlineto{\pgfqpoint{4.688249in}{2.637686in}}%
\pgfpathlineto{\pgfqpoint{4.653371in}{2.611729in}}%
\pgfpathclose%
\pgfusepath{fill}%
\end{pgfscope}%
\begin{pgfscope}%
\pgfpathrectangle{\pgfqpoint{1.020000in}{0.880000in}}{\pgfqpoint{6.160000in}{6.160000in}}%
\pgfusepath{clip}%
\pgfsetbuttcap%
\pgfsetroundjoin%
\definecolor{currentfill}{rgb}{0.554312,0.690097,0.995516}%
\pgfsetfillcolor{currentfill}%
\pgfsetlinewidth{0.000000pt}%
\definecolor{currentstroke}{rgb}{0.000000,0.000000,0.000000}%
\pgfsetstrokecolor{currentstroke}%
\pgfsetdash{}{0pt}%
\pgfpathmoveto{\pgfqpoint{3.050412in}{2.929370in}}%
\pgfpathlineto{\pgfqpoint{3.060692in}{2.770588in}}%
\pgfpathlineto{\pgfqpoint{3.067624in}{2.909544in}}%
\pgfpathlineto{\pgfqpoint{3.101468in}{3.046827in}}%
\pgfpathlineto{\pgfqpoint{3.138845in}{2.854875in}}%
\pgfpathlineto{\pgfqpoint{3.129578in}{2.920560in}}%
\pgfpathlineto{\pgfqpoint{3.120548in}{2.964825in}}%
\pgfpathlineto{\pgfqpoint{3.086124in}{2.889159in}}%
\pgfpathlineto{\pgfqpoint{3.050412in}{2.929370in}}%
\pgfpathclose%
\pgfusepath{fill}%
\end{pgfscope}%
\begin{pgfscope}%
\pgfpathrectangle{\pgfqpoint{1.020000in}{0.880000in}}{\pgfqpoint{6.160000in}{6.160000in}}%
\pgfusepath{clip}%
\pgfsetbuttcap%
\pgfsetroundjoin%
\definecolor{currentfill}{rgb}{0.521696,0.659599,0.987736}%
\pgfsetfillcolor{currentfill}%
\pgfsetlinewidth{0.000000pt}%
\definecolor{currentstroke}{rgb}{0.000000,0.000000,0.000000}%
\pgfsetstrokecolor{currentstroke}%
\pgfsetdash{}{0pt}%
\pgfpathmoveto{\pgfqpoint{2.980762in}{2.850741in}}%
\pgfpathlineto{\pgfqpoint{2.988965in}{2.867564in}}%
\pgfpathlineto{\pgfqpoint{2.998628in}{2.763717in}}%
\pgfpathlineto{\pgfqpoint{3.033660in}{2.789202in}}%
\pgfpathlineto{\pgfqpoint{3.067624in}{2.909544in}}%
\pgfpathlineto{\pgfqpoint{3.060692in}{2.770588in}}%
\pgfpathlineto{\pgfqpoint{3.050412in}{2.929370in}}%
\pgfpathlineto{\pgfqpoint{3.015266in}{2.917181in}}%
\pgfpathlineto{\pgfqpoint{2.980762in}{2.850741in}}%
\pgfpathclose%
\pgfusepath{fill}%
\end{pgfscope}%
\begin{pgfscope}%
\pgfpathrectangle{\pgfqpoint{1.020000in}{0.880000in}}{\pgfqpoint{6.160000in}{6.160000in}}%
\pgfusepath{clip}%
\pgfsetbuttcap%
\pgfsetroundjoin%
\definecolor{currentfill}{rgb}{0.651398,0.768121,0.995891}%
\pgfsetfillcolor{currentfill}%
\pgfsetlinewidth{0.000000pt}%
\definecolor{currentstroke}{rgb}{0.000000,0.000000,0.000000}%
\pgfsetstrokecolor{currentstroke}%
\pgfsetdash{}{0pt}%
\pgfpathmoveto{\pgfqpoint{3.259053in}{3.214090in}}%
\pgfpathlineto{\pgfqpoint{3.270370in}{2.932148in}}%
\pgfpathlineto{\pgfqpoint{3.276804in}{3.184195in}}%
\pgfpathlineto{\pgfqpoint{3.313695in}{3.002023in}}%
\pgfpathlineto{\pgfqpoint{3.349044in}{2.985747in}}%
\pgfpathlineto{\pgfqpoint{3.338949in}{3.138419in}}%
\pgfpathlineto{\pgfqpoint{3.329370in}{3.229879in}}%
\pgfpathlineto{\pgfqpoint{3.295254in}{3.105926in}}%
\pgfpathlineto{\pgfqpoint{3.259053in}{3.214090in}}%
\pgfpathclose%
\pgfusepath{fill}%
\end{pgfscope}%
\begin{pgfscope}%
\pgfpathrectangle{\pgfqpoint{1.020000in}{0.880000in}}{\pgfqpoint{6.160000in}{6.160000in}}%
\pgfusepath{clip}%
\pgfsetbuttcap%
\pgfsetroundjoin%
\definecolor{currentfill}{rgb}{0.368507,0.491141,0.905243}%
\pgfsetfillcolor{currentfill}%
\pgfsetlinewidth{0.000000pt}%
\definecolor{currentstroke}{rgb}{0.000000,0.000000,0.000000}%
\pgfsetstrokecolor{currentstroke}%
\pgfsetdash{}{0pt}%
\pgfpathmoveto{\pgfqpoint{4.744239in}{2.733040in}}%
\pgfpathlineto{\pgfqpoint{4.754673in}{2.738292in}}%
\pgfpathlineto{\pgfqpoint{4.787664in}{2.433212in}}%
\pgfpathlineto{\pgfqpoint{4.822699in}{2.481199in}}%
\pgfpathlineto{\pgfqpoint{4.812117in}{2.460316in}}%
\pgfpathlineto{\pgfqpoint{4.777688in}{2.505291in}}%
\pgfpathlineto{\pgfqpoint{4.744239in}{2.733040in}}%
\pgfpathclose%
\pgfusepath{fill}%
\end{pgfscope}%
\begin{pgfscope}%
\pgfpathrectangle{\pgfqpoint{1.020000in}{0.880000in}}{\pgfqpoint{6.160000in}{6.160000in}}%
\pgfusepath{clip}%
\pgfsetbuttcap%
\pgfsetroundjoin%
\definecolor{currentfill}{rgb}{0.698454,0.799450,0.984577}%
\pgfsetfillcolor{currentfill}%
\pgfsetlinewidth{0.000000pt}%
\definecolor{currentstroke}{rgb}{0.000000,0.000000,0.000000}%
\pgfsetstrokecolor{currentstroke}%
\pgfsetdash{}{0pt}%
\pgfpathmoveto{\pgfqpoint{4.156634in}{3.189722in}}%
\pgfpathlineto{\pgfqpoint{4.166348in}{3.239865in}}%
\pgfpathlineto{\pgfqpoint{4.176094in}{3.277921in}}%
\pgfpathlineto{\pgfqpoint{4.211006in}{2.872235in}}%
\pgfpathlineto{\pgfqpoint{4.246041in}{3.148837in}}%
\pgfpathlineto{\pgfqpoint{4.236310in}{3.301131in}}%
\pgfpathlineto{\pgfqpoint{4.226489in}{3.264426in}}%
\pgfpathlineto{\pgfqpoint{4.191554in}{3.157030in}}%
\pgfpathlineto{\pgfqpoint{4.156634in}{3.189722in}}%
\pgfpathclose%
\pgfusepath{fill}%
\end{pgfscope}%
\begin{pgfscope}%
\pgfpathrectangle{\pgfqpoint{1.020000in}{0.880000in}}{\pgfqpoint{6.160000in}{6.160000in}}%
\pgfusepath{clip}%
\pgfsetbuttcap%
\pgfsetroundjoin%
\definecolor{currentfill}{rgb}{0.473070,0.611077,0.970634}%
\pgfsetfillcolor{currentfill}%
\pgfsetlinewidth{0.000000pt}%
\definecolor{currentstroke}{rgb}{0.000000,0.000000,0.000000}%
\pgfsetstrokecolor{currentstroke}%
\pgfsetdash{}{0pt}%
\pgfpathmoveto{\pgfqpoint{4.494616in}{2.814763in}}%
\pgfpathlineto{\pgfqpoint{4.504677in}{2.801228in}}%
\pgfpathlineto{\pgfqpoint{4.515059in}{2.877747in}}%
\pgfpathlineto{\pgfqpoint{4.549340in}{2.702757in}}%
\pgfpathlineto{\pgfqpoint{4.583907in}{2.634563in}}%
\pgfpathlineto{\pgfqpoint{4.573865in}{2.669957in}}%
\pgfpathlineto{\pgfqpoint{4.563999in}{2.747020in}}%
\pgfpathlineto{\pgfqpoint{4.529585in}{2.853862in}}%
\pgfpathlineto{\pgfqpoint{4.494616in}{2.814763in}}%
\pgfpathclose%
\pgfusepath{fill}%
\end{pgfscope}%
\begin{pgfscope}%
\pgfpathrectangle{\pgfqpoint{1.020000in}{0.880000in}}{\pgfqpoint{6.160000in}{6.160000in}}%
\pgfusepath{clip}%
\pgfsetbuttcap%
\pgfsetroundjoin%
\definecolor{currentfill}{rgb}{0.494638,0.633022,0.978983}%
\pgfsetfillcolor{currentfill}%
\pgfsetlinewidth{0.000000pt}%
\definecolor{currentstroke}{rgb}{0.000000,0.000000,0.000000}%
\pgfsetstrokecolor{currentstroke}%
\pgfsetdash{}{0pt}%
\pgfpathmoveto{\pgfqpoint{2.910871in}{2.792702in}}%
\pgfpathlineto{\pgfqpoint{2.917990in}{2.887577in}}%
\pgfpathlineto{\pgfqpoint{2.930021in}{2.596500in}}%
\pgfpathlineto{\pgfqpoint{2.961122in}{2.939338in}}%
\pgfpathlineto{\pgfqpoint{2.998628in}{2.763717in}}%
\pgfpathlineto{\pgfqpoint{2.988965in}{2.867564in}}%
\pgfpathlineto{\pgfqpoint{2.980762in}{2.850741in}}%
\pgfpathlineto{\pgfqpoint{2.947689in}{2.671347in}}%
\pgfpathlineto{\pgfqpoint{2.910871in}{2.792702in}}%
\pgfpathclose%
\pgfusepath{fill}%
\end{pgfscope}%
\begin{pgfscope}%
\pgfpathrectangle{\pgfqpoint{1.020000in}{0.880000in}}{\pgfqpoint{6.160000in}{6.160000in}}%
\pgfusepath{clip}%
\pgfsetbuttcap%
\pgfsetroundjoin%
\definecolor{currentfill}{rgb}{0.624703,0.748318,0.998719}%
\pgfsetfillcolor{currentfill}%
\pgfsetlinewidth{0.000000pt}%
\definecolor{currentstroke}{rgb}{0.000000,0.000000,0.000000}%
\pgfsetstrokecolor{currentstroke}%
\pgfsetdash{}{0pt}%
\pgfpathmoveto{\pgfqpoint{3.189930in}{3.071973in}}%
\pgfpathlineto{\pgfqpoint{3.198018in}{3.128337in}}%
\pgfpathlineto{\pgfqpoint{3.208063in}{2.987731in}}%
\pgfpathlineto{\pgfqpoint{3.243307in}{2.990965in}}%
\pgfpathlineto{\pgfqpoint{3.276804in}{3.184195in}}%
\pgfpathlineto{\pgfqpoint{3.270370in}{2.932148in}}%
\pgfpathlineto{\pgfqpoint{3.259053in}{3.214090in}}%
\pgfpathlineto{\pgfqpoint{3.227221in}{2.857722in}}%
\pgfpathlineto{\pgfqpoint{3.189930in}{3.071973in}}%
\pgfpathclose%
\pgfusepath{fill}%
\end{pgfscope}%
\begin{pgfscope}%
\pgfpathrectangle{\pgfqpoint{1.020000in}{0.880000in}}{\pgfqpoint{6.160000in}{6.160000in}}%
\pgfusepath{clip}%
\pgfsetbuttcap%
\pgfsetroundjoin%
\definecolor{currentfill}{rgb}{0.613933,0.739923,0.999142}%
\pgfsetfillcolor{currentfill}%
\pgfsetlinewidth{0.000000pt}%
\definecolor{currentstroke}{rgb}{0.000000,0.000000,0.000000}%
\pgfsetstrokecolor{currentstroke}%
\pgfsetdash{}{0pt}%
\pgfpathmoveto{\pgfqpoint{4.246041in}{3.148837in}}%
\pgfpathlineto{\pgfqpoint{4.255869in}{3.141449in}}%
\pgfpathlineto{\pgfqpoint{4.265620in}{3.015943in}}%
\pgfpathlineto{\pgfqpoint{4.300525in}{2.967745in}}%
\pgfpathlineto{\pgfqpoint{4.335374in}{2.919414in}}%
\pgfpathlineto{\pgfqpoint{4.325420in}{2.885016in}}%
\pgfpathlineto{\pgfqpoint{4.315737in}{3.032783in}}%
\pgfpathlineto{\pgfqpoint{4.280918in}{3.088719in}}%
\pgfpathlineto{\pgfqpoint{4.246041in}{3.148837in}}%
\pgfpathclose%
\pgfusepath{fill}%
\end{pgfscope}%
\begin{pgfscope}%
\pgfpathrectangle{\pgfqpoint{1.020000in}{0.880000in}}{\pgfqpoint{6.160000in}{6.160000in}}%
\pgfusepath{clip}%
\pgfsetbuttcap%
\pgfsetroundjoin%
\definecolor{currentfill}{rgb}{0.394042,0.522413,0.924916}%
\pgfsetfillcolor{currentfill}%
\pgfsetlinewidth{0.000000pt}%
\definecolor{currentstroke}{rgb}{0.000000,0.000000,0.000000}%
\pgfsetstrokecolor{currentstroke}%
\pgfsetdash{}{0pt}%
\pgfpathmoveto{\pgfqpoint{4.583907in}{2.634563in}}%
\pgfpathlineto{\pgfqpoint{4.594281in}{2.675624in}}%
\pgfpathlineto{\pgfqpoint{4.604054in}{2.567855in}}%
\pgfpathlineto{\pgfqpoint{4.638113in}{2.396631in}}%
\pgfpathlineto{\pgfqpoint{4.674377in}{2.699875in}}%
\pgfpathlineto{\pgfqpoint{4.664219in}{2.730300in}}%
\pgfpathlineto{\pgfqpoint{4.653371in}{2.611729in}}%
\pgfpathlineto{\pgfqpoint{4.618401in}{2.565518in}}%
\pgfpathlineto{\pgfqpoint{4.583907in}{2.634563in}}%
\pgfpathclose%
\pgfusepath{fill}%
\end{pgfscope}%
\begin{pgfscope}%
\pgfpathrectangle{\pgfqpoint{1.020000in}{0.880000in}}{\pgfqpoint{6.160000in}{6.160000in}}%
\pgfusepath{clip}%
\pgfsetbuttcap%
\pgfsetroundjoin%
\definecolor{currentfill}{rgb}{0.597777,0.727330,0.999777}%
\pgfsetfillcolor{currentfill}%
\pgfsetlinewidth{0.000000pt}%
\definecolor{currentstroke}{rgb}{0.000000,0.000000,0.000000}%
\pgfsetstrokecolor{currentstroke}%
\pgfsetdash{}{0pt}%
\pgfpathmoveto{\pgfqpoint{3.120548in}{2.964825in}}%
\pgfpathlineto{\pgfqpoint{3.129578in}{2.920560in}}%
\pgfpathlineto{\pgfqpoint{3.138845in}{2.854875in}}%
\pgfpathlineto{\pgfqpoint{3.172613in}{3.002973in}}%
\pgfpathlineto{\pgfqpoint{3.208063in}{2.987731in}}%
\pgfpathlineto{\pgfqpoint{3.198018in}{3.128337in}}%
\pgfpathlineto{\pgfqpoint{3.189930in}{3.071973in}}%
\pgfpathlineto{\pgfqpoint{3.155638in}{2.978972in}}%
\pgfpathlineto{\pgfqpoint{3.120548in}{2.964825in}}%
\pgfpathclose%
\pgfusepath{fill}%
\end{pgfscope}%
\begin{pgfscope}%
\pgfpathrectangle{\pgfqpoint{1.020000in}{0.880000in}}{\pgfqpoint{6.160000in}{6.160000in}}%
\pgfusepath{clip}%
\pgfsetbuttcap%
\pgfsetroundjoin%
\definecolor{currentfill}{rgb}{0.753611,0.830233,0.960871}%
\pgfsetfillcolor{currentfill}%
\pgfsetlinewidth{0.000000pt}%
\definecolor{currentstroke}{rgb}{0.000000,0.000000,0.000000}%
\pgfsetstrokecolor{currentstroke}%
\pgfsetdash{}{0pt}%
\pgfpathmoveto{\pgfqpoint{3.997389in}{3.322409in}}%
\pgfpathlineto{\pgfqpoint{4.006910in}{3.338092in}}%
\pgfpathlineto{\pgfqpoint{4.016546in}{3.298539in}}%
\pgfpathlineto{\pgfqpoint{4.051608in}{3.287435in}}%
\pgfpathlineto{\pgfqpoint{4.086833in}{3.077301in}}%
\pgfpathlineto{\pgfqpoint{4.076957in}{3.320138in}}%
\pgfpathlineto{\pgfqpoint{4.067254in}{3.391660in}}%
\pgfpathlineto{\pgfqpoint{4.032402in}{3.303323in}}%
\pgfpathlineto{\pgfqpoint{3.997389in}{3.322409in}}%
\pgfpathclose%
\pgfusepath{fill}%
\end{pgfscope}%
\begin{pgfscope}%
\pgfpathrectangle{\pgfqpoint{1.020000in}{0.880000in}}{\pgfqpoint{6.160000in}{6.160000in}}%
\pgfusepath{clip}%
\pgfsetbuttcap%
\pgfsetroundjoin%
\definecolor{currentfill}{rgb}{0.809329,0.852974,0.922323}%
\pgfsetfillcolor{currentfill}%
\pgfsetlinewidth{0.000000pt}%
\definecolor{currentstroke}{rgb}{0.000000,0.000000,0.000000}%
\pgfsetstrokecolor{currentstroke}%
\pgfsetdash{}{0pt}%
\pgfpathmoveto{\pgfqpoint{3.838327in}{3.370465in}}%
\pgfpathlineto{\pgfqpoint{3.847468in}{3.437442in}}%
\pgfpathlineto{\pgfqpoint{3.857060in}{3.376577in}}%
\pgfpathlineto{\pgfqpoint{3.892072in}{3.403400in}}%
\pgfpathlineto{\pgfqpoint{3.927242in}{3.367098in}}%
\pgfpathlineto{\pgfqpoint{3.917360in}{3.526291in}}%
\pgfpathlineto{\pgfqpoint{3.908705in}{3.236251in}}%
\pgfpathlineto{\pgfqpoint{3.872710in}{3.574398in}}%
\pgfpathlineto{\pgfqpoint{3.838327in}{3.370465in}}%
\pgfpathclose%
\pgfusepath{fill}%
\end{pgfscope}%
\begin{pgfscope}%
\pgfpathrectangle{\pgfqpoint{1.020000in}{0.880000in}}{\pgfqpoint{6.160000in}{6.160000in}}%
\pgfusepath{clip}%
\pgfsetbuttcap%
\pgfsetroundjoin%
\definecolor{currentfill}{rgb}{0.559747,0.694768,0.996075}%
\pgfsetfillcolor{currentfill}%
\pgfsetlinewidth{0.000000pt}%
\definecolor{currentstroke}{rgb}{0.000000,0.000000,0.000000}%
\pgfsetstrokecolor{currentstroke}%
\pgfsetdash{}{0pt}%
\pgfpathmoveto{\pgfqpoint{4.335374in}{2.919414in}}%
\pgfpathlineto{\pgfqpoint{4.345507in}{3.045579in}}%
\pgfpathlineto{\pgfqpoint{4.355054in}{2.805987in}}%
\pgfpathlineto{\pgfqpoint{4.390168in}{2.900674in}}%
\pgfpathlineto{\pgfqpoint{4.425295in}{2.982865in}}%
\pgfpathlineto{\pgfqpoint{4.414854in}{2.805326in}}%
\pgfpathlineto{\pgfqpoint{4.405201in}{2.955313in}}%
\pgfpathlineto{\pgfqpoint{4.370230in}{2.904425in}}%
\pgfpathlineto{\pgfqpoint{4.335374in}{2.919414in}}%
\pgfpathclose%
\pgfusepath{fill}%
\end{pgfscope}%
\begin{pgfscope}%
\pgfpathrectangle{\pgfqpoint{1.020000in}{0.880000in}}{\pgfqpoint{6.160000in}{6.160000in}}%
\pgfusepath{clip}%
\pgfsetbuttcap%
\pgfsetroundjoin%
\definecolor{currentfill}{rgb}{0.698454,0.799450,0.984577}%
\pgfsetfillcolor{currentfill}%
\pgfsetlinewidth{0.000000pt}%
\definecolor{currentstroke}{rgb}{0.000000,0.000000,0.000000}%
\pgfsetstrokecolor{currentstroke}%
\pgfsetdash{}{0pt}%
\pgfpathmoveto{\pgfqpoint{4.086833in}{3.077301in}}%
\pgfpathlineto{\pgfqpoint{4.096442in}{3.112588in}}%
\pgfpathlineto{\pgfqpoint{4.106157in}{3.046332in}}%
\pgfpathlineto{\pgfqpoint{4.141064in}{3.236895in}}%
\pgfpathlineto{\pgfqpoint{4.176094in}{3.277921in}}%
\pgfpathlineto{\pgfqpoint{4.166348in}{3.239865in}}%
\pgfpathlineto{\pgfqpoint{4.156634in}{3.189722in}}%
\pgfpathlineto{\pgfqpoint{4.121674in}{3.216575in}}%
\pgfpathlineto{\pgfqpoint{4.086833in}{3.077301in}}%
\pgfpathclose%
\pgfusepath{fill}%
\end{pgfscope}%
\begin{pgfscope}%
\pgfpathrectangle{\pgfqpoint{1.020000in}{0.880000in}}{\pgfqpoint{6.160000in}{6.160000in}}%
\pgfusepath{clip}%
\pgfsetbuttcap%
\pgfsetroundjoin%
\definecolor{currentfill}{rgb}{0.796064,0.848693,0.933471}%
\pgfsetfillcolor{currentfill}%
\pgfsetlinewidth{0.000000pt}%
\definecolor{currentstroke}{rgb}{0.000000,0.000000,0.000000}%
\pgfsetstrokecolor{currentstroke}%
\pgfsetdash{}{0pt}%
\pgfpathmoveto{\pgfqpoint{3.767524in}{3.540610in}}%
\pgfpathlineto{\pgfqpoint{3.777598in}{3.352412in}}%
\pgfpathlineto{\pgfqpoint{3.786887in}{3.358404in}}%
\pgfpathlineto{\pgfqpoint{3.822394in}{3.253380in}}%
\pgfpathlineto{\pgfqpoint{3.857060in}{3.376577in}}%
\pgfpathlineto{\pgfqpoint{3.847468in}{3.437442in}}%
\pgfpathlineto{\pgfqpoint{3.838327in}{3.370465in}}%
\pgfpathlineto{\pgfqpoint{3.803385in}{3.345488in}}%
\pgfpathlineto{\pgfqpoint{3.767524in}{3.540610in}}%
\pgfpathclose%
\pgfusepath{fill}%
\end{pgfscope}%
\begin{pgfscope}%
\pgfpathrectangle{\pgfqpoint{1.020000in}{0.880000in}}{\pgfqpoint{6.160000in}{6.160000in}}%
\pgfusepath{clip}%
\pgfsetbuttcap%
\pgfsetroundjoin%
\definecolor{currentfill}{rgb}{0.753611,0.830233,0.960871}%
\pgfsetfillcolor{currentfill}%
\pgfsetlinewidth{0.000000pt}%
\definecolor{currentstroke}{rgb}{0.000000,0.000000,0.000000}%
\pgfsetstrokecolor{currentstroke}%
\pgfsetdash{}{0pt}%
\pgfpathmoveto{\pgfqpoint{3.927242in}{3.367098in}}%
\pgfpathlineto{\pgfqpoint{3.936827in}{3.324612in}}%
\pgfpathlineto{\pgfqpoint{3.946486in}{3.256268in}}%
\pgfpathlineto{\pgfqpoint{3.981890in}{3.094148in}}%
\pgfpathlineto{\pgfqpoint{4.016546in}{3.298539in}}%
\pgfpathlineto{\pgfqpoint{4.006910in}{3.338092in}}%
\pgfpathlineto{\pgfqpoint{3.997389in}{3.322409in}}%
\pgfpathlineto{\pgfqpoint{3.962463in}{3.286343in}}%
\pgfpathlineto{\pgfqpoint{3.927242in}{3.367098in}}%
\pgfpathclose%
\pgfusepath{fill}%
\end{pgfscope}%
\begin{pgfscope}%
\pgfpathrectangle{\pgfqpoint{1.020000in}{0.880000in}}{\pgfqpoint{6.160000in}{6.160000in}}%
\pgfusepath{clip}%
\pgfsetbuttcap%
\pgfsetroundjoin%
\definecolor{currentfill}{rgb}{0.743754,0.825125,0.965798}%
\pgfsetfillcolor{currentfill}%
\pgfsetlinewidth{0.000000pt}%
\definecolor{currentstroke}{rgb}{0.000000,0.000000,0.000000}%
\pgfsetstrokecolor{currentstroke}%
\pgfsetdash{}{0pt}%
\pgfpathmoveto{\pgfqpoint{3.628484in}{3.243302in}}%
\pgfpathlineto{\pgfqpoint{3.637422in}{3.279832in}}%
\pgfpathlineto{\pgfqpoint{3.646980in}{3.205881in}}%
\pgfpathlineto{\pgfqpoint{3.682296in}{3.174961in}}%
\pgfpathlineto{\pgfqpoint{3.717583in}{3.141082in}}%
\pgfpathlineto{\pgfqpoint{3.707367in}{3.349965in}}%
\pgfpathlineto{\pgfqpoint{3.698228in}{3.333429in}}%
\pgfpathlineto{\pgfqpoint{3.662708in}{3.411921in}}%
\pgfpathlineto{\pgfqpoint{3.628484in}{3.243302in}}%
\pgfpathclose%
\pgfusepath{fill}%
\end{pgfscope}%
\begin{pgfscope}%
\pgfpathrectangle{\pgfqpoint{1.020000in}{0.880000in}}{\pgfqpoint{6.160000in}{6.160000in}}%
\pgfusepath{clip}%
\pgfsetbuttcap%
\pgfsetroundjoin%
\definecolor{currentfill}{rgb}{0.554312,0.690097,0.995516}%
\pgfsetfillcolor{currentfill}%
\pgfsetlinewidth{0.000000pt}%
\definecolor{currentstroke}{rgb}{0.000000,0.000000,0.000000}%
\pgfsetstrokecolor{currentstroke}%
\pgfsetdash{}{0pt}%
\pgfpathmoveto{\pgfqpoint{3.208063in}{2.987731in}}%
\pgfpathlineto{\pgfqpoint{3.217286in}{2.931442in}}%
\pgfpathlineto{\pgfqpoint{3.227565in}{2.766242in}}%
\pgfpathlineto{\pgfqpoint{3.262340in}{2.819616in}}%
\pgfpathlineto{\pgfqpoint{3.297504in}{2.831007in}}%
\pgfpathlineto{\pgfqpoint{3.289613in}{2.735031in}}%
\pgfpathlineto{\pgfqpoint{3.276804in}{3.184195in}}%
\pgfpathlineto{\pgfqpoint{3.243307in}{2.990965in}}%
\pgfpathlineto{\pgfqpoint{3.208063in}{2.987731in}}%
\pgfpathclose%
\pgfusepath{fill}%
\end{pgfscope}%
\begin{pgfscope}%
\pgfpathrectangle{\pgfqpoint{1.020000in}{0.880000in}}{\pgfqpoint{6.160000in}{6.160000in}}%
\pgfusepath{clip}%
\pgfsetbuttcap%
\pgfsetroundjoin%
\definecolor{currentfill}{rgb}{0.688188,0.793178,0.988038}%
\pgfsetfillcolor{currentfill}%
\pgfsetlinewidth{0.000000pt}%
\definecolor{currentstroke}{rgb}{0.000000,0.000000,0.000000}%
\pgfsetstrokecolor{currentstroke}%
\pgfsetdash{}{0pt}%
\pgfpathmoveto{\pgfqpoint{3.489106in}{3.064501in}}%
\pgfpathlineto{\pgfqpoint{3.496424in}{3.305469in}}%
\pgfpathlineto{\pgfqpoint{3.506433in}{3.159405in}}%
\pgfpathlineto{\pgfqpoint{3.542045in}{3.100409in}}%
\pgfpathlineto{\pgfqpoint{3.577610in}{3.038593in}}%
\pgfpathlineto{\pgfqpoint{3.567540in}{3.201259in}}%
\pgfpathlineto{\pgfqpoint{3.558463in}{3.203828in}}%
\pgfpathlineto{\pgfqpoint{3.523379in}{3.193391in}}%
\pgfpathlineto{\pgfqpoint{3.489106in}{3.064501in}}%
\pgfpathclose%
\pgfusepath{fill}%
\end{pgfscope}%
\begin{pgfscope}%
\pgfpathrectangle{\pgfqpoint{1.020000in}{0.880000in}}{\pgfqpoint{6.160000in}{6.160000in}}%
\pgfusepath{clip}%
\pgfsetbuttcap%
\pgfsetroundjoin%
\definecolor{currentfill}{rgb}{0.630089,0.752516,0.998508}%
\pgfsetfillcolor{currentfill}%
\pgfsetlinewidth{0.000000pt}%
\definecolor{currentstroke}{rgb}{0.000000,0.000000,0.000000}%
\pgfsetstrokecolor{currentstroke}%
\pgfsetdash{}{0pt}%
\pgfpathmoveto{\pgfqpoint{3.349044in}{2.985747in}}%
\pgfpathlineto{\pgfqpoint{3.356982in}{3.092458in}}%
\pgfpathlineto{\pgfqpoint{3.366283in}{3.037668in}}%
\pgfpathlineto{\pgfqpoint{3.402980in}{2.856686in}}%
\pgfpathlineto{\pgfqpoint{3.437278in}{2.975951in}}%
\pgfpathlineto{\pgfqpoint{3.426802in}{3.181814in}}%
\pgfpathlineto{\pgfqpoint{3.418429in}{3.111412in}}%
\pgfpathlineto{\pgfqpoint{3.383075in}{3.129678in}}%
\pgfpathlineto{\pgfqpoint{3.349044in}{2.985747in}}%
\pgfpathclose%
\pgfusepath{fill}%
\end{pgfscope}%
\begin{pgfscope}%
\pgfpathrectangle{\pgfqpoint{1.020000in}{0.880000in}}{\pgfqpoint{6.160000in}{6.160000in}}%
\pgfusepath{clip}%
\pgfsetbuttcap%
\pgfsetroundjoin%
\definecolor{currentfill}{rgb}{0.597777,0.727330,0.999777}%
\pgfsetfillcolor{currentfill}%
\pgfsetlinewidth{0.000000pt}%
\definecolor{currentstroke}{rgb}{0.000000,0.000000,0.000000}%
\pgfsetstrokecolor{currentstroke}%
\pgfsetdash{}{0pt}%
\pgfpathmoveto{\pgfqpoint{3.276804in}{3.184195in}}%
\pgfpathlineto{\pgfqpoint{3.289613in}{2.735031in}}%
\pgfpathlineto{\pgfqpoint{3.297504in}{2.831007in}}%
\pgfpathlineto{\pgfqpoint{3.331167in}{3.016226in}}%
\pgfpathlineto{\pgfqpoint{3.366283in}{3.037668in}}%
\pgfpathlineto{\pgfqpoint{3.356982in}{3.092458in}}%
\pgfpathlineto{\pgfqpoint{3.349044in}{2.985747in}}%
\pgfpathlineto{\pgfqpoint{3.313695in}{3.002023in}}%
\pgfpathlineto{\pgfqpoint{3.276804in}{3.184195in}}%
\pgfpathclose%
\pgfusepath{fill}%
\end{pgfscope}%
\begin{pgfscope}%
\pgfpathrectangle{\pgfqpoint{1.020000in}{0.880000in}}{\pgfqpoint{6.160000in}{6.160000in}}%
\pgfusepath{clip}%
\pgfsetbuttcap%
\pgfsetroundjoin%
\definecolor{currentfill}{rgb}{0.791392,0.846750,0.936641}%
\pgfsetfillcolor{currentfill}%
\pgfsetlinewidth{0.000000pt}%
\definecolor{currentstroke}{rgb}{0.000000,0.000000,0.000000}%
\pgfsetstrokecolor{currentstroke}%
\pgfsetdash{}{0pt}%
\pgfpathmoveto{\pgfqpoint{3.698228in}{3.333429in}}%
\pgfpathlineto{\pgfqpoint{3.707367in}{3.349965in}}%
\pgfpathlineto{\pgfqpoint{3.717583in}{3.141082in}}%
\pgfpathlineto{\pgfqpoint{3.751936in}{3.312750in}}%
\pgfpathlineto{\pgfqpoint{3.786887in}{3.358404in}}%
\pgfpathlineto{\pgfqpoint{3.777598in}{3.352412in}}%
\pgfpathlineto{\pgfqpoint{3.767524in}{3.540610in}}%
\pgfpathlineto{\pgfqpoint{3.732536in}{3.506762in}}%
\pgfpathlineto{\pgfqpoint{3.698228in}{3.333429in}}%
\pgfpathclose%
\pgfusepath{fill}%
\end{pgfscope}%
\begin{pgfscope}%
\pgfpathrectangle{\pgfqpoint{1.020000in}{0.880000in}}{\pgfqpoint{6.160000in}{6.160000in}}%
\pgfusepath{clip}%
\pgfsetbuttcap%
\pgfsetroundjoin%
\definecolor{currentfill}{rgb}{0.667253,0.779176,0.992959}%
\pgfsetfillcolor{currentfill}%
\pgfsetlinewidth{0.000000pt}%
\definecolor{currentstroke}{rgb}{0.000000,0.000000,0.000000}%
\pgfsetstrokecolor{currentstroke}%
\pgfsetdash{}{0pt}%
\pgfpathmoveto{\pgfqpoint{3.418429in}{3.111412in}}%
\pgfpathlineto{\pgfqpoint{3.426802in}{3.181814in}}%
\pgfpathlineto{\pgfqpoint{3.437278in}{2.975951in}}%
\pgfpathlineto{\pgfqpoint{3.472356in}{2.994562in}}%
\pgfpathlineto{\pgfqpoint{3.506433in}{3.159405in}}%
\pgfpathlineto{\pgfqpoint{3.496424in}{3.305469in}}%
\pgfpathlineto{\pgfqpoint{3.489106in}{3.064501in}}%
\pgfpathlineto{\pgfqpoint{3.453572in}{3.117257in}}%
\pgfpathlineto{\pgfqpoint{3.418429in}{3.111412in}}%
\pgfpathclose%
\pgfusepath{fill}%
\end{pgfscope}%
\begin{pgfscope}%
\pgfpathrectangle{\pgfqpoint{1.020000in}{0.880000in}}{\pgfqpoint{6.160000in}{6.160000in}}%
\pgfusepath{clip}%
\pgfsetbuttcap%
\pgfsetroundjoin%
\definecolor{currentfill}{rgb}{0.733898,0.820018,0.970724}%
\pgfsetfillcolor{currentfill}%
\pgfsetlinewidth{0.000000pt}%
\definecolor{currentstroke}{rgb}{0.000000,0.000000,0.000000}%
\pgfsetstrokecolor{currentstroke}%
\pgfsetdash{}{0pt}%
\pgfpathmoveto{\pgfqpoint{3.558463in}{3.203828in}}%
\pgfpathlineto{\pgfqpoint{3.567540in}{3.201259in}}%
\pgfpathlineto{\pgfqpoint{3.577610in}{3.038593in}}%
\pgfpathlineto{\pgfqpoint{3.610316in}{3.464466in}}%
\pgfpathlineto{\pgfqpoint{3.646980in}{3.205881in}}%
\pgfpathlineto{\pgfqpoint{3.637422in}{3.279832in}}%
\pgfpathlineto{\pgfqpoint{3.628484in}{3.243302in}}%
\pgfpathlineto{\pgfqpoint{3.592993in}{3.305631in}}%
\pgfpathlineto{\pgfqpoint{3.558463in}{3.203828in}}%
\pgfpathclose%
\pgfusepath{fill}%
\end{pgfscope}%
\begin{pgfscope}%
\pgfpathrectangle{\pgfqpoint{1.020000in}{0.880000in}}{\pgfqpoint{6.160000in}{6.160000in}}%
\pgfusepath{clip}%
\pgfsetbuttcap%
\pgfsetroundjoin%
\definecolor{currentfill}{rgb}{0.646113,0.764436,0.996868}%
\pgfsetfillcolor{currentfill}%
\pgfsetlinewidth{0.000000pt}%
\definecolor{currentstroke}{rgb}{0.000000,0.000000,0.000000}%
\pgfsetstrokecolor{currentstroke}%
\pgfsetdash{}{0pt}%
\pgfpathmoveto{\pgfqpoint{4.176094in}{3.277921in}}%
\pgfpathlineto{\pgfqpoint{4.185859in}{2.915788in}}%
\pgfpathlineto{\pgfqpoint{4.195646in}{3.133322in}}%
\pgfpathlineto{\pgfqpoint{4.230673in}{3.091599in}}%
\pgfpathlineto{\pgfqpoint{4.265620in}{3.015943in}}%
\pgfpathlineto{\pgfqpoint{4.255869in}{3.141449in}}%
\pgfpathlineto{\pgfqpoint{4.246041in}{3.148837in}}%
\pgfpathlineto{\pgfqpoint{4.211006in}{2.872235in}}%
\pgfpathlineto{\pgfqpoint{4.176094in}{3.277921in}}%
\pgfpathclose%
\pgfusepath{fill}%
\end{pgfscope}%
\begin{pgfscope}%
\pgfpathrectangle{\pgfqpoint{1.020000in}{0.880000in}}{\pgfqpoint{6.160000in}{6.160000in}}%
\pgfusepath{clip}%
\pgfsetbuttcap%
\pgfsetroundjoin%
\definecolor{currentfill}{rgb}{0.478462,0.616564,0.972721}%
\pgfsetfillcolor{currentfill}%
\pgfsetlinewidth{0.000000pt}%
\definecolor{currentstroke}{rgb}{0.000000,0.000000,0.000000}%
\pgfsetstrokecolor{currentstroke}%
\pgfsetdash{}{0pt}%
\pgfpathmoveto{\pgfqpoint{2.998628in}{2.763717in}}%
\pgfpathlineto{\pgfqpoint{3.007692in}{2.710560in}}%
\pgfpathlineto{\pgfqpoint{3.015246in}{2.786795in}}%
\pgfpathlineto{\pgfqpoint{3.052452in}{2.629167in}}%
\pgfpathlineto{\pgfqpoint{3.086899in}{2.709181in}}%
\pgfpathlineto{\pgfqpoint{3.077942in}{2.748118in}}%
\pgfpathlineto{\pgfqpoint{3.067624in}{2.909544in}}%
\pgfpathlineto{\pgfqpoint{3.033660in}{2.789202in}}%
\pgfpathlineto{\pgfqpoint{2.998628in}{2.763717in}}%
\pgfpathclose%
\pgfusepath{fill}%
\end{pgfscope}%
\begin{pgfscope}%
\pgfpathrectangle{\pgfqpoint{1.020000in}{0.880000in}}{\pgfqpoint{6.160000in}{6.160000in}}%
\pgfusepath{clip}%
\pgfsetbuttcap%
\pgfsetroundjoin%
\definecolor{currentfill}{rgb}{0.516260,0.654498,0.986407}%
\pgfsetfillcolor{currentfill}%
\pgfsetlinewidth{0.000000pt}%
\definecolor{currentstroke}{rgb}{0.000000,0.000000,0.000000}%
\pgfsetstrokecolor{currentstroke}%
\pgfsetdash{}{0pt}%
\pgfpathmoveto{\pgfqpoint{4.425295in}{2.982865in}}%
\pgfpathlineto{\pgfqpoint{4.434785in}{2.761753in}}%
\pgfpathlineto{\pgfqpoint{4.444820in}{2.755980in}}%
\pgfpathlineto{\pgfqpoint{4.479644in}{2.721469in}}%
\pgfpathlineto{\pgfqpoint{4.515059in}{2.877747in}}%
\pgfpathlineto{\pgfqpoint{4.504677in}{2.801228in}}%
\pgfpathlineto{\pgfqpoint{4.494616in}{2.814763in}}%
\pgfpathlineto{\pgfqpoint{4.459917in}{2.870297in}}%
\pgfpathlineto{\pgfqpoint{4.425295in}{2.982865in}}%
\pgfpathclose%
\pgfusepath{fill}%
\end{pgfscope}%
\begin{pgfscope}%
\pgfpathrectangle{\pgfqpoint{1.020000in}{0.880000in}}{\pgfqpoint{6.160000in}{6.160000in}}%
\pgfusepath{clip}%
\pgfsetbuttcap%
\pgfsetroundjoin%
\definecolor{currentfill}{rgb}{0.708720,0.805721,0.981117}%
\pgfsetfillcolor{currentfill}%
\pgfsetlinewidth{0.000000pt}%
\definecolor{currentstroke}{rgb}{0.000000,0.000000,0.000000}%
\pgfsetstrokecolor{currentstroke}%
\pgfsetdash{}{0pt}%
\pgfpathmoveto{\pgfqpoint{4.016546in}{3.298539in}}%
\pgfpathlineto{\pgfqpoint{4.026169in}{3.276765in}}%
\pgfpathlineto{\pgfqpoint{4.035783in}{3.272661in}}%
\pgfpathlineto{\pgfqpoint{4.071042in}{3.140372in}}%
\pgfpathlineto{\pgfqpoint{4.106157in}{3.046332in}}%
\pgfpathlineto{\pgfqpoint{4.096442in}{3.112588in}}%
\pgfpathlineto{\pgfqpoint{4.086833in}{3.077301in}}%
\pgfpathlineto{\pgfqpoint{4.051608in}{3.287435in}}%
\pgfpathlineto{\pgfqpoint{4.016546in}{3.298539in}}%
\pgfpathclose%
\pgfusepath{fill}%
\end{pgfscope}%
\begin{pgfscope}%
\pgfpathrectangle{\pgfqpoint{1.020000in}{0.880000in}}{\pgfqpoint{6.160000in}{6.160000in}}%
\pgfusepath{clip}%
\pgfsetbuttcap%
\pgfsetroundjoin%
\definecolor{currentfill}{rgb}{0.383662,0.510183,0.917831}%
\pgfsetfillcolor{currentfill}%
\pgfsetlinewidth{0.000000pt}%
\definecolor{currentstroke}{rgb}{0.000000,0.000000,0.000000}%
\pgfsetstrokecolor{currentstroke}%
\pgfsetdash{}{0pt}%
\pgfpathmoveto{\pgfqpoint{4.604054in}{2.567855in}}%
\pgfpathlineto{\pgfqpoint{4.613912in}{2.479285in}}%
\pgfpathlineto{\pgfqpoint{4.649440in}{2.623534in}}%
\pgfpathlineto{\pgfqpoint{4.684424in}{2.644151in}}%
\pgfpathlineto{\pgfqpoint{4.674377in}{2.699875in}}%
\pgfpathlineto{\pgfqpoint{4.638113in}{2.396631in}}%
\pgfpathlineto{\pgfqpoint{4.604054in}{2.567855in}}%
\pgfpathclose%
\pgfusepath{fill}%
\end{pgfscope}%
\begin{pgfscope}%
\pgfpathrectangle{\pgfqpoint{1.020000in}{0.880000in}}{\pgfqpoint{6.160000in}{6.160000in}}%
\pgfusepath{clip}%
\pgfsetbuttcap%
\pgfsetroundjoin%
\definecolor{currentfill}{rgb}{0.419991,0.552989,0.942630}%
\pgfsetfillcolor{currentfill}%
\pgfsetlinewidth{0.000000pt}%
\definecolor{currentstroke}{rgb}{0.000000,0.000000,0.000000}%
\pgfsetstrokecolor{currentstroke}%
\pgfsetdash{}{0pt}%
\pgfpathmoveto{\pgfqpoint{2.861767in}{2.414118in}}%
\pgfpathlineto{\pgfqpoint{2.866435in}{2.684731in}}%
\pgfpathlineto{\pgfqpoint{2.875448in}{2.630553in}}%
\pgfpathlineto{\pgfqpoint{2.909811in}{2.714645in}}%
\pgfpathlineto{\pgfqpoint{2.945850in}{2.667407in}}%
\pgfpathlineto{\pgfqpoint{2.935203in}{2.848655in}}%
\pgfpathlineto{\pgfqpoint{2.930021in}{2.596500in}}%
\pgfpathlineto{\pgfqpoint{2.894721in}{2.593931in}}%
\pgfpathlineto{\pgfqpoint{2.861767in}{2.414118in}}%
\pgfpathclose%
\pgfusepath{fill}%
\end{pgfscope}%
\begin{pgfscope}%
\pgfpathrectangle{\pgfqpoint{1.020000in}{0.880000in}}{\pgfqpoint{6.160000in}{6.160000in}}%
\pgfusepath{clip}%
\pgfsetbuttcap%
\pgfsetroundjoin%
\definecolor{currentfill}{rgb}{0.435815,0.570707,0.951717}%
\pgfsetfillcolor{currentfill}%
\pgfsetlinewidth{0.000000pt}%
\definecolor{currentstroke}{rgb}{0.000000,0.000000,0.000000}%
\pgfsetstrokecolor{currentstroke}%
\pgfsetdash{}{0pt}%
\pgfpathmoveto{\pgfqpoint{4.515059in}{2.877747in}}%
\pgfpathlineto{\pgfqpoint{4.524630in}{2.707360in}}%
\pgfpathlineto{\pgfqpoint{4.534213in}{2.543997in}}%
\pgfpathlineto{\pgfqpoint{4.569671in}{2.692031in}}%
\pgfpathlineto{\pgfqpoint{4.604054in}{2.567855in}}%
\pgfpathlineto{\pgfqpoint{4.594281in}{2.675624in}}%
\pgfpathlineto{\pgfqpoint{4.583907in}{2.634563in}}%
\pgfpathlineto{\pgfqpoint{4.549340in}{2.702757in}}%
\pgfpathlineto{\pgfqpoint{4.515059in}{2.877747in}}%
\pgfpathclose%
\pgfusepath{fill}%
\end{pgfscope}%
\begin{pgfscope}%
\pgfpathrectangle{\pgfqpoint{1.020000in}{0.880000in}}{\pgfqpoint{6.160000in}{6.160000in}}%
\pgfusepath{clip}%
\pgfsetbuttcap%
\pgfsetroundjoin%
\definecolor{currentfill}{rgb}{0.763363,0.835092,0.955658}%
\pgfsetfillcolor{currentfill}%
\pgfsetlinewidth{0.000000pt}%
\definecolor{currentstroke}{rgb}{0.000000,0.000000,0.000000}%
\pgfsetstrokecolor{currentstroke}%
\pgfsetdash{}{0pt}%
\pgfpathmoveto{\pgfqpoint{3.857060in}{3.376577in}}%
\pgfpathlineto{\pgfqpoint{3.866806in}{3.270214in}}%
\pgfpathlineto{\pgfqpoint{3.876915in}{3.046084in}}%
\pgfpathlineto{\pgfqpoint{3.911195in}{3.327369in}}%
\pgfpathlineto{\pgfqpoint{3.946486in}{3.256268in}}%
\pgfpathlineto{\pgfqpoint{3.936827in}{3.324612in}}%
\pgfpathlineto{\pgfqpoint{3.927242in}{3.367098in}}%
\pgfpathlineto{\pgfqpoint{3.892072in}{3.403400in}}%
\pgfpathlineto{\pgfqpoint{3.857060in}{3.376577in}}%
\pgfpathclose%
\pgfusepath{fill}%
\end{pgfscope}%
\begin{pgfscope}%
\pgfpathrectangle{\pgfqpoint{1.020000in}{0.880000in}}{\pgfqpoint{6.160000in}{6.160000in}}%
\pgfusepath{clip}%
\pgfsetbuttcap%
\pgfsetroundjoin%
\definecolor{currentfill}{rgb}{0.559747,0.694768,0.996075}%
\pgfsetfillcolor{currentfill}%
\pgfsetlinewidth{0.000000pt}%
\definecolor{currentstroke}{rgb}{0.000000,0.000000,0.000000}%
\pgfsetstrokecolor{currentstroke}%
\pgfsetdash{}{0pt}%
\pgfpathmoveto{\pgfqpoint{3.138845in}{2.854875in}}%
\pgfpathlineto{\pgfqpoint{3.147402in}{2.858133in}}%
\pgfpathlineto{\pgfqpoint{3.156781in}{2.783713in}}%
\pgfpathlineto{\pgfqpoint{3.189780in}{3.016302in}}%
\pgfpathlineto{\pgfqpoint{3.227565in}{2.766242in}}%
\pgfpathlineto{\pgfqpoint{3.217286in}{2.931442in}}%
\pgfpathlineto{\pgfqpoint{3.208063in}{2.987731in}}%
\pgfpathlineto{\pgfqpoint{3.172613in}{3.002973in}}%
\pgfpathlineto{\pgfqpoint{3.138845in}{2.854875in}}%
\pgfpathclose%
\pgfusepath{fill}%
\end{pgfscope}%
\begin{pgfscope}%
\pgfpathrectangle{\pgfqpoint{1.020000in}{0.880000in}}{\pgfqpoint{6.160000in}{6.160000in}}%
\pgfusepath{clip}%
\pgfsetbuttcap%
\pgfsetroundjoin%
\definecolor{currentfill}{rgb}{0.581486,0.713451,0.998314}%
\pgfsetfillcolor{currentfill}%
\pgfsetlinewidth{0.000000pt}%
\definecolor{currentstroke}{rgb}{0.000000,0.000000,0.000000}%
\pgfsetstrokecolor{currentstroke}%
\pgfsetdash{}{0pt}%
\pgfpathmoveto{\pgfqpoint{4.265620in}{3.015943in}}%
\pgfpathlineto{\pgfqpoint{4.275462in}{2.991488in}}%
\pgfpathlineto{\pgfqpoint{4.285248in}{2.895121in}}%
\pgfpathlineto{\pgfqpoint{4.320269in}{2.916893in}}%
\pgfpathlineto{\pgfqpoint{4.355054in}{2.805987in}}%
\pgfpathlineto{\pgfqpoint{4.345507in}{3.045579in}}%
\pgfpathlineto{\pgfqpoint{4.335374in}{2.919414in}}%
\pgfpathlineto{\pgfqpoint{4.300525in}{2.967745in}}%
\pgfpathlineto{\pgfqpoint{4.265620in}{3.015943in}}%
\pgfpathclose%
\pgfusepath{fill}%
\end{pgfscope}%
\begin{pgfscope}%
\pgfpathrectangle{\pgfqpoint{1.020000in}{0.880000in}}{\pgfqpoint{6.160000in}{6.160000in}}%
\pgfusepath{clip}%
\pgfsetbuttcap%
\pgfsetroundjoin%
\definecolor{currentfill}{rgb}{0.532568,0.669801,0.990393}%
\pgfsetfillcolor{currentfill}%
\pgfsetlinewidth{0.000000pt}%
\definecolor{currentstroke}{rgb}{0.000000,0.000000,0.000000}%
\pgfsetstrokecolor{currentstroke}%
\pgfsetdash{}{0pt}%
\pgfpathmoveto{\pgfqpoint{3.067624in}{2.909544in}}%
\pgfpathlineto{\pgfqpoint{3.077942in}{2.748118in}}%
\pgfpathlineto{\pgfqpoint{3.086899in}{2.709181in}}%
\pgfpathlineto{\pgfqpoint{3.120455in}{2.875763in}}%
\pgfpathlineto{\pgfqpoint{3.156781in}{2.783713in}}%
\pgfpathlineto{\pgfqpoint{3.147402in}{2.858133in}}%
\pgfpathlineto{\pgfqpoint{3.138845in}{2.854875in}}%
\pgfpathlineto{\pgfqpoint{3.101468in}{3.046827in}}%
\pgfpathlineto{\pgfqpoint{3.067624in}{2.909544in}}%
\pgfpathclose%
\pgfusepath{fill}%
\end{pgfscope}%
\begin{pgfscope}%
\pgfpathrectangle{\pgfqpoint{1.020000in}{0.880000in}}{\pgfqpoint{6.160000in}{6.160000in}}%
\pgfusepath{clip}%
\pgfsetbuttcap%
\pgfsetroundjoin%
\definecolor{currentfill}{rgb}{0.430507,0.564883,0.948889}%
\pgfsetfillcolor{currentfill}%
\pgfsetlinewidth{0.000000pt}%
\definecolor{currentstroke}{rgb}{0.000000,0.000000,0.000000}%
\pgfsetstrokecolor{currentstroke}%
\pgfsetdash{}{0pt}%
\pgfpathmoveto{\pgfqpoint{4.674377in}{2.699875in}}%
\pgfpathlineto{\pgfqpoint{4.684424in}{2.644151in}}%
\pgfpathlineto{\pgfqpoint{4.718344in}{2.469322in}}%
\pgfpathlineto{\pgfqpoint{4.754673in}{2.738292in}}%
\pgfpathlineto{\pgfqpoint{4.744239in}{2.733040in}}%
\pgfpathlineto{\pgfqpoint{4.708988in}{2.655106in}}%
\pgfpathlineto{\pgfqpoint{4.674377in}{2.699875in}}%
\pgfpathclose%
\pgfusepath{fill}%
\end{pgfscope}%
\begin{pgfscope}%
\pgfpathrectangle{\pgfqpoint{1.020000in}{0.880000in}}{\pgfqpoint{6.160000in}{6.160000in}}%
\pgfusepath{clip}%
\pgfsetbuttcap%
\pgfsetroundjoin%
\definecolor{currentfill}{rgb}{0.743754,0.825125,0.965798}%
\pgfsetfillcolor{currentfill}%
\pgfsetlinewidth{0.000000pt}%
\definecolor{currentstroke}{rgb}{0.000000,0.000000,0.000000}%
\pgfsetstrokecolor{currentstroke}%
\pgfsetdash{}{0pt}%
\pgfpathmoveto{\pgfqpoint{3.786887in}{3.358404in}}%
\pgfpathlineto{\pgfqpoint{3.796432in}{3.304137in}}%
\pgfpathlineto{\pgfqpoint{3.805862in}{3.283074in}}%
\pgfpathlineto{\pgfqpoint{3.841465in}{3.157462in}}%
\pgfpathlineto{\pgfqpoint{3.876915in}{3.046084in}}%
\pgfpathlineto{\pgfqpoint{3.866806in}{3.270214in}}%
\pgfpathlineto{\pgfqpoint{3.857060in}{3.376577in}}%
\pgfpathlineto{\pgfqpoint{3.822394in}{3.253380in}}%
\pgfpathlineto{\pgfqpoint{3.786887in}{3.358404in}}%
\pgfpathclose%
\pgfusepath{fill}%
\end{pgfscope}%
\begin{pgfscope}%
\pgfpathrectangle{\pgfqpoint{1.020000in}{0.880000in}}{\pgfqpoint{6.160000in}{6.160000in}}%
\pgfusepath{clip}%
\pgfsetbuttcap%
\pgfsetroundjoin%
\definecolor{currentfill}{rgb}{0.473070,0.611077,0.970634}%
\pgfsetfillcolor{currentfill}%
\pgfsetlinewidth{0.000000pt}%
\definecolor{currentstroke}{rgb}{0.000000,0.000000,0.000000}%
\pgfsetstrokecolor{currentstroke}%
\pgfsetdash{}{0pt}%
\pgfpathmoveto{\pgfqpoint{2.930021in}{2.596500in}}%
\pgfpathlineto{\pgfqpoint{2.935203in}{2.848655in}}%
\pgfpathlineto{\pgfqpoint{2.945850in}{2.667407in}}%
\pgfpathlineto{\pgfqpoint{2.981916in}{2.613278in}}%
\pgfpathlineto{\pgfqpoint{3.015246in}{2.786795in}}%
\pgfpathlineto{\pgfqpoint{3.007692in}{2.710560in}}%
\pgfpathlineto{\pgfqpoint{2.998628in}{2.763717in}}%
\pgfpathlineto{\pgfqpoint{2.961122in}{2.939338in}}%
\pgfpathlineto{\pgfqpoint{2.930021in}{2.596500in}}%
\pgfpathclose%
\pgfusepath{fill}%
\end{pgfscope}%
\begin{pgfscope}%
\pgfpathrectangle{\pgfqpoint{1.020000in}{0.880000in}}{\pgfqpoint{6.160000in}{6.160000in}}%
\pgfusepath{clip}%
\pgfsetbuttcap%
\pgfsetroundjoin%
\definecolor{currentfill}{rgb}{0.651398,0.768121,0.995891}%
\pgfsetfillcolor{currentfill}%
\pgfsetlinewidth{0.000000pt}%
\definecolor{currentstroke}{rgb}{0.000000,0.000000,0.000000}%
\pgfsetstrokecolor{currentstroke}%
\pgfsetdash{}{0pt}%
\pgfpathmoveto{\pgfqpoint{4.106157in}{3.046332in}}%
\pgfpathlineto{\pgfqpoint{4.116000in}{2.805002in}}%
\pgfpathlineto{\pgfqpoint{4.125490in}{3.129592in}}%
\pgfpathlineto{\pgfqpoint{4.160604in}{3.016899in}}%
\pgfpathlineto{\pgfqpoint{4.195646in}{3.133322in}}%
\pgfpathlineto{\pgfqpoint{4.185859in}{2.915788in}}%
\pgfpathlineto{\pgfqpoint{4.176094in}{3.277921in}}%
\pgfpathlineto{\pgfqpoint{4.141064in}{3.236895in}}%
\pgfpathlineto{\pgfqpoint{4.106157in}{3.046332in}}%
\pgfpathclose%
\pgfusepath{fill}%
\end{pgfscope}%
\begin{pgfscope}%
\pgfpathrectangle{\pgfqpoint{1.020000in}{0.880000in}}{\pgfqpoint{6.160000in}{6.160000in}}%
\pgfusepath{clip}%
\pgfsetbuttcap%
\pgfsetroundjoin%
\definecolor{currentfill}{rgb}{0.414801,0.546874,0.939088}%
\pgfsetfillcolor{currentfill}%
\pgfsetlinewidth{0.000000pt}%
\definecolor{currentstroke}{rgb}{0.000000,0.000000,0.000000}%
\pgfsetstrokecolor{currentstroke}%
\pgfsetdash{}{0pt}%
\pgfpathmoveto{\pgfqpoint{2.786961in}{2.709822in}}%
\pgfpathlineto{\pgfqpoint{2.795819in}{2.663343in}}%
\pgfpathlineto{\pgfqpoint{2.803872in}{2.675512in}}%
\pgfpathlineto{\pgfqpoint{2.840674in}{2.579566in}}%
\pgfpathlineto{\pgfqpoint{2.875448in}{2.630553in}}%
\pgfpathlineto{\pgfqpoint{2.866435in}{2.684731in}}%
\pgfpathlineto{\pgfqpoint{2.861767in}{2.414118in}}%
\pgfpathlineto{\pgfqpoint{2.822910in}{2.672689in}}%
\pgfpathlineto{\pgfqpoint{2.786961in}{2.709822in}}%
\pgfpathclose%
\pgfusepath{fill}%
\end{pgfscope}%
\begin{pgfscope}%
\pgfpathrectangle{\pgfqpoint{1.020000in}{0.880000in}}{\pgfqpoint{6.160000in}{6.160000in}}%
\pgfusepath{clip}%
\pgfsetbuttcap%
\pgfsetroundjoin%
\definecolor{currentfill}{rgb}{0.693321,0.796314,0.986308}%
\pgfsetfillcolor{currentfill}%
\pgfsetlinewidth{0.000000pt}%
\definecolor{currentstroke}{rgb}{0.000000,0.000000,0.000000}%
\pgfsetstrokecolor{currentstroke}%
\pgfsetdash{}{0pt}%
\pgfpathmoveto{\pgfqpoint{3.646980in}{3.205881in}}%
\pgfpathlineto{\pgfqpoint{3.655585in}{3.313782in}}%
\pgfpathlineto{\pgfqpoint{3.666438in}{2.996242in}}%
\pgfpathlineto{\pgfqpoint{3.701805in}{2.953444in}}%
\pgfpathlineto{\pgfqpoint{3.735545in}{3.261841in}}%
\pgfpathlineto{\pgfqpoint{3.726491in}{3.213892in}}%
\pgfpathlineto{\pgfqpoint{3.717583in}{3.141082in}}%
\pgfpathlineto{\pgfqpoint{3.682296in}{3.174961in}}%
\pgfpathlineto{\pgfqpoint{3.646980in}{3.205881in}}%
\pgfpathclose%
\pgfusepath{fill}%
\end{pgfscope}%
\begin{pgfscope}%
\pgfpathrectangle{\pgfqpoint{1.020000in}{0.880000in}}{\pgfqpoint{6.160000in}{6.160000in}}%
\pgfusepath{clip}%
\pgfsetbuttcap%
\pgfsetroundjoin%
\definecolor{currentfill}{rgb}{0.527132,0.664700,0.989065}%
\pgfsetfillcolor{currentfill}%
\pgfsetlinewidth{0.000000pt}%
\definecolor{currentstroke}{rgb}{0.000000,0.000000,0.000000}%
\pgfsetstrokecolor{currentstroke}%
\pgfsetdash{}{0pt}%
\pgfpathmoveto{\pgfqpoint{4.355054in}{2.805987in}}%
\pgfpathlineto{\pgfqpoint{4.365037in}{2.818253in}}%
\pgfpathlineto{\pgfqpoint{4.375074in}{2.845912in}}%
\pgfpathlineto{\pgfqpoint{4.409990in}{2.808065in}}%
\pgfpathlineto{\pgfqpoint{4.444820in}{2.755980in}}%
\pgfpathlineto{\pgfqpoint{4.434785in}{2.761753in}}%
\pgfpathlineto{\pgfqpoint{4.425295in}{2.982865in}}%
\pgfpathlineto{\pgfqpoint{4.390168in}{2.900674in}}%
\pgfpathlineto{\pgfqpoint{4.355054in}{2.805987in}}%
\pgfpathclose%
\pgfusepath{fill}%
\end{pgfscope}%
\begin{pgfscope}%
\pgfpathrectangle{\pgfqpoint{1.020000in}{0.880000in}}{\pgfqpoint{6.160000in}{6.160000in}}%
\pgfusepath{clip}%
\pgfsetbuttcap%
\pgfsetroundjoin%
\definecolor{currentfill}{rgb}{0.713852,0.808857,0.979386}%
\pgfsetfillcolor{currentfill}%
\pgfsetlinewidth{0.000000pt}%
\definecolor{currentstroke}{rgb}{0.000000,0.000000,0.000000}%
\pgfsetstrokecolor{currentstroke}%
\pgfsetdash{}{0pt}%
\pgfpathmoveto{\pgfqpoint{3.946486in}{3.256268in}}%
\pgfpathlineto{\pgfqpoint{3.956260in}{3.141365in}}%
\pgfpathlineto{\pgfqpoint{3.965803in}{3.132011in}}%
\pgfpathlineto{\pgfqpoint{4.001007in}{3.077848in}}%
\pgfpathlineto{\pgfqpoint{4.035783in}{3.272661in}}%
\pgfpathlineto{\pgfqpoint{4.026169in}{3.276765in}}%
\pgfpathlineto{\pgfqpoint{4.016546in}{3.298539in}}%
\pgfpathlineto{\pgfqpoint{3.981890in}{3.094148in}}%
\pgfpathlineto{\pgfqpoint{3.946486in}{3.256268in}}%
\pgfpathclose%
\pgfusepath{fill}%
\end{pgfscope}%
\begin{pgfscope}%
\pgfpathrectangle{\pgfqpoint{1.020000in}{0.880000in}}{\pgfqpoint{6.160000in}{6.160000in}}%
\pgfusepath{clip}%
\pgfsetbuttcap%
\pgfsetroundjoin%
\definecolor{currentfill}{rgb}{0.394042,0.522413,0.924916}%
\pgfsetfillcolor{currentfill}%
\pgfsetlinewidth{0.000000pt}%
\definecolor{currentstroke}{rgb}{0.000000,0.000000,0.000000}%
\pgfsetstrokecolor{currentstroke}%
\pgfsetdash{}{0pt}%
\pgfpathmoveto{\pgfqpoint{4.534213in}{2.543997in}}%
\pgfpathlineto{\pgfqpoint{4.544602in}{2.604382in}}%
\pgfpathlineto{\pgfqpoint{4.579546in}{2.605947in}}%
\pgfpathlineto{\pgfqpoint{4.613912in}{2.479285in}}%
\pgfpathlineto{\pgfqpoint{4.604054in}{2.567855in}}%
\pgfpathlineto{\pgfqpoint{4.569671in}{2.692031in}}%
\pgfpathlineto{\pgfqpoint{4.534213in}{2.543997in}}%
\pgfpathclose%
\pgfusepath{fill}%
\end{pgfscope}%
\begin{pgfscope}%
\pgfpathrectangle{\pgfqpoint{1.020000in}{0.880000in}}{\pgfqpoint{6.160000in}{6.160000in}}%
\pgfusepath{clip}%
\pgfsetbuttcap%
\pgfsetroundjoin%
\definecolor{currentfill}{rgb}{0.548876,0.685104,0.994379}%
\pgfsetfillcolor{currentfill}%
\pgfsetlinewidth{0.000000pt}%
\definecolor{currentstroke}{rgb}{0.000000,0.000000,0.000000}%
\pgfsetstrokecolor{currentstroke}%
\pgfsetdash{}{0pt}%
\pgfpathmoveto{\pgfqpoint{3.297504in}{2.831007in}}%
\pgfpathlineto{\pgfqpoint{3.307327in}{2.712164in}}%
\pgfpathlineto{\pgfqpoint{3.315115in}{2.826230in}}%
\pgfpathlineto{\pgfqpoint{3.350332in}{2.838437in}}%
\pgfpathlineto{\pgfqpoint{3.384635in}{2.962987in}}%
\pgfpathlineto{\pgfqpoint{3.377004in}{2.809026in}}%
\pgfpathlineto{\pgfqpoint{3.366283in}{3.037668in}}%
\pgfpathlineto{\pgfqpoint{3.331167in}{3.016226in}}%
\pgfpathlineto{\pgfqpoint{3.297504in}{2.831007in}}%
\pgfpathclose%
\pgfusepath{fill}%
\end{pgfscope}%
\begin{pgfscope}%
\pgfpathrectangle{\pgfqpoint{1.020000in}{0.880000in}}{\pgfqpoint{6.160000in}{6.160000in}}%
\pgfusepath{clip}%
\pgfsetbuttcap%
\pgfsetroundjoin%
\definecolor{currentfill}{rgb}{0.743754,0.825125,0.965798}%
\pgfsetfillcolor{currentfill}%
\pgfsetlinewidth{0.000000pt}%
\definecolor{currentstroke}{rgb}{0.000000,0.000000,0.000000}%
\pgfsetstrokecolor{currentstroke}%
\pgfsetdash{}{0pt}%
\pgfpathmoveto{\pgfqpoint{3.717583in}{3.141082in}}%
\pgfpathlineto{\pgfqpoint{3.726491in}{3.213892in}}%
\pgfpathlineto{\pgfqpoint{3.735545in}{3.261841in}}%
\pgfpathlineto{\pgfqpoint{3.771179in}{3.159088in}}%
\pgfpathlineto{\pgfqpoint{3.805862in}{3.283074in}}%
\pgfpathlineto{\pgfqpoint{3.796432in}{3.304137in}}%
\pgfpathlineto{\pgfqpoint{3.786887in}{3.358404in}}%
\pgfpathlineto{\pgfqpoint{3.751936in}{3.312750in}}%
\pgfpathlineto{\pgfqpoint{3.717583in}{3.141082in}}%
\pgfpathclose%
\pgfusepath{fill}%
\end{pgfscope}%
\begin{pgfscope}%
\pgfpathrectangle{\pgfqpoint{1.020000in}{0.880000in}}{\pgfqpoint{6.160000in}{6.160000in}}%
\pgfusepath{clip}%
\pgfsetbuttcap%
\pgfsetroundjoin%
\definecolor{currentfill}{rgb}{0.521696,0.659599,0.987736}%
\pgfsetfillcolor{currentfill}%
\pgfsetlinewidth{0.000000pt}%
\definecolor{currentstroke}{rgb}{0.000000,0.000000,0.000000}%
\pgfsetstrokecolor{currentstroke}%
\pgfsetdash{}{0pt}%
\pgfpathmoveto{\pgfqpoint{3.227565in}{2.766242in}}%
\pgfpathlineto{\pgfqpoint{3.235335in}{2.863303in}}%
\pgfpathlineto{\pgfqpoint{3.243905in}{2.879518in}}%
\pgfpathlineto{\pgfqpoint{3.279379in}{2.869502in}}%
\pgfpathlineto{\pgfqpoint{3.315115in}{2.826230in}}%
\pgfpathlineto{\pgfqpoint{3.307327in}{2.712164in}}%
\pgfpathlineto{\pgfqpoint{3.297504in}{2.831007in}}%
\pgfpathlineto{\pgfqpoint{3.262340in}{2.819616in}}%
\pgfpathlineto{\pgfqpoint{3.227565in}{2.766242in}}%
\pgfpathclose%
\pgfusepath{fill}%
\end{pgfscope}%
\begin{pgfscope}%
\pgfpathrectangle{\pgfqpoint{1.020000in}{0.880000in}}{\pgfqpoint{6.160000in}{6.160000in}}%
\pgfusepath{clip}%
\pgfsetbuttcap%
\pgfsetroundjoin%
\definecolor{currentfill}{rgb}{0.698454,0.799450,0.984577}%
\pgfsetfillcolor{currentfill}%
\pgfsetlinewidth{0.000000pt}%
\definecolor{currentstroke}{rgb}{0.000000,0.000000,0.000000}%
\pgfsetstrokecolor{currentstroke}%
\pgfsetdash{}{0pt}%
\pgfpathmoveto{\pgfqpoint{3.876915in}{3.046084in}}%
\pgfpathlineto{\pgfqpoint{3.885971in}{3.167920in}}%
\pgfpathlineto{\pgfqpoint{3.895519in}{3.134644in}}%
\pgfpathlineto{\pgfqpoint{3.930843in}{3.065825in}}%
\pgfpathlineto{\pgfqpoint{3.965803in}{3.132011in}}%
\pgfpathlineto{\pgfqpoint{3.956260in}{3.141365in}}%
\pgfpathlineto{\pgfqpoint{3.946486in}{3.256268in}}%
\pgfpathlineto{\pgfqpoint{3.911195in}{3.327369in}}%
\pgfpathlineto{\pgfqpoint{3.876915in}{3.046084in}}%
\pgfpathclose%
\pgfusepath{fill}%
\end{pgfscope}%
\begin{pgfscope}%
\pgfpathrectangle{\pgfqpoint{1.020000in}{0.880000in}}{\pgfqpoint{6.160000in}{6.160000in}}%
\pgfusepath{clip}%
\pgfsetbuttcap%
\pgfsetroundjoin%
\definecolor{currentfill}{rgb}{0.586921,0.718121,0.998874}%
\pgfsetfillcolor{currentfill}%
\pgfsetlinewidth{0.000000pt}%
\definecolor{currentstroke}{rgb}{0.000000,0.000000,0.000000}%
\pgfsetstrokecolor{currentstroke}%
\pgfsetdash{}{0pt}%
\pgfpathmoveto{\pgfqpoint{3.366283in}{3.037668in}}%
\pgfpathlineto{\pgfqpoint{3.377004in}{2.809026in}}%
\pgfpathlineto{\pgfqpoint{3.384635in}{2.962987in}}%
\pgfpathlineto{\pgfqpoint{3.419633in}{3.005485in}}%
\pgfpathlineto{\pgfqpoint{3.455247in}{2.964991in}}%
\pgfpathlineto{\pgfqpoint{3.446231in}{2.973363in}}%
\pgfpathlineto{\pgfqpoint{3.437278in}{2.975951in}}%
\pgfpathlineto{\pgfqpoint{3.402980in}{2.856686in}}%
\pgfpathlineto{\pgfqpoint{3.366283in}{3.037668in}}%
\pgfpathclose%
\pgfusepath{fill}%
\end{pgfscope}%
\begin{pgfscope}%
\pgfpathrectangle{\pgfqpoint{1.020000in}{0.880000in}}{\pgfqpoint{6.160000in}{6.160000in}}%
\pgfusepath{clip}%
\pgfsetbuttcap%
\pgfsetroundjoin%
\definecolor{currentfill}{rgb}{0.693321,0.796314,0.986308}%
\pgfsetfillcolor{currentfill}%
\pgfsetlinewidth{0.000000pt}%
\definecolor{currentstroke}{rgb}{0.000000,0.000000,0.000000}%
\pgfsetstrokecolor{currentstroke}%
\pgfsetdash{}{0pt}%
\pgfpathmoveto{\pgfqpoint{3.577610in}{3.038593in}}%
\pgfpathlineto{\pgfqpoint{3.586375in}{3.092286in}}%
\pgfpathlineto{\pgfqpoint{3.595872in}{3.026988in}}%
\pgfpathlineto{\pgfqpoint{3.631046in}{3.034226in}}%
\pgfpathlineto{\pgfqpoint{3.666438in}{2.996242in}}%
\pgfpathlineto{\pgfqpoint{3.655585in}{3.313782in}}%
\pgfpathlineto{\pgfqpoint{3.646980in}{3.205881in}}%
\pgfpathlineto{\pgfqpoint{3.610316in}{3.464466in}}%
\pgfpathlineto{\pgfqpoint{3.577610in}{3.038593in}}%
\pgfpathclose%
\pgfusepath{fill}%
\end{pgfscope}%
\begin{pgfscope}%
\pgfpathrectangle{\pgfqpoint{1.020000in}{0.880000in}}{\pgfqpoint{6.160000in}{6.160000in}}%
\pgfusepath{clip}%
\pgfsetbuttcap%
\pgfsetroundjoin%
\definecolor{currentfill}{rgb}{0.473070,0.611077,0.970634}%
\pgfsetfillcolor{currentfill}%
\pgfsetlinewidth{0.000000pt}%
\definecolor{currentstroke}{rgb}{0.000000,0.000000,0.000000}%
\pgfsetstrokecolor{currentstroke}%
\pgfsetdash{}{0pt}%
\pgfpathmoveto{\pgfqpoint{4.444820in}{2.755980in}}%
\pgfpathlineto{\pgfqpoint{4.454972in}{2.785085in}}%
\pgfpathlineto{\pgfqpoint{4.464924in}{2.733580in}}%
\pgfpathlineto{\pgfqpoint{4.499727in}{2.674346in}}%
\pgfpathlineto{\pgfqpoint{4.534213in}{2.543997in}}%
\pgfpathlineto{\pgfqpoint{4.524630in}{2.707360in}}%
\pgfpathlineto{\pgfqpoint{4.515059in}{2.877747in}}%
\pgfpathlineto{\pgfqpoint{4.479644in}{2.721469in}}%
\pgfpathlineto{\pgfqpoint{4.444820in}{2.755980in}}%
\pgfpathclose%
\pgfusepath{fill}%
\end{pgfscope}%
\begin{pgfscope}%
\pgfpathrectangle{\pgfqpoint{1.020000in}{0.880000in}}{\pgfqpoint{6.160000in}{6.160000in}}%
\pgfusepath{clip}%
\pgfsetbuttcap%
\pgfsetroundjoin%
\definecolor{currentfill}{rgb}{0.661968,0.775491,0.993937}%
\pgfsetfillcolor{currentfill}%
\pgfsetlinewidth{0.000000pt}%
\definecolor{currentstroke}{rgb}{0.000000,0.000000,0.000000}%
\pgfsetstrokecolor{currentstroke}%
\pgfsetdash{}{0pt}%
\pgfpathmoveto{\pgfqpoint{4.035783in}{3.272661in}}%
\pgfpathlineto{\pgfqpoint{4.045559in}{3.165754in}}%
\pgfpathlineto{\pgfqpoint{4.055336in}{3.059495in}}%
\pgfpathlineto{\pgfqpoint{4.090412in}{3.094349in}}%
\pgfpathlineto{\pgfqpoint{4.125490in}{3.129592in}}%
\pgfpathlineto{\pgfqpoint{4.116000in}{2.805002in}}%
\pgfpathlineto{\pgfqpoint{4.106157in}{3.046332in}}%
\pgfpathlineto{\pgfqpoint{4.071042in}{3.140372in}}%
\pgfpathlineto{\pgfqpoint{4.035783in}{3.272661in}}%
\pgfpathclose%
\pgfusepath{fill}%
\end{pgfscope}%
\begin{pgfscope}%
\pgfpathrectangle{\pgfqpoint{1.020000in}{0.880000in}}{\pgfqpoint{6.160000in}{6.160000in}}%
\pgfusepath{clip}%
\pgfsetbuttcap%
\pgfsetroundjoin%
\definecolor{currentfill}{rgb}{0.613933,0.739923,0.999142}%
\pgfsetfillcolor{currentfill}%
\pgfsetlinewidth{0.000000pt}%
\definecolor{currentstroke}{rgb}{0.000000,0.000000,0.000000}%
\pgfsetstrokecolor{currentstroke}%
\pgfsetdash{}{0pt}%
\pgfpathmoveto{\pgfqpoint{4.195646in}{3.133322in}}%
\pgfpathlineto{\pgfqpoint{4.205412in}{2.961317in}}%
\pgfpathlineto{\pgfqpoint{4.215247in}{3.043825in}}%
\pgfpathlineto{\pgfqpoint{4.250181in}{2.816073in}}%
\pgfpathlineto{\pgfqpoint{4.285248in}{2.895121in}}%
\pgfpathlineto{\pgfqpoint{4.275462in}{2.991488in}}%
\pgfpathlineto{\pgfqpoint{4.265620in}{3.015943in}}%
\pgfpathlineto{\pgfqpoint{4.230673in}{3.091599in}}%
\pgfpathlineto{\pgfqpoint{4.195646in}{3.133322in}}%
\pgfpathclose%
\pgfusepath{fill}%
\end{pgfscope}%
\begin{pgfscope}%
\pgfpathrectangle{\pgfqpoint{1.020000in}{0.880000in}}{\pgfqpoint{6.160000in}{6.160000in}}%
\pgfusepath{clip}%
\pgfsetbuttcap%
\pgfsetroundjoin%
\definecolor{currentfill}{rgb}{0.677823,0.786546,0.991005}%
\pgfsetfillcolor{currentfill}%
\pgfsetlinewidth{0.000000pt}%
\definecolor{currentstroke}{rgb}{0.000000,0.000000,0.000000}%
\pgfsetstrokecolor{currentstroke}%
\pgfsetdash{}{0pt}%
\pgfpathmoveto{\pgfqpoint{3.506433in}{3.159405in}}%
\pgfpathlineto{\pgfqpoint{3.514669in}{3.277168in}}%
\pgfpathlineto{\pgfqpoint{3.524426in}{3.170892in}}%
\pgfpathlineto{\pgfqpoint{3.560512in}{3.046710in}}%
\pgfpathlineto{\pgfqpoint{3.595872in}{3.026988in}}%
\pgfpathlineto{\pgfqpoint{3.586375in}{3.092286in}}%
\pgfpathlineto{\pgfqpoint{3.577610in}{3.038593in}}%
\pgfpathlineto{\pgfqpoint{3.542045in}{3.100409in}}%
\pgfpathlineto{\pgfqpoint{3.506433in}{3.159405in}}%
\pgfpathclose%
\pgfusepath{fill}%
\end{pgfscope}%
\begin{pgfscope}%
\pgfpathrectangle{\pgfqpoint{1.020000in}{0.880000in}}{\pgfqpoint{6.160000in}{6.160000in}}%
\pgfusepath{clip}%
\pgfsetbuttcap%
\pgfsetroundjoin%
\definecolor{currentfill}{rgb}{0.538004,0.674902,0.991722}%
\pgfsetfillcolor{currentfill}%
\pgfsetlinewidth{0.000000pt}%
\definecolor{currentstroke}{rgb}{0.000000,0.000000,0.000000}%
\pgfsetstrokecolor{currentstroke}%
\pgfsetdash{}{0pt}%
\pgfpathmoveto{\pgfqpoint{4.285248in}{2.895121in}}%
\pgfpathlineto{\pgfqpoint{4.295162in}{2.910294in}}%
\pgfpathlineto{\pgfqpoint{4.304901in}{2.762011in}}%
\pgfpathlineto{\pgfqpoint{4.340036in}{2.836878in}}%
\pgfpathlineto{\pgfqpoint{4.375074in}{2.845912in}}%
\pgfpathlineto{\pgfqpoint{4.365037in}{2.818253in}}%
\pgfpathlineto{\pgfqpoint{4.355054in}{2.805987in}}%
\pgfpathlineto{\pgfqpoint{4.320269in}{2.916893in}}%
\pgfpathlineto{\pgfqpoint{4.285248in}{2.895121in}}%
\pgfpathclose%
\pgfusepath{fill}%
\end{pgfscope}%
\begin{pgfscope}%
\pgfpathrectangle{\pgfqpoint{1.020000in}{0.880000in}}{\pgfqpoint{6.160000in}{6.160000in}}%
\pgfusepath{clip}%
\pgfsetbuttcap%
\pgfsetroundjoin%
\definecolor{currentfill}{rgb}{0.516260,0.654498,0.986407}%
\pgfsetfillcolor{currentfill}%
\pgfsetlinewidth{0.000000pt}%
\definecolor{currentstroke}{rgb}{0.000000,0.000000,0.000000}%
\pgfsetstrokecolor{currentstroke}%
\pgfsetdash{}{0pt}%
\pgfpathmoveto{\pgfqpoint{3.156781in}{2.783713in}}%
\pgfpathlineto{\pgfqpoint{3.167068in}{2.621225in}}%
\pgfpathlineto{\pgfqpoint{3.173037in}{2.883064in}}%
\pgfpathlineto{\pgfqpoint{3.210687in}{2.656429in}}%
\pgfpathlineto{\pgfqpoint{3.243905in}{2.879518in}}%
\pgfpathlineto{\pgfqpoint{3.235335in}{2.863303in}}%
\pgfpathlineto{\pgfqpoint{3.227565in}{2.766242in}}%
\pgfpathlineto{\pgfqpoint{3.189780in}{3.016302in}}%
\pgfpathlineto{\pgfqpoint{3.156781in}{2.783713in}}%
\pgfpathclose%
\pgfusepath{fill}%
\end{pgfscope}%
\begin{pgfscope}%
\pgfpathrectangle{\pgfqpoint{1.020000in}{0.880000in}}{\pgfqpoint{6.160000in}{6.160000in}}%
\pgfusepath{clip}%
\pgfsetbuttcap%
\pgfsetroundjoin%
\definecolor{currentfill}{rgb}{0.698454,0.799450,0.984577}%
\pgfsetfillcolor{currentfill}%
\pgfsetlinewidth{0.000000pt}%
\definecolor{currentstroke}{rgb}{0.000000,0.000000,0.000000}%
\pgfsetstrokecolor{currentstroke}%
\pgfsetdash{}{0pt}%
\pgfpathmoveto{\pgfqpoint{3.805862in}{3.283074in}}%
\pgfpathlineto{\pgfqpoint{3.815817in}{3.124097in}}%
\pgfpathlineto{\pgfqpoint{3.825287in}{3.097656in}}%
\pgfpathlineto{\pgfqpoint{3.860169in}{3.189663in}}%
\pgfpathlineto{\pgfqpoint{3.895519in}{3.134644in}}%
\pgfpathlineto{\pgfqpoint{3.885971in}{3.167920in}}%
\pgfpathlineto{\pgfqpoint{3.876915in}{3.046084in}}%
\pgfpathlineto{\pgfqpoint{3.841465in}{3.157462in}}%
\pgfpathlineto{\pgfqpoint{3.805862in}{3.283074in}}%
\pgfpathclose%
\pgfusepath{fill}%
\end{pgfscope}%
\begin{pgfscope}%
\pgfpathrectangle{\pgfqpoint{1.020000in}{0.880000in}}{\pgfqpoint{6.160000in}{6.160000in}}%
\pgfusepath{clip}%
\pgfsetbuttcap%
\pgfsetroundjoin%
\definecolor{currentfill}{rgb}{0.656683,0.771806,0.994914}%
\pgfsetfillcolor{currentfill}%
\pgfsetlinewidth{0.000000pt}%
\definecolor{currentstroke}{rgb}{0.000000,0.000000,0.000000}%
\pgfsetstrokecolor{currentstroke}%
\pgfsetdash{}{0pt}%
\pgfpathmoveto{\pgfqpoint{3.437278in}{2.975951in}}%
\pgfpathlineto{\pgfqpoint{3.446231in}{2.973363in}}%
\pgfpathlineto{\pgfqpoint{3.455247in}{2.964991in}}%
\pgfpathlineto{\pgfqpoint{3.489722in}{3.080769in}}%
\pgfpathlineto{\pgfqpoint{3.524426in}{3.170892in}}%
\pgfpathlineto{\pgfqpoint{3.514669in}{3.277168in}}%
\pgfpathlineto{\pgfqpoint{3.506433in}{3.159405in}}%
\pgfpathlineto{\pgfqpoint{3.472356in}{2.994562in}}%
\pgfpathlineto{\pgfqpoint{3.437278in}{2.975951in}}%
\pgfpathclose%
\pgfusepath{fill}%
\end{pgfscope}%
\begin{pgfscope}%
\pgfpathrectangle{\pgfqpoint{1.020000in}{0.880000in}}{\pgfqpoint{6.160000in}{6.160000in}}%
\pgfusepath{clip}%
\pgfsetbuttcap%
\pgfsetroundjoin%
\definecolor{currentfill}{rgb}{0.505423,0.643995,0.983157}%
\pgfsetfillcolor{currentfill}%
\pgfsetlinewidth{0.000000pt}%
\definecolor{currentstroke}{rgb}{0.000000,0.000000,0.000000}%
\pgfsetstrokecolor{currentstroke}%
\pgfsetdash{}{0pt}%
\pgfpathmoveto{\pgfqpoint{3.086899in}{2.709181in}}%
\pgfpathlineto{\pgfqpoint{3.093569in}{2.880010in}}%
\pgfpathlineto{\pgfqpoint{3.103022in}{2.799384in}}%
\pgfpathlineto{\pgfqpoint{3.139813in}{2.671277in}}%
\pgfpathlineto{\pgfqpoint{3.173037in}{2.883064in}}%
\pgfpathlineto{\pgfqpoint{3.167068in}{2.621225in}}%
\pgfpathlineto{\pgfqpoint{3.156781in}{2.783713in}}%
\pgfpathlineto{\pgfqpoint{3.120455in}{2.875763in}}%
\pgfpathlineto{\pgfqpoint{3.086899in}{2.709181in}}%
\pgfpathclose%
\pgfusepath{fill}%
\end{pgfscope}%
\begin{pgfscope}%
\pgfpathrectangle{\pgfqpoint{1.020000in}{0.880000in}}{\pgfqpoint{6.160000in}{6.160000in}}%
\pgfusepath{clip}%
\pgfsetbuttcap%
\pgfsetroundjoin%
\definecolor{currentfill}{rgb}{0.667253,0.779176,0.992959}%
\pgfsetfillcolor{currentfill}%
\pgfsetlinewidth{0.000000pt}%
\definecolor{currentstroke}{rgb}{0.000000,0.000000,0.000000}%
\pgfsetstrokecolor{currentstroke}%
\pgfsetdash{}{0pt}%
\pgfpathmoveto{\pgfqpoint{3.965803in}{3.132011in}}%
\pgfpathlineto{\pgfqpoint{3.975378in}{3.116333in}}%
\pgfpathlineto{\pgfqpoint{3.985260in}{2.955625in}}%
\pgfpathlineto{\pgfqpoint{4.020413in}{2.932601in}}%
\pgfpathlineto{\pgfqpoint{4.055336in}{3.059495in}}%
\pgfpathlineto{\pgfqpoint{4.045559in}{3.165754in}}%
\pgfpathlineto{\pgfqpoint{4.035783in}{3.272661in}}%
\pgfpathlineto{\pgfqpoint{4.001007in}{3.077848in}}%
\pgfpathlineto{\pgfqpoint{3.965803in}{3.132011in}}%
\pgfpathclose%
\pgfusepath{fill}%
\end{pgfscope}%
\begin{pgfscope}%
\pgfpathrectangle{\pgfqpoint{1.020000in}{0.880000in}}{\pgfqpoint{6.160000in}{6.160000in}}%
\pgfusepath{clip}%
\pgfsetbuttcap%
\pgfsetroundjoin%
\definecolor{currentfill}{rgb}{0.414801,0.546874,0.939088}%
\pgfsetfillcolor{currentfill}%
\pgfsetlinewidth{0.000000pt}%
\definecolor{currentstroke}{rgb}{0.000000,0.000000,0.000000}%
\pgfsetstrokecolor{currentstroke}%
\pgfsetdash{}{0pt}%
\pgfpathmoveto{\pgfqpoint{2.875448in}{2.630553in}}%
\pgfpathlineto{\pgfqpoint{2.885966in}{2.462197in}}%
\pgfpathlineto{\pgfqpoint{2.892034in}{2.634975in}}%
\pgfpathlineto{\pgfqpoint{2.930022in}{2.439886in}}%
\pgfpathlineto{\pgfqpoint{2.963109in}{2.630083in}}%
\pgfpathlineto{\pgfqpoint{2.954117in}{2.677382in}}%
\pgfpathlineto{\pgfqpoint{2.945850in}{2.667407in}}%
\pgfpathlineto{\pgfqpoint{2.909811in}{2.714645in}}%
\pgfpathlineto{\pgfqpoint{2.875448in}{2.630553in}}%
\pgfpathclose%
\pgfusepath{fill}%
\end{pgfscope}%
\begin{pgfscope}%
\pgfpathrectangle{\pgfqpoint{1.020000in}{0.880000in}}{\pgfqpoint{6.160000in}{6.160000in}}%
\pgfusepath{clip}%
\pgfsetbuttcap%
\pgfsetroundjoin%
\definecolor{currentfill}{rgb}{0.478462,0.616564,0.972721}%
\pgfsetfillcolor{currentfill}%
\pgfsetlinewidth{0.000000pt}%
\definecolor{currentstroke}{rgb}{0.000000,0.000000,0.000000}%
\pgfsetstrokecolor{currentstroke}%
\pgfsetdash{}{0pt}%
\pgfpathmoveto{\pgfqpoint{3.015246in}{2.786795in}}%
\pgfpathlineto{\pgfqpoint{3.022920in}{2.855436in}}%
\pgfpathlineto{\pgfqpoint{3.033796in}{2.650273in}}%
\pgfpathlineto{\pgfqpoint{3.070386in}{2.547414in}}%
\pgfpathlineto{\pgfqpoint{3.103022in}{2.799384in}}%
\pgfpathlineto{\pgfqpoint{3.093569in}{2.880010in}}%
\pgfpathlineto{\pgfqpoint{3.086899in}{2.709181in}}%
\pgfpathlineto{\pgfqpoint{3.052452in}{2.629167in}}%
\pgfpathlineto{\pgfqpoint{3.015246in}{2.786795in}}%
\pgfpathclose%
\pgfusepath{fill}%
\end{pgfscope}%
\begin{pgfscope}%
\pgfpathrectangle{\pgfqpoint{1.020000in}{0.880000in}}{\pgfqpoint{6.160000in}{6.160000in}}%
\pgfusepath{clip}%
\pgfsetbuttcap%
\pgfsetroundjoin%
\definecolor{currentfill}{rgb}{0.483854,0.622050,0.974808}%
\pgfsetfillcolor{currentfill}%
\pgfsetlinewidth{0.000000pt}%
\definecolor{currentstroke}{rgb}{0.000000,0.000000,0.000000}%
\pgfsetstrokecolor{currentstroke}%
\pgfsetdash{}{0pt}%
\pgfpathmoveto{\pgfqpoint{4.375074in}{2.845912in}}%
\pgfpathlineto{\pgfqpoint{4.384762in}{2.687683in}}%
\pgfpathlineto{\pgfqpoint{4.394736in}{2.668253in}}%
\pgfpathlineto{\pgfqpoint{4.429613in}{2.613524in}}%
\pgfpathlineto{\pgfqpoint{4.464924in}{2.733580in}}%
\pgfpathlineto{\pgfqpoint{4.454972in}{2.785085in}}%
\pgfpathlineto{\pgfqpoint{4.444820in}{2.755980in}}%
\pgfpathlineto{\pgfqpoint{4.409990in}{2.808065in}}%
\pgfpathlineto{\pgfqpoint{4.375074in}{2.845912in}}%
\pgfpathclose%
\pgfusepath{fill}%
\end{pgfscope}%
\begin{pgfscope}%
\pgfpathrectangle{\pgfqpoint{1.020000in}{0.880000in}}{\pgfqpoint{6.160000in}{6.160000in}}%
\pgfusepath{clip}%
\pgfsetbuttcap%
\pgfsetroundjoin%
\definecolor{currentfill}{rgb}{0.693321,0.796314,0.986308}%
\pgfsetfillcolor{currentfill}%
\pgfsetlinewidth{0.000000pt}%
\definecolor{currentstroke}{rgb}{0.000000,0.000000,0.000000}%
\pgfsetstrokecolor{currentstroke}%
\pgfsetdash{}{0pt}%
\pgfpathmoveto{\pgfqpoint{3.735545in}{3.261841in}}%
\pgfpathlineto{\pgfqpoint{3.745535in}{3.103389in}}%
\pgfpathlineto{\pgfqpoint{3.755150in}{3.030086in}}%
\pgfpathlineto{\pgfqpoint{3.790334in}{3.034278in}}%
\pgfpathlineto{\pgfqpoint{3.825287in}{3.097656in}}%
\pgfpathlineto{\pgfqpoint{3.815817in}{3.124097in}}%
\pgfpathlineto{\pgfqpoint{3.805862in}{3.283074in}}%
\pgfpathlineto{\pgfqpoint{3.771179in}{3.159088in}}%
\pgfpathlineto{\pgfqpoint{3.735545in}{3.261841in}}%
\pgfpathclose%
\pgfusepath{fill}%
\end{pgfscope}%
\begin{pgfscope}%
\pgfpathrectangle{\pgfqpoint{1.020000in}{0.880000in}}{\pgfqpoint{6.160000in}{6.160000in}}%
\pgfusepath{clip}%
\pgfsetbuttcap%
\pgfsetroundjoin%
\definecolor{currentfill}{rgb}{0.457046,0.594006,0.963029}%
\pgfsetfillcolor{currentfill}%
\pgfsetlinewidth{0.000000pt}%
\definecolor{currentstroke}{rgb}{0.000000,0.000000,0.000000}%
\pgfsetstrokecolor{currentstroke}%
\pgfsetdash{}{0pt}%
\pgfpathmoveto{\pgfqpoint{2.945850in}{2.667407in}}%
\pgfpathlineto{\pgfqpoint{2.954117in}{2.677382in}}%
\pgfpathlineto{\pgfqpoint{2.963109in}{2.630083in}}%
\pgfpathlineto{\pgfqpoint{2.999020in}{2.593275in}}%
\pgfpathlineto{\pgfqpoint{3.033796in}{2.650273in}}%
\pgfpathlineto{\pgfqpoint{3.022920in}{2.855436in}}%
\pgfpathlineto{\pgfqpoint{3.015246in}{2.786795in}}%
\pgfpathlineto{\pgfqpoint{2.981916in}{2.613278in}}%
\pgfpathlineto{\pgfqpoint{2.945850in}{2.667407in}}%
\pgfpathclose%
\pgfusepath{fill}%
\end{pgfscope}%
\begin{pgfscope}%
\pgfpathrectangle{\pgfqpoint{1.020000in}{0.880000in}}{\pgfqpoint{6.160000in}{6.160000in}}%
\pgfusepath{clip}%
\pgfsetbuttcap%
\pgfsetroundjoin%
\definecolor{currentfill}{rgb}{0.500031,0.638508,0.981070}%
\pgfsetfillcolor{currentfill}%
\pgfsetlinewidth{0.000000pt}%
\definecolor{currentstroke}{rgb}{0.000000,0.000000,0.000000}%
\pgfsetstrokecolor{currentstroke}%
\pgfsetdash{}{0pt}%
\pgfpathmoveto{\pgfqpoint{3.315115in}{2.826230in}}%
\pgfpathlineto{\pgfqpoint{3.323925in}{2.826628in}}%
\pgfpathlineto{\pgfqpoint{3.333661in}{2.721141in}}%
\pgfpathlineto{\pgfqpoint{3.369893in}{2.614356in}}%
\pgfpathlineto{\pgfqpoint{3.404540in}{2.697074in}}%
\pgfpathlineto{\pgfqpoint{3.395621in}{2.700015in}}%
\pgfpathlineto{\pgfqpoint{3.384635in}{2.962987in}}%
\pgfpathlineto{\pgfqpoint{3.350332in}{2.838437in}}%
\pgfpathlineto{\pgfqpoint{3.315115in}{2.826230in}}%
\pgfpathclose%
\pgfusepath{fill}%
\end{pgfscope}%
\begin{pgfscope}%
\pgfpathrectangle{\pgfqpoint{1.020000in}{0.880000in}}{\pgfqpoint{6.160000in}{6.160000in}}%
\pgfusepath{clip}%
\pgfsetbuttcap%
\pgfsetroundjoin%
\definecolor{currentfill}{rgb}{0.630089,0.752516,0.998508}%
\pgfsetfillcolor{currentfill}%
\pgfsetlinewidth{0.000000pt}%
\definecolor{currentstroke}{rgb}{0.000000,0.000000,0.000000}%
\pgfsetstrokecolor{currentstroke}%
\pgfsetdash{}{0pt}%
\pgfpathmoveto{\pgfqpoint{4.125490in}{3.129592in}}%
\pgfpathlineto{\pgfqpoint{4.135300in}{2.965119in}}%
\pgfpathlineto{\pgfqpoint{4.145071in}{2.876438in}}%
\pgfpathlineto{\pgfqpoint{4.180139in}{2.999272in}}%
\pgfpathlineto{\pgfqpoint{4.215247in}{3.043825in}}%
\pgfpathlineto{\pgfqpoint{4.205412in}{2.961317in}}%
\pgfpathlineto{\pgfqpoint{4.195646in}{3.133322in}}%
\pgfpathlineto{\pgfqpoint{4.160604in}{3.016899in}}%
\pgfpathlineto{\pgfqpoint{4.125490in}{3.129592in}}%
\pgfpathclose%
\pgfusepath{fill}%
\end{pgfscope}%
\begin{pgfscope}%
\pgfpathrectangle{\pgfqpoint{1.020000in}{0.880000in}}{\pgfqpoint{6.160000in}{6.160000in}}%
\pgfusepath{clip}%
\pgfsetbuttcap%
\pgfsetroundjoin%
\definecolor{currentfill}{rgb}{0.446431,0.582356,0.957373}%
\pgfsetfillcolor{currentfill}%
\pgfsetlinewidth{0.000000pt}%
\definecolor{currentstroke}{rgb}{0.000000,0.000000,0.000000}%
\pgfsetstrokecolor{currentstroke}%
\pgfsetdash{}{0pt}%
\pgfpathmoveto{\pgfqpoint{4.464924in}{2.733580in}}%
\pgfpathlineto{\pgfqpoint{4.474709in}{2.621028in}}%
\pgfpathlineto{\pgfqpoint{4.510241in}{2.786466in}}%
\pgfpathlineto{\pgfqpoint{4.544602in}{2.604382in}}%
\pgfpathlineto{\pgfqpoint{4.534213in}{2.543997in}}%
\pgfpathlineto{\pgfqpoint{4.499727in}{2.674346in}}%
\pgfpathlineto{\pgfqpoint{4.464924in}{2.733580in}}%
\pgfpathclose%
\pgfusepath{fill}%
\end{pgfscope}%
\begin{pgfscope}%
\pgfpathrectangle{\pgfqpoint{1.020000in}{0.880000in}}{\pgfqpoint{6.160000in}{6.160000in}}%
\pgfusepath{clip}%
\pgfsetbuttcap%
\pgfsetroundjoin%
\definecolor{currentfill}{rgb}{0.554312,0.690097,0.995516}%
\pgfsetfillcolor{currentfill}%
\pgfsetlinewidth{0.000000pt}%
\definecolor{currentstroke}{rgb}{0.000000,0.000000,0.000000}%
\pgfsetstrokecolor{currentstroke}%
\pgfsetdash{}{0pt}%
\pgfpathmoveto{\pgfqpoint{4.215247in}{3.043825in}}%
\pgfpathlineto{\pgfqpoint{4.225004in}{2.848325in}}%
\pgfpathlineto{\pgfqpoint{4.234833in}{2.822707in}}%
\pgfpathlineto{\pgfqpoint{4.269938in}{2.848635in}}%
\pgfpathlineto{\pgfqpoint{4.304901in}{2.762011in}}%
\pgfpathlineto{\pgfqpoint{4.295162in}{2.910294in}}%
\pgfpathlineto{\pgfqpoint{4.285248in}{2.895121in}}%
\pgfpathlineto{\pgfqpoint{4.250181in}{2.816073in}}%
\pgfpathlineto{\pgfqpoint{4.215247in}{3.043825in}}%
\pgfpathclose%
\pgfusepath{fill}%
\end{pgfscope}%
\begin{pgfscope}%
\pgfpathrectangle{\pgfqpoint{1.020000in}{0.880000in}}{\pgfqpoint{6.160000in}{6.160000in}}%
\pgfusepath{clip}%
\pgfsetbuttcap%
\pgfsetroundjoin%
\definecolor{currentfill}{rgb}{0.677823,0.786546,0.991005}%
\pgfsetfillcolor{currentfill}%
\pgfsetlinewidth{0.000000pt}%
\definecolor{currentstroke}{rgb}{0.000000,0.000000,0.000000}%
\pgfsetstrokecolor{currentstroke}%
\pgfsetdash{}{0pt}%
\pgfpathmoveto{\pgfqpoint{3.666438in}{2.996242in}}%
\pgfpathlineto{\pgfqpoint{3.674538in}{3.211383in}}%
\pgfpathlineto{\pgfqpoint{3.683571in}{3.252007in}}%
\pgfpathlineto{\pgfqpoint{3.720021in}{3.009898in}}%
\pgfpathlineto{\pgfqpoint{3.755150in}{3.030086in}}%
\pgfpathlineto{\pgfqpoint{3.745535in}{3.103389in}}%
\pgfpathlineto{\pgfqpoint{3.735545in}{3.261841in}}%
\pgfpathlineto{\pgfqpoint{3.701805in}{2.953444in}}%
\pgfpathlineto{\pgfqpoint{3.666438in}{2.996242in}}%
\pgfpathclose%
\pgfusepath{fill}%
\end{pgfscope}%
\begin{pgfscope}%
\pgfpathrectangle{\pgfqpoint{1.020000in}{0.880000in}}{\pgfqpoint{6.160000in}{6.160000in}}%
\pgfusepath{clip}%
\pgfsetbuttcap%
\pgfsetroundjoin%
\definecolor{currentfill}{rgb}{0.613933,0.739923,0.999142}%
\pgfsetfillcolor{currentfill}%
\pgfsetlinewidth{0.000000pt}%
\definecolor{currentstroke}{rgb}{0.000000,0.000000,0.000000}%
\pgfsetstrokecolor{currentstroke}%
\pgfsetdash{}{0pt}%
\pgfpathmoveto{\pgfqpoint{3.524426in}{3.170892in}}%
\pgfpathlineto{\pgfqpoint{3.535357in}{2.884613in}}%
\pgfpathlineto{\pgfqpoint{3.544509in}{2.870287in}}%
\pgfpathlineto{\pgfqpoint{3.579771in}{2.872691in}}%
\pgfpathlineto{\pgfqpoint{3.613775in}{3.091639in}}%
\pgfpathlineto{\pgfqpoint{3.605747in}{2.897894in}}%
\pgfpathlineto{\pgfqpoint{3.595872in}{3.026988in}}%
\pgfpathlineto{\pgfqpoint{3.560512in}{3.046710in}}%
\pgfpathlineto{\pgfqpoint{3.524426in}{3.170892in}}%
\pgfpathclose%
\pgfusepath{fill}%
\end{pgfscope}%
\begin{pgfscope}%
\pgfpathrectangle{\pgfqpoint{1.020000in}{0.880000in}}{\pgfqpoint{6.160000in}{6.160000in}}%
\pgfusepath{clip}%
\pgfsetbuttcap%
\pgfsetroundjoin%
\definecolor{currentfill}{rgb}{0.661968,0.775491,0.993937}%
\pgfsetfillcolor{currentfill}%
\pgfsetlinewidth{0.000000pt}%
\definecolor{currentstroke}{rgb}{0.000000,0.000000,0.000000}%
\pgfsetstrokecolor{currentstroke}%
\pgfsetdash{}{0pt}%
\pgfpathmoveto{\pgfqpoint{3.895519in}{3.134644in}}%
\pgfpathlineto{\pgfqpoint{3.904592in}{3.277100in}}%
\pgfpathlineto{\pgfqpoint{3.915356in}{2.810128in}}%
\pgfpathlineto{\pgfqpoint{3.949771in}{3.105877in}}%
\pgfpathlineto{\pgfqpoint{3.985260in}{2.955625in}}%
\pgfpathlineto{\pgfqpoint{3.975378in}{3.116333in}}%
\pgfpathlineto{\pgfqpoint{3.965803in}{3.132011in}}%
\pgfpathlineto{\pgfqpoint{3.930843in}{3.065825in}}%
\pgfpathlineto{\pgfqpoint{3.895519in}{3.134644in}}%
\pgfpathclose%
\pgfusepath{fill}%
\end{pgfscope}%
\begin{pgfscope}%
\pgfpathrectangle{\pgfqpoint{1.020000in}{0.880000in}}{\pgfqpoint{6.160000in}{6.160000in}}%
\pgfusepath{clip}%
\pgfsetbuttcap%
\pgfsetroundjoin%
\definecolor{currentfill}{rgb}{0.619318,0.744121,0.998931}%
\pgfsetfillcolor{currentfill}%
\pgfsetlinewidth{0.000000pt}%
\definecolor{currentstroke}{rgb}{0.000000,0.000000,0.000000}%
\pgfsetstrokecolor{currentstroke}%
\pgfsetdash{}{0pt}%
\pgfpathmoveto{\pgfqpoint{4.055336in}{3.059495in}}%
\pgfpathlineto{\pgfqpoint{4.064930in}{3.107572in}}%
\pgfpathlineto{\pgfqpoint{4.074845in}{2.890186in}}%
\pgfpathlineto{\pgfqpoint{4.110027in}{2.807667in}}%
\pgfpathlineto{\pgfqpoint{4.145071in}{2.876438in}}%
\pgfpathlineto{\pgfqpoint{4.135300in}{2.965119in}}%
\pgfpathlineto{\pgfqpoint{4.125490in}{3.129592in}}%
\pgfpathlineto{\pgfqpoint{4.090412in}{3.094349in}}%
\pgfpathlineto{\pgfqpoint{4.055336in}{3.059495in}}%
\pgfpathclose%
\pgfusepath{fill}%
\end{pgfscope}%
\begin{pgfscope}%
\pgfpathrectangle{\pgfqpoint{1.020000in}{0.880000in}}{\pgfqpoint{6.160000in}{6.160000in}}%
\pgfusepath{clip}%
\pgfsetbuttcap%
\pgfsetroundjoin%
\definecolor{currentfill}{rgb}{0.414801,0.546874,0.939088}%
\pgfsetfillcolor{currentfill}%
\pgfsetlinewidth{0.000000pt}%
\definecolor{currentstroke}{rgb}{0.000000,0.000000,0.000000}%
\pgfsetstrokecolor{currentstroke}%
\pgfsetdash{}{0pt}%
\pgfpathmoveto{\pgfqpoint{2.803872in}{2.675512in}}%
\pgfpathlineto{\pgfqpoint{2.813060in}{2.607056in}}%
\pgfpathlineto{\pgfqpoint{2.820739in}{2.648971in}}%
\pgfpathlineto{\pgfqpoint{2.857228in}{2.580050in}}%
\pgfpathlineto{\pgfqpoint{2.892034in}{2.634975in}}%
\pgfpathlineto{\pgfqpoint{2.885966in}{2.462197in}}%
\pgfpathlineto{\pgfqpoint{2.875448in}{2.630553in}}%
\pgfpathlineto{\pgfqpoint{2.840674in}{2.579566in}}%
\pgfpathlineto{\pgfqpoint{2.803872in}{2.675512in}}%
\pgfpathclose%
\pgfusepath{fill}%
\end{pgfscope}%
\begin{pgfscope}%
\pgfpathrectangle{\pgfqpoint{1.020000in}{0.880000in}}{\pgfqpoint{6.160000in}{6.160000in}}%
\pgfusepath{clip}%
\pgfsetbuttcap%
\pgfsetroundjoin%
\definecolor{currentfill}{rgb}{0.483854,0.622050,0.974808}%
\pgfsetfillcolor{currentfill}%
\pgfsetlinewidth{0.000000pt}%
\definecolor{currentstroke}{rgb}{0.000000,0.000000,0.000000}%
\pgfsetstrokecolor{currentstroke}%
\pgfsetdash{}{0pt}%
\pgfpathmoveto{\pgfqpoint{4.304901in}{2.762011in}}%
\pgfpathlineto{\pgfqpoint{4.314720in}{2.679876in}}%
\pgfpathlineto{\pgfqpoint{4.324637in}{2.662147in}}%
\pgfpathlineto{\pgfqpoint{4.359853in}{2.753231in}}%
\pgfpathlineto{\pgfqpoint{4.394736in}{2.668253in}}%
\pgfpathlineto{\pgfqpoint{4.384762in}{2.687683in}}%
\pgfpathlineto{\pgfqpoint{4.375074in}{2.845912in}}%
\pgfpathlineto{\pgfqpoint{4.340036in}{2.836878in}}%
\pgfpathlineto{\pgfqpoint{4.304901in}{2.762011in}}%
\pgfpathclose%
\pgfusepath{fill}%
\end{pgfscope}%
\begin{pgfscope}%
\pgfpathrectangle{\pgfqpoint{1.020000in}{0.880000in}}{\pgfqpoint{6.160000in}{6.160000in}}%
\pgfusepath{clip}%
\pgfsetbuttcap%
\pgfsetroundjoin%
\definecolor{currentfill}{rgb}{0.667253,0.779176,0.992959}%
\pgfsetfillcolor{currentfill}%
\pgfsetlinewidth{0.000000pt}%
\definecolor{currentstroke}{rgb}{0.000000,0.000000,0.000000}%
\pgfsetstrokecolor{currentstroke}%
\pgfsetdash{}{0pt}%
\pgfpathmoveto{\pgfqpoint{3.595872in}{3.026988in}}%
\pgfpathlineto{\pgfqpoint{3.605747in}{2.897894in}}%
\pgfpathlineto{\pgfqpoint{3.613775in}{3.091639in}}%
\pgfpathlineto{\pgfqpoint{3.649065in}{3.095436in}}%
\pgfpathlineto{\pgfqpoint{3.683571in}{3.252007in}}%
\pgfpathlineto{\pgfqpoint{3.674538in}{3.211383in}}%
\pgfpathlineto{\pgfqpoint{3.666438in}{2.996242in}}%
\pgfpathlineto{\pgfqpoint{3.631046in}{3.034226in}}%
\pgfpathlineto{\pgfqpoint{3.595872in}{3.026988in}}%
\pgfpathclose%
\pgfusepath{fill}%
\end{pgfscope}%
\begin{pgfscope}%
\pgfpathrectangle{\pgfqpoint{1.020000in}{0.880000in}}{\pgfqpoint{6.160000in}{6.160000in}}%
\pgfusepath{clip}%
\pgfsetbuttcap%
\pgfsetroundjoin%
\definecolor{currentfill}{rgb}{0.576051,0.708780,0.997755}%
\pgfsetfillcolor{currentfill}%
\pgfsetlinewidth{0.000000pt}%
\definecolor{currentstroke}{rgb}{0.000000,0.000000,0.000000}%
\pgfsetstrokecolor{currentstroke}%
\pgfsetdash{}{0pt}%
\pgfpathmoveto{\pgfqpoint{3.384635in}{2.962987in}}%
\pgfpathlineto{\pgfqpoint{3.395621in}{2.700015in}}%
\pgfpathlineto{\pgfqpoint{3.404540in}{2.697074in}}%
\pgfpathlineto{\pgfqpoint{3.437687in}{2.985129in}}%
\pgfpathlineto{\pgfqpoint{3.473919in}{2.866201in}}%
\pgfpathlineto{\pgfqpoint{3.463876in}{3.012650in}}%
\pgfpathlineto{\pgfqpoint{3.455247in}{2.964991in}}%
\pgfpathlineto{\pgfqpoint{3.419633in}{3.005485in}}%
\pgfpathlineto{\pgfqpoint{3.384635in}{2.962987in}}%
\pgfpathclose%
\pgfusepath{fill}%
\end{pgfscope}%
\begin{pgfscope}%
\pgfpathrectangle{\pgfqpoint{1.020000in}{0.880000in}}{\pgfqpoint{6.160000in}{6.160000in}}%
\pgfusepath{clip}%
\pgfsetbuttcap%
\pgfsetroundjoin%
\definecolor{currentfill}{rgb}{0.441123,0.576532,0.954545}%
\pgfsetfillcolor{currentfill}%
\pgfsetlinewidth{0.000000pt}%
\definecolor{currentstroke}{rgb}{0.000000,0.000000,0.000000}%
\pgfsetstrokecolor{currentstroke}%
\pgfsetdash{}{0pt}%
\pgfpathmoveto{\pgfqpoint{4.394736in}{2.668253in}}%
\pgfpathlineto{\pgfqpoint{4.404818in}{2.688608in}}%
\pgfpathlineto{\pgfqpoint{4.439580in}{2.575138in}}%
\pgfpathlineto{\pgfqpoint{4.474709in}{2.621028in}}%
\pgfpathlineto{\pgfqpoint{4.464924in}{2.733580in}}%
\pgfpathlineto{\pgfqpoint{4.429613in}{2.613524in}}%
\pgfpathlineto{\pgfqpoint{4.394736in}{2.668253in}}%
\pgfpathclose%
\pgfusepath{fill}%
\end{pgfscope}%
\begin{pgfscope}%
\pgfpathrectangle{\pgfqpoint{1.020000in}{0.880000in}}{\pgfqpoint{6.160000in}{6.160000in}}%
\pgfusepath{clip}%
\pgfsetbuttcap%
\pgfsetroundjoin%
\definecolor{currentfill}{rgb}{0.613933,0.739923,0.999142}%
\pgfsetfillcolor{currentfill}%
\pgfsetlinewidth{0.000000pt}%
\definecolor{currentstroke}{rgb}{0.000000,0.000000,0.000000}%
\pgfsetstrokecolor{currentstroke}%
\pgfsetdash{}{0pt}%
\pgfpathmoveto{\pgfqpoint{3.455247in}{2.964991in}}%
\pgfpathlineto{\pgfqpoint{3.463876in}{3.012650in}}%
\pgfpathlineto{\pgfqpoint{3.473919in}{2.866201in}}%
\pgfpathlineto{\pgfqpoint{3.508789in}{2.932440in}}%
\pgfpathlineto{\pgfqpoint{3.544509in}{2.870287in}}%
\pgfpathlineto{\pgfqpoint{3.535357in}{2.884613in}}%
\pgfpathlineto{\pgfqpoint{3.524426in}{3.170892in}}%
\pgfpathlineto{\pgfqpoint{3.489722in}{3.080769in}}%
\pgfpathlineto{\pgfqpoint{3.455247in}{2.964991in}}%
\pgfpathclose%
\pgfusepath{fill}%
\end{pgfscope}%
\begin{pgfscope}%
\pgfpathrectangle{\pgfqpoint{1.020000in}{0.880000in}}{\pgfqpoint{6.160000in}{6.160000in}}%
\pgfusepath{clip}%
\pgfsetbuttcap%
\pgfsetroundjoin%
\definecolor{currentfill}{rgb}{0.661968,0.775491,0.993937}%
\pgfsetfillcolor{currentfill}%
\pgfsetlinewidth{0.000000pt}%
\definecolor{currentstroke}{rgb}{0.000000,0.000000,0.000000}%
\pgfsetstrokecolor{currentstroke}%
\pgfsetdash{}{0pt}%
\pgfpathmoveto{\pgfqpoint{3.825287in}{3.097656in}}%
\pgfpathlineto{\pgfqpoint{3.835029in}{2.998068in}}%
\pgfpathlineto{\pgfqpoint{3.844141in}{3.084217in}}%
\pgfpathlineto{\pgfqpoint{3.879865in}{2.928348in}}%
\pgfpathlineto{\pgfqpoint{3.915356in}{2.810128in}}%
\pgfpathlineto{\pgfqpoint{3.904592in}{3.277100in}}%
\pgfpathlineto{\pgfqpoint{3.895519in}{3.134644in}}%
\pgfpathlineto{\pgfqpoint{3.860169in}{3.189663in}}%
\pgfpathlineto{\pgfqpoint{3.825287in}{3.097656in}}%
\pgfpathclose%
\pgfusepath{fill}%
\end{pgfscope}%
\begin{pgfscope}%
\pgfpathrectangle{\pgfqpoint{1.020000in}{0.880000in}}{\pgfqpoint{6.160000in}{6.160000in}}%
\pgfusepath{clip}%
\pgfsetbuttcap%
\pgfsetroundjoin%
\definecolor{currentfill}{rgb}{0.430507,0.564883,0.948889}%
\pgfsetfillcolor{currentfill}%
\pgfsetlinewidth{0.000000pt}%
\definecolor{currentstroke}{rgb}{0.000000,0.000000,0.000000}%
\pgfsetstrokecolor{currentstroke}%
\pgfsetdash{}{0pt}%
\pgfpathmoveto{\pgfqpoint{3.033796in}{2.650273in}}%
\pgfpathlineto{\pgfqpoint{3.044170in}{2.487459in}}%
\pgfpathlineto{\pgfqpoint{3.050660in}{2.664147in}}%
\pgfpathlineto{\pgfqpoint{3.086540in}{2.631091in}}%
\pgfpathlineto{\pgfqpoint{3.122532in}{2.583158in}}%
\pgfpathlineto{\pgfqpoint{3.113411in}{2.632527in}}%
\pgfpathlineto{\pgfqpoint{3.103022in}{2.799384in}}%
\pgfpathlineto{\pgfqpoint{3.070386in}{2.547414in}}%
\pgfpathlineto{\pgfqpoint{3.033796in}{2.650273in}}%
\pgfpathclose%
\pgfusepath{fill}%
\end{pgfscope}%
\begin{pgfscope}%
\pgfpathrectangle{\pgfqpoint{1.020000in}{0.880000in}}{\pgfqpoint{6.160000in}{6.160000in}}%
\pgfusepath{clip}%
\pgfsetbuttcap%
\pgfsetroundjoin%
\definecolor{currentfill}{rgb}{0.624703,0.748318,0.998719}%
\pgfsetfillcolor{currentfill}%
\pgfsetlinewidth{0.000000pt}%
\definecolor{currentstroke}{rgb}{0.000000,0.000000,0.000000}%
\pgfsetstrokecolor{currentstroke}%
\pgfsetdash{}{0pt}%
\pgfpathmoveto{\pgfqpoint{3.985260in}{2.955625in}}%
\pgfpathlineto{\pgfqpoint{3.994649in}{3.050847in}}%
\pgfpathlineto{\pgfqpoint{4.004535in}{2.890867in}}%
\pgfpathlineto{\pgfqpoint{4.039470in}{3.051113in}}%
\pgfpathlineto{\pgfqpoint{4.074845in}{2.890186in}}%
\pgfpathlineto{\pgfqpoint{4.064930in}{3.107572in}}%
\pgfpathlineto{\pgfqpoint{4.055336in}{3.059495in}}%
\pgfpathlineto{\pgfqpoint{4.020413in}{2.932601in}}%
\pgfpathlineto{\pgfqpoint{3.985260in}{2.955625in}}%
\pgfpathclose%
\pgfusepath{fill}%
\end{pgfscope}%
\begin{pgfscope}%
\pgfpathrectangle{\pgfqpoint{1.020000in}{0.880000in}}{\pgfqpoint{6.160000in}{6.160000in}}%
\pgfusepath{clip}%
\pgfsetbuttcap%
\pgfsetroundjoin%
\definecolor{currentfill}{rgb}{0.651398,0.768121,0.995891}%
\pgfsetfillcolor{currentfill}%
\pgfsetlinewidth{0.000000pt}%
\definecolor{currentstroke}{rgb}{0.000000,0.000000,0.000000}%
\pgfsetstrokecolor{currentstroke}%
\pgfsetdash{}{0pt}%
\pgfpathmoveto{\pgfqpoint{3.755150in}{3.030086in}}%
\pgfpathlineto{\pgfqpoint{3.764600in}{2.997836in}}%
\pgfpathlineto{\pgfqpoint{3.773822in}{3.024666in}}%
\pgfpathlineto{\pgfqpoint{3.808989in}{3.052396in}}%
\pgfpathlineto{\pgfqpoint{3.844141in}{3.084217in}}%
\pgfpathlineto{\pgfqpoint{3.835029in}{2.998068in}}%
\pgfpathlineto{\pgfqpoint{3.825287in}{3.097656in}}%
\pgfpathlineto{\pgfqpoint{3.790334in}{3.034278in}}%
\pgfpathlineto{\pgfqpoint{3.755150in}{3.030086in}}%
\pgfpathclose%
\pgfusepath{fill}%
\end{pgfscope}%
\begin{pgfscope}%
\pgfpathrectangle{\pgfqpoint{1.020000in}{0.880000in}}{\pgfqpoint{6.160000in}{6.160000in}}%
\pgfusepath{clip}%
\pgfsetbuttcap%
\pgfsetroundjoin%
\definecolor{currentfill}{rgb}{0.505423,0.643995,0.983157}%
\pgfsetfillcolor{currentfill}%
\pgfsetlinewidth{0.000000pt}%
\definecolor{currentstroke}{rgb}{0.000000,0.000000,0.000000}%
\pgfsetstrokecolor{currentstroke}%
\pgfsetdash{}{0pt}%
\pgfpathmoveto{\pgfqpoint{3.173037in}{2.883064in}}%
\pgfpathlineto{\pgfqpoint{3.182664in}{2.788084in}}%
\pgfpathlineto{\pgfqpoint{3.192930in}{2.629367in}}%
\pgfpathlineto{\pgfqpoint{3.227583in}{2.708233in}}%
\pgfpathlineto{\pgfqpoint{3.261732in}{2.845197in}}%
\pgfpathlineto{\pgfqpoint{3.254021in}{2.732949in}}%
\pgfpathlineto{\pgfqpoint{3.243905in}{2.879518in}}%
\pgfpathlineto{\pgfqpoint{3.210687in}{2.656429in}}%
\pgfpathlineto{\pgfqpoint{3.173037in}{2.883064in}}%
\pgfpathclose%
\pgfusepath{fill}%
\end{pgfscope}%
\begin{pgfscope}%
\pgfpathrectangle{\pgfqpoint{1.020000in}{0.880000in}}{\pgfqpoint{6.160000in}{6.160000in}}%
\pgfusepath{clip}%
\pgfsetbuttcap%
\pgfsetroundjoin%
\definecolor{currentfill}{rgb}{0.538004,0.674902,0.991722}%
\pgfsetfillcolor{currentfill}%
\pgfsetlinewidth{0.000000pt}%
\definecolor{currentstroke}{rgb}{0.000000,0.000000,0.000000}%
\pgfsetstrokecolor{currentstroke}%
\pgfsetdash{}{0pt}%
\pgfpathmoveto{\pgfqpoint{3.243905in}{2.879518in}}%
\pgfpathlineto{\pgfqpoint{3.254021in}{2.732949in}}%
\pgfpathlineto{\pgfqpoint{3.261732in}{2.845197in}}%
\pgfpathlineto{\pgfqpoint{3.296481in}{2.923256in}}%
\pgfpathlineto{\pgfqpoint{3.333661in}{2.721141in}}%
\pgfpathlineto{\pgfqpoint{3.323925in}{2.826628in}}%
\pgfpathlineto{\pgfqpoint{3.315115in}{2.826230in}}%
\pgfpathlineto{\pgfqpoint{3.279379in}{2.869502in}}%
\pgfpathlineto{\pgfqpoint{3.243905in}{2.879518in}}%
\pgfpathclose%
\pgfusepath{fill}%
\end{pgfscope}%
\begin{pgfscope}%
\pgfpathrectangle{\pgfqpoint{1.020000in}{0.880000in}}{\pgfqpoint{6.160000in}{6.160000in}}%
\pgfusepath{clip}%
\pgfsetbuttcap%
\pgfsetroundjoin%
\definecolor{currentfill}{rgb}{0.581486,0.713451,0.998314}%
\pgfsetfillcolor{currentfill}%
\pgfsetlinewidth{0.000000pt}%
\definecolor{currentstroke}{rgb}{0.000000,0.000000,0.000000}%
\pgfsetstrokecolor{currentstroke}%
\pgfsetdash{}{0pt}%
\pgfpathmoveto{\pgfqpoint{4.145071in}{2.876438in}}%
\pgfpathlineto{\pgfqpoint{4.154816in}{2.892396in}}%
\pgfpathlineto{\pgfqpoint{4.164595in}{2.863224in}}%
\pgfpathlineto{\pgfqpoint{4.199742in}{2.902127in}}%
\pgfpathlineto{\pgfqpoint{4.234833in}{2.822707in}}%
\pgfpathlineto{\pgfqpoint{4.225004in}{2.848325in}}%
\pgfpathlineto{\pgfqpoint{4.215247in}{3.043825in}}%
\pgfpathlineto{\pgfqpoint{4.180139in}{2.999272in}}%
\pgfpathlineto{\pgfqpoint{4.145071in}{2.876438in}}%
\pgfpathclose%
\pgfusepath{fill}%
\end{pgfscope}%
\begin{pgfscope}%
\pgfpathrectangle{\pgfqpoint{1.020000in}{0.880000in}}{\pgfqpoint{6.160000in}{6.160000in}}%
\pgfusepath{clip}%
\pgfsetbuttcap%
\pgfsetroundjoin%
\definecolor{currentfill}{rgb}{0.473070,0.611077,0.970634}%
\pgfsetfillcolor{currentfill}%
\pgfsetlinewidth{0.000000pt}%
\definecolor{currentstroke}{rgb}{0.000000,0.000000,0.000000}%
\pgfsetstrokecolor{currentstroke}%
\pgfsetdash{}{0pt}%
\pgfpathmoveto{\pgfqpoint{3.103022in}{2.799384in}}%
\pgfpathlineto{\pgfqpoint{3.113411in}{2.632527in}}%
\pgfpathlineto{\pgfqpoint{3.122532in}{2.583158in}}%
\pgfpathlineto{\pgfqpoint{3.157260in}{2.652077in}}%
\pgfpathlineto{\pgfqpoint{3.192930in}{2.629367in}}%
\pgfpathlineto{\pgfqpoint{3.182664in}{2.788084in}}%
\pgfpathlineto{\pgfqpoint{3.173037in}{2.883064in}}%
\pgfpathlineto{\pgfqpoint{3.139813in}{2.671277in}}%
\pgfpathlineto{\pgfqpoint{3.103022in}{2.799384in}}%
\pgfpathclose%
\pgfusepath{fill}%
\end{pgfscope}%
\begin{pgfscope}%
\pgfpathrectangle{\pgfqpoint{1.020000in}{0.880000in}}{\pgfqpoint{6.160000in}{6.160000in}}%
\pgfusepath{clip}%
\pgfsetbuttcap%
\pgfsetroundjoin%
\definecolor{currentfill}{rgb}{0.494638,0.633022,0.978983}%
\pgfsetfillcolor{currentfill}%
\pgfsetlinewidth{0.000000pt}%
\definecolor{currentstroke}{rgb}{0.000000,0.000000,0.000000}%
\pgfsetstrokecolor{currentstroke}%
\pgfsetdash{}{0pt}%
\pgfpathmoveto{\pgfqpoint{4.234833in}{2.822707in}}%
\pgfpathlineto{\pgfqpoint{4.244703in}{2.836429in}}%
\pgfpathlineto{\pgfqpoint{4.254500in}{2.714089in}}%
\pgfpathlineto{\pgfqpoint{4.289539in}{2.641655in}}%
\pgfpathlineto{\pgfqpoint{4.324637in}{2.662147in}}%
\pgfpathlineto{\pgfqpoint{4.314720in}{2.679876in}}%
\pgfpathlineto{\pgfqpoint{4.304901in}{2.762011in}}%
\pgfpathlineto{\pgfqpoint{4.269938in}{2.848635in}}%
\pgfpathlineto{\pgfqpoint{4.234833in}{2.822707in}}%
\pgfpathclose%
\pgfusepath{fill}%
\end{pgfscope}%
\begin{pgfscope}%
\pgfpathrectangle{\pgfqpoint{1.020000in}{0.880000in}}{\pgfqpoint{6.160000in}{6.160000in}}%
\pgfusepath{clip}%
\pgfsetbuttcap%
\pgfsetroundjoin%
\definecolor{currentfill}{rgb}{0.635474,0.756714,0.998297}%
\pgfsetfillcolor{currentfill}%
\pgfsetlinewidth{0.000000pt}%
\definecolor{currentstroke}{rgb}{0.000000,0.000000,0.000000}%
\pgfsetstrokecolor{currentstroke}%
\pgfsetdash{}{0pt}%
\pgfpathmoveto{\pgfqpoint{3.683571in}{3.252007in}}%
\pgfpathlineto{\pgfqpoint{3.694133in}{2.987370in}}%
\pgfpathlineto{\pgfqpoint{3.703941in}{2.873174in}}%
\pgfpathlineto{\pgfqpoint{3.739371in}{2.835791in}}%
\pgfpathlineto{\pgfqpoint{3.773822in}{3.024666in}}%
\pgfpathlineto{\pgfqpoint{3.764600in}{2.997836in}}%
\pgfpathlineto{\pgfqpoint{3.755150in}{3.030086in}}%
\pgfpathlineto{\pgfqpoint{3.720021in}{3.009898in}}%
\pgfpathlineto{\pgfqpoint{3.683571in}{3.252007in}}%
\pgfpathclose%
\pgfusepath{fill}%
\end{pgfscope}%
\begin{pgfscope}%
\pgfpathrectangle{\pgfqpoint{1.020000in}{0.880000in}}{\pgfqpoint{6.160000in}{6.160000in}}%
\pgfusepath{clip}%
\pgfsetbuttcap%
\pgfsetroundjoin%
\definecolor{currentfill}{rgb}{0.419991,0.552989,0.942630}%
\pgfsetfillcolor{currentfill}%
\pgfsetlinewidth{0.000000pt}%
\definecolor{currentstroke}{rgb}{0.000000,0.000000,0.000000}%
\pgfsetstrokecolor{currentstroke}%
\pgfsetdash{}{0pt}%
\pgfpathmoveto{\pgfqpoint{2.963109in}{2.630083in}}%
\pgfpathlineto{\pgfqpoint{2.971732in}{2.613874in}}%
\pgfpathlineto{\pgfqpoint{2.979448in}{2.673794in}}%
\pgfpathlineto{\pgfqpoint{3.017075in}{2.498382in}}%
\pgfpathlineto{\pgfqpoint{3.050660in}{2.664147in}}%
\pgfpathlineto{\pgfqpoint{3.044170in}{2.487459in}}%
\pgfpathlineto{\pgfqpoint{3.033796in}{2.650273in}}%
\pgfpathlineto{\pgfqpoint{2.999020in}{2.593275in}}%
\pgfpathlineto{\pgfqpoint{2.963109in}{2.630083in}}%
\pgfpathclose%
\pgfusepath{fill}%
\end{pgfscope}%
\begin{pgfscope}%
\pgfpathrectangle{\pgfqpoint{1.020000in}{0.880000in}}{\pgfqpoint{6.160000in}{6.160000in}}%
\pgfusepath{clip}%
\pgfsetbuttcap%
\pgfsetroundjoin%
\definecolor{currentfill}{rgb}{0.538004,0.674902,0.991722}%
\pgfsetfillcolor{currentfill}%
\pgfsetlinewidth{0.000000pt}%
\definecolor{currentstroke}{rgb}{0.000000,0.000000,0.000000}%
\pgfsetstrokecolor{currentstroke}%
\pgfsetdash{}{0pt}%
\pgfpathmoveto{\pgfqpoint{3.473919in}{2.866201in}}%
\pgfpathlineto{\pgfqpoint{3.483405in}{2.797821in}}%
\pgfpathlineto{\pgfqpoint{3.492319in}{2.812949in}}%
\pgfpathlineto{\pgfqpoint{3.528837in}{2.639578in}}%
\pgfpathlineto{\pgfqpoint{3.563606in}{2.723100in}}%
\pgfpathlineto{\pgfqpoint{3.553063in}{2.953158in}}%
\pgfpathlineto{\pgfqpoint{3.544509in}{2.870287in}}%
\pgfpathlineto{\pgfqpoint{3.508789in}{2.932440in}}%
\pgfpathlineto{\pgfqpoint{3.473919in}{2.866201in}}%
\pgfpathclose%
\pgfusepath{fill}%
\end{pgfscope}%
\begin{pgfscope}%
\pgfpathrectangle{\pgfqpoint{1.020000in}{0.880000in}}{\pgfqpoint{6.160000in}{6.160000in}}%
\pgfusepath{clip}%
\pgfsetbuttcap%
\pgfsetroundjoin%
\definecolor{currentfill}{rgb}{0.624703,0.748318,0.998719}%
\pgfsetfillcolor{currentfill}%
\pgfsetlinewidth{0.000000pt}%
\definecolor{currentstroke}{rgb}{0.000000,0.000000,0.000000}%
\pgfsetstrokecolor{currentstroke}%
\pgfsetdash{}{0pt}%
\pgfpathmoveto{\pgfqpoint{3.915356in}{2.810128in}}%
\pgfpathlineto{\pgfqpoint{3.923949in}{3.151879in}}%
\pgfpathlineto{\pgfqpoint{3.934227in}{2.855578in}}%
\pgfpathlineto{\pgfqpoint{3.969027in}{3.040192in}}%
\pgfpathlineto{\pgfqpoint{4.004535in}{2.890867in}}%
\pgfpathlineto{\pgfqpoint{3.994649in}{3.050847in}}%
\pgfpathlineto{\pgfqpoint{3.985260in}{2.955625in}}%
\pgfpathlineto{\pgfqpoint{3.949771in}{3.105877in}}%
\pgfpathlineto{\pgfqpoint{3.915356in}{2.810128in}}%
\pgfpathclose%
\pgfusepath{fill}%
\end{pgfscope}%
\begin{pgfscope}%
\pgfpathrectangle{\pgfqpoint{1.020000in}{0.880000in}}{\pgfqpoint{6.160000in}{6.160000in}}%
\pgfusepath{clip}%
\pgfsetbuttcap%
\pgfsetroundjoin%
\definecolor{currentfill}{rgb}{0.576051,0.708780,0.997755}%
\pgfsetfillcolor{currentfill}%
\pgfsetlinewidth{0.000000pt}%
\definecolor{currentstroke}{rgb}{0.000000,0.000000,0.000000}%
\pgfsetstrokecolor{currentstroke}%
\pgfsetdash{}{0pt}%
\pgfpathmoveto{\pgfqpoint{4.074845in}{2.890186in}}%
\pgfpathlineto{\pgfqpoint{4.084532in}{2.881770in}}%
\pgfpathlineto{\pgfqpoint{4.094375in}{2.721690in}}%
\pgfpathlineto{\pgfqpoint{4.129271in}{3.162058in}}%
\pgfpathlineto{\pgfqpoint{4.164595in}{2.863224in}}%
\pgfpathlineto{\pgfqpoint{4.154816in}{2.892396in}}%
\pgfpathlineto{\pgfqpoint{4.145071in}{2.876438in}}%
\pgfpathlineto{\pgfqpoint{4.110027in}{2.807667in}}%
\pgfpathlineto{\pgfqpoint{4.074845in}{2.890186in}}%
\pgfpathclose%
\pgfusepath{fill}%
\end{pgfscope}%
\begin{pgfscope}%
\pgfpathrectangle{\pgfqpoint{1.020000in}{0.880000in}}{\pgfqpoint{6.160000in}{6.160000in}}%
\pgfusepath{clip}%
\pgfsetbuttcap%
\pgfsetroundjoin%
\definecolor{currentfill}{rgb}{0.467678,0.605591,0.968546}%
\pgfsetfillcolor{currentfill}%
\pgfsetlinewidth{0.000000pt}%
\definecolor{currentstroke}{rgb}{0.000000,0.000000,0.000000}%
\pgfsetstrokecolor{currentstroke}%
\pgfsetdash{}{0pt}%
\pgfpathmoveto{\pgfqpoint{4.324637in}{2.662147in}}%
\pgfpathlineto{\pgfqpoint{4.334576in}{2.646441in}}%
\pgfpathlineto{\pgfqpoint{4.369808in}{2.724944in}}%
\pgfpathlineto{\pgfqpoint{4.404818in}{2.688608in}}%
\pgfpathlineto{\pgfqpoint{4.394736in}{2.668253in}}%
\pgfpathlineto{\pgfqpoint{4.359853in}{2.753231in}}%
\pgfpathlineto{\pgfqpoint{4.324637in}{2.662147in}}%
\pgfpathclose%
\pgfusepath{fill}%
\end{pgfscope}%
\begin{pgfscope}%
\pgfpathrectangle{\pgfqpoint{1.020000in}{0.880000in}}{\pgfqpoint{6.160000in}{6.160000in}}%
\pgfusepath{clip}%
\pgfsetbuttcap%
\pgfsetroundjoin%
\definecolor{currentfill}{rgb}{0.586921,0.718121,0.998874}%
\pgfsetfillcolor{currentfill}%
\pgfsetlinewidth{0.000000pt}%
\definecolor{currentstroke}{rgb}{0.000000,0.000000,0.000000}%
\pgfsetstrokecolor{currentstroke}%
\pgfsetdash{}{0pt}%
\pgfpathmoveto{\pgfqpoint{3.544509in}{2.870287in}}%
\pgfpathlineto{\pgfqpoint{3.553063in}{2.953158in}}%
\pgfpathlineto{\pgfqpoint{3.563606in}{2.723100in}}%
\pgfpathlineto{\pgfqpoint{3.598157in}{2.849999in}}%
\pgfpathlineto{\pgfqpoint{3.632923in}{2.950403in}}%
\pgfpathlineto{\pgfqpoint{3.623756in}{2.947903in}}%
\pgfpathlineto{\pgfqpoint{3.613775in}{3.091639in}}%
\pgfpathlineto{\pgfqpoint{3.579771in}{2.872691in}}%
\pgfpathlineto{\pgfqpoint{3.544509in}{2.870287in}}%
\pgfpathclose%
\pgfusepath{fill}%
\end{pgfscope}%
\begin{pgfscope}%
\pgfpathrectangle{\pgfqpoint{1.020000in}{0.880000in}}{\pgfqpoint{6.160000in}{6.160000in}}%
\pgfusepath{clip}%
\pgfsetbuttcap%
\pgfsetroundjoin%
\definecolor{currentfill}{rgb}{0.494638,0.633022,0.978983}%
\pgfsetfillcolor{currentfill}%
\pgfsetlinewidth{0.000000pt}%
\definecolor{currentstroke}{rgb}{0.000000,0.000000,0.000000}%
\pgfsetstrokecolor{currentstroke}%
\pgfsetdash{}{0pt}%
\pgfpathmoveto{\pgfqpoint{3.333661in}{2.721141in}}%
\pgfpathlineto{\pgfqpoint{3.342449in}{2.727731in}}%
\pgfpathlineto{\pgfqpoint{3.352094in}{2.634598in}}%
\pgfpathlineto{\pgfqpoint{3.386109in}{2.807270in}}%
\pgfpathlineto{\pgfqpoint{3.421192in}{2.853721in}}%
\pgfpathlineto{\pgfqpoint{3.412490in}{2.821693in}}%
\pgfpathlineto{\pgfqpoint{3.404540in}{2.697074in}}%
\pgfpathlineto{\pgfqpoint{3.369893in}{2.614356in}}%
\pgfpathlineto{\pgfqpoint{3.333661in}{2.721141in}}%
\pgfpathclose%
\pgfusepath{fill}%
\end{pgfscope}%
\begin{pgfscope}%
\pgfpathrectangle{\pgfqpoint{1.020000in}{0.880000in}}{\pgfqpoint{6.160000in}{6.160000in}}%
\pgfusepath{clip}%
\pgfsetbuttcap%
\pgfsetroundjoin%
\definecolor{currentfill}{rgb}{0.608547,0.735725,0.999354}%
\pgfsetfillcolor{currentfill}%
\pgfsetlinewidth{0.000000pt}%
\definecolor{currentstroke}{rgb}{0.000000,0.000000,0.000000}%
\pgfsetstrokecolor{currentstroke}%
\pgfsetdash{}{0pt}%
\pgfpathmoveto{\pgfqpoint{3.844141in}{3.084217in}}%
\pgfpathlineto{\pgfqpoint{3.853990in}{2.959031in}}%
\pgfpathlineto{\pgfqpoint{3.863812in}{2.841215in}}%
\pgfpathlineto{\pgfqpoint{3.898618in}{2.991916in}}%
\pgfpathlineto{\pgfqpoint{3.934227in}{2.855578in}}%
\pgfpathlineto{\pgfqpoint{3.923949in}{3.151879in}}%
\pgfpathlineto{\pgfqpoint{3.915356in}{2.810128in}}%
\pgfpathlineto{\pgfqpoint{3.879865in}{2.928348in}}%
\pgfpathlineto{\pgfqpoint{3.844141in}{3.084217in}}%
\pgfpathclose%
\pgfusepath{fill}%
\end{pgfscope}%
\begin{pgfscope}%
\pgfpathrectangle{\pgfqpoint{1.020000in}{0.880000in}}{\pgfqpoint{6.160000in}{6.160000in}}%
\pgfusepath{clip}%
\pgfsetbuttcap%
\pgfsetroundjoin%
\definecolor{currentfill}{rgb}{0.565182,0.699438,0.996635}%
\pgfsetfillcolor{currentfill}%
\pgfsetlinewidth{0.000000pt}%
\definecolor{currentstroke}{rgb}{0.000000,0.000000,0.000000}%
\pgfsetstrokecolor{currentstroke}%
\pgfsetdash{}{0pt}%
\pgfpathmoveto{\pgfqpoint{4.004535in}{2.890867in}}%
\pgfpathlineto{\pgfqpoint{4.014179in}{2.868278in}}%
\pgfpathlineto{\pgfqpoint{4.023795in}{2.873724in}}%
\pgfpathlineto{\pgfqpoint{4.059208in}{2.729811in}}%
\pgfpathlineto{\pgfqpoint{4.094375in}{2.721690in}}%
\pgfpathlineto{\pgfqpoint{4.084532in}{2.881770in}}%
\pgfpathlineto{\pgfqpoint{4.074845in}{2.890186in}}%
\pgfpathlineto{\pgfqpoint{4.039470in}{3.051113in}}%
\pgfpathlineto{\pgfqpoint{4.004535in}{2.890867in}}%
\pgfpathclose%
\pgfusepath{fill}%
\end{pgfscope}%
\begin{pgfscope}%
\pgfpathrectangle{\pgfqpoint{1.020000in}{0.880000in}}{\pgfqpoint{6.160000in}{6.160000in}}%
\pgfusepath{clip}%
\pgfsetbuttcap%
\pgfsetroundjoin%
\definecolor{currentfill}{rgb}{0.462354,0.599830,0.965857}%
\pgfsetfillcolor{currentfill}%
\pgfsetlinewidth{0.000000pt}%
\definecolor{currentstroke}{rgb}{0.000000,0.000000,0.000000}%
\pgfsetstrokecolor{currentstroke}%
\pgfsetdash{}{0pt}%
\pgfpathmoveto{\pgfqpoint{4.254500in}{2.714089in}}%
\pgfpathlineto{\pgfqpoint{4.264410in}{2.731754in}}%
\pgfpathlineto{\pgfqpoint{4.299482in}{2.657180in}}%
\pgfpathlineto{\pgfqpoint{4.334576in}{2.646441in}}%
\pgfpathlineto{\pgfqpoint{4.324637in}{2.662147in}}%
\pgfpathlineto{\pgfqpoint{4.289539in}{2.641655in}}%
\pgfpathlineto{\pgfqpoint{4.254500in}{2.714089in}}%
\pgfpathclose%
\pgfusepath{fill}%
\end{pgfscope}%
\begin{pgfscope}%
\pgfpathrectangle{\pgfqpoint{1.020000in}{0.880000in}}{\pgfqpoint{6.160000in}{6.160000in}}%
\pgfusepath{clip}%
\pgfsetbuttcap%
\pgfsetroundjoin%
\definecolor{currentfill}{rgb}{0.548876,0.685104,0.994379}%
\pgfsetfillcolor{currentfill}%
\pgfsetlinewidth{0.000000pt}%
\definecolor{currentstroke}{rgb}{0.000000,0.000000,0.000000}%
\pgfsetstrokecolor{currentstroke}%
\pgfsetdash{}{0pt}%
\pgfpathmoveto{\pgfqpoint{3.404540in}{2.697074in}}%
\pgfpathlineto{\pgfqpoint{3.412490in}{2.821693in}}%
\pgfpathlineto{\pgfqpoint{3.421192in}{2.853721in}}%
\pgfpathlineto{\pgfqpoint{3.456844in}{2.823679in}}%
\pgfpathlineto{\pgfqpoint{3.492319in}{2.812949in}}%
\pgfpathlineto{\pgfqpoint{3.483405in}{2.797821in}}%
\pgfpathlineto{\pgfqpoint{3.473919in}{2.866201in}}%
\pgfpathlineto{\pgfqpoint{3.437687in}{2.985129in}}%
\pgfpathlineto{\pgfqpoint{3.404540in}{2.697074in}}%
\pgfpathclose%
\pgfusepath{fill}%
\end{pgfscope}%
\begin{pgfscope}%
\pgfpathrectangle{\pgfqpoint{1.020000in}{0.880000in}}{\pgfqpoint{6.160000in}{6.160000in}}%
\pgfusepath{clip}%
\pgfsetbuttcap%
\pgfsetroundjoin%
\definecolor{currentfill}{rgb}{0.651398,0.768121,0.995891}%
\pgfsetfillcolor{currentfill}%
\pgfsetlinewidth{0.000000pt}%
\definecolor{currentstroke}{rgb}{0.000000,0.000000,0.000000}%
\pgfsetstrokecolor{currentstroke}%
\pgfsetdash{}{0pt}%
\pgfpathmoveto{\pgfqpoint{3.613775in}{3.091639in}}%
\pgfpathlineto{\pgfqpoint{3.623756in}{2.947903in}}%
\pgfpathlineto{\pgfqpoint{3.632923in}{2.950403in}}%
\pgfpathlineto{\pgfqpoint{3.667923in}{3.014983in}}%
\pgfpathlineto{\pgfqpoint{3.703941in}{2.873174in}}%
\pgfpathlineto{\pgfqpoint{3.694133in}{2.987370in}}%
\pgfpathlineto{\pgfqpoint{3.683571in}{3.252007in}}%
\pgfpathlineto{\pgfqpoint{3.649065in}{3.095436in}}%
\pgfpathlineto{\pgfqpoint{3.613775in}{3.091639in}}%
\pgfpathclose%
\pgfusepath{fill}%
\end{pgfscope}%
\begin{pgfscope}%
\pgfpathrectangle{\pgfqpoint{1.020000in}{0.880000in}}{\pgfqpoint{6.160000in}{6.160000in}}%
\pgfusepath{clip}%
\pgfsetbuttcap%
\pgfsetroundjoin%
\definecolor{currentfill}{rgb}{0.489246,0.627536,0.976896}%
\pgfsetfillcolor{currentfill}%
\pgfsetlinewidth{0.000000pt}%
\definecolor{currentstroke}{rgb}{0.000000,0.000000,0.000000}%
\pgfsetstrokecolor{currentstroke}%
\pgfsetdash{}{0pt}%
\pgfpathmoveto{\pgfqpoint{3.261732in}{2.845197in}}%
\pgfpathlineto{\pgfqpoint{3.272153in}{2.666522in}}%
\pgfpathlineto{\pgfqpoint{3.281991in}{2.550658in}}%
\pgfpathlineto{\pgfqpoint{3.316498in}{2.654560in}}%
\pgfpathlineto{\pgfqpoint{3.352094in}{2.634598in}}%
\pgfpathlineto{\pgfqpoint{3.342449in}{2.727731in}}%
\pgfpathlineto{\pgfqpoint{3.333661in}{2.721141in}}%
\pgfpathlineto{\pgfqpoint{3.296481in}{2.923256in}}%
\pgfpathlineto{\pgfqpoint{3.261732in}{2.845197in}}%
\pgfpathclose%
\pgfusepath{fill}%
\end{pgfscope}%
\begin{pgfscope}%
\pgfpathrectangle{\pgfqpoint{1.020000in}{0.880000in}}{\pgfqpoint{6.160000in}{6.160000in}}%
\pgfusepath{clip}%
\pgfsetbuttcap%
\pgfsetroundjoin%
\definecolor{currentfill}{rgb}{0.441123,0.576532,0.954545}%
\pgfsetfillcolor{currentfill}%
\pgfsetlinewidth{0.000000pt}%
\definecolor{currentstroke}{rgb}{0.000000,0.000000,0.000000}%
\pgfsetstrokecolor{currentstroke}%
\pgfsetdash{}{0pt}%
\pgfpathmoveto{\pgfqpoint{2.892034in}{2.634975in}}%
\pgfpathlineto{\pgfqpoint{2.899221in}{2.725000in}}%
\pgfpathlineto{\pgfqpoint{2.909473in}{2.579043in}}%
\pgfpathlineto{\pgfqpoint{2.943237in}{2.723928in}}%
\pgfpathlineto{\pgfqpoint{2.979448in}{2.673794in}}%
\pgfpathlineto{\pgfqpoint{2.971732in}{2.613874in}}%
\pgfpathlineto{\pgfqpoint{2.963109in}{2.630083in}}%
\pgfpathlineto{\pgfqpoint{2.930022in}{2.439886in}}%
\pgfpathlineto{\pgfqpoint{2.892034in}{2.634975in}}%
\pgfpathclose%
\pgfusepath{fill}%
\end{pgfscope}%
\begin{pgfscope}%
\pgfpathrectangle{\pgfqpoint{1.020000in}{0.880000in}}{\pgfqpoint{6.160000in}{6.160000in}}%
\pgfusepath{clip}%
\pgfsetbuttcap%
\pgfsetroundjoin%
\definecolor{currentfill}{rgb}{0.457046,0.594006,0.963029}%
\pgfsetfillcolor{currentfill}%
\pgfsetlinewidth{0.000000pt}%
\definecolor{currentstroke}{rgb}{0.000000,0.000000,0.000000}%
\pgfsetstrokecolor{currentstroke}%
\pgfsetdash{}{0pt}%
\pgfpathmoveto{\pgfqpoint{3.192930in}{2.629367in}}%
\pgfpathlineto{\pgfqpoint{3.201648in}{2.626297in}}%
\pgfpathlineto{\pgfqpoint{3.209193in}{2.744685in}}%
\pgfpathlineto{\pgfqpoint{3.246965in}{2.507269in}}%
\pgfpathlineto{\pgfqpoint{3.281991in}{2.550658in}}%
\pgfpathlineto{\pgfqpoint{3.272153in}{2.666522in}}%
\pgfpathlineto{\pgfqpoint{3.261732in}{2.845197in}}%
\pgfpathlineto{\pgfqpoint{3.227583in}{2.708233in}}%
\pgfpathlineto{\pgfqpoint{3.192930in}{2.629367in}}%
\pgfpathclose%
\pgfusepath{fill}%
\end{pgfscope}%
\begin{pgfscope}%
\pgfpathrectangle{\pgfqpoint{1.020000in}{0.880000in}}{\pgfqpoint{6.160000in}{6.160000in}}%
\pgfusepath{clip}%
\pgfsetbuttcap%
\pgfsetroundjoin%
\definecolor{currentfill}{rgb}{0.538004,0.674902,0.991722}%
\pgfsetfillcolor{currentfill}%
\pgfsetlinewidth{0.000000pt}%
\definecolor{currentstroke}{rgb}{0.000000,0.000000,0.000000}%
\pgfsetstrokecolor{currentstroke}%
\pgfsetdash{}{0pt}%
\pgfpathmoveto{\pgfqpoint{4.164595in}{2.863224in}}%
\pgfpathlineto{\pgfqpoint{4.174386in}{2.878584in}}%
\pgfpathlineto{\pgfqpoint{4.184200in}{2.666689in}}%
\pgfpathlineto{\pgfqpoint{4.219387in}{2.784688in}}%
\pgfpathlineto{\pgfqpoint{4.254500in}{2.714089in}}%
\pgfpathlineto{\pgfqpoint{4.244703in}{2.836429in}}%
\pgfpathlineto{\pgfqpoint{4.234833in}{2.822707in}}%
\pgfpathlineto{\pgfqpoint{4.199742in}{2.902127in}}%
\pgfpathlineto{\pgfqpoint{4.164595in}{2.863224in}}%
\pgfpathclose%
\pgfusepath{fill}%
\end{pgfscope}%
\begin{pgfscope}%
\pgfpathrectangle{\pgfqpoint{1.020000in}{0.880000in}}{\pgfqpoint{6.160000in}{6.160000in}}%
\pgfusepath{clip}%
\pgfsetbuttcap%
\pgfsetroundjoin%
\definecolor{currentfill}{rgb}{0.435815,0.570707,0.951717}%
\pgfsetfillcolor{currentfill}%
\pgfsetlinewidth{0.000000pt}%
\definecolor{currentstroke}{rgb}{0.000000,0.000000,0.000000}%
\pgfsetstrokecolor{currentstroke}%
\pgfsetdash{}{0pt}%
\pgfpathmoveto{\pgfqpoint{3.122532in}{2.583158in}}%
\pgfpathlineto{\pgfqpoint{3.131907in}{2.511013in}}%
\pgfpathlineto{\pgfqpoint{3.139543in}{2.604667in}}%
\pgfpathlineto{\pgfqpoint{3.175506in}{2.561344in}}%
\pgfpathlineto{\pgfqpoint{3.209193in}{2.744685in}}%
\pgfpathlineto{\pgfqpoint{3.201648in}{2.626297in}}%
\pgfpathlineto{\pgfqpoint{3.192930in}{2.629367in}}%
\pgfpathlineto{\pgfqpoint{3.157260in}{2.652077in}}%
\pgfpathlineto{\pgfqpoint{3.122532in}{2.583158in}}%
\pgfpathclose%
\pgfusepath{fill}%
\end{pgfscope}%
\begin{pgfscope}%
\pgfpathrectangle{\pgfqpoint{1.020000in}{0.880000in}}{\pgfqpoint{6.160000in}{6.160000in}}%
\pgfusepath{clip}%
\pgfsetbuttcap%
\pgfsetroundjoin%
\definecolor{currentfill}{rgb}{0.619318,0.744121,0.998931}%
\pgfsetfillcolor{currentfill}%
\pgfsetlinewidth{0.000000pt}%
\definecolor{currentstroke}{rgb}{0.000000,0.000000,0.000000}%
\pgfsetstrokecolor{currentstroke}%
\pgfsetdash{}{0pt}%
\pgfpathmoveto{\pgfqpoint{3.773822in}{3.024666in}}%
\pgfpathlineto{\pgfqpoint{3.783854in}{2.856278in}}%
\pgfpathlineto{\pgfqpoint{3.792839in}{2.952095in}}%
\pgfpathlineto{\pgfqpoint{3.828247in}{2.927172in}}%
\pgfpathlineto{\pgfqpoint{3.863812in}{2.841215in}}%
\pgfpathlineto{\pgfqpoint{3.853990in}{2.959031in}}%
\pgfpathlineto{\pgfqpoint{3.844141in}{3.084217in}}%
\pgfpathlineto{\pgfqpoint{3.808989in}{3.052396in}}%
\pgfpathlineto{\pgfqpoint{3.773822in}{3.024666in}}%
\pgfpathclose%
\pgfusepath{fill}%
\end{pgfscope}%
\begin{pgfscope}%
\pgfpathrectangle{\pgfqpoint{1.020000in}{0.880000in}}{\pgfqpoint{6.160000in}{6.160000in}}%
\pgfusepath{clip}%
\pgfsetbuttcap%
\pgfsetroundjoin%
\definecolor{currentfill}{rgb}{0.430507,0.564883,0.948889}%
\pgfsetfillcolor{currentfill}%
\pgfsetlinewidth{0.000000pt}%
\definecolor{currentstroke}{rgb}{0.000000,0.000000,0.000000}%
\pgfsetstrokecolor{currentstroke}%
\pgfsetdash{}{0pt}%
\pgfpathmoveto{\pgfqpoint{2.820739in}{2.648971in}}%
\pgfpathlineto{\pgfqpoint{2.830786in}{2.519239in}}%
\pgfpathlineto{\pgfqpoint{2.838689in}{2.547426in}}%
\pgfpathlineto{\pgfqpoint{2.874014in}{2.568749in}}%
\pgfpathlineto{\pgfqpoint{2.909473in}{2.579043in}}%
\pgfpathlineto{\pgfqpoint{2.899221in}{2.725000in}}%
\pgfpathlineto{\pgfqpoint{2.892034in}{2.634975in}}%
\pgfpathlineto{\pgfqpoint{2.857228in}{2.580050in}}%
\pgfpathlineto{\pgfqpoint{2.820739in}{2.648971in}}%
\pgfpathclose%
\pgfusepath{fill}%
\end{pgfscope}%
\begin{pgfscope}%
\pgfpathrectangle{\pgfqpoint{1.020000in}{0.880000in}}{\pgfqpoint{6.160000in}{6.160000in}}%
\pgfusepath{clip}%
\pgfsetbuttcap%
\pgfsetroundjoin%
\definecolor{currentfill}{rgb}{0.597777,0.727330,0.999777}%
\pgfsetfillcolor{currentfill}%
\pgfsetlinewidth{0.000000pt}%
\definecolor{currentstroke}{rgb}{0.000000,0.000000,0.000000}%
\pgfsetstrokecolor{currentstroke}%
\pgfsetdash{}{0pt}%
\pgfpathmoveto{\pgfqpoint{3.703941in}{2.873174in}}%
\pgfpathlineto{\pgfqpoint{3.712916in}{2.935108in}}%
\pgfpathlineto{\pgfqpoint{3.722364in}{2.902142in}}%
\pgfpathlineto{\pgfqpoint{3.757543in}{2.941314in}}%
\pgfpathlineto{\pgfqpoint{3.792839in}{2.952095in}}%
\pgfpathlineto{\pgfqpoint{3.783854in}{2.856278in}}%
\pgfpathlineto{\pgfqpoint{3.773822in}{3.024666in}}%
\pgfpathlineto{\pgfqpoint{3.739371in}{2.835791in}}%
\pgfpathlineto{\pgfqpoint{3.703941in}{2.873174in}}%
\pgfpathclose%
\pgfusepath{fill}%
\end{pgfscope}%
\begin{pgfscope}%
\pgfpathrectangle{\pgfqpoint{1.020000in}{0.880000in}}{\pgfqpoint{6.160000in}{6.160000in}}%
\pgfusepath{clip}%
\pgfsetbuttcap%
\pgfsetroundjoin%
\definecolor{currentfill}{rgb}{0.532568,0.669801,0.990393}%
\pgfsetfillcolor{currentfill}%
\pgfsetlinewidth{0.000000pt}%
\definecolor{currentstroke}{rgb}{0.000000,0.000000,0.000000}%
\pgfsetstrokecolor{currentstroke}%
\pgfsetdash{}{0pt}%
\pgfpathmoveto{\pgfqpoint{3.563606in}{2.723100in}}%
\pgfpathlineto{\pgfqpoint{3.572081in}{2.825880in}}%
\pgfpathlineto{\pgfqpoint{3.582119in}{2.676998in}}%
\pgfpathlineto{\pgfqpoint{3.617013in}{2.758012in}}%
\pgfpathlineto{\pgfqpoint{3.651788in}{2.867701in}}%
\pgfpathlineto{\pgfqpoint{3.643450in}{2.706448in}}%
\pgfpathlineto{\pgfqpoint{3.632923in}{2.950403in}}%
\pgfpathlineto{\pgfqpoint{3.598157in}{2.849999in}}%
\pgfpathlineto{\pgfqpoint{3.563606in}{2.723100in}}%
\pgfpathclose%
\pgfusepath{fill}%
\end{pgfscope}%
\begin{pgfscope}%
\pgfpathrectangle{\pgfqpoint{1.020000in}{0.880000in}}{\pgfqpoint{6.160000in}{6.160000in}}%
\pgfusepath{clip}%
\pgfsetbuttcap%
\pgfsetroundjoin%
\definecolor{currentfill}{rgb}{0.419991,0.552989,0.942630}%
\pgfsetfillcolor{currentfill}%
\pgfsetlinewidth{0.000000pt}%
\definecolor{currentstroke}{rgb}{0.000000,0.000000,0.000000}%
\pgfsetstrokecolor{currentstroke}%
\pgfsetdash{}{0pt}%
\pgfpathmoveto{\pgfqpoint{3.050660in}{2.664147in}}%
\pgfpathlineto{\pgfqpoint{3.059544in}{2.633924in}}%
\pgfpathlineto{\pgfqpoint{3.069121in}{2.543229in}}%
\pgfpathlineto{\pgfqpoint{3.105319in}{2.483251in}}%
\pgfpathlineto{\pgfqpoint{3.139543in}{2.604667in}}%
\pgfpathlineto{\pgfqpoint{3.131907in}{2.511013in}}%
\pgfpathlineto{\pgfqpoint{3.122532in}{2.583158in}}%
\pgfpathlineto{\pgfqpoint{3.086540in}{2.631091in}}%
\pgfpathlineto{\pgfqpoint{3.050660in}{2.664147in}}%
\pgfpathclose%
\pgfusepath{fill}%
\end{pgfscope}%
\begin{pgfscope}%
\pgfpathrectangle{\pgfqpoint{1.020000in}{0.880000in}}{\pgfqpoint{6.160000in}{6.160000in}}%
\pgfusepath{clip}%
\pgfsetbuttcap%
\pgfsetroundjoin%
\definecolor{currentfill}{rgb}{0.473070,0.611077,0.970634}%
\pgfsetfillcolor{currentfill}%
\pgfsetlinewidth{0.000000pt}%
\definecolor{currentstroke}{rgb}{0.000000,0.000000,0.000000}%
\pgfsetstrokecolor{currentstroke}%
\pgfsetdash{}{0pt}%
\pgfpathmoveto{\pgfqpoint{4.184200in}{2.666689in}}%
\pgfpathlineto{\pgfqpoint{4.194010in}{2.595214in}}%
\pgfpathlineto{\pgfqpoint{4.229167in}{2.599004in}}%
\pgfpathlineto{\pgfqpoint{4.264410in}{2.731754in}}%
\pgfpathlineto{\pgfqpoint{4.254500in}{2.714089in}}%
\pgfpathlineto{\pgfqpoint{4.219387in}{2.784688in}}%
\pgfpathlineto{\pgfqpoint{4.184200in}{2.666689in}}%
\pgfpathclose%
\pgfusepath{fill}%
\end{pgfscope}%
\begin{pgfscope}%
\pgfpathrectangle{\pgfqpoint{1.020000in}{0.880000in}}{\pgfqpoint{6.160000in}{6.160000in}}%
\pgfusepath{clip}%
\pgfsetbuttcap%
\pgfsetroundjoin%
\definecolor{currentfill}{rgb}{0.586921,0.718121,0.998874}%
\pgfsetfillcolor{currentfill}%
\pgfsetlinewidth{0.000000pt}%
\definecolor{currentstroke}{rgb}{0.000000,0.000000,0.000000}%
\pgfsetstrokecolor{currentstroke}%
\pgfsetdash{}{0pt}%
\pgfpathmoveto{\pgfqpoint{3.934227in}{2.855578in}}%
\pgfpathlineto{\pgfqpoint{3.943935in}{2.781189in}}%
\pgfpathlineto{\pgfqpoint{3.953394in}{2.817290in}}%
\pgfpathlineto{\pgfqpoint{3.988268in}{3.013055in}}%
\pgfpathlineto{\pgfqpoint{4.023795in}{2.873724in}}%
\pgfpathlineto{\pgfqpoint{4.014179in}{2.868278in}}%
\pgfpathlineto{\pgfqpoint{4.004535in}{2.890867in}}%
\pgfpathlineto{\pgfqpoint{3.969027in}{3.040192in}}%
\pgfpathlineto{\pgfqpoint{3.934227in}{2.855578in}}%
\pgfpathclose%
\pgfusepath{fill}%
\end{pgfscope}%
\begin{pgfscope}%
\pgfpathrectangle{\pgfqpoint{1.020000in}{0.880000in}}{\pgfqpoint{6.160000in}{6.160000in}}%
\pgfusepath{clip}%
\pgfsetbuttcap%
\pgfsetroundjoin%
\definecolor{currentfill}{rgb}{0.581486,0.713451,0.998314}%
\pgfsetfillcolor{currentfill}%
\pgfsetlinewidth{0.000000pt}%
\definecolor{currentstroke}{rgb}{0.000000,0.000000,0.000000}%
\pgfsetstrokecolor{currentstroke}%
\pgfsetdash{}{0pt}%
\pgfpathmoveto{\pgfqpoint{3.632923in}{2.950403in}}%
\pgfpathlineto{\pgfqpoint{3.643450in}{2.706448in}}%
\pgfpathlineto{\pgfqpoint{3.651788in}{2.867701in}}%
\pgfpathlineto{\pgfqpoint{3.687567in}{2.787399in}}%
\pgfpathlineto{\pgfqpoint{3.722364in}{2.902142in}}%
\pgfpathlineto{\pgfqpoint{3.712916in}{2.935108in}}%
\pgfpathlineto{\pgfqpoint{3.703941in}{2.873174in}}%
\pgfpathlineto{\pgfqpoint{3.667923in}{3.014983in}}%
\pgfpathlineto{\pgfqpoint{3.632923in}{2.950403in}}%
\pgfpathclose%
\pgfusepath{fill}%
\end{pgfscope}%
\begin{pgfscope}%
\pgfpathrectangle{\pgfqpoint{1.020000in}{0.880000in}}{\pgfqpoint{6.160000in}{6.160000in}}%
\pgfusepath{clip}%
\pgfsetbuttcap%
\pgfsetroundjoin%
\definecolor{currentfill}{rgb}{0.559747,0.694768,0.996075}%
\pgfsetfillcolor{currentfill}%
\pgfsetlinewidth{0.000000pt}%
\definecolor{currentstroke}{rgb}{0.000000,0.000000,0.000000}%
\pgfsetstrokecolor{currentstroke}%
\pgfsetdash{}{0pt}%
\pgfpathmoveto{\pgfqpoint{3.863812in}{2.841215in}}%
\pgfpathlineto{\pgfqpoint{3.873194in}{2.864603in}}%
\pgfpathlineto{\pgfqpoint{3.882944in}{2.774358in}}%
\pgfpathlineto{\pgfqpoint{3.918110in}{2.819329in}}%
\pgfpathlineto{\pgfqpoint{3.953394in}{2.817290in}}%
\pgfpathlineto{\pgfqpoint{3.943935in}{2.781189in}}%
\pgfpathlineto{\pgfqpoint{3.934227in}{2.855578in}}%
\pgfpathlineto{\pgfqpoint{3.898618in}{2.991916in}}%
\pgfpathlineto{\pgfqpoint{3.863812in}{2.841215in}}%
\pgfpathclose%
\pgfusepath{fill}%
\end{pgfscope}%
\begin{pgfscope}%
\pgfpathrectangle{\pgfqpoint{1.020000in}{0.880000in}}{\pgfqpoint{6.160000in}{6.160000in}}%
\pgfusepath{clip}%
\pgfsetbuttcap%
\pgfsetroundjoin%
\definecolor{currentfill}{rgb}{0.516260,0.654498,0.986407}%
\pgfsetfillcolor{currentfill}%
\pgfsetlinewidth{0.000000pt}%
\definecolor{currentstroke}{rgb}{0.000000,0.000000,0.000000}%
\pgfsetstrokecolor{currentstroke}%
\pgfsetdash{}{0pt}%
\pgfpathmoveto{\pgfqpoint{3.492319in}{2.812949in}}%
\pgfpathlineto{\pgfqpoint{3.501702in}{2.763103in}}%
\pgfpathlineto{\pgfqpoint{3.510986in}{2.729493in}}%
\pgfpathlineto{\pgfqpoint{3.545011in}{2.946827in}}%
\pgfpathlineto{\pgfqpoint{3.582119in}{2.676998in}}%
\pgfpathlineto{\pgfqpoint{3.572081in}{2.825880in}}%
\pgfpathlineto{\pgfqpoint{3.563606in}{2.723100in}}%
\pgfpathlineto{\pgfqpoint{3.528837in}{2.639578in}}%
\pgfpathlineto{\pgfqpoint{3.492319in}{2.812949in}}%
\pgfpathclose%
\pgfusepath{fill}%
\end{pgfscope}%
\begin{pgfscope}%
\pgfpathrectangle{\pgfqpoint{1.020000in}{0.880000in}}{\pgfqpoint{6.160000in}{6.160000in}}%
\pgfusepath{clip}%
\pgfsetbuttcap%
\pgfsetroundjoin%
\definecolor{currentfill}{rgb}{0.409611,0.540759,0.935545}%
\pgfsetfillcolor{currentfill}%
\pgfsetlinewidth{0.000000pt}%
\definecolor{currentstroke}{rgb}{0.000000,0.000000,0.000000}%
\pgfsetstrokecolor{currentstroke}%
\pgfsetdash{}{0pt}%
\pgfpathmoveto{\pgfqpoint{2.979448in}{2.673794in}}%
\pgfpathlineto{\pgfqpoint{2.991314in}{2.391549in}}%
\pgfpathlineto{\pgfqpoint{2.999940in}{2.377828in}}%
\pgfpathlineto{\pgfqpoint{3.031881in}{2.687333in}}%
\pgfpathlineto{\pgfqpoint{3.069121in}{2.543229in}}%
\pgfpathlineto{\pgfqpoint{3.059544in}{2.633924in}}%
\pgfpathlineto{\pgfqpoint{3.050660in}{2.664147in}}%
\pgfpathlineto{\pgfqpoint{3.017075in}{2.498382in}}%
\pgfpathlineto{\pgfqpoint{2.979448in}{2.673794in}}%
\pgfpathclose%
\pgfusepath{fill}%
\end{pgfscope}%
\begin{pgfscope}%
\pgfpathrectangle{\pgfqpoint{1.020000in}{0.880000in}}{\pgfqpoint{6.160000in}{6.160000in}}%
\pgfusepath{clip}%
\pgfsetbuttcap%
\pgfsetroundjoin%
\definecolor{currentfill}{rgb}{0.554312,0.690097,0.995516}%
\pgfsetfillcolor{currentfill}%
\pgfsetlinewidth{0.000000pt}%
\definecolor{currentstroke}{rgb}{0.000000,0.000000,0.000000}%
\pgfsetstrokecolor{currentstroke}%
\pgfsetdash{}{0pt}%
\pgfpathmoveto{\pgfqpoint{4.094375in}{2.721690in}}%
\pgfpathlineto{\pgfqpoint{4.103936in}{2.902392in}}%
\pgfpathlineto{\pgfqpoint{4.113834in}{2.683651in}}%
\pgfpathlineto{\pgfqpoint{4.149012in}{2.731375in}}%
\pgfpathlineto{\pgfqpoint{4.184200in}{2.666689in}}%
\pgfpathlineto{\pgfqpoint{4.174386in}{2.878584in}}%
\pgfpathlineto{\pgfqpoint{4.164595in}{2.863224in}}%
\pgfpathlineto{\pgfqpoint{4.129271in}{3.162058in}}%
\pgfpathlineto{\pgfqpoint{4.094375in}{2.721690in}}%
\pgfpathclose%
\pgfusepath{fill}%
\end{pgfscope}%
\begin{pgfscope}%
\pgfpathrectangle{\pgfqpoint{1.020000in}{0.880000in}}{\pgfqpoint{6.160000in}{6.160000in}}%
\pgfusepath{clip}%
\pgfsetbuttcap%
\pgfsetroundjoin%
\definecolor{currentfill}{rgb}{0.473070,0.611077,0.970634}%
\pgfsetfillcolor{currentfill}%
\pgfsetlinewidth{0.000000pt}%
\definecolor{currentstroke}{rgb}{0.000000,0.000000,0.000000}%
\pgfsetstrokecolor{currentstroke}%
\pgfsetdash{}{0pt}%
\pgfpathmoveto{\pgfqpoint{4.113834in}{2.683651in}}%
\pgfpathlineto{\pgfqpoint{4.123550in}{2.733110in}}%
\pgfpathlineto{\pgfqpoint{4.158816in}{2.663701in}}%
\pgfpathlineto{\pgfqpoint{4.194010in}{2.595214in}}%
\pgfpathlineto{\pgfqpoint{4.184200in}{2.666689in}}%
\pgfpathlineto{\pgfqpoint{4.149012in}{2.731375in}}%
\pgfpathlineto{\pgfqpoint{4.113834in}{2.683651in}}%
\pgfpathclose%
\pgfusepath{fill}%
\end{pgfscope}%
\begin{pgfscope}%
\pgfpathrectangle{\pgfqpoint{1.020000in}{0.880000in}}{\pgfqpoint{6.160000in}{6.160000in}}%
\pgfusepath{clip}%
\pgfsetbuttcap%
\pgfsetroundjoin%
\definecolor{currentfill}{rgb}{0.554312,0.690097,0.995516}%
\pgfsetfillcolor{currentfill}%
\pgfsetlinewidth{0.000000pt}%
\definecolor{currentstroke}{rgb}{0.000000,0.000000,0.000000}%
\pgfsetstrokecolor{currentstroke}%
\pgfsetdash{}{0pt}%
\pgfpathmoveto{\pgfqpoint{3.792839in}{2.952095in}}%
\pgfpathlineto{\pgfqpoint{3.803177in}{2.704716in}}%
\pgfpathlineto{\pgfqpoint{3.811550in}{2.976695in}}%
\pgfpathlineto{\pgfqpoint{3.848234in}{2.597246in}}%
\pgfpathlineto{\pgfqpoint{3.882944in}{2.774358in}}%
\pgfpathlineto{\pgfqpoint{3.873194in}{2.864603in}}%
\pgfpathlineto{\pgfqpoint{3.863812in}{2.841215in}}%
\pgfpathlineto{\pgfqpoint{3.828247in}{2.927172in}}%
\pgfpathlineto{\pgfqpoint{3.792839in}{2.952095in}}%
\pgfpathclose%
\pgfusepath{fill}%
\end{pgfscope}%
\begin{pgfscope}%
\pgfpathrectangle{\pgfqpoint{1.020000in}{0.880000in}}{\pgfqpoint{6.160000in}{6.160000in}}%
\pgfusepath{clip}%
\pgfsetbuttcap%
\pgfsetroundjoin%
\definecolor{currentfill}{rgb}{0.538004,0.674902,0.991722}%
\pgfsetfillcolor{currentfill}%
\pgfsetlinewidth{0.000000pt}%
\definecolor{currentstroke}{rgb}{0.000000,0.000000,0.000000}%
\pgfsetstrokecolor{currentstroke}%
\pgfsetdash{}{0pt}%
\pgfpathmoveto{\pgfqpoint{4.023795in}{2.873724in}}%
\pgfpathlineto{\pgfqpoint{4.033519in}{2.823005in}}%
\pgfpathlineto{\pgfqpoint{4.043097in}{2.883887in}}%
\pgfpathlineto{\pgfqpoint{4.078509in}{2.786965in}}%
\pgfpathlineto{\pgfqpoint{4.113834in}{2.683651in}}%
\pgfpathlineto{\pgfqpoint{4.103936in}{2.902392in}}%
\pgfpathlineto{\pgfqpoint{4.094375in}{2.721690in}}%
\pgfpathlineto{\pgfqpoint{4.059208in}{2.729811in}}%
\pgfpathlineto{\pgfqpoint{4.023795in}{2.873724in}}%
\pgfpathclose%
\pgfusepath{fill}%
\end{pgfscope}%
\begin{pgfscope}%
\pgfpathrectangle{\pgfqpoint{1.020000in}{0.880000in}}{\pgfqpoint{6.160000in}{6.160000in}}%
\pgfusepath{clip}%
\pgfsetbuttcap%
\pgfsetroundjoin%
\definecolor{currentfill}{rgb}{0.409611,0.540759,0.935545}%
\pgfsetfillcolor{currentfill}%
\pgfsetlinewidth{0.000000pt}%
\definecolor{currentstroke}{rgb}{0.000000,0.000000,0.000000}%
\pgfsetstrokecolor{currentstroke}%
\pgfsetdash{}{0pt}%
\pgfpathmoveto{\pgfqpoint{2.909473in}{2.579043in}}%
\pgfpathlineto{\pgfqpoint{2.919065in}{2.484409in}}%
\pgfpathlineto{\pgfqpoint{2.926851in}{2.532409in}}%
\pgfpathlineto{\pgfqpoint{2.961554in}{2.607813in}}%
\pgfpathlineto{\pgfqpoint{2.999940in}{2.377828in}}%
\pgfpathlineto{\pgfqpoint{2.991314in}{2.391549in}}%
\pgfpathlineto{\pgfqpoint{2.979448in}{2.673794in}}%
\pgfpathlineto{\pgfqpoint{2.943237in}{2.723928in}}%
\pgfpathlineto{\pgfqpoint{2.909473in}{2.579043in}}%
\pgfpathclose%
\pgfusepath{fill}%
\end{pgfscope}%
\begin{pgfscope}%
\pgfpathrectangle{\pgfqpoint{1.020000in}{0.880000in}}{\pgfqpoint{6.160000in}{6.160000in}}%
\pgfusepath{clip}%
\pgfsetbuttcap%
\pgfsetroundjoin%
\definecolor{currentfill}{rgb}{0.521696,0.659599,0.987736}%
\pgfsetfillcolor{currentfill}%
\pgfsetlinewidth{0.000000pt}%
\definecolor{currentstroke}{rgb}{0.000000,0.000000,0.000000}%
\pgfsetstrokecolor{currentstroke}%
\pgfsetdash{}{0pt}%
\pgfpathmoveto{\pgfqpoint{3.421192in}{2.853721in}}%
\pgfpathlineto{\pgfqpoint{3.430787in}{2.771145in}}%
\pgfpathlineto{\pgfqpoint{3.440670in}{2.650984in}}%
\pgfpathlineto{\pgfqpoint{3.475538in}{2.730605in}}%
\pgfpathlineto{\pgfqpoint{3.510986in}{2.729493in}}%
\pgfpathlineto{\pgfqpoint{3.501702in}{2.763103in}}%
\pgfpathlineto{\pgfqpoint{3.492319in}{2.812949in}}%
\pgfpathlineto{\pgfqpoint{3.456844in}{2.823679in}}%
\pgfpathlineto{\pgfqpoint{3.421192in}{2.853721in}}%
\pgfpathclose%
\pgfusepath{fill}%
\end{pgfscope}%
\begin{pgfscope}%
\pgfpathrectangle{\pgfqpoint{1.020000in}{0.880000in}}{\pgfqpoint{6.160000in}{6.160000in}}%
\pgfusepath{clip}%
\pgfsetbuttcap%
\pgfsetroundjoin%
\definecolor{currentfill}{rgb}{0.462354,0.599830,0.965857}%
\pgfsetfillcolor{currentfill}%
\pgfsetlinewidth{0.000000pt}%
\definecolor{currentstroke}{rgb}{0.000000,0.000000,0.000000}%
\pgfsetstrokecolor{currentstroke}%
\pgfsetdash{}{0pt}%
\pgfpathmoveto{\pgfqpoint{3.281991in}{2.550658in}}%
\pgfpathlineto{\pgfqpoint{3.289758in}{2.664840in}}%
\pgfpathlineto{\pgfqpoint{3.298798in}{2.640584in}}%
\pgfpathlineto{\pgfqpoint{3.334040in}{2.673198in}}%
\pgfpathlineto{\pgfqpoint{3.368862in}{2.757158in}}%
\pgfpathlineto{\pgfqpoint{3.361063in}{2.623461in}}%
\pgfpathlineto{\pgfqpoint{3.352094in}{2.634598in}}%
\pgfpathlineto{\pgfqpoint{3.316498in}{2.654560in}}%
\pgfpathlineto{\pgfqpoint{3.281991in}{2.550658in}}%
\pgfpathclose%
\pgfusepath{fill}%
\end{pgfscope}%
\begin{pgfscope}%
\pgfpathrectangle{\pgfqpoint{1.020000in}{0.880000in}}{\pgfqpoint{6.160000in}{6.160000in}}%
\pgfusepath{clip}%
\pgfsetbuttcap%
\pgfsetroundjoin%
\definecolor{currentfill}{rgb}{0.394042,0.522413,0.924916}%
\pgfsetfillcolor{currentfill}%
\pgfsetlinewidth{0.000000pt}%
\definecolor{currentstroke}{rgb}{0.000000,0.000000,0.000000}%
\pgfsetstrokecolor{currentstroke}%
\pgfsetdash{}{0pt}%
\pgfpathmoveto{\pgfqpoint{2.838689in}{2.547426in}}%
\pgfpathlineto{\pgfqpoint{2.847226in}{2.530229in}}%
\pgfpathlineto{\pgfqpoint{2.856731in}{2.442020in}}%
\pgfpathlineto{\pgfqpoint{2.892128in}{2.461021in}}%
\pgfpathlineto{\pgfqpoint{2.926851in}{2.532409in}}%
\pgfpathlineto{\pgfqpoint{2.919065in}{2.484409in}}%
\pgfpathlineto{\pgfqpoint{2.909473in}{2.579043in}}%
\pgfpathlineto{\pgfqpoint{2.874014in}{2.568749in}}%
\pgfpathlineto{\pgfqpoint{2.838689in}{2.547426in}}%
\pgfpathclose%
\pgfusepath{fill}%
\end{pgfscope}%
\begin{pgfscope}%
\pgfpathrectangle{\pgfqpoint{1.020000in}{0.880000in}}{\pgfqpoint{6.160000in}{6.160000in}}%
\pgfusepath{clip}%
\pgfsetbuttcap%
\pgfsetroundjoin%
\definecolor{currentfill}{rgb}{0.505423,0.643995,0.983157}%
\pgfsetfillcolor{currentfill}%
\pgfsetlinewidth{0.000000pt}%
\definecolor{currentstroke}{rgb}{0.000000,0.000000,0.000000}%
\pgfsetstrokecolor{currentstroke}%
\pgfsetdash{}{0pt}%
\pgfpathmoveto{\pgfqpoint{3.352094in}{2.634598in}}%
\pgfpathlineto{\pgfqpoint{3.361063in}{2.623461in}}%
\pgfpathlineto{\pgfqpoint{3.368862in}{2.757158in}}%
\pgfpathlineto{\pgfqpoint{3.404344in}{2.761657in}}%
\pgfpathlineto{\pgfqpoint{3.440670in}{2.650984in}}%
\pgfpathlineto{\pgfqpoint{3.430787in}{2.771145in}}%
\pgfpathlineto{\pgfqpoint{3.421192in}{2.853721in}}%
\pgfpathlineto{\pgfqpoint{3.386109in}{2.807270in}}%
\pgfpathlineto{\pgfqpoint{3.352094in}{2.634598in}}%
\pgfpathclose%
\pgfusepath{fill}%
\end{pgfscope}%
\begin{pgfscope}%
\pgfpathrectangle{\pgfqpoint{1.020000in}{0.880000in}}{\pgfqpoint{6.160000in}{6.160000in}}%
\pgfusepath{clip}%
\pgfsetbuttcap%
\pgfsetroundjoin%
\definecolor{currentfill}{rgb}{0.446431,0.582356,0.957373}%
\pgfsetfillcolor{currentfill}%
\pgfsetlinewidth{0.000000pt}%
\definecolor{currentstroke}{rgb}{0.000000,0.000000,0.000000}%
\pgfsetstrokecolor{currentstroke}%
\pgfsetdash{}{0pt}%
\pgfpathmoveto{\pgfqpoint{3.209193in}{2.744685in}}%
\pgfpathlineto{\pgfqpoint{3.219980in}{2.533772in}}%
\pgfpathlineto{\pgfqpoint{3.228982in}{2.506354in}}%
\pgfpathlineto{\pgfqpoint{3.262134in}{2.761484in}}%
\pgfpathlineto{\pgfqpoint{3.298798in}{2.640584in}}%
\pgfpathlineto{\pgfqpoint{3.289758in}{2.664840in}}%
\pgfpathlineto{\pgfqpoint{3.281991in}{2.550658in}}%
\pgfpathlineto{\pgfqpoint{3.246965in}{2.507269in}}%
\pgfpathlineto{\pgfqpoint{3.209193in}{2.744685in}}%
\pgfpathclose%
\pgfusepath{fill}%
\end{pgfscope}%
\begin{pgfscope}%
\pgfpathrectangle{\pgfqpoint{1.020000in}{0.880000in}}{\pgfqpoint{6.160000in}{6.160000in}}%
\pgfusepath{clip}%
\pgfsetbuttcap%
\pgfsetroundjoin%
\definecolor{currentfill}{rgb}{0.383662,0.510183,0.917831}%
\pgfsetfillcolor{currentfill}%
\pgfsetlinewidth{0.000000pt}%
\definecolor{currentstroke}{rgb}{0.000000,0.000000,0.000000}%
\pgfsetstrokecolor{currentstroke}%
\pgfsetdash{}{0pt}%
\pgfpathmoveto{\pgfqpoint{3.069121in}{2.543229in}}%
\pgfpathlineto{\pgfqpoint{3.079992in}{2.336885in}}%
\pgfpathlineto{\pgfqpoint{3.087591in}{2.424675in}}%
\pgfpathlineto{\pgfqpoint{3.122927in}{2.447312in}}%
\pgfpathlineto{\pgfqpoint{3.156966in}{2.594669in}}%
\pgfpathlineto{\pgfqpoint{3.148956in}{2.531586in}}%
\pgfpathlineto{\pgfqpoint{3.139543in}{2.604667in}}%
\pgfpathlineto{\pgfqpoint{3.105319in}{2.483251in}}%
\pgfpathlineto{\pgfqpoint{3.069121in}{2.543229in}}%
\pgfpathclose%
\pgfusepath{fill}%
\end{pgfscope}%
\begin{pgfscope}%
\pgfpathrectangle{\pgfqpoint{1.020000in}{0.880000in}}{\pgfqpoint{6.160000in}{6.160000in}}%
\pgfusepath{clip}%
\pgfsetbuttcap%
\pgfsetroundjoin%
\definecolor{currentfill}{rgb}{0.430507,0.564883,0.948889}%
\pgfsetfillcolor{currentfill}%
\pgfsetlinewidth{0.000000pt}%
\definecolor{currentstroke}{rgb}{0.000000,0.000000,0.000000}%
\pgfsetstrokecolor{currentstroke}%
\pgfsetdash{}{0pt}%
\pgfpathmoveto{\pgfqpoint{3.139543in}{2.604667in}}%
\pgfpathlineto{\pgfqpoint{3.148956in}{2.531586in}}%
\pgfpathlineto{\pgfqpoint{3.156966in}{2.594669in}}%
\pgfpathlineto{\pgfqpoint{3.193125in}{2.538138in}}%
\pgfpathlineto{\pgfqpoint{3.228982in}{2.506354in}}%
\pgfpathlineto{\pgfqpoint{3.219980in}{2.533772in}}%
\pgfpathlineto{\pgfqpoint{3.209193in}{2.744685in}}%
\pgfpathlineto{\pgfqpoint{3.175506in}{2.561344in}}%
\pgfpathlineto{\pgfqpoint{3.139543in}{2.604667in}}%
\pgfpathclose%
\pgfusepath{fill}%
\end{pgfscope}%
\begin{pgfscope}%
\pgfpathrectangle{\pgfqpoint{1.020000in}{0.880000in}}{\pgfqpoint{6.160000in}{6.160000in}}%
\pgfusepath{clip}%
\pgfsetbuttcap%
\pgfsetroundjoin%
\definecolor{currentfill}{rgb}{0.368507,0.491141,0.905243}%
\pgfsetfillcolor{currentfill}%
\pgfsetlinewidth{0.000000pt}%
\definecolor{currentstroke}{rgb}{0.000000,0.000000,0.000000}%
\pgfsetstrokecolor{currentstroke}%
\pgfsetdash{}{0pt}%
\pgfpathmoveto{\pgfqpoint{2.999940in}{2.377828in}}%
\pgfpathlineto{\pgfqpoint{3.006844in}{2.510596in}}%
\pgfpathlineto{\pgfqpoint{3.017924in}{2.292642in}}%
\pgfpathlineto{\pgfqpoint{3.051269in}{2.487830in}}%
\pgfpathlineto{\pgfqpoint{3.087591in}{2.424675in}}%
\pgfpathlineto{\pgfqpoint{3.079992in}{2.336885in}}%
\pgfpathlineto{\pgfqpoint{3.069121in}{2.543229in}}%
\pgfpathlineto{\pgfqpoint{3.031881in}{2.687333in}}%
\pgfpathlineto{\pgfqpoint{2.999940in}{2.377828in}}%
\pgfpathclose%
\pgfusepath{fill}%
\end{pgfscope}%
\begin{pgfscope}%
\pgfpathrectangle{\pgfqpoint{1.020000in}{0.880000in}}{\pgfqpoint{6.160000in}{6.160000in}}%
\pgfusepath{clip}%
\pgfsetbuttcap%
\pgfsetroundjoin%
\definecolor{currentfill}{rgb}{0.500031,0.638508,0.981070}%
\pgfsetfillcolor{currentfill}%
\pgfsetlinewidth{0.000000pt}%
\definecolor{currentstroke}{rgb}{0.000000,0.000000,0.000000}%
\pgfsetstrokecolor{currentstroke}%
\pgfsetdash{}{0pt}%
\pgfpathmoveto{\pgfqpoint{3.811550in}{2.976695in}}%
\pgfpathlineto{\pgfqpoint{3.822187in}{2.646493in}}%
\pgfpathlineto{\pgfqpoint{3.831540in}{2.667026in}}%
\pgfpathlineto{\pgfqpoint{3.866933in}{2.658560in}}%
\pgfpathlineto{\pgfqpoint{3.902450in}{2.596354in}}%
\pgfpathlineto{\pgfqpoint{3.892396in}{2.787321in}}%
\pgfpathlineto{\pgfqpoint{3.882944in}{2.774358in}}%
\pgfpathlineto{\pgfqpoint{3.848234in}{2.597246in}}%
\pgfpathlineto{\pgfqpoint{3.811550in}{2.976695in}}%
\pgfpathclose%
\pgfusepath{fill}%
\end{pgfscope}%
\begin{pgfscope}%
\pgfpathrectangle{\pgfqpoint{1.020000in}{0.880000in}}{\pgfqpoint{6.160000in}{6.160000in}}%
\pgfusepath{clip}%
\pgfsetbuttcap%
\pgfsetroundjoin%
\definecolor{currentfill}{rgb}{0.521696,0.659599,0.987736}%
\pgfsetfillcolor{currentfill}%
\pgfsetlinewidth{0.000000pt}%
\definecolor{currentstroke}{rgb}{0.000000,0.000000,0.000000}%
\pgfsetstrokecolor{currentstroke}%
\pgfsetdash{}{0pt}%
\pgfpathmoveto{\pgfqpoint{3.582119in}{2.676998in}}%
\pgfpathlineto{\pgfqpoint{3.590227in}{2.850685in}}%
\pgfpathlineto{\pgfqpoint{3.600782in}{2.616444in}}%
\pgfpathlineto{\pgfqpoint{3.635979in}{2.654805in}}%
\pgfpathlineto{\pgfqpoint{3.670623in}{2.801051in}}%
\pgfpathlineto{\pgfqpoint{3.661486in}{2.779637in}}%
\pgfpathlineto{\pgfqpoint{3.651788in}{2.867701in}}%
\pgfpathlineto{\pgfqpoint{3.617013in}{2.758012in}}%
\pgfpathlineto{\pgfqpoint{3.582119in}{2.676998in}}%
\pgfpathclose%
\pgfusepath{fill}%
\end{pgfscope}%
\begin{pgfscope}%
\pgfpathrectangle{\pgfqpoint{1.020000in}{0.880000in}}{\pgfqpoint{6.160000in}{6.160000in}}%
\pgfusepath{clip}%
\pgfsetbuttcap%
\pgfsetroundjoin%
\definecolor{currentfill}{rgb}{0.559747,0.694768,0.996075}%
\pgfsetfillcolor{currentfill}%
\pgfsetlinewidth{0.000000pt}%
\definecolor{currentstroke}{rgb}{0.000000,0.000000,0.000000}%
\pgfsetstrokecolor{currentstroke}%
\pgfsetdash{}{0pt}%
\pgfpathmoveto{\pgfqpoint{3.651788in}{2.867701in}}%
\pgfpathlineto{\pgfqpoint{3.661486in}{2.779637in}}%
\pgfpathlineto{\pgfqpoint{3.670623in}{2.801051in}}%
\pgfpathlineto{\pgfqpoint{3.706451in}{2.718082in}}%
\pgfpathlineto{\pgfqpoint{3.741578in}{2.774271in}}%
\pgfpathlineto{\pgfqpoint{3.731468in}{2.947985in}}%
\pgfpathlineto{\pgfqpoint{3.722364in}{2.902142in}}%
\pgfpathlineto{\pgfqpoint{3.687567in}{2.787399in}}%
\pgfpathlineto{\pgfqpoint{3.651788in}{2.867701in}}%
\pgfpathclose%
\pgfusepath{fill}%
\end{pgfscope}%
\begin{pgfscope}%
\pgfpathrectangle{\pgfqpoint{1.020000in}{0.880000in}}{\pgfqpoint{6.160000in}{6.160000in}}%
\pgfusepath{clip}%
\pgfsetbuttcap%
\pgfsetroundjoin%
\definecolor{currentfill}{rgb}{0.592356,0.722792,0.999434}%
\pgfsetfillcolor{currentfill}%
\pgfsetlinewidth{0.000000pt}%
\definecolor{currentstroke}{rgb}{0.000000,0.000000,0.000000}%
\pgfsetstrokecolor{currentstroke}%
\pgfsetdash{}{0pt}%
\pgfpathmoveto{\pgfqpoint{3.722364in}{2.902142in}}%
\pgfpathlineto{\pgfqpoint{3.731468in}{2.947985in}}%
\pgfpathlineto{\pgfqpoint{3.741578in}{2.774271in}}%
\pgfpathlineto{\pgfqpoint{3.776439in}{2.900041in}}%
\pgfpathlineto{\pgfqpoint{3.811550in}{2.976695in}}%
\pgfpathlineto{\pgfqpoint{3.803177in}{2.704716in}}%
\pgfpathlineto{\pgfqpoint{3.792839in}{2.952095in}}%
\pgfpathlineto{\pgfqpoint{3.757543in}{2.941314in}}%
\pgfpathlineto{\pgfqpoint{3.722364in}{2.902142in}}%
\pgfpathclose%
\pgfusepath{fill}%
\end{pgfscope}%
\begin{pgfscope}%
\pgfpathrectangle{\pgfqpoint{1.020000in}{0.880000in}}{\pgfqpoint{6.160000in}{6.160000in}}%
\pgfusepath{clip}%
\pgfsetbuttcap%
\pgfsetroundjoin%
\definecolor{currentfill}{rgb}{0.581486,0.713451,0.998314}%
\pgfsetfillcolor{currentfill}%
\pgfsetlinewidth{0.000000pt}%
\definecolor{currentstroke}{rgb}{0.000000,0.000000,0.000000}%
\pgfsetstrokecolor{currentstroke}%
\pgfsetdash{}{0pt}%
\pgfpathmoveto{\pgfqpoint{3.953394in}{2.817290in}}%
\pgfpathlineto{\pgfqpoint{3.962870in}{2.859212in}}%
\pgfpathlineto{\pgfqpoint{3.972305in}{2.934074in}}%
\pgfpathlineto{\pgfqpoint{4.008144in}{2.669021in}}%
\pgfpathlineto{\pgfqpoint{4.043097in}{2.883887in}}%
\pgfpathlineto{\pgfqpoint{4.033519in}{2.823005in}}%
\pgfpathlineto{\pgfqpoint{4.023795in}{2.873724in}}%
\pgfpathlineto{\pgfqpoint{3.988268in}{3.013055in}}%
\pgfpathlineto{\pgfqpoint{3.953394in}{2.817290in}}%
\pgfpathclose%
\pgfusepath{fill}%
\end{pgfscope}%
\begin{pgfscope}%
\pgfpathrectangle{\pgfqpoint{1.020000in}{0.880000in}}{\pgfqpoint{6.160000in}{6.160000in}}%
\pgfusepath{clip}%
\pgfsetbuttcap%
\pgfsetroundjoin%
\definecolor{currentfill}{rgb}{0.548876,0.685104,0.994379}%
\pgfsetfillcolor{currentfill}%
\pgfsetlinewidth{0.000000pt}%
\definecolor{currentstroke}{rgb}{0.000000,0.000000,0.000000}%
\pgfsetstrokecolor{currentstroke}%
\pgfsetdash{}{0pt}%
\pgfpathmoveto{\pgfqpoint{3.882944in}{2.774358in}}%
\pgfpathlineto{\pgfqpoint{3.892396in}{2.787321in}}%
\pgfpathlineto{\pgfqpoint{3.902450in}{2.596354in}}%
\pgfpathlineto{\pgfqpoint{3.937228in}{2.805832in}}%
\pgfpathlineto{\pgfqpoint{3.972305in}{2.934074in}}%
\pgfpathlineto{\pgfqpoint{3.962870in}{2.859212in}}%
\pgfpathlineto{\pgfqpoint{3.953394in}{2.817290in}}%
\pgfpathlineto{\pgfqpoint{3.918110in}{2.819329in}}%
\pgfpathlineto{\pgfqpoint{3.882944in}{2.774358in}}%
\pgfpathclose%
\pgfusepath{fill}%
\end{pgfscope}%
\begin{pgfscope}%
\pgfpathrectangle{\pgfqpoint{1.020000in}{0.880000in}}{\pgfqpoint{6.160000in}{6.160000in}}%
\pgfusepath{clip}%
\pgfsetbuttcap%
\pgfsetroundjoin%
\definecolor{currentfill}{rgb}{0.527132,0.664700,0.989065}%
\pgfsetfillcolor{currentfill}%
\pgfsetlinewidth{0.000000pt}%
\definecolor{currentstroke}{rgb}{0.000000,0.000000,0.000000}%
\pgfsetstrokecolor{currentstroke}%
\pgfsetdash{}{0pt}%
\pgfpathmoveto{\pgfqpoint{4.043097in}{2.883887in}}%
\pgfpathlineto{\pgfqpoint{4.052992in}{2.729179in}}%
\pgfpathlineto{\pgfqpoint{4.088254in}{2.760325in}}%
\pgfpathlineto{\pgfqpoint{4.123550in}{2.733110in}}%
\pgfpathlineto{\pgfqpoint{4.113834in}{2.683651in}}%
\pgfpathlineto{\pgfqpoint{4.078509in}{2.786965in}}%
\pgfpathlineto{\pgfqpoint{4.043097in}{2.883887in}}%
\pgfpathclose%
\pgfusepath{fill}%
\end{pgfscope}%
\begin{pgfscope}%
\pgfpathrectangle{\pgfqpoint{1.020000in}{0.880000in}}{\pgfqpoint{6.160000in}{6.160000in}}%
\pgfusepath{clip}%
\pgfsetbuttcap%
\pgfsetroundjoin%
\definecolor{currentfill}{rgb}{0.378598,0.503856,0.913692}%
\pgfsetfillcolor{currentfill}%
\pgfsetlinewidth{0.000000pt}%
\definecolor{currentstroke}{rgb}{0.000000,0.000000,0.000000}%
\pgfsetstrokecolor{currentstroke}%
\pgfsetdash{}{0pt}%
\pgfpathmoveto{\pgfqpoint{2.926851in}{2.532409in}}%
\pgfpathlineto{\pgfqpoint{2.935022in}{2.552001in}}%
\pgfpathlineto{\pgfqpoint{2.944891in}{2.437665in}}%
\pgfpathlineto{\pgfqpoint{2.980944in}{2.406409in}}%
\pgfpathlineto{\pgfqpoint{3.017924in}{2.292642in}}%
\pgfpathlineto{\pgfqpoint{3.006844in}{2.510596in}}%
\pgfpathlineto{\pgfqpoint{2.999940in}{2.377828in}}%
\pgfpathlineto{\pgfqpoint{2.961554in}{2.607813in}}%
\pgfpathlineto{\pgfqpoint{2.926851in}{2.532409in}}%
\pgfpathclose%
\pgfusepath{fill}%
\end{pgfscope}%
\begin{pgfscope}%
\pgfpathrectangle{\pgfqpoint{1.020000in}{0.880000in}}{\pgfqpoint{6.160000in}{6.160000in}}%
\pgfusepath{clip}%
\pgfsetbuttcap%
\pgfsetroundjoin%
\definecolor{currentfill}{rgb}{0.489246,0.627536,0.976896}%
\pgfsetfillcolor{currentfill}%
\pgfsetlinewidth{0.000000pt}%
\definecolor{currentstroke}{rgb}{0.000000,0.000000,0.000000}%
\pgfsetstrokecolor{currentstroke}%
\pgfsetdash{}{0pt}%
\pgfpathmoveto{\pgfqpoint{3.440670in}{2.650984in}}%
\pgfpathlineto{\pgfqpoint{3.449345in}{2.693923in}}%
\pgfpathlineto{\pgfqpoint{3.459451in}{2.544723in}}%
\pgfpathlineto{\pgfqpoint{3.494463in}{2.610904in}}%
\pgfpathlineto{\pgfqpoint{3.529522in}{2.673717in}}%
\pgfpathlineto{\pgfqpoint{3.519547in}{2.806385in}}%
\pgfpathlineto{\pgfqpoint{3.510986in}{2.729493in}}%
\pgfpathlineto{\pgfqpoint{3.475538in}{2.730605in}}%
\pgfpathlineto{\pgfqpoint{3.440670in}{2.650984in}}%
\pgfpathclose%
\pgfusepath{fill}%
\end{pgfscope}%
\begin{pgfscope}%
\pgfpathrectangle{\pgfqpoint{1.020000in}{0.880000in}}{\pgfqpoint{6.160000in}{6.160000in}}%
\pgfusepath{clip}%
\pgfsetbuttcap%
\pgfsetroundjoin%
\definecolor{currentfill}{rgb}{0.521696,0.659599,0.987736}%
\pgfsetfillcolor{currentfill}%
\pgfsetlinewidth{0.000000pt}%
\definecolor{currentstroke}{rgb}{0.000000,0.000000,0.000000}%
\pgfsetstrokecolor{currentstroke}%
\pgfsetdash{}{0pt}%
\pgfpathmoveto{\pgfqpoint{3.510986in}{2.729493in}}%
\pgfpathlineto{\pgfqpoint{3.519547in}{2.806385in}}%
\pgfpathlineto{\pgfqpoint{3.529522in}{2.673717in}}%
\pgfpathlineto{\pgfqpoint{3.564905in}{2.688184in}}%
\pgfpathlineto{\pgfqpoint{3.600782in}{2.616444in}}%
\pgfpathlineto{\pgfqpoint{3.590227in}{2.850685in}}%
\pgfpathlineto{\pgfqpoint{3.582119in}{2.676998in}}%
\pgfpathlineto{\pgfqpoint{3.545011in}{2.946827in}}%
\pgfpathlineto{\pgfqpoint{3.510986in}{2.729493in}}%
\pgfpathclose%
\pgfusepath{fill}%
\end{pgfscope}%
\begin{pgfscope}%
\pgfpathrectangle{\pgfqpoint{1.020000in}{0.880000in}}{\pgfqpoint{6.160000in}{6.160000in}}%
\pgfusepath{clip}%
\pgfsetbuttcap%
\pgfsetroundjoin%
\definecolor{currentfill}{rgb}{0.462354,0.599830,0.965857}%
\pgfsetfillcolor{currentfill}%
\pgfsetlinewidth{0.000000pt}%
\definecolor{currentstroke}{rgb}{0.000000,0.000000,0.000000}%
\pgfsetstrokecolor{currentstroke}%
\pgfsetdash{}{0pt}%
\pgfpathmoveto{\pgfqpoint{3.368862in}{2.757158in}}%
\pgfpathlineto{\pgfqpoint{3.379499in}{2.546317in}}%
\pgfpathlineto{\pgfqpoint{3.388171in}{2.578409in}}%
\pgfpathlineto{\pgfqpoint{3.424240in}{2.507850in}}%
\pgfpathlineto{\pgfqpoint{3.459451in}{2.544723in}}%
\pgfpathlineto{\pgfqpoint{3.449345in}{2.693923in}}%
\pgfpathlineto{\pgfqpoint{3.440670in}{2.650984in}}%
\pgfpathlineto{\pgfqpoint{3.404344in}{2.761657in}}%
\pgfpathlineto{\pgfqpoint{3.368862in}{2.757158in}}%
\pgfpathclose%
\pgfusepath{fill}%
\end{pgfscope}%
\begin{pgfscope}%
\pgfpathrectangle{\pgfqpoint{1.020000in}{0.880000in}}{\pgfqpoint{6.160000in}{6.160000in}}%
\pgfusepath{clip}%
\pgfsetbuttcap%
\pgfsetroundjoin%
\definecolor{currentfill}{rgb}{0.527132,0.664700,0.989065}%
\pgfsetfillcolor{currentfill}%
\pgfsetlinewidth{0.000000pt}%
\definecolor{currentstroke}{rgb}{0.000000,0.000000,0.000000}%
\pgfsetstrokecolor{currentstroke}%
\pgfsetdash{}{0pt}%
\pgfpathmoveto{\pgfqpoint{3.972305in}{2.934074in}}%
\pgfpathlineto{\pgfqpoint{3.982473in}{2.664731in}}%
\pgfpathlineto{\pgfqpoint{4.017833in}{2.636190in}}%
\pgfpathlineto{\pgfqpoint{4.052992in}{2.729179in}}%
\pgfpathlineto{\pgfqpoint{4.043097in}{2.883887in}}%
\pgfpathlineto{\pgfqpoint{4.008144in}{2.669021in}}%
\pgfpathlineto{\pgfqpoint{3.972305in}{2.934074in}}%
\pgfpathclose%
\pgfusepath{fill}%
\end{pgfscope}%
\begin{pgfscope}%
\pgfpathrectangle{\pgfqpoint{1.020000in}{0.880000in}}{\pgfqpoint{6.160000in}{6.160000in}}%
\pgfusepath{clip}%
\pgfsetbuttcap%
\pgfsetroundjoin%
\definecolor{currentfill}{rgb}{0.373552,0.497499,0.909467}%
\pgfsetfillcolor{currentfill}%
\pgfsetlinewidth{0.000000pt}%
\definecolor{currentstroke}{rgb}{0.000000,0.000000,0.000000}%
\pgfsetstrokecolor{currentstroke}%
\pgfsetdash{}{0pt}%
\pgfpathmoveto{\pgfqpoint{3.156966in}{2.594669in}}%
\pgfpathlineto{\pgfqpoint{3.168866in}{2.281576in}}%
\pgfpathlineto{\pgfqpoint{3.176417in}{2.392589in}}%
\pgfpathlineto{\pgfqpoint{3.212008in}{2.393226in}}%
\pgfpathlineto{\pgfqpoint{3.247118in}{2.442340in}}%
\pgfpathlineto{\pgfqpoint{3.237689in}{2.511390in}}%
\pgfpathlineto{\pgfqpoint{3.228982in}{2.506354in}}%
\pgfpathlineto{\pgfqpoint{3.193125in}{2.538138in}}%
\pgfpathlineto{\pgfqpoint{3.156966in}{2.594669in}}%
\pgfpathclose%
\pgfusepath{fill}%
\end{pgfscope}%
\begin{pgfscope}%
\pgfpathrectangle{\pgfqpoint{1.020000in}{0.880000in}}{\pgfqpoint{6.160000in}{6.160000in}}%
\pgfusepath{clip}%
\pgfsetbuttcap%
\pgfsetroundjoin%
\definecolor{currentfill}{rgb}{0.441123,0.576532,0.954545}%
\pgfsetfillcolor{currentfill}%
\pgfsetlinewidth{0.000000pt}%
\definecolor{currentstroke}{rgb}{0.000000,0.000000,0.000000}%
\pgfsetstrokecolor{currentstroke}%
\pgfsetdash{}{0pt}%
\pgfpathmoveto{\pgfqpoint{3.298798in}{2.640584in}}%
\pgfpathlineto{\pgfqpoint{3.308558in}{2.536615in}}%
\pgfpathlineto{\pgfqpoint{3.317992in}{2.470143in}}%
\pgfpathlineto{\pgfqpoint{3.353536in}{2.469027in}}%
\pgfpathlineto{\pgfqpoint{3.388171in}{2.578409in}}%
\pgfpathlineto{\pgfqpoint{3.379499in}{2.546317in}}%
\pgfpathlineto{\pgfqpoint{3.368862in}{2.757158in}}%
\pgfpathlineto{\pgfqpoint{3.334040in}{2.673198in}}%
\pgfpathlineto{\pgfqpoint{3.298798in}{2.640584in}}%
\pgfpathclose%
\pgfusepath{fill}%
\end{pgfscope}%
\begin{pgfscope}%
\pgfpathrectangle{\pgfqpoint{1.020000in}{0.880000in}}{\pgfqpoint{6.160000in}{6.160000in}}%
\pgfusepath{clip}%
\pgfsetbuttcap%
\pgfsetroundjoin%
\definecolor{currentfill}{rgb}{0.467678,0.605591,0.968546}%
\pgfsetfillcolor{currentfill}%
\pgfsetlinewidth{0.000000pt}%
\definecolor{currentstroke}{rgb}{0.000000,0.000000,0.000000}%
\pgfsetstrokecolor{currentstroke}%
\pgfsetdash{}{0pt}%
\pgfpathmoveto{\pgfqpoint{3.831540in}{2.667026in}}%
\pgfpathlineto{\pgfqpoint{3.841434in}{2.538023in}}%
\pgfpathlineto{\pgfqpoint{3.876399in}{2.664977in}}%
\pgfpathlineto{\pgfqpoint{3.911723in}{2.685455in}}%
\pgfpathlineto{\pgfqpoint{3.902450in}{2.596354in}}%
\pgfpathlineto{\pgfqpoint{3.866933in}{2.658560in}}%
\pgfpathlineto{\pgfqpoint{3.831540in}{2.667026in}}%
\pgfpathclose%
\pgfusepath{fill}%
\end{pgfscope}%
\begin{pgfscope}%
\pgfpathrectangle{\pgfqpoint{1.020000in}{0.880000in}}{\pgfqpoint{6.160000in}{6.160000in}}%
\pgfusepath{clip}%
\pgfsetbuttcap%
\pgfsetroundjoin%
\definecolor{currentfill}{rgb}{0.527132,0.664700,0.989065}%
\pgfsetfillcolor{currentfill}%
\pgfsetlinewidth{0.000000pt}%
\definecolor{currentstroke}{rgb}{0.000000,0.000000,0.000000}%
\pgfsetstrokecolor{currentstroke}%
\pgfsetdash{}{0pt}%
\pgfpathmoveto{\pgfqpoint{3.741578in}{2.774271in}}%
\pgfpathlineto{\pgfqpoint{3.750654in}{2.835596in}}%
\pgfpathlineto{\pgfqpoint{3.760778in}{2.658391in}}%
\pgfpathlineto{\pgfqpoint{3.796514in}{2.574100in}}%
\pgfpathlineto{\pgfqpoint{3.831540in}{2.667026in}}%
\pgfpathlineto{\pgfqpoint{3.822187in}{2.646493in}}%
\pgfpathlineto{\pgfqpoint{3.811550in}{2.976695in}}%
\pgfpathlineto{\pgfqpoint{3.776439in}{2.900041in}}%
\pgfpathlineto{\pgfqpoint{3.741578in}{2.774271in}}%
\pgfpathclose%
\pgfusepath{fill}%
\end{pgfscope}%
\begin{pgfscope}%
\pgfpathrectangle{\pgfqpoint{1.020000in}{0.880000in}}{\pgfqpoint{6.160000in}{6.160000in}}%
\pgfusepath{clip}%
\pgfsetbuttcap%
\pgfsetroundjoin%
\definecolor{currentfill}{rgb}{0.441123,0.576532,0.954545}%
\pgfsetfillcolor{currentfill}%
\pgfsetlinewidth{0.000000pt}%
\definecolor{currentstroke}{rgb}{0.000000,0.000000,0.000000}%
\pgfsetstrokecolor{currentstroke}%
\pgfsetdash{}{0pt}%
\pgfpathmoveto{\pgfqpoint{3.760778in}{2.658391in}}%
\pgfpathlineto{\pgfqpoint{3.770231in}{2.639464in}}%
\pgfpathlineto{\pgfqpoint{3.806415in}{2.443295in}}%
\pgfpathlineto{\pgfqpoint{3.841434in}{2.538023in}}%
\pgfpathlineto{\pgfqpoint{3.831540in}{2.667026in}}%
\pgfpathlineto{\pgfqpoint{3.796514in}{2.574100in}}%
\pgfpathlineto{\pgfqpoint{3.760778in}{2.658391in}}%
\pgfpathclose%
\pgfusepath{fill}%
\end{pgfscope}%
\begin{pgfscope}%
\pgfpathrectangle{\pgfqpoint{1.020000in}{0.880000in}}{\pgfqpoint{6.160000in}{6.160000in}}%
\pgfusepath{clip}%
\pgfsetbuttcap%
\pgfsetroundjoin%
\definecolor{currentfill}{rgb}{0.383662,0.510183,0.917831}%
\pgfsetfillcolor{currentfill}%
\pgfsetlinewidth{0.000000pt}%
\definecolor{currentstroke}{rgb}{0.000000,0.000000,0.000000}%
\pgfsetstrokecolor{currentstroke}%
\pgfsetdash{}{0pt}%
\pgfpathmoveto{\pgfqpoint{2.856731in}{2.442020in}}%
\pgfpathlineto{\pgfqpoint{2.864613in}{2.476094in}}%
\pgfpathlineto{\pgfqpoint{2.873570in}{2.430640in}}%
\pgfpathlineto{\pgfqpoint{2.909639in}{2.403236in}}%
\pgfpathlineto{\pgfqpoint{2.944891in}{2.437665in}}%
\pgfpathlineto{\pgfqpoint{2.935022in}{2.552001in}}%
\pgfpathlineto{\pgfqpoint{2.926851in}{2.532409in}}%
\pgfpathlineto{\pgfqpoint{2.892128in}{2.461021in}}%
\pgfpathlineto{\pgfqpoint{2.856731in}{2.442020in}}%
\pgfpathclose%
\pgfusepath{fill}%
\end{pgfscope}%
\begin{pgfscope}%
\pgfpathrectangle{\pgfqpoint{1.020000in}{0.880000in}}{\pgfqpoint{6.160000in}{6.160000in}}%
\pgfusepath{clip}%
\pgfsetbuttcap%
\pgfsetroundjoin%
\definecolor{currentfill}{rgb}{0.430507,0.564883,0.948889}%
\pgfsetfillcolor{currentfill}%
\pgfsetlinewidth{0.000000pt}%
\definecolor{currentstroke}{rgb}{0.000000,0.000000,0.000000}%
\pgfsetstrokecolor{currentstroke}%
\pgfsetdash{}{0pt}%
\pgfpathmoveto{\pgfqpoint{3.228982in}{2.506354in}}%
\pgfpathlineto{\pgfqpoint{3.237689in}{2.511390in}}%
\pgfpathlineto{\pgfqpoint{3.247118in}{2.442340in}}%
\pgfpathlineto{\pgfqpoint{3.281092in}{2.617767in}}%
\pgfpathlineto{\pgfqpoint{3.317992in}{2.470143in}}%
\pgfpathlineto{\pgfqpoint{3.308558in}{2.536615in}}%
\pgfpathlineto{\pgfqpoint{3.298798in}{2.640584in}}%
\pgfpathlineto{\pgfqpoint{3.262134in}{2.761484in}}%
\pgfpathlineto{\pgfqpoint{3.228982in}{2.506354in}}%
\pgfpathclose%
\pgfusepath{fill}%
\end{pgfscope}%
\begin{pgfscope}%
\pgfpathrectangle{\pgfqpoint{1.020000in}{0.880000in}}{\pgfqpoint{6.160000in}{6.160000in}}%
\pgfusepath{clip}%
\pgfsetbuttcap%
\pgfsetroundjoin%
\definecolor{currentfill}{rgb}{0.521696,0.659599,0.987736}%
\pgfsetfillcolor{currentfill}%
\pgfsetlinewidth{0.000000pt}%
\definecolor{currentstroke}{rgb}{0.000000,0.000000,0.000000}%
\pgfsetstrokecolor{currentstroke}%
\pgfsetdash{}{0pt}%
\pgfpathmoveto{\pgfqpoint{3.902450in}{2.596354in}}%
\pgfpathlineto{\pgfqpoint{3.911723in}{2.685455in}}%
\pgfpathlineto{\pgfqpoint{3.946914in}{2.758773in}}%
\pgfpathlineto{\pgfqpoint{3.982473in}{2.664731in}}%
\pgfpathlineto{\pgfqpoint{3.972305in}{2.934074in}}%
\pgfpathlineto{\pgfqpoint{3.937228in}{2.805832in}}%
\pgfpathlineto{\pgfqpoint{3.902450in}{2.596354in}}%
\pgfpathclose%
\pgfusepath{fill}%
\end{pgfscope}%
\begin{pgfscope}%
\pgfpathrectangle{\pgfqpoint{1.020000in}{0.880000in}}{\pgfqpoint{6.160000in}{6.160000in}}%
\pgfusepath{clip}%
\pgfsetbuttcap%
\pgfsetroundjoin%
\definecolor{currentfill}{rgb}{0.373552,0.497499,0.909467}%
\pgfsetfillcolor{currentfill}%
\pgfsetlinewidth{0.000000pt}%
\definecolor{currentstroke}{rgb}{0.000000,0.000000,0.000000}%
\pgfsetstrokecolor{currentstroke}%
\pgfsetdash{}{0pt}%
\pgfpathmoveto{\pgfqpoint{3.087591in}{2.424675in}}%
\pgfpathlineto{\pgfqpoint{3.096992in}{2.351516in}}%
\pgfpathlineto{\pgfqpoint{3.104399in}{2.462084in}}%
\pgfpathlineto{\pgfqpoint{3.138282in}{2.631476in}}%
\pgfpathlineto{\pgfqpoint{3.176417in}{2.392589in}}%
\pgfpathlineto{\pgfqpoint{3.168866in}{2.281576in}}%
\pgfpathlineto{\pgfqpoint{3.156966in}{2.594669in}}%
\pgfpathlineto{\pgfqpoint{3.122927in}{2.447312in}}%
\pgfpathlineto{\pgfqpoint{3.087591in}{2.424675in}}%
\pgfpathclose%
\pgfusepath{fill}%
\end{pgfscope}%
\begin{pgfscope}%
\pgfpathrectangle{\pgfqpoint{1.020000in}{0.880000in}}{\pgfqpoint{6.160000in}{6.160000in}}%
\pgfusepath{clip}%
\pgfsetbuttcap%
\pgfsetroundjoin%
\definecolor{currentfill}{rgb}{0.521696,0.659599,0.987736}%
\pgfsetfillcolor{currentfill}%
\pgfsetlinewidth{0.000000pt}%
\definecolor{currentstroke}{rgb}{0.000000,0.000000,0.000000}%
\pgfsetstrokecolor{currentstroke}%
\pgfsetdash{}{0pt}%
\pgfpathmoveto{\pgfqpoint{3.670623in}{2.801051in}}%
\pgfpathlineto{\pgfqpoint{3.679944in}{2.791293in}}%
\pgfpathlineto{\pgfqpoint{3.689589in}{2.720043in}}%
\pgfpathlineto{\pgfqpoint{3.725728in}{2.576051in}}%
\pgfpathlineto{\pgfqpoint{3.760778in}{2.658391in}}%
\pgfpathlineto{\pgfqpoint{3.750654in}{2.835596in}}%
\pgfpathlineto{\pgfqpoint{3.741578in}{2.774271in}}%
\pgfpathlineto{\pgfqpoint{3.706451in}{2.718082in}}%
\pgfpathlineto{\pgfqpoint{3.670623in}{2.801051in}}%
\pgfpathclose%
\pgfusepath{fill}%
\end{pgfscope}%
\begin{pgfscope}%
\pgfpathrectangle{\pgfqpoint{1.020000in}{0.880000in}}{\pgfqpoint{6.160000in}{6.160000in}}%
\pgfusepath{clip}%
\pgfsetbuttcap%
\pgfsetroundjoin%
\definecolor{currentfill}{rgb}{0.353369,0.472069,0.892570}%
\pgfsetfillcolor{currentfill}%
\pgfsetlinewidth{0.000000pt}%
\definecolor{currentstroke}{rgb}{0.000000,0.000000,0.000000}%
\pgfsetstrokecolor{currentstroke}%
\pgfsetdash{}{0pt}%
\pgfpathmoveto{\pgfqpoint{3.017924in}{2.292642in}}%
\pgfpathlineto{\pgfqpoint{3.025572in}{2.366384in}}%
\pgfpathlineto{\pgfqpoint{3.033031in}{2.459168in}}%
\pgfpathlineto{\pgfqpoint{3.069731in}{2.371277in}}%
\pgfpathlineto{\pgfqpoint{3.104399in}{2.462084in}}%
\pgfpathlineto{\pgfqpoint{3.096992in}{2.351516in}}%
\pgfpathlineto{\pgfqpoint{3.087591in}{2.424675in}}%
\pgfpathlineto{\pgfqpoint{3.051269in}{2.487830in}}%
\pgfpathlineto{\pgfqpoint{3.017924in}{2.292642in}}%
\pgfpathclose%
\pgfusepath{fill}%
\end{pgfscope}%
\begin{pgfscope}%
\pgfpathrectangle{\pgfqpoint{1.020000in}{0.880000in}}{\pgfqpoint{6.160000in}{6.160000in}}%
\pgfusepath{clip}%
\pgfsetbuttcap%
\pgfsetroundjoin%
\definecolor{currentfill}{rgb}{0.505423,0.643995,0.983157}%
\pgfsetfillcolor{currentfill}%
\pgfsetlinewidth{0.000000pt}%
\definecolor{currentstroke}{rgb}{0.000000,0.000000,0.000000}%
\pgfsetstrokecolor{currentstroke}%
\pgfsetdash{}{0pt}%
\pgfpathmoveto{\pgfqpoint{3.600782in}{2.616444in}}%
\pgfpathlineto{\pgfqpoint{3.609626in}{2.674456in}}%
\pgfpathlineto{\pgfqpoint{3.619424in}{2.570271in}}%
\pgfpathlineto{\pgfqpoint{3.653824in}{2.770402in}}%
\pgfpathlineto{\pgfqpoint{3.689589in}{2.720043in}}%
\pgfpathlineto{\pgfqpoint{3.679944in}{2.791293in}}%
\pgfpathlineto{\pgfqpoint{3.670623in}{2.801051in}}%
\pgfpathlineto{\pgfqpoint{3.635979in}{2.654805in}}%
\pgfpathlineto{\pgfqpoint{3.600782in}{2.616444in}}%
\pgfpathclose%
\pgfusepath{fill}%
\end{pgfscope}%
\begin{pgfscope}%
\pgfpathrectangle{\pgfqpoint{1.020000in}{0.880000in}}{\pgfqpoint{6.160000in}{6.160000in}}%
\pgfusepath{clip}%
\pgfsetbuttcap%
\pgfsetroundjoin%
\definecolor{currentfill}{rgb}{0.473070,0.611077,0.970634}%
\pgfsetfillcolor{currentfill}%
\pgfsetlinewidth{0.000000pt}%
\definecolor{currentstroke}{rgb}{0.000000,0.000000,0.000000}%
\pgfsetstrokecolor{currentstroke}%
\pgfsetdash{}{0pt}%
\pgfpathmoveto{\pgfqpoint{3.529522in}{2.673717in}}%
\pgfpathlineto{\pgfqpoint{3.539566in}{2.529948in}}%
\pgfpathlineto{\pgfqpoint{3.547190in}{2.763281in}}%
\pgfpathlineto{\pgfqpoint{3.583608in}{2.624652in}}%
\pgfpathlineto{\pgfqpoint{3.619424in}{2.570271in}}%
\pgfpathlineto{\pgfqpoint{3.609626in}{2.674456in}}%
\pgfpathlineto{\pgfqpoint{3.600782in}{2.616444in}}%
\pgfpathlineto{\pgfqpoint{3.564905in}{2.688184in}}%
\pgfpathlineto{\pgfqpoint{3.529522in}{2.673717in}}%
\pgfpathclose%
\pgfusepath{fill}%
\end{pgfscope}%
\begin{pgfscope}%
\pgfpathrectangle{\pgfqpoint{1.020000in}{0.880000in}}{\pgfqpoint{6.160000in}{6.160000in}}%
\pgfusepath{clip}%
\pgfsetbuttcap%
\pgfsetroundjoin%
\definecolor{currentfill}{rgb}{0.446431,0.582356,0.957373}%
\pgfsetfillcolor{currentfill}%
\pgfsetlinewidth{0.000000pt}%
\definecolor{currentstroke}{rgb}{0.000000,0.000000,0.000000}%
\pgfsetstrokecolor{currentstroke}%
\pgfsetdash{}{0pt}%
\pgfpathmoveto{\pgfqpoint{3.459451in}{2.544723in}}%
\pgfpathlineto{\pgfqpoint{3.467620in}{2.663280in}}%
\pgfpathlineto{\pgfqpoint{3.478046in}{2.470703in}}%
\pgfpathlineto{\pgfqpoint{3.513900in}{2.421297in}}%
\pgfpathlineto{\pgfqpoint{3.547190in}{2.763281in}}%
\pgfpathlineto{\pgfqpoint{3.539566in}{2.529948in}}%
\pgfpathlineto{\pgfqpoint{3.529522in}{2.673717in}}%
\pgfpathlineto{\pgfqpoint{3.494463in}{2.610904in}}%
\pgfpathlineto{\pgfqpoint{3.459451in}{2.544723in}}%
\pgfpathclose%
\pgfusepath{fill}%
\end{pgfscope}%
\begin{pgfscope}%
\pgfpathrectangle{\pgfqpoint{1.020000in}{0.880000in}}{\pgfqpoint{6.160000in}{6.160000in}}%
\pgfusepath{clip}%
\pgfsetbuttcap%
\pgfsetroundjoin%
\definecolor{currentfill}{rgb}{0.419991,0.552989,0.942630}%
\pgfsetfillcolor{currentfill}%
\pgfsetlinewidth{0.000000pt}%
\definecolor{currentstroke}{rgb}{0.000000,0.000000,0.000000}%
\pgfsetstrokecolor{currentstroke}%
\pgfsetdash{}{0pt}%
\pgfpathmoveto{\pgfqpoint{3.388171in}{2.578409in}}%
\pgfpathlineto{\pgfqpoint{3.397756in}{2.498441in}}%
\pgfpathlineto{\pgfqpoint{3.406611in}{2.512575in}}%
\pgfpathlineto{\pgfqpoint{3.442580in}{2.460549in}}%
\pgfpathlineto{\pgfqpoint{3.478046in}{2.470703in}}%
\pgfpathlineto{\pgfqpoint{3.467620in}{2.663280in}}%
\pgfpathlineto{\pgfqpoint{3.459451in}{2.544723in}}%
\pgfpathlineto{\pgfqpoint{3.424240in}{2.507850in}}%
\pgfpathlineto{\pgfqpoint{3.388171in}{2.578409in}}%
\pgfpathclose%
\pgfusepath{fill}%
\end{pgfscope}%
\begin{pgfscope}%
\pgfpathrectangle{\pgfqpoint{1.020000in}{0.880000in}}{\pgfqpoint{6.160000in}{6.160000in}}%
\pgfusepath{clip}%
\pgfsetbuttcap%
\pgfsetroundjoin%
\definecolor{currentfill}{rgb}{0.333490,0.446265,0.874452}%
\pgfsetfillcolor{currentfill}%
\pgfsetlinewidth{0.000000pt}%
\definecolor{currentstroke}{rgb}{0.000000,0.000000,0.000000}%
\pgfsetstrokecolor{currentstroke}%
\pgfsetdash{}{0pt}%
\pgfpathmoveto{\pgfqpoint{3.176417in}{2.392589in}}%
\pgfpathlineto{\pgfqpoint{3.186468in}{2.259665in}}%
\pgfpathlineto{\pgfqpoint{3.194279in}{2.350321in}}%
\pgfpathlineto{\pgfqpoint{3.230413in}{2.300802in}}%
\pgfpathlineto{\pgfqpoint{3.265558in}{2.351648in}}%
\pgfpathlineto{\pgfqpoint{3.256863in}{2.340292in}}%
\pgfpathlineto{\pgfqpoint{3.247118in}{2.442340in}}%
\pgfpathlineto{\pgfqpoint{3.212008in}{2.393226in}}%
\pgfpathlineto{\pgfqpoint{3.176417in}{2.392589in}}%
\pgfpathclose%
\pgfusepath{fill}%
\end{pgfscope}%
\begin{pgfscope}%
\pgfpathrectangle{\pgfqpoint{1.020000in}{0.880000in}}{\pgfqpoint{6.160000in}{6.160000in}}%
\pgfusepath{clip}%
\pgfsetbuttcap%
\pgfsetroundjoin%
\definecolor{currentfill}{rgb}{0.353369,0.472069,0.892570}%
\pgfsetfillcolor{currentfill}%
\pgfsetlinewidth{0.000000pt}%
\definecolor{currentstroke}{rgb}{0.000000,0.000000,0.000000}%
\pgfsetstrokecolor{currentstroke}%
\pgfsetdash{}{0pt}%
\pgfpathmoveto{\pgfqpoint{2.944891in}{2.437665in}}%
\pgfpathlineto{\pgfqpoint{2.954071in}{2.378738in}}%
\pgfpathlineto{\pgfqpoint{2.961357in}{2.474055in}}%
\pgfpathlineto{\pgfqpoint{2.998611in}{2.349523in}}%
\pgfpathlineto{\pgfqpoint{3.033031in}{2.459168in}}%
\pgfpathlineto{\pgfqpoint{3.025572in}{2.366384in}}%
\pgfpathlineto{\pgfqpoint{3.017924in}{2.292642in}}%
\pgfpathlineto{\pgfqpoint{2.980944in}{2.406409in}}%
\pgfpathlineto{\pgfqpoint{2.944891in}{2.437665in}}%
\pgfpathclose%
\pgfusepath{fill}%
\end{pgfscope}%
\begin{pgfscope}%
\pgfpathrectangle{\pgfqpoint{1.020000in}{0.880000in}}{\pgfqpoint{6.160000in}{6.160000in}}%
\pgfusepath{clip}%
\pgfsetbuttcap%
\pgfsetroundjoin%
\definecolor{currentfill}{rgb}{0.409611,0.540759,0.935545}%
\pgfsetfillcolor{currentfill}%
\pgfsetlinewidth{0.000000pt}%
\definecolor{currentstroke}{rgb}{0.000000,0.000000,0.000000}%
\pgfsetstrokecolor{currentstroke}%
\pgfsetdash{}{0pt}%
\pgfpathmoveto{\pgfqpoint{3.317992in}{2.470143in}}%
\pgfpathlineto{\pgfqpoint{3.327003in}{2.453455in}}%
\pgfpathlineto{\pgfqpoint{3.334938in}{2.564630in}}%
\pgfpathlineto{\pgfqpoint{3.370995in}{2.514156in}}%
\pgfpathlineto{\pgfqpoint{3.406611in}{2.512575in}}%
\pgfpathlineto{\pgfqpoint{3.397756in}{2.498441in}}%
\pgfpathlineto{\pgfqpoint{3.388171in}{2.578409in}}%
\pgfpathlineto{\pgfqpoint{3.353536in}{2.469027in}}%
\pgfpathlineto{\pgfqpoint{3.317992in}{2.470143in}}%
\pgfpathclose%
\pgfusepath{fill}%
\end{pgfscope}%
\begin{pgfscope}%
\pgfpathrectangle{\pgfqpoint{1.020000in}{0.880000in}}{\pgfqpoint{6.160000in}{6.160000in}}%
\pgfusepath{clip}%
\pgfsetbuttcap%
\pgfsetroundjoin%
\definecolor{currentfill}{rgb}{0.489246,0.627536,0.976896}%
\pgfsetfillcolor{currentfill}%
\pgfsetlinewidth{0.000000pt}%
\definecolor{currentstroke}{rgb}{0.000000,0.000000,0.000000}%
\pgfsetstrokecolor{currentstroke}%
\pgfsetdash{}{0pt}%
\pgfpathmoveto{\pgfqpoint{3.689589in}{2.720043in}}%
\pgfpathlineto{\pgfqpoint{3.699415in}{2.613267in}}%
\pgfpathlineto{\pgfqpoint{3.734288in}{2.746115in}}%
\pgfpathlineto{\pgfqpoint{3.770231in}{2.639464in}}%
\pgfpathlineto{\pgfqpoint{3.760778in}{2.658391in}}%
\pgfpathlineto{\pgfqpoint{3.725728in}{2.576051in}}%
\pgfpathlineto{\pgfqpoint{3.689589in}{2.720043in}}%
\pgfpathclose%
\pgfusepath{fill}%
\end{pgfscope}%
\begin{pgfscope}%
\pgfpathrectangle{\pgfqpoint{1.020000in}{0.880000in}}{\pgfqpoint{6.160000in}{6.160000in}}%
\pgfusepath{clip}%
\pgfsetbuttcap%
\pgfsetroundjoin%
\definecolor{currentfill}{rgb}{0.383662,0.510183,0.917831}%
\pgfsetfillcolor{currentfill}%
\pgfsetlinewidth{0.000000pt}%
\definecolor{currentstroke}{rgb}{0.000000,0.000000,0.000000}%
\pgfsetstrokecolor{currentstroke}%
\pgfsetdash{}{0pt}%
\pgfpathmoveto{\pgfqpoint{3.247118in}{2.442340in}}%
\pgfpathlineto{\pgfqpoint{3.256863in}{2.340292in}}%
\pgfpathlineto{\pgfqpoint{3.265558in}{2.351648in}}%
\pgfpathlineto{\pgfqpoint{3.300767in}{2.396950in}}%
\pgfpathlineto{\pgfqpoint{3.334938in}{2.564630in}}%
\pgfpathlineto{\pgfqpoint{3.327003in}{2.453455in}}%
\pgfpathlineto{\pgfqpoint{3.317992in}{2.470143in}}%
\pgfpathlineto{\pgfqpoint{3.281092in}{2.617767in}}%
\pgfpathlineto{\pgfqpoint{3.247118in}{2.442340in}}%
\pgfpathclose%
\pgfusepath{fill}%
\end{pgfscope}%
\begin{pgfscope}%
\pgfpathrectangle{\pgfqpoint{1.020000in}{0.880000in}}{\pgfqpoint{6.160000in}{6.160000in}}%
\pgfusepath{clip}%
\pgfsetbuttcap%
\pgfsetroundjoin%
\definecolor{currentfill}{rgb}{0.478462,0.616564,0.972721}%
\pgfsetfillcolor{currentfill}%
\pgfsetlinewidth{0.000000pt}%
\definecolor{currentstroke}{rgb}{0.000000,0.000000,0.000000}%
\pgfsetstrokecolor{currentstroke}%
\pgfsetdash{}{0pt}%
\pgfpathmoveto{\pgfqpoint{3.619424in}{2.570271in}}%
\pgfpathlineto{\pgfqpoint{3.628936in}{2.517517in}}%
\pgfpathlineto{\pgfqpoint{3.663967in}{2.604168in}}%
\pgfpathlineto{\pgfqpoint{3.699415in}{2.613267in}}%
\pgfpathlineto{\pgfqpoint{3.689589in}{2.720043in}}%
\pgfpathlineto{\pgfqpoint{3.653824in}{2.770402in}}%
\pgfpathlineto{\pgfqpoint{3.619424in}{2.570271in}}%
\pgfpathclose%
\pgfusepath{fill}%
\end{pgfscope}%
\begin{pgfscope}%
\pgfpathrectangle{\pgfqpoint{1.020000in}{0.880000in}}{\pgfqpoint{6.160000in}{6.160000in}}%
\pgfusepath{clip}%
\pgfsetbuttcap%
\pgfsetroundjoin%
\definecolor{currentfill}{rgb}{0.353369,0.472069,0.892570}%
\pgfsetfillcolor{currentfill}%
\pgfsetlinewidth{0.000000pt}%
\definecolor{currentstroke}{rgb}{0.000000,0.000000,0.000000}%
\pgfsetstrokecolor{currentstroke}%
\pgfsetdash{}{0pt}%
\pgfpathmoveto{\pgfqpoint{3.104399in}{2.462084in}}%
\pgfpathlineto{\pgfqpoint{3.113942in}{2.378514in}}%
\pgfpathlineto{\pgfqpoint{3.123579in}{2.286765in}}%
\pgfpathlineto{\pgfqpoint{3.158801in}{2.330874in}}%
\pgfpathlineto{\pgfqpoint{3.194279in}{2.350321in}}%
\pgfpathlineto{\pgfqpoint{3.186468in}{2.259665in}}%
\pgfpathlineto{\pgfqpoint{3.176417in}{2.392589in}}%
\pgfpathlineto{\pgfqpoint{3.138282in}{2.631476in}}%
\pgfpathlineto{\pgfqpoint{3.104399in}{2.462084in}}%
\pgfpathclose%
\pgfusepath{fill}%
\end{pgfscope}%
\begin{pgfscope}%
\pgfpathrectangle{\pgfqpoint{1.020000in}{0.880000in}}{\pgfqpoint{6.160000in}{6.160000in}}%
\pgfusepath{clip}%
\pgfsetbuttcap%
\pgfsetroundjoin%
\definecolor{currentfill}{rgb}{0.373552,0.497499,0.909467}%
\pgfsetfillcolor{currentfill}%
\pgfsetlinewidth{0.000000pt}%
\definecolor{currentstroke}{rgb}{0.000000,0.000000,0.000000}%
\pgfsetstrokecolor{currentstroke}%
\pgfsetdash{}{0pt}%
\pgfpathmoveto{\pgfqpoint{2.873570in}{2.430640in}}%
\pgfpathlineto{\pgfqpoint{2.882727in}{2.370891in}}%
\pgfpathlineto{\pgfqpoint{2.889241in}{2.514257in}}%
\pgfpathlineto{\pgfqpoint{2.926418in}{2.407470in}}%
\pgfpathlineto{\pgfqpoint{2.961357in}{2.474055in}}%
\pgfpathlineto{\pgfqpoint{2.954071in}{2.378738in}}%
\pgfpathlineto{\pgfqpoint{2.944891in}{2.437665in}}%
\pgfpathlineto{\pgfqpoint{2.909639in}{2.403236in}}%
\pgfpathlineto{\pgfqpoint{2.873570in}{2.430640in}}%
\pgfpathclose%
\pgfusepath{fill}%
\end{pgfscope}%
\begin{pgfscope}%
\pgfpathrectangle{\pgfqpoint{1.020000in}{0.880000in}}{\pgfqpoint{6.160000in}{6.160000in}}%
\pgfusepath{clip}%
\pgfsetbuttcap%
\pgfsetroundjoin%
\definecolor{currentfill}{rgb}{0.467678,0.605591,0.968546}%
\pgfsetfillcolor{currentfill}%
\pgfsetlinewidth{0.000000pt}%
\definecolor{currentstroke}{rgb}{0.000000,0.000000,0.000000}%
\pgfsetstrokecolor{currentstroke}%
\pgfsetdash{}{0pt}%
\pgfpathmoveto{\pgfqpoint{3.547190in}{2.763281in}}%
\pgfpathlineto{\pgfqpoint{3.557431in}{2.591946in}}%
\pgfpathlineto{\pgfqpoint{3.593098in}{2.573267in}}%
\pgfpathlineto{\pgfqpoint{3.628936in}{2.517517in}}%
\pgfpathlineto{\pgfqpoint{3.619424in}{2.570271in}}%
\pgfpathlineto{\pgfqpoint{3.583608in}{2.624652in}}%
\pgfpathlineto{\pgfqpoint{3.547190in}{2.763281in}}%
\pgfpathclose%
\pgfusepath{fill}%
\end{pgfscope}%
\begin{pgfscope}%
\pgfpathrectangle{\pgfqpoint{1.020000in}{0.880000in}}{\pgfqpoint{6.160000in}{6.160000in}}%
\pgfusepath{clip}%
\pgfsetbuttcap%
\pgfsetroundjoin%
\definecolor{currentfill}{rgb}{0.399231,0.528528,0.928459}%
\pgfsetfillcolor{currentfill}%
\pgfsetlinewidth{0.000000pt}%
\definecolor{currentstroke}{rgb}{0.000000,0.000000,0.000000}%
\pgfsetstrokecolor{currentstroke}%
\pgfsetdash{}{0pt}%
\pgfpathmoveto{\pgfqpoint{3.406611in}{2.512575in}}%
\pgfpathlineto{\pgfqpoint{3.416129in}{2.443894in}}%
\pgfpathlineto{\pgfqpoint{3.452112in}{2.391097in}}%
\pgfpathlineto{\pgfqpoint{3.486122in}{2.610409in}}%
\pgfpathlineto{\pgfqpoint{3.478046in}{2.470703in}}%
\pgfpathlineto{\pgfqpoint{3.442580in}{2.460549in}}%
\pgfpathlineto{\pgfqpoint{3.406611in}{2.512575in}}%
\pgfpathclose%
\pgfusepath{fill}%
\end{pgfscope}%
\begin{pgfscope}%
\pgfpathrectangle{\pgfqpoint{1.020000in}{0.880000in}}{\pgfqpoint{6.160000in}{6.160000in}}%
\pgfusepath{clip}%
\pgfsetbuttcap%
\pgfsetroundjoin%
\definecolor{currentfill}{rgb}{0.446431,0.582356,0.957373}%
\pgfsetfillcolor{currentfill}%
\pgfsetlinewidth{0.000000pt}%
\definecolor{currentstroke}{rgb}{0.000000,0.000000,0.000000}%
\pgfsetstrokecolor{currentstroke}%
\pgfsetdash{}{0pt}%
\pgfpathmoveto{\pgfqpoint{3.478046in}{2.470703in}}%
\pgfpathlineto{\pgfqpoint{3.486122in}{2.610409in}}%
\pgfpathlineto{\pgfqpoint{3.521979in}{2.573060in}}%
\pgfpathlineto{\pgfqpoint{3.557431in}{2.591946in}}%
\pgfpathlineto{\pgfqpoint{3.547190in}{2.763281in}}%
\pgfpathlineto{\pgfqpoint{3.513900in}{2.421297in}}%
\pgfpathlineto{\pgfqpoint{3.478046in}{2.470703in}}%
\pgfpathclose%
\pgfusepath{fill}%
\end{pgfscope}%
\begin{pgfscope}%
\pgfpathrectangle{\pgfqpoint{1.020000in}{0.880000in}}{\pgfqpoint{6.160000in}{6.160000in}}%
\pgfusepath{clip}%
\pgfsetbuttcap%
\pgfsetroundjoin%
\definecolor{currentfill}{rgb}{0.348323,0.465711,0.888346}%
\pgfsetfillcolor{currentfill}%
\pgfsetlinewidth{0.000000pt}%
\definecolor{currentstroke}{rgb}{0.000000,0.000000,0.000000}%
\pgfsetstrokecolor{currentstroke}%
\pgfsetdash{}{0pt}%
\pgfpathmoveto{\pgfqpoint{3.033031in}{2.459168in}}%
\pgfpathlineto{\pgfqpoint{3.043924in}{2.257156in}}%
\pgfpathlineto{\pgfqpoint{3.051785in}{2.318864in}}%
\pgfpathlineto{\pgfqpoint{3.086412in}{2.418966in}}%
\pgfpathlineto{\pgfqpoint{3.123579in}{2.286765in}}%
\pgfpathlineto{\pgfqpoint{3.113942in}{2.378514in}}%
\pgfpathlineto{\pgfqpoint{3.104399in}{2.462084in}}%
\pgfpathlineto{\pgfqpoint{3.069731in}{2.371277in}}%
\pgfpathlineto{\pgfqpoint{3.033031in}{2.459168in}}%
\pgfpathclose%
\pgfusepath{fill}%
\end{pgfscope}%
\begin{pgfscope}%
\pgfpathrectangle{\pgfqpoint{1.020000in}{0.880000in}}{\pgfqpoint{6.160000in}{6.160000in}}%
\pgfusepath{clip}%
\pgfsetbuttcap%
\pgfsetroundjoin%
\definecolor{currentfill}{rgb}{0.414801,0.546874,0.939088}%
\pgfsetfillcolor{currentfill}%
\pgfsetlinewidth{0.000000pt}%
\definecolor{currentstroke}{rgb}{0.000000,0.000000,0.000000}%
\pgfsetstrokecolor{currentstroke}%
\pgfsetdash{}{0pt}%
\pgfpathmoveto{\pgfqpoint{3.334938in}{2.564630in}}%
\pgfpathlineto{\pgfqpoint{3.343524in}{2.603545in}}%
\pgfpathlineto{\pgfqpoint{3.381258in}{2.352180in}}%
\pgfpathlineto{\pgfqpoint{3.416129in}{2.443894in}}%
\pgfpathlineto{\pgfqpoint{3.406611in}{2.512575in}}%
\pgfpathlineto{\pgfqpoint{3.370995in}{2.514156in}}%
\pgfpathlineto{\pgfqpoint{3.334938in}{2.564630in}}%
\pgfpathclose%
\pgfusepath{fill}%
\end{pgfscope}%
\begin{pgfscope}%
\pgfpathrectangle{\pgfqpoint{1.020000in}{0.880000in}}{\pgfqpoint{6.160000in}{6.160000in}}%
\pgfusepath{clip}%
\pgfsetbuttcap%
\pgfsetroundjoin%
\definecolor{currentfill}{rgb}{0.348323,0.465711,0.888346}%
\pgfsetfillcolor{currentfill}%
\pgfsetlinewidth{0.000000pt}%
\definecolor{currentstroke}{rgb}{0.000000,0.000000,0.000000}%
\pgfsetstrokecolor{currentstroke}%
\pgfsetdash{}{0pt}%
\pgfpathmoveto{\pgfqpoint{2.961357in}{2.474055in}}%
\pgfpathlineto{\pgfqpoint{2.971790in}{2.315487in}}%
\pgfpathlineto{\pgfqpoint{2.979303in}{2.396186in}}%
\pgfpathlineto{\pgfqpoint{3.015477in}{2.365394in}}%
\pgfpathlineto{\pgfqpoint{3.051785in}{2.318864in}}%
\pgfpathlineto{\pgfqpoint{3.043924in}{2.257156in}}%
\pgfpathlineto{\pgfqpoint{3.033031in}{2.459168in}}%
\pgfpathlineto{\pgfqpoint{2.998611in}{2.349523in}}%
\pgfpathlineto{\pgfqpoint{2.961357in}{2.474055in}}%
\pgfpathclose%
\pgfusepath{fill}%
\end{pgfscope}%
\begin{pgfscope}%
\pgfpathrectangle{\pgfqpoint{1.020000in}{0.880000in}}{\pgfqpoint{6.160000in}{6.160000in}}%
\pgfusepath{clip}%
\pgfsetbuttcap%
\pgfsetroundjoin%
\definecolor{currentfill}{rgb}{0.363461,0.484784,0.901019}%
\pgfsetfillcolor{currentfill}%
\pgfsetlinewidth{0.000000pt}%
\definecolor{currentstroke}{rgb}{0.000000,0.000000,0.000000}%
\pgfsetstrokecolor{currentstroke}%
\pgfsetdash{}{0pt}%
\pgfpathmoveto{\pgfqpoint{3.194279in}{2.350321in}}%
\pgfpathlineto{\pgfqpoint{3.202097in}{2.443956in}}%
\pgfpathlineto{\pgfqpoint{3.238964in}{2.324790in}}%
\pgfpathlineto{\pgfqpoint{3.271982in}{2.612587in}}%
\pgfpathlineto{\pgfqpoint{3.265558in}{2.351648in}}%
\pgfpathlineto{\pgfqpoint{3.230413in}{2.300802in}}%
\pgfpathlineto{\pgfqpoint{3.194279in}{2.350321in}}%
\pgfpathclose%
\pgfusepath{fill}%
\end{pgfscope}%
\begin{pgfscope}%
\pgfpathrectangle{\pgfqpoint{1.020000in}{0.880000in}}{\pgfqpoint{6.160000in}{6.160000in}}%
\pgfusepath{clip}%
\pgfsetbuttcap%
\pgfsetroundjoin%
\definecolor{currentfill}{rgb}{0.343278,0.459354,0.884122}%
\pgfsetfillcolor{currentfill}%
\pgfsetlinewidth{0.000000pt}%
\definecolor{currentstroke}{rgb}{0.000000,0.000000,0.000000}%
\pgfsetstrokecolor{currentstroke}%
\pgfsetdash{}{0pt}%
\pgfpathmoveto{\pgfqpoint{3.123579in}{2.286765in}}%
\pgfpathlineto{\pgfqpoint{3.133337in}{2.183835in}}%
\pgfpathlineto{\pgfqpoint{3.165672in}{2.509595in}}%
\pgfpathlineto{\pgfqpoint{3.202097in}{2.443956in}}%
\pgfpathlineto{\pgfqpoint{3.194279in}{2.350321in}}%
\pgfpathlineto{\pgfqpoint{3.158801in}{2.330874in}}%
\pgfpathlineto{\pgfqpoint{3.123579in}{2.286765in}}%
\pgfpathclose%
\pgfusepath{fill}%
\end{pgfscope}%
\begin{pgfscope}%
\pgfpathrectangle{\pgfqpoint{1.020000in}{0.880000in}}{\pgfqpoint{6.160000in}{6.160000in}}%
\pgfusepath{clip}%
\pgfsetbuttcap%
\pgfsetroundjoin%
\definecolor{currentfill}{rgb}{0.414801,0.546874,0.939088}%
\pgfsetfillcolor{currentfill}%
\pgfsetlinewidth{0.000000pt}%
\definecolor{currentstroke}{rgb}{0.000000,0.000000,0.000000}%
\pgfsetstrokecolor{currentstroke}%
\pgfsetdash{}{0pt}%
\pgfpathmoveto{\pgfqpoint{3.265558in}{2.351648in}}%
\pgfpathlineto{\pgfqpoint{3.271982in}{2.612587in}}%
\pgfpathlineto{\pgfqpoint{3.309674in}{2.391957in}}%
\pgfpathlineto{\pgfqpoint{3.343524in}{2.603545in}}%
\pgfpathlineto{\pgfqpoint{3.334938in}{2.564630in}}%
\pgfpathlineto{\pgfqpoint{3.300767in}{2.396950in}}%
\pgfpathlineto{\pgfqpoint{3.265558in}{2.351648in}}%
\pgfpathclose%
\pgfusepath{fill}%
\end{pgfscope}%
\begin{pgfscope}%
\pgfpathrectangle{\pgfqpoint{1.020000in}{0.880000in}}{\pgfqpoint{6.160000in}{6.160000in}}%
\pgfusepath{clip}%
\pgfsetbuttcap%
\pgfsetroundjoin%
\definecolor{currentfill}{rgb}{0.338377,0.452819,0.879317}%
\pgfsetfillcolor{currentfill}%
\pgfsetlinewidth{0.000000pt}%
\definecolor{currentstroke}{rgb}{0.000000,0.000000,0.000000}%
\pgfsetstrokecolor{currentstroke}%
\pgfsetdash{}{0pt}%
\pgfpathmoveto{\pgfqpoint{3.051785in}{2.318864in}}%
\pgfpathlineto{\pgfqpoint{3.059613in}{2.386011in}}%
\pgfpathlineto{\pgfqpoint{3.095246in}{2.401132in}}%
\pgfpathlineto{\pgfqpoint{3.133337in}{2.183835in}}%
\pgfpathlineto{\pgfqpoint{3.123579in}{2.286765in}}%
\pgfpathlineto{\pgfqpoint{3.086412in}{2.418966in}}%
\pgfpathlineto{\pgfqpoint{3.051785in}{2.318864in}}%
\pgfpathclose%
\pgfusepath{fill}%
\end{pgfscope}%
\begin{pgfscope}%
\pgfpathrectangle{\pgfqpoint{1.020000in}{0.880000in}}{\pgfqpoint{6.160000in}{6.160000in}}%
\pgfusepath{clip}%
\pgfsetbuttcap%
\pgfsetroundjoin%
\definecolor{currentfill}{rgb}{0.373552,0.497499,0.909467}%
\pgfsetfillcolor{currentfill}%
\pgfsetlinewidth{0.000000pt}%
\definecolor{currentstroke}{rgb}{0.000000,0.000000,0.000000}%
\pgfsetstrokecolor{currentstroke}%
\pgfsetdash{}{0pt}%
\pgfpathmoveto{\pgfqpoint{2.889241in}{2.514257in}}%
\pgfpathlineto{\pgfqpoint{2.899665in}{2.359507in}}%
\pgfpathlineto{\pgfqpoint{2.908314in}{2.341982in}}%
\pgfpathlineto{\pgfqpoint{2.943124in}{2.423881in}}%
\pgfpathlineto{\pgfqpoint{2.979303in}{2.396186in}}%
\pgfpathlineto{\pgfqpoint{2.971790in}{2.315487in}}%
\pgfpathlineto{\pgfqpoint{2.961357in}{2.474055in}}%
\pgfpathlineto{\pgfqpoint{2.926418in}{2.407470in}}%
\pgfpathlineto{\pgfqpoint{2.889241in}{2.514257in}}%
\pgfpathclose%
\pgfusepath{fill}%
\end{pgfscope}%
\begin{pgfscope}%
\pgfpathrectangle{\pgfqpoint{1.020000in}{0.880000in}}{\pgfqpoint{6.160000in}{6.160000in}}%
\pgfusepath{clip}%
\pgfsetbuttcap%
\pgfsetroundjoin%
\definecolor{currentfill}{rgb}{0.348323,0.465711,0.888346}%
\pgfsetfillcolor{currentfill}%
\pgfsetlinewidth{0.000000pt}%
\definecolor{currentstroke}{rgb}{0.000000,0.000000,0.000000}%
\pgfsetstrokecolor{currentstroke}%
\pgfsetdash{}{0pt}%
\pgfpathmoveto{\pgfqpoint{2.908314in}{2.341982in}}%
\pgfpathlineto{\pgfqpoint{2.918963in}{2.169519in}}%
\pgfpathlineto{\pgfqpoint{2.952990in}{2.312941in}}%
\pgfpathlineto{\pgfqpoint{2.988071in}{2.375842in}}%
\pgfpathlineto{\pgfqpoint{2.979303in}{2.396186in}}%
\pgfpathlineto{\pgfqpoint{2.943124in}{2.423881in}}%
\pgfpathlineto{\pgfqpoint{2.908314in}{2.341982in}}%
\pgfpathclose%
\pgfusepath{fill}%
\end{pgfscope}%
\begin{pgfscope}%
\pgfpathrectangle{\pgfqpoint{1.020000in}{0.880000in}}{\pgfqpoint{6.160000in}{6.160000in}}%
\pgfusepath{clip}%
\pgfsetbuttcap%
\pgfsetroundjoin%
\definecolor{currentfill}{rgb}{0.383662,0.510183,0.917831}%
\pgfsetfillcolor{currentfill}%
\pgfsetlinewidth{0.000000pt}%
\definecolor{currentstroke}{rgb}{0.000000,0.000000,0.000000}%
\pgfsetstrokecolor{currentstroke}%
\pgfsetdash{}{0pt}%
\pgfpathmoveto{\pgfqpoint{2.979303in}{2.396186in}}%
\pgfpathlineto{\pgfqpoint{2.988071in}{2.375842in}}%
\pgfpathlineto{\pgfqpoint{3.021118in}{2.614281in}}%
\pgfpathlineto{\pgfqpoint{3.059613in}{2.386011in}}%
\pgfpathlineto{\pgfqpoint{3.051785in}{2.318864in}}%
\pgfpathlineto{\pgfqpoint{3.015477in}{2.365394in}}%
\pgfpathlineto{\pgfqpoint{2.979303in}{2.396186in}}%
\pgfpathclose%
\pgfusepath{fill}%
\end{pgfscope}%
\begin{pgfscope}%
\definecolor{textcolor}{rgb}{0.000000,0.000000,0.000000}%
\pgfsetstrokecolor{textcolor}%
\pgfsetfillcolor{textcolor}%
\pgftext[x=4.100000in,y=7.123333in,,base]{\color{textcolor}\rmfamily\fontsize{16.000000}{19.200000}\selectfont Noisy Franke's Function}%
\end{pgfscope}%
\end{pgfpicture}%
\makeatother%
\endgroup%

%     \end{adjustbox}
%     \caption{\small Franke's Function with noise}
%     \label{frankfuncnoise}
%     \end{center}
% \end{wrapfigure}

% \begin{wrapfigure}{l}{0.3\textwidth}
%     \caption{\small Franke's Function}
%     \label{frankfunc}
%     \begin{center}
%     \begin{adjustbox}{clip,trim=4cm 3.4cm 3cm 4.8cm, max width=0.3\textwidth}
%     %% Creator: Matplotlib, PGF backend
%%
%% To include the figure in your LaTeX document, write
%%   \input{<filename>.pgf}
%%
%% Make sure the required packages are loaded in your preamble
%%   \usepackage{pgf}
%%
%% Also ensure that all the required font packages are loaded; for instance,
%% the lmodern package is sometimes necessary when using math font.
%%   \usepackage{lmodern}
%%
%% Figures using additional raster images can only be included by \input if
%% they are in the same directory as the main LaTeX file. For loading figures
%% from other directories you can use the `import` package
%%   \usepackage{import}
%%
%% and then include the figures with
%%   \import{<path to file>}{<filename>.pgf}
%%
%% Matplotlib used the following preamble
%%   
%%   \usepackage{fontspec}
%%   \setmainfont{DejaVuSerif.ttf}[Path=\detokenize{/home/brage/anaconda3/lib/python3.10/site-packages/matplotlib/mpl-data/fonts/ttf/}]
%%   \setsansfont{DejaVuSans.ttf}[Path=\detokenize{/home/brage/anaconda3/lib/python3.10/site-packages/matplotlib/mpl-data/fonts/ttf/}]
%%   \setmonofont{DejaVuSansMono.ttf}[Path=\detokenize{/home/brage/anaconda3/lib/python3.10/site-packages/matplotlib/mpl-data/fonts/ttf/}]
%%   \makeatletter\@ifpackageloaded{underscore}{}{\usepackage[strings]{underscore}}\makeatother
%%
\begingroup%
\makeatletter%
\begin{pgfpicture}%
\pgfpathrectangle{\pgfpoint{0pt}{0.1in}}{\pgfqpoint{8.000000in}{7.900000in}}%
\pgfusepath{use as bounding box, clip}%
\begin{pgfscope}%
\pgfsetbuttcap%
\pgfsetmiterjoin%
\definecolor{currentfill}{rgb}{1.000000,1.000000,1.000000}%
\pgfsetfillcolor{currentfill}%
\pgfsetlinewidth{0.000000pt}%
\definecolor{currentstroke}{rgb}{1.000000,1.000000,1.000000}%
\pgfsetstrokecolor{currentstroke}%
\pgfsetdash{}{0pt}%
\pgfpathmoveto{\pgfqpoint{0.000000in}{0.000000in}}%
\pgfpathlineto{\pgfqpoint{8.000000in}{0.000000in}}%
\pgfpathlineto{\pgfqpoint{8.000000in}{8.000000in}}%
\pgfpathlineto{\pgfqpoint{0.000000in}{8.000000in}}%
\pgfpathlineto{\pgfqpoint{0.000000in}{0.000000in}}%
\pgfpathclose%
\pgfusepath{fill}%
\end{pgfscope}%
\begin{pgfscope}%
\pgfsetbuttcap%
\pgfsetmiterjoin%
\definecolor{currentfill}{rgb}{1.000000,1.000000,1.000000}%
\pgfsetfillcolor{currentfill}%
\pgfsetlinewidth{0.000000pt}%
\definecolor{currentstroke}{rgb}{0.000000,0.000000,0.000000}%
\pgfsetstrokecolor{currentstroke}%
\pgfsetstrokeopacity{0.000000}%
\pgfsetdash{}{0pt}%
\pgfpathmoveto{\pgfqpoint{1.020000in}{0.880000in}}%
\pgfpathlineto{\pgfqpoint{7.180000in}{0.880000in}}%
\pgfpathlineto{\pgfqpoint{7.180000in}{7.040000in}}%
\pgfpathlineto{\pgfqpoint{1.020000in}{7.040000in}}%
\pgfpathlineto{\pgfqpoint{1.020000in}{0.880000in}}%
\pgfpathclose%
\pgfusepath{fill}%
\end{pgfscope}%
\begin{pgfscope}%
\pgfsetbuttcap%
\pgfsetmiterjoin%
\definecolor{currentfill}{rgb}{0.950000,0.950000,0.950000}%
\pgfsetfillcolor{currentfill}%
\pgfsetfillopacity{0.500000}%
\pgfsetlinewidth{1.003750pt}%
\definecolor{currentstroke}{rgb}{0.950000,0.950000,0.950000}%
\pgfsetstrokecolor{currentstroke}%
\pgfsetstrokeopacity{0.500000}%
\pgfsetdash{}{0pt}%
\pgfpathmoveto{\pgfqpoint{1.820028in}{3.056209in}}%
\pgfpathlineto{\pgfqpoint{5.383253in}{3.408151in}}%
\pgfpathlineto{\pgfqpoint{5.413101in}{6.077288in}}%
\pgfpathlineto{\pgfqpoint{1.759303in}{5.804651in}}%
\pgfusepath{stroke,fill}%
\end{pgfscope}%
\begin{pgfscope}%
\pgfsetbuttcap%
\pgfsetmiterjoin%
\definecolor{currentfill}{rgb}{0.900000,0.900000,0.900000}%
\pgfsetfillcolor{currentfill}%
\pgfsetfillopacity{0.500000}%
\pgfsetlinewidth{1.003750pt}%
\definecolor{currentstroke}{rgb}{0.900000,0.900000,0.900000}%
\pgfsetstrokecolor{currentstroke}%
\pgfsetstrokeopacity{0.500000}%
\pgfsetdash{}{0pt}%
\pgfpathmoveto{\pgfqpoint{5.383253in}{3.408151in}}%
\pgfpathlineto{\pgfqpoint{6.727341in}{2.194520in}}%
\pgfpathlineto{\pgfqpoint{6.797857in}{5.135277in}}%
\pgfpathlineto{\pgfqpoint{5.413101in}{6.077288in}}%
\pgfusepath{stroke,fill}%
\end{pgfscope}%
\begin{pgfscope}%
\pgfsetbuttcap%
\pgfsetmiterjoin%
\definecolor{currentfill}{rgb}{0.925000,0.925000,0.925000}%
\pgfsetfillcolor{currentfill}%
\pgfsetfillopacity{0.500000}%
\pgfsetlinewidth{1.003750pt}%
\definecolor{currentstroke}{rgb}{0.925000,0.925000,0.925000}%
\pgfsetstrokecolor{currentstroke}%
\pgfsetstrokeopacity{0.500000}%
\pgfsetdash{}{0pt}%
\pgfpathmoveto{\pgfqpoint{1.820028in}{3.056209in}}%
\pgfpathlineto{\pgfqpoint{2.801699in}{1.758325in}}%
\pgfpathlineto{\pgfqpoint{6.727341in}{2.194520in}}%
\pgfpathlineto{\pgfqpoint{5.383253in}{3.408151in}}%
\pgfusepath{stroke,fill}%
\end{pgfscope}%
\begin{pgfscope}%
\pgfsetrectcap%
\pgfsetroundjoin%
\pgfsetlinewidth{0.803000pt}%
\definecolor{currentstroke}{rgb}{0.000000,0.000000,0.000000}%
\pgfsetstrokecolor{currentstroke}%
\pgfsetdash{}{0pt}%
\pgfpathmoveto{\pgfqpoint{1.820028in}{3.056209in}}%
\pgfpathlineto{\pgfqpoint{2.801699in}{1.758325in}}%
\pgfusepath{stroke}%
\end{pgfscope}%
\begin{pgfscope}%
\pgfsetbuttcap%
\pgfsetroundjoin%
\pgfsetlinewidth{0.803000pt}%
\definecolor{currentstroke}{rgb}{0.690196,0.690196,0.690196}%
\pgfsetstrokecolor{currentstroke}%
\pgfsetdash{}{0pt}%
\pgfpathmoveto{\pgfqpoint{1.876411in}{2.981664in}}%
\pgfpathlineto{\pgfqpoint{5.460715in}{3.338208in}}%
\pgfpathlineto{\pgfqpoint{5.492698in}{6.023140in}}%
\pgfusepath{stroke}%
\end{pgfscope}%
\begin{pgfscope}%
\pgfsetbuttcap%
\pgfsetroundjoin%
\pgfsetlinewidth{0.803000pt}%
\definecolor{currentstroke}{rgb}{0.690196,0.690196,0.690196}%
\pgfsetstrokecolor{currentstroke}%
\pgfsetdash{}{0pt}%
\pgfpathmoveto{\pgfqpoint{2.034966in}{2.772035in}}%
\pgfpathlineto{\pgfqpoint{5.678372in}{3.141677in}}%
\pgfpathlineto{\pgfqpoint{5.716491in}{5.870900in}}%
\pgfusepath{stroke}%
\end{pgfscope}%
\begin{pgfscope}%
\pgfsetbuttcap%
\pgfsetroundjoin%
\pgfsetlinewidth{0.803000pt}%
\definecolor{currentstroke}{rgb}{0.690196,0.690196,0.690196}%
\pgfsetstrokecolor{currentstroke}%
\pgfsetdash{}{0pt}%
\pgfpathmoveto{\pgfqpoint{2.199498in}{2.554505in}}%
\pgfpathlineto{\pgfqpoint{5.903965in}{2.937979in}}%
\pgfpathlineto{\pgfqpoint{5.948657in}{5.712964in}}%
\pgfusepath{stroke}%
\end{pgfscope}%
\begin{pgfscope}%
\pgfsetbuttcap%
\pgfsetroundjoin%
\pgfsetlinewidth{0.803000pt}%
\definecolor{currentstroke}{rgb}{0.690196,0.690196,0.690196}%
\pgfsetstrokecolor{currentstroke}%
\pgfsetdash{}{0pt}%
\pgfpathmoveto{\pgfqpoint{2.370351in}{2.328618in}}%
\pgfpathlineto{\pgfqpoint{6.137938in}{2.726716in}}%
\pgfpathlineto{\pgfqpoint{6.189674in}{5.549007in}}%
\pgfusepath{stroke}%
\end{pgfscope}%
\begin{pgfscope}%
\pgfsetbuttcap%
\pgfsetroundjoin%
\pgfsetlinewidth{0.803000pt}%
\definecolor{currentstroke}{rgb}{0.690196,0.690196,0.690196}%
\pgfsetstrokecolor{currentstroke}%
\pgfsetdash{}{0pt}%
\pgfpathmoveto{\pgfqpoint{2.547896in}{2.093882in}}%
\pgfpathlineto{\pgfqpoint{6.380765in}{2.507458in}}%
\pgfpathlineto{\pgfqpoint{6.440059in}{5.378677in}}%
\pgfusepath{stroke}%
\end{pgfscope}%
\begin{pgfscope}%
\pgfsetbuttcap%
\pgfsetroundjoin%
\pgfsetlinewidth{0.803000pt}%
\definecolor{currentstroke}{rgb}{0.690196,0.690196,0.690196}%
\pgfsetstrokecolor{currentstroke}%
\pgfsetdash{}{0pt}%
\pgfpathmoveto{\pgfqpoint{2.732534in}{1.849769in}}%
\pgfpathlineto{\pgfqpoint{6.632959in}{2.279742in}}%
\pgfpathlineto{\pgfqpoint{6.700367in}{5.201597in}}%
\pgfusepath{stroke}%
\end{pgfscope}%
\begin{pgfscope}%
\pgfsetrectcap%
\pgfsetroundjoin%
\pgfsetlinewidth{0.803000pt}%
\definecolor{currentstroke}{rgb}{0.000000,0.000000,0.000000}%
\pgfsetstrokecolor{currentstroke}%
\pgfsetdash{}{0pt}%
\pgfpathmoveto{\pgfqpoint{1.906009in}{2.984608in}}%
\pgfpathlineto{\pgfqpoint{1.817169in}{2.975771in}}%
\pgfusepath{stroke}%
\end{pgfscope}%
\begin{pgfscope}%
\definecolor{textcolor}{rgb}{0.000000,0.000000,0.000000}%
\pgfsetstrokecolor{textcolor}%
\pgfsetfillcolor{textcolor}%
\pgftext[x=1.695006in,y=2.829093in,,top]{\color{textcolor}\rmfamily\fontsize{12.000000}{14.400000}\selectfont 0.0}%
\end{pgfscope}%
\begin{pgfscope}%
\pgfsetrectcap%
\pgfsetroundjoin%
\pgfsetlinewidth{0.803000pt}%
\definecolor{currentstroke}{rgb}{0.000000,0.000000,0.000000}%
\pgfsetstrokecolor{currentstroke}%
\pgfsetdash{}{0pt}%
\pgfpathmoveto{\pgfqpoint{2.065070in}{2.775089in}}%
\pgfpathlineto{\pgfqpoint{1.974711in}{2.765922in}}%
\pgfusepath{stroke}%
\end{pgfscope}%
\begin{pgfscope}%
\definecolor{textcolor}{rgb}{0.000000,0.000000,0.000000}%
\pgfsetstrokecolor{textcolor}%
\pgfsetfillcolor{textcolor}%
\pgftext[x=1.850263in,y=2.616564in,,top]{\color{textcolor}\rmfamily\fontsize{12.000000}{14.400000}\selectfont 0.2}%
\end{pgfscope}%
\begin{pgfscope}%
\pgfsetrectcap%
\pgfsetroundjoin%
\pgfsetlinewidth{0.803000pt}%
\definecolor{currentstroke}{rgb}{0.000000,0.000000,0.000000}%
\pgfsetstrokecolor{currentstroke}%
\pgfsetdash{}{0pt}%
\pgfpathmoveto{\pgfqpoint{2.230125in}{2.557675in}}%
\pgfpathlineto{\pgfqpoint{2.138196in}{2.548159in}}%
\pgfusepath{stroke}%
\end{pgfscope}%
\begin{pgfscope}%
\definecolor{textcolor}{rgb}{0.000000,0.000000,0.000000}%
\pgfsetstrokecolor{textcolor}%
\pgfsetfillcolor{textcolor}%
\pgftext[x=2.011374in,y=2.396021in,,top]{\color{textcolor}\rmfamily\fontsize{12.000000}{14.400000}\selectfont 0.4}%
\end{pgfscope}%
\begin{pgfscope}%
\pgfsetrectcap%
\pgfsetroundjoin%
\pgfsetlinewidth{0.803000pt}%
\definecolor{currentstroke}{rgb}{0.000000,0.000000,0.000000}%
\pgfsetstrokecolor{currentstroke}%
\pgfsetdash{}{0pt}%
\pgfpathmoveto{\pgfqpoint{2.401519in}{2.331911in}}%
\pgfpathlineto{\pgfqpoint{2.307964in}{2.322026in}}%
\pgfusepath{stroke}%
\end{pgfscope}%
\begin{pgfscope}%
\definecolor{textcolor}{rgb}{0.000000,0.000000,0.000000}%
\pgfsetstrokecolor{textcolor}%
\pgfsetfillcolor{textcolor}%
\pgftext[x=2.178677in,y=2.167001in,,top]{\color{textcolor}\rmfamily\fontsize{12.000000}{14.400000}\selectfont 0.6}%
\end{pgfscope}%
\begin{pgfscope}%
\pgfsetrectcap%
\pgfsetroundjoin%
\pgfsetlinewidth{0.803000pt}%
\definecolor{currentstroke}{rgb}{0.000000,0.000000,0.000000}%
\pgfsetstrokecolor{currentstroke}%
\pgfsetdash{}{0pt}%
\pgfpathmoveto{\pgfqpoint{2.579624in}{2.097306in}}%
\pgfpathlineto{\pgfqpoint{2.484386in}{2.087029in}}%
\pgfusepath{stroke}%
\end{pgfscope}%
\begin{pgfscope}%
\definecolor{textcolor}{rgb}{0.000000,0.000000,0.000000}%
\pgfsetstrokecolor{textcolor}%
\pgfsetfillcolor{textcolor}%
\pgftext[x=2.352536in,y=1.929008in,,top]{\color{textcolor}\rmfamily\fontsize{12.000000}{14.400000}\selectfont 0.8}%
\end{pgfscope}%
\begin{pgfscope}%
\pgfsetrectcap%
\pgfsetroundjoin%
\pgfsetlinewidth{0.803000pt}%
\definecolor{currentstroke}{rgb}{0.000000,0.000000,0.000000}%
\pgfsetstrokecolor{currentstroke}%
\pgfsetdash{}{0pt}%
\pgfpathmoveto{\pgfqpoint{2.764843in}{1.853330in}}%
\pgfpathlineto{\pgfqpoint{2.667861in}{1.842639in}}%
\pgfusepath{stroke}%
\end{pgfscope}%
\begin{pgfscope}%
\definecolor{textcolor}{rgb}{0.000000,0.000000,0.000000}%
\pgfsetstrokecolor{textcolor}%
\pgfsetfillcolor{textcolor}%
\pgftext[x=2.533343in,y=1.681502in,,top]{\color{textcolor}\rmfamily\fontsize{12.000000}{14.400000}\selectfont 1.0}%
\end{pgfscope}%
\begin{pgfscope}%
\pgfsetrectcap%
\pgfsetroundjoin%
\pgfsetlinewidth{0.803000pt}%
\definecolor{currentstroke}{rgb}{0.000000,0.000000,0.000000}%
\pgfsetstrokecolor{currentstroke}%
\pgfsetdash{}{0pt}%
\pgfpathmoveto{\pgfqpoint{6.727341in}{2.194520in}}%
\pgfpathlineto{\pgfqpoint{2.801699in}{1.758325in}}%
\pgfusepath{stroke}%
\end{pgfscope}%
\begin{pgfscope}%
\pgfsetbuttcap%
\pgfsetroundjoin%
\pgfsetlinewidth{0.803000pt}%
\definecolor{currentstroke}{rgb}{0.690196,0.690196,0.690196}%
\pgfsetstrokecolor{currentstroke}%
\pgfsetdash{}{0pt}%
\pgfpathmoveto{\pgfqpoint{1.999008in}{5.822537in}}%
\pgfpathlineto{\pgfqpoint{2.053616in}{3.079281in}}%
\pgfpathlineto{\pgfqpoint{3.059912in}{1.787016in}}%
\pgfusepath{stroke}%
\end{pgfscope}%
\begin{pgfscope}%
\pgfsetbuttcap%
\pgfsetroundjoin%
\pgfsetlinewidth{0.803000pt}%
\definecolor{currentstroke}{rgb}{0.690196,0.690196,0.690196}%
\pgfsetstrokecolor{currentstroke}%
\pgfsetdash{}{0pt}%
\pgfpathmoveto{\pgfqpoint{2.651344in}{5.871213in}}%
\pgfpathlineto{\pgfqpoint{2.689429in}{3.142081in}}%
\pgfpathlineto{\pgfqpoint{3.762134in}{1.865043in}}%
\pgfusepath{stroke}%
\end{pgfscope}%
\begin{pgfscope}%
\pgfsetbuttcap%
\pgfsetroundjoin%
\pgfsetlinewidth{0.803000pt}%
\definecolor{currentstroke}{rgb}{0.690196,0.690196,0.690196}%
\pgfsetstrokecolor{currentstroke}%
\pgfsetdash{}{0pt}%
\pgfpathmoveto{\pgfqpoint{3.296244in}{5.919334in}}%
\pgfpathlineto{\pgfqpoint{3.318173in}{3.204182in}}%
\pgfpathlineto{\pgfqpoint{4.455663in}{1.942104in}}%
\pgfusepath{stroke}%
\end{pgfscope}%
\begin{pgfscope}%
\pgfsetbuttcap%
\pgfsetroundjoin%
\pgfsetlinewidth{0.803000pt}%
\definecolor{currentstroke}{rgb}{0.690196,0.690196,0.690196}%
\pgfsetstrokecolor{currentstroke}%
\pgfsetdash{}{0pt}%
\pgfpathmoveto{\pgfqpoint{3.933833in}{5.966909in}}%
\pgfpathlineto{\pgfqpoint{3.939966in}{3.265597in}}%
\pgfpathlineto{\pgfqpoint{5.140659in}{2.018217in}}%
\pgfusepath{stroke}%
\end{pgfscope}%
\begin{pgfscope}%
\pgfsetbuttcap%
\pgfsetroundjoin%
\pgfsetlinewidth{0.803000pt}%
\definecolor{currentstroke}{rgb}{0.690196,0.690196,0.690196}%
\pgfsetstrokecolor{currentstroke}%
\pgfsetdash{}{0pt}%
\pgfpathmoveto{\pgfqpoint{4.564237in}{6.013948in}}%
\pgfpathlineto{\pgfqpoint{4.554922in}{3.326336in}}%
\pgfpathlineto{\pgfqpoint{5.817277in}{2.093399in}}%
\pgfusepath{stroke}%
\end{pgfscope}%
\begin{pgfscope}%
\pgfsetbuttcap%
\pgfsetroundjoin%
\pgfsetlinewidth{0.803000pt}%
\definecolor{currentstroke}{rgb}{0.690196,0.690196,0.690196}%
\pgfsetstrokecolor{currentstroke}%
\pgfsetdash{}{0pt}%
\pgfpathmoveto{\pgfqpoint{5.187576in}{6.060460in}}%
\pgfpathlineto{\pgfqpoint{5.163153in}{3.386411in}}%
\pgfpathlineto{\pgfqpoint{6.485672in}{2.167667in}}%
\pgfusepath{stroke}%
\end{pgfscope}%
\begin{pgfscope}%
\pgfsetrectcap%
\pgfsetroundjoin%
\pgfsetlinewidth{0.803000pt}%
\definecolor{currentstroke}{rgb}{0.000000,0.000000,0.000000}%
\pgfsetstrokecolor{currentstroke}%
\pgfsetdash{}{0pt}%
\pgfpathmoveto{\pgfqpoint{3.050944in}{1.798533in}}%
\pgfpathlineto{\pgfqpoint{3.077898in}{1.763919in}}%
\pgfusepath{stroke}%
\end{pgfscope}%
\begin{pgfscope}%
\definecolor{textcolor}{rgb}{0.000000,0.000000,0.000000}%
\pgfsetstrokecolor{textcolor}%
\pgfsetfillcolor{textcolor}%
\pgftext[x=3.117580in,y=1.567654in,,top]{\color{textcolor}\rmfamily\fontsize{12.000000}{14.400000}\selectfont 0.0}%
\end{pgfscope}%
\begin{pgfscope}%
\pgfsetrectcap%
\pgfsetroundjoin%
\pgfsetlinewidth{0.803000pt}%
\definecolor{currentstroke}{rgb}{0.000000,0.000000,0.000000}%
\pgfsetstrokecolor{currentstroke}%
\pgfsetdash{}{0pt}%
\pgfpathmoveto{\pgfqpoint{3.752580in}{1.876417in}}%
\pgfpathlineto{\pgfqpoint{3.781295in}{1.842233in}}%
\pgfusepath{stroke}%
\end{pgfscope}%
\begin{pgfscope}%
\definecolor{textcolor}{rgb}{0.000000,0.000000,0.000000}%
\pgfsetstrokecolor{textcolor}%
\pgfsetfillcolor{textcolor}%
\pgftext[x=3.822483in,y=1.647391in,,top]{\color{textcolor}\rmfamily\fontsize{12.000000}{14.400000}\selectfont 0.2}%
\end{pgfscope}%
\begin{pgfscope}%
\pgfsetrectcap%
\pgfsetroundjoin%
\pgfsetlinewidth{0.803000pt}%
\definecolor{currentstroke}{rgb}{0.000000,0.000000,0.000000}%
\pgfsetstrokecolor{currentstroke}%
\pgfsetdash{}{0pt}%
\pgfpathmoveto{\pgfqpoint{4.445539in}{1.953338in}}%
\pgfpathlineto{\pgfqpoint{4.475967in}{1.919576in}}%
\pgfusepath{stroke}%
\end{pgfscope}%
\begin{pgfscope}%
\definecolor{textcolor}{rgb}{0.000000,0.000000,0.000000}%
\pgfsetstrokecolor{textcolor}%
\pgfsetfillcolor{textcolor}%
\pgftext[x=4.518621in,y=1.726137in,,top]{\color{textcolor}\rmfamily\fontsize{12.000000}{14.400000}\selectfont 0.4}%
\end{pgfscope}%
\begin{pgfscope}%
\pgfsetrectcap%
\pgfsetroundjoin%
\pgfsetlinewidth{0.803000pt}%
\definecolor{currentstroke}{rgb}{0.000000,0.000000,0.000000}%
\pgfsetstrokecolor{currentstroke}%
\pgfsetdash{}{0pt}%
\pgfpathmoveto{\pgfqpoint{5.129978in}{2.029313in}}%
\pgfpathlineto{\pgfqpoint{5.162077in}{1.995965in}}%
\pgfusepath{stroke}%
\end{pgfscope}%
\begin{pgfscope}%
\definecolor{textcolor}{rgb}{0.000000,0.000000,0.000000}%
\pgfsetstrokecolor{textcolor}%
\pgfsetfillcolor{textcolor}%
\pgftext[x=5.206157in,y=1.803909in,,top]{\color{textcolor}\rmfamily\fontsize{12.000000}{14.400000}\selectfont 0.6}%
\end{pgfscope}%
\begin{pgfscope}%
\pgfsetrectcap%
\pgfsetroundjoin%
\pgfsetlinewidth{0.803000pt}%
\definecolor{currentstroke}{rgb}{0.000000,0.000000,0.000000}%
\pgfsetstrokecolor{currentstroke}%
\pgfsetdash{}{0pt}%
\pgfpathmoveto{\pgfqpoint{5.806055in}{2.104359in}}%
\pgfpathlineto{\pgfqpoint{5.839782in}{2.071419in}}%
\pgfusepath{stroke}%
\end{pgfscope}%
\begin{pgfscope}%
\definecolor{textcolor}{rgb}{0.000000,0.000000,0.000000}%
\pgfsetstrokecolor{textcolor}%
\pgfsetfillcolor{textcolor}%
\pgftext[x=5.885249in,y=1.880727in,,top]{\color{textcolor}\rmfamily\fontsize{12.000000}{14.400000}\selectfont 0.8}%
\end{pgfscope}%
\begin{pgfscope}%
\pgfsetrectcap%
\pgfsetroundjoin%
\pgfsetlinewidth{0.803000pt}%
\definecolor{currentstroke}{rgb}{0.000000,0.000000,0.000000}%
\pgfsetstrokecolor{currentstroke}%
\pgfsetdash{}{0pt}%
\pgfpathmoveto{\pgfqpoint{6.473923in}{2.178495in}}%
\pgfpathlineto{\pgfqpoint{6.509235in}{2.145954in}}%
\pgfusepath{stroke}%
\end{pgfscope}%
\begin{pgfscope}%
\definecolor{textcolor}{rgb}{0.000000,0.000000,0.000000}%
\pgfsetstrokecolor{textcolor}%
\pgfsetfillcolor{textcolor}%
\pgftext[x=6.556053in,y=1.956607in,,top]{\color{textcolor}\rmfamily\fontsize{12.000000}{14.400000}\selectfont 1.0}%
\end{pgfscope}%
\begin{pgfscope}%
\pgfsetrectcap%
\pgfsetroundjoin%
\pgfsetlinewidth{0.803000pt}%
\definecolor{currentstroke}{rgb}{0.000000,0.000000,0.000000}%
\pgfsetstrokecolor{currentstroke}%
\pgfsetdash{}{0pt}%
\pgfpathmoveto{\pgfqpoint{6.727341in}{2.194520in}}%
\pgfpathlineto{\pgfqpoint{6.797857in}{5.135277in}}%
\pgfusepath{stroke}%
\end{pgfscope}%
\begin{pgfscope}%
\pgfsetbuttcap%
\pgfsetroundjoin%
\pgfsetlinewidth{0.803000pt}%
\definecolor{currentstroke}{rgb}{0.690196,0.690196,0.690196}%
\pgfsetstrokecolor{currentstroke}%
\pgfsetdash{}{0pt}%
\pgfpathmoveto{\pgfqpoint{6.728714in}{2.251780in}}%
\pgfpathlineto{\pgfqpoint{5.383836in}{3.460263in}}%
\pgfpathlineto{\pgfqpoint{1.818843in}{3.109828in}}%
\pgfusepath{stroke}%
\end{pgfscope}%
\begin{pgfscope}%
\pgfsetbuttcap%
\pgfsetroundjoin%
\pgfsetlinewidth{0.803000pt}%
\definecolor{currentstroke}{rgb}{0.690196,0.690196,0.690196}%
\pgfsetstrokecolor{currentstroke}%
\pgfsetdash{}{0pt}%
\pgfpathmoveto{\pgfqpoint{6.736062in}{2.558212in}}%
\pgfpathlineto{\pgfqpoint{5.386953in}{3.739054in}}%
\pgfpathlineto{\pgfqpoint{1.812505in}{3.396704in}}%
\pgfusepath{stroke}%
\end{pgfscope}%
\begin{pgfscope}%
\pgfsetbuttcap%
\pgfsetroundjoin%
\pgfsetlinewidth{0.803000pt}%
\definecolor{currentstroke}{rgb}{0.690196,0.690196,0.690196}%
\pgfsetstrokecolor{currentstroke}%
\pgfsetdash{}{0pt}%
\pgfpathmoveto{\pgfqpoint{6.743453in}{2.866419in}}%
\pgfpathlineto{\pgfqpoint{5.390087in}{4.019295in}}%
\pgfpathlineto{\pgfqpoint{1.806132in}{3.685123in}}%
\pgfusepath{stroke}%
\end{pgfscope}%
\begin{pgfscope}%
\pgfsetbuttcap%
\pgfsetroundjoin%
\pgfsetlinewidth{0.803000pt}%
\definecolor{currentstroke}{rgb}{0.690196,0.690196,0.690196}%
\pgfsetstrokecolor{currentstroke}%
\pgfsetdash{}{0pt}%
\pgfpathmoveto{\pgfqpoint{6.750886in}{3.176415in}}%
\pgfpathlineto{\pgfqpoint{5.393237in}{4.301000in}}%
\pgfpathlineto{\pgfqpoint{1.799726in}{3.975096in}}%
\pgfusepath{stroke}%
\end{pgfscope}%
\begin{pgfscope}%
\pgfsetbuttcap%
\pgfsetroundjoin%
\pgfsetlinewidth{0.803000pt}%
\definecolor{currentstroke}{rgb}{0.690196,0.690196,0.690196}%
\pgfsetstrokecolor{currentstroke}%
\pgfsetdash{}{0pt}%
\pgfpathmoveto{\pgfqpoint{6.758363in}{3.488217in}}%
\pgfpathlineto{\pgfqpoint{5.396404in}{4.584180in}}%
\pgfpathlineto{\pgfqpoint{1.793284in}{4.266636in}}%
\pgfusepath{stroke}%
\end{pgfscope}%
\begin{pgfscope}%
\pgfsetbuttcap%
\pgfsetroundjoin%
\pgfsetlinewidth{0.803000pt}%
\definecolor{currentstroke}{rgb}{0.690196,0.690196,0.690196}%
\pgfsetstrokecolor{currentstroke}%
\pgfsetdash{}{0pt}%
\pgfpathmoveto{\pgfqpoint{6.765883in}{3.801840in}}%
\pgfpathlineto{\pgfqpoint{5.399587in}{4.868845in}}%
\pgfpathlineto{\pgfqpoint{1.786808in}{4.559757in}}%
\pgfusepath{stroke}%
\end{pgfscope}%
\begin{pgfscope}%
\pgfsetbuttcap%
\pgfsetroundjoin%
\pgfsetlinewidth{0.803000pt}%
\definecolor{currentstroke}{rgb}{0.690196,0.690196,0.690196}%
\pgfsetstrokecolor{currentstroke}%
\pgfsetdash{}{0pt}%
\pgfpathmoveto{\pgfqpoint{6.773447in}{4.117300in}}%
\pgfpathlineto{\pgfqpoint{5.402787in}{5.155009in}}%
\pgfpathlineto{\pgfqpoint{1.780296in}{4.854470in}}%
\pgfusepath{stroke}%
\end{pgfscope}%
\begin{pgfscope}%
\pgfsetbuttcap%
\pgfsetroundjoin%
\pgfsetlinewidth{0.803000pt}%
\definecolor{currentstroke}{rgb}{0.690196,0.690196,0.690196}%
\pgfsetstrokecolor{currentstroke}%
\pgfsetdash{}{0pt}%
\pgfpathmoveto{\pgfqpoint{6.781056in}{4.434613in}}%
\pgfpathlineto{\pgfqpoint{5.406004in}{5.442682in}}%
\pgfpathlineto{\pgfqpoint{1.773749in}{5.150790in}}%
\pgfusepath{stroke}%
\end{pgfscope}%
\begin{pgfscope}%
\pgfsetbuttcap%
\pgfsetroundjoin%
\pgfsetlinewidth{0.803000pt}%
\definecolor{currentstroke}{rgb}{0.690196,0.690196,0.690196}%
\pgfsetstrokecolor{currentstroke}%
\pgfsetdash{}{0pt}%
\pgfpathmoveto{\pgfqpoint{6.788709in}{4.753796in}}%
\pgfpathlineto{\pgfqpoint{5.409238in}{5.731876in}}%
\pgfpathlineto{\pgfqpoint{1.767167in}{5.448728in}}%
\pgfusepath{stroke}%
\end{pgfscope}%
\begin{pgfscope}%
\pgfsetbuttcap%
\pgfsetroundjoin%
\pgfsetlinewidth{0.803000pt}%
\definecolor{currentstroke}{rgb}{0.690196,0.690196,0.690196}%
\pgfsetstrokecolor{currentstroke}%
\pgfsetdash{}{0pt}%
\pgfpathmoveto{\pgfqpoint{6.796408in}{5.074865in}}%
\pgfpathlineto{\pgfqpoint{5.412489in}{6.022605in}}%
\pgfpathlineto{\pgfqpoint{1.760548in}{5.748299in}}%
\pgfusepath{stroke}%
\end{pgfscope}%
\begin{pgfscope}%
\pgfsetrectcap%
\pgfsetroundjoin%
\pgfsetlinewidth{0.803000pt}%
\definecolor{currentstroke}{rgb}{0.000000,0.000000,0.000000}%
\pgfsetstrokecolor{currentstroke}%
\pgfsetdash{}{0pt}%
\pgfpathmoveto{\pgfqpoint{6.716768in}{2.262515in}}%
\pgfpathlineto{\pgfqpoint{6.752671in}{2.230253in}}%
\pgfusepath{stroke}%
\end{pgfscope}%
\begin{pgfscope}%
\definecolor{textcolor}{rgb}{0.000000,0.000000,0.000000}%
\pgfsetstrokecolor{textcolor}%
\pgfsetfillcolor{textcolor}%
\pgftext[x=6.992871in,y=2.205616in,,top]{\color{textcolor}\rmfamily\fontsize{12.000000}{14.400000}\selectfont -0.10}%
\end{pgfscope}%
\begin{pgfscope}%
\pgfsetrectcap%
\pgfsetroundjoin%
\pgfsetlinewidth{0.803000pt}%
\definecolor{currentstroke}{rgb}{0.000000,0.000000,0.000000}%
\pgfsetstrokecolor{currentstroke}%
\pgfsetdash{}{0pt}%
\pgfpathmoveto{\pgfqpoint{6.724075in}{2.568705in}}%
\pgfpathlineto{\pgfqpoint{6.760101in}{2.537172in}}%
\pgfusepath{stroke}%
\end{pgfscope}%
\begin{pgfscope}%
\definecolor{textcolor}{rgb}{0.000000,0.000000,0.000000}%
\pgfsetstrokecolor{textcolor}%
\pgfsetfillcolor{textcolor}%
\pgftext[x=7.001013in,y=2.513091in,,top]{\color{textcolor}\rmfamily\fontsize{12.000000}{14.400000}\selectfont 0.07}%
\end{pgfscope}%
\begin{pgfscope}%
\pgfsetrectcap%
\pgfsetroundjoin%
\pgfsetlinewidth{0.803000pt}%
\definecolor{currentstroke}{rgb}{0.000000,0.000000,0.000000}%
\pgfsetstrokecolor{currentstroke}%
\pgfsetdash{}{0pt}%
\pgfpathmoveto{\pgfqpoint{6.731424in}{2.876666in}}%
\pgfpathlineto{\pgfqpoint{6.767575in}{2.845870in}}%
\pgfusepath{stroke}%
\end{pgfscope}%
\begin{pgfscope}%
\definecolor{textcolor}{rgb}{0.000000,0.000000,0.000000}%
\pgfsetstrokecolor{textcolor}%
\pgfsetfillcolor{textcolor}%
\pgftext[x=7.009202in,y=2.822353in,,top]{\color{textcolor}\rmfamily\fontsize{12.000000}{14.400000}\selectfont 0.23}%
\end{pgfscope}%
\begin{pgfscope}%
\pgfsetrectcap%
\pgfsetroundjoin%
\pgfsetlinewidth{0.803000pt}%
\definecolor{currentstroke}{rgb}{0.000000,0.000000,0.000000}%
\pgfsetstrokecolor{currentstroke}%
\pgfsetdash{}{0pt}%
\pgfpathmoveto{\pgfqpoint{6.738816in}{3.186414in}}%
\pgfpathlineto{\pgfqpoint{6.775091in}{3.156365in}}%
\pgfusepath{stroke}%
\end{pgfscope}%
\begin{pgfscope}%
\definecolor{textcolor}{rgb}{0.000000,0.000000,0.000000}%
\pgfsetstrokecolor{textcolor}%
\pgfsetfillcolor{textcolor}%
\pgftext[x=7.017438in,y=3.133417in,,top]{\color{textcolor}\rmfamily\fontsize{12.000000}{14.400000}\selectfont 0.40}%
\end{pgfscope}%
\begin{pgfscope}%
\pgfsetrectcap%
\pgfsetroundjoin%
\pgfsetlinewidth{0.803000pt}%
\definecolor{currentstroke}{rgb}{0.000000,0.000000,0.000000}%
\pgfsetstrokecolor{currentstroke}%
\pgfsetdash{}{0pt}%
\pgfpathmoveto{\pgfqpoint{6.746251in}{3.497964in}}%
\pgfpathlineto{\pgfqpoint{6.782652in}{3.468671in}}%
\pgfusepath{stroke}%
\end{pgfscope}%
\begin{pgfscope}%
\definecolor{textcolor}{rgb}{0.000000,0.000000,0.000000}%
\pgfsetstrokecolor{textcolor}%
\pgfsetfillcolor{textcolor}%
\pgftext[x=7.025723in,y=3.446301in,,top]{\color{textcolor}\rmfamily\fontsize{12.000000}{14.400000}\selectfont 0.57}%
\end{pgfscope}%
\begin{pgfscope}%
\pgfsetrectcap%
\pgfsetroundjoin%
\pgfsetlinewidth{0.803000pt}%
\definecolor{currentstroke}{rgb}{0.000000,0.000000,0.000000}%
\pgfsetstrokecolor{currentstroke}%
\pgfsetdash{}{0pt}%
\pgfpathmoveto{\pgfqpoint{6.753729in}{3.811332in}}%
\pgfpathlineto{\pgfqpoint{6.790257in}{3.782805in}}%
\pgfusepath{stroke}%
\end{pgfscope}%
\begin{pgfscope}%
\definecolor{textcolor}{rgb}{0.000000,0.000000,0.000000}%
\pgfsetstrokecolor{textcolor}%
\pgfsetfillcolor{textcolor}%
\pgftext[x=7.034056in,y=3.761018in,,top]{\color{textcolor}\rmfamily\fontsize{12.000000}{14.400000}\selectfont 0.73}%
\end{pgfscope}%
\begin{pgfscope}%
\pgfsetrectcap%
\pgfsetroundjoin%
\pgfsetlinewidth{0.803000pt}%
\definecolor{currentstroke}{rgb}{0.000000,0.000000,0.000000}%
\pgfsetstrokecolor{currentstroke}%
\pgfsetdash{}{0pt}%
\pgfpathmoveto{\pgfqpoint{6.761251in}{4.126534in}}%
\pgfpathlineto{\pgfqpoint{6.797907in}{4.098782in}}%
\pgfusepath{stroke}%
\end{pgfscope}%
\begin{pgfscope}%
\definecolor{textcolor}{rgb}{0.000000,0.000000,0.000000}%
\pgfsetstrokecolor{textcolor}%
\pgfsetfillcolor{textcolor}%
\pgftext[x=7.042439in,y=4.077587in,,top]{\color{textcolor}\rmfamily\fontsize{12.000000}{14.400000}\selectfont 0.90}%
\end{pgfscope}%
\begin{pgfscope}%
\pgfsetrectcap%
\pgfsetroundjoin%
\pgfsetlinewidth{0.803000pt}%
\definecolor{currentstroke}{rgb}{0.000000,0.000000,0.000000}%
\pgfsetstrokecolor{currentstroke}%
\pgfsetdash{}{0pt}%
\pgfpathmoveto{\pgfqpoint{6.768817in}{4.443586in}}%
\pgfpathlineto{\pgfqpoint{6.805601in}{4.416619in}}%
\pgfusepath{stroke}%
\end{pgfscope}%
\begin{pgfscope}%
\definecolor{textcolor}{rgb}{0.000000,0.000000,0.000000}%
\pgfsetstrokecolor{textcolor}%
\pgfsetfillcolor{textcolor}%
\pgftext[x=7.050870in,y=4.396023in,,top]{\color{textcolor}\rmfamily\fontsize{12.000000}{14.400000}\selectfont 1.07}%
\end{pgfscope}%
\begin{pgfscope}%
\pgfsetrectcap%
\pgfsetroundjoin%
\pgfsetlinewidth{0.803000pt}%
\definecolor{currentstroke}{rgb}{0.000000,0.000000,0.000000}%
\pgfsetstrokecolor{currentstroke}%
\pgfsetdash{}{0pt}%
\pgfpathmoveto{\pgfqpoint{6.776427in}{4.762505in}}%
\pgfpathlineto{\pgfqpoint{6.813341in}{4.736332in}}%
\pgfusepath{stroke}%
\end{pgfscope}%
\begin{pgfscope}%
\definecolor{textcolor}{rgb}{0.000000,0.000000,0.000000}%
\pgfsetstrokecolor{textcolor}%
\pgfsetfillcolor{textcolor}%
\pgftext[x=7.059352in,y=4.716342in,,top]{\color{textcolor}\rmfamily\fontsize{12.000000}{14.400000}\selectfont 1.23}%
\end{pgfscope}%
\begin{pgfscope}%
\pgfsetrectcap%
\pgfsetroundjoin%
\pgfsetlinewidth{0.803000pt}%
\definecolor{currentstroke}{rgb}{0.000000,0.000000,0.000000}%
\pgfsetstrokecolor{currentstroke}%
\pgfsetdash{}{0pt}%
\pgfpathmoveto{\pgfqpoint{6.784083in}{5.083306in}}%
\pgfpathlineto{\pgfqpoint{6.821127in}{5.057937in}}%
\pgfusepath{stroke}%
\end{pgfscope}%
\begin{pgfscope}%
\definecolor{textcolor}{rgb}{0.000000,0.000000,0.000000}%
\pgfsetstrokecolor{textcolor}%
\pgfsetfillcolor{textcolor}%
\pgftext[x=7.067884in,y=5.038562in,,top]{\color{textcolor}\rmfamily\fontsize{12.000000}{14.400000}\selectfont 1.40}%
\end{pgfscope}%
\begin{pgfscope}%
\pgfpathrectangle{\pgfqpoint{1.020000in}{0.880000in}}{\pgfqpoint{6.160000in}{6.160000in}}%
\pgfusepath{clip}%
\pgfsetbuttcap%
\pgfsetroundjoin%
\definecolor{currentfill}{rgb}{0.510824,0.649397,0.985079}%
\pgfsetfillcolor{currentfill}%
\pgfsetlinewidth{0.000000pt}%
\definecolor{currentstroke}{rgb}{0.000000,0.000000,0.000000}%
\pgfsetstrokecolor{currentstroke}%
\pgfsetdash{}{0pt}%
\pgfpathmoveto{\pgfqpoint{5.215154in}{3.996199in}}%
\pgfpathlineto{\pgfqpoint{5.225819in}{3.985146in}}%
\pgfpathlineto{\pgfqpoint{5.236501in}{3.973926in}}%
\pgfpathlineto{\pgfqpoint{5.267436in}{3.972650in}}%
\pgfpathlineto{\pgfqpoint{5.256730in}{3.983851in}}%
\pgfpathlineto{\pgfqpoint{5.246041in}{3.994885in}}%
\pgfpathlineto{\pgfqpoint{5.215154in}{3.996199in}}%
\pgfpathclose%
\pgfusepath{fill}%
\end{pgfscope}%
\begin{pgfscope}%
\pgfpathrectangle{\pgfqpoint{1.020000in}{0.880000in}}{\pgfqpoint{6.160000in}{6.160000in}}%
\pgfusepath{clip}%
\pgfsetbuttcap%
\pgfsetroundjoin%
\definecolor{currentfill}{rgb}{0.510824,0.649397,0.985079}%
\pgfsetfillcolor{currentfill}%
\pgfsetlinewidth{0.000000pt}%
\definecolor{currentstroke}{rgb}{0.000000,0.000000,0.000000}%
\pgfsetstrokecolor{currentstroke}%
\pgfsetdash{}{0pt}%
\pgfpathmoveto{\pgfqpoint{5.153322in}{3.998946in}}%
\pgfpathlineto{\pgfqpoint{5.163939in}{3.987856in}}%
\pgfpathlineto{\pgfqpoint{5.174573in}{3.976595in}}%
\pgfpathlineto{\pgfqpoint{5.205547in}{3.975241in}}%
\pgfpathlineto{\pgfqpoint{5.236501in}{3.973926in}}%
\pgfpathlineto{\pgfqpoint{5.225819in}{3.985146in}}%
\pgfpathlineto{\pgfqpoint{5.215154in}{3.996199in}}%
\pgfpathlineto{\pgfqpoint{5.184247in}{3.997553in}}%
\pgfpathlineto{\pgfqpoint{5.153322in}{3.998946in}}%
\pgfpathclose%
\pgfusepath{fill}%
\end{pgfscope}%
\begin{pgfscope}%
\pgfpathrectangle{\pgfqpoint{1.020000in}{0.880000in}}{\pgfqpoint{6.160000in}{6.160000in}}%
\pgfusepath{clip}%
\pgfsetbuttcap%
\pgfsetroundjoin%
\definecolor{currentfill}{rgb}{0.521696,0.659599,0.987736}%
\pgfsetfillcolor{currentfill}%
\pgfsetlinewidth{0.000000pt}%
\definecolor{currentstroke}{rgb}{0.000000,0.000000,0.000000}%
\pgfsetstrokecolor{currentstroke}%
\pgfsetdash{}{0pt}%
\pgfpathmoveto{\pgfqpoint{5.091412in}{4.001855in}}%
\pgfpathlineto{\pgfqpoint{5.101981in}{3.990728in}}%
\pgfpathlineto{\pgfqpoint{5.112567in}{3.979426in}}%
\pgfpathlineto{\pgfqpoint{5.143580in}{3.977990in}}%
\pgfpathlineto{\pgfqpoint{5.174573in}{3.976595in}}%
\pgfpathlineto{\pgfqpoint{5.163939in}{3.987856in}}%
\pgfpathlineto{\pgfqpoint{5.153322in}{3.998946in}}%
\pgfpathlineto{\pgfqpoint{5.122376in}{4.000380in}}%
\pgfpathlineto{\pgfqpoint{5.091412in}{4.001855in}}%
\pgfpathclose%
\pgfusepath{fill}%
\end{pgfscope}%
\begin{pgfscope}%
\pgfpathrectangle{\pgfqpoint{1.020000in}{0.880000in}}{\pgfqpoint{6.160000in}{6.160000in}}%
\pgfusepath{clip}%
\pgfsetbuttcap%
\pgfsetroundjoin%
\definecolor{currentfill}{rgb}{0.505423,0.643995,0.983157}%
\pgfsetfillcolor{currentfill}%
\pgfsetlinewidth{0.000000pt}%
\definecolor{currentstroke}{rgb}{0.000000,0.000000,0.000000}%
\pgfsetstrokecolor{currentstroke}%
\pgfsetdash{}{0pt}%
\pgfpathmoveto{\pgfqpoint{5.236501in}{3.973926in}}%
\pgfpathlineto{\pgfqpoint{5.247201in}{3.962539in}}%
\pgfpathlineto{\pgfqpoint{5.257919in}{3.950987in}}%
\pgfpathlineto{\pgfqpoint{5.288901in}{3.949756in}}%
\pgfpathlineto{\pgfqpoint{5.278160in}{3.961285in}}%
\pgfpathlineto{\pgfqpoint{5.267436in}{3.972650in}}%
\pgfpathlineto{\pgfqpoint{5.236501in}{3.973926in}}%
\pgfpathclose%
\pgfusepath{fill}%
\end{pgfscope}%
\begin{pgfscope}%
\pgfpathrectangle{\pgfqpoint{1.020000in}{0.880000in}}{\pgfqpoint{6.160000in}{6.160000in}}%
\pgfusepath{clip}%
\pgfsetbuttcap%
\pgfsetroundjoin%
\definecolor{currentfill}{rgb}{0.527132,0.664700,0.989065}%
\pgfsetfillcolor{currentfill}%
\pgfsetlinewidth{0.000000pt}%
\definecolor{currentstroke}{rgb}{0.000000,0.000000,0.000000}%
\pgfsetstrokecolor{currentstroke}%
\pgfsetdash{}{0pt}%
\pgfpathmoveto{\pgfqpoint{5.029424in}{4.004931in}}%
\pgfpathlineto{\pgfqpoint{5.039945in}{3.993766in}}%
\pgfpathlineto{\pgfqpoint{5.050483in}{3.982423in}}%
\pgfpathlineto{\pgfqpoint{5.081535in}{3.980903in}}%
\pgfpathlineto{\pgfqpoint{5.112567in}{3.979426in}}%
\pgfpathlineto{\pgfqpoint{5.101981in}{3.990728in}}%
\pgfpathlineto{\pgfqpoint{5.091412in}{4.001855in}}%
\pgfpathlineto{\pgfqpoint{5.060427in}{4.003372in}}%
\pgfpathlineto{\pgfqpoint{5.029424in}{4.004931in}}%
\pgfpathclose%
\pgfusepath{fill}%
\end{pgfscope}%
\begin{pgfscope}%
\pgfpathrectangle{\pgfqpoint{1.020000in}{0.880000in}}{\pgfqpoint{6.160000in}{6.160000in}}%
\pgfusepath{clip}%
\pgfsetbuttcap%
\pgfsetroundjoin%
\definecolor{currentfill}{rgb}{0.510824,0.649397,0.985079}%
\pgfsetfillcolor{currentfill}%
\pgfsetlinewidth{0.000000pt}%
\definecolor{currentstroke}{rgb}{0.000000,0.000000,0.000000}%
\pgfsetstrokecolor{currentstroke}%
\pgfsetdash{}{0pt}%
\pgfpathmoveto{\pgfqpoint{5.174573in}{3.976595in}}%
\pgfpathlineto{\pgfqpoint{5.185225in}{3.965165in}}%
\pgfpathlineto{\pgfqpoint{5.195895in}{3.953567in}}%
\pgfpathlineto{\pgfqpoint{5.226916in}{3.952257in}}%
\pgfpathlineto{\pgfqpoint{5.257919in}{3.950987in}}%
\pgfpathlineto{\pgfqpoint{5.247201in}{3.962539in}}%
\pgfpathlineto{\pgfqpoint{5.236501in}{3.973926in}}%
\pgfpathlineto{\pgfqpoint{5.205547in}{3.975241in}}%
\pgfpathlineto{\pgfqpoint{5.174573in}{3.976595in}}%
\pgfpathclose%
\pgfusepath{fill}%
\end{pgfscope}%
\begin{pgfscope}%
\pgfpathrectangle{\pgfqpoint{1.020000in}{0.880000in}}{\pgfqpoint{6.160000in}{6.160000in}}%
\pgfusepath{clip}%
\pgfsetbuttcap%
\pgfsetroundjoin%
\definecolor{currentfill}{rgb}{0.532568,0.669801,0.990393}%
\pgfsetfillcolor{currentfill}%
\pgfsetlinewidth{0.000000pt}%
\definecolor{currentstroke}{rgb}{0.000000,0.000000,0.000000}%
\pgfsetstrokecolor{currentstroke}%
\pgfsetdash{}{0pt}%
\pgfpathmoveto{\pgfqpoint{4.967357in}{4.008180in}}%
\pgfpathlineto{\pgfqpoint{4.977830in}{3.996976in}}%
\pgfpathlineto{\pgfqpoint{4.988320in}{3.985593in}}%
\pgfpathlineto{\pgfqpoint{5.019412in}{3.983986in}}%
\pgfpathlineto{\pgfqpoint{5.050483in}{3.982423in}}%
\pgfpathlineto{\pgfqpoint{5.039945in}{3.993766in}}%
\pgfpathlineto{\pgfqpoint{5.029424in}{4.004931in}}%
\pgfpathlineto{\pgfqpoint{4.998400in}{4.006534in}}%
\pgfpathlineto{\pgfqpoint{4.967357in}{4.008180in}}%
\pgfpathclose%
\pgfusepath{fill}%
\end{pgfscope}%
\begin{pgfscope}%
\pgfpathrectangle{\pgfqpoint{1.020000in}{0.880000in}}{\pgfqpoint{6.160000in}{6.160000in}}%
\pgfusepath{clip}%
\pgfsetbuttcap%
\pgfsetroundjoin%
\definecolor{currentfill}{rgb}{0.516260,0.654498,0.986407}%
\pgfsetfillcolor{currentfill}%
\pgfsetlinewidth{0.000000pt}%
\definecolor{currentstroke}{rgb}{0.000000,0.000000,0.000000}%
\pgfsetstrokecolor{currentstroke}%
\pgfsetdash{}{0pt}%
\pgfpathmoveto{\pgfqpoint{5.112567in}{3.979426in}}%
\pgfpathlineto{\pgfqpoint{5.123171in}{3.967952in}}%
\pgfpathlineto{\pgfqpoint{5.133792in}{3.956308in}}%
\pgfpathlineto{\pgfqpoint{5.164853in}{3.954917in}}%
\pgfpathlineto{\pgfqpoint{5.195895in}{3.953567in}}%
\pgfpathlineto{\pgfqpoint{5.185225in}{3.965165in}}%
\pgfpathlineto{\pgfqpoint{5.174573in}{3.976595in}}%
\pgfpathlineto{\pgfqpoint{5.143580in}{3.977990in}}%
\pgfpathlineto{\pgfqpoint{5.112567in}{3.979426in}}%
\pgfpathclose%
\pgfusepath{fill}%
\end{pgfscope}%
\begin{pgfscope}%
\pgfpathrectangle{\pgfqpoint{1.020000in}{0.880000in}}{\pgfqpoint{6.160000in}{6.160000in}}%
\pgfusepath{clip}%
\pgfsetbuttcap%
\pgfsetroundjoin%
\definecolor{currentfill}{rgb}{0.538004,0.674902,0.991722}%
\pgfsetfillcolor{currentfill}%
\pgfsetlinewidth{0.000000pt}%
\definecolor{currentstroke}{rgb}{0.000000,0.000000,0.000000}%
\pgfsetstrokecolor{currentstroke}%
\pgfsetdash{}{0pt}%
\pgfpathmoveto{\pgfqpoint{4.905212in}{4.011608in}}%
\pgfpathlineto{\pgfqpoint{4.915636in}{4.000367in}}%
\pgfpathlineto{\pgfqpoint{4.926078in}{3.988943in}}%
\pgfpathlineto{\pgfqpoint{4.957209in}{3.987245in}}%
\pgfpathlineto{\pgfqpoint{4.988320in}{3.985593in}}%
\pgfpathlineto{\pgfqpoint{4.977830in}{3.996976in}}%
\pgfpathlineto{\pgfqpoint{4.967357in}{4.008180in}}%
\pgfpathlineto{\pgfqpoint{4.936294in}{4.009871in}}%
\pgfpathlineto{\pgfqpoint{4.905212in}{4.011608in}}%
\pgfpathclose%
\pgfusepath{fill}%
\end{pgfscope}%
\begin{pgfscope}%
\pgfpathrectangle{\pgfqpoint{1.020000in}{0.880000in}}{\pgfqpoint{6.160000in}{6.160000in}}%
\pgfusepath{clip}%
\pgfsetbuttcap%
\pgfsetroundjoin%
\definecolor{currentfill}{rgb}{0.505423,0.643995,0.983157}%
\pgfsetfillcolor{currentfill}%
\pgfsetlinewidth{0.000000pt}%
\definecolor{currentstroke}{rgb}{0.000000,0.000000,0.000000}%
\pgfsetstrokecolor{currentstroke}%
\pgfsetdash{}{0pt}%
\pgfpathmoveto{\pgfqpoint{5.257919in}{3.950987in}}%
\pgfpathlineto{\pgfqpoint{5.268653in}{3.939272in}}%
\pgfpathlineto{\pgfqpoint{5.279405in}{3.927396in}}%
\pgfpathlineto{\pgfqpoint{5.310436in}{3.926214in}}%
\pgfpathlineto{\pgfqpoint{5.299660in}{3.938065in}}%
\pgfpathlineto{\pgfqpoint{5.288901in}{3.949756in}}%
\pgfpathlineto{\pgfqpoint{5.257919in}{3.950987in}}%
\pgfpathclose%
\pgfusepath{fill}%
\end{pgfscope}%
\begin{pgfscope}%
\pgfpathrectangle{\pgfqpoint{1.020000in}{0.880000in}}{\pgfqpoint{6.160000in}{6.160000in}}%
\pgfusepath{clip}%
\pgfsetbuttcap%
\pgfsetroundjoin%
\definecolor{currentfill}{rgb}{0.521696,0.659599,0.987736}%
\pgfsetfillcolor{currentfill}%
\pgfsetlinewidth{0.000000pt}%
\definecolor{currentstroke}{rgb}{0.000000,0.000000,0.000000}%
\pgfsetstrokecolor{currentstroke}%
\pgfsetdash{}{0pt}%
\pgfpathmoveto{\pgfqpoint{5.050483in}{3.982423in}}%
\pgfpathlineto{\pgfqpoint{5.061039in}{3.970905in}}%
\pgfpathlineto{\pgfqpoint{5.071612in}{3.959214in}}%
\pgfpathlineto{\pgfqpoint{5.102712in}{3.957740in}}%
\pgfpathlineto{\pgfqpoint{5.133792in}{3.956308in}}%
\pgfpathlineto{\pgfqpoint{5.123171in}{3.967952in}}%
\pgfpathlineto{\pgfqpoint{5.112567in}{3.979426in}}%
\pgfpathlineto{\pgfqpoint{5.081535in}{3.980903in}}%
\pgfpathlineto{\pgfqpoint{5.050483in}{3.982423in}}%
\pgfpathclose%
\pgfusepath{fill}%
\end{pgfscope}%
\begin{pgfscope}%
\pgfpathrectangle{\pgfqpoint{1.020000in}{0.880000in}}{\pgfqpoint{6.160000in}{6.160000in}}%
\pgfusepath{clip}%
\pgfsetbuttcap%
\pgfsetroundjoin%
\definecolor{currentfill}{rgb}{0.543440,0.680003,0.993051}%
\pgfsetfillcolor{currentfill}%
\pgfsetlinewidth{0.000000pt}%
\definecolor{currentstroke}{rgb}{0.000000,0.000000,0.000000}%
\pgfsetstrokecolor{currentstroke}%
\pgfsetdash{}{0pt}%
\pgfpathmoveto{\pgfqpoint{4.842987in}{4.015227in}}%
\pgfpathlineto{\pgfqpoint{4.853363in}{4.003949in}}%
\pgfpathlineto{\pgfqpoint{4.863757in}{3.992485in}}%
\pgfpathlineto{\pgfqpoint{4.894928in}{3.990689in}}%
\pgfpathlineto{\pgfqpoint{4.926078in}{3.988943in}}%
\pgfpathlineto{\pgfqpoint{4.915636in}{4.000367in}}%
\pgfpathlineto{\pgfqpoint{4.905212in}{4.011608in}}%
\pgfpathlineto{\pgfqpoint{4.874109in}{4.013393in}}%
\pgfpathlineto{\pgfqpoint{4.842987in}{4.015227in}}%
\pgfpathclose%
\pgfusepath{fill}%
\end{pgfscope}%
\begin{pgfscope}%
\pgfpathrectangle{\pgfqpoint{1.020000in}{0.880000in}}{\pgfqpoint{6.160000in}{6.160000in}}%
\pgfusepath{clip}%
\pgfsetbuttcap%
\pgfsetroundjoin%
\definecolor{currentfill}{rgb}{0.505423,0.643995,0.983157}%
\pgfsetfillcolor{currentfill}%
\pgfsetlinewidth{0.000000pt}%
\definecolor{currentstroke}{rgb}{0.000000,0.000000,0.000000}%
\pgfsetstrokecolor{currentstroke}%
\pgfsetdash{}{0pt}%
\pgfpathmoveto{\pgfqpoint{5.195895in}{3.953567in}}%
\pgfpathlineto{\pgfqpoint{5.206581in}{3.941803in}}%
\pgfpathlineto{\pgfqpoint{5.217285in}{3.929876in}}%
\pgfpathlineto{\pgfqpoint{5.248355in}{3.928616in}}%
\pgfpathlineto{\pgfqpoint{5.279405in}{3.927396in}}%
\pgfpathlineto{\pgfqpoint{5.268653in}{3.939272in}}%
\pgfpathlineto{\pgfqpoint{5.257919in}{3.950987in}}%
\pgfpathlineto{\pgfqpoint{5.226916in}{3.952257in}}%
\pgfpathlineto{\pgfqpoint{5.195895in}{3.953567in}}%
\pgfpathclose%
\pgfusepath{fill}%
\end{pgfscope}%
\begin{pgfscope}%
\pgfpathrectangle{\pgfqpoint{1.020000in}{0.880000in}}{\pgfqpoint{6.160000in}{6.160000in}}%
\pgfusepath{clip}%
\pgfsetbuttcap%
\pgfsetroundjoin%
\definecolor{currentfill}{rgb}{0.527132,0.664700,0.989065}%
\pgfsetfillcolor{currentfill}%
\pgfsetlinewidth{0.000000pt}%
\definecolor{currentstroke}{rgb}{0.000000,0.000000,0.000000}%
\pgfsetstrokecolor{currentstroke}%
\pgfsetdash{}{0pt}%
\pgfpathmoveto{\pgfqpoint{4.988320in}{3.985593in}}%
\pgfpathlineto{\pgfqpoint{4.998828in}{3.974031in}}%
\pgfpathlineto{\pgfqpoint{5.009353in}{3.962293in}}%
\pgfpathlineto{\pgfqpoint{5.040492in}{3.960731in}}%
\pgfpathlineto{\pgfqpoint{5.071612in}{3.959214in}}%
\pgfpathlineto{\pgfqpoint{5.061039in}{3.970905in}}%
\pgfpathlineto{\pgfqpoint{5.050483in}{3.982423in}}%
\pgfpathlineto{\pgfqpoint{5.019412in}{3.983986in}}%
\pgfpathlineto{\pgfqpoint{4.988320in}{3.985593in}}%
\pgfpathclose%
\pgfusepath{fill}%
\end{pgfscope}%
\begin{pgfscope}%
\pgfpathrectangle{\pgfqpoint{1.020000in}{0.880000in}}{\pgfqpoint{6.160000in}{6.160000in}}%
\pgfusepath{clip}%
\pgfsetbuttcap%
\pgfsetroundjoin%
\definecolor{currentfill}{rgb}{0.548876,0.685104,0.994379}%
\pgfsetfillcolor{currentfill}%
\pgfsetlinewidth{0.000000pt}%
\definecolor{currentstroke}{rgb}{0.000000,0.000000,0.000000}%
\pgfsetstrokecolor{currentstroke}%
\pgfsetdash{}{0pt}%
\pgfpathmoveto{\pgfqpoint{4.780683in}{4.019048in}}%
\pgfpathlineto{\pgfqpoint{4.791011in}{4.007736in}}%
\pgfpathlineto{\pgfqpoint{4.801356in}{3.996235in}}%
\pgfpathlineto{\pgfqpoint{4.832566in}{3.994333in}}%
\pgfpathlineto{\pgfqpoint{4.863757in}{3.992485in}}%
\pgfpathlineto{\pgfqpoint{4.853363in}{4.003949in}}%
\pgfpathlineto{\pgfqpoint{4.842987in}{4.015227in}}%
\pgfpathlineto{\pgfqpoint{4.811845in}{4.017111in}}%
\pgfpathlineto{\pgfqpoint{4.780683in}{4.019048in}}%
\pgfpathclose%
\pgfusepath{fill}%
\end{pgfscope}%
\begin{pgfscope}%
\pgfpathrectangle{\pgfqpoint{1.020000in}{0.880000in}}{\pgfqpoint{6.160000in}{6.160000in}}%
\pgfusepath{clip}%
\pgfsetbuttcap%
\pgfsetroundjoin%
\definecolor{currentfill}{rgb}{0.510824,0.649397,0.985079}%
\pgfsetfillcolor{currentfill}%
\pgfsetlinewidth{0.000000pt}%
\definecolor{currentstroke}{rgb}{0.000000,0.000000,0.000000}%
\pgfsetstrokecolor{currentstroke}%
\pgfsetdash{}{0pt}%
\pgfpathmoveto{\pgfqpoint{5.133792in}{3.956308in}}%
\pgfpathlineto{\pgfqpoint{5.144431in}{3.944495in}}%
\pgfpathlineto{\pgfqpoint{5.155087in}{3.932515in}}%
\pgfpathlineto{\pgfqpoint{5.186196in}{3.931175in}}%
\pgfpathlineto{\pgfqpoint{5.217285in}{3.929876in}}%
\pgfpathlineto{\pgfqpoint{5.206581in}{3.941803in}}%
\pgfpathlineto{\pgfqpoint{5.195895in}{3.953567in}}%
\pgfpathlineto{\pgfqpoint{5.164853in}{3.954917in}}%
\pgfpathlineto{\pgfqpoint{5.133792in}{3.956308in}}%
\pgfpathclose%
\pgfusepath{fill}%
\end{pgfscope}%
\begin{pgfscope}%
\pgfpathrectangle{\pgfqpoint{1.020000in}{0.880000in}}{\pgfqpoint{6.160000in}{6.160000in}}%
\pgfusepath{clip}%
\pgfsetbuttcap%
\pgfsetroundjoin%
\definecolor{currentfill}{rgb}{0.532568,0.669801,0.990393}%
\pgfsetfillcolor{currentfill}%
\pgfsetlinewidth{0.000000pt}%
\definecolor{currentstroke}{rgb}{0.000000,0.000000,0.000000}%
\pgfsetstrokecolor{currentstroke}%
\pgfsetdash{}{0pt}%
\pgfpathmoveto{\pgfqpoint{4.926078in}{3.988943in}}%
\pgfpathlineto{\pgfqpoint{4.936538in}{3.977338in}}%
\pgfpathlineto{\pgfqpoint{4.947014in}{3.965553in}}%
\pgfpathlineto{\pgfqpoint{4.978193in}{3.963900in}}%
\pgfpathlineto{\pgfqpoint{5.009353in}{3.962293in}}%
\pgfpathlineto{\pgfqpoint{4.998828in}{3.974031in}}%
\pgfpathlineto{\pgfqpoint{4.988320in}{3.985593in}}%
\pgfpathlineto{\pgfqpoint{4.957209in}{3.987245in}}%
\pgfpathlineto{\pgfqpoint{4.926078in}{3.988943in}}%
\pgfpathclose%
\pgfusepath{fill}%
\end{pgfscope}%
\begin{pgfscope}%
\pgfpathrectangle{\pgfqpoint{1.020000in}{0.880000in}}{\pgfqpoint{6.160000in}{6.160000in}}%
\pgfusepath{clip}%
\pgfsetbuttcap%
\pgfsetroundjoin%
\definecolor{currentfill}{rgb}{0.554312,0.690097,0.995516}%
\pgfsetfillcolor{currentfill}%
\pgfsetlinewidth{0.000000pt}%
\definecolor{currentstroke}{rgb}{0.000000,0.000000,0.000000}%
\pgfsetstrokecolor{currentstroke}%
\pgfsetdash{}{0pt}%
\pgfpathmoveto{\pgfqpoint{4.718298in}{4.023092in}}%
\pgfpathlineto{\pgfqpoint{4.728578in}{4.011748in}}%
\pgfpathlineto{\pgfqpoint{4.738875in}{4.000213in}}%
\pgfpathlineto{\pgfqpoint{4.770125in}{3.998193in}}%
\pgfpathlineto{\pgfqpoint{4.801356in}{3.996235in}}%
\pgfpathlineto{\pgfqpoint{4.791011in}{4.007736in}}%
\pgfpathlineto{\pgfqpoint{4.780683in}{4.019048in}}%
\pgfpathlineto{\pgfqpoint{4.749500in}{4.021041in}}%
\pgfpathlineto{\pgfqpoint{4.718298in}{4.023092in}}%
\pgfpathclose%
\pgfusepath{fill}%
\end{pgfscope}%
\begin{pgfscope}%
\pgfpathrectangle{\pgfqpoint{1.020000in}{0.880000in}}{\pgfqpoint{6.160000in}{6.160000in}}%
\pgfusepath{clip}%
\pgfsetbuttcap%
\pgfsetroundjoin%
\definecolor{currentfill}{rgb}{0.500031,0.638508,0.981070}%
\pgfsetfillcolor{currentfill}%
\pgfsetlinewidth{0.000000pt}%
\definecolor{currentstroke}{rgb}{0.000000,0.000000,0.000000}%
\pgfsetstrokecolor{currentstroke}%
\pgfsetdash{}{0pt}%
\pgfpathmoveto{\pgfqpoint{5.279405in}{3.927396in}}%
\pgfpathlineto{\pgfqpoint{5.290175in}{3.915361in}}%
\pgfpathlineto{\pgfqpoint{5.300962in}{3.903169in}}%
\pgfpathlineto{\pgfqpoint{5.332041in}{3.902043in}}%
\pgfpathlineto{\pgfqpoint{5.321230in}{3.914206in}}%
\pgfpathlineto{\pgfqpoint{5.310436in}{3.926214in}}%
\pgfpathlineto{\pgfqpoint{5.279405in}{3.927396in}}%
\pgfpathclose%
\pgfusepath{fill}%
\end{pgfscope}%
\begin{pgfscope}%
\pgfpathrectangle{\pgfqpoint{1.020000in}{0.880000in}}{\pgfqpoint{6.160000in}{6.160000in}}%
\pgfusepath{clip}%
\pgfsetbuttcap%
\pgfsetroundjoin%
\definecolor{currentfill}{rgb}{0.521696,0.659599,0.987736}%
\pgfsetfillcolor{currentfill}%
\pgfsetlinewidth{0.000000pt}%
\definecolor{currentstroke}{rgb}{0.000000,0.000000,0.000000}%
\pgfsetstrokecolor{currentstroke}%
\pgfsetdash{}{0pt}%
\pgfpathmoveto{\pgfqpoint{5.071612in}{3.959214in}}%
\pgfpathlineto{\pgfqpoint{5.082202in}{3.947351in}}%
\pgfpathlineto{\pgfqpoint{5.092810in}{3.935319in}}%
\pgfpathlineto{\pgfqpoint{5.123958in}{3.933896in}}%
\pgfpathlineto{\pgfqpoint{5.155087in}{3.932515in}}%
\pgfpathlineto{\pgfqpoint{5.144431in}{3.944495in}}%
\pgfpathlineto{\pgfqpoint{5.133792in}{3.956308in}}%
\pgfpathlineto{\pgfqpoint{5.102712in}{3.957740in}}%
\pgfpathlineto{\pgfqpoint{5.071612in}{3.959214in}}%
\pgfpathclose%
\pgfusepath{fill}%
\end{pgfscope}%
\begin{pgfscope}%
\pgfpathrectangle{\pgfqpoint{1.020000in}{0.880000in}}{\pgfqpoint{6.160000in}{6.160000in}}%
\pgfusepath{clip}%
\pgfsetbuttcap%
\pgfsetroundjoin%
\definecolor{currentfill}{rgb}{0.538004,0.674902,0.991722}%
\pgfsetfillcolor{currentfill}%
\pgfsetlinewidth{0.000000pt}%
\definecolor{currentstroke}{rgb}{0.000000,0.000000,0.000000}%
\pgfsetstrokecolor{currentstroke}%
\pgfsetdash{}{0pt}%
\pgfpathmoveto{\pgfqpoint{4.863757in}{3.992485in}}%
\pgfpathlineto{\pgfqpoint{4.874168in}{3.980837in}}%
\pgfpathlineto{\pgfqpoint{4.884596in}{3.969007in}}%
\pgfpathlineto{\pgfqpoint{4.915815in}{3.967255in}}%
\pgfpathlineto{\pgfqpoint{4.947014in}{3.965553in}}%
\pgfpathlineto{\pgfqpoint{4.936538in}{3.977338in}}%
\pgfpathlineto{\pgfqpoint{4.926078in}{3.988943in}}%
\pgfpathlineto{\pgfqpoint{4.894928in}{3.990689in}}%
\pgfpathlineto{\pgfqpoint{4.863757in}{3.992485in}}%
\pgfpathclose%
\pgfusepath{fill}%
\end{pgfscope}%
\begin{pgfscope}%
\pgfpathrectangle{\pgfqpoint{1.020000in}{0.880000in}}{\pgfqpoint{6.160000in}{6.160000in}}%
\pgfusepath{clip}%
\pgfsetbuttcap%
\pgfsetroundjoin%
\definecolor{currentfill}{rgb}{0.565182,0.699438,0.996635}%
\pgfsetfillcolor{currentfill}%
\pgfsetlinewidth{0.000000pt}%
\definecolor{currentstroke}{rgb}{0.000000,0.000000,0.000000}%
\pgfsetstrokecolor{currentstroke}%
\pgfsetdash{}{0pt}%
\pgfpathmoveto{\pgfqpoint{4.655833in}{4.027384in}}%
\pgfpathlineto{\pgfqpoint{4.666064in}{4.016014in}}%
\pgfpathlineto{\pgfqpoint{4.676312in}{4.004450in}}%
\pgfpathlineto{\pgfqpoint{4.707604in}{4.002297in}}%
\pgfpathlineto{\pgfqpoint{4.738875in}{4.000213in}}%
\pgfpathlineto{\pgfqpoint{4.728578in}{4.011748in}}%
\pgfpathlineto{\pgfqpoint{4.718298in}{4.023092in}}%
\pgfpathlineto{\pgfqpoint{4.687076in}{4.025205in}}%
\pgfpathlineto{\pgfqpoint{4.655833in}{4.027384in}}%
\pgfpathclose%
\pgfusepath{fill}%
\end{pgfscope}%
\begin{pgfscope}%
\pgfpathrectangle{\pgfqpoint{1.020000in}{0.880000in}}{\pgfqpoint{6.160000in}{6.160000in}}%
\pgfusepath{clip}%
\pgfsetbuttcap%
\pgfsetroundjoin%
\definecolor{currentfill}{rgb}{0.505423,0.643995,0.983157}%
\pgfsetfillcolor{currentfill}%
\pgfsetlinewidth{0.000000pt}%
\definecolor{currentstroke}{rgb}{0.000000,0.000000,0.000000}%
\pgfsetstrokecolor{currentstroke}%
\pgfsetdash{}{0pt}%
\pgfpathmoveto{\pgfqpoint{5.217285in}{3.929876in}}%
\pgfpathlineto{\pgfqpoint{5.228007in}{3.917787in}}%
\pgfpathlineto{\pgfqpoint{5.238746in}{3.905538in}}%
\pgfpathlineto{\pgfqpoint{5.269864in}{3.904334in}}%
\pgfpathlineto{\pgfqpoint{5.300962in}{3.903169in}}%
\pgfpathlineto{\pgfqpoint{5.290175in}{3.915361in}}%
\pgfpathlineto{\pgfqpoint{5.279405in}{3.927396in}}%
\pgfpathlineto{\pgfqpoint{5.248355in}{3.928616in}}%
\pgfpathlineto{\pgfqpoint{5.217285in}{3.929876in}}%
\pgfpathclose%
\pgfusepath{fill}%
\end{pgfscope}%
\begin{pgfscope}%
\pgfpathrectangle{\pgfqpoint{1.020000in}{0.880000in}}{\pgfqpoint{6.160000in}{6.160000in}}%
\pgfusepath{clip}%
\pgfsetbuttcap%
\pgfsetroundjoin%
\definecolor{currentfill}{rgb}{0.527132,0.664700,0.989065}%
\pgfsetfillcolor{currentfill}%
\pgfsetlinewidth{0.000000pt}%
\definecolor{currentstroke}{rgb}{0.000000,0.000000,0.000000}%
\pgfsetstrokecolor{currentstroke}%
\pgfsetdash{}{0pt}%
\pgfpathmoveto{\pgfqpoint{5.009353in}{3.962293in}}%
\pgfpathlineto{\pgfqpoint{5.019895in}{3.950380in}}%
\pgfpathlineto{\pgfqpoint{5.030454in}{3.938295in}}%
\pgfpathlineto{\pgfqpoint{5.061642in}{3.936785in}}%
\pgfpathlineto{\pgfqpoint{5.092810in}{3.935319in}}%
\pgfpathlineto{\pgfqpoint{5.082202in}{3.947351in}}%
\pgfpathlineto{\pgfqpoint{5.071612in}{3.959214in}}%
\pgfpathlineto{\pgfqpoint{5.040492in}{3.960731in}}%
\pgfpathlineto{\pgfqpoint{5.009353in}{3.962293in}}%
\pgfpathclose%
\pgfusepath{fill}%
\end{pgfscope}%
\begin{pgfscope}%
\pgfpathrectangle{\pgfqpoint{1.020000in}{0.880000in}}{\pgfqpoint{6.160000in}{6.160000in}}%
\pgfusepath{clip}%
\pgfsetbuttcap%
\pgfsetroundjoin%
\definecolor{currentfill}{rgb}{0.548876,0.685104,0.994379}%
\pgfsetfillcolor{currentfill}%
\pgfsetlinewidth{0.000000pt}%
\definecolor{currentstroke}{rgb}{0.000000,0.000000,0.000000}%
\pgfsetstrokecolor{currentstroke}%
\pgfsetdash{}{0pt}%
\pgfpathmoveto{\pgfqpoint{4.801356in}{3.996235in}}%
\pgfpathlineto{\pgfqpoint{4.811718in}{3.984546in}}%
\pgfpathlineto{\pgfqpoint{4.822098in}{3.972672in}}%
\pgfpathlineto{\pgfqpoint{4.853357in}{3.970812in}}%
\pgfpathlineto{\pgfqpoint{4.884596in}{3.969007in}}%
\pgfpathlineto{\pgfqpoint{4.874168in}{3.980837in}}%
\pgfpathlineto{\pgfqpoint{4.863757in}{3.992485in}}%
\pgfpathlineto{\pgfqpoint{4.832566in}{3.994333in}}%
\pgfpathlineto{\pgfqpoint{4.801356in}{3.996235in}}%
\pgfpathclose%
\pgfusepath{fill}%
\end{pgfscope}%
\begin{pgfscope}%
\pgfpathrectangle{\pgfqpoint{1.020000in}{0.880000in}}{\pgfqpoint{6.160000in}{6.160000in}}%
\pgfusepath{clip}%
\pgfsetbuttcap%
\pgfsetroundjoin%
\definecolor{currentfill}{rgb}{0.570616,0.704109,0.997195}%
\pgfsetfillcolor{currentfill}%
\pgfsetlinewidth{0.000000pt}%
\definecolor{currentstroke}{rgb}{0.000000,0.000000,0.000000}%
\pgfsetstrokecolor{currentstroke}%
\pgfsetdash{}{0pt}%
\pgfpathmoveto{\pgfqpoint{4.593287in}{4.031961in}}%
\pgfpathlineto{\pgfqpoint{4.603470in}{4.020574in}}%
\pgfpathlineto{\pgfqpoint{4.613669in}{4.008991in}}%
\pgfpathlineto{\pgfqpoint{4.645001in}{4.006680in}}%
\pgfpathlineto{\pgfqpoint{4.676312in}{4.004450in}}%
\pgfpathlineto{\pgfqpoint{4.666064in}{4.016014in}}%
\pgfpathlineto{\pgfqpoint{4.655833in}{4.027384in}}%
\pgfpathlineto{\pgfqpoint{4.624570in}{4.029634in}}%
\pgfpathlineto{\pgfqpoint{4.593287in}{4.031961in}}%
\pgfpathclose%
\pgfusepath{fill}%
\end{pgfscope}%
\begin{pgfscope}%
\pgfpathrectangle{\pgfqpoint{1.020000in}{0.880000in}}{\pgfqpoint{6.160000in}{6.160000in}}%
\pgfusepath{clip}%
\pgfsetbuttcap%
\pgfsetroundjoin%
\definecolor{currentfill}{rgb}{0.510824,0.649397,0.985079}%
\pgfsetfillcolor{currentfill}%
\pgfsetlinewidth{0.000000pt}%
\definecolor{currentstroke}{rgb}{0.000000,0.000000,0.000000}%
\pgfsetstrokecolor{currentstroke}%
\pgfsetdash{}{0pt}%
\pgfpathmoveto{\pgfqpoint{5.155087in}{3.932515in}}%
\pgfpathlineto{\pgfqpoint{5.165760in}{3.920371in}}%
\pgfpathlineto{\pgfqpoint{5.176451in}{3.908064in}}%
\pgfpathlineto{\pgfqpoint{5.207608in}{3.906781in}}%
\pgfpathlineto{\pgfqpoint{5.238746in}{3.905538in}}%
\pgfpathlineto{\pgfqpoint{5.228007in}{3.917787in}}%
\pgfpathlineto{\pgfqpoint{5.217285in}{3.929876in}}%
\pgfpathlineto{\pgfqpoint{5.186196in}{3.931175in}}%
\pgfpathlineto{\pgfqpoint{5.155087in}{3.932515in}}%
\pgfpathclose%
\pgfusepath{fill}%
\end{pgfscope}%
\begin{pgfscope}%
\pgfpathrectangle{\pgfqpoint{1.020000in}{0.880000in}}{\pgfqpoint{6.160000in}{6.160000in}}%
\pgfusepath{clip}%
\pgfsetbuttcap%
\pgfsetroundjoin%
\definecolor{currentfill}{rgb}{0.532568,0.669801,0.990393}%
\pgfsetfillcolor{currentfill}%
\pgfsetlinewidth{0.000000pt}%
\definecolor{currentstroke}{rgb}{0.000000,0.000000,0.000000}%
\pgfsetstrokecolor{currentstroke}%
\pgfsetdash{}{0pt}%
\pgfpathmoveto{\pgfqpoint{4.947014in}{3.965553in}}%
\pgfpathlineto{\pgfqpoint{4.957508in}{3.953591in}}%
\pgfpathlineto{\pgfqpoint{4.968019in}{3.941453in}}%
\pgfpathlineto{\pgfqpoint{4.999247in}{3.939851in}}%
\pgfpathlineto{\pgfqpoint{5.030454in}{3.938295in}}%
\pgfpathlineto{\pgfqpoint{5.019895in}{3.950380in}}%
\pgfpathlineto{\pgfqpoint{5.009353in}{3.962293in}}%
\pgfpathlineto{\pgfqpoint{4.978193in}{3.963900in}}%
\pgfpathlineto{\pgfqpoint{4.947014in}{3.965553in}}%
\pgfpathclose%
\pgfusepath{fill}%
\end{pgfscope}%
\begin{pgfscope}%
\pgfpathrectangle{\pgfqpoint{1.020000in}{0.880000in}}{\pgfqpoint{6.160000in}{6.160000in}}%
\pgfusepath{clip}%
\pgfsetbuttcap%
\pgfsetroundjoin%
\definecolor{currentfill}{rgb}{0.554312,0.690097,0.995516}%
\pgfsetfillcolor{currentfill}%
\pgfsetlinewidth{0.000000pt}%
\definecolor{currentstroke}{rgb}{0.000000,0.000000,0.000000}%
\pgfsetstrokecolor{currentstroke}%
\pgfsetdash{}{0pt}%
\pgfpathmoveto{\pgfqpoint{4.738875in}{4.000213in}}%
\pgfpathlineto{\pgfqpoint{4.749189in}{3.988487in}}%
\pgfpathlineto{\pgfqpoint{4.759520in}{3.976573in}}%
\pgfpathlineto{\pgfqpoint{4.790819in}{3.974591in}}%
\pgfpathlineto{\pgfqpoint{4.822098in}{3.972672in}}%
\pgfpathlineto{\pgfqpoint{4.811718in}{3.984546in}}%
\pgfpathlineto{\pgfqpoint{4.801356in}{3.996235in}}%
\pgfpathlineto{\pgfqpoint{4.770125in}{3.998193in}}%
\pgfpathlineto{\pgfqpoint{4.738875in}{4.000213in}}%
\pgfpathclose%
\pgfusepath{fill}%
\end{pgfscope}%
\begin{pgfscope}%
\pgfpathrectangle{\pgfqpoint{1.020000in}{0.880000in}}{\pgfqpoint{6.160000in}{6.160000in}}%
\pgfusepath{clip}%
\pgfsetbuttcap%
\pgfsetroundjoin%
\definecolor{currentfill}{rgb}{0.576051,0.708780,0.997755}%
\pgfsetfillcolor{currentfill}%
\pgfsetlinewidth{0.000000pt}%
\definecolor{currentstroke}{rgb}{0.000000,0.000000,0.000000}%
\pgfsetstrokecolor{currentstroke}%
\pgfsetdash{}{0pt}%
\pgfpathmoveto{\pgfqpoint{4.530660in}{4.036875in}}%
\pgfpathlineto{\pgfqpoint{4.540794in}{4.025484in}}%
\pgfpathlineto{\pgfqpoint{4.550945in}{4.013894in}}%
\pgfpathlineto{\pgfqpoint{4.582317in}{4.011393in}}%
\pgfpathlineto{\pgfqpoint{4.613669in}{4.008991in}}%
\pgfpathlineto{\pgfqpoint{4.603470in}{4.020574in}}%
\pgfpathlineto{\pgfqpoint{4.593287in}{4.031961in}}%
\pgfpathlineto{\pgfqpoint{4.561984in}{4.034372in}}%
\pgfpathlineto{\pgfqpoint{4.530660in}{4.036875in}}%
\pgfpathclose%
\pgfusepath{fill}%
\end{pgfscope}%
\begin{pgfscope}%
\pgfpathrectangle{\pgfqpoint{1.020000in}{0.880000in}}{\pgfqpoint{6.160000in}{6.160000in}}%
\pgfusepath{clip}%
\pgfsetbuttcap%
\pgfsetroundjoin%
\definecolor{currentfill}{rgb}{0.494638,0.633022,0.978983}%
\pgfsetfillcolor{currentfill}%
\pgfsetlinewidth{0.000000pt}%
\definecolor{currentstroke}{rgb}{0.000000,0.000000,0.000000}%
\pgfsetstrokecolor{currentstroke}%
\pgfsetdash{}{0pt}%
\pgfpathmoveto{\pgfqpoint{5.300962in}{3.903169in}}%
\pgfpathlineto{\pgfqpoint{5.311767in}{3.890823in}}%
\pgfpathlineto{\pgfqpoint{5.322589in}{3.878325in}}%
\pgfpathlineto{\pgfqpoint{5.353716in}{3.877259in}}%
\pgfpathlineto{\pgfqpoint{5.342870in}{3.889726in}}%
\pgfpathlineto{\pgfqpoint{5.332041in}{3.902043in}}%
\pgfpathlineto{\pgfqpoint{5.300962in}{3.903169in}}%
\pgfpathclose%
\pgfusepath{fill}%
\end{pgfscope}%
\begin{pgfscope}%
\pgfpathrectangle{\pgfqpoint{1.020000in}{0.880000in}}{\pgfqpoint{6.160000in}{6.160000in}}%
\pgfusepath{clip}%
\pgfsetbuttcap%
\pgfsetroundjoin%
\definecolor{currentfill}{rgb}{0.516260,0.654498,0.986407}%
\pgfsetfillcolor{currentfill}%
\pgfsetlinewidth{0.000000pt}%
\definecolor{currentstroke}{rgb}{0.000000,0.000000,0.000000}%
\pgfsetstrokecolor{currentstroke}%
\pgfsetdash{}{0pt}%
\pgfpathmoveto{\pgfqpoint{5.092810in}{3.935319in}}%
\pgfpathlineto{\pgfqpoint{5.103435in}{3.923119in}}%
\pgfpathlineto{\pgfqpoint{5.114078in}{3.910754in}}%
\pgfpathlineto{\pgfqpoint{5.145274in}{3.909388in}}%
\pgfpathlineto{\pgfqpoint{5.176451in}{3.908064in}}%
\pgfpathlineto{\pgfqpoint{5.165760in}{3.920371in}}%
\pgfpathlineto{\pgfqpoint{5.155087in}{3.932515in}}%
\pgfpathlineto{\pgfqpoint{5.123958in}{3.933896in}}%
\pgfpathlineto{\pgfqpoint{5.092810in}{3.935319in}}%
\pgfpathclose%
\pgfusepath{fill}%
\end{pgfscope}%
\begin{pgfscope}%
\pgfpathrectangle{\pgfqpoint{1.020000in}{0.880000in}}{\pgfqpoint{6.160000in}{6.160000in}}%
\pgfusepath{clip}%
\pgfsetbuttcap%
\pgfsetroundjoin%
\definecolor{currentfill}{rgb}{0.538004,0.674902,0.991722}%
\pgfsetfillcolor{currentfill}%
\pgfsetlinewidth{0.000000pt}%
\definecolor{currentstroke}{rgb}{0.000000,0.000000,0.000000}%
\pgfsetstrokecolor{currentstroke}%
\pgfsetdash{}{0pt}%
\pgfpathmoveto{\pgfqpoint{4.884596in}{3.969007in}}%
\pgfpathlineto{\pgfqpoint{4.895042in}{3.956996in}}%
\pgfpathlineto{\pgfqpoint{4.905505in}{3.944807in}}%
\pgfpathlineto{\pgfqpoint{4.936772in}{3.943104in}}%
\pgfpathlineto{\pgfqpoint{4.968019in}{3.941453in}}%
\pgfpathlineto{\pgfqpoint{4.957508in}{3.953591in}}%
\pgfpathlineto{\pgfqpoint{4.947014in}{3.965553in}}%
\pgfpathlineto{\pgfqpoint{4.915815in}{3.967255in}}%
\pgfpathlineto{\pgfqpoint{4.884596in}{3.969007in}}%
\pgfpathclose%
\pgfusepath{fill}%
\end{pgfscope}%
\begin{pgfscope}%
\pgfpathrectangle{\pgfqpoint{1.020000in}{0.880000in}}{\pgfqpoint{6.160000in}{6.160000in}}%
\pgfusepath{clip}%
\pgfsetbuttcap%
\pgfsetroundjoin%
\definecolor{currentfill}{rgb}{0.559747,0.694768,0.996075}%
\pgfsetfillcolor{currentfill}%
\pgfsetlinewidth{0.000000pt}%
\definecolor{currentstroke}{rgb}{0.000000,0.000000,0.000000}%
\pgfsetstrokecolor{currentstroke}%
\pgfsetdash{}{0pt}%
\pgfpathmoveto{\pgfqpoint{4.676312in}{4.004450in}}%
\pgfpathlineto{\pgfqpoint{4.686578in}{3.992694in}}%
\pgfpathlineto{\pgfqpoint{4.696861in}{3.980745in}}%
\pgfpathlineto{\pgfqpoint{4.728201in}{3.978623in}}%
\pgfpathlineto{\pgfqpoint{4.759520in}{3.976573in}}%
\pgfpathlineto{\pgfqpoint{4.749189in}{3.988487in}}%
\pgfpathlineto{\pgfqpoint{4.738875in}{4.000213in}}%
\pgfpathlineto{\pgfqpoint{4.707604in}{4.002297in}}%
\pgfpathlineto{\pgfqpoint{4.676312in}{4.004450in}}%
\pgfpathclose%
\pgfusepath{fill}%
\end{pgfscope}%
\begin{pgfscope}%
\pgfpathrectangle{\pgfqpoint{1.020000in}{0.880000in}}{\pgfqpoint{6.160000in}{6.160000in}}%
\pgfusepath{clip}%
\pgfsetbuttcap%
\pgfsetroundjoin%
\definecolor{currentfill}{rgb}{0.500031,0.638508,0.981070}%
\pgfsetfillcolor{currentfill}%
\pgfsetlinewidth{0.000000pt}%
\definecolor{currentstroke}{rgb}{0.000000,0.000000,0.000000}%
\pgfsetstrokecolor{currentstroke}%
\pgfsetdash{}{0pt}%
\pgfpathmoveto{\pgfqpoint{5.238746in}{3.905538in}}%
\pgfpathlineto{\pgfqpoint{5.249502in}{3.893132in}}%
\pgfpathlineto{\pgfqpoint{5.260276in}{3.880571in}}%
\pgfpathlineto{\pgfqpoint{5.291442in}{3.879429in}}%
\pgfpathlineto{\pgfqpoint{5.322589in}{3.878325in}}%
\pgfpathlineto{\pgfqpoint{5.311767in}{3.890823in}}%
\pgfpathlineto{\pgfqpoint{5.300962in}{3.903169in}}%
\pgfpathlineto{\pgfqpoint{5.269864in}{3.904334in}}%
\pgfpathlineto{\pgfqpoint{5.238746in}{3.905538in}}%
\pgfpathclose%
\pgfusepath{fill}%
\end{pgfscope}%
\begin{pgfscope}%
\pgfpathrectangle{\pgfqpoint{1.020000in}{0.880000in}}{\pgfqpoint{6.160000in}{6.160000in}}%
\pgfusepath{clip}%
\pgfsetbuttcap%
\pgfsetroundjoin%
\definecolor{currentfill}{rgb}{0.581486,0.713451,0.998314}%
\pgfsetfillcolor{currentfill}%
\pgfsetlinewidth{0.000000pt}%
\definecolor{currentstroke}{rgb}{0.000000,0.000000,0.000000}%
\pgfsetstrokecolor{currentstroke}%
\pgfsetdash{}{0pt}%
\pgfpathmoveto{\pgfqpoint{4.467950in}{4.042195in}}%
\pgfpathlineto{\pgfqpoint{4.478035in}{4.030819in}}%
\pgfpathlineto{\pgfqpoint{4.488138in}{4.019244in}}%
\pgfpathlineto{\pgfqpoint{4.519552in}{4.016507in}}%
\pgfpathlineto{\pgfqpoint{4.550945in}{4.013894in}}%
\pgfpathlineto{\pgfqpoint{4.540794in}{4.025484in}}%
\pgfpathlineto{\pgfqpoint{4.530660in}{4.036875in}}%
\pgfpathlineto{\pgfqpoint{4.499315in}{4.039479in}}%
\pgfpathlineto{\pgfqpoint{4.467950in}{4.042195in}}%
\pgfpathclose%
\pgfusepath{fill}%
\end{pgfscope}%
\begin{pgfscope}%
\pgfpathrectangle{\pgfqpoint{1.020000in}{0.880000in}}{\pgfqpoint{6.160000in}{6.160000in}}%
\pgfusepath{clip}%
\pgfsetbuttcap%
\pgfsetroundjoin%
\definecolor{currentfill}{rgb}{0.521696,0.659599,0.987736}%
\pgfsetfillcolor{currentfill}%
\pgfsetlinewidth{0.000000pt}%
\definecolor{currentstroke}{rgb}{0.000000,0.000000,0.000000}%
\pgfsetstrokecolor{currentstroke}%
\pgfsetdash{}{0pt}%
\pgfpathmoveto{\pgfqpoint{5.030454in}{3.938295in}}%
\pgfpathlineto{\pgfqpoint{5.041031in}{3.926040in}}%
\pgfpathlineto{\pgfqpoint{5.051625in}{3.913616in}}%
\pgfpathlineto{\pgfqpoint{5.082862in}{3.912163in}}%
\pgfpathlineto{\pgfqpoint{5.114078in}{3.910754in}}%
\pgfpathlineto{\pgfqpoint{5.103435in}{3.923119in}}%
\pgfpathlineto{\pgfqpoint{5.092810in}{3.935319in}}%
\pgfpathlineto{\pgfqpoint{5.061642in}{3.936785in}}%
\pgfpathlineto{\pgfqpoint{5.030454in}{3.938295in}}%
\pgfpathclose%
\pgfusepath{fill}%
\end{pgfscope}%
\begin{pgfscope}%
\pgfpathrectangle{\pgfqpoint{1.020000in}{0.880000in}}{\pgfqpoint{6.160000in}{6.160000in}}%
\pgfusepath{clip}%
\pgfsetbuttcap%
\pgfsetroundjoin%
\definecolor{currentfill}{rgb}{0.543440,0.680003,0.993051}%
\pgfsetfillcolor{currentfill}%
\pgfsetlinewidth{0.000000pt}%
\definecolor{currentstroke}{rgb}{0.000000,0.000000,0.000000}%
\pgfsetstrokecolor{currentstroke}%
\pgfsetdash{}{0pt}%
\pgfpathmoveto{\pgfqpoint{4.822098in}{3.972672in}}%
\pgfpathlineto{\pgfqpoint{4.832495in}{3.960614in}}%
\pgfpathlineto{\pgfqpoint{4.842910in}{3.948375in}}%
\pgfpathlineto{\pgfqpoint{4.874217in}{3.946563in}}%
\pgfpathlineto{\pgfqpoint{4.905505in}{3.944807in}}%
\pgfpathlineto{\pgfqpoint{4.895042in}{3.956996in}}%
\pgfpathlineto{\pgfqpoint{4.884596in}{3.969007in}}%
\pgfpathlineto{\pgfqpoint{4.853357in}{3.970812in}}%
\pgfpathlineto{\pgfqpoint{4.822098in}{3.972672in}}%
\pgfpathclose%
\pgfusepath{fill}%
\end{pgfscope}%
\begin{pgfscope}%
\pgfpathrectangle{\pgfqpoint{1.020000in}{0.880000in}}{\pgfqpoint{6.160000in}{6.160000in}}%
\pgfusepath{clip}%
\pgfsetbuttcap%
\pgfsetroundjoin%
\definecolor{currentfill}{rgb}{0.565182,0.699438,0.996635}%
\pgfsetfillcolor{currentfill}%
\pgfsetlinewidth{0.000000pt}%
\definecolor{currentstroke}{rgb}{0.000000,0.000000,0.000000}%
\pgfsetstrokecolor{currentstroke}%
\pgfsetdash{}{0pt}%
\pgfpathmoveto{\pgfqpoint{4.613669in}{4.008991in}}%
\pgfpathlineto{\pgfqpoint{4.623886in}{3.997212in}}%
\pgfpathlineto{\pgfqpoint{4.634121in}{3.985239in}}%
\pgfpathlineto{\pgfqpoint{4.665501in}{3.982948in}}%
\pgfpathlineto{\pgfqpoint{4.696861in}{3.980745in}}%
\pgfpathlineto{\pgfqpoint{4.686578in}{3.992694in}}%
\pgfpathlineto{\pgfqpoint{4.676312in}{4.004450in}}%
\pgfpathlineto{\pgfqpoint{4.645001in}{4.006680in}}%
\pgfpathlineto{\pgfqpoint{4.613669in}{4.008991in}}%
\pgfpathclose%
\pgfusepath{fill}%
\end{pgfscope}%
\begin{pgfscope}%
\pgfpathrectangle{\pgfqpoint{1.020000in}{0.880000in}}{\pgfqpoint{6.160000in}{6.160000in}}%
\pgfusepath{clip}%
\pgfsetbuttcap%
\pgfsetroundjoin%
\definecolor{currentfill}{rgb}{0.505423,0.643995,0.983157}%
\pgfsetfillcolor{currentfill}%
\pgfsetlinewidth{0.000000pt}%
\definecolor{currentstroke}{rgb}{0.000000,0.000000,0.000000}%
\pgfsetstrokecolor{currentstroke}%
\pgfsetdash{}{0pt}%
\pgfpathmoveto{\pgfqpoint{5.176451in}{3.908064in}}%
\pgfpathlineto{\pgfqpoint{5.187159in}{3.895598in}}%
\pgfpathlineto{\pgfqpoint{5.197885in}{3.882974in}}%
\pgfpathlineto{\pgfqpoint{5.229090in}{3.881753in}}%
\pgfpathlineto{\pgfqpoint{5.260276in}{3.880571in}}%
\pgfpathlineto{\pgfqpoint{5.249502in}{3.893132in}}%
\pgfpathlineto{\pgfqpoint{5.238746in}{3.905538in}}%
\pgfpathlineto{\pgfqpoint{5.207608in}{3.906781in}}%
\pgfpathlineto{\pgfqpoint{5.176451in}{3.908064in}}%
\pgfpathclose%
\pgfusepath{fill}%
\end{pgfscope}%
\begin{pgfscope}%
\pgfpathrectangle{\pgfqpoint{1.020000in}{0.880000in}}{\pgfqpoint{6.160000in}{6.160000in}}%
\pgfusepath{clip}%
\pgfsetbuttcap%
\pgfsetroundjoin%
\definecolor{currentfill}{rgb}{0.592356,0.722792,0.999434}%
\pgfsetfillcolor{currentfill}%
\pgfsetlinewidth{0.000000pt}%
\definecolor{currentstroke}{rgb}{0.000000,0.000000,0.000000}%
\pgfsetstrokecolor{currentstroke}%
\pgfsetdash{}{0pt}%
\pgfpathmoveto{\pgfqpoint{4.405158in}{4.048016in}}%
\pgfpathlineto{\pgfqpoint{4.415195in}{4.036683in}}%
\pgfpathlineto{\pgfqpoint{4.425248in}{4.025150in}}%
\pgfpathlineto{\pgfqpoint{4.456704in}{4.022119in}}%
\pgfpathlineto{\pgfqpoint{4.488138in}{4.019244in}}%
\pgfpathlineto{\pgfqpoint{4.478035in}{4.030819in}}%
\pgfpathlineto{\pgfqpoint{4.467950in}{4.042195in}}%
\pgfpathlineto{\pgfqpoint{4.436564in}{4.045036in}}%
\pgfpathlineto{\pgfqpoint{4.405158in}{4.048016in}}%
\pgfpathclose%
\pgfusepath{fill}%
\end{pgfscope}%
\begin{pgfscope}%
\pgfpathrectangle{\pgfqpoint{1.020000in}{0.880000in}}{\pgfqpoint{6.160000in}{6.160000in}}%
\pgfusepath{clip}%
\pgfsetbuttcap%
\pgfsetroundjoin%
\definecolor{currentfill}{rgb}{0.527132,0.664700,0.989065}%
\pgfsetfillcolor{currentfill}%
\pgfsetlinewidth{0.000000pt}%
\definecolor{currentstroke}{rgb}{0.000000,0.000000,0.000000}%
\pgfsetstrokecolor{currentstroke}%
\pgfsetdash{}{0pt}%
\pgfpathmoveto{\pgfqpoint{4.968019in}{3.941453in}}%
\pgfpathlineto{\pgfqpoint{4.978548in}{3.929142in}}%
\pgfpathlineto{\pgfqpoint{4.989094in}{3.916660in}}%
\pgfpathlineto{\pgfqpoint{5.020370in}{3.915114in}}%
\pgfpathlineto{\pgfqpoint{5.051625in}{3.913616in}}%
\pgfpathlineto{\pgfqpoint{5.041031in}{3.926040in}}%
\pgfpathlineto{\pgfqpoint{5.030454in}{3.938295in}}%
\pgfpathlineto{\pgfqpoint{4.999247in}{3.939851in}}%
\pgfpathlineto{\pgfqpoint{4.968019in}{3.941453in}}%
\pgfpathclose%
\pgfusepath{fill}%
\end{pgfscope}%
\begin{pgfscope}%
\pgfpathrectangle{\pgfqpoint{1.020000in}{0.880000in}}{\pgfqpoint{6.160000in}{6.160000in}}%
\pgfusepath{clip}%
\pgfsetbuttcap%
\pgfsetroundjoin%
\definecolor{currentfill}{rgb}{0.548876,0.685104,0.994379}%
\pgfsetfillcolor{currentfill}%
\pgfsetlinewidth{0.000000pt}%
\definecolor{currentstroke}{rgb}{0.000000,0.000000,0.000000}%
\pgfsetstrokecolor{currentstroke}%
\pgfsetdash{}{0pt}%
\pgfpathmoveto{\pgfqpoint{4.759520in}{3.976573in}}%
\pgfpathlineto{\pgfqpoint{4.769869in}{3.964472in}}%
\pgfpathlineto{\pgfqpoint{4.780235in}{3.952187in}}%
\pgfpathlineto{\pgfqpoint{4.811582in}{3.950249in}}%
\pgfpathlineto{\pgfqpoint{4.842910in}{3.948375in}}%
\pgfpathlineto{\pgfqpoint{4.832495in}{3.960614in}}%
\pgfpathlineto{\pgfqpoint{4.822098in}{3.972672in}}%
\pgfpathlineto{\pgfqpoint{4.790819in}{3.974591in}}%
\pgfpathlineto{\pgfqpoint{4.759520in}{3.976573in}}%
\pgfpathclose%
\pgfusepath{fill}%
\end{pgfscope}%
\begin{pgfscope}%
\pgfpathrectangle{\pgfqpoint{1.020000in}{0.880000in}}{\pgfqpoint{6.160000in}{6.160000in}}%
\pgfusepath{clip}%
\pgfsetbuttcap%
\pgfsetroundjoin%
\definecolor{currentfill}{rgb}{0.576051,0.708780,0.997755}%
\pgfsetfillcolor{currentfill}%
\pgfsetlinewidth{0.000000pt}%
\definecolor{currentstroke}{rgb}{0.000000,0.000000,0.000000}%
\pgfsetstrokecolor{currentstroke}%
\pgfsetdash{}{0pt}%
\pgfpathmoveto{\pgfqpoint{4.550945in}{4.013894in}}%
\pgfpathlineto{\pgfqpoint{4.561113in}{4.002107in}}%
\pgfpathlineto{\pgfqpoint{4.571299in}{3.990124in}}%
\pgfpathlineto{\pgfqpoint{4.602720in}{3.987627in}}%
\pgfpathlineto{\pgfqpoint{4.634121in}{3.985239in}}%
\pgfpathlineto{\pgfqpoint{4.623886in}{3.997212in}}%
\pgfpathlineto{\pgfqpoint{4.613669in}{4.008991in}}%
\pgfpathlineto{\pgfqpoint{4.582317in}{4.011393in}}%
\pgfpathlineto{\pgfqpoint{4.550945in}{4.013894in}}%
\pgfpathclose%
\pgfusepath{fill}%
\end{pgfscope}%
\begin{pgfscope}%
\pgfpathrectangle{\pgfqpoint{1.020000in}{0.880000in}}{\pgfqpoint{6.160000in}{6.160000in}}%
\pgfusepath{clip}%
\pgfsetbuttcap%
\pgfsetroundjoin%
\definecolor{currentfill}{rgb}{0.494638,0.633022,0.978983}%
\pgfsetfillcolor{currentfill}%
\pgfsetlinewidth{0.000000pt}%
\definecolor{currentstroke}{rgb}{0.000000,0.000000,0.000000}%
\pgfsetstrokecolor{currentstroke}%
\pgfsetdash{}{0pt}%
\pgfpathmoveto{\pgfqpoint{5.322589in}{3.878325in}}%
\pgfpathlineto{\pgfqpoint{5.333429in}{3.865678in}}%
\pgfpathlineto{\pgfqpoint{5.344286in}{3.852883in}}%
\pgfpathlineto{\pgfqpoint{5.375461in}{3.851883in}}%
\pgfpathlineto{\pgfqpoint{5.364580in}{3.864644in}}%
\pgfpathlineto{\pgfqpoint{5.353716in}{3.877259in}}%
\pgfpathlineto{\pgfqpoint{5.322589in}{3.878325in}}%
\pgfpathclose%
\pgfusepath{fill}%
\end{pgfscope}%
\begin{pgfscope}%
\pgfpathrectangle{\pgfqpoint{1.020000in}{0.880000in}}{\pgfqpoint{6.160000in}{6.160000in}}%
\pgfusepath{clip}%
\pgfsetbuttcap%
\pgfsetroundjoin%
\definecolor{currentfill}{rgb}{0.510824,0.649397,0.985079}%
\pgfsetfillcolor{currentfill}%
\pgfsetlinewidth{0.000000pt}%
\definecolor{currentstroke}{rgb}{0.000000,0.000000,0.000000}%
\pgfsetstrokecolor{currentstroke}%
\pgfsetdash{}{0pt}%
\pgfpathmoveto{\pgfqpoint{5.114078in}{3.910754in}}%
\pgfpathlineto{\pgfqpoint{5.124738in}{3.898227in}}%
\pgfpathlineto{\pgfqpoint{5.135415in}{3.885538in}}%
\pgfpathlineto{\pgfqpoint{5.166660in}{3.884235in}}%
\pgfpathlineto{\pgfqpoint{5.197885in}{3.882974in}}%
\pgfpathlineto{\pgfqpoint{5.187159in}{3.895598in}}%
\pgfpathlineto{\pgfqpoint{5.176451in}{3.908064in}}%
\pgfpathlineto{\pgfqpoint{5.145274in}{3.909388in}}%
\pgfpathlineto{\pgfqpoint{5.114078in}{3.910754in}}%
\pgfpathclose%
\pgfusepath{fill}%
\end{pgfscope}%
\begin{pgfscope}%
\pgfpathrectangle{\pgfqpoint{1.020000in}{0.880000in}}{\pgfqpoint{6.160000in}{6.160000in}}%
\pgfusepath{clip}%
\pgfsetbuttcap%
\pgfsetroundjoin%
\definecolor{currentfill}{rgb}{0.532568,0.669801,0.990393}%
\pgfsetfillcolor{currentfill}%
\pgfsetlinewidth{0.000000pt}%
\definecolor{currentstroke}{rgb}{0.000000,0.000000,0.000000}%
\pgfsetstrokecolor{currentstroke}%
\pgfsetdash{}{0pt}%
\pgfpathmoveto{\pgfqpoint{4.905505in}{3.944807in}}%
\pgfpathlineto{\pgfqpoint{4.915985in}{3.932441in}}%
\pgfpathlineto{\pgfqpoint{4.926482in}{3.919900in}}%
\pgfpathlineto{\pgfqpoint{4.957798in}{3.918254in}}%
\pgfpathlineto{\pgfqpoint{4.989094in}{3.916660in}}%
\pgfpathlineto{\pgfqpoint{4.978548in}{3.929142in}}%
\pgfpathlineto{\pgfqpoint{4.968019in}{3.941453in}}%
\pgfpathlineto{\pgfqpoint{4.936772in}{3.943104in}}%
\pgfpathlineto{\pgfqpoint{4.905505in}{3.944807in}}%
\pgfpathclose%
\pgfusepath{fill}%
\end{pgfscope}%
\begin{pgfscope}%
\pgfpathrectangle{\pgfqpoint{1.020000in}{0.880000in}}{\pgfqpoint{6.160000in}{6.160000in}}%
\pgfusepath{clip}%
\pgfsetbuttcap%
\pgfsetroundjoin%
\definecolor{currentfill}{rgb}{0.597777,0.727330,0.999777}%
\pgfsetfillcolor{currentfill}%
\pgfsetlinewidth{0.000000pt}%
\definecolor{currentstroke}{rgb}{0.000000,0.000000,0.000000}%
\pgfsetstrokecolor{currentstroke}%
\pgfsetdash{}{0pt}%
\pgfpathmoveto{\pgfqpoint{4.342282in}{4.054462in}}%
\pgfpathlineto{\pgfqpoint{4.352270in}{4.043210in}}%
\pgfpathlineto{\pgfqpoint{4.362275in}{4.031760in}}%
\pgfpathlineto{\pgfqpoint{4.393772in}{4.028356in}}%
\pgfpathlineto{\pgfqpoint{4.425248in}{4.025150in}}%
\pgfpathlineto{\pgfqpoint{4.415195in}{4.036683in}}%
\pgfpathlineto{\pgfqpoint{4.405158in}{4.048016in}}%
\pgfpathlineto{\pgfqpoint{4.373731in}{4.051152in}}%
\pgfpathlineto{\pgfqpoint{4.342282in}{4.054462in}}%
\pgfpathclose%
\pgfusepath{fill}%
\end{pgfscope}%
\begin{pgfscope}%
\pgfpathrectangle{\pgfqpoint{1.020000in}{0.880000in}}{\pgfqpoint{6.160000in}{6.160000in}}%
\pgfusepath{clip}%
\pgfsetbuttcap%
\pgfsetroundjoin%
\definecolor{currentfill}{rgb}{0.554312,0.690097,0.995516}%
\pgfsetfillcolor{currentfill}%
\pgfsetlinewidth{0.000000pt}%
\definecolor{currentstroke}{rgb}{0.000000,0.000000,0.000000}%
\pgfsetstrokecolor{currentstroke}%
\pgfsetdash{}{0pt}%
\pgfpathmoveto{\pgfqpoint{4.696861in}{3.980745in}}%
\pgfpathlineto{\pgfqpoint{4.707161in}{3.968608in}}%
\pgfpathlineto{\pgfqpoint{4.717479in}{3.956282in}}%
\pgfpathlineto{\pgfqpoint{4.748867in}{3.954196in}}%
\pgfpathlineto{\pgfqpoint{4.780235in}{3.952187in}}%
\pgfpathlineto{\pgfqpoint{4.769869in}{3.964472in}}%
\pgfpathlineto{\pgfqpoint{4.759520in}{3.976573in}}%
\pgfpathlineto{\pgfqpoint{4.728201in}{3.978623in}}%
\pgfpathlineto{\pgfqpoint{4.696861in}{3.980745in}}%
\pgfpathclose%
\pgfusepath{fill}%
\end{pgfscope}%
\begin{pgfscope}%
\pgfpathrectangle{\pgfqpoint{1.020000in}{0.880000in}}{\pgfqpoint{6.160000in}{6.160000in}}%
\pgfusepath{clip}%
\pgfsetbuttcap%
\pgfsetroundjoin%
\definecolor{currentfill}{rgb}{0.494638,0.633022,0.978983}%
\pgfsetfillcolor{currentfill}%
\pgfsetlinewidth{0.000000pt}%
\definecolor{currentstroke}{rgb}{0.000000,0.000000,0.000000}%
\pgfsetstrokecolor{currentstroke}%
\pgfsetdash{}{0pt}%
\pgfpathmoveto{\pgfqpoint{5.260276in}{3.880571in}}%
\pgfpathlineto{\pgfqpoint{5.271068in}{3.867859in}}%
\pgfpathlineto{\pgfqpoint{5.281877in}{3.854996in}}%
\pgfpathlineto{\pgfqpoint{5.313091in}{3.853921in}}%
\pgfpathlineto{\pgfqpoint{5.344286in}{3.852883in}}%
\pgfpathlineto{\pgfqpoint{5.333429in}{3.865678in}}%
\pgfpathlineto{\pgfqpoint{5.322589in}{3.878325in}}%
\pgfpathlineto{\pgfqpoint{5.291442in}{3.879429in}}%
\pgfpathlineto{\pgfqpoint{5.260276in}{3.880571in}}%
\pgfpathclose%
\pgfusepath{fill}%
\end{pgfscope}%
\begin{pgfscope}%
\pgfpathrectangle{\pgfqpoint{1.020000in}{0.880000in}}{\pgfqpoint{6.160000in}{6.160000in}}%
\pgfusepath{clip}%
\pgfsetbuttcap%
\pgfsetroundjoin%
\definecolor{currentfill}{rgb}{0.581486,0.713451,0.998314}%
\pgfsetfillcolor{currentfill}%
\pgfsetlinewidth{0.000000pt}%
\definecolor{currentstroke}{rgb}{0.000000,0.000000,0.000000}%
\pgfsetstrokecolor{currentstroke}%
\pgfsetdash{}{0pt}%
\pgfpathmoveto{\pgfqpoint{4.488138in}{4.019244in}}%
\pgfpathlineto{\pgfqpoint{4.498258in}{4.007469in}}%
\pgfpathlineto{\pgfqpoint{4.508395in}{3.995496in}}%
\pgfpathlineto{\pgfqpoint{4.539857in}{3.992742in}}%
\pgfpathlineto{\pgfqpoint{4.571299in}{3.990124in}}%
\pgfpathlineto{\pgfqpoint{4.561113in}{4.002107in}}%
\pgfpathlineto{\pgfqpoint{4.550945in}{4.013894in}}%
\pgfpathlineto{\pgfqpoint{4.519552in}{4.016507in}}%
\pgfpathlineto{\pgfqpoint{4.488138in}{4.019244in}}%
\pgfpathclose%
\pgfusepath{fill}%
\end{pgfscope}%
\begin{pgfscope}%
\pgfpathrectangle{\pgfqpoint{1.020000in}{0.880000in}}{\pgfqpoint{6.160000in}{6.160000in}}%
\pgfusepath{clip}%
\pgfsetbuttcap%
\pgfsetroundjoin%
\definecolor{currentfill}{rgb}{0.516260,0.654498,0.986407}%
\pgfsetfillcolor{currentfill}%
\pgfsetlinewidth{0.000000pt}%
\definecolor{currentstroke}{rgb}{0.000000,0.000000,0.000000}%
\pgfsetstrokecolor{currentstroke}%
\pgfsetdash{}{0pt}%
\pgfpathmoveto{\pgfqpoint{5.051625in}{3.913616in}}%
\pgfpathlineto{\pgfqpoint{5.062237in}{3.901026in}}%
\pgfpathlineto{\pgfqpoint{5.072866in}{3.888273in}}%
\pgfpathlineto{\pgfqpoint{5.104151in}{3.886884in}}%
\pgfpathlineto{\pgfqpoint{5.135415in}{3.885538in}}%
\pgfpathlineto{\pgfqpoint{5.124738in}{3.898227in}}%
\pgfpathlineto{\pgfqpoint{5.114078in}{3.910754in}}%
\pgfpathlineto{\pgfqpoint{5.082862in}{3.912163in}}%
\pgfpathlineto{\pgfqpoint{5.051625in}{3.913616in}}%
\pgfpathclose%
\pgfusepath{fill}%
\end{pgfscope}%
\begin{pgfscope}%
\pgfpathrectangle{\pgfqpoint{1.020000in}{0.880000in}}{\pgfqpoint{6.160000in}{6.160000in}}%
\pgfusepath{clip}%
\pgfsetbuttcap%
\pgfsetroundjoin%
\definecolor{currentfill}{rgb}{0.538004,0.674902,0.991722}%
\pgfsetfillcolor{currentfill}%
\pgfsetlinewidth{0.000000pt}%
\definecolor{currentstroke}{rgb}{0.000000,0.000000,0.000000}%
\pgfsetstrokecolor{currentstroke}%
\pgfsetdash{}{0pt}%
\pgfpathmoveto{\pgfqpoint{4.842910in}{3.948375in}}%
\pgfpathlineto{\pgfqpoint{4.853342in}{3.935956in}}%
\pgfpathlineto{\pgfqpoint{4.863791in}{3.923360in}}%
\pgfpathlineto{\pgfqpoint{4.895147in}{3.921601in}}%
\pgfpathlineto{\pgfqpoint{4.926482in}{3.919900in}}%
\pgfpathlineto{\pgfqpoint{4.915985in}{3.932441in}}%
\pgfpathlineto{\pgfqpoint{4.905505in}{3.944807in}}%
\pgfpathlineto{\pgfqpoint{4.874217in}{3.946563in}}%
\pgfpathlineto{\pgfqpoint{4.842910in}{3.948375in}}%
\pgfpathclose%
\pgfusepath{fill}%
\end{pgfscope}%
\begin{pgfscope}%
\pgfpathrectangle{\pgfqpoint{1.020000in}{0.880000in}}{\pgfqpoint{6.160000in}{6.160000in}}%
\pgfusepath{clip}%
\pgfsetbuttcap%
\pgfsetroundjoin%
\definecolor{currentfill}{rgb}{0.608547,0.735725,0.999354}%
\pgfsetfillcolor{currentfill}%
\pgfsetlinewidth{0.000000pt}%
\definecolor{currentstroke}{rgb}{0.000000,0.000000,0.000000}%
\pgfsetstrokecolor{currentstroke}%
\pgfsetdash{}{0pt}%
\pgfpathmoveto{\pgfqpoint{4.279322in}{4.061691in}}%
\pgfpathlineto{\pgfqpoint{4.289261in}{4.050574in}}%
\pgfpathlineto{\pgfqpoint{4.299217in}{4.039263in}}%
\pgfpathlineto{\pgfqpoint{4.330757in}{4.035386in}}%
\pgfpathlineto{\pgfqpoint{4.362275in}{4.031760in}}%
\pgfpathlineto{\pgfqpoint{4.352270in}{4.043210in}}%
\pgfpathlineto{\pgfqpoint{4.342282in}{4.054462in}}%
\pgfpathlineto{\pgfqpoint{4.310813in}{4.057967in}}%
\pgfpathlineto{\pgfqpoint{4.279322in}{4.061691in}}%
\pgfpathclose%
\pgfusepath{fill}%
\end{pgfscope}%
\begin{pgfscope}%
\pgfpathrectangle{\pgfqpoint{1.020000in}{0.880000in}}{\pgfqpoint{6.160000in}{6.160000in}}%
\pgfusepath{clip}%
\pgfsetbuttcap%
\pgfsetroundjoin%
\definecolor{currentfill}{rgb}{0.565182,0.699438,0.996635}%
\pgfsetfillcolor{currentfill}%
\pgfsetlinewidth{0.000000pt}%
\definecolor{currentstroke}{rgb}{0.000000,0.000000,0.000000}%
\pgfsetstrokecolor{currentstroke}%
\pgfsetdash{}{0pt}%
\pgfpathmoveto{\pgfqpoint{4.634121in}{3.985239in}}%
\pgfpathlineto{\pgfqpoint{4.644373in}{3.973074in}}%
\pgfpathlineto{\pgfqpoint{4.654642in}{3.960718in}}%
\pgfpathlineto{\pgfqpoint{4.686070in}{3.958453in}}%
\pgfpathlineto{\pgfqpoint{4.717479in}{3.956282in}}%
\pgfpathlineto{\pgfqpoint{4.707161in}{3.968608in}}%
\pgfpathlineto{\pgfqpoint{4.696861in}{3.980745in}}%
\pgfpathlineto{\pgfqpoint{4.665501in}{3.982948in}}%
\pgfpathlineto{\pgfqpoint{4.634121in}{3.985239in}}%
\pgfpathclose%
\pgfusepath{fill}%
\end{pgfscope}%
\begin{pgfscope}%
\pgfpathrectangle{\pgfqpoint{1.020000in}{0.880000in}}{\pgfqpoint{6.160000in}{6.160000in}}%
\pgfusepath{clip}%
\pgfsetbuttcap%
\pgfsetroundjoin%
\definecolor{currentfill}{rgb}{0.500031,0.638508,0.981070}%
\pgfsetfillcolor{currentfill}%
\pgfsetlinewidth{0.000000pt}%
\definecolor{currentstroke}{rgb}{0.000000,0.000000,0.000000}%
\pgfsetstrokecolor{currentstroke}%
\pgfsetdash{}{0pt}%
\pgfpathmoveto{\pgfqpoint{5.197885in}{3.882974in}}%
\pgfpathlineto{\pgfqpoint{5.208628in}{3.870195in}}%
\pgfpathlineto{\pgfqpoint{5.219389in}{3.857263in}}%
\pgfpathlineto{\pgfqpoint{5.250643in}{3.856110in}}%
\pgfpathlineto{\pgfqpoint{5.281877in}{3.854996in}}%
\pgfpathlineto{\pgfqpoint{5.271068in}{3.867859in}}%
\pgfpathlineto{\pgfqpoint{5.260276in}{3.880571in}}%
\pgfpathlineto{\pgfqpoint{5.229090in}{3.881753in}}%
\pgfpathlineto{\pgfqpoint{5.197885in}{3.882974in}}%
\pgfpathclose%
\pgfusepath{fill}%
\end{pgfscope}%
\begin{pgfscope}%
\pgfpathrectangle{\pgfqpoint{1.020000in}{0.880000in}}{\pgfqpoint{6.160000in}{6.160000in}}%
\pgfusepath{clip}%
\pgfsetbuttcap%
\pgfsetroundjoin%
\definecolor{currentfill}{rgb}{0.521696,0.659599,0.987736}%
\pgfsetfillcolor{currentfill}%
\pgfsetlinewidth{0.000000pt}%
\definecolor{currentstroke}{rgb}{0.000000,0.000000,0.000000}%
\pgfsetstrokecolor{currentstroke}%
\pgfsetdash{}{0pt}%
\pgfpathmoveto{\pgfqpoint{4.989094in}{3.916660in}}%
\pgfpathlineto{\pgfqpoint{4.999657in}{3.904008in}}%
\pgfpathlineto{\pgfqpoint{5.010238in}{3.891190in}}%
\pgfpathlineto{\pgfqpoint{5.041562in}{3.889708in}}%
\pgfpathlineto{\pgfqpoint{5.072866in}{3.888273in}}%
\pgfpathlineto{\pgfqpoint{5.062237in}{3.901026in}}%
\pgfpathlineto{\pgfqpoint{5.051625in}{3.913616in}}%
\pgfpathlineto{\pgfqpoint{5.020370in}{3.915114in}}%
\pgfpathlineto{\pgfqpoint{4.989094in}{3.916660in}}%
\pgfpathclose%
\pgfusepath{fill}%
\end{pgfscope}%
\begin{pgfscope}%
\pgfpathrectangle{\pgfqpoint{1.020000in}{0.880000in}}{\pgfqpoint{6.160000in}{6.160000in}}%
\pgfusepath{clip}%
\pgfsetbuttcap%
\pgfsetroundjoin%
\definecolor{currentfill}{rgb}{0.586921,0.718121,0.998874}%
\pgfsetfillcolor{currentfill}%
\pgfsetlinewidth{0.000000pt}%
\definecolor{currentstroke}{rgb}{0.000000,0.000000,0.000000}%
\pgfsetstrokecolor{currentstroke}%
\pgfsetdash{}{0pt}%
\pgfpathmoveto{\pgfqpoint{4.425248in}{4.025150in}}%
\pgfpathlineto{\pgfqpoint{4.435320in}{4.013417in}}%
\pgfpathlineto{\pgfqpoint{4.445408in}{4.001486in}}%
\pgfpathlineto{\pgfqpoint{4.476912in}{3.998404in}}%
\pgfpathlineto{\pgfqpoint{4.508395in}{3.995496in}}%
\pgfpathlineto{\pgfqpoint{4.498258in}{4.007469in}}%
\pgfpathlineto{\pgfqpoint{4.488138in}{4.019244in}}%
\pgfpathlineto{\pgfqpoint{4.456704in}{4.022119in}}%
\pgfpathlineto{\pgfqpoint{4.425248in}{4.025150in}}%
\pgfpathclose%
\pgfusepath{fill}%
\end{pgfscope}%
\begin{pgfscope}%
\pgfpathrectangle{\pgfqpoint{1.020000in}{0.880000in}}{\pgfqpoint{6.160000in}{6.160000in}}%
\pgfusepath{clip}%
\pgfsetbuttcap%
\pgfsetroundjoin%
\definecolor{currentfill}{rgb}{0.543440,0.680003,0.993051}%
\pgfsetfillcolor{currentfill}%
\pgfsetlinewidth{0.000000pt}%
\definecolor{currentstroke}{rgb}{0.000000,0.000000,0.000000}%
\pgfsetstrokecolor{currentstroke}%
\pgfsetdash{}{0pt}%
\pgfpathmoveto{\pgfqpoint{4.780235in}{3.952187in}}%
\pgfpathlineto{\pgfqpoint{4.790618in}{3.939719in}}%
\pgfpathlineto{\pgfqpoint{4.801019in}{3.927070in}}%
\pgfpathlineto{\pgfqpoint{4.832415in}{3.925181in}}%
\pgfpathlineto{\pgfqpoint{4.863791in}{3.923360in}}%
\pgfpathlineto{\pgfqpoint{4.853342in}{3.935956in}}%
\pgfpathlineto{\pgfqpoint{4.842910in}{3.948375in}}%
\pgfpathlineto{\pgfqpoint{4.811582in}{3.950249in}}%
\pgfpathlineto{\pgfqpoint{4.780235in}{3.952187in}}%
\pgfpathclose%
\pgfusepath{fill}%
\end{pgfscope}%
\begin{pgfscope}%
\pgfpathrectangle{\pgfqpoint{1.020000in}{0.880000in}}{\pgfqpoint{6.160000in}{6.160000in}}%
\pgfusepath{clip}%
\pgfsetbuttcap%
\pgfsetroundjoin%
\definecolor{currentfill}{rgb}{0.619318,0.744121,0.998931}%
\pgfsetfillcolor{currentfill}%
\pgfsetlinewidth{0.000000pt}%
\definecolor{currentstroke}{rgb}{0.000000,0.000000,0.000000}%
\pgfsetstrokecolor{currentstroke}%
\pgfsetdash{}{0pt}%
\pgfpathmoveto{\pgfqpoint{4.216276in}{4.069905in}}%
\pgfpathlineto{\pgfqpoint{4.226165in}{4.058996in}}%
\pgfpathlineto{\pgfqpoint{4.236073in}{4.047897in}}%
\pgfpathlineto{\pgfqpoint{4.267656in}{4.043422in}}%
\pgfpathlineto{\pgfqpoint{4.299217in}{4.039263in}}%
\pgfpathlineto{\pgfqpoint{4.289261in}{4.050574in}}%
\pgfpathlineto{\pgfqpoint{4.279322in}{4.061691in}}%
\pgfpathlineto{\pgfqpoint{4.247810in}{4.065661in}}%
\pgfpathlineto{\pgfqpoint{4.216276in}{4.069905in}}%
\pgfpathclose%
\pgfusepath{fill}%
\end{pgfscope}%
\begin{pgfscope}%
\pgfpathrectangle{\pgfqpoint{1.020000in}{0.880000in}}{\pgfqpoint{6.160000in}{6.160000in}}%
\pgfusepath{clip}%
\pgfsetbuttcap%
\pgfsetroundjoin%
\definecolor{currentfill}{rgb}{0.570616,0.704109,0.997195}%
\pgfsetfillcolor{currentfill}%
\pgfsetlinewidth{0.000000pt}%
\definecolor{currentstroke}{rgb}{0.000000,0.000000,0.000000}%
\pgfsetstrokecolor{currentstroke}%
\pgfsetdash{}{0pt}%
\pgfpathmoveto{\pgfqpoint{4.571299in}{3.990124in}}%
\pgfpathlineto{\pgfqpoint{4.581502in}{3.977946in}}%
\pgfpathlineto{\pgfqpoint{4.591723in}{3.965574in}}%
\pgfpathlineto{\pgfqpoint{4.623192in}{3.963088in}}%
\pgfpathlineto{\pgfqpoint{4.654642in}{3.960718in}}%
\pgfpathlineto{\pgfqpoint{4.644373in}{3.973074in}}%
\pgfpathlineto{\pgfqpoint{4.634121in}{3.985239in}}%
\pgfpathlineto{\pgfqpoint{4.602720in}{3.987627in}}%
\pgfpathlineto{\pgfqpoint{4.571299in}{3.990124in}}%
\pgfpathclose%
\pgfusepath{fill}%
\end{pgfscope}%
\begin{pgfscope}%
\pgfpathrectangle{\pgfqpoint{1.020000in}{0.880000in}}{\pgfqpoint{6.160000in}{6.160000in}}%
\pgfusepath{clip}%
\pgfsetbuttcap%
\pgfsetroundjoin%
\definecolor{currentfill}{rgb}{0.489246,0.627536,0.976896}%
\pgfsetfillcolor{currentfill}%
\pgfsetlinewidth{0.000000pt}%
\definecolor{currentstroke}{rgb}{0.000000,0.000000,0.000000}%
\pgfsetstrokecolor{currentstroke}%
\pgfsetdash{}{0pt}%
\pgfpathmoveto{\pgfqpoint{5.344286in}{3.852883in}}%
\pgfpathlineto{\pgfqpoint{5.355161in}{3.839945in}}%
\pgfpathlineto{\pgfqpoint{5.366054in}{3.826865in}}%
\pgfpathlineto{\pgfqpoint{5.397277in}{3.825935in}}%
\pgfpathlineto{\pgfqpoint{5.386360in}{3.838979in}}%
\pgfpathlineto{\pgfqpoint{5.375461in}{3.851883in}}%
\pgfpathlineto{\pgfqpoint{5.344286in}{3.852883in}}%
\pgfpathclose%
\pgfusepath{fill}%
\end{pgfscope}%
\begin{pgfscope}%
\pgfpathrectangle{\pgfqpoint{1.020000in}{0.880000in}}{\pgfqpoint{6.160000in}{6.160000in}}%
\pgfusepath{clip}%
\pgfsetbuttcap%
\pgfsetroundjoin%
\definecolor{currentfill}{rgb}{0.505423,0.643995,0.983157}%
\pgfsetfillcolor{currentfill}%
\pgfsetlinewidth{0.000000pt}%
\definecolor{currentstroke}{rgb}{0.000000,0.000000,0.000000}%
\pgfsetstrokecolor{currentstroke}%
\pgfsetdash{}{0pt}%
\pgfpathmoveto{\pgfqpoint{5.135415in}{3.885538in}}%
\pgfpathlineto{\pgfqpoint{5.146110in}{3.872692in}}%
\pgfpathlineto{\pgfqpoint{5.156823in}{3.859690in}}%
\pgfpathlineto{\pgfqpoint{5.188116in}{3.858456in}}%
\pgfpathlineto{\pgfqpoint{5.219389in}{3.857263in}}%
\pgfpathlineto{\pgfqpoint{5.208628in}{3.870195in}}%
\pgfpathlineto{\pgfqpoint{5.197885in}{3.882974in}}%
\pgfpathlineto{\pgfqpoint{5.166660in}{3.884235in}}%
\pgfpathlineto{\pgfqpoint{5.135415in}{3.885538in}}%
\pgfpathclose%
\pgfusepath{fill}%
\end{pgfscope}%
\begin{pgfscope}%
\pgfpathrectangle{\pgfqpoint{1.020000in}{0.880000in}}{\pgfqpoint{6.160000in}{6.160000in}}%
\pgfusepath{clip}%
\pgfsetbuttcap%
\pgfsetroundjoin%
\definecolor{currentfill}{rgb}{0.527132,0.664700,0.989065}%
\pgfsetfillcolor{currentfill}%
\pgfsetlinewidth{0.000000pt}%
\definecolor{currentstroke}{rgb}{0.000000,0.000000,0.000000}%
\pgfsetstrokecolor{currentstroke}%
\pgfsetdash{}{0pt}%
\pgfpathmoveto{\pgfqpoint{4.926482in}{3.919900in}}%
\pgfpathlineto{\pgfqpoint{4.936997in}{3.907188in}}%
\pgfpathlineto{\pgfqpoint{4.947530in}{3.894304in}}%
\pgfpathlineto{\pgfqpoint{4.978894in}{3.892721in}}%
\pgfpathlineto{\pgfqpoint{5.010238in}{3.891190in}}%
\pgfpathlineto{\pgfqpoint{4.999657in}{3.904008in}}%
\pgfpathlineto{\pgfqpoint{4.989094in}{3.916660in}}%
\pgfpathlineto{\pgfqpoint{4.957798in}{3.918254in}}%
\pgfpathlineto{\pgfqpoint{4.926482in}{3.919900in}}%
\pgfpathclose%
\pgfusepath{fill}%
\end{pgfscope}%
\begin{pgfscope}%
\pgfpathrectangle{\pgfqpoint{1.020000in}{0.880000in}}{\pgfqpoint{6.160000in}{6.160000in}}%
\pgfusepath{clip}%
\pgfsetbuttcap%
\pgfsetroundjoin%
\definecolor{currentfill}{rgb}{0.597777,0.727330,0.999777}%
\pgfsetfillcolor{currentfill}%
\pgfsetlinewidth{0.000000pt}%
\definecolor{currentstroke}{rgb}{0.000000,0.000000,0.000000}%
\pgfsetstrokecolor{currentstroke}%
\pgfsetdash{}{0pt}%
\pgfpathmoveto{\pgfqpoint{4.362275in}{4.031760in}}%
\pgfpathlineto{\pgfqpoint{4.372298in}{4.020111in}}%
\pgfpathlineto{\pgfqpoint{4.382338in}{4.008264in}}%
\pgfpathlineto{\pgfqpoint{4.413884in}{4.004764in}}%
\pgfpathlineto{\pgfqpoint{4.445408in}{4.001486in}}%
\pgfpathlineto{\pgfqpoint{4.435320in}{4.013417in}}%
\pgfpathlineto{\pgfqpoint{4.425248in}{4.025150in}}%
\pgfpathlineto{\pgfqpoint{4.393772in}{4.028356in}}%
\pgfpathlineto{\pgfqpoint{4.362275in}{4.031760in}}%
\pgfpathclose%
\pgfusepath{fill}%
\end{pgfscope}%
\begin{pgfscope}%
\pgfpathrectangle{\pgfqpoint{1.020000in}{0.880000in}}{\pgfqpoint{6.160000in}{6.160000in}}%
\pgfusepath{clip}%
\pgfsetbuttcap%
\pgfsetroundjoin%
\definecolor{currentfill}{rgb}{0.554312,0.690097,0.995516}%
\pgfsetfillcolor{currentfill}%
\pgfsetlinewidth{0.000000pt}%
\definecolor{currentstroke}{rgb}{0.000000,0.000000,0.000000}%
\pgfsetstrokecolor{currentstroke}%
\pgfsetdash{}{0pt}%
\pgfpathmoveto{\pgfqpoint{4.717479in}{3.956282in}}%
\pgfpathlineto{\pgfqpoint{4.727814in}{3.943771in}}%
\pgfpathlineto{\pgfqpoint{4.738166in}{3.931076in}}%
\pgfpathlineto{\pgfqpoint{4.769602in}{3.929032in}}%
\pgfpathlineto{\pgfqpoint{4.801019in}{3.927070in}}%
\pgfpathlineto{\pgfqpoint{4.790618in}{3.939719in}}%
\pgfpathlineto{\pgfqpoint{4.780235in}{3.952187in}}%
\pgfpathlineto{\pgfqpoint{4.748867in}{3.954196in}}%
\pgfpathlineto{\pgfqpoint{4.717479in}{3.956282in}}%
\pgfpathclose%
\pgfusepath{fill}%
\end{pgfscope}%
\begin{pgfscope}%
\pgfpathrectangle{\pgfqpoint{1.020000in}{0.880000in}}{\pgfqpoint{6.160000in}{6.160000in}}%
\pgfusepath{clip}%
\pgfsetbuttcap%
\pgfsetroundjoin%
\definecolor{currentfill}{rgb}{0.489246,0.627536,0.976896}%
\pgfsetfillcolor{currentfill}%
\pgfsetlinewidth{0.000000pt}%
\definecolor{currentstroke}{rgb}{0.000000,0.000000,0.000000}%
\pgfsetstrokecolor{currentstroke}%
\pgfsetdash{}{0pt}%
\pgfpathmoveto{\pgfqpoint{5.281877in}{3.854996in}}%
\pgfpathlineto{\pgfqpoint{5.292704in}{3.841987in}}%
\pgfpathlineto{\pgfqpoint{5.303548in}{3.828833in}}%
\pgfpathlineto{\pgfqpoint{5.334811in}{3.827830in}}%
\pgfpathlineto{\pgfqpoint{5.366054in}{3.826865in}}%
\pgfpathlineto{\pgfqpoint{5.355161in}{3.839945in}}%
\pgfpathlineto{\pgfqpoint{5.344286in}{3.852883in}}%
\pgfpathlineto{\pgfqpoint{5.313091in}{3.853921in}}%
\pgfpathlineto{\pgfqpoint{5.281877in}{3.854996in}}%
\pgfpathclose%
\pgfusepath{fill}%
\end{pgfscope}%
\begin{pgfscope}%
\pgfpathrectangle{\pgfqpoint{1.020000in}{0.880000in}}{\pgfqpoint{6.160000in}{6.160000in}}%
\pgfusepath{clip}%
\pgfsetbuttcap%
\pgfsetroundjoin%
\definecolor{currentfill}{rgb}{0.624703,0.748318,0.998719}%
\pgfsetfillcolor{currentfill}%
\pgfsetlinewidth{0.000000pt}%
\definecolor{currentstroke}{rgb}{0.000000,0.000000,0.000000}%
\pgfsetstrokecolor{currentstroke}%
\pgfsetdash{}{0pt}%
\pgfpathmoveto{\pgfqpoint{4.153141in}{4.079349in}}%
\pgfpathlineto{\pgfqpoint{4.162982in}{4.068741in}}%
\pgfpathlineto{\pgfqpoint{4.172840in}{4.057954in}}%
\pgfpathlineto{\pgfqpoint{4.204467in}{4.052728in}}%
\pgfpathlineto{\pgfqpoint{4.236073in}{4.047897in}}%
\pgfpathlineto{\pgfqpoint{4.226165in}{4.058996in}}%
\pgfpathlineto{\pgfqpoint{4.216276in}{4.069905in}}%
\pgfpathlineto{\pgfqpoint{4.184720in}{4.074457in}}%
\pgfpathlineto{\pgfqpoint{4.153141in}{4.079349in}}%
\pgfpathclose%
\pgfusepath{fill}%
\end{pgfscope}%
\begin{pgfscope}%
\pgfpathrectangle{\pgfqpoint{1.020000in}{0.880000in}}{\pgfqpoint{6.160000in}{6.160000in}}%
\pgfusepath{clip}%
\pgfsetbuttcap%
\pgfsetroundjoin%
\definecolor{currentfill}{rgb}{0.576051,0.708780,0.997755}%
\pgfsetfillcolor{currentfill}%
\pgfsetlinewidth{0.000000pt}%
\definecolor{currentstroke}{rgb}{0.000000,0.000000,0.000000}%
\pgfsetstrokecolor{currentstroke}%
\pgfsetdash{}{0pt}%
\pgfpathmoveto{\pgfqpoint{4.508395in}{3.995496in}}%
\pgfpathlineto{\pgfqpoint{4.518550in}{3.983327in}}%
\pgfpathlineto{\pgfqpoint{4.528722in}{3.970961in}}%
\pgfpathlineto{\pgfqpoint{4.560233in}{3.968193in}}%
\pgfpathlineto{\pgfqpoint{4.591723in}{3.965574in}}%
\pgfpathlineto{\pgfqpoint{4.581502in}{3.977946in}}%
\pgfpathlineto{\pgfqpoint{4.571299in}{3.990124in}}%
\pgfpathlineto{\pgfqpoint{4.539857in}{3.992742in}}%
\pgfpathlineto{\pgfqpoint{4.508395in}{3.995496in}}%
\pgfpathclose%
\pgfusepath{fill}%
\end{pgfscope}%
\begin{pgfscope}%
\pgfpathrectangle{\pgfqpoint{1.020000in}{0.880000in}}{\pgfqpoint{6.160000in}{6.160000in}}%
\pgfusepath{clip}%
\pgfsetbuttcap%
\pgfsetroundjoin%
\definecolor{currentfill}{rgb}{0.510824,0.649397,0.985079}%
\pgfsetfillcolor{currentfill}%
\pgfsetlinewidth{0.000000pt}%
\definecolor{currentstroke}{rgb}{0.000000,0.000000,0.000000}%
\pgfsetstrokecolor{currentstroke}%
\pgfsetdash{}{0pt}%
\pgfpathmoveto{\pgfqpoint{5.072866in}{3.888273in}}%
\pgfpathlineto{\pgfqpoint{5.083513in}{3.875359in}}%
\pgfpathlineto{\pgfqpoint{5.094177in}{3.862284in}}%
\pgfpathlineto{\pgfqpoint{5.125510in}{3.860965in}}%
\pgfpathlineto{\pgfqpoint{5.156823in}{3.859690in}}%
\pgfpathlineto{\pgfqpoint{5.146110in}{3.872692in}}%
\pgfpathlineto{\pgfqpoint{5.135415in}{3.885538in}}%
\pgfpathlineto{\pgfqpoint{5.104151in}{3.886884in}}%
\pgfpathlineto{\pgfqpoint{5.072866in}{3.888273in}}%
\pgfpathclose%
\pgfusepath{fill}%
\end{pgfscope}%
\begin{pgfscope}%
\pgfpathrectangle{\pgfqpoint{1.020000in}{0.880000in}}{\pgfqpoint{6.160000in}{6.160000in}}%
\pgfusepath{clip}%
\pgfsetbuttcap%
\pgfsetroundjoin%
\definecolor{currentfill}{rgb}{0.532568,0.669801,0.990393}%
\pgfsetfillcolor{currentfill}%
\pgfsetlinewidth{0.000000pt}%
\definecolor{currentstroke}{rgb}{0.000000,0.000000,0.000000}%
\pgfsetstrokecolor{currentstroke}%
\pgfsetdash{}{0pt}%
\pgfpathmoveto{\pgfqpoint{4.863791in}{3.923360in}}%
\pgfpathlineto{\pgfqpoint{4.874257in}{3.910588in}}%
\pgfpathlineto{\pgfqpoint{4.884741in}{3.897641in}}%
\pgfpathlineto{\pgfqpoint{4.916145in}{3.895943in}}%
\pgfpathlineto{\pgfqpoint{4.947530in}{3.894304in}}%
\pgfpathlineto{\pgfqpoint{4.936997in}{3.907188in}}%
\pgfpathlineto{\pgfqpoint{4.926482in}{3.919900in}}%
\pgfpathlineto{\pgfqpoint{4.895147in}{3.921601in}}%
\pgfpathlineto{\pgfqpoint{4.863791in}{3.923360in}}%
\pgfpathclose%
\pgfusepath{fill}%
\end{pgfscope}%
\begin{pgfscope}%
\pgfpathrectangle{\pgfqpoint{1.020000in}{0.880000in}}{\pgfqpoint{6.160000in}{6.160000in}}%
\pgfusepath{clip}%
\pgfsetbuttcap%
\pgfsetroundjoin%
\definecolor{currentfill}{rgb}{0.603162,0.731527,0.999565}%
\pgfsetfillcolor{currentfill}%
\pgfsetlinewidth{0.000000pt}%
\definecolor{currentstroke}{rgb}{0.000000,0.000000,0.000000}%
\pgfsetstrokecolor{currentstroke}%
\pgfsetdash{}{0pt}%
\pgfpathmoveto{\pgfqpoint{4.299217in}{4.039263in}}%
\pgfpathlineto{\pgfqpoint{4.309191in}{4.027756in}}%
\pgfpathlineto{\pgfqpoint{4.319182in}{4.016051in}}%
\pgfpathlineto{\pgfqpoint{4.350771in}{4.012015in}}%
\pgfpathlineto{\pgfqpoint{4.382338in}{4.008264in}}%
\pgfpathlineto{\pgfqpoint{4.372298in}{4.020111in}}%
\pgfpathlineto{\pgfqpoint{4.362275in}{4.031760in}}%
\pgfpathlineto{\pgfqpoint{4.330757in}{4.035386in}}%
\pgfpathlineto{\pgfqpoint{4.299217in}{4.039263in}}%
\pgfpathclose%
\pgfusepath{fill}%
\end{pgfscope}%
\begin{pgfscope}%
\pgfpathrectangle{\pgfqpoint{1.020000in}{0.880000in}}{\pgfqpoint{6.160000in}{6.160000in}}%
\pgfusepath{clip}%
\pgfsetbuttcap%
\pgfsetroundjoin%
\definecolor{currentfill}{rgb}{0.559747,0.694768,0.996075}%
\pgfsetfillcolor{currentfill}%
\pgfsetlinewidth{0.000000pt}%
\definecolor{currentstroke}{rgb}{0.000000,0.000000,0.000000}%
\pgfsetstrokecolor{currentstroke}%
\pgfsetdash{}{0pt}%
\pgfpathmoveto{\pgfqpoint{4.654642in}{3.960718in}}%
\pgfpathlineto{\pgfqpoint{4.664928in}{3.948173in}}%
\pgfpathlineto{\pgfqpoint{4.675232in}{3.935441in}}%
\pgfpathlineto{\pgfqpoint{4.706709in}{3.933208in}}%
\pgfpathlineto{\pgfqpoint{4.738166in}{3.931076in}}%
\pgfpathlineto{\pgfqpoint{4.727814in}{3.943771in}}%
\pgfpathlineto{\pgfqpoint{4.717479in}{3.956282in}}%
\pgfpathlineto{\pgfqpoint{4.686070in}{3.958453in}}%
\pgfpathlineto{\pgfqpoint{4.654642in}{3.960718in}}%
\pgfpathclose%
\pgfusepath{fill}%
\end{pgfscope}%
\begin{pgfscope}%
\pgfpathrectangle{\pgfqpoint{1.020000in}{0.880000in}}{\pgfqpoint{6.160000in}{6.160000in}}%
\pgfusepath{clip}%
\pgfsetbuttcap%
\pgfsetroundjoin%
\definecolor{currentfill}{rgb}{0.494638,0.633022,0.978983}%
\pgfsetfillcolor{currentfill}%
\pgfsetlinewidth{0.000000pt}%
\definecolor{currentstroke}{rgb}{0.000000,0.000000,0.000000}%
\pgfsetstrokecolor{currentstroke}%
\pgfsetdash{}{0pt}%
\pgfpathmoveto{\pgfqpoint{5.219389in}{3.857263in}}%
\pgfpathlineto{\pgfqpoint{5.230168in}{3.844181in}}%
\pgfpathlineto{\pgfqpoint{5.240964in}{3.830951in}}%
\pgfpathlineto{\pgfqpoint{5.272266in}{3.829873in}}%
\pgfpathlineto{\pgfqpoint{5.303548in}{3.828833in}}%
\pgfpathlineto{\pgfqpoint{5.292704in}{3.841987in}}%
\pgfpathlineto{\pgfqpoint{5.281877in}{3.854996in}}%
\pgfpathlineto{\pgfqpoint{5.250643in}{3.856110in}}%
\pgfpathlineto{\pgfqpoint{5.219389in}{3.857263in}}%
\pgfpathclose%
\pgfusepath{fill}%
\end{pgfscope}%
\begin{pgfscope}%
\pgfpathrectangle{\pgfqpoint{1.020000in}{0.880000in}}{\pgfqpoint{6.160000in}{6.160000in}}%
\pgfusepath{clip}%
\pgfsetbuttcap%
\pgfsetroundjoin%
\definecolor{currentfill}{rgb}{0.635474,0.756714,0.998297}%
\pgfsetfillcolor{currentfill}%
\pgfsetlinewidth{0.000000pt}%
\definecolor{currentstroke}{rgb}{0.000000,0.000000,0.000000}%
\pgfsetstrokecolor{currentstroke}%
\pgfsetdash{}{0pt}%
\pgfpathmoveto{\pgfqpoint{4.089916in}{4.090313in}}%
\pgfpathlineto{\pgfqpoint{4.099707in}{4.080127in}}%
\pgfpathlineto{\pgfqpoint{4.109515in}{4.069779in}}%
\pgfpathlineto{\pgfqpoint{4.141189in}{4.063622in}}%
\pgfpathlineto{\pgfqpoint{4.172840in}{4.057954in}}%
\pgfpathlineto{\pgfqpoint{4.162982in}{4.068741in}}%
\pgfpathlineto{\pgfqpoint{4.153141in}{4.079349in}}%
\pgfpathlineto{\pgfqpoint{4.121540in}{4.084621in}}%
\pgfpathlineto{\pgfqpoint{4.089916in}{4.090313in}}%
\pgfpathclose%
\pgfusepath{fill}%
\end{pgfscope}%
\begin{pgfscope}%
\pgfpathrectangle{\pgfqpoint{1.020000in}{0.880000in}}{\pgfqpoint{6.160000in}{6.160000in}}%
\pgfusepath{clip}%
\pgfsetbuttcap%
\pgfsetroundjoin%
\definecolor{currentfill}{rgb}{0.516260,0.654498,0.986407}%
\pgfsetfillcolor{currentfill}%
\pgfsetlinewidth{0.000000pt}%
\definecolor{currentstroke}{rgb}{0.000000,0.000000,0.000000}%
\pgfsetstrokecolor{currentstroke}%
\pgfsetdash{}{0pt}%
\pgfpathmoveto{\pgfqpoint{5.010238in}{3.891190in}}%
\pgfpathlineto{\pgfqpoint{5.020836in}{3.878206in}}%
\pgfpathlineto{\pgfqpoint{5.031452in}{3.865058in}}%
\pgfpathlineto{\pgfqpoint{5.062824in}{3.863648in}}%
\pgfpathlineto{\pgfqpoint{5.094177in}{3.862284in}}%
\pgfpathlineto{\pgfqpoint{5.083513in}{3.875359in}}%
\pgfpathlineto{\pgfqpoint{5.072866in}{3.888273in}}%
\pgfpathlineto{\pgfqpoint{5.041562in}{3.889708in}}%
\pgfpathlineto{\pgfqpoint{5.010238in}{3.891190in}}%
\pgfpathclose%
\pgfusepath{fill}%
\end{pgfscope}%
\begin{pgfscope}%
\pgfpathrectangle{\pgfqpoint{1.020000in}{0.880000in}}{\pgfqpoint{6.160000in}{6.160000in}}%
\pgfusepath{clip}%
\pgfsetbuttcap%
\pgfsetroundjoin%
\definecolor{currentfill}{rgb}{0.586921,0.718121,0.998874}%
\pgfsetfillcolor{currentfill}%
\pgfsetlinewidth{0.000000pt}%
\definecolor{currentstroke}{rgb}{0.000000,0.000000,0.000000}%
\pgfsetstrokecolor{currentstroke}%
\pgfsetdash{}{0pt}%
\pgfpathmoveto{\pgfqpoint{4.445408in}{4.001486in}}%
\pgfpathlineto{\pgfqpoint{4.455515in}{3.989356in}}%
\pgfpathlineto{\pgfqpoint{4.465638in}{3.977028in}}%
\pgfpathlineto{\pgfqpoint{4.497190in}{3.973899in}}%
\pgfpathlineto{\pgfqpoint{4.528722in}{3.970961in}}%
\pgfpathlineto{\pgfqpoint{4.518550in}{3.983327in}}%
\pgfpathlineto{\pgfqpoint{4.508395in}{3.995496in}}%
\pgfpathlineto{\pgfqpoint{4.476912in}{3.998404in}}%
\pgfpathlineto{\pgfqpoint{4.445408in}{4.001486in}}%
\pgfpathclose%
\pgfusepath{fill}%
\end{pgfscope}%
\begin{pgfscope}%
\pgfpathrectangle{\pgfqpoint{1.020000in}{0.880000in}}{\pgfqpoint{6.160000in}{6.160000in}}%
\pgfusepath{clip}%
\pgfsetbuttcap%
\pgfsetroundjoin%
\definecolor{currentfill}{rgb}{0.543440,0.680003,0.993051}%
\pgfsetfillcolor{currentfill}%
\pgfsetlinewidth{0.000000pt}%
\definecolor{currentstroke}{rgb}{0.000000,0.000000,0.000000}%
\pgfsetstrokecolor{currentstroke}%
\pgfsetdash{}{0pt}%
\pgfpathmoveto{\pgfqpoint{4.801019in}{3.927070in}}%
\pgfpathlineto{\pgfqpoint{4.811437in}{3.914242in}}%
\pgfpathlineto{\pgfqpoint{4.821872in}{3.901235in}}%
\pgfpathlineto{\pgfqpoint{4.853317in}{3.899403in}}%
\pgfpathlineto{\pgfqpoint{4.884741in}{3.897641in}}%
\pgfpathlineto{\pgfqpoint{4.874257in}{3.910588in}}%
\pgfpathlineto{\pgfqpoint{4.863791in}{3.923360in}}%
\pgfpathlineto{\pgfqpoint{4.832415in}{3.925181in}}%
\pgfpathlineto{\pgfqpoint{4.801019in}{3.927070in}}%
\pgfpathclose%
\pgfusepath{fill}%
\end{pgfscope}%
\begin{pgfscope}%
\pgfpathrectangle{\pgfqpoint{1.020000in}{0.880000in}}{\pgfqpoint{6.160000in}{6.160000in}}%
\pgfusepath{clip}%
\pgfsetbuttcap%
\pgfsetroundjoin%
\definecolor{currentfill}{rgb}{0.613933,0.739923,0.999142}%
\pgfsetfillcolor{currentfill}%
\pgfsetlinewidth{0.000000pt}%
\definecolor{currentstroke}{rgb}{0.000000,0.000000,0.000000}%
\pgfsetstrokecolor{currentstroke}%
\pgfsetdash{}{0pt}%
\pgfpathmoveto{\pgfqpoint{4.236073in}{4.047897in}}%
\pgfpathlineto{\pgfqpoint{4.245998in}{4.036609in}}%
\pgfpathlineto{\pgfqpoint{4.255940in}{4.025127in}}%
\pgfpathlineto{\pgfqpoint{4.287572in}{4.020409in}}%
\pgfpathlineto{\pgfqpoint{4.319182in}{4.016051in}}%
\pgfpathlineto{\pgfqpoint{4.309191in}{4.027756in}}%
\pgfpathlineto{\pgfqpoint{4.299217in}{4.039263in}}%
\pgfpathlineto{\pgfqpoint{4.267656in}{4.043422in}}%
\pgfpathlineto{\pgfqpoint{4.236073in}{4.047897in}}%
\pgfpathclose%
\pgfusepath{fill}%
\end{pgfscope}%
\begin{pgfscope}%
\pgfpathrectangle{\pgfqpoint{1.020000in}{0.880000in}}{\pgfqpoint{6.160000in}{6.160000in}}%
\pgfusepath{clip}%
\pgfsetbuttcap%
\pgfsetroundjoin%
\definecolor{currentfill}{rgb}{0.565182,0.699438,0.996635}%
\pgfsetfillcolor{currentfill}%
\pgfsetlinewidth{0.000000pt}%
\definecolor{currentstroke}{rgb}{0.000000,0.000000,0.000000}%
\pgfsetstrokecolor{currentstroke}%
\pgfsetdash{}{0pt}%
\pgfpathmoveto{\pgfqpoint{4.591723in}{3.965574in}}%
\pgfpathlineto{\pgfqpoint{4.601961in}{3.953011in}}%
\pgfpathlineto{\pgfqpoint{4.612216in}{3.940256in}}%
\pgfpathlineto{\pgfqpoint{4.643734in}{3.937786in}}%
\pgfpathlineto{\pgfqpoint{4.675232in}{3.935441in}}%
\pgfpathlineto{\pgfqpoint{4.664928in}{3.948173in}}%
\pgfpathlineto{\pgfqpoint{4.654642in}{3.960718in}}%
\pgfpathlineto{\pgfqpoint{4.623192in}{3.963088in}}%
\pgfpathlineto{\pgfqpoint{4.591723in}{3.965574in}}%
\pgfpathclose%
\pgfusepath{fill}%
\end{pgfscope}%
\begin{pgfscope}%
\pgfpathrectangle{\pgfqpoint{1.020000in}{0.880000in}}{\pgfqpoint{6.160000in}{6.160000in}}%
\pgfusepath{clip}%
\pgfsetbuttcap%
\pgfsetroundjoin%
\definecolor{currentfill}{rgb}{0.483854,0.622050,0.974808}%
\pgfsetfillcolor{currentfill}%
\pgfsetlinewidth{0.000000pt}%
\definecolor{currentstroke}{rgb}{0.000000,0.000000,0.000000}%
\pgfsetstrokecolor{currentstroke}%
\pgfsetdash{}{0pt}%
\pgfpathmoveto{\pgfqpoint{5.366054in}{3.826865in}}%
\pgfpathlineto{\pgfqpoint{5.376965in}{3.813646in}}%
\pgfpathlineto{\pgfqpoint{5.387893in}{3.800290in}}%
\pgfpathlineto{\pgfqpoint{5.419165in}{3.799438in}}%
\pgfpathlineto{\pgfqpoint{5.408212in}{3.812754in}}%
\pgfpathlineto{\pgfqpoint{5.397277in}{3.825935in}}%
\pgfpathlineto{\pgfqpoint{5.366054in}{3.826865in}}%
\pgfpathclose%
\pgfusepath{fill}%
\end{pgfscope}%
\begin{pgfscope}%
\pgfpathrectangle{\pgfqpoint{1.020000in}{0.880000in}}{\pgfqpoint{6.160000in}{6.160000in}}%
\pgfusepath{clip}%
\pgfsetbuttcap%
\pgfsetroundjoin%
\definecolor{currentfill}{rgb}{0.500031,0.638508,0.981070}%
\pgfsetfillcolor{currentfill}%
\pgfsetlinewidth{0.000000pt}%
\definecolor{currentstroke}{rgb}{0.000000,0.000000,0.000000}%
\pgfsetstrokecolor{currentstroke}%
\pgfsetdash{}{0pt}%
\pgfpathmoveto{\pgfqpoint{5.156823in}{3.859690in}}%
\pgfpathlineto{\pgfqpoint{5.167553in}{3.846533in}}%
\pgfpathlineto{\pgfqpoint{5.178300in}{3.833224in}}%
\pgfpathlineto{\pgfqpoint{5.209642in}{3.832068in}}%
\pgfpathlineto{\pgfqpoint{5.240964in}{3.830951in}}%
\pgfpathlineto{\pgfqpoint{5.230168in}{3.844181in}}%
\pgfpathlineto{\pgfqpoint{5.219389in}{3.857263in}}%
\pgfpathlineto{\pgfqpoint{5.188116in}{3.858456in}}%
\pgfpathlineto{\pgfqpoint{5.156823in}{3.859690in}}%
\pgfpathclose%
\pgfusepath{fill}%
\end{pgfscope}%
\begin{pgfscope}%
\pgfpathrectangle{\pgfqpoint{1.020000in}{0.880000in}}{\pgfqpoint{6.160000in}{6.160000in}}%
\pgfusepath{clip}%
\pgfsetbuttcap%
\pgfsetroundjoin%
\definecolor{currentfill}{rgb}{0.521696,0.659599,0.987736}%
\pgfsetfillcolor{currentfill}%
\pgfsetlinewidth{0.000000pt}%
\definecolor{currentstroke}{rgb}{0.000000,0.000000,0.000000}%
\pgfsetstrokecolor{currentstroke}%
\pgfsetdash{}{0pt}%
\pgfpathmoveto{\pgfqpoint{4.947530in}{3.894304in}}%
\pgfpathlineto{\pgfqpoint{4.958079in}{3.881252in}}%
\pgfpathlineto{\pgfqpoint{4.968646in}{3.868031in}}%
\pgfpathlineto{\pgfqpoint{5.000059in}{3.866518in}}%
\pgfpathlineto{\pgfqpoint{5.031452in}{3.865058in}}%
\pgfpathlineto{\pgfqpoint{5.020836in}{3.878206in}}%
\pgfpathlineto{\pgfqpoint{5.010238in}{3.891190in}}%
\pgfpathlineto{\pgfqpoint{4.978894in}{3.892721in}}%
\pgfpathlineto{\pgfqpoint{4.947530in}{3.894304in}}%
\pgfpathclose%
\pgfusepath{fill}%
\end{pgfscope}%
\begin{pgfscope}%
\pgfpathrectangle{\pgfqpoint{1.020000in}{0.880000in}}{\pgfqpoint{6.160000in}{6.160000in}}%
\pgfusepath{clip}%
\pgfsetbuttcap%
\pgfsetroundjoin%
\definecolor{currentfill}{rgb}{0.651398,0.768121,0.995891}%
\pgfsetfillcolor{currentfill}%
\pgfsetlinewidth{0.000000pt}%
\definecolor{currentstroke}{rgb}{0.000000,0.000000,0.000000}%
\pgfsetstrokecolor{currentstroke}%
\pgfsetdash{}{0pt}%
\pgfpathmoveto{\pgfqpoint{4.026597in}{4.103130in}}%
\pgfpathlineto{\pgfqpoint{4.036337in}{4.093519in}}%
\pgfpathlineto{\pgfqpoint{4.046095in}{4.083767in}}%
\pgfpathlineto{\pgfqpoint{4.077817in}{4.076476in}}%
\pgfpathlineto{\pgfqpoint{4.109515in}{4.069779in}}%
\pgfpathlineto{\pgfqpoint{4.099707in}{4.080127in}}%
\pgfpathlineto{\pgfqpoint{4.089916in}{4.090313in}}%
\pgfpathlineto{\pgfqpoint{4.058268in}{4.096468in}}%
\pgfpathlineto{\pgfqpoint{4.026597in}{4.103130in}}%
\pgfpathclose%
\pgfusepath{fill}%
\end{pgfscope}%
\begin{pgfscope}%
\pgfpathrectangle{\pgfqpoint{1.020000in}{0.880000in}}{\pgfqpoint{6.160000in}{6.160000in}}%
\pgfusepath{clip}%
\pgfsetbuttcap%
\pgfsetroundjoin%
\definecolor{currentfill}{rgb}{0.592356,0.722792,0.999434}%
\pgfsetfillcolor{currentfill}%
\pgfsetlinewidth{0.000000pt}%
\definecolor{currentstroke}{rgb}{0.000000,0.000000,0.000000}%
\pgfsetstrokecolor{currentstroke}%
\pgfsetdash{}{0pt}%
\pgfpathmoveto{\pgfqpoint{4.382338in}{4.008264in}}%
\pgfpathlineto{\pgfqpoint{4.392395in}{3.996217in}}%
\pgfpathlineto{\pgfqpoint{4.402470in}{3.983971in}}%
\pgfpathlineto{\pgfqpoint{4.434065in}{3.980376in}}%
\pgfpathlineto{\pgfqpoint{4.465638in}{3.977028in}}%
\pgfpathlineto{\pgfqpoint{4.455515in}{3.989356in}}%
\pgfpathlineto{\pgfqpoint{4.445408in}{4.001486in}}%
\pgfpathlineto{\pgfqpoint{4.413884in}{4.004764in}}%
\pgfpathlineto{\pgfqpoint{4.382338in}{4.008264in}}%
\pgfpathclose%
\pgfusepath{fill}%
\end{pgfscope}%
\begin{pgfscope}%
\pgfpathrectangle{\pgfqpoint{1.020000in}{0.880000in}}{\pgfqpoint{6.160000in}{6.160000in}}%
\pgfusepath{clip}%
\pgfsetbuttcap%
\pgfsetroundjoin%
\definecolor{currentfill}{rgb}{0.548876,0.685104,0.994379}%
\pgfsetfillcolor{currentfill}%
\pgfsetlinewidth{0.000000pt}%
\definecolor{currentstroke}{rgb}{0.000000,0.000000,0.000000}%
\pgfsetstrokecolor{currentstroke}%
\pgfsetdash{}{0pt}%
\pgfpathmoveto{\pgfqpoint{4.738166in}{3.931076in}}%
\pgfpathlineto{\pgfqpoint{4.748535in}{3.918197in}}%
\pgfpathlineto{\pgfqpoint{4.758922in}{3.905137in}}%
\pgfpathlineto{\pgfqpoint{4.790407in}{3.903144in}}%
\pgfpathlineto{\pgfqpoint{4.821872in}{3.901235in}}%
\pgfpathlineto{\pgfqpoint{4.811437in}{3.914242in}}%
\pgfpathlineto{\pgfqpoint{4.801019in}{3.927070in}}%
\pgfpathlineto{\pgfqpoint{4.769602in}{3.929032in}}%
\pgfpathlineto{\pgfqpoint{4.738166in}{3.931076in}}%
\pgfpathclose%
\pgfusepath{fill}%
\end{pgfscope}%
\begin{pgfscope}%
\pgfpathrectangle{\pgfqpoint{1.020000in}{0.880000in}}{\pgfqpoint{6.160000in}{6.160000in}}%
\pgfusepath{clip}%
\pgfsetbuttcap%
\pgfsetroundjoin%
\definecolor{currentfill}{rgb}{0.489246,0.627536,0.976896}%
\pgfsetfillcolor{currentfill}%
\pgfsetlinewidth{0.000000pt}%
\definecolor{currentstroke}{rgb}{0.000000,0.000000,0.000000}%
\pgfsetstrokecolor{currentstroke}%
\pgfsetdash{}{0pt}%
\pgfpathmoveto{\pgfqpoint{5.303548in}{3.828833in}}%
\pgfpathlineto{\pgfqpoint{5.314410in}{3.815537in}}%
\pgfpathlineto{\pgfqpoint{5.325290in}{3.802100in}}%
\pgfpathlineto{\pgfqpoint{5.356602in}{3.801177in}}%
\pgfpathlineto{\pgfqpoint{5.387893in}{3.800290in}}%
\pgfpathlineto{\pgfqpoint{5.376965in}{3.813646in}}%
\pgfpathlineto{\pgfqpoint{5.366054in}{3.826865in}}%
\pgfpathlineto{\pgfqpoint{5.334811in}{3.827830in}}%
\pgfpathlineto{\pgfqpoint{5.303548in}{3.828833in}}%
\pgfpathclose%
\pgfusepath{fill}%
\end{pgfscope}%
\begin{pgfscope}%
\pgfpathrectangle{\pgfqpoint{1.020000in}{0.880000in}}{\pgfqpoint{6.160000in}{6.160000in}}%
\pgfusepath{clip}%
\pgfsetbuttcap%
\pgfsetroundjoin%
\definecolor{currentfill}{rgb}{0.624703,0.748318,0.998719}%
\pgfsetfillcolor{currentfill}%
\pgfsetlinewidth{0.000000pt}%
\definecolor{currentstroke}{rgb}{0.000000,0.000000,0.000000}%
\pgfsetstrokecolor{currentstroke}%
\pgfsetdash{}{0pt}%
\pgfpathmoveto{\pgfqpoint{4.172840in}{4.057954in}}%
\pgfpathlineto{\pgfqpoint{4.182715in}{4.046986in}}%
\pgfpathlineto{\pgfqpoint{4.192609in}{4.035832in}}%
\pgfpathlineto{\pgfqpoint{4.224286in}{4.030252in}}%
\pgfpathlineto{\pgfqpoint{4.255940in}{4.025127in}}%
\pgfpathlineto{\pgfqpoint{4.245998in}{4.036609in}}%
\pgfpathlineto{\pgfqpoint{4.236073in}{4.047897in}}%
\pgfpathlineto{\pgfqpoint{4.204467in}{4.052728in}}%
\pgfpathlineto{\pgfqpoint{4.172840in}{4.057954in}}%
\pgfpathclose%
\pgfusepath{fill}%
\end{pgfscope}%
\begin{pgfscope}%
\pgfpathrectangle{\pgfqpoint{1.020000in}{0.880000in}}{\pgfqpoint{6.160000in}{6.160000in}}%
\pgfusepath{clip}%
\pgfsetbuttcap%
\pgfsetroundjoin%
\definecolor{currentfill}{rgb}{0.570616,0.704109,0.997195}%
\pgfsetfillcolor{currentfill}%
\pgfsetlinewidth{0.000000pt}%
\definecolor{currentstroke}{rgb}{0.000000,0.000000,0.000000}%
\pgfsetstrokecolor{currentstroke}%
\pgfsetdash{}{0pt}%
\pgfpathmoveto{\pgfqpoint{4.528722in}{3.970961in}}%
\pgfpathlineto{\pgfqpoint{4.538911in}{3.958401in}}%
\pgfpathlineto{\pgfqpoint{4.549118in}{3.945646in}}%
\pgfpathlineto{\pgfqpoint{4.580677in}{3.942870in}}%
\pgfpathlineto{\pgfqpoint{4.612216in}{3.940256in}}%
\pgfpathlineto{\pgfqpoint{4.601961in}{3.953011in}}%
\pgfpathlineto{\pgfqpoint{4.591723in}{3.965574in}}%
\pgfpathlineto{\pgfqpoint{4.560233in}{3.968193in}}%
\pgfpathlineto{\pgfqpoint{4.528722in}{3.970961in}}%
\pgfpathclose%
\pgfusepath{fill}%
\end{pgfscope}%
\begin{pgfscope}%
\pgfpathrectangle{\pgfqpoint{1.020000in}{0.880000in}}{\pgfqpoint{6.160000in}{6.160000in}}%
\pgfusepath{clip}%
\pgfsetbuttcap%
\pgfsetroundjoin%
\definecolor{currentfill}{rgb}{0.505423,0.643995,0.983157}%
\pgfsetfillcolor{currentfill}%
\pgfsetlinewidth{0.000000pt}%
\definecolor{currentstroke}{rgb}{0.000000,0.000000,0.000000}%
\pgfsetstrokecolor{currentstroke}%
\pgfsetdash{}{0pt}%
\pgfpathmoveto{\pgfqpoint{5.094177in}{3.862284in}}%
\pgfpathlineto{\pgfqpoint{5.104859in}{3.849051in}}%
\pgfpathlineto{\pgfqpoint{5.115558in}{3.835659in}}%
\pgfpathlineto{\pgfqpoint{5.146939in}{3.834420in}}%
\pgfpathlineto{\pgfqpoint{5.178300in}{3.833224in}}%
\pgfpathlineto{\pgfqpoint{5.167553in}{3.846533in}}%
\pgfpathlineto{\pgfqpoint{5.156823in}{3.859690in}}%
\pgfpathlineto{\pgfqpoint{5.125510in}{3.860965in}}%
\pgfpathlineto{\pgfqpoint{5.094177in}{3.862284in}}%
\pgfpathclose%
\pgfusepath{fill}%
\end{pgfscope}%
\begin{pgfscope}%
\pgfpathrectangle{\pgfqpoint{1.020000in}{0.880000in}}{\pgfqpoint{6.160000in}{6.160000in}}%
\pgfusepath{clip}%
\pgfsetbuttcap%
\pgfsetroundjoin%
\definecolor{currentfill}{rgb}{0.532568,0.669801,0.990393}%
\pgfsetfillcolor{currentfill}%
\pgfsetlinewidth{0.000000pt}%
\definecolor{currentstroke}{rgb}{0.000000,0.000000,0.000000}%
\pgfsetstrokecolor{currentstroke}%
\pgfsetdash{}{0pt}%
\pgfpathmoveto{\pgfqpoint{4.884741in}{3.897641in}}%
\pgfpathlineto{\pgfqpoint{4.895242in}{3.884521in}}%
\pgfpathlineto{\pgfqpoint{4.905761in}{3.871227in}}%
\pgfpathlineto{\pgfqpoint{4.937214in}{3.869599in}}%
\pgfpathlineto{\pgfqpoint{4.968646in}{3.868031in}}%
\pgfpathlineto{\pgfqpoint{4.958079in}{3.881252in}}%
\pgfpathlineto{\pgfqpoint{4.947530in}{3.894304in}}%
\pgfpathlineto{\pgfqpoint{4.916145in}{3.895943in}}%
\pgfpathlineto{\pgfqpoint{4.884741in}{3.897641in}}%
\pgfpathclose%
\pgfusepath{fill}%
\end{pgfscope}%
\begin{pgfscope}%
\pgfpathrectangle{\pgfqpoint{1.020000in}{0.880000in}}{\pgfqpoint{6.160000in}{6.160000in}}%
\pgfusepath{clip}%
\pgfsetbuttcap%
\pgfsetroundjoin%
\definecolor{currentfill}{rgb}{0.603162,0.731527,0.999565}%
\pgfsetfillcolor{currentfill}%
\pgfsetlinewidth{0.000000pt}%
\definecolor{currentstroke}{rgb}{0.000000,0.000000,0.000000}%
\pgfsetstrokecolor{currentstroke}%
\pgfsetdash{}{0pt}%
\pgfpathmoveto{\pgfqpoint{4.319182in}{4.016051in}}%
\pgfpathlineto{\pgfqpoint{4.329191in}{4.004149in}}%
\pgfpathlineto{\pgfqpoint{4.339218in}{3.992046in}}%
\pgfpathlineto{\pgfqpoint{4.370855in}{3.987849in}}%
\pgfpathlineto{\pgfqpoint{4.402470in}{3.983971in}}%
\pgfpathlineto{\pgfqpoint{4.392395in}{3.996217in}}%
\pgfpathlineto{\pgfqpoint{4.382338in}{4.008264in}}%
\pgfpathlineto{\pgfqpoint{4.350771in}{4.012015in}}%
\pgfpathlineto{\pgfqpoint{4.319182in}{4.016051in}}%
\pgfpathclose%
\pgfusepath{fill}%
\end{pgfscope}%
\begin{pgfscope}%
\pgfpathrectangle{\pgfqpoint{1.020000in}{0.880000in}}{\pgfqpoint{6.160000in}{6.160000in}}%
\pgfusepath{clip}%
\pgfsetbuttcap%
\pgfsetroundjoin%
\definecolor{currentfill}{rgb}{0.661968,0.775491,0.993937}%
\pgfsetfillcolor{currentfill}%
\pgfsetlinewidth{0.000000pt}%
\definecolor{currentstroke}{rgb}{0.000000,0.000000,0.000000}%
\pgfsetstrokecolor{currentstroke}%
\pgfsetdash{}{0pt}%
\pgfpathmoveto{\pgfqpoint{3.963180in}{4.118167in}}%
\pgfpathlineto{\pgfqpoint{3.972868in}{4.109317in}}%
\pgfpathlineto{\pgfqpoint{3.982575in}{4.100355in}}%
\pgfpathlineto{\pgfqpoint{4.014348in}{4.091708in}}%
\pgfpathlineto{\pgfqpoint{4.046095in}{4.083767in}}%
\pgfpathlineto{\pgfqpoint{4.036337in}{4.093519in}}%
\pgfpathlineto{\pgfqpoint{4.026597in}{4.103130in}}%
\pgfpathlineto{\pgfqpoint{3.994901in}{4.110347in}}%
\pgfpathlineto{\pgfqpoint{3.963180in}{4.118167in}}%
\pgfpathclose%
\pgfusepath{fill}%
\end{pgfscope}%
\begin{pgfscope}%
\pgfpathrectangle{\pgfqpoint{1.020000in}{0.880000in}}{\pgfqpoint{6.160000in}{6.160000in}}%
\pgfusepath{clip}%
\pgfsetbuttcap%
\pgfsetroundjoin%
\definecolor{currentfill}{rgb}{0.554312,0.690097,0.995516}%
\pgfsetfillcolor{currentfill}%
\pgfsetlinewidth{0.000000pt}%
\definecolor{currentstroke}{rgb}{0.000000,0.000000,0.000000}%
\pgfsetstrokecolor{currentstroke}%
\pgfsetdash{}{0pt}%
\pgfpathmoveto{\pgfqpoint{4.675232in}{3.935441in}}%
\pgfpathlineto{\pgfqpoint{4.685553in}{3.922522in}}%
\pgfpathlineto{\pgfqpoint{4.695891in}{3.909417in}}%
\pgfpathlineto{\pgfqpoint{4.727417in}{3.907224in}}%
\pgfpathlineto{\pgfqpoint{4.758922in}{3.905137in}}%
\pgfpathlineto{\pgfqpoint{4.748535in}{3.918197in}}%
\pgfpathlineto{\pgfqpoint{4.738166in}{3.931076in}}%
\pgfpathlineto{\pgfqpoint{4.706709in}{3.933208in}}%
\pgfpathlineto{\pgfqpoint{4.675232in}{3.935441in}}%
\pgfpathclose%
\pgfusepath{fill}%
\end{pgfscope}%
\begin{pgfscope}%
\pgfpathrectangle{\pgfqpoint{1.020000in}{0.880000in}}{\pgfqpoint{6.160000in}{6.160000in}}%
\pgfusepath{clip}%
\pgfsetbuttcap%
\pgfsetroundjoin%
\definecolor{currentfill}{rgb}{0.489246,0.627536,0.976896}%
\pgfsetfillcolor{currentfill}%
\pgfsetlinewidth{0.000000pt}%
\definecolor{currentstroke}{rgb}{0.000000,0.000000,0.000000}%
\pgfsetstrokecolor{currentstroke}%
\pgfsetdash{}{0pt}%
\pgfpathmoveto{\pgfqpoint{5.240964in}{3.830951in}}%
\pgfpathlineto{\pgfqpoint{5.251778in}{3.817574in}}%
\pgfpathlineto{\pgfqpoint{5.262609in}{3.804051in}}%
\pgfpathlineto{\pgfqpoint{5.293960in}{3.803057in}}%
\pgfpathlineto{\pgfqpoint{5.325290in}{3.802100in}}%
\pgfpathlineto{\pgfqpoint{5.314410in}{3.815537in}}%
\pgfpathlineto{\pgfqpoint{5.303548in}{3.828833in}}%
\pgfpathlineto{\pgfqpoint{5.272266in}{3.829873in}}%
\pgfpathlineto{\pgfqpoint{5.240964in}{3.830951in}}%
\pgfpathclose%
\pgfusepath{fill}%
\end{pgfscope}%
\begin{pgfscope}%
\pgfpathrectangle{\pgfqpoint{1.020000in}{0.880000in}}{\pgfqpoint{6.160000in}{6.160000in}}%
\pgfusepath{clip}%
\pgfsetbuttcap%
\pgfsetroundjoin%
\definecolor{currentfill}{rgb}{0.510824,0.649397,0.985079}%
\pgfsetfillcolor{currentfill}%
\pgfsetlinewidth{0.000000pt}%
\definecolor{currentstroke}{rgb}{0.000000,0.000000,0.000000}%
\pgfsetstrokecolor{currentstroke}%
\pgfsetdash{}{0pt}%
\pgfpathmoveto{\pgfqpoint{5.031452in}{3.865058in}}%
\pgfpathlineto{\pgfqpoint{5.042085in}{3.851747in}}%
\pgfpathlineto{\pgfqpoint{5.052735in}{3.838271in}}%
\pgfpathlineto{\pgfqpoint{5.084156in}{3.836942in}}%
\pgfpathlineto{\pgfqpoint{5.115558in}{3.835659in}}%
\pgfpathlineto{\pgfqpoint{5.104859in}{3.849051in}}%
\pgfpathlineto{\pgfqpoint{5.094177in}{3.862284in}}%
\pgfpathlineto{\pgfqpoint{5.062824in}{3.863648in}}%
\pgfpathlineto{\pgfqpoint{5.031452in}{3.865058in}}%
\pgfpathclose%
\pgfusepath{fill}%
\end{pgfscope}%
\begin{pgfscope}%
\pgfpathrectangle{\pgfqpoint{1.020000in}{0.880000in}}{\pgfqpoint{6.160000in}{6.160000in}}%
\pgfusepath{clip}%
\pgfsetbuttcap%
\pgfsetroundjoin%
\definecolor{currentfill}{rgb}{0.635474,0.756714,0.998297}%
\pgfsetfillcolor{currentfill}%
\pgfsetlinewidth{0.000000pt}%
\definecolor{currentstroke}{rgb}{0.000000,0.000000,0.000000}%
\pgfsetstrokecolor{currentstroke}%
\pgfsetdash{}{0pt}%
\pgfpathmoveto{\pgfqpoint{4.109515in}{4.069779in}}%
\pgfpathlineto{\pgfqpoint{4.119341in}{4.059262in}}%
\pgfpathlineto{\pgfqpoint{4.129185in}{4.048569in}}%
\pgfpathlineto{\pgfqpoint{4.160908in}{4.041918in}}%
\pgfpathlineto{\pgfqpoint{4.192609in}{4.035832in}}%
\pgfpathlineto{\pgfqpoint{4.182715in}{4.046986in}}%
\pgfpathlineto{\pgfqpoint{4.172840in}{4.057954in}}%
\pgfpathlineto{\pgfqpoint{4.141189in}{4.063622in}}%
\pgfpathlineto{\pgfqpoint{4.109515in}{4.069779in}}%
\pgfpathclose%
\pgfusepath{fill}%
\end{pgfscope}%
\begin{pgfscope}%
\pgfpathrectangle{\pgfqpoint{1.020000in}{0.880000in}}{\pgfqpoint{6.160000in}{6.160000in}}%
\pgfusepath{clip}%
\pgfsetbuttcap%
\pgfsetroundjoin%
\definecolor{currentfill}{rgb}{0.581486,0.713451,0.998314}%
\pgfsetfillcolor{currentfill}%
\pgfsetlinewidth{0.000000pt}%
\definecolor{currentstroke}{rgb}{0.000000,0.000000,0.000000}%
\pgfsetstrokecolor{currentstroke}%
\pgfsetdash{}{0pt}%
\pgfpathmoveto{\pgfqpoint{4.465638in}{3.977028in}}%
\pgfpathlineto{\pgfqpoint{4.475779in}{3.964503in}}%
\pgfpathlineto{\pgfqpoint{4.485937in}{3.951779in}}%
\pgfpathlineto{\pgfqpoint{4.517538in}{3.948607in}}%
\pgfpathlineto{\pgfqpoint{4.549118in}{3.945646in}}%
\pgfpathlineto{\pgfqpoint{4.538911in}{3.958401in}}%
\pgfpathlineto{\pgfqpoint{4.528722in}{3.970961in}}%
\pgfpathlineto{\pgfqpoint{4.497190in}{3.973899in}}%
\pgfpathlineto{\pgfqpoint{4.465638in}{3.977028in}}%
\pgfpathclose%
\pgfusepath{fill}%
\end{pgfscope}%
\begin{pgfscope}%
\pgfpathrectangle{\pgfqpoint{1.020000in}{0.880000in}}{\pgfqpoint{6.160000in}{6.160000in}}%
\pgfusepath{clip}%
\pgfsetbuttcap%
\pgfsetroundjoin%
\definecolor{currentfill}{rgb}{0.538004,0.674902,0.991722}%
\pgfsetfillcolor{currentfill}%
\pgfsetlinewidth{0.000000pt}%
\definecolor{currentstroke}{rgb}{0.000000,0.000000,0.000000}%
\pgfsetstrokecolor{currentstroke}%
\pgfsetdash{}{0pt}%
\pgfpathmoveto{\pgfqpoint{4.821872in}{3.901235in}}%
\pgfpathlineto{\pgfqpoint{4.832325in}{3.888051in}}%
\pgfpathlineto{\pgfqpoint{4.842795in}{3.874688in}}%
\pgfpathlineto{\pgfqpoint{4.874288in}{3.872922in}}%
\pgfpathlineto{\pgfqpoint{4.905761in}{3.871227in}}%
\pgfpathlineto{\pgfqpoint{4.895242in}{3.884521in}}%
\pgfpathlineto{\pgfqpoint{4.884741in}{3.897641in}}%
\pgfpathlineto{\pgfqpoint{4.853317in}{3.899403in}}%
\pgfpathlineto{\pgfqpoint{4.821872in}{3.901235in}}%
\pgfpathclose%
\pgfusepath{fill}%
\end{pgfscope}%
\begin{pgfscope}%
\pgfpathrectangle{\pgfqpoint{1.020000in}{0.880000in}}{\pgfqpoint{6.160000in}{6.160000in}}%
\pgfusepath{clip}%
\pgfsetbuttcap%
\pgfsetroundjoin%
\definecolor{currentfill}{rgb}{0.613933,0.739923,0.999142}%
\pgfsetfillcolor{currentfill}%
\pgfsetlinewidth{0.000000pt}%
\definecolor{currentstroke}{rgb}{0.000000,0.000000,0.000000}%
\pgfsetstrokecolor{currentstroke}%
\pgfsetdash{}{0pt}%
\pgfpathmoveto{\pgfqpoint{4.255940in}{4.025127in}}%
\pgfpathlineto{\pgfqpoint{4.265900in}{4.013449in}}%
\pgfpathlineto{\pgfqpoint{4.275878in}{4.001571in}}%
\pgfpathlineto{\pgfqpoint{4.307559in}{3.996604in}}%
\pgfpathlineto{\pgfqpoint{4.339218in}{3.992046in}}%
\pgfpathlineto{\pgfqpoint{4.329191in}{4.004149in}}%
\pgfpathlineto{\pgfqpoint{4.319182in}{4.016051in}}%
\pgfpathlineto{\pgfqpoint{4.287572in}{4.020409in}}%
\pgfpathlineto{\pgfqpoint{4.255940in}{4.025127in}}%
\pgfpathclose%
\pgfusepath{fill}%
\end{pgfscope}%
\begin{pgfscope}%
\pgfpathrectangle{\pgfqpoint{1.020000in}{0.880000in}}{\pgfqpoint{6.160000in}{6.160000in}}%
\pgfusepath{clip}%
\pgfsetbuttcap%
\pgfsetroundjoin%
\definecolor{currentfill}{rgb}{0.478462,0.616564,0.972721}%
\pgfsetfillcolor{currentfill}%
\pgfsetlinewidth{0.000000pt}%
\definecolor{currentstroke}{rgb}{0.000000,0.000000,0.000000}%
\pgfsetstrokecolor{currentstroke}%
\pgfsetdash{}{0pt}%
\pgfpathmoveto{\pgfqpoint{5.387893in}{3.800290in}}%
\pgfpathlineto{\pgfqpoint{5.398839in}{3.786801in}}%
\pgfpathlineto{\pgfqpoint{5.409804in}{3.773179in}}%
\pgfpathlineto{\pgfqpoint{5.441124in}{3.772412in}}%
\pgfpathlineto{\pgfqpoint{5.430135in}{3.785990in}}%
\pgfpathlineto{\pgfqpoint{5.419165in}{3.799438in}}%
\pgfpathlineto{\pgfqpoint{5.387893in}{3.800290in}}%
\pgfpathclose%
\pgfusepath{fill}%
\end{pgfscope}%
\begin{pgfscope}%
\pgfpathrectangle{\pgfqpoint{1.020000in}{0.880000in}}{\pgfqpoint{6.160000in}{6.160000in}}%
\pgfusepath{clip}%
\pgfsetbuttcap%
\pgfsetroundjoin%
\definecolor{currentfill}{rgb}{0.559747,0.694768,0.996075}%
\pgfsetfillcolor{currentfill}%
\pgfsetlinewidth{0.000000pt}%
\definecolor{currentstroke}{rgb}{0.000000,0.000000,0.000000}%
\pgfsetstrokecolor{currentstroke}%
\pgfsetdash{}{0pt}%
\pgfpathmoveto{\pgfqpoint{4.612216in}{3.940256in}}%
\pgfpathlineto{\pgfqpoint{4.622488in}{3.927312in}}%
\pgfpathlineto{\pgfqpoint{4.632778in}{3.914177in}}%
\pgfpathlineto{\pgfqpoint{4.664345in}{3.911729in}}%
\pgfpathlineto{\pgfqpoint{4.695891in}{3.909417in}}%
\pgfpathlineto{\pgfqpoint{4.685553in}{3.922522in}}%
\pgfpathlineto{\pgfqpoint{4.675232in}{3.935441in}}%
\pgfpathlineto{\pgfqpoint{4.643734in}{3.937786in}}%
\pgfpathlineto{\pgfqpoint{4.612216in}{3.940256in}}%
\pgfpathclose%
\pgfusepath{fill}%
\end{pgfscope}%
\begin{pgfscope}%
\pgfpathrectangle{\pgfqpoint{1.020000in}{0.880000in}}{\pgfqpoint{6.160000in}{6.160000in}}%
\pgfusepath{clip}%
\pgfsetbuttcap%
\pgfsetroundjoin%
\definecolor{currentfill}{rgb}{0.494638,0.633022,0.978983}%
\pgfsetfillcolor{currentfill}%
\pgfsetlinewidth{0.000000pt}%
\definecolor{currentstroke}{rgb}{0.000000,0.000000,0.000000}%
\pgfsetstrokecolor{currentstroke}%
\pgfsetdash{}{0pt}%
\pgfpathmoveto{\pgfqpoint{5.178300in}{3.833224in}}%
\pgfpathlineto{\pgfqpoint{5.189066in}{3.819762in}}%
\pgfpathlineto{\pgfqpoint{5.199849in}{3.806148in}}%
\pgfpathlineto{\pgfqpoint{5.231239in}{3.805081in}}%
\pgfpathlineto{\pgfqpoint{5.262609in}{3.804051in}}%
\pgfpathlineto{\pgfqpoint{5.251778in}{3.817574in}}%
\pgfpathlineto{\pgfqpoint{5.240964in}{3.830951in}}%
\pgfpathlineto{\pgfqpoint{5.209642in}{3.832068in}}%
\pgfpathlineto{\pgfqpoint{5.178300in}{3.833224in}}%
\pgfpathclose%
\pgfusepath{fill}%
\end{pgfscope}%
\begin{pgfscope}%
\pgfpathrectangle{\pgfqpoint{1.020000in}{0.880000in}}{\pgfqpoint{6.160000in}{6.160000in}}%
\pgfusepath{clip}%
\pgfsetbuttcap%
\pgfsetroundjoin%
\definecolor{currentfill}{rgb}{0.677823,0.786546,0.991005}%
\pgfsetfillcolor{currentfill}%
\pgfsetlinewidth{0.000000pt}%
\definecolor{currentstroke}{rgb}{0.000000,0.000000,0.000000}%
\pgfsetstrokecolor{currentstroke}%
\pgfsetdash{}{0pt}%
\pgfpathmoveto{\pgfqpoint{3.899659in}{4.135809in}}%
\pgfpathlineto{\pgfqpoint{3.909295in}{4.127942in}}%
\pgfpathlineto{\pgfqpoint{3.918948in}{4.120002in}}%
\pgfpathlineto{\pgfqpoint{3.950775in}{4.109767in}}%
\pgfpathlineto{\pgfqpoint{3.982575in}{4.100355in}}%
\pgfpathlineto{\pgfqpoint{3.972868in}{4.109317in}}%
\pgfpathlineto{\pgfqpoint{3.963180in}{4.118167in}}%
\pgfpathlineto{\pgfqpoint{3.931433in}{4.126638in}}%
\pgfpathlineto{\pgfqpoint{3.899659in}{4.135809in}}%
\pgfpathclose%
\pgfusepath{fill}%
\end{pgfscope}%
\begin{pgfscope}%
\pgfpathrectangle{\pgfqpoint{1.020000in}{0.880000in}}{\pgfqpoint{6.160000in}{6.160000in}}%
\pgfusepath{clip}%
\pgfsetbuttcap%
\pgfsetroundjoin%
\definecolor{currentfill}{rgb}{0.521696,0.659599,0.987736}%
\pgfsetfillcolor{currentfill}%
\pgfsetlinewidth{0.000000pt}%
\definecolor{currentstroke}{rgb}{0.000000,0.000000,0.000000}%
\pgfsetstrokecolor{currentstroke}%
\pgfsetdash{}{0pt}%
\pgfpathmoveto{\pgfqpoint{4.968646in}{3.868031in}}%
\pgfpathlineto{\pgfqpoint{4.979231in}{3.854640in}}%
\pgfpathlineto{\pgfqpoint{4.989833in}{3.841077in}}%
\pgfpathlineto{\pgfqpoint{5.021294in}{3.839648in}}%
\pgfpathlineto{\pgfqpoint{5.052735in}{3.838271in}}%
\pgfpathlineto{\pgfqpoint{5.042085in}{3.851747in}}%
\pgfpathlineto{\pgfqpoint{5.031452in}{3.865058in}}%
\pgfpathlineto{\pgfqpoint{5.000059in}{3.866518in}}%
\pgfpathlineto{\pgfqpoint{4.968646in}{3.868031in}}%
\pgfpathclose%
\pgfusepath{fill}%
\end{pgfscope}%
\begin{pgfscope}%
\pgfpathrectangle{\pgfqpoint{1.020000in}{0.880000in}}{\pgfqpoint{6.160000in}{6.160000in}}%
\pgfusepath{clip}%
\pgfsetbuttcap%
\pgfsetroundjoin%
\definecolor{currentfill}{rgb}{0.586921,0.718121,0.998874}%
\pgfsetfillcolor{currentfill}%
\pgfsetlinewidth{0.000000pt}%
\definecolor{currentstroke}{rgb}{0.000000,0.000000,0.000000}%
\pgfsetstrokecolor{currentstroke}%
\pgfsetdash{}{0pt}%
\pgfpathmoveto{\pgfqpoint{4.402470in}{3.983971in}}%
\pgfpathlineto{\pgfqpoint{4.412563in}{3.971525in}}%
\pgfpathlineto{\pgfqpoint{4.422672in}{3.958877in}}%
\pgfpathlineto{\pgfqpoint{4.454315in}{3.955191in}}%
\pgfpathlineto{\pgfqpoint{4.485937in}{3.951779in}}%
\pgfpathlineto{\pgfqpoint{4.475779in}{3.964503in}}%
\pgfpathlineto{\pgfqpoint{4.465638in}{3.977028in}}%
\pgfpathlineto{\pgfqpoint{4.434065in}{3.980376in}}%
\pgfpathlineto{\pgfqpoint{4.402470in}{3.983971in}}%
\pgfpathclose%
\pgfusepath{fill}%
\end{pgfscope}%
\begin{pgfscope}%
\pgfpathrectangle{\pgfqpoint{1.020000in}{0.880000in}}{\pgfqpoint{6.160000in}{6.160000in}}%
\pgfusepath{clip}%
\pgfsetbuttcap%
\pgfsetroundjoin%
\definecolor{currentfill}{rgb}{0.646113,0.764436,0.996868}%
\pgfsetfillcolor{currentfill}%
\pgfsetlinewidth{0.000000pt}%
\definecolor{currentstroke}{rgb}{0.000000,0.000000,0.000000}%
\pgfsetstrokecolor{currentstroke}%
\pgfsetdash{}{0pt}%
\pgfpathmoveto{\pgfqpoint{4.046095in}{4.083767in}}%
\pgfpathlineto{\pgfqpoint{4.055870in}{4.073865in}}%
\pgfpathlineto{\pgfqpoint{4.065664in}{4.063801in}}%
\pgfpathlineto{\pgfqpoint{4.097437in}{4.055842in}}%
\pgfpathlineto{\pgfqpoint{4.129185in}{4.048569in}}%
\pgfpathlineto{\pgfqpoint{4.119341in}{4.059262in}}%
\pgfpathlineto{\pgfqpoint{4.109515in}{4.069779in}}%
\pgfpathlineto{\pgfqpoint{4.077817in}{4.076476in}}%
\pgfpathlineto{\pgfqpoint{4.046095in}{4.083767in}}%
\pgfpathclose%
\pgfusepath{fill}%
\end{pgfscope}%
\begin{pgfscope}%
\pgfpathrectangle{\pgfqpoint{1.020000in}{0.880000in}}{\pgfqpoint{6.160000in}{6.160000in}}%
\pgfusepath{clip}%
\pgfsetbuttcap%
\pgfsetroundjoin%
\definecolor{currentfill}{rgb}{0.543440,0.680003,0.993051}%
\pgfsetfillcolor{currentfill}%
\pgfsetlinewidth{0.000000pt}%
\definecolor{currentstroke}{rgb}{0.000000,0.000000,0.000000}%
\pgfsetstrokecolor{currentstroke}%
\pgfsetdash{}{0pt}%
\pgfpathmoveto{\pgfqpoint{4.758922in}{3.905137in}}%
\pgfpathlineto{\pgfqpoint{4.769326in}{3.891894in}}%
\pgfpathlineto{\pgfqpoint{4.779748in}{3.878467in}}%
\pgfpathlineto{\pgfqpoint{4.811281in}{3.876533in}}%
\pgfpathlineto{\pgfqpoint{4.842795in}{3.874688in}}%
\pgfpathlineto{\pgfqpoint{4.832325in}{3.888051in}}%
\pgfpathlineto{\pgfqpoint{4.821872in}{3.901235in}}%
\pgfpathlineto{\pgfqpoint{4.790407in}{3.903144in}}%
\pgfpathlineto{\pgfqpoint{4.758922in}{3.905137in}}%
\pgfpathclose%
\pgfusepath{fill}%
\end{pgfscope}%
\begin{pgfscope}%
\pgfpathrectangle{\pgfqpoint{1.020000in}{0.880000in}}{\pgfqpoint{6.160000in}{6.160000in}}%
\pgfusepath{clip}%
\pgfsetbuttcap%
\pgfsetroundjoin%
\definecolor{currentfill}{rgb}{0.483854,0.622050,0.974808}%
\pgfsetfillcolor{currentfill}%
\pgfsetlinewidth{0.000000pt}%
\definecolor{currentstroke}{rgb}{0.000000,0.000000,0.000000}%
\pgfsetstrokecolor{currentstroke}%
\pgfsetdash{}{0pt}%
\pgfpathmoveto{\pgfqpoint{5.325290in}{3.802100in}}%
\pgfpathlineto{\pgfqpoint{5.336188in}{3.788524in}}%
\pgfpathlineto{\pgfqpoint{5.347104in}{3.774810in}}%
\pgfpathlineto{\pgfqpoint{5.378463in}{3.773979in}}%
\pgfpathlineto{\pgfqpoint{5.409804in}{3.773179in}}%
\pgfpathlineto{\pgfqpoint{5.398839in}{3.786801in}}%
\pgfpathlineto{\pgfqpoint{5.387893in}{3.800290in}}%
\pgfpathlineto{\pgfqpoint{5.356602in}{3.801177in}}%
\pgfpathlineto{\pgfqpoint{5.325290in}{3.802100in}}%
\pgfpathclose%
\pgfusepath{fill}%
\end{pgfscope}%
\begin{pgfscope}%
\pgfpathrectangle{\pgfqpoint{1.020000in}{0.880000in}}{\pgfqpoint{6.160000in}{6.160000in}}%
\pgfusepath{clip}%
\pgfsetbuttcap%
\pgfsetroundjoin%
\definecolor{currentfill}{rgb}{0.624703,0.748318,0.998719}%
\pgfsetfillcolor{currentfill}%
\pgfsetlinewidth{0.000000pt}%
\definecolor{currentstroke}{rgb}{0.000000,0.000000,0.000000}%
\pgfsetstrokecolor{currentstroke}%
\pgfsetdash{}{0pt}%
\pgfpathmoveto{\pgfqpoint{4.192609in}{4.035832in}}%
\pgfpathlineto{\pgfqpoint{4.202520in}{4.024485in}}%
\pgfpathlineto{\pgfqpoint{4.212448in}{4.012939in}}%
\pgfpathlineto{\pgfqpoint{4.244174in}{4.006997in}}%
\pgfpathlineto{\pgfqpoint{4.275878in}{4.001571in}}%
\pgfpathlineto{\pgfqpoint{4.265900in}{4.013449in}}%
\pgfpathlineto{\pgfqpoint{4.255940in}{4.025127in}}%
\pgfpathlineto{\pgfqpoint{4.224286in}{4.030252in}}%
\pgfpathlineto{\pgfqpoint{4.192609in}{4.035832in}}%
\pgfpathclose%
\pgfusepath{fill}%
\end{pgfscope}%
\begin{pgfscope}%
\pgfpathrectangle{\pgfqpoint{1.020000in}{0.880000in}}{\pgfqpoint{6.160000in}{6.160000in}}%
\pgfusepath{clip}%
\pgfsetbuttcap%
\pgfsetroundjoin%
\definecolor{currentfill}{rgb}{0.500031,0.638508,0.981070}%
\pgfsetfillcolor{currentfill}%
\pgfsetlinewidth{0.000000pt}%
\definecolor{currentstroke}{rgb}{0.000000,0.000000,0.000000}%
\pgfsetstrokecolor{currentstroke}%
\pgfsetdash{}{0pt}%
\pgfpathmoveto{\pgfqpoint{5.115558in}{3.835659in}}%
\pgfpathlineto{\pgfqpoint{5.126274in}{3.822109in}}%
\pgfpathlineto{\pgfqpoint{5.137009in}{3.808396in}}%
\pgfpathlineto{\pgfqpoint{5.168438in}{3.807252in}}%
\pgfpathlineto{\pgfqpoint{5.199849in}{3.806148in}}%
\pgfpathlineto{\pgfqpoint{5.189066in}{3.819762in}}%
\pgfpathlineto{\pgfqpoint{5.178300in}{3.833224in}}%
\pgfpathlineto{\pgfqpoint{5.146939in}{3.834420in}}%
\pgfpathlineto{\pgfqpoint{5.115558in}{3.835659in}}%
\pgfpathclose%
\pgfusepath{fill}%
\end{pgfscope}%
\begin{pgfscope}%
\pgfpathrectangle{\pgfqpoint{1.020000in}{0.880000in}}{\pgfqpoint{6.160000in}{6.160000in}}%
\pgfusepath{clip}%
\pgfsetbuttcap%
\pgfsetroundjoin%
\definecolor{currentfill}{rgb}{0.570616,0.704109,0.997195}%
\pgfsetfillcolor{currentfill}%
\pgfsetlinewidth{0.000000pt}%
\definecolor{currentstroke}{rgb}{0.000000,0.000000,0.000000}%
\pgfsetstrokecolor{currentstroke}%
\pgfsetdash{}{0pt}%
\pgfpathmoveto{\pgfqpoint{4.549118in}{3.945646in}}%
\pgfpathlineto{\pgfqpoint{4.559342in}{3.932697in}}%
\pgfpathlineto{\pgfqpoint{4.569583in}{3.919554in}}%
\pgfpathlineto{\pgfqpoint{4.601191in}{3.916778in}}%
\pgfpathlineto{\pgfqpoint{4.632778in}{3.914177in}}%
\pgfpathlineto{\pgfqpoint{4.622488in}{3.927312in}}%
\pgfpathlineto{\pgfqpoint{4.612216in}{3.940256in}}%
\pgfpathlineto{\pgfqpoint{4.580677in}{3.942870in}}%
\pgfpathlineto{\pgfqpoint{4.549118in}{3.945646in}}%
\pgfpathclose%
\pgfusepath{fill}%
\end{pgfscope}%
\begin{pgfscope}%
\pgfpathrectangle{\pgfqpoint{1.020000in}{0.880000in}}{\pgfqpoint{6.160000in}{6.160000in}}%
\pgfusepath{clip}%
\pgfsetbuttcap%
\pgfsetroundjoin%
\definecolor{currentfill}{rgb}{0.693321,0.796314,0.986308}%
\pgfsetfillcolor{currentfill}%
\pgfsetlinewidth{0.000000pt}%
\definecolor{currentstroke}{rgb}{0.000000,0.000000,0.000000}%
\pgfsetstrokecolor{currentstroke}%
\pgfsetdash{}{0pt}%
\pgfpathmoveto{\pgfqpoint{3.836030in}{4.156439in}}%
\pgfpathlineto{\pgfqpoint{3.845610in}{4.149812in}}%
\pgfpathlineto{\pgfqpoint{3.855209in}{4.143162in}}%
\pgfpathlineto{\pgfqpoint{3.887093in}{4.131115in}}%
\pgfpathlineto{\pgfqpoint{3.918948in}{4.120002in}}%
\pgfpathlineto{\pgfqpoint{3.909295in}{4.127942in}}%
\pgfpathlineto{\pgfqpoint{3.899659in}{4.135809in}}%
\pgfpathlineto{\pgfqpoint{3.867858in}{4.145728in}}%
\pgfpathlineto{\pgfqpoint{3.836030in}{4.156439in}}%
\pgfpathclose%
\pgfusepath{fill}%
\end{pgfscope}%
\begin{pgfscope}%
\pgfpathrectangle{\pgfqpoint{1.020000in}{0.880000in}}{\pgfqpoint{6.160000in}{6.160000in}}%
\pgfusepath{clip}%
\pgfsetbuttcap%
\pgfsetroundjoin%
\definecolor{currentfill}{rgb}{0.527132,0.664700,0.989065}%
\pgfsetfillcolor{currentfill}%
\pgfsetlinewidth{0.000000pt}%
\definecolor{currentstroke}{rgb}{0.000000,0.000000,0.000000}%
\pgfsetstrokecolor{currentstroke}%
\pgfsetdash{}{0pt}%
\pgfpathmoveto{\pgfqpoint{4.905761in}{3.871227in}}%
\pgfpathlineto{\pgfqpoint{4.916297in}{3.857758in}}%
\pgfpathlineto{\pgfqpoint{4.926850in}{3.844109in}}%
\pgfpathlineto{\pgfqpoint{4.958352in}{3.842563in}}%
\pgfpathlineto{\pgfqpoint{4.989833in}{3.841077in}}%
\pgfpathlineto{\pgfqpoint{4.979231in}{3.854640in}}%
\pgfpathlineto{\pgfqpoint{4.968646in}{3.868031in}}%
\pgfpathlineto{\pgfqpoint{4.937214in}{3.869599in}}%
\pgfpathlineto{\pgfqpoint{4.905761in}{3.871227in}}%
\pgfpathclose%
\pgfusepath{fill}%
\end{pgfscope}%
\begin{pgfscope}%
\pgfpathrectangle{\pgfqpoint{1.020000in}{0.880000in}}{\pgfqpoint{6.160000in}{6.160000in}}%
\pgfusepath{clip}%
\pgfsetbuttcap%
\pgfsetroundjoin%
\definecolor{currentfill}{rgb}{0.597777,0.727330,0.999777}%
\pgfsetfillcolor{currentfill}%
\pgfsetlinewidth{0.000000pt}%
\definecolor{currentstroke}{rgb}{0.000000,0.000000,0.000000}%
\pgfsetstrokecolor{currentstroke}%
\pgfsetdash{}{0pt}%
\pgfpathmoveto{\pgfqpoint{4.339218in}{3.992046in}}%
\pgfpathlineto{\pgfqpoint{4.349261in}{3.979740in}}%
\pgfpathlineto{\pgfqpoint{4.359323in}{3.967228in}}%
\pgfpathlineto{\pgfqpoint{4.391008in}{3.962875in}}%
\pgfpathlineto{\pgfqpoint{4.422672in}{3.958877in}}%
\pgfpathlineto{\pgfqpoint{4.412563in}{3.971525in}}%
\pgfpathlineto{\pgfqpoint{4.402470in}{3.983971in}}%
\pgfpathlineto{\pgfqpoint{4.370855in}{3.987849in}}%
\pgfpathlineto{\pgfqpoint{4.339218in}{3.992046in}}%
\pgfpathclose%
\pgfusepath{fill}%
\end{pgfscope}%
\begin{pgfscope}%
\pgfpathrectangle{\pgfqpoint{1.020000in}{0.880000in}}{\pgfqpoint{6.160000in}{6.160000in}}%
\pgfusepath{clip}%
\pgfsetbuttcap%
\pgfsetroundjoin%
\definecolor{currentfill}{rgb}{0.661968,0.775491,0.993937}%
\pgfsetfillcolor{currentfill}%
\pgfsetlinewidth{0.000000pt}%
\definecolor{currentstroke}{rgb}{0.000000,0.000000,0.000000}%
\pgfsetstrokecolor{currentstroke}%
\pgfsetdash{}{0pt}%
\pgfpathmoveto{\pgfqpoint{3.982575in}{4.100355in}}%
\pgfpathlineto{\pgfqpoint{3.992299in}{4.091268in}}%
\pgfpathlineto{\pgfqpoint{4.002041in}{4.082039in}}%
\pgfpathlineto{\pgfqpoint{4.033866in}{4.072511in}}%
\pgfpathlineto{\pgfqpoint{4.065664in}{4.063801in}}%
\pgfpathlineto{\pgfqpoint{4.055870in}{4.073865in}}%
\pgfpathlineto{\pgfqpoint{4.046095in}{4.083767in}}%
\pgfpathlineto{\pgfqpoint{4.014348in}{4.091708in}}%
\pgfpathlineto{\pgfqpoint{3.982575in}{4.100355in}}%
\pgfpathclose%
\pgfusepath{fill}%
\end{pgfscope}%
\begin{pgfscope}%
\pgfpathrectangle{\pgfqpoint{1.020000in}{0.880000in}}{\pgfqpoint{6.160000in}{6.160000in}}%
\pgfusepath{clip}%
\pgfsetbuttcap%
\pgfsetroundjoin%
\definecolor{currentfill}{rgb}{0.548876,0.685104,0.994379}%
\pgfsetfillcolor{currentfill}%
\pgfsetlinewidth{0.000000pt}%
\definecolor{currentstroke}{rgb}{0.000000,0.000000,0.000000}%
\pgfsetstrokecolor{currentstroke}%
\pgfsetdash{}{0pt}%
\pgfpathmoveto{\pgfqpoint{4.695891in}{3.909417in}}%
\pgfpathlineto{\pgfqpoint{4.706247in}{3.896125in}}%
\pgfpathlineto{\pgfqpoint{4.716620in}{3.882645in}}%
\pgfpathlineto{\pgfqpoint{4.748194in}{3.880500in}}%
\pgfpathlineto{\pgfqpoint{4.779748in}{3.878467in}}%
\pgfpathlineto{\pgfqpoint{4.769326in}{3.891894in}}%
\pgfpathlineto{\pgfqpoint{4.758922in}{3.905137in}}%
\pgfpathlineto{\pgfqpoint{4.727417in}{3.907224in}}%
\pgfpathlineto{\pgfqpoint{4.695891in}{3.909417in}}%
\pgfpathclose%
\pgfusepath{fill}%
\end{pgfscope}%
\begin{pgfscope}%
\pgfpathrectangle{\pgfqpoint{1.020000in}{0.880000in}}{\pgfqpoint{6.160000in}{6.160000in}}%
\pgfusepath{clip}%
\pgfsetbuttcap%
\pgfsetroundjoin%
\definecolor{currentfill}{rgb}{0.489246,0.627536,0.976896}%
\pgfsetfillcolor{currentfill}%
\pgfsetlinewidth{0.000000pt}%
\definecolor{currentstroke}{rgb}{0.000000,0.000000,0.000000}%
\pgfsetstrokecolor{currentstroke}%
\pgfsetdash{}{0pt}%
\pgfpathmoveto{\pgfqpoint{5.262609in}{3.804051in}}%
\pgfpathlineto{\pgfqpoint{5.273458in}{3.790383in}}%
\pgfpathlineto{\pgfqpoint{5.284325in}{3.776568in}}%
\pgfpathlineto{\pgfqpoint{5.315724in}{3.775673in}}%
\pgfpathlineto{\pgfqpoint{5.347104in}{3.774810in}}%
\pgfpathlineto{\pgfqpoint{5.336188in}{3.788524in}}%
\pgfpathlineto{\pgfqpoint{5.325290in}{3.802100in}}%
\pgfpathlineto{\pgfqpoint{5.293960in}{3.803057in}}%
\pgfpathlineto{\pgfqpoint{5.262609in}{3.804051in}}%
\pgfpathclose%
\pgfusepath{fill}%
\end{pgfscope}%
\begin{pgfscope}%
\pgfpathrectangle{\pgfqpoint{1.020000in}{0.880000in}}{\pgfqpoint{6.160000in}{6.160000in}}%
\pgfusepath{clip}%
\pgfsetbuttcap%
\pgfsetroundjoin%
\definecolor{currentfill}{rgb}{0.510824,0.649397,0.985079}%
\pgfsetfillcolor{currentfill}%
\pgfsetlinewidth{0.000000pt}%
\definecolor{currentstroke}{rgb}{0.000000,0.000000,0.000000}%
\pgfsetstrokecolor{currentstroke}%
\pgfsetdash{}{0pt}%
\pgfpathmoveto{\pgfqpoint{5.052735in}{3.838271in}}%
\pgfpathlineto{\pgfqpoint{5.063403in}{3.824627in}}%
\pgfpathlineto{\pgfqpoint{5.074089in}{3.810810in}}%
\pgfpathlineto{\pgfqpoint{5.105559in}{3.809582in}}%
\pgfpathlineto{\pgfqpoint{5.137009in}{3.808396in}}%
\pgfpathlineto{\pgfqpoint{5.126274in}{3.822109in}}%
\pgfpathlineto{\pgfqpoint{5.115558in}{3.835659in}}%
\pgfpathlineto{\pgfqpoint{5.084156in}{3.836942in}}%
\pgfpathlineto{\pgfqpoint{5.052735in}{3.838271in}}%
\pgfpathclose%
\pgfusepath{fill}%
\end{pgfscope}%
\begin{pgfscope}%
\pgfpathrectangle{\pgfqpoint{1.020000in}{0.880000in}}{\pgfqpoint{6.160000in}{6.160000in}}%
\pgfusepath{clip}%
\pgfsetbuttcap%
\pgfsetroundjoin%
\definecolor{currentfill}{rgb}{0.576051,0.708780,0.997755}%
\pgfsetfillcolor{currentfill}%
\pgfsetlinewidth{0.000000pt}%
\definecolor{currentstroke}{rgb}{0.000000,0.000000,0.000000}%
\pgfsetstrokecolor{currentstroke}%
\pgfsetdash{}{0pt}%
\pgfpathmoveto{\pgfqpoint{4.485937in}{3.951779in}}%
\pgfpathlineto{\pgfqpoint{4.496113in}{3.938857in}}%
\pgfpathlineto{\pgfqpoint{4.506306in}{3.925737in}}%
\pgfpathlineto{\pgfqpoint{4.537955in}{3.922531in}}%
\pgfpathlineto{\pgfqpoint{4.569583in}{3.919554in}}%
\pgfpathlineto{\pgfqpoint{4.559342in}{3.932697in}}%
\pgfpathlineto{\pgfqpoint{4.549118in}{3.945646in}}%
\pgfpathlineto{\pgfqpoint{4.517538in}{3.948607in}}%
\pgfpathlineto{\pgfqpoint{4.485937in}{3.951779in}}%
\pgfpathclose%
\pgfusepath{fill}%
\end{pgfscope}%
\begin{pgfscope}%
\pgfpathrectangle{\pgfqpoint{1.020000in}{0.880000in}}{\pgfqpoint{6.160000in}{6.160000in}}%
\pgfusepath{clip}%
\pgfsetbuttcap%
\pgfsetroundjoin%
\definecolor{currentfill}{rgb}{0.635474,0.756714,0.998297}%
\pgfsetfillcolor{currentfill}%
\pgfsetlinewidth{0.000000pt}%
\definecolor{currentstroke}{rgb}{0.000000,0.000000,0.000000}%
\pgfsetstrokecolor{currentstroke}%
\pgfsetdash{}{0pt}%
\pgfpathmoveto{\pgfqpoint{4.129185in}{4.048569in}}%
\pgfpathlineto{\pgfqpoint{4.139046in}{4.037689in}}%
\pgfpathlineto{\pgfqpoint{4.148925in}{4.026614in}}%
\pgfpathlineto{\pgfqpoint{4.180699in}{4.019456in}}%
\pgfpathlineto{\pgfqpoint{4.212448in}{4.012939in}}%
\pgfpathlineto{\pgfqpoint{4.202520in}{4.024485in}}%
\pgfpathlineto{\pgfqpoint{4.192609in}{4.035832in}}%
\pgfpathlineto{\pgfqpoint{4.160908in}{4.041918in}}%
\pgfpathlineto{\pgfqpoint{4.129185in}{4.048569in}}%
\pgfpathclose%
\pgfusepath{fill}%
\end{pgfscope}%
\begin{pgfscope}%
\pgfpathrectangle{\pgfqpoint{1.020000in}{0.880000in}}{\pgfqpoint{6.160000in}{6.160000in}}%
\pgfusepath{clip}%
\pgfsetbuttcap%
\pgfsetroundjoin%
\definecolor{currentfill}{rgb}{0.532568,0.669801,0.990393}%
\pgfsetfillcolor{currentfill}%
\pgfsetlinewidth{0.000000pt}%
\definecolor{currentstroke}{rgb}{0.000000,0.000000,0.000000}%
\pgfsetstrokecolor{currentstroke}%
\pgfsetdash{}{0pt}%
\pgfpathmoveto{\pgfqpoint{4.842795in}{3.874688in}}%
\pgfpathlineto{\pgfqpoint{4.853282in}{3.861143in}}%
\pgfpathlineto{\pgfqpoint{4.863787in}{3.847410in}}%
\pgfpathlineto{\pgfqpoint{4.895329in}{3.845723in}}%
\pgfpathlineto{\pgfqpoint{4.926850in}{3.844109in}}%
\pgfpathlineto{\pgfqpoint{4.916297in}{3.857758in}}%
\pgfpathlineto{\pgfqpoint{4.905761in}{3.871227in}}%
\pgfpathlineto{\pgfqpoint{4.874288in}{3.872922in}}%
\pgfpathlineto{\pgfqpoint{4.842795in}{3.874688in}}%
\pgfpathclose%
\pgfusepath{fill}%
\end{pgfscope}%
\begin{pgfscope}%
\pgfpathrectangle{\pgfqpoint{1.020000in}{0.880000in}}{\pgfqpoint{6.160000in}{6.160000in}}%
\pgfusepath{clip}%
\pgfsetbuttcap%
\pgfsetroundjoin%
\definecolor{currentfill}{rgb}{0.608547,0.735725,0.999354}%
\pgfsetfillcolor{currentfill}%
\pgfsetlinewidth{0.000000pt}%
\definecolor{currentstroke}{rgb}{0.000000,0.000000,0.000000}%
\pgfsetstrokecolor{currentstroke}%
\pgfsetdash{}{0pt}%
\pgfpathmoveto{\pgfqpoint{4.275878in}{4.001571in}}%
\pgfpathlineto{\pgfqpoint{4.285873in}{3.989488in}}%
\pgfpathlineto{\pgfqpoint{4.295885in}{3.977195in}}%
\pgfpathlineto{\pgfqpoint{4.327615in}{3.971984in}}%
\pgfpathlineto{\pgfqpoint{4.359323in}{3.967228in}}%
\pgfpathlineto{\pgfqpoint{4.349261in}{3.979740in}}%
\pgfpathlineto{\pgfqpoint{4.339218in}{3.992046in}}%
\pgfpathlineto{\pgfqpoint{4.307559in}{3.996604in}}%
\pgfpathlineto{\pgfqpoint{4.275878in}{4.001571in}}%
\pgfpathclose%
\pgfusepath{fill}%
\end{pgfscope}%
\begin{pgfscope}%
\pgfpathrectangle{\pgfqpoint{1.020000in}{0.880000in}}{\pgfqpoint{6.160000in}{6.160000in}}%
\pgfusepath{clip}%
\pgfsetbuttcap%
\pgfsetroundjoin%
\definecolor{currentfill}{rgb}{0.713852,0.808857,0.979386}%
\pgfsetfillcolor{currentfill}%
\pgfsetlinewidth{0.000000pt}%
\definecolor{currentstroke}{rgb}{0.000000,0.000000,0.000000}%
\pgfsetstrokecolor{currentstroke}%
\pgfsetdash{}{0pt}%
\pgfpathmoveto{\pgfqpoint{3.772284in}{4.180405in}}%
\pgfpathlineto{\pgfqpoint{3.781807in}{4.175309in}}%
\pgfpathlineto{\pgfqpoint{3.791349in}{4.170250in}}%
\pgfpathlineto{\pgfqpoint{3.823295in}{4.156191in}}%
\pgfpathlineto{\pgfqpoint{3.855209in}{4.143162in}}%
\pgfpathlineto{\pgfqpoint{3.845610in}{4.149812in}}%
\pgfpathlineto{\pgfqpoint{3.836030in}{4.156439in}}%
\pgfpathlineto{\pgfqpoint{3.804172in}{4.167985in}}%
\pgfpathlineto{\pgfqpoint{3.772284in}{4.180405in}}%
\pgfpathclose%
\pgfusepath{fill}%
\end{pgfscope}%
\begin{pgfscope}%
\pgfpathrectangle{\pgfqpoint{1.020000in}{0.880000in}}{\pgfqpoint{6.160000in}{6.160000in}}%
\pgfusepath{clip}%
\pgfsetbuttcap%
\pgfsetroundjoin%
\definecolor{currentfill}{rgb}{0.473070,0.611077,0.970634}%
\pgfsetfillcolor{currentfill}%
\pgfsetlinewidth{0.000000pt}%
\definecolor{currentstroke}{rgb}{0.000000,0.000000,0.000000}%
\pgfsetstrokecolor{currentstroke}%
\pgfsetdash{}{0pt}%
\pgfpathmoveto{\pgfqpoint{5.409804in}{3.773179in}}%
\pgfpathlineto{\pgfqpoint{5.420786in}{3.759428in}}%
\pgfpathlineto{\pgfqpoint{5.431786in}{3.745547in}}%
\pgfpathlineto{\pgfqpoint{5.463155in}{3.744875in}}%
\pgfpathlineto{\pgfqpoint{5.452130in}{3.758707in}}%
\pgfpathlineto{\pgfqpoint{5.441124in}{3.772412in}}%
\pgfpathlineto{\pgfqpoint{5.409804in}{3.773179in}}%
\pgfpathclose%
\pgfusepath{fill}%
\end{pgfscope}%
\begin{pgfscope}%
\pgfpathrectangle{\pgfqpoint{1.020000in}{0.880000in}}{\pgfqpoint{6.160000in}{6.160000in}}%
\pgfusepath{clip}%
\pgfsetbuttcap%
\pgfsetroundjoin%
\definecolor{currentfill}{rgb}{0.554312,0.690097,0.995516}%
\pgfsetfillcolor{currentfill}%
\pgfsetlinewidth{0.000000pt}%
\definecolor{currentstroke}{rgb}{0.000000,0.000000,0.000000}%
\pgfsetstrokecolor{currentstroke}%
\pgfsetdash{}{0pt}%
\pgfpathmoveto{\pgfqpoint{4.632778in}{3.914177in}}%
\pgfpathlineto{\pgfqpoint{4.643085in}{3.900851in}}%
\pgfpathlineto{\pgfqpoint{4.653410in}{3.887332in}}%
\pgfpathlineto{\pgfqpoint{4.685025in}{3.884917in}}%
\pgfpathlineto{\pgfqpoint{4.716620in}{3.882645in}}%
\pgfpathlineto{\pgfqpoint{4.706247in}{3.896125in}}%
\pgfpathlineto{\pgfqpoint{4.695891in}{3.909417in}}%
\pgfpathlineto{\pgfqpoint{4.664345in}{3.911729in}}%
\pgfpathlineto{\pgfqpoint{4.632778in}{3.914177in}}%
\pgfpathclose%
\pgfusepath{fill}%
\end{pgfscope}%
\begin{pgfscope}%
\pgfpathrectangle{\pgfqpoint{1.020000in}{0.880000in}}{\pgfqpoint{6.160000in}{6.160000in}}%
\pgfusepath{clip}%
\pgfsetbuttcap%
\pgfsetroundjoin%
\definecolor{currentfill}{rgb}{0.489246,0.627536,0.976896}%
\pgfsetfillcolor{currentfill}%
\pgfsetlinewidth{0.000000pt}%
\definecolor{currentstroke}{rgb}{0.000000,0.000000,0.000000}%
\pgfsetstrokecolor{currentstroke}%
\pgfsetdash{}{0pt}%
\pgfpathmoveto{\pgfqpoint{5.199849in}{3.806148in}}%
\pgfpathlineto{\pgfqpoint{5.210649in}{3.792379in}}%
\pgfpathlineto{\pgfqpoint{5.221467in}{3.778451in}}%
\pgfpathlineto{\pgfqpoint{5.252906in}{3.777494in}}%
\pgfpathlineto{\pgfqpoint{5.284325in}{3.776568in}}%
\pgfpathlineto{\pgfqpoint{5.273458in}{3.790383in}}%
\pgfpathlineto{\pgfqpoint{5.262609in}{3.804051in}}%
\pgfpathlineto{\pgfqpoint{5.231239in}{3.805081in}}%
\pgfpathlineto{\pgfqpoint{5.199849in}{3.806148in}}%
\pgfpathclose%
\pgfusepath{fill}%
\end{pgfscope}%
\begin{pgfscope}%
\pgfpathrectangle{\pgfqpoint{1.020000in}{0.880000in}}{\pgfqpoint{6.160000in}{6.160000in}}%
\pgfusepath{clip}%
\pgfsetbuttcap%
\pgfsetroundjoin%
\definecolor{currentfill}{rgb}{0.677823,0.786546,0.991005}%
\pgfsetfillcolor{currentfill}%
\pgfsetlinewidth{0.000000pt}%
\definecolor{currentstroke}{rgb}{0.000000,0.000000,0.000000}%
\pgfsetstrokecolor{currentstroke}%
\pgfsetdash{}{0pt}%
\pgfpathmoveto{\pgfqpoint{3.918948in}{4.120002in}}%
\pgfpathlineto{\pgfqpoint{3.928620in}{4.111969in}}%
\pgfpathlineto{\pgfqpoint{3.938310in}{4.103821in}}%
\pgfpathlineto{\pgfqpoint{3.970190in}{4.092454in}}%
\pgfpathlineto{\pgfqpoint{4.002041in}{4.082039in}}%
\pgfpathlineto{\pgfqpoint{3.992299in}{4.091268in}}%
\pgfpathlineto{\pgfqpoint{3.982575in}{4.100355in}}%
\pgfpathlineto{\pgfqpoint{3.950775in}{4.109767in}}%
\pgfpathlineto{\pgfqpoint{3.918948in}{4.120002in}}%
\pgfpathclose%
\pgfusepath{fill}%
\end{pgfscope}%
\begin{pgfscope}%
\pgfpathrectangle{\pgfqpoint{1.020000in}{0.880000in}}{\pgfqpoint{6.160000in}{6.160000in}}%
\pgfusepath{clip}%
\pgfsetbuttcap%
\pgfsetroundjoin%
\definecolor{currentfill}{rgb}{0.516260,0.654498,0.986407}%
\pgfsetfillcolor{currentfill}%
\pgfsetlinewidth{0.000000pt}%
\definecolor{currentstroke}{rgb}{0.000000,0.000000,0.000000}%
\pgfsetstrokecolor{currentstroke}%
\pgfsetdash{}{0pt}%
\pgfpathmoveto{\pgfqpoint{4.989833in}{3.841077in}}%
\pgfpathlineto{\pgfqpoint{5.000452in}{3.827337in}}%
\pgfpathlineto{\pgfqpoint{5.011089in}{3.813411in}}%
\pgfpathlineto{\pgfqpoint{5.042599in}{3.812086in}}%
\pgfpathlineto{\pgfqpoint{5.074089in}{3.810810in}}%
\pgfpathlineto{\pgfqpoint{5.063403in}{3.824627in}}%
\pgfpathlineto{\pgfqpoint{5.052735in}{3.838271in}}%
\pgfpathlineto{\pgfqpoint{5.021294in}{3.839648in}}%
\pgfpathlineto{\pgfqpoint{4.989833in}{3.841077in}}%
\pgfpathclose%
\pgfusepath{fill}%
\end{pgfscope}%
\begin{pgfscope}%
\pgfpathrectangle{\pgfqpoint{1.020000in}{0.880000in}}{\pgfqpoint{6.160000in}{6.160000in}}%
\pgfusepath{clip}%
\pgfsetbuttcap%
\pgfsetroundjoin%
\definecolor{currentfill}{rgb}{0.586921,0.718121,0.998874}%
\pgfsetfillcolor{currentfill}%
\pgfsetlinewidth{0.000000pt}%
\definecolor{currentstroke}{rgb}{0.000000,0.000000,0.000000}%
\pgfsetstrokecolor{currentstroke}%
\pgfsetdash{}{0pt}%
\pgfpathmoveto{\pgfqpoint{4.422672in}{3.958877in}}%
\pgfpathlineto{\pgfqpoint{4.432800in}{3.946026in}}%
\pgfpathlineto{\pgfqpoint{4.442944in}{3.932971in}}%
\pgfpathlineto{\pgfqpoint{4.474635in}{3.929204in}}%
\pgfpathlineto{\pgfqpoint{4.506306in}{3.925737in}}%
\pgfpathlineto{\pgfqpoint{4.496113in}{3.938857in}}%
\pgfpathlineto{\pgfqpoint{4.485937in}{3.951779in}}%
\pgfpathlineto{\pgfqpoint{4.454315in}{3.955191in}}%
\pgfpathlineto{\pgfqpoint{4.422672in}{3.958877in}}%
\pgfpathclose%
\pgfusepath{fill}%
\end{pgfscope}%
\begin{pgfscope}%
\pgfpathrectangle{\pgfqpoint{1.020000in}{0.880000in}}{\pgfqpoint{6.160000in}{6.160000in}}%
\pgfusepath{clip}%
\pgfsetbuttcap%
\pgfsetroundjoin%
\definecolor{currentfill}{rgb}{0.646113,0.764436,0.996868}%
\pgfsetfillcolor{currentfill}%
\pgfsetlinewidth{0.000000pt}%
\definecolor{currentstroke}{rgb}{0.000000,0.000000,0.000000}%
\pgfsetstrokecolor{currentstroke}%
\pgfsetdash{}{0pt}%
\pgfpathmoveto{\pgfqpoint{4.065664in}{4.063801in}}%
\pgfpathlineto{\pgfqpoint{4.075475in}{4.053561in}}%
\pgfpathlineto{\pgfqpoint{4.085305in}{4.043131in}}%
\pgfpathlineto{\pgfqpoint{4.117128in}{4.034482in}}%
\pgfpathlineto{\pgfqpoint{4.148925in}{4.026614in}}%
\pgfpathlineto{\pgfqpoint{4.139046in}{4.037689in}}%
\pgfpathlineto{\pgfqpoint{4.129185in}{4.048569in}}%
\pgfpathlineto{\pgfqpoint{4.097437in}{4.055842in}}%
\pgfpathlineto{\pgfqpoint{4.065664in}{4.063801in}}%
\pgfpathclose%
\pgfusepath{fill}%
\end{pgfscope}%
\begin{pgfscope}%
\pgfpathrectangle{\pgfqpoint{1.020000in}{0.880000in}}{\pgfqpoint{6.160000in}{6.160000in}}%
\pgfusepath{clip}%
\pgfsetbuttcap%
\pgfsetroundjoin%
\definecolor{currentfill}{rgb}{0.538004,0.674902,0.991722}%
\pgfsetfillcolor{currentfill}%
\pgfsetlinewidth{0.000000pt}%
\definecolor{currentstroke}{rgb}{0.000000,0.000000,0.000000}%
\pgfsetstrokecolor{currentstroke}%
\pgfsetdash{}{0pt}%
\pgfpathmoveto{\pgfqpoint{4.779748in}{3.878467in}}%
\pgfpathlineto{\pgfqpoint{4.790187in}{3.864853in}}%
\pgfpathlineto{\pgfqpoint{4.800643in}{3.851044in}}%
\pgfpathlineto{\pgfqpoint{4.832225in}{3.849181in}}%
\pgfpathlineto{\pgfqpoint{4.863787in}{3.847410in}}%
\pgfpathlineto{\pgfqpoint{4.853282in}{3.861143in}}%
\pgfpathlineto{\pgfqpoint{4.842795in}{3.874688in}}%
\pgfpathlineto{\pgfqpoint{4.811281in}{3.876533in}}%
\pgfpathlineto{\pgfqpoint{4.779748in}{3.878467in}}%
\pgfpathclose%
\pgfusepath{fill}%
\end{pgfscope}%
\begin{pgfscope}%
\pgfpathrectangle{\pgfqpoint{1.020000in}{0.880000in}}{\pgfqpoint{6.160000in}{6.160000in}}%
\pgfusepath{clip}%
\pgfsetbuttcap%
\pgfsetroundjoin%
\definecolor{currentfill}{rgb}{0.478462,0.616564,0.972721}%
\pgfsetfillcolor{currentfill}%
\pgfsetlinewidth{0.000000pt}%
\definecolor{currentstroke}{rgb}{0.000000,0.000000,0.000000}%
\pgfsetstrokecolor{currentstroke}%
\pgfsetdash{}{0pt}%
\pgfpathmoveto{\pgfqpoint{5.347104in}{3.774810in}}%
\pgfpathlineto{\pgfqpoint{5.358037in}{3.760959in}}%
\pgfpathlineto{\pgfqpoint{5.368988in}{3.746969in}}%
\pgfpathlineto{\pgfqpoint{5.400397in}{3.746245in}}%
\pgfpathlineto{\pgfqpoint{5.431786in}{3.745547in}}%
\pgfpathlineto{\pgfqpoint{5.420786in}{3.759428in}}%
\pgfpathlineto{\pgfqpoint{5.409804in}{3.773179in}}%
\pgfpathlineto{\pgfqpoint{5.378463in}{3.773979in}}%
\pgfpathlineto{\pgfqpoint{5.347104in}{3.774810in}}%
\pgfpathclose%
\pgfusepath{fill}%
\end{pgfscope}%
\begin{pgfscope}%
\pgfpathrectangle{\pgfqpoint{1.020000in}{0.880000in}}{\pgfqpoint{6.160000in}{6.160000in}}%
\pgfusepath{clip}%
\pgfsetbuttcap%
\pgfsetroundjoin%
\definecolor{currentfill}{rgb}{0.619318,0.744121,0.998931}%
\pgfsetfillcolor{currentfill}%
\pgfsetlinewidth{0.000000pt}%
\definecolor{currentstroke}{rgb}{0.000000,0.000000,0.000000}%
\pgfsetstrokecolor{currentstroke}%
\pgfsetdash{}{0pt}%
\pgfpathmoveto{\pgfqpoint{4.212448in}{4.012939in}}%
\pgfpathlineto{\pgfqpoint{4.222394in}{4.001187in}}%
\pgfpathlineto{\pgfqpoint{4.232358in}{3.989220in}}%
\pgfpathlineto{\pgfqpoint{4.264133in}{3.982919in}}%
\pgfpathlineto{\pgfqpoint{4.295885in}{3.977195in}}%
\pgfpathlineto{\pgfqpoint{4.285873in}{3.989488in}}%
\pgfpathlineto{\pgfqpoint{4.275878in}{4.001571in}}%
\pgfpathlineto{\pgfqpoint{4.244174in}{4.006997in}}%
\pgfpathlineto{\pgfqpoint{4.212448in}{4.012939in}}%
\pgfpathclose%
\pgfusepath{fill}%
\end{pgfscope}%
\begin{pgfscope}%
\pgfpathrectangle{\pgfqpoint{1.020000in}{0.880000in}}{\pgfqpoint{6.160000in}{6.160000in}}%
\pgfusepath{clip}%
\pgfsetbuttcap%
\pgfsetroundjoin%
\definecolor{currentfill}{rgb}{0.494638,0.633022,0.978983}%
\pgfsetfillcolor{currentfill}%
\pgfsetlinewidth{0.000000pt}%
\definecolor{currentstroke}{rgb}{0.000000,0.000000,0.000000}%
\pgfsetstrokecolor{currentstroke}%
\pgfsetdash{}{0pt}%
\pgfpathmoveto{\pgfqpoint{5.137009in}{3.808396in}}%
\pgfpathlineto{\pgfqpoint{5.147760in}{3.794517in}}%
\pgfpathlineto{\pgfqpoint{5.158529in}{3.780463in}}%
\pgfpathlineto{\pgfqpoint{5.190008in}{3.779440in}}%
\pgfpathlineto{\pgfqpoint{5.221467in}{3.778451in}}%
\pgfpathlineto{\pgfqpoint{5.210649in}{3.792379in}}%
\pgfpathlineto{\pgfqpoint{5.199849in}{3.806148in}}%
\pgfpathlineto{\pgfqpoint{5.168438in}{3.807252in}}%
\pgfpathlineto{\pgfqpoint{5.137009in}{3.808396in}}%
\pgfpathclose%
\pgfusepath{fill}%
\end{pgfscope}%
\begin{pgfscope}%
\pgfpathrectangle{\pgfqpoint{1.020000in}{0.880000in}}{\pgfqpoint{6.160000in}{6.160000in}}%
\pgfusepath{clip}%
\pgfsetbuttcap%
\pgfsetroundjoin%
\definecolor{currentfill}{rgb}{0.565182,0.699438,0.996635}%
\pgfsetfillcolor{currentfill}%
\pgfsetlinewidth{0.000000pt}%
\definecolor{currentstroke}{rgb}{0.000000,0.000000,0.000000}%
\pgfsetstrokecolor{currentstroke}%
\pgfsetdash{}{0pt}%
\pgfpathmoveto{\pgfqpoint{4.569583in}{3.919554in}}%
\pgfpathlineto{\pgfqpoint{4.579842in}{3.906216in}}%
\pgfpathlineto{\pgfqpoint{4.590118in}{3.892679in}}%
\pgfpathlineto{\pgfqpoint{4.621774in}{3.889911in}}%
\pgfpathlineto{\pgfqpoint{4.653410in}{3.887332in}}%
\pgfpathlineto{\pgfqpoint{4.643085in}{3.900851in}}%
\pgfpathlineto{\pgfqpoint{4.632778in}{3.914177in}}%
\pgfpathlineto{\pgfqpoint{4.601191in}{3.916778in}}%
\pgfpathlineto{\pgfqpoint{4.569583in}{3.919554in}}%
\pgfpathclose%
\pgfusepath{fill}%
\end{pgfscope}%
\begin{pgfscope}%
\pgfpathrectangle{\pgfqpoint{1.020000in}{0.880000in}}{\pgfqpoint{6.160000in}{6.160000in}}%
\pgfusepath{clip}%
\pgfsetbuttcap%
\pgfsetroundjoin%
\definecolor{currentfill}{rgb}{0.733898,0.820018,0.970724}%
\pgfsetfillcolor{currentfill}%
\pgfsetlinewidth{0.000000pt}%
\definecolor{currentstroke}{rgb}{0.000000,0.000000,0.000000}%
\pgfsetstrokecolor{currentstroke}%
\pgfsetdash{}{0pt}%
\pgfpathmoveto{\pgfqpoint{3.708415in}{4.207987in}}%
\pgfpathlineto{\pgfqpoint{3.717878in}{4.204738in}}%
\pgfpathlineto{\pgfqpoint{3.727360in}{4.201601in}}%
\pgfpathlineto{\pgfqpoint{3.759371in}{4.185376in}}%
\pgfpathlineto{\pgfqpoint{3.791349in}{4.170250in}}%
\pgfpathlineto{\pgfqpoint{3.781807in}{4.175309in}}%
\pgfpathlineto{\pgfqpoint{3.772284in}{4.180405in}}%
\pgfpathlineto{\pgfqpoint{3.740365in}{4.193730in}}%
\pgfpathlineto{\pgfqpoint{3.708415in}{4.207987in}}%
\pgfpathclose%
\pgfusepath{fill}%
\end{pgfscope}%
\begin{pgfscope}%
\pgfpathrectangle{\pgfqpoint{1.020000in}{0.880000in}}{\pgfqpoint{6.160000in}{6.160000in}}%
\pgfusepath{clip}%
\pgfsetbuttcap%
\pgfsetroundjoin%
\definecolor{currentfill}{rgb}{0.521696,0.659599,0.987736}%
\pgfsetfillcolor{currentfill}%
\pgfsetlinewidth{0.000000pt}%
\definecolor{currentstroke}{rgb}{0.000000,0.000000,0.000000}%
\pgfsetstrokecolor{currentstroke}%
\pgfsetdash{}{0pt}%
\pgfpathmoveto{\pgfqpoint{4.926850in}{3.844109in}}%
\pgfpathlineto{\pgfqpoint{4.937421in}{3.830272in}}%
\pgfpathlineto{\pgfqpoint{4.948009in}{3.816236in}}%
\pgfpathlineto{\pgfqpoint{4.979559in}{3.814793in}}%
\pgfpathlineto{\pgfqpoint{5.011089in}{3.813411in}}%
\pgfpathlineto{\pgfqpoint{5.000452in}{3.827337in}}%
\pgfpathlineto{\pgfqpoint{4.989833in}{3.841077in}}%
\pgfpathlineto{\pgfqpoint{4.958352in}{3.842563in}}%
\pgfpathlineto{\pgfqpoint{4.926850in}{3.844109in}}%
\pgfpathclose%
\pgfusepath{fill}%
\end{pgfscope}%
\begin{pgfscope}%
\pgfpathrectangle{\pgfqpoint{1.020000in}{0.880000in}}{\pgfqpoint{6.160000in}{6.160000in}}%
\pgfusepath{clip}%
\pgfsetbuttcap%
\pgfsetroundjoin%
\definecolor{currentfill}{rgb}{0.698454,0.799450,0.984577}%
\pgfsetfillcolor{currentfill}%
\pgfsetlinewidth{0.000000pt}%
\definecolor{currentstroke}{rgb}{0.000000,0.000000,0.000000}%
\pgfsetstrokecolor{currentstroke}%
\pgfsetdash{}{0pt}%
\pgfpathmoveto{\pgfqpoint{3.855209in}{4.143162in}}%
\pgfpathlineto{\pgfqpoint{3.864826in}{4.136461in}}%
\pgfpathlineto{\pgfqpoint{3.874461in}{4.129679in}}%
\pgfpathlineto{\pgfqpoint{3.906401in}{4.116209in}}%
\pgfpathlineto{\pgfqpoint{3.938310in}{4.103821in}}%
\pgfpathlineto{\pgfqpoint{3.928620in}{4.111969in}}%
\pgfpathlineto{\pgfqpoint{3.918948in}{4.120002in}}%
\pgfpathlineto{\pgfqpoint{3.887093in}{4.131115in}}%
\pgfpathlineto{\pgfqpoint{3.855209in}{4.143162in}}%
\pgfpathclose%
\pgfusepath{fill}%
\end{pgfscope}%
\begin{pgfscope}%
\pgfpathrectangle{\pgfqpoint{1.020000in}{0.880000in}}{\pgfqpoint{6.160000in}{6.160000in}}%
\pgfusepath{clip}%
\pgfsetbuttcap%
\pgfsetroundjoin%
\definecolor{currentfill}{rgb}{0.592356,0.722792,0.999434}%
\pgfsetfillcolor{currentfill}%
\pgfsetlinewidth{0.000000pt}%
\definecolor{currentstroke}{rgb}{0.000000,0.000000,0.000000}%
\pgfsetstrokecolor{currentstroke}%
\pgfsetdash{}{0pt}%
\pgfpathmoveto{\pgfqpoint{4.359323in}{3.967228in}}%
\pgfpathlineto{\pgfqpoint{4.369401in}{3.954509in}}%
\pgfpathlineto{\pgfqpoint{4.379497in}{3.941578in}}%
\pgfpathlineto{\pgfqpoint{4.411232in}{3.937080in}}%
\pgfpathlineto{\pgfqpoint{4.442944in}{3.932971in}}%
\pgfpathlineto{\pgfqpoint{4.432800in}{3.946026in}}%
\pgfpathlineto{\pgfqpoint{4.422672in}{3.958877in}}%
\pgfpathlineto{\pgfqpoint{4.391008in}{3.962875in}}%
\pgfpathlineto{\pgfqpoint{4.359323in}{3.967228in}}%
\pgfpathclose%
\pgfusepath{fill}%
\end{pgfscope}%
\begin{pgfscope}%
\pgfpathrectangle{\pgfqpoint{1.020000in}{0.880000in}}{\pgfqpoint{6.160000in}{6.160000in}}%
\pgfusepath{clip}%
\pgfsetbuttcap%
\pgfsetroundjoin%
\definecolor{currentfill}{rgb}{0.543440,0.680003,0.993051}%
\pgfsetfillcolor{currentfill}%
\pgfsetlinewidth{0.000000pt}%
\definecolor{currentstroke}{rgb}{0.000000,0.000000,0.000000}%
\pgfsetstrokecolor{currentstroke}%
\pgfsetdash{}{0pt}%
\pgfpathmoveto{\pgfqpoint{4.716620in}{3.882645in}}%
\pgfpathlineto{\pgfqpoint{4.727010in}{3.868972in}}%
\pgfpathlineto{\pgfqpoint{4.737417in}{3.855097in}}%
\pgfpathlineto{\pgfqpoint{4.769040in}{3.853011in}}%
\pgfpathlineto{\pgfqpoint{4.800643in}{3.851044in}}%
\pgfpathlineto{\pgfqpoint{4.790187in}{3.864853in}}%
\pgfpathlineto{\pgfqpoint{4.779748in}{3.878467in}}%
\pgfpathlineto{\pgfqpoint{4.748194in}{3.880500in}}%
\pgfpathlineto{\pgfqpoint{4.716620in}{3.882645in}}%
\pgfpathclose%
\pgfusepath{fill}%
\end{pgfscope}%
\begin{pgfscope}%
\pgfpathrectangle{\pgfqpoint{1.020000in}{0.880000in}}{\pgfqpoint{6.160000in}{6.160000in}}%
\pgfusepath{clip}%
\pgfsetbuttcap%
\pgfsetroundjoin%
\definecolor{currentfill}{rgb}{0.478462,0.616564,0.972721}%
\pgfsetfillcolor{currentfill}%
\pgfsetlinewidth{0.000000pt}%
\definecolor{currentstroke}{rgb}{0.000000,0.000000,0.000000}%
\pgfsetstrokecolor{currentstroke}%
\pgfsetdash{}{0pt}%
\pgfpathmoveto{\pgfqpoint{5.284325in}{3.776568in}}%
\pgfpathlineto{\pgfqpoint{5.295210in}{3.762604in}}%
\pgfpathlineto{\pgfqpoint{5.306112in}{3.748487in}}%
\pgfpathlineto{\pgfqpoint{5.337560in}{3.747717in}}%
\pgfpathlineto{\pgfqpoint{5.368988in}{3.746969in}}%
\pgfpathlineto{\pgfqpoint{5.358037in}{3.760959in}}%
\pgfpathlineto{\pgfqpoint{5.347104in}{3.774810in}}%
\pgfpathlineto{\pgfqpoint{5.315724in}{3.775673in}}%
\pgfpathlineto{\pgfqpoint{5.284325in}{3.776568in}}%
\pgfpathclose%
\pgfusepath{fill}%
\end{pgfscope}%
\begin{pgfscope}%
\pgfpathrectangle{\pgfqpoint{1.020000in}{0.880000in}}{\pgfqpoint{6.160000in}{6.160000in}}%
\pgfusepath{clip}%
\pgfsetbuttcap%
\pgfsetroundjoin%
\definecolor{currentfill}{rgb}{0.661968,0.775491,0.993937}%
\pgfsetfillcolor{currentfill}%
\pgfsetlinewidth{0.000000pt}%
\definecolor{currentstroke}{rgb}{0.000000,0.000000,0.000000}%
\pgfsetstrokecolor{currentstroke}%
\pgfsetdash{}{0pt}%
\pgfpathmoveto{\pgfqpoint{4.002041in}{4.082039in}}%
\pgfpathlineto{\pgfqpoint{4.011802in}{4.072649in}}%
\pgfpathlineto{\pgfqpoint{4.021580in}{4.063077in}}%
\pgfpathlineto{\pgfqpoint{4.053456in}{4.052637in}}%
\pgfpathlineto{\pgfqpoint{4.085305in}{4.043131in}}%
\pgfpathlineto{\pgfqpoint{4.075475in}{4.053561in}}%
\pgfpathlineto{\pgfqpoint{4.065664in}{4.063801in}}%
\pgfpathlineto{\pgfqpoint{4.033866in}{4.072511in}}%
\pgfpathlineto{\pgfqpoint{4.002041in}{4.082039in}}%
\pgfpathclose%
\pgfusepath{fill}%
\end{pgfscope}%
\begin{pgfscope}%
\pgfpathrectangle{\pgfqpoint{1.020000in}{0.880000in}}{\pgfqpoint{6.160000in}{6.160000in}}%
\pgfusepath{clip}%
\pgfsetbuttcap%
\pgfsetroundjoin%
\definecolor{currentfill}{rgb}{0.500031,0.638508,0.981070}%
\pgfsetfillcolor{currentfill}%
\pgfsetlinewidth{0.000000pt}%
\definecolor{currentstroke}{rgb}{0.000000,0.000000,0.000000}%
\pgfsetstrokecolor{currentstroke}%
\pgfsetdash{}{0pt}%
\pgfpathmoveto{\pgfqpoint{5.074089in}{3.810810in}}%
\pgfpathlineto{\pgfqpoint{5.084791in}{3.796811in}}%
\pgfpathlineto{\pgfqpoint{5.095511in}{3.782616in}}%
\pgfpathlineto{\pgfqpoint{5.127030in}{3.781521in}}%
\pgfpathlineto{\pgfqpoint{5.158529in}{3.780463in}}%
\pgfpathlineto{\pgfqpoint{5.147760in}{3.794517in}}%
\pgfpathlineto{\pgfqpoint{5.137009in}{3.808396in}}%
\pgfpathlineto{\pgfqpoint{5.105559in}{3.809582in}}%
\pgfpathlineto{\pgfqpoint{5.074089in}{3.810810in}}%
\pgfpathclose%
\pgfusepath{fill}%
\end{pgfscope}%
\begin{pgfscope}%
\pgfpathrectangle{\pgfqpoint{1.020000in}{0.880000in}}{\pgfqpoint{6.160000in}{6.160000in}}%
\pgfusepath{clip}%
\pgfsetbuttcap%
\pgfsetroundjoin%
\definecolor{currentfill}{rgb}{0.570616,0.704109,0.997195}%
\pgfsetfillcolor{currentfill}%
\pgfsetlinewidth{0.000000pt}%
\definecolor{currentstroke}{rgb}{0.000000,0.000000,0.000000}%
\pgfsetstrokecolor{currentstroke}%
\pgfsetdash{}{0pt}%
\pgfpathmoveto{\pgfqpoint{4.506306in}{3.925737in}}%
\pgfpathlineto{\pgfqpoint{4.516516in}{3.912415in}}%
\pgfpathlineto{\pgfqpoint{4.526744in}{3.898890in}}%
\pgfpathlineto{\pgfqpoint{4.558441in}{3.895661in}}%
\pgfpathlineto{\pgfqpoint{4.590118in}{3.892679in}}%
\pgfpathlineto{\pgfqpoint{4.579842in}{3.906216in}}%
\pgfpathlineto{\pgfqpoint{4.569583in}{3.919554in}}%
\pgfpathlineto{\pgfqpoint{4.537955in}{3.922531in}}%
\pgfpathlineto{\pgfqpoint{4.506306in}{3.925737in}}%
\pgfpathclose%
\pgfusepath{fill}%
\end{pgfscope}%
\begin{pgfscope}%
\pgfpathrectangle{\pgfqpoint{1.020000in}{0.880000in}}{\pgfqpoint{6.160000in}{6.160000in}}%
\pgfusepath{clip}%
\pgfsetbuttcap%
\pgfsetroundjoin%
\definecolor{currentfill}{rgb}{0.630089,0.752516,0.998508}%
\pgfsetfillcolor{currentfill}%
\pgfsetlinewidth{0.000000pt}%
\definecolor{currentstroke}{rgb}{0.000000,0.000000,0.000000}%
\pgfsetstrokecolor{currentstroke}%
\pgfsetdash{}{0pt}%
\pgfpathmoveto{\pgfqpoint{4.148925in}{4.026614in}}%
\pgfpathlineto{\pgfqpoint{4.158823in}{4.015332in}}%
\pgfpathlineto{\pgfqpoint{4.168737in}{4.003831in}}%
\pgfpathlineto{\pgfqpoint{4.200560in}{3.996166in}}%
\pgfpathlineto{\pgfqpoint{4.232358in}{3.989220in}}%
\pgfpathlineto{\pgfqpoint{4.222394in}{4.001187in}}%
\pgfpathlineto{\pgfqpoint{4.212448in}{4.012939in}}%
\pgfpathlineto{\pgfqpoint{4.180699in}{4.019456in}}%
\pgfpathlineto{\pgfqpoint{4.148925in}{4.026614in}}%
\pgfpathclose%
\pgfusepath{fill}%
\end{pgfscope}%
\begin{pgfscope}%
\pgfpathrectangle{\pgfqpoint{1.020000in}{0.880000in}}{\pgfqpoint{6.160000in}{6.160000in}}%
\pgfusepath{clip}%
\pgfsetbuttcap%
\pgfsetroundjoin%
\definecolor{currentfill}{rgb}{0.527132,0.664700,0.989065}%
\pgfsetfillcolor{currentfill}%
\pgfsetlinewidth{0.000000pt}%
\definecolor{currentstroke}{rgb}{0.000000,0.000000,0.000000}%
\pgfsetstrokecolor{currentstroke}%
\pgfsetdash{}{0pt}%
\pgfpathmoveto{\pgfqpoint{4.863787in}{3.847410in}}%
\pgfpathlineto{\pgfqpoint{4.874309in}{3.833480in}}%
\pgfpathlineto{\pgfqpoint{4.884848in}{3.819335in}}%
\pgfpathlineto{\pgfqpoint{4.916438in}{3.817747in}}%
\pgfpathlineto{\pgfqpoint{4.948009in}{3.816236in}}%
\pgfpathlineto{\pgfqpoint{4.937421in}{3.830272in}}%
\pgfpathlineto{\pgfqpoint{4.926850in}{3.844109in}}%
\pgfpathlineto{\pgfqpoint{4.895329in}{3.845723in}}%
\pgfpathlineto{\pgfqpoint{4.863787in}{3.847410in}}%
\pgfpathclose%
\pgfusepath{fill}%
\end{pgfscope}%
\begin{pgfscope}%
\pgfpathrectangle{\pgfqpoint{1.020000in}{0.880000in}}{\pgfqpoint{6.160000in}{6.160000in}}%
\pgfusepath{clip}%
\pgfsetbuttcap%
\pgfsetroundjoin%
\definecolor{currentfill}{rgb}{0.603162,0.731527,0.999565}%
\pgfsetfillcolor{currentfill}%
\pgfsetlinewidth{0.000000pt}%
\definecolor{currentstroke}{rgb}{0.000000,0.000000,0.000000}%
\pgfsetstrokecolor{currentstroke}%
\pgfsetdash{}{0pt}%
\pgfpathmoveto{\pgfqpoint{4.295885in}{3.977195in}}%
\pgfpathlineto{\pgfqpoint{4.305916in}{3.964687in}}%
\pgfpathlineto{\pgfqpoint{4.315963in}{3.951960in}}%
\pgfpathlineto{\pgfqpoint{4.347741in}{3.946518in}}%
\pgfpathlineto{\pgfqpoint{4.379497in}{3.941578in}}%
\pgfpathlineto{\pgfqpoint{4.369401in}{3.954509in}}%
\pgfpathlineto{\pgfqpoint{4.359323in}{3.967228in}}%
\pgfpathlineto{\pgfqpoint{4.327615in}{3.971984in}}%
\pgfpathlineto{\pgfqpoint{4.295885in}{3.977195in}}%
\pgfpathclose%
\pgfusepath{fill}%
\end{pgfscope}%
\begin{pgfscope}%
\pgfpathrectangle{\pgfqpoint{1.020000in}{0.880000in}}{\pgfqpoint{6.160000in}{6.160000in}}%
\pgfusepath{clip}%
\pgfsetbuttcap%
\pgfsetroundjoin%
\definecolor{currentfill}{rgb}{0.753611,0.830233,0.960871}%
\pgfsetfillcolor{currentfill}%
\pgfsetlinewidth{0.000000pt}%
\definecolor{currentstroke}{rgb}{0.000000,0.000000,0.000000}%
\pgfsetstrokecolor{currentstroke}%
\pgfsetdash{}{0pt}%
\pgfpathmoveto{\pgfqpoint{3.644415in}{4.239356in}}%
\pgfpathlineto{\pgfqpoint{3.653814in}{4.238287in}}%
\pgfpathlineto{\pgfqpoint{3.663232in}{4.237416in}}%
\pgfpathlineto{\pgfqpoint{3.695313in}{4.218944in}}%
\pgfpathlineto{\pgfqpoint{3.727360in}{4.201601in}}%
\pgfpathlineto{\pgfqpoint{3.717878in}{4.204738in}}%
\pgfpathlineto{\pgfqpoint{3.708415in}{4.207987in}}%
\pgfpathlineto{\pgfqpoint{3.676432in}{4.223194in}}%
\pgfpathlineto{\pgfqpoint{3.644415in}{4.239356in}}%
\pgfpathclose%
\pgfusepath{fill}%
\end{pgfscope}%
\begin{pgfscope}%
\pgfpathrectangle{\pgfqpoint{1.020000in}{0.880000in}}{\pgfqpoint{6.160000in}{6.160000in}}%
\pgfusepath{clip}%
\pgfsetbuttcap%
\pgfsetroundjoin%
\definecolor{currentfill}{rgb}{0.467678,0.605591,0.968546}%
\pgfsetfillcolor{currentfill}%
\pgfsetlinewidth{0.000000pt}%
\definecolor{currentstroke}{rgb}{0.000000,0.000000,0.000000}%
\pgfsetstrokecolor{currentstroke}%
\pgfsetdash{}{0pt}%
\pgfpathmoveto{\pgfqpoint{5.431786in}{3.745547in}}%
\pgfpathlineto{\pgfqpoint{5.442804in}{3.731537in}}%
\pgfpathlineto{\pgfqpoint{5.453840in}{3.717399in}}%
\pgfpathlineto{\pgfqpoint{5.485258in}{3.716841in}}%
\pgfpathlineto{\pgfqpoint{5.474197in}{3.730920in}}%
\pgfpathlineto{\pgfqpoint{5.463155in}{3.744875in}}%
\pgfpathlineto{\pgfqpoint{5.431786in}{3.745547in}}%
\pgfpathclose%
\pgfusepath{fill}%
\end{pgfscope}%
\begin{pgfscope}%
\pgfpathrectangle{\pgfqpoint{1.020000in}{0.880000in}}{\pgfqpoint{6.160000in}{6.160000in}}%
\pgfusepath{clip}%
\pgfsetbuttcap%
\pgfsetroundjoin%
\definecolor{currentfill}{rgb}{0.548876,0.685104,0.994379}%
\pgfsetfillcolor{currentfill}%
\pgfsetlinewidth{0.000000pt}%
\definecolor{currentstroke}{rgb}{0.000000,0.000000,0.000000}%
\pgfsetstrokecolor{currentstroke}%
\pgfsetdash{}{0pt}%
\pgfpathmoveto{\pgfqpoint{4.653410in}{3.887332in}}%
\pgfpathlineto{\pgfqpoint{4.663752in}{3.873614in}}%
\pgfpathlineto{\pgfqpoint{4.674111in}{3.859690in}}%
\pgfpathlineto{\pgfqpoint{4.705774in}{3.857317in}}%
\pgfpathlineto{\pgfqpoint{4.737417in}{3.855097in}}%
\pgfpathlineto{\pgfqpoint{4.727010in}{3.868972in}}%
\pgfpathlineto{\pgfqpoint{4.716620in}{3.882645in}}%
\pgfpathlineto{\pgfqpoint{4.685025in}{3.884917in}}%
\pgfpathlineto{\pgfqpoint{4.653410in}{3.887332in}}%
\pgfpathclose%
\pgfusepath{fill}%
\end{pgfscope}%
\begin{pgfscope}%
\pgfpathrectangle{\pgfqpoint{1.020000in}{0.880000in}}{\pgfqpoint{6.160000in}{6.160000in}}%
\pgfusepath{clip}%
\pgfsetbuttcap%
\pgfsetroundjoin%
\definecolor{currentfill}{rgb}{0.483854,0.622050,0.974808}%
\pgfsetfillcolor{currentfill}%
\pgfsetlinewidth{0.000000pt}%
\definecolor{currentstroke}{rgb}{0.000000,0.000000,0.000000}%
\pgfsetstrokecolor{currentstroke}%
\pgfsetdash{}{0pt}%
\pgfpathmoveto{\pgfqpoint{5.221467in}{3.778451in}}%
\pgfpathlineto{\pgfqpoint{5.232302in}{3.764358in}}%
\pgfpathlineto{\pgfqpoint{5.243155in}{3.750090in}}%
\pgfpathlineto{\pgfqpoint{5.274644in}{3.749278in}}%
\pgfpathlineto{\pgfqpoint{5.306112in}{3.748487in}}%
\pgfpathlineto{\pgfqpoint{5.295210in}{3.762604in}}%
\pgfpathlineto{\pgfqpoint{5.284325in}{3.776568in}}%
\pgfpathlineto{\pgfqpoint{5.252906in}{3.777494in}}%
\pgfpathlineto{\pgfqpoint{5.221467in}{3.778451in}}%
\pgfpathclose%
\pgfusepath{fill}%
\end{pgfscope}%
\begin{pgfscope}%
\pgfpathrectangle{\pgfqpoint{1.020000in}{0.880000in}}{\pgfqpoint{6.160000in}{6.160000in}}%
\pgfusepath{clip}%
\pgfsetbuttcap%
\pgfsetroundjoin%
\definecolor{currentfill}{rgb}{0.718985,0.811993,0.977656}%
\pgfsetfillcolor{currentfill}%
\pgfsetlinewidth{0.000000pt}%
\definecolor{currentstroke}{rgb}{0.000000,0.000000,0.000000}%
\pgfsetstrokecolor{currentstroke}%
\pgfsetdash{}{0pt}%
\pgfpathmoveto{\pgfqpoint{3.791349in}{4.170250in}}%
\pgfpathlineto{\pgfqpoint{3.800909in}{4.165194in}}%
\pgfpathlineto{\pgfqpoint{3.810487in}{4.160099in}}%
\pgfpathlineto{\pgfqpoint{3.842490in}{4.144292in}}%
\pgfpathlineto{\pgfqpoint{3.874461in}{4.129679in}}%
\pgfpathlineto{\pgfqpoint{3.864826in}{4.136461in}}%
\pgfpathlineto{\pgfqpoint{3.855209in}{4.143162in}}%
\pgfpathlineto{\pgfqpoint{3.823295in}{4.156191in}}%
\pgfpathlineto{\pgfqpoint{3.791349in}{4.170250in}}%
\pgfpathclose%
\pgfusepath{fill}%
\end{pgfscope}%
\begin{pgfscope}%
\pgfpathrectangle{\pgfqpoint{1.020000in}{0.880000in}}{\pgfqpoint{6.160000in}{6.160000in}}%
\pgfusepath{clip}%
\pgfsetbuttcap%
\pgfsetroundjoin%
\definecolor{currentfill}{rgb}{0.505423,0.643995,0.983157}%
\pgfsetfillcolor{currentfill}%
\pgfsetlinewidth{0.000000pt}%
\definecolor{currentstroke}{rgb}{0.000000,0.000000,0.000000}%
\pgfsetstrokecolor{currentstroke}%
\pgfsetdash{}{0pt}%
\pgfpathmoveto{\pgfqpoint{5.011089in}{3.813411in}}%
\pgfpathlineto{\pgfqpoint{5.021743in}{3.799285in}}%
\pgfpathlineto{\pgfqpoint{5.032413in}{3.784938in}}%
\pgfpathlineto{\pgfqpoint{5.063972in}{3.783754in}}%
\pgfpathlineto{\pgfqpoint{5.095511in}{3.782616in}}%
\pgfpathlineto{\pgfqpoint{5.084791in}{3.796811in}}%
\pgfpathlineto{\pgfqpoint{5.074089in}{3.810810in}}%
\pgfpathlineto{\pgfqpoint{5.042599in}{3.812086in}}%
\pgfpathlineto{\pgfqpoint{5.011089in}{3.813411in}}%
\pgfpathclose%
\pgfusepath{fill}%
\end{pgfscope}%
\begin{pgfscope}%
\pgfpathrectangle{\pgfqpoint{1.020000in}{0.880000in}}{\pgfqpoint{6.160000in}{6.160000in}}%
\pgfusepath{clip}%
\pgfsetbuttcap%
\pgfsetroundjoin%
\definecolor{currentfill}{rgb}{0.677823,0.786546,0.991005}%
\pgfsetfillcolor{currentfill}%
\pgfsetlinewidth{0.000000pt}%
\definecolor{currentstroke}{rgb}{0.000000,0.000000,0.000000}%
\pgfsetstrokecolor{currentstroke}%
\pgfsetdash{}{0pt}%
\pgfpathmoveto{\pgfqpoint{3.938310in}{4.103821in}}%
\pgfpathlineto{\pgfqpoint{3.948018in}{4.095532in}}%
\pgfpathlineto{\pgfqpoint{3.957744in}{4.087071in}}%
\pgfpathlineto{\pgfqpoint{3.989676in}{4.074529in}}%
\pgfpathlineto{\pgfqpoint{4.021580in}{4.063077in}}%
\pgfpathlineto{\pgfqpoint{4.011802in}{4.072649in}}%
\pgfpathlineto{\pgfqpoint{4.002041in}{4.082039in}}%
\pgfpathlineto{\pgfqpoint{3.970190in}{4.092454in}}%
\pgfpathlineto{\pgfqpoint{3.938310in}{4.103821in}}%
\pgfpathclose%
\pgfusepath{fill}%
\end{pgfscope}%
\begin{pgfscope}%
\pgfpathrectangle{\pgfqpoint{1.020000in}{0.880000in}}{\pgfqpoint{6.160000in}{6.160000in}}%
\pgfusepath{clip}%
\pgfsetbuttcap%
\pgfsetroundjoin%
\definecolor{currentfill}{rgb}{0.581486,0.713451,0.998314}%
\pgfsetfillcolor{currentfill}%
\pgfsetlinewidth{0.000000pt}%
\definecolor{currentstroke}{rgb}{0.000000,0.000000,0.000000}%
\pgfsetstrokecolor{currentstroke}%
\pgfsetdash{}{0pt}%
\pgfpathmoveto{\pgfqpoint{4.442944in}{3.932971in}}%
\pgfpathlineto{\pgfqpoint{4.453106in}{3.919708in}}%
\pgfpathlineto{\pgfqpoint{4.463285in}{3.906235in}}%
\pgfpathlineto{\pgfqpoint{4.495025in}{3.902401in}}%
\pgfpathlineto{\pgfqpoint{4.526744in}{3.898890in}}%
\pgfpathlineto{\pgfqpoint{4.516516in}{3.912415in}}%
\pgfpathlineto{\pgfqpoint{4.506306in}{3.925737in}}%
\pgfpathlineto{\pgfqpoint{4.474635in}{3.929204in}}%
\pgfpathlineto{\pgfqpoint{4.442944in}{3.932971in}}%
\pgfpathclose%
\pgfusepath{fill}%
\end{pgfscope}%
\begin{pgfscope}%
\pgfpathrectangle{\pgfqpoint{1.020000in}{0.880000in}}{\pgfqpoint{6.160000in}{6.160000in}}%
\pgfusepath{clip}%
\pgfsetbuttcap%
\pgfsetroundjoin%
\definecolor{currentfill}{rgb}{0.646113,0.764436,0.996868}%
\pgfsetfillcolor{currentfill}%
\pgfsetlinewidth{0.000000pt}%
\definecolor{currentstroke}{rgb}{0.000000,0.000000,0.000000}%
\pgfsetstrokecolor{currentstroke}%
\pgfsetdash{}{0pt}%
\pgfpathmoveto{\pgfqpoint{4.085305in}{4.043131in}}%
\pgfpathlineto{\pgfqpoint{4.095152in}{4.032494in}}%
\pgfpathlineto{\pgfqpoint{4.105017in}{4.021632in}}%
\pgfpathlineto{\pgfqpoint{4.136890in}{4.012292in}}%
\pgfpathlineto{\pgfqpoint{4.168737in}{4.003831in}}%
\pgfpathlineto{\pgfqpoint{4.158823in}{4.015332in}}%
\pgfpathlineto{\pgfqpoint{4.148925in}{4.026614in}}%
\pgfpathlineto{\pgfqpoint{4.117128in}{4.034482in}}%
\pgfpathlineto{\pgfqpoint{4.085305in}{4.043131in}}%
\pgfpathclose%
\pgfusepath{fill}%
\end{pgfscope}%
\begin{pgfscope}%
\pgfpathrectangle{\pgfqpoint{1.020000in}{0.880000in}}{\pgfqpoint{6.160000in}{6.160000in}}%
\pgfusepath{clip}%
\pgfsetbuttcap%
\pgfsetroundjoin%
\definecolor{currentfill}{rgb}{0.532568,0.669801,0.990393}%
\pgfsetfillcolor{currentfill}%
\pgfsetlinewidth{0.000000pt}%
\definecolor{currentstroke}{rgb}{0.000000,0.000000,0.000000}%
\pgfsetstrokecolor{currentstroke}%
\pgfsetdash{}{0pt}%
\pgfpathmoveto{\pgfqpoint{4.800643in}{3.851044in}}%
\pgfpathlineto{\pgfqpoint{4.811116in}{3.837027in}}%
\pgfpathlineto{\pgfqpoint{4.821606in}{3.822784in}}%
\pgfpathlineto{\pgfqpoint{4.853237in}{3.821011in}}%
\pgfpathlineto{\pgfqpoint{4.884848in}{3.819335in}}%
\pgfpathlineto{\pgfqpoint{4.874309in}{3.833480in}}%
\pgfpathlineto{\pgfqpoint{4.863787in}{3.847410in}}%
\pgfpathlineto{\pgfqpoint{4.832225in}{3.849181in}}%
\pgfpathlineto{\pgfqpoint{4.800643in}{3.851044in}}%
\pgfpathclose%
\pgfusepath{fill}%
\end{pgfscope}%
\begin{pgfscope}%
\pgfpathrectangle{\pgfqpoint{1.020000in}{0.880000in}}{\pgfqpoint{6.160000in}{6.160000in}}%
\pgfusepath{clip}%
\pgfsetbuttcap%
\pgfsetroundjoin%
\definecolor{currentfill}{rgb}{0.467678,0.605591,0.968546}%
\pgfsetfillcolor{currentfill}%
\pgfsetlinewidth{0.000000pt}%
\definecolor{currentstroke}{rgb}{0.000000,0.000000,0.000000}%
\pgfsetstrokecolor{currentstroke}%
\pgfsetdash{}{0pt}%
\pgfpathmoveto{\pgfqpoint{5.368988in}{3.746969in}}%
\pgfpathlineto{\pgfqpoint{5.379957in}{3.732838in}}%
\pgfpathlineto{\pgfqpoint{5.390944in}{3.718563in}}%
\pgfpathlineto{\pgfqpoint{5.422402in}{3.717974in}}%
\pgfpathlineto{\pgfqpoint{5.453840in}{3.717399in}}%
\pgfpathlineto{\pgfqpoint{5.442804in}{3.731537in}}%
\pgfpathlineto{\pgfqpoint{5.431786in}{3.745547in}}%
\pgfpathlineto{\pgfqpoint{5.400397in}{3.746245in}}%
\pgfpathlineto{\pgfqpoint{5.368988in}{3.746969in}}%
\pgfpathclose%
\pgfusepath{fill}%
\end{pgfscope}%
\begin{pgfscope}%
\pgfpathrectangle{\pgfqpoint{1.020000in}{0.880000in}}{\pgfqpoint{6.160000in}{6.160000in}}%
\pgfusepath{clip}%
\pgfsetbuttcap%
\pgfsetroundjoin%
\definecolor{currentfill}{rgb}{0.489246,0.627536,0.976896}%
\pgfsetfillcolor{currentfill}%
\pgfsetlinewidth{0.000000pt}%
\definecolor{currentstroke}{rgb}{0.000000,0.000000,0.000000}%
\pgfsetstrokecolor{currentstroke}%
\pgfsetdash{}{0pt}%
\pgfpathmoveto{\pgfqpoint{5.158529in}{3.780463in}}%
\pgfpathlineto{\pgfqpoint{5.169315in}{3.766221in}}%
\pgfpathlineto{\pgfqpoint{5.180119in}{3.751773in}}%
\pgfpathlineto{\pgfqpoint{5.211647in}{3.750921in}}%
\pgfpathlineto{\pgfqpoint{5.243155in}{3.750090in}}%
\pgfpathlineto{\pgfqpoint{5.232302in}{3.764358in}}%
\pgfpathlineto{\pgfqpoint{5.221467in}{3.778451in}}%
\pgfpathlineto{\pgfqpoint{5.190008in}{3.779440in}}%
\pgfpathlineto{\pgfqpoint{5.158529in}{3.780463in}}%
\pgfpathclose%
\pgfusepath{fill}%
\end{pgfscope}%
\begin{pgfscope}%
\pgfpathrectangle{\pgfqpoint{1.020000in}{0.880000in}}{\pgfqpoint{6.160000in}{6.160000in}}%
\pgfusepath{clip}%
\pgfsetbuttcap%
\pgfsetroundjoin%
\definecolor{currentfill}{rgb}{0.613933,0.739923,0.999142}%
\pgfsetfillcolor{currentfill}%
\pgfsetlinewidth{0.000000pt}%
\definecolor{currentstroke}{rgb}{0.000000,0.000000,0.000000}%
\pgfsetstrokecolor{currentstroke}%
\pgfsetdash{}{0pt}%
\pgfpathmoveto{\pgfqpoint{4.232358in}{3.989220in}}%
\pgfpathlineto{\pgfqpoint{4.242340in}{3.977031in}}%
\pgfpathlineto{\pgfqpoint{4.252339in}{3.964611in}}%
\pgfpathlineto{\pgfqpoint{4.284163in}{3.957967in}}%
\pgfpathlineto{\pgfqpoint{4.315963in}{3.951960in}}%
\pgfpathlineto{\pgfqpoint{4.305916in}{3.964687in}}%
\pgfpathlineto{\pgfqpoint{4.295885in}{3.977195in}}%
\pgfpathlineto{\pgfqpoint{4.264133in}{3.982919in}}%
\pgfpathlineto{\pgfqpoint{4.232358in}{3.989220in}}%
\pgfpathclose%
\pgfusepath{fill}%
\end{pgfscope}%
\begin{pgfscope}%
\pgfpathrectangle{\pgfqpoint{1.020000in}{0.880000in}}{\pgfqpoint{6.160000in}{6.160000in}}%
\pgfusepath{clip}%
\pgfsetbuttcap%
\pgfsetroundjoin%
\definecolor{currentfill}{rgb}{0.559747,0.694768,0.996075}%
\pgfsetfillcolor{currentfill}%
\pgfsetlinewidth{0.000000pt}%
\definecolor{currentstroke}{rgb}{0.000000,0.000000,0.000000}%
\pgfsetstrokecolor{currentstroke}%
\pgfsetdash{}{0pt}%
\pgfpathmoveto{\pgfqpoint{4.590118in}{3.892679in}}%
\pgfpathlineto{\pgfqpoint{4.600411in}{3.878938in}}%
\pgfpathlineto{\pgfqpoint{4.610722in}{3.864986in}}%
\pgfpathlineto{\pgfqpoint{4.642427in}{3.862238in}}%
\pgfpathlineto{\pgfqpoint{4.674111in}{3.859690in}}%
\pgfpathlineto{\pgfqpoint{4.663752in}{3.873614in}}%
\pgfpathlineto{\pgfqpoint{4.653410in}{3.887332in}}%
\pgfpathlineto{\pgfqpoint{4.621774in}{3.889911in}}%
\pgfpathlineto{\pgfqpoint{4.590118in}{3.892679in}}%
\pgfpathclose%
\pgfusepath{fill}%
\end{pgfscope}%
\begin{pgfscope}%
\pgfpathrectangle{\pgfqpoint{1.020000in}{0.880000in}}{\pgfqpoint{6.160000in}{6.160000in}}%
\pgfusepath{clip}%
\pgfsetbuttcap%
\pgfsetroundjoin%
\definecolor{currentfill}{rgb}{0.516260,0.654498,0.986407}%
\pgfsetfillcolor{currentfill}%
\pgfsetlinewidth{0.000000pt}%
\definecolor{currentstroke}{rgb}{0.000000,0.000000,0.000000}%
\pgfsetstrokecolor{currentstroke}%
\pgfsetdash{}{0pt}%
\pgfpathmoveto{\pgfqpoint{4.948009in}{3.816236in}}%
\pgfpathlineto{\pgfqpoint{4.958613in}{3.801979in}}%
\pgfpathlineto{\pgfqpoint{4.969235in}{3.787474in}}%
\pgfpathlineto{\pgfqpoint{5.000834in}{3.786175in}}%
\pgfpathlineto{\pgfqpoint{5.032413in}{3.784938in}}%
\pgfpathlineto{\pgfqpoint{5.021743in}{3.799285in}}%
\pgfpathlineto{\pgfqpoint{5.011089in}{3.813411in}}%
\pgfpathlineto{\pgfqpoint{4.979559in}{3.814793in}}%
\pgfpathlineto{\pgfqpoint{4.948009in}{3.816236in}}%
\pgfpathclose%
\pgfusepath{fill}%
\end{pgfscope}%
\begin{pgfscope}%
\pgfpathrectangle{\pgfqpoint{1.020000in}{0.880000in}}{\pgfqpoint{6.160000in}{6.160000in}}%
\pgfusepath{clip}%
\pgfsetbuttcap%
\pgfsetroundjoin%
\definecolor{currentfill}{rgb}{0.777378,0.840921,0.946149}%
\pgfsetfillcolor{currentfill}%
\pgfsetlinewidth{0.000000pt}%
\definecolor{currentstroke}{rgb}{0.000000,0.000000,0.000000}%
\pgfsetstrokecolor{currentstroke}%
\pgfsetdash{}{0pt}%
\pgfpathmoveto{\pgfqpoint{3.580277in}{4.274530in}}%
\pgfpathlineto{\pgfqpoint{3.589609in}{4.275973in}}%
\pgfpathlineto{\pgfqpoint{3.598957in}{4.277717in}}%
\pgfpathlineto{\pgfqpoint{3.631113in}{4.257013in}}%
\pgfpathlineto{\pgfqpoint{3.663232in}{4.237416in}}%
\pgfpathlineto{\pgfqpoint{3.653814in}{4.238287in}}%
\pgfpathlineto{\pgfqpoint{3.644415in}{4.239356in}}%
\pgfpathlineto{\pgfqpoint{3.612364in}{4.256474in}}%
\pgfpathlineto{\pgfqpoint{3.580277in}{4.274530in}}%
\pgfpathclose%
\pgfusepath{fill}%
\end{pgfscope}%
\begin{pgfscope}%
\pgfpathrectangle{\pgfqpoint{1.020000in}{0.880000in}}{\pgfqpoint{6.160000in}{6.160000in}}%
\pgfusepath{clip}%
\pgfsetbuttcap%
\pgfsetroundjoin%
\definecolor{currentfill}{rgb}{0.586921,0.718121,0.998874}%
\pgfsetfillcolor{currentfill}%
\pgfsetlinewidth{0.000000pt}%
\definecolor{currentstroke}{rgb}{0.000000,0.000000,0.000000}%
\pgfsetstrokecolor{currentstroke}%
\pgfsetdash{}{0pt}%
\pgfpathmoveto{\pgfqpoint{4.379497in}{3.941578in}}%
\pgfpathlineto{\pgfqpoint{4.389611in}{3.928431in}}%
\pgfpathlineto{\pgfqpoint{4.399742in}{3.915064in}}%
\pgfpathlineto{\pgfqpoint{4.431524in}{3.910439in}}%
\pgfpathlineto{\pgfqpoint{4.463285in}{3.906235in}}%
\pgfpathlineto{\pgfqpoint{4.453106in}{3.919708in}}%
\pgfpathlineto{\pgfqpoint{4.442944in}{3.932971in}}%
\pgfpathlineto{\pgfqpoint{4.411232in}{3.937080in}}%
\pgfpathlineto{\pgfqpoint{4.379497in}{3.941578in}}%
\pgfpathclose%
\pgfusepath{fill}%
\end{pgfscope}%
\begin{pgfscope}%
\pgfpathrectangle{\pgfqpoint{1.020000in}{0.880000in}}{\pgfqpoint{6.160000in}{6.160000in}}%
\pgfusepath{clip}%
\pgfsetbuttcap%
\pgfsetroundjoin%
\definecolor{currentfill}{rgb}{0.738826,0.822572,0.968261}%
\pgfsetfillcolor{currentfill}%
\pgfsetlinewidth{0.000000pt}%
\definecolor{currentstroke}{rgb}{0.000000,0.000000,0.000000}%
\pgfsetstrokecolor{currentstroke}%
\pgfsetdash{}{0pt}%
\pgfpathmoveto{\pgfqpoint{3.727360in}{4.201601in}}%
\pgfpathlineto{\pgfqpoint{3.736859in}{4.198530in}}%
\pgfpathlineto{\pgfqpoint{3.746377in}{4.195473in}}%
\pgfpathlineto{\pgfqpoint{3.778449in}{4.177148in}}%
\pgfpathlineto{\pgfqpoint{3.810487in}{4.160099in}}%
\pgfpathlineto{\pgfqpoint{3.800909in}{4.165194in}}%
\pgfpathlineto{\pgfqpoint{3.791349in}{4.170250in}}%
\pgfpathlineto{\pgfqpoint{3.759371in}{4.185376in}}%
\pgfpathlineto{\pgfqpoint{3.727360in}{4.201601in}}%
\pgfpathclose%
\pgfusepath{fill}%
\end{pgfscope}%
\begin{pgfscope}%
\pgfpathrectangle{\pgfqpoint{1.020000in}{0.880000in}}{\pgfqpoint{6.160000in}{6.160000in}}%
\pgfusepath{clip}%
\pgfsetbuttcap%
\pgfsetroundjoin%
\definecolor{currentfill}{rgb}{0.698454,0.799450,0.984577}%
\pgfsetfillcolor{currentfill}%
\pgfsetlinewidth{0.000000pt}%
\definecolor{currentstroke}{rgb}{0.000000,0.000000,0.000000}%
\pgfsetstrokecolor{currentstroke}%
\pgfsetdash{}{0pt}%
\pgfpathmoveto{\pgfqpoint{3.874461in}{4.129679in}}%
\pgfpathlineto{\pgfqpoint{3.884115in}{4.122781in}}%
\pgfpathlineto{\pgfqpoint{3.893787in}{4.115726in}}%
\pgfpathlineto{\pgfqpoint{3.925781in}{4.100779in}}%
\pgfpathlineto{\pgfqpoint{3.957744in}{4.087071in}}%
\pgfpathlineto{\pgfqpoint{3.948018in}{4.095532in}}%
\pgfpathlineto{\pgfqpoint{3.938310in}{4.103821in}}%
\pgfpathlineto{\pgfqpoint{3.906401in}{4.116209in}}%
\pgfpathlineto{\pgfqpoint{3.874461in}{4.129679in}}%
\pgfpathclose%
\pgfusepath{fill}%
\end{pgfscope}%
\begin{pgfscope}%
\pgfpathrectangle{\pgfqpoint{1.020000in}{0.880000in}}{\pgfqpoint{6.160000in}{6.160000in}}%
\pgfusepath{clip}%
\pgfsetbuttcap%
\pgfsetroundjoin%
\definecolor{currentfill}{rgb}{0.538004,0.674902,0.991722}%
\pgfsetfillcolor{currentfill}%
\pgfsetlinewidth{0.000000pt}%
\definecolor{currentstroke}{rgb}{0.000000,0.000000,0.000000}%
\pgfsetstrokecolor{currentstroke}%
\pgfsetdash{}{0pt}%
\pgfpathmoveto{\pgfqpoint{4.737417in}{3.855097in}}%
\pgfpathlineto{\pgfqpoint{4.747842in}{3.841007in}}%
\pgfpathlineto{\pgfqpoint{4.758284in}{3.826682in}}%
\pgfpathlineto{\pgfqpoint{4.789955in}{3.824670in}}%
\pgfpathlineto{\pgfqpoint{4.821606in}{3.822784in}}%
\pgfpathlineto{\pgfqpoint{4.811116in}{3.837027in}}%
\pgfpathlineto{\pgfqpoint{4.800643in}{3.851044in}}%
\pgfpathlineto{\pgfqpoint{4.769040in}{3.853011in}}%
\pgfpathlineto{\pgfqpoint{4.737417in}{3.855097in}}%
\pgfpathclose%
\pgfusepath{fill}%
\end{pgfscope}%
\begin{pgfscope}%
\pgfpathrectangle{\pgfqpoint{1.020000in}{0.880000in}}{\pgfqpoint{6.160000in}{6.160000in}}%
\pgfusepath{clip}%
\pgfsetbuttcap%
\pgfsetroundjoin%
\definecolor{currentfill}{rgb}{0.473070,0.611077,0.970634}%
\pgfsetfillcolor{currentfill}%
\pgfsetlinewidth{0.000000pt}%
\definecolor{currentstroke}{rgb}{0.000000,0.000000,0.000000}%
\pgfsetstrokecolor{currentstroke}%
\pgfsetdash{}{0pt}%
\pgfpathmoveto{\pgfqpoint{5.306112in}{3.748487in}}%
\pgfpathlineto{\pgfqpoint{5.317031in}{3.734210in}}%
\pgfpathlineto{\pgfqpoint{5.327969in}{3.719763in}}%
\pgfpathlineto{\pgfqpoint{5.359467in}{3.719160in}}%
\pgfpathlineto{\pgfqpoint{5.390944in}{3.718563in}}%
\pgfpathlineto{\pgfqpoint{5.379957in}{3.732838in}}%
\pgfpathlineto{\pgfqpoint{5.368988in}{3.746969in}}%
\pgfpathlineto{\pgfqpoint{5.337560in}{3.747717in}}%
\pgfpathlineto{\pgfqpoint{5.306112in}{3.748487in}}%
\pgfpathclose%
\pgfusepath{fill}%
\end{pgfscope}%
\begin{pgfscope}%
\pgfpathrectangle{\pgfqpoint{1.020000in}{0.880000in}}{\pgfqpoint{6.160000in}{6.160000in}}%
\pgfusepath{clip}%
\pgfsetbuttcap%
\pgfsetroundjoin%
\definecolor{currentfill}{rgb}{0.661968,0.775491,0.993937}%
\pgfsetfillcolor{currentfill}%
\pgfsetlinewidth{0.000000pt}%
\definecolor{currentstroke}{rgb}{0.000000,0.000000,0.000000}%
\pgfsetstrokecolor{currentstroke}%
\pgfsetdash{}{0pt}%
\pgfpathmoveto{\pgfqpoint{4.021580in}{4.063077in}}%
\pgfpathlineto{\pgfqpoint{4.031377in}{4.053298in}}%
\pgfpathlineto{\pgfqpoint{4.041191in}{4.043288in}}%
\pgfpathlineto{\pgfqpoint{4.073118in}{4.031934in}}%
\pgfpathlineto{\pgfqpoint{4.105017in}{4.021632in}}%
\pgfpathlineto{\pgfqpoint{4.095152in}{4.032494in}}%
\pgfpathlineto{\pgfqpoint{4.085305in}{4.043131in}}%
\pgfpathlineto{\pgfqpoint{4.053456in}{4.052637in}}%
\pgfpathlineto{\pgfqpoint{4.021580in}{4.063077in}}%
\pgfpathclose%
\pgfusepath{fill}%
\end{pgfscope}%
\begin{pgfscope}%
\pgfpathrectangle{\pgfqpoint{1.020000in}{0.880000in}}{\pgfqpoint{6.160000in}{6.160000in}}%
\pgfusepath{clip}%
\pgfsetbuttcap%
\pgfsetroundjoin%
\definecolor{currentfill}{rgb}{0.494638,0.633022,0.978983}%
\pgfsetfillcolor{currentfill}%
\pgfsetlinewidth{0.000000pt}%
\definecolor{currentstroke}{rgb}{0.000000,0.000000,0.000000}%
\pgfsetstrokecolor{currentstroke}%
\pgfsetdash{}{0pt}%
\pgfpathmoveto{\pgfqpoint{5.095511in}{3.782616in}}%
\pgfpathlineto{\pgfqpoint{5.106248in}{3.768204in}}%
\pgfpathlineto{\pgfqpoint{5.117002in}{3.753547in}}%
\pgfpathlineto{\pgfqpoint{5.148571in}{3.752647in}}%
\pgfpathlineto{\pgfqpoint{5.180119in}{3.751773in}}%
\pgfpathlineto{\pgfqpoint{5.169315in}{3.766221in}}%
\pgfpathlineto{\pgfqpoint{5.158529in}{3.780463in}}%
\pgfpathlineto{\pgfqpoint{5.127030in}{3.781521in}}%
\pgfpathlineto{\pgfqpoint{5.095511in}{3.782616in}}%
\pgfpathclose%
\pgfusepath{fill}%
\end{pgfscope}%
\begin{pgfscope}%
\pgfpathrectangle{\pgfqpoint{1.020000in}{0.880000in}}{\pgfqpoint{6.160000in}{6.160000in}}%
\pgfusepath{clip}%
\pgfsetbuttcap%
\pgfsetroundjoin%
\definecolor{currentfill}{rgb}{0.565182,0.699438,0.996635}%
\pgfsetfillcolor{currentfill}%
\pgfsetlinewidth{0.000000pt}%
\definecolor{currentstroke}{rgb}{0.000000,0.000000,0.000000}%
\pgfsetstrokecolor{currentstroke}%
\pgfsetdash{}{0pt}%
\pgfpathmoveto{\pgfqpoint{4.526744in}{3.898890in}}%
\pgfpathlineto{\pgfqpoint{4.536989in}{3.885155in}}%
\pgfpathlineto{\pgfqpoint{4.547251in}{3.871204in}}%
\pgfpathlineto{\pgfqpoint{4.578997in}{3.867964in}}%
\pgfpathlineto{\pgfqpoint{4.610722in}{3.864986in}}%
\pgfpathlineto{\pgfqpoint{4.600411in}{3.878938in}}%
\pgfpathlineto{\pgfqpoint{4.590118in}{3.892679in}}%
\pgfpathlineto{\pgfqpoint{4.558441in}{3.895661in}}%
\pgfpathlineto{\pgfqpoint{4.526744in}{3.898890in}}%
\pgfpathclose%
\pgfusepath{fill}%
\end{pgfscope}%
\begin{pgfscope}%
\pgfpathrectangle{\pgfqpoint{1.020000in}{0.880000in}}{\pgfqpoint{6.160000in}{6.160000in}}%
\pgfusepath{clip}%
\pgfsetbuttcap%
\pgfsetroundjoin%
\definecolor{currentfill}{rgb}{0.630089,0.752516,0.998508}%
\pgfsetfillcolor{currentfill}%
\pgfsetlinewidth{0.000000pt}%
\definecolor{currentstroke}{rgb}{0.000000,0.000000,0.000000}%
\pgfsetstrokecolor{currentstroke}%
\pgfsetdash{}{0pt}%
\pgfpathmoveto{\pgfqpoint{4.168737in}{4.003831in}}%
\pgfpathlineto{\pgfqpoint{4.178670in}{3.992098in}}%
\pgfpathlineto{\pgfqpoint{4.188620in}{3.980118in}}%
\pgfpathlineto{\pgfqpoint{4.220492in}{3.971967in}}%
\pgfpathlineto{\pgfqpoint{4.252339in}{3.964611in}}%
\pgfpathlineto{\pgfqpoint{4.242340in}{3.977031in}}%
\pgfpathlineto{\pgfqpoint{4.232358in}{3.989220in}}%
\pgfpathlineto{\pgfqpoint{4.200560in}{3.996166in}}%
\pgfpathlineto{\pgfqpoint{4.168737in}{4.003831in}}%
\pgfpathclose%
\pgfusepath{fill}%
\end{pgfscope}%
\begin{pgfscope}%
\pgfpathrectangle{\pgfqpoint{1.020000in}{0.880000in}}{\pgfqpoint{6.160000in}{6.160000in}}%
\pgfusepath{clip}%
\pgfsetbuttcap%
\pgfsetroundjoin%
\definecolor{currentfill}{rgb}{0.521696,0.659599,0.987736}%
\pgfsetfillcolor{currentfill}%
\pgfsetlinewidth{0.000000pt}%
\definecolor{currentstroke}{rgb}{0.000000,0.000000,0.000000}%
\pgfsetstrokecolor{currentstroke}%
\pgfsetdash{}{0pt}%
\pgfpathmoveto{\pgfqpoint{4.884848in}{3.819335in}}%
\pgfpathlineto{\pgfqpoint{4.895404in}{3.804952in}}%
\pgfpathlineto{\pgfqpoint{4.905977in}{3.790294in}}%
\pgfpathlineto{\pgfqpoint{4.937616in}{3.788843in}}%
\pgfpathlineto{\pgfqpoint{4.969235in}{3.787474in}}%
\pgfpathlineto{\pgfqpoint{4.958613in}{3.801979in}}%
\pgfpathlineto{\pgfqpoint{4.948009in}{3.816236in}}%
\pgfpathlineto{\pgfqpoint{4.916438in}{3.817747in}}%
\pgfpathlineto{\pgfqpoint{4.884848in}{3.819335in}}%
\pgfpathclose%
\pgfusepath{fill}%
\end{pgfscope}%
\begin{pgfscope}%
\pgfpathrectangle{\pgfqpoint{1.020000in}{0.880000in}}{\pgfqpoint{6.160000in}{6.160000in}}%
\pgfusepath{clip}%
\pgfsetbuttcap%
\pgfsetroundjoin%
\definecolor{currentfill}{rgb}{0.597777,0.727330,0.999777}%
\pgfsetfillcolor{currentfill}%
\pgfsetlinewidth{0.000000pt}%
\definecolor{currentstroke}{rgb}{0.000000,0.000000,0.000000}%
\pgfsetstrokecolor{currentstroke}%
\pgfsetdash{}{0pt}%
\pgfpathmoveto{\pgfqpoint{4.315963in}{3.951960in}}%
\pgfpathlineto{\pgfqpoint{4.326029in}{3.939005in}}%
\pgfpathlineto{\pgfqpoint{4.336111in}{3.925818in}}%
\pgfpathlineto{\pgfqpoint{4.367938in}{3.920169in}}%
\pgfpathlineto{\pgfqpoint{4.399742in}{3.915064in}}%
\pgfpathlineto{\pgfqpoint{4.389611in}{3.928431in}}%
\pgfpathlineto{\pgfqpoint{4.379497in}{3.941578in}}%
\pgfpathlineto{\pgfqpoint{4.347741in}{3.946518in}}%
\pgfpathlineto{\pgfqpoint{4.315963in}{3.951960in}}%
\pgfpathclose%
\pgfusepath{fill}%
\end{pgfscope}%
\begin{pgfscope}%
\pgfpathrectangle{\pgfqpoint{1.020000in}{0.880000in}}{\pgfqpoint{6.160000in}{6.160000in}}%
\pgfusepath{clip}%
\pgfsetbuttcap%
\pgfsetroundjoin%
\definecolor{currentfill}{rgb}{0.462354,0.599830,0.965857}%
\pgfsetfillcolor{currentfill}%
\pgfsetlinewidth{0.000000pt}%
\definecolor{currentstroke}{rgb}{0.000000,0.000000,0.000000}%
\pgfsetstrokecolor{currentstroke}%
\pgfsetdash{}{0pt}%
\pgfpathmoveto{\pgfqpoint{5.453840in}{3.717399in}}%
\pgfpathlineto{\pgfqpoint{5.464894in}{3.703132in}}%
\pgfpathlineto{\pgfqpoint{5.475966in}{3.688733in}}%
\pgfpathlineto{\pgfqpoint{5.507433in}{3.688315in}}%
\pgfpathlineto{\pgfqpoint{5.496337in}{3.702640in}}%
\pgfpathlineto{\pgfqpoint{5.485258in}{3.716841in}}%
\pgfpathlineto{\pgfqpoint{5.453840in}{3.717399in}}%
\pgfpathclose%
\pgfusepath{fill}%
\end{pgfscope}%
\begin{pgfscope}%
\pgfpathrectangle{\pgfqpoint{1.020000in}{0.880000in}}{\pgfqpoint{6.160000in}{6.160000in}}%
\pgfusepath{clip}%
\pgfsetbuttcap%
\pgfsetroundjoin%
\definecolor{currentfill}{rgb}{0.543440,0.680003,0.993051}%
\pgfsetfillcolor{currentfill}%
\pgfsetlinewidth{0.000000pt}%
\definecolor{currentstroke}{rgb}{0.000000,0.000000,0.000000}%
\pgfsetstrokecolor{currentstroke}%
\pgfsetdash{}{0pt}%
\pgfpathmoveto{\pgfqpoint{4.674111in}{3.859690in}}%
\pgfpathlineto{\pgfqpoint{4.684487in}{3.845545in}}%
\pgfpathlineto{\pgfqpoint{4.694880in}{3.831159in}}%
\pgfpathlineto{\pgfqpoint{4.726592in}{3.828838in}}%
\pgfpathlineto{\pgfqpoint{4.758284in}{3.826682in}}%
\pgfpathlineto{\pgfqpoint{4.747842in}{3.841007in}}%
\pgfpathlineto{\pgfqpoint{4.737417in}{3.855097in}}%
\pgfpathlineto{\pgfqpoint{4.705774in}{3.857317in}}%
\pgfpathlineto{\pgfqpoint{4.674111in}{3.859690in}}%
\pgfpathclose%
\pgfusepath{fill}%
\end{pgfscope}%
\begin{pgfscope}%
\pgfpathrectangle{\pgfqpoint{1.020000in}{0.880000in}}{\pgfqpoint{6.160000in}{6.160000in}}%
\pgfusepath{clip}%
\pgfsetbuttcap%
\pgfsetroundjoin%
\definecolor{currentfill}{rgb}{0.478462,0.616564,0.972721}%
\pgfsetfillcolor{currentfill}%
\pgfsetlinewidth{0.000000pt}%
\definecolor{currentstroke}{rgb}{0.000000,0.000000,0.000000}%
\pgfsetstrokecolor{currentstroke}%
\pgfsetdash{}{0pt}%
\pgfpathmoveto{\pgfqpoint{5.243155in}{3.750090in}}%
\pgfpathlineto{\pgfqpoint{5.254026in}{3.735632in}}%
\pgfpathlineto{\pgfqpoint{5.264913in}{3.720967in}}%
\pgfpathlineto{\pgfqpoint{5.296451in}{3.720366in}}%
\pgfpathlineto{\pgfqpoint{5.327969in}{3.719763in}}%
\pgfpathlineto{\pgfqpoint{5.317031in}{3.734210in}}%
\pgfpathlineto{\pgfqpoint{5.306112in}{3.748487in}}%
\pgfpathlineto{\pgfqpoint{5.274644in}{3.749278in}}%
\pgfpathlineto{\pgfqpoint{5.243155in}{3.750090in}}%
\pgfpathclose%
\pgfusepath{fill}%
\end{pgfscope}%
\begin{pgfscope}%
\pgfpathrectangle{\pgfqpoint{1.020000in}{0.880000in}}{\pgfqpoint{6.160000in}{6.160000in}}%
\pgfusepath{clip}%
\pgfsetbuttcap%
\pgfsetroundjoin%
\definecolor{currentfill}{rgb}{0.804965,0.851666,0.926165}%
\pgfsetfillcolor{currentfill}%
\pgfsetlinewidth{0.000000pt}%
\definecolor{currentstroke}{rgb}{0.000000,0.000000,0.000000}%
\pgfsetstrokecolor{currentstroke}%
\pgfsetdash{}{0pt}%
\pgfpathmoveto{\pgfqpoint{3.515995in}{4.313335in}}%
\pgfpathlineto{\pgfqpoint{3.525254in}{4.317608in}}%
\pgfpathlineto{\pgfqpoint{3.534530in}{4.322298in}}%
\pgfpathlineto{\pgfqpoint{3.566763in}{4.299496in}}%
\pgfpathlineto{\pgfqpoint{3.598957in}{4.277717in}}%
\pgfpathlineto{\pgfqpoint{3.589609in}{4.275973in}}%
\pgfpathlineto{\pgfqpoint{3.580277in}{4.274530in}}%
\pgfpathlineto{\pgfqpoint{3.548155in}{4.293498in}}%
\pgfpathlineto{\pgfqpoint{3.515995in}{4.313335in}}%
\pgfpathclose%
\pgfusepath{fill}%
\end{pgfscope}%
\begin{pgfscope}%
\pgfpathrectangle{\pgfqpoint{1.020000in}{0.880000in}}{\pgfqpoint{6.160000in}{6.160000in}}%
\pgfusepath{clip}%
\pgfsetbuttcap%
\pgfsetroundjoin%
\definecolor{currentfill}{rgb}{0.500031,0.638508,0.981070}%
\pgfsetfillcolor{currentfill}%
\pgfsetlinewidth{0.000000pt}%
\definecolor{currentstroke}{rgb}{0.000000,0.000000,0.000000}%
\pgfsetstrokecolor{currentstroke}%
\pgfsetdash{}{0pt}%
\pgfpathmoveto{\pgfqpoint{5.032413in}{3.784938in}}%
\pgfpathlineto{\pgfqpoint{5.043101in}{3.770339in}}%
\pgfpathlineto{\pgfqpoint{5.053805in}{3.755447in}}%
\pgfpathlineto{\pgfqpoint{5.085414in}{3.754478in}}%
\pgfpathlineto{\pgfqpoint{5.117002in}{3.753547in}}%
\pgfpathlineto{\pgfqpoint{5.106248in}{3.768204in}}%
\pgfpathlineto{\pgfqpoint{5.095511in}{3.782616in}}%
\pgfpathlineto{\pgfqpoint{5.063972in}{3.783754in}}%
\pgfpathlineto{\pgfqpoint{5.032413in}{3.784938in}}%
\pgfpathclose%
\pgfusepath{fill}%
\end{pgfscope}%
\begin{pgfscope}%
\pgfpathrectangle{\pgfqpoint{1.020000in}{0.880000in}}{\pgfqpoint{6.160000in}{6.160000in}}%
\pgfusepath{clip}%
\pgfsetbuttcap%
\pgfsetroundjoin%
\definecolor{currentfill}{rgb}{0.763363,0.835092,0.955658}%
\pgfsetfillcolor{currentfill}%
\pgfsetlinewidth{0.000000pt}%
\definecolor{currentstroke}{rgb}{0.000000,0.000000,0.000000}%
\pgfsetstrokecolor{currentstroke}%
\pgfsetdash{}{0pt}%
\pgfpathmoveto{\pgfqpoint{3.663232in}{4.237416in}}%
\pgfpathlineto{\pgfqpoint{3.672667in}{4.236688in}}%
\pgfpathlineto{\pgfqpoint{3.682121in}{4.236038in}}%
\pgfpathlineto{\pgfqpoint{3.714268in}{4.215099in}}%
\pgfpathlineto{\pgfqpoint{3.746377in}{4.195473in}}%
\pgfpathlineto{\pgfqpoint{3.736859in}{4.198530in}}%
\pgfpathlineto{\pgfqpoint{3.727360in}{4.201601in}}%
\pgfpathlineto{\pgfqpoint{3.695313in}{4.218944in}}%
\pgfpathlineto{\pgfqpoint{3.663232in}{4.237416in}}%
\pgfpathclose%
\pgfusepath{fill}%
\end{pgfscope}%
\begin{pgfscope}%
\pgfpathrectangle{\pgfqpoint{1.020000in}{0.880000in}}{\pgfqpoint{6.160000in}{6.160000in}}%
\pgfusepath{clip}%
\pgfsetbuttcap%
\pgfsetroundjoin%
\definecolor{currentfill}{rgb}{0.718985,0.811993,0.977656}%
\pgfsetfillcolor{currentfill}%
\pgfsetlinewidth{0.000000pt}%
\definecolor{currentstroke}{rgb}{0.000000,0.000000,0.000000}%
\pgfsetstrokecolor{currentstroke}%
\pgfsetdash{}{0pt}%
\pgfpathmoveto{\pgfqpoint{3.810487in}{4.160099in}}%
\pgfpathlineto{\pgfqpoint{3.820084in}{4.154920in}}%
\pgfpathlineto{\pgfqpoint{3.829699in}{4.149604in}}%
\pgfpathlineto{\pgfqpoint{3.861760in}{4.131981in}}%
\pgfpathlineto{\pgfqpoint{3.893787in}{4.115726in}}%
\pgfpathlineto{\pgfqpoint{3.884115in}{4.122781in}}%
\pgfpathlineto{\pgfqpoint{3.874461in}{4.129679in}}%
\pgfpathlineto{\pgfqpoint{3.842490in}{4.144292in}}%
\pgfpathlineto{\pgfqpoint{3.810487in}{4.160099in}}%
\pgfpathclose%
\pgfusepath{fill}%
\end{pgfscope}%
\begin{pgfscope}%
\pgfpathrectangle{\pgfqpoint{1.020000in}{0.880000in}}{\pgfqpoint{6.160000in}{6.160000in}}%
\pgfusepath{clip}%
\pgfsetbuttcap%
\pgfsetroundjoin%
\definecolor{currentfill}{rgb}{0.576051,0.708780,0.997755}%
\pgfsetfillcolor{currentfill}%
\pgfsetlinewidth{0.000000pt}%
\definecolor{currentstroke}{rgb}{0.000000,0.000000,0.000000}%
\pgfsetstrokecolor{currentstroke}%
\pgfsetdash{}{0pt}%
\pgfpathmoveto{\pgfqpoint{4.463285in}{3.906235in}}%
\pgfpathlineto{\pgfqpoint{4.473482in}{3.892545in}}%
\pgfpathlineto{\pgfqpoint{4.483696in}{3.878632in}}%
\pgfpathlineto{\pgfqpoint{4.515484in}{3.874746in}}%
\pgfpathlineto{\pgfqpoint{4.547251in}{3.871204in}}%
\pgfpathlineto{\pgfqpoint{4.536989in}{3.885155in}}%
\pgfpathlineto{\pgfqpoint{4.526744in}{3.898890in}}%
\pgfpathlineto{\pgfqpoint{4.495025in}{3.902401in}}%
\pgfpathlineto{\pgfqpoint{4.463285in}{3.906235in}}%
\pgfpathclose%
\pgfusepath{fill}%
\end{pgfscope}%
\begin{pgfscope}%
\pgfpathrectangle{\pgfqpoint{1.020000in}{0.880000in}}{\pgfqpoint{6.160000in}{6.160000in}}%
\pgfusepath{clip}%
\pgfsetbuttcap%
\pgfsetroundjoin%
\definecolor{currentfill}{rgb}{0.677823,0.786546,0.991005}%
\pgfsetfillcolor{currentfill}%
\pgfsetlinewidth{0.000000pt}%
\definecolor{currentstroke}{rgb}{0.000000,0.000000,0.000000}%
\pgfsetstrokecolor{currentstroke}%
\pgfsetdash{}{0pt}%
\pgfpathmoveto{\pgfqpoint{3.957744in}{4.087071in}}%
\pgfpathlineto{\pgfqpoint{3.967488in}{4.078406in}}%
\pgfpathlineto{\pgfqpoint{3.977251in}{4.069502in}}%
\pgfpathlineto{\pgfqpoint{4.009236in}{4.055781in}}%
\pgfpathlineto{\pgfqpoint{4.041191in}{4.043288in}}%
\pgfpathlineto{\pgfqpoint{4.031377in}{4.053298in}}%
\pgfpathlineto{\pgfqpoint{4.021580in}{4.063077in}}%
\pgfpathlineto{\pgfqpoint{3.989676in}{4.074529in}}%
\pgfpathlineto{\pgfqpoint{3.957744in}{4.087071in}}%
\pgfpathclose%
\pgfusepath{fill}%
\end{pgfscope}%
\begin{pgfscope}%
\pgfpathrectangle{\pgfqpoint{1.020000in}{0.880000in}}{\pgfqpoint{6.160000in}{6.160000in}}%
\pgfusepath{clip}%
\pgfsetbuttcap%
\pgfsetroundjoin%
\definecolor{currentfill}{rgb}{0.527132,0.664700,0.989065}%
\pgfsetfillcolor{currentfill}%
\pgfsetlinewidth{0.000000pt}%
\definecolor{currentstroke}{rgb}{0.000000,0.000000,0.000000}%
\pgfsetstrokecolor{currentstroke}%
\pgfsetdash{}{0pt}%
\pgfpathmoveto{\pgfqpoint{4.821606in}{3.822784in}}%
\pgfpathlineto{\pgfqpoint{4.832114in}{3.808286in}}%
\pgfpathlineto{\pgfqpoint{4.842638in}{3.793491in}}%
\pgfpathlineto{\pgfqpoint{4.874317in}{3.791838in}}%
\pgfpathlineto{\pgfqpoint{4.905977in}{3.790294in}}%
\pgfpathlineto{\pgfqpoint{4.895404in}{3.804952in}}%
\pgfpathlineto{\pgfqpoint{4.884848in}{3.819335in}}%
\pgfpathlineto{\pgfqpoint{4.853237in}{3.821011in}}%
\pgfpathlineto{\pgfqpoint{4.821606in}{3.822784in}}%
\pgfpathclose%
\pgfusepath{fill}%
\end{pgfscope}%
\begin{pgfscope}%
\pgfpathrectangle{\pgfqpoint{1.020000in}{0.880000in}}{\pgfqpoint{6.160000in}{6.160000in}}%
\pgfusepath{clip}%
\pgfsetbuttcap%
\pgfsetroundjoin%
\definecolor{currentfill}{rgb}{0.640828,0.760752,0.997846}%
\pgfsetfillcolor{currentfill}%
\pgfsetlinewidth{0.000000pt}%
\definecolor{currentstroke}{rgb}{0.000000,0.000000,0.000000}%
\pgfsetstrokecolor{currentstroke}%
\pgfsetdash{}{0pt}%
\pgfpathmoveto{\pgfqpoint{4.105017in}{4.021632in}}%
\pgfpathlineto{\pgfqpoint{4.114900in}{4.010525in}}%
\pgfpathlineto{\pgfqpoint{4.124801in}{3.999155in}}%
\pgfpathlineto{\pgfqpoint{4.156724in}{3.989150in}}%
\pgfpathlineto{\pgfqpoint{4.188620in}{3.980118in}}%
\pgfpathlineto{\pgfqpoint{4.178670in}{3.992098in}}%
\pgfpathlineto{\pgfqpoint{4.168737in}{4.003831in}}%
\pgfpathlineto{\pgfqpoint{4.136890in}{4.012292in}}%
\pgfpathlineto{\pgfqpoint{4.105017in}{4.021632in}}%
\pgfpathclose%
\pgfusepath{fill}%
\end{pgfscope}%
\begin{pgfscope}%
\pgfpathrectangle{\pgfqpoint{1.020000in}{0.880000in}}{\pgfqpoint{6.160000in}{6.160000in}}%
\pgfusepath{clip}%
\pgfsetbuttcap%
\pgfsetroundjoin%
\definecolor{currentfill}{rgb}{0.462354,0.599830,0.965857}%
\pgfsetfillcolor{currentfill}%
\pgfsetlinewidth{0.000000pt}%
\definecolor{currentstroke}{rgb}{0.000000,0.000000,0.000000}%
\pgfsetstrokecolor{currentstroke}%
\pgfsetdash{}{0pt}%
\pgfpathmoveto{\pgfqpoint{5.390944in}{3.718563in}}%
\pgfpathlineto{\pgfqpoint{5.401949in}{3.704138in}}%
\pgfpathlineto{\pgfqpoint{5.412971in}{3.689557in}}%
\pgfpathlineto{\pgfqpoint{5.444479in}{3.689149in}}%
\pgfpathlineto{\pgfqpoint{5.475966in}{3.688733in}}%
\pgfpathlineto{\pgfqpoint{5.464894in}{3.703132in}}%
\pgfpathlineto{\pgfqpoint{5.453840in}{3.717399in}}%
\pgfpathlineto{\pgfqpoint{5.422402in}{3.717974in}}%
\pgfpathlineto{\pgfqpoint{5.390944in}{3.718563in}}%
\pgfpathclose%
\pgfusepath{fill}%
\end{pgfscope}%
\begin{pgfscope}%
\pgfpathrectangle{\pgfqpoint{1.020000in}{0.880000in}}{\pgfqpoint{6.160000in}{6.160000in}}%
\pgfusepath{clip}%
\pgfsetbuttcap%
\pgfsetroundjoin%
\definecolor{currentfill}{rgb}{0.483854,0.622050,0.974808}%
\pgfsetfillcolor{currentfill}%
\pgfsetlinewidth{0.000000pt}%
\definecolor{currentstroke}{rgb}{0.000000,0.000000,0.000000}%
\pgfsetstrokecolor{currentstroke}%
\pgfsetdash{}{0pt}%
\pgfpathmoveto{\pgfqpoint{5.180119in}{3.751773in}}%
\pgfpathlineto{\pgfqpoint{5.190939in}{3.737095in}}%
\pgfpathlineto{\pgfqpoint{5.201776in}{3.722155in}}%
\pgfpathlineto{\pgfqpoint{5.233355in}{3.721564in}}%
\pgfpathlineto{\pgfqpoint{5.264913in}{3.720967in}}%
\pgfpathlineto{\pgfqpoint{5.254026in}{3.735632in}}%
\pgfpathlineto{\pgfqpoint{5.243155in}{3.750090in}}%
\pgfpathlineto{\pgfqpoint{5.211647in}{3.750921in}}%
\pgfpathlineto{\pgfqpoint{5.180119in}{3.751773in}}%
\pgfpathclose%
\pgfusepath{fill}%
\end{pgfscope}%
\begin{pgfscope}%
\pgfpathrectangle{\pgfqpoint{1.020000in}{0.880000in}}{\pgfqpoint{6.160000in}{6.160000in}}%
\pgfusepath{clip}%
\pgfsetbuttcap%
\pgfsetroundjoin%
\definecolor{currentfill}{rgb}{0.554312,0.690097,0.995516}%
\pgfsetfillcolor{currentfill}%
\pgfsetlinewidth{0.000000pt}%
\definecolor{currentstroke}{rgb}{0.000000,0.000000,0.000000}%
\pgfsetstrokecolor{currentstroke}%
\pgfsetdash{}{0pt}%
\pgfpathmoveto{\pgfqpoint{4.610722in}{3.864986in}}%
\pgfpathlineto{\pgfqpoint{4.621050in}{3.850810in}}%
\pgfpathlineto{\pgfqpoint{4.631395in}{3.836388in}}%
\pgfpathlineto{\pgfqpoint{4.663148in}{3.833666in}}%
\pgfpathlineto{\pgfqpoint{4.694880in}{3.831159in}}%
\pgfpathlineto{\pgfqpoint{4.684487in}{3.845545in}}%
\pgfpathlineto{\pgfqpoint{4.674111in}{3.859690in}}%
\pgfpathlineto{\pgfqpoint{4.642427in}{3.862238in}}%
\pgfpathlineto{\pgfqpoint{4.610722in}{3.864986in}}%
\pgfpathclose%
\pgfusepath{fill}%
\end{pgfscope}%
\begin{pgfscope}%
\pgfpathrectangle{\pgfqpoint{1.020000in}{0.880000in}}{\pgfqpoint{6.160000in}{6.160000in}}%
\pgfusepath{clip}%
\pgfsetbuttcap%
\pgfsetroundjoin%
\definecolor{currentfill}{rgb}{0.608547,0.735725,0.999354}%
\pgfsetfillcolor{currentfill}%
\pgfsetlinewidth{0.000000pt}%
\definecolor{currentstroke}{rgb}{0.000000,0.000000,0.000000}%
\pgfsetstrokecolor{currentstroke}%
\pgfsetdash{}{0pt}%
\pgfpathmoveto{\pgfqpoint{4.252339in}{3.964611in}}%
\pgfpathlineto{\pgfqpoint{4.262356in}{3.951949in}}%
\pgfpathlineto{\pgfqpoint{4.272390in}{3.939037in}}%
\pgfpathlineto{\pgfqpoint{4.304262in}{3.932081in}}%
\pgfpathlineto{\pgfqpoint{4.336111in}{3.925818in}}%
\pgfpathlineto{\pgfqpoint{4.326029in}{3.939005in}}%
\pgfpathlineto{\pgfqpoint{4.315963in}{3.951960in}}%
\pgfpathlineto{\pgfqpoint{4.284163in}{3.957967in}}%
\pgfpathlineto{\pgfqpoint{4.252339in}{3.964611in}}%
\pgfpathclose%
\pgfusepath{fill}%
\end{pgfscope}%
\begin{pgfscope}%
\pgfpathrectangle{\pgfqpoint{1.020000in}{0.880000in}}{\pgfqpoint{6.160000in}{6.160000in}}%
\pgfusepath{clip}%
\pgfsetbuttcap%
\pgfsetroundjoin%
\definecolor{currentfill}{rgb}{0.505423,0.643995,0.983157}%
\pgfsetfillcolor{currentfill}%
\pgfsetlinewidth{0.000000pt}%
\definecolor{currentstroke}{rgb}{0.000000,0.000000,0.000000}%
\pgfsetstrokecolor{currentstroke}%
\pgfsetdash{}{0pt}%
\pgfpathmoveto{\pgfqpoint{4.969235in}{3.787474in}}%
\pgfpathlineto{\pgfqpoint{4.979873in}{3.772680in}}%
\pgfpathlineto{\pgfqpoint{4.990528in}{3.757542in}}%
\pgfpathlineto{\pgfqpoint{5.022176in}{3.756465in}}%
\pgfpathlineto{\pgfqpoint{5.053805in}{3.755447in}}%
\pgfpathlineto{\pgfqpoint{5.043101in}{3.770339in}}%
\pgfpathlineto{\pgfqpoint{5.032413in}{3.784938in}}%
\pgfpathlineto{\pgfqpoint{5.000834in}{3.786175in}}%
\pgfpathlineto{\pgfqpoint{4.969235in}{3.787474in}}%
\pgfpathclose%
\pgfusepath{fill}%
\end{pgfscope}%
\begin{pgfscope}%
\pgfpathrectangle{\pgfqpoint{1.020000in}{0.880000in}}{\pgfqpoint{6.160000in}{6.160000in}}%
\pgfusepath{clip}%
\pgfsetbuttcap%
\pgfsetroundjoin%
\definecolor{currentfill}{rgb}{0.581486,0.713451,0.998314}%
\pgfsetfillcolor{currentfill}%
\pgfsetlinewidth{0.000000pt}%
\definecolor{currentstroke}{rgb}{0.000000,0.000000,0.000000}%
\pgfsetstrokecolor{currentstroke}%
\pgfsetdash{}{0pt}%
\pgfpathmoveto{\pgfqpoint{4.399742in}{3.915064in}}%
\pgfpathlineto{\pgfqpoint{4.409890in}{3.901471in}}%
\pgfpathlineto{\pgfqpoint{4.420056in}{3.887644in}}%
\pgfpathlineto{\pgfqpoint{4.451887in}{3.882913in}}%
\pgfpathlineto{\pgfqpoint{4.483696in}{3.878632in}}%
\pgfpathlineto{\pgfqpoint{4.473482in}{3.892545in}}%
\pgfpathlineto{\pgfqpoint{4.463285in}{3.906235in}}%
\pgfpathlineto{\pgfqpoint{4.431524in}{3.910439in}}%
\pgfpathlineto{\pgfqpoint{4.399742in}{3.915064in}}%
\pgfpathclose%
\pgfusepath{fill}%
\end{pgfscope}%
\begin{pgfscope}%
\pgfpathrectangle{\pgfqpoint{1.020000in}{0.880000in}}{\pgfqpoint{6.160000in}{6.160000in}}%
\pgfusepath{clip}%
\pgfsetbuttcap%
\pgfsetroundjoin%
\definecolor{currentfill}{rgb}{0.532568,0.669801,0.990393}%
\pgfsetfillcolor{currentfill}%
\pgfsetlinewidth{0.000000pt}%
\definecolor{currentstroke}{rgb}{0.000000,0.000000,0.000000}%
\pgfsetstrokecolor{currentstroke}%
\pgfsetdash{}{0pt}%
\pgfpathmoveto{\pgfqpoint{4.758284in}{3.826682in}}%
\pgfpathlineto{\pgfqpoint{4.768743in}{3.812089in}}%
\pgfpathlineto{\pgfqpoint{4.779218in}{3.797185in}}%
\pgfpathlineto{\pgfqpoint{4.810938in}{3.795268in}}%
\pgfpathlineto{\pgfqpoint{4.842638in}{3.793491in}}%
\pgfpathlineto{\pgfqpoint{4.832114in}{3.808286in}}%
\pgfpathlineto{\pgfqpoint{4.821606in}{3.822784in}}%
\pgfpathlineto{\pgfqpoint{4.789955in}{3.824670in}}%
\pgfpathlineto{\pgfqpoint{4.758284in}{3.826682in}}%
\pgfpathclose%
\pgfusepath{fill}%
\end{pgfscope}%
\begin{pgfscope}%
\pgfpathrectangle{\pgfqpoint{1.020000in}{0.880000in}}{\pgfqpoint{6.160000in}{6.160000in}}%
\pgfusepath{clip}%
\pgfsetbuttcap%
\pgfsetroundjoin%
\definecolor{currentfill}{rgb}{0.467678,0.605591,0.968546}%
\pgfsetfillcolor{currentfill}%
\pgfsetlinewidth{0.000000pt}%
\definecolor{currentstroke}{rgb}{0.000000,0.000000,0.000000}%
\pgfsetstrokecolor{currentstroke}%
\pgfsetdash{}{0pt}%
\pgfpathmoveto{\pgfqpoint{5.327969in}{3.719763in}}%
\pgfpathlineto{\pgfqpoint{5.338923in}{3.705134in}}%
\pgfpathlineto{\pgfqpoint{5.349895in}{3.690308in}}%
\pgfpathlineto{\pgfqpoint{5.381443in}{3.689946in}}%
\pgfpathlineto{\pgfqpoint{5.412971in}{3.689557in}}%
\pgfpathlineto{\pgfqpoint{5.401949in}{3.704138in}}%
\pgfpathlineto{\pgfqpoint{5.390944in}{3.718563in}}%
\pgfpathlineto{\pgfqpoint{5.359467in}{3.719160in}}%
\pgfpathlineto{\pgfqpoint{5.327969in}{3.719763in}}%
\pgfpathclose%
\pgfusepath{fill}%
\end{pgfscope}%
\begin{pgfscope}%
\pgfpathrectangle{\pgfqpoint{1.020000in}{0.880000in}}{\pgfqpoint{6.160000in}{6.160000in}}%
\pgfusepath{clip}%
\pgfsetbuttcap%
\pgfsetroundjoin%
\definecolor{currentfill}{rgb}{0.743754,0.825125,0.965798}%
\pgfsetfillcolor{currentfill}%
\pgfsetlinewidth{0.000000pt}%
\definecolor{currentstroke}{rgb}{0.000000,0.000000,0.000000}%
\pgfsetstrokecolor{currentstroke}%
\pgfsetdash{}{0pt}%
\pgfpathmoveto{\pgfqpoint{3.746377in}{4.195473in}}%
\pgfpathlineto{\pgfqpoint{3.755913in}{4.192371in}}%
\pgfpathlineto{\pgfqpoint{3.765469in}{4.189156in}}%
\pgfpathlineto{\pgfqpoint{3.797602in}{4.168649in}}%
\pgfpathlineto{\pgfqpoint{3.829699in}{4.149604in}}%
\pgfpathlineto{\pgfqpoint{3.820084in}{4.154920in}}%
\pgfpathlineto{\pgfqpoint{3.810487in}{4.160099in}}%
\pgfpathlineto{\pgfqpoint{3.778449in}{4.177148in}}%
\pgfpathlineto{\pgfqpoint{3.746377in}{4.195473in}}%
\pgfpathclose%
\pgfusepath{fill}%
\end{pgfscope}%
\begin{pgfscope}%
\pgfpathrectangle{\pgfqpoint{1.020000in}{0.880000in}}{\pgfqpoint{6.160000in}{6.160000in}}%
\pgfusepath{clip}%
\pgfsetbuttcap%
\pgfsetroundjoin%
\definecolor{currentfill}{rgb}{0.698454,0.799450,0.984577}%
\pgfsetfillcolor{currentfill}%
\pgfsetlinewidth{0.000000pt}%
\definecolor{currentstroke}{rgb}{0.000000,0.000000,0.000000}%
\pgfsetstrokecolor{currentstroke}%
\pgfsetdash{}{0pt}%
\pgfpathmoveto{\pgfqpoint{3.893787in}{4.115726in}}%
\pgfpathlineto{\pgfqpoint{3.903477in}{4.108471in}}%
\pgfpathlineto{\pgfqpoint{3.913187in}{4.100969in}}%
\pgfpathlineto{\pgfqpoint{3.945235in}{4.084537in}}%
\pgfpathlineto{\pgfqpoint{3.977251in}{4.069502in}}%
\pgfpathlineto{\pgfqpoint{3.967488in}{4.078406in}}%
\pgfpathlineto{\pgfqpoint{3.957744in}{4.087071in}}%
\pgfpathlineto{\pgfqpoint{3.925781in}{4.100779in}}%
\pgfpathlineto{\pgfqpoint{3.893787in}{4.115726in}}%
\pgfpathclose%
\pgfusepath{fill}%
\end{pgfscope}%
\begin{pgfscope}%
\pgfpathrectangle{\pgfqpoint{1.020000in}{0.880000in}}{\pgfqpoint{6.160000in}{6.160000in}}%
\pgfusepath{clip}%
\pgfsetbuttcap%
\pgfsetroundjoin%
\definecolor{currentfill}{rgb}{0.791392,0.846750,0.936641}%
\pgfsetfillcolor{currentfill}%
\pgfsetlinewidth{0.000000pt}%
\definecolor{currentstroke}{rgb}{0.000000,0.000000,0.000000}%
\pgfsetstrokecolor{currentstroke}%
\pgfsetdash{}{0pt}%
\pgfpathmoveto{\pgfqpoint{3.598957in}{4.277717in}}%
\pgfpathlineto{\pgfqpoint{3.608324in}{4.279693in}}%
\pgfpathlineto{\pgfqpoint{3.617709in}{4.281819in}}%
\pgfpathlineto{\pgfqpoint{3.649935in}{4.258285in}}%
\pgfpathlineto{\pgfqpoint{3.682121in}{4.236038in}}%
\pgfpathlineto{\pgfqpoint{3.672667in}{4.236688in}}%
\pgfpathlineto{\pgfqpoint{3.663232in}{4.237416in}}%
\pgfpathlineto{\pgfqpoint{3.631113in}{4.257013in}}%
\pgfpathlineto{\pgfqpoint{3.598957in}{4.277717in}}%
\pgfpathclose%
\pgfusepath{fill}%
\end{pgfscope}%
\begin{pgfscope}%
\pgfpathrectangle{\pgfqpoint{1.020000in}{0.880000in}}{\pgfqpoint{6.160000in}{6.160000in}}%
\pgfusepath{clip}%
\pgfsetbuttcap%
\pgfsetroundjoin%
\definecolor{currentfill}{rgb}{0.831148,0.859513,0.903110}%
\pgfsetfillcolor{currentfill}%
\pgfsetlinewidth{0.000000pt}%
\definecolor{currentstroke}{rgb}{0.000000,0.000000,0.000000}%
\pgfsetstrokecolor{currentstroke}%
\pgfsetdash{}{0pt}%
\pgfpathmoveto{\pgfqpoint{3.451566in}{4.355376in}}%
\pgfpathlineto{\pgfqpoint{3.460747in}{4.362760in}}%
\pgfpathlineto{\pgfqpoint{3.469943in}{4.370689in}}%
\pgfpathlineto{\pgfqpoint{3.502256in}{4.346058in}}%
\pgfpathlineto{\pgfqpoint{3.534530in}{4.322298in}}%
\pgfpathlineto{\pgfqpoint{3.525254in}{4.317608in}}%
\pgfpathlineto{\pgfqpoint{3.515995in}{4.313335in}}%
\pgfpathlineto{\pgfqpoint{3.483799in}{4.333985in}}%
\pgfpathlineto{\pgfqpoint{3.451566in}{4.355376in}}%
\pgfpathclose%
\pgfusepath{fill}%
\end{pgfscope}%
\begin{pgfscope}%
\pgfpathrectangle{\pgfqpoint{1.020000in}{0.880000in}}{\pgfqpoint{6.160000in}{6.160000in}}%
\pgfusepath{clip}%
\pgfsetbuttcap%
\pgfsetroundjoin%
\definecolor{currentfill}{rgb}{0.661968,0.775491,0.993937}%
\pgfsetfillcolor{currentfill}%
\pgfsetlinewidth{0.000000pt}%
\definecolor{currentstroke}{rgb}{0.000000,0.000000,0.000000}%
\pgfsetstrokecolor{currentstroke}%
\pgfsetdash{}{0pt}%
\pgfpathmoveto{\pgfqpoint{4.041191in}{4.043288in}}%
\pgfpathlineto{\pgfqpoint{4.051024in}{4.033019in}}%
\pgfpathlineto{\pgfqpoint{4.060875in}{4.022464in}}%
\pgfpathlineto{\pgfqpoint{4.092852in}{4.010227in}}%
\pgfpathlineto{\pgfqpoint{4.124801in}{3.999155in}}%
\pgfpathlineto{\pgfqpoint{4.114900in}{4.010525in}}%
\pgfpathlineto{\pgfqpoint{4.105017in}{4.021632in}}%
\pgfpathlineto{\pgfqpoint{4.073118in}{4.031934in}}%
\pgfpathlineto{\pgfqpoint{4.041191in}{4.043288in}}%
\pgfpathclose%
\pgfusepath{fill}%
\end{pgfscope}%
\begin{pgfscope}%
\pgfpathrectangle{\pgfqpoint{1.020000in}{0.880000in}}{\pgfqpoint{6.160000in}{6.160000in}}%
\pgfusepath{clip}%
\pgfsetbuttcap%
\pgfsetroundjoin%
\definecolor{currentfill}{rgb}{0.489246,0.627536,0.976896}%
\pgfsetfillcolor{currentfill}%
\pgfsetlinewidth{0.000000pt}%
\definecolor{currentstroke}{rgb}{0.000000,0.000000,0.000000}%
\pgfsetstrokecolor{currentstroke}%
\pgfsetdash{}{0pt}%
\pgfpathmoveto{\pgfqpoint{5.117002in}{3.753547in}}%
\pgfpathlineto{\pgfqpoint{5.127772in}{3.738606in}}%
\pgfpathlineto{\pgfqpoint{5.138558in}{3.723333in}}%
\pgfpathlineto{\pgfqpoint{5.170177in}{3.722743in}}%
\pgfpathlineto{\pgfqpoint{5.201776in}{3.722155in}}%
\pgfpathlineto{\pgfqpoint{5.190939in}{3.737095in}}%
\pgfpathlineto{\pgfqpoint{5.180119in}{3.751773in}}%
\pgfpathlineto{\pgfqpoint{5.148571in}{3.752647in}}%
\pgfpathlineto{\pgfqpoint{5.117002in}{3.753547in}}%
\pgfpathclose%
\pgfusepath{fill}%
\end{pgfscope}%
\begin{pgfscope}%
\pgfpathrectangle{\pgfqpoint{1.020000in}{0.880000in}}{\pgfqpoint{6.160000in}{6.160000in}}%
\pgfusepath{clip}%
\pgfsetbuttcap%
\pgfsetroundjoin%
\definecolor{currentfill}{rgb}{0.559747,0.694768,0.996075}%
\pgfsetfillcolor{currentfill}%
\pgfsetlinewidth{0.000000pt}%
\definecolor{currentstroke}{rgb}{0.000000,0.000000,0.000000}%
\pgfsetstrokecolor{currentstroke}%
\pgfsetdash{}{0pt}%
\pgfpathmoveto{\pgfqpoint{4.547251in}{3.871204in}}%
\pgfpathlineto{\pgfqpoint{4.557530in}{3.857024in}}%
\pgfpathlineto{\pgfqpoint{4.567827in}{3.842597in}}%
\pgfpathlineto{\pgfqpoint{4.599621in}{3.839353in}}%
\pgfpathlineto{\pgfqpoint{4.631395in}{3.836388in}}%
\pgfpathlineto{\pgfqpoint{4.621050in}{3.850810in}}%
\pgfpathlineto{\pgfqpoint{4.610722in}{3.864986in}}%
\pgfpathlineto{\pgfqpoint{4.578997in}{3.867964in}}%
\pgfpathlineto{\pgfqpoint{4.547251in}{3.871204in}}%
\pgfpathclose%
\pgfusepath{fill}%
\end{pgfscope}%
\begin{pgfscope}%
\pgfpathrectangle{\pgfqpoint{1.020000in}{0.880000in}}{\pgfqpoint{6.160000in}{6.160000in}}%
\pgfusepath{clip}%
\pgfsetbuttcap%
\pgfsetroundjoin%
\definecolor{currentfill}{rgb}{0.624703,0.748318,0.998719}%
\pgfsetfillcolor{currentfill}%
\pgfsetlinewidth{0.000000pt}%
\definecolor{currentstroke}{rgb}{0.000000,0.000000,0.000000}%
\pgfsetstrokecolor{currentstroke}%
\pgfsetdash{}{0pt}%
\pgfpathmoveto{\pgfqpoint{4.188620in}{3.980118in}}%
\pgfpathlineto{\pgfqpoint{4.198588in}{3.967879in}}%
\pgfpathlineto{\pgfqpoint{4.208574in}{3.955365in}}%
\pgfpathlineto{\pgfqpoint{4.240494in}{3.946769in}}%
\pgfpathlineto{\pgfqpoint{4.272390in}{3.939037in}}%
\pgfpathlineto{\pgfqpoint{4.262356in}{3.951949in}}%
\pgfpathlineto{\pgfqpoint{4.252339in}{3.964611in}}%
\pgfpathlineto{\pgfqpoint{4.220492in}{3.971967in}}%
\pgfpathlineto{\pgfqpoint{4.188620in}{3.980118in}}%
\pgfpathclose%
\pgfusepath{fill}%
\end{pgfscope}%
\begin{pgfscope}%
\pgfpathrectangle{\pgfqpoint{1.020000in}{0.880000in}}{\pgfqpoint{6.160000in}{6.160000in}}%
\pgfusepath{clip}%
\pgfsetbuttcap%
\pgfsetroundjoin%
\definecolor{currentfill}{rgb}{0.510824,0.649397,0.985079}%
\pgfsetfillcolor{currentfill}%
\pgfsetlinewidth{0.000000pt}%
\definecolor{currentstroke}{rgb}{0.000000,0.000000,0.000000}%
\pgfsetstrokecolor{currentstroke}%
\pgfsetdash{}{0pt}%
\pgfpathmoveto{\pgfqpoint{4.905977in}{3.790294in}}%
\pgfpathlineto{\pgfqpoint{4.916566in}{3.775311in}}%
\pgfpathlineto{\pgfqpoint{4.927171in}{3.759936in}}%
\pgfpathlineto{\pgfqpoint{4.958859in}{3.758694in}}%
\pgfpathlineto{\pgfqpoint{4.990528in}{3.757542in}}%
\pgfpathlineto{\pgfqpoint{4.979873in}{3.772680in}}%
\pgfpathlineto{\pgfqpoint{4.969235in}{3.787474in}}%
\pgfpathlineto{\pgfqpoint{4.937616in}{3.788843in}}%
\pgfpathlineto{\pgfqpoint{4.905977in}{3.790294in}}%
\pgfpathclose%
\pgfusepath{fill}%
\end{pgfscope}%
\begin{pgfscope}%
\pgfpathrectangle{\pgfqpoint{1.020000in}{0.880000in}}{\pgfqpoint{6.160000in}{6.160000in}}%
\pgfusepath{clip}%
\pgfsetbuttcap%
\pgfsetroundjoin%
\definecolor{currentfill}{rgb}{0.592356,0.722792,0.999434}%
\pgfsetfillcolor{currentfill}%
\pgfsetlinewidth{0.000000pt}%
\definecolor{currentstroke}{rgb}{0.000000,0.000000,0.000000}%
\pgfsetstrokecolor{currentstroke}%
\pgfsetdash{}{0pt}%
\pgfpathmoveto{\pgfqpoint{4.336111in}{3.925818in}}%
\pgfpathlineto{\pgfqpoint{4.346211in}{3.912390in}}%
\pgfpathlineto{\pgfqpoint{4.356329in}{3.898712in}}%
\pgfpathlineto{\pgfqpoint{4.388204in}{3.892887in}}%
\pgfpathlineto{\pgfqpoint{4.420056in}{3.887644in}}%
\pgfpathlineto{\pgfqpoint{4.409890in}{3.901471in}}%
\pgfpathlineto{\pgfqpoint{4.399742in}{3.915064in}}%
\pgfpathlineto{\pgfqpoint{4.367938in}{3.920169in}}%
\pgfpathlineto{\pgfqpoint{4.336111in}{3.925818in}}%
\pgfpathclose%
\pgfusepath{fill}%
\end{pgfscope}%
\begin{pgfscope}%
\pgfpathrectangle{\pgfqpoint{1.020000in}{0.880000in}}{\pgfqpoint{6.160000in}{6.160000in}}%
\pgfusepath{clip}%
\pgfsetbuttcap%
\pgfsetroundjoin%
\definecolor{currentfill}{rgb}{0.457046,0.594006,0.963029}%
\pgfsetfillcolor{currentfill}%
\pgfsetlinewidth{0.000000pt}%
\definecolor{currentstroke}{rgb}{0.000000,0.000000,0.000000}%
\pgfsetstrokecolor{currentstroke}%
\pgfsetdash{}{0pt}%
\pgfpathmoveto{\pgfqpoint{5.475966in}{3.688733in}}%
\pgfpathlineto{\pgfqpoint{5.487056in}{3.674200in}}%
\pgfpathlineto{\pgfqpoint{5.498164in}{3.659531in}}%
\pgfpathlineto{\pgfqpoint{5.529682in}{3.659294in}}%
\pgfpathlineto{\pgfqpoint{5.518548in}{3.673867in}}%
\pgfpathlineto{\pgfqpoint{5.507433in}{3.688315in}}%
\pgfpathlineto{\pgfqpoint{5.475966in}{3.688733in}}%
\pgfpathclose%
\pgfusepath{fill}%
\end{pgfscope}%
\begin{pgfscope}%
\pgfpathrectangle{\pgfqpoint{1.020000in}{0.880000in}}{\pgfqpoint{6.160000in}{6.160000in}}%
\pgfusepath{clip}%
\pgfsetbuttcap%
\pgfsetroundjoin%
\definecolor{currentfill}{rgb}{0.538004,0.674902,0.991722}%
\pgfsetfillcolor{currentfill}%
\pgfsetlinewidth{0.000000pt}%
\definecolor{currentstroke}{rgb}{0.000000,0.000000,0.000000}%
\pgfsetstrokecolor{currentstroke}%
\pgfsetdash{}{0pt}%
\pgfpathmoveto{\pgfqpoint{4.694880in}{3.831159in}}%
\pgfpathlineto{\pgfqpoint{4.705291in}{3.816499in}}%
\pgfpathlineto{\pgfqpoint{4.715718in}{3.801521in}}%
\pgfpathlineto{\pgfqpoint{4.747478in}{3.799262in}}%
\pgfpathlineto{\pgfqpoint{4.779218in}{3.797185in}}%
\pgfpathlineto{\pgfqpoint{4.768743in}{3.812089in}}%
\pgfpathlineto{\pgfqpoint{4.758284in}{3.826682in}}%
\pgfpathlineto{\pgfqpoint{4.726592in}{3.828838in}}%
\pgfpathlineto{\pgfqpoint{4.694880in}{3.831159in}}%
\pgfpathclose%
\pgfusepath{fill}%
\end{pgfscope}%
\begin{pgfscope}%
\pgfpathrectangle{\pgfqpoint{1.020000in}{0.880000in}}{\pgfqpoint{6.160000in}{6.160000in}}%
\pgfusepath{clip}%
\pgfsetbuttcap%
\pgfsetroundjoin%
\definecolor{currentfill}{rgb}{0.473070,0.611077,0.970634}%
\pgfsetfillcolor{currentfill}%
\pgfsetlinewidth{0.000000pt}%
\definecolor{currentstroke}{rgb}{0.000000,0.000000,0.000000}%
\pgfsetstrokecolor{currentstroke}%
\pgfsetdash{}{0pt}%
\pgfpathmoveto{\pgfqpoint{5.264913in}{3.720967in}}%
\pgfpathlineto{\pgfqpoint{5.275817in}{3.706071in}}%
\pgfpathlineto{\pgfqpoint{5.286738in}{3.690915in}}%
\pgfpathlineto{\pgfqpoint{5.318327in}{3.690634in}}%
\pgfpathlineto{\pgfqpoint{5.349895in}{3.690308in}}%
\pgfpathlineto{\pgfqpoint{5.338923in}{3.705134in}}%
\pgfpathlineto{\pgfqpoint{5.327969in}{3.719763in}}%
\pgfpathlineto{\pgfqpoint{5.296451in}{3.720366in}}%
\pgfpathlineto{\pgfqpoint{5.264913in}{3.720967in}}%
\pgfpathclose%
\pgfusepath{fill}%
\end{pgfscope}%
\begin{pgfscope}%
\pgfpathrectangle{\pgfqpoint{1.020000in}{0.880000in}}{\pgfqpoint{6.160000in}{6.160000in}}%
\pgfusepath{clip}%
\pgfsetbuttcap%
\pgfsetroundjoin%
\definecolor{currentfill}{rgb}{0.494638,0.633022,0.978983}%
\pgfsetfillcolor{currentfill}%
\pgfsetlinewidth{0.000000pt}%
\definecolor{currentstroke}{rgb}{0.000000,0.000000,0.000000}%
\pgfsetstrokecolor{currentstroke}%
\pgfsetdash{}{0pt}%
\pgfpathmoveto{\pgfqpoint{5.053805in}{3.755447in}}%
\pgfpathlineto{\pgfqpoint{5.064525in}{3.740209in}}%
\pgfpathlineto{\pgfqpoint{5.075260in}{3.724554in}}%
\pgfpathlineto{\pgfqpoint{5.106919in}{3.723932in}}%
\pgfpathlineto{\pgfqpoint{5.138558in}{3.723333in}}%
\pgfpathlineto{\pgfqpoint{5.127772in}{3.738606in}}%
\pgfpathlineto{\pgfqpoint{5.117002in}{3.753547in}}%
\pgfpathlineto{\pgfqpoint{5.085414in}{3.754478in}}%
\pgfpathlineto{\pgfqpoint{5.053805in}{3.755447in}}%
\pgfpathclose%
\pgfusepath{fill}%
\end{pgfscope}%
\begin{pgfscope}%
\pgfpathrectangle{\pgfqpoint{1.020000in}{0.880000in}}{\pgfqpoint{6.160000in}{6.160000in}}%
\pgfusepath{clip}%
\pgfsetbuttcap%
\pgfsetroundjoin%
\definecolor{currentfill}{rgb}{0.570616,0.704109,0.997195}%
\pgfsetfillcolor{currentfill}%
\pgfsetlinewidth{0.000000pt}%
\definecolor{currentstroke}{rgb}{0.000000,0.000000,0.000000}%
\pgfsetstrokecolor{currentstroke}%
\pgfsetdash{}{0pt}%
\pgfpathmoveto{\pgfqpoint{4.483696in}{3.878632in}}%
\pgfpathlineto{\pgfqpoint{4.493927in}{3.864484in}}%
\pgfpathlineto{\pgfqpoint{4.504176in}{3.850087in}}%
\pgfpathlineto{\pgfqpoint{4.536012in}{3.846160in}}%
\pgfpathlineto{\pgfqpoint{4.567827in}{3.842597in}}%
\pgfpathlineto{\pgfqpoint{4.557530in}{3.857024in}}%
\pgfpathlineto{\pgfqpoint{4.547251in}{3.871204in}}%
\pgfpathlineto{\pgfqpoint{4.515484in}{3.874746in}}%
\pgfpathlineto{\pgfqpoint{4.483696in}{3.878632in}}%
\pgfpathclose%
\pgfusepath{fill}%
\end{pgfscope}%
\begin{pgfscope}%
\pgfpathrectangle{\pgfqpoint{1.020000in}{0.880000in}}{\pgfqpoint{6.160000in}{6.160000in}}%
\pgfusepath{clip}%
\pgfsetbuttcap%
\pgfsetroundjoin%
\definecolor{currentfill}{rgb}{0.724041,0.814910,0.975651}%
\pgfsetfillcolor{currentfill}%
\pgfsetlinewidth{0.000000pt}%
\definecolor{currentstroke}{rgb}{0.000000,0.000000,0.000000}%
\pgfsetstrokecolor{currentstroke}%
\pgfsetdash{}{0pt}%
\pgfpathmoveto{\pgfqpoint{3.829699in}{4.149604in}}%
\pgfpathlineto{\pgfqpoint{3.839333in}{4.144094in}}%
\pgfpathlineto{\pgfqpoint{3.848986in}{4.138326in}}%
\pgfpathlineto{\pgfqpoint{3.881104in}{4.118876in}}%
\pgfpathlineto{\pgfqpoint{3.913187in}{4.100969in}}%
\pgfpathlineto{\pgfqpoint{3.903477in}{4.108471in}}%
\pgfpathlineto{\pgfqpoint{3.893787in}{4.115726in}}%
\pgfpathlineto{\pgfqpoint{3.861760in}{4.131981in}}%
\pgfpathlineto{\pgfqpoint{3.829699in}{4.149604in}}%
\pgfpathclose%
\pgfusepath{fill}%
\end{pgfscope}%
\begin{pgfscope}%
\pgfpathrectangle{\pgfqpoint{1.020000in}{0.880000in}}{\pgfqpoint{6.160000in}{6.160000in}}%
\pgfusepath{clip}%
\pgfsetbuttcap%
\pgfsetroundjoin%
\definecolor{currentfill}{rgb}{0.677823,0.786546,0.991005}%
\pgfsetfillcolor{currentfill}%
\pgfsetlinewidth{0.000000pt}%
\definecolor{currentstroke}{rgb}{0.000000,0.000000,0.000000}%
\pgfsetstrokecolor{currentstroke}%
\pgfsetdash{}{0pt}%
\pgfpathmoveto{\pgfqpoint{3.977251in}{4.069502in}}%
\pgfpathlineto{\pgfqpoint{3.987032in}{4.060322in}}%
\pgfpathlineto{\pgfqpoint{3.996832in}{4.050827in}}%
\pgfpathlineto{\pgfqpoint{4.028869in}{4.035964in}}%
\pgfpathlineto{\pgfqpoint{4.060875in}{4.022464in}}%
\pgfpathlineto{\pgfqpoint{4.051024in}{4.033019in}}%
\pgfpathlineto{\pgfqpoint{4.041191in}{4.043288in}}%
\pgfpathlineto{\pgfqpoint{4.009236in}{4.055781in}}%
\pgfpathlineto{\pgfqpoint{3.977251in}{4.069502in}}%
\pgfpathclose%
\pgfusepath{fill}%
\end{pgfscope}%
\begin{pgfscope}%
\pgfpathrectangle{\pgfqpoint{1.020000in}{0.880000in}}{\pgfqpoint{6.160000in}{6.160000in}}%
\pgfusepath{clip}%
\pgfsetbuttcap%
\pgfsetroundjoin%
\definecolor{currentfill}{rgb}{0.516260,0.654498,0.986407}%
\pgfsetfillcolor{currentfill}%
\pgfsetlinewidth{0.000000pt}%
\definecolor{currentstroke}{rgb}{0.000000,0.000000,0.000000}%
\pgfsetstrokecolor{currentstroke}%
\pgfsetdash{}{0pt}%
\pgfpathmoveto{\pgfqpoint{4.842638in}{3.793491in}}%
\pgfpathlineto{\pgfqpoint{4.853178in}{3.778342in}}%
\pgfpathlineto{\pgfqpoint{4.863734in}{3.762761in}}%
\pgfpathlineto{\pgfqpoint{4.895462in}{3.761286in}}%
\pgfpathlineto{\pgfqpoint{4.927171in}{3.759936in}}%
\pgfpathlineto{\pgfqpoint{4.916566in}{3.775311in}}%
\pgfpathlineto{\pgfqpoint{4.905977in}{3.790294in}}%
\pgfpathlineto{\pgfqpoint{4.874317in}{3.791838in}}%
\pgfpathlineto{\pgfqpoint{4.842638in}{3.793491in}}%
\pgfpathclose%
\pgfusepath{fill}%
\end{pgfscope}%
\begin{pgfscope}%
\pgfpathrectangle{\pgfqpoint{1.020000in}{0.880000in}}{\pgfqpoint{6.160000in}{6.160000in}}%
\pgfusepath{clip}%
\pgfsetbuttcap%
\pgfsetroundjoin%
\definecolor{currentfill}{rgb}{0.772706,0.838978,0.949319}%
\pgfsetfillcolor{currentfill}%
\pgfsetlinewidth{0.000000pt}%
\definecolor{currentstroke}{rgb}{0.000000,0.000000,0.000000}%
\pgfsetstrokecolor{currentstroke}%
\pgfsetdash{}{0pt}%
\pgfpathmoveto{\pgfqpoint{3.682121in}{4.236038in}}%
\pgfpathlineto{\pgfqpoint{3.691593in}{4.235389in}}%
\pgfpathlineto{\pgfqpoint{3.701085in}{4.234655in}}%
\pgfpathlineto{\pgfqpoint{3.733297in}{4.211154in}}%
\pgfpathlineto{\pgfqpoint{3.765469in}{4.189156in}}%
\pgfpathlineto{\pgfqpoint{3.755913in}{4.192371in}}%
\pgfpathlineto{\pgfqpoint{3.746377in}{4.195473in}}%
\pgfpathlineto{\pgfqpoint{3.714268in}{4.215099in}}%
\pgfpathlineto{\pgfqpoint{3.682121in}{4.236038in}}%
\pgfpathclose%
\pgfusepath{fill}%
\end{pgfscope}%
\begin{pgfscope}%
\pgfpathrectangle{\pgfqpoint{1.020000in}{0.880000in}}{\pgfqpoint{6.160000in}{6.160000in}}%
\pgfusepath{clip}%
\pgfsetbuttcap%
\pgfsetroundjoin%
\definecolor{currentfill}{rgb}{0.457046,0.594006,0.963029}%
\pgfsetfillcolor{currentfill}%
\pgfsetlinewidth{0.000000pt}%
\definecolor{currentstroke}{rgb}{0.000000,0.000000,0.000000}%
\pgfsetstrokecolor{currentstroke}%
\pgfsetdash{}{0pt}%
\pgfpathmoveto{\pgfqpoint{5.412971in}{3.689557in}}%
\pgfpathlineto{\pgfqpoint{5.424011in}{3.674811in}}%
\pgfpathlineto{\pgfqpoint{5.435068in}{3.659892in}}%
\pgfpathlineto{\pgfqpoint{5.466627in}{3.659736in}}%
\pgfpathlineto{\pgfqpoint{5.498164in}{3.659531in}}%
\pgfpathlineto{\pgfqpoint{5.487056in}{3.674200in}}%
\pgfpathlineto{\pgfqpoint{5.475966in}{3.688733in}}%
\pgfpathlineto{\pgfqpoint{5.444479in}{3.689149in}}%
\pgfpathlineto{\pgfqpoint{5.412971in}{3.689557in}}%
\pgfpathclose%
\pgfusepath{fill}%
\end{pgfscope}%
\begin{pgfscope}%
\pgfpathrectangle{\pgfqpoint{1.020000in}{0.880000in}}{\pgfqpoint{6.160000in}{6.160000in}}%
\pgfusepath{clip}%
\pgfsetbuttcap%
\pgfsetroundjoin%
\definecolor{currentfill}{rgb}{0.640828,0.760752,0.997846}%
\pgfsetfillcolor{currentfill}%
\pgfsetlinewidth{0.000000pt}%
\definecolor{currentstroke}{rgb}{0.000000,0.000000,0.000000}%
\pgfsetstrokecolor{currentstroke}%
\pgfsetdash{}{0pt}%
\pgfpathmoveto{\pgfqpoint{4.124801in}{3.999155in}}%
\pgfpathlineto{\pgfqpoint{4.134720in}{3.987500in}}%
\pgfpathlineto{\pgfqpoint{4.144657in}{3.975540in}}%
\pgfpathlineto{\pgfqpoint{4.176629in}{3.964922in}}%
\pgfpathlineto{\pgfqpoint{4.208574in}{3.955365in}}%
\pgfpathlineto{\pgfqpoint{4.198588in}{3.967879in}}%
\pgfpathlineto{\pgfqpoint{4.188620in}{3.980118in}}%
\pgfpathlineto{\pgfqpoint{4.156724in}{3.989150in}}%
\pgfpathlineto{\pgfqpoint{4.124801in}{3.999155in}}%
\pgfpathclose%
\pgfusepath{fill}%
\end{pgfscope}%
\begin{pgfscope}%
\pgfpathrectangle{\pgfqpoint{1.020000in}{0.880000in}}{\pgfqpoint{6.160000in}{6.160000in}}%
\pgfusepath{clip}%
\pgfsetbuttcap%
\pgfsetroundjoin%
\definecolor{currentfill}{rgb}{0.818056,0.855590,0.914638}%
\pgfsetfillcolor{currentfill}%
\pgfsetlinewidth{0.000000pt}%
\definecolor{currentstroke}{rgb}{0.000000,0.000000,0.000000}%
\pgfsetstrokecolor{currentstroke}%
\pgfsetdash{}{0pt}%
\pgfpathmoveto{\pgfqpoint{3.534530in}{4.322298in}}%
\pgfpathlineto{\pgfqpoint{3.543822in}{4.327321in}}%
\pgfpathlineto{\pgfqpoint{3.553134in}{4.332576in}}%
\pgfpathlineto{\pgfqpoint{3.585442in}{4.306602in}}%
\pgfpathlineto{\pgfqpoint{3.617709in}{4.281819in}}%
\pgfpathlineto{\pgfqpoint{3.608324in}{4.279693in}}%
\pgfpathlineto{\pgfqpoint{3.598957in}{4.277717in}}%
\pgfpathlineto{\pgfqpoint{3.566763in}{4.299496in}}%
\pgfpathlineto{\pgfqpoint{3.534530in}{4.322298in}}%
\pgfpathclose%
\pgfusepath{fill}%
\end{pgfscope}%
\begin{pgfscope}%
\pgfpathrectangle{\pgfqpoint{1.020000in}{0.880000in}}{\pgfqpoint{6.160000in}{6.160000in}}%
\pgfusepath{clip}%
\pgfsetbuttcap%
\pgfsetroundjoin%
\definecolor{currentfill}{rgb}{0.855378,0.863778,0.876587}%
\pgfsetfillcolor{currentfill}%
\pgfsetlinewidth{0.000000pt}%
\definecolor{currentstroke}{rgb}{0.000000,0.000000,0.000000}%
\pgfsetstrokecolor{currentstroke}%
\pgfsetdash{}{0pt}%
\pgfpathmoveto{\pgfqpoint{3.386988in}{4.400019in}}%
\pgfpathlineto{\pgfqpoint{3.396085in}{4.410734in}}%
\pgfpathlineto{\pgfqpoint{3.405198in}{4.422131in}}%
\pgfpathlineto{\pgfqpoint{3.437590in}{4.396087in}}%
\pgfpathlineto{\pgfqpoint{3.469943in}{4.370689in}}%
\pgfpathlineto{\pgfqpoint{3.460747in}{4.362760in}}%
\pgfpathlineto{\pgfqpoint{3.451566in}{4.355376in}}%
\pgfpathlineto{\pgfqpoint{3.419296in}{4.377422in}}%
\pgfpathlineto{\pgfqpoint{3.386988in}{4.400019in}}%
\pgfpathclose%
\pgfusepath{fill}%
\end{pgfscope}%
\begin{pgfscope}%
\pgfpathrectangle{\pgfqpoint{1.020000in}{0.880000in}}{\pgfqpoint{6.160000in}{6.160000in}}%
\pgfusepath{clip}%
\pgfsetbuttcap%
\pgfsetroundjoin%
\definecolor{currentfill}{rgb}{0.478462,0.616564,0.972721}%
\pgfsetfillcolor{currentfill}%
\pgfsetlinewidth{0.000000pt}%
\definecolor{currentstroke}{rgb}{0.000000,0.000000,0.000000}%
\pgfsetstrokecolor{currentstroke}%
\pgfsetdash{}{0pt}%
\pgfpathmoveto{\pgfqpoint{5.201776in}{3.722155in}}%
\pgfpathlineto{\pgfqpoint{5.212629in}{3.706914in}}%
\pgfpathlineto{\pgfqpoint{5.223498in}{3.691324in}}%
\pgfpathlineto{\pgfqpoint{5.255128in}{3.691146in}}%
\pgfpathlineto{\pgfqpoint{5.286738in}{3.690915in}}%
\pgfpathlineto{\pgfqpoint{5.275817in}{3.706071in}}%
\pgfpathlineto{\pgfqpoint{5.264913in}{3.720967in}}%
\pgfpathlineto{\pgfqpoint{5.233355in}{3.721564in}}%
\pgfpathlineto{\pgfqpoint{5.201776in}{3.722155in}}%
\pgfpathclose%
\pgfusepath{fill}%
\end{pgfscope}%
\begin{pgfscope}%
\pgfpathrectangle{\pgfqpoint{1.020000in}{0.880000in}}{\pgfqpoint{6.160000in}{6.160000in}}%
\pgfusepath{clip}%
\pgfsetbuttcap%
\pgfsetroundjoin%
\definecolor{currentfill}{rgb}{0.543440,0.680003,0.993051}%
\pgfsetfillcolor{currentfill}%
\pgfsetlinewidth{0.000000pt}%
\definecolor{currentstroke}{rgb}{0.000000,0.000000,0.000000}%
\pgfsetstrokecolor{currentstroke}%
\pgfsetdash{}{0pt}%
\pgfpathmoveto{\pgfqpoint{4.631395in}{3.836388in}}%
\pgfpathlineto{\pgfqpoint{4.641757in}{3.821691in}}%
\pgfpathlineto{\pgfqpoint{4.652136in}{3.806678in}}%
\pgfpathlineto{\pgfqpoint{4.683937in}{3.803984in}}%
\pgfpathlineto{\pgfqpoint{4.715718in}{3.801521in}}%
\pgfpathlineto{\pgfqpoint{4.705291in}{3.816499in}}%
\pgfpathlineto{\pgfqpoint{4.694880in}{3.831159in}}%
\pgfpathlineto{\pgfqpoint{4.663148in}{3.833666in}}%
\pgfpathlineto{\pgfqpoint{4.631395in}{3.836388in}}%
\pgfpathclose%
\pgfusepath{fill}%
\end{pgfscope}%
\begin{pgfscope}%
\pgfpathrectangle{\pgfqpoint{1.020000in}{0.880000in}}{\pgfqpoint{6.160000in}{6.160000in}}%
\pgfusepath{clip}%
\pgfsetbuttcap%
\pgfsetroundjoin%
\definecolor{currentfill}{rgb}{0.608547,0.735725,0.999354}%
\pgfsetfillcolor{currentfill}%
\pgfsetlinewidth{0.000000pt}%
\definecolor{currentstroke}{rgb}{0.000000,0.000000,0.000000}%
\pgfsetstrokecolor{currentstroke}%
\pgfsetdash{}{0pt}%
\pgfpathmoveto{\pgfqpoint{4.272390in}{3.939037in}}%
\pgfpathlineto{\pgfqpoint{4.282442in}{3.925864in}}%
\pgfpathlineto{\pgfqpoint{4.292511in}{3.912420in}}%
\pgfpathlineto{\pgfqpoint{4.324432in}{3.905196in}}%
\pgfpathlineto{\pgfqpoint{4.356329in}{3.898712in}}%
\pgfpathlineto{\pgfqpoint{4.346211in}{3.912390in}}%
\pgfpathlineto{\pgfqpoint{4.336111in}{3.925818in}}%
\pgfpathlineto{\pgfqpoint{4.304262in}{3.932081in}}%
\pgfpathlineto{\pgfqpoint{4.272390in}{3.939037in}}%
\pgfpathclose%
\pgfusepath{fill}%
\end{pgfscope}%
\begin{pgfscope}%
\pgfpathrectangle{\pgfqpoint{1.020000in}{0.880000in}}{\pgfqpoint{6.160000in}{6.160000in}}%
\pgfusepath{clip}%
\pgfsetbuttcap%
\pgfsetroundjoin%
\definecolor{currentfill}{rgb}{0.500031,0.638508,0.981070}%
\pgfsetfillcolor{currentfill}%
\pgfsetlinewidth{0.000000pt}%
\definecolor{currentstroke}{rgb}{0.000000,0.000000,0.000000}%
\pgfsetstrokecolor{currentstroke}%
\pgfsetdash{}{0pt}%
\pgfpathmoveto{\pgfqpoint{4.990528in}{3.757542in}}%
\pgfpathlineto{\pgfqpoint{5.001198in}{3.741990in}}%
\pgfpathlineto{\pgfqpoint{5.011882in}{3.725930in}}%
\pgfpathlineto{\pgfqpoint{5.043581in}{3.725213in}}%
\pgfpathlineto{\pgfqpoint{5.075260in}{3.724554in}}%
\pgfpathlineto{\pgfqpoint{5.064525in}{3.740209in}}%
\pgfpathlineto{\pgfqpoint{5.053805in}{3.755447in}}%
\pgfpathlineto{\pgfqpoint{5.022176in}{3.756465in}}%
\pgfpathlineto{\pgfqpoint{4.990528in}{3.757542in}}%
\pgfpathclose%
\pgfusepath{fill}%
\end{pgfscope}%
\begin{pgfscope}%
\pgfpathrectangle{\pgfqpoint{1.020000in}{0.880000in}}{\pgfqpoint{6.160000in}{6.160000in}}%
\pgfusepath{clip}%
\pgfsetbuttcap%
\pgfsetroundjoin%
\definecolor{currentfill}{rgb}{0.576051,0.708780,0.997755}%
\pgfsetfillcolor{currentfill}%
\pgfsetlinewidth{0.000000pt}%
\definecolor{currentstroke}{rgb}{0.000000,0.000000,0.000000}%
\pgfsetstrokecolor{currentstroke}%
\pgfsetdash{}{0pt}%
\pgfpathmoveto{\pgfqpoint{4.420056in}{3.887644in}}%
\pgfpathlineto{\pgfqpoint{4.430239in}{3.873573in}}%
\pgfpathlineto{\pgfqpoint{4.440440in}{3.859246in}}%
\pgfpathlineto{\pgfqpoint{4.472318in}{3.854429in}}%
\pgfpathlineto{\pgfqpoint{4.504176in}{3.850087in}}%
\pgfpathlineto{\pgfqpoint{4.493927in}{3.864484in}}%
\pgfpathlineto{\pgfqpoint{4.483696in}{3.878632in}}%
\pgfpathlineto{\pgfqpoint{4.451887in}{3.882913in}}%
\pgfpathlineto{\pgfqpoint{4.420056in}{3.887644in}}%
\pgfpathclose%
\pgfusepath{fill}%
\end{pgfscope}%
\begin{pgfscope}%
\pgfpathrectangle{\pgfqpoint{1.020000in}{0.880000in}}{\pgfqpoint{6.160000in}{6.160000in}}%
\pgfusepath{clip}%
\pgfsetbuttcap%
\pgfsetroundjoin%
\definecolor{currentfill}{rgb}{0.527132,0.664700,0.989065}%
\pgfsetfillcolor{currentfill}%
\pgfsetlinewidth{0.000000pt}%
\definecolor{currentstroke}{rgb}{0.000000,0.000000,0.000000}%
\pgfsetstrokecolor{currentstroke}%
\pgfsetdash{}{0pt}%
\pgfpathmoveto{\pgfqpoint{4.779218in}{3.797185in}}%
\pgfpathlineto{\pgfqpoint{4.789710in}{3.781907in}}%
\pgfpathlineto{\pgfqpoint{4.800217in}{3.766173in}}%
\pgfpathlineto{\pgfqpoint{4.831985in}{3.764384in}}%
\pgfpathlineto{\pgfqpoint{4.863734in}{3.762761in}}%
\pgfpathlineto{\pgfqpoint{4.853178in}{3.778342in}}%
\pgfpathlineto{\pgfqpoint{4.842638in}{3.793491in}}%
\pgfpathlineto{\pgfqpoint{4.810938in}{3.795268in}}%
\pgfpathlineto{\pgfqpoint{4.779218in}{3.797185in}}%
\pgfpathclose%
\pgfusepath{fill}%
\end{pgfscope}%
\begin{pgfscope}%
\pgfpathrectangle{\pgfqpoint{1.020000in}{0.880000in}}{\pgfqpoint{6.160000in}{6.160000in}}%
\pgfusepath{clip}%
\pgfsetbuttcap%
\pgfsetroundjoin%
\definecolor{currentfill}{rgb}{0.462354,0.599830,0.965857}%
\pgfsetfillcolor{currentfill}%
\pgfsetlinewidth{0.000000pt}%
\definecolor{currentstroke}{rgb}{0.000000,0.000000,0.000000}%
\pgfsetstrokecolor{currentstroke}%
\pgfsetdash{}{0pt}%
\pgfpathmoveto{\pgfqpoint{5.349895in}{3.690308in}}%
\pgfpathlineto{\pgfqpoint{5.360884in}{3.675267in}}%
\pgfpathlineto{\pgfqpoint{5.371890in}{3.659991in}}%
\pgfpathlineto{\pgfqpoint{5.403489in}{3.659983in}}%
\pgfpathlineto{\pgfqpoint{5.435068in}{3.659892in}}%
\pgfpathlineto{\pgfqpoint{5.424011in}{3.674811in}}%
\pgfpathlineto{\pgfqpoint{5.412971in}{3.689557in}}%
\pgfpathlineto{\pgfqpoint{5.381443in}{3.689946in}}%
\pgfpathlineto{\pgfqpoint{5.349895in}{3.690308in}}%
\pgfpathclose%
\pgfusepath{fill}%
\end{pgfscope}%
\begin{pgfscope}%
\pgfpathrectangle{\pgfqpoint{1.020000in}{0.880000in}}{\pgfqpoint{6.160000in}{6.160000in}}%
\pgfusepath{clip}%
\pgfsetbuttcap%
\pgfsetroundjoin%
\definecolor{currentfill}{rgb}{0.483854,0.622050,0.974808}%
\pgfsetfillcolor{currentfill}%
\pgfsetlinewidth{0.000000pt}%
\definecolor{currentstroke}{rgb}{0.000000,0.000000,0.000000}%
\pgfsetstrokecolor{currentstroke}%
\pgfsetdash{}{0pt}%
\pgfpathmoveto{\pgfqpoint{5.138558in}{3.723333in}}%
\pgfpathlineto{\pgfqpoint{5.149360in}{3.707667in}}%
\pgfpathlineto{\pgfqpoint{5.160176in}{3.691536in}}%
\pgfpathlineto{\pgfqpoint{5.191847in}{3.691452in}}%
\pgfpathlineto{\pgfqpoint{5.223498in}{3.691324in}}%
\pgfpathlineto{\pgfqpoint{5.212629in}{3.706914in}}%
\pgfpathlineto{\pgfqpoint{5.201776in}{3.722155in}}%
\pgfpathlineto{\pgfqpoint{5.170177in}{3.722743in}}%
\pgfpathlineto{\pgfqpoint{5.138558in}{3.723333in}}%
\pgfpathclose%
\pgfusepath{fill}%
\end{pgfscope}%
\begin{pgfscope}%
\pgfpathrectangle{\pgfqpoint{1.020000in}{0.880000in}}{\pgfqpoint{6.160000in}{6.160000in}}%
\pgfusepath{clip}%
\pgfsetbuttcap%
\pgfsetroundjoin%
\definecolor{currentfill}{rgb}{0.703587,0.802586,0.982847}%
\pgfsetfillcolor{currentfill}%
\pgfsetlinewidth{0.000000pt}%
\definecolor{currentstroke}{rgb}{0.000000,0.000000,0.000000}%
\pgfsetstrokecolor{currentstroke}%
\pgfsetdash{}{0pt}%
\pgfpathmoveto{\pgfqpoint{3.913187in}{4.100969in}}%
\pgfpathlineto{\pgfqpoint{3.922915in}{4.093170in}}%
\pgfpathlineto{\pgfqpoint{3.932662in}{4.085022in}}%
\pgfpathlineto{\pgfqpoint{3.964764in}{4.067149in}}%
\pgfpathlineto{\pgfqpoint{3.996832in}{4.050827in}}%
\pgfpathlineto{\pgfqpoint{3.987032in}{4.060322in}}%
\pgfpathlineto{\pgfqpoint{3.977251in}{4.069502in}}%
\pgfpathlineto{\pgfqpoint{3.945235in}{4.084537in}}%
\pgfpathlineto{\pgfqpoint{3.913187in}{4.100969in}}%
\pgfpathclose%
\pgfusepath{fill}%
\end{pgfscope}%
\begin{pgfscope}%
\pgfpathrectangle{\pgfqpoint{1.020000in}{0.880000in}}{\pgfqpoint{6.160000in}{6.160000in}}%
\pgfusepath{clip}%
\pgfsetbuttcap%
\pgfsetroundjoin%
\definecolor{currentfill}{rgb}{0.656683,0.771806,0.994914}%
\pgfsetfillcolor{currentfill}%
\pgfsetlinewidth{0.000000pt}%
\definecolor{currentstroke}{rgb}{0.000000,0.000000,0.000000}%
\pgfsetstrokecolor{currentstroke}%
\pgfsetdash{}{0pt}%
\pgfpathmoveto{\pgfqpoint{4.060875in}{4.022464in}}%
\pgfpathlineto{\pgfqpoint{4.070744in}{4.011593in}}%
\pgfpathlineto{\pgfqpoint{4.080631in}{4.000377in}}%
\pgfpathlineto{\pgfqpoint{4.112658in}{3.987323in}}%
\pgfpathlineto{\pgfqpoint{4.144657in}{3.975540in}}%
\pgfpathlineto{\pgfqpoint{4.134720in}{3.987500in}}%
\pgfpathlineto{\pgfqpoint{4.124801in}{3.999155in}}%
\pgfpathlineto{\pgfqpoint{4.092852in}{4.010227in}}%
\pgfpathlineto{\pgfqpoint{4.060875in}{4.022464in}}%
\pgfpathclose%
\pgfusepath{fill}%
\end{pgfscope}%
\begin{pgfscope}%
\pgfpathrectangle{\pgfqpoint{1.020000in}{0.880000in}}{\pgfqpoint{6.160000in}{6.160000in}}%
\pgfusepath{clip}%
\pgfsetbuttcap%
\pgfsetroundjoin%
\definecolor{currentfill}{rgb}{0.554312,0.690097,0.995516}%
\pgfsetfillcolor{currentfill}%
\pgfsetlinewidth{0.000000pt}%
\definecolor{currentstroke}{rgb}{0.000000,0.000000,0.000000}%
\pgfsetstrokecolor{currentstroke}%
\pgfsetdash{}{0pt}%
\pgfpathmoveto{\pgfqpoint{4.567827in}{3.842597in}}%
\pgfpathlineto{\pgfqpoint{4.578141in}{3.827897in}}%
\pgfpathlineto{\pgfqpoint{4.588472in}{3.812887in}}%
\pgfpathlineto{\pgfqpoint{4.620314in}{3.809635in}}%
\pgfpathlineto{\pgfqpoint{4.652136in}{3.806678in}}%
\pgfpathlineto{\pgfqpoint{4.641757in}{3.821691in}}%
\pgfpathlineto{\pgfqpoint{4.631395in}{3.836388in}}%
\pgfpathlineto{\pgfqpoint{4.599621in}{3.839353in}}%
\pgfpathlineto{\pgfqpoint{4.567827in}{3.842597in}}%
\pgfpathclose%
\pgfusepath{fill}%
\end{pgfscope}%
\begin{pgfscope}%
\pgfpathrectangle{\pgfqpoint{1.020000in}{0.880000in}}{\pgfqpoint{6.160000in}{6.160000in}}%
\pgfusepath{clip}%
\pgfsetbuttcap%
\pgfsetroundjoin%
\definecolor{currentfill}{rgb}{0.748682,0.827679,0.963334}%
\pgfsetfillcolor{currentfill}%
\pgfsetlinewidth{0.000000pt}%
\definecolor{currentstroke}{rgb}{0.000000,0.000000,0.000000}%
\pgfsetstrokecolor{currentstroke}%
\pgfsetdash{}{0pt}%
\pgfpathmoveto{\pgfqpoint{3.765469in}{4.189156in}}%
\pgfpathlineto{\pgfqpoint{3.775043in}{4.185754in}}%
\pgfpathlineto{\pgfqpoint{3.784638in}{4.182085in}}%
\pgfpathlineto{\pgfqpoint{3.816831in}{4.159380in}}%
\pgfpathlineto{\pgfqpoint{3.848986in}{4.138326in}}%
\pgfpathlineto{\pgfqpoint{3.839333in}{4.144094in}}%
\pgfpathlineto{\pgfqpoint{3.829699in}{4.149604in}}%
\pgfpathlineto{\pgfqpoint{3.797602in}{4.168649in}}%
\pgfpathlineto{\pgfqpoint{3.765469in}{4.189156in}}%
\pgfpathclose%
\pgfusepath{fill}%
\end{pgfscope}%
\begin{pgfscope}%
\pgfpathrectangle{\pgfqpoint{1.020000in}{0.880000in}}{\pgfqpoint{6.160000in}{6.160000in}}%
\pgfusepath{clip}%
\pgfsetbuttcap%
\pgfsetroundjoin%
\definecolor{currentfill}{rgb}{0.505423,0.643995,0.983157}%
\pgfsetfillcolor{currentfill}%
\pgfsetlinewidth{0.000000pt}%
\definecolor{currentstroke}{rgb}{0.000000,0.000000,0.000000}%
\pgfsetstrokecolor{currentstroke}%
\pgfsetdash{}{0pt}%
\pgfpathmoveto{\pgfqpoint{4.927171in}{3.759936in}}%
\pgfpathlineto{\pgfqpoint{4.937791in}{3.744081in}}%
\pgfpathlineto{\pgfqpoint{4.948425in}{3.727632in}}%
\pgfpathlineto{\pgfqpoint{4.980163in}{3.726728in}}%
\pgfpathlineto{\pgfqpoint{5.011882in}{3.725930in}}%
\pgfpathlineto{\pgfqpoint{5.001198in}{3.741990in}}%
\pgfpathlineto{\pgfqpoint{4.990528in}{3.757542in}}%
\pgfpathlineto{\pgfqpoint{4.958859in}{3.758694in}}%
\pgfpathlineto{\pgfqpoint{4.927171in}{3.759936in}}%
\pgfpathclose%
\pgfusepath{fill}%
\end{pgfscope}%
\begin{pgfscope}%
\pgfpathrectangle{\pgfqpoint{1.020000in}{0.880000in}}{\pgfqpoint{6.160000in}{6.160000in}}%
\pgfusepath{clip}%
\pgfsetbuttcap%
\pgfsetroundjoin%
\definecolor{currentfill}{rgb}{0.800601,0.850358,0.930008}%
\pgfsetfillcolor{currentfill}%
\pgfsetlinewidth{0.000000pt}%
\definecolor{currentstroke}{rgb}{0.000000,0.000000,0.000000}%
\pgfsetstrokecolor{currentstroke}%
\pgfsetdash{}{0pt}%
\pgfpathmoveto{\pgfqpoint{3.617709in}{4.281819in}}%
\pgfpathlineto{\pgfqpoint{3.627113in}{4.284001in}}%
\pgfpathlineto{\pgfqpoint{3.636536in}{4.286133in}}%
\pgfpathlineto{\pgfqpoint{3.668832in}{4.259656in}}%
\pgfpathlineto{\pgfqpoint{3.701085in}{4.234655in}}%
\pgfpathlineto{\pgfqpoint{3.691593in}{4.235389in}}%
\pgfpathlineto{\pgfqpoint{3.682121in}{4.236038in}}%
\pgfpathlineto{\pgfqpoint{3.649935in}{4.258285in}}%
\pgfpathlineto{\pgfqpoint{3.617709in}{4.281819in}}%
\pgfpathclose%
\pgfusepath{fill}%
\end{pgfscope}%
\begin{pgfscope}%
\pgfpathrectangle{\pgfqpoint{1.020000in}{0.880000in}}{\pgfqpoint{6.160000in}{6.160000in}}%
\pgfusepath{clip}%
\pgfsetbuttcap%
\pgfsetroundjoin%
\definecolor{currentfill}{rgb}{0.619318,0.744121,0.998931}%
\pgfsetfillcolor{currentfill}%
\pgfsetlinewidth{0.000000pt}%
\definecolor{currentstroke}{rgb}{0.000000,0.000000,0.000000}%
\pgfsetstrokecolor{currentstroke}%
\pgfsetdash{}{0pt}%
\pgfpathmoveto{\pgfqpoint{4.208574in}{3.955365in}}%
\pgfpathlineto{\pgfqpoint{4.218577in}{3.942563in}}%
\pgfpathlineto{\pgfqpoint{4.228598in}{3.929459in}}%
\pgfpathlineto{\pgfqpoint{4.260567in}{3.920475in}}%
\pgfpathlineto{\pgfqpoint{4.292511in}{3.912420in}}%
\pgfpathlineto{\pgfqpoint{4.282442in}{3.925864in}}%
\pgfpathlineto{\pgfqpoint{4.272390in}{3.939037in}}%
\pgfpathlineto{\pgfqpoint{4.240494in}{3.946769in}}%
\pgfpathlineto{\pgfqpoint{4.208574in}{3.955365in}}%
\pgfpathclose%
\pgfusepath{fill}%
\end{pgfscope}%
\begin{pgfscope}%
\pgfpathrectangle{\pgfqpoint{1.020000in}{0.880000in}}{\pgfqpoint{6.160000in}{6.160000in}}%
\pgfusepath{clip}%
\pgfsetbuttcap%
\pgfsetroundjoin%
\definecolor{currentfill}{rgb}{0.847365,0.862472,0.885540}%
\pgfsetfillcolor{currentfill}%
\pgfsetlinewidth{0.000000pt}%
\definecolor{currentstroke}{rgb}{0.000000,0.000000,0.000000}%
\pgfsetstrokecolor{currentstroke}%
\pgfsetdash{}{0pt}%
\pgfpathmoveto{\pgfqpoint{3.469943in}{4.370689in}}%
\pgfpathlineto{\pgfqpoint{3.479157in}{4.379061in}}%
\pgfpathlineto{\pgfqpoint{3.488389in}{4.387757in}}%
\pgfpathlineto{\pgfqpoint{3.520783in}{4.359661in}}%
\pgfpathlineto{\pgfqpoint{3.553134in}{4.332576in}}%
\pgfpathlineto{\pgfqpoint{3.543822in}{4.327321in}}%
\pgfpathlineto{\pgfqpoint{3.534530in}{4.322298in}}%
\pgfpathlineto{\pgfqpoint{3.502256in}{4.346058in}}%
\pgfpathlineto{\pgfqpoint{3.469943in}{4.370689in}}%
\pgfpathclose%
\pgfusepath{fill}%
\end{pgfscope}%
\begin{pgfscope}%
\pgfpathrectangle{\pgfqpoint{1.020000in}{0.880000in}}{\pgfqpoint{6.160000in}{6.160000in}}%
\pgfusepath{clip}%
\pgfsetbuttcap%
\pgfsetroundjoin%
\definecolor{currentfill}{rgb}{0.883687,0.856108,0.840258}%
\pgfsetfillcolor{currentfill}%
\pgfsetlinewidth{0.000000pt}%
\definecolor{currentstroke}{rgb}{0.000000,0.000000,0.000000}%
\pgfsetstrokecolor{currentstroke}%
\pgfsetdash{}{0pt}%
\pgfpathmoveto{\pgfqpoint{3.322262in}{4.446385in}}%
\pgfpathlineto{\pgfqpoint{3.331271in}{4.460570in}}%
\pgfpathlineto{\pgfqpoint{3.340294in}{4.475579in}}%
\pgfpathlineto{\pgfqpoint{3.372765in}{4.448680in}}%
\pgfpathlineto{\pgfqpoint{3.405198in}{4.422131in}}%
\pgfpathlineto{\pgfqpoint{3.396085in}{4.410734in}}%
\pgfpathlineto{\pgfqpoint{3.386988in}{4.400019in}}%
\pgfpathlineto{\pgfqpoint{3.354643in}{4.423050in}}%
\pgfpathlineto{\pgfqpoint{3.322262in}{4.446385in}}%
\pgfpathclose%
\pgfusepath{fill}%
\end{pgfscope}%
\begin{pgfscope}%
\pgfpathrectangle{\pgfqpoint{1.020000in}{0.880000in}}{\pgfqpoint{6.160000in}{6.160000in}}%
\pgfusepath{clip}%
\pgfsetbuttcap%
\pgfsetroundjoin%
\definecolor{currentfill}{rgb}{0.586921,0.718121,0.998874}%
\pgfsetfillcolor{currentfill}%
\pgfsetlinewidth{0.000000pt}%
\definecolor{currentstroke}{rgb}{0.000000,0.000000,0.000000}%
\pgfsetstrokecolor{currentstroke}%
\pgfsetdash{}{0pt}%
\pgfpathmoveto{\pgfqpoint{4.356329in}{3.898712in}}%
\pgfpathlineto{\pgfqpoint{4.366464in}{3.884776in}}%
\pgfpathlineto{\pgfqpoint{4.376616in}{3.870569in}}%
\pgfpathlineto{\pgfqpoint{4.408539in}{3.864601in}}%
\pgfpathlineto{\pgfqpoint{4.440440in}{3.859246in}}%
\pgfpathlineto{\pgfqpoint{4.430239in}{3.873573in}}%
\pgfpathlineto{\pgfqpoint{4.420056in}{3.887644in}}%
\pgfpathlineto{\pgfqpoint{4.388204in}{3.892887in}}%
\pgfpathlineto{\pgfqpoint{4.356329in}{3.898712in}}%
\pgfpathclose%
\pgfusepath{fill}%
\end{pgfscope}%
\begin{pgfscope}%
\pgfpathrectangle{\pgfqpoint{1.020000in}{0.880000in}}{\pgfqpoint{6.160000in}{6.160000in}}%
\pgfusepath{clip}%
\pgfsetbuttcap%
\pgfsetroundjoin%
\definecolor{currentfill}{rgb}{0.532568,0.669801,0.990393}%
\pgfsetfillcolor{currentfill}%
\pgfsetlinewidth{0.000000pt}%
\definecolor{currentstroke}{rgb}{0.000000,0.000000,0.000000}%
\pgfsetstrokecolor{currentstroke}%
\pgfsetdash{}{0pt}%
\pgfpathmoveto{\pgfqpoint{4.715718in}{3.801521in}}%
\pgfpathlineto{\pgfqpoint{4.726161in}{3.786162in}}%
\pgfpathlineto{\pgfqpoint{4.736620in}{3.770341in}}%
\pgfpathlineto{\pgfqpoint{4.768428in}{3.768151in}}%
\pgfpathlineto{\pgfqpoint{4.800217in}{3.766173in}}%
\pgfpathlineto{\pgfqpoint{4.789710in}{3.781907in}}%
\pgfpathlineto{\pgfqpoint{4.779218in}{3.797185in}}%
\pgfpathlineto{\pgfqpoint{4.747478in}{3.799262in}}%
\pgfpathlineto{\pgfqpoint{4.715718in}{3.801521in}}%
\pgfpathclose%
\pgfusepath{fill}%
\end{pgfscope}%
\begin{pgfscope}%
\pgfpathrectangle{\pgfqpoint{1.020000in}{0.880000in}}{\pgfqpoint{6.160000in}{6.160000in}}%
\pgfusepath{clip}%
\pgfsetbuttcap%
\pgfsetroundjoin%
\definecolor{currentfill}{rgb}{0.467678,0.605591,0.968546}%
\pgfsetfillcolor{currentfill}%
\pgfsetlinewidth{0.000000pt}%
\definecolor{currentstroke}{rgb}{0.000000,0.000000,0.000000}%
\pgfsetstrokecolor{currentstroke}%
\pgfsetdash{}{0pt}%
\pgfpathmoveto{\pgfqpoint{5.286738in}{3.690915in}}%
\pgfpathlineto{\pgfqpoint{5.297674in}{3.675466in}}%
\pgfpathlineto{\pgfqpoint{5.308627in}{3.659687in}}%
\pgfpathlineto{\pgfqpoint{5.340269in}{3.659898in}}%
\pgfpathlineto{\pgfqpoint{5.371890in}{3.659991in}}%
\pgfpathlineto{\pgfqpoint{5.360884in}{3.675267in}}%
\pgfpathlineto{\pgfqpoint{5.349895in}{3.690308in}}%
\pgfpathlineto{\pgfqpoint{5.318327in}{3.690634in}}%
\pgfpathlineto{\pgfqpoint{5.286738in}{3.690915in}}%
\pgfpathclose%
\pgfusepath{fill}%
\end{pgfscope}%
\begin{pgfscope}%
\pgfpathrectangle{\pgfqpoint{1.020000in}{0.880000in}}{\pgfqpoint{6.160000in}{6.160000in}}%
\pgfusepath{clip}%
\pgfsetbuttcap%
\pgfsetroundjoin%
\definecolor{currentfill}{rgb}{0.446431,0.582356,0.957373}%
\pgfsetfillcolor{currentfill}%
\pgfsetlinewidth{0.000000pt}%
\definecolor{currentstroke}{rgb}{0.000000,0.000000,0.000000}%
\pgfsetstrokecolor{currentstroke}%
\pgfsetdash{}{0pt}%
\pgfpathmoveto{\pgfqpoint{5.498164in}{3.659531in}}%
\pgfpathlineto{\pgfqpoint{5.509290in}{3.644721in}}%
\pgfpathlineto{\pgfqpoint{5.520434in}{3.629768in}}%
\pgfpathlineto{\pgfqpoint{5.552002in}{3.629769in}}%
\pgfpathlineto{\pgfqpoint{5.540833in}{3.644595in}}%
\pgfpathlineto{\pgfqpoint{5.529682in}{3.659294in}}%
\pgfpathlineto{\pgfqpoint{5.498164in}{3.659531in}}%
\pgfpathclose%
\pgfusepath{fill}%
\end{pgfscope}%
\begin{pgfscope}%
\pgfpathrectangle{\pgfqpoint{1.020000in}{0.880000in}}{\pgfqpoint{6.160000in}{6.160000in}}%
\pgfusepath{clip}%
\pgfsetbuttcap%
\pgfsetroundjoin%
\definecolor{currentfill}{rgb}{0.489246,0.627536,0.976896}%
\pgfsetfillcolor{currentfill}%
\pgfsetlinewidth{0.000000pt}%
\definecolor{currentstroke}{rgb}{0.000000,0.000000,0.000000}%
\pgfsetstrokecolor{currentstroke}%
\pgfsetdash{}{0pt}%
\pgfpathmoveto{\pgfqpoint{5.075260in}{3.724554in}}%
\pgfpathlineto{\pgfqpoint{5.086010in}{3.708397in}}%
\pgfpathlineto{\pgfqpoint{5.096774in}{3.691635in}}%
\pgfpathlineto{\pgfqpoint{5.128485in}{3.691590in}}%
\pgfpathlineto{\pgfqpoint{5.160176in}{3.691536in}}%
\pgfpathlineto{\pgfqpoint{5.149360in}{3.707667in}}%
\pgfpathlineto{\pgfqpoint{5.138558in}{3.723333in}}%
\pgfpathlineto{\pgfqpoint{5.106919in}{3.723932in}}%
\pgfpathlineto{\pgfqpoint{5.075260in}{3.724554in}}%
\pgfpathclose%
\pgfusepath{fill}%
\end{pgfscope}%
\begin{pgfscope}%
\pgfpathrectangle{\pgfqpoint{1.020000in}{0.880000in}}{\pgfqpoint{6.160000in}{6.160000in}}%
\pgfusepath{clip}%
\pgfsetbuttcap%
\pgfsetroundjoin%
\definecolor{currentfill}{rgb}{0.559747,0.694768,0.996075}%
\pgfsetfillcolor{currentfill}%
\pgfsetlinewidth{0.000000pt}%
\definecolor{currentstroke}{rgb}{0.000000,0.000000,0.000000}%
\pgfsetstrokecolor{currentstroke}%
\pgfsetdash{}{0pt}%
\pgfpathmoveto{\pgfqpoint{4.504176in}{3.850087in}}%
\pgfpathlineto{\pgfqpoint{4.514441in}{3.835417in}}%
\pgfpathlineto{\pgfqpoint{4.524724in}{3.820445in}}%
\pgfpathlineto{\pgfqpoint{4.556608in}{3.816476in}}%
\pgfpathlineto{\pgfqpoint{4.588472in}{3.812887in}}%
\pgfpathlineto{\pgfqpoint{4.578141in}{3.827897in}}%
\pgfpathlineto{\pgfqpoint{4.567827in}{3.842597in}}%
\pgfpathlineto{\pgfqpoint{4.536012in}{3.846160in}}%
\pgfpathlineto{\pgfqpoint{4.504176in}{3.850087in}}%
\pgfpathclose%
\pgfusepath{fill}%
\end{pgfscope}%
\begin{pgfscope}%
\pgfpathrectangle{\pgfqpoint{1.020000in}{0.880000in}}{\pgfqpoint{6.160000in}{6.160000in}}%
\pgfusepath{clip}%
\pgfsetbuttcap%
\pgfsetroundjoin%
\definecolor{currentfill}{rgb}{0.510824,0.649397,0.985079}%
\pgfsetfillcolor{currentfill}%
\pgfsetlinewidth{0.000000pt}%
\definecolor{currentstroke}{rgb}{0.000000,0.000000,0.000000}%
\pgfsetstrokecolor{currentstroke}%
\pgfsetdash{}{0pt}%
\pgfpathmoveto{\pgfqpoint{4.863734in}{3.762761in}}%
\pgfpathlineto{\pgfqpoint{4.874304in}{3.746647in}}%
\pgfpathlineto{\pgfqpoint{4.884889in}{3.729870in}}%
\pgfpathlineto{\pgfqpoint{4.916667in}{3.728670in}}%
\pgfpathlineto{\pgfqpoint{4.948425in}{3.727632in}}%
\pgfpathlineto{\pgfqpoint{4.937791in}{3.744081in}}%
\pgfpathlineto{\pgfqpoint{4.927171in}{3.759936in}}%
\pgfpathlineto{\pgfqpoint{4.895462in}{3.761286in}}%
\pgfpathlineto{\pgfqpoint{4.863734in}{3.762761in}}%
\pgfpathclose%
\pgfusepath{fill}%
\end{pgfscope}%
\begin{pgfscope}%
\pgfpathrectangle{\pgfqpoint{1.020000in}{0.880000in}}{\pgfqpoint{6.160000in}{6.160000in}}%
\pgfusepath{clip}%
\pgfsetbuttcap%
\pgfsetroundjoin%
\definecolor{currentfill}{rgb}{0.451739,0.588181,0.960201}%
\pgfsetfillcolor{currentfill}%
\pgfsetlinewidth{0.000000pt}%
\definecolor{currentstroke}{rgb}{0.000000,0.000000,0.000000}%
\pgfsetstrokecolor{currentstroke}%
\pgfsetdash{}{0pt}%
\pgfpathmoveto{\pgfqpoint{5.435068in}{3.659892in}}%
\pgfpathlineto{\pgfqpoint{5.446143in}{3.644789in}}%
\pgfpathlineto{\pgfqpoint{5.457235in}{3.629492in}}%
\pgfpathlineto{\pgfqpoint{5.488845in}{3.629685in}}%
\pgfpathlineto{\pgfqpoint{5.520434in}{3.629768in}}%
\pgfpathlineto{\pgfqpoint{5.509290in}{3.644721in}}%
\pgfpathlineto{\pgfqpoint{5.498164in}{3.659531in}}%
\pgfpathlineto{\pgfqpoint{5.466627in}{3.659736in}}%
\pgfpathlineto{\pgfqpoint{5.435068in}{3.659892in}}%
\pgfpathclose%
\pgfusepath{fill}%
\end{pgfscope}%
\begin{pgfscope}%
\pgfpathrectangle{\pgfqpoint{1.020000in}{0.880000in}}{\pgfqpoint{6.160000in}{6.160000in}}%
\pgfusepath{clip}%
\pgfsetbuttcap%
\pgfsetroundjoin%
\definecolor{currentfill}{rgb}{0.677823,0.786546,0.991005}%
\pgfsetfillcolor{currentfill}%
\pgfsetlinewidth{0.000000pt}%
\definecolor{currentstroke}{rgb}{0.000000,0.000000,0.000000}%
\pgfsetstrokecolor{currentstroke}%
\pgfsetdash{}{0pt}%
\pgfpathmoveto{\pgfqpoint{3.996832in}{4.050827in}}%
\pgfpathlineto{\pgfqpoint{4.006650in}{4.040978in}}%
\pgfpathlineto{\pgfqpoint{4.016487in}{4.030735in}}%
\pgfpathlineto{\pgfqpoint{4.048575in}{4.014812in}}%
\pgfpathlineto{\pgfqpoint{4.080631in}{4.000377in}}%
\pgfpathlineto{\pgfqpoint{4.070744in}{4.011593in}}%
\pgfpathlineto{\pgfqpoint{4.060875in}{4.022464in}}%
\pgfpathlineto{\pgfqpoint{4.028869in}{4.035964in}}%
\pgfpathlineto{\pgfqpoint{3.996832in}{4.050827in}}%
\pgfpathclose%
\pgfusepath{fill}%
\end{pgfscope}%
\begin{pgfscope}%
\pgfpathrectangle{\pgfqpoint{1.020000in}{0.880000in}}{\pgfqpoint{6.160000in}{6.160000in}}%
\pgfusepath{clip}%
\pgfsetbuttcap%
\pgfsetroundjoin%
\definecolor{currentfill}{rgb}{0.728970,0.817464,0.973188}%
\pgfsetfillcolor{currentfill}%
\pgfsetlinewidth{0.000000pt}%
\definecolor{currentstroke}{rgb}{0.000000,0.000000,0.000000}%
\pgfsetstrokecolor{currentstroke}%
\pgfsetdash{}{0pt}%
\pgfpathmoveto{\pgfqpoint{3.848986in}{4.138326in}}%
\pgfpathlineto{\pgfqpoint{3.858659in}{4.132237in}}%
\pgfpathlineto{\pgfqpoint{3.868351in}{4.125757in}}%
\pgfpathlineto{\pgfqpoint{3.900525in}{4.104532in}}%
\pgfpathlineto{\pgfqpoint{3.932662in}{4.085022in}}%
\pgfpathlineto{\pgfqpoint{3.922915in}{4.093170in}}%
\pgfpathlineto{\pgfqpoint{3.913187in}{4.100969in}}%
\pgfpathlineto{\pgfqpoint{3.881104in}{4.118876in}}%
\pgfpathlineto{\pgfqpoint{3.848986in}{4.138326in}}%
\pgfpathclose%
\pgfusepath{fill}%
\end{pgfscope}%
\begin{pgfscope}%
\pgfpathrectangle{\pgfqpoint{1.020000in}{0.880000in}}{\pgfqpoint{6.160000in}{6.160000in}}%
\pgfusepath{clip}%
\pgfsetbuttcap%
\pgfsetroundjoin%
\definecolor{currentfill}{rgb}{0.635474,0.756714,0.998297}%
\pgfsetfillcolor{currentfill}%
\pgfsetlinewidth{0.000000pt}%
\definecolor{currentstroke}{rgb}{0.000000,0.000000,0.000000}%
\pgfsetstrokecolor{currentstroke}%
\pgfsetdash{}{0pt}%
\pgfpathmoveto{\pgfqpoint{4.144657in}{3.975540in}}%
\pgfpathlineto{\pgfqpoint{4.154612in}{3.963255in}}%
\pgfpathlineto{\pgfqpoint{4.164584in}{3.950625in}}%
\pgfpathlineto{\pgfqpoint{4.196604in}{3.939473in}}%
\pgfpathlineto{\pgfqpoint{4.228598in}{3.929459in}}%
\pgfpathlineto{\pgfqpoint{4.218577in}{3.942563in}}%
\pgfpathlineto{\pgfqpoint{4.208574in}{3.955365in}}%
\pgfpathlineto{\pgfqpoint{4.176629in}{3.964922in}}%
\pgfpathlineto{\pgfqpoint{4.144657in}{3.975540in}}%
\pgfpathclose%
\pgfusepath{fill}%
\end{pgfscope}%
\begin{pgfscope}%
\pgfpathrectangle{\pgfqpoint{1.020000in}{0.880000in}}{\pgfqpoint{6.160000in}{6.160000in}}%
\pgfusepath{clip}%
\pgfsetbuttcap%
\pgfsetroundjoin%
\definecolor{currentfill}{rgb}{0.473070,0.611077,0.970634}%
\pgfsetfillcolor{currentfill}%
\pgfsetlinewidth{0.000000pt}%
\definecolor{currentstroke}{rgb}{0.000000,0.000000,0.000000}%
\pgfsetstrokecolor{currentstroke}%
\pgfsetdash{}{0pt}%
\pgfpathmoveto{\pgfqpoint{5.223498in}{3.691324in}}%
\pgfpathlineto{\pgfqpoint{5.234382in}{3.675331in}}%
\pgfpathlineto{\pgfqpoint{5.245280in}{3.658872in}}%
\pgfpathlineto{\pgfqpoint{5.276964in}{3.659347in}}%
\pgfpathlineto{\pgfqpoint{5.308627in}{3.659687in}}%
\pgfpathlineto{\pgfqpoint{5.297674in}{3.675466in}}%
\pgfpathlineto{\pgfqpoint{5.286738in}{3.690915in}}%
\pgfpathlineto{\pgfqpoint{5.255128in}{3.691146in}}%
\pgfpathlineto{\pgfqpoint{5.223498in}{3.691324in}}%
\pgfpathclose%
\pgfusepath{fill}%
\end{pgfscope}%
\begin{pgfscope}%
\pgfpathrectangle{\pgfqpoint{1.020000in}{0.880000in}}{\pgfqpoint{6.160000in}{6.160000in}}%
\pgfusepath{clip}%
\pgfsetbuttcap%
\pgfsetroundjoin%
\definecolor{currentfill}{rgb}{0.782049,0.842864,0.942980}%
\pgfsetfillcolor{currentfill}%
\pgfsetlinewidth{0.000000pt}%
\definecolor{currentstroke}{rgb}{0.000000,0.000000,0.000000}%
\pgfsetstrokecolor{currentstroke}%
\pgfsetdash{}{0pt}%
\pgfpathmoveto{\pgfqpoint{3.701085in}{4.234655in}}%
\pgfpathlineto{\pgfqpoint{3.710596in}{4.233745in}}%
\pgfpathlineto{\pgfqpoint{3.720128in}{4.232556in}}%
\pgfpathlineto{\pgfqpoint{3.752404in}{4.206472in}}%
\pgfpathlineto{\pgfqpoint{3.784638in}{4.182085in}}%
\pgfpathlineto{\pgfqpoint{3.775043in}{4.185754in}}%
\pgfpathlineto{\pgfqpoint{3.765469in}{4.189156in}}%
\pgfpathlineto{\pgfqpoint{3.733297in}{4.211154in}}%
\pgfpathlineto{\pgfqpoint{3.701085in}{4.234655in}}%
\pgfpathclose%
\pgfusepath{fill}%
\end{pgfscope}%
\begin{pgfscope}%
\pgfpathrectangle{\pgfqpoint{1.020000in}{0.880000in}}{\pgfqpoint{6.160000in}{6.160000in}}%
\pgfusepath{clip}%
\pgfsetbuttcap%
\pgfsetroundjoin%
\definecolor{currentfill}{rgb}{0.538004,0.674902,0.991722}%
\pgfsetfillcolor{currentfill}%
\pgfsetlinewidth{0.000000pt}%
\definecolor{currentstroke}{rgb}{0.000000,0.000000,0.000000}%
\pgfsetstrokecolor{currentstroke}%
\pgfsetdash{}{0pt}%
\pgfpathmoveto{\pgfqpoint{4.652136in}{3.806678in}}%
\pgfpathlineto{\pgfqpoint{4.662531in}{3.791290in}}%
\pgfpathlineto{\pgfqpoint{4.672942in}{3.775450in}}%
\pgfpathlineto{\pgfqpoint{4.704791in}{3.772765in}}%
\pgfpathlineto{\pgfqpoint{4.736620in}{3.770341in}}%
\pgfpathlineto{\pgfqpoint{4.726161in}{3.786162in}}%
\pgfpathlineto{\pgfqpoint{4.715718in}{3.801521in}}%
\pgfpathlineto{\pgfqpoint{4.683937in}{3.803984in}}%
\pgfpathlineto{\pgfqpoint{4.652136in}{3.806678in}}%
\pgfpathclose%
\pgfusepath{fill}%
\end{pgfscope}%
\begin{pgfscope}%
\pgfpathrectangle{\pgfqpoint{1.020000in}{0.880000in}}{\pgfqpoint{6.160000in}{6.160000in}}%
\pgfusepath{clip}%
\pgfsetbuttcap%
\pgfsetroundjoin%
\definecolor{currentfill}{rgb}{0.603162,0.731527,0.999565}%
\pgfsetfillcolor{currentfill}%
\pgfsetlinewidth{0.000000pt}%
\definecolor{currentstroke}{rgb}{0.000000,0.000000,0.000000}%
\pgfsetstrokecolor{currentstroke}%
\pgfsetdash{}{0pt}%
\pgfpathmoveto{\pgfqpoint{4.292511in}{3.912420in}}%
\pgfpathlineto{\pgfqpoint{4.302598in}{3.898693in}}%
\pgfpathlineto{\pgfqpoint{4.312702in}{3.884671in}}%
\pgfpathlineto{\pgfqpoint{4.344671in}{3.877230in}}%
\pgfpathlineto{\pgfqpoint{4.376616in}{3.870569in}}%
\pgfpathlineto{\pgfqpoint{4.366464in}{3.884776in}}%
\pgfpathlineto{\pgfqpoint{4.356329in}{3.898712in}}%
\pgfpathlineto{\pgfqpoint{4.324432in}{3.905196in}}%
\pgfpathlineto{\pgfqpoint{4.292511in}{3.912420in}}%
\pgfpathclose%
\pgfusepath{fill}%
\end{pgfscope}%
\begin{pgfscope}%
\pgfpathrectangle{\pgfqpoint{1.020000in}{0.880000in}}{\pgfqpoint{6.160000in}{6.160000in}}%
\pgfusepath{clip}%
\pgfsetbuttcap%
\pgfsetroundjoin%
\definecolor{currentfill}{rgb}{0.489246,0.627536,0.976896}%
\pgfsetfillcolor{currentfill}%
\pgfsetlinewidth{0.000000pt}%
\definecolor{currentstroke}{rgb}{0.000000,0.000000,0.000000}%
\pgfsetstrokecolor{currentstroke}%
\pgfsetdash{}{0pt}%
\pgfpathmoveto{\pgfqpoint{5.011882in}{3.725930in}}%
\pgfpathlineto{\pgfqpoint{5.022581in}{3.709250in}}%
\pgfpathlineto{\pgfqpoint{5.033291in}{3.691815in}}%
\pgfpathlineto{\pgfqpoint{5.065042in}{3.691698in}}%
\pgfpathlineto{\pgfqpoint{5.096774in}{3.691635in}}%
\pgfpathlineto{\pgfqpoint{5.086010in}{3.708397in}}%
\pgfpathlineto{\pgfqpoint{5.075260in}{3.724554in}}%
\pgfpathlineto{\pgfqpoint{5.043581in}{3.725213in}}%
\pgfpathlineto{\pgfqpoint{5.011882in}{3.725930in}}%
\pgfpathclose%
\pgfusepath{fill}%
\end{pgfscope}%
\begin{pgfscope}%
\pgfpathrectangle{\pgfqpoint{1.020000in}{0.880000in}}{\pgfqpoint{6.160000in}{6.160000in}}%
\pgfusepath{clip}%
\pgfsetbuttcap%
\pgfsetroundjoin%
\definecolor{currentfill}{rgb}{0.831148,0.859513,0.903110}%
\pgfsetfillcolor{currentfill}%
\pgfsetlinewidth{0.000000pt}%
\definecolor{currentstroke}{rgb}{0.000000,0.000000,0.000000}%
\pgfsetstrokecolor{currentstroke}%
\pgfsetdash{}{0pt}%
\pgfpathmoveto{\pgfqpoint{3.553134in}{4.332576in}}%
\pgfpathlineto{\pgfqpoint{3.562464in}{4.337949in}}%
\pgfpathlineto{\pgfqpoint{3.571813in}{4.343312in}}%
\pgfpathlineto{\pgfqpoint{3.604197in}{4.314041in}}%
\pgfpathlineto{\pgfqpoint{3.636536in}{4.286133in}}%
\pgfpathlineto{\pgfqpoint{3.627113in}{4.284001in}}%
\pgfpathlineto{\pgfqpoint{3.617709in}{4.281819in}}%
\pgfpathlineto{\pgfqpoint{3.585442in}{4.306602in}}%
\pgfpathlineto{\pgfqpoint{3.553134in}{4.332576in}}%
\pgfpathclose%
\pgfusepath{fill}%
\end{pgfscope}%
\begin{pgfscope}%
\pgfpathrectangle{\pgfqpoint{1.020000in}{0.880000in}}{\pgfqpoint{6.160000in}{6.160000in}}%
\pgfusepath{clip}%
\pgfsetbuttcap%
\pgfsetroundjoin%
\definecolor{currentfill}{rgb}{0.909460,0.839386,0.800331}%
\pgfsetfillcolor{currentfill}%
\pgfsetlinewidth{0.000000pt}%
\definecolor{currentstroke}{rgb}{0.000000,0.000000,0.000000}%
\pgfsetstrokecolor{currentstroke}%
\pgfsetdash{}{0pt}%
\pgfpathmoveto{\pgfqpoint{3.257396in}{4.493379in}}%
\pgfpathlineto{\pgfqpoint{3.266312in}{4.511067in}}%
\pgfpathlineto{\pgfqpoint{3.275241in}{4.529725in}}%
\pgfpathlineto{\pgfqpoint{3.307786in}{4.502656in}}%
\pgfpathlineto{\pgfqpoint{3.340294in}{4.475579in}}%
\pgfpathlineto{\pgfqpoint{3.331271in}{4.460570in}}%
\pgfpathlineto{\pgfqpoint{3.322262in}{4.446385in}}%
\pgfpathlineto{\pgfqpoint{3.289846in}{4.469880in}}%
\pgfpathlineto{\pgfqpoint{3.257396in}{4.493379in}}%
\pgfpathclose%
\pgfusepath{fill}%
\end{pgfscope}%
\begin{pgfscope}%
\pgfpathrectangle{\pgfqpoint{1.020000in}{0.880000in}}{\pgfqpoint{6.160000in}{6.160000in}}%
\pgfusepath{clip}%
\pgfsetbuttcap%
\pgfsetroundjoin%
\definecolor{currentfill}{rgb}{0.875557,0.860242,0.851430}%
\pgfsetfillcolor{currentfill}%
\pgfsetlinewidth{0.000000pt}%
\definecolor{currentstroke}{rgb}{0.000000,0.000000,0.000000}%
\pgfsetstrokecolor{currentstroke}%
\pgfsetdash{}{0pt}%
\pgfpathmoveto{\pgfqpoint{3.405198in}{4.422131in}}%
\pgfpathlineto{\pgfqpoint{3.414326in}{4.434091in}}%
\pgfpathlineto{\pgfqpoint{3.423473in}{4.446473in}}%
\pgfpathlineto{\pgfqpoint{3.455952in}{4.416742in}}%
\pgfpathlineto{\pgfqpoint{3.488389in}{4.387757in}}%
\pgfpathlineto{\pgfqpoint{3.479157in}{4.379061in}}%
\pgfpathlineto{\pgfqpoint{3.469943in}{4.370689in}}%
\pgfpathlineto{\pgfqpoint{3.437590in}{4.396087in}}%
\pgfpathlineto{\pgfqpoint{3.405198in}{4.422131in}}%
\pgfpathclose%
\pgfusepath{fill}%
\end{pgfscope}%
\begin{pgfscope}%
\pgfpathrectangle{\pgfqpoint{1.020000in}{0.880000in}}{\pgfqpoint{6.160000in}{6.160000in}}%
\pgfusepath{clip}%
\pgfsetbuttcap%
\pgfsetroundjoin%
\definecolor{currentfill}{rgb}{0.570616,0.704109,0.997195}%
\pgfsetfillcolor{currentfill}%
\pgfsetlinewidth{0.000000pt}%
\definecolor{currentstroke}{rgb}{0.000000,0.000000,0.000000}%
\pgfsetstrokecolor{currentstroke}%
\pgfsetdash{}{0pt}%
\pgfpathmoveto{\pgfqpoint{4.440440in}{3.859246in}}%
\pgfpathlineto{\pgfqpoint{4.450657in}{3.844643in}}%
\pgfpathlineto{\pgfqpoint{4.460892in}{3.829742in}}%
\pgfpathlineto{\pgfqpoint{4.492819in}{3.824847in}}%
\pgfpathlineto{\pgfqpoint{4.524724in}{3.820445in}}%
\pgfpathlineto{\pgfqpoint{4.514441in}{3.835417in}}%
\pgfpathlineto{\pgfqpoint{4.504176in}{3.850087in}}%
\pgfpathlineto{\pgfqpoint{4.472318in}{3.854429in}}%
\pgfpathlineto{\pgfqpoint{4.440440in}{3.859246in}}%
\pgfpathclose%
\pgfusepath{fill}%
\end{pgfscope}%
\begin{pgfscope}%
\pgfpathrectangle{\pgfqpoint{1.020000in}{0.880000in}}{\pgfqpoint{6.160000in}{6.160000in}}%
\pgfusepath{clip}%
\pgfsetbuttcap%
\pgfsetroundjoin%
\definecolor{currentfill}{rgb}{0.516260,0.654498,0.986407}%
\pgfsetfillcolor{currentfill}%
\pgfsetlinewidth{0.000000pt}%
\definecolor{currentstroke}{rgb}{0.000000,0.000000,0.000000}%
\pgfsetstrokecolor{currentstroke}%
\pgfsetdash{}{0pt}%
\pgfpathmoveto{\pgfqpoint{4.800217in}{3.766173in}}%
\pgfpathlineto{\pgfqpoint{4.810739in}{3.749873in}}%
\pgfpathlineto{\pgfqpoint{4.821275in}{3.732869in}}%
\pgfpathlineto{\pgfqpoint{4.853092in}{3.731261in}}%
\pgfpathlineto{\pgfqpoint{4.884889in}{3.729870in}}%
\pgfpathlineto{\pgfqpoint{4.874304in}{3.746647in}}%
\pgfpathlineto{\pgfqpoint{4.863734in}{3.762761in}}%
\pgfpathlineto{\pgfqpoint{4.831985in}{3.764384in}}%
\pgfpathlineto{\pgfqpoint{4.800217in}{3.766173in}}%
\pgfpathclose%
\pgfusepath{fill}%
\end{pgfscope}%
\begin{pgfscope}%
\pgfpathrectangle{\pgfqpoint{1.020000in}{0.880000in}}{\pgfqpoint{6.160000in}{6.160000in}}%
\pgfusepath{clip}%
\pgfsetbuttcap%
\pgfsetroundjoin%
\definecolor{currentfill}{rgb}{0.457046,0.594006,0.963029}%
\pgfsetfillcolor{currentfill}%
\pgfsetlinewidth{0.000000pt}%
\definecolor{currentstroke}{rgb}{0.000000,0.000000,0.000000}%
\pgfsetstrokecolor{currentstroke}%
\pgfsetdash{}{0pt}%
\pgfpathmoveto{\pgfqpoint{5.371890in}{3.659991in}}%
\pgfpathlineto{\pgfqpoint{5.382912in}{3.644458in}}%
\pgfpathlineto{\pgfqpoint{5.393950in}{3.628648in}}%
\pgfpathlineto{\pgfqpoint{5.425604in}{3.629157in}}%
\pgfpathlineto{\pgfqpoint{5.457235in}{3.629492in}}%
\pgfpathlineto{\pgfqpoint{5.446143in}{3.644789in}}%
\pgfpathlineto{\pgfqpoint{5.435068in}{3.659892in}}%
\pgfpathlineto{\pgfqpoint{5.403489in}{3.659983in}}%
\pgfpathlineto{\pgfqpoint{5.371890in}{3.659991in}}%
\pgfpathclose%
\pgfusepath{fill}%
\end{pgfscope}%
\begin{pgfscope}%
\pgfpathrectangle{\pgfqpoint{1.020000in}{0.880000in}}{\pgfqpoint{6.160000in}{6.160000in}}%
\pgfusepath{clip}%
\pgfsetbuttcap%
\pgfsetroundjoin%
\definecolor{currentfill}{rgb}{0.473070,0.611077,0.970634}%
\pgfsetfillcolor{currentfill}%
\pgfsetlinewidth{0.000000pt}%
\definecolor{currentstroke}{rgb}{0.000000,0.000000,0.000000}%
\pgfsetstrokecolor{currentstroke}%
\pgfsetdash{}{0pt}%
\pgfpathmoveto{\pgfqpoint{5.160176in}{3.691536in}}%
\pgfpathlineto{\pgfqpoint{5.171006in}{3.674855in}}%
\pgfpathlineto{\pgfqpoint{5.181849in}{3.657532in}}%
\pgfpathlineto{\pgfqpoint{5.213575in}{3.658263in}}%
\pgfpathlineto{\pgfqpoint{5.245280in}{3.658872in}}%
\pgfpathlineto{\pgfqpoint{5.234382in}{3.675331in}}%
\pgfpathlineto{\pgfqpoint{5.223498in}{3.691324in}}%
\pgfpathlineto{\pgfqpoint{5.191847in}{3.691452in}}%
\pgfpathlineto{\pgfqpoint{5.160176in}{3.691536in}}%
\pgfpathclose%
\pgfusepath{fill}%
\end{pgfscope}%
\begin{pgfscope}%
\pgfpathrectangle{\pgfqpoint{1.020000in}{0.880000in}}{\pgfqpoint{6.160000in}{6.160000in}}%
\pgfusepath{clip}%
\pgfsetbuttcap%
\pgfsetroundjoin%
\definecolor{currentfill}{rgb}{0.543440,0.680003,0.993051}%
\pgfsetfillcolor{currentfill}%
\pgfsetlinewidth{0.000000pt}%
\definecolor{currentstroke}{rgb}{0.000000,0.000000,0.000000}%
\pgfsetstrokecolor{currentstroke}%
\pgfsetdash{}{0pt}%
\pgfpathmoveto{\pgfqpoint{4.588472in}{3.812887in}}%
\pgfpathlineto{\pgfqpoint{4.598819in}{3.797516in}}%
\pgfpathlineto{\pgfqpoint{4.609182in}{3.781715in}}%
\pgfpathlineto{\pgfqpoint{4.641073in}{3.778423in}}%
\pgfpathlineto{\pgfqpoint{4.672942in}{3.775450in}}%
\pgfpathlineto{\pgfqpoint{4.662531in}{3.791290in}}%
\pgfpathlineto{\pgfqpoint{4.652136in}{3.806678in}}%
\pgfpathlineto{\pgfqpoint{4.620314in}{3.809635in}}%
\pgfpathlineto{\pgfqpoint{4.588472in}{3.812887in}}%
\pgfpathclose%
\pgfusepath{fill}%
\end{pgfscope}%
\begin{pgfscope}%
\pgfpathrectangle{\pgfqpoint{1.020000in}{0.880000in}}{\pgfqpoint{6.160000in}{6.160000in}}%
\pgfusepath{clip}%
\pgfsetbuttcap%
\pgfsetroundjoin%
\definecolor{currentfill}{rgb}{0.656683,0.771806,0.994914}%
\pgfsetfillcolor{currentfill}%
\pgfsetlinewidth{0.000000pt}%
\definecolor{currentstroke}{rgb}{0.000000,0.000000,0.000000}%
\pgfsetstrokecolor{currentstroke}%
\pgfsetdash{}{0pt}%
\pgfpathmoveto{\pgfqpoint{4.080631in}{4.000377in}}%
\pgfpathlineto{\pgfqpoint{4.090537in}{3.988789in}}%
\pgfpathlineto{\pgfqpoint{4.100461in}{3.976800in}}%
\pgfpathlineto{\pgfqpoint{4.132537in}{3.963029in}}%
\pgfpathlineto{\pgfqpoint{4.164584in}{3.950625in}}%
\pgfpathlineto{\pgfqpoint{4.154612in}{3.963255in}}%
\pgfpathlineto{\pgfqpoint{4.144657in}{3.975540in}}%
\pgfpathlineto{\pgfqpoint{4.112658in}{3.987323in}}%
\pgfpathlineto{\pgfqpoint{4.080631in}{4.000377in}}%
\pgfpathclose%
\pgfusepath{fill}%
\end{pgfscope}%
\begin{pgfscope}%
\pgfpathrectangle{\pgfqpoint{1.020000in}{0.880000in}}{\pgfqpoint{6.160000in}{6.160000in}}%
\pgfusepath{clip}%
\pgfsetbuttcap%
\pgfsetroundjoin%
\definecolor{currentfill}{rgb}{0.703587,0.802586,0.982847}%
\pgfsetfillcolor{currentfill}%
\pgfsetlinewidth{0.000000pt}%
\definecolor{currentstroke}{rgb}{0.000000,0.000000,0.000000}%
\pgfsetstrokecolor{currentstroke}%
\pgfsetdash{}{0pt}%
\pgfpathmoveto{\pgfqpoint{3.932662in}{4.085022in}}%
\pgfpathlineto{\pgfqpoint{3.942428in}{4.076471in}}%
\pgfpathlineto{\pgfqpoint{3.952213in}{4.067464in}}%
\pgfpathlineto{\pgfqpoint{3.984367in}{4.048252in}}%
\pgfpathlineto{\pgfqpoint{4.016487in}{4.030735in}}%
\pgfpathlineto{\pgfqpoint{4.006650in}{4.040978in}}%
\pgfpathlineto{\pgfqpoint{3.996832in}{4.050827in}}%
\pgfpathlineto{\pgfqpoint{3.964764in}{4.067149in}}%
\pgfpathlineto{\pgfqpoint{3.932662in}{4.085022in}}%
\pgfpathclose%
\pgfusepath{fill}%
\end{pgfscope}%
\begin{pgfscope}%
\pgfpathrectangle{\pgfqpoint{1.020000in}{0.880000in}}{\pgfqpoint{6.160000in}{6.160000in}}%
\pgfusepath{clip}%
\pgfsetbuttcap%
\pgfsetroundjoin%
\definecolor{currentfill}{rgb}{0.494638,0.633022,0.978983}%
\pgfsetfillcolor{currentfill}%
\pgfsetlinewidth{0.000000pt}%
\definecolor{currentstroke}{rgb}{0.000000,0.000000,0.000000}%
\pgfsetstrokecolor{currentstroke}%
\pgfsetdash{}{0pt}%
\pgfpathmoveto{\pgfqpoint{4.948425in}{3.727632in}}%
\pgfpathlineto{\pgfqpoint{4.959072in}{3.710451in}}%
\pgfpathlineto{\pgfqpoint{4.969732in}{3.692371in}}%
\pgfpathlineto{\pgfqpoint{5.001521in}{3.692025in}}%
\pgfpathlineto{\pgfqpoint{5.033291in}{3.691815in}}%
\pgfpathlineto{\pgfqpoint{5.022581in}{3.709250in}}%
\pgfpathlineto{\pgfqpoint{5.011882in}{3.725930in}}%
\pgfpathlineto{\pgfqpoint{4.980163in}{3.726728in}}%
\pgfpathlineto{\pgfqpoint{4.948425in}{3.727632in}}%
\pgfpathclose%
\pgfusepath{fill}%
\end{pgfscope}%
\begin{pgfscope}%
\pgfpathrectangle{\pgfqpoint{1.020000in}{0.880000in}}{\pgfqpoint{6.160000in}{6.160000in}}%
\pgfusepath{clip}%
\pgfsetbuttcap%
\pgfsetroundjoin%
\definecolor{currentfill}{rgb}{0.613933,0.739923,0.999142}%
\pgfsetfillcolor{currentfill}%
\pgfsetlinewidth{0.000000pt}%
\definecolor{currentstroke}{rgb}{0.000000,0.000000,0.000000}%
\pgfsetstrokecolor{currentstroke}%
\pgfsetdash{}{0pt}%
\pgfpathmoveto{\pgfqpoint{4.228598in}{3.929459in}}%
\pgfpathlineto{\pgfqpoint{4.238637in}{3.916037in}}%
\pgfpathlineto{\pgfqpoint{4.248693in}{3.902284in}}%
\pgfpathlineto{\pgfqpoint{4.280710in}{3.892989in}}%
\pgfpathlineto{\pgfqpoint{4.312702in}{3.884671in}}%
\pgfpathlineto{\pgfqpoint{4.302598in}{3.898693in}}%
\pgfpathlineto{\pgfqpoint{4.292511in}{3.912420in}}%
\pgfpathlineto{\pgfqpoint{4.260567in}{3.920475in}}%
\pgfpathlineto{\pgfqpoint{4.228598in}{3.929459in}}%
\pgfpathclose%
\pgfusepath{fill}%
\end{pgfscope}%
\begin{pgfscope}%
\pgfpathrectangle{\pgfqpoint{1.020000in}{0.880000in}}{\pgfqpoint{6.160000in}{6.160000in}}%
\pgfusepath{clip}%
\pgfsetbuttcap%
\pgfsetroundjoin%
\definecolor{currentfill}{rgb}{0.758539,0.832787,0.958408}%
\pgfsetfillcolor{currentfill}%
\pgfsetlinewidth{0.000000pt}%
\definecolor{currentstroke}{rgb}{0.000000,0.000000,0.000000}%
\pgfsetstrokecolor{currentstroke}%
\pgfsetdash{}{0pt}%
\pgfpathmoveto{\pgfqpoint{3.784638in}{4.182085in}}%
\pgfpathlineto{\pgfqpoint{3.794252in}{4.178065in}}%
\pgfpathlineto{\pgfqpoint{3.803886in}{4.173605in}}%
\pgfpathlineto{\pgfqpoint{3.836139in}{4.148764in}}%
\pgfpathlineto{\pgfqpoint{3.868351in}{4.125757in}}%
\pgfpathlineto{\pgfqpoint{3.858659in}{4.132237in}}%
\pgfpathlineto{\pgfqpoint{3.848986in}{4.138326in}}%
\pgfpathlineto{\pgfqpoint{3.816831in}{4.159380in}}%
\pgfpathlineto{\pgfqpoint{3.784638in}{4.182085in}}%
\pgfpathclose%
\pgfusepath{fill}%
\end{pgfscope}%
\begin{pgfscope}%
\pgfpathrectangle{\pgfqpoint{1.020000in}{0.880000in}}{\pgfqpoint{6.160000in}{6.160000in}}%
\pgfusepath{clip}%
\pgfsetbuttcap%
\pgfsetroundjoin%
\definecolor{currentfill}{rgb}{0.521696,0.659599,0.987736}%
\pgfsetfillcolor{currentfill}%
\pgfsetlinewidth{0.000000pt}%
\definecolor{currentstroke}{rgb}{0.000000,0.000000,0.000000}%
\pgfsetstrokecolor{currentstroke}%
\pgfsetdash{}{0pt}%
\pgfpathmoveto{\pgfqpoint{4.736620in}{3.770341in}}%
\pgfpathlineto{\pgfqpoint{4.747094in}{3.753947in}}%
\pgfpathlineto{\pgfqpoint{4.757583in}{3.736841in}}%
\pgfpathlineto{\pgfqpoint{4.789439in}{3.734721in}}%
\pgfpathlineto{\pgfqpoint{4.821275in}{3.732869in}}%
\pgfpathlineto{\pgfqpoint{4.810739in}{3.749873in}}%
\pgfpathlineto{\pgfqpoint{4.800217in}{3.766173in}}%
\pgfpathlineto{\pgfqpoint{4.768428in}{3.768151in}}%
\pgfpathlineto{\pgfqpoint{4.736620in}{3.770341in}}%
\pgfpathclose%
\pgfusepath{fill}%
\end{pgfscope}%
\begin{pgfscope}%
\pgfpathrectangle{\pgfqpoint{1.020000in}{0.880000in}}{\pgfqpoint{6.160000in}{6.160000in}}%
\pgfusepath{clip}%
\pgfsetbuttcap%
\pgfsetroundjoin%
\definecolor{currentfill}{rgb}{0.457046,0.594006,0.963029}%
\pgfsetfillcolor{currentfill}%
\pgfsetlinewidth{0.000000pt}%
\definecolor{currentstroke}{rgb}{0.000000,0.000000,0.000000}%
\pgfsetstrokecolor{currentstroke}%
\pgfsetdash{}{0pt}%
\pgfpathmoveto{\pgfqpoint{5.308627in}{3.659687in}}%
\pgfpathlineto{\pgfqpoint{5.319595in}{3.643539in}}%
\pgfpathlineto{\pgfqpoint{5.330578in}{3.626983in}}%
\pgfpathlineto{\pgfqpoint{5.362275in}{3.627933in}}%
\pgfpathlineto{\pgfqpoint{5.393950in}{3.628648in}}%
\pgfpathlineto{\pgfqpoint{5.382912in}{3.644458in}}%
\pgfpathlineto{\pgfqpoint{5.371890in}{3.659991in}}%
\pgfpathlineto{\pgfqpoint{5.340269in}{3.659898in}}%
\pgfpathlineto{\pgfqpoint{5.308627in}{3.659687in}}%
\pgfpathclose%
\pgfusepath{fill}%
\end{pgfscope}%
\begin{pgfscope}%
\pgfpathrectangle{\pgfqpoint{1.020000in}{0.880000in}}{\pgfqpoint{6.160000in}{6.160000in}}%
\pgfusepath{clip}%
\pgfsetbuttcap%
\pgfsetroundjoin%
\definecolor{currentfill}{rgb}{0.441123,0.576532,0.954545}%
\pgfsetfillcolor{currentfill}%
\pgfsetlinewidth{0.000000pt}%
\definecolor{currentstroke}{rgb}{0.000000,0.000000,0.000000}%
\pgfsetstrokecolor{currentstroke}%
\pgfsetdash{}{0pt}%
\pgfpathmoveto{\pgfqpoint{5.520434in}{3.629768in}}%
\pgfpathlineto{\pgfqpoint{5.531596in}{3.614670in}}%
\pgfpathlineto{\pgfqpoint{5.542776in}{3.599424in}}%
\pgfpathlineto{\pgfqpoint{5.574396in}{3.599734in}}%
\pgfpathlineto{\pgfqpoint{5.563190in}{3.614816in}}%
\pgfpathlineto{\pgfqpoint{5.552002in}{3.629769in}}%
\pgfpathlineto{\pgfqpoint{5.520434in}{3.629768in}}%
\pgfpathclose%
\pgfusepath{fill}%
\end{pgfscope}%
\begin{pgfscope}%
\pgfpathrectangle{\pgfqpoint{1.020000in}{0.880000in}}{\pgfqpoint{6.160000in}{6.160000in}}%
\pgfusepath{clip}%
\pgfsetbuttcap%
\pgfsetroundjoin%
\definecolor{currentfill}{rgb}{0.581486,0.713451,0.998314}%
\pgfsetfillcolor{currentfill}%
\pgfsetlinewidth{0.000000pt}%
\definecolor{currentstroke}{rgb}{0.000000,0.000000,0.000000}%
\pgfsetstrokecolor{currentstroke}%
\pgfsetdash{}{0pt}%
\pgfpathmoveto{\pgfqpoint{4.376616in}{3.870569in}}%
\pgfpathlineto{\pgfqpoint{4.386786in}{3.856077in}}%
\pgfpathlineto{\pgfqpoint{4.396973in}{3.841279in}}%
\pgfpathlineto{\pgfqpoint{4.428943in}{3.835194in}}%
\pgfpathlineto{\pgfqpoint{4.460892in}{3.829742in}}%
\pgfpathlineto{\pgfqpoint{4.450657in}{3.844643in}}%
\pgfpathlineto{\pgfqpoint{4.440440in}{3.859246in}}%
\pgfpathlineto{\pgfqpoint{4.408539in}{3.864601in}}%
\pgfpathlineto{\pgfqpoint{4.376616in}{3.870569in}}%
\pgfpathclose%
\pgfusepath{fill}%
\end{pgfscope}%
\begin{pgfscope}%
\pgfpathrectangle{\pgfqpoint{1.020000in}{0.880000in}}{\pgfqpoint{6.160000in}{6.160000in}}%
\pgfusepath{clip}%
\pgfsetbuttcap%
\pgfsetroundjoin%
\definecolor{currentfill}{rgb}{0.809329,0.852974,0.922323}%
\pgfsetfillcolor{currentfill}%
\pgfsetlinewidth{0.000000pt}%
\definecolor{currentstroke}{rgb}{0.000000,0.000000,0.000000}%
\pgfsetstrokecolor{currentstroke}%
\pgfsetdash{}{0pt}%
\pgfpathmoveto{\pgfqpoint{3.636536in}{4.286133in}}%
\pgfpathlineto{\pgfqpoint{3.645980in}{4.288100in}}%
\pgfpathlineto{\pgfqpoint{3.655444in}{4.289777in}}%
\pgfpathlineto{\pgfqpoint{3.687808in}{4.260334in}}%
\pgfpathlineto{\pgfqpoint{3.720128in}{4.232556in}}%
\pgfpathlineto{\pgfqpoint{3.710596in}{4.233745in}}%
\pgfpathlineto{\pgfqpoint{3.701085in}{4.234655in}}%
\pgfpathlineto{\pgfqpoint{3.668832in}{4.259656in}}%
\pgfpathlineto{\pgfqpoint{3.636536in}{4.286133in}}%
\pgfpathclose%
\pgfusepath{fill}%
\end{pgfscope}%
\begin{pgfscope}%
\pgfpathrectangle{\pgfqpoint{1.020000in}{0.880000in}}{\pgfqpoint{6.160000in}{6.160000in}}%
\pgfusepath{clip}%
\pgfsetbuttcap%
\pgfsetroundjoin%
\definecolor{currentfill}{rgb}{0.478462,0.616564,0.972721}%
\pgfsetfillcolor{currentfill}%
\pgfsetlinewidth{0.000000pt}%
\definecolor{currentstroke}{rgb}{0.000000,0.000000,0.000000}%
\pgfsetstrokecolor{currentstroke}%
\pgfsetdash{}{0pt}%
\pgfpathmoveto{\pgfqpoint{5.096774in}{3.691635in}}%
\pgfpathlineto{\pgfqpoint{5.107549in}{3.674149in}}%
\pgfpathlineto{\pgfqpoint{5.118336in}{3.655809in}}%
\pgfpathlineto{\pgfqpoint{5.150103in}{3.656702in}}%
\pgfpathlineto{\pgfqpoint{5.181849in}{3.657532in}}%
\pgfpathlineto{\pgfqpoint{5.171006in}{3.674855in}}%
\pgfpathlineto{\pgfqpoint{5.160176in}{3.691536in}}%
\pgfpathlineto{\pgfqpoint{5.128485in}{3.691590in}}%
\pgfpathlineto{\pgfqpoint{5.096774in}{3.691635in}}%
\pgfpathclose%
\pgfusepath{fill}%
\end{pgfscope}%
\begin{pgfscope}%
\pgfpathrectangle{\pgfqpoint{1.020000in}{0.880000in}}{\pgfqpoint{6.160000in}{6.160000in}}%
\pgfusepath{clip}%
\pgfsetbuttcap%
\pgfsetroundjoin%
\definecolor{currentfill}{rgb}{0.863392,0.865084,0.867634}%
\pgfsetfillcolor{currentfill}%
\pgfsetlinewidth{0.000000pt}%
\definecolor{currentstroke}{rgb}{0.000000,0.000000,0.000000}%
\pgfsetstrokecolor{currentstroke}%
\pgfsetdash{}{0pt}%
\pgfpathmoveto{\pgfqpoint{3.488389in}{4.387757in}}%
\pgfpathlineto{\pgfqpoint{3.497639in}{4.396640in}}%
\pgfpathlineto{\pgfqpoint{3.506910in}{4.405557in}}%
\pgfpathlineto{\pgfqpoint{3.539385in}{4.373856in}}%
\pgfpathlineto{\pgfqpoint{3.571813in}{4.343312in}}%
\pgfpathlineto{\pgfqpoint{3.562464in}{4.337949in}}%
\pgfpathlineto{\pgfqpoint{3.553134in}{4.332576in}}%
\pgfpathlineto{\pgfqpoint{3.520783in}{4.359661in}}%
\pgfpathlineto{\pgfqpoint{3.488389in}{4.387757in}}%
\pgfpathclose%
\pgfusepath{fill}%
\end{pgfscope}%
\begin{pgfscope}%
\pgfpathrectangle{\pgfqpoint{1.020000in}{0.880000in}}{\pgfqpoint{6.160000in}{6.160000in}}%
\pgfusepath{clip}%
\pgfsetbuttcap%
\pgfsetroundjoin%
\definecolor{currentfill}{rgb}{0.554312,0.690097,0.995516}%
\pgfsetfillcolor{currentfill}%
\pgfsetlinewidth{0.000000pt}%
\definecolor{currentstroke}{rgb}{0.000000,0.000000,0.000000}%
\pgfsetstrokecolor{currentstroke}%
\pgfsetdash{}{0pt}%
\pgfpathmoveto{\pgfqpoint{4.524724in}{3.820445in}}%
\pgfpathlineto{\pgfqpoint{4.535024in}{3.805128in}}%
\pgfpathlineto{\pgfqpoint{4.545340in}{3.789411in}}%
\pgfpathlineto{\pgfqpoint{4.577272in}{3.785364in}}%
\pgfpathlineto{\pgfqpoint{4.609182in}{3.781715in}}%
\pgfpathlineto{\pgfqpoint{4.598819in}{3.797516in}}%
\pgfpathlineto{\pgfqpoint{4.588472in}{3.812887in}}%
\pgfpathlineto{\pgfqpoint{4.556608in}{3.816476in}}%
\pgfpathlineto{\pgfqpoint{4.524724in}{3.820445in}}%
\pgfpathclose%
\pgfusepath{fill}%
\end{pgfscope}%
\begin{pgfscope}%
\pgfpathrectangle{\pgfqpoint{1.020000in}{0.880000in}}{\pgfqpoint{6.160000in}{6.160000in}}%
\pgfusepath{clip}%
\pgfsetbuttcap%
\pgfsetroundjoin%
\definecolor{currentfill}{rgb}{0.928116,0.822197,0.765141}%
\pgfsetfillcolor{currentfill}%
\pgfsetlinewidth{0.000000pt}%
\definecolor{currentstroke}{rgb}{0.000000,0.000000,0.000000}%
\pgfsetstrokecolor{currentstroke}%
\pgfsetdash{}{0pt}%
\pgfpathmoveto{\pgfqpoint{3.192399in}{4.539728in}}%
\pgfpathlineto{\pgfqpoint{3.201218in}{4.560833in}}%
\pgfpathlineto{\pgfqpoint{3.210049in}{4.583050in}}%
\pgfpathlineto{\pgfqpoint{3.242661in}{4.556592in}}%
\pgfpathlineto{\pgfqpoint{3.275241in}{4.529725in}}%
\pgfpathlineto{\pgfqpoint{3.266312in}{4.511067in}}%
\pgfpathlineto{\pgfqpoint{3.257396in}{4.493379in}}%
\pgfpathlineto{\pgfqpoint{3.224913in}{4.516719in}}%
\pgfpathlineto{\pgfqpoint{3.192399in}{4.539728in}}%
\pgfpathclose%
\pgfusepath{fill}%
\end{pgfscope}%
\begin{pgfscope}%
\pgfpathrectangle{\pgfqpoint{1.020000in}{0.880000in}}{\pgfqpoint{6.160000in}{6.160000in}}%
\pgfusepath{clip}%
\pgfsetbuttcap%
\pgfsetroundjoin%
\definecolor{currentfill}{rgb}{0.500031,0.638508,0.981070}%
\pgfsetfillcolor{currentfill}%
\pgfsetlinewidth{0.000000pt}%
\definecolor{currentstroke}{rgb}{0.000000,0.000000,0.000000}%
\pgfsetstrokecolor{currentstroke}%
\pgfsetdash{}{0pt}%
\pgfpathmoveto{\pgfqpoint{4.884889in}{3.729870in}}%
\pgfpathlineto{\pgfqpoint{4.895487in}{3.712271in}}%
\pgfpathlineto{\pgfqpoint{4.906096in}{3.693660in}}%
\pgfpathlineto{\pgfqpoint{4.937923in}{3.692901in}}%
\pgfpathlineto{\pgfqpoint{4.969732in}{3.692371in}}%
\pgfpathlineto{\pgfqpoint{4.959072in}{3.710451in}}%
\pgfpathlineto{\pgfqpoint{4.948425in}{3.727632in}}%
\pgfpathlineto{\pgfqpoint{4.916667in}{3.728670in}}%
\pgfpathlineto{\pgfqpoint{4.884889in}{3.729870in}}%
\pgfpathclose%
\pgfusepath{fill}%
\end{pgfscope}%
\begin{pgfscope}%
\pgfpathrectangle{\pgfqpoint{1.020000in}{0.880000in}}{\pgfqpoint{6.160000in}{6.160000in}}%
\pgfusepath{clip}%
\pgfsetbuttcap%
\pgfsetroundjoin%
\definecolor{currentfill}{rgb}{0.902849,0.844796,0.811970}%
\pgfsetfillcolor{currentfill}%
\pgfsetlinewidth{0.000000pt}%
\definecolor{currentstroke}{rgb}{0.000000,0.000000,0.000000}%
\pgfsetstrokecolor{currentstroke}%
\pgfsetdash{}{0pt}%
\pgfpathmoveto{\pgfqpoint{3.340294in}{4.475579in}}%
\pgfpathlineto{\pgfqpoint{3.349333in}{4.491277in}}%
\pgfpathlineto{\pgfqpoint{3.358389in}{4.507500in}}%
\pgfpathlineto{\pgfqpoint{3.390951in}{4.476787in}}%
\pgfpathlineto{\pgfqpoint{3.423473in}{4.446473in}}%
\pgfpathlineto{\pgfqpoint{3.414326in}{4.434091in}}%
\pgfpathlineto{\pgfqpoint{3.405198in}{4.422131in}}%
\pgfpathlineto{\pgfqpoint{3.372765in}{4.448680in}}%
\pgfpathlineto{\pgfqpoint{3.340294in}{4.475579in}}%
\pgfpathclose%
\pgfusepath{fill}%
\end{pgfscope}%
\begin{pgfscope}%
\pgfpathrectangle{\pgfqpoint{1.020000in}{0.880000in}}{\pgfqpoint{6.160000in}{6.160000in}}%
\pgfusepath{clip}%
\pgfsetbuttcap%
\pgfsetroundjoin%
\definecolor{currentfill}{rgb}{0.446431,0.582356,0.957373}%
\pgfsetfillcolor{currentfill}%
\pgfsetlinewidth{0.000000pt}%
\definecolor{currentstroke}{rgb}{0.000000,0.000000,0.000000}%
\pgfsetstrokecolor{currentstroke}%
\pgfsetdash{}{0pt}%
\pgfpathmoveto{\pgfqpoint{5.457235in}{3.629492in}}%
\pgfpathlineto{\pgfqpoint{5.468344in}{3.613992in}}%
\pgfpathlineto{\pgfqpoint{5.479471in}{3.598284in}}%
\pgfpathlineto{\pgfqpoint{5.511134in}{3.598956in}}%
\pgfpathlineto{\pgfqpoint{5.542776in}{3.599424in}}%
\pgfpathlineto{\pgfqpoint{5.531596in}{3.614670in}}%
\pgfpathlineto{\pgfqpoint{5.520434in}{3.629768in}}%
\pgfpathlineto{\pgfqpoint{5.488845in}{3.629685in}}%
\pgfpathlineto{\pgfqpoint{5.457235in}{3.629492in}}%
\pgfpathclose%
\pgfusepath{fill}%
\end{pgfscope}%
\begin{pgfscope}%
\pgfpathrectangle{\pgfqpoint{1.020000in}{0.880000in}}{\pgfqpoint{6.160000in}{6.160000in}}%
\pgfusepath{clip}%
\pgfsetbuttcap%
\pgfsetroundjoin%
\definecolor{currentfill}{rgb}{0.677823,0.786546,0.991005}%
\pgfsetfillcolor{currentfill}%
\pgfsetlinewidth{0.000000pt}%
\definecolor{currentstroke}{rgb}{0.000000,0.000000,0.000000}%
\pgfsetstrokecolor{currentstroke}%
\pgfsetdash{}{0pt}%
\pgfpathmoveto{\pgfqpoint{4.016487in}{4.030735in}}%
\pgfpathlineto{\pgfqpoint{4.026343in}{4.020058in}}%
\pgfpathlineto{\pgfqpoint{4.036217in}{4.008909in}}%
\pgfpathlineto{\pgfqpoint{4.068354in}{3.992054in}}%
\pgfpathlineto{\pgfqpoint{4.100461in}{3.976800in}}%
\pgfpathlineto{\pgfqpoint{4.090537in}{3.988789in}}%
\pgfpathlineto{\pgfqpoint{4.080631in}{4.000377in}}%
\pgfpathlineto{\pgfqpoint{4.048575in}{4.014812in}}%
\pgfpathlineto{\pgfqpoint{4.016487in}{4.030735in}}%
\pgfpathclose%
\pgfusepath{fill}%
\end{pgfscope}%
\begin{pgfscope}%
\pgfpathrectangle{\pgfqpoint{1.020000in}{0.880000in}}{\pgfqpoint{6.160000in}{6.160000in}}%
\pgfusepath{clip}%
\pgfsetbuttcap%
\pgfsetroundjoin%
\definecolor{currentfill}{rgb}{0.630089,0.752516,0.998508}%
\pgfsetfillcolor{currentfill}%
\pgfsetlinewidth{0.000000pt}%
\definecolor{currentstroke}{rgb}{0.000000,0.000000,0.000000}%
\pgfsetstrokecolor{currentstroke}%
\pgfsetdash{}{0pt}%
\pgfpathmoveto{\pgfqpoint{4.164584in}{3.950625in}}%
\pgfpathlineto{\pgfqpoint{4.174575in}{3.937631in}}%
\pgfpathlineto{\pgfqpoint{4.184583in}{3.924254in}}%
\pgfpathlineto{\pgfqpoint{4.216651in}{3.912668in}}%
\pgfpathlineto{\pgfqpoint{4.248693in}{3.902284in}}%
\pgfpathlineto{\pgfqpoint{4.238637in}{3.916037in}}%
\pgfpathlineto{\pgfqpoint{4.228598in}{3.929459in}}%
\pgfpathlineto{\pgfqpoint{4.196604in}{3.939473in}}%
\pgfpathlineto{\pgfqpoint{4.164584in}{3.950625in}}%
\pgfpathclose%
\pgfusepath{fill}%
\end{pgfscope}%
\begin{pgfscope}%
\pgfpathrectangle{\pgfqpoint{1.020000in}{0.880000in}}{\pgfqpoint{6.160000in}{6.160000in}}%
\pgfusepath{clip}%
\pgfsetbuttcap%
\pgfsetroundjoin%
\definecolor{currentfill}{rgb}{0.462354,0.599830,0.965857}%
\pgfsetfillcolor{currentfill}%
\pgfsetlinewidth{0.000000pt}%
\definecolor{currentstroke}{rgb}{0.000000,0.000000,0.000000}%
\pgfsetstrokecolor{currentstroke}%
\pgfsetdash{}{0pt}%
\pgfpathmoveto{\pgfqpoint{5.245280in}{3.658872in}}%
\pgfpathlineto{\pgfqpoint{5.256192in}{3.641883in}}%
\pgfpathlineto{\pgfqpoint{5.267117in}{3.624302in}}%
\pgfpathlineto{\pgfqpoint{5.298858in}{3.625777in}}%
\pgfpathlineto{\pgfqpoint{5.330578in}{3.626983in}}%
\pgfpathlineto{\pgfqpoint{5.319595in}{3.643539in}}%
\pgfpathlineto{\pgfqpoint{5.308627in}{3.659687in}}%
\pgfpathlineto{\pgfqpoint{5.276964in}{3.659347in}}%
\pgfpathlineto{\pgfqpoint{5.245280in}{3.658872in}}%
\pgfpathclose%
\pgfusepath{fill}%
\end{pgfscope}%
\begin{pgfscope}%
\pgfpathrectangle{\pgfqpoint{1.020000in}{0.880000in}}{\pgfqpoint{6.160000in}{6.160000in}}%
\pgfusepath{clip}%
\pgfsetbuttcap%
\pgfsetroundjoin%
\definecolor{currentfill}{rgb}{0.532568,0.669801,0.990393}%
\pgfsetfillcolor{currentfill}%
\pgfsetlinewidth{0.000000pt}%
\definecolor{currentstroke}{rgb}{0.000000,0.000000,0.000000}%
\pgfsetstrokecolor{currentstroke}%
\pgfsetdash{}{0pt}%
\pgfpathmoveto{\pgfqpoint{4.672942in}{3.775450in}}%
\pgfpathlineto{\pgfqpoint{4.683369in}{3.759054in}}%
\pgfpathlineto{\pgfqpoint{4.693810in}{3.741974in}}%
\pgfpathlineto{\pgfqpoint{4.725707in}{3.739252in}}%
\pgfpathlineto{\pgfqpoint{4.757583in}{3.736841in}}%
\pgfpathlineto{\pgfqpoint{4.747094in}{3.753947in}}%
\pgfpathlineto{\pgfqpoint{4.736620in}{3.770341in}}%
\pgfpathlineto{\pgfqpoint{4.704791in}{3.772765in}}%
\pgfpathlineto{\pgfqpoint{4.672942in}{3.775450in}}%
\pgfpathclose%
\pgfusepath{fill}%
\end{pgfscope}%
\begin{pgfscope}%
\pgfpathrectangle{\pgfqpoint{1.020000in}{0.880000in}}{\pgfqpoint{6.160000in}{6.160000in}}%
\pgfusepath{clip}%
\pgfsetbuttcap%
\pgfsetroundjoin%
\definecolor{currentfill}{rgb}{0.728970,0.817464,0.973188}%
\pgfsetfillcolor{currentfill}%
\pgfsetlinewidth{0.000000pt}%
\definecolor{currentstroke}{rgb}{0.000000,0.000000,0.000000}%
\pgfsetstrokecolor{currentstroke}%
\pgfsetdash{}{0pt}%
\pgfpathmoveto{\pgfqpoint{3.868351in}{4.125757in}}%
\pgfpathlineto{\pgfqpoint{3.878063in}{4.118817in}}%
\pgfpathlineto{\pgfqpoint{3.887794in}{4.111345in}}%
\pgfpathlineto{\pgfqpoint{3.920023in}{4.088467in}}%
\pgfpathlineto{\pgfqpoint{3.952213in}{4.067464in}}%
\pgfpathlineto{\pgfqpoint{3.942428in}{4.076471in}}%
\pgfpathlineto{\pgfqpoint{3.932662in}{4.085022in}}%
\pgfpathlineto{\pgfqpoint{3.900525in}{4.104532in}}%
\pgfpathlineto{\pgfqpoint{3.868351in}{4.125757in}}%
\pgfpathclose%
\pgfusepath{fill}%
\end{pgfscope}%
\begin{pgfscope}%
\pgfpathrectangle{\pgfqpoint{1.020000in}{0.880000in}}{\pgfqpoint{6.160000in}{6.160000in}}%
\pgfusepath{clip}%
\pgfsetbuttcap%
\pgfsetroundjoin%
\definecolor{currentfill}{rgb}{0.592356,0.722792,0.999434}%
\pgfsetfillcolor{currentfill}%
\pgfsetlinewidth{0.000000pt}%
\definecolor{currentstroke}{rgb}{0.000000,0.000000,0.000000}%
\pgfsetstrokecolor{currentstroke}%
\pgfsetdash{}{0pt}%
\pgfpathmoveto{\pgfqpoint{4.312702in}{3.884671in}}%
\pgfpathlineto{\pgfqpoint{4.322824in}{3.870341in}}%
\pgfpathlineto{\pgfqpoint{4.332963in}{3.855688in}}%
\pgfpathlineto{\pgfqpoint{4.364979in}{3.848080in}}%
\pgfpathlineto{\pgfqpoint{4.396973in}{3.841279in}}%
\pgfpathlineto{\pgfqpoint{4.386786in}{3.856077in}}%
\pgfpathlineto{\pgfqpoint{4.376616in}{3.870569in}}%
\pgfpathlineto{\pgfqpoint{4.344671in}{3.877230in}}%
\pgfpathlineto{\pgfqpoint{4.312702in}{3.884671in}}%
\pgfpathclose%
\pgfusepath{fill}%
\end{pgfscope}%
\begin{pgfscope}%
\pgfpathrectangle{\pgfqpoint{1.020000in}{0.880000in}}{\pgfqpoint{6.160000in}{6.160000in}}%
\pgfusepath{clip}%
\pgfsetbuttcap%
\pgfsetroundjoin%
\definecolor{currentfill}{rgb}{0.483854,0.622050,0.974808}%
\pgfsetfillcolor{currentfill}%
\pgfsetlinewidth{0.000000pt}%
\definecolor{currentstroke}{rgb}{0.000000,0.000000,0.000000}%
\pgfsetstrokecolor{currentstroke}%
\pgfsetdash{}{0pt}%
\pgfpathmoveto{\pgfqpoint{5.033291in}{3.691815in}}%
\pgfpathlineto{\pgfqpoint{5.044013in}{3.673469in}}%
\pgfpathlineto{\pgfqpoint{5.054745in}{3.654040in}}%
\pgfpathlineto{\pgfqpoint{5.086550in}{3.654902in}}%
\pgfpathlineto{\pgfqpoint{5.118336in}{3.655809in}}%
\pgfpathlineto{\pgfqpoint{5.107549in}{3.674149in}}%
\pgfpathlineto{\pgfqpoint{5.096774in}{3.691635in}}%
\pgfpathlineto{\pgfqpoint{5.065042in}{3.691698in}}%
\pgfpathlineto{\pgfqpoint{5.033291in}{3.691815in}}%
\pgfpathclose%
\pgfusepath{fill}%
\end{pgfscope}%
\begin{pgfscope}%
\pgfpathrectangle{\pgfqpoint{1.020000in}{0.880000in}}{\pgfqpoint{6.160000in}{6.160000in}}%
\pgfusepath{clip}%
\pgfsetbuttcap%
\pgfsetroundjoin%
\definecolor{currentfill}{rgb}{0.786721,0.844807,0.939810}%
\pgfsetfillcolor{currentfill}%
\pgfsetlinewidth{0.000000pt}%
\definecolor{currentstroke}{rgb}{0.000000,0.000000,0.000000}%
\pgfsetstrokecolor{currentstroke}%
\pgfsetdash{}{0pt}%
\pgfpathmoveto{\pgfqpoint{3.720128in}{4.232556in}}%
\pgfpathlineto{\pgfqpoint{3.729680in}{4.230983in}}%
\pgfpathlineto{\pgfqpoint{3.739253in}{4.228914in}}%
\pgfpathlineto{\pgfqpoint{3.771592in}{4.200316in}}%
\pgfpathlineto{\pgfqpoint{3.803886in}{4.173605in}}%
\pgfpathlineto{\pgfqpoint{3.794252in}{4.178065in}}%
\pgfpathlineto{\pgfqpoint{3.784638in}{4.182085in}}%
\pgfpathlineto{\pgfqpoint{3.752404in}{4.206472in}}%
\pgfpathlineto{\pgfqpoint{3.720128in}{4.232556in}}%
\pgfpathclose%
\pgfusepath{fill}%
\end{pgfscope}%
\begin{pgfscope}%
\pgfpathrectangle{\pgfqpoint{1.020000in}{0.880000in}}{\pgfqpoint{6.160000in}{6.160000in}}%
\pgfusepath{clip}%
\pgfsetbuttcap%
\pgfsetroundjoin%
\definecolor{currentfill}{rgb}{0.505423,0.643995,0.983157}%
\pgfsetfillcolor{currentfill}%
\pgfsetlinewidth{0.000000pt}%
\definecolor{currentstroke}{rgb}{0.000000,0.000000,0.000000}%
\pgfsetstrokecolor{currentstroke}%
\pgfsetdash{}{0pt}%
\pgfpathmoveto{\pgfqpoint{4.821275in}{3.732869in}}%
\pgfpathlineto{\pgfqpoint{4.831824in}{3.714990in}}%
\pgfpathlineto{\pgfqpoint{4.842385in}{3.696034in}}%
\pgfpathlineto{\pgfqpoint{4.874250in}{3.694691in}}%
\pgfpathlineto{\pgfqpoint{4.906096in}{3.693660in}}%
\pgfpathlineto{\pgfqpoint{4.895487in}{3.712271in}}%
\pgfpathlineto{\pgfqpoint{4.884889in}{3.729870in}}%
\pgfpathlineto{\pgfqpoint{4.853092in}{3.731261in}}%
\pgfpathlineto{\pgfqpoint{4.821275in}{3.732869in}}%
\pgfpathclose%
\pgfusepath{fill}%
\end{pgfscope}%
\begin{pgfscope}%
\pgfpathrectangle{\pgfqpoint{1.020000in}{0.880000in}}{\pgfqpoint{6.160000in}{6.160000in}}%
\pgfusepath{clip}%
\pgfsetbuttcap%
\pgfsetroundjoin%
\definecolor{currentfill}{rgb}{0.565182,0.699438,0.996635}%
\pgfsetfillcolor{currentfill}%
\pgfsetlinewidth{0.000000pt}%
\definecolor{currentstroke}{rgb}{0.000000,0.000000,0.000000}%
\pgfsetstrokecolor{currentstroke}%
\pgfsetdash{}{0pt}%
\pgfpathmoveto{\pgfqpoint{4.460892in}{3.829742in}}%
\pgfpathlineto{\pgfqpoint{4.471144in}{3.814508in}}%
\pgfpathlineto{\pgfqpoint{4.481412in}{3.798900in}}%
\pgfpathlineto{\pgfqpoint{4.513387in}{3.793904in}}%
\pgfpathlineto{\pgfqpoint{4.545340in}{3.789411in}}%
\pgfpathlineto{\pgfqpoint{4.535024in}{3.805128in}}%
\pgfpathlineto{\pgfqpoint{4.524724in}{3.820445in}}%
\pgfpathlineto{\pgfqpoint{4.492819in}{3.824847in}}%
\pgfpathlineto{\pgfqpoint{4.460892in}{3.829742in}}%
\pgfpathclose%
\pgfusepath{fill}%
\end{pgfscope}%
\begin{pgfscope}%
\pgfpathrectangle{\pgfqpoint{1.020000in}{0.880000in}}{\pgfqpoint{6.160000in}{6.160000in}}%
\pgfusepath{clip}%
\pgfsetbuttcap%
\pgfsetroundjoin%
\definecolor{currentfill}{rgb}{0.446431,0.582356,0.957373}%
\pgfsetfillcolor{currentfill}%
\pgfsetlinewidth{0.000000pt}%
\definecolor{currentstroke}{rgb}{0.000000,0.000000,0.000000}%
\pgfsetstrokecolor{currentstroke}%
\pgfsetdash{}{0pt}%
\pgfpathmoveto{\pgfqpoint{5.393950in}{3.628648in}}%
\pgfpathlineto{\pgfqpoint{5.405005in}{3.612540in}}%
\pgfpathlineto{\pgfqpoint{5.416075in}{3.596122in}}%
\pgfpathlineto{\pgfqpoint{5.447784in}{3.597358in}}%
\pgfpathlineto{\pgfqpoint{5.479471in}{3.598284in}}%
\pgfpathlineto{\pgfqpoint{5.468344in}{3.613992in}}%
\pgfpathlineto{\pgfqpoint{5.457235in}{3.629492in}}%
\pgfpathlineto{\pgfqpoint{5.425604in}{3.629157in}}%
\pgfpathlineto{\pgfqpoint{5.393950in}{3.628648in}}%
\pgfpathclose%
\pgfusepath{fill}%
\end{pgfscope}%
\begin{pgfscope}%
\pgfpathrectangle{\pgfqpoint{1.020000in}{0.880000in}}{\pgfqpoint{6.160000in}{6.160000in}}%
\pgfusepath{clip}%
\pgfsetbuttcap%
\pgfsetroundjoin%
\definecolor{currentfill}{rgb}{0.843358,0.861820,0.890017}%
\pgfsetfillcolor{currentfill}%
\pgfsetlinewidth{0.000000pt}%
\definecolor{currentstroke}{rgb}{0.000000,0.000000,0.000000}%
\pgfsetstrokecolor{currentstroke}%
\pgfsetdash{}{0pt}%
\pgfpathmoveto{\pgfqpoint{3.571813in}{4.343312in}}%
\pgfpathlineto{\pgfqpoint{3.581184in}{4.348524in}}%
\pgfpathlineto{\pgfqpoint{3.590576in}{4.353433in}}%
\pgfpathlineto{\pgfqpoint{3.623034in}{4.320835in}}%
\pgfpathlineto{\pgfqpoint{3.655444in}{4.289777in}}%
\pgfpathlineto{\pgfqpoint{3.645980in}{4.288100in}}%
\pgfpathlineto{\pgfqpoint{3.636536in}{4.286133in}}%
\pgfpathlineto{\pgfqpoint{3.604197in}{4.314041in}}%
\pgfpathlineto{\pgfqpoint{3.571813in}{4.343312in}}%
\pgfpathclose%
\pgfusepath{fill}%
\end{pgfscope}%
\begin{pgfscope}%
\pgfpathrectangle{\pgfqpoint{1.020000in}{0.880000in}}{\pgfqpoint{6.160000in}{6.160000in}}%
\pgfusepath{clip}%
\pgfsetbuttcap%
\pgfsetroundjoin%
\definecolor{currentfill}{rgb}{0.462354,0.599830,0.965857}%
\pgfsetfillcolor{currentfill}%
\pgfsetlinewidth{0.000000pt}%
\definecolor{currentstroke}{rgb}{0.000000,0.000000,0.000000}%
\pgfsetstrokecolor{currentstroke}%
\pgfsetdash{}{0pt}%
\pgfpathmoveto{\pgfqpoint{5.181849in}{3.657532in}}%
\pgfpathlineto{\pgfqpoint{5.192704in}{3.639468in}}%
\pgfpathlineto{\pgfqpoint{5.203569in}{3.620568in}}%
\pgfpathlineto{\pgfqpoint{5.235354in}{3.622559in}}%
\pgfpathlineto{\pgfqpoint{5.267117in}{3.624302in}}%
\pgfpathlineto{\pgfqpoint{5.256192in}{3.641883in}}%
\pgfpathlineto{\pgfqpoint{5.245280in}{3.658872in}}%
\pgfpathlineto{\pgfqpoint{5.213575in}{3.658263in}}%
\pgfpathlineto{\pgfqpoint{5.181849in}{3.657532in}}%
\pgfpathclose%
\pgfusepath{fill}%
\end{pgfscope}%
\begin{pgfscope}%
\pgfpathrectangle{\pgfqpoint{1.020000in}{0.880000in}}{\pgfqpoint{6.160000in}{6.160000in}}%
\pgfusepath{clip}%
\pgfsetbuttcap%
\pgfsetroundjoin%
\definecolor{currentfill}{rgb}{0.538004,0.674902,0.991722}%
\pgfsetfillcolor{currentfill}%
\pgfsetlinewidth{0.000000pt}%
\definecolor{currentstroke}{rgb}{0.000000,0.000000,0.000000}%
\pgfsetstrokecolor{currentstroke}%
\pgfsetdash{}{0pt}%
\pgfpathmoveto{\pgfqpoint{4.609182in}{3.781715in}}%
\pgfpathlineto{\pgfqpoint{4.619562in}{3.765397in}}%
\pgfpathlineto{\pgfqpoint{4.629957in}{3.748447in}}%
\pgfpathlineto{\pgfqpoint{4.661894in}{3.745031in}}%
\pgfpathlineto{\pgfqpoint{4.693810in}{3.741974in}}%
\pgfpathlineto{\pgfqpoint{4.683369in}{3.759054in}}%
\pgfpathlineto{\pgfqpoint{4.672942in}{3.775450in}}%
\pgfpathlineto{\pgfqpoint{4.641073in}{3.778423in}}%
\pgfpathlineto{\pgfqpoint{4.609182in}{3.781715in}}%
\pgfpathclose%
\pgfusepath{fill}%
\end{pgfscope}%
\begin{pgfscope}%
\pgfpathrectangle{\pgfqpoint{1.020000in}{0.880000in}}{\pgfqpoint{6.160000in}{6.160000in}}%
\pgfusepath{clip}%
\pgfsetbuttcap%
\pgfsetroundjoin%
\definecolor{currentfill}{rgb}{0.945540,0.798606,0.723105}%
\pgfsetfillcolor{currentfill}%
\pgfsetlinewidth{0.000000pt}%
\definecolor{currentstroke}{rgb}{0.000000,0.000000,0.000000}%
\pgfsetstrokecolor{currentstroke}%
\pgfsetdash{}{0pt}%
\pgfpathmoveto{\pgfqpoint{3.127286in}{4.584049in}}%
\pgfpathlineto{\pgfqpoint{3.136006in}{4.608353in}}%
\pgfpathlineto{\pgfqpoint{3.144736in}{4.633902in}}%
\pgfpathlineto{\pgfqpoint{3.177406in}{4.608891in}}%
\pgfpathlineto{\pgfqpoint{3.210049in}{4.583050in}}%
\pgfpathlineto{\pgfqpoint{3.201218in}{4.560833in}}%
\pgfpathlineto{\pgfqpoint{3.192399in}{4.539728in}}%
\pgfpathlineto{\pgfqpoint{3.159856in}{4.562231in}}%
\pgfpathlineto{\pgfqpoint{3.127286in}{4.584049in}}%
\pgfpathclose%
\pgfusepath{fill}%
\end{pgfscope}%
\begin{pgfscope}%
\pgfpathrectangle{\pgfqpoint{1.020000in}{0.880000in}}{\pgfqpoint{6.160000in}{6.160000in}}%
\pgfusepath{clip}%
\pgfsetbuttcap%
\pgfsetroundjoin%
\definecolor{currentfill}{rgb}{0.483854,0.622050,0.974808}%
\pgfsetfillcolor{currentfill}%
\pgfsetlinewidth{0.000000pt}%
\definecolor{currentstroke}{rgb}{0.000000,0.000000,0.000000}%
\pgfsetstrokecolor{currentstroke}%
\pgfsetdash{}{0pt}%
\pgfpathmoveto{\pgfqpoint{4.969732in}{3.692371in}}%
\pgfpathlineto{\pgfqpoint{4.980401in}{3.673203in}}%
\pgfpathlineto{\pgfqpoint{4.991079in}{3.652736in}}%
\pgfpathlineto{\pgfqpoint{5.022921in}{3.653292in}}%
\pgfpathlineto{\pgfqpoint{5.054745in}{3.654040in}}%
\pgfpathlineto{\pgfqpoint{5.044013in}{3.673469in}}%
\pgfpathlineto{\pgfqpoint{5.033291in}{3.691815in}}%
\pgfpathlineto{\pgfqpoint{5.001521in}{3.692025in}}%
\pgfpathlineto{\pgfqpoint{4.969732in}{3.692371in}}%
\pgfpathclose%
\pgfusepath{fill}%
\end{pgfscope}%
\begin{pgfscope}%
\pgfpathrectangle{\pgfqpoint{1.020000in}{0.880000in}}{\pgfqpoint{6.160000in}{6.160000in}}%
\pgfusepath{clip}%
\pgfsetbuttcap%
\pgfsetroundjoin%
\definecolor{currentfill}{rgb}{0.651398,0.768121,0.995891}%
\pgfsetfillcolor{currentfill}%
\pgfsetlinewidth{0.000000pt}%
\definecolor{currentstroke}{rgb}{0.000000,0.000000,0.000000}%
\pgfsetstrokecolor{currentstroke}%
\pgfsetdash{}{0pt}%
\pgfpathmoveto{\pgfqpoint{4.100461in}{3.976800in}}%
\pgfpathlineto{\pgfqpoint{4.110402in}{3.964384in}}%
\pgfpathlineto{\pgfqpoint{4.120362in}{3.951516in}}%
\pgfpathlineto{\pgfqpoint{4.152487in}{3.937163in}}%
\pgfpathlineto{\pgfqpoint{4.184583in}{3.924254in}}%
\pgfpathlineto{\pgfqpoint{4.174575in}{3.937631in}}%
\pgfpathlineto{\pgfqpoint{4.164584in}{3.950625in}}%
\pgfpathlineto{\pgfqpoint{4.132537in}{3.963029in}}%
\pgfpathlineto{\pgfqpoint{4.100461in}{3.976800in}}%
\pgfpathclose%
\pgfusepath{fill}%
\end{pgfscope}%
\begin{pgfscope}%
\pgfpathrectangle{\pgfqpoint{1.020000in}{0.880000in}}{\pgfqpoint{6.160000in}{6.160000in}}%
\pgfusepath{clip}%
\pgfsetbuttcap%
\pgfsetroundjoin%
\definecolor{currentfill}{rgb}{0.891817,0.851973,0.829085}%
\pgfsetfillcolor{currentfill}%
\pgfsetlinewidth{0.000000pt}%
\definecolor{currentstroke}{rgb}{0.000000,0.000000,0.000000}%
\pgfsetstrokecolor{currentstroke}%
\pgfsetdash{}{0pt}%
\pgfpathmoveto{\pgfqpoint{3.423473in}{4.446473in}}%
\pgfpathlineto{\pgfqpoint{3.432638in}{4.459118in}}%
\pgfpathlineto{\pgfqpoint{3.441824in}{4.471844in}}%
\pgfpathlineto{\pgfqpoint{3.474390in}{4.438275in}}%
\pgfpathlineto{\pgfqpoint{3.506910in}{4.405557in}}%
\pgfpathlineto{\pgfqpoint{3.497639in}{4.396640in}}%
\pgfpathlineto{\pgfqpoint{3.488389in}{4.387757in}}%
\pgfpathlineto{\pgfqpoint{3.455952in}{4.416742in}}%
\pgfpathlineto{\pgfqpoint{3.423473in}{4.446473in}}%
\pgfpathclose%
\pgfusepath{fill}%
\end{pgfscope}%
\begin{pgfscope}%
\pgfpathrectangle{\pgfqpoint{1.020000in}{0.880000in}}{\pgfqpoint{6.160000in}{6.160000in}}%
\pgfusepath{clip}%
\pgfsetbuttcap%
\pgfsetroundjoin%
\definecolor{currentfill}{rgb}{0.703587,0.802586,0.982847}%
\pgfsetfillcolor{currentfill}%
\pgfsetlinewidth{0.000000pt}%
\definecolor{currentstroke}{rgb}{0.000000,0.000000,0.000000}%
\pgfsetstrokecolor{currentstroke}%
\pgfsetdash{}{0pt}%
\pgfpathmoveto{\pgfqpoint{3.952213in}{4.067464in}}%
\pgfpathlineto{\pgfqpoint{3.962017in}{4.057948in}}%
\pgfpathlineto{\pgfqpoint{3.971840in}{4.047872in}}%
\pgfpathlineto{\pgfqpoint{4.004046in}{4.027479in}}%
\pgfpathlineto{\pgfqpoint{4.036217in}{4.008909in}}%
\pgfpathlineto{\pgfqpoint{4.026343in}{4.020058in}}%
\pgfpathlineto{\pgfqpoint{4.016487in}{4.030735in}}%
\pgfpathlineto{\pgfqpoint{3.984367in}{4.048252in}}%
\pgfpathlineto{\pgfqpoint{3.952213in}{4.067464in}}%
\pgfpathclose%
\pgfusepath{fill}%
\end{pgfscope}%
\begin{pgfscope}%
\pgfpathrectangle{\pgfqpoint{1.020000in}{0.880000in}}{\pgfqpoint{6.160000in}{6.160000in}}%
\pgfusepath{clip}%
\pgfsetbuttcap%
\pgfsetroundjoin%
\definecolor{currentfill}{rgb}{0.608547,0.735725,0.999354}%
\pgfsetfillcolor{currentfill}%
\pgfsetlinewidth{0.000000pt}%
\definecolor{currentstroke}{rgb}{0.000000,0.000000,0.000000}%
\pgfsetstrokecolor{currentstroke}%
\pgfsetdash{}{0pt}%
\pgfpathmoveto{\pgfqpoint{4.248693in}{3.902284in}}%
\pgfpathlineto{\pgfqpoint{4.258767in}{3.888187in}}%
\pgfpathlineto{\pgfqpoint{4.268858in}{3.873729in}}%
\pgfpathlineto{\pgfqpoint{4.300923in}{3.864202in}}%
\pgfpathlineto{\pgfqpoint{4.332963in}{3.855688in}}%
\pgfpathlineto{\pgfqpoint{4.322824in}{3.870341in}}%
\pgfpathlineto{\pgfqpoint{4.312702in}{3.884671in}}%
\pgfpathlineto{\pgfqpoint{4.280710in}{3.892989in}}%
\pgfpathlineto{\pgfqpoint{4.248693in}{3.902284in}}%
\pgfpathclose%
\pgfusepath{fill}%
\end{pgfscope}%
\begin{pgfscope}%
\pgfpathrectangle{\pgfqpoint{1.020000in}{0.880000in}}{\pgfqpoint{6.160000in}{6.160000in}}%
\pgfusepath{clip}%
\pgfsetbuttcap%
\pgfsetroundjoin%
\definecolor{currentfill}{rgb}{0.928116,0.822197,0.765141}%
\pgfsetfillcolor{currentfill}%
\pgfsetlinewidth{0.000000pt}%
\definecolor{currentstroke}{rgb}{0.000000,0.000000,0.000000}%
\pgfsetstrokecolor{currentstroke}%
\pgfsetdash{}{0pt}%
\pgfpathmoveto{\pgfqpoint{3.275241in}{4.529725in}}%
\pgfpathlineto{\pgfqpoint{3.284184in}{4.549198in}}%
\pgfpathlineto{\pgfqpoint{3.293144in}{4.569302in}}%
\pgfpathlineto{\pgfqpoint{3.325786in}{4.538410in}}%
\pgfpathlineto{\pgfqpoint{3.358389in}{4.507500in}}%
\pgfpathlineto{\pgfqpoint{3.349333in}{4.491277in}}%
\pgfpathlineto{\pgfqpoint{3.340294in}{4.475579in}}%
\pgfpathlineto{\pgfqpoint{3.307786in}{4.502656in}}%
\pgfpathlineto{\pgfqpoint{3.275241in}{4.529725in}}%
\pgfpathclose%
\pgfusepath{fill}%
\end{pgfscope}%
\begin{pgfscope}%
\pgfpathrectangle{\pgfqpoint{1.020000in}{0.880000in}}{\pgfqpoint{6.160000in}{6.160000in}}%
\pgfusepath{clip}%
\pgfsetbuttcap%
\pgfsetroundjoin%
\definecolor{currentfill}{rgb}{0.510824,0.649397,0.985079}%
\pgfsetfillcolor{currentfill}%
\pgfsetlinewidth{0.000000pt}%
\definecolor{currentstroke}{rgb}{0.000000,0.000000,0.000000}%
\pgfsetstrokecolor{currentstroke}%
\pgfsetdash{}{0pt}%
\pgfpathmoveto{\pgfqpoint{4.757583in}{3.736841in}}%
\pgfpathlineto{\pgfqpoint{4.768084in}{3.718855in}}%
\pgfpathlineto{\pgfqpoint{4.778598in}{3.699783in}}%
\pgfpathlineto{\pgfqpoint{4.810501in}{3.697723in}}%
\pgfpathlineto{\pgfqpoint{4.842385in}{3.696034in}}%
\pgfpathlineto{\pgfqpoint{4.831824in}{3.714990in}}%
\pgfpathlineto{\pgfqpoint{4.821275in}{3.732869in}}%
\pgfpathlineto{\pgfqpoint{4.789439in}{3.734721in}}%
\pgfpathlineto{\pgfqpoint{4.757583in}{3.736841in}}%
\pgfpathclose%
\pgfusepath{fill}%
\end{pgfscope}%
\begin{pgfscope}%
\pgfpathrectangle{\pgfqpoint{1.020000in}{0.880000in}}{\pgfqpoint{6.160000in}{6.160000in}}%
\pgfusepath{clip}%
\pgfsetbuttcap%
\pgfsetroundjoin%
\definecolor{currentfill}{rgb}{0.451739,0.588181,0.960201}%
\pgfsetfillcolor{currentfill}%
\pgfsetlinewidth{0.000000pt}%
\definecolor{currentstroke}{rgb}{0.000000,0.000000,0.000000}%
\pgfsetstrokecolor{currentstroke}%
\pgfsetdash{}{0pt}%
\pgfpathmoveto{\pgfqpoint{5.330578in}{3.626983in}}%
\pgfpathlineto{\pgfqpoint{5.341575in}{3.609985in}}%
\pgfpathlineto{\pgfqpoint{5.352586in}{3.592519in}}%
\pgfpathlineto{\pgfqpoint{5.384342in}{3.594525in}}%
\pgfpathlineto{\pgfqpoint{5.416075in}{3.596122in}}%
\pgfpathlineto{\pgfqpoint{5.405005in}{3.612540in}}%
\pgfpathlineto{\pgfqpoint{5.393950in}{3.628648in}}%
\pgfpathlineto{\pgfqpoint{5.362275in}{3.627933in}}%
\pgfpathlineto{\pgfqpoint{5.330578in}{3.626983in}}%
\pgfpathclose%
\pgfusepath{fill}%
\end{pgfscope}%
\begin{pgfscope}%
\pgfpathrectangle{\pgfqpoint{1.020000in}{0.880000in}}{\pgfqpoint{6.160000in}{6.160000in}}%
\pgfusepath{clip}%
\pgfsetbuttcap%
\pgfsetroundjoin%
\definecolor{currentfill}{rgb}{0.758539,0.832787,0.958408}%
\pgfsetfillcolor{currentfill}%
\pgfsetlinewidth{0.000000pt}%
\definecolor{currentstroke}{rgb}{0.000000,0.000000,0.000000}%
\pgfsetstrokecolor{currentstroke}%
\pgfsetdash{}{0pt}%
\pgfpathmoveto{\pgfqpoint{3.803886in}{4.173605in}}%
\pgfpathlineto{\pgfqpoint{3.813541in}{4.168616in}}%
\pgfpathlineto{\pgfqpoint{3.823217in}{4.163006in}}%
\pgfpathlineto{\pgfqpoint{3.855526in}{4.136173in}}%
\pgfpathlineto{\pgfqpoint{3.887794in}{4.111345in}}%
\pgfpathlineto{\pgfqpoint{3.878063in}{4.118817in}}%
\pgfpathlineto{\pgfqpoint{3.868351in}{4.125757in}}%
\pgfpathlineto{\pgfqpoint{3.836139in}{4.148764in}}%
\pgfpathlineto{\pgfqpoint{3.803886in}{4.173605in}}%
\pgfpathclose%
\pgfusepath{fill}%
\end{pgfscope}%
\begin{pgfscope}%
\pgfpathrectangle{\pgfqpoint{1.020000in}{0.880000in}}{\pgfqpoint{6.160000in}{6.160000in}}%
\pgfusepath{clip}%
\pgfsetbuttcap%
\pgfsetroundjoin%
\definecolor{currentfill}{rgb}{0.435815,0.570707,0.951717}%
\pgfsetfillcolor{currentfill}%
\pgfsetlinewidth{0.000000pt}%
\definecolor{currentstroke}{rgb}{0.000000,0.000000,0.000000}%
\pgfsetstrokecolor{currentstroke}%
\pgfsetdash{}{0pt}%
\pgfpathmoveto{\pgfqpoint{5.542776in}{3.599424in}}%
\pgfpathlineto{\pgfqpoint{5.553974in}{3.584031in}}%
\pgfpathlineto{\pgfqpoint{5.565189in}{3.568496in}}%
\pgfpathlineto{\pgfqpoint{5.596862in}{3.569199in}}%
\pgfpathlineto{\pgfqpoint{5.585620in}{3.584528in}}%
\pgfpathlineto{\pgfqpoint{5.574396in}{3.599734in}}%
\pgfpathlineto{\pgfqpoint{5.542776in}{3.599424in}}%
\pgfpathclose%
\pgfusepath{fill}%
\end{pgfscope}%
\begin{pgfscope}%
\pgfpathrectangle{\pgfqpoint{1.020000in}{0.880000in}}{\pgfqpoint{6.160000in}{6.160000in}}%
\pgfusepath{clip}%
\pgfsetbuttcap%
\pgfsetroundjoin%
\definecolor{currentfill}{rgb}{0.576051,0.708780,0.997755}%
\pgfsetfillcolor{currentfill}%
\pgfsetlinewidth{0.000000pt}%
\definecolor{currentstroke}{rgb}{0.000000,0.000000,0.000000}%
\pgfsetstrokecolor{currentstroke}%
\pgfsetdash{}{0pt}%
\pgfpathmoveto{\pgfqpoint{4.396973in}{3.841279in}}%
\pgfpathlineto{\pgfqpoint{4.407177in}{3.826152in}}%
\pgfpathlineto{\pgfqpoint{4.417398in}{3.810664in}}%
\pgfpathlineto{\pgfqpoint{4.449416in}{3.804462in}}%
\pgfpathlineto{\pgfqpoint{4.481412in}{3.798900in}}%
\pgfpathlineto{\pgfqpoint{4.471144in}{3.814508in}}%
\pgfpathlineto{\pgfqpoint{4.460892in}{3.829742in}}%
\pgfpathlineto{\pgfqpoint{4.428943in}{3.835194in}}%
\pgfpathlineto{\pgfqpoint{4.396973in}{3.841279in}}%
\pgfpathclose%
\pgfusepath{fill}%
\end{pgfscope}%
\begin{pgfscope}%
\pgfpathrectangle{\pgfqpoint{1.020000in}{0.880000in}}{\pgfqpoint{6.160000in}{6.160000in}}%
\pgfusepath{clip}%
\pgfsetbuttcap%
\pgfsetroundjoin%
\definecolor{currentfill}{rgb}{0.467678,0.605591,0.968546}%
\pgfsetfillcolor{currentfill}%
\pgfsetlinewidth{0.000000pt}%
\definecolor{currentstroke}{rgb}{0.000000,0.000000,0.000000}%
\pgfsetstrokecolor{currentstroke}%
\pgfsetdash{}{0pt}%
\pgfpathmoveto{\pgfqpoint{5.118336in}{3.655809in}}%
\pgfpathlineto{\pgfqpoint{5.129133in}{3.636477in}}%
\pgfpathlineto{\pgfqpoint{5.139938in}{3.616017in}}%
\pgfpathlineto{\pgfqpoint{5.171763in}{3.618367in}}%
\pgfpathlineto{\pgfqpoint{5.203569in}{3.620568in}}%
\pgfpathlineto{\pgfqpoint{5.192704in}{3.639468in}}%
\pgfpathlineto{\pgfqpoint{5.181849in}{3.657532in}}%
\pgfpathlineto{\pgfqpoint{5.150103in}{3.656702in}}%
\pgfpathlineto{\pgfqpoint{5.118336in}{3.655809in}}%
\pgfpathclose%
\pgfusepath{fill}%
\end{pgfscope}%
\begin{pgfscope}%
\pgfpathrectangle{\pgfqpoint{1.020000in}{0.880000in}}{\pgfqpoint{6.160000in}{6.160000in}}%
\pgfusepath{clip}%
\pgfsetbuttcap%
\pgfsetroundjoin%
\definecolor{currentfill}{rgb}{0.489246,0.627536,0.976896}%
\pgfsetfillcolor{currentfill}%
\pgfsetlinewidth{0.000000pt}%
\definecolor{currentstroke}{rgb}{0.000000,0.000000,0.000000}%
\pgfsetstrokecolor{currentstroke}%
\pgfsetdash{}{0pt}%
\pgfpathmoveto{\pgfqpoint{4.906096in}{3.693660in}}%
\pgfpathlineto{\pgfqpoint{4.916715in}{3.673818in}}%
\pgfpathlineto{\pgfqpoint{4.927341in}{3.652506in}}%
\pgfpathlineto{\pgfqpoint{4.959219in}{3.652448in}}%
\pgfpathlineto{\pgfqpoint{4.991079in}{3.652736in}}%
\pgfpathlineto{\pgfqpoint{4.980401in}{3.673203in}}%
\pgfpathlineto{\pgfqpoint{4.969732in}{3.692371in}}%
\pgfpathlineto{\pgfqpoint{4.937923in}{3.692901in}}%
\pgfpathlineto{\pgfqpoint{4.906096in}{3.693660in}}%
\pgfpathclose%
\pgfusepath{fill}%
\end{pgfscope}%
\begin{pgfscope}%
\pgfpathrectangle{\pgfqpoint{1.020000in}{0.880000in}}{\pgfqpoint{6.160000in}{6.160000in}}%
\pgfusepath{clip}%
\pgfsetbuttcap%
\pgfsetroundjoin%
\definecolor{currentfill}{rgb}{0.548876,0.685104,0.994379}%
\pgfsetfillcolor{currentfill}%
\pgfsetlinewidth{0.000000pt}%
\definecolor{currentstroke}{rgb}{0.000000,0.000000,0.000000}%
\pgfsetstrokecolor{currentstroke}%
\pgfsetdash{}{0pt}%
\pgfpathmoveto{\pgfqpoint{4.545340in}{3.789411in}}%
\pgfpathlineto{\pgfqpoint{4.555672in}{3.773221in}}%
\pgfpathlineto{\pgfqpoint{4.566021in}{3.756466in}}%
\pgfpathlineto{\pgfqpoint{4.597999in}{3.752247in}}%
\pgfpathlineto{\pgfqpoint{4.629957in}{3.748447in}}%
\pgfpathlineto{\pgfqpoint{4.619562in}{3.765397in}}%
\pgfpathlineto{\pgfqpoint{4.609182in}{3.781715in}}%
\pgfpathlineto{\pgfqpoint{4.577272in}{3.785364in}}%
\pgfpathlineto{\pgfqpoint{4.545340in}{3.789411in}}%
\pgfpathclose%
\pgfusepath{fill}%
\end{pgfscope}%
\begin{pgfscope}%
\pgfpathrectangle{\pgfqpoint{1.020000in}{0.880000in}}{\pgfqpoint{6.160000in}{6.160000in}}%
\pgfusepath{clip}%
\pgfsetbuttcap%
\pgfsetroundjoin%
\definecolor{currentfill}{rgb}{0.822420,0.856898,0.910795}%
\pgfsetfillcolor{currentfill}%
\pgfsetlinewidth{0.000000pt}%
\definecolor{currentstroke}{rgb}{0.000000,0.000000,0.000000}%
\pgfsetstrokecolor{currentstroke}%
\pgfsetdash{}{0pt}%
\pgfpathmoveto{\pgfqpoint{3.655444in}{4.289777in}}%
\pgfpathlineto{\pgfqpoint{3.664930in}{4.291031in}}%
\pgfpathlineto{\pgfqpoint{3.674438in}{4.291725in}}%
\pgfpathlineto{\pgfqpoint{3.706869in}{4.259393in}}%
\pgfpathlineto{\pgfqpoint{3.739253in}{4.228914in}}%
\pgfpathlineto{\pgfqpoint{3.729680in}{4.230983in}}%
\pgfpathlineto{\pgfqpoint{3.720128in}{4.232556in}}%
\pgfpathlineto{\pgfqpoint{3.687808in}{4.260334in}}%
\pgfpathlineto{\pgfqpoint{3.655444in}{4.289777in}}%
\pgfpathclose%
\pgfusepath{fill}%
\end{pgfscope}%
\begin{pgfscope}%
\pgfpathrectangle{\pgfqpoint{1.020000in}{0.880000in}}{\pgfqpoint{6.160000in}{6.160000in}}%
\pgfusepath{clip}%
\pgfsetbuttcap%
\pgfsetroundjoin%
\definecolor{currentfill}{rgb}{0.435815,0.570707,0.951717}%
\pgfsetfillcolor{currentfill}%
\pgfsetlinewidth{0.000000pt}%
\definecolor{currentstroke}{rgb}{0.000000,0.000000,0.000000}%
\pgfsetstrokecolor{currentstroke}%
\pgfsetdash{}{0pt}%
\pgfpathmoveto{\pgfqpoint{5.479471in}{3.598284in}}%
\pgfpathlineto{\pgfqpoint{5.490614in}{3.582366in}}%
\pgfpathlineto{\pgfqpoint{5.501775in}{3.566241in}}%
\pgfpathlineto{\pgfqpoint{5.533494in}{3.567534in}}%
\pgfpathlineto{\pgfqpoint{5.565189in}{3.568496in}}%
\pgfpathlineto{\pgfqpoint{5.553974in}{3.584031in}}%
\pgfpathlineto{\pgfqpoint{5.542776in}{3.599424in}}%
\pgfpathlineto{\pgfqpoint{5.511134in}{3.598956in}}%
\pgfpathlineto{\pgfqpoint{5.479471in}{3.598284in}}%
\pgfpathclose%
\pgfusepath{fill}%
\end{pgfscope}%
\begin{pgfscope}%
\pgfpathrectangle{\pgfqpoint{1.020000in}{0.880000in}}{\pgfqpoint{6.160000in}{6.160000in}}%
\pgfusepath{clip}%
\pgfsetbuttcap%
\pgfsetroundjoin%
\definecolor{currentfill}{rgb}{0.451739,0.588181,0.960201}%
\pgfsetfillcolor{currentfill}%
\pgfsetlinewidth{0.000000pt}%
\definecolor{currentstroke}{rgb}{0.000000,0.000000,0.000000}%
\pgfsetstrokecolor{currentstroke}%
\pgfsetdash{}{0pt}%
\pgfpathmoveto{\pgfqpoint{5.267117in}{3.624302in}}%
\pgfpathlineto{\pgfqpoint{5.278054in}{3.606070in}}%
\pgfpathlineto{\pgfqpoint{5.289003in}{3.587143in}}%
\pgfpathlineto{\pgfqpoint{5.320806in}{3.590065in}}%
\pgfpathlineto{\pgfqpoint{5.352586in}{3.592519in}}%
\pgfpathlineto{\pgfqpoint{5.341575in}{3.609985in}}%
\pgfpathlineto{\pgfqpoint{5.330578in}{3.626983in}}%
\pgfpathlineto{\pgfqpoint{5.298858in}{3.625777in}}%
\pgfpathlineto{\pgfqpoint{5.267117in}{3.624302in}}%
\pgfpathclose%
\pgfusepath{fill}%
\end{pgfscope}%
\begin{pgfscope}%
\pgfpathrectangle{\pgfqpoint{1.020000in}{0.880000in}}{\pgfqpoint{6.160000in}{6.160000in}}%
\pgfusepath{clip}%
\pgfsetbuttcap%
\pgfsetroundjoin%
\definecolor{currentfill}{rgb}{0.521696,0.659599,0.987736}%
\pgfsetfillcolor{currentfill}%
\pgfsetlinewidth{0.000000pt}%
\definecolor{currentstroke}{rgb}{0.000000,0.000000,0.000000}%
\pgfsetstrokecolor{currentstroke}%
\pgfsetdash{}{0pt}%
\pgfpathmoveto{\pgfqpoint{4.693810in}{3.741974in}}%
\pgfpathlineto{\pgfqpoint{4.704266in}{3.724051in}}%
\pgfpathlineto{\pgfqpoint{4.714734in}{3.705095in}}%
\pgfpathlineto{\pgfqpoint{4.746676in}{3.702236in}}%
\pgfpathlineto{\pgfqpoint{4.778598in}{3.699783in}}%
\pgfpathlineto{\pgfqpoint{4.768084in}{3.718855in}}%
\pgfpathlineto{\pgfqpoint{4.757583in}{3.736841in}}%
\pgfpathlineto{\pgfqpoint{4.725707in}{3.739252in}}%
\pgfpathlineto{\pgfqpoint{4.693810in}{3.741974in}}%
\pgfpathclose%
\pgfusepath{fill}%
\end{pgfscope}%
\begin{pgfscope}%
\pgfpathrectangle{\pgfqpoint{1.020000in}{0.880000in}}{\pgfqpoint{6.160000in}{6.160000in}}%
\pgfusepath{clip}%
\pgfsetbuttcap%
\pgfsetroundjoin%
\definecolor{currentfill}{rgb}{0.672538,0.782861,0.991982}%
\pgfsetfillcolor{currentfill}%
\pgfsetlinewidth{0.000000pt}%
\definecolor{currentstroke}{rgb}{0.000000,0.000000,0.000000}%
\pgfsetstrokecolor{currentstroke}%
\pgfsetdash{}{0pt}%
\pgfpathmoveto{\pgfqpoint{4.036217in}{4.008909in}}%
\pgfpathlineto{\pgfqpoint{4.046109in}{3.997253in}}%
\pgfpathlineto{\pgfqpoint{4.056020in}{3.985055in}}%
\pgfpathlineto{\pgfqpoint{4.088207in}{3.967439in}}%
\pgfpathlineto{\pgfqpoint{4.120362in}{3.951516in}}%
\pgfpathlineto{\pgfqpoint{4.110402in}{3.964384in}}%
\pgfpathlineto{\pgfqpoint{4.100461in}{3.976800in}}%
\pgfpathlineto{\pgfqpoint{4.068354in}{3.992054in}}%
\pgfpathlineto{\pgfqpoint{4.036217in}{4.008909in}}%
\pgfpathclose%
\pgfusepath{fill}%
\end{pgfscope}%
\begin{pgfscope}%
\pgfpathrectangle{\pgfqpoint{1.020000in}{0.880000in}}{\pgfqpoint{6.160000in}{6.160000in}}%
\pgfusepath{clip}%
\pgfsetbuttcap%
\pgfsetroundjoin%
\definecolor{currentfill}{rgb}{0.624703,0.748318,0.998719}%
\pgfsetfillcolor{currentfill}%
\pgfsetlinewidth{0.000000pt}%
\definecolor{currentstroke}{rgb}{0.000000,0.000000,0.000000}%
\pgfsetstrokecolor{currentstroke}%
\pgfsetdash{}{0pt}%
\pgfpathmoveto{\pgfqpoint{4.184583in}{3.924254in}}%
\pgfpathlineto{\pgfqpoint{4.194608in}{3.910478in}}%
\pgfpathlineto{\pgfqpoint{4.204652in}{3.896287in}}%
\pgfpathlineto{\pgfqpoint{4.236768in}{3.884384in}}%
\pgfpathlineto{\pgfqpoint{4.268858in}{3.873729in}}%
\pgfpathlineto{\pgfqpoint{4.258767in}{3.888187in}}%
\pgfpathlineto{\pgfqpoint{4.248693in}{3.902284in}}%
\pgfpathlineto{\pgfqpoint{4.216651in}{3.912668in}}%
\pgfpathlineto{\pgfqpoint{4.184583in}{3.924254in}}%
\pgfpathclose%
\pgfusepath{fill}%
\end{pgfscope}%
\begin{pgfscope}%
\pgfpathrectangle{\pgfqpoint{1.020000in}{0.880000in}}{\pgfqpoint{6.160000in}{6.160000in}}%
\pgfusepath{clip}%
\pgfsetbuttcap%
\pgfsetroundjoin%
\definecolor{currentfill}{rgb}{0.875557,0.860242,0.851430}%
\pgfsetfillcolor{currentfill}%
\pgfsetlinewidth{0.000000pt}%
\definecolor{currentstroke}{rgb}{0.000000,0.000000,0.000000}%
\pgfsetstrokecolor{currentstroke}%
\pgfsetdash{}{0pt}%
\pgfpathmoveto{\pgfqpoint{3.506910in}{4.405557in}}%
\pgfpathlineto{\pgfqpoint{3.516202in}{4.414339in}}%
\pgfpathlineto{\pgfqpoint{3.525516in}{4.422806in}}%
\pgfpathlineto{\pgfqpoint{3.558070in}{4.387467in}}%
\pgfpathlineto{\pgfqpoint{3.590576in}{4.353433in}}%
\pgfpathlineto{\pgfqpoint{3.581184in}{4.348524in}}%
\pgfpathlineto{\pgfqpoint{3.571813in}{4.343312in}}%
\pgfpathlineto{\pgfqpoint{3.539385in}{4.373856in}}%
\pgfpathlineto{\pgfqpoint{3.506910in}{4.405557in}}%
\pgfpathclose%
\pgfusepath{fill}%
\end{pgfscope}%
\begin{pgfscope}%
\pgfpathrectangle{\pgfqpoint{1.020000in}{0.880000in}}{\pgfqpoint{6.160000in}{6.160000in}}%
\pgfusepath{clip}%
\pgfsetbuttcap%
\pgfsetroundjoin%
\definecolor{currentfill}{rgb}{0.467678,0.605591,0.968546}%
\pgfsetfillcolor{currentfill}%
\pgfsetlinewidth{0.000000pt}%
\definecolor{currentstroke}{rgb}{0.000000,0.000000,0.000000}%
\pgfsetstrokecolor{currentstroke}%
\pgfsetdash{}{0pt}%
\pgfpathmoveto{\pgfqpoint{5.054745in}{3.654040in}}%
\pgfpathlineto{\pgfqpoint{5.065484in}{3.633347in}}%
\pgfpathlineto{\pgfqpoint{5.076229in}{3.611213in}}%
\pgfpathlineto{\pgfqpoint{5.108093in}{3.613598in}}%
\pgfpathlineto{\pgfqpoint{5.139938in}{3.616017in}}%
\pgfpathlineto{\pgfqpoint{5.129133in}{3.636477in}}%
\pgfpathlineto{\pgfqpoint{5.118336in}{3.655809in}}%
\pgfpathlineto{\pgfqpoint{5.086550in}{3.654902in}}%
\pgfpathlineto{\pgfqpoint{5.054745in}{3.654040in}}%
\pgfpathclose%
\pgfusepath{fill}%
\end{pgfscope}%
\begin{pgfscope}%
\pgfpathrectangle{\pgfqpoint{1.020000in}{0.880000in}}{\pgfqpoint{6.160000in}{6.160000in}}%
\pgfusepath{clip}%
\pgfsetbuttcap%
\pgfsetroundjoin%
\definecolor{currentfill}{rgb}{0.956371,0.775144,0.686416}%
\pgfsetfillcolor{currentfill}%
\pgfsetlinewidth{0.000000pt}%
\definecolor{currentstroke}{rgb}{0.000000,0.000000,0.000000}%
\pgfsetstrokecolor{currentstroke}%
\pgfsetdash{}{0pt}%
\pgfpathmoveto{\pgfqpoint{3.062074in}{4.624930in}}%
\pgfpathlineto{\pgfqpoint{3.070694in}{4.652081in}}%
\pgfpathlineto{\pgfqpoint{3.079323in}{4.680595in}}%
\pgfpathlineto{\pgfqpoint{3.112040in}{4.657872in}}%
\pgfpathlineto{\pgfqpoint{3.144736in}{4.633902in}}%
\pgfpathlineto{\pgfqpoint{3.136006in}{4.608353in}}%
\pgfpathlineto{\pgfqpoint{3.127286in}{4.584049in}}%
\pgfpathlineto{\pgfqpoint{3.094691in}{4.605006in}}%
\pgfpathlineto{\pgfqpoint{3.062074in}{4.624930in}}%
\pgfpathclose%
\pgfusepath{fill}%
\end{pgfscope}%
\begin{pgfscope}%
\pgfpathrectangle{\pgfqpoint{1.020000in}{0.880000in}}{\pgfqpoint{6.160000in}{6.160000in}}%
\pgfusepath{clip}%
\pgfsetbuttcap%
\pgfsetroundjoin%
\definecolor{currentfill}{rgb}{0.586921,0.718121,0.998874}%
\pgfsetfillcolor{currentfill}%
\pgfsetlinewidth{0.000000pt}%
\definecolor{currentstroke}{rgb}{0.000000,0.000000,0.000000}%
\pgfsetstrokecolor{currentstroke}%
\pgfsetdash{}{0pt}%
\pgfpathmoveto{\pgfqpoint{4.332963in}{3.855688in}}%
\pgfpathlineto{\pgfqpoint{4.343119in}{3.840692in}}%
\pgfpathlineto{\pgfqpoint{4.353293in}{3.825332in}}%
\pgfpathlineto{\pgfqpoint{4.385357in}{3.817590in}}%
\pgfpathlineto{\pgfqpoint{4.417398in}{3.810664in}}%
\pgfpathlineto{\pgfqpoint{4.407177in}{3.826152in}}%
\pgfpathlineto{\pgfqpoint{4.396973in}{3.841279in}}%
\pgfpathlineto{\pgfqpoint{4.364979in}{3.848080in}}%
\pgfpathlineto{\pgfqpoint{4.332963in}{3.855688in}}%
\pgfpathclose%
\pgfusepath{fill}%
\end{pgfscope}%
\begin{pgfscope}%
\pgfpathrectangle{\pgfqpoint{1.020000in}{0.880000in}}{\pgfqpoint{6.160000in}{6.160000in}}%
\pgfusepath{clip}%
\pgfsetbuttcap%
\pgfsetroundjoin%
\definecolor{currentfill}{rgb}{0.728970,0.817464,0.973188}%
\pgfsetfillcolor{currentfill}%
\pgfsetlinewidth{0.000000pt}%
\definecolor{currentstroke}{rgb}{0.000000,0.000000,0.000000}%
\pgfsetstrokecolor{currentstroke}%
\pgfsetdash{}{0pt}%
\pgfpathmoveto{\pgfqpoint{3.887794in}{4.111345in}}%
\pgfpathlineto{\pgfqpoint{3.897546in}{4.103273in}}%
\pgfpathlineto{\pgfqpoint{3.907317in}{4.094533in}}%
\pgfpathlineto{\pgfqpoint{3.939598in}{4.070193in}}%
\pgfpathlineto{\pgfqpoint{3.971840in}{4.047872in}}%
\pgfpathlineto{\pgfqpoint{3.962017in}{4.057948in}}%
\pgfpathlineto{\pgfqpoint{3.952213in}{4.067464in}}%
\pgfpathlineto{\pgfqpoint{3.920023in}{4.088467in}}%
\pgfpathlineto{\pgfqpoint{3.887794in}{4.111345in}}%
\pgfpathclose%
\pgfusepath{fill}%
\end{pgfscope}%
\begin{pgfscope}%
\pgfpathrectangle{\pgfqpoint{1.020000in}{0.880000in}}{\pgfqpoint{6.160000in}{6.160000in}}%
\pgfusepath{clip}%
\pgfsetbuttcap%
\pgfsetroundjoin%
\definecolor{currentfill}{rgb}{0.922681,0.828568,0.777054}%
\pgfsetfillcolor{currentfill}%
\pgfsetlinewidth{0.000000pt}%
\definecolor{currentstroke}{rgb}{0.000000,0.000000,0.000000}%
\pgfsetstrokecolor{currentstroke}%
\pgfsetdash{}{0pt}%
\pgfpathmoveto{\pgfqpoint{3.358389in}{4.507500in}}%
\pgfpathlineto{\pgfqpoint{3.367463in}{4.524064in}}%
\pgfpathlineto{\pgfqpoint{3.376558in}{4.540760in}}%
\pgfpathlineto{\pgfqpoint{3.409213in}{4.506077in}}%
\pgfpathlineto{\pgfqpoint{3.441824in}{4.471844in}}%
\pgfpathlineto{\pgfqpoint{3.432638in}{4.459118in}}%
\pgfpathlineto{\pgfqpoint{3.423473in}{4.446473in}}%
\pgfpathlineto{\pgfqpoint{3.390951in}{4.476787in}}%
\pgfpathlineto{\pgfqpoint{3.358389in}{4.507500in}}%
\pgfpathclose%
\pgfusepath{fill}%
\end{pgfscope}%
\begin{pgfscope}%
\pgfpathrectangle{\pgfqpoint{1.020000in}{0.880000in}}{\pgfqpoint{6.160000in}{6.160000in}}%
\pgfusepath{clip}%
\pgfsetbuttcap%
\pgfsetroundjoin%
\definecolor{currentfill}{rgb}{0.494638,0.633022,0.978983}%
\pgfsetfillcolor{currentfill}%
\pgfsetlinewidth{0.000000pt}%
\definecolor{currentstroke}{rgb}{0.000000,0.000000,0.000000}%
\pgfsetstrokecolor{currentstroke}%
\pgfsetdash{}{0pt}%
\pgfpathmoveto{\pgfqpoint{4.842385in}{3.696034in}}%
\pgfpathlineto{\pgfqpoint{4.852955in}{3.675767in}}%
\pgfpathlineto{\pgfqpoint{4.863533in}{3.653931in}}%
\pgfpathlineto{\pgfqpoint{4.895446in}{3.652981in}}%
\pgfpathlineto{\pgfqpoint{4.927341in}{3.652506in}}%
\pgfpathlineto{\pgfqpoint{4.916715in}{3.673818in}}%
\pgfpathlineto{\pgfqpoint{4.906096in}{3.693660in}}%
\pgfpathlineto{\pgfqpoint{4.874250in}{3.694691in}}%
\pgfpathlineto{\pgfqpoint{4.842385in}{3.696034in}}%
\pgfpathclose%
\pgfusepath{fill}%
\end{pgfscope}%
\begin{pgfscope}%
\pgfpathrectangle{\pgfqpoint{1.020000in}{0.880000in}}{\pgfqpoint{6.160000in}{6.160000in}}%
\pgfusepath{clip}%
\pgfsetbuttcap%
\pgfsetroundjoin%
\definecolor{currentfill}{rgb}{0.945540,0.798606,0.723105}%
\pgfsetfillcolor{currentfill}%
\pgfsetlinewidth{0.000000pt}%
\definecolor{currentstroke}{rgb}{0.000000,0.000000,0.000000}%
\pgfsetstrokecolor{currentstroke}%
\pgfsetdash{}{0pt}%
\pgfpathmoveto{\pgfqpoint{3.210049in}{4.583050in}}%
\pgfpathlineto{\pgfqpoint{3.218893in}{4.606206in}}%
\pgfpathlineto{\pgfqpoint{3.227753in}{4.630095in}}%
\pgfpathlineto{\pgfqpoint{3.260465in}{4.599943in}}%
\pgfpathlineto{\pgfqpoint{3.293144in}{4.569302in}}%
\pgfpathlineto{\pgfqpoint{3.284184in}{4.549198in}}%
\pgfpathlineto{\pgfqpoint{3.275241in}{4.529725in}}%
\pgfpathlineto{\pgfqpoint{3.242661in}{4.556592in}}%
\pgfpathlineto{\pgfqpoint{3.210049in}{4.583050in}}%
\pgfpathclose%
\pgfusepath{fill}%
\end{pgfscope}%
\begin{pgfscope}%
\pgfpathrectangle{\pgfqpoint{1.020000in}{0.880000in}}{\pgfqpoint{6.160000in}{6.160000in}}%
\pgfusepath{clip}%
\pgfsetbuttcap%
\pgfsetroundjoin%
\definecolor{currentfill}{rgb}{0.554312,0.690097,0.995516}%
\pgfsetfillcolor{currentfill}%
\pgfsetlinewidth{0.000000pt}%
\definecolor{currentstroke}{rgb}{0.000000,0.000000,0.000000}%
\pgfsetstrokecolor{currentstroke}%
\pgfsetdash{}{0pt}%
\pgfpathmoveto{\pgfqpoint{4.481412in}{3.798900in}}%
\pgfpathlineto{\pgfqpoint{4.491698in}{3.782861in}}%
\pgfpathlineto{\pgfqpoint{4.501999in}{3.766323in}}%
\pgfpathlineto{\pgfqpoint{4.534021in}{3.761142in}}%
\pgfpathlineto{\pgfqpoint{4.566021in}{3.756466in}}%
\pgfpathlineto{\pgfqpoint{4.555672in}{3.773221in}}%
\pgfpathlineto{\pgfqpoint{4.545340in}{3.789411in}}%
\pgfpathlineto{\pgfqpoint{4.513387in}{3.793904in}}%
\pgfpathlineto{\pgfqpoint{4.481412in}{3.798900in}}%
\pgfpathclose%
\pgfusepath{fill}%
\end{pgfscope}%
\begin{pgfscope}%
\pgfpathrectangle{\pgfqpoint{1.020000in}{0.880000in}}{\pgfqpoint{6.160000in}{6.160000in}}%
\pgfusepath{clip}%
\pgfsetbuttcap%
\pgfsetroundjoin%
\definecolor{currentfill}{rgb}{0.441123,0.576532,0.954545}%
\pgfsetfillcolor{currentfill}%
\pgfsetlinewidth{0.000000pt}%
\definecolor{currentstroke}{rgb}{0.000000,0.000000,0.000000}%
\pgfsetstrokecolor{currentstroke}%
\pgfsetdash{}{0pt}%
\pgfpathmoveto{\pgfqpoint{5.416075in}{3.596122in}}%
\pgfpathlineto{\pgfqpoint{5.427161in}{3.579387in}}%
\pgfpathlineto{\pgfqpoint{5.438263in}{3.562339in}}%
\pgfpathlineto{\pgfqpoint{5.470031in}{3.564537in}}%
\pgfpathlineto{\pgfqpoint{5.501775in}{3.566241in}}%
\pgfpathlineto{\pgfqpoint{5.490614in}{3.582366in}}%
\pgfpathlineto{\pgfqpoint{5.479471in}{3.598284in}}%
\pgfpathlineto{\pgfqpoint{5.447784in}{3.597358in}}%
\pgfpathlineto{\pgfqpoint{5.416075in}{3.596122in}}%
\pgfpathclose%
\pgfusepath{fill}%
\end{pgfscope}%
\begin{pgfscope}%
\pgfpathrectangle{\pgfqpoint{1.020000in}{0.880000in}}{\pgfqpoint{6.160000in}{6.160000in}}%
\pgfusepath{clip}%
\pgfsetbuttcap%
\pgfsetroundjoin%
\definecolor{currentfill}{rgb}{0.796064,0.848693,0.933471}%
\pgfsetfillcolor{currentfill}%
\pgfsetlinewidth{0.000000pt}%
\definecolor{currentstroke}{rgb}{0.000000,0.000000,0.000000}%
\pgfsetstrokecolor{currentstroke}%
\pgfsetdash{}{0pt}%
\pgfpathmoveto{\pgfqpoint{3.739253in}{4.228914in}}%
\pgfpathlineto{\pgfqpoint{3.748848in}{4.226234in}}%
\pgfpathlineto{\pgfqpoint{3.758464in}{4.222830in}}%
\pgfpathlineto{\pgfqpoint{3.790863in}{4.191886in}}%
\pgfpathlineto{\pgfqpoint{3.823217in}{4.163006in}}%
\pgfpathlineto{\pgfqpoint{3.813541in}{4.168616in}}%
\pgfpathlineto{\pgfqpoint{3.803886in}{4.173605in}}%
\pgfpathlineto{\pgfqpoint{3.771592in}{4.200316in}}%
\pgfpathlineto{\pgfqpoint{3.739253in}{4.228914in}}%
\pgfpathclose%
\pgfusepath{fill}%
\end{pgfscope}%
\begin{pgfscope}%
\pgfpathrectangle{\pgfqpoint{1.020000in}{0.880000in}}{\pgfqpoint{6.160000in}{6.160000in}}%
\pgfusepath{clip}%
\pgfsetbuttcap%
\pgfsetroundjoin%
\definecolor{currentfill}{rgb}{0.451739,0.588181,0.960201}%
\pgfsetfillcolor{currentfill}%
\pgfsetlinewidth{0.000000pt}%
\definecolor{currentstroke}{rgb}{0.000000,0.000000,0.000000}%
\pgfsetstrokecolor{currentstroke}%
\pgfsetdash{}{0pt}%
\pgfpathmoveto{\pgfqpoint{5.203569in}{3.620568in}}%
\pgfpathlineto{\pgfqpoint{5.214444in}{3.600745in}}%
\pgfpathlineto{\pgfqpoint{5.225327in}{3.579929in}}%
\pgfpathlineto{\pgfqpoint{5.257176in}{3.583754in}}%
\pgfpathlineto{\pgfqpoint{5.289003in}{3.587143in}}%
\pgfpathlineto{\pgfqpoint{5.278054in}{3.606070in}}%
\pgfpathlineto{\pgfqpoint{5.267117in}{3.624302in}}%
\pgfpathlineto{\pgfqpoint{5.235354in}{3.622559in}}%
\pgfpathlineto{\pgfqpoint{5.203569in}{3.620568in}}%
\pgfpathclose%
\pgfusepath{fill}%
\end{pgfscope}%
\begin{pgfscope}%
\pgfpathrectangle{\pgfqpoint{1.020000in}{0.880000in}}{\pgfqpoint{6.160000in}{6.160000in}}%
\pgfusepath{clip}%
\pgfsetbuttcap%
\pgfsetroundjoin%
\definecolor{currentfill}{rgb}{0.527132,0.664700,0.989065}%
\pgfsetfillcolor{currentfill}%
\pgfsetlinewidth{0.000000pt}%
\definecolor{currentstroke}{rgb}{0.000000,0.000000,0.000000}%
\pgfsetstrokecolor{currentstroke}%
\pgfsetdash{}{0pt}%
\pgfpathmoveto{\pgfqpoint{4.629957in}{3.748447in}}%
\pgfpathlineto{\pgfqpoint{4.640366in}{3.730726in}}%
\pgfpathlineto{\pgfqpoint{4.650790in}{3.712071in}}%
\pgfpathlineto{\pgfqpoint{4.682772in}{3.708371in}}%
\pgfpathlineto{\pgfqpoint{4.714734in}{3.705095in}}%
\pgfpathlineto{\pgfqpoint{4.704266in}{3.724051in}}%
\pgfpathlineto{\pgfqpoint{4.693810in}{3.741974in}}%
\pgfpathlineto{\pgfqpoint{4.661894in}{3.745031in}}%
\pgfpathlineto{\pgfqpoint{4.629957in}{3.748447in}}%
\pgfpathclose%
\pgfusepath{fill}%
\end{pgfscope}%
\begin{pgfscope}%
\pgfpathrectangle{\pgfqpoint{1.020000in}{0.880000in}}{\pgfqpoint{6.160000in}{6.160000in}}%
\pgfusepath{clip}%
\pgfsetbuttcap%
\pgfsetroundjoin%
\definecolor{currentfill}{rgb}{0.473070,0.611077,0.970634}%
\pgfsetfillcolor{currentfill}%
\pgfsetlinewidth{0.000000pt}%
\definecolor{currentstroke}{rgb}{0.000000,0.000000,0.000000}%
\pgfsetstrokecolor{currentstroke}%
\pgfsetdash{}{0pt}%
\pgfpathmoveto{\pgfqpoint{4.991079in}{3.652736in}}%
\pgfpathlineto{\pgfqpoint{5.001763in}{3.630748in}}%
\pgfpathlineto{\pgfqpoint{5.012451in}{3.607022in}}%
\pgfpathlineto{\pgfqpoint{5.044348in}{3.608979in}}%
\pgfpathlineto{\pgfqpoint{5.076229in}{3.611213in}}%
\pgfpathlineto{\pgfqpoint{5.065484in}{3.633347in}}%
\pgfpathlineto{\pgfqpoint{5.054745in}{3.654040in}}%
\pgfpathlineto{\pgfqpoint{5.022921in}{3.653292in}}%
\pgfpathlineto{\pgfqpoint{4.991079in}{3.652736in}}%
\pgfpathclose%
\pgfusepath{fill}%
\end{pgfscope}%
\begin{pgfscope}%
\pgfpathrectangle{\pgfqpoint{1.020000in}{0.880000in}}{\pgfqpoint{6.160000in}{6.160000in}}%
\pgfusepath{clip}%
\pgfsetbuttcap%
\pgfsetroundjoin%
\definecolor{currentfill}{rgb}{0.646113,0.764436,0.996868}%
\pgfsetfillcolor{currentfill}%
\pgfsetlinewidth{0.000000pt}%
\definecolor{currentstroke}{rgb}{0.000000,0.000000,0.000000}%
\pgfsetstrokecolor{currentstroke}%
\pgfsetdash{}{0pt}%
\pgfpathmoveto{\pgfqpoint{4.120362in}{3.951516in}}%
\pgfpathlineto{\pgfqpoint{4.130340in}{3.938175in}}%
\pgfpathlineto{\pgfqpoint{4.140335in}{3.924340in}}%
\pgfpathlineto{\pgfqpoint{4.172508in}{3.909563in}}%
\pgfpathlineto{\pgfqpoint{4.204652in}{3.896287in}}%
\pgfpathlineto{\pgfqpoint{4.194608in}{3.910478in}}%
\pgfpathlineto{\pgfqpoint{4.184583in}{3.924254in}}%
\pgfpathlineto{\pgfqpoint{4.152487in}{3.937163in}}%
\pgfpathlineto{\pgfqpoint{4.120362in}{3.951516in}}%
\pgfpathclose%
\pgfusepath{fill}%
\end{pgfscope}%
\begin{pgfscope}%
\pgfpathrectangle{\pgfqpoint{1.020000in}{0.880000in}}{\pgfqpoint{6.160000in}{6.160000in}}%
\pgfusepath{clip}%
\pgfsetbuttcap%
\pgfsetroundjoin%
\definecolor{currentfill}{rgb}{0.855378,0.863778,0.876587}%
\pgfsetfillcolor{currentfill}%
\pgfsetlinewidth{0.000000pt}%
\definecolor{currentstroke}{rgb}{0.000000,0.000000,0.000000}%
\pgfsetstrokecolor{currentstroke}%
\pgfsetdash{}{0pt}%
\pgfpathmoveto{\pgfqpoint{3.590576in}{4.353433in}}%
\pgfpathlineto{\pgfqpoint{3.599990in}{4.357878in}}%
\pgfpathlineto{\pgfqpoint{3.609428in}{4.361690in}}%
\pgfpathlineto{\pgfqpoint{3.641958in}{4.325852in}}%
\pgfpathlineto{\pgfqpoint{3.674438in}{4.291725in}}%
\pgfpathlineto{\pgfqpoint{3.664930in}{4.291031in}}%
\pgfpathlineto{\pgfqpoint{3.655444in}{4.289777in}}%
\pgfpathlineto{\pgfqpoint{3.623034in}{4.320835in}}%
\pgfpathlineto{\pgfqpoint{3.590576in}{4.353433in}}%
\pgfpathclose%
\pgfusepath{fill}%
\end{pgfscope}%
\begin{pgfscope}%
\pgfpathrectangle{\pgfqpoint{1.020000in}{0.880000in}}{\pgfqpoint{6.160000in}{6.160000in}}%
\pgfusepath{clip}%
\pgfsetbuttcap%
\pgfsetroundjoin%
\definecolor{currentfill}{rgb}{0.603162,0.731527,0.999565}%
\pgfsetfillcolor{currentfill}%
\pgfsetlinewidth{0.000000pt}%
\definecolor{currentstroke}{rgb}{0.000000,0.000000,0.000000}%
\pgfsetstrokecolor{currentstroke}%
\pgfsetdash{}{0pt}%
\pgfpathmoveto{\pgfqpoint{4.268858in}{3.873729in}}%
\pgfpathlineto{\pgfqpoint{4.278967in}{3.858897in}}%
\pgfpathlineto{\pgfqpoint{4.289093in}{3.843675in}}%
\pgfpathlineto{\pgfqpoint{4.321205in}{3.833990in}}%
\pgfpathlineto{\pgfqpoint{4.353293in}{3.825332in}}%
\pgfpathlineto{\pgfqpoint{4.343119in}{3.840692in}}%
\pgfpathlineto{\pgfqpoint{4.332963in}{3.855688in}}%
\pgfpathlineto{\pgfqpoint{4.300923in}{3.864202in}}%
\pgfpathlineto{\pgfqpoint{4.268858in}{3.873729in}}%
\pgfpathclose%
\pgfusepath{fill}%
\end{pgfscope}%
\begin{pgfscope}%
\pgfpathrectangle{\pgfqpoint{1.020000in}{0.880000in}}{\pgfqpoint{6.160000in}{6.160000in}}%
\pgfusepath{clip}%
\pgfsetbuttcap%
\pgfsetroundjoin%
\definecolor{currentfill}{rgb}{0.698454,0.799450,0.984577}%
\pgfsetfillcolor{currentfill}%
\pgfsetlinewidth{0.000000pt}%
\definecolor{currentstroke}{rgb}{0.000000,0.000000,0.000000}%
\pgfsetstrokecolor{currentstroke}%
\pgfsetdash{}{0pt}%
\pgfpathmoveto{\pgfqpoint{3.971840in}{4.047872in}}%
\pgfpathlineto{\pgfqpoint{3.981683in}{4.037188in}}%
\pgfpathlineto{\pgfqpoint{3.991544in}{4.025849in}}%
\pgfpathlineto{\pgfqpoint{4.023800in}{4.004486in}}%
\pgfpathlineto{\pgfqpoint{4.056020in}{3.985055in}}%
\pgfpathlineto{\pgfqpoint{4.046109in}{3.997253in}}%
\pgfpathlineto{\pgfqpoint{4.036217in}{4.008909in}}%
\pgfpathlineto{\pgfqpoint{4.004046in}{4.027479in}}%
\pgfpathlineto{\pgfqpoint{3.971840in}{4.047872in}}%
\pgfpathclose%
\pgfusepath{fill}%
\end{pgfscope}%
\begin{pgfscope}%
\pgfpathrectangle{\pgfqpoint{1.020000in}{0.880000in}}{\pgfqpoint{6.160000in}{6.160000in}}%
\pgfusepath{clip}%
\pgfsetbuttcap%
\pgfsetroundjoin%
\definecolor{currentfill}{rgb}{0.500031,0.638508,0.981070}%
\pgfsetfillcolor{currentfill}%
\pgfsetlinewidth{0.000000pt}%
\definecolor{currentstroke}{rgb}{0.000000,0.000000,0.000000}%
\pgfsetstrokecolor{currentstroke}%
\pgfsetdash{}{0pt}%
\pgfpathmoveto{\pgfqpoint{4.778598in}{3.699783in}}%
\pgfpathlineto{\pgfqpoint{4.789122in}{3.679395in}}%
\pgfpathlineto{\pgfqpoint{4.799655in}{3.657431in}}%
\pgfpathlineto{\pgfqpoint{4.831603in}{3.655404in}}%
\pgfpathlineto{\pgfqpoint{4.863533in}{3.653931in}}%
\pgfpathlineto{\pgfqpoint{4.852955in}{3.675767in}}%
\pgfpathlineto{\pgfqpoint{4.842385in}{3.696034in}}%
\pgfpathlineto{\pgfqpoint{4.810501in}{3.697723in}}%
\pgfpathlineto{\pgfqpoint{4.778598in}{3.699783in}}%
\pgfpathclose%
\pgfusepath{fill}%
\end{pgfscope}%
\begin{pgfscope}%
\pgfpathrectangle{\pgfqpoint{1.020000in}{0.880000in}}{\pgfqpoint{6.160000in}{6.160000in}}%
\pgfusepath{clip}%
\pgfsetbuttcap%
\pgfsetroundjoin%
\definecolor{currentfill}{rgb}{0.441123,0.576532,0.954545}%
\pgfsetfillcolor{currentfill}%
\pgfsetlinewidth{0.000000pt}%
\definecolor{currentstroke}{rgb}{0.000000,0.000000,0.000000}%
\pgfsetstrokecolor{currentstroke}%
\pgfsetdash{}{0pt}%
\pgfpathmoveto{\pgfqpoint{5.352586in}{3.592519in}}%
\pgfpathlineto{\pgfqpoint{5.363611in}{3.574569in}}%
\pgfpathlineto{\pgfqpoint{5.374650in}{3.556140in}}%
\pgfpathlineto{\pgfqpoint{5.406469in}{3.559565in}}%
\pgfpathlineto{\pgfqpoint{5.438263in}{3.562339in}}%
\pgfpathlineto{\pgfqpoint{5.427161in}{3.579387in}}%
\pgfpathlineto{\pgfqpoint{5.416075in}{3.596122in}}%
\pgfpathlineto{\pgfqpoint{5.384342in}{3.594525in}}%
\pgfpathlineto{\pgfqpoint{5.352586in}{3.592519in}}%
\pgfpathclose%
\pgfusepath{fill}%
\end{pgfscope}%
\begin{pgfscope}%
\pgfpathrectangle{\pgfqpoint{1.020000in}{0.880000in}}{\pgfqpoint{6.160000in}{6.160000in}}%
\pgfusepath{clip}%
\pgfsetbuttcap%
\pgfsetroundjoin%
\definecolor{currentfill}{rgb}{0.963772,0.749086,0.649420}%
\pgfsetfillcolor{currentfill}%
\pgfsetlinewidth{0.000000pt}%
\definecolor{currentstroke}{rgb}{0.000000,0.000000,0.000000}%
\pgfsetstrokecolor{currentstroke}%
\pgfsetdash{}{0pt}%
\pgfpathmoveto{\pgfqpoint{2.996784in}{4.661030in}}%
\pgfpathlineto{\pgfqpoint{3.005306in}{4.690546in}}%
\pgfpathlineto{\pgfqpoint{3.013834in}{4.721524in}}%
\pgfpathlineto{\pgfqpoint{3.046586in}{4.701874in}}%
\pgfpathlineto{\pgfqpoint{3.079323in}{4.680595in}}%
\pgfpathlineto{\pgfqpoint{3.070694in}{4.652081in}}%
\pgfpathlineto{\pgfqpoint{3.062074in}{4.624930in}}%
\pgfpathlineto{\pgfqpoint{3.029437in}{4.643656in}}%
\pgfpathlineto{\pgfqpoint{2.996784in}{4.661030in}}%
\pgfpathclose%
\pgfusepath{fill}%
\end{pgfscope}%
\begin{pgfscope}%
\pgfpathrectangle{\pgfqpoint{1.020000in}{0.880000in}}{\pgfqpoint{6.160000in}{6.160000in}}%
\pgfusepath{clip}%
\pgfsetbuttcap%
\pgfsetroundjoin%
\definecolor{currentfill}{rgb}{0.565182,0.699438,0.996635}%
\pgfsetfillcolor{currentfill}%
\pgfsetlinewidth{0.000000pt}%
\definecolor{currentstroke}{rgb}{0.000000,0.000000,0.000000}%
\pgfsetstrokecolor{currentstroke}%
\pgfsetdash{}{0pt}%
\pgfpathmoveto{\pgfqpoint{4.417398in}{3.810664in}}%
\pgfpathlineto{\pgfqpoint{4.427636in}{3.794777in}}%
\pgfpathlineto{\pgfqpoint{4.437890in}{3.778440in}}%
\pgfpathlineto{\pgfqpoint{4.469956in}{3.772066in}}%
\pgfpathlineto{\pgfqpoint{4.501999in}{3.766323in}}%
\pgfpathlineto{\pgfqpoint{4.491698in}{3.782861in}}%
\pgfpathlineto{\pgfqpoint{4.481412in}{3.798900in}}%
\pgfpathlineto{\pgfqpoint{4.449416in}{3.804462in}}%
\pgfpathlineto{\pgfqpoint{4.417398in}{3.810664in}}%
\pgfpathclose%
\pgfusepath{fill}%
\end{pgfscope}%
\begin{pgfscope}%
\pgfpathrectangle{\pgfqpoint{1.020000in}{0.880000in}}{\pgfqpoint{6.160000in}{6.160000in}}%
\pgfusepath{clip}%
\pgfsetbuttcap%
\pgfsetroundjoin%
\definecolor{currentfill}{rgb}{0.451739,0.588181,0.960201}%
\pgfsetfillcolor{currentfill}%
\pgfsetlinewidth{0.000000pt}%
\definecolor{currentstroke}{rgb}{0.000000,0.000000,0.000000}%
\pgfsetstrokecolor{currentstroke}%
\pgfsetdash{}{0pt}%
\pgfpathmoveto{\pgfqpoint{5.139938in}{3.616017in}}%
\pgfpathlineto{\pgfqpoint{5.150749in}{3.594306in}}%
\pgfpathlineto{\pgfqpoint{5.161567in}{3.571247in}}%
\pgfpathlineto{\pgfqpoint{5.193457in}{3.575727in}}%
\pgfpathlineto{\pgfqpoint{5.225327in}{3.579929in}}%
\pgfpathlineto{\pgfqpoint{5.214444in}{3.600745in}}%
\pgfpathlineto{\pgfqpoint{5.203569in}{3.620568in}}%
\pgfpathlineto{\pgfqpoint{5.171763in}{3.618367in}}%
\pgfpathlineto{\pgfqpoint{5.139938in}{3.616017in}}%
\pgfpathclose%
\pgfusepath{fill}%
\end{pgfscope}%
\begin{pgfscope}%
\pgfpathrectangle{\pgfqpoint{1.020000in}{0.880000in}}{\pgfqpoint{6.160000in}{6.160000in}}%
\pgfusepath{clip}%
\pgfsetbuttcap%
\pgfsetroundjoin%
\definecolor{currentfill}{rgb}{0.425199,0.559058,0.946061}%
\pgfsetfillcolor{currentfill}%
\pgfsetlinewidth{0.000000pt}%
\definecolor{currentstroke}{rgb}{0.000000,0.000000,0.000000}%
\pgfsetstrokecolor{currentstroke}%
\pgfsetdash{}{0pt}%
\pgfpathmoveto{\pgfqpoint{5.565189in}{3.568496in}}%
\pgfpathlineto{\pgfqpoint{5.576423in}{3.552827in}}%
\pgfpathlineto{\pgfqpoint{5.587675in}{3.537034in}}%
\pgfpathlineto{\pgfqpoint{5.619402in}{3.538205in}}%
\pgfpathlineto{\pgfqpoint{5.608123in}{3.553755in}}%
\pgfpathlineto{\pgfqpoint{5.596862in}{3.569199in}}%
\pgfpathlineto{\pgfqpoint{5.565189in}{3.568496in}}%
\pgfpathclose%
\pgfusepath{fill}%
\end{pgfscope}%
\begin{pgfscope}%
\pgfpathrectangle{\pgfqpoint{1.020000in}{0.880000in}}{\pgfqpoint{6.160000in}{6.160000in}}%
\pgfusepath{clip}%
\pgfsetbuttcap%
\pgfsetroundjoin%
\definecolor{currentfill}{rgb}{0.763363,0.835092,0.955658}%
\pgfsetfillcolor{currentfill}%
\pgfsetlinewidth{0.000000pt}%
\definecolor{currentstroke}{rgb}{0.000000,0.000000,0.000000}%
\pgfsetstrokecolor{currentstroke}%
\pgfsetdash{}{0pt}%
\pgfpathmoveto{\pgfqpoint{3.823217in}{4.163006in}}%
\pgfpathlineto{\pgfqpoint{3.832913in}{4.156686in}}%
\pgfpathlineto{\pgfqpoint{3.842630in}{4.149568in}}%
\pgfpathlineto{\pgfqpoint{3.874995in}{4.120970in}}%
\pgfpathlineto{\pgfqpoint{3.907317in}{4.094533in}}%
\pgfpathlineto{\pgfqpoint{3.897546in}{4.103273in}}%
\pgfpathlineto{\pgfqpoint{3.887794in}{4.111345in}}%
\pgfpathlineto{\pgfqpoint{3.855526in}{4.136173in}}%
\pgfpathlineto{\pgfqpoint{3.823217in}{4.163006in}}%
\pgfpathclose%
\pgfusepath{fill}%
\end{pgfscope}%
\begin{pgfscope}%
\pgfpathrectangle{\pgfqpoint{1.020000in}{0.880000in}}{\pgfqpoint{6.160000in}{6.160000in}}%
\pgfusepath{clip}%
\pgfsetbuttcap%
\pgfsetroundjoin%
\definecolor{currentfill}{rgb}{0.473070,0.611077,0.970634}%
\pgfsetfillcolor{currentfill}%
\pgfsetlinewidth{0.000000pt}%
\definecolor{currentstroke}{rgb}{0.000000,0.000000,0.000000}%
\pgfsetstrokecolor{currentstroke}%
\pgfsetdash{}{0pt}%
\pgfpathmoveto{\pgfqpoint{4.927341in}{3.652506in}}%
\pgfpathlineto{\pgfqpoint{4.937973in}{3.629471in}}%
\pgfpathlineto{\pgfqpoint{4.948608in}{3.604463in}}%
\pgfpathlineto{\pgfqpoint{4.980537in}{3.605475in}}%
\pgfpathlineto{\pgfqpoint{5.012451in}{3.607022in}}%
\pgfpathlineto{\pgfqpoint{5.001763in}{3.630748in}}%
\pgfpathlineto{\pgfqpoint{4.991079in}{3.652736in}}%
\pgfpathlineto{\pgfqpoint{4.959219in}{3.652448in}}%
\pgfpathlineto{\pgfqpoint{4.927341in}{3.652506in}}%
\pgfpathclose%
\pgfusepath{fill}%
\end{pgfscope}%
\begin{pgfscope}%
\pgfpathrectangle{\pgfqpoint{1.020000in}{0.880000in}}{\pgfqpoint{6.160000in}{6.160000in}}%
\pgfusepath{clip}%
\pgfsetbuttcap%
\pgfsetroundjoin%
\definecolor{currentfill}{rgb}{0.909460,0.839386,0.800331}%
\pgfsetfillcolor{currentfill}%
\pgfsetlinewidth{0.000000pt}%
\definecolor{currentstroke}{rgb}{0.000000,0.000000,0.000000}%
\pgfsetstrokecolor{currentstroke}%
\pgfsetdash{}{0pt}%
\pgfpathmoveto{\pgfqpoint{3.441824in}{4.471844in}}%
\pgfpathlineto{\pgfqpoint{3.451031in}{4.484455in}}%
\pgfpathlineto{\pgfqpoint{3.460261in}{4.496735in}}%
\pgfpathlineto{\pgfqpoint{3.492913in}{4.459291in}}%
\pgfpathlineto{\pgfqpoint{3.525516in}{4.422806in}}%
\pgfpathlineto{\pgfqpoint{3.516202in}{4.414339in}}%
\pgfpathlineto{\pgfqpoint{3.506910in}{4.405557in}}%
\pgfpathlineto{\pgfqpoint{3.474390in}{4.438275in}}%
\pgfpathlineto{\pgfqpoint{3.441824in}{4.471844in}}%
\pgfpathclose%
\pgfusepath{fill}%
\end{pgfscope}%
\begin{pgfscope}%
\pgfpathrectangle{\pgfqpoint{1.020000in}{0.880000in}}{\pgfqpoint{6.160000in}{6.160000in}}%
\pgfusepath{clip}%
\pgfsetbuttcap%
\pgfsetroundjoin%
\definecolor{currentfill}{rgb}{0.538004,0.674902,0.991722}%
\pgfsetfillcolor{currentfill}%
\pgfsetlinewidth{0.000000pt}%
\definecolor{currentstroke}{rgb}{0.000000,0.000000,0.000000}%
\pgfsetstrokecolor{currentstroke}%
\pgfsetdash{}{0pt}%
\pgfpathmoveto{\pgfqpoint{4.566021in}{3.756466in}}%
\pgfpathlineto{\pgfqpoint{4.576384in}{3.739033in}}%
\pgfpathlineto{\pgfqpoint{4.586763in}{3.720789in}}%
\pgfpathlineto{\pgfqpoint{4.618787in}{3.716206in}}%
\pgfpathlineto{\pgfqpoint{4.650790in}{3.712071in}}%
\pgfpathlineto{\pgfqpoint{4.640366in}{3.730726in}}%
\pgfpathlineto{\pgfqpoint{4.629957in}{3.748447in}}%
\pgfpathlineto{\pgfqpoint{4.597999in}{3.752247in}}%
\pgfpathlineto{\pgfqpoint{4.566021in}{3.756466in}}%
\pgfpathclose%
\pgfusepath{fill}%
\end{pgfscope}%
\begin{pgfscope}%
\pgfpathrectangle{\pgfqpoint{1.020000in}{0.880000in}}{\pgfqpoint{6.160000in}{6.160000in}}%
\pgfusepath{clip}%
\pgfsetbuttcap%
\pgfsetroundjoin%
\definecolor{currentfill}{rgb}{0.959518,0.766973,0.674145}%
\pgfsetfillcolor{currentfill}%
\pgfsetlinewidth{0.000000pt}%
\definecolor{currentstroke}{rgb}{0.000000,0.000000,0.000000}%
\pgfsetstrokecolor{currentstroke}%
\pgfsetdash{}{0pt}%
\pgfpathmoveto{\pgfqpoint{3.144736in}{4.633902in}}%
\pgfpathlineto{\pgfqpoint{3.153478in}{4.660506in}}%
\pgfpathlineto{\pgfqpoint{3.162235in}{4.687939in}}%
\pgfpathlineto{\pgfqpoint{3.195008in}{4.659510in}}%
\pgfpathlineto{\pgfqpoint{3.227753in}{4.630095in}}%
\pgfpathlineto{\pgfqpoint{3.218893in}{4.606206in}}%
\pgfpathlineto{\pgfqpoint{3.210049in}{4.583050in}}%
\pgfpathlineto{\pgfqpoint{3.177406in}{4.608891in}}%
\pgfpathlineto{\pgfqpoint{3.144736in}{4.633902in}}%
\pgfpathclose%
\pgfusepath{fill}%
\end{pgfscope}%
\begin{pgfscope}%
\pgfpathrectangle{\pgfqpoint{1.020000in}{0.880000in}}{\pgfqpoint{6.160000in}{6.160000in}}%
\pgfusepath{clip}%
\pgfsetbuttcap%
\pgfsetroundjoin%
\definecolor{currentfill}{rgb}{0.943432,0.802276,0.729172}%
\pgfsetfillcolor{currentfill}%
\pgfsetlinewidth{0.000000pt}%
\definecolor{currentstroke}{rgb}{0.000000,0.000000,0.000000}%
\pgfsetstrokecolor{currentstroke}%
\pgfsetdash{}{0pt}%
\pgfpathmoveto{\pgfqpoint{3.293144in}{4.569302in}}%
\pgfpathlineto{\pgfqpoint{3.302122in}{4.589826in}}%
\pgfpathlineto{\pgfqpoint{3.311120in}{4.610534in}}%
\pgfpathlineto{\pgfqpoint{3.343860in}{4.575663in}}%
\pgfpathlineto{\pgfqpoint{3.376558in}{4.540760in}}%
\pgfpathlineto{\pgfqpoint{3.367463in}{4.524064in}}%
\pgfpathlineto{\pgfqpoint{3.358389in}{4.507500in}}%
\pgfpathlineto{\pgfqpoint{3.325786in}{4.538410in}}%
\pgfpathlineto{\pgfqpoint{3.293144in}{4.569302in}}%
\pgfpathclose%
\pgfusepath{fill}%
\end{pgfscope}%
\begin{pgfscope}%
\pgfpathrectangle{\pgfqpoint{1.020000in}{0.880000in}}{\pgfqpoint{6.160000in}{6.160000in}}%
\pgfusepath{clip}%
\pgfsetbuttcap%
\pgfsetroundjoin%
\definecolor{currentfill}{rgb}{0.430507,0.564883,0.948889}%
\pgfsetfillcolor{currentfill}%
\pgfsetlinewidth{0.000000pt}%
\definecolor{currentstroke}{rgb}{0.000000,0.000000,0.000000}%
\pgfsetstrokecolor{currentstroke}%
\pgfsetdash{}{0pt}%
\pgfpathmoveto{\pgfqpoint{5.501775in}{3.566241in}}%
\pgfpathlineto{\pgfqpoint{5.512952in}{3.549921in}}%
\pgfpathlineto{\pgfqpoint{5.524147in}{3.533423in}}%
\pgfpathlineto{\pgfqpoint{5.555924in}{3.535474in}}%
\pgfpathlineto{\pgfqpoint{5.587675in}{3.537034in}}%
\pgfpathlineto{\pgfqpoint{5.576423in}{3.552827in}}%
\pgfpathlineto{\pgfqpoint{5.565189in}{3.568496in}}%
\pgfpathlineto{\pgfqpoint{5.533494in}{3.567534in}}%
\pgfpathlineto{\pgfqpoint{5.501775in}{3.566241in}}%
\pgfpathclose%
\pgfusepath{fill}%
\end{pgfscope}%
\begin{pgfscope}%
\pgfpathrectangle{\pgfqpoint{1.020000in}{0.880000in}}{\pgfqpoint{6.160000in}{6.160000in}}%
\pgfusepath{clip}%
\pgfsetbuttcap%
\pgfsetroundjoin%
\definecolor{currentfill}{rgb}{0.441123,0.576532,0.954545}%
\pgfsetfillcolor{currentfill}%
\pgfsetlinewidth{0.000000pt}%
\definecolor{currentstroke}{rgb}{0.000000,0.000000,0.000000}%
\pgfsetstrokecolor{currentstroke}%
\pgfsetdash{}{0pt}%
\pgfpathmoveto{\pgfqpoint{5.289003in}{3.587143in}}%
\pgfpathlineto{\pgfqpoint{5.299963in}{3.567496in}}%
\pgfpathlineto{\pgfqpoint{5.310934in}{3.547133in}}%
\pgfpathlineto{\pgfqpoint{5.342804in}{3.552008in}}%
\pgfpathlineto{\pgfqpoint{5.374650in}{3.556140in}}%
\pgfpathlineto{\pgfqpoint{5.363611in}{3.574569in}}%
\pgfpathlineto{\pgfqpoint{5.352586in}{3.592519in}}%
\pgfpathlineto{\pgfqpoint{5.320806in}{3.590065in}}%
\pgfpathlineto{\pgfqpoint{5.289003in}{3.587143in}}%
\pgfpathclose%
\pgfusepath{fill}%
\end{pgfscope}%
\begin{pgfscope}%
\pgfpathrectangle{\pgfqpoint{1.020000in}{0.880000in}}{\pgfqpoint{6.160000in}{6.160000in}}%
\pgfusepath{clip}%
\pgfsetbuttcap%
\pgfsetroundjoin%
\definecolor{currentfill}{rgb}{0.505423,0.643995,0.983157}%
\pgfsetfillcolor{currentfill}%
\pgfsetlinewidth{0.000000pt}%
\definecolor{currentstroke}{rgb}{0.000000,0.000000,0.000000}%
\pgfsetstrokecolor{currentstroke}%
\pgfsetdash{}{0pt}%
\pgfpathmoveto{\pgfqpoint{4.714734in}{3.705095in}}%
\pgfpathlineto{\pgfqpoint{4.725214in}{3.684889in}}%
\pgfpathlineto{\pgfqpoint{4.735703in}{3.663191in}}%
\pgfpathlineto{\pgfqpoint{4.767688in}{3.660027in}}%
\pgfpathlineto{\pgfqpoint{4.799655in}{3.657431in}}%
\pgfpathlineto{\pgfqpoint{4.789122in}{3.679395in}}%
\pgfpathlineto{\pgfqpoint{4.778598in}{3.699783in}}%
\pgfpathlineto{\pgfqpoint{4.746676in}{3.702236in}}%
\pgfpathlineto{\pgfqpoint{4.714734in}{3.705095in}}%
\pgfpathclose%
\pgfusepath{fill}%
\end{pgfscope}%
\begin{pgfscope}%
\pgfpathrectangle{\pgfqpoint{1.020000in}{0.880000in}}{\pgfqpoint{6.160000in}{6.160000in}}%
\pgfusepath{clip}%
\pgfsetbuttcap%
\pgfsetroundjoin%
\definecolor{currentfill}{rgb}{0.831148,0.859513,0.903110}%
\pgfsetfillcolor{currentfill}%
\pgfsetlinewidth{0.000000pt}%
\definecolor{currentstroke}{rgb}{0.000000,0.000000,0.000000}%
\pgfsetstrokecolor{currentstroke}%
\pgfsetdash{}{0pt}%
\pgfpathmoveto{\pgfqpoint{3.674438in}{4.291725in}}%
\pgfpathlineto{\pgfqpoint{3.683968in}{4.291716in}}%
\pgfpathlineto{\pgfqpoint{3.693521in}{4.290863in}}%
\pgfpathlineto{\pgfqpoint{3.726017in}{4.255833in}}%
\pgfpathlineto{\pgfqpoint{3.758464in}{4.222830in}}%
\pgfpathlineto{\pgfqpoint{3.748848in}{4.226234in}}%
\pgfpathlineto{\pgfqpoint{3.739253in}{4.228914in}}%
\pgfpathlineto{\pgfqpoint{3.706869in}{4.259393in}}%
\pgfpathlineto{\pgfqpoint{3.674438in}{4.291725in}}%
\pgfpathclose%
\pgfusepath{fill}%
\end{pgfscope}%
\begin{pgfscope}%
\pgfpathrectangle{\pgfqpoint{1.020000in}{0.880000in}}{\pgfqpoint{6.160000in}{6.160000in}}%
\pgfusepath{clip}%
\pgfsetbuttcap%
\pgfsetroundjoin%
\definecolor{currentfill}{rgb}{0.451739,0.588181,0.960201}%
\pgfsetfillcolor{currentfill}%
\pgfsetlinewidth{0.000000pt}%
\definecolor{currentstroke}{rgb}{0.000000,0.000000,0.000000}%
\pgfsetstrokecolor{currentstroke}%
\pgfsetdash{}{0pt}%
\pgfpathmoveto{\pgfqpoint{5.076229in}{3.611213in}}%
\pgfpathlineto{\pgfqpoint{5.086979in}{3.587476in}}%
\pgfpathlineto{\pgfqpoint{5.097731in}{3.562009in}}%
\pgfpathlineto{\pgfqpoint{5.129657in}{3.566620in}}%
\pgfpathlineto{\pgfqpoint{5.161567in}{3.571247in}}%
\pgfpathlineto{\pgfqpoint{5.150749in}{3.594306in}}%
\pgfpathlineto{\pgfqpoint{5.139938in}{3.616017in}}%
\pgfpathlineto{\pgfqpoint{5.108093in}{3.613598in}}%
\pgfpathlineto{\pgfqpoint{5.076229in}{3.611213in}}%
\pgfpathclose%
\pgfusepath{fill}%
\end{pgfscope}%
\begin{pgfscope}%
\pgfpathrectangle{\pgfqpoint{1.020000in}{0.880000in}}{\pgfqpoint{6.160000in}{6.160000in}}%
\pgfusepath{clip}%
\pgfsetbuttcap%
\pgfsetroundjoin%
\definecolor{currentfill}{rgb}{0.619318,0.744121,0.998931}%
\pgfsetfillcolor{currentfill}%
\pgfsetlinewidth{0.000000pt}%
\definecolor{currentstroke}{rgb}{0.000000,0.000000,0.000000}%
\pgfsetstrokecolor{currentstroke}%
\pgfsetdash{}{0pt}%
\pgfpathmoveto{\pgfqpoint{4.204652in}{3.896287in}}%
\pgfpathlineto{\pgfqpoint{4.214713in}{3.881666in}}%
\pgfpathlineto{\pgfqpoint{4.224791in}{3.866602in}}%
\pgfpathlineto{\pgfqpoint{4.256955in}{3.854504in}}%
\pgfpathlineto{\pgfqpoint{4.289093in}{3.843675in}}%
\pgfpathlineto{\pgfqpoint{4.278967in}{3.858897in}}%
\pgfpathlineto{\pgfqpoint{4.268858in}{3.873729in}}%
\pgfpathlineto{\pgfqpoint{4.236768in}{3.884384in}}%
\pgfpathlineto{\pgfqpoint{4.204652in}{3.896287in}}%
\pgfpathclose%
\pgfusepath{fill}%
\end{pgfscope}%
\begin{pgfscope}%
\pgfpathrectangle{\pgfqpoint{1.020000in}{0.880000in}}{\pgfqpoint{6.160000in}{6.160000in}}%
\pgfusepath{clip}%
\pgfsetbuttcap%
\pgfsetroundjoin%
\definecolor{currentfill}{rgb}{0.667253,0.779176,0.992959}%
\pgfsetfillcolor{currentfill}%
\pgfsetlinewidth{0.000000pt}%
\definecolor{currentstroke}{rgb}{0.000000,0.000000,0.000000}%
\pgfsetstrokecolor{currentstroke}%
\pgfsetdash{}{0pt}%
\pgfpathmoveto{\pgfqpoint{4.056020in}{3.985055in}}%
\pgfpathlineto{\pgfqpoint{4.065950in}{3.972287in}}%
\pgfpathlineto{\pgfqpoint{4.075897in}{3.958921in}}%
\pgfpathlineto{\pgfqpoint{4.108132in}{3.940749in}}%
\pgfpathlineto{\pgfqpoint{4.140335in}{3.924340in}}%
\pgfpathlineto{\pgfqpoint{4.130340in}{3.938175in}}%
\pgfpathlineto{\pgfqpoint{4.120362in}{3.951516in}}%
\pgfpathlineto{\pgfqpoint{4.088207in}{3.967439in}}%
\pgfpathlineto{\pgfqpoint{4.056020in}{3.985055in}}%
\pgfpathclose%
\pgfusepath{fill}%
\end{pgfscope}%
\begin{pgfscope}%
\pgfpathrectangle{\pgfqpoint{1.020000in}{0.880000in}}{\pgfqpoint{6.160000in}{6.160000in}}%
\pgfusepath{clip}%
\pgfsetbuttcap%
\pgfsetroundjoin%
\definecolor{currentfill}{rgb}{0.581486,0.713451,0.998314}%
\pgfsetfillcolor{currentfill}%
\pgfsetlinewidth{0.000000pt}%
\definecolor{currentstroke}{rgb}{0.000000,0.000000,0.000000}%
\pgfsetstrokecolor{currentstroke}%
\pgfsetdash{}{0pt}%
\pgfpathmoveto{\pgfqpoint{4.353293in}{3.825332in}}%
\pgfpathlineto{\pgfqpoint{4.363483in}{3.809581in}}%
\pgfpathlineto{\pgfqpoint{4.373691in}{3.793407in}}%
\pgfpathlineto{\pgfqpoint{4.405802in}{3.785523in}}%
\pgfpathlineto{\pgfqpoint{4.437890in}{3.778440in}}%
\pgfpathlineto{\pgfqpoint{4.427636in}{3.794777in}}%
\pgfpathlineto{\pgfqpoint{4.417398in}{3.810664in}}%
\pgfpathlineto{\pgfqpoint{4.385357in}{3.817590in}}%
\pgfpathlineto{\pgfqpoint{4.353293in}{3.825332in}}%
\pgfpathclose%
\pgfusepath{fill}%
\end{pgfscope}%
\begin{pgfscope}%
\pgfpathrectangle{\pgfqpoint{1.020000in}{0.880000in}}{\pgfqpoint{6.160000in}{6.160000in}}%
\pgfusepath{clip}%
\pgfsetbuttcap%
\pgfsetroundjoin%
\definecolor{currentfill}{rgb}{0.478462,0.616564,0.972721}%
\pgfsetfillcolor{currentfill}%
\pgfsetlinewidth{0.000000pt}%
\definecolor{currentstroke}{rgb}{0.000000,0.000000,0.000000}%
\pgfsetstrokecolor{currentstroke}%
\pgfsetdash{}{0pt}%
\pgfpathmoveto{\pgfqpoint{4.863533in}{3.653931in}}%
\pgfpathlineto{\pgfqpoint{4.874117in}{3.630257in}}%
\pgfpathlineto{\pgfqpoint{4.884704in}{3.604479in}}%
\pgfpathlineto{\pgfqpoint{4.916663in}{3.604100in}}%
\pgfpathlineto{\pgfqpoint{4.948608in}{3.604463in}}%
\pgfpathlineto{\pgfqpoint{4.937973in}{3.629471in}}%
\pgfpathlineto{\pgfqpoint{4.927341in}{3.652506in}}%
\pgfpathlineto{\pgfqpoint{4.895446in}{3.652981in}}%
\pgfpathlineto{\pgfqpoint{4.863533in}{3.653931in}}%
\pgfpathclose%
\pgfusepath{fill}%
\end{pgfscope}%
\begin{pgfscope}%
\pgfpathrectangle{\pgfqpoint{1.020000in}{0.880000in}}{\pgfqpoint{6.160000in}{6.160000in}}%
\pgfusepath{clip}%
\pgfsetbuttcap%
\pgfsetroundjoin%
\definecolor{currentfill}{rgb}{0.728970,0.817464,0.973188}%
\pgfsetfillcolor{currentfill}%
\pgfsetlinewidth{0.000000pt}%
\definecolor{currentstroke}{rgb}{0.000000,0.000000,0.000000}%
\pgfsetstrokecolor{currentstroke}%
\pgfsetdash{}{0pt}%
\pgfpathmoveto{\pgfqpoint{3.907317in}{4.094533in}}%
\pgfpathlineto{\pgfqpoint{3.917108in}{4.085060in}}%
\pgfpathlineto{\pgfqpoint{3.926919in}{4.074795in}}%
\pgfpathlineto{\pgfqpoint{3.959252in}{4.049253in}}%
\pgfpathlineto{\pgfqpoint{3.991544in}{4.025849in}}%
\pgfpathlineto{\pgfqpoint{3.981683in}{4.037188in}}%
\pgfpathlineto{\pgfqpoint{3.971840in}{4.047872in}}%
\pgfpathlineto{\pgfqpoint{3.939598in}{4.070193in}}%
\pgfpathlineto{\pgfqpoint{3.907317in}{4.094533in}}%
\pgfpathclose%
\pgfusepath{fill}%
\end{pgfscope}%
\begin{pgfscope}%
\pgfpathrectangle{\pgfqpoint{1.020000in}{0.880000in}}{\pgfqpoint{6.160000in}{6.160000in}}%
\pgfusepath{clip}%
\pgfsetbuttcap%
\pgfsetroundjoin%
\definecolor{currentfill}{rgb}{0.967544,0.730850,0.624685}%
\pgfsetfillcolor{currentfill}%
\pgfsetlinewidth{0.000000pt}%
\definecolor{currentstroke}{rgb}{0.000000,0.000000,0.000000}%
\pgfsetstrokecolor{currentstroke}%
\pgfsetdash{}{0pt}%
\pgfpathmoveto{\pgfqpoint{2.931439in}{4.691170in}}%
\pgfpathlineto{\pgfqpoint{2.939865in}{4.722459in}}%
\pgfpathlineto{\pgfqpoint{2.948297in}{4.755282in}}%
\pgfpathlineto{\pgfqpoint{2.981070in}{4.739376in}}%
\pgfpathlineto{\pgfqpoint{3.013834in}{4.721524in}}%
\pgfpathlineto{\pgfqpoint{3.005306in}{4.690546in}}%
\pgfpathlineto{\pgfqpoint{2.996784in}{4.661030in}}%
\pgfpathlineto{\pgfqpoint{2.964117in}{4.676909in}}%
\pgfpathlineto{\pgfqpoint{2.931439in}{4.691170in}}%
\pgfpathclose%
\pgfusepath{fill}%
\end{pgfscope}%
\begin{pgfscope}%
\pgfpathrectangle{\pgfqpoint{1.020000in}{0.880000in}}{\pgfqpoint{6.160000in}{6.160000in}}%
\pgfusepath{clip}%
\pgfsetbuttcap%
\pgfsetroundjoin%
\definecolor{currentfill}{rgb}{0.548876,0.685104,0.994379}%
\pgfsetfillcolor{currentfill}%
\pgfsetlinewidth{0.000000pt}%
\definecolor{currentstroke}{rgb}{0.000000,0.000000,0.000000}%
\pgfsetstrokecolor{currentstroke}%
\pgfsetdash{}{0pt}%
\pgfpathmoveto{\pgfqpoint{4.501999in}{3.766323in}}%
\pgfpathlineto{\pgfqpoint{4.512317in}{3.749200in}}%
\pgfpathlineto{\pgfqpoint{4.522650in}{3.731390in}}%
\pgfpathlineto{\pgfqpoint{4.554717in}{3.725840in}}%
\pgfpathlineto{\pgfqpoint{4.586763in}{3.720789in}}%
\pgfpathlineto{\pgfqpoint{4.576384in}{3.739033in}}%
\pgfpathlineto{\pgfqpoint{4.566021in}{3.756466in}}%
\pgfpathlineto{\pgfqpoint{4.534021in}{3.761142in}}%
\pgfpathlineto{\pgfqpoint{4.501999in}{3.766323in}}%
\pgfpathclose%
\pgfusepath{fill}%
\end{pgfscope}%
\begin{pgfscope}%
\pgfpathrectangle{\pgfqpoint{1.020000in}{0.880000in}}{\pgfqpoint{6.160000in}{6.160000in}}%
\pgfusepath{clip}%
\pgfsetbuttcap%
\pgfsetroundjoin%
\definecolor{currentfill}{rgb}{0.891817,0.851973,0.829085}%
\pgfsetfillcolor{currentfill}%
\pgfsetlinewidth{0.000000pt}%
\definecolor{currentstroke}{rgb}{0.000000,0.000000,0.000000}%
\pgfsetstrokecolor{currentstroke}%
\pgfsetdash{}{0pt}%
\pgfpathmoveto{\pgfqpoint{3.525516in}{4.422806in}}%
\pgfpathlineto{\pgfqpoint{3.534853in}{4.430762in}}%
\pgfpathlineto{\pgfqpoint{3.544215in}{4.438009in}}%
\pgfpathlineto{\pgfqpoint{3.576847in}{4.399124in}}%
\pgfpathlineto{\pgfqpoint{3.609428in}{4.361690in}}%
\pgfpathlineto{\pgfqpoint{3.599990in}{4.357878in}}%
\pgfpathlineto{\pgfqpoint{3.590576in}{4.353433in}}%
\pgfpathlineto{\pgfqpoint{3.558070in}{4.387467in}}%
\pgfpathlineto{\pgfqpoint{3.525516in}{4.422806in}}%
\pgfpathclose%
\pgfusepath{fill}%
\end{pgfscope}%
\begin{pgfscope}%
\pgfpathrectangle{\pgfqpoint{1.020000in}{0.880000in}}{\pgfqpoint{6.160000in}{6.160000in}}%
\pgfusepath{clip}%
\pgfsetbuttcap%
\pgfsetroundjoin%
\definecolor{currentfill}{rgb}{0.430507,0.564883,0.948889}%
\pgfsetfillcolor{currentfill}%
\pgfsetlinewidth{0.000000pt}%
\definecolor{currentstroke}{rgb}{0.000000,0.000000,0.000000}%
\pgfsetstrokecolor{currentstroke}%
\pgfsetdash{}{0pt}%
\pgfpathmoveto{\pgfqpoint{5.438263in}{3.562339in}}%
\pgfpathlineto{\pgfqpoint{5.449380in}{3.544993in}}%
\pgfpathlineto{\pgfqpoint{5.460514in}{3.527382in}}%
\pgfpathlineto{\pgfqpoint{5.492344in}{3.530766in}}%
\pgfpathlineto{\pgfqpoint{5.524147in}{3.533423in}}%
\pgfpathlineto{\pgfqpoint{5.512952in}{3.549921in}}%
\pgfpathlineto{\pgfqpoint{5.501775in}{3.566241in}}%
\pgfpathlineto{\pgfqpoint{5.470031in}{3.564537in}}%
\pgfpathlineto{\pgfqpoint{5.438263in}{3.562339in}}%
\pgfpathclose%
\pgfusepath{fill}%
\end{pgfscope}%
\begin{pgfscope}%
\pgfpathrectangle{\pgfqpoint{1.020000in}{0.880000in}}{\pgfqpoint{6.160000in}{6.160000in}}%
\pgfusepath{clip}%
\pgfsetbuttcap%
\pgfsetroundjoin%
\definecolor{currentfill}{rgb}{0.441123,0.576532,0.954545}%
\pgfsetfillcolor{currentfill}%
\pgfsetlinewidth{0.000000pt}%
\definecolor{currentstroke}{rgb}{0.000000,0.000000,0.000000}%
\pgfsetstrokecolor{currentstroke}%
\pgfsetdash{}{0pt}%
\pgfpathmoveto{\pgfqpoint{5.225327in}{3.579929in}}%
\pgfpathlineto{\pgfqpoint{5.236219in}{3.558082in}}%
\pgfpathlineto{\pgfqpoint{5.247119in}{3.535208in}}%
\pgfpathlineto{\pgfqpoint{5.279038in}{3.541519in}}%
\pgfpathlineto{\pgfqpoint{5.310934in}{3.547133in}}%
\pgfpathlineto{\pgfqpoint{5.299963in}{3.567496in}}%
\pgfpathlineto{\pgfqpoint{5.289003in}{3.587143in}}%
\pgfpathlineto{\pgfqpoint{5.257176in}{3.583754in}}%
\pgfpathlineto{\pgfqpoint{5.225327in}{3.579929in}}%
\pgfpathclose%
\pgfusepath{fill}%
\end{pgfscope}%
\begin{pgfscope}%
\pgfpathrectangle{\pgfqpoint{1.020000in}{0.880000in}}{\pgfqpoint{6.160000in}{6.160000in}}%
\pgfusepath{clip}%
\pgfsetbuttcap%
\pgfsetroundjoin%
\definecolor{currentfill}{rgb}{0.451739,0.588181,0.960201}%
\pgfsetfillcolor{currentfill}%
\pgfsetlinewidth{0.000000pt}%
\definecolor{currentstroke}{rgb}{0.000000,0.000000,0.000000}%
\pgfsetstrokecolor{currentstroke}%
\pgfsetdash{}{0pt}%
\pgfpathmoveto{\pgfqpoint{5.012451in}{3.607022in}}%
\pgfpathlineto{\pgfqpoint{5.023141in}{3.581360in}}%
\pgfpathlineto{\pgfqpoint{5.033831in}{3.553607in}}%
\pgfpathlineto{\pgfqpoint{5.065788in}{3.557604in}}%
\pgfpathlineto{\pgfqpoint{5.097731in}{3.562009in}}%
\pgfpathlineto{\pgfqpoint{5.086979in}{3.587476in}}%
\pgfpathlineto{\pgfqpoint{5.076229in}{3.611213in}}%
\pgfpathlineto{\pgfqpoint{5.044348in}{3.608979in}}%
\pgfpathlineto{\pgfqpoint{5.012451in}{3.607022in}}%
\pgfpathclose%
\pgfusepath{fill}%
\end{pgfscope}%
\begin{pgfscope}%
\pgfpathrectangle{\pgfqpoint{1.020000in}{0.880000in}}{\pgfqpoint{6.160000in}{6.160000in}}%
\pgfusepath{clip}%
\pgfsetbuttcap%
\pgfsetroundjoin%
\definecolor{currentfill}{rgb}{0.516260,0.654498,0.986407}%
\pgfsetfillcolor{currentfill}%
\pgfsetlinewidth{0.000000pt}%
\definecolor{currentstroke}{rgb}{0.000000,0.000000,0.000000}%
\pgfsetstrokecolor{currentstroke}%
\pgfsetdash{}{0pt}%
\pgfpathmoveto{\pgfqpoint{4.650790in}{3.712071in}}%
\pgfpathlineto{\pgfqpoint{4.661226in}{3.692291in}}%
\pgfpathlineto{\pgfqpoint{4.671673in}{3.671177in}}%
\pgfpathlineto{\pgfqpoint{4.703698in}{3.666915in}}%
\pgfpathlineto{\pgfqpoint{4.735703in}{3.663191in}}%
\pgfpathlineto{\pgfqpoint{4.725214in}{3.684889in}}%
\pgfpathlineto{\pgfqpoint{4.714734in}{3.705095in}}%
\pgfpathlineto{\pgfqpoint{4.682772in}{3.708371in}}%
\pgfpathlineto{\pgfqpoint{4.650790in}{3.712071in}}%
\pgfpathclose%
\pgfusepath{fill}%
\end{pgfscope}%
\begin{pgfscope}%
\pgfpathrectangle{\pgfqpoint{1.020000in}{0.880000in}}{\pgfqpoint{6.160000in}{6.160000in}}%
\pgfusepath{clip}%
\pgfsetbuttcap%
\pgfsetroundjoin%
\definecolor{currentfill}{rgb}{0.800601,0.850358,0.930008}%
\pgfsetfillcolor{currentfill}%
\pgfsetlinewidth{0.000000pt}%
\definecolor{currentstroke}{rgb}{0.000000,0.000000,0.000000}%
\pgfsetstrokecolor{currentstroke}%
\pgfsetdash{}{0pt}%
\pgfpathmoveto{\pgfqpoint{3.758464in}{4.222830in}}%
\pgfpathlineto{\pgfqpoint{3.768102in}{4.218586in}}%
\pgfpathlineto{\pgfqpoint{3.777763in}{4.213393in}}%
\pgfpathlineto{\pgfqpoint{3.810220in}{4.180369in}}%
\pgfpathlineto{\pgfqpoint{3.842630in}{4.149568in}}%
\pgfpathlineto{\pgfqpoint{3.832913in}{4.156686in}}%
\pgfpathlineto{\pgfqpoint{3.823217in}{4.163006in}}%
\pgfpathlineto{\pgfqpoint{3.790863in}{4.191886in}}%
\pgfpathlineto{\pgfqpoint{3.758464in}{4.222830in}}%
\pgfpathclose%
\pgfusepath{fill}%
\end{pgfscope}%
\begin{pgfscope}%
\pgfpathrectangle{\pgfqpoint{1.020000in}{0.880000in}}{\pgfqpoint{6.160000in}{6.160000in}}%
\pgfusepath{clip}%
\pgfsetbuttcap%
\pgfsetroundjoin%
\definecolor{currentfill}{rgb}{0.483854,0.622050,0.974808}%
\pgfsetfillcolor{currentfill}%
\pgfsetlinewidth{0.000000pt}%
\definecolor{currentstroke}{rgb}{0.000000,0.000000,0.000000}%
\pgfsetstrokecolor{currentstroke}%
\pgfsetdash{}{0pt}%
\pgfpathmoveto{\pgfqpoint{4.799655in}{3.657431in}}%
\pgfpathlineto{\pgfqpoint{4.810194in}{3.633623in}}%
\pgfpathlineto{\pgfqpoint{4.820738in}{3.607704in}}%
\pgfpathlineto{\pgfqpoint{4.852729in}{3.605667in}}%
\pgfpathlineto{\pgfqpoint{4.884704in}{3.604479in}}%
\pgfpathlineto{\pgfqpoint{4.874117in}{3.630257in}}%
\pgfpathlineto{\pgfqpoint{4.863533in}{3.653931in}}%
\pgfpathlineto{\pgfqpoint{4.831603in}{3.655404in}}%
\pgfpathlineto{\pgfqpoint{4.799655in}{3.657431in}}%
\pgfpathclose%
\pgfusepath{fill}%
\end{pgfscope}%
\begin{pgfscope}%
\pgfpathrectangle{\pgfqpoint{1.020000in}{0.880000in}}{\pgfqpoint{6.160000in}{6.160000in}}%
\pgfusepath{clip}%
\pgfsetbuttcap%
\pgfsetroundjoin%
\definecolor{currentfill}{rgb}{0.938326,0.808917,0.741162}%
\pgfsetfillcolor{currentfill}%
\pgfsetlinewidth{0.000000pt}%
\definecolor{currentstroke}{rgb}{0.000000,0.000000,0.000000}%
\pgfsetstrokecolor{currentstroke}%
\pgfsetdash{}{0pt}%
\pgfpathmoveto{\pgfqpoint{3.376558in}{4.540760in}}%
\pgfpathlineto{\pgfqpoint{3.385674in}{4.557360in}}%
\pgfpathlineto{\pgfqpoint{3.394815in}{4.573617in}}%
\pgfpathlineto{\pgfqpoint{3.427562in}{4.534923in}}%
\pgfpathlineto{\pgfqpoint{3.460261in}{4.496735in}}%
\pgfpathlineto{\pgfqpoint{3.451031in}{4.484455in}}%
\pgfpathlineto{\pgfqpoint{3.441824in}{4.471844in}}%
\pgfpathlineto{\pgfqpoint{3.409213in}{4.506077in}}%
\pgfpathlineto{\pgfqpoint{3.376558in}{4.540760in}}%
\pgfpathclose%
\pgfusepath{fill}%
\end{pgfscope}%
\begin{pgfscope}%
\pgfpathrectangle{\pgfqpoint{1.020000in}{0.880000in}}{\pgfqpoint{6.160000in}{6.160000in}}%
\pgfusepath{clip}%
\pgfsetbuttcap%
\pgfsetroundjoin%
\definecolor{currentfill}{rgb}{0.966962,0.735670,0.630877}%
\pgfsetfillcolor{currentfill}%
\pgfsetlinewidth{0.000000pt}%
\definecolor{currentstroke}{rgb}{0.000000,0.000000,0.000000}%
\pgfsetstrokecolor{currentstroke}%
\pgfsetdash{}{0pt}%
\pgfpathmoveto{\pgfqpoint{3.079323in}{4.680595in}}%
\pgfpathlineto{\pgfqpoint{3.087962in}{4.710267in}}%
\pgfpathlineto{\pgfqpoint{3.096616in}{4.740854in}}%
\pgfpathlineto{\pgfqpoint{3.129437in}{4.715135in}}%
\pgfpathlineto{\pgfqpoint{3.162235in}{4.687939in}}%
\pgfpathlineto{\pgfqpoint{3.153478in}{4.660506in}}%
\pgfpathlineto{\pgfqpoint{3.144736in}{4.633902in}}%
\pgfpathlineto{\pgfqpoint{3.112040in}{4.657872in}}%
\pgfpathlineto{\pgfqpoint{3.079323in}{4.680595in}}%
\pgfpathclose%
\pgfusepath{fill}%
\end{pgfscope}%
\begin{pgfscope}%
\pgfpathrectangle{\pgfqpoint{1.020000in}{0.880000in}}{\pgfqpoint{6.160000in}{6.160000in}}%
\pgfusepath{clip}%
\pgfsetbuttcap%
\pgfsetroundjoin%
\definecolor{currentfill}{rgb}{0.640828,0.760752,0.997846}%
\pgfsetfillcolor{currentfill}%
\pgfsetlinewidth{0.000000pt}%
\definecolor{currentstroke}{rgb}{0.000000,0.000000,0.000000}%
\pgfsetstrokecolor{currentstroke}%
\pgfsetdash{}{0pt}%
\pgfpathmoveto{\pgfqpoint{4.140335in}{3.924340in}}%
\pgfpathlineto{\pgfqpoint{4.150349in}{3.909995in}}%
\pgfpathlineto{\pgfqpoint{4.160380in}{3.895127in}}%
\pgfpathlineto{\pgfqpoint{4.192600in}{3.880099in}}%
\pgfpathlineto{\pgfqpoint{4.224791in}{3.866602in}}%
\pgfpathlineto{\pgfqpoint{4.214713in}{3.881666in}}%
\pgfpathlineto{\pgfqpoint{4.204652in}{3.896287in}}%
\pgfpathlineto{\pgfqpoint{4.172508in}{3.909563in}}%
\pgfpathlineto{\pgfqpoint{4.140335in}{3.924340in}}%
\pgfpathclose%
\pgfusepath{fill}%
\end{pgfscope}%
\begin{pgfscope}%
\pgfpathrectangle{\pgfqpoint{1.020000in}{0.880000in}}{\pgfqpoint{6.160000in}{6.160000in}}%
\pgfusepath{clip}%
\pgfsetbuttcap%
\pgfsetroundjoin%
\definecolor{currentfill}{rgb}{0.592356,0.722792,0.999434}%
\pgfsetfillcolor{currentfill}%
\pgfsetlinewidth{0.000000pt}%
\definecolor{currentstroke}{rgb}{0.000000,0.000000,0.000000}%
\pgfsetstrokecolor{currentstroke}%
\pgfsetdash{}{0pt}%
\pgfpathmoveto{\pgfqpoint{4.289093in}{3.843675in}}%
\pgfpathlineto{\pgfqpoint{4.299236in}{3.828044in}}%
\pgfpathlineto{\pgfqpoint{4.309396in}{3.811986in}}%
\pgfpathlineto{\pgfqpoint{4.341556in}{3.802190in}}%
\pgfpathlineto{\pgfqpoint{4.373691in}{3.793407in}}%
\pgfpathlineto{\pgfqpoint{4.363483in}{3.809581in}}%
\pgfpathlineto{\pgfqpoint{4.353293in}{3.825332in}}%
\pgfpathlineto{\pgfqpoint{4.321205in}{3.833990in}}%
\pgfpathlineto{\pgfqpoint{4.289093in}{3.843675in}}%
\pgfpathclose%
\pgfusepath{fill}%
\end{pgfscope}%
\begin{pgfscope}%
\pgfpathrectangle{\pgfqpoint{1.020000in}{0.880000in}}{\pgfqpoint{6.160000in}{6.160000in}}%
\pgfusepath{clip}%
\pgfsetbuttcap%
\pgfsetroundjoin%
\definecolor{currentfill}{rgb}{0.435815,0.570707,0.951717}%
\pgfsetfillcolor{currentfill}%
\pgfsetlinewidth{0.000000pt}%
\definecolor{currentstroke}{rgb}{0.000000,0.000000,0.000000}%
\pgfsetstrokecolor{currentstroke}%
\pgfsetdash{}{0pt}%
\pgfpathmoveto{\pgfqpoint{5.161567in}{3.571247in}}%
\pgfpathlineto{\pgfqpoint{5.172389in}{3.546787in}}%
\pgfpathlineto{\pgfqpoint{5.183215in}{3.520926in}}%
\pgfpathlineto{\pgfqpoint{5.215177in}{3.528295in}}%
\pgfpathlineto{\pgfqpoint{5.247119in}{3.535208in}}%
\pgfpathlineto{\pgfqpoint{5.236219in}{3.558082in}}%
\pgfpathlineto{\pgfqpoint{5.225327in}{3.579929in}}%
\pgfpathlineto{\pgfqpoint{5.193457in}{3.575727in}}%
\pgfpathlineto{\pgfqpoint{5.161567in}{3.571247in}}%
\pgfpathclose%
\pgfusepath{fill}%
\end{pgfscope}%
\begin{pgfscope}%
\pgfpathrectangle{\pgfqpoint{1.020000in}{0.880000in}}{\pgfqpoint{6.160000in}{6.160000in}}%
\pgfusepath{clip}%
\pgfsetbuttcap%
\pgfsetroundjoin%
\definecolor{currentfill}{rgb}{0.698454,0.799450,0.984577}%
\pgfsetfillcolor{currentfill}%
\pgfsetlinewidth{0.000000pt}%
\definecolor{currentstroke}{rgb}{0.000000,0.000000,0.000000}%
\pgfsetstrokecolor{currentstroke}%
\pgfsetdash{}{0pt}%
\pgfpathmoveto{\pgfqpoint{3.991544in}{4.025849in}}%
\pgfpathlineto{\pgfqpoint{4.001425in}{4.013817in}}%
\pgfpathlineto{\pgfqpoint{4.011325in}{4.001057in}}%
\pgfpathlineto{\pgfqpoint{4.043629in}{3.978983in}}%
\pgfpathlineto{\pgfqpoint{4.075897in}{3.958921in}}%
\pgfpathlineto{\pgfqpoint{4.065950in}{3.972287in}}%
\pgfpathlineto{\pgfqpoint{4.056020in}{3.985055in}}%
\pgfpathlineto{\pgfqpoint{4.023800in}{4.004486in}}%
\pgfpathlineto{\pgfqpoint{3.991544in}{4.025849in}}%
\pgfpathclose%
\pgfusepath{fill}%
\end{pgfscope}%
\begin{pgfscope}%
\pgfpathrectangle{\pgfqpoint{1.020000in}{0.880000in}}{\pgfqpoint{6.160000in}{6.160000in}}%
\pgfusepath{clip}%
\pgfsetbuttcap%
\pgfsetroundjoin%
\definecolor{currentfill}{rgb}{0.430507,0.564883,0.948889}%
\pgfsetfillcolor{currentfill}%
\pgfsetlinewidth{0.000000pt}%
\definecolor{currentstroke}{rgb}{0.000000,0.000000,0.000000}%
\pgfsetstrokecolor{currentstroke}%
\pgfsetdash{}{0pt}%
\pgfpathmoveto{\pgfqpoint{5.374650in}{3.556140in}}%
\pgfpathlineto{\pgfqpoint{5.385702in}{3.537257in}}%
\pgfpathlineto{\pgfqpoint{5.396769in}{3.517971in}}%
\pgfpathlineto{\pgfqpoint{5.428656in}{3.523153in}}%
\pgfpathlineto{\pgfqpoint{5.460514in}{3.527382in}}%
\pgfpathlineto{\pgfqpoint{5.449380in}{3.544993in}}%
\pgfpathlineto{\pgfqpoint{5.438263in}{3.562339in}}%
\pgfpathlineto{\pgfqpoint{5.406469in}{3.559565in}}%
\pgfpathlineto{\pgfqpoint{5.374650in}{3.556140in}}%
\pgfpathclose%
\pgfusepath{fill}%
\end{pgfscope}%
\begin{pgfscope}%
\pgfpathrectangle{\pgfqpoint{1.020000in}{0.880000in}}{\pgfqpoint{6.160000in}{6.160000in}}%
\pgfusepath{clip}%
\pgfsetbuttcap%
\pgfsetroundjoin%
\definecolor{currentfill}{rgb}{0.959518,0.766973,0.674145}%
\pgfsetfillcolor{currentfill}%
\pgfsetlinewidth{0.000000pt}%
\definecolor{currentstroke}{rgb}{0.000000,0.000000,0.000000}%
\pgfsetstrokecolor{currentstroke}%
\pgfsetdash{}{0pt}%
\pgfpathmoveto{\pgfqpoint{3.227753in}{4.630095in}}%
\pgfpathlineto{\pgfqpoint{3.236631in}{4.654483in}}%
\pgfpathlineto{\pgfqpoint{3.245529in}{4.679105in}}%
\pgfpathlineto{\pgfqpoint{3.278343in}{4.645107in}}%
\pgfpathlineto{\pgfqpoint{3.311120in}{4.610534in}}%
\pgfpathlineto{\pgfqpoint{3.302122in}{4.589826in}}%
\pgfpathlineto{\pgfqpoint{3.293144in}{4.569302in}}%
\pgfpathlineto{\pgfqpoint{3.260465in}{4.599943in}}%
\pgfpathlineto{\pgfqpoint{3.227753in}{4.630095in}}%
\pgfpathclose%
\pgfusepath{fill}%
\end{pgfscope}%
\begin{pgfscope}%
\pgfpathrectangle{\pgfqpoint{1.020000in}{0.880000in}}{\pgfqpoint{6.160000in}{6.160000in}}%
\pgfusepath{clip}%
\pgfsetbuttcap%
\pgfsetroundjoin%
\definecolor{currentfill}{rgb}{0.559747,0.694768,0.996075}%
\pgfsetfillcolor{currentfill}%
\pgfsetlinewidth{0.000000pt}%
\definecolor{currentstroke}{rgb}{0.000000,0.000000,0.000000}%
\pgfsetstrokecolor{currentstroke}%
\pgfsetdash{}{0pt}%
\pgfpathmoveto{\pgfqpoint{4.437890in}{3.778440in}}%
\pgfpathlineto{\pgfqpoint{4.448161in}{3.761595in}}%
\pgfpathlineto{\pgfqpoint{4.458449in}{3.744170in}}%
\pgfpathlineto{\pgfqpoint{4.490561in}{3.737482in}}%
\pgfpathlineto{\pgfqpoint{4.522650in}{3.731390in}}%
\pgfpathlineto{\pgfqpoint{4.512317in}{3.749200in}}%
\pgfpathlineto{\pgfqpoint{4.501999in}{3.766323in}}%
\pgfpathlineto{\pgfqpoint{4.469956in}{3.772066in}}%
\pgfpathlineto{\pgfqpoint{4.437890in}{3.778440in}}%
\pgfpathclose%
\pgfusepath{fill}%
\end{pgfscope}%
\begin{pgfscope}%
\pgfpathrectangle{\pgfqpoint{1.020000in}{0.880000in}}{\pgfqpoint{6.160000in}{6.160000in}}%
\pgfusepath{clip}%
\pgfsetbuttcap%
\pgfsetroundjoin%
\definecolor{currentfill}{rgb}{0.451739,0.588181,0.960201}%
\pgfsetfillcolor{currentfill}%
\pgfsetlinewidth{0.000000pt}%
\definecolor{currentstroke}{rgb}{0.000000,0.000000,0.000000}%
\pgfsetstrokecolor{currentstroke}%
\pgfsetdash{}{0pt}%
\pgfpathmoveto{\pgfqpoint{4.948608in}{3.604463in}}%
\pgfpathlineto{\pgfqpoint{4.959244in}{3.577256in}}%
\pgfpathlineto{\pgfqpoint{4.969879in}{3.547672in}}%
\pgfpathlineto{\pgfqpoint{5.001861in}{3.550230in}}%
\pgfpathlineto{\pgfqpoint{5.033831in}{3.553607in}}%
\pgfpathlineto{\pgfqpoint{5.023141in}{3.581360in}}%
\pgfpathlineto{\pgfqpoint{5.012451in}{3.607022in}}%
\pgfpathlineto{\pgfqpoint{4.980537in}{3.605475in}}%
\pgfpathlineto{\pgfqpoint{4.948608in}{3.604463in}}%
\pgfpathclose%
\pgfusepath{fill}%
\end{pgfscope}%
\begin{pgfscope}%
\pgfpathrectangle{\pgfqpoint{1.020000in}{0.880000in}}{\pgfqpoint{6.160000in}{6.160000in}}%
\pgfusepath{clip}%
\pgfsetbuttcap%
\pgfsetroundjoin%
\definecolor{currentfill}{rgb}{0.419991,0.552989,0.942630}%
\pgfsetfillcolor{currentfill}%
\pgfsetlinewidth{0.000000pt}%
\definecolor{currentstroke}{rgb}{0.000000,0.000000,0.000000}%
\pgfsetstrokecolor{currentstroke}%
\pgfsetdash{}{0pt}%
\pgfpathmoveto{\pgfqpoint{5.587675in}{3.537034in}}%
\pgfpathlineto{\pgfqpoint{5.598946in}{3.521134in}}%
\pgfpathlineto{\pgfqpoint{5.610235in}{3.505147in}}%
\pgfpathlineto{\pgfqpoint{5.642017in}{3.506840in}}%
\pgfpathlineto{\pgfqpoint{5.630700in}{3.522562in}}%
\pgfpathlineto{\pgfqpoint{5.619402in}{3.538205in}}%
\pgfpathlineto{\pgfqpoint{5.587675in}{3.537034in}}%
\pgfpathclose%
\pgfusepath{fill}%
\end{pgfscope}%
\begin{pgfscope}%
\pgfpathrectangle{\pgfqpoint{1.020000in}{0.880000in}}{\pgfqpoint{6.160000in}{6.160000in}}%
\pgfusepath{clip}%
\pgfsetbuttcap%
\pgfsetroundjoin%
\definecolor{currentfill}{rgb}{0.968863,0.710838,0.599901}%
\pgfsetfillcolor{currentfill}%
\pgfsetlinewidth{0.000000pt}%
\definecolor{currentstroke}{rgb}{0.000000,0.000000,0.000000}%
\pgfsetstrokecolor{currentstroke}%
\pgfsetdash{}{0pt}%
\pgfpathmoveto{\pgfqpoint{2.866059in}{4.714429in}}%
\pgfpathlineto{\pgfqpoint{2.874395in}{4.746809in}}%
\pgfpathlineto{\pgfqpoint{2.882736in}{4.780766in}}%
\pgfpathlineto{\pgfqpoint{2.915518in}{4.769112in}}%
\pgfpathlineto{\pgfqpoint{2.948297in}{4.755282in}}%
\pgfpathlineto{\pgfqpoint{2.939865in}{4.722459in}}%
\pgfpathlineto{\pgfqpoint{2.931439in}{4.691170in}}%
\pgfpathlineto{\pgfqpoint{2.898752in}{4.703705in}}%
\pgfpathlineto{\pgfqpoint{2.866059in}{4.714429in}}%
\pgfpathclose%
\pgfusepath{fill}%
\end{pgfscope}%
\begin{pgfscope}%
\pgfpathrectangle{\pgfqpoint{1.020000in}{0.880000in}}{\pgfqpoint{6.160000in}{6.160000in}}%
\pgfusepath{clip}%
\pgfsetbuttcap%
\pgfsetroundjoin%
\definecolor{currentfill}{rgb}{0.867428,0.864377,0.862602}%
\pgfsetfillcolor{currentfill}%
\pgfsetlinewidth{0.000000pt}%
\definecolor{currentstroke}{rgb}{0.000000,0.000000,0.000000}%
\pgfsetstrokecolor{currentstroke}%
\pgfsetdash{}{0pt}%
\pgfpathmoveto{\pgfqpoint{3.609428in}{4.361690in}}%
\pgfpathlineto{\pgfqpoint{3.618889in}{4.364697in}}%
\pgfpathlineto{\pgfqpoint{3.628375in}{4.366726in}}%
\pgfpathlineto{\pgfqpoint{3.660974in}{4.327858in}}%
\pgfpathlineto{\pgfqpoint{3.693521in}{4.290863in}}%
\pgfpathlineto{\pgfqpoint{3.683968in}{4.291716in}}%
\pgfpathlineto{\pgfqpoint{3.674438in}{4.291725in}}%
\pgfpathlineto{\pgfqpoint{3.641958in}{4.325852in}}%
\pgfpathlineto{\pgfqpoint{3.609428in}{4.361690in}}%
\pgfpathclose%
\pgfusepath{fill}%
\end{pgfscope}%
\begin{pgfscope}%
\pgfpathrectangle{\pgfqpoint{1.020000in}{0.880000in}}{\pgfqpoint{6.160000in}{6.160000in}}%
\pgfusepath{clip}%
\pgfsetbuttcap%
\pgfsetroundjoin%
\definecolor{currentfill}{rgb}{0.521696,0.659599,0.987736}%
\pgfsetfillcolor{currentfill}%
\pgfsetlinewidth{0.000000pt}%
\definecolor{currentstroke}{rgb}{0.000000,0.000000,0.000000}%
\pgfsetstrokecolor{currentstroke}%
\pgfsetdash{}{0pt}%
\pgfpathmoveto{\pgfqpoint{4.586763in}{3.720789in}}%
\pgfpathlineto{\pgfqpoint{4.597155in}{3.701578in}}%
\pgfpathlineto{\pgfqpoint{4.607561in}{3.681233in}}%
\pgfpathlineto{\pgfqpoint{4.639627in}{3.675956in}}%
\pgfpathlineto{\pgfqpoint{4.671673in}{3.671177in}}%
\pgfpathlineto{\pgfqpoint{4.661226in}{3.692291in}}%
\pgfpathlineto{\pgfqpoint{4.650790in}{3.712071in}}%
\pgfpathlineto{\pgfqpoint{4.618787in}{3.716206in}}%
\pgfpathlineto{\pgfqpoint{4.586763in}{3.720789in}}%
\pgfpathclose%
\pgfusepath{fill}%
\end{pgfscope}%
\begin{pgfscope}%
\pgfpathrectangle{\pgfqpoint{1.020000in}{0.880000in}}{\pgfqpoint{6.160000in}{6.160000in}}%
\pgfusepath{clip}%
\pgfsetbuttcap%
\pgfsetroundjoin%
\definecolor{currentfill}{rgb}{0.763363,0.835092,0.955658}%
\pgfsetfillcolor{currentfill}%
\pgfsetlinewidth{0.000000pt}%
\definecolor{currentstroke}{rgb}{0.000000,0.000000,0.000000}%
\pgfsetstrokecolor{currentstroke}%
\pgfsetdash{}{0pt}%
\pgfpathmoveto{\pgfqpoint{3.842630in}{4.149568in}}%
\pgfpathlineto{\pgfqpoint{3.852368in}{4.141569in}}%
\pgfpathlineto{\pgfqpoint{3.862127in}{4.132613in}}%
\pgfpathlineto{\pgfqpoint{3.894545in}{4.102559in}}%
\pgfpathlineto{\pgfqpoint{3.926919in}{4.074795in}}%
\pgfpathlineto{\pgfqpoint{3.917108in}{4.085060in}}%
\pgfpathlineto{\pgfqpoint{3.907317in}{4.094533in}}%
\pgfpathlineto{\pgfqpoint{3.874995in}{4.120970in}}%
\pgfpathlineto{\pgfqpoint{3.842630in}{4.149568in}}%
\pgfpathclose%
\pgfusepath{fill}%
\end{pgfscope}%
\begin{pgfscope}%
\pgfpathrectangle{\pgfqpoint{1.020000in}{0.880000in}}{\pgfqpoint{6.160000in}{6.160000in}}%
\pgfusepath{clip}%
\pgfsetbuttcap%
\pgfsetroundjoin%
\definecolor{currentfill}{rgb}{0.489246,0.627536,0.976896}%
\pgfsetfillcolor{currentfill}%
\pgfsetlinewidth{0.000000pt}%
\definecolor{currentstroke}{rgb}{0.000000,0.000000,0.000000}%
\pgfsetstrokecolor{currentstroke}%
\pgfsetdash{}{0pt}%
\pgfpathmoveto{\pgfqpoint{4.735703in}{3.663191in}}%
\pgfpathlineto{\pgfqpoint{4.746200in}{3.639752in}}%
\pgfpathlineto{\pgfqpoint{4.756704in}{3.614321in}}%
\pgfpathlineto{\pgfqpoint{4.788730in}{3.610596in}}%
\pgfpathlineto{\pgfqpoint{4.820738in}{3.607704in}}%
\pgfpathlineto{\pgfqpoint{4.810194in}{3.633623in}}%
\pgfpathlineto{\pgfqpoint{4.799655in}{3.657431in}}%
\pgfpathlineto{\pgfqpoint{4.767688in}{3.660027in}}%
\pgfpathlineto{\pgfqpoint{4.735703in}{3.663191in}}%
\pgfpathclose%
\pgfusepath{fill}%
\end{pgfscope}%
\begin{pgfscope}%
\pgfpathrectangle{\pgfqpoint{1.020000in}{0.880000in}}{\pgfqpoint{6.160000in}{6.160000in}}%
\pgfusepath{clip}%
\pgfsetbuttcap%
\pgfsetroundjoin%
\definecolor{currentfill}{rgb}{0.435815,0.570707,0.951717}%
\pgfsetfillcolor{currentfill}%
\pgfsetlinewidth{0.000000pt}%
\definecolor{currentstroke}{rgb}{0.000000,0.000000,0.000000}%
\pgfsetstrokecolor{currentstroke}%
\pgfsetdash{}{0pt}%
\pgfpathmoveto{\pgfqpoint{5.097731in}{3.562009in}}%
\pgfpathlineto{\pgfqpoint{5.108484in}{3.534741in}}%
\pgfpathlineto{\pgfqpoint{5.119239in}{3.505672in}}%
\pgfpathlineto{\pgfqpoint{5.151235in}{3.513302in}}%
\pgfpathlineto{\pgfqpoint{5.183215in}{3.520926in}}%
\pgfpathlineto{\pgfqpoint{5.172389in}{3.546787in}}%
\pgfpathlineto{\pgfqpoint{5.161567in}{3.571247in}}%
\pgfpathlineto{\pgfqpoint{5.129657in}{3.566620in}}%
\pgfpathlineto{\pgfqpoint{5.097731in}{3.562009in}}%
\pgfpathclose%
\pgfusepath{fill}%
\end{pgfscope}%
\begin{pgfscope}%
\pgfpathrectangle{\pgfqpoint{1.020000in}{0.880000in}}{\pgfqpoint{6.160000in}{6.160000in}}%
\pgfusepath{clip}%
\pgfsetbuttcap%
\pgfsetroundjoin%
\definecolor{currentfill}{rgb}{0.430507,0.564883,0.948889}%
\pgfsetfillcolor{currentfill}%
\pgfsetlinewidth{0.000000pt}%
\definecolor{currentstroke}{rgb}{0.000000,0.000000,0.000000}%
\pgfsetstrokecolor{currentstroke}%
\pgfsetdash{}{0pt}%
\pgfpathmoveto{\pgfqpoint{5.310934in}{3.547133in}}%
\pgfpathlineto{\pgfqpoint{5.321916in}{3.526092in}}%
\pgfpathlineto{\pgfqpoint{5.332911in}{3.504449in}}%
\pgfpathlineto{\pgfqpoint{5.364854in}{3.511753in}}%
\pgfpathlineto{\pgfqpoint{5.396769in}{3.517971in}}%
\pgfpathlineto{\pgfqpoint{5.385702in}{3.537257in}}%
\pgfpathlineto{\pgfqpoint{5.374650in}{3.556140in}}%
\pgfpathlineto{\pgfqpoint{5.342804in}{3.552008in}}%
\pgfpathlineto{\pgfqpoint{5.310934in}{3.547133in}}%
\pgfpathclose%
\pgfusepath{fill}%
\end{pgfscope}%
\begin{pgfscope}%
\pgfpathrectangle{\pgfqpoint{1.020000in}{0.880000in}}{\pgfqpoint{6.160000in}{6.160000in}}%
\pgfusepath{clip}%
\pgfsetbuttcap%
\pgfsetroundjoin%
\definecolor{currentfill}{rgb}{0.419991,0.552989,0.942630}%
\pgfsetfillcolor{currentfill}%
\pgfsetlinewidth{0.000000pt}%
\definecolor{currentstroke}{rgb}{0.000000,0.000000,0.000000}%
\pgfsetstrokecolor{currentstroke}%
\pgfsetdash{}{0pt}%
\pgfpathmoveto{\pgfqpoint{5.524147in}{3.533423in}}%
\pgfpathlineto{\pgfqpoint{5.535360in}{3.516777in}}%
\pgfpathlineto{\pgfqpoint{5.546592in}{3.500019in}}%
\pgfpathlineto{\pgfqpoint{5.578427in}{3.502920in}}%
\pgfpathlineto{\pgfqpoint{5.610235in}{3.505147in}}%
\pgfpathlineto{\pgfqpoint{5.598946in}{3.521134in}}%
\pgfpathlineto{\pgfqpoint{5.587675in}{3.537034in}}%
\pgfpathlineto{\pgfqpoint{5.555924in}{3.535474in}}%
\pgfpathlineto{\pgfqpoint{5.524147in}{3.533423in}}%
\pgfpathclose%
\pgfusepath{fill}%
\end{pgfscope}%
\begin{pgfscope}%
\pgfpathrectangle{\pgfqpoint{1.020000in}{0.880000in}}{\pgfqpoint{6.160000in}{6.160000in}}%
\pgfusepath{clip}%
\pgfsetbuttcap%
\pgfsetroundjoin%
\definecolor{currentfill}{rgb}{0.457046,0.594006,0.963029}%
\pgfsetfillcolor{currentfill}%
\pgfsetlinewidth{0.000000pt}%
\definecolor{currentstroke}{rgb}{0.000000,0.000000,0.000000}%
\pgfsetstrokecolor{currentstroke}%
\pgfsetdash{}{0pt}%
\pgfpathmoveto{\pgfqpoint{4.884704in}{3.604479in}}%
\pgfpathlineto{\pgfqpoint{4.895292in}{3.576353in}}%
\pgfpathlineto{\pgfqpoint{4.905879in}{3.545692in}}%
\pgfpathlineto{\pgfqpoint{4.937885in}{3.546112in}}%
\pgfpathlineto{\pgfqpoint{4.969879in}{3.547672in}}%
\pgfpathlineto{\pgfqpoint{4.959244in}{3.577256in}}%
\pgfpathlineto{\pgfqpoint{4.948608in}{3.604463in}}%
\pgfpathlineto{\pgfqpoint{4.916663in}{3.604100in}}%
\pgfpathlineto{\pgfqpoint{4.884704in}{3.604479in}}%
\pgfpathclose%
\pgfusepath{fill}%
\end{pgfscope}%
\begin{pgfscope}%
\pgfpathrectangle{\pgfqpoint{1.020000in}{0.880000in}}{\pgfqpoint{6.160000in}{6.160000in}}%
\pgfusepath{clip}%
\pgfsetbuttcap%
\pgfsetroundjoin%
\definecolor{currentfill}{rgb}{0.613933,0.739923,0.999142}%
\pgfsetfillcolor{currentfill}%
\pgfsetlinewidth{0.000000pt}%
\definecolor{currentstroke}{rgb}{0.000000,0.000000,0.000000}%
\pgfsetstrokecolor{currentstroke}%
\pgfsetdash{}{0pt}%
\pgfpathmoveto{\pgfqpoint{4.224791in}{3.866602in}}%
\pgfpathlineto{\pgfqpoint{4.234887in}{3.851084in}}%
\pgfpathlineto{\pgfqpoint{4.245000in}{3.835099in}}%
\pgfpathlineto{\pgfqpoint{4.277211in}{3.822913in}}%
\pgfpathlineto{\pgfqpoint{4.309396in}{3.811986in}}%
\pgfpathlineto{\pgfqpoint{4.299236in}{3.828044in}}%
\pgfpathlineto{\pgfqpoint{4.289093in}{3.843675in}}%
\pgfpathlineto{\pgfqpoint{4.256955in}{3.854504in}}%
\pgfpathlineto{\pgfqpoint{4.224791in}{3.866602in}}%
\pgfpathclose%
\pgfusepath{fill}%
\end{pgfscope}%
\begin{pgfscope}%
\pgfpathrectangle{\pgfqpoint{1.020000in}{0.880000in}}{\pgfqpoint{6.160000in}{6.160000in}}%
\pgfusepath{clip}%
\pgfsetbuttcap%
\pgfsetroundjoin%
\definecolor{currentfill}{rgb}{0.925563,0.825517,0.771136}%
\pgfsetfillcolor{currentfill}%
\pgfsetlinewidth{0.000000pt}%
\definecolor{currentstroke}{rgb}{0.000000,0.000000,0.000000}%
\pgfsetstrokecolor{currentstroke}%
\pgfsetdash{}{0pt}%
\pgfpathmoveto{\pgfqpoint{3.460261in}{4.496735in}}%
\pgfpathlineto{\pgfqpoint{3.469516in}{4.508458in}}%
\pgfpathlineto{\pgfqpoint{3.478797in}{4.519388in}}%
\pgfpathlineto{\pgfqpoint{3.511532in}{4.478166in}}%
\pgfpathlineto{\pgfqpoint{3.544215in}{4.438009in}}%
\pgfpathlineto{\pgfqpoint{3.534853in}{4.430762in}}%
\pgfpathlineto{\pgfqpoint{3.525516in}{4.422806in}}%
\pgfpathlineto{\pgfqpoint{3.492913in}{4.459291in}}%
\pgfpathlineto{\pgfqpoint{3.460261in}{4.496735in}}%
\pgfpathclose%
\pgfusepath{fill}%
\end{pgfscope}%
\begin{pgfscope}%
\pgfpathrectangle{\pgfqpoint{1.020000in}{0.880000in}}{\pgfqpoint{6.160000in}{6.160000in}}%
\pgfusepath{clip}%
\pgfsetbuttcap%
\pgfsetroundjoin%
\definecolor{currentfill}{rgb}{0.661968,0.775491,0.993937}%
\pgfsetfillcolor{currentfill}%
\pgfsetlinewidth{0.000000pt}%
\definecolor{currentstroke}{rgb}{0.000000,0.000000,0.000000}%
\pgfsetstrokecolor{currentstroke}%
\pgfsetdash{}{0pt}%
\pgfpathmoveto{\pgfqpoint{4.075897in}{3.958921in}}%
\pgfpathlineto{\pgfqpoint{4.085863in}{3.944938in}}%
\pgfpathlineto{\pgfqpoint{4.095847in}{3.930321in}}%
\pgfpathlineto{\pgfqpoint{4.128130in}{3.911823in}}%
\pgfpathlineto{\pgfqpoint{4.160380in}{3.895127in}}%
\pgfpathlineto{\pgfqpoint{4.150349in}{3.909995in}}%
\pgfpathlineto{\pgfqpoint{4.140335in}{3.924340in}}%
\pgfpathlineto{\pgfqpoint{4.108132in}{3.940749in}}%
\pgfpathlineto{\pgfqpoint{4.075897in}{3.958921in}}%
\pgfpathclose%
\pgfusepath{fill}%
\end{pgfscope}%
\begin{pgfscope}%
\pgfpathrectangle{\pgfqpoint{1.020000in}{0.880000in}}{\pgfqpoint{6.160000in}{6.160000in}}%
\pgfusepath{clip}%
\pgfsetbuttcap%
\pgfsetroundjoin%
\definecolor{currentfill}{rgb}{0.570616,0.704109,0.997195}%
\pgfsetfillcolor{currentfill}%
\pgfsetlinewidth{0.000000pt}%
\definecolor{currentstroke}{rgb}{0.000000,0.000000,0.000000}%
\pgfsetstrokecolor{currentstroke}%
\pgfsetdash{}{0pt}%
\pgfpathmoveto{\pgfqpoint{4.373691in}{3.793407in}}%
\pgfpathlineto{\pgfqpoint{4.383915in}{3.776771in}}%
\pgfpathlineto{\pgfqpoint{4.394156in}{3.759631in}}%
\pgfpathlineto{\pgfqpoint{4.426314in}{3.751525in}}%
\pgfpathlineto{\pgfqpoint{4.458449in}{3.744170in}}%
\pgfpathlineto{\pgfqpoint{4.448161in}{3.761595in}}%
\pgfpathlineto{\pgfqpoint{4.437890in}{3.778440in}}%
\pgfpathlineto{\pgfqpoint{4.405802in}{3.785523in}}%
\pgfpathlineto{\pgfqpoint{4.373691in}{3.793407in}}%
\pgfpathclose%
\pgfusepath{fill}%
\end{pgfscope}%
\begin{pgfscope}%
\pgfpathrectangle{\pgfqpoint{1.020000in}{0.880000in}}{\pgfqpoint{6.160000in}{6.160000in}}%
\pgfusepath{clip}%
\pgfsetbuttcap%
\pgfsetroundjoin%
\definecolor{currentfill}{rgb}{0.835345,0.860514,0.898970}%
\pgfsetfillcolor{currentfill}%
\pgfsetlinewidth{0.000000pt}%
\definecolor{currentstroke}{rgb}{0.000000,0.000000,0.000000}%
\pgfsetstrokecolor{currentstroke}%
\pgfsetdash{}{0pt}%
\pgfpathmoveto{\pgfqpoint{3.693521in}{4.290863in}}%
\pgfpathlineto{\pgfqpoint{3.703098in}{4.289024in}}%
\pgfpathlineto{\pgfqpoint{3.712699in}{4.286060in}}%
\pgfpathlineto{\pgfqpoint{3.745256in}{4.248635in}}%
\pgfpathlineto{\pgfqpoint{3.777763in}{4.213393in}}%
\pgfpathlineto{\pgfqpoint{3.768102in}{4.218586in}}%
\pgfpathlineto{\pgfqpoint{3.758464in}{4.222830in}}%
\pgfpathlineto{\pgfqpoint{3.726017in}{4.255833in}}%
\pgfpathlineto{\pgfqpoint{3.693521in}{4.290863in}}%
\pgfpathclose%
\pgfusepath{fill}%
\end{pgfscope}%
\begin{pgfscope}%
\pgfpathrectangle{\pgfqpoint{1.020000in}{0.880000in}}{\pgfqpoint{6.160000in}{6.160000in}}%
\pgfusepath{clip}%
\pgfsetbuttcap%
\pgfsetroundjoin%
\definecolor{currentfill}{rgb}{0.969192,0.705836,0.593704}%
\pgfsetfillcolor{currentfill}%
\pgfsetlinewidth{0.000000pt}%
\definecolor{currentstroke}{rgb}{0.000000,0.000000,0.000000}%
\pgfsetstrokecolor{currentstroke}%
\pgfsetdash{}{0pt}%
\pgfpathmoveto{\pgfqpoint{3.013834in}{4.721524in}}%
\pgfpathlineto{\pgfqpoint{3.022373in}{4.753745in}}%
\pgfpathlineto{\pgfqpoint{3.030923in}{4.786953in}}%
\pgfpathlineto{\pgfqpoint{3.063777in}{4.764866in}}%
\pgfpathlineto{\pgfqpoint{3.096616in}{4.740854in}}%
\pgfpathlineto{\pgfqpoint{3.087962in}{4.710267in}}%
\pgfpathlineto{\pgfqpoint{3.079323in}{4.680595in}}%
\pgfpathlineto{\pgfqpoint{3.046586in}{4.701874in}}%
\pgfpathlineto{\pgfqpoint{3.013834in}{4.721524in}}%
\pgfpathclose%
\pgfusepath{fill}%
\end{pgfscope}%
\begin{pgfscope}%
\pgfpathrectangle{\pgfqpoint{1.020000in}{0.880000in}}{\pgfqpoint{6.160000in}{6.160000in}}%
\pgfusepath{clip}%
\pgfsetbuttcap%
\pgfsetroundjoin%
\definecolor{currentfill}{rgb}{0.969683,0.690484,0.575138}%
\pgfsetfillcolor{currentfill}%
\pgfsetlinewidth{0.000000pt}%
\definecolor{currentstroke}{rgb}{0.000000,0.000000,0.000000}%
\pgfsetstrokecolor{currentstroke}%
\pgfsetdash{}{0pt}%
\pgfpathmoveto{\pgfqpoint{2.800665in}{4.730217in}}%
\pgfpathlineto{\pgfqpoint{2.808918in}{4.762948in}}%
\pgfpathlineto{\pgfqpoint{2.817175in}{4.797269in}}%
\pgfpathlineto{\pgfqpoint{2.849954in}{4.790167in}}%
\pgfpathlineto{\pgfqpoint{2.882736in}{4.780766in}}%
\pgfpathlineto{\pgfqpoint{2.874395in}{4.746809in}}%
\pgfpathlineto{\pgfqpoint{2.866059in}{4.714429in}}%
\pgfpathlineto{\pgfqpoint{2.833363in}{4.723279in}}%
\pgfpathlineto{\pgfqpoint{2.800665in}{4.730217in}}%
\pgfpathclose%
\pgfusepath{fill}%
\end{pgfscope}%
\begin{pgfscope}%
\pgfpathrectangle{\pgfqpoint{1.020000in}{0.880000in}}{\pgfqpoint{6.160000in}{6.160000in}}%
\pgfusepath{clip}%
\pgfsetbuttcap%
\pgfsetroundjoin%
\definecolor{currentfill}{rgb}{0.430507,0.564883,0.948889}%
\pgfsetfillcolor{currentfill}%
\pgfsetlinewidth{0.000000pt}%
\definecolor{currentstroke}{rgb}{0.000000,0.000000,0.000000}%
\pgfsetstrokecolor{currentstroke}%
\pgfsetdash{}{0pt}%
\pgfpathmoveto{\pgfqpoint{5.033831in}{3.553607in}}%
\pgfpathlineto{\pgfqpoint{5.044521in}{3.523674in}}%
\pgfpathlineto{\pgfqpoint{5.055209in}{3.491563in}}%
\pgfpathlineto{\pgfqpoint{5.087230in}{3.498322in}}%
\pgfpathlineto{\pgfqpoint{5.119239in}{3.505672in}}%
\pgfpathlineto{\pgfqpoint{5.108484in}{3.534741in}}%
\pgfpathlineto{\pgfqpoint{5.097731in}{3.562009in}}%
\pgfpathlineto{\pgfqpoint{5.065788in}{3.557604in}}%
\pgfpathlineto{\pgfqpoint{5.033831in}{3.553607in}}%
\pgfpathclose%
\pgfusepath{fill}%
\end{pgfscope}%
\begin{pgfscope}%
\pgfpathrectangle{\pgfqpoint{1.020000in}{0.880000in}}{\pgfqpoint{6.160000in}{6.160000in}}%
\pgfusepath{clip}%
\pgfsetbuttcap%
\pgfsetroundjoin%
\definecolor{currentfill}{rgb}{0.532568,0.669801,0.990393}%
\pgfsetfillcolor{currentfill}%
\pgfsetlinewidth{0.000000pt}%
\definecolor{currentstroke}{rgb}{0.000000,0.000000,0.000000}%
\pgfsetstrokecolor{currentstroke}%
\pgfsetdash{}{0pt}%
\pgfpathmoveto{\pgfqpoint{4.522650in}{3.731390in}}%
\pgfpathlineto{\pgfqpoint{4.532998in}{3.712779in}}%
\pgfpathlineto{\pgfqpoint{4.543361in}{3.693238in}}%
\pgfpathlineto{\pgfqpoint{4.575472in}{3.686993in}}%
\pgfpathlineto{\pgfqpoint{4.607561in}{3.681233in}}%
\pgfpathlineto{\pgfqpoint{4.597155in}{3.701578in}}%
\pgfpathlineto{\pgfqpoint{4.586763in}{3.720789in}}%
\pgfpathlineto{\pgfqpoint{4.554717in}{3.725840in}}%
\pgfpathlineto{\pgfqpoint{4.522650in}{3.731390in}}%
\pgfpathclose%
\pgfusepath{fill}%
\end{pgfscope}%
\begin{pgfscope}%
\pgfpathrectangle{\pgfqpoint{1.020000in}{0.880000in}}{\pgfqpoint{6.160000in}{6.160000in}}%
\pgfusepath{clip}%
\pgfsetbuttcap%
\pgfsetroundjoin%
\definecolor{currentfill}{rgb}{0.728970,0.817464,0.973188}%
\pgfsetfillcolor{currentfill}%
\pgfsetlinewidth{0.000000pt}%
\definecolor{currentstroke}{rgb}{0.000000,0.000000,0.000000}%
\pgfsetstrokecolor{currentstroke}%
\pgfsetdash{}{0pt}%
\pgfpathmoveto{\pgfqpoint{3.926919in}{4.074795in}}%
\pgfpathlineto{\pgfqpoint{3.936750in}{4.063685in}}%
\pgfpathlineto{\pgfqpoint{3.946601in}{4.051684in}}%
\pgfpathlineto{\pgfqpoint{3.978983in}{4.025256in}}%
\pgfpathlineto{\pgfqpoint{4.011325in}{4.001057in}}%
\pgfpathlineto{\pgfqpoint{4.001425in}{4.013817in}}%
\pgfpathlineto{\pgfqpoint{3.991544in}{4.025849in}}%
\pgfpathlineto{\pgfqpoint{3.959252in}{4.049253in}}%
\pgfpathlineto{\pgfqpoint{3.926919in}{4.074795in}}%
\pgfpathclose%
\pgfusepath{fill}%
\end{pgfscope}%
\begin{pgfscope}%
\pgfpathrectangle{\pgfqpoint{1.020000in}{0.880000in}}{\pgfqpoint{6.160000in}{6.160000in}}%
\pgfusepath{clip}%
\pgfsetbuttcap%
\pgfsetroundjoin%
\definecolor{currentfill}{rgb}{0.425199,0.559058,0.946061}%
\pgfsetfillcolor{currentfill}%
\pgfsetlinewidth{0.000000pt}%
\definecolor{currentstroke}{rgb}{0.000000,0.000000,0.000000}%
\pgfsetstrokecolor{currentstroke}%
\pgfsetdash{}{0pt}%
\pgfpathmoveto{\pgfqpoint{5.247119in}{3.535208in}}%
\pgfpathlineto{\pgfqpoint{5.258027in}{3.511361in}}%
\pgfpathlineto{\pgfqpoint{5.268944in}{3.486654in}}%
\pgfpathlineto{\pgfqpoint{5.300940in}{3.496062in}}%
\pgfpathlineto{\pgfqpoint{5.332911in}{3.504449in}}%
\pgfpathlineto{\pgfqpoint{5.321916in}{3.526092in}}%
\pgfpathlineto{\pgfqpoint{5.310934in}{3.547133in}}%
\pgfpathlineto{\pgfqpoint{5.279038in}{3.541519in}}%
\pgfpathlineto{\pgfqpoint{5.247119in}{3.535208in}}%
\pgfpathclose%
\pgfusepath{fill}%
\end{pgfscope}%
\begin{pgfscope}%
\pgfpathrectangle{\pgfqpoint{1.020000in}{0.880000in}}{\pgfqpoint{6.160000in}{6.160000in}}%
\pgfusepath{clip}%
\pgfsetbuttcap%
\pgfsetroundjoin%
\definecolor{currentfill}{rgb}{0.956371,0.775144,0.686416}%
\pgfsetfillcolor{currentfill}%
\pgfsetlinewidth{0.000000pt}%
\definecolor{currentstroke}{rgb}{0.000000,0.000000,0.000000}%
\pgfsetstrokecolor{currentstroke}%
\pgfsetdash{}{0pt}%
\pgfpathmoveto{\pgfqpoint{3.311120in}{4.610534in}}%
\pgfpathlineto{\pgfqpoint{3.320142in}{4.631166in}}%
\pgfpathlineto{\pgfqpoint{3.329188in}{4.651441in}}%
\pgfpathlineto{\pgfqpoint{3.362023in}{4.612551in}}%
\pgfpathlineto{\pgfqpoint{3.394815in}{4.573617in}}%
\pgfpathlineto{\pgfqpoint{3.385674in}{4.557360in}}%
\pgfpathlineto{\pgfqpoint{3.376558in}{4.540760in}}%
\pgfpathlineto{\pgfqpoint{3.343860in}{4.575663in}}%
\pgfpathlineto{\pgfqpoint{3.311120in}{4.610534in}}%
\pgfpathclose%
\pgfusepath{fill}%
\end{pgfscope}%
\begin{pgfscope}%
\pgfpathrectangle{\pgfqpoint{1.020000in}{0.880000in}}{\pgfqpoint{6.160000in}{6.160000in}}%
\pgfusepath{clip}%
\pgfsetbuttcap%
\pgfsetroundjoin%
\definecolor{currentfill}{rgb}{0.419991,0.552989,0.942630}%
\pgfsetfillcolor{currentfill}%
\pgfsetlinewidth{0.000000pt}%
\definecolor{currentstroke}{rgb}{0.000000,0.000000,0.000000}%
\pgfsetstrokecolor{currentstroke}%
\pgfsetdash{}{0pt}%
\pgfpathmoveto{\pgfqpoint{5.460514in}{3.527382in}}%
\pgfpathlineto{\pgfqpoint{5.471665in}{3.509552in}}%
\pgfpathlineto{\pgfqpoint{5.482833in}{3.491567in}}%
\pgfpathlineto{\pgfqpoint{5.514727in}{3.496289in}}%
\pgfpathlineto{\pgfqpoint{5.546592in}{3.500019in}}%
\pgfpathlineto{\pgfqpoint{5.535360in}{3.516777in}}%
\pgfpathlineto{\pgfqpoint{5.524147in}{3.533423in}}%
\pgfpathlineto{\pgfqpoint{5.492344in}{3.530766in}}%
\pgfpathlineto{\pgfqpoint{5.460514in}{3.527382in}}%
\pgfpathclose%
\pgfusepath{fill}%
\end{pgfscope}%
\begin{pgfscope}%
\pgfpathrectangle{\pgfqpoint{1.020000in}{0.880000in}}{\pgfqpoint{6.160000in}{6.160000in}}%
\pgfusepath{clip}%
\pgfsetbuttcap%
\pgfsetroundjoin%
\definecolor{currentfill}{rgb}{0.500031,0.638508,0.981070}%
\pgfsetfillcolor{currentfill}%
\pgfsetlinewidth{0.000000pt}%
\definecolor{currentstroke}{rgb}{0.000000,0.000000,0.000000}%
\pgfsetstrokecolor{currentstroke}%
\pgfsetdash{}{0pt}%
\pgfpathmoveto{\pgfqpoint{4.671673in}{3.671177in}}%
\pgfpathlineto{\pgfqpoint{4.682131in}{3.648509in}}%
\pgfpathlineto{\pgfqpoint{4.692597in}{3.624073in}}%
\pgfpathlineto{\pgfqpoint{4.724660in}{3.618835in}}%
\pgfpathlineto{\pgfqpoint{4.756704in}{3.614321in}}%
\pgfpathlineto{\pgfqpoint{4.746200in}{3.639752in}}%
\pgfpathlineto{\pgfqpoint{4.735703in}{3.663191in}}%
\pgfpathlineto{\pgfqpoint{4.703698in}{3.666915in}}%
\pgfpathlineto{\pgfqpoint{4.671673in}{3.671177in}}%
\pgfpathclose%
\pgfusepath{fill}%
\end{pgfscope}%
\begin{pgfscope}%
\pgfpathrectangle{\pgfqpoint{1.020000in}{0.880000in}}{\pgfqpoint{6.160000in}{6.160000in}}%
\pgfusepath{clip}%
\pgfsetbuttcap%
\pgfsetroundjoin%
\definecolor{currentfill}{rgb}{0.967544,0.730850,0.624685}%
\pgfsetfillcolor{currentfill}%
\pgfsetlinewidth{0.000000pt}%
\definecolor{currentstroke}{rgb}{0.000000,0.000000,0.000000}%
\pgfsetstrokecolor{currentstroke}%
\pgfsetdash{}{0pt}%
\pgfpathmoveto{\pgfqpoint{3.162235in}{4.687939in}}%
\pgfpathlineto{\pgfqpoint{3.171010in}{4.715945in}}%
\pgfpathlineto{\pgfqpoint{3.179804in}{4.744233in}}%
\pgfpathlineto{\pgfqpoint{3.212681in}{4.712243in}}%
\pgfpathlineto{\pgfqpoint{3.245529in}{4.679105in}}%
\pgfpathlineto{\pgfqpoint{3.236631in}{4.654483in}}%
\pgfpathlineto{\pgfqpoint{3.227753in}{4.630095in}}%
\pgfpathlineto{\pgfqpoint{3.195008in}{4.659510in}}%
\pgfpathlineto{\pgfqpoint{3.162235in}{4.687939in}}%
\pgfpathclose%
\pgfusepath{fill}%
\end{pgfscope}%
\begin{pgfscope}%
\pgfpathrectangle{\pgfqpoint{1.020000in}{0.880000in}}{\pgfqpoint{6.160000in}{6.160000in}}%
\pgfusepath{clip}%
\pgfsetbuttcap%
\pgfsetroundjoin%
\definecolor{currentfill}{rgb}{0.462354,0.599830,0.965857}%
\pgfsetfillcolor{currentfill}%
\pgfsetlinewidth{0.000000pt}%
\definecolor{currentstroke}{rgb}{0.000000,0.000000,0.000000}%
\pgfsetstrokecolor{currentstroke}%
\pgfsetdash{}{0pt}%
\pgfpathmoveto{\pgfqpoint{4.820738in}{3.607704in}}%
\pgfpathlineto{\pgfqpoint{4.831284in}{3.579433in}}%
\pgfpathlineto{\pgfqpoint{4.841830in}{3.548622in}}%
\pgfpathlineto{\pgfqpoint{4.873861in}{3.546512in}}%
\pgfpathlineto{\pgfqpoint{4.905879in}{3.545692in}}%
\pgfpathlineto{\pgfqpoint{4.895292in}{3.576353in}}%
\pgfpathlineto{\pgfqpoint{4.884704in}{3.604479in}}%
\pgfpathlineto{\pgfqpoint{4.852729in}{3.605667in}}%
\pgfpathlineto{\pgfqpoint{4.820738in}{3.607704in}}%
\pgfpathclose%
\pgfusepath{fill}%
\end{pgfscope}%
\begin{pgfscope}%
\pgfpathrectangle{\pgfqpoint{1.020000in}{0.880000in}}{\pgfqpoint{6.160000in}{6.160000in}}%
\pgfusepath{clip}%
\pgfsetbuttcap%
\pgfsetroundjoin%
\definecolor{currentfill}{rgb}{0.906154,0.842091,0.806151}%
\pgfsetfillcolor{currentfill}%
\pgfsetlinewidth{0.000000pt}%
\definecolor{currentstroke}{rgb}{0.000000,0.000000,0.000000}%
\pgfsetstrokecolor{currentstroke}%
\pgfsetdash{}{0pt}%
\pgfpathmoveto{\pgfqpoint{3.544215in}{4.438009in}}%
\pgfpathlineto{\pgfqpoint{3.553603in}{4.444338in}}%
\pgfpathlineto{\pgfqpoint{3.563017in}{4.449541in}}%
\pgfpathlineto{\pgfqpoint{3.595723in}{4.407340in}}%
\pgfpathlineto{\pgfqpoint{3.628375in}{4.366726in}}%
\pgfpathlineto{\pgfqpoint{3.618889in}{4.364697in}}%
\pgfpathlineto{\pgfqpoint{3.609428in}{4.361690in}}%
\pgfpathlineto{\pgfqpoint{3.576847in}{4.399124in}}%
\pgfpathlineto{\pgfqpoint{3.544215in}{4.438009in}}%
\pgfpathclose%
\pgfusepath{fill}%
\end{pgfscope}%
\begin{pgfscope}%
\pgfpathrectangle{\pgfqpoint{1.020000in}{0.880000in}}{\pgfqpoint{6.160000in}{6.160000in}}%
\pgfusepath{clip}%
\pgfsetbuttcap%
\pgfsetroundjoin%
\definecolor{currentfill}{rgb}{0.419991,0.552989,0.942630}%
\pgfsetfillcolor{currentfill}%
\pgfsetlinewidth{0.000000pt}%
\definecolor{currentstroke}{rgb}{0.000000,0.000000,0.000000}%
\pgfsetstrokecolor{currentstroke}%
\pgfsetdash{}{0pt}%
\pgfpathmoveto{\pgfqpoint{5.183215in}{3.520926in}}%
\pgfpathlineto{\pgfqpoint{5.194047in}{3.493740in}}%
\pgfpathlineto{\pgfqpoint{5.204884in}{3.465385in}}%
\pgfpathlineto{\pgfqpoint{5.236924in}{3.476358in}}%
\pgfpathlineto{\pgfqpoint{5.268944in}{3.486654in}}%
\pgfpathlineto{\pgfqpoint{5.258027in}{3.511361in}}%
\pgfpathlineto{\pgfqpoint{5.247119in}{3.535208in}}%
\pgfpathlineto{\pgfqpoint{5.215177in}{3.528295in}}%
\pgfpathlineto{\pgfqpoint{5.183215in}{3.520926in}}%
\pgfpathclose%
\pgfusepath{fill}%
\end{pgfscope}%
\begin{pgfscope}%
\pgfpathrectangle{\pgfqpoint{1.020000in}{0.880000in}}{\pgfqpoint{6.160000in}{6.160000in}}%
\pgfusepath{clip}%
\pgfsetbuttcap%
\pgfsetroundjoin%
\definecolor{currentfill}{rgb}{0.800601,0.850358,0.930008}%
\pgfsetfillcolor{currentfill}%
\pgfsetlinewidth{0.000000pt}%
\definecolor{currentstroke}{rgb}{0.000000,0.000000,0.000000}%
\pgfsetstrokecolor{currentstroke}%
\pgfsetdash{}{0pt}%
\pgfpathmoveto{\pgfqpoint{3.777763in}{4.213393in}}%
\pgfpathlineto{\pgfqpoint{3.787446in}{4.207145in}}%
\pgfpathlineto{\pgfqpoint{3.797151in}{4.199744in}}%
\pgfpathlineto{\pgfqpoint{3.829663in}{4.165000in}}%
\pgfpathlineto{\pgfqpoint{3.862127in}{4.132613in}}%
\pgfpathlineto{\pgfqpoint{3.852368in}{4.141569in}}%
\pgfpathlineto{\pgfqpoint{3.842630in}{4.149568in}}%
\pgfpathlineto{\pgfqpoint{3.810220in}{4.180369in}}%
\pgfpathlineto{\pgfqpoint{3.777763in}{4.213393in}}%
\pgfpathclose%
\pgfusepath{fill}%
\end{pgfscope}%
\begin{pgfscope}%
\pgfpathrectangle{\pgfqpoint{1.020000in}{0.880000in}}{\pgfqpoint{6.160000in}{6.160000in}}%
\pgfusepath{clip}%
\pgfsetbuttcap%
\pgfsetroundjoin%
\definecolor{currentfill}{rgb}{0.425199,0.559058,0.946061}%
\pgfsetfillcolor{currentfill}%
\pgfsetlinewidth{0.000000pt}%
\definecolor{currentstroke}{rgb}{0.000000,0.000000,0.000000}%
\pgfsetstrokecolor{currentstroke}%
\pgfsetdash{}{0pt}%
\pgfpathmoveto{\pgfqpoint{4.969879in}{3.547672in}}%
\pgfpathlineto{\pgfqpoint{4.980512in}{3.515611in}}%
\pgfpathlineto{\pgfqpoint{4.991142in}{3.481074in}}%
\pgfpathlineto{\pgfqpoint{5.023180in}{3.485713in}}%
\pgfpathlineto{\pgfqpoint{5.055209in}{3.491563in}}%
\pgfpathlineto{\pgfqpoint{5.044521in}{3.523674in}}%
\pgfpathlineto{\pgfqpoint{5.033831in}{3.553607in}}%
\pgfpathlineto{\pgfqpoint{5.001861in}{3.550230in}}%
\pgfpathlineto{\pgfqpoint{4.969879in}{3.547672in}}%
\pgfpathclose%
\pgfusepath{fill}%
\end{pgfscope}%
\begin{pgfscope}%
\pgfpathrectangle{\pgfqpoint{1.020000in}{0.880000in}}{\pgfqpoint{6.160000in}{6.160000in}}%
\pgfusepath{clip}%
\pgfsetbuttcap%
\pgfsetroundjoin%
\definecolor{currentfill}{rgb}{0.630089,0.752516,0.998508}%
\pgfsetfillcolor{currentfill}%
\pgfsetlinewidth{0.000000pt}%
\definecolor{currentstroke}{rgb}{0.000000,0.000000,0.000000}%
\pgfsetstrokecolor{currentstroke}%
\pgfsetdash{}{0pt}%
\pgfpathmoveto{\pgfqpoint{4.160380in}{3.895127in}}%
\pgfpathlineto{\pgfqpoint{4.170429in}{3.879726in}}%
\pgfpathlineto{\pgfqpoint{4.180495in}{3.863786in}}%
\pgfpathlineto{\pgfqpoint{4.212762in}{3.848678in}}%
\pgfpathlineto{\pgfqpoint{4.245000in}{3.835099in}}%
\pgfpathlineto{\pgfqpoint{4.234887in}{3.851084in}}%
\pgfpathlineto{\pgfqpoint{4.224791in}{3.866602in}}%
\pgfpathlineto{\pgfqpoint{4.192600in}{3.880099in}}%
\pgfpathlineto{\pgfqpoint{4.160380in}{3.895127in}}%
\pgfpathclose%
\pgfusepath{fill}%
\end{pgfscope}%
\begin{pgfscope}%
\pgfpathrectangle{\pgfqpoint{1.020000in}{0.880000in}}{\pgfqpoint{6.160000in}{6.160000in}}%
\pgfusepath{clip}%
\pgfsetbuttcap%
\pgfsetroundjoin%
\definecolor{currentfill}{rgb}{0.586921,0.718121,0.998874}%
\pgfsetfillcolor{currentfill}%
\pgfsetlinewidth{0.000000pt}%
\definecolor{currentstroke}{rgb}{0.000000,0.000000,0.000000}%
\pgfsetstrokecolor{currentstroke}%
\pgfsetdash{}{0pt}%
\pgfpathmoveto{\pgfqpoint{4.309396in}{3.811986in}}%
\pgfpathlineto{\pgfqpoint{4.319574in}{3.795478in}}%
\pgfpathlineto{\pgfqpoint{4.329768in}{3.778497in}}%
\pgfpathlineto{\pgfqpoint{4.361974in}{3.768585in}}%
\pgfpathlineto{\pgfqpoint{4.394156in}{3.759631in}}%
\pgfpathlineto{\pgfqpoint{4.383915in}{3.776771in}}%
\pgfpathlineto{\pgfqpoint{4.373691in}{3.793407in}}%
\pgfpathlineto{\pgfqpoint{4.341556in}{3.802190in}}%
\pgfpathlineto{\pgfqpoint{4.309396in}{3.811986in}}%
\pgfpathclose%
\pgfusepath{fill}%
\end{pgfscope}%
\begin{pgfscope}%
\pgfpathrectangle{\pgfqpoint{1.020000in}{0.880000in}}{\pgfqpoint{6.160000in}{6.160000in}}%
\pgfusepath{clip}%
\pgfsetbuttcap%
\pgfsetroundjoin%
\definecolor{currentfill}{rgb}{0.419991,0.552989,0.942630}%
\pgfsetfillcolor{currentfill}%
\pgfsetlinewidth{0.000000pt}%
\definecolor{currentstroke}{rgb}{0.000000,0.000000,0.000000}%
\pgfsetstrokecolor{currentstroke}%
\pgfsetdash{}{0pt}%
\pgfpathmoveto{\pgfqpoint{5.396769in}{3.517971in}}%
\pgfpathlineto{\pgfqpoint{5.407851in}{3.498358in}}%
\pgfpathlineto{\pgfqpoint{5.418950in}{3.478524in}}%
\pgfpathlineto{\pgfqpoint{5.450907in}{3.485694in}}%
\pgfpathlineto{\pgfqpoint{5.482833in}{3.491567in}}%
\pgfpathlineto{\pgfqpoint{5.471665in}{3.509552in}}%
\pgfpathlineto{\pgfqpoint{5.460514in}{3.527382in}}%
\pgfpathlineto{\pgfqpoint{5.428656in}{3.523153in}}%
\pgfpathlineto{\pgfqpoint{5.396769in}{3.517971in}}%
\pgfpathclose%
\pgfusepath{fill}%
\end{pgfscope}%
\begin{pgfscope}%
\pgfpathrectangle{\pgfqpoint{1.020000in}{0.880000in}}{\pgfqpoint{6.160000in}{6.160000in}}%
\pgfusepath{clip}%
\pgfsetbuttcap%
\pgfsetroundjoin%
\definecolor{currentfill}{rgb}{0.968894,0.679480,0.562812}%
\pgfsetfillcolor{currentfill}%
\pgfsetlinewidth{0.000000pt}%
\definecolor{currentstroke}{rgb}{0.000000,0.000000,0.000000}%
\pgfsetstrokecolor{currentstroke}%
\pgfsetdash{}{0pt}%
\pgfpathmoveto{\pgfqpoint{2.735272in}{4.738322in}}%
\pgfpathlineto{\pgfqpoint{2.743452in}{4.770644in}}%
\pgfpathlineto{\pgfqpoint{2.751633in}{4.804536in}}%
\pgfpathlineto{\pgfqpoint{2.784401in}{4.802054in}}%
\pgfpathlineto{\pgfqpoint{2.817175in}{4.797269in}}%
\pgfpathlineto{\pgfqpoint{2.808918in}{4.762948in}}%
\pgfpathlineto{\pgfqpoint{2.800665in}{4.730217in}}%
\pgfpathlineto{\pgfqpoint{2.767967in}{4.735227in}}%
\pgfpathlineto{\pgfqpoint{2.735272in}{4.738322in}}%
\pgfpathclose%
\pgfusepath{fill}%
\end{pgfscope}%
\begin{pgfscope}%
\pgfpathrectangle{\pgfqpoint{1.020000in}{0.880000in}}{\pgfqpoint{6.160000in}{6.160000in}}%
\pgfusepath{clip}%
\pgfsetbuttcap%
\pgfsetroundjoin%
\definecolor{currentfill}{rgb}{0.548876,0.685104,0.994379}%
\pgfsetfillcolor{currentfill}%
\pgfsetlinewidth{0.000000pt}%
\definecolor{currentstroke}{rgb}{0.000000,0.000000,0.000000}%
\pgfsetstrokecolor{currentstroke}%
\pgfsetdash{}{0pt}%
\pgfpathmoveto{\pgfqpoint{4.458449in}{3.744170in}}%
\pgfpathlineto{\pgfqpoint{4.468752in}{3.726087in}}%
\pgfpathlineto{\pgfqpoint{4.479071in}{3.707256in}}%
\pgfpathlineto{\pgfqpoint{4.511228in}{3.699981in}}%
\pgfpathlineto{\pgfqpoint{4.543361in}{3.693238in}}%
\pgfpathlineto{\pgfqpoint{4.532998in}{3.712779in}}%
\pgfpathlineto{\pgfqpoint{4.522650in}{3.731390in}}%
\pgfpathlineto{\pgfqpoint{4.490561in}{3.737482in}}%
\pgfpathlineto{\pgfqpoint{4.458449in}{3.744170in}}%
\pgfpathclose%
\pgfusepath{fill}%
\end{pgfscope}%
\begin{pgfscope}%
\pgfpathrectangle{\pgfqpoint{1.020000in}{0.880000in}}{\pgfqpoint{6.160000in}{6.160000in}}%
\pgfusepath{clip}%
\pgfsetbuttcap%
\pgfsetroundjoin%
\definecolor{currentfill}{rgb}{0.693321,0.796314,0.986308}%
\pgfsetfillcolor{currentfill}%
\pgfsetlinewidth{0.000000pt}%
\definecolor{currentstroke}{rgb}{0.000000,0.000000,0.000000}%
\pgfsetstrokecolor{currentstroke}%
\pgfsetdash{}{0pt}%
\pgfpathmoveto{\pgfqpoint{4.011325in}{4.001057in}}%
\pgfpathlineto{\pgfqpoint{4.021243in}{3.987539in}}%
\pgfpathlineto{\pgfqpoint{4.031180in}{3.973244in}}%
\pgfpathlineto{\pgfqpoint{4.063531in}{3.950752in}}%
\pgfpathlineto{\pgfqpoint{4.095847in}{3.930321in}}%
\pgfpathlineto{\pgfqpoint{4.085863in}{3.944938in}}%
\pgfpathlineto{\pgfqpoint{4.075897in}{3.958921in}}%
\pgfpathlineto{\pgfqpoint{4.043629in}{3.978983in}}%
\pgfpathlineto{\pgfqpoint{4.011325in}{4.001057in}}%
\pgfpathclose%
\pgfusepath{fill}%
\end{pgfscope}%
\begin{pgfscope}%
\pgfpathrectangle{\pgfqpoint{1.020000in}{0.880000in}}{\pgfqpoint{6.160000in}{6.160000in}}%
\pgfusepath{clip}%
\pgfsetbuttcap%
\pgfsetroundjoin%
\definecolor{currentfill}{rgb}{0.414801,0.546874,0.939088}%
\pgfsetfillcolor{currentfill}%
\pgfsetlinewidth{0.000000pt}%
\definecolor{currentstroke}{rgb}{0.000000,0.000000,0.000000}%
\pgfsetstrokecolor{currentstroke}%
\pgfsetdash{}{0pt}%
\pgfpathmoveto{\pgfqpoint{5.610235in}{3.505147in}}%
\pgfpathlineto{\pgfqpoint{5.621544in}{3.489099in}}%
\pgfpathlineto{\pgfqpoint{5.632872in}{3.473018in}}%
\pgfpathlineto{\pgfqpoint{5.664710in}{3.475240in}}%
\pgfpathlineto{\pgfqpoint{5.653354in}{3.491059in}}%
\pgfpathlineto{\pgfqpoint{5.642017in}{3.506840in}}%
\pgfpathlineto{\pgfqpoint{5.610235in}{3.505147in}}%
\pgfpathclose%
\pgfusepath{fill}%
\end{pgfscope}%
\begin{pgfscope}%
\pgfpathrectangle{\pgfqpoint{1.020000in}{0.880000in}}{\pgfqpoint{6.160000in}{6.160000in}}%
\pgfusepath{clip}%
\pgfsetbuttcap%
\pgfsetroundjoin%
\definecolor{currentfill}{rgb}{0.968894,0.679480,0.562812}%
\pgfsetfillcolor{currentfill}%
\pgfsetlinewidth{0.000000pt}%
\definecolor{currentstroke}{rgb}{0.000000,0.000000,0.000000}%
\pgfsetstrokecolor{currentstroke}%
\pgfsetdash{}{0pt}%
\pgfpathmoveto{\pgfqpoint{2.948297in}{4.755282in}}%
\pgfpathlineto{\pgfqpoint{2.956737in}{4.789412in}}%
\pgfpathlineto{\pgfqpoint{2.965188in}{4.824580in}}%
\pgfpathlineto{\pgfqpoint{2.998059in}{4.806917in}}%
\pgfpathlineto{\pgfqpoint{3.030923in}{4.786953in}}%
\pgfpathlineto{\pgfqpoint{3.022373in}{4.753745in}}%
\pgfpathlineto{\pgfqpoint{3.013834in}{4.721524in}}%
\pgfpathlineto{\pgfqpoint{2.981070in}{4.739376in}}%
\pgfpathlineto{\pgfqpoint{2.948297in}{4.755282in}}%
\pgfpathclose%
\pgfusepath{fill}%
\end{pgfscope}%
\begin{pgfscope}%
\pgfpathrectangle{\pgfqpoint{1.020000in}{0.880000in}}{\pgfqpoint{6.160000in}{6.160000in}}%
\pgfusepath{clip}%
\pgfsetbuttcap%
\pgfsetroundjoin%
\definecolor{currentfill}{rgb}{0.510824,0.649397,0.985079}%
\pgfsetfillcolor{currentfill}%
\pgfsetlinewidth{0.000000pt}%
\definecolor{currentstroke}{rgb}{0.000000,0.000000,0.000000}%
\pgfsetstrokecolor{currentstroke}%
\pgfsetdash{}{0pt}%
\pgfpathmoveto{\pgfqpoint{4.607561in}{3.681233in}}%
\pgfpathlineto{\pgfqpoint{4.617978in}{3.659574in}}%
\pgfpathlineto{\pgfqpoint{4.628407in}{3.636428in}}%
\pgfpathlineto{\pgfqpoint{4.660512in}{3.629962in}}%
\pgfpathlineto{\pgfqpoint{4.692597in}{3.624073in}}%
\pgfpathlineto{\pgfqpoint{4.682131in}{3.648509in}}%
\pgfpathlineto{\pgfqpoint{4.671673in}{3.671177in}}%
\pgfpathlineto{\pgfqpoint{4.639627in}{3.675956in}}%
\pgfpathlineto{\pgfqpoint{4.607561in}{3.681233in}}%
\pgfpathclose%
\pgfusepath{fill}%
\end{pgfscope}%
\begin{pgfscope}%
\pgfpathrectangle{\pgfqpoint{1.020000in}{0.880000in}}{\pgfqpoint{6.160000in}{6.160000in}}%
\pgfusepath{clip}%
\pgfsetbuttcap%
\pgfsetroundjoin%
\definecolor{currentfill}{rgb}{0.409611,0.540759,0.935545}%
\pgfsetfillcolor{currentfill}%
\pgfsetlinewidth{0.000000pt}%
\definecolor{currentstroke}{rgb}{0.000000,0.000000,0.000000}%
\pgfsetstrokecolor{currentstroke}%
\pgfsetdash{}{0pt}%
\pgfpathmoveto{\pgfqpoint{5.119239in}{3.505672in}}%
\pgfpathlineto{\pgfqpoint{5.129995in}{3.474899in}}%
\pgfpathlineto{\pgfqpoint{5.140754in}{3.442626in}}%
\pgfpathlineto{\pgfqpoint{5.172826in}{3.454023in}}%
\pgfpathlineto{\pgfqpoint{5.204884in}{3.465385in}}%
\pgfpathlineto{\pgfqpoint{5.194047in}{3.493740in}}%
\pgfpathlineto{\pgfqpoint{5.183215in}{3.520926in}}%
\pgfpathlineto{\pgfqpoint{5.151235in}{3.513302in}}%
\pgfpathlineto{\pgfqpoint{5.119239in}{3.505672in}}%
\pgfpathclose%
\pgfusepath{fill}%
\end{pgfscope}%
\begin{pgfscope}%
\pgfpathrectangle{\pgfqpoint{1.020000in}{0.880000in}}{\pgfqpoint{6.160000in}{6.160000in}}%
\pgfusepath{clip}%
\pgfsetbuttcap%
\pgfsetroundjoin%
\definecolor{currentfill}{rgb}{0.467678,0.605591,0.968546}%
\pgfsetfillcolor{currentfill}%
\pgfsetlinewidth{0.000000pt}%
\definecolor{currentstroke}{rgb}{0.000000,0.000000,0.000000}%
\pgfsetstrokecolor{currentstroke}%
\pgfsetdash{}{0pt}%
\pgfpathmoveto{\pgfqpoint{4.756704in}{3.614321in}}%
\pgfpathlineto{\pgfqpoint{4.767213in}{3.586676in}}%
\pgfpathlineto{\pgfqpoint{4.777724in}{3.556639in}}%
\pgfpathlineto{\pgfqpoint{4.809785in}{3.552015in}}%
\pgfpathlineto{\pgfqpoint{4.841830in}{3.548622in}}%
\pgfpathlineto{\pgfqpoint{4.831284in}{3.579433in}}%
\pgfpathlineto{\pgfqpoint{4.820738in}{3.607704in}}%
\pgfpathlineto{\pgfqpoint{4.788730in}{3.610596in}}%
\pgfpathlineto{\pgfqpoint{4.756704in}{3.614321in}}%
\pgfpathclose%
\pgfusepath{fill}%
\end{pgfscope}%
\begin{pgfscope}%
\pgfpathrectangle{\pgfqpoint{1.020000in}{0.880000in}}{\pgfqpoint{6.160000in}{6.160000in}}%
\pgfusepath{clip}%
\pgfsetbuttcap%
\pgfsetroundjoin%
\definecolor{currentfill}{rgb}{0.949151,0.790785,0.710876}%
\pgfsetfillcolor{currentfill}%
\pgfsetlinewidth{0.000000pt}%
\definecolor{currentstroke}{rgb}{0.000000,0.000000,0.000000}%
\pgfsetstrokecolor{currentstroke}%
\pgfsetdash{}{0pt}%
\pgfpathmoveto{\pgfqpoint{3.394815in}{4.573617in}}%
\pgfpathlineto{\pgfqpoint{3.403982in}{4.589267in}}%
\pgfpathlineto{\pgfqpoint{3.413176in}{4.604038in}}%
\pgfpathlineto{\pgfqpoint{3.446011in}{4.561434in}}%
\pgfpathlineto{\pgfqpoint{3.478797in}{4.519388in}}%
\pgfpathlineto{\pgfqpoint{3.469516in}{4.508458in}}%
\pgfpathlineto{\pgfqpoint{3.460261in}{4.496735in}}%
\pgfpathlineto{\pgfqpoint{3.427562in}{4.534923in}}%
\pgfpathlineto{\pgfqpoint{3.394815in}{4.573617in}}%
\pgfpathclose%
\pgfusepath{fill}%
\end{pgfscope}%
\begin{pgfscope}%
\pgfpathrectangle{\pgfqpoint{1.020000in}{0.880000in}}{\pgfqpoint{6.160000in}{6.160000in}}%
\pgfusepath{clip}%
\pgfsetbuttcap%
\pgfsetroundjoin%
\definecolor{currentfill}{rgb}{0.430507,0.564883,0.948889}%
\pgfsetfillcolor{currentfill}%
\pgfsetlinewidth{0.000000pt}%
\definecolor{currentstroke}{rgb}{0.000000,0.000000,0.000000}%
\pgfsetstrokecolor{currentstroke}%
\pgfsetdash{}{0pt}%
\pgfpathmoveto{\pgfqpoint{4.905879in}{3.545692in}}%
\pgfpathlineto{\pgfqpoint{4.916464in}{3.512387in}}%
\pgfpathlineto{\pgfqpoint{4.927046in}{3.476440in}}%
\pgfpathlineto{\pgfqpoint{4.959098in}{3.477912in}}%
\pgfpathlineto{\pgfqpoint{4.991142in}{3.481074in}}%
\pgfpathlineto{\pgfqpoint{4.980512in}{3.515611in}}%
\pgfpathlineto{\pgfqpoint{4.969879in}{3.547672in}}%
\pgfpathlineto{\pgfqpoint{4.937885in}{3.546112in}}%
\pgfpathlineto{\pgfqpoint{4.905879in}{3.545692in}}%
\pgfpathclose%
\pgfusepath{fill}%
\end{pgfscope}%
\begin{pgfscope}%
\pgfpathrectangle{\pgfqpoint{1.020000in}{0.880000in}}{\pgfqpoint{6.160000in}{6.160000in}}%
\pgfusepath{clip}%
\pgfsetbuttcap%
\pgfsetroundjoin%
\definecolor{currentfill}{rgb}{0.763363,0.835092,0.955658}%
\pgfsetfillcolor{currentfill}%
\pgfsetlinewidth{0.000000pt}%
\definecolor{currentstroke}{rgb}{0.000000,0.000000,0.000000}%
\pgfsetstrokecolor{currentstroke}%
\pgfsetdash{}{0pt}%
\pgfpathmoveto{\pgfqpoint{3.862127in}{4.132613in}}%
\pgfpathlineto{\pgfqpoint{3.871908in}{4.122629in}}%
\pgfpathlineto{\pgfqpoint{3.881709in}{4.111559in}}%
\pgfpathlineto{\pgfqpoint{3.914177in}{4.080428in}}%
\pgfpathlineto{\pgfqpoint{3.946601in}{4.051684in}}%
\pgfpathlineto{\pgfqpoint{3.936750in}{4.063685in}}%
\pgfpathlineto{\pgfqpoint{3.926919in}{4.074795in}}%
\pgfpathlineto{\pgfqpoint{3.894545in}{4.102559in}}%
\pgfpathlineto{\pgfqpoint{3.862127in}{4.132613in}}%
\pgfpathclose%
\pgfusepath{fill}%
\end{pgfscope}%
\begin{pgfscope}%
\pgfpathrectangle{\pgfqpoint{1.020000in}{0.880000in}}{\pgfqpoint{6.160000in}{6.160000in}}%
\pgfusepath{clip}%
\pgfsetbuttcap%
\pgfsetroundjoin%
\definecolor{currentfill}{rgb}{0.414801,0.546874,0.939088}%
\pgfsetfillcolor{currentfill}%
\pgfsetlinewidth{0.000000pt}%
\definecolor{currentstroke}{rgb}{0.000000,0.000000,0.000000}%
\pgfsetstrokecolor{currentstroke}%
\pgfsetdash{}{0pt}%
\pgfpathmoveto{\pgfqpoint{5.332911in}{3.504449in}}%
\pgfpathlineto{\pgfqpoint{5.343918in}{3.482327in}}%
\pgfpathlineto{\pgfqpoint{5.354941in}{3.459888in}}%
\pgfpathlineto{\pgfqpoint{5.386961in}{3.469944in}}%
\pgfpathlineto{\pgfqpoint{5.418950in}{3.478524in}}%
\pgfpathlineto{\pgfqpoint{5.407851in}{3.498358in}}%
\pgfpathlineto{\pgfqpoint{5.396769in}{3.517971in}}%
\pgfpathlineto{\pgfqpoint{5.364854in}{3.511753in}}%
\pgfpathlineto{\pgfqpoint{5.332911in}{3.504449in}}%
\pgfpathclose%
\pgfusepath{fill}%
\end{pgfscope}%
\begin{pgfscope}%
\pgfpathrectangle{\pgfqpoint{1.020000in}{0.880000in}}{\pgfqpoint{6.160000in}{6.160000in}}%
\pgfusepath{clip}%
\pgfsetbuttcap%
\pgfsetroundjoin%
\definecolor{currentfill}{rgb}{0.875557,0.860242,0.851430}%
\pgfsetfillcolor{currentfill}%
\pgfsetlinewidth{0.000000pt}%
\definecolor{currentstroke}{rgb}{0.000000,0.000000,0.000000}%
\pgfsetstrokecolor{currentstroke}%
\pgfsetdash{}{0pt}%
\pgfpathmoveto{\pgfqpoint{3.628375in}{4.366726in}}%
\pgfpathlineto{\pgfqpoint{3.637887in}{4.367603in}}%
\pgfpathlineto{\pgfqpoint{3.647424in}{4.367160in}}%
\pgfpathlineto{\pgfqpoint{3.680089in}{4.325602in}}%
\pgfpathlineto{\pgfqpoint{3.712699in}{4.286060in}}%
\pgfpathlineto{\pgfqpoint{3.703098in}{4.289024in}}%
\pgfpathlineto{\pgfqpoint{3.693521in}{4.290863in}}%
\pgfpathlineto{\pgfqpoint{3.660974in}{4.327858in}}%
\pgfpathlineto{\pgfqpoint{3.628375in}{4.366726in}}%
\pgfpathclose%
\pgfusepath{fill}%
\end{pgfscope}%
\begin{pgfscope}%
\pgfpathrectangle{\pgfqpoint{1.020000in}{0.880000in}}{\pgfqpoint{6.160000in}{6.160000in}}%
\pgfusepath{clip}%
\pgfsetbuttcap%
\pgfsetroundjoin%
\definecolor{currentfill}{rgb}{0.414801,0.546874,0.939088}%
\pgfsetfillcolor{currentfill}%
\pgfsetlinewidth{0.000000pt}%
\definecolor{currentstroke}{rgb}{0.000000,0.000000,0.000000}%
\pgfsetstrokecolor{currentstroke}%
\pgfsetdash{}{0pt}%
\pgfpathmoveto{\pgfqpoint{5.546592in}{3.500019in}}%
\pgfpathlineto{\pgfqpoint{5.557842in}{3.483194in}}%
\pgfpathlineto{\pgfqpoint{5.569111in}{3.466351in}}%
\pgfpathlineto{\pgfqpoint{5.601007in}{3.470114in}}%
\pgfpathlineto{\pgfqpoint{5.632872in}{3.473018in}}%
\pgfpathlineto{\pgfqpoint{5.621544in}{3.489099in}}%
\pgfpathlineto{\pgfqpoint{5.610235in}{3.505147in}}%
\pgfpathlineto{\pgfqpoint{5.578427in}{3.502920in}}%
\pgfpathlineto{\pgfqpoint{5.546592in}{3.500019in}}%
\pgfpathclose%
\pgfusepath{fill}%
\end{pgfscope}%
\begin{pgfscope}%
\pgfpathrectangle{\pgfqpoint{1.020000in}{0.880000in}}{\pgfqpoint{6.160000in}{6.160000in}}%
\pgfusepath{clip}%
\pgfsetbuttcap%
\pgfsetroundjoin%
\definecolor{currentfill}{rgb}{0.968500,0.673977,0.556649}%
\pgfsetfillcolor{currentfill}%
\pgfsetlinewidth{0.000000pt}%
\definecolor{currentstroke}{rgb}{0.000000,0.000000,0.000000}%
\pgfsetstrokecolor{currentstroke}%
\pgfsetdash{}{0pt}%
\pgfpathmoveto{\pgfqpoint{2.669892in}{4.738928in}}%
\pgfpathlineto{\pgfqpoint{2.678008in}{4.770099in}}%
\pgfpathlineto{\pgfqpoint{2.686125in}{4.802787in}}%
\pgfpathlineto{\pgfqpoint{2.718874in}{4.804756in}}%
\pgfpathlineto{\pgfqpoint{2.751633in}{4.804536in}}%
\pgfpathlineto{\pgfqpoint{2.743452in}{4.770644in}}%
\pgfpathlineto{\pgfqpoint{2.735272in}{4.738322in}}%
\pgfpathlineto{\pgfqpoint{2.702580in}{4.739535in}}%
\pgfpathlineto{\pgfqpoint{2.669892in}{4.738928in}}%
\pgfpathclose%
\pgfusepath{fill}%
\end{pgfscope}%
\begin{pgfscope}%
\pgfpathrectangle{\pgfqpoint{1.020000in}{0.880000in}}{\pgfqpoint{6.160000in}{6.160000in}}%
\pgfusepath{clip}%
\pgfsetbuttcap%
\pgfsetroundjoin%
\definecolor{currentfill}{rgb}{0.969683,0.690484,0.575138}%
\pgfsetfillcolor{currentfill}%
\pgfsetlinewidth{0.000000pt}%
\definecolor{currentstroke}{rgb}{0.000000,0.000000,0.000000}%
\pgfsetstrokecolor{currentstroke}%
\pgfsetdash{}{0pt}%
\pgfpathmoveto{\pgfqpoint{3.096616in}{4.740854in}}%
\pgfpathlineto{\pgfqpoint{3.105286in}{4.772079in}}%
\pgfpathlineto{\pgfqpoint{3.113976in}{4.803629in}}%
\pgfpathlineto{\pgfqpoint{3.146901in}{4.774788in}}%
\pgfpathlineto{\pgfqpoint{3.179804in}{4.744233in}}%
\pgfpathlineto{\pgfqpoint{3.171010in}{4.715945in}}%
\pgfpathlineto{\pgfqpoint{3.162235in}{4.687939in}}%
\pgfpathlineto{\pgfqpoint{3.129437in}{4.715135in}}%
\pgfpathlineto{\pgfqpoint{3.096616in}{4.740854in}}%
\pgfpathclose%
\pgfusepath{fill}%
\end{pgfscope}%
\begin{pgfscope}%
\pgfpathrectangle{\pgfqpoint{1.020000in}{0.880000in}}{\pgfqpoint{6.160000in}{6.160000in}}%
\pgfusepath{clip}%
\pgfsetbuttcap%
\pgfsetroundjoin%
\definecolor{currentfill}{rgb}{0.966962,0.735670,0.630877}%
\pgfsetfillcolor{currentfill}%
\pgfsetlinewidth{0.000000pt}%
\definecolor{currentstroke}{rgb}{0.000000,0.000000,0.000000}%
\pgfsetstrokecolor{currentstroke}%
\pgfsetdash{}{0pt}%
\pgfpathmoveto{\pgfqpoint{3.245529in}{4.679105in}}%
\pgfpathlineto{\pgfqpoint{3.254450in}{4.703672in}}%
\pgfpathlineto{\pgfqpoint{3.263396in}{4.727869in}}%
\pgfpathlineto{\pgfqpoint{3.296311in}{4.689986in}}%
\pgfpathlineto{\pgfqpoint{3.329188in}{4.651441in}}%
\pgfpathlineto{\pgfqpoint{3.320142in}{4.631166in}}%
\pgfpathlineto{\pgfqpoint{3.311120in}{4.610534in}}%
\pgfpathlineto{\pgfqpoint{3.278343in}{4.645107in}}%
\pgfpathlineto{\pgfqpoint{3.245529in}{4.679105in}}%
\pgfpathclose%
\pgfusepath{fill}%
\end{pgfscope}%
\begin{pgfscope}%
\pgfpathrectangle{\pgfqpoint{1.020000in}{0.880000in}}{\pgfqpoint{6.160000in}{6.160000in}}%
\pgfusepath{clip}%
\pgfsetbuttcap%
\pgfsetroundjoin%
\definecolor{currentfill}{rgb}{0.603162,0.731527,0.999565}%
\pgfsetfillcolor{currentfill}%
\pgfsetlinewidth{0.000000pt}%
\definecolor{currentstroke}{rgb}{0.000000,0.000000,0.000000}%
\pgfsetstrokecolor{currentstroke}%
\pgfsetdash{}{0pt}%
\pgfpathmoveto{\pgfqpoint{4.245000in}{3.835099in}}%
\pgfpathlineto{\pgfqpoint{4.255131in}{3.818638in}}%
\pgfpathlineto{\pgfqpoint{4.265279in}{3.801692in}}%
\pgfpathlineto{\pgfqpoint{4.297536in}{3.789490in}}%
\pgfpathlineto{\pgfqpoint{4.329768in}{3.778497in}}%
\pgfpathlineto{\pgfqpoint{4.319574in}{3.795478in}}%
\pgfpathlineto{\pgfqpoint{4.309396in}{3.811986in}}%
\pgfpathlineto{\pgfqpoint{4.277211in}{3.822913in}}%
\pgfpathlineto{\pgfqpoint{4.245000in}{3.835099in}}%
\pgfpathclose%
\pgfusepath{fill}%
\end{pgfscope}%
\begin{pgfscope}%
\pgfpathrectangle{\pgfqpoint{1.020000in}{0.880000in}}{\pgfqpoint{6.160000in}{6.160000in}}%
\pgfusepath{clip}%
\pgfsetbuttcap%
\pgfsetroundjoin%
\definecolor{currentfill}{rgb}{0.404421,0.534643,0.932002}%
\pgfsetfillcolor{currentfill}%
\pgfsetlinewidth{0.000000pt}%
\definecolor{currentstroke}{rgb}{0.000000,0.000000,0.000000}%
\pgfsetstrokecolor{currentstroke}%
\pgfsetdash{}{0pt}%
\pgfpathmoveto{\pgfqpoint{5.055209in}{3.491563in}}%
\pgfpathlineto{\pgfqpoint{5.065896in}{3.457391in}}%
\pgfpathlineto{\pgfqpoint{5.076584in}{3.421406in}}%
\pgfpathlineto{\pgfqpoint{5.108672in}{3.431606in}}%
\pgfpathlineto{\pgfqpoint{5.140754in}{3.442626in}}%
\pgfpathlineto{\pgfqpoint{5.129995in}{3.474899in}}%
\pgfpathlineto{\pgfqpoint{5.119239in}{3.505672in}}%
\pgfpathlineto{\pgfqpoint{5.087230in}{3.498322in}}%
\pgfpathlineto{\pgfqpoint{5.055209in}{3.491563in}}%
\pgfpathclose%
\pgfusepath{fill}%
\end{pgfscope}%
\begin{pgfscope}%
\pgfpathrectangle{\pgfqpoint{1.020000in}{0.880000in}}{\pgfqpoint{6.160000in}{6.160000in}}%
\pgfusepath{clip}%
\pgfsetbuttcap%
\pgfsetroundjoin%
\definecolor{currentfill}{rgb}{0.656683,0.771806,0.994914}%
\pgfsetfillcolor{currentfill}%
\pgfsetlinewidth{0.000000pt}%
\definecolor{currentstroke}{rgb}{0.000000,0.000000,0.000000}%
\pgfsetstrokecolor{currentstroke}%
\pgfsetdash{}{0pt}%
\pgfpathmoveto{\pgfqpoint{4.095847in}{3.930321in}}%
\pgfpathlineto{\pgfqpoint{4.105849in}{3.915059in}}%
\pgfpathlineto{\pgfqpoint{4.115869in}{3.899148in}}%
\pgfpathlineto{\pgfqpoint{4.148198in}{3.880563in}}%
\pgfpathlineto{\pgfqpoint{4.180495in}{3.863786in}}%
\pgfpathlineto{\pgfqpoint{4.170429in}{3.879726in}}%
\pgfpathlineto{\pgfqpoint{4.160380in}{3.895127in}}%
\pgfpathlineto{\pgfqpoint{4.128130in}{3.911823in}}%
\pgfpathlineto{\pgfqpoint{4.095847in}{3.930321in}}%
\pgfpathclose%
\pgfusepath{fill}%
\end{pgfscope}%
\begin{pgfscope}%
\pgfpathrectangle{\pgfqpoint{1.020000in}{0.880000in}}{\pgfqpoint{6.160000in}{6.160000in}}%
\pgfusepath{clip}%
\pgfsetbuttcap%
\pgfsetroundjoin%
\definecolor{currentfill}{rgb}{0.559747,0.694768,0.996075}%
\pgfsetfillcolor{currentfill}%
\pgfsetlinewidth{0.000000pt}%
\definecolor{currentstroke}{rgb}{0.000000,0.000000,0.000000}%
\pgfsetstrokecolor{currentstroke}%
\pgfsetdash{}{0pt}%
\pgfpathmoveto{\pgfqpoint{4.394156in}{3.759631in}}%
\pgfpathlineto{\pgfqpoint{4.404414in}{3.741935in}}%
\pgfpathlineto{\pgfqpoint{4.414688in}{3.723632in}}%
\pgfpathlineto{\pgfqpoint{4.446891in}{3.715116in}}%
\pgfpathlineto{\pgfqpoint{4.479071in}{3.707256in}}%
\pgfpathlineto{\pgfqpoint{4.468752in}{3.726087in}}%
\pgfpathlineto{\pgfqpoint{4.458449in}{3.744170in}}%
\pgfpathlineto{\pgfqpoint{4.426314in}{3.751525in}}%
\pgfpathlineto{\pgfqpoint{4.394156in}{3.759631in}}%
\pgfpathclose%
\pgfusepath{fill}%
\end{pgfscope}%
\begin{pgfscope}%
\pgfpathrectangle{\pgfqpoint{1.020000in}{0.880000in}}{\pgfqpoint{6.160000in}{6.160000in}}%
\pgfusepath{clip}%
\pgfsetbuttcap%
\pgfsetroundjoin%
\definecolor{currentfill}{rgb}{0.409611,0.540759,0.935545}%
\pgfsetfillcolor{currentfill}%
\pgfsetlinewidth{0.000000pt}%
\definecolor{currentstroke}{rgb}{0.000000,0.000000,0.000000}%
\pgfsetstrokecolor{currentstroke}%
\pgfsetdash{}{0pt}%
\pgfpathmoveto{\pgfqpoint{5.268944in}{3.486654in}}%
\pgfpathlineto{\pgfqpoint{5.279871in}{3.461265in}}%
\pgfpathlineto{\pgfqpoint{5.290812in}{3.435437in}}%
\pgfpathlineto{\pgfqpoint{5.322890in}{3.448358in}}%
\pgfpathlineto{\pgfqpoint{5.354941in}{3.459888in}}%
\pgfpathlineto{\pgfqpoint{5.343918in}{3.482327in}}%
\pgfpathlineto{\pgfqpoint{5.332911in}{3.504449in}}%
\pgfpathlineto{\pgfqpoint{5.300940in}{3.496062in}}%
\pgfpathlineto{\pgfqpoint{5.268944in}{3.486654in}}%
\pgfpathclose%
\pgfusepath{fill}%
\end{pgfscope}%
\begin{pgfscope}%
\pgfpathrectangle{\pgfqpoint{1.020000in}{0.880000in}}{\pgfqpoint{6.160000in}{6.160000in}}%
\pgfusepath{clip}%
\pgfsetbuttcap%
\pgfsetroundjoin%
\definecolor{currentfill}{rgb}{0.521696,0.659599,0.987736}%
\pgfsetfillcolor{currentfill}%
\pgfsetlinewidth{0.000000pt}%
\definecolor{currentstroke}{rgb}{0.000000,0.000000,0.000000}%
\pgfsetstrokecolor{currentstroke}%
\pgfsetdash{}{0pt}%
\pgfpathmoveto{\pgfqpoint{4.543361in}{3.693238in}}%
\pgfpathlineto{\pgfqpoint{4.553738in}{3.672635in}}%
\pgfpathlineto{\pgfqpoint{4.564127in}{3.650839in}}%
\pgfpathlineto{\pgfqpoint{4.596278in}{3.643404in}}%
\pgfpathlineto{\pgfqpoint{4.628407in}{3.636428in}}%
\pgfpathlineto{\pgfqpoint{4.617978in}{3.659574in}}%
\pgfpathlineto{\pgfqpoint{4.607561in}{3.681233in}}%
\pgfpathlineto{\pgfqpoint{4.575472in}{3.686993in}}%
\pgfpathlineto{\pgfqpoint{4.543361in}{3.693238in}}%
\pgfpathclose%
\pgfusepath{fill}%
\end{pgfscope}%
\begin{pgfscope}%
\pgfpathrectangle{\pgfqpoint{1.020000in}{0.880000in}}{\pgfqpoint{6.160000in}{6.160000in}}%
\pgfusepath{clip}%
\pgfsetbuttcap%
\pgfsetroundjoin%
\definecolor{currentfill}{rgb}{0.478462,0.616564,0.972721}%
\pgfsetfillcolor{currentfill}%
\pgfsetlinewidth{0.000000pt}%
\definecolor{currentstroke}{rgb}{0.000000,0.000000,0.000000}%
\pgfsetstrokecolor{currentstroke}%
\pgfsetdash{}{0pt}%
\pgfpathmoveto{\pgfqpoint{4.692597in}{3.624073in}}%
\pgfpathlineto{\pgfqpoint{4.703070in}{3.597672in}}%
\pgfpathlineto{\pgfqpoint{4.713549in}{3.569155in}}%
\pgfpathlineto{\pgfqpoint{4.745646in}{3.562396in}}%
\pgfpathlineto{\pgfqpoint{4.777724in}{3.556639in}}%
\pgfpathlineto{\pgfqpoint{4.767213in}{3.586676in}}%
\pgfpathlineto{\pgfqpoint{4.756704in}{3.614321in}}%
\pgfpathlineto{\pgfqpoint{4.724660in}{3.618835in}}%
\pgfpathlineto{\pgfqpoint{4.692597in}{3.624073in}}%
\pgfpathclose%
\pgfusepath{fill}%
\end{pgfscope}%
\begin{pgfscope}%
\pgfpathrectangle{\pgfqpoint{1.020000in}{0.880000in}}{\pgfqpoint{6.160000in}{6.160000in}}%
\pgfusepath{clip}%
\pgfsetbuttcap%
\pgfsetroundjoin%
\definecolor{currentfill}{rgb}{0.430507,0.564883,0.948889}%
\pgfsetfillcolor{currentfill}%
\pgfsetlinewidth{0.000000pt}%
\definecolor{currentstroke}{rgb}{0.000000,0.000000,0.000000}%
\pgfsetstrokecolor{currentstroke}%
\pgfsetdash{}{0pt}%
\pgfpathmoveto{\pgfqpoint{4.841830in}{3.548622in}}%
\pgfpathlineto{\pgfqpoint{4.852376in}{3.515162in}}%
\pgfpathlineto{\pgfqpoint{4.862920in}{3.479057in}}%
\pgfpathlineto{\pgfqpoint{4.894988in}{3.476801in}}%
\pgfpathlineto{\pgfqpoint{4.927046in}{3.476440in}}%
\pgfpathlineto{\pgfqpoint{4.916464in}{3.512387in}}%
\pgfpathlineto{\pgfqpoint{4.905879in}{3.545692in}}%
\pgfpathlineto{\pgfqpoint{4.873861in}{3.546512in}}%
\pgfpathlineto{\pgfqpoint{4.841830in}{3.548622in}}%
\pgfpathclose%
\pgfusepath{fill}%
\end{pgfscope}%
\begin{pgfscope}%
\pgfpathrectangle{\pgfqpoint{1.020000in}{0.880000in}}{\pgfqpoint{6.160000in}{6.160000in}}%
\pgfusepath{clip}%
\pgfsetbuttcap%
\pgfsetroundjoin%
\definecolor{currentfill}{rgb}{0.966922,0.651969,0.531997}%
\pgfsetfillcolor{currentfill}%
\pgfsetlinewidth{0.000000pt}%
\definecolor{currentstroke}{rgb}{0.000000,0.000000,0.000000}%
\pgfsetstrokecolor{currentstroke}%
\pgfsetdash{}{0pt}%
\pgfpathmoveto{\pgfqpoint{2.882736in}{4.780766in}}%
\pgfpathlineto{\pgfqpoint{2.891083in}{4.816067in}}%
\pgfpathlineto{\pgfqpoint{2.899441in}{4.852439in}}%
\pgfpathlineto{\pgfqpoint{2.932315in}{4.839794in}}%
\pgfpathlineto{\pgfqpoint{2.965188in}{4.824580in}}%
\pgfpathlineto{\pgfqpoint{2.956737in}{4.789412in}}%
\pgfpathlineto{\pgfqpoint{2.948297in}{4.755282in}}%
\pgfpathlineto{\pgfqpoint{2.915518in}{4.769112in}}%
\pgfpathlineto{\pgfqpoint{2.882736in}{4.780766in}}%
\pgfpathclose%
\pgfusepath{fill}%
\end{pgfscope}%
\begin{pgfscope}%
\pgfpathrectangle{\pgfqpoint{1.020000in}{0.880000in}}{\pgfqpoint{6.160000in}{6.160000in}}%
\pgfusepath{clip}%
\pgfsetbuttcap%
\pgfsetroundjoin%
\definecolor{currentfill}{rgb}{0.839351,0.861167,0.894494}%
\pgfsetfillcolor{currentfill}%
\pgfsetlinewidth{0.000000pt}%
\definecolor{currentstroke}{rgb}{0.000000,0.000000,0.000000}%
\pgfsetstrokecolor{currentstroke}%
\pgfsetdash{}{0pt}%
\pgfpathmoveto{\pgfqpoint{3.712699in}{4.286060in}}%
\pgfpathlineto{\pgfqpoint{3.722324in}{4.281842in}}%
\pgfpathlineto{\pgfqpoint{3.731973in}{4.276246in}}%
\pgfpathlineto{\pgfqpoint{3.764589in}{4.236839in}}%
\pgfpathlineto{\pgfqpoint{3.797151in}{4.199744in}}%
\pgfpathlineto{\pgfqpoint{3.787446in}{4.207145in}}%
\pgfpathlineto{\pgfqpoint{3.777763in}{4.213393in}}%
\pgfpathlineto{\pgfqpoint{3.745256in}{4.248635in}}%
\pgfpathlineto{\pgfqpoint{3.712699in}{4.286060in}}%
\pgfpathclose%
\pgfusepath{fill}%
\end{pgfscope}%
\begin{pgfscope}%
\pgfpathrectangle{\pgfqpoint{1.020000in}{0.880000in}}{\pgfqpoint{6.160000in}{6.160000in}}%
\pgfusepath{clip}%
\pgfsetbuttcap%
\pgfsetroundjoin%
\definecolor{currentfill}{rgb}{0.724041,0.814910,0.975651}%
\pgfsetfillcolor{currentfill}%
\pgfsetlinewidth{0.000000pt}%
\definecolor{currentstroke}{rgb}{0.000000,0.000000,0.000000}%
\pgfsetstrokecolor{currentstroke}%
\pgfsetdash{}{0pt}%
\pgfpathmoveto{\pgfqpoint{3.946601in}{4.051684in}}%
\pgfpathlineto{\pgfqpoint{3.956471in}{4.038755in}}%
\pgfpathlineto{\pgfqpoint{3.966362in}{4.024868in}}%
\pgfpathlineto{\pgfqpoint{3.998791in}{3.997914in}}%
\pgfpathlineto{\pgfqpoint{4.031180in}{3.973244in}}%
\pgfpathlineto{\pgfqpoint{4.021243in}{3.987539in}}%
\pgfpathlineto{\pgfqpoint{4.011325in}{4.001057in}}%
\pgfpathlineto{\pgfqpoint{3.978983in}{4.025256in}}%
\pgfpathlineto{\pgfqpoint{3.946601in}{4.051684in}}%
\pgfpathclose%
\pgfusepath{fill}%
\end{pgfscope}%
\begin{pgfscope}%
\pgfpathrectangle{\pgfqpoint{1.020000in}{0.880000in}}{\pgfqpoint{6.160000in}{6.160000in}}%
\pgfusepath{clip}%
\pgfsetbuttcap%
\pgfsetroundjoin%
\definecolor{currentfill}{rgb}{0.414801,0.546874,0.939088}%
\pgfsetfillcolor{currentfill}%
\pgfsetlinewidth{0.000000pt}%
\definecolor{currentstroke}{rgb}{0.000000,0.000000,0.000000}%
\pgfsetstrokecolor{currentstroke}%
\pgfsetdash{}{0pt}%
\pgfpathmoveto{\pgfqpoint{5.482833in}{3.491567in}}%
\pgfpathlineto{\pgfqpoint{5.494019in}{3.473505in}}%
\pgfpathlineto{\pgfqpoint{5.505225in}{3.455454in}}%
\pgfpathlineto{\pgfqpoint{5.537185in}{3.461533in}}%
\pgfpathlineto{\pgfqpoint{5.569111in}{3.466351in}}%
\pgfpathlineto{\pgfqpoint{5.557842in}{3.483194in}}%
\pgfpathlineto{\pgfqpoint{5.546592in}{3.500019in}}%
\pgfpathlineto{\pgfqpoint{5.514727in}{3.496289in}}%
\pgfpathlineto{\pgfqpoint{5.482833in}{3.491567in}}%
\pgfpathclose%
\pgfusepath{fill}%
\end{pgfscope}%
\begin{pgfscope}%
\pgfpathrectangle{\pgfqpoint{1.020000in}{0.880000in}}{\pgfqpoint{6.160000in}{6.160000in}}%
\pgfusepath{clip}%
\pgfsetbuttcap%
\pgfsetroundjoin%
\definecolor{currentfill}{rgb}{0.935774,0.812237,0.747156}%
\pgfsetfillcolor{currentfill}%
\pgfsetlinewidth{0.000000pt}%
\definecolor{currentstroke}{rgb}{0.000000,0.000000,0.000000}%
\pgfsetstrokecolor{currentstroke}%
\pgfsetdash{}{0pt}%
\pgfpathmoveto{\pgfqpoint{3.478797in}{4.519388in}}%
\pgfpathlineto{\pgfqpoint{3.488105in}{4.529281in}}%
\pgfpathlineto{\pgfqpoint{3.497442in}{4.537891in}}%
\pgfpathlineto{\pgfqpoint{3.530257in}{4.493135in}}%
\pgfpathlineto{\pgfqpoint{3.563017in}{4.449541in}}%
\pgfpathlineto{\pgfqpoint{3.553603in}{4.444338in}}%
\pgfpathlineto{\pgfqpoint{3.544215in}{4.438009in}}%
\pgfpathlineto{\pgfqpoint{3.511532in}{4.478166in}}%
\pgfpathlineto{\pgfqpoint{3.478797in}{4.519388in}}%
\pgfpathclose%
\pgfusepath{fill}%
\end{pgfscope}%
\begin{pgfscope}%
\pgfpathrectangle{\pgfqpoint{1.020000in}{0.880000in}}{\pgfqpoint{6.160000in}{6.160000in}}%
\pgfusepath{clip}%
\pgfsetbuttcap%
\pgfsetroundjoin%
\definecolor{currentfill}{rgb}{0.399231,0.528528,0.928459}%
\pgfsetfillcolor{currentfill}%
\pgfsetlinewidth{0.000000pt}%
\definecolor{currentstroke}{rgb}{0.000000,0.000000,0.000000}%
\pgfsetstrokecolor{currentstroke}%
\pgfsetdash{}{0pt}%
\pgfpathmoveto{\pgfqpoint{4.991142in}{3.481074in}}%
\pgfpathlineto{\pgfqpoint{5.001769in}{3.444195in}}%
\pgfpathlineto{\pgfqpoint{5.012396in}{3.405257in}}%
\pgfpathlineto{\pgfqpoint{5.044491in}{3.412480in}}%
\pgfpathlineto{\pgfqpoint{5.076584in}{3.421406in}}%
\pgfpathlineto{\pgfqpoint{5.065896in}{3.457391in}}%
\pgfpathlineto{\pgfqpoint{5.055209in}{3.491563in}}%
\pgfpathlineto{\pgfqpoint{5.023180in}{3.485713in}}%
\pgfpathlineto{\pgfqpoint{4.991142in}{3.481074in}}%
\pgfpathclose%
\pgfusepath{fill}%
\end{pgfscope}%
\begin{pgfscope}%
\pgfpathrectangle{\pgfqpoint{1.020000in}{0.880000in}}{\pgfqpoint{6.160000in}{6.160000in}}%
\pgfusepath{clip}%
\pgfsetbuttcap%
\pgfsetroundjoin%
\definecolor{currentfill}{rgb}{0.968500,0.673977,0.556649}%
\pgfsetfillcolor{currentfill}%
\pgfsetlinewidth{0.000000pt}%
\definecolor{currentstroke}{rgb}{0.000000,0.000000,0.000000}%
\pgfsetstrokecolor{currentstroke}%
\pgfsetdash{}{0pt}%
\pgfpathmoveto{\pgfqpoint{2.604530in}{4.732603in}}%
\pgfpathlineto{\pgfqpoint{2.612592in}{4.761934in}}%
\pgfpathlineto{\pgfqpoint{2.620656in}{4.792699in}}%
\pgfpathlineto{\pgfqpoint{2.653385in}{4.798727in}}%
\pgfpathlineto{\pgfqpoint{2.686125in}{4.802787in}}%
\pgfpathlineto{\pgfqpoint{2.678008in}{4.770099in}}%
\pgfpathlineto{\pgfqpoint{2.669892in}{4.738928in}}%
\pgfpathlineto{\pgfqpoint{2.637208in}{4.736582in}}%
\pgfpathlineto{\pgfqpoint{2.604530in}{4.732603in}}%
\pgfpathclose%
\pgfusepath{fill}%
\end{pgfscope}%
\begin{pgfscope}%
\pgfpathrectangle{\pgfqpoint{1.020000in}{0.880000in}}{\pgfqpoint{6.160000in}{6.160000in}}%
\pgfusepath{clip}%
\pgfsetbuttcap%
\pgfsetroundjoin%
\definecolor{currentfill}{rgb}{0.399231,0.528528,0.928459}%
\pgfsetfillcolor{currentfill}%
\pgfsetlinewidth{0.000000pt}%
\definecolor{currentstroke}{rgb}{0.000000,0.000000,0.000000}%
\pgfsetstrokecolor{currentstroke}%
\pgfsetdash{}{0pt}%
\pgfpathmoveto{\pgfqpoint{5.204884in}{3.465385in}}%
\pgfpathlineto{\pgfqpoint{5.215729in}{3.436109in}}%
\pgfpathlineto{\pgfqpoint{5.226585in}{3.406246in}}%
\pgfpathlineto{\pgfqpoint{5.258709in}{3.421306in}}%
\pgfpathlineto{\pgfqpoint{5.290812in}{3.435437in}}%
\pgfpathlineto{\pgfqpoint{5.279871in}{3.461265in}}%
\pgfpathlineto{\pgfqpoint{5.268944in}{3.486654in}}%
\pgfpathlineto{\pgfqpoint{5.236924in}{3.476358in}}%
\pgfpathlineto{\pgfqpoint{5.204884in}{3.465385in}}%
\pgfpathclose%
\pgfusepath{fill}%
\end{pgfscope}%
\begin{pgfscope}%
\pgfpathrectangle{\pgfqpoint{1.020000in}{0.880000in}}{\pgfqpoint{6.160000in}{6.160000in}}%
\pgfusepath{clip}%
\pgfsetbuttcap%
\pgfsetroundjoin%
\definecolor{currentfill}{rgb}{0.624703,0.748318,0.998719}%
\pgfsetfillcolor{currentfill}%
\pgfsetlinewidth{0.000000pt}%
\definecolor{currentstroke}{rgb}{0.000000,0.000000,0.000000}%
\pgfsetstrokecolor{currentstroke}%
\pgfsetdash{}{0pt}%
\pgfpathmoveto{\pgfqpoint{4.180495in}{3.863786in}}%
\pgfpathlineto{\pgfqpoint{4.190579in}{3.847303in}}%
\pgfpathlineto{\pgfqpoint{4.200680in}{3.830279in}}%
\pgfpathlineto{\pgfqpoint{4.232994in}{3.815241in}}%
\pgfpathlineto{\pgfqpoint{4.265279in}{3.801692in}}%
\pgfpathlineto{\pgfqpoint{4.255131in}{3.818638in}}%
\pgfpathlineto{\pgfqpoint{4.245000in}{3.835099in}}%
\pgfpathlineto{\pgfqpoint{4.212762in}{3.848678in}}%
\pgfpathlineto{\pgfqpoint{4.180495in}{3.863786in}}%
\pgfpathclose%
\pgfusepath{fill}%
\end{pgfscope}%
\begin{pgfscope}%
\pgfpathrectangle{\pgfqpoint{1.020000in}{0.880000in}}{\pgfqpoint{6.160000in}{6.160000in}}%
\pgfusepath{clip}%
\pgfsetbuttcap%
\pgfsetroundjoin%
\definecolor{currentfill}{rgb}{0.576051,0.708780,0.997755}%
\pgfsetfillcolor{currentfill}%
\pgfsetlinewidth{0.000000pt}%
\definecolor{currentstroke}{rgb}{0.000000,0.000000,0.000000}%
\pgfsetstrokecolor{currentstroke}%
\pgfsetdash{}{0pt}%
\pgfpathmoveto{\pgfqpoint{4.329768in}{3.778497in}}%
\pgfpathlineto{\pgfqpoint{4.339979in}{3.761018in}}%
\pgfpathlineto{\pgfqpoint{4.350207in}{3.743013in}}%
\pgfpathlineto{\pgfqpoint{4.382460in}{3.732895in}}%
\pgfpathlineto{\pgfqpoint{4.414688in}{3.723632in}}%
\pgfpathlineto{\pgfqpoint{4.404414in}{3.741935in}}%
\pgfpathlineto{\pgfqpoint{4.394156in}{3.759631in}}%
\pgfpathlineto{\pgfqpoint{4.361974in}{3.768585in}}%
\pgfpathlineto{\pgfqpoint{4.329768in}{3.778497in}}%
\pgfpathclose%
\pgfusepath{fill}%
\end{pgfscope}%
\begin{pgfscope}%
\pgfpathrectangle{\pgfqpoint{1.020000in}{0.880000in}}{\pgfqpoint{6.160000in}{6.160000in}}%
\pgfusepath{clip}%
\pgfsetbuttcap%
\pgfsetroundjoin%
\definecolor{currentfill}{rgb}{0.409611,0.540759,0.935545}%
\pgfsetfillcolor{currentfill}%
\pgfsetlinewidth{0.000000pt}%
\definecolor{currentstroke}{rgb}{0.000000,0.000000,0.000000}%
\pgfsetstrokecolor{currentstroke}%
\pgfsetdash{}{0pt}%
\pgfpathmoveto{\pgfqpoint{5.418950in}{3.478524in}}%
\pgfpathlineto{\pgfqpoint{5.430066in}{3.458598in}}%
\pgfpathlineto{\pgfqpoint{5.441202in}{3.438728in}}%
\pgfpathlineto{\pgfqpoint{5.473231in}{3.447914in}}%
\pgfpathlineto{\pgfqpoint{5.505225in}{3.455454in}}%
\pgfpathlineto{\pgfqpoint{5.494019in}{3.473505in}}%
\pgfpathlineto{\pgfqpoint{5.482833in}{3.491567in}}%
\pgfpathlineto{\pgfqpoint{5.450907in}{3.485694in}}%
\pgfpathlineto{\pgfqpoint{5.418950in}{3.478524in}}%
\pgfpathclose%
\pgfusepath{fill}%
\end{pgfscope}%
\begin{pgfscope}%
\pgfpathrectangle{\pgfqpoint{1.020000in}{0.880000in}}{\pgfqpoint{6.160000in}{6.160000in}}%
\pgfusepath{clip}%
\pgfsetbuttcap%
\pgfsetroundjoin%
\definecolor{currentfill}{rgb}{0.800601,0.850358,0.930008}%
\pgfsetfillcolor{currentfill}%
\pgfsetlinewidth{0.000000pt}%
\definecolor{currentstroke}{rgb}{0.000000,0.000000,0.000000}%
\pgfsetstrokecolor{currentstroke}%
\pgfsetdash{}{0pt}%
\pgfpathmoveto{\pgfqpoint{3.797151in}{4.199744in}}%
\pgfpathlineto{\pgfqpoint{3.806879in}{4.191103in}}%
\pgfpathlineto{\pgfqpoint{3.816630in}{4.181144in}}%
\pgfpathlineto{\pgfqpoint{3.849194in}{4.145123in}}%
\pgfpathlineto{\pgfqpoint{3.881709in}{4.111559in}}%
\pgfpathlineto{\pgfqpoint{3.871908in}{4.122629in}}%
\pgfpathlineto{\pgfqpoint{3.862127in}{4.132613in}}%
\pgfpathlineto{\pgfqpoint{3.829663in}{4.165000in}}%
\pgfpathlineto{\pgfqpoint{3.797151in}{4.199744in}}%
\pgfpathclose%
\pgfusepath{fill}%
\end{pgfscope}%
\begin{pgfscope}%
\pgfpathrectangle{\pgfqpoint{1.020000in}{0.880000in}}{\pgfqpoint{6.160000in}{6.160000in}}%
\pgfusepath{clip}%
\pgfsetbuttcap%
\pgfsetroundjoin%
\definecolor{currentfill}{rgb}{0.532568,0.669801,0.990393}%
\pgfsetfillcolor{currentfill}%
\pgfsetlinewidth{0.000000pt}%
\definecolor{currentstroke}{rgb}{0.000000,0.000000,0.000000}%
\pgfsetstrokecolor{currentstroke}%
\pgfsetdash{}{0pt}%
\pgfpathmoveto{\pgfqpoint{4.479071in}{3.707256in}}%
\pgfpathlineto{\pgfqpoint{4.489405in}{3.687589in}}%
\pgfpathlineto{\pgfqpoint{4.499754in}{3.666996in}}%
\pgfpathlineto{\pgfqpoint{4.531952in}{3.658704in}}%
\pgfpathlineto{\pgfqpoint{4.564127in}{3.650839in}}%
\pgfpathlineto{\pgfqpoint{4.553738in}{3.672635in}}%
\pgfpathlineto{\pgfqpoint{4.543361in}{3.693238in}}%
\pgfpathlineto{\pgfqpoint{4.511228in}{3.699981in}}%
\pgfpathlineto{\pgfqpoint{4.479071in}{3.707256in}}%
\pgfpathclose%
\pgfusepath{fill}%
\end{pgfscope}%
\begin{pgfscope}%
\pgfpathrectangle{\pgfqpoint{1.020000in}{0.880000in}}{\pgfqpoint{6.160000in}{6.160000in}}%
\pgfusepath{clip}%
\pgfsetbuttcap%
\pgfsetroundjoin%
\definecolor{currentfill}{rgb}{0.441123,0.576532,0.954545}%
\pgfsetfillcolor{currentfill}%
\pgfsetlinewidth{0.000000pt}%
\definecolor{currentstroke}{rgb}{0.000000,0.000000,0.000000}%
\pgfsetstrokecolor{currentstroke}%
\pgfsetdash{}{0pt}%
\pgfpathmoveto{\pgfqpoint{4.777724in}{3.556639in}}%
\pgfpathlineto{\pgfqpoint{4.788238in}{3.524112in}}%
\pgfpathlineto{\pgfqpoint{4.798752in}{3.489099in}}%
\pgfpathlineto{\pgfqpoint{4.830843in}{3.483189in}}%
\pgfpathlineto{\pgfqpoint{4.862920in}{3.479057in}}%
\pgfpathlineto{\pgfqpoint{4.852376in}{3.515162in}}%
\pgfpathlineto{\pgfqpoint{4.841830in}{3.548622in}}%
\pgfpathlineto{\pgfqpoint{4.809785in}{3.552015in}}%
\pgfpathlineto{\pgfqpoint{4.777724in}{3.556639in}}%
\pgfpathclose%
\pgfusepath{fill}%
\end{pgfscope}%
\begin{pgfscope}%
\pgfpathrectangle{\pgfqpoint{1.020000in}{0.880000in}}{\pgfqpoint{6.160000in}{6.160000in}}%
\pgfusepath{clip}%
\pgfsetbuttcap%
\pgfsetroundjoin%
\definecolor{currentfill}{rgb}{0.383662,0.510183,0.917831}%
\pgfsetfillcolor{currentfill}%
\pgfsetlinewidth{0.000000pt}%
\definecolor{currentstroke}{rgb}{0.000000,0.000000,0.000000}%
\pgfsetstrokecolor{currentstroke}%
\pgfsetdash{}{0pt}%
\pgfpathmoveto{\pgfqpoint{5.140754in}{3.442626in}}%
\pgfpathlineto{\pgfqpoint{5.151519in}{3.409174in}}%
\pgfpathlineto{\pgfqpoint{5.162292in}{3.374977in}}%
\pgfpathlineto{\pgfqpoint{5.194444in}{3.390644in}}%
\pgfpathlineto{\pgfqpoint{5.226585in}{3.406246in}}%
\pgfpathlineto{\pgfqpoint{5.215729in}{3.436109in}}%
\pgfpathlineto{\pgfqpoint{5.204884in}{3.465385in}}%
\pgfpathlineto{\pgfqpoint{5.172826in}{3.454023in}}%
\pgfpathlineto{\pgfqpoint{5.140754in}{3.442626in}}%
\pgfpathclose%
\pgfusepath{fill}%
\end{pgfscope}%
\begin{pgfscope}%
\pgfpathrectangle{\pgfqpoint{1.020000in}{0.880000in}}{\pgfqpoint{6.160000in}{6.160000in}}%
\pgfusepath{clip}%
\pgfsetbuttcap%
\pgfsetroundjoin%
\definecolor{currentfill}{rgb}{0.963772,0.749086,0.649420}%
\pgfsetfillcolor{currentfill}%
\pgfsetlinewidth{0.000000pt}%
\definecolor{currentstroke}{rgb}{0.000000,0.000000,0.000000}%
\pgfsetstrokecolor{currentstroke}%
\pgfsetdash{}{0pt}%
\pgfpathmoveto{\pgfqpoint{3.329188in}{4.651441in}}%
\pgfpathlineto{\pgfqpoint{3.338261in}{4.671061in}}%
\pgfpathlineto{\pgfqpoint{3.347363in}{4.689715in}}%
\pgfpathlineto{\pgfqpoint{3.380292in}{4.646904in}}%
\pgfpathlineto{\pgfqpoint{3.413176in}{4.604038in}}%
\pgfpathlineto{\pgfqpoint{3.403982in}{4.589267in}}%
\pgfpathlineto{\pgfqpoint{3.394815in}{4.573617in}}%
\pgfpathlineto{\pgfqpoint{3.362023in}{4.612551in}}%
\pgfpathlineto{\pgfqpoint{3.329188in}{4.651441in}}%
\pgfpathclose%
\pgfusepath{fill}%
\end{pgfscope}%
\begin{pgfscope}%
\pgfpathrectangle{\pgfqpoint{1.020000in}{0.880000in}}{\pgfqpoint{6.160000in}{6.160000in}}%
\pgfusepath{clip}%
\pgfsetbuttcap%
\pgfsetroundjoin%
\definecolor{currentfill}{rgb}{0.968894,0.679480,0.562812}%
\pgfsetfillcolor{currentfill}%
\pgfsetlinewidth{0.000000pt}%
\definecolor{currentstroke}{rgb}{0.000000,0.000000,0.000000}%
\pgfsetstrokecolor{currentstroke}%
\pgfsetdash{}{0pt}%
\pgfpathmoveto{\pgfqpoint{2.539184in}{4.720251in}}%
\pgfpathlineto{\pgfqpoint{2.547204in}{4.747138in}}%
\pgfpathlineto{\pgfqpoint{2.555225in}{4.775352in}}%
\pgfpathlineto{\pgfqpoint{2.587936in}{4.784852in}}%
\pgfpathlineto{\pgfqpoint{2.620656in}{4.792699in}}%
\pgfpathlineto{\pgfqpoint{2.612592in}{4.761934in}}%
\pgfpathlineto{\pgfqpoint{2.604530in}{4.732603in}}%
\pgfpathlineto{\pgfqpoint{2.571855in}{4.727113in}}%
\pgfpathlineto{\pgfqpoint{2.539184in}{4.720251in}}%
\pgfpathclose%
\pgfusepath{fill}%
\end{pgfscope}%
\begin{pgfscope}%
\pgfpathrectangle{\pgfqpoint{1.020000in}{0.880000in}}{\pgfqpoint{6.160000in}{6.160000in}}%
\pgfusepath{clip}%
\pgfsetbuttcap%
\pgfsetroundjoin%
\definecolor{currentfill}{rgb}{0.966922,0.651969,0.531997}%
\pgfsetfillcolor{currentfill}%
\pgfsetlinewidth{0.000000pt}%
\definecolor{currentstroke}{rgb}{0.000000,0.000000,0.000000}%
\pgfsetstrokecolor{currentstroke}%
\pgfsetdash{}{0pt}%
\pgfpathmoveto{\pgfqpoint{3.030923in}{4.786953in}}%
\pgfpathlineto{\pgfqpoint{3.039490in}{4.820852in}}%
\pgfpathlineto{\pgfqpoint{3.048077in}{4.855110in}}%
\pgfpathlineto{\pgfqpoint{3.081034in}{4.830487in}}%
\pgfpathlineto{\pgfqpoint{3.113976in}{4.803629in}}%
\pgfpathlineto{\pgfqpoint{3.105286in}{4.772079in}}%
\pgfpathlineto{\pgfqpoint{3.096616in}{4.740854in}}%
\pgfpathlineto{\pgfqpoint{3.063777in}{4.764866in}}%
\pgfpathlineto{\pgfqpoint{3.030923in}{4.786953in}}%
\pgfpathclose%
\pgfusepath{fill}%
\end{pgfscope}%
\begin{pgfscope}%
\pgfpathrectangle{\pgfqpoint{1.020000in}{0.880000in}}{\pgfqpoint{6.160000in}{6.160000in}}%
\pgfusepath{clip}%
\pgfsetbuttcap%
\pgfsetroundjoin%
\definecolor{currentfill}{rgb}{0.489246,0.627536,0.976896}%
\pgfsetfillcolor{currentfill}%
\pgfsetlinewidth{0.000000pt}%
\definecolor{currentstroke}{rgb}{0.000000,0.000000,0.000000}%
\pgfsetstrokecolor{currentstroke}%
\pgfsetdash{}{0pt}%
\pgfpathmoveto{\pgfqpoint{4.628407in}{3.636428in}}%
\pgfpathlineto{\pgfqpoint{4.638845in}{3.611637in}}%
\pgfpathlineto{\pgfqpoint{4.649293in}{3.585081in}}%
\pgfpathlineto{\pgfqpoint{4.681432in}{3.576769in}}%
\pgfpathlineto{\pgfqpoint{4.713549in}{3.569155in}}%
\pgfpathlineto{\pgfqpoint{4.703070in}{3.597672in}}%
\pgfpathlineto{\pgfqpoint{4.692597in}{3.624073in}}%
\pgfpathlineto{\pgfqpoint{4.660512in}{3.629962in}}%
\pgfpathlineto{\pgfqpoint{4.628407in}{3.636428in}}%
\pgfpathclose%
\pgfusepath{fill}%
\end{pgfscope}%
\begin{pgfscope}%
\pgfpathrectangle{\pgfqpoint{1.020000in}{0.880000in}}{\pgfqpoint{6.160000in}{6.160000in}}%
\pgfusepath{clip}%
\pgfsetbuttcap%
\pgfsetroundjoin%
\definecolor{currentfill}{rgb}{0.394042,0.522413,0.924916}%
\pgfsetfillcolor{currentfill}%
\pgfsetlinewidth{0.000000pt}%
\definecolor{currentstroke}{rgb}{0.000000,0.000000,0.000000}%
\pgfsetstrokecolor{currentstroke}%
\pgfsetdash{}{0pt}%
\pgfpathmoveto{\pgfqpoint{4.927046in}{3.476440in}}%
\pgfpathlineto{\pgfqpoint{4.937626in}{3.437995in}}%
\pgfpathlineto{\pgfqpoint{4.948204in}{3.397355in}}%
\pgfpathlineto{\pgfqpoint{4.980300in}{3.400116in}}%
\pgfpathlineto{\pgfqpoint{5.012396in}{3.405257in}}%
\pgfpathlineto{\pgfqpoint{5.001769in}{3.444195in}}%
\pgfpathlineto{\pgfqpoint{4.991142in}{3.481074in}}%
\pgfpathlineto{\pgfqpoint{4.959098in}{3.477912in}}%
\pgfpathlineto{\pgfqpoint{4.927046in}{3.476440in}}%
\pgfpathclose%
\pgfusepath{fill}%
\end{pgfscope}%
\begin{pgfscope}%
\pgfpathrectangle{\pgfqpoint{1.020000in}{0.880000in}}{\pgfqpoint{6.160000in}{6.160000in}}%
\pgfusepath{clip}%
\pgfsetbuttcap%
\pgfsetroundjoin%
\definecolor{currentfill}{rgb}{0.683056,0.790043,0.989768}%
\pgfsetfillcolor{currentfill}%
\pgfsetlinewidth{0.000000pt}%
\definecolor{currentstroke}{rgb}{0.000000,0.000000,0.000000}%
\pgfsetstrokecolor{currentstroke}%
\pgfsetdash{}{0pt}%
\pgfpathmoveto{\pgfqpoint{4.031180in}{3.973244in}}%
\pgfpathlineto{\pgfqpoint{4.041136in}{3.958158in}}%
\pgfpathlineto{\pgfqpoint{4.051110in}{3.942276in}}%
\pgfpathlineto{\pgfqpoint{4.083507in}{3.919676in}}%
\pgfpathlineto{\pgfqpoint{4.115869in}{3.899148in}}%
\pgfpathlineto{\pgfqpoint{4.105849in}{3.915059in}}%
\pgfpathlineto{\pgfqpoint{4.095847in}{3.930321in}}%
\pgfpathlineto{\pgfqpoint{4.063531in}{3.950752in}}%
\pgfpathlineto{\pgfqpoint{4.031180in}{3.973244in}}%
\pgfpathclose%
\pgfusepath{fill}%
\end{pgfscope}%
\begin{pgfscope}%
\pgfpathrectangle{\pgfqpoint{1.020000in}{0.880000in}}{\pgfqpoint{6.160000in}{6.160000in}}%
\pgfusepath{clip}%
\pgfsetbuttcap%
\pgfsetroundjoin%
\definecolor{currentfill}{rgb}{0.404421,0.534643,0.932002}%
\pgfsetfillcolor{currentfill}%
\pgfsetlinewidth{0.000000pt}%
\definecolor{currentstroke}{rgb}{0.000000,0.000000,0.000000}%
\pgfsetstrokecolor{currentstroke}%
\pgfsetdash{}{0pt}%
\pgfpathmoveto{\pgfqpoint{5.632872in}{3.473018in}}%
\pgfpathlineto{\pgfqpoint{5.644221in}{3.456934in}}%
\pgfpathlineto{\pgfqpoint{5.655589in}{3.440877in}}%
\pgfpathlineto{\pgfqpoint{5.687482in}{3.443575in}}%
\pgfpathlineto{\pgfqpoint{5.676086in}{3.459405in}}%
\pgfpathlineto{\pgfqpoint{5.664710in}{3.475240in}}%
\pgfpathlineto{\pgfqpoint{5.632872in}{3.473018in}}%
\pgfpathclose%
\pgfusepath{fill}%
\end{pgfscope}%
\begin{pgfscope}%
\pgfpathrectangle{\pgfqpoint{1.020000in}{0.880000in}}{\pgfqpoint{6.160000in}{6.160000in}}%
\pgfusepath{clip}%
\pgfsetbuttcap%
\pgfsetroundjoin%
\definecolor{currentfill}{rgb}{0.912765,0.836682,0.794512}%
\pgfsetfillcolor{currentfill}%
\pgfsetlinewidth{0.000000pt}%
\definecolor{currentstroke}{rgb}{0.000000,0.000000,0.000000}%
\pgfsetstrokecolor{currentstroke}%
\pgfsetdash{}{0pt}%
\pgfpathmoveto{\pgfqpoint{3.563017in}{4.449541in}}%
\pgfpathlineto{\pgfqpoint{3.572459in}{4.453412in}}%
\pgfpathlineto{\pgfqpoint{3.581929in}{4.455749in}}%
\pgfpathlineto{\pgfqpoint{3.614705in}{4.410600in}}%
\pgfpathlineto{\pgfqpoint{3.647424in}{4.367160in}}%
\pgfpathlineto{\pgfqpoint{3.637887in}{4.367603in}}%
\pgfpathlineto{\pgfqpoint{3.628375in}{4.366726in}}%
\pgfpathlineto{\pgfqpoint{3.595723in}{4.407340in}}%
\pgfpathlineto{\pgfqpoint{3.563017in}{4.449541in}}%
\pgfpathclose%
\pgfusepath{fill}%
\end{pgfscope}%
\begin{pgfscope}%
\pgfpathrectangle{\pgfqpoint{1.020000in}{0.880000in}}{\pgfqpoint{6.160000in}{6.160000in}}%
\pgfusepath{clip}%
\pgfsetbuttcap%
\pgfsetroundjoin%
\definecolor{currentfill}{rgb}{0.963806,0.634188,0.513721}%
\pgfsetfillcolor{currentfill}%
\pgfsetlinewidth{0.000000pt}%
\definecolor{currentstroke}{rgb}{0.000000,0.000000,0.000000}%
\pgfsetstrokecolor{currentstroke}%
\pgfsetdash{}{0pt}%
\pgfpathmoveto{\pgfqpoint{2.817175in}{4.797269in}}%
\pgfpathlineto{\pgfqpoint{2.825437in}{4.832943in}}%
\pgfpathlineto{\pgfqpoint{2.833709in}{4.869695in}}%
\pgfpathlineto{\pgfqpoint{2.866572in}{4.862424in}}%
\pgfpathlineto{\pgfqpoint{2.899441in}{4.852439in}}%
\pgfpathlineto{\pgfqpoint{2.891083in}{4.816067in}}%
\pgfpathlineto{\pgfqpoint{2.882736in}{4.780766in}}%
\pgfpathlineto{\pgfqpoint{2.849954in}{4.790167in}}%
\pgfpathlineto{\pgfqpoint{2.817175in}{4.797269in}}%
\pgfpathclose%
\pgfusepath{fill}%
\end{pgfscope}%
\begin{pgfscope}%
\pgfpathrectangle{\pgfqpoint{1.020000in}{0.880000in}}{\pgfqpoint{6.160000in}{6.160000in}}%
\pgfusepath{clip}%
\pgfsetbuttcap%
\pgfsetroundjoin%
\definecolor{currentfill}{rgb}{0.969683,0.690484,0.575138}%
\pgfsetfillcolor{currentfill}%
\pgfsetlinewidth{0.000000pt}%
\definecolor{currentstroke}{rgb}{0.000000,0.000000,0.000000}%
\pgfsetstrokecolor{currentstroke}%
\pgfsetdash{}{0pt}%
\pgfpathmoveto{\pgfqpoint{3.179804in}{4.744233in}}%
\pgfpathlineto{\pgfqpoint{3.188623in}{4.772485in}}%
\pgfpathlineto{\pgfqpoint{3.197467in}{4.800355in}}%
\pgfpathlineto{\pgfqpoint{3.230447in}{4.764768in}}%
\pgfpathlineto{\pgfqpoint{3.263396in}{4.727869in}}%
\pgfpathlineto{\pgfqpoint{3.254450in}{4.703672in}}%
\pgfpathlineto{\pgfqpoint{3.245529in}{4.679105in}}%
\pgfpathlineto{\pgfqpoint{3.212681in}{4.712243in}}%
\pgfpathlineto{\pgfqpoint{3.179804in}{4.744233in}}%
\pgfpathclose%
\pgfusepath{fill}%
\end{pgfscope}%
\begin{pgfscope}%
\pgfpathrectangle{\pgfqpoint{1.020000in}{0.880000in}}{\pgfqpoint{6.160000in}{6.160000in}}%
\pgfusepath{clip}%
\pgfsetbuttcap%
\pgfsetroundjoin%
\definecolor{currentfill}{rgb}{0.399231,0.528528,0.928459}%
\pgfsetfillcolor{currentfill}%
\pgfsetlinewidth{0.000000pt}%
\definecolor{currentstroke}{rgb}{0.000000,0.000000,0.000000}%
\pgfsetstrokecolor{currentstroke}%
\pgfsetdash{}{0pt}%
\pgfpathmoveto{\pgfqpoint{5.354941in}{3.459888in}}%
\pgfpathlineto{\pgfqpoint{5.365980in}{3.437337in}}%
\pgfpathlineto{\pgfqpoint{5.377039in}{3.414907in}}%
\pgfpathlineto{\pgfqpoint{5.409138in}{3.427753in}}%
\pgfpathlineto{\pgfqpoint{5.441202in}{3.438728in}}%
\pgfpathlineto{\pgfqpoint{5.430066in}{3.458598in}}%
\pgfpathlineto{\pgfqpoint{5.418950in}{3.478524in}}%
\pgfpathlineto{\pgfqpoint{5.386961in}{3.469944in}}%
\pgfpathlineto{\pgfqpoint{5.354941in}{3.459888in}}%
\pgfpathclose%
\pgfusepath{fill}%
\end{pgfscope}%
\begin{pgfscope}%
\pgfpathrectangle{\pgfqpoint{1.020000in}{0.880000in}}{\pgfqpoint{6.160000in}{6.160000in}}%
\pgfusepath{clip}%
\pgfsetbuttcap%
\pgfsetroundjoin%
\definecolor{currentfill}{rgb}{0.373552,0.497499,0.909467}%
\pgfsetfillcolor{currentfill}%
\pgfsetlinewidth{0.000000pt}%
\definecolor{currentstroke}{rgb}{0.000000,0.000000,0.000000}%
\pgfsetstrokecolor{currentstroke}%
\pgfsetdash{}{0pt}%
\pgfpathmoveto{\pgfqpoint{5.076584in}{3.421406in}}%
\pgfpathlineto{\pgfqpoint{5.087274in}{3.383999in}}%
\pgfpathlineto{\pgfqpoint{5.097972in}{3.345700in}}%
\pgfpathlineto{\pgfqpoint{5.130133in}{3.359797in}}%
\pgfpathlineto{\pgfqpoint{5.162292in}{3.374977in}}%
\pgfpathlineto{\pgfqpoint{5.151519in}{3.409174in}}%
\pgfpathlineto{\pgfqpoint{5.140754in}{3.442626in}}%
\pgfpathlineto{\pgfqpoint{5.108672in}{3.431606in}}%
\pgfpathlineto{\pgfqpoint{5.076584in}{3.421406in}}%
\pgfpathclose%
\pgfusepath{fill}%
\end{pgfscope}%
\begin{pgfscope}%
\pgfpathrectangle{\pgfqpoint{1.020000in}{0.880000in}}{\pgfqpoint{6.160000in}{6.160000in}}%
\pgfusepath{clip}%
\pgfsetbuttcap%
\pgfsetroundjoin%
\definecolor{currentfill}{rgb}{0.969289,0.684982,0.568975}%
\pgfsetfillcolor{currentfill}%
\pgfsetlinewidth{0.000000pt}%
\definecolor{currentstroke}{rgb}{0.000000,0.000000,0.000000}%
\pgfsetstrokecolor{currentstroke}%
\pgfsetdash{}{0pt}%
\pgfpathmoveto{\pgfqpoint{2.473847in}{4.703041in}}%
\pgfpathlineto{\pgfqpoint{2.481834in}{4.726991in}}%
\pgfpathlineto{\pgfqpoint{2.489823in}{4.752138in}}%
\pgfpathlineto{\pgfqpoint{2.522521in}{4.764382in}}%
\pgfpathlineto{\pgfqpoint{2.555225in}{4.775352in}}%
\pgfpathlineto{\pgfqpoint{2.547204in}{4.747138in}}%
\pgfpathlineto{\pgfqpoint{2.539184in}{4.720251in}}%
\pgfpathlineto{\pgfqpoint{2.506515in}{4.712172in}}%
\pgfpathlineto{\pgfqpoint{2.473847in}{4.703041in}}%
\pgfpathclose%
\pgfusepath{fill}%
\end{pgfscope}%
\begin{pgfscope}%
\pgfpathrectangle{\pgfqpoint{1.020000in}{0.880000in}}{\pgfqpoint{6.160000in}{6.160000in}}%
\pgfusepath{clip}%
\pgfsetbuttcap%
\pgfsetroundjoin%
\definecolor{currentfill}{rgb}{0.758539,0.832787,0.958408}%
\pgfsetfillcolor{currentfill}%
\pgfsetlinewidth{0.000000pt}%
\definecolor{currentstroke}{rgb}{0.000000,0.000000,0.000000}%
\pgfsetstrokecolor{currentstroke}%
\pgfsetdash{}{0pt}%
\pgfpathmoveto{\pgfqpoint{3.881709in}{4.111559in}}%
\pgfpathlineto{\pgfqpoint{3.891531in}{4.099352in}}%
\pgfpathlineto{\pgfqpoint{3.901374in}{4.085972in}}%
\pgfpathlineto{\pgfqpoint{3.933890in}{4.054197in}}%
\pgfpathlineto{\pgfqpoint{3.966362in}{4.024868in}}%
\pgfpathlineto{\pgfqpoint{3.956471in}{4.038755in}}%
\pgfpathlineto{\pgfqpoint{3.946601in}{4.051684in}}%
\pgfpathlineto{\pgfqpoint{3.914177in}{4.080428in}}%
\pgfpathlineto{\pgfqpoint{3.881709in}{4.111559in}}%
\pgfpathclose%
\pgfusepath{fill}%
\end{pgfscope}%
\begin{pgfscope}%
\pgfpathrectangle{\pgfqpoint{1.020000in}{0.880000in}}{\pgfqpoint{6.160000in}{6.160000in}}%
\pgfusepath{clip}%
\pgfsetbuttcap%
\pgfsetroundjoin%
\definecolor{currentfill}{rgb}{0.404421,0.534643,0.932002}%
\pgfsetfillcolor{currentfill}%
\pgfsetlinewidth{0.000000pt}%
\definecolor{currentstroke}{rgb}{0.000000,0.000000,0.000000}%
\pgfsetstrokecolor{currentstroke}%
\pgfsetdash{}{0pt}%
\pgfpathmoveto{\pgfqpoint{5.569111in}{3.466351in}}%
\pgfpathlineto{\pgfqpoint{5.580402in}{3.449547in}}%
\pgfpathlineto{\pgfqpoint{5.591713in}{3.432836in}}%
\pgfpathlineto{\pgfqpoint{5.623667in}{3.437367in}}%
\pgfpathlineto{\pgfqpoint{5.655589in}{3.440877in}}%
\pgfpathlineto{\pgfqpoint{5.644221in}{3.456934in}}%
\pgfpathlineto{\pgfqpoint{5.632872in}{3.473018in}}%
\pgfpathlineto{\pgfqpoint{5.601007in}{3.470114in}}%
\pgfpathlineto{\pgfqpoint{5.569111in}{3.466351in}}%
\pgfpathclose%
\pgfusepath{fill}%
\end{pgfscope}%
\begin{pgfscope}%
\pgfpathrectangle{\pgfqpoint{1.020000in}{0.880000in}}{\pgfqpoint{6.160000in}{6.160000in}}%
\pgfusepath{clip}%
\pgfsetbuttcap%
\pgfsetroundjoin%
\definecolor{currentfill}{rgb}{0.399231,0.528528,0.928459}%
\pgfsetfillcolor{currentfill}%
\pgfsetlinewidth{0.000000pt}%
\definecolor{currentstroke}{rgb}{0.000000,0.000000,0.000000}%
\pgfsetstrokecolor{currentstroke}%
\pgfsetdash{}{0pt}%
\pgfpathmoveto{\pgfqpoint{4.862920in}{3.479057in}}%
\pgfpathlineto{\pgfqpoint{4.873463in}{3.440452in}}%
\pgfpathlineto{\pgfqpoint{4.884006in}{3.399650in}}%
\pgfpathlineto{\pgfqpoint{4.916107in}{3.397173in}}%
\pgfpathlineto{\pgfqpoint{4.948204in}{3.397355in}}%
\pgfpathlineto{\pgfqpoint{4.937626in}{3.437995in}}%
\pgfpathlineto{\pgfqpoint{4.927046in}{3.476440in}}%
\pgfpathlineto{\pgfqpoint{4.894988in}{3.476801in}}%
\pgfpathlineto{\pgfqpoint{4.862920in}{3.479057in}}%
\pgfpathclose%
\pgfusepath{fill}%
\end{pgfscope}%
\begin{pgfscope}%
\pgfpathrectangle{\pgfqpoint{1.020000in}{0.880000in}}{\pgfqpoint{6.160000in}{6.160000in}}%
\pgfusepath{clip}%
\pgfsetbuttcap%
\pgfsetroundjoin%
\definecolor{currentfill}{rgb}{0.592356,0.722792,0.999434}%
\pgfsetfillcolor{currentfill}%
\pgfsetlinewidth{0.000000pt}%
\definecolor{currentstroke}{rgb}{0.000000,0.000000,0.000000}%
\pgfsetstrokecolor{currentstroke}%
\pgfsetdash{}{0pt}%
\pgfpathmoveto{\pgfqpoint{4.265279in}{3.801692in}}%
\pgfpathlineto{\pgfqpoint{4.275443in}{3.784253in}}%
\pgfpathlineto{\pgfqpoint{4.285625in}{3.766313in}}%
\pgfpathlineto{\pgfqpoint{4.317929in}{3.754108in}}%
\pgfpathlineto{\pgfqpoint{4.350207in}{3.743013in}}%
\pgfpathlineto{\pgfqpoint{4.339979in}{3.761018in}}%
\pgfpathlineto{\pgfqpoint{4.329768in}{3.778497in}}%
\pgfpathlineto{\pgfqpoint{4.297536in}{3.789490in}}%
\pgfpathlineto{\pgfqpoint{4.265279in}{3.801692in}}%
\pgfpathclose%
\pgfusepath{fill}%
\end{pgfscope}%
\begin{pgfscope}%
\pgfpathrectangle{\pgfqpoint{1.020000in}{0.880000in}}{\pgfqpoint{6.160000in}{6.160000in}}%
\pgfusepath{clip}%
\pgfsetbuttcap%
\pgfsetroundjoin%
\definecolor{currentfill}{rgb}{0.883687,0.856108,0.840258}%
\pgfsetfillcolor{currentfill}%
\pgfsetlinewidth{0.000000pt}%
\definecolor{currentstroke}{rgb}{0.000000,0.000000,0.000000}%
\pgfsetstrokecolor{currentstroke}%
\pgfsetdash{}{0pt}%
\pgfpathmoveto{\pgfqpoint{3.647424in}{4.367160in}}%
\pgfpathlineto{\pgfqpoint{3.656988in}{4.365238in}}%
\pgfpathlineto{\pgfqpoint{3.666579in}{4.361686in}}%
\pgfpathlineto{\pgfqpoint{3.699304in}{4.317898in}}%
\pgfpathlineto{\pgfqpoint{3.731973in}{4.276246in}}%
\pgfpathlineto{\pgfqpoint{3.722324in}{4.281842in}}%
\pgfpathlineto{\pgfqpoint{3.712699in}{4.286060in}}%
\pgfpathlineto{\pgfqpoint{3.680089in}{4.325602in}}%
\pgfpathlineto{\pgfqpoint{3.647424in}{4.367160in}}%
\pgfpathclose%
\pgfusepath{fill}%
\end{pgfscope}%
\begin{pgfscope}%
\pgfpathrectangle{\pgfqpoint{1.020000in}{0.880000in}}{\pgfqpoint{6.160000in}{6.160000in}}%
\pgfusepath{clip}%
\pgfsetbuttcap%
\pgfsetroundjoin%
\definecolor{currentfill}{rgb}{0.451739,0.588181,0.960201}%
\pgfsetfillcolor{currentfill}%
\pgfsetlinewidth{0.000000pt}%
\definecolor{currentstroke}{rgb}{0.000000,0.000000,0.000000}%
\pgfsetstrokecolor{currentstroke}%
\pgfsetdash{}{0pt}%
\pgfpathmoveto{\pgfqpoint{4.713549in}{3.569155in}}%
\pgfpathlineto{\pgfqpoint{4.724034in}{3.538437in}}%
\pgfpathlineto{\pgfqpoint{4.734523in}{3.505525in}}%
\pgfpathlineto{\pgfqpoint{4.766646in}{3.496618in}}%
\pgfpathlineto{\pgfqpoint{4.798752in}{3.489099in}}%
\pgfpathlineto{\pgfqpoint{4.788238in}{3.524112in}}%
\pgfpathlineto{\pgfqpoint{4.777724in}{3.556639in}}%
\pgfpathlineto{\pgfqpoint{4.745646in}{3.562396in}}%
\pgfpathlineto{\pgfqpoint{4.713549in}{3.569155in}}%
\pgfpathclose%
\pgfusepath{fill}%
\end{pgfscope}%
\begin{pgfscope}%
\pgfpathrectangle{\pgfqpoint{1.020000in}{0.880000in}}{\pgfqpoint{6.160000in}{6.160000in}}%
\pgfusepath{clip}%
\pgfsetbuttcap%
\pgfsetroundjoin%
\definecolor{currentfill}{rgb}{0.548876,0.685104,0.994379}%
\pgfsetfillcolor{currentfill}%
\pgfsetlinewidth{0.000000pt}%
\definecolor{currentstroke}{rgb}{0.000000,0.000000,0.000000}%
\pgfsetstrokecolor{currentstroke}%
\pgfsetdash{}{0pt}%
\pgfpathmoveto{\pgfqpoint{4.414688in}{3.723632in}}%
\pgfpathlineto{\pgfqpoint{4.424977in}{3.704665in}}%
\pgfpathlineto{\pgfqpoint{4.435283in}{3.684983in}}%
\pgfpathlineto{\pgfqpoint{4.467531in}{3.675738in}}%
\pgfpathlineto{\pgfqpoint{4.499754in}{3.666996in}}%
\pgfpathlineto{\pgfqpoint{4.489405in}{3.687589in}}%
\pgfpathlineto{\pgfqpoint{4.479071in}{3.707256in}}%
\pgfpathlineto{\pgfqpoint{4.446891in}{3.715116in}}%
\pgfpathlineto{\pgfqpoint{4.414688in}{3.723632in}}%
\pgfpathclose%
\pgfusepath{fill}%
\end{pgfscope}%
\begin{pgfscope}%
\pgfpathrectangle{\pgfqpoint{1.020000in}{0.880000in}}{\pgfqpoint{6.160000in}{6.160000in}}%
\pgfusepath{clip}%
\pgfsetbuttcap%
\pgfsetroundjoin%
\definecolor{currentfill}{rgb}{0.388852,0.516298,0.921373}%
\pgfsetfillcolor{currentfill}%
\pgfsetlinewidth{0.000000pt}%
\definecolor{currentstroke}{rgb}{0.000000,0.000000,0.000000}%
\pgfsetstrokecolor{currentstroke}%
\pgfsetdash{}{0pt}%
\pgfpathmoveto{\pgfqpoint{5.290812in}{3.435437in}}%
\pgfpathlineto{\pgfqpoint{5.301768in}{3.409470in}}%
\pgfpathlineto{\pgfqpoint{5.312745in}{3.383709in}}%
\pgfpathlineto{\pgfqpoint{5.344907in}{3.400190in}}%
\pgfpathlineto{\pgfqpoint{5.377039in}{3.414907in}}%
\pgfpathlineto{\pgfqpoint{5.365980in}{3.437337in}}%
\pgfpathlineto{\pgfqpoint{5.354941in}{3.459888in}}%
\pgfpathlineto{\pgfqpoint{5.322890in}{3.448358in}}%
\pgfpathlineto{\pgfqpoint{5.290812in}{3.435437in}}%
\pgfpathclose%
\pgfusepath{fill}%
\end{pgfscope}%
\begin{pgfscope}%
\pgfpathrectangle{\pgfqpoint{1.020000in}{0.880000in}}{\pgfqpoint{6.160000in}{6.160000in}}%
\pgfusepath{clip}%
\pgfsetbuttcap%
\pgfsetroundjoin%
\definecolor{currentfill}{rgb}{0.646113,0.764436,0.996868}%
\pgfsetfillcolor{currentfill}%
\pgfsetlinewidth{0.000000pt}%
\definecolor{currentstroke}{rgb}{0.000000,0.000000,0.000000}%
\pgfsetstrokecolor{currentstroke}%
\pgfsetdash{}{0pt}%
\pgfpathmoveto{\pgfqpoint{4.115869in}{3.899148in}}%
\pgfpathlineto{\pgfqpoint{4.125907in}{3.882589in}}%
\pgfpathlineto{\pgfqpoint{4.135963in}{3.865391in}}%
\pgfpathlineto{\pgfqpoint{4.168337in}{3.846948in}}%
\pgfpathlineto{\pgfqpoint{4.200680in}{3.830279in}}%
\pgfpathlineto{\pgfqpoint{4.190579in}{3.847303in}}%
\pgfpathlineto{\pgfqpoint{4.180495in}{3.863786in}}%
\pgfpathlineto{\pgfqpoint{4.148198in}{3.880563in}}%
\pgfpathlineto{\pgfqpoint{4.115869in}{3.899148in}}%
\pgfpathclose%
\pgfusepath{fill}%
\end{pgfscope}%
\begin{pgfscope}%
\pgfpathrectangle{\pgfqpoint{1.020000in}{0.880000in}}{\pgfqpoint{6.160000in}{6.160000in}}%
\pgfusepath{clip}%
\pgfsetbuttcap%
\pgfsetroundjoin%
\definecolor{currentfill}{rgb}{0.505423,0.643995,0.983157}%
\pgfsetfillcolor{currentfill}%
\pgfsetlinewidth{0.000000pt}%
\definecolor{currentstroke}{rgb}{0.000000,0.000000,0.000000}%
\pgfsetstrokecolor{currentstroke}%
\pgfsetdash{}{0pt}%
\pgfpathmoveto{\pgfqpoint{4.564127in}{3.650839in}}%
\pgfpathlineto{\pgfqpoint{4.574529in}{3.627735in}}%
\pgfpathlineto{\pgfqpoint{4.584943in}{3.603233in}}%
\pgfpathlineto{\pgfqpoint{4.617130in}{3.593944in}}%
\pgfpathlineto{\pgfqpoint{4.649293in}{3.585081in}}%
\pgfpathlineto{\pgfqpoint{4.638845in}{3.611637in}}%
\pgfpathlineto{\pgfqpoint{4.628407in}{3.636428in}}%
\pgfpathlineto{\pgfqpoint{4.596278in}{3.643404in}}%
\pgfpathlineto{\pgfqpoint{4.564127in}{3.650839in}}%
\pgfpathclose%
\pgfusepath{fill}%
\end{pgfscope}%
\begin{pgfscope}%
\pgfpathrectangle{\pgfqpoint{1.020000in}{0.880000in}}{\pgfqpoint{6.160000in}{6.160000in}}%
\pgfusepath{clip}%
\pgfsetbuttcap%
\pgfsetroundjoin%
\definecolor{currentfill}{rgb}{0.363461,0.484784,0.901019}%
\pgfsetfillcolor{currentfill}%
\pgfsetlinewidth{0.000000pt}%
\definecolor{currentstroke}{rgb}{0.000000,0.000000,0.000000}%
\pgfsetstrokecolor{currentstroke}%
\pgfsetdash{}{0pt}%
\pgfpathmoveto{\pgfqpoint{5.012396in}{3.405257in}}%
\pgfpathlineto{\pgfqpoint{5.023024in}{3.364706in}}%
\pgfpathlineto{\pgfqpoint{5.033658in}{3.323149in}}%
\pgfpathlineto{\pgfqpoint{5.065812in}{3.333292in}}%
\pgfpathlineto{\pgfqpoint{5.097972in}{3.345700in}}%
\pgfpathlineto{\pgfqpoint{5.087274in}{3.383999in}}%
\pgfpathlineto{\pgfqpoint{5.076584in}{3.421406in}}%
\pgfpathlineto{\pgfqpoint{5.044491in}{3.412480in}}%
\pgfpathlineto{\pgfqpoint{5.012396in}{3.405257in}}%
\pgfpathclose%
\pgfusepath{fill}%
\end{pgfscope}%
\begin{pgfscope}%
\pgfpathrectangle{\pgfqpoint{1.020000in}{0.880000in}}{\pgfqpoint{6.160000in}{6.160000in}}%
\pgfusepath{clip}%
\pgfsetbuttcap%
\pgfsetroundjoin%
\definecolor{currentfill}{rgb}{0.958176,0.771234,0.680301}%
\pgfsetfillcolor{currentfill}%
\pgfsetlinewidth{0.000000pt}%
\definecolor{currentstroke}{rgb}{0.000000,0.000000,0.000000}%
\pgfsetstrokecolor{currentstroke}%
\pgfsetdash{}{0pt}%
\pgfpathmoveto{\pgfqpoint{3.413176in}{4.604038in}}%
\pgfpathlineto{\pgfqpoint{3.422400in}{4.617648in}}%
\pgfpathlineto{\pgfqpoint{3.431655in}{4.629812in}}%
\pgfpathlineto{\pgfqpoint{3.464575in}{4.583548in}}%
\pgfpathlineto{\pgfqpoint{3.497442in}{4.537891in}}%
\pgfpathlineto{\pgfqpoint{3.488105in}{4.529281in}}%
\pgfpathlineto{\pgfqpoint{3.478797in}{4.519388in}}%
\pgfpathlineto{\pgfqpoint{3.446011in}{4.561434in}}%
\pgfpathlineto{\pgfqpoint{3.413176in}{4.604038in}}%
\pgfpathclose%
\pgfusepath{fill}%
\end{pgfscope}%
\begin{pgfscope}%
\pgfpathrectangle{\pgfqpoint{1.020000in}{0.880000in}}{\pgfqpoint{6.160000in}{6.160000in}}%
\pgfusepath{clip}%
\pgfsetbuttcap%
\pgfsetroundjoin%
\definecolor{currentfill}{rgb}{0.961595,0.622247,0.501551}%
\pgfsetfillcolor{currentfill}%
\pgfsetlinewidth{0.000000pt}%
\definecolor{currentstroke}{rgb}{0.000000,0.000000,0.000000}%
\pgfsetstrokecolor{currentstroke}%
\pgfsetdash{}{0pt}%
\pgfpathmoveto{\pgfqpoint{2.751633in}{4.804536in}}%
\pgfpathlineto{\pgfqpoint{2.759820in}{4.839762in}}%
\pgfpathlineto{\pgfqpoint{2.768016in}{4.876049in}}%
\pgfpathlineto{\pgfqpoint{2.800857in}{4.874232in}}%
\pgfpathlineto{\pgfqpoint{2.833709in}{4.869695in}}%
\pgfpathlineto{\pgfqpoint{2.825437in}{4.832943in}}%
\pgfpathlineto{\pgfqpoint{2.817175in}{4.797269in}}%
\pgfpathlineto{\pgfqpoint{2.784401in}{4.802054in}}%
\pgfpathlineto{\pgfqpoint{2.751633in}{4.804536in}}%
\pgfpathclose%
\pgfusepath{fill}%
\end{pgfscope}%
\begin{pgfscope}%
\pgfpathrectangle{\pgfqpoint{1.020000in}{0.880000in}}{\pgfqpoint{6.160000in}{6.160000in}}%
\pgfusepath{clip}%
\pgfsetbuttcap%
\pgfsetroundjoin%
\definecolor{currentfill}{rgb}{0.969851,0.695830,0.581312}%
\pgfsetfillcolor{currentfill}%
\pgfsetlinewidth{0.000000pt}%
\definecolor{currentstroke}{rgb}{0.000000,0.000000,0.000000}%
\pgfsetstrokecolor{currentstroke}%
\pgfsetdash{}{0pt}%
\pgfpathmoveto{\pgfqpoint{2.408505in}{4.682315in}}%
\pgfpathlineto{\pgfqpoint{2.416469in}{4.702962in}}%
\pgfpathlineto{\pgfqpoint{2.424436in}{4.724663in}}%
\pgfpathlineto{\pgfqpoint{2.457129in}{4.738828in}}%
\pgfpathlineto{\pgfqpoint{2.489823in}{4.752138in}}%
\pgfpathlineto{\pgfqpoint{2.481834in}{4.726991in}}%
\pgfpathlineto{\pgfqpoint{2.473847in}{4.703041in}}%
\pgfpathlineto{\pgfqpoint{2.441178in}{4.693029in}}%
\pgfpathlineto{\pgfqpoint{2.408505in}{4.682315in}}%
\pgfpathclose%
\pgfusepath{fill}%
\end{pgfscope}%
\begin{pgfscope}%
\pgfpathrectangle{\pgfqpoint{1.020000in}{0.880000in}}{\pgfqpoint{6.160000in}{6.160000in}}%
\pgfusepath{clip}%
\pgfsetbuttcap%
\pgfsetroundjoin%
\definecolor{currentfill}{rgb}{0.960490,0.616276,0.495467}%
\pgfsetfillcolor{currentfill}%
\pgfsetlinewidth{0.000000pt}%
\definecolor{currentstroke}{rgb}{0.000000,0.000000,0.000000}%
\pgfsetstrokecolor{currentstroke}%
\pgfsetdash{}{0pt}%
\pgfpathmoveto{\pgfqpoint{2.965188in}{4.824580in}}%
\pgfpathlineto{\pgfqpoint{2.973656in}{4.860479in}}%
\pgfpathlineto{\pgfqpoint{2.982142in}{4.896762in}}%
\pgfpathlineto{\pgfqpoint{3.015112in}{4.877270in}}%
\pgfpathlineto{\pgfqpoint{3.048077in}{4.855110in}}%
\pgfpathlineto{\pgfqpoint{3.039490in}{4.820852in}}%
\pgfpathlineto{\pgfqpoint{3.030923in}{4.786953in}}%
\pgfpathlineto{\pgfqpoint{2.998059in}{4.806917in}}%
\pgfpathlineto{\pgfqpoint{2.965188in}{4.824580in}}%
\pgfpathclose%
\pgfusepath{fill}%
\end{pgfscope}%
\begin{pgfscope}%
\pgfpathrectangle{\pgfqpoint{1.020000in}{0.880000in}}{\pgfqpoint{6.160000in}{6.160000in}}%
\pgfusepath{clip}%
\pgfsetbuttcap%
\pgfsetroundjoin%
\definecolor{currentfill}{rgb}{0.404421,0.534643,0.932002}%
\pgfsetfillcolor{currentfill}%
\pgfsetlinewidth{0.000000pt}%
\definecolor{currentstroke}{rgb}{0.000000,0.000000,0.000000}%
\pgfsetstrokecolor{currentstroke}%
\pgfsetdash{}{0pt}%
\pgfpathmoveto{\pgfqpoint{5.505225in}{3.455454in}}%
\pgfpathlineto{\pgfqpoint{5.516452in}{3.437512in}}%
\pgfpathlineto{\pgfqpoint{5.527701in}{3.419775in}}%
\pgfpathlineto{\pgfqpoint{5.559725in}{3.427053in}}%
\pgfpathlineto{\pgfqpoint{5.591713in}{3.432836in}}%
\pgfpathlineto{\pgfqpoint{5.580402in}{3.449547in}}%
\pgfpathlineto{\pgfqpoint{5.569111in}{3.466351in}}%
\pgfpathlineto{\pgfqpoint{5.537185in}{3.461533in}}%
\pgfpathlineto{\pgfqpoint{5.505225in}{3.455454in}}%
\pgfpathclose%
\pgfusepath{fill}%
\end{pgfscope}%
\begin{pgfscope}%
\pgfpathrectangle{\pgfqpoint{1.020000in}{0.880000in}}{\pgfqpoint{6.160000in}{6.160000in}}%
\pgfusepath{clip}%
\pgfsetbuttcap%
\pgfsetroundjoin%
\definecolor{currentfill}{rgb}{0.718985,0.811993,0.977656}%
\pgfsetfillcolor{currentfill}%
\pgfsetlinewidth{0.000000pt}%
\definecolor{currentstroke}{rgb}{0.000000,0.000000,0.000000}%
\pgfsetstrokecolor{currentstroke}%
\pgfsetdash{}{0pt}%
\pgfpathmoveto{\pgfqpoint{3.966362in}{4.024868in}}%
\pgfpathlineto{\pgfqpoint{3.976271in}{4.010007in}}%
\pgfpathlineto{\pgfqpoint{3.986200in}{3.994164in}}%
\pgfpathlineto{\pgfqpoint{4.018674in}{3.967070in}}%
\pgfpathlineto{\pgfqpoint{4.051110in}{3.942276in}}%
\pgfpathlineto{\pgfqpoint{4.041136in}{3.958158in}}%
\pgfpathlineto{\pgfqpoint{4.031180in}{3.973244in}}%
\pgfpathlineto{\pgfqpoint{3.998791in}{3.997914in}}%
\pgfpathlineto{\pgfqpoint{3.966362in}{4.024868in}}%
\pgfpathclose%
\pgfusepath{fill}%
\end{pgfscope}%
\begin{pgfscope}%
\pgfpathrectangle{\pgfqpoint{1.020000in}{0.880000in}}{\pgfqpoint{6.160000in}{6.160000in}}%
\pgfusepath{clip}%
\pgfsetbuttcap%
\pgfsetroundjoin%
\definecolor{currentfill}{rgb}{0.373552,0.497499,0.909467}%
\pgfsetfillcolor{currentfill}%
\pgfsetlinewidth{0.000000pt}%
\definecolor{currentstroke}{rgb}{0.000000,0.000000,0.000000}%
\pgfsetstrokecolor{currentstroke}%
\pgfsetdash{}{0pt}%
\pgfpathmoveto{\pgfqpoint{5.226585in}{3.406246in}}%
\pgfpathlineto{\pgfqpoint{5.237456in}{3.376213in}}%
\pgfpathlineto{\pgfqpoint{5.248346in}{3.346489in}}%
\pgfpathlineto{\pgfqpoint{5.280556in}{3.365689in}}%
\pgfpathlineto{\pgfqpoint{5.312745in}{3.383709in}}%
\pgfpathlineto{\pgfqpoint{5.301768in}{3.409470in}}%
\pgfpathlineto{\pgfqpoint{5.290812in}{3.435437in}}%
\pgfpathlineto{\pgfqpoint{5.258709in}{3.421306in}}%
\pgfpathlineto{\pgfqpoint{5.226585in}{3.406246in}}%
\pgfpathclose%
\pgfusepath{fill}%
\end{pgfscope}%
\begin{pgfscope}%
\pgfpathrectangle{\pgfqpoint{1.020000in}{0.880000in}}{\pgfqpoint{6.160000in}{6.160000in}}%
\pgfusepath{clip}%
\pgfsetbuttcap%
\pgfsetroundjoin%
\definecolor{currentfill}{rgb}{0.843358,0.861820,0.890017}%
\pgfsetfillcolor{currentfill}%
\pgfsetlinewidth{0.000000pt}%
\definecolor{currentstroke}{rgb}{0.000000,0.000000,0.000000}%
\pgfsetstrokecolor{currentstroke}%
\pgfsetdash{}{0pt}%
\pgfpathmoveto{\pgfqpoint{3.731973in}{4.276246in}}%
\pgfpathlineto{\pgfqpoint{3.741648in}{4.269164in}}%
\pgfpathlineto{\pgfqpoint{3.751346in}{4.260498in}}%
\pgfpathlineto{\pgfqpoint{3.784015in}{4.219615in}}%
\pgfpathlineto{\pgfqpoint{3.816630in}{4.181144in}}%
\pgfpathlineto{\pgfqpoint{3.806879in}{4.191103in}}%
\pgfpathlineto{\pgfqpoint{3.797151in}{4.199744in}}%
\pgfpathlineto{\pgfqpoint{3.764589in}{4.236839in}}%
\pgfpathlineto{\pgfqpoint{3.731973in}{4.276246in}}%
\pgfpathclose%
\pgfusepath{fill}%
\end{pgfscope}%
\begin{pgfscope}%
\pgfpathrectangle{\pgfqpoint{1.020000in}{0.880000in}}{\pgfqpoint{6.160000in}{6.160000in}}%
\pgfusepath{clip}%
\pgfsetbuttcap%
\pgfsetroundjoin%
\definecolor{currentfill}{rgb}{0.968863,0.710838,0.599901}%
\pgfsetfillcolor{currentfill}%
\pgfsetlinewidth{0.000000pt}%
\definecolor{currentstroke}{rgb}{0.000000,0.000000,0.000000}%
\pgfsetstrokecolor{currentstroke}%
\pgfsetdash{}{0pt}%
\pgfpathmoveto{\pgfqpoint{2.343142in}{4.659496in}}%
\pgfpathlineto{\pgfqpoint{2.351090in}{4.676608in}}%
\pgfpathlineto{\pgfqpoint{2.359042in}{4.694621in}}%
\pgfpathlineto{\pgfqpoint{2.391741in}{4.709856in}}%
\pgfpathlineto{\pgfqpoint{2.424436in}{4.724663in}}%
\pgfpathlineto{\pgfqpoint{2.416469in}{4.702962in}}%
\pgfpathlineto{\pgfqpoint{2.408505in}{4.682315in}}%
\pgfpathlineto{\pgfqpoint{2.375828in}{4.671078in}}%
\pgfpathlineto{\pgfqpoint{2.343142in}{4.659496in}}%
\pgfpathclose%
\pgfusepath{fill}%
\end{pgfscope}%
\begin{pgfscope}%
\pgfpathrectangle{\pgfqpoint{1.020000in}{0.880000in}}{\pgfqpoint{6.160000in}{6.160000in}}%
\pgfusepath{clip}%
\pgfsetbuttcap%
\pgfsetroundjoin%
\definecolor{currentfill}{rgb}{0.358415,0.478426,0.896795}%
\pgfsetfillcolor{currentfill}%
\pgfsetlinewidth{0.000000pt}%
\definecolor{currentstroke}{rgb}{0.000000,0.000000,0.000000}%
\pgfsetstrokecolor{currentstroke}%
\pgfsetdash{}{0pt}%
\pgfpathmoveto{\pgfqpoint{4.948204in}{3.397355in}}%
\pgfpathlineto{\pgfqpoint{4.958784in}{3.354997in}}%
\pgfpathlineto{\pgfqpoint{4.969369in}{3.311570in}}%
\pgfpathlineto{\pgfqpoint{5.001510in}{3.315776in}}%
\pgfpathlineto{\pgfqpoint{5.033658in}{3.323149in}}%
\pgfpathlineto{\pgfqpoint{5.023024in}{3.364706in}}%
\pgfpathlineto{\pgfqpoint{5.012396in}{3.405257in}}%
\pgfpathlineto{\pgfqpoint{4.980300in}{3.400116in}}%
\pgfpathlineto{\pgfqpoint{4.948204in}{3.397355in}}%
\pgfpathclose%
\pgfusepath{fill}%
\end{pgfscope}%
\begin{pgfscope}%
\pgfpathrectangle{\pgfqpoint{1.020000in}{0.880000in}}{\pgfqpoint{6.160000in}{6.160000in}}%
\pgfusepath{clip}%
\pgfsetbuttcap%
\pgfsetroundjoin%
\definecolor{currentfill}{rgb}{0.404421,0.534643,0.932002}%
\pgfsetfillcolor{currentfill}%
\pgfsetlinewidth{0.000000pt}%
\definecolor{currentstroke}{rgb}{0.000000,0.000000,0.000000}%
\pgfsetstrokecolor{currentstroke}%
\pgfsetdash{}{0pt}%
\pgfpathmoveto{\pgfqpoint{4.798752in}{3.489099in}}%
\pgfpathlineto{\pgfqpoint{4.809268in}{3.451737in}}%
\pgfpathlineto{\pgfqpoint{4.819786in}{3.412313in}}%
\pgfpathlineto{\pgfqpoint{4.851900in}{3.404749in}}%
\pgfpathlineto{\pgfqpoint{4.884006in}{3.399650in}}%
\pgfpathlineto{\pgfqpoint{4.873463in}{3.440452in}}%
\pgfpathlineto{\pgfqpoint{4.862920in}{3.479057in}}%
\pgfpathlineto{\pgfqpoint{4.830843in}{3.483189in}}%
\pgfpathlineto{\pgfqpoint{4.798752in}{3.489099in}}%
\pgfpathclose%
\pgfusepath{fill}%
\end{pgfscope}%
\begin{pgfscope}%
\pgfpathrectangle{\pgfqpoint{1.020000in}{0.880000in}}{\pgfqpoint{6.160000in}{6.160000in}}%
\pgfusepath{clip}%
\pgfsetbuttcap%
\pgfsetroundjoin%
\definecolor{currentfill}{rgb}{0.969851,0.695830,0.581312}%
\pgfsetfillcolor{currentfill}%
\pgfsetlinewidth{0.000000pt}%
\definecolor{currentstroke}{rgb}{0.000000,0.000000,0.000000}%
\pgfsetstrokecolor{currentstroke}%
\pgfsetdash{}{0pt}%
\pgfpathmoveto{\pgfqpoint{3.263396in}{4.727869in}}%
\pgfpathlineto{\pgfqpoint{3.272372in}{4.751364in}}%
\pgfpathlineto{\pgfqpoint{3.281378in}{4.773809in}}%
\pgfpathlineto{\pgfqpoint{3.314391in}{4.732134in}}%
\pgfpathlineto{\pgfqpoint{3.347363in}{4.689715in}}%
\pgfpathlineto{\pgfqpoint{3.338261in}{4.671061in}}%
\pgfpathlineto{\pgfqpoint{3.329188in}{4.651441in}}%
\pgfpathlineto{\pgfqpoint{3.296311in}{4.689986in}}%
\pgfpathlineto{\pgfqpoint{3.263396in}{4.727869in}}%
\pgfpathclose%
\pgfusepath{fill}%
\end{pgfscope}%
\begin{pgfscope}%
\pgfpathrectangle{\pgfqpoint{1.020000in}{0.880000in}}{\pgfqpoint{6.160000in}{6.160000in}}%
\pgfusepath{clip}%
\pgfsetbuttcap%
\pgfsetroundjoin%
\definecolor{currentfill}{rgb}{0.358415,0.478426,0.896795}%
\pgfsetfillcolor{currentfill}%
\pgfsetlinewidth{0.000000pt}%
\definecolor{currentstroke}{rgb}{0.000000,0.000000,0.000000}%
\pgfsetstrokecolor{currentstroke}%
\pgfsetdash{}{0pt}%
\pgfpathmoveto{\pgfqpoint{5.162292in}{3.374977in}}%
\pgfpathlineto{\pgfqpoint{5.173079in}{3.340578in}}%
\pgfpathlineto{\pgfqpoint{5.183885in}{3.306600in}}%
\pgfpathlineto{\pgfqpoint{5.216120in}{3.326591in}}%
\pgfpathlineto{\pgfqpoint{5.248346in}{3.346489in}}%
\pgfpathlineto{\pgfqpoint{5.237456in}{3.376213in}}%
\pgfpathlineto{\pgfqpoint{5.226585in}{3.406246in}}%
\pgfpathlineto{\pgfqpoint{5.194444in}{3.390644in}}%
\pgfpathlineto{\pgfqpoint{5.162292in}{3.374977in}}%
\pgfpathclose%
\pgfusepath{fill}%
\end{pgfscope}%
\begin{pgfscope}%
\pgfpathrectangle{\pgfqpoint{1.020000in}{0.880000in}}{\pgfqpoint{6.160000in}{6.160000in}}%
\pgfusepath{clip}%
\pgfsetbuttcap%
\pgfsetroundjoin%
\definecolor{currentfill}{rgb}{0.964911,0.640159,0.519806}%
\pgfsetfillcolor{currentfill}%
\pgfsetlinewidth{0.000000pt}%
\definecolor{currentstroke}{rgb}{0.000000,0.000000,0.000000}%
\pgfsetstrokecolor{currentstroke}%
\pgfsetdash{}{0pt}%
\pgfpathmoveto{\pgfqpoint{3.113976in}{4.803629in}}%
\pgfpathlineto{\pgfqpoint{3.122690in}{4.835159in}}%
\pgfpathlineto{\pgfqpoint{3.131431in}{4.866297in}}%
\pgfpathlineto{\pgfqpoint{3.164460in}{4.834304in}}%
\pgfpathlineto{\pgfqpoint{3.197467in}{4.800355in}}%
\pgfpathlineto{\pgfqpoint{3.188623in}{4.772485in}}%
\pgfpathlineto{\pgfqpoint{3.179804in}{4.744233in}}%
\pgfpathlineto{\pgfqpoint{3.146901in}{4.774788in}}%
\pgfpathlineto{\pgfqpoint{3.113976in}{4.803629in}}%
\pgfpathclose%
\pgfusepath{fill}%
\end{pgfscope}%
\begin{pgfscope}%
\pgfpathrectangle{\pgfqpoint{1.020000in}{0.880000in}}{\pgfqpoint{6.160000in}{6.160000in}}%
\pgfusepath{clip}%
\pgfsetbuttcap%
\pgfsetroundjoin%
\definecolor{currentfill}{rgb}{0.467678,0.605591,0.968546}%
\pgfsetfillcolor{currentfill}%
\pgfsetlinewidth{0.000000pt}%
\definecolor{currentstroke}{rgb}{0.000000,0.000000,0.000000}%
\pgfsetstrokecolor{currentstroke}%
\pgfsetdash{}{0pt}%
\pgfpathmoveto{\pgfqpoint{4.649293in}{3.585081in}}%
\pgfpathlineto{\pgfqpoint{4.659748in}{3.556691in}}%
\pgfpathlineto{\pgfqpoint{4.670212in}{3.526479in}}%
\pgfpathlineto{\pgfqpoint{4.702378in}{3.515568in}}%
\pgfpathlineto{\pgfqpoint{4.734523in}{3.505525in}}%
\pgfpathlineto{\pgfqpoint{4.724034in}{3.538437in}}%
\pgfpathlineto{\pgfqpoint{4.713549in}{3.569155in}}%
\pgfpathlineto{\pgfqpoint{4.681432in}{3.576769in}}%
\pgfpathlineto{\pgfqpoint{4.649293in}{3.585081in}}%
\pgfpathclose%
\pgfusepath{fill}%
\end{pgfscope}%
\begin{pgfscope}%
\pgfpathrectangle{\pgfqpoint{1.020000in}{0.880000in}}{\pgfqpoint{6.160000in}{6.160000in}}%
\pgfusepath{clip}%
\pgfsetbuttcap%
\pgfsetroundjoin%
\definecolor{currentfill}{rgb}{0.960490,0.616276,0.495467}%
\pgfsetfillcolor{currentfill}%
\pgfsetlinewidth{0.000000pt}%
\definecolor{currentstroke}{rgb}{0.000000,0.000000,0.000000}%
\pgfsetstrokecolor{currentstroke}%
\pgfsetdash{}{0pt}%
\pgfpathmoveto{\pgfqpoint{2.686125in}{4.802787in}}%
\pgfpathlineto{\pgfqpoint{2.694247in}{4.836763in}}%
\pgfpathlineto{\pgfqpoint{2.702377in}{4.871759in}}%
\pgfpathlineto{\pgfqpoint{2.735189in}{4.875197in}}%
\pgfpathlineto{\pgfqpoint{2.768016in}{4.876049in}}%
\pgfpathlineto{\pgfqpoint{2.759820in}{4.839762in}}%
\pgfpathlineto{\pgfqpoint{2.751633in}{4.804536in}}%
\pgfpathlineto{\pgfqpoint{2.718874in}{4.804756in}}%
\pgfpathlineto{\pgfqpoint{2.686125in}{4.802787in}}%
\pgfpathclose%
\pgfusepath{fill}%
\end{pgfscope}%
\begin{pgfscope}%
\pgfpathrectangle{\pgfqpoint{1.020000in}{0.880000in}}{\pgfqpoint{6.160000in}{6.160000in}}%
\pgfusepath{clip}%
\pgfsetbuttcap%
\pgfsetroundjoin%
\definecolor{currentfill}{rgb}{0.565182,0.699438,0.996635}%
\pgfsetfillcolor{currentfill}%
\pgfsetlinewidth{0.000000pt}%
\definecolor{currentstroke}{rgb}{0.000000,0.000000,0.000000}%
\pgfsetstrokecolor{currentstroke}%
\pgfsetdash{}{0pt}%
\pgfpathmoveto{\pgfqpoint{4.350207in}{3.743013in}}%
\pgfpathlineto{\pgfqpoint{4.360452in}{3.724457in}}%
\pgfpathlineto{\pgfqpoint{4.370713in}{3.705327in}}%
\pgfpathlineto{\pgfqpoint{4.403011in}{3.694812in}}%
\pgfpathlineto{\pgfqpoint{4.435283in}{3.684983in}}%
\pgfpathlineto{\pgfqpoint{4.424977in}{3.704665in}}%
\pgfpathlineto{\pgfqpoint{4.414688in}{3.723632in}}%
\pgfpathlineto{\pgfqpoint{4.382460in}{3.732895in}}%
\pgfpathlineto{\pgfqpoint{4.350207in}{3.743013in}}%
\pgfpathclose%
\pgfusepath{fill}%
\end{pgfscope}%
\begin{pgfscope}%
\pgfpathrectangle{\pgfqpoint{1.020000in}{0.880000in}}{\pgfqpoint{6.160000in}{6.160000in}}%
\pgfusepath{clip}%
\pgfsetbuttcap%
\pgfsetroundjoin%
\definecolor{currentfill}{rgb}{0.613933,0.739923,0.999142}%
\pgfsetfillcolor{currentfill}%
\pgfsetlinewidth{0.000000pt}%
\definecolor{currentstroke}{rgb}{0.000000,0.000000,0.000000}%
\pgfsetstrokecolor{currentstroke}%
\pgfsetdash{}{0pt}%
\pgfpathmoveto{\pgfqpoint{4.200680in}{3.830279in}}%
\pgfpathlineto{\pgfqpoint{4.210799in}{3.812718in}}%
\pgfpathlineto{\pgfqpoint{4.220934in}{3.794629in}}%
\pgfpathlineto{\pgfqpoint{4.253294in}{3.779771in}}%
\pgfpathlineto{\pgfqpoint{4.285625in}{3.766313in}}%
\pgfpathlineto{\pgfqpoint{4.275443in}{3.784253in}}%
\pgfpathlineto{\pgfqpoint{4.265279in}{3.801692in}}%
\pgfpathlineto{\pgfqpoint{4.232994in}{3.815241in}}%
\pgfpathlineto{\pgfqpoint{4.200680in}{3.830279in}}%
\pgfpathclose%
\pgfusepath{fill}%
\end{pgfscope}%
\begin{pgfscope}%
\pgfpathrectangle{\pgfqpoint{1.020000in}{0.880000in}}{\pgfqpoint{6.160000in}{6.160000in}}%
\pgfusepath{clip}%
\pgfsetbuttcap%
\pgfsetroundjoin%
\definecolor{currentfill}{rgb}{0.945540,0.798606,0.723105}%
\pgfsetfillcolor{currentfill}%
\pgfsetlinewidth{0.000000pt}%
\definecolor{currentstroke}{rgb}{0.000000,0.000000,0.000000}%
\pgfsetstrokecolor{currentstroke}%
\pgfsetdash{}{0pt}%
\pgfpathmoveto{\pgfqpoint{3.497442in}{4.537891in}}%
\pgfpathlineto{\pgfqpoint{3.506809in}{4.544975in}}%
\pgfpathlineto{\pgfqpoint{3.516208in}{4.550295in}}%
\pgfpathlineto{\pgfqpoint{3.549096in}{4.502396in}}%
\pgfpathlineto{\pgfqpoint{3.581929in}{4.455749in}}%
\pgfpathlineto{\pgfqpoint{3.572459in}{4.453412in}}%
\pgfpathlineto{\pgfqpoint{3.563017in}{4.449541in}}%
\pgfpathlineto{\pgfqpoint{3.530257in}{4.493135in}}%
\pgfpathlineto{\pgfqpoint{3.497442in}{4.537891in}}%
\pgfpathclose%
\pgfusepath{fill}%
\end{pgfscope}%
\begin{pgfscope}%
\pgfpathrectangle{\pgfqpoint{1.020000in}{0.880000in}}{\pgfqpoint{6.160000in}{6.160000in}}%
\pgfusepath{clip}%
\pgfsetbuttcap%
\pgfsetroundjoin%
\definecolor{currentfill}{rgb}{0.399231,0.528528,0.928459}%
\pgfsetfillcolor{currentfill}%
\pgfsetlinewidth{0.000000pt}%
\definecolor{currentstroke}{rgb}{0.000000,0.000000,0.000000}%
\pgfsetstrokecolor{currentstroke}%
\pgfsetdash{}{0pt}%
\pgfpathmoveto{\pgfqpoint{5.441202in}{3.438728in}}%
\pgfpathlineto{\pgfqpoint{5.452360in}{3.419075in}}%
\pgfpathlineto{\pgfqpoint{5.463541in}{3.399802in}}%
\pgfpathlineto{\pgfqpoint{5.495641in}{3.410763in}}%
\pgfpathlineto{\pgfqpoint{5.527701in}{3.419775in}}%
\pgfpathlineto{\pgfqpoint{5.516452in}{3.437512in}}%
\pgfpathlineto{\pgfqpoint{5.505225in}{3.455454in}}%
\pgfpathlineto{\pgfqpoint{5.473231in}{3.447914in}}%
\pgfpathlineto{\pgfqpoint{5.441202in}{3.438728in}}%
\pgfpathclose%
\pgfusepath{fill}%
\end{pgfscope}%
\begin{pgfscope}%
\pgfpathrectangle{\pgfqpoint{1.020000in}{0.880000in}}{\pgfqpoint{6.160000in}{6.160000in}}%
\pgfusepath{clip}%
\pgfsetbuttcap%
\pgfsetroundjoin%
\definecolor{currentfill}{rgb}{0.516260,0.654498,0.986407}%
\pgfsetfillcolor{currentfill}%
\pgfsetlinewidth{0.000000pt}%
\definecolor{currentstroke}{rgb}{0.000000,0.000000,0.000000}%
\pgfsetstrokecolor{currentstroke}%
\pgfsetdash{}{0pt}%
\pgfpathmoveto{\pgfqpoint{4.499754in}{3.666996in}}%
\pgfpathlineto{\pgfqpoint{4.510117in}{3.645399in}}%
\pgfpathlineto{\pgfqpoint{4.520493in}{3.622743in}}%
\pgfpathlineto{\pgfqpoint{4.552731in}{3.612852in}}%
\pgfpathlineto{\pgfqpoint{4.584943in}{3.603233in}}%
\pgfpathlineto{\pgfqpoint{4.574529in}{3.627735in}}%
\pgfpathlineto{\pgfqpoint{4.564127in}{3.650839in}}%
\pgfpathlineto{\pgfqpoint{4.531952in}{3.658704in}}%
\pgfpathlineto{\pgfqpoint{4.499754in}{3.666996in}}%
\pgfpathclose%
\pgfusepath{fill}%
\end{pgfscope}%
\begin{pgfscope}%
\pgfpathrectangle{\pgfqpoint{1.020000in}{0.880000in}}{\pgfqpoint{6.160000in}{6.160000in}}%
\pgfusepath{clip}%
\pgfsetbuttcap%
\pgfsetroundjoin%
\definecolor{currentfill}{rgb}{0.968203,0.720844,0.612293}%
\pgfsetfillcolor{currentfill}%
\pgfsetlinewidth{0.000000pt}%
\definecolor{currentstroke}{rgb}{0.000000,0.000000,0.000000}%
\pgfsetstrokecolor{currentstroke}%
\pgfsetdash{}{0pt}%
\pgfpathmoveto{\pgfqpoint{2.277735in}{4.635986in}}%
\pgfpathlineto{\pgfqpoint{2.285673in}{4.649464in}}%
\pgfpathlineto{\pgfqpoint{2.293617in}{4.663685in}}%
\pgfpathlineto{\pgfqpoint{2.326335in}{4.679164in}}%
\pgfpathlineto{\pgfqpoint{2.359042in}{4.694621in}}%
\pgfpathlineto{\pgfqpoint{2.351090in}{4.676608in}}%
\pgfpathlineto{\pgfqpoint{2.343142in}{4.659496in}}%
\pgfpathlineto{\pgfqpoint{2.310445in}{4.647743in}}%
\pgfpathlineto{\pgfqpoint{2.277735in}{4.635986in}}%
\pgfpathclose%
\pgfusepath{fill}%
\end{pgfscope}%
\begin{pgfscope}%
\pgfpathrectangle{\pgfqpoint{1.020000in}{0.880000in}}{\pgfqpoint{6.160000in}{6.160000in}}%
\pgfusepath{clip}%
\pgfsetbuttcap%
\pgfsetroundjoin%
\definecolor{currentfill}{rgb}{0.800601,0.850358,0.930008}%
\pgfsetfillcolor{currentfill}%
\pgfsetlinewidth{0.000000pt}%
\definecolor{currentstroke}{rgb}{0.000000,0.000000,0.000000}%
\pgfsetstrokecolor{currentstroke}%
\pgfsetdash{}{0pt}%
\pgfpathmoveto{\pgfqpoint{3.816630in}{4.181144in}}%
\pgfpathlineto{\pgfqpoint{3.826403in}{4.169804in}}%
\pgfpathlineto{\pgfqpoint{3.836199in}{4.157035in}}%
\pgfpathlineto{\pgfqpoint{3.868811in}{4.120244in}}%
\pgfpathlineto{\pgfqpoint{3.901374in}{4.085972in}}%
\pgfpathlineto{\pgfqpoint{3.891531in}{4.099352in}}%
\pgfpathlineto{\pgfqpoint{3.881709in}{4.111559in}}%
\pgfpathlineto{\pgfqpoint{3.849194in}{4.145123in}}%
\pgfpathlineto{\pgfqpoint{3.816630in}{4.181144in}}%
\pgfpathclose%
\pgfusepath{fill}%
\end{pgfscope}%
\begin{pgfscope}%
\pgfpathrectangle{\pgfqpoint{1.020000in}{0.880000in}}{\pgfqpoint{6.160000in}{6.160000in}}%
\pgfusepath{clip}%
\pgfsetbuttcap%
\pgfsetroundjoin%
\definecolor{currentfill}{rgb}{0.343278,0.459354,0.884122}%
\pgfsetfillcolor{currentfill}%
\pgfsetlinewidth{0.000000pt}%
\definecolor{currentstroke}{rgb}{0.000000,0.000000,0.000000}%
\pgfsetstrokecolor{currentstroke}%
\pgfsetdash{}{0pt}%
\pgfpathmoveto{\pgfqpoint{5.097972in}{3.345700in}}%
\pgfpathlineto{\pgfqpoint{5.108682in}{3.307169in}}%
\pgfpathlineto{\pgfqpoint{5.119411in}{3.269166in}}%
\pgfpathlineto{\pgfqpoint{5.151647in}{3.287208in}}%
\pgfpathlineto{\pgfqpoint{5.183885in}{3.306600in}}%
\pgfpathlineto{\pgfqpoint{5.173079in}{3.340578in}}%
\pgfpathlineto{\pgfqpoint{5.162292in}{3.374977in}}%
\pgfpathlineto{\pgfqpoint{5.130133in}{3.359797in}}%
\pgfpathlineto{\pgfqpoint{5.097972in}{3.345700in}}%
\pgfpathclose%
\pgfusepath{fill}%
\end{pgfscope}%
\begin{pgfscope}%
\pgfpathrectangle{\pgfqpoint{1.020000in}{0.880000in}}{\pgfqpoint{6.160000in}{6.160000in}}%
\pgfusepath{clip}%
\pgfsetbuttcap%
\pgfsetroundjoin%
\definecolor{currentfill}{rgb}{0.677823,0.786546,0.991005}%
\pgfsetfillcolor{currentfill}%
\pgfsetlinewidth{0.000000pt}%
\definecolor{currentstroke}{rgb}{0.000000,0.000000,0.000000}%
\pgfsetstrokecolor{currentstroke}%
\pgfsetdash{}{0pt}%
\pgfpathmoveto{\pgfqpoint{4.051110in}{3.942276in}}%
\pgfpathlineto{\pgfqpoint{4.061102in}{3.925602in}}%
\pgfpathlineto{\pgfqpoint{4.071112in}{3.908147in}}%
\pgfpathlineto{\pgfqpoint{4.103555in}{3.885746in}}%
\pgfpathlineto{\pgfqpoint{4.135963in}{3.865391in}}%
\pgfpathlineto{\pgfqpoint{4.125907in}{3.882589in}}%
\pgfpathlineto{\pgfqpoint{4.115869in}{3.899148in}}%
\pgfpathlineto{\pgfqpoint{4.083507in}{3.919676in}}%
\pgfpathlineto{\pgfqpoint{4.051110in}{3.942276in}}%
\pgfpathclose%
\pgfusepath{fill}%
\end{pgfscope}%
\begin{pgfscope}%
\pgfpathrectangle{\pgfqpoint{1.020000in}{0.880000in}}{\pgfqpoint{6.160000in}{6.160000in}}%
\pgfusepath{clip}%
\pgfsetbuttcap%
\pgfsetroundjoin%
\definecolor{currentfill}{rgb}{0.399231,0.528528,0.928459}%
\pgfsetfillcolor{currentfill}%
\pgfsetlinewidth{0.000000pt}%
\definecolor{currentstroke}{rgb}{0.000000,0.000000,0.000000}%
\pgfsetstrokecolor{currentstroke}%
\pgfsetdash{}{0pt}%
\pgfpathmoveto{\pgfqpoint{5.655589in}{3.440877in}}%
\pgfpathlineto{\pgfqpoint{5.666979in}{3.424876in}}%
\pgfpathlineto{\pgfqpoint{5.678391in}{3.408958in}}%
\pgfpathlineto{\pgfqpoint{5.710338in}{3.412019in}}%
\pgfpathlineto{\pgfqpoint{5.698899in}{3.427774in}}%
\pgfpathlineto{\pgfqpoint{5.687482in}{3.443575in}}%
\pgfpathlineto{\pgfqpoint{5.655589in}{3.440877in}}%
\pgfpathclose%
\pgfusepath{fill}%
\end{pgfscope}%
\begin{pgfscope}%
\pgfpathrectangle{\pgfqpoint{1.020000in}{0.880000in}}{\pgfqpoint{6.160000in}{6.160000in}}%
\pgfusepath{clip}%
\pgfsetbuttcap%
\pgfsetroundjoin%
\definecolor{currentfill}{rgb}{0.358415,0.478426,0.896795}%
\pgfsetfillcolor{currentfill}%
\pgfsetlinewidth{0.000000pt}%
\definecolor{currentstroke}{rgb}{0.000000,0.000000,0.000000}%
\pgfsetstrokecolor{currentstroke}%
\pgfsetdash{}{0pt}%
\pgfpathmoveto{\pgfqpoint{4.884006in}{3.399650in}}%
\pgfpathlineto{\pgfqpoint{4.894552in}{3.357131in}}%
\pgfpathlineto{\pgfqpoint{4.905104in}{3.313544in}}%
\pgfpathlineto{\pgfqpoint{4.937235in}{3.310792in}}%
\pgfpathlineto{\pgfqpoint{4.969369in}{3.311570in}}%
\pgfpathlineto{\pgfqpoint{4.958784in}{3.354997in}}%
\pgfpathlineto{\pgfqpoint{4.948204in}{3.397355in}}%
\pgfpathlineto{\pgfqpoint{4.916107in}{3.397173in}}%
\pgfpathlineto{\pgfqpoint{4.884006in}{3.399650in}}%
\pgfpathclose%
\pgfusepath{fill}%
\end{pgfscope}%
\begin{pgfscope}%
\pgfpathrectangle{\pgfqpoint{1.020000in}{0.880000in}}{\pgfqpoint{6.160000in}{6.160000in}}%
\pgfusepath{clip}%
\pgfsetbuttcap%
\pgfsetroundjoin%
\definecolor{currentfill}{rgb}{0.954853,0.591622,0.471337}%
\pgfsetfillcolor{currentfill}%
\pgfsetlinewidth{0.000000pt}%
\definecolor{currentstroke}{rgb}{0.000000,0.000000,0.000000}%
\pgfsetstrokecolor{currentstroke}%
\pgfsetdash{}{0pt}%
\pgfpathmoveto{\pgfqpoint{2.899441in}{4.852439in}}%
\pgfpathlineto{\pgfqpoint{2.907814in}{4.889563in}}%
\pgfpathlineto{\pgfqpoint{2.916206in}{4.927084in}}%
\pgfpathlineto{\pgfqpoint{2.949172in}{4.913414in}}%
\pgfpathlineto{\pgfqpoint{2.982142in}{4.896762in}}%
\pgfpathlineto{\pgfqpoint{2.973656in}{4.860479in}}%
\pgfpathlineto{\pgfqpoint{2.965188in}{4.824580in}}%
\pgfpathlineto{\pgfqpoint{2.932315in}{4.839794in}}%
\pgfpathlineto{\pgfqpoint{2.899441in}{4.852439in}}%
\pgfpathclose%
\pgfusepath{fill}%
\end{pgfscope}%
\begin{pgfscope}%
\pgfpathrectangle{\pgfqpoint{1.020000in}{0.880000in}}{\pgfqpoint{6.160000in}{6.160000in}}%
\pgfusepath{clip}%
\pgfsetbuttcap%
\pgfsetroundjoin%
\definecolor{currentfill}{rgb}{0.388852,0.516298,0.921373}%
\pgfsetfillcolor{currentfill}%
\pgfsetlinewidth{0.000000pt}%
\definecolor{currentstroke}{rgb}{0.000000,0.000000,0.000000}%
\pgfsetstrokecolor{currentstroke}%
\pgfsetdash{}{0pt}%
\pgfpathmoveto{\pgfqpoint{5.377039in}{3.414907in}}%
\pgfpathlineto{\pgfqpoint{5.388121in}{3.392850in}}%
\pgfpathlineto{\pgfqpoint{5.399228in}{3.371424in}}%
\pgfpathlineto{\pgfqpoint{5.431404in}{3.386721in}}%
\pgfpathlineto{\pgfqpoint{5.463541in}{3.399802in}}%
\pgfpathlineto{\pgfqpoint{5.452360in}{3.419075in}}%
\pgfpathlineto{\pgfqpoint{5.441202in}{3.438728in}}%
\pgfpathlineto{\pgfqpoint{5.409138in}{3.427753in}}%
\pgfpathlineto{\pgfqpoint{5.377039in}{3.414907in}}%
\pgfpathclose%
\pgfusepath{fill}%
\end{pgfscope}%
\begin{pgfscope}%
\pgfpathrectangle{\pgfqpoint{1.020000in}{0.880000in}}{\pgfqpoint{6.160000in}{6.160000in}}%
\pgfusepath{clip}%
\pgfsetbuttcap%
\pgfsetroundjoin%
\definecolor{currentfill}{rgb}{0.966962,0.735670,0.630877}%
\pgfsetfillcolor{currentfill}%
\pgfsetlinewidth{0.000000pt}%
\definecolor{currentstroke}{rgb}{0.000000,0.000000,0.000000}%
\pgfsetstrokecolor{currentstroke}%
\pgfsetdash{}{0pt}%
\pgfpathmoveto{\pgfqpoint{2.212263in}{4.613087in}}%
\pgfpathlineto{\pgfqpoint{2.220195in}{4.622953in}}%
\pgfpathlineto{\pgfqpoint{2.228134in}{4.633406in}}%
\pgfpathlineto{\pgfqpoint{2.260884in}{4.648374in}}%
\pgfpathlineto{\pgfqpoint{2.293617in}{4.663685in}}%
\pgfpathlineto{\pgfqpoint{2.285673in}{4.649464in}}%
\pgfpathlineto{\pgfqpoint{2.277735in}{4.635986in}}%
\pgfpathlineto{\pgfqpoint{2.245008in}{4.624385in}}%
\pgfpathlineto{\pgfqpoint{2.212263in}{4.613087in}}%
\pgfpathclose%
\pgfusepath{fill}%
\end{pgfscope}%
\begin{pgfscope}%
\pgfpathrectangle{\pgfqpoint{1.020000in}{0.880000in}}{\pgfqpoint{6.160000in}{6.160000in}}%
\pgfusepath{clip}%
\pgfsetbuttcap%
\pgfsetroundjoin%
\definecolor{currentfill}{rgb}{0.960490,0.616276,0.495467}%
\pgfsetfillcolor{currentfill}%
\pgfsetlinewidth{0.000000pt}%
\definecolor{currentstroke}{rgb}{0.000000,0.000000,0.000000}%
\pgfsetstrokecolor{currentstroke}%
\pgfsetdash{}{0pt}%
\pgfpathmoveto{\pgfqpoint{2.620656in}{4.792699in}}%
\pgfpathlineto{\pgfqpoint{2.628724in}{4.824680in}}%
\pgfpathlineto{\pgfqpoint{2.636801in}{4.857618in}}%
\pgfpathlineto{\pgfqpoint{2.669581in}{4.865850in}}%
\pgfpathlineto{\pgfqpoint{2.702377in}{4.871759in}}%
\pgfpathlineto{\pgfqpoint{2.694247in}{4.836763in}}%
\pgfpathlineto{\pgfqpoint{2.686125in}{4.802787in}}%
\pgfpathlineto{\pgfqpoint{2.653385in}{4.798727in}}%
\pgfpathlineto{\pgfqpoint{2.620656in}{4.792699in}}%
\pgfpathclose%
\pgfusepath{fill}%
\end{pgfscope}%
\begin{pgfscope}%
\pgfpathrectangle{\pgfqpoint{1.020000in}{0.880000in}}{\pgfqpoint{6.160000in}{6.160000in}}%
\pgfusepath{clip}%
\pgfsetbuttcap%
\pgfsetroundjoin%
\definecolor{currentfill}{rgb}{0.328604,0.439712,0.869587}%
\pgfsetfillcolor{currentfill}%
\pgfsetlinewidth{0.000000pt}%
\definecolor{currentstroke}{rgb}{0.000000,0.000000,0.000000}%
\pgfsetstrokecolor{currentstroke}%
\pgfsetdash{}{0pt}%
\pgfpathmoveto{\pgfqpoint{5.033658in}{3.323149in}}%
\pgfpathlineto{\pgfqpoint{5.044304in}{3.281337in}}%
\pgfpathlineto{\pgfqpoint{5.054969in}{3.240139in}}%
\pgfpathlineto{\pgfqpoint{5.087184in}{3.253235in}}%
\pgfpathlineto{\pgfqpoint{5.119411in}{3.269166in}}%
\pgfpathlineto{\pgfqpoint{5.108682in}{3.307169in}}%
\pgfpathlineto{\pgfqpoint{5.097972in}{3.345700in}}%
\pgfpathlineto{\pgfqpoint{5.065812in}{3.333292in}}%
\pgfpathlineto{\pgfqpoint{5.033658in}{3.323149in}}%
\pgfpathclose%
\pgfusepath{fill}%
\end{pgfscope}%
\begin{pgfscope}%
\pgfpathrectangle{\pgfqpoint{1.020000in}{0.880000in}}{\pgfqpoint{6.160000in}{6.160000in}}%
\pgfusepath{clip}%
\pgfsetbuttcap%
\pgfsetroundjoin%
\definecolor{currentfill}{rgb}{0.419991,0.552989,0.942630}%
\pgfsetfillcolor{currentfill}%
\pgfsetlinewidth{0.000000pt}%
\definecolor{currentstroke}{rgb}{0.000000,0.000000,0.000000}%
\pgfsetstrokecolor{currentstroke}%
\pgfsetdash{}{0pt}%
\pgfpathmoveto{\pgfqpoint{4.734523in}{3.505525in}}%
\pgfpathlineto{\pgfqpoint{4.745016in}{3.470540in}}%
\pgfpathlineto{\pgfqpoint{4.755514in}{3.433737in}}%
\pgfpathlineto{\pgfqpoint{4.787658in}{3.422087in}}%
\pgfpathlineto{\pgfqpoint{4.819786in}{3.412313in}}%
\pgfpathlineto{\pgfqpoint{4.809268in}{3.451737in}}%
\pgfpathlineto{\pgfqpoint{4.798752in}{3.489099in}}%
\pgfpathlineto{\pgfqpoint{4.766646in}{3.496618in}}%
\pgfpathlineto{\pgfqpoint{4.734523in}{3.505525in}}%
\pgfpathclose%
\pgfusepath{fill}%
\end{pgfscope}%
\begin{pgfscope}%
\pgfpathrectangle{\pgfqpoint{1.020000in}{0.880000in}}{\pgfqpoint{6.160000in}{6.160000in}}%
\pgfusepath{clip}%
\pgfsetbuttcap%
\pgfsetroundjoin%
\definecolor{currentfill}{rgb}{0.922681,0.828568,0.777054}%
\pgfsetfillcolor{currentfill}%
\pgfsetlinewidth{0.000000pt}%
\definecolor{currentstroke}{rgb}{0.000000,0.000000,0.000000}%
\pgfsetstrokecolor{currentstroke}%
\pgfsetdash{}{0pt}%
\pgfpathmoveto{\pgfqpoint{3.581929in}{4.455749in}}%
\pgfpathlineto{\pgfqpoint{3.591428in}{4.456358in}}%
\pgfpathlineto{\pgfqpoint{3.600957in}{4.455062in}}%
\pgfpathlineto{\pgfqpoint{3.633797in}{4.407469in}}%
\pgfpathlineto{\pgfqpoint{3.666579in}{4.361686in}}%
\pgfpathlineto{\pgfqpoint{3.656988in}{4.365238in}}%
\pgfpathlineto{\pgfqpoint{3.647424in}{4.367160in}}%
\pgfpathlineto{\pgfqpoint{3.614705in}{4.410600in}}%
\pgfpathlineto{\pgfqpoint{3.581929in}{4.455749in}}%
\pgfpathclose%
\pgfusepath{fill}%
\end{pgfscope}%
\begin{pgfscope}%
\pgfpathrectangle{\pgfqpoint{1.020000in}{0.880000in}}{\pgfqpoint{6.160000in}{6.160000in}}%
\pgfusepath{clip}%
\pgfsetbuttcap%
\pgfsetroundjoin%
\definecolor{currentfill}{rgb}{0.753611,0.830233,0.960871}%
\pgfsetfillcolor{currentfill}%
\pgfsetlinewidth{0.000000pt}%
\definecolor{currentstroke}{rgb}{0.000000,0.000000,0.000000}%
\pgfsetstrokecolor{currentstroke}%
\pgfsetdash{}{0pt}%
\pgfpathmoveto{\pgfqpoint{3.901374in}{4.085972in}}%
\pgfpathlineto{\pgfqpoint{3.911238in}{4.071395in}}%
\pgfpathlineto{\pgfqpoint{3.921122in}{4.055609in}}%
\pgfpathlineto{\pgfqpoint{3.953683in}{4.023653in}}%
\pgfpathlineto{\pgfqpoint{3.986200in}{3.994164in}}%
\pgfpathlineto{\pgfqpoint{3.976271in}{4.010007in}}%
\pgfpathlineto{\pgfqpoint{3.966362in}{4.024868in}}%
\pgfpathlineto{\pgfqpoint{3.933890in}{4.054197in}}%
\pgfpathlineto{\pgfqpoint{3.901374in}{4.085972in}}%
\pgfpathclose%
\pgfusepath{fill}%
\end{pgfscope}%
\begin{pgfscope}%
\pgfpathrectangle{\pgfqpoint{1.020000in}{0.880000in}}{\pgfqpoint{6.160000in}{6.160000in}}%
\pgfusepath{clip}%
\pgfsetbuttcap%
\pgfsetroundjoin%
\definecolor{currentfill}{rgb}{0.399231,0.528528,0.928459}%
\pgfsetfillcolor{currentfill}%
\pgfsetlinewidth{0.000000pt}%
\definecolor{currentstroke}{rgb}{0.000000,0.000000,0.000000}%
\pgfsetstrokecolor{currentstroke}%
\pgfsetdash{}{0pt}%
\pgfpathmoveto{\pgfqpoint{5.591713in}{3.432836in}}%
\pgfpathlineto{\pgfqpoint{5.603046in}{3.416272in}}%
\pgfpathlineto{\pgfqpoint{5.614402in}{3.399902in}}%
\pgfpathlineto{\pgfqpoint{5.646413in}{3.404997in}}%
\pgfpathlineto{\pgfqpoint{5.678391in}{3.408958in}}%
\pgfpathlineto{\pgfqpoint{5.666979in}{3.424876in}}%
\pgfpathlineto{\pgfqpoint{5.655589in}{3.440877in}}%
\pgfpathlineto{\pgfqpoint{5.623667in}{3.437367in}}%
\pgfpathlineto{\pgfqpoint{5.591713in}{3.432836in}}%
\pgfpathclose%
\pgfusepath{fill}%
\end{pgfscope}%
\begin{pgfscope}%
\pgfpathrectangle{\pgfqpoint{1.020000in}{0.880000in}}{\pgfqpoint{6.160000in}{6.160000in}}%
\pgfusepath{clip}%
\pgfsetbuttcap%
\pgfsetroundjoin%
\definecolor{currentfill}{rgb}{0.968203,0.720844,0.612293}%
\pgfsetfillcolor{currentfill}%
\pgfsetlinewidth{0.000000pt}%
\definecolor{currentstroke}{rgb}{0.000000,0.000000,0.000000}%
\pgfsetstrokecolor{currentstroke}%
\pgfsetdash{}{0pt}%
\pgfpathmoveto{\pgfqpoint{3.347363in}{4.689715in}}%
\pgfpathlineto{\pgfqpoint{3.356497in}{4.707083in}}%
\pgfpathlineto{\pgfqpoint{3.365665in}{4.722841in}}%
\pgfpathlineto{\pgfqpoint{3.398684in}{4.676359in}}%
\pgfpathlineto{\pgfqpoint{3.431655in}{4.629812in}}%
\pgfpathlineto{\pgfqpoint{3.422400in}{4.617648in}}%
\pgfpathlineto{\pgfqpoint{3.413176in}{4.604038in}}%
\pgfpathlineto{\pgfqpoint{3.380292in}{4.646904in}}%
\pgfpathlineto{\pgfqpoint{3.347363in}{4.689715in}}%
\pgfpathclose%
\pgfusepath{fill}%
\end{pgfscope}%
\begin{pgfscope}%
\pgfpathrectangle{\pgfqpoint{1.020000in}{0.880000in}}{\pgfqpoint{6.160000in}{6.160000in}}%
\pgfusepath{clip}%
\pgfsetbuttcap%
\pgfsetroundjoin%
\definecolor{currentfill}{rgb}{0.483854,0.622050,0.974808}%
\pgfsetfillcolor{currentfill}%
\pgfsetlinewidth{0.000000pt}%
\definecolor{currentstroke}{rgb}{0.000000,0.000000,0.000000}%
\pgfsetstrokecolor{currentstroke}%
\pgfsetdash{}{0pt}%
\pgfpathmoveto{\pgfqpoint{4.584943in}{3.603233in}}%
\pgfpathlineto{\pgfqpoint{4.595367in}{3.577287in}}%
\pgfpathlineto{\pgfqpoint{4.605803in}{3.549910in}}%
\pgfpathlineto{\pgfqpoint{4.638020in}{3.538003in}}%
\pgfpathlineto{\pgfqpoint{4.670212in}{3.526479in}}%
\pgfpathlineto{\pgfqpoint{4.659748in}{3.556691in}}%
\pgfpathlineto{\pgfqpoint{4.649293in}{3.585081in}}%
\pgfpathlineto{\pgfqpoint{4.617130in}{3.593944in}}%
\pgfpathlineto{\pgfqpoint{4.584943in}{3.603233in}}%
\pgfpathclose%
\pgfusepath{fill}%
\end{pgfscope}%
\begin{pgfscope}%
\pgfpathrectangle{\pgfqpoint{1.020000in}{0.880000in}}{\pgfqpoint{6.160000in}{6.160000in}}%
\pgfusepath{clip}%
\pgfsetbuttcap%
\pgfsetroundjoin%
\definecolor{currentfill}{rgb}{0.373552,0.497499,0.909467}%
\pgfsetfillcolor{currentfill}%
\pgfsetlinewidth{0.000000pt}%
\definecolor{currentstroke}{rgb}{0.000000,0.000000,0.000000}%
\pgfsetstrokecolor{currentstroke}%
\pgfsetdash{}{0pt}%
\pgfpathmoveto{\pgfqpoint{5.312745in}{3.383709in}}%
\pgfpathlineto{\pgfqpoint{5.323745in}{3.358528in}}%
\pgfpathlineto{\pgfqpoint{5.334773in}{3.334307in}}%
\pgfpathlineto{\pgfqpoint{5.367017in}{3.353911in}}%
\pgfpathlineto{\pgfqpoint{5.399228in}{3.371424in}}%
\pgfpathlineto{\pgfqpoint{5.388121in}{3.392850in}}%
\pgfpathlineto{\pgfqpoint{5.377039in}{3.414907in}}%
\pgfpathlineto{\pgfqpoint{5.344907in}{3.400190in}}%
\pgfpathlineto{\pgfqpoint{5.312745in}{3.383709in}}%
\pgfpathclose%
\pgfusepath{fill}%
\end{pgfscope}%
\begin{pgfscope}%
\pgfpathrectangle{\pgfqpoint{1.020000in}{0.880000in}}{\pgfqpoint{6.160000in}{6.160000in}}%
\pgfusepath{clip}%
\pgfsetbuttcap%
\pgfsetroundjoin%
\definecolor{currentfill}{rgb}{0.581486,0.713451,0.998314}%
\pgfsetfillcolor{currentfill}%
\pgfsetlinewidth{0.000000pt}%
\definecolor{currentstroke}{rgb}{0.000000,0.000000,0.000000}%
\pgfsetstrokecolor{currentstroke}%
\pgfsetdash{}{0pt}%
\pgfpathmoveto{\pgfqpoint{4.285625in}{3.766313in}}%
\pgfpathlineto{\pgfqpoint{4.295824in}{3.747871in}}%
\pgfpathlineto{\pgfqpoint{4.306039in}{3.728927in}}%
\pgfpathlineto{\pgfqpoint{4.338389in}{3.716653in}}%
\pgfpathlineto{\pgfqpoint{4.370713in}{3.705327in}}%
\pgfpathlineto{\pgfqpoint{4.360452in}{3.724457in}}%
\pgfpathlineto{\pgfqpoint{4.350207in}{3.743013in}}%
\pgfpathlineto{\pgfqpoint{4.317929in}{3.754108in}}%
\pgfpathlineto{\pgfqpoint{4.285625in}{3.766313in}}%
\pgfpathclose%
\pgfusepath{fill}%
\end{pgfscope}%
\begin{pgfscope}%
\pgfpathrectangle{\pgfqpoint{1.020000in}{0.880000in}}{\pgfqpoint{6.160000in}{6.160000in}}%
\pgfusepath{clip}%
\pgfsetbuttcap%
\pgfsetroundjoin%
\definecolor{currentfill}{rgb}{0.532568,0.669801,0.990393}%
\pgfsetfillcolor{currentfill}%
\pgfsetlinewidth{0.000000pt}%
\definecolor{currentstroke}{rgb}{0.000000,0.000000,0.000000}%
\pgfsetstrokecolor{currentstroke}%
\pgfsetdash{}{0pt}%
\pgfpathmoveto{\pgfqpoint{4.435283in}{3.684983in}}%
\pgfpathlineto{\pgfqpoint{4.445604in}{3.664543in}}%
\pgfpathlineto{\pgfqpoint{4.455941in}{3.643315in}}%
\pgfpathlineto{\pgfqpoint{4.488230in}{3.632889in}}%
\pgfpathlineto{\pgfqpoint{4.520493in}{3.622743in}}%
\pgfpathlineto{\pgfqpoint{4.510117in}{3.645399in}}%
\pgfpathlineto{\pgfqpoint{4.499754in}{3.666996in}}%
\pgfpathlineto{\pgfqpoint{4.467531in}{3.675738in}}%
\pgfpathlineto{\pgfqpoint{4.435283in}{3.684983in}}%
\pgfpathclose%
\pgfusepath{fill}%
\end{pgfscope}%
\begin{pgfscope}%
\pgfpathrectangle{\pgfqpoint{1.020000in}{0.880000in}}{\pgfqpoint{6.160000in}{6.160000in}}%
\pgfusepath{clip}%
\pgfsetbuttcap%
\pgfsetroundjoin%
\definecolor{currentfill}{rgb}{0.964835,0.744614,0.643239}%
\pgfsetfillcolor{currentfill}%
\pgfsetlinewidth{0.000000pt}%
\definecolor{currentstroke}{rgb}{0.000000,0.000000,0.000000}%
\pgfsetstrokecolor{currentstroke}%
\pgfsetdash{}{0pt}%
\pgfpathmoveto{\pgfqpoint{2.146702in}{4.591928in}}%
\pgfpathlineto{\pgfqpoint{2.154630in}{4.598309in}}%
\pgfpathlineto{\pgfqpoint{2.162567in}{4.605129in}}%
\pgfpathlineto{\pgfqpoint{2.195363in}{4.618943in}}%
\pgfpathlineto{\pgfqpoint{2.228134in}{4.633406in}}%
\pgfpathlineto{\pgfqpoint{2.220195in}{4.622953in}}%
\pgfpathlineto{\pgfqpoint{2.212263in}{4.613087in}}%
\pgfpathlineto{\pgfqpoint{2.179495in}{4.602227in}}%
\pgfpathlineto{\pgfqpoint{2.146702in}{4.591928in}}%
\pgfpathclose%
\pgfusepath{fill}%
\end{pgfscope}%
\begin{pgfscope}%
\pgfpathrectangle{\pgfqpoint{1.020000in}{0.880000in}}{\pgfqpoint{6.160000in}{6.160000in}}%
\pgfusepath{clip}%
\pgfsetbuttcap%
\pgfsetroundjoin%
\definecolor{currentfill}{rgb}{0.640828,0.760752,0.997846}%
\pgfsetfillcolor{currentfill}%
\pgfsetlinewidth{0.000000pt}%
\definecolor{currentstroke}{rgb}{0.000000,0.000000,0.000000}%
\pgfsetstrokecolor{currentstroke}%
\pgfsetdash{}{0pt}%
\pgfpathmoveto{\pgfqpoint{4.135963in}{3.865391in}}%
\pgfpathlineto{\pgfqpoint{4.146035in}{3.847566in}}%
\pgfpathlineto{\pgfqpoint{4.156126in}{3.829136in}}%
\pgfpathlineto{\pgfqpoint{4.188545in}{3.811034in}}%
\pgfpathlineto{\pgfqpoint{4.220934in}{3.794629in}}%
\pgfpathlineto{\pgfqpoint{4.210799in}{3.812718in}}%
\pgfpathlineto{\pgfqpoint{4.200680in}{3.830279in}}%
\pgfpathlineto{\pgfqpoint{4.168337in}{3.846948in}}%
\pgfpathlineto{\pgfqpoint{4.135963in}{3.865391in}}%
\pgfpathclose%
\pgfusepath{fill}%
\end{pgfscope}%
\begin{pgfscope}%
\pgfpathrectangle{\pgfqpoint{1.020000in}{0.880000in}}{\pgfqpoint{6.160000in}{6.160000in}}%
\pgfusepath{clip}%
\pgfsetbuttcap%
\pgfsetroundjoin%
\definecolor{currentfill}{rgb}{0.318832,0.426605,0.859857}%
\pgfsetfillcolor{currentfill}%
\pgfsetlinewidth{0.000000pt}%
\definecolor{currentstroke}{rgb}{0.000000,0.000000,0.000000}%
\pgfsetstrokecolor{currentstroke}%
\pgfsetdash{}{0pt}%
\pgfpathmoveto{\pgfqpoint{4.969369in}{3.311570in}}%
\pgfpathlineto{\pgfqpoint{4.979966in}{3.267878in}}%
\pgfpathlineto{\pgfqpoint{4.990582in}{3.224849in}}%
\pgfpathlineto{\pgfqpoint{5.022768in}{3.230511in}}%
\pgfpathlineto{\pgfqpoint{5.054969in}{3.240139in}}%
\pgfpathlineto{\pgfqpoint{5.044304in}{3.281337in}}%
\pgfpathlineto{\pgfqpoint{5.033658in}{3.323149in}}%
\pgfpathlineto{\pgfqpoint{5.001510in}{3.315776in}}%
\pgfpathlineto{\pgfqpoint{4.969369in}{3.311570in}}%
\pgfpathclose%
\pgfusepath{fill}%
\end{pgfscope}%
\begin{pgfscope}%
\pgfpathrectangle{\pgfqpoint{1.020000in}{0.880000in}}{\pgfqpoint{6.160000in}{6.160000in}}%
\pgfusepath{clip}%
\pgfsetbuttcap%
\pgfsetroundjoin%
\definecolor{currentfill}{rgb}{0.962701,0.628218,0.507636}%
\pgfsetfillcolor{currentfill}%
\pgfsetlinewidth{0.000000pt}%
\definecolor{currentstroke}{rgb}{0.000000,0.000000,0.000000}%
\pgfsetstrokecolor{currentstroke}%
\pgfsetdash{}{0pt}%
\pgfpathmoveto{\pgfqpoint{2.555225in}{4.775352in}}%
\pgfpathlineto{\pgfqpoint{2.563251in}{4.804685in}}%
\pgfpathlineto{\pgfqpoint{2.571285in}{4.834894in}}%
\pgfpathlineto{\pgfqpoint{2.604036in}{4.847234in}}%
\pgfpathlineto{\pgfqpoint{2.636801in}{4.857618in}}%
\pgfpathlineto{\pgfqpoint{2.628724in}{4.824680in}}%
\pgfpathlineto{\pgfqpoint{2.620656in}{4.792699in}}%
\pgfpathlineto{\pgfqpoint{2.587936in}{4.784852in}}%
\pgfpathlineto{\pgfqpoint{2.555225in}{4.775352in}}%
\pgfpathclose%
\pgfusepath{fill}%
\end{pgfscope}%
\begin{pgfscope}%
\pgfpathrectangle{\pgfqpoint{1.020000in}{0.880000in}}{\pgfqpoint{6.160000in}{6.160000in}}%
\pgfusepath{clip}%
\pgfsetbuttcap%
\pgfsetroundjoin%
\definecolor{currentfill}{rgb}{0.883687,0.856108,0.840258}%
\pgfsetfillcolor{currentfill}%
\pgfsetlinewidth{0.000000pt}%
\definecolor{currentstroke}{rgb}{0.000000,0.000000,0.000000}%
\pgfsetstrokecolor{currentstroke}%
\pgfsetdash{}{0pt}%
\pgfpathmoveto{\pgfqpoint{3.666579in}{4.361686in}}%
\pgfpathlineto{\pgfqpoint{3.676197in}{4.356371in}}%
\pgfpathlineto{\pgfqpoint{3.685842in}{4.349173in}}%
\pgfpathlineto{\pgfqpoint{3.718622in}{4.303721in}}%
\pgfpathlineto{\pgfqpoint{3.751346in}{4.260498in}}%
\pgfpathlineto{\pgfqpoint{3.741648in}{4.269164in}}%
\pgfpathlineto{\pgfqpoint{3.731973in}{4.276246in}}%
\pgfpathlineto{\pgfqpoint{3.699304in}{4.317898in}}%
\pgfpathlineto{\pgfqpoint{3.666579in}{4.361686in}}%
\pgfpathclose%
\pgfusepath{fill}%
\end{pgfscope}%
\begin{pgfscope}%
\pgfpathrectangle{\pgfqpoint{1.020000in}{0.880000in}}{\pgfqpoint{6.160000in}{6.160000in}}%
\pgfusepath{clip}%
\pgfsetbuttcap%
\pgfsetroundjoin%
\definecolor{currentfill}{rgb}{0.368507,0.491141,0.905243}%
\pgfsetfillcolor{currentfill}%
\pgfsetlinewidth{0.000000pt}%
\definecolor{currentstroke}{rgb}{0.000000,0.000000,0.000000}%
\pgfsetstrokecolor{currentstroke}%
\pgfsetdash{}{0pt}%
\pgfpathmoveto{\pgfqpoint{4.819786in}{3.412313in}}%
\pgfpathlineto{\pgfqpoint{4.830308in}{3.371275in}}%
\pgfpathlineto{\pgfqpoint{4.840838in}{3.329236in}}%
\pgfpathlineto{\pgfqpoint{4.872973in}{3.319765in}}%
\pgfpathlineto{\pgfqpoint{4.905104in}{3.313544in}}%
\pgfpathlineto{\pgfqpoint{4.894552in}{3.357131in}}%
\pgfpathlineto{\pgfqpoint{4.884006in}{3.399650in}}%
\pgfpathlineto{\pgfqpoint{4.851900in}{3.404749in}}%
\pgfpathlineto{\pgfqpoint{4.819786in}{3.412313in}}%
\pgfpathclose%
\pgfusepath{fill}%
\end{pgfscope}%
\begin{pgfscope}%
\pgfpathrectangle{\pgfqpoint{1.020000in}{0.880000in}}{\pgfqpoint{6.160000in}{6.160000in}}%
\pgfusepath{clip}%
\pgfsetbuttcap%
\pgfsetroundjoin%
\definecolor{currentfill}{rgb}{0.954853,0.591622,0.471337}%
\pgfsetfillcolor{currentfill}%
\pgfsetlinewidth{0.000000pt}%
\definecolor{currentstroke}{rgb}{0.000000,0.000000,0.000000}%
\pgfsetstrokecolor{currentstroke}%
\pgfsetdash{}{0pt}%
\pgfpathmoveto{\pgfqpoint{3.048077in}{4.855110in}}%
\pgfpathlineto{\pgfqpoint{3.056688in}{4.889362in}}%
\pgfpathlineto{\pgfqpoint{3.065327in}{4.923213in}}%
\pgfpathlineto{\pgfqpoint{3.098385in}{4.896028in}}%
\pgfpathlineto{\pgfqpoint{3.131431in}{4.866297in}}%
\pgfpathlineto{\pgfqpoint{3.122690in}{4.835159in}}%
\pgfpathlineto{\pgfqpoint{3.113976in}{4.803629in}}%
\pgfpathlineto{\pgfqpoint{3.081034in}{4.830487in}}%
\pgfpathlineto{\pgfqpoint{3.048077in}{4.855110in}}%
\pgfpathclose%
\pgfusepath{fill}%
\end{pgfscope}%
\begin{pgfscope}%
\pgfpathrectangle{\pgfqpoint{1.020000in}{0.880000in}}{\pgfqpoint{6.160000in}{6.160000in}}%
\pgfusepath{clip}%
\pgfsetbuttcap%
\pgfsetroundjoin%
\definecolor{currentfill}{rgb}{0.353369,0.472069,0.892570}%
\pgfsetfillcolor{currentfill}%
\pgfsetlinewidth{0.000000pt}%
\definecolor{currentstroke}{rgb}{0.000000,0.000000,0.000000}%
\pgfsetstrokecolor{currentstroke}%
\pgfsetdash{}{0pt}%
\pgfpathmoveto{\pgfqpoint{5.248346in}{3.346489in}}%
\pgfpathlineto{\pgfqpoint{5.259262in}{3.317592in}}%
\pgfpathlineto{\pgfqpoint{5.270208in}{3.290052in}}%
\pgfpathlineto{\pgfqpoint{5.302502in}{3.312880in}}%
\pgfpathlineto{\pgfqpoint{5.334773in}{3.334307in}}%
\pgfpathlineto{\pgfqpoint{5.323745in}{3.358528in}}%
\pgfpathlineto{\pgfqpoint{5.312745in}{3.383709in}}%
\pgfpathlineto{\pgfqpoint{5.280556in}{3.365689in}}%
\pgfpathlineto{\pgfqpoint{5.248346in}{3.346489in}}%
\pgfpathclose%
\pgfusepath{fill}%
\end{pgfscope}%
\begin{pgfscope}%
\pgfpathrectangle{\pgfqpoint{1.020000in}{0.880000in}}{\pgfqpoint{6.160000in}{6.160000in}}%
\pgfusepath{clip}%
\pgfsetbuttcap%
\pgfsetroundjoin%
\definecolor{currentfill}{rgb}{0.964911,0.640159,0.519806}%
\pgfsetfillcolor{currentfill}%
\pgfsetlinewidth{0.000000pt}%
\definecolor{currentstroke}{rgb}{0.000000,0.000000,0.000000}%
\pgfsetstrokecolor{currentstroke}%
\pgfsetdash{}{0pt}%
\pgfpathmoveto{\pgfqpoint{3.197467in}{4.800355in}}%
\pgfpathlineto{\pgfqpoint{3.206342in}{4.827479in}}%
\pgfpathlineto{\pgfqpoint{3.215249in}{4.853472in}}%
\pgfpathlineto{\pgfqpoint{3.248330in}{4.814377in}}%
\pgfpathlineto{\pgfqpoint{3.281378in}{4.773809in}}%
\pgfpathlineto{\pgfqpoint{3.272372in}{4.751364in}}%
\pgfpathlineto{\pgfqpoint{3.263396in}{4.727869in}}%
\pgfpathlineto{\pgfqpoint{3.230447in}{4.764768in}}%
\pgfpathlineto{\pgfqpoint{3.197467in}{4.800355in}}%
\pgfpathclose%
\pgfusepath{fill}%
\end{pgfscope}%
\begin{pgfscope}%
\pgfpathrectangle{\pgfqpoint{1.020000in}{0.880000in}}{\pgfqpoint{6.160000in}{6.160000in}}%
\pgfusepath{clip}%
\pgfsetbuttcap%
\pgfsetroundjoin%
\definecolor{currentfill}{rgb}{0.394042,0.522413,0.924916}%
\pgfsetfillcolor{currentfill}%
\pgfsetlinewidth{0.000000pt}%
\definecolor{currentstroke}{rgb}{0.000000,0.000000,0.000000}%
\pgfsetstrokecolor{currentstroke}%
\pgfsetdash{}{0pt}%
\pgfpathmoveto{\pgfqpoint{5.527701in}{3.419775in}}%
\pgfpathlineto{\pgfqpoint{5.538974in}{3.402337in}}%
\pgfpathlineto{\pgfqpoint{5.550271in}{3.385279in}}%
\pgfpathlineto{\pgfqpoint{5.582355in}{3.393419in}}%
\pgfpathlineto{\pgfqpoint{5.614402in}{3.399902in}}%
\pgfpathlineto{\pgfqpoint{5.603046in}{3.416272in}}%
\pgfpathlineto{\pgfqpoint{5.591713in}{3.432836in}}%
\pgfpathlineto{\pgfqpoint{5.559725in}{3.427053in}}%
\pgfpathlineto{\pgfqpoint{5.527701in}{3.419775in}}%
\pgfpathclose%
\pgfusepath{fill}%
\end{pgfscope}%
\begin{pgfscope}%
\pgfpathrectangle{\pgfqpoint{1.020000in}{0.880000in}}{\pgfqpoint{6.160000in}{6.160000in}}%
\pgfusepath{clip}%
\pgfsetbuttcap%
\pgfsetroundjoin%
\definecolor{currentfill}{rgb}{0.947654,0.565976,0.447478}%
\pgfsetfillcolor{currentfill}%
\pgfsetlinewidth{0.000000pt}%
\definecolor{currentstroke}{rgb}{0.000000,0.000000,0.000000}%
\pgfsetstrokecolor{currentstroke}%
\pgfsetdash{}{0pt}%
\pgfpathmoveto{\pgfqpoint{2.833709in}{4.869695in}}%
\pgfpathlineto{\pgfqpoint{2.841996in}{4.907204in}}%
\pgfpathlineto{\pgfqpoint{2.850302in}{4.945112in}}%
\pgfpathlineto{\pgfqpoint{2.883248in}{4.937672in}}%
\pgfpathlineto{\pgfqpoint{2.916206in}{4.927084in}}%
\pgfpathlineto{\pgfqpoint{2.907814in}{4.889563in}}%
\pgfpathlineto{\pgfqpoint{2.899441in}{4.852439in}}%
\pgfpathlineto{\pgfqpoint{2.866572in}{4.862424in}}%
\pgfpathlineto{\pgfqpoint{2.833709in}{4.869695in}}%
\pgfpathclose%
\pgfusepath{fill}%
\end{pgfscope}%
\begin{pgfscope}%
\pgfpathrectangle{\pgfqpoint{1.020000in}{0.880000in}}{\pgfqpoint{6.160000in}{6.160000in}}%
\pgfusepath{clip}%
\pgfsetbuttcap%
\pgfsetroundjoin%
\definecolor{currentfill}{rgb}{0.708720,0.805721,0.981117}%
\pgfsetfillcolor{currentfill}%
\pgfsetlinewidth{0.000000pt}%
\definecolor{currentstroke}{rgb}{0.000000,0.000000,0.000000}%
\pgfsetstrokecolor{currentstroke}%
\pgfsetdash{}{0pt}%
\pgfpathmoveto{\pgfqpoint{3.986200in}{3.994164in}}%
\pgfpathlineto{\pgfqpoint{3.996147in}{3.977344in}}%
\pgfpathlineto{\pgfqpoint{4.006114in}{3.959563in}}%
\pgfpathlineto{\pgfqpoint{4.038632in}{3.932716in}}%
\pgfpathlineto{\pgfqpoint{4.071112in}{3.908147in}}%
\pgfpathlineto{\pgfqpoint{4.061102in}{3.925602in}}%
\pgfpathlineto{\pgfqpoint{4.051110in}{3.942276in}}%
\pgfpathlineto{\pgfqpoint{4.018674in}{3.967070in}}%
\pgfpathlineto{\pgfqpoint{3.986200in}{3.994164in}}%
\pgfpathclose%
\pgfusepath{fill}%
\end{pgfscope}%
\begin{pgfscope}%
\pgfpathrectangle{\pgfqpoint{1.020000in}{0.880000in}}{\pgfqpoint{6.160000in}{6.160000in}}%
\pgfusepath{clip}%
\pgfsetbuttcap%
\pgfsetroundjoin%
\definecolor{currentfill}{rgb}{0.962708,0.753557,0.655601}%
\pgfsetfillcolor{currentfill}%
\pgfsetlinewidth{0.000000pt}%
\definecolor{currentstroke}{rgb}{0.000000,0.000000,0.000000}%
\pgfsetstrokecolor{currentstroke}%
\pgfsetdash{}{0pt}%
\pgfpathmoveto{\pgfqpoint{2.081033in}{4.573422in}}%
\pgfpathlineto{\pgfqpoint{2.088957in}{4.576533in}}%
\pgfpathlineto{\pgfqpoint{2.096892in}{4.579940in}}%
\pgfpathlineto{\pgfqpoint{2.129745in}{4.592092in}}%
\pgfpathlineto{\pgfqpoint{2.162567in}{4.605129in}}%
\pgfpathlineto{\pgfqpoint{2.154630in}{4.598309in}}%
\pgfpathlineto{\pgfqpoint{2.146702in}{4.591928in}}%
\pgfpathlineto{\pgfqpoint{2.113882in}{4.582295in}}%
\pgfpathlineto{\pgfqpoint{2.081033in}{4.573422in}}%
\pgfpathclose%
\pgfusepath{fill}%
\end{pgfscope}%
\begin{pgfscope}%
\pgfpathrectangle{\pgfqpoint{1.020000in}{0.880000in}}{\pgfqpoint{6.160000in}{6.160000in}}%
\pgfusepath{clip}%
\pgfsetbuttcap%
\pgfsetroundjoin%
\definecolor{currentfill}{rgb}{0.333490,0.446265,0.874452}%
\pgfsetfillcolor{currentfill}%
\pgfsetlinewidth{0.000000pt}%
\definecolor{currentstroke}{rgb}{0.000000,0.000000,0.000000}%
\pgfsetstrokecolor{currentstroke}%
\pgfsetdash{}{0pt}%
\pgfpathmoveto{\pgfqpoint{5.183885in}{3.306600in}}%
\pgfpathlineto{\pgfqpoint{5.194718in}{3.273717in}}%
\pgfpathlineto{\pgfqpoint{5.205583in}{3.242615in}}%
\pgfpathlineto{\pgfqpoint{5.237900in}{3.266393in}}%
\pgfpathlineto{\pgfqpoint{5.270208in}{3.290052in}}%
\pgfpathlineto{\pgfqpoint{5.259262in}{3.317592in}}%
\pgfpathlineto{\pgfqpoint{5.248346in}{3.346489in}}%
\pgfpathlineto{\pgfqpoint{5.216120in}{3.326591in}}%
\pgfpathlineto{\pgfqpoint{5.183885in}{3.306600in}}%
\pgfpathclose%
\pgfusepath{fill}%
\end{pgfscope}%
\begin{pgfscope}%
\pgfpathrectangle{\pgfqpoint{1.020000in}{0.880000in}}{\pgfqpoint{6.160000in}{6.160000in}}%
\pgfusepath{clip}%
\pgfsetbuttcap%
\pgfsetroundjoin%
\definecolor{currentfill}{rgb}{0.435815,0.570707,0.951717}%
\pgfsetfillcolor{currentfill}%
\pgfsetlinewidth{0.000000pt}%
\definecolor{currentstroke}{rgb}{0.000000,0.000000,0.000000}%
\pgfsetstrokecolor{currentstroke}%
\pgfsetdash{}{0pt}%
\pgfpathmoveto{\pgfqpoint{4.670212in}{3.526479in}}%
\pgfpathlineto{\pgfqpoint{4.680683in}{3.494547in}}%
\pgfpathlineto{\pgfqpoint{4.691162in}{3.461108in}}%
\pgfpathlineto{\pgfqpoint{4.723350in}{3.446879in}}%
\pgfpathlineto{\pgfqpoint{4.755514in}{3.433737in}}%
\pgfpathlineto{\pgfqpoint{4.745016in}{3.470540in}}%
\pgfpathlineto{\pgfqpoint{4.734523in}{3.505525in}}%
\pgfpathlineto{\pgfqpoint{4.702378in}{3.515568in}}%
\pgfpathlineto{\pgfqpoint{4.670212in}{3.526479in}}%
\pgfpathclose%
\pgfusepath{fill}%
\end{pgfscope}%
\begin{pgfscope}%
\pgfpathrectangle{\pgfqpoint{1.020000in}{0.880000in}}{\pgfqpoint{6.160000in}{6.160000in}}%
\pgfusepath{clip}%
\pgfsetbuttcap%
\pgfsetroundjoin%
\definecolor{currentfill}{rgb}{0.964911,0.640159,0.519806}%
\pgfsetfillcolor{currentfill}%
\pgfsetlinewidth{0.000000pt}%
\definecolor{currentstroke}{rgb}{0.000000,0.000000,0.000000}%
\pgfsetstrokecolor{currentstroke}%
\pgfsetdash{}{0pt}%
\pgfpathmoveto{\pgfqpoint{2.489823in}{4.752138in}}%
\pgfpathlineto{\pgfqpoint{2.497818in}{4.778292in}}%
\pgfpathlineto{\pgfqpoint{2.505821in}{4.805226in}}%
\pgfpathlineto{\pgfqpoint{2.538548in}{4.820815in}}%
\pgfpathlineto{\pgfqpoint{2.571285in}{4.834894in}}%
\pgfpathlineto{\pgfqpoint{2.563251in}{4.804685in}}%
\pgfpathlineto{\pgfqpoint{2.555225in}{4.775352in}}%
\pgfpathlineto{\pgfqpoint{2.522521in}{4.764382in}}%
\pgfpathlineto{\pgfqpoint{2.489823in}{4.752138in}}%
\pgfpathclose%
\pgfusepath{fill}%
\end{pgfscope}%
\begin{pgfscope}%
\pgfpathrectangle{\pgfqpoint{1.020000in}{0.880000in}}{\pgfqpoint{6.160000in}{6.160000in}}%
\pgfusepath{clip}%
\pgfsetbuttcap%
\pgfsetroundjoin%
\definecolor{currentfill}{rgb}{0.962708,0.753557,0.655601}%
\pgfsetfillcolor{currentfill}%
\pgfsetlinewidth{0.000000pt}%
\definecolor{currentstroke}{rgb}{0.000000,0.000000,0.000000}%
\pgfsetstrokecolor{currentstroke}%
\pgfsetdash{}{0pt}%
\pgfpathmoveto{\pgfqpoint{3.431655in}{4.629812in}}%
\pgfpathlineto{\pgfqpoint{3.440942in}{4.640248in}}%
\pgfpathlineto{\pgfqpoint{3.450264in}{4.648682in}}%
\pgfpathlineto{\pgfqpoint{3.483263in}{4.599163in}}%
\pgfpathlineto{\pgfqpoint{3.516208in}{4.550295in}}%
\pgfpathlineto{\pgfqpoint{3.506809in}{4.544975in}}%
\pgfpathlineto{\pgfqpoint{3.497442in}{4.537891in}}%
\pgfpathlineto{\pgfqpoint{3.464575in}{4.583548in}}%
\pgfpathlineto{\pgfqpoint{3.431655in}{4.629812in}}%
\pgfpathclose%
\pgfusepath{fill}%
\end{pgfscope}%
\begin{pgfscope}%
\pgfpathrectangle{\pgfqpoint{1.020000in}{0.880000in}}{\pgfqpoint{6.160000in}{6.160000in}}%
\pgfusepath{clip}%
\pgfsetbuttcap%
\pgfsetroundjoin%
\definecolor{currentfill}{rgb}{0.843358,0.861820,0.890017}%
\pgfsetfillcolor{currentfill}%
\pgfsetlinewidth{0.000000pt}%
\definecolor{currentstroke}{rgb}{0.000000,0.000000,0.000000}%
\pgfsetstrokecolor{currentstroke}%
\pgfsetdash{}{0pt}%
\pgfpathmoveto{\pgfqpoint{3.751346in}{4.260498in}}%
\pgfpathlineto{\pgfqpoint{3.761070in}{4.250170in}}%
\pgfpathlineto{\pgfqpoint{3.770818in}{4.238118in}}%
\pgfpathlineto{\pgfqpoint{3.803535in}{4.196339in}}%
\pgfpathlineto{\pgfqpoint{3.836199in}{4.157035in}}%
\pgfpathlineto{\pgfqpoint{3.826403in}{4.169804in}}%
\pgfpathlineto{\pgfqpoint{3.816630in}{4.181144in}}%
\pgfpathlineto{\pgfqpoint{3.784015in}{4.219615in}}%
\pgfpathlineto{\pgfqpoint{3.751346in}{4.260498in}}%
\pgfpathclose%
\pgfusepath{fill}%
\end{pgfscope}%
\begin{pgfscope}%
\pgfpathrectangle{\pgfqpoint{1.020000in}{0.880000in}}{\pgfqpoint{6.160000in}{6.160000in}}%
\pgfusepath{clip}%
\pgfsetbuttcap%
\pgfsetroundjoin%
\definecolor{currentfill}{rgb}{0.313946,0.420052,0.854993}%
\pgfsetfillcolor{currentfill}%
\pgfsetlinewidth{0.000000pt}%
\definecolor{currentstroke}{rgb}{0.000000,0.000000,0.000000}%
\pgfsetstrokecolor{currentstroke}%
\pgfsetdash{}{0pt}%
\pgfpathmoveto{\pgfqpoint{5.119411in}{3.269166in}}%
\pgfpathlineto{\pgfqpoint{5.130168in}{3.232512in}}%
\pgfpathlineto{\pgfqpoint{5.140959in}{3.198041in}}%
\pgfpathlineto{\pgfqpoint{5.173267in}{3.219537in}}%
\pgfpathlineto{\pgfqpoint{5.205583in}{3.242615in}}%
\pgfpathlineto{\pgfqpoint{5.194718in}{3.273717in}}%
\pgfpathlineto{\pgfqpoint{5.183885in}{3.306600in}}%
\pgfpathlineto{\pgfqpoint{5.151647in}{3.287208in}}%
\pgfpathlineto{\pgfqpoint{5.119411in}{3.269166in}}%
\pgfpathclose%
\pgfusepath{fill}%
\end{pgfscope}%
\begin{pgfscope}%
\pgfpathrectangle{\pgfqpoint{1.020000in}{0.880000in}}{\pgfqpoint{6.160000in}{6.160000in}}%
\pgfusepath{clip}%
\pgfsetbuttcap%
\pgfsetroundjoin%
\definecolor{currentfill}{rgb}{0.318832,0.426605,0.859857}%
\pgfsetfillcolor{currentfill}%
\pgfsetlinewidth{0.000000pt}%
\definecolor{currentstroke}{rgb}{0.000000,0.000000,0.000000}%
\pgfsetstrokecolor{currentstroke}%
\pgfsetdash{}{0pt}%
\pgfpathmoveto{\pgfqpoint{4.905104in}{3.313544in}}%
\pgfpathlineto{\pgfqpoint{4.915667in}{3.269695in}}%
\pgfpathlineto{\pgfqpoint{4.926249in}{3.226512in}}%
\pgfpathlineto{\pgfqpoint{4.958410in}{3.223474in}}%
\pgfpathlineto{\pgfqpoint{4.990582in}{3.224849in}}%
\pgfpathlineto{\pgfqpoint{4.979966in}{3.267878in}}%
\pgfpathlineto{\pgfqpoint{4.969369in}{3.311570in}}%
\pgfpathlineto{\pgfqpoint{4.937235in}{3.310792in}}%
\pgfpathlineto{\pgfqpoint{4.905104in}{3.313544in}}%
\pgfpathclose%
\pgfusepath{fill}%
\end{pgfscope}%
\begin{pgfscope}%
\pgfpathrectangle{\pgfqpoint{1.020000in}{0.880000in}}{\pgfqpoint{6.160000in}{6.160000in}}%
\pgfusepath{clip}%
\pgfsetbuttcap%
\pgfsetroundjoin%
\definecolor{currentfill}{rgb}{0.500031,0.638508,0.981070}%
\pgfsetfillcolor{currentfill}%
\pgfsetlinewidth{0.000000pt}%
\definecolor{currentstroke}{rgb}{0.000000,0.000000,0.000000}%
\pgfsetstrokecolor{currentstroke}%
\pgfsetdash{}{0pt}%
\pgfpathmoveto{\pgfqpoint{4.520493in}{3.622743in}}%
\pgfpathlineto{\pgfqpoint{4.530883in}{3.599000in}}%
\pgfpathlineto{\pgfqpoint{4.541287in}{3.574189in}}%
\pgfpathlineto{\pgfqpoint{4.573559in}{3.562016in}}%
\pgfpathlineto{\pgfqpoint{4.605803in}{3.549910in}}%
\pgfpathlineto{\pgfqpoint{4.595367in}{3.577287in}}%
\pgfpathlineto{\pgfqpoint{4.584943in}{3.603233in}}%
\pgfpathlineto{\pgfqpoint{4.552731in}{3.612852in}}%
\pgfpathlineto{\pgfqpoint{4.520493in}{3.622743in}}%
\pgfpathclose%
\pgfusepath{fill}%
\end{pgfscope}%
\begin{pgfscope}%
\pgfpathrectangle{\pgfqpoint{1.020000in}{0.880000in}}{\pgfqpoint{6.160000in}{6.160000in}}%
\pgfusepath{clip}%
\pgfsetbuttcap%
\pgfsetroundjoin%
\definecolor{currentfill}{rgb}{0.554312,0.690097,0.995516}%
\pgfsetfillcolor{currentfill}%
\pgfsetlinewidth{0.000000pt}%
\definecolor{currentstroke}{rgb}{0.000000,0.000000,0.000000}%
\pgfsetstrokecolor{currentstroke}%
\pgfsetdash{}{0pt}%
\pgfpathmoveto{\pgfqpoint{4.370713in}{3.705327in}}%
\pgfpathlineto{\pgfqpoint{4.380990in}{3.685609in}}%
\pgfpathlineto{\pgfqpoint{4.391283in}{3.665295in}}%
\pgfpathlineto{\pgfqpoint{4.423625in}{3.654085in}}%
\pgfpathlineto{\pgfqpoint{4.455941in}{3.643315in}}%
\pgfpathlineto{\pgfqpoint{4.445604in}{3.664543in}}%
\pgfpathlineto{\pgfqpoint{4.435283in}{3.684983in}}%
\pgfpathlineto{\pgfqpoint{4.403011in}{3.694812in}}%
\pgfpathlineto{\pgfqpoint{4.370713in}{3.705327in}}%
\pgfpathclose%
\pgfusepath{fill}%
\end{pgfscope}%
\begin{pgfscope}%
\pgfpathrectangle{\pgfqpoint{1.020000in}{0.880000in}}{\pgfqpoint{6.160000in}{6.160000in}}%
\pgfusepath{clip}%
\pgfsetbuttcap%
\pgfsetroundjoin%
\definecolor{currentfill}{rgb}{0.603162,0.731527,0.999565}%
\pgfsetfillcolor{currentfill}%
\pgfsetlinewidth{0.000000pt}%
\definecolor{currentstroke}{rgb}{0.000000,0.000000,0.000000}%
\pgfsetstrokecolor{currentstroke}%
\pgfsetdash{}{0pt}%
\pgfpathmoveto{\pgfqpoint{4.220934in}{3.794629in}}%
\pgfpathlineto{\pgfqpoint{4.231087in}{3.776025in}}%
\pgfpathlineto{\pgfqpoint{4.241257in}{3.756926in}}%
\pgfpathlineto{\pgfqpoint{4.273662in}{3.742299in}}%
\pgfpathlineto{\pgfqpoint{4.306039in}{3.728927in}}%
\pgfpathlineto{\pgfqpoint{4.295824in}{3.747871in}}%
\pgfpathlineto{\pgfqpoint{4.285625in}{3.766313in}}%
\pgfpathlineto{\pgfqpoint{4.253294in}{3.779771in}}%
\pgfpathlineto{\pgfqpoint{4.220934in}{3.794629in}}%
\pgfpathclose%
\pgfusepath{fill}%
\end{pgfscope}%
\begin{pgfscope}%
\pgfpathrectangle{\pgfqpoint{1.020000in}{0.880000in}}{\pgfqpoint{6.160000in}{6.160000in}}%
\pgfusepath{clip}%
\pgfsetbuttcap%
\pgfsetroundjoin%
\definecolor{currentfill}{rgb}{0.388852,0.516298,0.921373}%
\pgfsetfillcolor{currentfill}%
\pgfsetlinewidth{0.000000pt}%
\definecolor{currentstroke}{rgb}{0.000000,0.000000,0.000000}%
\pgfsetstrokecolor{currentstroke}%
\pgfsetdash{}{0pt}%
\pgfpathmoveto{\pgfqpoint{5.463541in}{3.399802in}}%
\pgfpathlineto{\pgfqpoint{5.474749in}{3.381065in}}%
\pgfpathlineto{\pgfqpoint{5.485983in}{3.362999in}}%
\pgfpathlineto{\pgfqpoint{5.518148in}{3.375219in}}%
\pgfpathlineto{\pgfqpoint{5.550271in}{3.385279in}}%
\pgfpathlineto{\pgfqpoint{5.538974in}{3.402337in}}%
\pgfpathlineto{\pgfqpoint{5.527701in}{3.419775in}}%
\pgfpathlineto{\pgfqpoint{5.495641in}{3.410763in}}%
\pgfpathlineto{\pgfqpoint{5.463541in}{3.399802in}}%
\pgfpathclose%
\pgfusepath{fill}%
\end{pgfscope}%
\begin{pgfscope}%
\pgfpathrectangle{\pgfqpoint{1.020000in}{0.880000in}}{\pgfqpoint{6.160000in}{6.160000in}}%
\pgfusepath{clip}%
\pgfsetbuttcap%
\pgfsetroundjoin%
\definecolor{currentfill}{rgb}{0.967317,0.657471,0.538160}%
\pgfsetfillcolor{currentfill}%
\pgfsetlinewidth{0.000000pt}%
\definecolor{currentstroke}{rgb}{0.000000,0.000000,0.000000}%
\pgfsetstrokecolor{currentstroke}%
\pgfsetdash{}{0pt}%
\pgfpathmoveto{\pgfqpoint{2.424436in}{4.724663in}}%
\pgfpathlineto{\pgfqpoint{2.432409in}{4.747243in}}%
\pgfpathlineto{\pgfqpoint{2.440391in}{4.770495in}}%
\pgfpathlineto{\pgfqpoint{2.473104in}{4.788370in}}%
\pgfpathlineto{\pgfqpoint{2.505821in}{4.805226in}}%
\pgfpathlineto{\pgfqpoint{2.497818in}{4.778292in}}%
\pgfpathlineto{\pgfqpoint{2.489823in}{4.752138in}}%
\pgfpathlineto{\pgfqpoint{2.457129in}{4.738828in}}%
\pgfpathlineto{\pgfqpoint{2.424436in}{4.724663in}}%
\pgfpathclose%
\pgfusepath{fill}%
\end{pgfscope}%
\begin{pgfscope}%
\pgfpathrectangle{\pgfqpoint{1.020000in}{0.880000in}}{\pgfqpoint{6.160000in}{6.160000in}}%
\pgfusepath{clip}%
\pgfsetbuttcap%
\pgfsetroundjoin%
\definecolor{currentfill}{rgb}{0.294718,0.393542,0.834384}%
\pgfsetfillcolor{currentfill}%
\pgfsetlinewidth{0.000000pt}%
\definecolor{currentstroke}{rgb}{0.000000,0.000000,0.000000}%
\pgfsetstrokecolor{currentstroke}%
\pgfsetdash{}{0pt}%
\pgfpathmoveto{\pgfqpoint{5.054969in}{3.240139in}}%
\pgfpathlineto{\pgfqpoint{5.065661in}{3.200491in}}%
\pgfpathlineto{\pgfqpoint{5.076388in}{3.163345in}}%
\pgfpathlineto{\pgfqpoint{5.108664in}{3.179027in}}%
\pgfpathlineto{\pgfqpoint{5.140959in}{3.198041in}}%
\pgfpathlineto{\pgfqpoint{5.130168in}{3.232512in}}%
\pgfpathlineto{\pgfqpoint{5.119411in}{3.269166in}}%
\pgfpathlineto{\pgfqpoint{5.087184in}{3.253235in}}%
\pgfpathlineto{\pgfqpoint{5.054969in}{3.240139in}}%
\pgfpathclose%
\pgfusepath{fill}%
\end{pgfscope}%
\begin{pgfscope}%
\pgfpathrectangle{\pgfqpoint{1.020000in}{0.880000in}}{\pgfqpoint{6.160000in}{6.160000in}}%
\pgfusepath{clip}%
\pgfsetbuttcap%
\pgfsetroundjoin%
\definecolor{currentfill}{rgb}{0.383662,0.510183,0.917831}%
\pgfsetfillcolor{currentfill}%
\pgfsetlinewidth{0.000000pt}%
\definecolor{currentstroke}{rgb}{0.000000,0.000000,0.000000}%
\pgfsetstrokecolor{currentstroke}%
\pgfsetdash{}{0pt}%
\pgfpathmoveto{\pgfqpoint{4.755514in}{3.433737in}}%
\pgfpathlineto{\pgfqpoint{4.766019in}{3.395512in}}%
\pgfpathlineto{\pgfqpoint{4.776534in}{3.356405in}}%
\pgfpathlineto{\pgfqpoint{4.808693in}{3.341604in}}%
\pgfpathlineto{\pgfqpoint{4.840838in}{3.329236in}}%
\pgfpathlineto{\pgfqpoint{4.830308in}{3.371275in}}%
\pgfpathlineto{\pgfqpoint{4.819786in}{3.412313in}}%
\pgfpathlineto{\pgfqpoint{4.787658in}{3.422087in}}%
\pgfpathlineto{\pgfqpoint{4.755514in}{3.433737in}}%
\pgfpathclose%
\pgfusepath{fill}%
\end{pgfscope}%
\begin{pgfscope}%
\pgfpathrectangle{\pgfqpoint{1.020000in}{0.880000in}}{\pgfqpoint{6.160000in}{6.160000in}}%
\pgfusepath{clip}%
\pgfsetbuttcap%
\pgfsetroundjoin%
\definecolor{currentfill}{rgb}{0.667253,0.779176,0.992959}%
\pgfsetfillcolor{currentfill}%
\pgfsetlinewidth{0.000000pt}%
\definecolor{currentstroke}{rgb}{0.000000,0.000000,0.000000}%
\pgfsetstrokecolor{currentstroke}%
\pgfsetdash{}{0pt}%
\pgfpathmoveto{\pgfqpoint{4.071112in}{3.908147in}}%
\pgfpathlineto{\pgfqpoint{4.081140in}{3.889934in}}%
\pgfpathlineto{\pgfqpoint{4.091186in}{3.870992in}}%
\pgfpathlineto{\pgfqpoint{4.123673in}{3.849076in}}%
\pgfpathlineto{\pgfqpoint{4.156126in}{3.829136in}}%
\pgfpathlineto{\pgfqpoint{4.146035in}{3.847566in}}%
\pgfpathlineto{\pgfqpoint{4.135963in}{3.865391in}}%
\pgfpathlineto{\pgfqpoint{4.103555in}{3.885746in}}%
\pgfpathlineto{\pgfqpoint{4.071112in}{3.908147in}}%
\pgfpathclose%
\pgfusepath{fill}%
\end{pgfscope}%
\begin{pgfscope}%
\pgfpathrectangle{\pgfqpoint{1.020000in}{0.880000in}}{\pgfqpoint{6.160000in}{6.160000in}}%
\pgfusepath{clip}%
\pgfsetbuttcap%
\pgfsetroundjoin%
\definecolor{currentfill}{rgb}{0.796064,0.848693,0.933471}%
\pgfsetfillcolor{currentfill}%
\pgfsetlinewidth{0.000000pt}%
\definecolor{currentstroke}{rgb}{0.000000,0.000000,0.000000}%
\pgfsetstrokecolor{currentstroke}%
\pgfsetdash{}{0pt}%
\pgfpathmoveto{\pgfqpoint{3.836199in}{4.157035in}}%
\pgfpathlineto{\pgfqpoint{3.846017in}{4.142804in}}%
\pgfpathlineto{\pgfqpoint{3.855856in}{4.127098in}}%
\pgfpathlineto{\pgfqpoint{3.888514in}{4.090083in}}%
\pgfpathlineto{\pgfqpoint{3.921122in}{4.055609in}}%
\pgfpathlineto{\pgfqpoint{3.911238in}{4.071395in}}%
\pgfpathlineto{\pgfqpoint{3.901374in}{4.085972in}}%
\pgfpathlineto{\pgfqpoint{3.868811in}{4.120244in}}%
\pgfpathlineto{\pgfqpoint{3.836199in}{4.157035in}}%
\pgfpathclose%
\pgfusepath{fill}%
\end{pgfscope}%
\begin{pgfscope}%
\pgfpathrectangle{\pgfqpoint{1.020000in}{0.880000in}}{\pgfqpoint{6.160000in}{6.160000in}}%
\pgfusepath{clip}%
\pgfsetbuttcap%
\pgfsetroundjoin%
\definecolor{currentfill}{rgb}{0.944055,0.553153,0.435548}%
\pgfsetfillcolor{currentfill}%
\pgfsetlinewidth{0.000000pt}%
\definecolor{currentstroke}{rgb}{0.000000,0.000000,0.000000}%
\pgfsetstrokecolor{currentstroke}%
\pgfsetdash{}{0pt}%
\pgfpathmoveto{\pgfqpoint{2.768016in}{4.876049in}}%
\pgfpathlineto{\pgfqpoint{2.776226in}{4.913079in}}%
\pgfpathlineto{\pgfqpoint{2.784454in}{4.950495in}}%
\pgfpathlineto{\pgfqpoint{2.817369in}{4.949381in}}%
\pgfpathlineto{\pgfqpoint{2.850302in}{4.945112in}}%
\pgfpathlineto{\pgfqpoint{2.841996in}{4.907204in}}%
\pgfpathlineto{\pgfqpoint{2.833709in}{4.869695in}}%
\pgfpathlineto{\pgfqpoint{2.800857in}{4.874232in}}%
\pgfpathlineto{\pgfqpoint{2.768016in}{4.876049in}}%
\pgfpathclose%
\pgfusepath{fill}%
\end{pgfscope}%
\begin{pgfscope}%
\pgfpathrectangle{\pgfqpoint{1.020000in}{0.880000in}}{\pgfqpoint{6.160000in}{6.160000in}}%
\pgfusepath{clip}%
\pgfsetbuttcap%
\pgfsetroundjoin%
\definecolor{currentfill}{rgb}{0.388852,0.516298,0.921373}%
\pgfsetfillcolor{currentfill}%
\pgfsetlinewidth{0.000000pt}%
\definecolor{currentstroke}{rgb}{0.000000,0.000000,0.000000}%
\pgfsetstrokecolor{currentstroke}%
\pgfsetdash{}{0pt}%
\pgfpathmoveto{\pgfqpoint{5.678391in}{3.408958in}}%
\pgfpathlineto{\pgfqpoint{5.689824in}{3.393144in}}%
\pgfpathlineto{\pgfqpoint{5.701279in}{3.377447in}}%
\pgfpathlineto{\pgfqpoint{5.733278in}{3.380710in}}%
\pgfpathlineto{\pgfqpoint{5.721797in}{3.396327in}}%
\pgfpathlineto{\pgfqpoint{5.710338in}{3.412019in}}%
\pgfpathlineto{\pgfqpoint{5.678391in}{3.408958in}}%
\pgfpathclose%
\pgfusepath{fill}%
\end{pgfscope}%
\begin{pgfscope}%
\pgfpathrectangle{\pgfqpoint{1.020000in}{0.880000in}}{\pgfqpoint{6.160000in}{6.160000in}}%
\pgfusepath{clip}%
\pgfsetbuttcap%
\pgfsetroundjoin%
\definecolor{currentfill}{rgb}{0.950956,0.786875,0.704761}%
\pgfsetfillcolor{currentfill}%
\pgfsetlinewidth{0.000000pt}%
\definecolor{currentstroke}{rgb}{0.000000,0.000000,0.000000}%
\pgfsetstrokecolor{currentstroke}%
\pgfsetdash{}{0pt}%
\pgfpathmoveto{\pgfqpoint{3.516208in}{4.550295in}}%
\pgfpathlineto{\pgfqpoint{3.525638in}{4.553624in}}%
\pgfpathlineto{\pgfqpoint{3.535102in}{4.554750in}}%
\pgfpathlineto{\pgfqpoint{3.568058in}{4.504243in}}%
\pgfpathlineto{\pgfqpoint{3.600957in}{4.455062in}}%
\pgfpathlineto{\pgfqpoint{3.591428in}{4.456358in}}%
\pgfpathlineto{\pgfqpoint{3.581929in}{4.455749in}}%
\pgfpathlineto{\pgfqpoint{3.549096in}{4.502396in}}%
\pgfpathlineto{\pgfqpoint{3.516208in}{4.550295in}}%
\pgfpathclose%
\pgfusepath{fill}%
\end{pgfscope}%
\begin{pgfscope}%
\pgfpathrectangle{\pgfqpoint{1.020000in}{0.880000in}}{\pgfqpoint{6.160000in}{6.160000in}}%
\pgfusepath{clip}%
\pgfsetbuttcap%
\pgfsetroundjoin%
\definecolor{currentfill}{rgb}{0.373552,0.497499,0.909467}%
\pgfsetfillcolor{currentfill}%
\pgfsetlinewidth{0.000000pt}%
\definecolor{currentstroke}{rgb}{0.000000,0.000000,0.000000}%
\pgfsetstrokecolor{currentstroke}%
\pgfsetdash{}{0pt}%
\pgfpathmoveto{\pgfqpoint{5.399228in}{3.371424in}}%
\pgfpathlineto{\pgfqpoint{5.410364in}{3.350873in}}%
\pgfpathlineto{\pgfqpoint{5.421532in}{3.331410in}}%
\pgfpathlineto{\pgfqpoint{5.453778in}{3.348432in}}%
\pgfpathlineto{\pgfqpoint{5.485983in}{3.362999in}}%
\pgfpathlineto{\pgfqpoint{5.474749in}{3.381065in}}%
\pgfpathlineto{\pgfqpoint{5.463541in}{3.399802in}}%
\pgfpathlineto{\pgfqpoint{5.431404in}{3.386721in}}%
\pgfpathlineto{\pgfqpoint{5.399228in}{3.371424in}}%
\pgfpathclose%
\pgfusepath{fill}%
\end{pgfscope}%
\begin{pgfscope}%
\pgfpathrectangle{\pgfqpoint{1.020000in}{0.880000in}}{\pgfqpoint{6.160000in}{6.160000in}}%
\pgfusepath{clip}%
\pgfsetbuttcap%
\pgfsetroundjoin%
\definecolor{currentfill}{rgb}{0.968500,0.673977,0.556649}%
\pgfsetfillcolor{currentfill}%
\pgfsetlinewidth{0.000000pt}%
\definecolor{currentstroke}{rgb}{0.000000,0.000000,0.000000}%
\pgfsetstrokecolor{currentstroke}%
\pgfsetdash{}{0pt}%
\pgfpathmoveto{\pgfqpoint{2.359042in}{4.694621in}}%
\pgfpathlineto{\pgfqpoint{2.367001in}{4.713377in}}%
\pgfpathlineto{\pgfqpoint{2.374971in}{4.732692in}}%
\pgfpathlineto{\pgfqpoint{2.407682in}{4.751853in}}%
\pgfpathlineto{\pgfqpoint{2.440391in}{4.770495in}}%
\pgfpathlineto{\pgfqpoint{2.432409in}{4.747243in}}%
\pgfpathlineto{\pgfqpoint{2.424436in}{4.724663in}}%
\pgfpathlineto{\pgfqpoint{2.391741in}{4.709856in}}%
\pgfpathlineto{\pgfqpoint{2.359042in}{4.694621in}}%
\pgfpathclose%
\pgfusepath{fill}%
\end{pgfscope}%
\begin{pgfscope}%
\pgfpathrectangle{\pgfqpoint{1.020000in}{0.880000in}}{\pgfqpoint{6.160000in}{6.160000in}}%
\pgfusepath{clip}%
\pgfsetbuttcap%
\pgfsetroundjoin%
\definecolor{currentfill}{rgb}{0.944055,0.553153,0.435548}%
\pgfsetfillcolor{currentfill}%
\pgfsetlinewidth{0.000000pt}%
\definecolor{currentstroke}{rgb}{0.000000,0.000000,0.000000}%
\pgfsetstrokecolor{currentstroke}%
\pgfsetdash{}{0pt}%
\pgfpathmoveto{\pgfqpoint{2.982142in}{4.896762in}}%
\pgfpathlineto{\pgfqpoint{2.990653in}{4.933048in}}%
\pgfpathlineto{\pgfqpoint{2.999192in}{4.968924in}}%
\pgfpathlineto{\pgfqpoint{3.032260in}{4.947588in}}%
\pgfpathlineto{\pgfqpoint{3.065327in}{4.923213in}}%
\pgfpathlineto{\pgfqpoint{3.056688in}{4.889362in}}%
\pgfpathlineto{\pgfqpoint{3.048077in}{4.855110in}}%
\pgfpathlineto{\pgfqpoint{3.015112in}{4.877270in}}%
\pgfpathlineto{\pgfqpoint{2.982142in}{4.896762in}}%
\pgfpathclose%
\pgfusepath{fill}%
\end{pgfscope}%
\begin{pgfscope}%
\pgfpathrectangle{\pgfqpoint{1.020000in}{0.880000in}}{\pgfqpoint{6.160000in}{6.160000in}}%
\pgfusepath{clip}%
\pgfsetbuttcap%
\pgfsetroundjoin%
\definecolor{currentfill}{rgb}{0.967317,0.657471,0.538160}%
\pgfsetfillcolor{currentfill}%
\pgfsetlinewidth{0.000000pt}%
\definecolor{currentstroke}{rgb}{0.000000,0.000000,0.000000}%
\pgfsetstrokecolor{currentstroke}%
\pgfsetdash{}{0pt}%
\pgfpathmoveto{\pgfqpoint{3.281378in}{4.773809in}}%
\pgfpathlineto{\pgfqpoint{3.290419in}{4.794845in}}%
\pgfpathlineto{\pgfqpoint{3.299496in}{4.814110in}}%
\pgfpathlineto{\pgfqpoint{3.332602in}{4.768887in}}%
\pgfpathlineto{\pgfqpoint{3.365665in}{4.722841in}}%
\pgfpathlineto{\pgfqpoint{3.356497in}{4.707083in}}%
\pgfpathlineto{\pgfqpoint{3.347363in}{4.689715in}}%
\pgfpathlineto{\pgfqpoint{3.314391in}{4.732134in}}%
\pgfpathlineto{\pgfqpoint{3.281378in}{4.773809in}}%
\pgfpathclose%
\pgfusepath{fill}%
\end{pgfscope}%
\begin{pgfscope}%
\pgfpathrectangle{\pgfqpoint{1.020000in}{0.880000in}}{\pgfqpoint{6.160000in}{6.160000in}}%
\pgfusepath{clip}%
\pgfsetbuttcap%
\pgfsetroundjoin%
\definecolor{currentfill}{rgb}{0.457046,0.594006,0.963029}%
\pgfsetfillcolor{currentfill}%
\pgfsetlinewidth{0.000000pt}%
\definecolor{currentstroke}{rgb}{0.000000,0.000000,0.000000}%
\pgfsetstrokecolor{currentstroke}%
\pgfsetdash{}{0pt}%
\pgfpathmoveto{\pgfqpoint{4.605803in}{3.549910in}}%
\pgfpathlineto{\pgfqpoint{4.616249in}{3.521188in}}%
\pgfpathlineto{\pgfqpoint{4.626707in}{3.491290in}}%
\pgfpathlineto{\pgfqpoint{4.658949in}{3.476031in}}%
\pgfpathlineto{\pgfqpoint{4.691162in}{3.461108in}}%
\pgfpathlineto{\pgfqpoint{4.680683in}{3.494547in}}%
\pgfpathlineto{\pgfqpoint{4.670212in}{3.526479in}}%
\pgfpathlineto{\pgfqpoint{4.638020in}{3.538003in}}%
\pgfpathlineto{\pgfqpoint{4.605803in}{3.549910in}}%
\pgfpathclose%
\pgfusepath{fill}%
\end{pgfscope}%
\begin{pgfscope}%
\pgfpathrectangle{\pgfqpoint{1.020000in}{0.880000in}}{\pgfqpoint{6.160000in}{6.160000in}}%
\pgfusepath{clip}%
\pgfsetbuttcap%
\pgfsetroundjoin%
\definecolor{currentfill}{rgb}{0.285273,0.380129,0.823469}%
\pgfsetfillcolor{currentfill}%
\pgfsetlinewidth{0.000000pt}%
\definecolor{currentstroke}{rgb}{0.000000,0.000000,0.000000}%
\pgfsetstrokecolor{currentstroke}%
\pgfsetdash{}{0pt}%
\pgfpathmoveto{\pgfqpoint{4.990582in}{3.224849in}}%
\pgfpathlineto{\pgfqpoint{5.001224in}{3.183484in}}%
\pgfpathlineto{\pgfqpoint{5.011901in}{3.144800in}}%
\pgfpathlineto{\pgfqpoint{5.044133in}{3.151742in}}%
\pgfpathlineto{\pgfqpoint{5.076388in}{3.163345in}}%
\pgfpathlineto{\pgfqpoint{5.065661in}{3.200491in}}%
\pgfpathlineto{\pgfqpoint{5.054969in}{3.240139in}}%
\pgfpathlineto{\pgfqpoint{5.022768in}{3.230511in}}%
\pgfpathlineto{\pgfqpoint{4.990582in}{3.224849in}}%
\pgfpathclose%
\pgfusepath{fill}%
\end{pgfscope}%
\begin{pgfscope}%
\pgfpathrectangle{\pgfqpoint{1.020000in}{0.880000in}}{\pgfqpoint{6.160000in}{6.160000in}}%
\pgfusepath{clip}%
\pgfsetbuttcap%
\pgfsetroundjoin%
\definecolor{currentfill}{rgb}{0.328604,0.439712,0.869587}%
\pgfsetfillcolor{currentfill}%
\pgfsetlinewidth{0.000000pt}%
\definecolor{currentstroke}{rgb}{0.000000,0.000000,0.000000}%
\pgfsetstrokecolor{currentstroke}%
\pgfsetdash{}{0pt}%
\pgfpathmoveto{\pgfqpoint{4.840838in}{3.329236in}}%
\pgfpathlineto{\pgfqpoint{4.851380in}{3.286952in}}%
\pgfpathlineto{\pgfqpoint{4.861940in}{3.245295in}}%
\pgfpathlineto{\pgfqpoint{4.894094in}{3.233880in}}%
\pgfpathlineto{\pgfqpoint{4.926249in}{3.226512in}}%
\pgfpathlineto{\pgfqpoint{4.915667in}{3.269695in}}%
\pgfpathlineto{\pgfqpoint{4.905104in}{3.313544in}}%
\pgfpathlineto{\pgfqpoint{4.872973in}{3.319765in}}%
\pgfpathlineto{\pgfqpoint{4.840838in}{3.329236in}}%
\pgfpathclose%
\pgfusepath{fill}%
\end{pgfscope}%
\begin{pgfscope}%
\pgfpathrectangle{\pgfqpoint{1.020000in}{0.880000in}}{\pgfqpoint{6.160000in}{6.160000in}}%
\pgfusepath{clip}%
\pgfsetbuttcap%
\pgfsetroundjoin%
\definecolor{currentfill}{rgb}{0.969683,0.690484,0.575138}%
\pgfsetfillcolor{currentfill}%
\pgfsetlinewidth{0.000000pt}%
\definecolor{currentstroke}{rgb}{0.000000,0.000000,0.000000}%
\pgfsetstrokecolor{currentstroke}%
\pgfsetdash{}{0pt}%
\pgfpathmoveto{\pgfqpoint{2.293617in}{4.663685in}}%
\pgfpathlineto{\pgfqpoint{2.301569in}{4.678512in}}%
\pgfpathlineto{\pgfqpoint{2.309532in}{4.693780in}}%
\pgfpathlineto{\pgfqpoint{2.342256in}{4.713257in}}%
\pgfpathlineto{\pgfqpoint{2.374971in}{4.732692in}}%
\pgfpathlineto{\pgfqpoint{2.367001in}{4.713377in}}%
\pgfpathlineto{\pgfqpoint{2.359042in}{4.694621in}}%
\pgfpathlineto{\pgfqpoint{2.326335in}{4.679164in}}%
\pgfpathlineto{\pgfqpoint{2.293617in}{4.663685in}}%
\pgfpathclose%
\pgfusepath{fill}%
\end{pgfscope}%
\begin{pgfscope}%
\pgfpathrectangle{\pgfqpoint{1.020000in}{0.880000in}}{\pgfqpoint{6.160000in}{6.160000in}}%
\pgfusepath{clip}%
\pgfsetbuttcap%
\pgfsetroundjoin%
\definecolor{currentfill}{rgb}{0.358415,0.478426,0.896795}%
\pgfsetfillcolor{currentfill}%
\pgfsetlinewidth{0.000000pt}%
\definecolor{currentstroke}{rgb}{0.000000,0.000000,0.000000}%
\pgfsetstrokecolor{currentstroke}%
\pgfsetdash{}{0pt}%
\pgfpathmoveto{\pgfqpoint{5.334773in}{3.334307in}}%
\pgfpathlineto{\pgfqpoint{5.345834in}{3.311408in}}%
\pgfpathlineto{\pgfqpoint{5.356931in}{3.290146in}}%
\pgfpathlineto{\pgfqpoint{5.389249in}{3.311936in}}%
\pgfpathlineto{\pgfqpoint{5.421532in}{3.331410in}}%
\pgfpathlineto{\pgfqpoint{5.410364in}{3.350873in}}%
\pgfpathlineto{\pgfqpoint{5.399228in}{3.371424in}}%
\pgfpathlineto{\pgfqpoint{5.367017in}{3.353911in}}%
\pgfpathlineto{\pgfqpoint{5.334773in}{3.334307in}}%
\pgfpathclose%
\pgfusepath{fill}%
\end{pgfscope}%
\begin{pgfscope}%
\pgfpathrectangle{\pgfqpoint{1.020000in}{0.880000in}}{\pgfqpoint{6.160000in}{6.160000in}}%
\pgfusepath{clip}%
\pgfsetbuttcap%
\pgfsetroundjoin%
\definecolor{currentfill}{rgb}{0.748682,0.827679,0.963334}%
\pgfsetfillcolor{currentfill}%
\pgfsetlinewidth{0.000000pt}%
\definecolor{currentstroke}{rgb}{0.000000,0.000000,0.000000}%
\pgfsetstrokecolor{currentstroke}%
\pgfsetdash{}{0pt}%
\pgfpathmoveto{\pgfqpoint{3.921122in}{4.055609in}}%
\pgfpathlineto{\pgfqpoint{3.931025in}{4.038620in}}%
\pgfpathlineto{\pgfqpoint{3.940949in}{4.020446in}}%
\pgfpathlineto{\pgfqpoint{3.973553in}{3.988781in}}%
\pgfpathlineto{\pgfqpoint{4.006114in}{3.959563in}}%
\pgfpathlineto{\pgfqpoint{3.996147in}{3.977344in}}%
\pgfpathlineto{\pgfqpoint{3.986200in}{3.994164in}}%
\pgfpathlineto{\pgfqpoint{3.953683in}{4.023653in}}%
\pgfpathlineto{\pgfqpoint{3.921122in}{4.055609in}}%
\pgfpathclose%
\pgfusepath{fill}%
\end{pgfscope}%
\begin{pgfscope}%
\pgfpathrectangle{\pgfqpoint{1.020000in}{0.880000in}}{\pgfqpoint{6.160000in}{6.160000in}}%
\pgfusepath{clip}%
\pgfsetbuttcap%
\pgfsetroundjoin%
\definecolor{currentfill}{rgb}{0.953054,0.585211,0.465373}%
\pgfsetfillcolor{currentfill}%
\pgfsetlinewidth{0.000000pt}%
\definecolor{currentstroke}{rgb}{0.000000,0.000000,0.000000}%
\pgfsetstrokecolor{currentstroke}%
\pgfsetdash{}{0pt}%
\pgfpathmoveto{\pgfqpoint{3.131431in}{4.866297in}}%
\pgfpathlineto{\pgfqpoint{3.140204in}{4.896647in}}%
\pgfpathlineto{\pgfqpoint{3.149011in}{4.925798in}}%
\pgfpathlineto{\pgfqpoint{3.182141in}{4.890731in}}%
\pgfpathlineto{\pgfqpoint{3.215249in}{4.853472in}}%
\pgfpathlineto{\pgfqpoint{3.206342in}{4.827479in}}%
\pgfpathlineto{\pgfqpoint{3.197467in}{4.800355in}}%
\pgfpathlineto{\pgfqpoint{3.164460in}{4.834304in}}%
\pgfpathlineto{\pgfqpoint{3.131431in}{4.866297in}}%
\pgfpathclose%
\pgfusepath{fill}%
\end{pgfscope}%
\begin{pgfscope}%
\pgfpathrectangle{\pgfqpoint{1.020000in}{0.880000in}}{\pgfqpoint{6.160000in}{6.160000in}}%
\pgfusepath{clip}%
\pgfsetbuttcap%
\pgfsetroundjoin%
\definecolor{currentfill}{rgb}{0.516260,0.654498,0.986407}%
\pgfsetfillcolor{currentfill}%
\pgfsetlinewidth{0.000000pt}%
\definecolor{currentstroke}{rgb}{0.000000,0.000000,0.000000}%
\pgfsetstrokecolor{currentstroke}%
\pgfsetdash{}{0pt}%
\pgfpathmoveto{\pgfqpoint{4.455941in}{3.643315in}}%
\pgfpathlineto{\pgfqpoint{4.466292in}{3.621294in}}%
\pgfpathlineto{\pgfqpoint{4.476658in}{3.598500in}}%
\pgfpathlineto{\pgfqpoint{4.508987in}{3.586355in}}%
\pgfpathlineto{\pgfqpoint{4.541287in}{3.574189in}}%
\pgfpathlineto{\pgfqpoint{4.530883in}{3.599000in}}%
\pgfpathlineto{\pgfqpoint{4.520493in}{3.622743in}}%
\pgfpathlineto{\pgfqpoint{4.488230in}{3.632889in}}%
\pgfpathlineto{\pgfqpoint{4.455941in}{3.643315in}}%
\pgfpathclose%
\pgfusepath{fill}%
\end{pgfscope}%
\begin{pgfscope}%
\pgfpathrectangle{\pgfqpoint{1.020000in}{0.880000in}}{\pgfqpoint{6.160000in}{6.160000in}}%
\pgfusepath{clip}%
\pgfsetbuttcap%
\pgfsetroundjoin%
\definecolor{currentfill}{rgb}{0.388852,0.516298,0.921373}%
\pgfsetfillcolor{currentfill}%
\pgfsetlinewidth{0.000000pt}%
\definecolor{currentstroke}{rgb}{0.000000,0.000000,0.000000}%
\pgfsetstrokecolor{currentstroke}%
\pgfsetdash{}{0pt}%
\pgfpathmoveto{\pgfqpoint{5.614402in}{3.399902in}}%
\pgfpathlineto{\pgfqpoint{5.625781in}{3.383763in}}%
\pgfpathlineto{\pgfqpoint{5.637184in}{3.367880in}}%
\pgfpathlineto{\pgfqpoint{5.669249in}{3.373251in}}%
\pgfpathlineto{\pgfqpoint{5.701279in}{3.377447in}}%
\pgfpathlineto{\pgfqpoint{5.689824in}{3.393144in}}%
\pgfpathlineto{\pgfqpoint{5.678391in}{3.408958in}}%
\pgfpathlineto{\pgfqpoint{5.646413in}{3.404997in}}%
\pgfpathlineto{\pgfqpoint{5.614402in}{3.399902in}}%
\pgfpathclose%
\pgfusepath{fill}%
\end{pgfscope}%
\begin{pgfscope}%
\pgfpathrectangle{\pgfqpoint{1.020000in}{0.880000in}}{\pgfqpoint{6.160000in}{6.160000in}}%
\pgfusepath{clip}%
\pgfsetbuttcap%
\pgfsetroundjoin%
\definecolor{currentfill}{rgb}{0.570616,0.704109,0.997195}%
\pgfsetfillcolor{currentfill}%
\pgfsetlinewidth{0.000000pt}%
\definecolor{currentstroke}{rgb}{0.000000,0.000000,0.000000}%
\pgfsetstrokecolor{currentstroke}%
\pgfsetdash{}{0pt}%
\pgfpathmoveto{\pgfqpoint{4.306039in}{3.728927in}}%
\pgfpathlineto{\pgfqpoint{4.316271in}{3.709486in}}%
\pgfpathlineto{\pgfqpoint{4.326520in}{3.689564in}}%
\pgfpathlineto{\pgfqpoint{4.358915in}{3.677073in}}%
\pgfpathlineto{\pgfqpoint{4.391283in}{3.665295in}}%
\pgfpathlineto{\pgfqpoint{4.380990in}{3.685609in}}%
\pgfpathlineto{\pgfqpoint{4.370713in}{3.705327in}}%
\pgfpathlineto{\pgfqpoint{4.338389in}{3.716653in}}%
\pgfpathlineto{\pgfqpoint{4.306039in}{3.728927in}}%
\pgfpathclose%
\pgfusepath{fill}%
\end{pgfscope}%
\begin{pgfscope}%
\pgfpathrectangle{\pgfqpoint{1.020000in}{0.880000in}}{\pgfqpoint{6.160000in}{6.160000in}}%
\pgfusepath{clip}%
\pgfsetbuttcap%
\pgfsetroundjoin%
\definecolor{currentfill}{rgb}{0.925563,0.825517,0.771136}%
\pgfsetfillcolor{currentfill}%
\pgfsetlinewidth{0.000000pt}%
\definecolor{currentstroke}{rgb}{0.000000,0.000000,0.000000}%
\pgfsetstrokecolor{currentstroke}%
\pgfsetdash{}{0pt}%
\pgfpathmoveto{\pgfqpoint{3.600957in}{4.455062in}}%
\pgfpathlineto{\pgfqpoint{3.610515in}{4.451697in}}%
\pgfpathlineto{\pgfqpoint{3.620105in}{4.446121in}}%
\pgfpathlineto{\pgfqpoint{3.653003in}{4.396703in}}%
\pgfpathlineto{\pgfqpoint{3.685842in}{4.349173in}}%
\pgfpathlineto{\pgfqpoint{3.676197in}{4.356371in}}%
\pgfpathlineto{\pgfqpoint{3.666579in}{4.361686in}}%
\pgfpathlineto{\pgfqpoint{3.633797in}{4.407469in}}%
\pgfpathlineto{\pgfqpoint{3.600957in}{4.455062in}}%
\pgfpathclose%
\pgfusepath{fill}%
\end{pgfscope}%
\begin{pgfscope}%
\pgfpathrectangle{\pgfqpoint{1.020000in}{0.880000in}}{\pgfqpoint{6.160000in}{6.160000in}}%
\pgfusepath{clip}%
\pgfsetbuttcap%
\pgfsetroundjoin%
\definecolor{currentfill}{rgb}{0.630089,0.752516,0.998508}%
\pgfsetfillcolor{currentfill}%
\pgfsetlinewidth{0.000000pt}%
\definecolor{currentstroke}{rgb}{0.000000,0.000000,0.000000}%
\pgfsetstrokecolor{currentstroke}%
\pgfsetdash{}{0pt}%
\pgfpathmoveto{\pgfqpoint{4.156126in}{3.829136in}}%
\pgfpathlineto{\pgfqpoint{4.166233in}{3.810127in}}%
\pgfpathlineto{\pgfqpoint{4.176358in}{3.790573in}}%
\pgfpathlineto{\pgfqpoint{4.208823in}{3.772965in}}%
\pgfpathlineto{\pgfqpoint{4.241257in}{3.756926in}}%
\pgfpathlineto{\pgfqpoint{4.231087in}{3.776025in}}%
\pgfpathlineto{\pgfqpoint{4.220934in}{3.794629in}}%
\pgfpathlineto{\pgfqpoint{4.188545in}{3.811034in}}%
\pgfpathlineto{\pgfqpoint{4.156126in}{3.829136in}}%
\pgfpathclose%
\pgfusepath{fill}%
\end{pgfscope}%
\begin{pgfscope}%
\pgfpathrectangle{\pgfqpoint{1.020000in}{0.880000in}}{\pgfqpoint{6.160000in}{6.160000in}}%
\pgfusepath{clip}%
\pgfsetbuttcap%
\pgfsetroundjoin%
\definecolor{currentfill}{rgb}{0.944055,0.553153,0.435548}%
\pgfsetfillcolor{currentfill}%
\pgfsetlinewidth{0.000000pt}%
\definecolor{currentstroke}{rgb}{0.000000,0.000000,0.000000}%
\pgfsetstrokecolor{currentstroke}%
\pgfsetdash{}{0pt}%
\pgfpathmoveto{\pgfqpoint{2.702377in}{4.871759in}}%
\pgfpathlineto{\pgfqpoint{2.710521in}{4.907464in}}%
\pgfpathlineto{\pgfqpoint{2.718684in}{4.943533in}}%
\pgfpathlineto{\pgfqpoint{2.751559in}{4.948514in}}%
\pgfpathlineto{\pgfqpoint{2.784454in}{4.950495in}}%
\pgfpathlineto{\pgfqpoint{2.776226in}{4.913079in}}%
\pgfpathlineto{\pgfqpoint{2.768016in}{4.876049in}}%
\pgfpathlineto{\pgfqpoint{2.735189in}{4.875197in}}%
\pgfpathlineto{\pgfqpoint{2.702377in}{4.871759in}}%
\pgfpathclose%
\pgfusepath{fill}%
\end{pgfscope}%
\begin{pgfscope}%
\pgfpathrectangle{\pgfqpoint{1.020000in}{0.880000in}}{\pgfqpoint{6.160000in}{6.160000in}}%
\pgfusepath{clip}%
\pgfsetbuttcap%
\pgfsetroundjoin%
\definecolor{currentfill}{rgb}{0.338377,0.452819,0.879317}%
\pgfsetfillcolor{currentfill}%
\pgfsetlinewidth{0.000000pt}%
\definecolor{currentstroke}{rgb}{0.000000,0.000000,0.000000}%
\pgfsetstrokecolor{currentstroke}%
\pgfsetdash{}{0pt}%
\pgfpathmoveto{\pgfqpoint{5.270208in}{3.290052in}}%
\pgfpathlineto{\pgfqpoint{5.281191in}{3.264366in}}%
\pgfpathlineto{\pgfqpoint{5.292215in}{3.240975in}}%
\pgfpathlineto{\pgfqpoint{5.324584in}{3.266337in}}%
\pgfpathlineto{\pgfqpoint{5.356931in}{3.290146in}}%
\pgfpathlineto{\pgfqpoint{5.345834in}{3.311408in}}%
\pgfpathlineto{\pgfqpoint{5.334773in}{3.334307in}}%
\pgfpathlineto{\pgfqpoint{5.302502in}{3.312880in}}%
\pgfpathlineto{\pgfqpoint{5.270208in}{3.290052in}}%
\pgfpathclose%
\pgfusepath{fill}%
\end{pgfscope}%
\begin{pgfscope}%
\pgfpathrectangle{\pgfqpoint{1.020000in}{0.880000in}}{\pgfqpoint{6.160000in}{6.160000in}}%
\pgfusepath{clip}%
\pgfsetbuttcap%
\pgfsetroundjoin%
\definecolor{currentfill}{rgb}{0.968863,0.710838,0.599901}%
\pgfsetfillcolor{currentfill}%
\pgfsetlinewidth{0.000000pt}%
\definecolor{currentstroke}{rgb}{0.000000,0.000000,0.000000}%
\pgfsetstrokecolor{currentstroke}%
\pgfsetdash{}{0pt}%
\pgfpathmoveto{\pgfqpoint{2.228134in}{4.633406in}}%
\pgfpathlineto{\pgfqpoint{2.236083in}{4.644328in}}%
\pgfpathlineto{\pgfqpoint{2.244044in}{4.655576in}}%
\pgfpathlineto{\pgfqpoint{2.276796in}{4.674485in}}%
\pgfpathlineto{\pgfqpoint{2.309532in}{4.693780in}}%
\pgfpathlineto{\pgfqpoint{2.301569in}{4.678512in}}%
\pgfpathlineto{\pgfqpoint{2.293617in}{4.663685in}}%
\pgfpathlineto{\pgfqpoint{2.260884in}{4.648374in}}%
\pgfpathlineto{\pgfqpoint{2.228134in}{4.633406in}}%
\pgfpathclose%
\pgfusepath{fill}%
\end{pgfscope}%
\begin{pgfscope}%
\pgfpathrectangle{\pgfqpoint{1.020000in}{0.880000in}}{\pgfqpoint{6.160000in}{6.160000in}}%
\pgfusepath{clip}%
\pgfsetbuttcap%
\pgfsetroundjoin%
\definecolor{currentfill}{rgb}{0.404421,0.534643,0.932002}%
\pgfsetfillcolor{currentfill}%
\pgfsetlinewidth{0.000000pt}%
\definecolor{currentstroke}{rgb}{0.000000,0.000000,0.000000}%
\pgfsetstrokecolor{currentstroke}%
\pgfsetdash{}{0pt}%
\pgfpathmoveto{\pgfqpoint{4.691162in}{3.461108in}}%
\pgfpathlineto{\pgfqpoint{4.701651in}{3.426493in}}%
\pgfpathlineto{\pgfqpoint{4.712152in}{3.391147in}}%
\pgfpathlineto{\pgfqpoint{4.744355in}{3.373106in}}%
\pgfpathlineto{\pgfqpoint{4.776534in}{3.356405in}}%
\pgfpathlineto{\pgfqpoint{4.766019in}{3.395512in}}%
\pgfpathlineto{\pgfqpoint{4.755514in}{3.433737in}}%
\pgfpathlineto{\pgfqpoint{4.723350in}{3.446879in}}%
\pgfpathlineto{\pgfqpoint{4.691162in}{3.461108in}}%
\pgfpathclose%
\pgfusepath{fill}%
\end{pgfscope}%
\begin{pgfscope}%
\pgfpathrectangle{\pgfqpoint{1.020000in}{0.880000in}}{\pgfqpoint{6.160000in}{6.160000in}}%
\pgfusepath{clip}%
\pgfsetbuttcap%
\pgfsetroundjoin%
\definecolor{currentfill}{rgb}{0.285273,0.380129,0.823469}%
\pgfsetfillcolor{currentfill}%
\pgfsetlinewidth{0.000000pt}%
\definecolor{currentstroke}{rgb}{0.000000,0.000000,0.000000}%
\pgfsetstrokecolor{currentstroke}%
\pgfsetdash{}{0pt}%
\pgfpathmoveto{\pgfqpoint{4.926249in}{3.226512in}}%
\pgfpathlineto{\pgfqpoint{4.936856in}{3.185001in}}%
\pgfpathlineto{\pgfqpoint{4.947496in}{3.146179in}}%
\pgfpathlineto{\pgfqpoint{4.979690in}{3.142898in}}%
\pgfpathlineto{\pgfqpoint{5.011901in}{3.144800in}}%
\pgfpathlineto{\pgfqpoint{5.001224in}{3.183484in}}%
\pgfpathlineto{\pgfqpoint{4.990582in}{3.224849in}}%
\pgfpathlineto{\pgfqpoint{4.958410in}{3.223474in}}%
\pgfpathlineto{\pgfqpoint{4.926249in}{3.226512in}}%
\pgfpathclose%
\pgfusepath{fill}%
\end{pgfscope}%
\begin{pgfscope}%
\pgfpathrectangle{\pgfqpoint{1.020000in}{0.880000in}}{\pgfqpoint{6.160000in}{6.160000in}}%
\pgfusepath{clip}%
\pgfsetbuttcap%
\pgfsetroundjoin%
\definecolor{currentfill}{rgb}{0.313946,0.420052,0.854993}%
\pgfsetfillcolor{currentfill}%
\pgfsetlinewidth{0.000000pt}%
\definecolor{currentstroke}{rgb}{0.000000,0.000000,0.000000}%
\pgfsetstrokecolor{currentstroke}%
\pgfsetdash{}{0pt}%
\pgfpathmoveto{\pgfqpoint{5.205583in}{3.242615in}}%
\pgfpathlineto{\pgfqpoint{5.216489in}{3.213944in}}%
\pgfpathlineto{\pgfqpoint{5.227441in}{3.188274in}}%
\pgfpathlineto{\pgfqpoint{5.259831in}{3.214692in}}%
\pgfpathlineto{\pgfqpoint{5.292215in}{3.240975in}}%
\pgfpathlineto{\pgfqpoint{5.281191in}{3.264366in}}%
\pgfpathlineto{\pgfqpoint{5.270208in}{3.290052in}}%
\pgfpathlineto{\pgfqpoint{5.237900in}{3.266393in}}%
\pgfpathlineto{\pgfqpoint{5.205583in}{3.242615in}}%
\pgfpathclose%
\pgfusepath{fill}%
\end{pgfscope}%
\begin{pgfscope}%
\pgfpathrectangle{\pgfqpoint{1.020000in}{0.880000in}}{\pgfqpoint{6.160000in}{6.160000in}}%
\pgfusepath{clip}%
\pgfsetbuttcap%
\pgfsetroundjoin%
\definecolor{currentfill}{rgb}{0.887752,0.854040,0.834671}%
\pgfsetfillcolor{currentfill}%
\pgfsetlinewidth{0.000000pt}%
\definecolor{currentstroke}{rgb}{0.000000,0.000000,0.000000}%
\pgfsetstrokecolor{currentstroke}%
\pgfsetdash{}{0pt}%
\pgfpathmoveto{\pgfqpoint{3.685842in}{4.349173in}}%
\pgfpathlineto{\pgfqpoint{3.695514in}{4.339995in}}%
\pgfpathlineto{\pgfqpoint{3.705213in}{4.328760in}}%
\pgfpathlineto{\pgfqpoint{3.738044in}{4.282296in}}%
\pgfpathlineto{\pgfqpoint{3.770818in}{4.238118in}}%
\pgfpathlineto{\pgfqpoint{3.761070in}{4.250170in}}%
\pgfpathlineto{\pgfqpoint{3.751346in}{4.260498in}}%
\pgfpathlineto{\pgfqpoint{3.718622in}{4.303721in}}%
\pgfpathlineto{\pgfqpoint{3.685842in}{4.349173in}}%
\pgfpathclose%
\pgfusepath{fill}%
\end{pgfscope}%
\begin{pgfscope}%
\pgfpathrectangle{\pgfqpoint{1.020000in}{0.880000in}}{\pgfqpoint{6.160000in}{6.160000in}}%
\pgfusepath{clip}%
\pgfsetbuttcap%
\pgfsetroundjoin%
\definecolor{currentfill}{rgb}{0.388852,0.516298,0.921373}%
\pgfsetfillcolor{currentfill}%
\pgfsetlinewidth{0.000000pt}%
\definecolor{currentstroke}{rgb}{0.000000,0.000000,0.000000}%
\pgfsetstrokecolor{currentstroke}%
\pgfsetdash{}{0pt}%
\pgfpathmoveto{\pgfqpoint{5.550271in}{3.385279in}}%
\pgfpathlineto{\pgfqpoint{5.561594in}{3.368667in}}%
\pgfpathlineto{\pgfqpoint{5.572944in}{3.352543in}}%
\pgfpathlineto{\pgfqpoint{5.605083in}{3.361069in}}%
\pgfpathlineto{\pgfqpoint{5.637184in}{3.367880in}}%
\pgfpathlineto{\pgfqpoint{5.625781in}{3.383763in}}%
\pgfpathlineto{\pgfqpoint{5.614402in}{3.399902in}}%
\pgfpathlineto{\pgfqpoint{5.582355in}{3.393419in}}%
\pgfpathlineto{\pgfqpoint{5.550271in}{3.385279in}}%
\pgfpathclose%
\pgfusepath{fill}%
\end{pgfscope}%
\begin{pgfscope}%
\pgfpathrectangle{\pgfqpoint{1.020000in}{0.880000in}}{\pgfqpoint{6.160000in}{6.160000in}}%
\pgfusepath{clip}%
\pgfsetbuttcap%
\pgfsetroundjoin%
\definecolor{currentfill}{rgb}{0.698454,0.799450,0.984577}%
\pgfsetfillcolor{currentfill}%
\pgfsetlinewidth{0.000000pt}%
\definecolor{currentstroke}{rgb}{0.000000,0.000000,0.000000}%
\pgfsetstrokecolor{currentstroke}%
\pgfsetdash{}{0pt}%
\pgfpathmoveto{\pgfqpoint{4.006114in}{3.959563in}}%
\pgfpathlineto{\pgfqpoint{4.016098in}{3.940848in}}%
\pgfpathlineto{\pgfqpoint{4.026101in}{3.921238in}}%
\pgfpathlineto{\pgfqpoint{4.058663in}{3.895009in}}%
\pgfpathlineto{\pgfqpoint{4.091186in}{3.870992in}}%
\pgfpathlineto{\pgfqpoint{4.081140in}{3.889934in}}%
\pgfpathlineto{\pgfqpoint{4.071112in}{3.908147in}}%
\pgfpathlineto{\pgfqpoint{4.038632in}{3.932716in}}%
\pgfpathlineto{\pgfqpoint{4.006114in}{3.959563in}}%
\pgfpathclose%
\pgfusepath{fill}%
\end{pgfscope}%
\begin{pgfscope}%
\pgfpathrectangle{\pgfqpoint{1.020000in}{0.880000in}}{\pgfqpoint{6.160000in}{6.160000in}}%
\pgfusepath{clip}%
\pgfsetbuttcap%
\pgfsetroundjoin%
\definecolor{currentfill}{rgb}{0.969683,0.690484,0.575138}%
\pgfsetfillcolor{currentfill}%
\pgfsetlinewidth{0.000000pt}%
\definecolor{currentstroke}{rgb}{0.000000,0.000000,0.000000}%
\pgfsetstrokecolor{currentstroke}%
\pgfsetdash{}{0pt}%
\pgfpathmoveto{\pgfqpoint{3.365665in}{4.722841in}}%
\pgfpathlineto{\pgfqpoint{3.374869in}{4.736668in}}%
\pgfpathlineto{\pgfqpoint{3.384111in}{4.748249in}}%
\pgfpathlineto{\pgfqpoint{3.417213in}{4.698503in}}%
\pgfpathlineto{\pgfqpoint{3.450264in}{4.648682in}}%
\pgfpathlineto{\pgfqpoint{3.440942in}{4.640248in}}%
\pgfpathlineto{\pgfqpoint{3.431655in}{4.629812in}}%
\pgfpathlineto{\pgfqpoint{3.398684in}{4.676359in}}%
\pgfpathlineto{\pgfqpoint{3.365665in}{4.722841in}}%
\pgfpathclose%
\pgfusepath{fill}%
\end{pgfscope}%
\begin{pgfscope}%
\pgfpathrectangle{\pgfqpoint{1.020000in}{0.880000in}}{\pgfqpoint{6.160000in}{6.160000in}}%
\pgfusepath{clip}%
\pgfsetbuttcap%
\pgfsetroundjoin%
\definecolor{currentfill}{rgb}{0.285273,0.380129,0.823469}%
\pgfsetfillcolor{currentfill}%
\pgfsetlinewidth{0.000000pt}%
\definecolor{currentstroke}{rgb}{0.000000,0.000000,0.000000}%
\pgfsetstrokecolor{currentstroke}%
\pgfsetdash{}{0pt}%
\pgfpathmoveto{\pgfqpoint{5.140959in}{3.198041in}}%
\pgfpathlineto{\pgfqpoint{5.151793in}{3.166546in}}%
\pgfpathlineto{\pgfqpoint{5.162676in}{3.138719in}}%
\pgfpathlineto{\pgfqpoint{5.195053in}{3.162624in}}%
\pgfpathlineto{\pgfqpoint{5.227441in}{3.188274in}}%
\pgfpathlineto{\pgfqpoint{5.216489in}{3.213944in}}%
\pgfpathlineto{\pgfqpoint{5.205583in}{3.242615in}}%
\pgfpathlineto{\pgfqpoint{5.173267in}{3.219537in}}%
\pgfpathlineto{\pgfqpoint{5.140959in}{3.198041in}}%
\pgfpathclose%
\pgfusepath{fill}%
\end{pgfscope}%
\begin{pgfscope}%
\pgfpathrectangle{\pgfqpoint{1.020000in}{0.880000in}}{\pgfqpoint{6.160000in}{6.160000in}}%
\pgfusepath{clip}%
\pgfsetbuttcap%
\pgfsetroundjoin%
\definecolor{currentfill}{rgb}{0.967874,0.725847,0.618489}%
\pgfsetfillcolor{currentfill}%
\pgfsetlinewidth{0.000000pt}%
\definecolor{currentstroke}{rgb}{0.000000,0.000000,0.000000}%
\pgfsetstrokecolor{currentstroke}%
\pgfsetdash{}{0pt}%
\pgfpathmoveto{\pgfqpoint{2.162567in}{4.605129in}}%
\pgfpathlineto{\pgfqpoint{2.170515in}{4.612285in}}%
\pgfpathlineto{\pgfqpoint{2.178477in}{4.619656in}}%
\pgfpathlineto{\pgfqpoint{2.211273in}{4.637243in}}%
\pgfpathlineto{\pgfqpoint{2.244044in}{4.655576in}}%
\pgfpathlineto{\pgfqpoint{2.236083in}{4.644328in}}%
\pgfpathlineto{\pgfqpoint{2.228134in}{4.633406in}}%
\pgfpathlineto{\pgfqpoint{2.195363in}{4.618943in}}%
\pgfpathlineto{\pgfqpoint{2.162567in}{4.605129in}}%
\pgfpathclose%
\pgfusepath{fill}%
\end{pgfscope}%
\begin{pgfscope}%
\pgfpathrectangle{\pgfqpoint{1.020000in}{0.880000in}}{\pgfqpoint{6.160000in}{6.160000in}}%
\pgfusepath{clip}%
\pgfsetbuttcap%
\pgfsetroundjoin%
\definecolor{currentfill}{rgb}{0.931831,0.519086,0.406480}%
\pgfsetfillcolor{currentfill}%
\pgfsetlinewidth{0.000000pt}%
\definecolor{currentstroke}{rgb}{0.000000,0.000000,0.000000}%
\pgfsetstrokecolor{currentstroke}%
\pgfsetdash{}{0pt}%
\pgfpathmoveto{\pgfqpoint{2.916206in}{4.927084in}}%
\pgfpathlineto{\pgfqpoint{2.924622in}{4.964612in}}%
\pgfpathlineto{\pgfqpoint{2.933067in}{5.001722in}}%
\pgfpathlineto{\pgfqpoint{2.966126in}{4.987021in}}%
\pgfpathlineto{\pgfqpoint{2.999192in}{4.968924in}}%
\pgfpathlineto{\pgfqpoint{2.990653in}{4.933048in}}%
\pgfpathlineto{\pgfqpoint{2.982142in}{4.896762in}}%
\pgfpathlineto{\pgfqpoint{2.949172in}{4.913414in}}%
\pgfpathlineto{\pgfqpoint{2.916206in}{4.927084in}}%
\pgfpathclose%
\pgfusepath{fill}%
\end{pgfscope}%
\begin{pgfscope}%
\pgfpathrectangle{\pgfqpoint{1.020000in}{0.880000in}}{\pgfqpoint{6.160000in}{6.160000in}}%
\pgfusepath{clip}%
\pgfsetbuttcap%
\pgfsetroundjoin%
\definecolor{currentfill}{rgb}{0.478462,0.616564,0.972721}%
\pgfsetfillcolor{currentfill}%
\pgfsetlinewidth{0.000000pt}%
\definecolor{currentstroke}{rgb}{0.000000,0.000000,0.000000}%
\pgfsetstrokecolor{currentstroke}%
\pgfsetdash{}{0pt}%
\pgfpathmoveto{\pgfqpoint{4.541287in}{3.574189in}}%
\pgfpathlineto{\pgfqpoint{4.551703in}{3.548378in}}%
\pgfpathlineto{\pgfqpoint{4.562133in}{3.521699in}}%
\pgfpathlineto{\pgfqpoint{4.594435in}{3.506587in}}%
\pgfpathlineto{\pgfqpoint{4.626707in}{3.491290in}}%
\pgfpathlineto{\pgfqpoint{4.616249in}{3.521188in}}%
\pgfpathlineto{\pgfqpoint{4.605803in}{3.549910in}}%
\pgfpathlineto{\pgfqpoint{4.573559in}{3.562016in}}%
\pgfpathlineto{\pgfqpoint{4.541287in}{3.574189in}}%
\pgfpathclose%
\pgfusepath{fill}%
\end{pgfscope}%
\begin{pgfscope}%
\pgfpathrectangle{\pgfqpoint{1.020000in}{0.880000in}}{\pgfqpoint{6.160000in}{6.160000in}}%
\pgfusepath{clip}%
\pgfsetbuttcap%
\pgfsetroundjoin%
\definecolor{currentfill}{rgb}{0.944055,0.553153,0.435548}%
\pgfsetfillcolor{currentfill}%
\pgfsetlinewidth{0.000000pt}%
\definecolor{currentstroke}{rgb}{0.000000,0.000000,0.000000}%
\pgfsetstrokecolor{currentstroke}%
\pgfsetdash{}{0pt}%
\pgfpathmoveto{\pgfqpoint{2.636801in}{4.857618in}}%
\pgfpathlineto{\pgfqpoint{2.644891in}{4.891216in}}%
\pgfpathlineto{\pgfqpoint{2.652999in}{4.925142in}}%
\pgfpathlineto{\pgfqpoint{2.685830in}{4.935686in}}%
\pgfpathlineto{\pgfqpoint{2.718684in}{4.943533in}}%
\pgfpathlineto{\pgfqpoint{2.710521in}{4.907464in}}%
\pgfpathlineto{\pgfqpoint{2.702377in}{4.871759in}}%
\pgfpathlineto{\pgfqpoint{2.669581in}{4.865850in}}%
\pgfpathlineto{\pgfqpoint{2.636801in}{4.857618in}}%
\pgfpathclose%
\pgfusepath{fill}%
\end{pgfscope}%
\begin{pgfscope}%
\pgfpathrectangle{\pgfqpoint{1.020000in}{0.880000in}}{\pgfqpoint{6.160000in}{6.160000in}}%
\pgfusepath{clip}%
\pgfsetbuttcap%
\pgfsetroundjoin%
\definecolor{currentfill}{rgb}{0.348323,0.465711,0.888346}%
\pgfsetfillcolor{currentfill}%
\pgfsetlinewidth{0.000000pt}%
\definecolor{currentstroke}{rgb}{0.000000,0.000000,0.000000}%
\pgfsetstrokecolor{currentstroke}%
\pgfsetdash{}{0pt}%
\pgfpathmoveto{\pgfqpoint{4.776534in}{3.356405in}}%
\pgfpathlineto{\pgfqpoint{4.787062in}{3.317081in}}%
\pgfpathlineto{\pgfqpoint{4.797607in}{3.278310in}}%
\pgfpathlineto{\pgfqpoint{4.829779in}{3.260303in}}%
\pgfpathlineto{\pgfqpoint{4.861940in}{3.245295in}}%
\pgfpathlineto{\pgfqpoint{4.851380in}{3.286952in}}%
\pgfpathlineto{\pgfqpoint{4.840838in}{3.329236in}}%
\pgfpathlineto{\pgfqpoint{4.808693in}{3.341604in}}%
\pgfpathlineto{\pgfqpoint{4.776534in}{3.356405in}}%
\pgfpathclose%
\pgfusepath{fill}%
\end{pgfscope}%
\begin{pgfscope}%
\pgfpathrectangle{\pgfqpoint{1.020000in}{0.880000in}}{\pgfqpoint{6.160000in}{6.160000in}}%
\pgfusepath{clip}%
\pgfsetbuttcap%
\pgfsetroundjoin%
\definecolor{currentfill}{rgb}{0.266381,0.353304,0.801637}%
\pgfsetfillcolor{currentfill}%
\pgfsetlinewidth{0.000000pt}%
\definecolor{currentstroke}{rgb}{0.000000,0.000000,0.000000}%
\pgfsetstrokecolor{currentstroke}%
\pgfsetdash{}{0pt}%
\pgfpathmoveto{\pgfqpoint{5.076388in}{3.163345in}}%
\pgfpathlineto{\pgfqpoint{5.087159in}{3.129605in}}%
\pgfpathlineto{\pgfqpoint{5.097982in}{3.100059in}}%
\pgfpathlineto{\pgfqpoint{5.130317in}{3.117551in}}%
\pgfpathlineto{\pgfqpoint{5.162676in}{3.138719in}}%
\pgfpathlineto{\pgfqpoint{5.151793in}{3.166546in}}%
\pgfpathlineto{\pgfqpoint{5.140959in}{3.198041in}}%
\pgfpathlineto{\pgfqpoint{5.108664in}{3.179027in}}%
\pgfpathlineto{\pgfqpoint{5.076388in}{3.163345in}}%
\pgfpathclose%
\pgfusepath{fill}%
\end{pgfscope}%
\begin{pgfscope}%
\pgfpathrectangle{\pgfqpoint{1.020000in}{0.880000in}}{\pgfqpoint{6.160000in}{6.160000in}}%
\pgfusepath{clip}%
\pgfsetbuttcap%
\pgfsetroundjoin%
\definecolor{currentfill}{rgb}{0.839351,0.861167,0.894494}%
\pgfsetfillcolor{currentfill}%
\pgfsetlinewidth{0.000000pt}%
\definecolor{currentstroke}{rgb}{0.000000,0.000000,0.000000}%
\pgfsetstrokecolor{currentstroke}%
\pgfsetdash{}{0pt}%
\pgfpathmoveto{\pgfqpoint{3.770818in}{4.238118in}}%
\pgfpathlineto{\pgfqpoint{3.780590in}{4.224300in}}%
\pgfpathlineto{\pgfqpoint{3.790385in}{4.208696in}}%
\pgfpathlineto{\pgfqpoint{3.823148in}{4.166649in}}%
\pgfpathlineto{\pgfqpoint{3.855856in}{4.127098in}}%
\pgfpathlineto{\pgfqpoint{3.846017in}{4.142804in}}%
\pgfpathlineto{\pgfqpoint{3.836199in}{4.157035in}}%
\pgfpathlineto{\pgfqpoint{3.803535in}{4.196339in}}%
\pgfpathlineto{\pgfqpoint{3.770818in}{4.238118in}}%
\pgfpathclose%
\pgfusepath{fill}%
\end{pgfscope}%
\begin{pgfscope}%
\pgfpathrectangle{\pgfqpoint{1.020000in}{0.880000in}}{\pgfqpoint{6.160000in}{6.160000in}}%
\pgfusepath{clip}%
\pgfsetbuttcap%
\pgfsetroundjoin%
\definecolor{currentfill}{rgb}{0.538004,0.674902,0.991722}%
\pgfsetfillcolor{currentfill}%
\pgfsetlinewidth{0.000000pt}%
\definecolor{currentstroke}{rgb}{0.000000,0.000000,0.000000}%
\pgfsetstrokecolor{currentstroke}%
\pgfsetdash{}{0pt}%
\pgfpathmoveto{\pgfqpoint{4.391283in}{3.665295in}}%
\pgfpathlineto{\pgfqpoint{4.401593in}{3.644398in}}%
\pgfpathlineto{\pgfqpoint{4.411918in}{3.622945in}}%
\pgfpathlineto{\pgfqpoint{4.444302in}{3.610666in}}%
\pgfpathlineto{\pgfqpoint{4.476658in}{3.598500in}}%
\pgfpathlineto{\pgfqpoint{4.466292in}{3.621294in}}%
\pgfpathlineto{\pgfqpoint{4.455941in}{3.643315in}}%
\pgfpathlineto{\pgfqpoint{4.423625in}{3.654085in}}%
\pgfpathlineto{\pgfqpoint{4.391283in}{3.665295in}}%
\pgfpathclose%
\pgfusepath{fill}%
\end{pgfscope}%
\begin{pgfscope}%
\pgfpathrectangle{\pgfqpoint{1.020000in}{0.880000in}}{\pgfqpoint{6.160000in}{6.160000in}}%
\pgfusepath{clip}%
\pgfsetbuttcap%
\pgfsetroundjoin%
\definecolor{currentfill}{rgb}{0.592356,0.722792,0.999434}%
\pgfsetfillcolor{currentfill}%
\pgfsetlinewidth{0.000000pt}%
\definecolor{currentstroke}{rgb}{0.000000,0.000000,0.000000}%
\pgfsetstrokecolor{currentstroke}%
\pgfsetdash{}{0pt}%
\pgfpathmoveto{\pgfqpoint{4.241257in}{3.756926in}}%
\pgfpathlineto{\pgfqpoint{4.251444in}{3.737356in}}%
\pgfpathlineto{\pgfqpoint{4.261647in}{3.717346in}}%
\pgfpathlineto{\pgfqpoint{4.294098in}{3.702932in}}%
\pgfpathlineto{\pgfqpoint{4.326520in}{3.689564in}}%
\pgfpathlineto{\pgfqpoint{4.316271in}{3.709486in}}%
\pgfpathlineto{\pgfqpoint{4.306039in}{3.728927in}}%
\pgfpathlineto{\pgfqpoint{4.273662in}{3.742299in}}%
\pgfpathlineto{\pgfqpoint{4.241257in}{3.756926in}}%
\pgfpathclose%
\pgfusepath{fill}%
\end{pgfscope}%
\begin{pgfscope}%
\pgfpathrectangle{\pgfqpoint{1.020000in}{0.880000in}}{\pgfqpoint{6.160000in}{6.160000in}}%
\pgfusepath{clip}%
\pgfsetbuttcap%
\pgfsetroundjoin%
\definecolor{currentfill}{rgb}{0.965899,0.740142,0.637058}%
\pgfsetfillcolor{currentfill}%
\pgfsetlinewidth{0.000000pt}%
\definecolor{currentstroke}{rgb}{0.000000,0.000000,0.000000}%
\pgfsetstrokecolor{currentstroke}%
\pgfsetdash{}{0pt}%
\pgfpathmoveto{\pgfqpoint{2.096892in}{4.579940in}}%
\pgfpathlineto{\pgfqpoint{2.104839in}{4.583560in}}%
\pgfpathlineto{\pgfqpoint{2.112802in}{4.587292in}}%
\pgfpathlineto{\pgfqpoint{2.145655in}{4.602963in}}%
\pgfpathlineto{\pgfqpoint{2.178477in}{4.619656in}}%
\pgfpathlineto{\pgfqpoint{2.170515in}{4.612285in}}%
\pgfpathlineto{\pgfqpoint{2.162567in}{4.605129in}}%
\pgfpathlineto{\pgfqpoint{2.129745in}{4.592092in}}%
\pgfpathlineto{\pgfqpoint{2.096892in}{4.579940in}}%
\pgfpathclose%
\pgfusepath{fill}%
\end{pgfscope}%
\begin{pgfscope}%
\pgfpathrectangle{\pgfqpoint{1.020000in}{0.880000in}}{\pgfqpoint{6.160000in}{6.160000in}}%
\pgfusepath{clip}%
\pgfsetbuttcap%
\pgfsetroundjoin%
\definecolor{currentfill}{rgb}{0.378598,0.503856,0.913692}%
\pgfsetfillcolor{currentfill}%
\pgfsetlinewidth{0.000000pt}%
\definecolor{currentstroke}{rgb}{0.000000,0.000000,0.000000}%
\pgfsetstrokecolor{currentstroke}%
\pgfsetdash{}{0pt}%
\pgfpathmoveto{\pgfqpoint{5.485983in}{3.362999in}}%
\pgfpathlineto{\pgfqpoint{5.497247in}{3.345712in}}%
\pgfpathlineto{\pgfqpoint{5.508541in}{3.329275in}}%
\pgfpathlineto{\pgfqpoint{5.540764in}{3.342027in}}%
\pgfpathlineto{\pgfqpoint{5.572944in}{3.352543in}}%
\pgfpathlineto{\pgfqpoint{5.561594in}{3.368667in}}%
\pgfpathlineto{\pgfqpoint{5.550271in}{3.385279in}}%
\pgfpathlineto{\pgfqpoint{5.518148in}{3.375219in}}%
\pgfpathlineto{\pgfqpoint{5.485983in}{3.362999in}}%
\pgfpathclose%
\pgfusepath{fill}%
\end{pgfscope}%
\begin{pgfscope}%
\pgfpathrectangle{\pgfqpoint{1.020000in}{0.880000in}}{\pgfqpoint{6.160000in}{6.160000in}}%
\pgfusepath{clip}%
\pgfsetbuttcap%
\pgfsetroundjoin%
\definecolor{currentfill}{rgb}{0.294718,0.393542,0.834384}%
\pgfsetfillcolor{currentfill}%
\pgfsetlinewidth{0.000000pt}%
\definecolor{currentstroke}{rgb}{0.000000,0.000000,0.000000}%
\pgfsetstrokecolor{currentstroke}%
\pgfsetdash{}{0pt}%
\pgfpathmoveto{\pgfqpoint{4.861940in}{3.245295in}}%
\pgfpathlineto{\pgfqpoint{4.872523in}{3.205208in}}%
\pgfpathlineto{\pgfqpoint{4.883138in}{3.167646in}}%
\pgfpathlineto{\pgfqpoint{4.915314in}{3.154541in}}%
\pgfpathlineto{\pgfqpoint{4.947496in}{3.146179in}}%
\pgfpathlineto{\pgfqpoint{4.936856in}{3.185001in}}%
\pgfpathlineto{\pgfqpoint{4.926249in}{3.226512in}}%
\pgfpathlineto{\pgfqpoint{4.894094in}{3.233880in}}%
\pgfpathlineto{\pgfqpoint{4.861940in}{3.245295in}}%
\pgfpathclose%
\pgfusepath{fill}%
\end{pgfscope}%
\begin{pgfscope}%
\pgfpathrectangle{\pgfqpoint{1.020000in}{0.880000in}}{\pgfqpoint{6.160000in}{6.160000in}}%
\pgfusepath{clip}%
\pgfsetbuttcap%
\pgfsetroundjoin%
\definecolor{currentfill}{rgb}{0.956653,0.598034,0.477302}%
\pgfsetfillcolor{currentfill}%
\pgfsetlinewidth{0.000000pt}%
\definecolor{currentstroke}{rgb}{0.000000,0.000000,0.000000}%
\pgfsetstrokecolor{currentstroke}%
\pgfsetdash{}{0pt}%
\pgfpathmoveto{\pgfqpoint{3.215249in}{4.853472in}}%
\pgfpathlineto{\pgfqpoint{3.224193in}{4.877943in}}%
\pgfpathlineto{\pgfqpoint{3.233176in}{4.900492in}}%
\pgfpathlineto{\pgfqpoint{3.266353in}{4.858113in}}%
\pgfpathlineto{\pgfqpoint{3.299496in}{4.814110in}}%
\pgfpathlineto{\pgfqpoint{3.290419in}{4.794845in}}%
\pgfpathlineto{\pgfqpoint{3.281378in}{4.773809in}}%
\pgfpathlineto{\pgfqpoint{3.248330in}{4.814377in}}%
\pgfpathlineto{\pgfqpoint{3.215249in}{4.853472in}}%
\pgfpathclose%
\pgfusepath{fill}%
\end{pgfscope}%
\begin{pgfscope}%
\pgfpathrectangle{\pgfqpoint{1.020000in}{0.880000in}}{\pgfqpoint{6.160000in}{6.160000in}}%
\pgfusepath{clip}%
\pgfsetbuttcap%
\pgfsetroundjoin%
\definecolor{currentfill}{rgb}{0.656683,0.771806,0.994914}%
\pgfsetfillcolor{currentfill}%
\pgfsetlinewidth{0.000000pt}%
\definecolor{currentstroke}{rgb}{0.000000,0.000000,0.000000}%
\pgfsetstrokecolor{currentstroke}%
\pgfsetdash{}{0pt}%
\pgfpathmoveto{\pgfqpoint{4.091186in}{3.870992in}}%
\pgfpathlineto{\pgfqpoint{4.101250in}{3.851361in}}%
\pgfpathlineto{\pgfqpoint{4.111330in}{3.831088in}}%
\pgfpathlineto{\pgfqpoint{4.143861in}{3.809900in}}%
\pgfpathlineto{\pgfqpoint{4.176358in}{3.790573in}}%
\pgfpathlineto{\pgfqpoint{4.166233in}{3.810127in}}%
\pgfpathlineto{\pgfqpoint{4.156126in}{3.829136in}}%
\pgfpathlineto{\pgfqpoint{4.123673in}{3.849076in}}%
\pgfpathlineto{\pgfqpoint{4.091186in}{3.870992in}}%
\pgfpathclose%
\pgfusepath{fill}%
\end{pgfscope}%
\begin{pgfscope}%
\pgfpathrectangle{\pgfqpoint{1.020000in}{0.880000in}}{\pgfqpoint{6.160000in}{6.160000in}}%
\pgfusepath{clip}%
\pgfsetbuttcap%
\pgfsetroundjoin%
\definecolor{currentfill}{rgb}{0.947654,0.565976,0.447478}%
\pgfsetfillcolor{currentfill}%
\pgfsetlinewidth{0.000000pt}%
\definecolor{currentstroke}{rgb}{0.000000,0.000000,0.000000}%
\pgfsetstrokecolor{currentstroke}%
\pgfsetdash{}{0pt}%
\pgfpathmoveto{\pgfqpoint{2.571285in}{4.834894in}}%
\pgfpathlineto{\pgfqpoint{2.579333in}{4.865700in}}%
\pgfpathlineto{\pgfqpoint{2.587399in}{4.896788in}}%
\pgfpathlineto{\pgfqpoint{2.620189in}{4.912100in}}%
\pgfpathlineto{\pgfqpoint{2.652999in}{4.925142in}}%
\pgfpathlineto{\pgfqpoint{2.644891in}{4.891216in}}%
\pgfpathlineto{\pgfqpoint{2.636801in}{4.857618in}}%
\pgfpathlineto{\pgfqpoint{2.604036in}{4.847234in}}%
\pgfpathlineto{\pgfqpoint{2.571285in}{4.834894in}}%
\pgfpathclose%
\pgfusepath{fill}%
\end{pgfscope}%
\begin{pgfscope}%
\pgfpathrectangle{\pgfqpoint{1.020000in}{0.880000in}}{\pgfqpoint{6.160000in}{6.160000in}}%
\pgfusepath{clip}%
\pgfsetbuttcap%
\pgfsetroundjoin%
\definecolor{currentfill}{rgb}{0.966962,0.735670,0.630877}%
\pgfsetfillcolor{currentfill}%
\pgfsetlinewidth{0.000000pt}%
\definecolor{currentstroke}{rgb}{0.000000,0.000000,0.000000}%
\pgfsetstrokecolor{currentstroke}%
\pgfsetdash{}{0pt}%
\pgfpathmoveto{\pgfqpoint{3.450264in}{4.648682in}}%
\pgfpathlineto{\pgfqpoint{3.459622in}{4.654849in}}%
\pgfpathlineto{\pgfqpoint{3.469017in}{4.658503in}}%
\pgfpathlineto{\pgfqpoint{3.502087in}{4.606283in}}%
\pgfpathlineto{\pgfqpoint{3.535102in}{4.554750in}}%
\pgfpathlineto{\pgfqpoint{3.525638in}{4.553624in}}%
\pgfpathlineto{\pgfqpoint{3.516208in}{4.550295in}}%
\pgfpathlineto{\pgfqpoint{3.483263in}{4.599163in}}%
\pgfpathlineto{\pgfqpoint{3.450264in}{4.648682in}}%
\pgfpathclose%
\pgfusepath{fill}%
\end{pgfscope}%
\begin{pgfscope}%
\pgfpathrectangle{\pgfqpoint{1.020000in}{0.880000in}}{\pgfqpoint{6.160000in}{6.160000in}}%
\pgfusepath{clip}%
\pgfsetbuttcap%
\pgfsetroundjoin%
\definecolor{currentfill}{rgb}{0.257234,0.339661,0.789661}%
\pgfsetfillcolor{currentfill}%
\pgfsetlinewidth{0.000000pt}%
\definecolor{currentstroke}{rgb}{0.000000,0.000000,0.000000}%
\pgfsetstrokecolor{currentstroke}%
\pgfsetdash{}{0pt}%
\pgfpathmoveto{\pgfqpoint{5.011901in}{3.144800in}}%
\pgfpathlineto{\pgfqpoint{5.022621in}{3.109763in}}%
\pgfpathlineto{\pgfqpoint{5.033392in}{3.079217in}}%
\pgfpathlineto{\pgfqpoint{5.065673in}{3.087067in}}%
\pgfpathlineto{\pgfqpoint{5.097982in}{3.100059in}}%
\pgfpathlineto{\pgfqpoint{5.087159in}{3.129605in}}%
\pgfpathlineto{\pgfqpoint{5.076388in}{3.163345in}}%
\pgfpathlineto{\pgfqpoint{5.044133in}{3.151742in}}%
\pgfpathlineto{\pgfqpoint{5.011901in}{3.144800in}}%
\pgfpathclose%
\pgfusepath{fill}%
\end{pgfscope}%
\begin{pgfscope}%
\pgfpathrectangle{\pgfqpoint{1.020000in}{0.880000in}}{\pgfqpoint{6.160000in}{6.160000in}}%
\pgfusepath{clip}%
\pgfsetbuttcap%
\pgfsetroundjoin%
\definecolor{currentfill}{rgb}{0.936780,0.532750,0.418093}%
\pgfsetfillcolor{currentfill}%
\pgfsetlinewidth{0.000000pt}%
\definecolor{currentstroke}{rgb}{0.000000,0.000000,0.000000}%
\pgfsetstrokecolor{currentstroke}%
\pgfsetdash{}{0pt}%
\pgfpathmoveto{\pgfqpoint{3.065327in}{4.923213in}}%
\pgfpathlineto{\pgfqpoint{3.073997in}{4.956241in}}%
\pgfpathlineto{\pgfqpoint{3.082704in}{4.988009in}}%
\pgfpathlineto{\pgfqpoint{3.115863in}{4.958330in}}%
\pgfpathlineto{\pgfqpoint{3.149011in}{4.925798in}}%
\pgfpathlineto{\pgfqpoint{3.140204in}{4.896647in}}%
\pgfpathlineto{\pgfqpoint{3.131431in}{4.866297in}}%
\pgfpathlineto{\pgfqpoint{3.098385in}{4.896028in}}%
\pgfpathlineto{\pgfqpoint{3.065327in}{4.923213in}}%
\pgfpathclose%
\pgfusepath{fill}%
\end{pgfscope}%
\begin{pgfscope}%
\pgfpathrectangle{\pgfqpoint{1.020000in}{0.880000in}}{\pgfqpoint{6.160000in}{6.160000in}}%
\pgfusepath{clip}%
\pgfsetbuttcap%
\pgfsetroundjoin%
\definecolor{currentfill}{rgb}{0.786721,0.844807,0.939810}%
\pgfsetfillcolor{currentfill}%
\pgfsetlinewidth{0.000000pt}%
\definecolor{currentstroke}{rgb}{0.000000,0.000000,0.000000}%
\pgfsetstrokecolor{currentstroke}%
\pgfsetdash{}{0pt}%
\pgfpathmoveto{\pgfqpoint{3.855856in}{4.127098in}}%
\pgfpathlineto{\pgfqpoint{3.865717in}{4.109919in}}%
\pgfpathlineto{\pgfqpoint{3.875600in}{4.091290in}}%
\pgfpathlineto{\pgfqpoint{3.908299in}{4.054608in}}%
\pgfpathlineto{\pgfqpoint{3.940949in}{4.020446in}}%
\pgfpathlineto{\pgfqpoint{3.931025in}{4.038620in}}%
\pgfpathlineto{\pgfqpoint{3.921122in}{4.055609in}}%
\pgfpathlineto{\pgfqpoint{3.888514in}{4.090083in}}%
\pgfpathlineto{\pgfqpoint{3.855856in}{4.127098in}}%
\pgfpathclose%
\pgfusepath{fill}%
\end{pgfscope}%
\begin{pgfscope}%
\pgfpathrectangle{\pgfqpoint{1.020000in}{0.880000in}}{\pgfqpoint{6.160000in}{6.160000in}}%
\pgfusepath{clip}%
\pgfsetbuttcap%
\pgfsetroundjoin%
\definecolor{currentfill}{rgb}{0.430507,0.564883,0.948889}%
\pgfsetfillcolor{currentfill}%
\pgfsetlinewidth{0.000000pt}%
\definecolor{currentstroke}{rgb}{0.000000,0.000000,0.000000}%
\pgfsetstrokecolor{currentstroke}%
\pgfsetdash{}{0pt}%
\pgfpathmoveto{\pgfqpoint{4.626707in}{3.491290in}}%
\pgfpathlineto{\pgfqpoint{4.637176in}{3.460475in}}%
\pgfpathlineto{\pgfqpoint{4.647659in}{3.429094in}}%
\pgfpathlineto{\pgfqpoint{4.679921in}{3.409977in}}%
\pgfpathlineto{\pgfqpoint{4.712152in}{3.391147in}}%
\pgfpathlineto{\pgfqpoint{4.701651in}{3.426493in}}%
\pgfpathlineto{\pgfqpoint{4.691162in}{3.461108in}}%
\pgfpathlineto{\pgfqpoint{4.658949in}{3.476031in}}%
\pgfpathlineto{\pgfqpoint{4.626707in}{3.491290in}}%
\pgfpathclose%
\pgfusepath{fill}%
\end{pgfscope}%
\begin{pgfscope}%
\pgfpathrectangle{\pgfqpoint{1.020000in}{0.880000in}}{\pgfqpoint{6.160000in}{6.160000in}}%
\pgfusepath{clip}%
\pgfsetbuttcap%
\pgfsetroundjoin%
\definecolor{currentfill}{rgb}{0.383662,0.510183,0.917831}%
\pgfsetfillcolor{currentfill}%
\pgfsetlinewidth{0.000000pt}%
\definecolor{currentstroke}{rgb}{0.000000,0.000000,0.000000}%
\pgfsetstrokecolor{currentstroke}%
\pgfsetdash{}{0pt}%
\pgfpathmoveto{\pgfqpoint{5.701279in}{3.377447in}}%
\pgfpathlineto{\pgfqpoint{5.712757in}{3.361877in}}%
\pgfpathlineto{\pgfqpoint{5.724257in}{3.346433in}}%
\pgfpathlineto{\pgfqpoint{5.756306in}{3.349718in}}%
\pgfpathlineto{\pgfqpoint{5.744781in}{3.365173in}}%
\pgfpathlineto{\pgfqpoint{5.733278in}{3.380710in}}%
\pgfpathlineto{\pgfqpoint{5.701279in}{3.377447in}}%
\pgfpathclose%
\pgfusepath{fill}%
\end{pgfscope}%
\begin{pgfscope}%
\pgfpathrectangle{\pgfqpoint{1.020000in}{0.880000in}}{\pgfqpoint{6.160000in}{6.160000in}}%
\pgfusepath{clip}%
\pgfsetbuttcap%
\pgfsetroundjoin%
\definecolor{currentfill}{rgb}{0.368507,0.491141,0.905243}%
\pgfsetfillcolor{currentfill}%
\pgfsetlinewidth{0.000000pt}%
\definecolor{currentstroke}{rgb}{0.000000,0.000000,0.000000}%
\pgfsetstrokecolor{currentstroke}%
\pgfsetdash{}{0pt}%
\pgfpathmoveto{\pgfqpoint{5.421532in}{3.331410in}}%
\pgfpathlineto{\pgfqpoint{5.432734in}{3.313205in}}%
\pgfpathlineto{\pgfqpoint{5.443971in}{3.296367in}}%
\pgfpathlineto{\pgfqpoint{5.476277in}{3.314092in}}%
\pgfpathlineto{\pgfqpoint{5.508541in}{3.329275in}}%
\pgfpathlineto{\pgfqpoint{5.497247in}{3.345712in}}%
\pgfpathlineto{\pgfqpoint{5.485983in}{3.362999in}}%
\pgfpathlineto{\pgfqpoint{5.453778in}{3.348432in}}%
\pgfpathlineto{\pgfqpoint{5.421532in}{3.331410in}}%
\pgfpathclose%
\pgfusepath{fill}%
\end{pgfscope}%
\begin{pgfscope}%
\pgfpathrectangle{\pgfqpoint{1.020000in}{0.880000in}}{\pgfqpoint{6.160000in}{6.160000in}}%
\pgfusepath{clip}%
\pgfsetbuttcap%
\pgfsetroundjoin%
\definecolor{currentfill}{rgb}{0.953054,0.585211,0.465373}%
\pgfsetfillcolor{currentfill}%
\pgfsetlinewidth{0.000000pt}%
\definecolor{currentstroke}{rgb}{0.000000,0.000000,0.000000}%
\pgfsetstrokecolor{currentstroke}%
\pgfsetdash{}{0pt}%
\pgfpathmoveto{\pgfqpoint{2.505821in}{4.805226in}}%
\pgfpathlineto{\pgfqpoint{2.513838in}{4.832680in}}%
\pgfpathlineto{\pgfqpoint{2.521873in}{4.860362in}}%
\pgfpathlineto{\pgfqpoint{2.554628in}{4.879453in}}%
\pgfpathlineto{\pgfqpoint{2.587399in}{4.896788in}}%
\pgfpathlineto{\pgfqpoint{2.579333in}{4.865700in}}%
\pgfpathlineto{\pgfqpoint{2.571285in}{4.834894in}}%
\pgfpathlineto{\pgfqpoint{2.538548in}{4.820815in}}%
\pgfpathlineto{\pgfqpoint{2.505821in}{4.805226in}}%
\pgfpathclose%
\pgfusepath{fill}%
\end{pgfscope}%
\begin{pgfscope}%
\pgfpathrectangle{\pgfqpoint{1.020000in}{0.880000in}}{\pgfqpoint{6.160000in}{6.160000in}}%
\pgfusepath{clip}%
\pgfsetbuttcap%
\pgfsetroundjoin%
\definecolor{currentfill}{rgb}{0.921406,0.491420,0.383408}%
\pgfsetfillcolor{currentfill}%
\pgfsetlinewidth{0.000000pt}%
\definecolor{currentstroke}{rgb}{0.000000,0.000000,0.000000}%
\pgfsetstrokecolor{currentstroke}%
\pgfsetdash{}{0pt}%
\pgfpathmoveto{\pgfqpoint{2.850302in}{4.945112in}}%
\pgfpathlineto{\pgfqpoint{2.858631in}{4.983023in}}%
\pgfpathlineto{\pgfqpoint{2.866989in}{5.020510in}}%
\pgfpathlineto{\pgfqpoint{2.900020in}{5.012909in}}%
\pgfpathlineto{\pgfqpoint{2.933067in}{5.001722in}}%
\pgfpathlineto{\pgfqpoint{2.924622in}{4.964612in}}%
\pgfpathlineto{\pgfqpoint{2.916206in}{4.927084in}}%
\pgfpathlineto{\pgfqpoint{2.883248in}{4.937672in}}%
\pgfpathlineto{\pgfqpoint{2.850302in}{4.945112in}}%
\pgfpathclose%
\pgfusepath{fill}%
\end{pgfscope}%
\begin{pgfscope}%
\pgfpathrectangle{\pgfqpoint{1.020000in}{0.880000in}}{\pgfqpoint{6.160000in}{6.160000in}}%
\pgfusepath{clip}%
\pgfsetbuttcap%
\pgfsetroundjoin%
\definecolor{currentfill}{rgb}{0.500031,0.638508,0.981070}%
\pgfsetfillcolor{currentfill}%
\pgfsetlinewidth{0.000000pt}%
\definecolor{currentstroke}{rgb}{0.000000,0.000000,0.000000}%
\pgfsetstrokecolor{currentstroke}%
\pgfsetdash{}{0pt}%
\pgfpathmoveto{\pgfqpoint{4.476658in}{3.598500in}}%
\pgfpathlineto{\pgfqpoint{4.487040in}{3.574993in}}%
\pgfpathlineto{\pgfqpoint{4.497436in}{3.550872in}}%
\pgfpathlineto{\pgfqpoint{4.529800in}{3.536481in}}%
\pgfpathlineto{\pgfqpoint{4.562133in}{3.521699in}}%
\pgfpathlineto{\pgfqpoint{4.551703in}{3.548378in}}%
\pgfpathlineto{\pgfqpoint{4.541287in}{3.574189in}}%
\pgfpathlineto{\pgfqpoint{4.508987in}{3.586355in}}%
\pgfpathlineto{\pgfqpoint{4.476658in}{3.598500in}}%
\pgfpathclose%
\pgfusepath{fill}%
\end{pgfscope}%
\begin{pgfscope}%
\pgfpathrectangle{\pgfqpoint{1.020000in}{0.880000in}}{\pgfqpoint{6.160000in}{6.160000in}}%
\pgfusepath{clip}%
\pgfsetbuttcap%
\pgfsetroundjoin%
\definecolor{currentfill}{rgb}{0.952761,0.782965,0.698646}%
\pgfsetfillcolor{currentfill}%
\pgfsetlinewidth{0.000000pt}%
\definecolor{currentstroke}{rgb}{0.000000,0.000000,0.000000}%
\pgfsetstrokecolor{currentstroke}%
\pgfsetdash{}{0pt}%
\pgfpathmoveto{\pgfqpoint{3.535102in}{4.554750in}}%
\pgfpathlineto{\pgfqpoint{3.544598in}{4.553482in}}%
\pgfpathlineto{\pgfqpoint{3.554129in}{4.549649in}}%
\pgfpathlineto{\pgfqpoint{3.587147in}{4.497194in}}%
\pgfpathlineto{\pgfqpoint{3.620105in}{4.446121in}}%
\pgfpathlineto{\pgfqpoint{3.610515in}{4.451697in}}%
\pgfpathlineto{\pgfqpoint{3.600957in}{4.455062in}}%
\pgfpathlineto{\pgfqpoint{3.568058in}{4.504243in}}%
\pgfpathlineto{\pgfqpoint{3.535102in}{4.554750in}}%
\pgfpathclose%
\pgfusepath{fill}%
\end{pgfscope}%
\begin{pgfscope}%
\pgfpathrectangle{\pgfqpoint{1.020000in}{0.880000in}}{\pgfqpoint{6.160000in}{6.160000in}}%
\pgfusepath{clip}%
\pgfsetbuttcap%
\pgfsetroundjoin%
\definecolor{currentfill}{rgb}{0.348323,0.465711,0.888346}%
\pgfsetfillcolor{currentfill}%
\pgfsetlinewidth{0.000000pt}%
\definecolor{currentstroke}{rgb}{0.000000,0.000000,0.000000}%
\pgfsetstrokecolor{currentstroke}%
\pgfsetdash{}{0pt}%
\pgfpathmoveto{\pgfqpoint{5.356931in}{3.290146in}}%
\pgfpathlineto{\pgfqpoint{5.368067in}{3.270770in}}%
\pgfpathlineto{\pgfqpoint{5.379246in}{3.253440in}}%
\pgfpathlineto{\pgfqpoint{5.411626in}{3.276102in}}%
\pgfpathlineto{\pgfqpoint{5.443971in}{3.296367in}}%
\pgfpathlineto{\pgfqpoint{5.432734in}{3.313205in}}%
\pgfpathlineto{\pgfqpoint{5.421532in}{3.331410in}}%
\pgfpathlineto{\pgfqpoint{5.389249in}{3.311936in}}%
\pgfpathlineto{\pgfqpoint{5.356931in}{3.290146in}}%
\pgfpathclose%
\pgfusepath{fill}%
\end{pgfscope}%
\begin{pgfscope}%
\pgfpathrectangle{\pgfqpoint{1.020000in}{0.880000in}}{\pgfqpoint{6.160000in}{6.160000in}}%
\pgfusepath{clip}%
\pgfsetbuttcap%
\pgfsetroundjoin%
\definecolor{currentfill}{rgb}{0.738826,0.822572,0.968261}%
\pgfsetfillcolor{currentfill}%
\pgfsetlinewidth{0.000000pt}%
\definecolor{currentstroke}{rgb}{0.000000,0.000000,0.000000}%
\pgfsetstrokecolor{currentstroke}%
\pgfsetdash{}{0pt}%
\pgfpathmoveto{\pgfqpoint{3.940949in}{4.020446in}}%
\pgfpathlineto{\pgfqpoint{3.950892in}{4.001121in}}%
\pgfpathlineto{\pgfqpoint{3.960854in}{3.980695in}}%
\pgfpathlineto{\pgfqpoint{3.993499in}{3.949775in}}%
\pgfpathlineto{\pgfqpoint{4.026101in}{3.921238in}}%
\pgfpathlineto{\pgfqpoint{4.016098in}{3.940848in}}%
\pgfpathlineto{\pgfqpoint{4.006114in}{3.959563in}}%
\pgfpathlineto{\pgfqpoint{3.973553in}{3.988781in}}%
\pgfpathlineto{\pgfqpoint{3.940949in}{4.020446in}}%
\pgfpathclose%
\pgfusepath{fill}%
\end{pgfscope}%
\begin{pgfscope}%
\pgfpathrectangle{\pgfqpoint{1.020000in}{0.880000in}}{\pgfqpoint{6.160000in}{6.160000in}}%
\pgfusepath{clip}%
\pgfsetbuttcap%
\pgfsetroundjoin%
\definecolor{currentfill}{rgb}{0.373552,0.497499,0.909467}%
\pgfsetfillcolor{currentfill}%
\pgfsetlinewidth{0.000000pt}%
\definecolor{currentstroke}{rgb}{0.000000,0.000000,0.000000}%
\pgfsetstrokecolor{currentstroke}%
\pgfsetdash{}{0pt}%
\pgfpathmoveto{\pgfqpoint{4.712152in}{3.391147in}}%
\pgfpathlineto{\pgfqpoint{4.722666in}{3.355623in}}%
\pgfpathlineto{\pgfqpoint{4.733198in}{3.320556in}}%
\pgfpathlineto{\pgfqpoint{4.765415in}{3.298634in}}%
\pgfpathlineto{\pgfqpoint{4.797607in}{3.278310in}}%
\pgfpathlineto{\pgfqpoint{4.787062in}{3.317081in}}%
\pgfpathlineto{\pgfqpoint{4.776534in}{3.356405in}}%
\pgfpathlineto{\pgfqpoint{4.744355in}{3.373106in}}%
\pgfpathlineto{\pgfqpoint{4.712152in}{3.391147in}}%
\pgfpathclose%
\pgfusepath{fill}%
\end{pgfscope}%
\begin{pgfscope}%
\pgfpathrectangle{\pgfqpoint{1.020000in}{0.880000in}}{\pgfqpoint{6.160000in}{6.160000in}}%
\pgfusepath{clip}%
\pgfsetbuttcap%
\pgfsetroundjoin%
\definecolor{currentfill}{rgb}{0.554312,0.690097,0.995516}%
\pgfsetfillcolor{currentfill}%
\pgfsetlinewidth{0.000000pt}%
\definecolor{currentstroke}{rgb}{0.000000,0.000000,0.000000}%
\pgfsetstrokecolor{currentstroke}%
\pgfsetdash{}{0pt}%
\pgfpathmoveto{\pgfqpoint{4.326520in}{3.689564in}}%
\pgfpathlineto{\pgfqpoint{4.336785in}{3.669186in}}%
\pgfpathlineto{\pgfqpoint{4.347067in}{3.648390in}}%
\pgfpathlineto{\pgfqpoint{4.379507in}{3.635466in}}%
\pgfpathlineto{\pgfqpoint{4.411918in}{3.622945in}}%
\pgfpathlineto{\pgfqpoint{4.401593in}{3.644398in}}%
\pgfpathlineto{\pgfqpoint{4.391283in}{3.665295in}}%
\pgfpathlineto{\pgfqpoint{4.358915in}{3.677073in}}%
\pgfpathlineto{\pgfqpoint{4.326520in}{3.689564in}}%
\pgfpathclose%
\pgfusepath{fill}%
\end{pgfscope}%
\begin{pgfscope}%
\pgfpathrectangle{\pgfqpoint{1.020000in}{0.880000in}}{\pgfqpoint{6.160000in}{6.160000in}}%
\pgfusepath{clip}%
\pgfsetbuttcap%
\pgfsetroundjoin%
\definecolor{currentfill}{rgb}{0.383662,0.510183,0.917831}%
\pgfsetfillcolor{currentfill}%
\pgfsetlinewidth{0.000000pt}%
\definecolor{currentstroke}{rgb}{0.000000,0.000000,0.000000}%
\pgfsetstrokecolor{currentstroke}%
\pgfsetdash{}{0pt}%
\pgfpathmoveto{\pgfqpoint{5.637184in}{3.367880in}}%
\pgfpathlineto{\pgfqpoint{5.648611in}{3.352264in}}%
\pgfpathlineto{\pgfqpoint{5.660062in}{3.336914in}}%
\pgfpathlineto{\pgfqpoint{5.692177in}{3.342242in}}%
\pgfpathlineto{\pgfqpoint{5.724257in}{3.346433in}}%
\pgfpathlineto{\pgfqpoint{5.712757in}{3.361877in}}%
\pgfpathlineto{\pgfqpoint{5.701279in}{3.377447in}}%
\pgfpathlineto{\pgfqpoint{5.669249in}{3.373251in}}%
\pgfpathlineto{\pgfqpoint{5.637184in}{3.367880in}}%
\pgfpathclose%
\pgfusepath{fill}%
\end{pgfscope}%
\begin{pgfscope}%
\pgfpathrectangle{\pgfqpoint{1.020000in}{0.880000in}}{\pgfqpoint{6.160000in}{6.160000in}}%
\pgfusepath{clip}%
\pgfsetbuttcap%
\pgfsetroundjoin%
\definecolor{currentfill}{rgb}{0.252663,0.332837,0.783665}%
\pgfsetfillcolor{currentfill}%
\pgfsetlinewidth{0.000000pt}%
\definecolor{currentstroke}{rgb}{0.000000,0.000000,0.000000}%
\pgfsetstrokecolor{currentstroke}%
\pgfsetdash{}{0pt}%
\pgfpathmoveto{\pgfqpoint{4.947496in}{3.146179in}}%
\pgfpathlineto{\pgfqpoint{4.958177in}{3.111013in}}%
\pgfpathlineto{\pgfqpoint{4.968907in}{3.080349in}}%
\pgfpathlineto{\pgfqpoint{5.001138in}{3.076926in}}%
\pgfpathlineto{\pgfqpoint{5.033392in}{3.079217in}}%
\pgfpathlineto{\pgfqpoint{5.022621in}{3.109763in}}%
\pgfpathlineto{\pgfqpoint{5.011901in}{3.144800in}}%
\pgfpathlineto{\pgfqpoint{4.979690in}{3.142898in}}%
\pgfpathlineto{\pgfqpoint{4.947496in}{3.146179in}}%
\pgfpathclose%
\pgfusepath{fill}%
\end{pgfscope}%
\begin{pgfscope}%
\pgfpathrectangle{\pgfqpoint{1.020000in}{0.880000in}}{\pgfqpoint{6.160000in}{6.160000in}}%
\pgfusepath{clip}%
\pgfsetbuttcap%
\pgfsetroundjoin%
\definecolor{currentfill}{rgb}{0.958279,0.604335,0.483297}%
\pgfsetfillcolor{currentfill}%
\pgfsetlinewidth{0.000000pt}%
\definecolor{currentstroke}{rgb}{0.000000,0.000000,0.000000}%
\pgfsetstrokecolor{currentstroke}%
\pgfsetdash{}{0pt}%
\pgfpathmoveto{\pgfqpoint{2.440391in}{4.770495in}}%
\pgfpathlineto{\pgfqpoint{2.448387in}{4.794184in}}%
\pgfpathlineto{\pgfqpoint{2.456401in}{4.818042in}}%
\pgfpathlineto{\pgfqpoint{2.489132in}{4.839796in}}%
\pgfpathlineto{\pgfqpoint{2.521873in}{4.860362in}}%
\pgfpathlineto{\pgfqpoint{2.513838in}{4.832680in}}%
\pgfpathlineto{\pgfqpoint{2.505821in}{4.805226in}}%
\pgfpathlineto{\pgfqpoint{2.473104in}{4.788370in}}%
\pgfpathlineto{\pgfqpoint{2.440391in}{4.770495in}}%
\pgfpathclose%
\pgfusepath{fill}%
\end{pgfscope}%
\begin{pgfscope}%
\pgfpathrectangle{\pgfqpoint{1.020000in}{0.880000in}}{\pgfqpoint{6.160000in}{6.160000in}}%
\pgfusepath{clip}%
\pgfsetbuttcap%
\pgfsetroundjoin%
\definecolor{currentfill}{rgb}{0.613933,0.739923,0.999142}%
\pgfsetfillcolor{currentfill}%
\pgfsetlinewidth{0.000000pt}%
\definecolor{currentstroke}{rgb}{0.000000,0.000000,0.000000}%
\pgfsetstrokecolor{currentstroke}%
\pgfsetdash{}{0pt}%
\pgfpathmoveto{\pgfqpoint{4.176358in}{3.790573in}}%
\pgfpathlineto{\pgfqpoint{4.186499in}{3.770516in}}%
\pgfpathlineto{\pgfqpoint{4.196658in}{3.750001in}}%
\pgfpathlineto{\pgfqpoint{4.229168in}{3.732980in}}%
\pgfpathlineto{\pgfqpoint{4.261647in}{3.717346in}}%
\pgfpathlineto{\pgfqpoint{4.251444in}{3.737356in}}%
\pgfpathlineto{\pgfqpoint{4.241257in}{3.756926in}}%
\pgfpathlineto{\pgfqpoint{4.208823in}{3.772965in}}%
\pgfpathlineto{\pgfqpoint{4.176358in}{3.790573in}}%
\pgfpathclose%
\pgfusepath{fill}%
\end{pgfscope}%
\begin{pgfscope}%
\pgfpathrectangle{\pgfqpoint{1.020000in}{0.880000in}}{\pgfqpoint{6.160000in}{6.160000in}}%
\pgfusepath{clip}%
\pgfsetbuttcap%
\pgfsetroundjoin%
\definecolor{currentfill}{rgb}{0.323718,0.433158,0.864722}%
\pgfsetfillcolor{currentfill}%
\pgfsetlinewidth{0.000000pt}%
\definecolor{currentstroke}{rgb}{0.000000,0.000000,0.000000}%
\pgfsetstrokecolor{currentstroke}%
\pgfsetdash{}{0pt}%
\pgfpathmoveto{\pgfqpoint{5.292215in}{3.240975in}}%
\pgfpathlineto{\pgfqpoint{5.303285in}{3.220221in}}%
\pgfpathlineto{\pgfqpoint{5.314404in}{3.202327in}}%
\pgfpathlineto{\pgfqpoint{5.346836in}{3.228688in}}%
\pgfpathlineto{\pgfqpoint{5.379246in}{3.253440in}}%
\pgfpathlineto{\pgfqpoint{5.368067in}{3.270770in}}%
\pgfpathlineto{\pgfqpoint{5.356931in}{3.290146in}}%
\pgfpathlineto{\pgfqpoint{5.324584in}{3.266337in}}%
\pgfpathlineto{\pgfqpoint{5.292215in}{3.240975in}}%
\pgfpathclose%
\pgfusepath{fill}%
\end{pgfscope}%
\begin{pgfscope}%
\pgfpathrectangle{\pgfqpoint{1.020000in}{0.880000in}}{\pgfqpoint{6.160000in}{6.160000in}}%
\pgfusepath{clip}%
\pgfsetbuttcap%
\pgfsetroundjoin%
\definecolor{currentfill}{rgb}{0.925563,0.825517,0.771136}%
\pgfsetfillcolor{currentfill}%
\pgfsetlinewidth{0.000000pt}%
\definecolor{currentstroke}{rgb}{0.000000,0.000000,0.000000}%
\pgfsetstrokecolor{currentstroke}%
\pgfsetdash{}{0pt}%
\pgfpathmoveto{\pgfqpoint{3.620105in}{4.446121in}}%
\pgfpathlineto{\pgfqpoint{3.629724in}{4.438215in}}%
\pgfpathlineto{\pgfqpoint{3.639374in}{4.427887in}}%
\pgfpathlineto{\pgfqpoint{3.672324in}{4.377355in}}%
\pgfpathlineto{\pgfqpoint{3.705213in}{4.328760in}}%
\pgfpathlineto{\pgfqpoint{3.695514in}{4.339995in}}%
\pgfpathlineto{\pgfqpoint{3.685842in}{4.349173in}}%
\pgfpathlineto{\pgfqpoint{3.653003in}{4.396703in}}%
\pgfpathlineto{\pgfqpoint{3.620105in}{4.446121in}}%
\pgfpathclose%
\pgfusepath{fill}%
\end{pgfscope}%
\begin{pgfscope}%
\pgfpathrectangle{\pgfqpoint{1.020000in}{0.880000in}}{\pgfqpoint{6.160000in}{6.160000in}}%
\pgfusepath{clip}%
\pgfsetbuttcap%
\pgfsetroundjoin%
\definecolor{currentfill}{rgb}{0.299441,0.400248,0.839842}%
\pgfsetfillcolor{currentfill}%
\pgfsetlinewidth{0.000000pt}%
\definecolor{currentstroke}{rgb}{0.000000,0.000000,0.000000}%
\pgfsetstrokecolor{currentstroke}%
\pgfsetdash{}{0pt}%
\pgfpathmoveto{\pgfqpoint{5.227441in}{3.188274in}}%
\pgfpathlineto{\pgfqpoint{5.238444in}{3.166049in}}%
\pgfpathlineto{\pgfqpoint{5.249503in}{3.147558in}}%
\pgfpathlineto{\pgfqpoint{5.281956in}{3.175013in}}%
\pgfpathlineto{\pgfqpoint{5.314404in}{3.202327in}}%
\pgfpathlineto{\pgfqpoint{5.303285in}{3.220221in}}%
\pgfpathlineto{\pgfqpoint{5.292215in}{3.240975in}}%
\pgfpathlineto{\pgfqpoint{5.259831in}{3.214692in}}%
\pgfpathlineto{\pgfqpoint{5.227441in}{3.188274in}}%
\pgfpathclose%
\pgfusepath{fill}%
\end{pgfscope}%
\begin{pgfscope}%
\pgfpathrectangle{\pgfqpoint{1.020000in}{0.880000in}}{\pgfqpoint{6.160000in}{6.160000in}}%
\pgfusepath{clip}%
\pgfsetbuttcap%
\pgfsetroundjoin%
\definecolor{currentfill}{rgb}{0.313946,0.420052,0.854993}%
\pgfsetfillcolor{currentfill}%
\pgfsetlinewidth{0.000000pt}%
\definecolor{currentstroke}{rgb}{0.000000,0.000000,0.000000}%
\pgfsetstrokecolor{currentstroke}%
\pgfsetdash{}{0pt}%
\pgfpathmoveto{\pgfqpoint{4.797607in}{3.278310in}}%
\pgfpathlineto{\pgfqpoint{4.808174in}{3.240920in}}%
\pgfpathlineto{\pgfqpoint{4.818769in}{3.205755in}}%
\pgfpathlineto{\pgfqpoint{4.850959in}{3.184953in}}%
\pgfpathlineto{\pgfqpoint{4.883138in}{3.167646in}}%
\pgfpathlineto{\pgfqpoint{4.872523in}{3.205208in}}%
\pgfpathlineto{\pgfqpoint{4.861940in}{3.245295in}}%
\pgfpathlineto{\pgfqpoint{4.829779in}{3.260303in}}%
\pgfpathlineto{\pgfqpoint{4.797607in}{3.278310in}}%
\pgfpathclose%
\pgfusepath{fill}%
\end{pgfscope}%
\begin{pgfscope}%
\pgfpathrectangle{\pgfqpoint{1.020000in}{0.880000in}}{\pgfqpoint{6.160000in}{6.160000in}}%
\pgfusepath{clip}%
\pgfsetbuttcap%
\pgfsetroundjoin%
\definecolor{currentfill}{rgb}{0.961595,0.622247,0.501551}%
\pgfsetfillcolor{currentfill}%
\pgfsetlinewidth{0.000000pt}%
\definecolor{currentstroke}{rgb}{0.000000,0.000000,0.000000}%
\pgfsetstrokecolor{currentstroke}%
\pgfsetdash{}{0pt}%
\pgfpathmoveto{\pgfqpoint{3.299496in}{4.814110in}}%
\pgfpathlineto{\pgfqpoint{3.308612in}{4.831245in}}%
\pgfpathlineto{\pgfqpoint{3.317770in}{4.845899in}}%
\pgfpathlineto{\pgfqpoint{3.350962in}{4.797521in}}%
\pgfpathlineto{\pgfqpoint{3.384111in}{4.748249in}}%
\pgfpathlineto{\pgfqpoint{3.374869in}{4.736668in}}%
\pgfpathlineto{\pgfqpoint{3.365665in}{4.722841in}}%
\pgfpathlineto{\pgfqpoint{3.332602in}{4.768887in}}%
\pgfpathlineto{\pgfqpoint{3.299496in}{4.814110in}}%
\pgfpathclose%
\pgfusepath{fill}%
\end{pgfscope}%
\begin{pgfscope}%
\pgfpathrectangle{\pgfqpoint{1.020000in}{0.880000in}}{\pgfqpoint{6.160000in}{6.160000in}}%
\pgfusepath{clip}%
\pgfsetbuttcap%
\pgfsetroundjoin%
\definecolor{currentfill}{rgb}{0.963806,0.634188,0.513721}%
\pgfsetfillcolor{currentfill}%
\pgfsetlinewidth{0.000000pt}%
\definecolor{currentstroke}{rgb}{0.000000,0.000000,0.000000}%
\pgfsetstrokecolor{currentstroke}%
\pgfsetdash{}{0pt}%
\pgfpathmoveto{\pgfqpoint{2.374971in}{4.732692in}}%
\pgfpathlineto{\pgfqpoint{2.382955in}{4.752354in}}%
\pgfpathlineto{\pgfqpoint{2.390957in}{4.772123in}}%
\pgfpathlineto{\pgfqpoint{2.423677in}{4.795388in}}%
\pgfpathlineto{\pgfqpoint{2.456401in}{4.818042in}}%
\pgfpathlineto{\pgfqpoint{2.448387in}{4.794184in}}%
\pgfpathlineto{\pgfqpoint{2.440391in}{4.770495in}}%
\pgfpathlineto{\pgfqpoint{2.407682in}{4.751853in}}%
\pgfpathlineto{\pgfqpoint{2.374971in}{4.732692in}}%
\pgfpathclose%
\pgfusepath{fill}%
\end{pgfscope}%
\begin{pgfscope}%
\pgfpathrectangle{\pgfqpoint{1.020000in}{0.880000in}}{\pgfqpoint{6.160000in}{6.160000in}}%
\pgfusepath{clip}%
\pgfsetbuttcap%
\pgfsetroundjoin%
\definecolor{currentfill}{rgb}{0.271104,0.360011,0.807095}%
\pgfsetfillcolor{currentfill}%
\pgfsetlinewidth{0.000000pt}%
\definecolor{currentstroke}{rgb}{0.000000,0.000000,0.000000}%
\pgfsetstrokecolor{currentstroke}%
\pgfsetdash{}{0pt}%
\pgfpathmoveto{\pgfqpoint{5.162676in}{3.138719in}}%
\pgfpathlineto{\pgfqpoint{5.173616in}{3.115102in}}%
\pgfpathlineto{\pgfqpoint{5.184616in}{3.096045in}}%
\pgfpathlineto{\pgfqpoint{5.217054in}{3.120899in}}%
\pgfpathlineto{\pgfqpoint{5.249503in}{3.147558in}}%
\pgfpathlineto{\pgfqpoint{5.238444in}{3.166049in}}%
\pgfpathlineto{\pgfqpoint{5.227441in}{3.188274in}}%
\pgfpathlineto{\pgfqpoint{5.195053in}{3.162624in}}%
\pgfpathlineto{\pgfqpoint{5.162676in}{3.138719in}}%
\pgfpathclose%
\pgfusepath{fill}%
\end{pgfscope}%
\begin{pgfscope}%
\pgfpathrectangle{\pgfqpoint{1.020000in}{0.880000in}}{\pgfqpoint{6.160000in}{6.160000in}}%
\pgfusepath{clip}%
\pgfsetbuttcap%
\pgfsetroundjoin%
\definecolor{currentfill}{rgb}{0.688188,0.793178,0.988038}%
\pgfsetfillcolor{currentfill}%
\pgfsetlinewidth{0.000000pt}%
\definecolor{currentstroke}{rgb}{0.000000,0.000000,0.000000}%
\pgfsetstrokecolor{currentstroke}%
\pgfsetdash{}{0pt}%
\pgfpathmoveto{\pgfqpoint{4.026101in}{3.921238in}}%
\pgfpathlineto{\pgfqpoint{4.036122in}{3.900785in}}%
\pgfpathlineto{\pgfqpoint{4.046161in}{3.879550in}}%
\pgfpathlineto{\pgfqpoint{4.078764in}{3.854266in}}%
\pgfpathlineto{\pgfqpoint{4.111330in}{3.831088in}}%
\pgfpathlineto{\pgfqpoint{4.101250in}{3.851361in}}%
\pgfpathlineto{\pgfqpoint{4.091186in}{3.870992in}}%
\pgfpathlineto{\pgfqpoint{4.058663in}{3.895009in}}%
\pgfpathlineto{\pgfqpoint{4.026101in}{3.921238in}}%
\pgfpathclose%
\pgfusepath{fill}%
\end{pgfscope}%
\begin{pgfscope}%
\pgfpathrectangle{\pgfqpoint{1.020000in}{0.880000in}}{\pgfqpoint{6.160000in}{6.160000in}}%
\pgfusepath{clip}%
\pgfsetbuttcap%
\pgfsetroundjoin%
\definecolor{currentfill}{rgb}{0.457046,0.594006,0.963029}%
\pgfsetfillcolor{currentfill}%
\pgfsetlinewidth{0.000000pt}%
\definecolor{currentstroke}{rgb}{0.000000,0.000000,0.000000}%
\pgfsetstrokecolor{currentstroke}%
\pgfsetdash{}{0pt}%
\pgfpathmoveto{\pgfqpoint{4.562133in}{3.521699in}}%
\pgfpathlineto{\pgfqpoint{4.572577in}{3.494346in}}%
\pgfpathlineto{\pgfqpoint{4.583036in}{3.466581in}}%
\pgfpathlineto{\pgfqpoint{4.615365in}{3.448072in}}%
\pgfpathlineto{\pgfqpoint{4.647659in}{3.429094in}}%
\pgfpathlineto{\pgfqpoint{4.637176in}{3.460475in}}%
\pgfpathlineto{\pgfqpoint{4.626707in}{3.491290in}}%
\pgfpathlineto{\pgfqpoint{4.594435in}{3.506587in}}%
\pgfpathlineto{\pgfqpoint{4.562133in}{3.521699in}}%
\pgfpathclose%
\pgfusepath{fill}%
\end{pgfscope}%
\begin{pgfscope}%
\pgfpathrectangle{\pgfqpoint{1.020000in}{0.880000in}}{\pgfqpoint{6.160000in}{6.160000in}}%
\pgfusepath{clip}%
\pgfsetbuttcap%
\pgfsetroundjoin%
\definecolor{currentfill}{rgb}{0.915157,0.476927,0.372179}%
\pgfsetfillcolor{currentfill}%
\pgfsetlinewidth{0.000000pt}%
\definecolor{currentstroke}{rgb}{0.000000,0.000000,0.000000}%
\pgfsetstrokecolor{currentstroke}%
\pgfsetdash{}{0pt}%
\pgfpathmoveto{\pgfqpoint{2.784454in}{4.950495in}}%
\pgfpathlineto{\pgfqpoint{2.792706in}{4.987907in}}%
\pgfpathlineto{\pgfqpoint{2.800988in}{5.024889in}}%
\pgfpathlineto{\pgfqpoint{2.833977in}{5.024496in}}%
\pgfpathlineto{\pgfqpoint{2.866989in}{5.020510in}}%
\pgfpathlineto{\pgfqpoint{2.858631in}{4.983023in}}%
\pgfpathlineto{\pgfqpoint{2.850302in}{4.945112in}}%
\pgfpathlineto{\pgfqpoint{2.817369in}{4.949381in}}%
\pgfpathlineto{\pgfqpoint{2.784454in}{4.950495in}}%
\pgfpathclose%
\pgfusepath{fill}%
\end{pgfscope}%
\begin{pgfscope}%
\pgfpathrectangle{\pgfqpoint{1.020000in}{0.880000in}}{\pgfqpoint{6.160000in}{6.160000in}}%
\pgfusepath{clip}%
\pgfsetbuttcap%
\pgfsetroundjoin%
\definecolor{currentfill}{rgb}{0.883687,0.856108,0.840258}%
\pgfsetfillcolor{currentfill}%
\pgfsetlinewidth{0.000000pt}%
\definecolor{currentstroke}{rgb}{0.000000,0.000000,0.000000}%
\pgfsetstrokecolor{currentstroke}%
\pgfsetdash{}{0pt}%
\pgfpathmoveto{\pgfqpoint{3.705213in}{4.328760in}}%
\pgfpathlineto{\pgfqpoint{3.714939in}{4.315416in}}%
\pgfpathlineto{\pgfqpoint{3.724692in}{4.299937in}}%
\pgfpathlineto{\pgfqpoint{3.757567in}{4.253164in}}%
\pgfpathlineto{\pgfqpoint{3.790385in}{4.208696in}}%
\pgfpathlineto{\pgfqpoint{3.780590in}{4.224300in}}%
\pgfpathlineto{\pgfqpoint{3.770818in}{4.238118in}}%
\pgfpathlineto{\pgfqpoint{3.738044in}{4.282296in}}%
\pgfpathlineto{\pgfqpoint{3.705213in}{4.328760in}}%
\pgfpathclose%
\pgfusepath{fill}%
\end{pgfscope}%
\begin{pgfscope}%
\pgfpathrectangle{\pgfqpoint{1.020000in}{0.880000in}}{\pgfqpoint{6.160000in}{6.160000in}}%
\pgfusepath{clip}%
\pgfsetbuttcap%
\pgfsetroundjoin%
\definecolor{currentfill}{rgb}{0.378598,0.503856,0.913692}%
\pgfsetfillcolor{currentfill}%
\pgfsetlinewidth{0.000000pt}%
\definecolor{currentstroke}{rgb}{0.000000,0.000000,0.000000}%
\pgfsetstrokecolor{currentstroke}%
\pgfsetdash{}{0pt}%
\pgfpathmoveto{\pgfqpoint{5.572944in}{3.352543in}}%
\pgfpathlineto{\pgfqpoint{5.584320in}{3.336926in}}%
\pgfpathlineto{\pgfqpoint{5.595723in}{3.321806in}}%
\pgfpathlineto{\pgfqpoint{5.627912in}{3.330191in}}%
\pgfpathlineto{\pgfqpoint{5.660062in}{3.336914in}}%
\pgfpathlineto{\pgfqpoint{5.648611in}{3.352264in}}%
\pgfpathlineto{\pgfqpoint{5.637184in}{3.367880in}}%
\pgfpathlineto{\pgfqpoint{5.605083in}{3.361069in}}%
\pgfpathlineto{\pgfqpoint{5.572944in}{3.352543in}}%
\pgfpathclose%
\pgfusepath{fill}%
\end{pgfscope}%
\begin{pgfscope}%
\pgfpathrectangle{\pgfqpoint{1.020000in}{0.880000in}}{\pgfqpoint{6.160000in}{6.160000in}}%
\pgfusepath{clip}%
\pgfsetbuttcap%
\pgfsetroundjoin%
\definecolor{currentfill}{rgb}{0.967317,0.657471,0.538160}%
\pgfsetfillcolor{currentfill}%
\pgfsetlinewidth{0.000000pt}%
\definecolor{currentstroke}{rgb}{0.000000,0.000000,0.000000}%
\pgfsetstrokecolor{currentstroke}%
\pgfsetdash{}{0pt}%
\pgfpathmoveto{\pgfqpoint{2.309532in}{4.693780in}}%
\pgfpathlineto{\pgfqpoint{2.317510in}{4.709304in}}%
\pgfpathlineto{\pgfqpoint{2.325507in}{4.724872in}}%
\pgfpathlineto{\pgfqpoint{2.358234in}{4.748527in}}%
\pgfpathlineto{\pgfqpoint{2.390957in}{4.772123in}}%
\pgfpathlineto{\pgfqpoint{2.382955in}{4.752354in}}%
\pgfpathlineto{\pgfqpoint{2.374971in}{4.732692in}}%
\pgfpathlineto{\pgfqpoint{2.342256in}{4.713257in}}%
\pgfpathlineto{\pgfqpoint{2.309532in}{4.693780in}}%
\pgfpathclose%
\pgfusepath{fill}%
\end{pgfscope}%
\begin{pgfscope}%
\pgfpathrectangle{\pgfqpoint{1.020000in}{0.880000in}}{\pgfqpoint{6.160000in}{6.160000in}}%
\pgfusepath{clip}%
\pgfsetbuttcap%
\pgfsetroundjoin%
\definecolor{currentfill}{rgb}{0.918282,0.484173,0.377794}%
\pgfsetfillcolor{currentfill}%
\pgfsetlinewidth{0.000000pt}%
\definecolor{currentstroke}{rgb}{0.000000,0.000000,0.000000}%
\pgfsetstrokecolor{currentstroke}%
\pgfsetdash{}{0pt}%
\pgfpathmoveto{\pgfqpoint{2.999192in}{4.968924in}}%
\pgfpathlineto{\pgfqpoint{3.007764in}{5.003951in}}%
\pgfpathlineto{\pgfqpoint{3.016374in}{5.037670in}}%
\pgfpathlineto{\pgfqpoint{3.049539in}{5.014543in}}%
\pgfpathlineto{\pgfqpoint{3.082704in}{4.988009in}}%
\pgfpathlineto{\pgfqpoint{3.073997in}{4.956241in}}%
\pgfpathlineto{\pgfqpoint{3.065327in}{4.923213in}}%
\pgfpathlineto{\pgfqpoint{3.032260in}{4.947588in}}%
\pgfpathlineto{\pgfqpoint{2.999192in}{4.968924in}}%
\pgfpathclose%
\pgfusepath{fill}%
\end{pgfscope}%
\begin{pgfscope}%
\pgfpathrectangle{\pgfqpoint{1.020000in}{0.880000in}}{\pgfqpoint{6.160000in}{6.160000in}}%
\pgfusepath{clip}%
\pgfsetbuttcap%
\pgfsetroundjoin%
\definecolor{currentfill}{rgb}{0.248091,0.326013,0.777669}%
\pgfsetfillcolor{currentfill}%
\pgfsetlinewidth{0.000000pt}%
\definecolor{currentstroke}{rgb}{0.000000,0.000000,0.000000}%
\pgfsetstrokecolor{currentstroke}%
\pgfsetdash{}{0pt}%
\pgfpathmoveto{\pgfqpoint{5.097982in}{3.100059in}}%
\pgfpathlineto{\pgfqpoint{5.108863in}{3.075327in}}%
\pgfpathlineto{\pgfqpoint{5.119807in}{3.055807in}}%
\pgfpathlineto{\pgfqpoint{5.152199in}{3.074024in}}%
\pgfpathlineto{\pgfqpoint{5.184616in}{3.096045in}}%
\pgfpathlineto{\pgfqpoint{5.173616in}{3.115102in}}%
\pgfpathlineto{\pgfqpoint{5.162676in}{3.138719in}}%
\pgfpathlineto{\pgfqpoint{5.130317in}{3.117551in}}%
\pgfpathlineto{\pgfqpoint{5.097982in}{3.100059in}}%
\pgfpathclose%
\pgfusepath{fill}%
\end{pgfscope}%
\begin{pgfscope}%
\pgfpathrectangle{\pgfqpoint{1.020000in}{0.880000in}}{\pgfqpoint{6.160000in}{6.160000in}}%
\pgfusepath{clip}%
\pgfsetbuttcap%
\pgfsetroundjoin%
\definecolor{currentfill}{rgb}{0.521696,0.659599,0.987736}%
\pgfsetfillcolor{currentfill}%
\pgfsetlinewidth{0.000000pt}%
\definecolor{currentstroke}{rgb}{0.000000,0.000000,0.000000}%
\pgfsetstrokecolor{currentstroke}%
\pgfsetdash{}{0pt}%
\pgfpathmoveto{\pgfqpoint{4.411918in}{3.622945in}}%
\pgfpathlineto{\pgfqpoint{4.422260in}{3.600990in}}%
\pgfpathlineto{\pgfqpoint{4.432618in}{3.578614in}}%
\pgfpathlineto{\pgfqpoint{4.465042in}{3.564889in}}%
\pgfpathlineto{\pgfqpoint{4.497436in}{3.550872in}}%
\pgfpathlineto{\pgfqpoint{4.487040in}{3.574993in}}%
\pgfpathlineto{\pgfqpoint{4.476658in}{3.598500in}}%
\pgfpathlineto{\pgfqpoint{4.444302in}{3.610666in}}%
\pgfpathlineto{\pgfqpoint{4.411918in}{3.622945in}}%
\pgfpathclose%
\pgfusepath{fill}%
\end{pgfscope}%
\begin{pgfscope}%
\pgfpathrectangle{\pgfqpoint{1.020000in}{0.880000in}}{\pgfqpoint{6.160000in}{6.160000in}}%
\pgfusepath{clip}%
\pgfsetbuttcap%
\pgfsetroundjoin%
\definecolor{currentfill}{rgb}{0.936780,0.532750,0.418093}%
\pgfsetfillcolor{currentfill}%
\pgfsetlinewidth{0.000000pt}%
\definecolor{currentstroke}{rgb}{0.000000,0.000000,0.000000}%
\pgfsetstrokecolor{currentstroke}%
\pgfsetdash{}{0pt}%
\pgfpathmoveto{\pgfqpoint{3.149011in}{4.925798in}}%
\pgfpathlineto{\pgfqpoint{3.157856in}{4.953322in}}%
\pgfpathlineto{\pgfqpoint{3.166744in}{4.978789in}}%
\pgfpathlineto{\pgfqpoint{3.199972in}{4.940848in}}%
\pgfpathlineto{\pgfqpoint{3.233176in}{4.900492in}}%
\pgfpathlineto{\pgfqpoint{3.224193in}{4.877943in}}%
\pgfpathlineto{\pgfqpoint{3.215249in}{4.853472in}}%
\pgfpathlineto{\pgfqpoint{3.182141in}{4.890731in}}%
\pgfpathlineto{\pgfqpoint{3.149011in}{4.925798in}}%
\pgfpathclose%
\pgfusepath{fill}%
\end{pgfscope}%
\begin{pgfscope}%
\pgfpathrectangle{\pgfqpoint{1.020000in}{0.880000in}}{\pgfqpoint{6.160000in}{6.160000in}}%
\pgfusepath{clip}%
\pgfsetbuttcap%
\pgfsetroundjoin%
\definecolor{currentfill}{rgb}{0.261805,0.346484,0.795658}%
\pgfsetfillcolor{currentfill}%
\pgfsetlinewidth{0.000000pt}%
\definecolor{currentstroke}{rgb}{0.000000,0.000000,0.000000}%
\pgfsetstrokecolor{currentstroke}%
\pgfsetdash{}{0pt}%
\pgfpathmoveto{\pgfqpoint{4.883138in}{3.167646in}}%
\pgfpathlineto{\pgfqpoint{4.893790in}{3.133519in}}%
\pgfpathlineto{\pgfqpoint{4.904486in}{3.103620in}}%
\pgfpathlineto{\pgfqpoint{4.936692in}{3.089372in}}%
\pgfpathlineto{\pgfqpoint{4.968907in}{3.080349in}}%
\pgfpathlineto{\pgfqpoint{4.958177in}{3.111013in}}%
\pgfpathlineto{\pgfqpoint{4.947496in}{3.146179in}}%
\pgfpathlineto{\pgfqpoint{4.915314in}{3.154541in}}%
\pgfpathlineto{\pgfqpoint{4.883138in}{3.167646in}}%
\pgfpathclose%
\pgfusepath{fill}%
\end{pgfscope}%
\begin{pgfscope}%
\pgfpathrectangle{\pgfqpoint{1.020000in}{0.880000in}}{\pgfqpoint{6.160000in}{6.160000in}}%
\pgfusepath{clip}%
\pgfsetbuttcap%
\pgfsetroundjoin%
\definecolor{currentfill}{rgb}{0.968894,0.679480,0.562812}%
\pgfsetfillcolor{currentfill}%
\pgfsetlinewidth{0.000000pt}%
\definecolor{currentstroke}{rgb}{0.000000,0.000000,0.000000}%
\pgfsetstrokecolor{currentstroke}%
\pgfsetdash{}{0pt}%
\pgfpathmoveto{\pgfqpoint{2.244044in}{4.655576in}}%
\pgfpathlineto{\pgfqpoint{2.252021in}{4.666989in}}%
\pgfpathlineto{\pgfqpoint{2.260017in}{4.678384in}}%
\pgfpathlineto{\pgfqpoint{2.292769in}{4.701412in}}%
\pgfpathlineto{\pgfqpoint{2.325507in}{4.724872in}}%
\pgfpathlineto{\pgfqpoint{2.317510in}{4.709304in}}%
\pgfpathlineto{\pgfqpoint{2.309532in}{4.693780in}}%
\pgfpathlineto{\pgfqpoint{2.276796in}{4.674485in}}%
\pgfpathlineto{\pgfqpoint{2.244044in}{4.655576in}}%
\pgfpathclose%
\pgfusepath{fill}%
\end{pgfscope}%
\begin{pgfscope}%
\pgfpathrectangle{\pgfqpoint{1.020000in}{0.880000in}}{\pgfqpoint{6.160000in}{6.160000in}}%
\pgfusepath{clip}%
\pgfsetbuttcap%
\pgfsetroundjoin%
\definecolor{currentfill}{rgb}{0.831148,0.859513,0.903110}%
\pgfsetfillcolor{currentfill}%
\pgfsetlinewidth{0.000000pt}%
\definecolor{currentstroke}{rgb}{0.000000,0.000000,0.000000}%
\pgfsetstrokecolor{currentstroke}%
\pgfsetdash{}{0pt}%
\pgfpathmoveto{\pgfqpoint{3.790385in}{4.208696in}}%
\pgfpathlineto{\pgfqpoint{3.800205in}{4.191309in}}%
\pgfpathlineto{\pgfqpoint{3.810047in}{4.172164in}}%
\pgfpathlineto{\pgfqpoint{3.842850in}{4.130489in}}%
\pgfpathlineto{\pgfqpoint{3.875600in}{4.091290in}}%
\pgfpathlineto{\pgfqpoint{3.865717in}{4.109919in}}%
\pgfpathlineto{\pgfqpoint{3.855856in}{4.127098in}}%
\pgfpathlineto{\pgfqpoint{3.823148in}{4.166649in}}%
\pgfpathlineto{\pgfqpoint{3.790385in}{4.208696in}}%
\pgfpathclose%
\pgfusepath{fill}%
\end{pgfscope}%
\begin{pgfscope}%
\pgfpathrectangle{\pgfqpoint{1.020000in}{0.880000in}}{\pgfqpoint{6.160000in}{6.160000in}}%
\pgfusepath{clip}%
\pgfsetbuttcap%
\pgfsetroundjoin%
\definecolor{currentfill}{rgb}{0.576051,0.708780,0.997755}%
\pgfsetfillcolor{currentfill}%
\pgfsetlinewidth{0.000000pt}%
\definecolor{currentstroke}{rgb}{0.000000,0.000000,0.000000}%
\pgfsetstrokecolor{currentstroke}%
\pgfsetdash{}{0pt}%
\pgfpathmoveto{\pgfqpoint{4.261647in}{3.717346in}}%
\pgfpathlineto{\pgfqpoint{4.271867in}{3.696938in}}%
\pgfpathlineto{\pgfqpoint{4.282104in}{3.676179in}}%
\pgfpathlineto{\pgfqpoint{4.314600in}{3.661897in}}%
\pgfpathlineto{\pgfqpoint{4.347067in}{3.648390in}}%
\pgfpathlineto{\pgfqpoint{4.336785in}{3.669186in}}%
\pgfpathlineto{\pgfqpoint{4.326520in}{3.689564in}}%
\pgfpathlineto{\pgfqpoint{4.294098in}{3.702932in}}%
\pgfpathlineto{\pgfqpoint{4.261647in}{3.717346in}}%
\pgfpathclose%
\pgfusepath{fill}%
\end{pgfscope}%
\begin{pgfscope}%
\pgfpathrectangle{\pgfqpoint{1.020000in}{0.880000in}}{\pgfqpoint{6.160000in}{6.160000in}}%
\pgfusepath{clip}%
\pgfsetbuttcap%
\pgfsetroundjoin%
\definecolor{currentfill}{rgb}{0.399231,0.528528,0.928459}%
\pgfsetfillcolor{currentfill}%
\pgfsetlinewidth{0.000000pt}%
\definecolor{currentstroke}{rgb}{0.000000,0.000000,0.000000}%
\pgfsetstrokecolor{currentstroke}%
\pgfsetdash{}{0pt}%
\pgfpathmoveto{\pgfqpoint{4.647659in}{3.429094in}}%
\pgfpathlineto{\pgfqpoint{4.658157in}{3.397576in}}%
\pgfpathlineto{\pgfqpoint{4.668672in}{3.366418in}}%
\pgfpathlineto{\pgfqpoint{4.700952in}{3.343367in}}%
\pgfpathlineto{\pgfqpoint{4.733198in}{3.320556in}}%
\pgfpathlineto{\pgfqpoint{4.722666in}{3.355623in}}%
\pgfpathlineto{\pgfqpoint{4.712152in}{3.391147in}}%
\pgfpathlineto{\pgfqpoint{4.679921in}{3.409977in}}%
\pgfpathlineto{\pgfqpoint{4.647659in}{3.429094in}}%
\pgfpathclose%
\pgfusepath{fill}%
\end{pgfscope}%
\begin{pgfscope}%
\pgfpathrectangle{\pgfqpoint{1.020000in}{0.880000in}}{\pgfqpoint{6.160000in}{6.160000in}}%
\pgfusepath{clip}%
\pgfsetbuttcap%
\pgfsetroundjoin%
\definecolor{currentfill}{rgb}{0.968105,0.668475,0.550486}%
\pgfsetfillcolor{currentfill}%
\pgfsetlinewidth{0.000000pt}%
\definecolor{currentstroke}{rgb}{0.000000,0.000000,0.000000}%
\pgfsetstrokecolor{currentstroke}%
\pgfsetdash{}{0pt}%
\pgfpathmoveto{\pgfqpoint{3.384111in}{4.748249in}}%
\pgfpathlineto{\pgfqpoint{3.393392in}{4.757286in}}%
\pgfpathlineto{\pgfqpoint{3.402714in}{4.763497in}}%
\pgfpathlineto{\pgfqpoint{3.435891in}{4.711042in}}%
\pgfpathlineto{\pgfqpoint{3.469017in}{4.658503in}}%
\pgfpathlineto{\pgfqpoint{3.459622in}{4.654849in}}%
\pgfpathlineto{\pgfqpoint{3.450264in}{4.648682in}}%
\pgfpathlineto{\pgfqpoint{3.417213in}{4.698503in}}%
\pgfpathlineto{\pgfqpoint{3.384111in}{4.748249in}}%
\pgfpathclose%
\pgfusepath{fill}%
\end{pgfscope}%
\begin{pgfscope}%
\pgfpathrectangle{\pgfqpoint{1.020000in}{0.880000in}}{\pgfqpoint{6.160000in}{6.160000in}}%
\pgfusepath{clip}%
\pgfsetbuttcap%
\pgfsetroundjoin%
\definecolor{currentfill}{rgb}{0.373552,0.497499,0.909467}%
\pgfsetfillcolor{currentfill}%
\pgfsetlinewidth{0.000000pt}%
\definecolor{currentstroke}{rgb}{0.000000,0.000000,0.000000}%
\pgfsetstrokecolor{currentstroke}%
\pgfsetdash{}{0pt}%
\pgfpathmoveto{\pgfqpoint{5.508541in}{3.329275in}}%
\pgfpathlineto{\pgfqpoint{5.519867in}{3.313715in}}%
\pgfpathlineto{\pgfqpoint{5.531223in}{3.299016in}}%
\pgfpathlineto{\pgfqpoint{5.563494in}{3.311495in}}%
\pgfpathlineto{\pgfqpoint{5.595723in}{3.321806in}}%
\pgfpathlineto{\pgfqpoint{5.584320in}{3.336926in}}%
\pgfpathlineto{\pgfqpoint{5.572944in}{3.352543in}}%
\pgfpathlineto{\pgfqpoint{5.540764in}{3.342027in}}%
\pgfpathlineto{\pgfqpoint{5.508541in}{3.329275in}}%
\pgfpathclose%
\pgfusepath{fill}%
\end{pgfscope}%
\begin{pgfscope}%
\pgfpathrectangle{\pgfqpoint{1.020000in}{0.880000in}}{\pgfqpoint{6.160000in}{6.160000in}}%
\pgfusepath{clip}%
\pgfsetbuttcap%
\pgfsetroundjoin%
\definecolor{currentfill}{rgb}{0.234377,0.305542,0.759680}%
\pgfsetfillcolor{currentfill}%
\pgfsetlinewidth{0.000000pt}%
\definecolor{currentstroke}{rgb}{0.000000,0.000000,0.000000}%
\pgfsetstrokecolor{currentstroke}%
\pgfsetdash{}{0pt}%
\pgfpathmoveto{\pgfqpoint{5.033392in}{3.079217in}}%
\pgfpathlineto{\pgfqpoint{5.044221in}{3.053823in}}%
\pgfpathlineto{\pgfqpoint{5.055114in}{3.034006in}}%
\pgfpathlineto{\pgfqpoint{5.087445in}{3.042246in}}%
\pgfpathlineto{\pgfqpoint{5.119807in}{3.055807in}}%
\pgfpathlineto{\pgfqpoint{5.108863in}{3.075327in}}%
\pgfpathlineto{\pgfqpoint{5.097982in}{3.100059in}}%
\pgfpathlineto{\pgfqpoint{5.065673in}{3.087067in}}%
\pgfpathlineto{\pgfqpoint{5.033392in}{3.079217in}}%
\pgfpathclose%
\pgfusepath{fill}%
\end{pgfscope}%
\begin{pgfscope}%
\pgfpathrectangle{\pgfqpoint{1.020000in}{0.880000in}}{\pgfqpoint{6.160000in}{6.160000in}}%
\pgfusepath{clip}%
\pgfsetbuttcap%
\pgfsetroundjoin%
\definecolor{currentfill}{rgb}{0.640828,0.760752,0.997846}%
\pgfsetfillcolor{currentfill}%
\pgfsetlinewidth{0.000000pt}%
\definecolor{currentstroke}{rgb}{0.000000,0.000000,0.000000}%
\pgfsetstrokecolor{currentstroke}%
\pgfsetdash{}{0pt}%
\pgfpathmoveto{\pgfqpoint{4.111330in}{3.831088in}}%
\pgfpathlineto{\pgfqpoint{4.121428in}{3.810229in}}%
\pgfpathlineto{\pgfqpoint{4.131543in}{3.788848in}}%
\pgfpathlineto{\pgfqpoint{4.164117in}{3.768574in}}%
\pgfpathlineto{\pgfqpoint{4.196658in}{3.750001in}}%
\pgfpathlineto{\pgfqpoint{4.186499in}{3.770516in}}%
\pgfpathlineto{\pgfqpoint{4.176358in}{3.790573in}}%
\pgfpathlineto{\pgfqpoint{4.143861in}{3.809900in}}%
\pgfpathlineto{\pgfqpoint{4.111330in}{3.831088in}}%
\pgfpathclose%
\pgfusepath{fill}%
\end{pgfscope}%
\begin{pgfscope}%
\pgfpathrectangle{\pgfqpoint{1.020000in}{0.880000in}}{\pgfqpoint{6.160000in}{6.160000in}}%
\pgfusepath{clip}%
\pgfsetbuttcap%
\pgfsetroundjoin%
\definecolor{currentfill}{rgb}{0.969192,0.705836,0.593704}%
\pgfsetfillcolor{currentfill}%
\pgfsetlinewidth{0.000000pt}%
\definecolor{currentstroke}{rgb}{0.000000,0.000000,0.000000}%
\pgfsetstrokecolor{currentstroke}%
\pgfsetdash{}{0pt}%
\pgfpathmoveto{\pgfqpoint{2.178477in}{4.619656in}}%
\pgfpathlineto{\pgfqpoint{2.186455in}{4.627105in}}%
\pgfpathlineto{\pgfqpoint{2.194452in}{4.634477in}}%
\pgfpathlineto{\pgfqpoint{2.227246in}{4.656008in}}%
\pgfpathlineto{\pgfqpoint{2.260017in}{4.678384in}}%
\pgfpathlineto{\pgfqpoint{2.252021in}{4.666989in}}%
\pgfpathlineto{\pgfqpoint{2.244044in}{4.655576in}}%
\pgfpathlineto{\pgfqpoint{2.211273in}{4.637243in}}%
\pgfpathlineto{\pgfqpoint{2.178477in}{4.619656in}}%
\pgfpathclose%
\pgfusepath{fill}%
\end{pgfscope}%
\begin{pgfscope}%
\pgfpathrectangle{\pgfqpoint{1.020000in}{0.880000in}}{\pgfqpoint{6.160000in}{6.160000in}}%
\pgfusepath{clip}%
\pgfsetbuttcap%
\pgfsetroundjoin%
\definecolor{currentfill}{rgb}{0.915157,0.476927,0.372179}%
\pgfsetfillcolor{currentfill}%
\pgfsetlinewidth{0.000000pt}%
\definecolor{currentstroke}{rgb}{0.000000,0.000000,0.000000}%
\pgfsetstrokecolor{currentstroke}%
\pgfsetdash{}{0pt}%
\pgfpathmoveto{\pgfqpoint{2.718684in}{4.943533in}}%
\pgfpathlineto{\pgfqpoint{2.726870in}{4.979581in}}%
\pgfpathlineto{\pgfqpoint{2.735085in}{5.015196in}}%
\pgfpathlineto{\pgfqpoint{2.768023in}{5.021752in}}%
\pgfpathlineto{\pgfqpoint{2.800988in}{5.024889in}}%
\pgfpathlineto{\pgfqpoint{2.792706in}{4.987907in}}%
\pgfpathlineto{\pgfqpoint{2.784454in}{4.950495in}}%
\pgfpathlineto{\pgfqpoint{2.751559in}{4.948514in}}%
\pgfpathlineto{\pgfqpoint{2.718684in}{4.943533in}}%
\pgfpathclose%
\pgfusepath{fill}%
\end{pgfscope}%
\begin{pgfscope}%
\pgfpathrectangle{\pgfqpoint{1.020000in}{0.880000in}}{\pgfqpoint{6.160000in}{6.160000in}}%
\pgfusepath{clip}%
\pgfsetbuttcap%
\pgfsetroundjoin%
\definecolor{currentfill}{rgb}{0.777378,0.840921,0.946149}%
\pgfsetfillcolor{currentfill}%
\pgfsetlinewidth{0.000000pt}%
\definecolor{currentstroke}{rgb}{0.000000,0.000000,0.000000}%
\pgfsetstrokecolor{currentstroke}%
\pgfsetdash{}{0pt}%
\pgfpathmoveto{\pgfqpoint{3.875600in}{4.091290in}}%
\pgfpathlineto{\pgfqpoint{3.885502in}{4.071251in}}%
\pgfpathlineto{\pgfqpoint{3.895425in}{4.049861in}}%
\pgfpathlineto{\pgfqpoint{3.928163in}{4.014048in}}%
\pgfpathlineto{\pgfqpoint{3.960854in}{3.980695in}}%
\pgfpathlineto{\pgfqpoint{3.950892in}{4.001121in}}%
\pgfpathlineto{\pgfqpoint{3.940949in}{4.020446in}}%
\pgfpathlineto{\pgfqpoint{3.908299in}{4.054608in}}%
\pgfpathlineto{\pgfqpoint{3.875600in}{4.091290in}}%
\pgfpathclose%
\pgfusepath{fill}%
\end{pgfscope}%
\begin{pgfscope}%
\pgfpathrectangle{\pgfqpoint{1.020000in}{0.880000in}}{\pgfqpoint{6.160000in}{6.160000in}}%
\pgfusepath{clip}%
\pgfsetbuttcap%
\pgfsetroundjoin%
\definecolor{currentfill}{rgb}{0.338377,0.452819,0.879317}%
\pgfsetfillcolor{currentfill}%
\pgfsetlinewidth{0.000000pt}%
\definecolor{currentstroke}{rgb}{0.000000,0.000000,0.000000}%
\pgfsetstrokecolor{currentstroke}%
\pgfsetdash{}{0pt}%
\pgfpathmoveto{\pgfqpoint{4.733198in}{3.320556in}}%
\pgfpathlineto{\pgfqpoint{4.743751in}{3.286632in}}%
\pgfpathlineto{\pgfqpoint{4.754329in}{3.254548in}}%
\pgfpathlineto{\pgfqpoint{4.786562in}{3.229239in}}%
\pgfpathlineto{\pgfqpoint{4.818769in}{3.205755in}}%
\pgfpathlineto{\pgfqpoint{4.808174in}{3.240920in}}%
\pgfpathlineto{\pgfqpoint{4.797607in}{3.278310in}}%
\pgfpathlineto{\pgfqpoint{4.765415in}{3.298634in}}%
\pgfpathlineto{\pgfqpoint{4.733198in}{3.320556in}}%
\pgfpathclose%
\pgfusepath{fill}%
\end{pgfscope}%
\begin{pgfscope}%
\pgfpathrectangle{\pgfqpoint{1.020000in}{0.880000in}}{\pgfqpoint{6.160000in}{6.160000in}}%
\pgfusepath{clip}%
\pgfsetbuttcap%
\pgfsetroundjoin%
\definecolor{currentfill}{rgb}{0.378598,0.503856,0.913692}%
\pgfsetfillcolor{currentfill}%
\pgfsetlinewidth{0.000000pt}%
\definecolor{currentstroke}{rgb}{0.000000,0.000000,0.000000}%
\pgfsetstrokecolor{currentstroke}%
\pgfsetdash{}{0pt}%
\pgfpathmoveto{\pgfqpoint{5.724257in}{3.346433in}}%
\pgfpathlineto{\pgfqpoint{5.735780in}{3.331107in}}%
\pgfpathlineto{\pgfqpoint{5.747325in}{3.315888in}}%
\pgfpathlineto{\pgfqpoint{5.779422in}{3.319034in}}%
\pgfpathlineto{\pgfqpoint{5.767853in}{3.334341in}}%
\pgfpathlineto{\pgfqpoint{5.756306in}{3.349718in}}%
\pgfpathlineto{\pgfqpoint{5.724257in}{3.346433in}}%
\pgfpathclose%
\pgfusepath{fill}%
\end{pgfscope}%
\begin{pgfscope}%
\pgfpathrectangle{\pgfqpoint{1.020000in}{0.880000in}}{\pgfqpoint{6.160000in}{6.160000in}}%
\pgfusepath{clip}%
\pgfsetbuttcap%
\pgfsetroundjoin%
\definecolor{currentfill}{rgb}{0.968203,0.720844,0.612293}%
\pgfsetfillcolor{currentfill}%
\pgfsetlinewidth{0.000000pt}%
\definecolor{currentstroke}{rgb}{0.000000,0.000000,0.000000}%
\pgfsetstrokecolor{currentstroke}%
\pgfsetdash{}{0pt}%
\pgfpathmoveto{\pgfqpoint{2.112802in}{4.587292in}}%
\pgfpathlineto{\pgfqpoint{2.120781in}{4.591020in}}%
\pgfpathlineto{\pgfqpoint{2.128780in}{4.594615in}}%
\pgfpathlineto{\pgfqpoint{2.161631in}{4.613964in}}%
\pgfpathlineto{\pgfqpoint{2.194452in}{4.634477in}}%
\pgfpathlineto{\pgfqpoint{2.186455in}{4.627105in}}%
\pgfpathlineto{\pgfqpoint{2.178477in}{4.619656in}}%
\pgfpathlineto{\pgfqpoint{2.145655in}{4.602963in}}%
\pgfpathlineto{\pgfqpoint{2.112802in}{4.587292in}}%
\pgfpathclose%
\pgfusepath{fill}%
\end{pgfscope}%
\begin{pgfscope}%
\pgfpathrectangle{\pgfqpoint{1.020000in}{0.880000in}}{\pgfqpoint{6.160000in}{6.160000in}}%
\pgfusepath{clip}%
\pgfsetbuttcap%
\pgfsetroundjoin%
\definecolor{currentfill}{rgb}{0.358415,0.478426,0.896795}%
\pgfsetfillcolor{currentfill}%
\pgfsetlinewidth{0.000000pt}%
\definecolor{currentstroke}{rgb}{0.000000,0.000000,0.000000}%
\pgfsetstrokecolor{currentstroke}%
\pgfsetdash{}{0pt}%
\pgfpathmoveto{\pgfqpoint{5.443971in}{3.296367in}}%
\pgfpathlineto{\pgfqpoint{5.455245in}{3.280938in}}%
\pgfpathlineto{\pgfqpoint{5.466556in}{3.266889in}}%
\pgfpathlineto{\pgfqpoint{5.498910in}{3.284184in}}%
\pgfpathlineto{\pgfqpoint{5.531223in}{3.299016in}}%
\pgfpathlineto{\pgfqpoint{5.519867in}{3.313715in}}%
\pgfpathlineto{\pgfqpoint{5.508541in}{3.329275in}}%
\pgfpathlineto{\pgfqpoint{5.476277in}{3.314092in}}%
\pgfpathlineto{\pgfqpoint{5.443971in}{3.296367in}}%
\pgfpathclose%
\pgfusepath{fill}%
\end{pgfscope}%
\begin{pgfscope}%
\pgfpathrectangle{\pgfqpoint{1.020000in}{0.880000in}}{\pgfqpoint{6.160000in}{6.160000in}}%
\pgfusepath{clip}%
\pgfsetbuttcap%
\pgfsetroundjoin%
\definecolor{currentfill}{rgb}{0.478462,0.616564,0.972721}%
\pgfsetfillcolor{currentfill}%
\pgfsetlinewidth{0.000000pt}%
\definecolor{currentstroke}{rgb}{0.000000,0.000000,0.000000}%
\pgfsetstrokecolor{currentstroke}%
\pgfsetdash{}{0pt}%
\pgfpathmoveto{\pgfqpoint{4.497436in}{3.550872in}}%
\pgfpathlineto{\pgfqpoint{4.507848in}{3.526283in}}%
\pgfpathlineto{\pgfqpoint{4.518276in}{3.501414in}}%
\pgfpathlineto{\pgfqpoint{4.550673in}{3.484401in}}%
\pgfpathlineto{\pgfqpoint{4.583036in}{3.466581in}}%
\pgfpathlineto{\pgfqpoint{4.572577in}{3.494346in}}%
\pgfpathlineto{\pgfqpoint{4.562133in}{3.521699in}}%
\pgfpathlineto{\pgfqpoint{4.529800in}{3.536481in}}%
\pgfpathlineto{\pgfqpoint{4.497436in}{3.550872in}}%
\pgfpathclose%
\pgfusepath{fill}%
\end{pgfscope}%
\begin{pgfscope}%
\pgfpathrectangle{\pgfqpoint{1.020000in}{0.880000in}}{\pgfqpoint{6.160000in}{6.160000in}}%
\pgfusepath{clip}%
\pgfsetbuttcap%
\pgfsetroundjoin%
\definecolor{currentfill}{rgb}{0.967874,0.725847,0.618489}%
\pgfsetfillcolor{currentfill}%
\pgfsetlinewidth{0.000000pt}%
\definecolor{currentstroke}{rgb}{0.000000,0.000000,0.000000}%
\pgfsetstrokecolor{currentstroke}%
\pgfsetdash{}{0pt}%
\pgfpathmoveto{\pgfqpoint{3.469017in}{4.658503in}}%
\pgfpathlineto{\pgfqpoint{3.478449in}{4.659423in}}%
\pgfpathlineto{\pgfqpoint{3.487919in}{4.657410in}}%
\pgfpathlineto{\pgfqpoint{3.521053in}{4.603172in}}%
\pgfpathlineto{\pgfqpoint{3.554129in}{4.549649in}}%
\pgfpathlineto{\pgfqpoint{3.544598in}{4.553482in}}%
\pgfpathlineto{\pgfqpoint{3.535102in}{4.554750in}}%
\pgfpathlineto{\pgfqpoint{3.502087in}{4.606283in}}%
\pgfpathlineto{\pgfqpoint{3.469017in}{4.658503in}}%
\pgfpathclose%
\pgfusepath{fill}%
\end{pgfscope}%
\begin{pgfscope}%
\pgfpathrectangle{\pgfqpoint{1.020000in}{0.880000in}}{\pgfqpoint{6.160000in}{6.160000in}}%
\pgfusepath{clip}%
\pgfsetbuttcap%
\pgfsetroundjoin%
\definecolor{currentfill}{rgb}{0.234377,0.305542,0.759680}%
\pgfsetfillcolor{currentfill}%
\pgfsetlinewidth{0.000000pt}%
\definecolor{currentstroke}{rgb}{0.000000,0.000000,0.000000}%
\pgfsetstrokecolor{currentstroke}%
\pgfsetdash{}{0pt}%
\pgfpathmoveto{\pgfqpoint{4.968907in}{3.080349in}}%
\pgfpathlineto{\pgfqpoint{4.979691in}{3.054849in}}%
\pgfpathlineto{\pgfqpoint{4.990535in}{3.034941in}}%
\pgfpathlineto{\pgfqpoint{5.022812in}{3.031518in}}%
\pgfpathlineto{\pgfqpoint{5.055114in}{3.034006in}}%
\pgfpathlineto{\pgfqpoint{5.044221in}{3.053823in}}%
\pgfpathlineto{\pgfqpoint{5.033392in}{3.079217in}}%
\pgfpathlineto{\pgfqpoint{5.001138in}{3.076926in}}%
\pgfpathlineto{\pgfqpoint{4.968907in}{3.080349in}}%
\pgfpathclose%
\pgfusepath{fill}%
\end{pgfscope}%
\begin{pgfscope}%
\pgfpathrectangle{\pgfqpoint{1.020000in}{0.880000in}}{\pgfqpoint{6.160000in}{6.160000in}}%
\pgfusepath{clip}%
\pgfsetbuttcap%
\pgfsetroundjoin%
\definecolor{currentfill}{rgb}{0.724041,0.814910,0.975651}%
\pgfsetfillcolor{currentfill}%
\pgfsetlinewidth{0.000000pt}%
\definecolor{currentstroke}{rgb}{0.000000,0.000000,0.000000}%
\pgfsetstrokecolor{currentstroke}%
\pgfsetdash{}{0pt}%
\pgfpathmoveto{\pgfqpoint{3.960854in}{3.980695in}}%
\pgfpathlineto{\pgfqpoint{3.970834in}{3.959229in}}%
\pgfpathlineto{\pgfqpoint{3.980833in}{3.936802in}}%
\pgfpathlineto{\pgfqpoint{4.013518in}{3.907037in}}%
\pgfpathlineto{\pgfqpoint{4.046161in}{3.879550in}}%
\pgfpathlineto{\pgfqpoint{4.036122in}{3.900785in}}%
\pgfpathlineto{\pgfqpoint{4.026101in}{3.921238in}}%
\pgfpathlineto{\pgfqpoint{3.993499in}{3.949775in}}%
\pgfpathlineto{\pgfqpoint{3.960854in}{3.980695in}}%
\pgfpathclose%
\pgfusepath{fill}%
\end{pgfscope}%
\begin{pgfscope}%
\pgfpathrectangle{\pgfqpoint{1.020000in}{0.880000in}}{\pgfqpoint{6.160000in}{6.160000in}}%
\pgfusepath{clip}%
\pgfsetbuttcap%
\pgfsetroundjoin%
\definecolor{currentfill}{rgb}{0.899534,0.440692,0.344107}%
\pgfsetfillcolor{currentfill}%
\pgfsetlinewidth{0.000000pt}%
\definecolor{currentstroke}{rgb}{0.000000,0.000000,0.000000}%
\pgfsetstrokecolor{currentstroke}%
\pgfsetdash{}{0pt}%
\pgfpathmoveto{\pgfqpoint{2.933067in}{5.001722in}}%
\pgfpathlineto{\pgfqpoint{2.941546in}{5.037964in}}%
\pgfpathlineto{\pgfqpoint{2.950064in}{5.072868in}}%
\pgfpathlineto{\pgfqpoint{2.983214in}{5.057171in}}%
\pgfpathlineto{\pgfqpoint{3.016374in}{5.037670in}}%
\pgfpathlineto{\pgfqpoint{3.007764in}{5.003951in}}%
\pgfpathlineto{\pgfqpoint{2.999192in}{4.968924in}}%
\pgfpathlineto{\pgfqpoint{2.966126in}{4.987021in}}%
\pgfpathlineto{\pgfqpoint{2.933067in}{5.001722in}}%
\pgfpathclose%
\pgfusepath{fill}%
\end{pgfscope}%
\begin{pgfscope}%
\pgfpathrectangle{\pgfqpoint{1.020000in}{0.880000in}}{\pgfqpoint{6.160000in}{6.160000in}}%
\pgfusepath{clip}%
\pgfsetbuttcap%
\pgfsetroundjoin%
\definecolor{currentfill}{rgb}{0.543440,0.680003,0.993051}%
\pgfsetfillcolor{currentfill}%
\pgfsetlinewidth{0.000000pt}%
\definecolor{currentstroke}{rgb}{0.000000,0.000000,0.000000}%
\pgfsetstrokecolor{currentstroke}%
\pgfsetdash{}{0pt}%
\pgfpathmoveto{\pgfqpoint{4.347067in}{3.648390in}}%
\pgfpathlineto{\pgfqpoint{4.357366in}{3.627230in}}%
\pgfpathlineto{\pgfqpoint{4.367681in}{3.605776in}}%
\pgfpathlineto{\pgfqpoint{4.400164in}{3.592184in}}%
\pgfpathlineto{\pgfqpoint{4.432618in}{3.578614in}}%
\pgfpathlineto{\pgfqpoint{4.422260in}{3.600990in}}%
\pgfpathlineto{\pgfqpoint{4.411918in}{3.622945in}}%
\pgfpathlineto{\pgfqpoint{4.379507in}{3.635466in}}%
\pgfpathlineto{\pgfqpoint{4.347067in}{3.648390in}}%
\pgfpathclose%
\pgfusepath{fill}%
\end{pgfscope}%
\begin{pgfscope}%
\pgfpathrectangle{\pgfqpoint{1.020000in}{0.880000in}}{\pgfqpoint{6.160000in}{6.160000in}}%
\pgfusepath{clip}%
\pgfsetbuttcap%
\pgfsetroundjoin%
\definecolor{currentfill}{rgb}{0.918282,0.484173,0.377794}%
\pgfsetfillcolor{currentfill}%
\pgfsetlinewidth{0.000000pt}%
\definecolor{currentstroke}{rgb}{0.000000,0.000000,0.000000}%
\pgfsetstrokecolor{currentstroke}%
\pgfsetdash{}{0pt}%
\pgfpathmoveto{\pgfqpoint{2.652999in}{4.925142in}}%
\pgfpathlineto{\pgfqpoint{2.661130in}{4.959028in}}%
\pgfpathlineto{\pgfqpoint{2.669291in}{4.992476in}}%
\pgfpathlineto{\pgfqpoint{2.702174in}{5.005374in}}%
\pgfpathlineto{\pgfqpoint{2.735085in}{5.015196in}}%
\pgfpathlineto{\pgfqpoint{2.726870in}{4.979581in}}%
\pgfpathlineto{\pgfqpoint{2.718684in}{4.943533in}}%
\pgfpathlineto{\pgfqpoint{2.685830in}{4.935686in}}%
\pgfpathlineto{\pgfqpoint{2.652999in}{4.925142in}}%
\pgfpathclose%
\pgfusepath{fill}%
\end{pgfscope}%
\begin{pgfscope}%
\pgfpathrectangle{\pgfqpoint{1.020000in}{0.880000in}}{\pgfqpoint{6.160000in}{6.160000in}}%
\pgfusepath{clip}%
\pgfsetbuttcap%
\pgfsetroundjoin%
\definecolor{currentfill}{rgb}{0.338377,0.452819,0.879317}%
\pgfsetfillcolor{currentfill}%
\pgfsetlinewidth{0.000000pt}%
\definecolor{currentstroke}{rgb}{0.000000,0.000000,0.000000}%
\pgfsetstrokecolor{currentstroke}%
\pgfsetdash{}{0pt}%
\pgfpathmoveto{\pgfqpoint{5.379246in}{3.253440in}}%
\pgfpathlineto{\pgfqpoint{5.390468in}{3.238217in}}%
\pgfpathlineto{\pgfqpoint{5.401734in}{3.225059in}}%
\pgfpathlineto{\pgfqpoint{5.434163in}{3.247134in}}%
\pgfpathlineto{\pgfqpoint{5.466556in}{3.266889in}}%
\pgfpathlineto{\pgfqpoint{5.455245in}{3.280938in}}%
\pgfpathlineto{\pgfqpoint{5.443971in}{3.296367in}}%
\pgfpathlineto{\pgfqpoint{5.411626in}{3.276102in}}%
\pgfpathlineto{\pgfqpoint{5.379246in}{3.253440in}}%
\pgfpathclose%
\pgfusepath{fill}%
\end{pgfscope}%
\begin{pgfscope}%
\pgfpathrectangle{\pgfqpoint{1.020000in}{0.880000in}}{\pgfqpoint{6.160000in}{6.160000in}}%
\pgfusepath{clip}%
\pgfsetbuttcap%
\pgfsetroundjoin%
\definecolor{currentfill}{rgb}{0.285273,0.380129,0.823469}%
\pgfsetfillcolor{currentfill}%
\pgfsetlinewidth{0.000000pt}%
\definecolor{currentstroke}{rgb}{0.000000,0.000000,0.000000}%
\pgfsetstrokecolor{currentstroke}%
\pgfsetdash{}{0pt}%
\pgfpathmoveto{\pgfqpoint{4.818769in}{3.205755in}}%
\pgfpathlineto{\pgfqpoint{4.829398in}{3.173614in}}%
\pgfpathlineto{\pgfqpoint{4.840066in}{3.145197in}}%
\pgfpathlineto{\pgfqpoint{4.872280in}{3.122490in}}%
\pgfpathlineto{\pgfqpoint{4.904486in}{3.103620in}}%
\pgfpathlineto{\pgfqpoint{4.893790in}{3.133519in}}%
\pgfpathlineto{\pgfqpoint{4.883138in}{3.167646in}}%
\pgfpathlineto{\pgfqpoint{4.850959in}{3.184953in}}%
\pgfpathlineto{\pgfqpoint{4.818769in}{3.205755in}}%
\pgfpathclose%
\pgfusepath{fill}%
\end{pgfscope}%
\begin{pgfscope}%
\pgfpathrectangle{\pgfqpoint{1.020000in}{0.880000in}}{\pgfqpoint{6.160000in}{6.160000in}}%
\pgfusepath{clip}%
\pgfsetbuttcap%
\pgfsetroundjoin%
\definecolor{currentfill}{rgb}{0.603162,0.731527,0.999565}%
\pgfsetfillcolor{currentfill}%
\pgfsetlinewidth{0.000000pt}%
\definecolor{currentstroke}{rgb}{0.000000,0.000000,0.000000}%
\pgfsetstrokecolor{currentstroke}%
\pgfsetdash{}{0pt}%
\pgfpathmoveto{\pgfqpoint{4.196658in}{3.750001in}}%
\pgfpathlineto{\pgfqpoint{4.206834in}{3.729086in}}%
\pgfpathlineto{\pgfqpoint{4.217026in}{3.707832in}}%
\pgfpathlineto{\pgfqpoint{4.249580in}{3.691427in}}%
\pgfpathlineto{\pgfqpoint{4.282104in}{3.676179in}}%
\pgfpathlineto{\pgfqpoint{4.271867in}{3.696938in}}%
\pgfpathlineto{\pgfqpoint{4.261647in}{3.717346in}}%
\pgfpathlineto{\pgfqpoint{4.229168in}{3.732980in}}%
\pgfpathlineto{\pgfqpoint{4.196658in}{3.750001in}}%
\pgfpathclose%
\pgfusepath{fill}%
\end{pgfscope}%
\begin{pgfscope}%
\pgfpathrectangle{\pgfqpoint{1.020000in}{0.880000in}}{\pgfqpoint{6.160000in}{6.160000in}}%
\pgfusepath{clip}%
\pgfsetbuttcap%
\pgfsetroundjoin%
\definecolor{currentfill}{rgb}{0.378598,0.503856,0.913692}%
\pgfsetfillcolor{currentfill}%
\pgfsetlinewidth{0.000000pt}%
\definecolor{currentstroke}{rgb}{0.000000,0.000000,0.000000}%
\pgfsetstrokecolor{currentstroke}%
\pgfsetdash{}{0pt}%
\pgfpathmoveto{\pgfqpoint{5.660062in}{3.336914in}}%
\pgfpathlineto{\pgfqpoint{5.671538in}{3.321811in}}%
\pgfpathlineto{\pgfqpoint{5.683037in}{3.306928in}}%
\pgfpathlineto{\pgfqpoint{5.715197in}{3.311923in}}%
\pgfpathlineto{\pgfqpoint{5.747325in}{3.315888in}}%
\pgfpathlineto{\pgfqpoint{5.735780in}{3.331107in}}%
\pgfpathlineto{\pgfqpoint{5.724257in}{3.346433in}}%
\pgfpathlineto{\pgfqpoint{5.692177in}{3.342242in}}%
\pgfpathlineto{\pgfqpoint{5.660062in}{3.336914in}}%
\pgfpathclose%
\pgfusepath{fill}%
\end{pgfscope}%
\begin{pgfscope}%
\pgfpathrectangle{\pgfqpoint{1.020000in}{0.880000in}}{\pgfqpoint{6.160000in}{6.160000in}}%
\pgfusepath{clip}%
\pgfsetbuttcap%
\pgfsetroundjoin%
\definecolor{currentfill}{rgb}{0.944055,0.553153,0.435548}%
\pgfsetfillcolor{currentfill}%
\pgfsetlinewidth{0.000000pt}%
\definecolor{currentstroke}{rgb}{0.000000,0.000000,0.000000}%
\pgfsetstrokecolor{currentstroke}%
\pgfsetdash{}{0pt}%
\pgfpathmoveto{\pgfqpoint{3.233176in}{4.900492in}}%
\pgfpathlineto{\pgfqpoint{3.242202in}{4.920724in}}%
\pgfpathlineto{\pgfqpoint{3.251273in}{4.938252in}}%
\pgfpathlineto{\pgfqpoint{3.284539in}{4.892955in}}%
\pgfpathlineto{\pgfqpoint{3.317770in}{4.845899in}}%
\pgfpathlineto{\pgfqpoint{3.308612in}{4.831245in}}%
\pgfpathlineto{\pgfqpoint{3.299496in}{4.814110in}}%
\pgfpathlineto{\pgfqpoint{3.266353in}{4.858113in}}%
\pgfpathlineto{\pgfqpoint{3.233176in}{4.900492in}}%
\pgfpathclose%
\pgfusepath{fill}%
\end{pgfscope}%
\begin{pgfscope}%
\pgfpathrectangle{\pgfqpoint{1.020000in}{0.880000in}}{\pgfqpoint{6.160000in}{6.160000in}}%
\pgfusepath{clip}%
\pgfsetbuttcap%
\pgfsetroundjoin%
\definecolor{currentfill}{rgb}{0.954566,0.779055,0.692531}%
\pgfsetfillcolor{currentfill}%
\pgfsetlinewidth{0.000000pt}%
\definecolor{currentstroke}{rgb}{0.000000,0.000000,0.000000}%
\pgfsetstrokecolor{currentstroke}%
\pgfsetdash{}{0pt}%
\pgfpathmoveto{\pgfqpoint{3.554129in}{4.549649in}}%
\pgfpathlineto{\pgfqpoint{3.563694in}{4.543112in}}%
\pgfpathlineto{\pgfqpoint{3.573294in}{4.533759in}}%
\pgfpathlineto{\pgfqpoint{3.606364in}{4.480115in}}%
\pgfpathlineto{\pgfqpoint{3.639374in}{4.427887in}}%
\pgfpathlineto{\pgfqpoint{3.629724in}{4.438215in}}%
\pgfpathlineto{\pgfqpoint{3.620105in}{4.446121in}}%
\pgfpathlineto{\pgfqpoint{3.587147in}{4.497194in}}%
\pgfpathlineto{\pgfqpoint{3.554129in}{4.549649in}}%
\pgfpathclose%
\pgfusepath{fill}%
\end{pgfscope}%
\begin{pgfscope}%
\pgfpathrectangle{\pgfqpoint{1.020000in}{0.880000in}}{\pgfqpoint{6.160000in}{6.160000in}}%
\pgfusepath{clip}%
\pgfsetbuttcap%
\pgfsetroundjoin%
\definecolor{currentfill}{rgb}{0.318832,0.426605,0.859857}%
\pgfsetfillcolor{currentfill}%
\pgfsetlinewidth{0.000000pt}%
\definecolor{currentstroke}{rgb}{0.000000,0.000000,0.000000}%
\pgfsetstrokecolor{currentstroke}%
\pgfsetdash{}{0pt}%
\pgfpathmoveto{\pgfqpoint{5.314404in}{3.202327in}}%
\pgfpathlineto{\pgfqpoint{5.325573in}{3.187375in}}%
\pgfpathlineto{\pgfqpoint{5.336794in}{3.175304in}}%
\pgfpathlineto{\pgfqpoint{5.369276in}{3.200959in}}%
\pgfpathlineto{\pgfqpoint{5.401734in}{3.225059in}}%
\pgfpathlineto{\pgfqpoint{5.390468in}{3.238217in}}%
\pgfpathlineto{\pgfqpoint{5.379246in}{3.253440in}}%
\pgfpathlineto{\pgfqpoint{5.346836in}{3.228688in}}%
\pgfpathlineto{\pgfqpoint{5.314404in}{3.202327in}}%
\pgfpathclose%
\pgfusepath{fill}%
\end{pgfscope}%
\begin{pgfscope}%
\pgfpathrectangle{\pgfqpoint{1.020000in}{0.880000in}}{\pgfqpoint{6.160000in}{6.160000in}}%
\pgfusepath{clip}%
\pgfsetbuttcap%
\pgfsetroundjoin%
\definecolor{currentfill}{rgb}{0.430507,0.564883,0.948889}%
\pgfsetfillcolor{currentfill}%
\pgfsetlinewidth{0.000000pt}%
\definecolor{currentstroke}{rgb}{0.000000,0.000000,0.000000}%
\pgfsetstrokecolor{currentstroke}%
\pgfsetdash{}{0pt}%
\pgfpathmoveto{\pgfqpoint{4.583036in}{3.466581in}}%
\pgfpathlineto{\pgfqpoint{4.593511in}{3.438724in}}%
\pgfpathlineto{\pgfqpoint{4.604003in}{3.411140in}}%
\pgfpathlineto{\pgfqpoint{4.636356in}{3.389154in}}%
\pgfpathlineto{\pgfqpoint{4.668672in}{3.366418in}}%
\pgfpathlineto{\pgfqpoint{4.658157in}{3.397576in}}%
\pgfpathlineto{\pgfqpoint{4.647659in}{3.429094in}}%
\pgfpathlineto{\pgfqpoint{4.615365in}{3.448072in}}%
\pgfpathlineto{\pgfqpoint{4.583036in}{3.466581in}}%
\pgfpathclose%
\pgfusepath{fill}%
\end{pgfscope}%
\begin{pgfscope}%
\pgfpathrectangle{\pgfqpoint{1.020000in}{0.880000in}}{\pgfqpoint{6.160000in}{6.160000in}}%
\pgfusepath{clip}%
\pgfsetbuttcap%
\pgfsetroundjoin%
\definecolor{currentfill}{rgb}{0.924409,0.498590,0.389059}%
\pgfsetfillcolor{currentfill}%
\pgfsetlinewidth{0.000000pt}%
\definecolor{currentstroke}{rgb}{0.000000,0.000000,0.000000}%
\pgfsetstrokecolor{currentstroke}%
\pgfsetdash{}{0pt}%
\pgfpathmoveto{\pgfqpoint{2.587399in}{4.896788in}}%
\pgfpathlineto{\pgfqpoint{2.595488in}{4.927810in}}%
\pgfpathlineto{\pgfqpoint{2.603605in}{4.958394in}}%
\pgfpathlineto{\pgfqpoint{2.636435in}{4.976731in}}%
\pgfpathlineto{\pgfqpoint{2.669291in}{4.992476in}}%
\pgfpathlineto{\pgfqpoint{2.661130in}{4.959028in}}%
\pgfpathlineto{\pgfqpoint{2.652999in}{4.925142in}}%
\pgfpathlineto{\pgfqpoint{2.620189in}{4.912100in}}%
\pgfpathlineto{\pgfqpoint{2.587399in}{4.896788in}}%
\pgfpathclose%
\pgfusepath{fill}%
\end{pgfscope}%
\begin{pgfscope}%
\pgfpathrectangle{\pgfqpoint{1.020000in}{0.880000in}}{\pgfqpoint{6.160000in}{6.160000in}}%
\pgfusepath{clip}%
\pgfsetbuttcap%
\pgfsetroundjoin%
\definecolor{currentfill}{rgb}{0.912033,0.469680,0.366565}%
\pgfsetfillcolor{currentfill}%
\pgfsetlinewidth{0.000000pt}%
\definecolor{currentstroke}{rgb}{0.000000,0.000000,0.000000}%
\pgfsetstrokecolor{currentstroke}%
\pgfsetdash{}{0pt}%
\pgfpathmoveto{\pgfqpoint{3.082704in}{4.988009in}}%
\pgfpathlineto{\pgfqpoint{3.091452in}{5.018065in}}%
\pgfpathlineto{\pgfqpoint{3.100245in}{5.045950in}}%
\pgfpathlineto{\pgfqpoint{3.133500in}{5.013941in}}%
\pgfpathlineto{\pgfqpoint{3.166744in}{4.978789in}}%
\pgfpathlineto{\pgfqpoint{3.157856in}{4.953322in}}%
\pgfpathlineto{\pgfqpoint{3.149011in}{4.925798in}}%
\pgfpathlineto{\pgfqpoint{3.115863in}{4.958330in}}%
\pgfpathlineto{\pgfqpoint{3.082704in}{4.988009in}}%
\pgfpathclose%
\pgfusepath{fill}%
\end{pgfscope}%
\begin{pgfscope}%
\pgfpathrectangle{\pgfqpoint{1.020000in}{0.880000in}}{\pgfqpoint{6.160000in}{6.160000in}}%
\pgfusepath{clip}%
\pgfsetbuttcap%
\pgfsetroundjoin%
\definecolor{currentfill}{rgb}{0.289996,0.386836,0.828926}%
\pgfsetfillcolor{currentfill}%
\pgfsetlinewidth{0.000000pt}%
\definecolor{currentstroke}{rgb}{0.000000,0.000000,0.000000}%
\pgfsetstrokecolor{currentstroke}%
\pgfsetdash{}{0pt}%
\pgfpathmoveto{\pgfqpoint{5.249503in}{3.147558in}}%
\pgfpathlineto{\pgfqpoint{5.260620in}{3.132908in}}%
\pgfpathlineto{\pgfqpoint{5.271796in}{3.122017in}}%
\pgfpathlineto{\pgfqpoint{5.304298in}{3.148727in}}%
\pgfpathlineto{\pgfqpoint{5.336794in}{3.175304in}}%
\pgfpathlineto{\pgfqpoint{5.325573in}{3.187375in}}%
\pgfpathlineto{\pgfqpoint{5.314404in}{3.202327in}}%
\pgfpathlineto{\pgfqpoint{5.281956in}{3.175013in}}%
\pgfpathlineto{\pgfqpoint{5.249503in}{3.147558in}}%
\pgfpathclose%
\pgfusepath{fill}%
\end{pgfscope}%
\begin{pgfscope}%
\pgfpathrectangle{\pgfqpoint{1.020000in}{0.880000in}}{\pgfqpoint{6.160000in}{6.160000in}}%
\pgfusepath{clip}%
\pgfsetbuttcap%
\pgfsetroundjoin%
\definecolor{currentfill}{rgb}{0.925563,0.825517,0.771136}%
\pgfsetfillcolor{currentfill}%
\pgfsetlinewidth{0.000000pt}%
\definecolor{currentstroke}{rgb}{0.000000,0.000000,0.000000}%
\pgfsetstrokecolor{currentstroke}%
\pgfsetdash{}{0pt}%
\pgfpathmoveto{\pgfqpoint{3.639374in}{4.427887in}}%
\pgfpathlineto{\pgfqpoint{3.649054in}{4.415071in}}%
\pgfpathlineto{\pgfqpoint{3.658763in}{4.399736in}}%
\pgfpathlineto{\pgfqpoint{3.691757in}{4.348861in}}%
\pgfpathlineto{\pgfqpoint{3.724692in}{4.299937in}}%
\pgfpathlineto{\pgfqpoint{3.714939in}{4.315416in}}%
\pgfpathlineto{\pgfqpoint{3.705213in}{4.328760in}}%
\pgfpathlineto{\pgfqpoint{3.672324in}{4.377355in}}%
\pgfpathlineto{\pgfqpoint{3.639374in}{4.427887in}}%
\pgfpathclose%
\pgfusepath{fill}%
\end{pgfscope}%
\begin{pgfscope}%
\pgfpathrectangle{\pgfqpoint{1.020000in}{0.880000in}}{\pgfqpoint{6.160000in}{6.160000in}}%
\pgfusepath{clip}%
\pgfsetbuttcap%
\pgfsetroundjoin%
\definecolor{currentfill}{rgb}{0.672538,0.782861,0.991982}%
\pgfsetfillcolor{currentfill}%
\pgfsetlinewidth{0.000000pt}%
\definecolor{currentstroke}{rgb}{0.000000,0.000000,0.000000}%
\pgfsetstrokecolor{currentstroke}%
\pgfsetdash{}{0pt}%
\pgfpathmoveto{\pgfqpoint{4.046161in}{3.879550in}}%
\pgfpathlineto{\pgfqpoint{4.056217in}{3.857605in}}%
\pgfpathlineto{\pgfqpoint{4.066290in}{3.835031in}}%
\pgfpathlineto{\pgfqpoint{4.098935in}{3.810960in}}%
\pgfpathlineto{\pgfqpoint{4.131543in}{3.788848in}}%
\pgfpathlineto{\pgfqpoint{4.121428in}{3.810229in}}%
\pgfpathlineto{\pgfqpoint{4.111330in}{3.831088in}}%
\pgfpathlineto{\pgfqpoint{4.078764in}{3.854266in}}%
\pgfpathlineto{\pgfqpoint{4.046161in}{3.879550in}}%
\pgfpathclose%
\pgfusepath{fill}%
\end{pgfscope}%
\begin{pgfscope}%
\pgfpathrectangle{\pgfqpoint{1.020000in}{0.880000in}}{\pgfqpoint{6.160000in}{6.160000in}}%
\pgfusepath{clip}%
\pgfsetbuttcap%
\pgfsetroundjoin%
\definecolor{currentfill}{rgb}{0.266381,0.353304,0.801637}%
\pgfsetfillcolor{currentfill}%
\pgfsetlinewidth{0.000000pt}%
\definecolor{currentstroke}{rgb}{0.000000,0.000000,0.000000}%
\pgfsetstrokecolor{currentstroke}%
\pgfsetdash{}{0pt}%
\pgfpathmoveto{\pgfqpoint{5.184616in}{3.096045in}}%
\pgfpathlineto{\pgfqpoint{5.195680in}{3.081678in}}%
\pgfpathlineto{\pgfqpoint{5.206808in}{3.071901in}}%
\pgfpathlineto{\pgfqpoint{5.239296in}{3.096082in}}%
\pgfpathlineto{\pgfqpoint{5.271796in}{3.122017in}}%
\pgfpathlineto{\pgfqpoint{5.260620in}{3.132908in}}%
\pgfpathlineto{\pgfqpoint{5.249503in}{3.147558in}}%
\pgfpathlineto{\pgfqpoint{5.217054in}{3.120899in}}%
\pgfpathlineto{\pgfqpoint{5.184616in}{3.096045in}}%
\pgfpathclose%
\pgfusepath{fill}%
\end{pgfscope}%
\begin{pgfscope}%
\pgfpathrectangle{\pgfqpoint{1.020000in}{0.880000in}}{\pgfqpoint{6.160000in}{6.160000in}}%
\pgfusepath{clip}%
\pgfsetbuttcap%
\pgfsetroundjoin%
\definecolor{currentfill}{rgb}{0.875557,0.860242,0.851430}%
\pgfsetfillcolor{currentfill}%
\pgfsetlinewidth{0.000000pt}%
\definecolor{currentstroke}{rgb}{0.000000,0.000000,0.000000}%
\pgfsetstrokecolor{currentstroke}%
\pgfsetdash{}{0pt}%
\pgfpathmoveto{\pgfqpoint{3.724692in}{4.299937in}}%
\pgfpathlineto{\pgfqpoint{3.734470in}{4.282324in}}%
\pgfpathlineto{\pgfqpoint{3.744274in}{4.262603in}}%
\pgfpathlineto{\pgfqpoint{3.777189in}{4.216240in}}%
\pgfpathlineto{\pgfqpoint{3.810047in}{4.172164in}}%
\pgfpathlineto{\pgfqpoint{3.800205in}{4.191309in}}%
\pgfpathlineto{\pgfqpoint{3.790385in}{4.208696in}}%
\pgfpathlineto{\pgfqpoint{3.757567in}{4.253164in}}%
\pgfpathlineto{\pgfqpoint{3.724692in}{4.299937in}}%
\pgfpathclose%
\pgfusepath{fill}%
\end{pgfscope}%
\begin{pgfscope}%
\pgfpathrectangle{\pgfqpoint{1.020000in}{0.880000in}}{\pgfqpoint{6.160000in}{6.160000in}}%
\pgfusepath{clip}%
\pgfsetbuttcap%
\pgfsetroundjoin%
\definecolor{currentfill}{rgb}{0.373552,0.497499,0.909467}%
\pgfsetfillcolor{currentfill}%
\pgfsetlinewidth{0.000000pt}%
\definecolor{currentstroke}{rgb}{0.000000,0.000000,0.000000}%
\pgfsetstrokecolor{currentstroke}%
\pgfsetdash{}{0pt}%
\pgfpathmoveto{\pgfqpoint{5.595723in}{3.321806in}}%
\pgfpathlineto{\pgfqpoint{5.607154in}{3.307152in}}%
\pgfpathlineto{\pgfqpoint{5.618610in}{3.292907in}}%
\pgfpathlineto{\pgfqpoint{5.650842in}{3.300670in}}%
\pgfpathlineto{\pgfqpoint{5.683037in}{3.306928in}}%
\pgfpathlineto{\pgfqpoint{5.671538in}{3.321811in}}%
\pgfpathlineto{\pgfqpoint{5.660062in}{3.336914in}}%
\pgfpathlineto{\pgfqpoint{5.627912in}{3.330191in}}%
\pgfpathlineto{\pgfqpoint{5.595723in}{3.321806in}}%
\pgfpathclose%
\pgfusepath{fill}%
\end{pgfscope}%
\begin{pgfscope}%
\pgfpathrectangle{\pgfqpoint{1.020000in}{0.880000in}}{\pgfqpoint{6.160000in}{6.160000in}}%
\pgfusepath{clip}%
\pgfsetbuttcap%
\pgfsetroundjoin%
\definecolor{currentfill}{rgb}{0.243520,0.319189,0.771672}%
\pgfsetfillcolor{currentfill}%
\pgfsetlinewidth{0.000000pt}%
\definecolor{currentstroke}{rgb}{0.000000,0.000000,0.000000}%
\pgfsetstrokecolor{currentstroke}%
\pgfsetdash{}{0pt}%
\pgfpathmoveto{\pgfqpoint{4.904486in}{3.103620in}}%
\pgfpathlineto{\pgfqpoint{4.915231in}{3.078572in}}%
\pgfpathlineto{\pgfqpoint{4.926031in}{3.058777in}}%
\pgfpathlineto{\pgfqpoint{4.958277in}{3.044157in}}%
\pgfpathlineto{\pgfqpoint{4.990535in}{3.034941in}}%
\pgfpathlineto{\pgfqpoint{4.979691in}{3.054849in}}%
\pgfpathlineto{\pgfqpoint{4.968907in}{3.080349in}}%
\pgfpathlineto{\pgfqpoint{4.936692in}{3.089372in}}%
\pgfpathlineto{\pgfqpoint{4.904486in}{3.103620in}}%
\pgfpathclose%
\pgfusepath{fill}%
\end{pgfscope}%
\begin{pgfscope}%
\pgfpathrectangle{\pgfqpoint{1.020000in}{0.880000in}}{\pgfqpoint{6.160000in}{6.160000in}}%
\pgfusepath{clip}%
\pgfsetbuttcap%
\pgfsetroundjoin%
\definecolor{currentfill}{rgb}{0.934305,0.525918,0.412286}%
\pgfsetfillcolor{currentfill}%
\pgfsetlinewidth{0.000000pt}%
\definecolor{currentstroke}{rgb}{0.000000,0.000000,0.000000}%
\pgfsetstrokecolor{currentstroke}%
\pgfsetdash{}{0pt}%
\pgfpathmoveto{\pgfqpoint{2.521873in}{4.860362in}}%
\pgfpathlineto{\pgfqpoint{2.529931in}{4.887952in}}%
\pgfpathlineto{\pgfqpoint{2.538016in}{4.915102in}}%
\pgfpathlineto{\pgfqpoint{2.570800in}{4.937751in}}%
\pgfpathlineto{\pgfqpoint{2.603605in}{4.958394in}}%
\pgfpathlineto{\pgfqpoint{2.595488in}{4.927810in}}%
\pgfpathlineto{\pgfqpoint{2.587399in}{4.896788in}}%
\pgfpathlineto{\pgfqpoint{2.554628in}{4.879453in}}%
\pgfpathlineto{\pgfqpoint{2.521873in}{4.860362in}}%
\pgfpathclose%
\pgfusepath{fill}%
\end{pgfscope}%
\begin{pgfscope}%
\pgfpathrectangle{\pgfqpoint{1.020000in}{0.880000in}}{\pgfqpoint{6.160000in}{6.160000in}}%
\pgfusepath{clip}%
\pgfsetbuttcap%
\pgfsetroundjoin%
\definecolor{currentfill}{rgb}{0.505423,0.643995,0.983157}%
\pgfsetfillcolor{currentfill}%
\pgfsetlinewidth{0.000000pt}%
\definecolor{currentstroke}{rgb}{0.000000,0.000000,0.000000}%
\pgfsetstrokecolor{currentstroke}%
\pgfsetdash{}{0pt}%
\pgfpathmoveto{\pgfqpoint{4.432618in}{3.578614in}}%
\pgfpathlineto{\pgfqpoint{4.442992in}{3.555924in}}%
\pgfpathlineto{\pgfqpoint{4.453382in}{3.533059in}}%
\pgfpathlineto{\pgfqpoint{4.485845in}{3.517609in}}%
\pgfpathlineto{\pgfqpoint{4.518276in}{3.501414in}}%
\pgfpathlineto{\pgfqpoint{4.507848in}{3.526283in}}%
\pgfpathlineto{\pgfqpoint{4.497436in}{3.550872in}}%
\pgfpathlineto{\pgfqpoint{4.465042in}{3.564889in}}%
\pgfpathlineto{\pgfqpoint{4.432618in}{3.578614in}}%
\pgfpathclose%
\pgfusepath{fill}%
\end{pgfscope}%
\begin{pgfscope}%
\pgfpathrectangle{\pgfqpoint{1.020000in}{0.880000in}}{\pgfqpoint{6.160000in}{6.160000in}}%
\pgfusepath{clip}%
\pgfsetbuttcap%
\pgfsetroundjoin%
\definecolor{currentfill}{rgb}{0.373552,0.497499,0.909467}%
\pgfsetfillcolor{currentfill}%
\pgfsetlinewidth{0.000000pt}%
\definecolor{currentstroke}{rgb}{0.000000,0.000000,0.000000}%
\pgfsetstrokecolor{currentstroke}%
\pgfsetdash{}{0pt}%
\pgfpathmoveto{\pgfqpoint{4.668672in}{3.366418in}}%
\pgfpathlineto{\pgfqpoint{4.679207in}{3.336153in}}%
\pgfpathlineto{\pgfqpoint{4.689765in}{3.307325in}}%
\pgfpathlineto{\pgfqpoint{4.722065in}{3.280836in}}%
\pgfpathlineto{\pgfqpoint{4.754329in}{3.254548in}}%
\pgfpathlineto{\pgfqpoint{4.743751in}{3.286632in}}%
\pgfpathlineto{\pgfqpoint{4.733198in}{3.320556in}}%
\pgfpathlineto{\pgfqpoint{4.700952in}{3.343367in}}%
\pgfpathlineto{\pgfqpoint{4.668672in}{3.366418in}}%
\pgfpathclose%
\pgfusepath{fill}%
\end{pgfscope}%
\begin{pgfscope}%
\pgfpathrectangle{\pgfqpoint{1.020000in}{0.880000in}}{\pgfqpoint{6.160000in}{6.160000in}}%
\pgfusepath{clip}%
\pgfsetbuttcap%
\pgfsetroundjoin%
\definecolor{currentfill}{rgb}{0.243520,0.319189,0.771672}%
\pgfsetfillcolor{currentfill}%
\pgfsetlinewidth{0.000000pt}%
\definecolor{currentstroke}{rgb}{0.000000,0.000000,0.000000}%
\pgfsetstrokecolor{currentstroke}%
\pgfsetdash{}{0pt}%
\pgfpathmoveto{\pgfqpoint{5.119807in}{3.055807in}}%
\pgfpathlineto{\pgfqpoint{5.130818in}{3.041647in}}%
\pgfpathlineto{\pgfqpoint{5.141896in}{3.032732in}}%
\pgfpathlineto{\pgfqpoint{5.174339in}{3.050470in}}%
\pgfpathlineto{\pgfqpoint{5.206808in}{3.071901in}}%
\pgfpathlineto{\pgfqpoint{5.195680in}{3.081678in}}%
\pgfpathlineto{\pgfqpoint{5.184616in}{3.096045in}}%
\pgfpathlineto{\pgfqpoint{5.152199in}{3.074024in}}%
\pgfpathlineto{\pgfqpoint{5.119807in}{3.055807in}}%
\pgfpathclose%
\pgfusepath{fill}%
\end{pgfscope}%
\begin{pgfscope}%
\pgfpathrectangle{\pgfqpoint{1.020000in}{0.880000in}}{\pgfqpoint{6.160000in}{6.160000in}}%
\pgfusepath{clip}%
\pgfsetbuttcap%
\pgfsetroundjoin%
\definecolor{currentfill}{rgb}{0.888390,0.417703,0.327898}%
\pgfsetfillcolor{currentfill}%
\pgfsetlinewidth{0.000000pt}%
\definecolor{currentstroke}{rgb}{0.000000,0.000000,0.000000}%
\pgfsetstrokecolor{currentstroke}%
\pgfsetdash{}{0pt}%
\pgfpathmoveto{\pgfqpoint{2.866989in}{5.020510in}}%
\pgfpathlineto{\pgfqpoint{2.875382in}{5.057118in}}%
\pgfpathlineto{\pgfqpoint{2.883815in}{5.092372in}}%
\pgfpathlineto{\pgfqpoint{2.916930in}{5.084629in}}%
\pgfpathlineto{\pgfqpoint{2.950064in}{5.072868in}}%
\pgfpathlineto{\pgfqpoint{2.941546in}{5.037964in}}%
\pgfpathlineto{\pgfqpoint{2.933067in}{5.001722in}}%
\pgfpathlineto{\pgfqpoint{2.900020in}{5.012909in}}%
\pgfpathlineto{\pgfqpoint{2.866989in}{5.020510in}}%
\pgfpathclose%
\pgfusepath{fill}%
\end{pgfscope}%
\begin{pgfscope}%
\pgfpathrectangle{\pgfqpoint{1.020000in}{0.880000in}}{\pgfqpoint{6.160000in}{6.160000in}}%
\pgfusepath{clip}%
\pgfsetbuttcap%
\pgfsetroundjoin%
\definecolor{currentfill}{rgb}{0.822420,0.856898,0.910795}%
\pgfsetfillcolor{currentfill}%
\pgfsetlinewidth{0.000000pt}%
\definecolor{currentstroke}{rgb}{0.000000,0.000000,0.000000}%
\pgfsetstrokecolor{currentstroke}%
\pgfsetdash{}{0pt}%
\pgfpathmoveto{\pgfqpoint{3.810047in}{4.172164in}}%
\pgfpathlineto{\pgfqpoint{3.819911in}{4.151307in}}%
\pgfpathlineto{\pgfqpoint{3.829797in}{4.128809in}}%
\pgfpathlineto{\pgfqpoint{3.862637in}{4.088127in}}%
\pgfpathlineto{\pgfqpoint{3.895425in}{4.049861in}}%
\pgfpathlineto{\pgfqpoint{3.885502in}{4.071251in}}%
\pgfpathlineto{\pgfqpoint{3.875600in}{4.091290in}}%
\pgfpathlineto{\pgfqpoint{3.842850in}{4.130489in}}%
\pgfpathlineto{\pgfqpoint{3.810047in}{4.172164in}}%
\pgfpathclose%
\pgfusepath{fill}%
\end{pgfscope}%
\begin{pgfscope}%
\pgfpathrectangle{\pgfqpoint{1.020000in}{0.880000in}}{\pgfqpoint{6.160000in}{6.160000in}}%
\pgfusepath{clip}%
\pgfsetbuttcap%
\pgfsetroundjoin%
\definecolor{currentfill}{rgb}{0.565182,0.699438,0.996635}%
\pgfsetfillcolor{currentfill}%
\pgfsetlinewidth{0.000000pt}%
\definecolor{currentstroke}{rgb}{0.000000,0.000000,0.000000}%
\pgfsetstrokecolor{currentstroke}%
\pgfsetdash{}{0pt}%
\pgfpathmoveto{\pgfqpoint{4.282104in}{3.676179in}}%
\pgfpathlineto{\pgfqpoint{4.292358in}{3.655129in}}%
\pgfpathlineto{\pgfqpoint{4.302629in}{3.633858in}}%
\pgfpathlineto{\pgfqpoint{4.335170in}{3.619595in}}%
\pgfpathlineto{\pgfqpoint{4.367681in}{3.605776in}}%
\pgfpathlineto{\pgfqpoint{4.357366in}{3.627230in}}%
\pgfpathlineto{\pgfqpoint{4.347067in}{3.648390in}}%
\pgfpathlineto{\pgfqpoint{4.314600in}{3.661897in}}%
\pgfpathlineto{\pgfqpoint{4.282104in}{3.676179in}}%
\pgfpathclose%
\pgfusepath{fill}%
\end{pgfscope}%
\begin{pgfscope}%
\pgfpathrectangle{\pgfqpoint{1.020000in}{0.880000in}}{\pgfqpoint{6.160000in}{6.160000in}}%
\pgfusepath{clip}%
\pgfsetbuttcap%
\pgfsetroundjoin%
\definecolor{currentfill}{rgb}{0.956653,0.598034,0.477302}%
\pgfsetfillcolor{currentfill}%
\pgfsetlinewidth{0.000000pt}%
\definecolor{currentstroke}{rgb}{0.000000,0.000000,0.000000}%
\pgfsetstrokecolor{currentstroke}%
\pgfsetdash{}{0pt}%
\pgfpathmoveto{\pgfqpoint{3.317770in}{4.845899in}}%
\pgfpathlineto{\pgfqpoint{3.326971in}{4.857735in}}%
\pgfpathlineto{\pgfqpoint{3.336218in}{4.866440in}}%
\pgfpathlineto{\pgfqpoint{3.369489in}{4.815444in}}%
\pgfpathlineto{\pgfqpoint{3.402714in}{4.763497in}}%
\pgfpathlineto{\pgfqpoint{3.393392in}{4.757286in}}%
\pgfpathlineto{\pgfqpoint{3.384111in}{4.748249in}}%
\pgfpathlineto{\pgfqpoint{3.350962in}{4.797521in}}%
\pgfpathlineto{\pgfqpoint{3.317770in}{4.845899in}}%
\pgfpathclose%
\pgfusepath{fill}%
\end{pgfscope}%
\begin{pgfscope}%
\pgfpathrectangle{\pgfqpoint{1.020000in}{0.880000in}}{\pgfqpoint{6.160000in}{6.160000in}}%
\pgfusepath{clip}%
\pgfsetbuttcap%
\pgfsetroundjoin%
\definecolor{currentfill}{rgb}{0.944055,0.553153,0.435548}%
\pgfsetfillcolor{currentfill}%
\pgfsetlinewidth{0.000000pt}%
\definecolor{currentstroke}{rgb}{0.000000,0.000000,0.000000}%
\pgfsetstrokecolor{currentstroke}%
\pgfsetdash{}{0pt}%
\pgfpathmoveto{\pgfqpoint{2.456401in}{4.818042in}}%
\pgfpathlineto{\pgfqpoint{2.464438in}{4.841776in}}%
\pgfpathlineto{\pgfqpoint{2.472502in}{4.865072in}}%
\pgfpathlineto{\pgfqpoint{2.505252in}{4.890767in}}%
\pgfpathlineto{\pgfqpoint{2.538016in}{4.915102in}}%
\pgfpathlineto{\pgfqpoint{2.529931in}{4.887952in}}%
\pgfpathlineto{\pgfqpoint{2.521873in}{4.860362in}}%
\pgfpathlineto{\pgfqpoint{2.489132in}{4.839796in}}%
\pgfpathlineto{\pgfqpoint{2.456401in}{4.818042in}}%
\pgfpathclose%
\pgfusepath{fill}%
\end{pgfscope}%
\begin{pgfscope}%
\pgfpathrectangle{\pgfqpoint{1.020000in}{0.880000in}}{\pgfqpoint{6.160000in}{6.160000in}}%
\pgfusepath{clip}%
\pgfsetbuttcap%
\pgfsetroundjoin%
\definecolor{currentfill}{rgb}{0.630089,0.752516,0.998508}%
\pgfsetfillcolor{currentfill}%
\pgfsetlinewidth{0.000000pt}%
\definecolor{currentstroke}{rgb}{0.000000,0.000000,0.000000}%
\pgfsetstrokecolor{currentstroke}%
\pgfsetdash{}{0pt}%
\pgfpathmoveto{\pgfqpoint{4.131543in}{3.788848in}}%
\pgfpathlineto{\pgfqpoint{4.141675in}{3.767016in}}%
\pgfpathlineto{\pgfqpoint{4.151824in}{3.744813in}}%
\pgfpathlineto{\pgfqpoint{4.184441in}{3.725572in}}%
\pgfpathlineto{\pgfqpoint{4.217026in}{3.707832in}}%
\pgfpathlineto{\pgfqpoint{4.206834in}{3.729086in}}%
\pgfpathlineto{\pgfqpoint{4.196658in}{3.750001in}}%
\pgfpathlineto{\pgfqpoint{4.164117in}{3.768574in}}%
\pgfpathlineto{\pgfqpoint{4.131543in}{3.788848in}}%
\pgfpathclose%
\pgfusepath{fill}%
\end{pgfscope}%
\begin{pgfscope}%
\pgfpathrectangle{\pgfqpoint{1.020000in}{0.880000in}}{\pgfqpoint{6.160000in}{6.160000in}}%
\pgfusepath{clip}%
\pgfsetbuttcap%
\pgfsetroundjoin%
\definecolor{currentfill}{rgb}{0.953054,0.585211,0.465373}%
\pgfsetfillcolor{currentfill}%
\pgfsetlinewidth{0.000000pt}%
\definecolor{currentstroke}{rgb}{0.000000,0.000000,0.000000}%
\pgfsetstrokecolor{currentstroke}%
\pgfsetdash{}{0pt}%
\pgfpathmoveto{\pgfqpoint{2.390957in}{4.772123in}}%
\pgfpathlineto{\pgfqpoint{2.398981in}{4.791737in}}%
\pgfpathlineto{\pgfqpoint{2.407032in}{4.810913in}}%
\pgfpathlineto{\pgfqpoint{2.439764in}{4.838345in}}%
\pgfpathlineto{\pgfqpoint{2.472502in}{4.865072in}}%
\pgfpathlineto{\pgfqpoint{2.464438in}{4.841776in}}%
\pgfpathlineto{\pgfqpoint{2.456401in}{4.818042in}}%
\pgfpathlineto{\pgfqpoint{2.423677in}{4.795388in}}%
\pgfpathlineto{\pgfqpoint{2.390957in}{4.772123in}}%
\pgfpathclose%
\pgfusepath{fill}%
\end{pgfscope}%
\begin{pgfscope}%
\pgfpathrectangle{\pgfqpoint{1.020000in}{0.880000in}}{\pgfqpoint{6.160000in}{6.160000in}}%
\pgfusepath{clip}%
\pgfsetbuttcap%
\pgfsetroundjoin%
\definecolor{currentfill}{rgb}{0.229806,0.298718,0.753683}%
\pgfsetfillcolor{currentfill}%
\pgfsetlinewidth{0.000000pt}%
\definecolor{currentstroke}{rgb}{0.000000,0.000000,0.000000}%
\pgfsetstrokecolor{currentstroke}%
\pgfsetdash{}{0pt}%
\pgfpathmoveto{\pgfqpoint{5.055114in}{3.034006in}}%
\pgfpathlineto{\pgfqpoint{5.066072in}{3.019925in}}%
\pgfpathlineto{\pgfqpoint{5.077097in}{3.011458in}}%
\pgfpathlineto{\pgfqpoint{5.109481in}{3.019513in}}%
\pgfpathlineto{\pgfqpoint{5.141896in}{3.032732in}}%
\pgfpathlineto{\pgfqpoint{5.130818in}{3.041647in}}%
\pgfpathlineto{\pgfqpoint{5.119807in}{3.055807in}}%
\pgfpathlineto{\pgfqpoint{5.087445in}{3.042246in}}%
\pgfpathlineto{\pgfqpoint{5.055114in}{3.034006in}}%
\pgfpathclose%
\pgfusepath{fill}%
\end{pgfscope}%
\begin{pgfscope}%
\pgfpathrectangle{\pgfqpoint{1.020000in}{0.880000in}}{\pgfqpoint{6.160000in}{6.160000in}}%
\pgfusepath{clip}%
\pgfsetbuttcap%
\pgfsetroundjoin%
\definecolor{currentfill}{rgb}{0.313946,0.420052,0.854993}%
\pgfsetfillcolor{currentfill}%
\pgfsetlinewidth{0.000000pt}%
\definecolor{currentstroke}{rgb}{0.000000,0.000000,0.000000}%
\pgfsetstrokecolor{currentstroke}%
\pgfsetdash{}{0pt}%
\pgfpathmoveto{\pgfqpoint{4.754329in}{3.254548in}}%
\pgfpathlineto{\pgfqpoint{4.764936in}{3.224966in}}%
\pgfpathlineto{\pgfqpoint{4.775576in}{3.198466in}}%
\pgfpathlineto{\pgfqpoint{4.807834in}{3.170841in}}%
\pgfpathlineto{\pgfqpoint{4.840066in}{3.145197in}}%
\pgfpathlineto{\pgfqpoint{4.829398in}{3.173614in}}%
\pgfpathlineto{\pgfqpoint{4.818769in}{3.205755in}}%
\pgfpathlineto{\pgfqpoint{4.786562in}{3.229239in}}%
\pgfpathlineto{\pgfqpoint{4.754329in}{3.254548in}}%
\pgfpathclose%
\pgfusepath{fill}%
\end{pgfscope}%
\begin{pgfscope}%
\pgfpathrectangle{\pgfqpoint{1.020000in}{0.880000in}}{\pgfqpoint{6.160000in}{6.160000in}}%
\pgfusepath{clip}%
\pgfsetbuttcap%
\pgfsetroundjoin%
\definecolor{currentfill}{rgb}{0.368507,0.491141,0.905243}%
\pgfsetfillcolor{currentfill}%
\pgfsetlinewidth{0.000000pt}%
\definecolor{currentstroke}{rgb}{0.000000,0.000000,0.000000}%
\pgfsetstrokecolor{currentstroke}%
\pgfsetdash{}{0pt}%
\pgfpathmoveto{\pgfqpoint{5.531223in}{3.299016in}}%
\pgfpathlineto{\pgfqpoint{5.542611in}{3.285119in}}%
\pgfpathlineto{\pgfqpoint{5.554028in}{3.271927in}}%
\pgfpathlineto{\pgfqpoint{5.586339in}{3.283398in}}%
\pgfpathlineto{\pgfqpoint{5.618610in}{3.292907in}}%
\pgfpathlineto{\pgfqpoint{5.607154in}{3.307152in}}%
\pgfpathlineto{\pgfqpoint{5.595723in}{3.321806in}}%
\pgfpathlineto{\pgfqpoint{5.563494in}{3.311495in}}%
\pgfpathlineto{\pgfqpoint{5.531223in}{3.299016in}}%
\pgfpathclose%
\pgfusepath{fill}%
\end{pgfscope}%
\begin{pgfscope}%
\pgfpathrectangle{\pgfqpoint{1.020000in}{0.880000in}}{\pgfqpoint{6.160000in}{6.160000in}}%
\pgfusepath{clip}%
\pgfsetbuttcap%
\pgfsetroundjoin%
\definecolor{currentfill}{rgb}{0.763363,0.835092,0.955658}%
\pgfsetfillcolor{currentfill}%
\pgfsetlinewidth{0.000000pt}%
\definecolor{currentstroke}{rgb}{0.000000,0.000000,0.000000}%
\pgfsetstrokecolor{currentstroke}%
\pgfsetdash{}{0pt}%
\pgfpathmoveto{\pgfqpoint{3.895425in}{4.049861in}}%
\pgfpathlineto{\pgfqpoint{3.905367in}{4.027196in}}%
\pgfpathlineto{\pgfqpoint{3.915329in}{4.003350in}}%
\pgfpathlineto{\pgfqpoint{3.948104in}{3.968897in}}%
\pgfpathlineto{\pgfqpoint{3.980833in}{3.936802in}}%
\pgfpathlineto{\pgfqpoint{3.970834in}{3.959229in}}%
\pgfpathlineto{\pgfqpoint{3.960854in}{3.980695in}}%
\pgfpathlineto{\pgfqpoint{3.928163in}{4.014048in}}%
\pgfpathlineto{\pgfqpoint{3.895425in}{4.049861in}}%
\pgfpathclose%
\pgfusepath{fill}%
\end{pgfscope}%
\begin{pgfscope}%
\pgfpathrectangle{\pgfqpoint{1.020000in}{0.880000in}}{\pgfqpoint{6.160000in}{6.160000in}}%
\pgfusepath{clip}%
\pgfsetbuttcap%
\pgfsetroundjoin%
\definecolor{currentfill}{rgb}{0.961595,0.622247,0.501551}%
\pgfsetfillcolor{currentfill}%
\pgfsetlinewidth{0.000000pt}%
\definecolor{currentstroke}{rgb}{0.000000,0.000000,0.000000}%
\pgfsetstrokecolor{currentstroke}%
\pgfsetdash{}{0pt}%
\pgfpathmoveto{\pgfqpoint{2.325507in}{4.724872in}}%
\pgfpathlineto{\pgfqpoint{2.333525in}{4.740253in}}%
\pgfpathlineto{\pgfqpoint{2.341571in}{4.755198in}}%
\pgfpathlineto{\pgfqpoint{2.374303in}{4.783095in}}%
\pgfpathlineto{\pgfqpoint{2.407032in}{4.810913in}}%
\pgfpathlineto{\pgfqpoint{2.398981in}{4.791737in}}%
\pgfpathlineto{\pgfqpoint{2.390957in}{4.772123in}}%
\pgfpathlineto{\pgfqpoint{2.358234in}{4.748527in}}%
\pgfpathlineto{\pgfqpoint{2.325507in}{4.724872in}}%
\pgfpathclose%
\pgfusepath{fill}%
\end{pgfscope}%
\begin{pgfscope}%
\pgfpathrectangle{\pgfqpoint{1.020000in}{0.880000in}}{\pgfqpoint{6.160000in}{6.160000in}}%
\pgfusepath{clip}%
\pgfsetbuttcap%
\pgfsetroundjoin%
\definecolor{currentfill}{rgb}{0.457046,0.594006,0.963029}%
\pgfsetfillcolor{currentfill}%
\pgfsetlinewidth{0.000000pt}%
\definecolor{currentstroke}{rgb}{0.000000,0.000000,0.000000}%
\pgfsetstrokecolor{currentstroke}%
\pgfsetdash{}{0pt}%
\pgfpathmoveto{\pgfqpoint{4.518276in}{3.501414in}}%
\pgfpathlineto{\pgfqpoint{4.528720in}{3.476494in}}%
\pgfpathlineto{\pgfqpoint{4.539182in}{3.451781in}}%
\pgfpathlineto{\pgfqpoint{4.571611in}{3.432072in}}%
\pgfpathlineto{\pgfqpoint{4.604003in}{3.411140in}}%
\pgfpathlineto{\pgfqpoint{4.593511in}{3.438724in}}%
\pgfpathlineto{\pgfqpoint{4.583036in}{3.466581in}}%
\pgfpathlineto{\pgfqpoint{4.550673in}{3.484401in}}%
\pgfpathlineto{\pgfqpoint{4.518276in}{3.501414in}}%
\pgfpathclose%
\pgfusepath{fill}%
\end{pgfscope}%
\begin{pgfscope}%
\pgfpathrectangle{\pgfqpoint{1.020000in}{0.880000in}}{\pgfqpoint{6.160000in}{6.160000in}}%
\pgfusepath{clip}%
\pgfsetbuttcap%
\pgfsetroundjoin%
\definecolor{currentfill}{rgb}{0.966922,0.651969,0.531997}%
\pgfsetfillcolor{currentfill}%
\pgfsetlinewidth{0.000000pt}%
\definecolor{currentstroke}{rgb}{0.000000,0.000000,0.000000}%
\pgfsetstrokecolor{currentstroke}%
\pgfsetdash{}{0pt}%
\pgfpathmoveto{\pgfqpoint{3.402714in}{4.763497in}}%
\pgfpathlineto{\pgfqpoint{3.412078in}{4.766628in}}%
\pgfpathlineto{\pgfqpoint{3.421485in}{4.766454in}}%
\pgfpathlineto{\pgfqpoint{3.454729in}{4.711977in}}%
\pgfpathlineto{\pgfqpoint{3.487919in}{4.657410in}}%
\pgfpathlineto{\pgfqpoint{3.478449in}{4.659423in}}%
\pgfpathlineto{\pgfqpoint{3.469017in}{4.658503in}}%
\pgfpathlineto{\pgfqpoint{3.435891in}{4.711042in}}%
\pgfpathlineto{\pgfqpoint{3.402714in}{4.763497in}}%
\pgfpathclose%
\pgfusepath{fill}%
\end{pgfscope}%
\begin{pgfscope}%
\pgfpathrectangle{\pgfqpoint{1.020000in}{0.880000in}}{\pgfqpoint{6.160000in}{6.160000in}}%
\pgfusepath{clip}%
\pgfsetbuttcap%
\pgfsetroundjoin%
\definecolor{currentfill}{rgb}{0.966922,0.651969,0.531997}%
\pgfsetfillcolor{currentfill}%
\pgfsetlinewidth{0.000000pt}%
\definecolor{currentstroke}{rgb}{0.000000,0.000000,0.000000}%
\pgfsetstrokecolor{currentstroke}%
\pgfsetdash{}{0pt}%
\pgfpathmoveto{\pgfqpoint{2.260017in}{4.678384in}}%
\pgfpathlineto{\pgfqpoint{2.268034in}{4.689562in}}%
\pgfpathlineto{\pgfqpoint{2.276078in}{4.700306in}}%
\pgfpathlineto{\pgfqpoint{2.308831in}{4.727512in}}%
\pgfpathlineto{\pgfqpoint{2.341571in}{4.755198in}}%
\pgfpathlineto{\pgfqpoint{2.333525in}{4.740253in}}%
\pgfpathlineto{\pgfqpoint{2.325507in}{4.724872in}}%
\pgfpathlineto{\pgfqpoint{2.292769in}{4.701412in}}%
\pgfpathlineto{\pgfqpoint{2.260017in}{4.678384in}}%
\pgfpathclose%
\pgfusepath{fill}%
\end{pgfscope}%
\begin{pgfscope}%
\pgfpathrectangle{\pgfqpoint{1.020000in}{0.880000in}}{\pgfqpoint{6.160000in}{6.160000in}}%
\pgfusepath{clip}%
\pgfsetbuttcap%
\pgfsetroundjoin%
\definecolor{currentfill}{rgb}{0.373552,0.497499,0.909467}%
\pgfsetfillcolor{currentfill}%
\pgfsetlinewidth{0.000000pt}%
\definecolor{currentstroke}{rgb}{0.000000,0.000000,0.000000}%
\pgfsetstrokecolor{currentstroke}%
\pgfsetdash{}{0pt}%
\pgfpathmoveto{\pgfqpoint{5.747325in}{3.315888in}}%
\pgfpathlineto{\pgfqpoint{5.758892in}{3.300755in}}%
\pgfpathlineto{\pgfqpoint{5.770481in}{3.285688in}}%
\pgfpathlineto{\pgfqpoint{5.802624in}{3.288577in}}%
\pgfpathlineto{\pgfqpoint{5.791012in}{3.303784in}}%
\pgfpathlineto{\pgfqpoint{5.779422in}{3.319034in}}%
\pgfpathlineto{\pgfqpoint{5.747325in}{3.315888in}}%
\pgfpathclose%
\pgfusepath{fill}%
\end{pgfscope}%
\begin{pgfscope}%
\pgfpathrectangle{\pgfqpoint{1.020000in}{0.880000in}}{\pgfqpoint{6.160000in}{6.160000in}}%
\pgfusepath{clip}%
\pgfsetbuttcap%
\pgfsetroundjoin%
\definecolor{currentfill}{rgb}{0.353369,0.472069,0.892570}%
\pgfsetfillcolor{currentfill}%
\pgfsetlinewidth{0.000000pt}%
\definecolor{currentstroke}{rgb}{0.000000,0.000000,0.000000}%
\pgfsetstrokecolor{currentstroke}%
\pgfsetdash{}{0pt}%
\pgfpathmoveto{\pgfqpoint{5.466556in}{3.266889in}}%
\pgfpathlineto{\pgfqpoint{5.477903in}{3.254125in}}%
\pgfpathlineto{\pgfqpoint{5.489285in}{3.242490in}}%
\pgfpathlineto{\pgfqpoint{5.521676in}{3.258324in}}%
\pgfpathlineto{\pgfqpoint{5.554028in}{3.271927in}}%
\pgfpathlineto{\pgfqpoint{5.542611in}{3.285119in}}%
\pgfpathlineto{\pgfqpoint{5.531223in}{3.299016in}}%
\pgfpathlineto{\pgfqpoint{5.498910in}{3.284184in}}%
\pgfpathlineto{\pgfqpoint{5.466556in}{3.266889in}}%
\pgfpathclose%
\pgfusepath{fill}%
\end{pgfscope}%
\begin{pgfscope}%
\pgfpathrectangle{\pgfqpoint{1.020000in}{0.880000in}}{\pgfqpoint{6.160000in}{6.160000in}}%
\pgfusepath{clip}%
\pgfsetbuttcap%
\pgfsetroundjoin%
\definecolor{currentfill}{rgb}{0.884643,0.410017,0.322507}%
\pgfsetfillcolor{currentfill}%
\pgfsetlinewidth{0.000000pt}%
\definecolor{currentstroke}{rgb}{0.000000,0.000000,0.000000}%
\pgfsetstrokecolor{currentstroke}%
\pgfsetdash{}{0pt}%
\pgfpathmoveto{\pgfqpoint{3.016374in}{5.037670in}}%
\pgfpathlineto{\pgfqpoint{3.025027in}{5.069610in}}%
\pgfpathlineto{\pgfqpoint{3.033727in}{5.099292in}}%
\pgfpathlineto{\pgfqpoint{3.066986in}{5.074496in}}%
\pgfpathlineto{\pgfqpoint{3.100245in}{5.045950in}}%
\pgfpathlineto{\pgfqpoint{3.091452in}{5.018065in}}%
\pgfpathlineto{\pgfqpoint{3.082704in}{4.988009in}}%
\pgfpathlineto{\pgfqpoint{3.049539in}{5.014543in}}%
\pgfpathlineto{\pgfqpoint{3.016374in}{5.037670in}}%
\pgfpathclose%
\pgfusepath{fill}%
\end{pgfscope}%
\begin{pgfscope}%
\pgfpathrectangle{\pgfqpoint{1.020000in}{0.880000in}}{\pgfqpoint{6.160000in}{6.160000in}}%
\pgfusepath{clip}%
\pgfsetbuttcap%
\pgfsetroundjoin%
\definecolor{currentfill}{rgb}{0.915157,0.476927,0.372179}%
\pgfsetfillcolor{currentfill}%
\pgfsetlinewidth{0.000000pt}%
\definecolor{currentstroke}{rgb}{0.000000,0.000000,0.000000}%
\pgfsetstrokecolor{currentstroke}%
\pgfsetdash{}{0pt}%
\pgfpathmoveto{\pgfqpoint{3.166744in}{4.978789in}}%
\pgfpathlineto{\pgfqpoint{3.175678in}{5.001772in}}%
\pgfpathlineto{\pgfqpoint{3.184661in}{5.021853in}}%
\pgfpathlineto{\pgfqpoint{3.217978in}{4.981359in}}%
\pgfpathlineto{\pgfqpoint{3.251273in}{4.938252in}}%
\pgfpathlineto{\pgfqpoint{3.242202in}{4.920724in}}%
\pgfpathlineto{\pgfqpoint{3.233176in}{4.900492in}}%
\pgfpathlineto{\pgfqpoint{3.199972in}{4.940848in}}%
\pgfpathlineto{\pgfqpoint{3.166744in}{4.978789in}}%
\pgfpathclose%
\pgfusepath{fill}%
\end{pgfscope}%
\begin{pgfscope}%
\pgfpathrectangle{\pgfqpoint{1.020000in}{0.880000in}}{\pgfqpoint{6.160000in}{6.160000in}}%
\pgfusepath{clip}%
\pgfsetbuttcap%
\pgfsetroundjoin%
\definecolor{currentfill}{rgb}{0.880896,0.402331,0.317115}%
\pgfsetfillcolor{currentfill}%
\pgfsetlinewidth{0.000000pt}%
\definecolor{currentstroke}{rgb}{0.000000,0.000000,0.000000}%
\pgfsetstrokecolor{currentstroke}%
\pgfsetdash{}{0pt}%
\pgfpathmoveto{\pgfqpoint{2.800988in}{5.024889in}}%
\pgfpathlineto{\pgfqpoint{2.809304in}{5.060990in}}%
\pgfpathlineto{\pgfqpoint{2.817661in}{5.095739in}}%
\pgfpathlineto{\pgfqpoint{2.850724in}{5.096069in}}%
\pgfpathlineto{\pgfqpoint{2.883815in}{5.092372in}}%
\pgfpathlineto{\pgfqpoint{2.875382in}{5.057118in}}%
\pgfpathlineto{\pgfqpoint{2.866989in}{5.020510in}}%
\pgfpathlineto{\pgfqpoint{2.833977in}{5.024496in}}%
\pgfpathlineto{\pgfqpoint{2.800988in}{5.024889in}}%
\pgfpathclose%
\pgfusepath{fill}%
\end{pgfscope}%
\begin{pgfscope}%
\pgfpathrectangle{\pgfqpoint{1.020000in}{0.880000in}}{\pgfqpoint{6.160000in}{6.160000in}}%
\pgfusepath{clip}%
\pgfsetbuttcap%
\pgfsetroundjoin%
\definecolor{currentfill}{rgb}{0.708720,0.805721,0.981117}%
\pgfsetfillcolor{currentfill}%
\pgfsetlinewidth{0.000000pt}%
\definecolor{currentstroke}{rgb}{0.000000,0.000000,0.000000}%
\pgfsetstrokecolor{currentstroke}%
\pgfsetdash{}{0pt}%
\pgfpathmoveto{\pgfqpoint{3.980833in}{3.936802in}}%
\pgfpathlineto{\pgfqpoint{3.990849in}{3.913501in}}%
\pgfpathlineto{\pgfqpoint{4.000883in}{3.889429in}}%
\pgfpathlineto{\pgfqpoint{4.033607in}{3.861161in}}%
\pgfpathlineto{\pgfqpoint{4.066290in}{3.835031in}}%
\pgfpathlineto{\pgfqpoint{4.056217in}{3.857605in}}%
\pgfpathlineto{\pgfqpoint{4.046161in}{3.879550in}}%
\pgfpathlineto{\pgfqpoint{4.013518in}{3.907037in}}%
\pgfpathlineto{\pgfqpoint{3.980833in}{3.936802in}}%
\pgfpathclose%
\pgfusepath{fill}%
\end{pgfscope}%
\begin{pgfscope}%
\pgfpathrectangle{\pgfqpoint{1.020000in}{0.880000in}}{\pgfqpoint{6.160000in}{6.160000in}}%
\pgfusepath{clip}%
\pgfsetbuttcap%
\pgfsetroundjoin%
\definecolor{currentfill}{rgb}{0.527132,0.664700,0.989065}%
\pgfsetfillcolor{currentfill}%
\pgfsetlinewidth{0.000000pt}%
\definecolor{currentstroke}{rgb}{0.000000,0.000000,0.000000}%
\pgfsetstrokecolor{currentstroke}%
\pgfsetdash{}{0pt}%
\pgfpathmoveto{\pgfqpoint{4.367681in}{3.605776in}}%
\pgfpathlineto{\pgfqpoint{4.378013in}{3.584116in}}%
\pgfpathlineto{\pgfqpoint{4.388362in}{3.562355in}}%
\pgfpathlineto{\pgfqpoint{4.420887in}{3.547908in}}%
\pgfpathlineto{\pgfqpoint{4.453382in}{3.533059in}}%
\pgfpathlineto{\pgfqpoint{4.442992in}{3.555924in}}%
\pgfpathlineto{\pgfqpoint{4.432618in}{3.578614in}}%
\pgfpathlineto{\pgfqpoint{4.400164in}{3.592184in}}%
\pgfpathlineto{\pgfqpoint{4.367681in}{3.605776in}}%
\pgfpathclose%
\pgfusepath{fill}%
\end{pgfscope}%
\begin{pgfscope}%
\pgfpathrectangle{\pgfqpoint{1.020000in}{0.880000in}}{\pgfqpoint{6.160000in}{6.160000in}}%
\pgfusepath{clip}%
\pgfsetbuttcap%
\pgfsetroundjoin%
\definecolor{currentfill}{rgb}{0.968894,0.679480,0.562812}%
\pgfsetfillcolor{currentfill}%
\pgfsetlinewidth{0.000000pt}%
\definecolor{currentstroke}{rgb}{0.000000,0.000000,0.000000}%
\pgfsetstrokecolor{currentstroke}%
\pgfsetdash{}{0pt}%
\pgfpathmoveto{\pgfqpoint{2.194452in}{4.634477in}}%
\pgfpathlineto{\pgfqpoint{2.202471in}{4.641602in}}%
\pgfpathlineto{\pgfqpoint{2.210516in}{4.648295in}}%
\pgfpathlineto{\pgfqpoint{2.243308in}{4.673826in}}%
\pgfpathlineto{\pgfqpoint{2.276078in}{4.700306in}}%
\pgfpathlineto{\pgfqpoint{2.268034in}{4.689562in}}%
\pgfpathlineto{\pgfqpoint{2.260017in}{4.678384in}}%
\pgfpathlineto{\pgfqpoint{2.227246in}{4.656008in}}%
\pgfpathlineto{\pgfqpoint{2.194452in}{4.634477in}}%
\pgfpathclose%
\pgfusepath{fill}%
\end{pgfscope}%
\begin{pgfscope}%
\pgfpathrectangle{\pgfqpoint{1.020000in}{0.880000in}}{\pgfqpoint{6.160000in}{6.160000in}}%
\pgfusepath{clip}%
\pgfsetbuttcap%
\pgfsetroundjoin%
\definecolor{currentfill}{rgb}{0.266381,0.353304,0.801637}%
\pgfsetfillcolor{currentfill}%
\pgfsetlinewidth{0.000000pt}%
\definecolor{currentstroke}{rgb}{0.000000,0.000000,0.000000}%
\pgfsetstrokecolor{currentstroke}%
\pgfsetdash{}{0pt}%
\pgfpathmoveto{\pgfqpoint{4.840066in}{3.145197in}}%
\pgfpathlineto{\pgfqpoint{4.850776in}{3.121055in}}%
\pgfpathlineto{\pgfqpoint{4.861534in}{3.101541in}}%
\pgfpathlineto{\pgfqpoint{4.893786in}{3.078176in}}%
\pgfpathlineto{\pgfqpoint{4.926031in}{3.058777in}}%
\pgfpathlineto{\pgfqpoint{4.915231in}{3.078572in}}%
\pgfpathlineto{\pgfqpoint{4.904486in}{3.103620in}}%
\pgfpathlineto{\pgfqpoint{4.872280in}{3.122490in}}%
\pgfpathlineto{\pgfqpoint{4.840066in}{3.145197in}}%
\pgfpathclose%
\pgfusepath{fill}%
\end{pgfscope}%
\begin{pgfscope}%
\pgfpathrectangle{\pgfqpoint{1.020000in}{0.880000in}}{\pgfqpoint{6.160000in}{6.160000in}}%
\pgfusepath{clip}%
\pgfsetbuttcap%
\pgfsetroundjoin%
\definecolor{currentfill}{rgb}{0.229806,0.298718,0.753683}%
\pgfsetfillcolor{currentfill}%
\pgfsetlinewidth{0.000000pt}%
\definecolor{currentstroke}{rgb}{0.000000,0.000000,0.000000}%
\pgfsetstrokecolor{currentstroke}%
\pgfsetdash{}{0pt}%
\pgfpathmoveto{\pgfqpoint{4.990535in}{3.034941in}}%
\pgfpathlineto{\pgfqpoint{5.001442in}{3.020783in}}%
\pgfpathlineto{\pgfqpoint{5.012412in}{3.012253in}}%
\pgfpathlineto{\pgfqpoint{5.044742in}{3.008986in}}%
\pgfpathlineto{\pgfqpoint{5.077097in}{3.011458in}}%
\pgfpathlineto{\pgfqpoint{5.066072in}{3.019925in}}%
\pgfpathlineto{\pgfqpoint{5.055114in}{3.034006in}}%
\pgfpathlineto{\pgfqpoint{5.022812in}{3.031518in}}%
\pgfpathlineto{\pgfqpoint{4.990535in}{3.034941in}}%
\pgfpathclose%
\pgfusepath{fill}%
\end{pgfscope}%
\begin{pgfscope}%
\pgfpathrectangle{\pgfqpoint{1.020000in}{0.880000in}}{\pgfqpoint{6.160000in}{6.160000in}}%
\pgfusepath{clip}%
\pgfsetbuttcap%
\pgfsetroundjoin%
\definecolor{currentfill}{rgb}{0.968203,0.720844,0.612293}%
\pgfsetfillcolor{currentfill}%
\pgfsetlinewidth{0.000000pt}%
\definecolor{currentstroke}{rgb}{0.000000,0.000000,0.000000}%
\pgfsetstrokecolor{currentstroke}%
\pgfsetdash{}{0pt}%
\pgfpathmoveto{\pgfqpoint{3.487919in}{4.657410in}}%
\pgfpathlineto{\pgfqpoint{3.497427in}{4.652301in}}%
\pgfpathlineto{\pgfqpoint{3.506975in}{4.643966in}}%
\pgfpathlineto{\pgfqpoint{3.540163in}{4.588496in}}%
\pgfpathlineto{\pgfqpoint{3.573294in}{4.533759in}}%
\pgfpathlineto{\pgfqpoint{3.563694in}{4.543112in}}%
\pgfpathlineto{\pgfqpoint{3.554129in}{4.549649in}}%
\pgfpathlineto{\pgfqpoint{3.521053in}{4.603172in}}%
\pgfpathlineto{\pgfqpoint{3.487919in}{4.657410in}}%
\pgfpathclose%
\pgfusepath{fill}%
\end{pgfscope}%
\begin{pgfscope}%
\pgfpathrectangle{\pgfqpoint{1.020000in}{0.880000in}}{\pgfqpoint{6.160000in}{6.160000in}}%
\pgfusepath{clip}%
\pgfsetbuttcap%
\pgfsetroundjoin%
\definecolor{currentfill}{rgb}{0.586921,0.718121,0.998874}%
\pgfsetfillcolor{currentfill}%
\pgfsetlinewidth{0.000000pt}%
\definecolor{currentstroke}{rgb}{0.000000,0.000000,0.000000}%
\pgfsetstrokecolor{currentstroke}%
\pgfsetdash{}{0pt}%
\pgfpathmoveto{\pgfqpoint{4.217026in}{3.707832in}}%
\pgfpathlineto{\pgfqpoint{4.227235in}{3.686309in}}%
\pgfpathlineto{\pgfqpoint{4.237461in}{3.664594in}}%
\pgfpathlineto{\pgfqpoint{4.270060in}{3.648786in}}%
\pgfpathlineto{\pgfqpoint{4.302629in}{3.633858in}}%
\pgfpathlineto{\pgfqpoint{4.292358in}{3.655129in}}%
\pgfpathlineto{\pgfqpoint{4.282104in}{3.676179in}}%
\pgfpathlineto{\pgfqpoint{4.249580in}{3.691427in}}%
\pgfpathlineto{\pgfqpoint{4.217026in}{3.707832in}}%
\pgfpathclose%
\pgfusepath{fill}%
\end{pgfscope}%
\begin{pgfscope}%
\pgfpathrectangle{\pgfqpoint{1.020000in}{0.880000in}}{\pgfqpoint{6.160000in}{6.160000in}}%
\pgfusepath{clip}%
\pgfsetbuttcap%
\pgfsetroundjoin%
\definecolor{currentfill}{rgb}{0.969192,0.705836,0.593704}%
\pgfsetfillcolor{currentfill}%
\pgfsetlinewidth{0.000000pt}%
\definecolor{currentstroke}{rgb}{0.000000,0.000000,0.000000}%
\pgfsetstrokecolor{currentstroke}%
\pgfsetdash{}{0pt}%
\pgfpathmoveto{\pgfqpoint{2.128780in}{4.594615in}}%
\pgfpathlineto{\pgfqpoint{2.136801in}{4.597935in}}%
\pgfpathlineto{\pgfqpoint{2.144847in}{4.600826in}}%
\pgfpathlineto{\pgfqpoint{2.177697in}{4.623906in}}%
\pgfpathlineto{\pgfqpoint{2.210516in}{4.648295in}}%
\pgfpathlineto{\pgfqpoint{2.202471in}{4.641602in}}%
\pgfpathlineto{\pgfqpoint{2.194452in}{4.634477in}}%
\pgfpathlineto{\pgfqpoint{2.161631in}{4.613964in}}%
\pgfpathlineto{\pgfqpoint{2.128780in}{4.594615in}}%
\pgfpathclose%
\pgfusepath{fill}%
\end{pgfscope}%
\begin{pgfscope}%
\pgfpathrectangle{\pgfqpoint{1.020000in}{0.880000in}}{\pgfqpoint{6.160000in}{6.160000in}}%
\pgfusepath{clip}%
\pgfsetbuttcap%
\pgfsetroundjoin%
\definecolor{currentfill}{rgb}{0.338377,0.452819,0.879317}%
\pgfsetfillcolor{currentfill}%
\pgfsetlinewidth{0.000000pt}%
\definecolor{currentstroke}{rgb}{0.000000,0.000000,0.000000}%
\pgfsetstrokecolor{currentstroke}%
\pgfsetdash{}{0pt}%
\pgfpathmoveto{\pgfqpoint{5.401734in}{3.225059in}}%
\pgfpathlineto{\pgfqpoint{5.413043in}{3.213819in}}%
\pgfpathlineto{\pgfqpoint{5.424393in}{3.204267in}}%
\pgfpathlineto{\pgfqpoint{5.456856in}{3.224429in}}%
\pgfpathlineto{\pgfqpoint{5.489285in}{3.242490in}}%
\pgfpathlineto{\pgfqpoint{5.477903in}{3.254125in}}%
\pgfpathlineto{\pgfqpoint{5.466556in}{3.266889in}}%
\pgfpathlineto{\pgfqpoint{5.434163in}{3.247134in}}%
\pgfpathlineto{\pgfqpoint{5.401734in}{3.225059in}}%
\pgfpathclose%
\pgfusepath{fill}%
\end{pgfscope}%
\begin{pgfscope}%
\pgfpathrectangle{\pgfqpoint{1.020000in}{0.880000in}}{\pgfqpoint{6.160000in}{6.160000in}}%
\pgfusepath{clip}%
\pgfsetbuttcap%
\pgfsetroundjoin%
\definecolor{currentfill}{rgb}{0.409611,0.540759,0.935545}%
\pgfsetfillcolor{currentfill}%
\pgfsetlinewidth{0.000000pt}%
\definecolor{currentstroke}{rgb}{0.000000,0.000000,0.000000}%
\pgfsetstrokecolor{currentstroke}%
\pgfsetdash{}{0pt}%
\pgfpathmoveto{\pgfqpoint{4.604003in}{3.411140in}}%
\pgfpathlineto{\pgfqpoint{4.614514in}{3.384222in}}%
\pgfpathlineto{\pgfqpoint{4.625046in}{3.358372in}}%
\pgfpathlineto{\pgfqpoint{4.657425in}{3.333346in}}%
\pgfpathlineto{\pgfqpoint{4.689765in}{3.307325in}}%
\pgfpathlineto{\pgfqpoint{4.679207in}{3.336153in}}%
\pgfpathlineto{\pgfqpoint{4.668672in}{3.366418in}}%
\pgfpathlineto{\pgfqpoint{4.636356in}{3.389154in}}%
\pgfpathlineto{\pgfqpoint{4.604003in}{3.411140in}}%
\pgfpathclose%
\pgfusepath{fill}%
\end{pgfscope}%
\begin{pgfscope}%
\pgfpathrectangle{\pgfqpoint{1.020000in}{0.880000in}}{\pgfqpoint{6.160000in}{6.160000in}}%
\pgfusepath{clip}%
\pgfsetbuttcap%
\pgfsetroundjoin%
\definecolor{currentfill}{rgb}{0.373552,0.497499,0.909467}%
\pgfsetfillcolor{currentfill}%
\pgfsetlinewidth{0.000000pt}%
\definecolor{currentstroke}{rgb}{0.000000,0.000000,0.000000}%
\pgfsetstrokecolor{currentstroke}%
\pgfsetdash{}{0pt}%
\pgfpathmoveto{\pgfqpoint{5.683037in}{3.306928in}}%
\pgfpathlineto{\pgfqpoint{5.694560in}{3.292224in}}%
\pgfpathlineto{\pgfqpoint{5.706106in}{3.277655in}}%
\pgfpathlineto{\pgfqpoint{5.738309in}{3.282107in}}%
\pgfpathlineto{\pgfqpoint{5.770481in}{3.285688in}}%
\pgfpathlineto{\pgfqpoint{5.758892in}{3.300755in}}%
\pgfpathlineto{\pgfqpoint{5.747325in}{3.315888in}}%
\pgfpathlineto{\pgfqpoint{5.715197in}{3.311923in}}%
\pgfpathlineto{\pgfqpoint{5.683037in}{3.306928in}}%
\pgfpathclose%
\pgfusepath{fill}%
\end{pgfscope}%
\begin{pgfscope}%
\pgfpathrectangle{\pgfqpoint{1.020000in}{0.880000in}}{\pgfqpoint{6.160000in}{6.160000in}}%
\pgfusepath{clip}%
\pgfsetbuttcap%
\pgfsetroundjoin%
\definecolor{currentfill}{rgb}{0.954566,0.779055,0.692531}%
\pgfsetfillcolor{currentfill}%
\pgfsetlinewidth{0.000000pt}%
\definecolor{currentstroke}{rgb}{0.000000,0.000000,0.000000}%
\pgfsetstrokecolor{currentstroke}%
\pgfsetdash{}{0pt}%
\pgfpathmoveto{\pgfqpoint{3.573294in}{4.533759in}}%
\pgfpathlineto{\pgfqpoint{3.582927in}{4.521513in}}%
\pgfpathlineto{\pgfqpoint{3.592593in}{4.506334in}}%
\pgfpathlineto{\pgfqpoint{3.625708in}{4.452321in}}%
\pgfpathlineto{\pgfqpoint{3.658763in}{4.399736in}}%
\pgfpathlineto{\pgfqpoint{3.649054in}{4.415071in}}%
\pgfpathlineto{\pgfqpoint{3.639374in}{4.427887in}}%
\pgfpathlineto{\pgfqpoint{3.606364in}{4.480115in}}%
\pgfpathlineto{\pgfqpoint{3.573294in}{4.533759in}}%
\pgfpathclose%
\pgfusepath{fill}%
\end{pgfscope}%
\begin{pgfscope}%
\pgfpathrectangle{\pgfqpoint{1.020000in}{0.880000in}}{\pgfqpoint{6.160000in}{6.160000in}}%
\pgfusepath{clip}%
\pgfsetbuttcap%
\pgfsetroundjoin%
\definecolor{currentfill}{rgb}{0.313946,0.420052,0.854993}%
\pgfsetfillcolor{currentfill}%
\pgfsetlinewidth{0.000000pt}%
\definecolor{currentstroke}{rgb}{0.000000,0.000000,0.000000}%
\pgfsetstrokecolor{currentstroke}%
\pgfsetdash{}{0pt}%
\pgfpathmoveto{\pgfqpoint{5.336794in}{3.175304in}}%
\pgfpathlineto{\pgfqpoint{5.348066in}{3.165911in}}%
\pgfpathlineto{\pgfqpoint{5.359384in}{3.158872in}}%
\pgfpathlineto{\pgfqpoint{5.391900in}{3.182273in}}%
\pgfpathlineto{\pgfqpoint{5.424393in}{3.204267in}}%
\pgfpathlineto{\pgfqpoint{5.413043in}{3.213819in}}%
\pgfpathlineto{\pgfqpoint{5.401734in}{3.225059in}}%
\pgfpathlineto{\pgfqpoint{5.369276in}{3.200959in}}%
\pgfpathlineto{\pgfqpoint{5.336794in}{3.175304in}}%
\pgfpathclose%
\pgfusepath{fill}%
\end{pgfscope}%
\begin{pgfscope}%
\pgfpathrectangle{\pgfqpoint{1.020000in}{0.880000in}}{\pgfqpoint{6.160000in}{6.160000in}}%
\pgfusepath{clip}%
\pgfsetbuttcap%
\pgfsetroundjoin%
\definecolor{currentfill}{rgb}{0.656683,0.771806,0.994914}%
\pgfsetfillcolor{currentfill}%
\pgfsetlinewidth{0.000000pt}%
\definecolor{currentstroke}{rgb}{0.000000,0.000000,0.000000}%
\pgfsetstrokecolor{currentstroke}%
\pgfsetdash{}{0pt}%
\pgfpathmoveto{\pgfqpoint{4.066290in}{3.835031in}}%
\pgfpathlineto{\pgfqpoint{4.076380in}{3.811919in}}%
\pgfpathlineto{\pgfqpoint{4.086487in}{3.788366in}}%
\pgfpathlineto{\pgfqpoint{4.119173in}{3.765701in}}%
\pgfpathlineto{\pgfqpoint{4.151824in}{3.744813in}}%
\pgfpathlineto{\pgfqpoint{4.141675in}{3.767016in}}%
\pgfpathlineto{\pgfqpoint{4.131543in}{3.788848in}}%
\pgfpathlineto{\pgfqpoint{4.098935in}{3.810960in}}%
\pgfpathlineto{\pgfqpoint{4.066290in}{3.835031in}}%
\pgfpathclose%
\pgfusepath{fill}%
\end{pgfscope}%
\begin{pgfscope}%
\pgfpathrectangle{\pgfqpoint{1.020000in}{0.880000in}}{\pgfqpoint{6.160000in}{6.160000in}}%
\pgfusepath{clip}%
\pgfsetbuttcap%
\pgfsetroundjoin%
\definecolor{currentfill}{rgb}{0.919376,0.831273,0.782874}%
\pgfsetfillcolor{currentfill}%
\pgfsetlinewidth{0.000000pt}%
\definecolor{currentstroke}{rgb}{0.000000,0.000000,0.000000}%
\pgfsetstrokecolor{currentstroke}%
\pgfsetdash{}{0pt}%
\pgfpathmoveto{\pgfqpoint{3.658763in}{4.399736in}}%
\pgfpathlineto{\pgfqpoint{3.668501in}{4.381878in}}%
\pgfpathlineto{\pgfqpoint{3.678268in}{4.361529in}}%
\pgfpathlineto{\pgfqpoint{3.711301in}{4.311099in}}%
\pgfpathlineto{\pgfqpoint{3.744274in}{4.262603in}}%
\pgfpathlineto{\pgfqpoint{3.734470in}{4.282324in}}%
\pgfpathlineto{\pgfqpoint{3.724692in}{4.299937in}}%
\pgfpathlineto{\pgfqpoint{3.691757in}{4.348861in}}%
\pgfpathlineto{\pgfqpoint{3.658763in}{4.399736in}}%
\pgfpathclose%
\pgfusepath{fill}%
\end{pgfscope}%
\begin{pgfscope}%
\pgfpathrectangle{\pgfqpoint{1.020000in}{0.880000in}}{\pgfqpoint{6.160000in}{6.160000in}}%
\pgfusepath{clip}%
\pgfsetbuttcap%
\pgfsetroundjoin%
\definecolor{currentfill}{rgb}{0.877149,0.394645,0.311724}%
\pgfsetfillcolor{currentfill}%
\pgfsetlinewidth{0.000000pt}%
\definecolor{currentstroke}{rgb}{0.000000,0.000000,0.000000}%
\pgfsetstrokecolor{currentstroke}%
\pgfsetdash{}{0pt}%
\pgfpathmoveto{\pgfqpoint{2.735085in}{5.015196in}}%
\pgfpathlineto{\pgfqpoint{2.743334in}{5.049937in}}%
\pgfpathlineto{\pgfqpoint{2.751625in}{5.083344in}}%
\pgfpathlineto{\pgfqpoint{2.784627in}{5.091456in}}%
\pgfpathlineto{\pgfqpoint{2.817661in}{5.095739in}}%
\pgfpathlineto{\pgfqpoint{2.809304in}{5.060990in}}%
\pgfpathlineto{\pgfqpoint{2.800988in}{5.024889in}}%
\pgfpathlineto{\pgfqpoint{2.768023in}{5.021752in}}%
\pgfpathlineto{\pgfqpoint{2.735085in}{5.015196in}}%
\pgfpathclose%
\pgfusepath{fill}%
\end{pgfscope}%
\begin{pgfscope}%
\pgfpathrectangle{\pgfqpoint{1.020000in}{0.880000in}}{\pgfqpoint{6.160000in}{6.160000in}}%
\pgfusepath{clip}%
\pgfsetbuttcap%
\pgfsetroundjoin%
\definecolor{currentfill}{rgb}{0.289996,0.386836,0.828926}%
\pgfsetfillcolor{currentfill}%
\pgfsetlinewidth{0.000000pt}%
\definecolor{currentstroke}{rgb}{0.000000,0.000000,0.000000}%
\pgfsetstrokecolor{currentstroke}%
\pgfsetdash{}{0pt}%
\pgfpathmoveto{\pgfqpoint{5.271796in}{3.122017in}}%
\pgfpathlineto{\pgfqpoint{5.283028in}{3.114622in}}%
\pgfpathlineto{\pgfqpoint{5.294314in}{3.110298in}}%
\pgfpathlineto{\pgfqpoint{5.326853in}{3.134642in}}%
\pgfpathlineto{\pgfqpoint{5.359384in}{3.158872in}}%
\pgfpathlineto{\pgfqpoint{5.348066in}{3.165911in}}%
\pgfpathlineto{\pgfqpoint{5.336794in}{3.175304in}}%
\pgfpathlineto{\pgfqpoint{5.304298in}{3.148727in}}%
\pgfpathlineto{\pgfqpoint{5.271796in}{3.122017in}}%
\pgfpathclose%
\pgfusepath{fill}%
\end{pgfscope}%
\begin{pgfscope}%
\pgfpathrectangle{\pgfqpoint{1.020000in}{0.880000in}}{\pgfqpoint{6.160000in}{6.160000in}}%
\pgfusepath{clip}%
\pgfsetbuttcap%
\pgfsetroundjoin%
\definecolor{currentfill}{rgb}{0.348323,0.465711,0.888346}%
\pgfsetfillcolor{currentfill}%
\pgfsetlinewidth{0.000000pt}%
\definecolor{currentstroke}{rgb}{0.000000,0.000000,0.000000}%
\pgfsetstrokecolor{currentstroke}%
\pgfsetdash{}{0pt}%
\pgfpathmoveto{\pgfqpoint{4.689765in}{3.307325in}}%
\pgfpathlineto{\pgfqpoint{4.700348in}{3.280448in}}%
\pgfpathlineto{\pgfqpoint{4.710960in}{3.255975in}}%
\pgfpathlineto{\pgfqpoint{4.743287in}{3.227133in}}%
\pgfpathlineto{\pgfqpoint{4.775576in}{3.198466in}}%
\pgfpathlineto{\pgfqpoint{4.764936in}{3.224966in}}%
\pgfpathlineto{\pgfqpoint{4.754329in}{3.254548in}}%
\pgfpathlineto{\pgfqpoint{4.722065in}{3.280836in}}%
\pgfpathlineto{\pgfqpoint{4.689765in}{3.307325in}}%
\pgfpathclose%
\pgfusepath{fill}%
\end{pgfscope}%
\begin{pgfscope}%
\pgfpathrectangle{\pgfqpoint{1.020000in}{0.880000in}}{\pgfqpoint{6.160000in}{6.160000in}}%
\pgfusepath{clip}%
\pgfsetbuttcap%
\pgfsetroundjoin%
\definecolor{currentfill}{rgb}{0.863392,0.865084,0.867634}%
\pgfsetfillcolor{currentfill}%
\pgfsetlinewidth{0.000000pt}%
\definecolor{currentstroke}{rgb}{0.000000,0.000000,0.000000}%
\pgfsetstrokecolor{currentstroke}%
\pgfsetdash{}{0pt}%
\pgfpathmoveto{\pgfqpoint{3.744274in}{4.262603in}}%
\pgfpathlineto{\pgfqpoint{3.754102in}{4.240831in}}%
\pgfpathlineto{\pgfqpoint{3.763954in}{4.217089in}}%
\pgfpathlineto{\pgfqpoint{3.796903in}{4.171834in}}%
\pgfpathlineto{\pgfqpoint{3.829797in}{4.128809in}}%
\pgfpathlineto{\pgfqpoint{3.819911in}{4.151307in}}%
\pgfpathlineto{\pgfqpoint{3.810047in}{4.172164in}}%
\pgfpathlineto{\pgfqpoint{3.777189in}{4.216240in}}%
\pgfpathlineto{\pgfqpoint{3.744274in}{4.262603in}}%
\pgfpathclose%
\pgfusepath{fill}%
\end{pgfscope}%
\begin{pgfscope}%
\pgfpathrectangle{\pgfqpoint{1.020000in}{0.880000in}}{\pgfqpoint{6.160000in}{6.160000in}}%
\pgfusepath{clip}%
\pgfsetbuttcap%
\pgfsetroundjoin%
\definecolor{currentfill}{rgb}{0.483854,0.622050,0.974808}%
\pgfsetfillcolor{currentfill}%
\pgfsetlinewidth{0.000000pt}%
\definecolor{currentstroke}{rgb}{0.000000,0.000000,0.000000}%
\pgfsetstrokecolor{currentstroke}%
\pgfsetdash{}{0pt}%
\pgfpathmoveto{\pgfqpoint{4.453382in}{3.533059in}}%
\pgfpathlineto{\pgfqpoint{4.463790in}{3.510179in}}%
\pgfpathlineto{\pgfqpoint{4.474215in}{3.487467in}}%
\pgfpathlineto{\pgfqpoint{4.506716in}{3.470224in}}%
\pgfpathlineto{\pgfqpoint{4.539182in}{3.451781in}}%
\pgfpathlineto{\pgfqpoint{4.528720in}{3.476494in}}%
\pgfpathlineto{\pgfqpoint{4.518276in}{3.501414in}}%
\pgfpathlineto{\pgfqpoint{4.485845in}{3.517609in}}%
\pgfpathlineto{\pgfqpoint{4.453382in}{3.533059in}}%
\pgfpathclose%
\pgfusepath{fill}%
\end{pgfscope}%
\begin{pgfscope}%
\pgfpathrectangle{\pgfqpoint{1.020000in}{0.880000in}}{\pgfqpoint{6.160000in}{6.160000in}}%
\pgfusepath{clip}%
\pgfsetbuttcap%
\pgfsetroundjoin%
\definecolor{currentfill}{rgb}{0.238948,0.312365,0.765676}%
\pgfsetfillcolor{currentfill}%
\pgfsetlinewidth{0.000000pt}%
\definecolor{currentstroke}{rgb}{0.000000,0.000000,0.000000}%
\pgfsetstrokecolor{currentstroke}%
\pgfsetdash{}{0pt}%
\pgfpathmoveto{\pgfqpoint{4.926031in}{3.058777in}}%
\pgfpathlineto{\pgfqpoint{4.936887in}{3.044386in}}%
\pgfpathlineto{\pgfqpoint{4.947801in}{3.035285in}}%
\pgfpathlineto{\pgfqpoint{4.980101in}{3.021145in}}%
\pgfpathlineto{\pgfqpoint{5.012412in}{3.012253in}}%
\pgfpathlineto{\pgfqpoint{5.001442in}{3.020783in}}%
\pgfpathlineto{\pgfqpoint{4.990535in}{3.034941in}}%
\pgfpathlineto{\pgfqpoint{4.958277in}{3.044157in}}%
\pgfpathlineto{\pgfqpoint{4.926031in}{3.058777in}}%
\pgfpathclose%
\pgfusepath{fill}%
\end{pgfscope}%
\begin{pgfscope}%
\pgfpathrectangle{\pgfqpoint{1.020000in}{0.880000in}}{\pgfqpoint{6.160000in}{6.160000in}}%
\pgfusepath{clip}%
\pgfsetbuttcap%
\pgfsetroundjoin%
\definecolor{currentfill}{rgb}{0.929357,0.512254,0.400673}%
\pgfsetfillcolor{currentfill}%
\pgfsetlinewidth{0.000000pt}%
\definecolor{currentstroke}{rgb}{0.000000,0.000000,0.000000}%
\pgfsetstrokecolor{currentstroke}%
\pgfsetdash{}{0pt}%
\pgfpathmoveto{\pgfqpoint{3.251273in}{4.938252in}}%
\pgfpathlineto{\pgfqpoint{3.260392in}{4.952707in}}%
\pgfpathlineto{\pgfqpoint{3.269561in}{4.963741in}}%
\pgfpathlineto{\pgfqpoint{3.302907in}{4.916026in}}%
\pgfpathlineto{\pgfqpoint{3.336218in}{4.866440in}}%
\pgfpathlineto{\pgfqpoint{3.326971in}{4.857735in}}%
\pgfpathlineto{\pgfqpoint{3.317770in}{4.845899in}}%
\pgfpathlineto{\pgfqpoint{3.284539in}{4.892955in}}%
\pgfpathlineto{\pgfqpoint{3.251273in}{4.938252in}}%
\pgfpathclose%
\pgfusepath{fill}%
\end{pgfscope}%
\begin{pgfscope}%
\pgfpathrectangle{\pgfqpoint{1.020000in}{0.880000in}}{\pgfqpoint{6.160000in}{6.160000in}}%
\pgfusepath{clip}%
\pgfsetbuttcap%
\pgfsetroundjoin%
\definecolor{currentfill}{rgb}{0.266381,0.353304,0.801637}%
\pgfsetfillcolor{currentfill}%
\pgfsetlinewidth{0.000000pt}%
\definecolor{currentstroke}{rgb}{0.000000,0.000000,0.000000}%
\pgfsetstrokecolor{currentstroke}%
\pgfsetdash{}{0pt}%
\pgfpathmoveto{\pgfqpoint{5.206808in}{3.071901in}}%
\pgfpathlineto{\pgfqpoint{5.217998in}{3.066392in}}%
\pgfpathlineto{\pgfqpoint{5.229248in}{3.064633in}}%
\pgfpathlineto{\pgfqpoint{5.261776in}{3.086665in}}%
\pgfpathlineto{\pgfqpoint{5.294314in}{3.110298in}}%
\pgfpathlineto{\pgfqpoint{5.283028in}{3.114622in}}%
\pgfpathlineto{\pgfqpoint{5.271796in}{3.122017in}}%
\pgfpathlineto{\pgfqpoint{5.239296in}{3.096082in}}%
\pgfpathlineto{\pgfqpoint{5.206808in}{3.071901in}}%
\pgfpathclose%
\pgfusepath{fill}%
\end{pgfscope}%
\begin{pgfscope}%
\pgfpathrectangle{\pgfqpoint{1.020000in}{0.880000in}}{\pgfqpoint{6.160000in}{6.160000in}}%
\pgfusepath{clip}%
\pgfsetbuttcap%
\pgfsetroundjoin%
\definecolor{currentfill}{rgb}{0.368507,0.491141,0.905243}%
\pgfsetfillcolor{currentfill}%
\pgfsetlinewidth{0.000000pt}%
\definecolor{currentstroke}{rgb}{0.000000,0.000000,0.000000}%
\pgfsetstrokecolor{currentstroke}%
\pgfsetdash{}{0pt}%
\pgfpathmoveto{\pgfqpoint{5.618610in}{3.292907in}}%
\pgfpathlineto{\pgfqpoint{5.630092in}{3.278998in}}%
\pgfpathlineto{\pgfqpoint{5.641599in}{3.265340in}}%
\pgfpathlineto{\pgfqpoint{5.673869in}{3.272135in}}%
\pgfpathlineto{\pgfqpoint{5.706106in}{3.277655in}}%
\pgfpathlineto{\pgfqpoint{5.694560in}{3.292224in}}%
\pgfpathlineto{\pgfqpoint{5.683037in}{3.306928in}}%
\pgfpathlineto{\pgfqpoint{5.650842in}{3.300670in}}%
\pgfpathlineto{\pgfqpoint{5.618610in}{3.292907in}}%
\pgfpathclose%
\pgfusepath{fill}%
\end{pgfscope}%
\begin{pgfscope}%
\pgfpathrectangle{\pgfqpoint{1.020000in}{0.880000in}}{\pgfqpoint{6.160000in}{6.160000in}}%
\pgfusepath{clip}%
\pgfsetbuttcap%
\pgfsetroundjoin%
\definecolor{currentfill}{rgb}{0.809329,0.852974,0.922323}%
\pgfsetfillcolor{currentfill}%
\pgfsetlinewidth{0.000000pt}%
\definecolor{currentstroke}{rgb}{0.000000,0.000000,0.000000}%
\pgfsetstrokecolor{currentstroke}%
\pgfsetdash{}{0pt}%
\pgfpathmoveto{\pgfqpoint{3.829797in}{4.128809in}}%
\pgfpathlineto{\pgfqpoint{3.839704in}{4.104762in}}%
\pgfpathlineto{\pgfqpoint{3.849631in}{4.079278in}}%
\pgfpathlineto{\pgfqpoint{3.882505in}{4.040155in}}%
\pgfpathlineto{\pgfqpoint{3.915329in}{4.003350in}}%
\pgfpathlineto{\pgfqpoint{3.905367in}{4.027196in}}%
\pgfpathlineto{\pgfqpoint{3.895425in}{4.049861in}}%
\pgfpathlineto{\pgfqpoint{3.862637in}{4.088127in}}%
\pgfpathlineto{\pgfqpoint{3.829797in}{4.128809in}}%
\pgfpathclose%
\pgfusepath{fill}%
\end{pgfscope}%
\begin{pgfscope}%
\pgfpathrectangle{\pgfqpoint{1.020000in}{0.880000in}}{\pgfqpoint{6.160000in}{6.160000in}}%
\pgfusepath{clip}%
\pgfsetbuttcap%
\pgfsetroundjoin%
\definecolor{currentfill}{rgb}{0.548876,0.685104,0.994379}%
\pgfsetfillcolor{currentfill}%
\pgfsetlinewidth{0.000000pt}%
\definecolor{currentstroke}{rgb}{0.000000,0.000000,0.000000}%
\pgfsetstrokecolor{currentstroke}%
\pgfsetdash{}{0pt}%
\pgfpathmoveto{\pgfqpoint{4.302629in}{3.633858in}}%
\pgfpathlineto{\pgfqpoint{4.312917in}{3.612447in}}%
\pgfpathlineto{\pgfqpoint{4.323222in}{3.590987in}}%
\pgfpathlineto{\pgfqpoint{4.355807in}{3.576632in}}%
\pgfpathlineto{\pgfqpoint{4.388362in}{3.562355in}}%
\pgfpathlineto{\pgfqpoint{4.378013in}{3.584116in}}%
\pgfpathlineto{\pgfqpoint{4.367681in}{3.605776in}}%
\pgfpathlineto{\pgfqpoint{4.335170in}{3.619595in}}%
\pgfpathlineto{\pgfqpoint{4.302629in}{3.633858in}}%
\pgfpathclose%
\pgfusepath{fill}%
\end{pgfscope}%
\begin{pgfscope}%
\pgfpathrectangle{\pgfqpoint{1.020000in}{0.880000in}}{\pgfqpoint{6.160000in}{6.160000in}}%
\pgfusepath{clip}%
\pgfsetbuttcap%
\pgfsetroundjoin%
\definecolor{currentfill}{rgb}{0.861054,0.362916,0.290628}%
\pgfsetfillcolor{currentfill}%
\pgfsetlinewidth{0.000000pt}%
\definecolor{currentstroke}{rgb}{0.000000,0.000000,0.000000}%
\pgfsetstrokecolor{currentstroke}%
\pgfsetdash{}{0pt}%
\pgfpathmoveto{\pgfqpoint{2.950064in}{5.072868in}}%
\pgfpathlineto{\pgfqpoint{2.958626in}{5.105947in}}%
\pgfpathlineto{\pgfqpoint{2.967238in}{5.136712in}}%
\pgfpathlineto{\pgfqpoint{3.000476in}{5.120096in}}%
\pgfpathlineto{\pgfqpoint{3.033727in}{5.099292in}}%
\pgfpathlineto{\pgfqpoint{3.025027in}{5.069610in}}%
\pgfpathlineto{\pgfqpoint{3.016374in}{5.037670in}}%
\pgfpathlineto{\pgfqpoint{2.983214in}{5.057171in}}%
\pgfpathlineto{\pgfqpoint{2.950064in}{5.072868in}}%
\pgfpathclose%
\pgfusepath{fill}%
\end{pgfscope}%
\begin{pgfscope}%
\pgfpathrectangle{\pgfqpoint{1.020000in}{0.880000in}}{\pgfqpoint{6.160000in}{6.160000in}}%
\pgfusepath{clip}%
\pgfsetbuttcap%
\pgfsetroundjoin%
\definecolor{currentfill}{rgb}{0.884643,0.410017,0.322507}%
\pgfsetfillcolor{currentfill}%
\pgfsetlinewidth{0.000000pt}%
\definecolor{currentstroke}{rgb}{0.000000,0.000000,0.000000}%
\pgfsetstrokecolor{currentstroke}%
\pgfsetdash{}{0pt}%
\pgfpathmoveto{\pgfqpoint{2.669291in}{4.992476in}}%
\pgfpathlineto{\pgfqpoint{2.677486in}{5.025066in}}%
\pgfpathlineto{\pgfqpoint{2.685721in}{5.056355in}}%
\pgfpathlineto{\pgfqpoint{2.718656in}{5.071572in}}%
\pgfpathlineto{\pgfqpoint{2.751625in}{5.083344in}}%
\pgfpathlineto{\pgfqpoint{2.743334in}{5.049937in}}%
\pgfpathlineto{\pgfqpoint{2.735085in}{5.015196in}}%
\pgfpathlineto{\pgfqpoint{2.702174in}{5.005374in}}%
\pgfpathlineto{\pgfqpoint{2.669291in}{4.992476in}}%
\pgfpathclose%
\pgfusepath{fill}%
\end{pgfscope}%
\begin{pgfscope}%
\pgfpathrectangle{\pgfqpoint{1.020000in}{0.880000in}}{\pgfqpoint{6.160000in}{6.160000in}}%
\pgfusepath{clip}%
\pgfsetbuttcap%
\pgfsetroundjoin%
\definecolor{currentfill}{rgb}{0.248091,0.326013,0.777669}%
\pgfsetfillcolor{currentfill}%
\pgfsetlinewidth{0.000000pt}%
\definecolor{currentstroke}{rgb}{0.000000,0.000000,0.000000}%
\pgfsetstrokecolor{currentstroke}%
\pgfsetdash{}{0pt}%
\pgfpathmoveto{\pgfqpoint{5.141896in}{3.032732in}}%
\pgfpathlineto{\pgfqpoint{5.153039in}{3.028695in}}%
\pgfpathlineto{\pgfqpoint{5.164244in}{3.028946in}}%
\pgfpathlineto{\pgfqpoint{5.196735in}{3.045107in}}%
\pgfpathlineto{\pgfqpoint{5.229248in}{3.064633in}}%
\pgfpathlineto{\pgfqpoint{5.217998in}{3.066392in}}%
\pgfpathlineto{\pgfqpoint{5.206808in}{3.071901in}}%
\pgfpathlineto{\pgfqpoint{5.174339in}{3.050470in}}%
\pgfpathlineto{\pgfqpoint{5.141896in}{3.032732in}}%
\pgfpathclose%
\pgfusepath{fill}%
\end{pgfscope}%
\begin{pgfscope}%
\pgfpathrectangle{\pgfqpoint{1.020000in}{0.880000in}}{\pgfqpoint{6.160000in}{6.160000in}}%
\pgfusepath{clip}%
\pgfsetbuttcap%
\pgfsetroundjoin%
\definecolor{currentfill}{rgb}{0.299441,0.400248,0.839842}%
\pgfsetfillcolor{currentfill}%
\pgfsetlinewidth{0.000000pt}%
\definecolor{currentstroke}{rgb}{0.000000,0.000000,0.000000}%
\pgfsetstrokecolor{currentstroke}%
\pgfsetdash{}{0pt}%
\pgfpathmoveto{\pgfqpoint{4.775576in}{3.198466in}}%
\pgfpathlineto{\pgfqpoint{4.786254in}{3.175504in}}%
\pgfpathlineto{\pgfqpoint{4.796971in}{3.156375in}}%
\pgfpathlineto{\pgfqpoint{4.829265in}{3.127937in}}%
\pgfpathlineto{\pgfqpoint{4.861534in}{3.101541in}}%
\pgfpathlineto{\pgfqpoint{4.850776in}{3.121055in}}%
\pgfpathlineto{\pgfqpoint{4.840066in}{3.145197in}}%
\pgfpathlineto{\pgfqpoint{4.807834in}{3.170841in}}%
\pgfpathlineto{\pgfqpoint{4.775576in}{3.198466in}}%
\pgfpathclose%
\pgfusepath{fill}%
\end{pgfscope}%
\begin{pgfscope}%
\pgfpathrectangle{\pgfqpoint{1.020000in}{0.880000in}}{\pgfqpoint{6.160000in}{6.160000in}}%
\pgfusepath{clip}%
\pgfsetbuttcap%
\pgfsetroundjoin%
\definecolor{currentfill}{rgb}{0.613933,0.739923,0.999142}%
\pgfsetfillcolor{currentfill}%
\pgfsetlinewidth{0.000000pt}%
\definecolor{currentstroke}{rgb}{0.000000,0.000000,0.000000}%
\pgfsetstrokecolor{currentstroke}%
\pgfsetdash{}{0pt}%
\pgfpathmoveto{\pgfqpoint{4.151824in}{3.744813in}}%
\pgfpathlineto{\pgfqpoint{4.161989in}{3.722323in}}%
\pgfpathlineto{\pgfqpoint{4.172171in}{3.699638in}}%
\pgfpathlineto{\pgfqpoint{4.204832in}{3.681484in}}%
\pgfpathlineto{\pgfqpoint{4.237461in}{3.664594in}}%
\pgfpathlineto{\pgfqpoint{4.227235in}{3.686309in}}%
\pgfpathlineto{\pgfqpoint{4.217026in}{3.707832in}}%
\pgfpathlineto{\pgfqpoint{4.184441in}{3.725572in}}%
\pgfpathlineto{\pgfqpoint{4.151824in}{3.744813in}}%
\pgfpathclose%
\pgfusepath{fill}%
\end{pgfscope}%
\begin{pgfscope}%
\pgfpathrectangle{\pgfqpoint{1.020000in}{0.880000in}}{\pgfqpoint{6.160000in}{6.160000in}}%
\pgfusepath{clip}%
\pgfsetbuttcap%
\pgfsetroundjoin%
\definecolor{currentfill}{rgb}{0.748682,0.827679,0.963334}%
\pgfsetfillcolor{currentfill}%
\pgfsetlinewidth{0.000000pt}%
\definecolor{currentstroke}{rgb}{0.000000,0.000000,0.000000}%
\pgfsetstrokecolor{currentstroke}%
\pgfsetdash{}{0pt}%
\pgfpathmoveto{\pgfqpoint{3.915329in}{4.003350in}}%
\pgfpathlineto{\pgfqpoint{3.925309in}{3.978432in}}%
\pgfpathlineto{\pgfqpoint{3.935307in}{3.952565in}}%
\pgfpathlineto{\pgfqpoint{3.968117in}{3.919888in}}%
\pgfpathlineto{\pgfqpoint{4.000883in}{3.889429in}}%
\pgfpathlineto{\pgfqpoint{3.990849in}{3.913501in}}%
\pgfpathlineto{\pgfqpoint{3.980833in}{3.936802in}}%
\pgfpathlineto{\pgfqpoint{3.948104in}{3.968897in}}%
\pgfpathlineto{\pgfqpoint{3.915329in}{4.003350in}}%
\pgfpathclose%
\pgfusepath{fill}%
\end{pgfscope}%
\begin{pgfscope}%
\pgfpathrectangle{\pgfqpoint{1.020000in}{0.880000in}}{\pgfqpoint{6.160000in}{6.160000in}}%
\pgfusepath{clip}%
\pgfsetbuttcap%
\pgfsetroundjoin%
\definecolor{currentfill}{rgb}{0.884643,0.410017,0.322507}%
\pgfsetfillcolor{currentfill}%
\pgfsetlinewidth{0.000000pt}%
\definecolor{currentstroke}{rgb}{0.000000,0.000000,0.000000}%
\pgfsetstrokecolor{currentstroke}%
\pgfsetdash{}{0pt}%
\pgfpathmoveto{\pgfqpoint{3.100245in}{5.045950in}}%
\pgfpathlineto{\pgfqpoint{3.109087in}{5.071210in}}%
\pgfpathlineto{\pgfqpoint{3.117982in}{5.093399in}}%
\pgfpathlineto{\pgfqpoint{3.151327in}{5.059326in}}%
\pgfpathlineto{\pgfqpoint{3.184661in}{5.021853in}}%
\pgfpathlineto{\pgfqpoint{3.175678in}{5.001772in}}%
\pgfpathlineto{\pgfqpoint{3.166744in}{4.978789in}}%
\pgfpathlineto{\pgfqpoint{3.133500in}{5.013941in}}%
\pgfpathlineto{\pgfqpoint{3.100245in}{5.045950in}}%
\pgfpathclose%
\pgfusepath{fill}%
\end{pgfscope}%
\begin{pgfscope}%
\pgfpathrectangle{\pgfqpoint{1.020000in}{0.880000in}}{\pgfqpoint{6.160000in}{6.160000in}}%
\pgfusepath{clip}%
\pgfsetbuttcap%
\pgfsetroundjoin%
\definecolor{currentfill}{rgb}{0.441123,0.576532,0.954545}%
\pgfsetfillcolor{currentfill}%
\pgfsetlinewidth{0.000000pt}%
\definecolor{currentstroke}{rgb}{0.000000,0.000000,0.000000}%
\pgfsetstrokecolor{currentstroke}%
\pgfsetdash{}{0pt}%
\pgfpathmoveto{\pgfqpoint{4.539182in}{3.451781in}}%
\pgfpathlineto{\pgfqpoint{4.549662in}{3.427555in}}%
\pgfpathlineto{\pgfqpoint{4.560162in}{3.404098in}}%
\pgfpathlineto{\pgfqpoint{4.592625in}{3.382028in}}%
\pgfpathlineto{\pgfqpoint{4.625046in}{3.358372in}}%
\pgfpathlineto{\pgfqpoint{4.614514in}{3.384222in}}%
\pgfpathlineto{\pgfqpoint{4.604003in}{3.411140in}}%
\pgfpathlineto{\pgfqpoint{4.571611in}{3.432072in}}%
\pgfpathlineto{\pgfqpoint{4.539182in}{3.451781in}}%
\pgfpathclose%
\pgfusepath{fill}%
\end{pgfscope}%
\begin{pgfscope}%
\pgfpathrectangle{\pgfqpoint{1.020000in}{0.880000in}}{\pgfqpoint{6.160000in}{6.160000in}}%
\pgfusepath{clip}%
\pgfsetbuttcap%
\pgfsetroundjoin%
\definecolor{currentfill}{rgb}{0.363461,0.484784,0.901019}%
\pgfsetfillcolor{currentfill}%
\pgfsetlinewidth{0.000000pt}%
\definecolor{currentstroke}{rgb}{0.000000,0.000000,0.000000}%
\pgfsetstrokecolor{currentstroke}%
\pgfsetdash{}{0pt}%
\pgfpathmoveto{\pgfqpoint{5.554028in}{3.271927in}}%
\pgfpathlineto{\pgfqpoint{5.565474in}{3.259314in}}%
\pgfpathlineto{\pgfqpoint{5.576947in}{3.247133in}}%
\pgfpathlineto{\pgfqpoint{5.609291in}{3.257067in}}%
\pgfpathlineto{\pgfqpoint{5.641599in}{3.265340in}}%
\pgfpathlineto{\pgfqpoint{5.630092in}{3.278998in}}%
\pgfpathlineto{\pgfqpoint{5.618610in}{3.292907in}}%
\pgfpathlineto{\pgfqpoint{5.586339in}{3.283398in}}%
\pgfpathlineto{\pgfqpoint{5.554028in}{3.271927in}}%
\pgfpathclose%
\pgfusepath{fill}%
\end{pgfscope}%
\begin{pgfscope}%
\pgfpathrectangle{\pgfqpoint{1.020000in}{0.880000in}}{\pgfqpoint{6.160000in}{6.160000in}}%
\pgfusepath{clip}%
\pgfsetbuttcap%
\pgfsetroundjoin%
\definecolor{currentfill}{rgb}{0.895885,0.433075,0.338681}%
\pgfsetfillcolor{currentfill}%
\pgfsetlinewidth{0.000000pt}%
\definecolor{currentstroke}{rgb}{0.000000,0.000000,0.000000}%
\pgfsetstrokecolor{currentstroke}%
\pgfsetdash{}{0pt}%
\pgfpathmoveto{\pgfqpoint{2.603605in}{4.958394in}}%
\pgfpathlineto{\pgfqpoint{2.611757in}{4.988141in}}%
\pgfpathlineto{\pgfqpoint{2.619949in}{5.016635in}}%
\pgfpathlineto{\pgfqpoint{2.652819in}{5.037947in}}%
\pgfpathlineto{\pgfqpoint{2.685721in}{5.056355in}}%
\pgfpathlineto{\pgfqpoint{2.677486in}{5.025066in}}%
\pgfpathlineto{\pgfqpoint{2.669291in}{4.992476in}}%
\pgfpathlineto{\pgfqpoint{2.636435in}{4.976731in}}%
\pgfpathlineto{\pgfqpoint{2.603605in}{4.958394in}}%
\pgfpathclose%
\pgfusepath{fill}%
\end{pgfscope}%
\begin{pgfscope}%
\pgfpathrectangle{\pgfqpoint{1.020000in}{0.880000in}}{\pgfqpoint{6.160000in}{6.160000in}}%
\pgfusepath{clip}%
\pgfsetbuttcap%
\pgfsetroundjoin%
\definecolor{currentfill}{rgb}{0.949454,0.572388,0.453443}%
\pgfsetfillcolor{currentfill}%
\pgfsetlinewidth{0.000000pt}%
\definecolor{currentstroke}{rgb}{0.000000,0.000000,0.000000}%
\pgfsetstrokecolor{currentstroke}%
\pgfsetdash{}{0pt}%
\pgfpathmoveto{\pgfqpoint{3.336218in}{4.866440in}}%
\pgfpathlineto{\pgfqpoint{3.345511in}{4.871726in}}%
\pgfpathlineto{\pgfqpoint{3.354853in}{4.873343in}}%
\pgfpathlineto{\pgfqpoint{3.388192in}{4.820397in}}%
\pgfpathlineto{\pgfqpoint{3.421485in}{4.766454in}}%
\pgfpathlineto{\pgfqpoint{3.412078in}{4.766628in}}%
\pgfpathlineto{\pgfqpoint{3.402714in}{4.763497in}}%
\pgfpathlineto{\pgfqpoint{3.369489in}{4.815444in}}%
\pgfpathlineto{\pgfqpoint{3.336218in}{4.866440in}}%
\pgfpathclose%
\pgfusepath{fill}%
\end{pgfscope}%
\begin{pgfscope}%
\pgfpathrectangle{\pgfqpoint{1.020000in}{0.880000in}}{\pgfqpoint{6.160000in}{6.160000in}}%
\pgfusepath{clip}%
\pgfsetbuttcap%
\pgfsetroundjoin%
\definecolor{currentfill}{rgb}{0.234377,0.305542,0.759680}%
\pgfsetfillcolor{currentfill}%
\pgfsetlinewidth{0.000000pt}%
\definecolor{currentstroke}{rgb}{0.000000,0.000000,0.000000}%
\pgfsetstrokecolor{currentstroke}%
\pgfsetdash{}{0pt}%
\pgfpathmoveto{\pgfqpoint{5.077097in}{3.011458in}}%
\pgfpathlineto{\pgfqpoint{5.088188in}{3.008212in}}%
\pgfpathlineto{\pgfqpoint{5.099340in}{3.009557in}}%
\pgfpathlineto{\pgfqpoint{5.131779in}{3.016899in}}%
\pgfpathlineto{\pgfqpoint{5.164244in}{3.028946in}}%
\pgfpathlineto{\pgfqpoint{5.153039in}{3.028695in}}%
\pgfpathlineto{\pgfqpoint{5.141896in}{3.032732in}}%
\pgfpathlineto{\pgfqpoint{5.109481in}{3.019513in}}%
\pgfpathlineto{\pgfqpoint{5.077097in}{3.011458in}}%
\pgfpathclose%
\pgfusepath{fill}%
\end{pgfscope}%
\begin{pgfscope}%
\pgfpathrectangle{\pgfqpoint{1.020000in}{0.880000in}}{\pgfqpoint{6.160000in}{6.160000in}}%
\pgfusepath{clip}%
\pgfsetbuttcap%
\pgfsetroundjoin%
\definecolor{currentfill}{rgb}{0.693321,0.796314,0.986308}%
\pgfsetfillcolor{currentfill}%
\pgfsetlinewidth{0.000000pt}%
\definecolor{currentstroke}{rgb}{0.000000,0.000000,0.000000}%
\pgfsetstrokecolor{currentstroke}%
\pgfsetdash{}{0pt}%
\pgfpathmoveto{\pgfqpoint{4.000883in}{3.889429in}}%
\pgfpathlineto{\pgfqpoint{4.010934in}{3.864697in}}%
\pgfpathlineto{\pgfqpoint{4.021002in}{3.839425in}}%
\pgfpathlineto{\pgfqpoint{4.053764in}{3.812914in}}%
\pgfpathlineto{\pgfqpoint{4.086487in}{3.788366in}}%
\pgfpathlineto{\pgfqpoint{4.076380in}{3.811919in}}%
\pgfpathlineto{\pgfqpoint{4.066290in}{3.835031in}}%
\pgfpathlineto{\pgfqpoint{4.033607in}{3.861161in}}%
\pgfpathlineto{\pgfqpoint{4.000883in}{3.889429in}}%
\pgfpathclose%
\pgfusepath{fill}%
\end{pgfscope}%
\begin{pgfscope}%
\pgfpathrectangle{\pgfqpoint{1.020000in}{0.880000in}}{\pgfqpoint{6.160000in}{6.160000in}}%
\pgfusepath{clip}%
\pgfsetbuttcap%
\pgfsetroundjoin%
\definecolor{currentfill}{rgb}{0.908908,0.462433,0.360950}%
\pgfsetfillcolor{currentfill}%
\pgfsetlinewidth{0.000000pt}%
\definecolor{currentstroke}{rgb}{0.000000,0.000000,0.000000}%
\pgfsetstrokecolor{currentstroke}%
\pgfsetdash{}{0pt}%
\pgfpathmoveto{\pgfqpoint{2.538016in}{4.915102in}}%
\pgfpathlineto{\pgfqpoint{2.546136in}{4.941444in}}%
\pgfpathlineto{\pgfqpoint{2.554295in}{4.966591in}}%
\pgfpathlineto{\pgfqpoint{2.587108in}{4.992737in}}%
\pgfpathlineto{\pgfqpoint{2.619949in}{5.016635in}}%
\pgfpathlineto{\pgfqpoint{2.611757in}{4.988141in}}%
\pgfpathlineto{\pgfqpoint{2.603605in}{4.958394in}}%
\pgfpathlineto{\pgfqpoint{2.570800in}{4.937751in}}%
\pgfpathlineto{\pgfqpoint{2.538016in}{4.915102in}}%
\pgfpathclose%
\pgfusepath{fill}%
\end{pgfscope}%
\begin{pgfscope}%
\pgfpathrectangle{\pgfqpoint{1.020000in}{0.880000in}}{\pgfqpoint{6.160000in}{6.160000in}}%
\pgfusepath{clip}%
\pgfsetbuttcap%
\pgfsetroundjoin%
\definecolor{currentfill}{rgb}{0.510824,0.649397,0.985079}%
\pgfsetfillcolor{currentfill}%
\pgfsetlinewidth{0.000000pt}%
\definecolor{currentstroke}{rgb}{0.000000,0.000000,0.000000}%
\pgfsetstrokecolor{currentstroke}%
\pgfsetdash{}{0pt}%
\pgfpathmoveto{\pgfqpoint{4.388362in}{3.562355in}}%
\pgfpathlineto{\pgfqpoint{4.398728in}{3.540614in}}%
\pgfpathlineto{\pgfqpoint{4.409112in}{3.519025in}}%
\pgfpathlineto{\pgfqpoint{4.441680in}{3.503661in}}%
\pgfpathlineto{\pgfqpoint{4.474215in}{3.487467in}}%
\pgfpathlineto{\pgfqpoint{4.463790in}{3.510179in}}%
\pgfpathlineto{\pgfqpoint{4.453382in}{3.533059in}}%
\pgfpathlineto{\pgfqpoint{4.420887in}{3.547908in}}%
\pgfpathlineto{\pgfqpoint{4.388362in}{3.562355in}}%
\pgfpathclose%
\pgfusepath{fill}%
\end{pgfscope}%
\begin{pgfscope}%
\pgfpathrectangle{\pgfqpoint{1.020000in}{0.880000in}}{\pgfqpoint{6.160000in}{6.160000in}}%
\pgfusepath{clip}%
\pgfsetbuttcap%
\pgfsetroundjoin%
\definecolor{currentfill}{rgb}{0.368507,0.491141,0.905243}%
\pgfsetfillcolor{currentfill}%
\pgfsetlinewidth{0.000000pt}%
\definecolor{currentstroke}{rgb}{0.000000,0.000000,0.000000}%
\pgfsetstrokecolor{currentstroke}%
\pgfsetdash{}{0pt}%
\pgfpathmoveto{\pgfqpoint{5.770481in}{3.285688in}}%
\pgfpathlineto{\pgfqpoint{5.782092in}{3.270663in}}%
\pgfpathlineto{\pgfqpoint{5.793724in}{3.255657in}}%
\pgfpathlineto{\pgfqpoint{5.825913in}{3.258232in}}%
\pgfpathlineto{\pgfqpoint{5.814258in}{3.273398in}}%
\pgfpathlineto{\pgfqpoint{5.802624in}{3.288577in}}%
\pgfpathlineto{\pgfqpoint{5.770481in}{3.285688in}}%
\pgfpathclose%
\pgfusepath{fill}%
\end{pgfscope}%
\begin{pgfscope}%
\pgfpathrectangle{\pgfqpoint{1.020000in}{0.880000in}}{\pgfqpoint{6.160000in}{6.160000in}}%
\pgfusepath{clip}%
\pgfsetbuttcap%
\pgfsetroundjoin%
\definecolor{currentfill}{rgb}{0.353369,0.472069,0.892570}%
\pgfsetfillcolor{currentfill}%
\pgfsetlinewidth{0.000000pt}%
\definecolor{currentstroke}{rgb}{0.000000,0.000000,0.000000}%
\pgfsetstrokecolor{currentstroke}%
\pgfsetdash{}{0pt}%
\pgfpathmoveto{\pgfqpoint{5.489285in}{3.242490in}}%
\pgfpathlineto{\pgfqpoint{5.500700in}{3.231782in}}%
\pgfpathlineto{\pgfqpoint{5.512144in}{3.221766in}}%
\pgfpathlineto{\pgfqpoint{5.544564in}{3.235394in}}%
\pgfpathlineto{\pgfqpoint{5.576947in}{3.247133in}}%
\pgfpathlineto{\pgfqpoint{5.565474in}{3.259314in}}%
\pgfpathlineto{\pgfqpoint{5.554028in}{3.271927in}}%
\pgfpathlineto{\pgfqpoint{5.521676in}{3.258324in}}%
\pgfpathlineto{\pgfqpoint{5.489285in}{3.242490in}}%
\pgfpathclose%
\pgfusepath{fill}%
\end{pgfscope}%
\begin{pgfscope}%
\pgfpathrectangle{\pgfqpoint{1.020000in}{0.880000in}}{\pgfqpoint{6.160000in}{6.160000in}}%
\pgfusepath{clip}%
\pgfsetbuttcap%
\pgfsetroundjoin%
\definecolor{currentfill}{rgb}{0.257234,0.339661,0.789661}%
\pgfsetfillcolor{currentfill}%
\pgfsetlinewidth{0.000000pt}%
\definecolor{currentstroke}{rgb}{0.000000,0.000000,0.000000}%
\pgfsetstrokecolor{currentstroke}%
\pgfsetdash{}{0pt}%
\pgfpathmoveto{\pgfqpoint{4.861534in}{3.101541in}}%
\pgfpathlineto{\pgfqpoint{4.872341in}{3.086792in}}%
\pgfpathlineto{\pgfqpoint{4.883198in}{3.076710in}}%
\pgfpathlineto{\pgfqpoint{4.915503in}{3.054069in}}%
\pgfpathlineto{\pgfqpoint{4.947801in}{3.035285in}}%
\pgfpathlineto{\pgfqpoint{4.936887in}{3.044386in}}%
\pgfpathlineto{\pgfqpoint{4.926031in}{3.058777in}}%
\pgfpathlineto{\pgfqpoint{4.893786in}{3.078176in}}%
\pgfpathlineto{\pgfqpoint{4.861534in}{3.101541in}}%
\pgfpathclose%
\pgfusepath{fill}%
\end{pgfscope}%
\begin{pgfscope}%
\pgfpathrectangle{\pgfqpoint{1.020000in}{0.880000in}}{\pgfqpoint{6.160000in}{6.160000in}}%
\pgfusepath{clip}%
\pgfsetbuttcap%
\pgfsetroundjoin%
\definecolor{currentfill}{rgb}{0.924409,0.498590,0.389059}%
\pgfsetfillcolor{currentfill}%
\pgfsetlinewidth{0.000000pt}%
\definecolor{currentstroke}{rgb}{0.000000,0.000000,0.000000}%
\pgfsetstrokecolor{currentstroke}%
\pgfsetdash{}{0pt}%
\pgfpathmoveto{\pgfqpoint{2.472502in}{4.865072in}}%
\pgfpathlineto{\pgfqpoint{2.480599in}{4.887591in}}%
\pgfpathlineto{\pgfqpoint{2.488735in}{4.908984in}}%
\pgfpathlineto{\pgfqpoint{2.521505in}{4.938551in}}%
\pgfpathlineto{\pgfqpoint{2.554295in}{4.966591in}}%
\pgfpathlineto{\pgfqpoint{2.546136in}{4.941444in}}%
\pgfpathlineto{\pgfqpoint{2.538016in}{4.915102in}}%
\pgfpathlineto{\pgfqpoint{2.505252in}{4.890767in}}%
\pgfpathlineto{\pgfqpoint{2.472502in}{4.865072in}}%
\pgfpathclose%
\pgfusepath{fill}%
\end{pgfscope}%
\begin{pgfscope}%
\pgfpathrectangle{\pgfqpoint{1.020000in}{0.880000in}}{\pgfqpoint{6.160000in}{6.160000in}}%
\pgfusepath{clip}%
\pgfsetbuttcap%
\pgfsetroundjoin%
\definecolor{currentfill}{rgb}{0.966017,0.646130,0.525890}%
\pgfsetfillcolor{currentfill}%
\pgfsetlinewidth{0.000000pt}%
\definecolor{currentstroke}{rgb}{0.000000,0.000000,0.000000}%
\pgfsetstrokecolor{currentstroke}%
\pgfsetdash{}{0pt}%
\pgfpathmoveto{\pgfqpoint{3.421485in}{4.766454in}}%
\pgfpathlineto{\pgfqpoint{3.430936in}{4.762786in}}%
\pgfpathlineto{\pgfqpoint{3.440429in}{4.755477in}}%
\pgfpathlineto{\pgfqpoint{3.473729in}{4.699769in}}%
\pgfpathlineto{\pgfqpoint{3.506975in}{4.643966in}}%
\pgfpathlineto{\pgfqpoint{3.497427in}{4.652301in}}%
\pgfpathlineto{\pgfqpoint{3.487919in}{4.657410in}}%
\pgfpathlineto{\pgfqpoint{3.454729in}{4.711977in}}%
\pgfpathlineto{\pgfqpoint{3.421485in}{4.766454in}}%
\pgfpathclose%
\pgfusepath{fill}%
\end{pgfscope}%
\begin{pgfscope}%
\pgfpathrectangle{\pgfqpoint{1.020000in}{0.880000in}}{\pgfqpoint{6.160000in}{6.160000in}}%
\pgfusepath{clip}%
\pgfsetbuttcap%
\pgfsetroundjoin%
\definecolor{currentfill}{rgb}{0.388852,0.516298,0.921373}%
\pgfsetfillcolor{currentfill}%
\pgfsetlinewidth{0.000000pt}%
\definecolor{currentstroke}{rgb}{0.000000,0.000000,0.000000}%
\pgfsetstrokecolor{currentstroke}%
\pgfsetdash{}{0pt}%
\pgfpathmoveto{\pgfqpoint{4.625046in}{3.358372in}}%
\pgfpathlineto{\pgfqpoint{4.635601in}{3.333969in}}%
\pgfpathlineto{\pgfqpoint{4.646181in}{3.311349in}}%
\pgfpathlineto{\pgfqpoint{4.678592in}{3.284245in}}%
\pgfpathlineto{\pgfqpoint{4.710960in}{3.255975in}}%
\pgfpathlineto{\pgfqpoint{4.700348in}{3.280448in}}%
\pgfpathlineto{\pgfqpoint{4.689765in}{3.307325in}}%
\pgfpathlineto{\pgfqpoint{4.657425in}{3.333346in}}%
\pgfpathlineto{\pgfqpoint{4.625046in}{3.358372in}}%
\pgfpathclose%
\pgfusepath{fill}%
\end{pgfscope}%
\begin{pgfscope}%
\pgfpathrectangle{\pgfqpoint{1.020000in}{0.880000in}}{\pgfqpoint{6.160000in}{6.160000in}}%
\pgfusepath{clip}%
\pgfsetbuttcap%
\pgfsetroundjoin%
\definecolor{currentfill}{rgb}{0.570616,0.704109,0.997195}%
\pgfsetfillcolor{currentfill}%
\pgfsetlinewidth{0.000000pt}%
\definecolor{currentstroke}{rgb}{0.000000,0.000000,0.000000}%
\pgfsetstrokecolor{currentstroke}%
\pgfsetdash{}{0pt}%
\pgfpathmoveto{\pgfqpoint{4.237461in}{3.664594in}}%
\pgfpathlineto{\pgfqpoint{4.247704in}{3.642773in}}%
\pgfpathlineto{\pgfqpoint{4.257964in}{3.620934in}}%
\pgfpathlineto{\pgfqpoint{4.290608in}{3.605673in}}%
\pgfpathlineto{\pgfqpoint{4.323222in}{3.590987in}}%
\pgfpathlineto{\pgfqpoint{4.312917in}{3.612447in}}%
\pgfpathlineto{\pgfqpoint{4.302629in}{3.633858in}}%
\pgfpathlineto{\pgfqpoint{4.270060in}{3.648786in}}%
\pgfpathlineto{\pgfqpoint{4.237461in}{3.664594in}}%
\pgfpathclose%
\pgfusepath{fill}%
\end{pgfscope}%
\begin{pgfscope}%
\pgfpathrectangle{\pgfqpoint{1.020000in}{0.880000in}}{\pgfqpoint{6.160000in}{6.160000in}}%
\pgfusepath{clip}%
\pgfsetbuttcap%
\pgfsetroundjoin%
\definecolor{currentfill}{rgb}{0.848040,0.338280,0.275206}%
\pgfsetfillcolor{currentfill}%
\pgfsetlinewidth{0.000000pt}%
\definecolor{currentstroke}{rgb}{0.000000,0.000000,0.000000}%
\pgfsetstrokecolor{currentstroke}%
\pgfsetdash{}{0pt}%
\pgfpathmoveto{\pgfqpoint{2.883815in}{5.092372in}}%
\pgfpathlineto{\pgfqpoint{2.892293in}{5.125783in}}%
\pgfpathlineto{\pgfqpoint{2.900823in}{5.156855in}}%
\pgfpathlineto{\pgfqpoint{2.934019in}{5.148994in}}%
\pgfpathlineto{\pgfqpoint{2.967238in}{5.136712in}}%
\pgfpathlineto{\pgfqpoint{2.958626in}{5.105947in}}%
\pgfpathlineto{\pgfqpoint{2.950064in}{5.072868in}}%
\pgfpathlineto{\pgfqpoint{2.916930in}{5.084629in}}%
\pgfpathlineto{\pgfqpoint{2.883815in}{5.092372in}}%
\pgfpathclose%
\pgfusepath{fill}%
\end{pgfscope}%
\begin{pgfscope}%
\pgfpathrectangle{\pgfqpoint{1.020000in}{0.880000in}}{\pgfqpoint{6.160000in}{6.160000in}}%
\pgfusepath{clip}%
\pgfsetbuttcap%
\pgfsetroundjoin%
\definecolor{currentfill}{rgb}{0.939254,0.539581,0.423900}%
\pgfsetfillcolor{currentfill}%
\pgfsetlinewidth{0.000000pt}%
\definecolor{currentstroke}{rgb}{0.000000,0.000000,0.000000}%
\pgfsetstrokecolor{currentstroke}%
\pgfsetdash{}{0pt}%
\pgfpathmoveto{\pgfqpoint{2.407032in}{4.810913in}}%
\pgfpathlineto{\pgfqpoint{2.415116in}{4.829350in}}%
\pgfpathlineto{\pgfqpoint{2.423236in}{4.846732in}}%
\pgfpathlineto{\pgfqpoint{2.455980in}{4.878256in}}%
\pgfpathlineto{\pgfqpoint{2.488735in}{4.908984in}}%
\pgfpathlineto{\pgfqpoint{2.480599in}{4.887591in}}%
\pgfpathlineto{\pgfqpoint{2.472502in}{4.865072in}}%
\pgfpathlineto{\pgfqpoint{2.439764in}{4.838345in}}%
\pgfpathlineto{\pgfqpoint{2.407032in}{4.810913in}}%
\pgfpathclose%
\pgfusepath{fill}%
\end{pgfscope}%
\begin{pgfscope}%
\pgfpathrectangle{\pgfqpoint{1.020000in}{0.880000in}}{\pgfqpoint{6.160000in}{6.160000in}}%
\pgfusepath{clip}%
\pgfsetbuttcap%
\pgfsetroundjoin%
\definecolor{currentfill}{rgb}{0.968203,0.720844,0.612293}%
\pgfsetfillcolor{currentfill}%
\pgfsetlinewidth{0.000000pt}%
\definecolor{currentstroke}{rgb}{0.000000,0.000000,0.000000}%
\pgfsetstrokecolor{currentstroke}%
\pgfsetdash{}{0pt}%
\pgfpathmoveto{\pgfqpoint{3.506975in}{4.643966in}}%
\pgfpathlineto{\pgfqpoint{3.516560in}{4.632314in}}%
\pgfpathlineto{\pgfqpoint{3.526183in}{4.617295in}}%
\pgfpathlineto{\pgfqpoint{3.559418in}{4.561447in}}%
\pgfpathlineto{\pgfqpoint{3.592593in}{4.506334in}}%
\pgfpathlineto{\pgfqpoint{3.582927in}{4.521513in}}%
\pgfpathlineto{\pgfqpoint{3.573294in}{4.533759in}}%
\pgfpathlineto{\pgfqpoint{3.540163in}{4.588496in}}%
\pgfpathlineto{\pgfqpoint{3.506975in}{4.643966in}}%
\pgfpathclose%
\pgfusepath{fill}%
\end{pgfscope}%
\begin{pgfscope}%
\pgfpathrectangle{\pgfqpoint{1.020000in}{0.880000in}}{\pgfqpoint{6.160000in}{6.160000in}}%
\pgfusepath{clip}%
\pgfsetbuttcap%
\pgfsetroundjoin%
\definecolor{currentfill}{rgb}{0.234377,0.305542,0.759680}%
\pgfsetfillcolor{currentfill}%
\pgfsetlinewidth{0.000000pt}%
\definecolor{currentstroke}{rgb}{0.000000,0.000000,0.000000}%
\pgfsetstrokecolor{currentstroke}%
\pgfsetdash{}{0pt}%
\pgfpathmoveto{\pgfqpoint{5.012412in}{3.012253in}}%
\pgfpathlineto{\pgfqpoint{5.023444in}{3.008960in}}%
\pgfpathlineto{\pgfqpoint{5.034536in}{3.010274in}}%
\pgfpathlineto{\pgfqpoint{5.066927in}{3.007300in}}%
\pgfpathlineto{\pgfqpoint{5.099340in}{3.009557in}}%
\pgfpathlineto{\pgfqpoint{5.088188in}{3.008212in}}%
\pgfpathlineto{\pgfqpoint{5.077097in}{3.011458in}}%
\pgfpathlineto{\pgfqpoint{5.044742in}{3.008986in}}%
\pgfpathlineto{\pgfqpoint{5.012412in}{3.012253in}}%
\pgfpathclose%
\pgfusepath{fill}%
\end{pgfscope}%
\begin{pgfscope}%
\pgfpathrectangle{\pgfqpoint{1.020000in}{0.880000in}}{\pgfqpoint{6.160000in}{6.160000in}}%
\pgfusepath{clip}%
\pgfsetbuttcap%
\pgfsetroundjoin%
\definecolor{currentfill}{rgb}{0.953054,0.585211,0.465373}%
\pgfsetfillcolor{currentfill}%
\pgfsetlinewidth{0.000000pt}%
\definecolor{currentstroke}{rgb}{0.000000,0.000000,0.000000}%
\pgfsetstrokecolor{currentstroke}%
\pgfsetdash{}{0pt}%
\pgfpathmoveto{\pgfqpoint{2.341571in}{4.755198in}}%
\pgfpathlineto{\pgfqpoint{2.349647in}{4.769442in}}%
\pgfpathlineto{\pgfqpoint{2.357759in}{4.782708in}}%
\pgfpathlineto{\pgfqpoint{2.390498in}{4.814769in}}%
\pgfpathlineto{\pgfqpoint{2.423236in}{4.846732in}}%
\pgfpathlineto{\pgfqpoint{2.415116in}{4.829350in}}%
\pgfpathlineto{\pgfqpoint{2.407032in}{4.810913in}}%
\pgfpathlineto{\pgfqpoint{2.374303in}{4.783095in}}%
\pgfpathlineto{\pgfqpoint{2.341571in}{4.755198in}}%
\pgfpathclose%
\pgfusepath{fill}%
\end{pgfscope}%
\begin{pgfscope}%
\pgfpathrectangle{\pgfqpoint{1.020000in}{0.880000in}}{\pgfqpoint{6.160000in}{6.160000in}}%
\pgfusepath{clip}%
\pgfsetbuttcap%
\pgfsetroundjoin%
\definecolor{currentfill}{rgb}{0.368507,0.491141,0.905243}%
\pgfsetfillcolor{currentfill}%
\pgfsetlinewidth{0.000000pt}%
\definecolor{currentstroke}{rgb}{0.000000,0.000000,0.000000}%
\pgfsetstrokecolor{currentstroke}%
\pgfsetdash{}{0pt}%
\pgfpathmoveto{\pgfqpoint{5.706106in}{3.277655in}}%
\pgfpathlineto{\pgfqpoint{5.717674in}{3.263173in}}%
\pgfpathlineto{\pgfqpoint{5.729263in}{3.248729in}}%
\pgfpathlineto{\pgfqpoint{5.761508in}{3.252537in}}%
\pgfpathlineto{\pgfqpoint{5.793724in}{3.255657in}}%
\pgfpathlineto{\pgfqpoint{5.782092in}{3.270663in}}%
\pgfpathlineto{\pgfqpoint{5.770481in}{3.285688in}}%
\pgfpathlineto{\pgfqpoint{5.738309in}{3.282107in}}%
\pgfpathlineto{\pgfqpoint{5.706106in}{3.277655in}}%
\pgfpathclose%
\pgfusepath{fill}%
\end{pgfscope}%
\begin{pgfscope}%
\pgfpathrectangle{\pgfqpoint{1.020000in}{0.880000in}}{\pgfqpoint{6.160000in}{6.160000in}}%
\pgfusepath{clip}%
\pgfsetbuttcap%
\pgfsetroundjoin%
\definecolor{currentfill}{rgb}{0.961595,0.622247,0.501551}%
\pgfsetfillcolor{currentfill}%
\pgfsetlinewidth{0.000000pt}%
\definecolor{currentstroke}{rgb}{0.000000,0.000000,0.000000}%
\pgfsetstrokecolor{currentstroke}%
\pgfsetdash{}{0pt}%
\pgfpathmoveto{\pgfqpoint{2.276078in}{4.700306in}}%
\pgfpathlineto{\pgfqpoint{2.284152in}{4.710386in}}%
\pgfpathlineto{\pgfqpoint{2.292260in}{4.719564in}}%
\pgfpathlineto{\pgfqpoint{2.325015in}{4.750872in}}%
\pgfpathlineto{\pgfqpoint{2.357759in}{4.782708in}}%
\pgfpathlineto{\pgfqpoint{2.349647in}{4.769442in}}%
\pgfpathlineto{\pgfqpoint{2.341571in}{4.755198in}}%
\pgfpathlineto{\pgfqpoint{2.308831in}{4.727512in}}%
\pgfpathlineto{\pgfqpoint{2.276078in}{4.700306in}}%
\pgfpathclose%
\pgfusepath{fill}%
\end{pgfscope}%
\begin{pgfscope}%
\pgfpathrectangle{\pgfqpoint{1.020000in}{0.880000in}}{\pgfqpoint{6.160000in}{6.160000in}}%
\pgfusepath{clip}%
\pgfsetbuttcap%
\pgfsetroundjoin%
\definecolor{currentfill}{rgb}{0.950956,0.786875,0.704761}%
\pgfsetfillcolor{currentfill}%
\pgfsetlinewidth{0.000000pt}%
\definecolor{currentstroke}{rgb}{0.000000,0.000000,0.000000}%
\pgfsetstrokecolor{currentstroke}%
\pgfsetdash{}{0pt}%
\pgfpathmoveto{\pgfqpoint{3.592593in}{4.506334in}}%
\pgfpathlineto{\pgfqpoint{3.602292in}{4.488216in}}%
\pgfpathlineto{\pgfqpoint{3.612023in}{4.467193in}}%
\pgfpathlineto{\pgfqpoint{3.645176in}{4.413655in}}%
\pgfpathlineto{\pgfqpoint{3.678268in}{4.361529in}}%
\pgfpathlineto{\pgfqpoint{3.668501in}{4.381878in}}%
\pgfpathlineto{\pgfqpoint{3.658763in}{4.399736in}}%
\pgfpathlineto{\pgfqpoint{3.625708in}{4.452321in}}%
\pgfpathlineto{\pgfqpoint{3.592593in}{4.506334in}}%
\pgfpathclose%
\pgfusepath{fill}%
\end{pgfscope}%
\begin{pgfscope}%
\pgfpathrectangle{\pgfqpoint{1.020000in}{0.880000in}}{\pgfqpoint{6.160000in}{6.160000in}}%
\pgfusepath{clip}%
\pgfsetbuttcap%
\pgfsetroundjoin%
\definecolor{currentfill}{rgb}{0.338377,0.452819,0.879317}%
\pgfsetfillcolor{currentfill}%
\pgfsetlinewidth{0.000000pt}%
\definecolor{currentstroke}{rgb}{0.000000,0.000000,0.000000}%
\pgfsetstrokecolor{currentstroke}%
\pgfsetdash{}{0pt}%
\pgfpathmoveto{\pgfqpoint{5.424393in}{3.204267in}}%
\pgfpathlineto{\pgfqpoint{5.435780in}{3.196097in}}%
\pgfpathlineto{\pgfqpoint{5.447202in}{3.188962in}}%
\pgfpathlineto{\pgfqpoint{5.479689in}{3.206253in}}%
\pgfpathlineto{\pgfqpoint{5.512144in}{3.221766in}}%
\pgfpathlineto{\pgfqpoint{5.500700in}{3.231782in}}%
\pgfpathlineto{\pgfqpoint{5.489285in}{3.242490in}}%
\pgfpathlineto{\pgfqpoint{5.456856in}{3.224429in}}%
\pgfpathlineto{\pgfqpoint{5.424393in}{3.204267in}}%
\pgfpathclose%
\pgfusepath{fill}%
\end{pgfscope}%
\begin{pgfscope}%
\pgfpathrectangle{\pgfqpoint{1.020000in}{0.880000in}}{\pgfqpoint{6.160000in}{6.160000in}}%
\pgfusepath{clip}%
\pgfsetbuttcap%
\pgfsetroundjoin%
\definecolor{currentfill}{rgb}{0.640828,0.760752,0.997846}%
\pgfsetfillcolor{currentfill}%
\pgfsetlinewidth{0.000000pt}%
\definecolor{currentstroke}{rgb}{0.000000,0.000000,0.000000}%
\pgfsetstrokecolor{currentstroke}%
\pgfsetdash{}{0pt}%
\pgfpathmoveto{\pgfqpoint{4.086487in}{3.788366in}}%
\pgfpathlineto{\pgfqpoint{4.096611in}{3.764478in}}%
\pgfpathlineto{\pgfqpoint{4.106751in}{3.740366in}}%
\pgfpathlineto{\pgfqpoint{4.139478in}{3.719219in}}%
\pgfpathlineto{\pgfqpoint{4.172171in}{3.699638in}}%
\pgfpathlineto{\pgfqpoint{4.161989in}{3.722323in}}%
\pgfpathlineto{\pgfqpoint{4.151824in}{3.744813in}}%
\pgfpathlineto{\pgfqpoint{4.119173in}{3.765701in}}%
\pgfpathlineto{\pgfqpoint{4.086487in}{3.788366in}}%
\pgfpathclose%
\pgfusepath{fill}%
\end{pgfscope}%
\begin{pgfscope}%
\pgfpathrectangle{\pgfqpoint{1.020000in}{0.880000in}}{\pgfqpoint{6.160000in}{6.160000in}}%
\pgfusepath{clip}%
\pgfsetbuttcap%
\pgfsetroundjoin%
\definecolor{currentfill}{rgb}{0.909460,0.839386,0.800331}%
\pgfsetfillcolor{currentfill}%
\pgfsetlinewidth{0.000000pt}%
\definecolor{currentstroke}{rgb}{0.000000,0.000000,0.000000}%
\pgfsetstrokecolor{currentstroke}%
\pgfsetdash{}{0pt}%
\pgfpathmoveto{\pgfqpoint{3.678268in}{4.361529in}}%
\pgfpathlineto{\pgfqpoint{3.688062in}{4.338752in}}%
\pgfpathlineto{\pgfqpoint{3.697882in}{4.313642in}}%
\pgfpathlineto{\pgfqpoint{3.730947in}{4.264423in}}%
\pgfpathlineto{\pgfqpoint{3.763954in}{4.217089in}}%
\pgfpathlineto{\pgfqpoint{3.754102in}{4.240831in}}%
\pgfpathlineto{\pgfqpoint{3.744274in}{4.262603in}}%
\pgfpathlineto{\pgfqpoint{3.711301in}{4.311099in}}%
\pgfpathlineto{\pgfqpoint{3.678268in}{4.361529in}}%
\pgfpathclose%
\pgfusepath{fill}%
\end{pgfscope}%
\begin{pgfscope}%
\pgfpathrectangle{\pgfqpoint{1.020000in}{0.880000in}}{\pgfqpoint{6.160000in}{6.160000in}}%
\pgfusepath{clip}%
\pgfsetbuttcap%
\pgfsetroundjoin%
\definecolor{currentfill}{rgb}{0.967317,0.657471,0.538160}%
\pgfsetfillcolor{currentfill}%
\pgfsetlinewidth{0.000000pt}%
\definecolor{currentstroke}{rgb}{0.000000,0.000000,0.000000}%
\pgfsetstrokecolor{currentstroke}%
\pgfsetdash{}{0pt}%
\pgfpathmoveto{\pgfqpoint{2.210516in}{4.648295in}}%
\pgfpathlineto{\pgfqpoint{2.218589in}{4.654363in}}%
\pgfpathlineto{\pgfqpoint{2.226695in}{4.659600in}}%
\pgfpathlineto{\pgfqpoint{2.259488in}{4.689058in}}%
\pgfpathlineto{\pgfqpoint{2.292260in}{4.719564in}}%
\pgfpathlineto{\pgfqpoint{2.284152in}{4.710386in}}%
\pgfpathlineto{\pgfqpoint{2.276078in}{4.700306in}}%
\pgfpathlineto{\pgfqpoint{2.243308in}{4.673826in}}%
\pgfpathlineto{\pgfqpoint{2.210516in}{4.648295in}}%
\pgfpathclose%
\pgfusepath{fill}%
\end{pgfscope}%
\begin{pgfscope}%
\pgfpathrectangle{\pgfqpoint{1.020000in}{0.880000in}}{\pgfqpoint{6.160000in}{6.160000in}}%
\pgfusepath{clip}%
\pgfsetbuttcap%
\pgfsetroundjoin%
\definecolor{currentfill}{rgb}{0.851372,0.863125,0.881064}%
\pgfsetfillcolor{currentfill}%
\pgfsetlinewidth{0.000000pt}%
\definecolor{currentstroke}{rgb}{0.000000,0.000000,0.000000}%
\pgfsetstrokecolor{currentstroke}%
\pgfsetdash{}{0pt}%
\pgfpathmoveto{\pgfqpoint{3.763954in}{4.217089in}}%
\pgfpathlineto{\pgfqpoint{3.773828in}{4.191485in}}%
\pgfpathlineto{\pgfqpoint{3.783725in}{4.164152in}}%
\pgfpathlineto{\pgfqpoint{3.816705in}{4.120646in}}%
\pgfpathlineto{\pgfqpoint{3.849631in}{4.079278in}}%
\pgfpathlineto{\pgfqpoint{3.839704in}{4.104762in}}%
\pgfpathlineto{\pgfqpoint{3.829797in}{4.128809in}}%
\pgfpathlineto{\pgfqpoint{3.796903in}{4.171834in}}%
\pgfpathlineto{\pgfqpoint{3.763954in}{4.217089in}}%
\pgfpathclose%
\pgfusepath{fill}%
\end{pgfscope}%
\begin{pgfscope}%
\pgfpathrectangle{\pgfqpoint{1.020000in}{0.880000in}}{\pgfqpoint{6.160000in}{6.160000in}}%
\pgfusepath{clip}%
\pgfsetbuttcap%
\pgfsetroundjoin%
\definecolor{currentfill}{rgb}{0.333490,0.446265,0.874452}%
\pgfsetfillcolor{currentfill}%
\pgfsetlinewidth{0.000000pt}%
\definecolor{currentstroke}{rgb}{0.000000,0.000000,0.000000}%
\pgfsetstrokecolor{currentstroke}%
\pgfsetdash{}{0pt}%
\pgfpathmoveto{\pgfqpoint{4.710960in}{3.255975in}}%
\pgfpathlineto{\pgfqpoint{4.721603in}{3.234263in}}%
\pgfpathlineto{\pgfqpoint{4.732280in}{3.215546in}}%
\pgfpathlineto{\pgfqpoint{4.764644in}{3.185878in}}%
\pgfpathlineto{\pgfqpoint{4.796971in}{3.156375in}}%
\pgfpathlineto{\pgfqpoint{4.786254in}{3.175504in}}%
\pgfpathlineto{\pgfqpoint{4.775576in}{3.198466in}}%
\pgfpathlineto{\pgfqpoint{4.743287in}{3.227133in}}%
\pgfpathlineto{\pgfqpoint{4.710960in}{3.255975in}}%
\pgfpathclose%
\pgfusepath{fill}%
\end{pgfscope}%
\begin{pgfscope}%
\pgfpathrectangle{\pgfqpoint{1.020000in}{0.880000in}}{\pgfqpoint{6.160000in}{6.160000in}}%
\pgfusepath{clip}%
\pgfsetbuttcap%
\pgfsetroundjoin%
\definecolor{currentfill}{rgb}{0.895885,0.433075,0.338681}%
\pgfsetfillcolor{currentfill}%
\pgfsetlinewidth{0.000000pt}%
\definecolor{currentstroke}{rgb}{0.000000,0.000000,0.000000}%
\pgfsetstrokecolor{currentstroke}%
\pgfsetdash{}{0pt}%
\pgfpathmoveto{\pgfqpoint{3.184661in}{5.021853in}}%
\pgfpathlineto{\pgfqpoint{3.193696in}{5.038629in}}%
\pgfpathlineto{\pgfqpoint{3.202785in}{5.051728in}}%
\pgfpathlineto{\pgfqpoint{3.236185in}{5.009125in}}%
\pgfpathlineto{\pgfqpoint{3.269561in}{4.963741in}}%
\pgfpathlineto{\pgfqpoint{3.260392in}{4.952707in}}%
\pgfpathlineto{\pgfqpoint{3.251273in}{4.938252in}}%
\pgfpathlineto{\pgfqpoint{3.217978in}{4.981359in}}%
\pgfpathlineto{\pgfqpoint{3.184661in}{5.021853in}}%
\pgfpathclose%
\pgfusepath{fill}%
\end{pgfscope}%
\begin{pgfscope}%
\pgfpathrectangle{\pgfqpoint{1.020000in}{0.880000in}}{\pgfqpoint{6.160000in}{6.160000in}}%
\pgfusepath{clip}%
\pgfsetbuttcap%
\pgfsetroundjoin%
\definecolor{currentfill}{rgb}{0.467678,0.605591,0.968546}%
\pgfsetfillcolor{currentfill}%
\pgfsetlinewidth{0.000000pt}%
\definecolor{currentstroke}{rgb}{0.000000,0.000000,0.000000}%
\pgfsetstrokecolor{currentstroke}%
\pgfsetdash{}{0pt}%
\pgfpathmoveto{\pgfqpoint{4.474215in}{3.487467in}}%
\pgfpathlineto{\pgfqpoint{4.484658in}{3.465116in}}%
\pgfpathlineto{\pgfqpoint{4.495121in}{3.443324in}}%
\pgfpathlineto{\pgfqpoint{4.527661in}{3.424511in}}%
\pgfpathlineto{\pgfqpoint{4.560162in}{3.404098in}}%
\pgfpathlineto{\pgfqpoint{4.549662in}{3.427555in}}%
\pgfpathlineto{\pgfqpoint{4.539182in}{3.451781in}}%
\pgfpathlineto{\pgfqpoint{4.506716in}{3.470224in}}%
\pgfpathlineto{\pgfqpoint{4.474215in}{3.487467in}}%
\pgfpathclose%
\pgfusepath{fill}%
\end{pgfscope}%
\begin{pgfscope}%
\pgfpathrectangle{\pgfqpoint{1.020000in}{0.880000in}}{\pgfqpoint{6.160000in}{6.160000in}}%
\pgfusepath{clip}%
\pgfsetbuttcap%
\pgfsetroundjoin%
\definecolor{currentfill}{rgb}{0.969683,0.690484,0.575138}%
\pgfsetfillcolor{currentfill}%
\pgfsetlinewidth{0.000000pt}%
\definecolor{currentstroke}{rgb}{0.000000,0.000000,0.000000}%
\pgfsetstrokecolor{currentstroke}%
\pgfsetdash{}{0pt}%
\pgfpathmoveto{\pgfqpoint{2.144847in}{4.600826in}}%
\pgfpathlineto{\pgfqpoint{2.152921in}{4.603126in}}%
\pgfpathlineto{\pgfqpoint{2.161026in}{4.604664in}}%
\pgfpathlineto{\pgfqpoint{2.193876in}{4.631407in}}%
\pgfpathlineto{\pgfqpoint{2.226695in}{4.659600in}}%
\pgfpathlineto{\pgfqpoint{2.218589in}{4.654363in}}%
\pgfpathlineto{\pgfqpoint{2.210516in}{4.648295in}}%
\pgfpathlineto{\pgfqpoint{2.177697in}{4.623906in}}%
\pgfpathlineto{\pgfqpoint{2.144847in}{4.600826in}}%
\pgfpathclose%
\pgfusepath{fill}%
\end{pgfscope}%
\begin{pgfscope}%
\pgfpathrectangle{\pgfqpoint{1.020000in}{0.880000in}}{\pgfqpoint{6.160000in}{6.160000in}}%
\pgfusepath{clip}%
\pgfsetbuttcap%
\pgfsetroundjoin%
\definecolor{currentfill}{rgb}{0.318832,0.426605,0.859857}%
\pgfsetfillcolor{currentfill}%
\pgfsetlinewidth{0.000000pt}%
\definecolor{currentstroke}{rgb}{0.000000,0.000000,0.000000}%
\pgfsetstrokecolor{currentstroke}%
\pgfsetdash{}{0pt}%
\pgfpathmoveto{\pgfqpoint{5.359384in}{3.158872in}}%
\pgfpathlineto{\pgfqpoint{5.370746in}{3.153764in}}%
\pgfpathlineto{\pgfqpoint{5.382147in}{3.150099in}}%
\pgfpathlineto{\pgfqpoint{5.414686in}{3.170123in}}%
\pgfpathlineto{\pgfqpoint{5.447202in}{3.188962in}}%
\pgfpathlineto{\pgfqpoint{5.435780in}{3.196097in}}%
\pgfpathlineto{\pgfqpoint{5.424393in}{3.204267in}}%
\pgfpathlineto{\pgfqpoint{5.391900in}{3.182273in}}%
\pgfpathlineto{\pgfqpoint{5.359384in}{3.158872in}}%
\pgfpathclose%
\pgfusepath{fill}%
\end{pgfscope}%
\begin{pgfscope}%
\pgfpathrectangle{\pgfqpoint{1.020000in}{0.880000in}}{\pgfqpoint{6.160000in}{6.160000in}}%
\pgfusepath{clip}%
\pgfsetbuttcap%
\pgfsetroundjoin%
\definecolor{currentfill}{rgb}{0.791392,0.846750,0.936641}%
\pgfsetfillcolor{currentfill}%
\pgfsetlinewidth{0.000000pt}%
\definecolor{currentstroke}{rgb}{0.000000,0.000000,0.000000}%
\pgfsetstrokecolor{currentstroke}%
\pgfsetdash{}{0pt}%
\pgfpathmoveto{\pgfqpoint{3.849631in}{4.079278in}}%
\pgfpathlineto{\pgfqpoint{3.859578in}{4.052487in}}%
\pgfpathlineto{\pgfqpoint{3.869544in}{4.024538in}}%
\pgfpathlineto{\pgfqpoint{3.902449in}{3.987460in}}%
\pgfpathlineto{\pgfqpoint{3.935307in}{3.952565in}}%
\pgfpathlineto{\pgfqpoint{3.925309in}{3.978432in}}%
\pgfpathlineto{\pgfqpoint{3.915329in}{4.003350in}}%
\pgfpathlineto{\pgfqpoint{3.882505in}{4.040155in}}%
\pgfpathlineto{\pgfqpoint{3.849631in}{4.079278in}}%
\pgfpathclose%
\pgfusepath{fill}%
\end{pgfscope}%
\begin{pgfscope}%
\pgfpathrectangle{\pgfqpoint{1.020000in}{0.880000in}}{\pgfqpoint{6.160000in}{6.160000in}}%
\pgfusepath{clip}%
\pgfsetbuttcap%
\pgfsetroundjoin%
\definecolor{currentfill}{rgb}{0.852378,0.346492,0.280346}%
\pgfsetfillcolor{currentfill}%
\pgfsetlinewidth{0.000000pt}%
\definecolor{currentstroke}{rgb}{0.000000,0.000000,0.000000}%
\pgfsetstrokecolor{currentstroke}%
\pgfsetdash{}{0pt}%
\pgfpathmoveto{\pgfqpoint{3.033727in}{5.099292in}}%
\pgfpathlineto{\pgfqpoint{3.042479in}{5.126241in}}%
\pgfpathlineto{\pgfqpoint{3.051287in}{5.149992in}}%
\pgfpathlineto{\pgfqpoint{3.084633in}{5.123724in}}%
\pgfpathlineto{\pgfqpoint{3.117982in}{5.093399in}}%
\pgfpathlineto{\pgfqpoint{3.109087in}{5.071210in}}%
\pgfpathlineto{\pgfqpoint{3.100245in}{5.045950in}}%
\pgfpathlineto{\pgfqpoint{3.066986in}{5.074496in}}%
\pgfpathlineto{\pgfqpoint{3.033727in}{5.099292in}}%
\pgfpathclose%
\pgfusepath{fill}%
\end{pgfscope}%
\begin{pgfscope}%
\pgfpathrectangle{\pgfqpoint{1.020000in}{0.880000in}}{\pgfqpoint{6.160000in}{6.160000in}}%
\pgfusepath{clip}%
\pgfsetbuttcap%
\pgfsetroundjoin%
\definecolor{currentfill}{rgb}{0.532568,0.669801,0.990393}%
\pgfsetfillcolor{currentfill}%
\pgfsetlinewidth{0.000000pt}%
\definecolor{currentstroke}{rgb}{0.000000,0.000000,0.000000}%
\pgfsetstrokecolor{currentstroke}%
\pgfsetdash{}{0pt}%
\pgfpathmoveto{\pgfqpoint{4.323222in}{3.590987in}}%
\pgfpathlineto{\pgfqpoint{4.333544in}{3.569577in}}%
\pgfpathlineto{\pgfqpoint{4.343884in}{3.548324in}}%
\pgfpathlineto{\pgfqpoint{4.376513in}{3.533819in}}%
\pgfpathlineto{\pgfqpoint{4.409112in}{3.519025in}}%
\pgfpathlineto{\pgfqpoint{4.398728in}{3.540614in}}%
\pgfpathlineto{\pgfqpoint{4.388362in}{3.562355in}}%
\pgfpathlineto{\pgfqpoint{4.355807in}{3.576632in}}%
\pgfpathlineto{\pgfqpoint{4.323222in}{3.590987in}}%
\pgfpathclose%
\pgfusepath{fill}%
\end{pgfscope}%
\begin{pgfscope}%
\pgfpathrectangle{\pgfqpoint{1.020000in}{0.880000in}}{\pgfqpoint{6.160000in}{6.160000in}}%
\pgfusepath{clip}%
\pgfsetbuttcap%
\pgfsetroundjoin%
\definecolor{currentfill}{rgb}{0.839365,0.321856,0.264924}%
\pgfsetfillcolor{currentfill}%
\pgfsetlinewidth{0.000000pt}%
\definecolor{currentstroke}{rgb}{0.000000,0.000000,0.000000}%
\pgfsetstrokecolor{currentstroke}%
\pgfsetdash{}{0pt}%
\pgfpathmoveto{\pgfqpoint{2.817661in}{5.095739in}}%
\pgfpathlineto{\pgfqpoint{2.826064in}{5.128650in}}%
\pgfpathlineto{\pgfqpoint{2.834519in}{5.159231in}}%
\pgfpathlineto{\pgfqpoint{2.867655in}{5.160259in}}%
\pgfpathlineto{\pgfqpoint{2.900823in}{5.156855in}}%
\pgfpathlineto{\pgfqpoint{2.892293in}{5.125783in}}%
\pgfpathlineto{\pgfqpoint{2.883815in}{5.092372in}}%
\pgfpathlineto{\pgfqpoint{2.850724in}{5.096069in}}%
\pgfpathlineto{\pgfqpoint{2.817661in}{5.095739in}}%
\pgfpathclose%
\pgfusepath{fill}%
\end{pgfscope}%
\begin{pgfscope}%
\pgfpathrectangle{\pgfqpoint{1.020000in}{0.880000in}}{\pgfqpoint{6.160000in}{6.160000in}}%
\pgfusepath{clip}%
\pgfsetbuttcap%
\pgfsetroundjoin%
\definecolor{currentfill}{rgb}{0.299441,0.400248,0.839842}%
\pgfsetfillcolor{currentfill}%
\pgfsetlinewidth{0.000000pt}%
\definecolor{currentstroke}{rgb}{0.000000,0.000000,0.000000}%
\pgfsetstrokecolor{currentstroke}%
\pgfsetdash{}{0pt}%
\pgfpathmoveto{\pgfqpoint{5.294314in}{3.110298in}}%
\pgfpathlineto{\pgfqpoint{5.305649in}{3.108495in}}%
\pgfpathlineto{\pgfqpoint{5.317027in}{3.108575in}}%
\pgfpathlineto{\pgfqpoint{5.349592in}{3.129380in}}%
\pgfpathlineto{\pgfqpoint{5.382147in}{3.150099in}}%
\pgfpathlineto{\pgfqpoint{5.370746in}{3.153764in}}%
\pgfpathlineto{\pgfqpoint{5.359384in}{3.158872in}}%
\pgfpathlineto{\pgfqpoint{5.326853in}{3.134642in}}%
\pgfpathlineto{\pgfqpoint{5.294314in}{3.110298in}}%
\pgfpathclose%
\pgfusepath{fill}%
\end{pgfscope}%
\begin{pgfscope}%
\pgfpathrectangle{\pgfqpoint{1.020000in}{0.880000in}}{\pgfqpoint{6.160000in}{6.160000in}}%
\pgfusepath{clip}%
\pgfsetbuttcap%
\pgfsetroundjoin%
\definecolor{currentfill}{rgb}{0.363461,0.484784,0.901019}%
\pgfsetfillcolor{currentfill}%
\pgfsetlinewidth{0.000000pt}%
\definecolor{currentstroke}{rgb}{0.000000,0.000000,0.000000}%
\pgfsetstrokecolor{currentstroke}%
\pgfsetdash{}{0pt}%
\pgfpathmoveto{\pgfqpoint{5.641599in}{3.265340in}}%
\pgfpathlineto{\pgfqpoint{5.653128in}{3.251843in}}%
\pgfpathlineto{\pgfqpoint{5.664681in}{3.238418in}}%
\pgfpathlineto{\pgfqpoint{5.696988in}{3.244077in}}%
\pgfpathlineto{\pgfqpoint{5.729263in}{3.248729in}}%
\pgfpathlineto{\pgfqpoint{5.717674in}{3.263173in}}%
\pgfpathlineto{\pgfqpoint{5.706106in}{3.277655in}}%
\pgfpathlineto{\pgfqpoint{5.673869in}{3.272135in}}%
\pgfpathlineto{\pgfqpoint{5.641599in}{3.265340in}}%
\pgfpathclose%
\pgfusepath{fill}%
\end{pgfscope}%
\begin{pgfscope}%
\pgfpathrectangle{\pgfqpoint{1.020000in}{0.880000in}}{\pgfqpoint{6.160000in}{6.160000in}}%
\pgfusepath{clip}%
\pgfsetbuttcap%
\pgfsetroundjoin%
\definecolor{currentfill}{rgb}{0.728970,0.817464,0.973188}%
\pgfsetfillcolor{currentfill}%
\pgfsetlinewidth{0.000000pt}%
\definecolor{currentstroke}{rgb}{0.000000,0.000000,0.000000}%
\pgfsetstrokecolor{currentstroke}%
\pgfsetdash{}{0pt}%
\pgfpathmoveto{\pgfqpoint{3.935307in}{3.952565in}}%
\pgfpathlineto{\pgfqpoint{3.945322in}{3.925886in}}%
\pgfpathlineto{\pgfqpoint{3.955354in}{3.898542in}}%
\pgfpathlineto{\pgfqpoint{3.988200in}{3.867958in}}%
\pgfpathlineto{\pgfqpoint{4.021002in}{3.839425in}}%
\pgfpathlineto{\pgfqpoint{4.010934in}{3.864697in}}%
\pgfpathlineto{\pgfqpoint{4.000883in}{3.889429in}}%
\pgfpathlineto{\pgfqpoint{3.968117in}{3.919888in}}%
\pgfpathlineto{\pgfqpoint{3.935307in}{3.952565in}}%
\pgfpathclose%
\pgfusepath{fill}%
\end{pgfscope}%
\begin{pgfscope}%
\pgfpathrectangle{\pgfqpoint{1.020000in}{0.880000in}}{\pgfqpoint{6.160000in}{6.160000in}}%
\pgfusepath{clip}%
\pgfsetbuttcap%
\pgfsetroundjoin%
\definecolor{currentfill}{rgb}{0.243520,0.319189,0.771672}%
\pgfsetfillcolor{currentfill}%
\pgfsetlinewidth{0.000000pt}%
\definecolor{currentstroke}{rgb}{0.000000,0.000000,0.000000}%
\pgfsetstrokecolor{currentstroke}%
\pgfsetdash{}{0pt}%
\pgfpathmoveto{\pgfqpoint{4.947801in}{3.035285in}}%
\pgfpathlineto{\pgfqpoint{4.958771in}{3.031110in}}%
\pgfpathlineto{\pgfqpoint{4.969795in}{3.031269in}}%
\pgfpathlineto{\pgfqpoint{5.002161in}{3.018376in}}%
\pgfpathlineto{\pgfqpoint{5.034536in}{3.010274in}}%
\pgfpathlineto{\pgfqpoint{5.023444in}{3.008960in}}%
\pgfpathlineto{\pgfqpoint{5.012412in}{3.012253in}}%
\pgfpathlineto{\pgfqpoint{4.980101in}{3.021145in}}%
\pgfpathlineto{\pgfqpoint{4.947801in}{3.035285in}}%
\pgfpathclose%
\pgfusepath{fill}%
\end{pgfscope}%
\begin{pgfscope}%
\pgfpathrectangle{\pgfqpoint{1.020000in}{0.880000in}}{\pgfqpoint{6.160000in}{6.160000in}}%
\pgfusepath{clip}%
\pgfsetbuttcap%
\pgfsetroundjoin%
\definecolor{currentfill}{rgb}{0.597777,0.727330,0.999777}%
\pgfsetfillcolor{currentfill}%
\pgfsetlinewidth{0.000000pt}%
\definecolor{currentstroke}{rgb}{0.000000,0.000000,0.000000}%
\pgfsetstrokecolor{currentstroke}%
\pgfsetdash{}{0pt}%
\pgfpathmoveto{\pgfqpoint{4.172171in}{3.699638in}}%
\pgfpathlineto{\pgfqpoint{4.182370in}{3.676854in}}%
\pgfpathlineto{\pgfqpoint{4.192586in}{3.654071in}}%
\pgfpathlineto{\pgfqpoint{4.225290in}{3.636999in}}%
\pgfpathlineto{\pgfqpoint{4.257964in}{3.620934in}}%
\pgfpathlineto{\pgfqpoint{4.247704in}{3.642773in}}%
\pgfpathlineto{\pgfqpoint{4.237461in}{3.664594in}}%
\pgfpathlineto{\pgfqpoint{4.204832in}{3.681484in}}%
\pgfpathlineto{\pgfqpoint{4.172171in}{3.699638in}}%
\pgfpathclose%
\pgfusepath{fill}%
\end{pgfscope}%
\begin{pgfscope}%
\pgfpathrectangle{\pgfqpoint{1.020000in}{0.880000in}}{\pgfqpoint{6.160000in}{6.160000in}}%
\pgfusepath{clip}%
\pgfsetbuttcap%
\pgfsetroundjoin%
\definecolor{currentfill}{rgb}{0.289996,0.386836,0.828926}%
\pgfsetfillcolor{currentfill}%
\pgfsetlinewidth{0.000000pt}%
\definecolor{currentstroke}{rgb}{0.000000,0.000000,0.000000}%
\pgfsetstrokecolor{currentstroke}%
\pgfsetdash{}{0pt}%
\pgfpathmoveto{\pgfqpoint{4.796971in}{3.156375in}}%
\pgfpathlineto{\pgfqpoint{4.807730in}{3.141194in}}%
\pgfpathlineto{\pgfqpoint{4.818531in}{3.129886in}}%
\pgfpathlineto{\pgfqpoint{4.850877in}{3.102302in}}%
\pgfpathlineto{\pgfqpoint{4.883198in}{3.076710in}}%
\pgfpathlineto{\pgfqpoint{4.872341in}{3.086792in}}%
\pgfpathlineto{\pgfqpoint{4.861534in}{3.101541in}}%
\pgfpathlineto{\pgfqpoint{4.829265in}{3.127937in}}%
\pgfpathlineto{\pgfqpoint{4.796971in}{3.156375in}}%
\pgfpathclose%
\pgfusepath{fill}%
\end{pgfscope}%
\begin{pgfscope}%
\pgfpathrectangle{\pgfqpoint{1.020000in}{0.880000in}}{\pgfqpoint{6.160000in}{6.160000in}}%
\pgfusepath{clip}%
\pgfsetbuttcap%
\pgfsetroundjoin%
\definecolor{currentfill}{rgb}{0.275827,0.366717,0.812553}%
\pgfsetfillcolor{currentfill}%
\pgfsetlinewidth{0.000000pt}%
\definecolor{currentstroke}{rgb}{0.000000,0.000000,0.000000}%
\pgfsetstrokecolor{currentstroke}%
\pgfsetdash{}{0pt}%
\pgfpathmoveto{\pgfqpoint{5.229248in}{3.064633in}}%
\pgfpathlineto{\pgfqpoint{5.240550in}{3.065952in}}%
\pgfpathlineto{\pgfqpoint{5.251900in}{3.069574in}}%
\pgfpathlineto{\pgfqpoint{5.284461in}{3.088387in}}%
\pgfpathlineto{\pgfqpoint{5.317027in}{3.108575in}}%
\pgfpathlineto{\pgfqpoint{5.305649in}{3.108495in}}%
\pgfpathlineto{\pgfqpoint{5.294314in}{3.110298in}}%
\pgfpathlineto{\pgfqpoint{5.261776in}{3.086665in}}%
\pgfpathlineto{\pgfqpoint{5.229248in}{3.064633in}}%
\pgfpathclose%
\pgfusepath{fill}%
\end{pgfscope}%
\begin{pgfscope}%
\pgfpathrectangle{\pgfqpoint{1.020000in}{0.880000in}}{\pgfqpoint{6.160000in}{6.160000in}}%
\pgfusepath{clip}%
\pgfsetbuttcap%
\pgfsetroundjoin%
\definecolor{currentfill}{rgb}{0.419991,0.552989,0.942630}%
\pgfsetfillcolor{currentfill}%
\pgfsetlinewidth{0.000000pt}%
\definecolor{currentstroke}{rgb}{0.000000,0.000000,0.000000}%
\pgfsetstrokecolor{currentstroke}%
\pgfsetdash{}{0pt}%
\pgfpathmoveto{\pgfqpoint{4.560162in}{3.404098in}}%
\pgfpathlineto{\pgfqpoint{4.570684in}{3.381680in}}%
\pgfpathlineto{\pgfqpoint{4.581227in}{3.360537in}}%
\pgfpathlineto{\pgfqpoint{4.613726in}{3.336863in}}%
\pgfpathlineto{\pgfqpoint{4.646181in}{3.311349in}}%
\pgfpathlineto{\pgfqpoint{4.635601in}{3.333969in}}%
\pgfpathlineto{\pgfqpoint{4.625046in}{3.358372in}}%
\pgfpathlineto{\pgfqpoint{4.592625in}{3.382028in}}%
\pgfpathlineto{\pgfqpoint{4.560162in}{3.404098in}}%
\pgfpathclose%
\pgfusepath{fill}%
\end{pgfscope}%
\begin{pgfscope}%
\pgfpathrectangle{\pgfqpoint{1.020000in}{0.880000in}}{\pgfqpoint{6.160000in}{6.160000in}}%
\pgfusepath{clip}%
\pgfsetbuttcap%
\pgfsetroundjoin%
\definecolor{currentfill}{rgb}{0.921406,0.491420,0.383408}%
\pgfsetfillcolor{currentfill}%
\pgfsetlinewidth{0.000000pt}%
\definecolor{currentstroke}{rgb}{0.000000,0.000000,0.000000}%
\pgfsetstrokecolor{currentstroke}%
\pgfsetdash{}{0pt}%
\pgfpathmoveto{\pgfqpoint{3.269561in}{4.963741in}}%
\pgfpathlineto{\pgfqpoint{3.278781in}{4.971040in}}%
\pgfpathlineto{\pgfqpoint{3.288055in}{4.974325in}}%
\pgfpathlineto{\pgfqpoint{3.321472in}{4.924813in}}%
\pgfpathlineto{\pgfqpoint{3.354853in}{4.873343in}}%
\pgfpathlineto{\pgfqpoint{3.345511in}{4.871726in}}%
\pgfpathlineto{\pgfqpoint{3.336218in}{4.866440in}}%
\pgfpathlineto{\pgfqpoint{3.302907in}{4.916026in}}%
\pgfpathlineto{\pgfqpoint{3.269561in}{4.963741in}}%
\pgfpathclose%
\pgfusepath{fill}%
\end{pgfscope}%
\begin{pgfscope}%
\pgfpathrectangle{\pgfqpoint{1.020000in}{0.880000in}}{\pgfqpoint{6.160000in}{6.160000in}}%
\pgfusepath{clip}%
\pgfsetbuttcap%
\pgfsetroundjoin%
\definecolor{currentfill}{rgb}{0.257234,0.339661,0.789661}%
\pgfsetfillcolor{currentfill}%
\pgfsetlinewidth{0.000000pt}%
\definecolor{currentstroke}{rgb}{0.000000,0.000000,0.000000}%
\pgfsetstrokecolor{currentstroke}%
\pgfsetdash{}{0pt}%
\pgfpathmoveto{\pgfqpoint{5.164244in}{3.028946in}}%
\pgfpathlineto{\pgfqpoint{5.175505in}{3.032717in}}%
\pgfpathlineto{\pgfqpoint{5.186815in}{3.039123in}}%
\pgfpathlineto{\pgfqpoint{5.219349in}{3.052909in}}%
\pgfpathlineto{\pgfqpoint{5.251900in}{3.069574in}}%
\pgfpathlineto{\pgfqpoint{5.240550in}{3.065952in}}%
\pgfpathlineto{\pgfqpoint{5.229248in}{3.064633in}}%
\pgfpathlineto{\pgfqpoint{5.196735in}{3.045107in}}%
\pgfpathlineto{\pgfqpoint{5.164244in}{3.028946in}}%
\pgfpathclose%
\pgfusepath{fill}%
\end{pgfscope}%
\begin{pgfscope}%
\pgfpathrectangle{\pgfqpoint{1.020000in}{0.880000in}}{\pgfqpoint{6.160000in}{6.160000in}}%
\pgfusepath{clip}%
\pgfsetbuttcap%
\pgfsetroundjoin%
\definecolor{currentfill}{rgb}{0.358415,0.478426,0.896795}%
\pgfsetfillcolor{currentfill}%
\pgfsetlinewidth{0.000000pt}%
\definecolor{currentstroke}{rgb}{0.000000,0.000000,0.000000}%
\pgfsetstrokecolor{currentstroke}%
\pgfsetdash{}{0pt}%
\pgfpathmoveto{\pgfqpoint{5.576947in}{3.247133in}}%
\pgfpathlineto{\pgfqpoint{5.588444in}{3.235230in}}%
\pgfpathlineto{\pgfqpoint{5.599965in}{3.223449in}}%
\pgfpathlineto{\pgfqpoint{5.632340in}{3.231590in}}%
\pgfpathlineto{\pgfqpoint{5.664681in}{3.238418in}}%
\pgfpathlineto{\pgfqpoint{5.653128in}{3.251843in}}%
\pgfpathlineto{\pgfqpoint{5.641599in}{3.265340in}}%
\pgfpathlineto{\pgfqpoint{5.609291in}{3.257067in}}%
\pgfpathlineto{\pgfqpoint{5.576947in}{3.247133in}}%
\pgfpathclose%
\pgfusepath{fill}%
\end{pgfscope}%
\begin{pgfscope}%
\pgfpathrectangle{\pgfqpoint{1.020000in}{0.880000in}}{\pgfqpoint{6.160000in}{6.160000in}}%
\pgfusepath{clip}%
\pgfsetbuttcap%
\pgfsetroundjoin%
\definecolor{currentfill}{rgb}{0.672538,0.782861,0.991982}%
\pgfsetfillcolor{currentfill}%
\pgfsetlinewidth{0.000000pt}%
\definecolor{currentstroke}{rgb}{0.000000,0.000000,0.000000}%
\pgfsetstrokecolor{currentstroke}%
\pgfsetdash{}{0pt}%
\pgfpathmoveto{\pgfqpoint{4.021002in}{3.839425in}}%
\pgfpathlineto{\pgfqpoint{4.031087in}{3.813744in}}%
\pgfpathlineto{\pgfqpoint{4.041188in}{3.787786in}}%
\pgfpathlineto{\pgfqpoint{4.073988in}{3.763192in}}%
\pgfpathlineto{\pgfqpoint{4.106751in}{3.740366in}}%
\pgfpathlineto{\pgfqpoint{4.096611in}{3.764478in}}%
\pgfpathlineto{\pgfqpoint{4.086487in}{3.788366in}}%
\pgfpathlineto{\pgfqpoint{4.053764in}{3.812914in}}%
\pgfpathlineto{\pgfqpoint{4.021002in}{3.839425in}}%
\pgfpathclose%
\pgfusepath{fill}%
\end{pgfscope}%
\begin{pgfscope}%
\pgfpathrectangle{\pgfqpoint{1.020000in}{0.880000in}}{\pgfqpoint{6.160000in}{6.160000in}}%
\pgfusepath{clip}%
\pgfsetbuttcap%
\pgfsetroundjoin%
\definecolor{currentfill}{rgb}{0.839365,0.321856,0.264924}%
\pgfsetfillcolor{currentfill}%
\pgfsetlinewidth{0.000000pt}%
\definecolor{currentstroke}{rgb}{0.000000,0.000000,0.000000}%
\pgfsetstrokecolor{currentstroke}%
\pgfsetdash{}{0pt}%
\pgfpathmoveto{\pgfqpoint{2.751625in}{5.083344in}}%
\pgfpathlineto{\pgfqpoint{2.759963in}{5.114943in}}%
\pgfpathlineto{\pgfqpoint{2.768353in}{5.144254in}}%
\pgfpathlineto{\pgfqpoint{2.801417in}{5.153851in}}%
\pgfpathlineto{\pgfqpoint{2.834519in}{5.159231in}}%
\pgfpathlineto{\pgfqpoint{2.826064in}{5.128650in}}%
\pgfpathlineto{\pgfqpoint{2.817661in}{5.095739in}}%
\pgfpathlineto{\pgfqpoint{2.784627in}{5.091456in}}%
\pgfpathlineto{\pgfqpoint{2.751625in}{5.083344in}}%
\pgfpathclose%
\pgfusepath{fill}%
\end{pgfscope}%
\begin{pgfscope}%
\pgfpathrectangle{\pgfqpoint{1.020000in}{0.880000in}}{\pgfqpoint{6.160000in}{6.160000in}}%
\pgfusepath{clip}%
\pgfsetbuttcap%
\pgfsetroundjoin%
\definecolor{currentfill}{rgb}{0.494638,0.633022,0.978983}%
\pgfsetfillcolor{currentfill}%
\pgfsetlinewidth{0.000000pt}%
\definecolor{currentstroke}{rgb}{0.000000,0.000000,0.000000}%
\pgfsetstrokecolor{currentstroke}%
\pgfsetdash{}{0pt}%
\pgfpathmoveto{\pgfqpoint{4.409112in}{3.519025in}}%
\pgfpathlineto{\pgfqpoint{4.419514in}{3.497727in}}%
\pgfpathlineto{\pgfqpoint{4.429934in}{3.476861in}}%
\pgfpathlineto{\pgfqpoint{4.462545in}{3.460696in}}%
\pgfpathlineto{\pgfqpoint{4.495121in}{3.443324in}}%
\pgfpathlineto{\pgfqpoint{4.484658in}{3.465116in}}%
\pgfpathlineto{\pgfqpoint{4.474215in}{3.487467in}}%
\pgfpathlineto{\pgfqpoint{4.441680in}{3.503661in}}%
\pgfpathlineto{\pgfqpoint{4.409112in}{3.519025in}}%
\pgfpathclose%
\pgfusepath{fill}%
\end{pgfscope}%
\begin{pgfscope}%
\pgfpathrectangle{\pgfqpoint{1.020000in}{0.880000in}}{\pgfqpoint{6.160000in}{6.160000in}}%
\pgfusepath{clip}%
\pgfsetbuttcap%
\pgfsetroundjoin%
\definecolor{currentfill}{rgb}{0.358415,0.478426,0.896795}%
\pgfsetfillcolor{currentfill}%
\pgfsetlinewidth{0.000000pt}%
\definecolor{currentstroke}{rgb}{0.000000,0.000000,0.000000}%
\pgfsetstrokecolor{currentstroke}%
\pgfsetdash{}{0pt}%
\pgfpathmoveto{\pgfqpoint{5.793724in}{3.255657in}}%
\pgfpathlineto{\pgfqpoint{5.805378in}{3.240648in}}%
\pgfpathlineto{\pgfqpoint{5.817052in}{3.225618in}}%
\pgfpathlineto{\pgfqpoint{5.849287in}{3.227881in}}%
\pgfpathlineto{\pgfqpoint{5.837589in}{3.243063in}}%
\pgfpathlineto{\pgfqpoint{5.825913in}{3.258232in}}%
\pgfpathlineto{\pgfqpoint{5.793724in}{3.255657in}}%
\pgfpathclose%
\pgfusepath{fill}%
\end{pgfscope}%
\begin{pgfscope}%
\pgfpathrectangle{\pgfqpoint{1.020000in}{0.880000in}}{\pgfqpoint{6.160000in}{6.160000in}}%
\pgfusepath{clip}%
\pgfsetbuttcap%
\pgfsetroundjoin%
\definecolor{currentfill}{rgb}{0.945854,0.559565,0.441513}%
\pgfsetfillcolor{currentfill}%
\pgfsetlinewidth{0.000000pt}%
\definecolor{currentstroke}{rgb}{0.000000,0.000000,0.000000}%
\pgfsetstrokecolor{currentstroke}%
\pgfsetdash{}{0pt}%
\pgfpathmoveto{\pgfqpoint{3.354853in}{4.873343in}}%
\pgfpathlineto{\pgfqpoint{3.364243in}{4.871077in}}%
\pgfpathlineto{\pgfqpoint{3.373682in}{4.864762in}}%
\pgfpathlineto{\pgfqpoint{3.407079in}{4.810633in}}%
\pgfpathlineto{\pgfqpoint{3.440429in}{4.755477in}}%
\pgfpathlineto{\pgfqpoint{3.430936in}{4.762786in}}%
\pgfpathlineto{\pgfqpoint{3.421485in}{4.766454in}}%
\pgfpathlineto{\pgfqpoint{3.388192in}{4.820397in}}%
\pgfpathlineto{\pgfqpoint{3.354853in}{4.873343in}}%
\pgfpathclose%
\pgfusepath{fill}%
\end{pgfscope}%
\begin{pgfscope}%
\pgfpathrectangle{\pgfqpoint{1.020000in}{0.880000in}}{\pgfqpoint{6.160000in}{6.160000in}}%
\pgfusepath{clip}%
\pgfsetbuttcap%
\pgfsetroundjoin%
\definecolor{currentfill}{rgb}{0.373552,0.497499,0.909467}%
\pgfsetfillcolor{currentfill}%
\pgfsetlinewidth{0.000000pt}%
\definecolor{currentstroke}{rgb}{0.000000,0.000000,0.000000}%
\pgfsetstrokecolor{currentstroke}%
\pgfsetdash{}{0pt}%
\pgfpathmoveto{\pgfqpoint{4.646181in}{3.311349in}}%
\pgfpathlineto{\pgfqpoint{4.656787in}{3.290776in}}%
\pgfpathlineto{\pgfqpoint{4.667422in}{3.272426in}}%
\pgfpathlineto{\pgfqpoint{4.699873in}{3.244602in}}%
\pgfpathlineto{\pgfqpoint{4.732280in}{3.215546in}}%
\pgfpathlineto{\pgfqpoint{4.721603in}{3.234263in}}%
\pgfpathlineto{\pgfqpoint{4.710960in}{3.255975in}}%
\pgfpathlineto{\pgfqpoint{4.678592in}{3.284245in}}%
\pgfpathlineto{\pgfqpoint{4.646181in}{3.311349in}}%
\pgfpathclose%
\pgfusepath{fill}%
\end{pgfscope}%
\begin{pgfscope}%
\pgfpathrectangle{\pgfqpoint{1.020000in}{0.880000in}}{\pgfqpoint{6.160000in}{6.160000in}}%
\pgfusepath{clip}%
\pgfsetbuttcap%
\pgfsetroundjoin%
\definecolor{currentfill}{rgb}{0.559747,0.694768,0.996075}%
\pgfsetfillcolor{currentfill}%
\pgfsetlinewidth{0.000000pt}%
\definecolor{currentstroke}{rgb}{0.000000,0.000000,0.000000}%
\pgfsetstrokecolor{currentstroke}%
\pgfsetdash{}{0pt}%
\pgfpathmoveto{\pgfqpoint{4.257964in}{3.620934in}}%
\pgfpathlineto{\pgfqpoint{4.268241in}{3.599175in}}%
\pgfpathlineto{\pgfqpoint{4.278535in}{3.577593in}}%
\pgfpathlineto{\pgfqpoint{4.311224in}{3.562824in}}%
\pgfpathlineto{\pgfqpoint{4.343884in}{3.548324in}}%
\pgfpathlineto{\pgfqpoint{4.333544in}{3.569577in}}%
\pgfpathlineto{\pgfqpoint{4.323222in}{3.590987in}}%
\pgfpathlineto{\pgfqpoint{4.290608in}{3.605673in}}%
\pgfpathlineto{\pgfqpoint{4.257964in}{3.620934in}}%
\pgfpathclose%
\pgfusepath{fill}%
\end{pgfscope}%
\begin{pgfscope}%
\pgfpathrectangle{\pgfqpoint{1.020000in}{0.880000in}}{\pgfqpoint{6.160000in}{6.160000in}}%
\pgfusepath{clip}%
\pgfsetbuttcap%
\pgfsetroundjoin%
\definecolor{currentfill}{rgb}{0.825294,0.295749,0.250025}%
\pgfsetfillcolor{currentfill}%
\pgfsetlinewidth{0.000000pt}%
\definecolor{currentstroke}{rgb}{0.000000,0.000000,0.000000}%
\pgfsetstrokecolor{currentstroke}%
\pgfsetdash{}{0pt}%
\pgfpathmoveto{\pgfqpoint{2.967238in}{5.136712in}}%
\pgfpathlineto{\pgfqpoint{2.975904in}{5.164673in}}%
\pgfpathlineto{\pgfqpoint{2.984629in}{5.189353in}}%
\pgfpathlineto{\pgfqpoint{3.017951in}{5.171938in}}%
\pgfpathlineto{\pgfqpoint{3.051287in}{5.149992in}}%
\pgfpathlineto{\pgfqpoint{3.042479in}{5.126241in}}%
\pgfpathlineto{\pgfqpoint{3.033727in}{5.099292in}}%
\pgfpathlineto{\pgfqpoint{3.000476in}{5.120096in}}%
\pgfpathlineto{\pgfqpoint{2.967238in}{5.136712in}}%
\pgfpathclose%
\pgfusepath{fill}%
\end{pgfscope}%
\begin{pgfscope}%
\pgfpathrectangle{\pgfqpoint{1.020000in}{0.880000in}}{\pgfqpoint{6.160000in}{6.160000in}}%
\pgfusepath{clip}%
\pgfsetbuttcap%
\pgfsetroundjoin%
\definecolor{currentfill}{rgb}{0.248091,0.326013,0.777669}%
\pgfsetfillcolor{currentfill}%
\pgfsetlinewidth{0.000000pt}%
\definecolor{currentstroke}{rgb}{0.000000,0.000000,0.000000}%
\pgfsetstrokecolor{currentstroke}%
\pgfsetdash{}{0pt}%
\pgfpathmoveto{\pgfqpoint{5.099340in}{3.009557in}}%
\pgfpathlineto{\pgfqpoint{5.110549in}{3.014673in}}%
\pgfpathlineto{\pgfqpoint{5.121806in}{3.022615in}}%
\pgfpathlineto{\pgfqpoint{5.154300in}{3.028857in}}%
\pgfpathlineto{\pgfqpoint{5.186815in}{3.039123in}}%
\pgfpathlineto{\pgfqpoint{5.175505in}{3.032717in}}%
\pgfpathlineto{\pgfqpoint{5.164244in}{3.028946in}}%
\pgfpathlineto{\pgfqpoint{5.131779in}{3.016899in}}%
\pgfpathlineto{\pgfqpoint{5.099340in}{3.009557in}}%
\pgfpathclose%
\pgfusepath{fill}%
\end{pgfscope}%
\begin{pgfscope}%
\pgfpathrectangle{\pgfqpoint{1.020000in}{0.880000in}}{\pgfqpoint{6.160000in}{6.160000in}}%
\pgfusepath{clip}%
\pgfsetbuttcap%
\pgfsetroundjoin%
\definecolor{currentfill}{rgb}{0.353369,0.472069,0.892570}%
\pgfsetfillcolor{currentfill}%
\pgfsetlinewidth{0.000000pt}%
\definecolor{currentstroke}{rgb}{0.000000,0.000000,0.000000}%
\pgfsetstrokecolor{currentstroke}%
\pgfsetdash{}{0pt}%
\pgfpathmoveto{\pgfqpoint{5.512144in}{3.221766in}}%
\pgfpathlineto{\pgfqpoint{5.523616in}{3.212193in}}%
\pgfpathlineto{\pgfqpoint{5.535113in}{3.202817in}}%
\pgfpathlineto{\pgfqpoint{5.567556in}{3.213879in}}%
\pgfpathlineto{\pgfqpoint{5.599965in}{3.223449in}}%
\pgfpathlineto{\pgfqpoint{5.588444in}{3.235230in}}%
\pgfpathlineto{\pgfqpoint{5.576947in}{3.247133in}}%
\pgfpathlineto{\pgfqpoint{5.544564in}{3.235394in}}%
\pgfpathlineto{\pgfqpoint{5.512144in}{3.221766in}}%
\pgfpathclose%
\pgfusepath{fill}%
\end{pgfscope}%
\begin{pgfscope}%
\pgfpathrectangle{\pgfqpoint{1.020000in}{0.880000in}}{\pgfqpoint{6.160000in}{6.160000in}}%
\pgfusepath{clip}%
\pgfsetbuttcap%
\pgfsetroundjoin%
\definecolor{currentfill}{rgb}{0.843703,0.330068,0.270065}%
\pgfsetfillcolor{currentfill}%
\pgfsetlinewidth{0.000000pt}%
\definecolor{currentstroke}{rgb}{0.000000,0.000000,0.000000}%
\pgfsetstrokecolor{currentstroke}%
\pgfsetdash{}{0pt}%
\pgfpathmoveto{\pgfqpoint{2.685721in}{5.056355in}}%
\pgfpathlineto{\pgfqpoint{2.694004in}{5.085890in}}%
\pgfpathlineto{\pgfqpoint{2.702338in}{5.113210in}}%
\pgfpathlineto{\pgfqpoint{2.735326in}{5.130628in}}%
\pgfpathlineto{\pgfqpoint{2.768353in}{5.144254in}}%
\pgfpathlineto{\pgfqpoint{2.759963in}{5.114943in}}%
\pgfpathlineto{\pgfqpoint{2.751625in}{5.083344in}}%
\pgfpathlineto{\pgfqpoint{2.718656in}{5.071572in}}%
\pgfpathlineto{\pgfqpoint{2.685721in}{5.056355in}}%
\pgfpathclose%
\pgfusepath{fill}%
\end{pgfscope}%
\begin{pgfscope}%
\pgfpathrectangle{\pgfqpoint{1.020000in}{0.880000in}}{\pgfqpoint{6.160000in}{6.160000in}}%
\pgfusepath{clip}%
\pgfsetbuttcap%
\pgfsetroundjoin%
\definecolor{currentfill}{rgb}{0.856716,0.354704,0.285487}%
\pgfsetfillcolor{currentfill}%
\pgfsetlinewidth{0.000000pt}%
\definecolor{currentstroke}{rgb}{0.000000,0.000000,0.000000}%
\pgfsetstrokecolor{currentstroke}%
\pgfsetdash{}{0pt}%
\pgfpathmoveto{\pgfqpoint{3.117982in}{5.093399in}}%
\pgfpathlineto{\pgfqpoint{3.126933in}{5.112093in}}%
\pgfpathlineto{\pgfqpoint{3.135943in}{5.126890in}}%
\pgfpathlineto{\pgfqpoint{3.169369in}{5.091118in}}%
\pgfpathlineto{\pgfqpoint{3.202785in}{5.051728in}}%
\pgfpathlineto{\pgfqpoint{3.193696in}{5.038629in}}%
\pgfpathlineto{\pgfqpoint{3.184661in}{5.021853in}}%
\pgfpathlineto{\pgfqpoint{3.151327in}{5.059326in}}%
\pgfpathlineto{\pgfqpoint{3.117982in}{5.093399in}}%
\pgfpathclose%
\pgfusepath{fill}%
\end{pgfscope}%
\begin{pgfscope}%
\pgfpathrectangle{\pgfqpoint{1.020000in}{0.880000in}}{\pgfqpoint{6.160000in}{6.160000in}}%
\pgfusepath{clip}%
\pgfsetbuttcap%
\pgfsetroundjoin%
\definecolor{currentfill}{rgb}{0.966017,0.646130,0.525890}%
\pgfsetfillcolor{currentfill}%
\pgfsetlinewidth{0.000000pt}%
\definecolor{currentstroke}{rgb}{0.000000,0.000000,0.000000}%
\pgfsetstrokecolor{currentstroke}%
\pgfsetdash{}{0pt}%
\pgfpathmoveto{\pgfqpoint{3.440429in}{4.755477in}}%
\pgfpathlineto{\pgfqpoint{3.449967in}{4.744421in}}%
\pgfpathlineto{\pgfqpoint{3.459546in}{4.729558in}}%
\pgfpathlineto{\pgfqpoint{3.492892in}{4.673477in}}%
\pgfpathlineto{\pgfqpoint{3.526183in}{4.617295in}}%
\pgfpathlineto{\pgfqpoint{3.516560in}{4.632314in}}%
\pgfpathlineto{\pgfqpoint{3.506975in}{4.643966in}}%
\pgfpathlineto{\pgfqpoint{3.473729in}{4.699769in}}%
\pgfpathlineto{\pgfqpoint{3.440429in}{4.755477in}}%
\pgfpathclose%
\pgfusepath{fill}%
\end{pgfscope}%
\begin{pgfscope}%
\pgfpathrectangle{\pgfqpoint{1.020000in}{0.880000in}}{\pgfqpoint{6.160000in}{6.160000in}}%
\pgfusepath{clip}%
\pgfsetbuttcap%
\pgfsetroundjoin%
\definecolor{currentfill}{rgb}{0.261805,0.346484,0.795658}%
\pgfsetfillcolor{currentfill}%
\pgfsetlinewidth{0.000000pt}%
\definecolor{currentstroke}{rgb}{0.000000,0.000000,0.000000}%
\pgfsetstrokecolor{currentstroke}%
\pgfsetdash{}{0pt}%
\pgfpathmoveto{\pgfqpoint{4.883198in}{3.076710in}}%
\pgfpathlineto{\pgfqpoint{4.894105in}{3.070976in}}%
\pgfpathlineto{\pgfqpoint{4.905059in}{3.069074in}}%
\pgfpathlineto{\pgfqpoint{4.937431in}{3.048405in}}%
\pgfpathlineto{\pgfqpoint{4.969795in}{3.031269in}}%
\pgfpathlineto{\pgfqpoint{4.958771in}{3.031110in}}%
\pgfpathlineto{\pgfqpoint{4.947801in}{3.035285in}}%
\pgfpathlineto{\pgfqpoint{4.915503in}{3.054069in}}%
\pgfpathlineto{\pgfqpoint{4.883198in}{3.076710in}}%
\pgfpathclose%
\pgfusepath{fill}%
\end{pgfscope}%
\begin{pgfscope}%
\pgfpathrectangle{\pgfqpoint{1.020000in}{0.880000in}}{\pgfqpoint{6.160000in}{6.160000in}}%
\pgfusepath{clip}%
\pgfsetbuttcap%
\pgfsetroundjoin%
\definecolor{currentfill}{rgb}{0.967544,0.730850,0.624685}%
\pgfsetfillcolor{currentfill}%
\pgfsetlinewidth{0.000000pt}%
\definecolor{currentstroke}{rgb}{0.000000,0.000000,0.000000}%
\pgfsetstrokecolor{currentstroke}%
\pgfsetdash{}{0pt}%
\pgfpathmoveto{\pgfqpoint{3.526183in}{4.617295in}}%
\pgfpathlineto{\pgfqpoint{3.535844in}{4.598902in}}%
\pgfpathlineto{\pgfqpoint{3.545540in}{4.577172in}}%
\pgfpathlineto{\pgfqpoint{3.578811in}{4.521820in}}%
\pgfpathlineto{\pgfqpoint{3.612023in}{4.467193in}}%
\pgfpathlineto{\pgfqpoint{3.602292in}{4.488216in}}%
\pgfpathlineto{\pgfqpoint{3.592593in}{4.506334in}}%
\pgfpathlineto{\pgfqpoint{3.559418in}{4.561447in}}%
\pgfpathlineto{\pgfqpoint{3.526183in}{4.617295in}}%
\pgfpathclose%
\pgfusepath{fill}%
\end{pgfscope}%
\begin{pgfscope}%
\pgfpathrectangle{\pgfqpoint{1.020000in}{0.880000in}}{\pgfqpoint{6.160000in}{6.160000in}}%
\pgfusepath{clip}%
\pgfsetbuttcap%
\pgfsetroundjoin%
\definecolor{currentfill}{rgb}{0.945540,0.798606,0.723105}%
\pgfsetfillcolor{currentfill}%
\pgfsetlinewidth{0.000000pt}%
\definecolor{currentstroke}{rgb}{0.000000,0.000000,0.000000}%
\pgfsetstrokecolor{currentstroke}%
\pgfsetdash{}{0pt}%
\pgfpathmoveto{\pgfqpoint{3.612023in}{4.467193in}}%
\pgfpathlineto{\pgfqpoint{3.621784in}{4.443337in}}%
\pgfpathlineto{\pgfqpoint{3.631575in}{4.416756in}}%
\pgfpathlineto{\pgfqpoint{3.664758in}{4.364512in}}%
\pgfpathlineto{\pgfqpoint{3.697882in}{4.313642in}}%
\pgfpathlineto{\pgfqpoint{3.688062in}{4.338752in}}%
\pgfpathlineto{\pgfqpoint{3.678268in}{4.361529in}}%
\pgfpathlineto{\pgfqpoint{3.645176in}{4.413655in}}%
\pgfpathlineto{\pgfqpoint{3.612023in}{4.467193in}}%
\pgfpathclose%
\pgfusepath{fill}%
\end{pgfscope}%
\begin{pgfscope}%
\pgfpathrectangle{\pgfqpoint{1.020000in}{0.880000in}}{\pgfqpoint{6.160000in}{6.160000in}}%
\pgfusepath{clip}%
\pgfsetbuttcap%
\pgfsetroundjoin%
\definecolor{currentfill}{rgb}{0.624703,0.748318,0.998719}%
\pgfsetfillcolor{currentfill}%
\pgfsetlinewidth{0.000000pt}%
\definecolor{currentstroke}{rgb}{0.000000,0.000000,0.000000}%
\pgfsetstrokecolor{currentstroke}%
\pgfsetdash{}{0pt}%
\pgfpathmoveto{\pgfqpoint{4.106751in}{3.740366in}}%
\pgfpathlineto{\pgfqpoint{4.116907in}{3.716143in}}%
\pgfpathlineto{\pgfqpoint{4.127080in}{3.691928in}}%
\pgfpathlineto{\pgfqpoint{4.159849in}{3.672330in}}%
\pgfpathlineto{\pgfqpoint{4.192586in}{3.654071in}}%
\pgfpathlineto{\pgfqpoint{4.182370in}{3.676854in}}%
\pgfpathlineto{\pgfqpoint{4.172171in}{3.699638in}}%
\pgfpathlineto{\pgfqpoint{4.139478in}{3.719219in}}%
\pgfpathlineto{\pgfqpoint{4.106751in}{3.740366in}}%
\pgfpathclose%
\pgfusepath{fill}%
\end{pgfscope}%
\begin{pgfscope}%
\pgfpathrectangle{\pgfqpoint{1.020000in}{0.880000in}}{\pgfqpoint{6.160000in}{6.160000in}}%
\pgfusepath{clip}%
\pgfsetbuttcap%
\pgfsetroundjoin%
\definecolor{currentfill}{rgb}{0.899543,0.847500,0.817789}%
\pgfsetfillcolor{currentfill}%
\pgfsetlinewidth{0.000000pt}%
\definecolor{currentstroke}{rgb}{0.000000,0.000000,0.000000}%
\pgfsetstrokecolor{currentstroke}%
\pgfsetdash{}{0pt}%
\pgfpathmoveto{\pgfqpoint{3.697882in}{4.313642in}}%
\pgfpathlineto{\pgfqpoint{3.707727in}{4.286324in}}%
\pgfpathlineto{\pgfqpoint{3.717596in}{4.256954in}}%
\pgfpathlineto{\pgfqpoint{3.750689in}{4.209650in}}%
\pgfpathlineto{\pgfqpoint{3.783725in}{4.164152in}}%
\pgfpathlineto{\pgfqpoint{3.773828in}{4.191485in}}%
\pgfpathlineto{\pgfqpoint{3.763954in}{4.217089in}}%
\pgfpathlineto{\pgfqpoint{3.730947in}{4.264423in}}%
\pgfpathlineto{\pgfqpoint{3.697882in}{4.313642in}}%
\pgfpathclose%
\pgfusepath{fill}%
\end{pgfscope}%
\begin{pgfscope}%
\pgfpathrectangle{\pgfqpoint{1.020000in}{0.880000in}}{\pgfqpoint{6.160000in}{6.160000in}}%
\pgfusepath{clip}%
\pgfsetbuttcap%
\pgfsetroundjoin%
\definecolor{currentfill}{rgb}{0.835345,0.860514,0.898970}%
\pgfsetfillcolor{currentfill}%
\pgfsetlinewidth{0.000000pt}%
\definecolor{currentstroke}{rgb}{0.000000,0.000000,0.000000}%
\pgfsetstrokecolor{currentstroke}%
\pgfsetdash{}{0pt}%
\pgfpathmoveto{\pgfqpoint{3.783725in}{4.164152in}}%
\pgfpathlineto{\pgfqpoint{3.793642in}{4.135245in}}%
\pgfpathlineto{\pgfqpoint{3.803579in}{4.104941in}}%
\pgfpathlineto{\pgfqpoint{3.836587in}{4.063732in}}%
\pgfpathlineto{\pgfqpoint{3.869544in}{4.024538in}}%
\pgfpathlineto{\pgfqpoint{3.859578in}{4.052487in}}%
\pgfpathlineto{\pgfqpoint{3.849631in}{4.079278in}}%
\pgfpathlineto{\pgfqpoint{3.816705in}{4.120646in}}%
\pgfpathlineto{\pgfqpoint{3.783725in}{4.164152in}}%
\pgfpathclose%
\pgfusepath{fill}%
\end{pgfscope}%
\begin{pgfscope}%
\pgfpathrectangle{\pgfqpoint{1.020000in}{0.880000in}}{\pgfqpoint{6.160000in}{6.160000in}}%
\pgfusepath{clip}%
\pgfsetbuttcap%
\pgfsetroundjoin%
\definecolor{currentfill}{rgb}{0.363461,0.484784,0.901019}%
\pgfsetfillcolor{currentfill}%
\pgfsetlinewidth{0.000000pt}%
\definecolor{currentstroke}{rgb}{0.000000,0.000000,0.000000}%
\pgfsetstrokecolor{currentstroke}%
\pgfsetdash{}{0pt}%
\pgfpathmoveto{\pgfqpoint{5.729263in}{3.248729in}}%
\pgfpathlineto{\pgfqpoint{5.740874in}{3.234279in}}%
\pgfpathlineto{\pgfqpoint{5.752506in}{3.219785in}}%
\pgfpathlineto{\pgfqpoint{5.784792in}{3.222954in}}%
\pgfpathlineto{\pgfqpoint{5.817052in}{3.225618in}}%
\pgfpathlineto{\pgfqpoint{5.805378in}{3.240648in}}%
\pgfpathlineto{\pgfqpoint{5.793724in}{3.255657in}}%
\pgfpathlineto{\pgfqpoint{5.761508in}{3.252537in}}%
\pgfpathlineto{\pgfqpoint{5.729263in}{3.248729in}}%
\pgfpathclose%
\pgfusepath{fill}%
\end{pgfscope}%
\begin{pgfscope}%
\pgfpathrectangle{\pgfqpoint{1.020000in}{0.880000in}}{\pgfqpoint{6.160000in}{6.160000in}}%
\pgfusepath{clip}%
\pgfsetbuttcap%
\pgfsetroundjoin%
\definecolor{currentfill}{rgb}{0.861054,0.362916,0.290628}%
\pgfsetfillcolor{currentfill}%
\pgfsetlinewidth{0.000000pt}%
\definecolor{currentstroke}{rgb}{0.000000,0.000000,0.000000}%
\pgfsetstrokecolor{currentstroke}%
\pgfsetdash{}{0pt}%
\pgfpathmoveto{\pgfqpoint{2.619949in}{5.016635in}}%
\pgfpathlineto{\pgfqpoint{2.628187in}{5.043448in}}%
\pgfpathlineto{\pgfqpoint{2.636476in}{5.068146in}}%
\pgfpathlineto{\pgfqpoint{2.669389in}{5.092277in}}%
\pgfpathlineto{\pgfqpoint{2.702338in}{5.113210in}}%
\pgfpathlineto{\pgfqpoint{2.694004in}{5.085890in}}%
\pgfpathlineto{\pgfqpoint{2.685721in}{5.056355in}}%
\pgfpathlineto{\pgfqpoint{2.652819in}{5.037947in}}%
\pgfpathlineto{\pgfqpoint{2.619949in}{5.016635in}}%
\pgfpathclose%
\pgfusepath{fill}%
\end{pgfscope}%
\begin{pgfscope}%
\pgfpathrectangle{\pgfqpoint{1.020000in}{0.880000in}}{\pgfqpoint{6.160000in}{6.160000in}}%
\pgfusepath{clip}%
\pgfsetbuttcap%
\pgfsetroundjoin%
\definecolor{currentfill}{rgb}{0.451739,0.588181,0.960201}%
\pgfsetfillcolor{currentfill}%
\pgfsetlinewidth{0.000000pt}%
\definecolor{currentstroke}{rgb}{0.000000,0.000000,0.000000}%
\pgfsetstrokecolor{currentstroke}%
\pgfsetdash{}{0pt}%
\pgfpathmoveto{\pgfqpoint{4.495121in}{3.443324in}}%
\pgfpathlineto{\pgfqpoint{4.505603in}{3.422277in}}%
\pgfpathlineto{\pgfqpoint{4.516106in}{3.402141in}}%
\pgfpathlineto{\pgfqpoint{4.548686in}{3.382280in}}%
\pgfpathlineto{\pgfqpoint{4.581227in}{3.360537in}}%
\pgfpathlineto{\pgfqpoint{4.570684in}{3.381680in}}%
\pgfpathlineto{\pgfqpoint{4.560162in}{3.404098in}}%
\pgfpathlineto{\pgfqpoint{4.527661in}{3.424511in}}%
\pgfpathlineto{\pgfqpoint{4.495121in}{3.443324in}}%
\pgfpathclose%
\pgfusepath{fill}%
\end{pgfscope}%
\begin{pgfscope}%
\pgfpathrectangle{\pgfqpoint{1.020000in}{0.880000in}}{\pgfqpoint{6.160000in}{6.160000in}}%
\pgfusepath{clip}%
\pgfsetbuttcap%
\pgfsetroundjoin%
\definecolor{currentfill}{rgb}{0.772706,0.838978,0.949319}%
\pgfsetfillcolor{currentfill}%
\pgfsetlinewidth{0.000000pt}%
\definecolor{currentstroke}{rgb}{0.000000,0.000000,0.000000}%
\pgfsetstrokecolor{currentstroke}%
\pgfsetdash{}{0pt}%
\pgfpathmoveto{\pgfqpoint{3.869544in}{4.024538in}}%
\pgfpathlineto{\pgfqpoint{3.879527in}{3.995595in}}%
\pgfpathlineto{\pgfqpoint{3.889528in}{3.965834in}}%
\pgfpathlineto{\pgfqpoint{3.922464in}{3.931177in}}%
\pgfpathlineto{\pgfqpoint{3.955354in}{3.898542in}}%
\pgfpathlineto{\pgfqpoint{3.945322in}{3.925886in}}%
\pgfpathlineto{\pgfqpoint{3.935307in}{3.952565in}}%
\pgfpathlineto{\pgfqpoint{3.902449in}{3.987460in}}%
\pgfpathlineto{\pgfqpoint{3.869544in}{4.024538in}}%
\pgfpathclose%
\pgfusepath{fill}%
\end{pgfscope}%
\begin{pgfscope}%
\pgfpathrectangle{\pgfqpoint{1.020000in}{0.880000in}}{\pgfqpoint{6.160000in}{6.160000in}}%
\pgfusepath{clip}%
\pgfsetbuttcap%
\pgfsetroundjoin%
\definecolor{currentfill}{rgb}{0.338377,0.452819,0.879317}%
\pgfsetfillcolor{currentfill}%
\pgfsetlinewidth{0.000000pt}%
\definecolor{currentstroke}{rgb}{0.000000,0.000000,0.000000}%
\pgfsetstrokecolor{currentstroke}%
\pgfsetdash{}{0pt}%
\pgfpathmoveto{\pgfqpoint{5.447202in}{3.188962in}}%
\pgfpathlineto{\pgfqpoint{5.458653in}{3.182488in}}%
\pgfpathlineto{\pgfqpoint{5.470131in}{3.176309in}}%
\pgfpathlineto{\pgfqpoint{5.502637in}{3.190265in}}%
\pgfpathlineto{\pgfqpoint{5.535113in}{3.202817in}}%
\pgfpathlineto{\pgfqpoint{5.523616in}{3.212193in}}%
\pgfpathlineto{\pgfqpoint{5.512144in}{3.221766in}}%
\pgfpathlineto{\pgfqpoint{5.479689in}{3.206253in}}%
\pgfpathlineto{\pgfqpoint{5.447202in}{3.188962in}}%
\pgfpathclose%
\pgfusepath{fill}%
\end{pgfscope}%
\begin{pgfscope}%
\pgfpathrectangle{\pgfqpoint{1.020000in}{0.880000in}}{\pgfqpoint{6.160000in}{6.160000in}}%
\pgfusepath{clip}%
\pgfsetbuttcap%
\pgfsetroundjoin%
\definecolor{currentfill}{rgb}{0.328604,0.439712,0.869587}%
\pgfsetfillcolor{currentfill}%
\pgfsetlinewidth{0.000000pt}%
\definecolor{currentstroke}{rgb}{0.000000,0.000000,0.000000}%
\pgfsetstrokecolor{currentstroke}%
\pgfsetdash{}{0pt}%
\pgfpathmoveto{\pgfqpoint{4.732280in}{3.215546in}}%
\pgfpathlineto{\pgfqpoint{4.742991in}{3.199917in}}%
\pgfpathlineto{\pgfqpoint{4.753738in}{3.187322in}}%
\pgfpathlineto{\pgfqpoint{4.786153in}{3.158517in}}%
\pgfpathlineto{\pgfqpoint{4.818531in}{3.129886in}}%
\pgfpathlineto{\pgfqpoint{4.807730in}{3.141194in}}%
\pgfpathlineto{\pgfqpoint{4.796971in}{3.156375in}}%
\pgfpathlineto{\pgfqpoint{4.764644in}{3.185878in}}%
\pgfpathlineto{\pgfqpoint{4.732280in}{3.215546in}}%
\pgfpathclose%
\pgfusepath{fill}%
\end{pgfscope}%
\begin{pgfscope}%
\pgfpathrectangle{\pgfqpoint{1.020000in}{0.880000in}}{\pgfqpoint{6.160000in}{6.160000in}}%
\pgfusepath{clip}%
\pgfsetbuttcap%
\pgfsetroundjoin%
\definecolor{currentfill}{rgb}{0.248091,0.326013,0.777669}%
\pgfsetfillcolor{currentfill}%
\pgfsetlinewidth{0.000000pt}%
\definecolor{currentstroke}{rgb}{0.000000,0.000000,0.000000}%
\pgfsetstrokecolor{currentstroke}%
\pgfsetdash{}{0pt}%
\pgfpathmoveto{\pgfqpoint{5.034536in}{3.010274in}}%
\pgfpathlineto{\pgfqpoint{5.045681in}{3.015375in}}%
\pgfpathlineto{\pgfqpoint{5.056873in}{3.023316in}}%
\pgfpathlineto{\pgfqpoint{5.089331in}{3.020725in}}%
\pgfpathlineto{\pgfqpoint{5.121806in}{3.022615in}}%
\pgfpathlineto{\pgfqpoint{5.110549in}{3.014673in}}%
\pgfpathlineto{\pgfqpoint{5.099340in}{3.009557in}}%
\pgfpathlineto{\pgfqpoint{5.066927in}{3.007300in}}%
\pgfpathlineto{\pgfqpoint{5.034536in}{3.010274in}}%
\pgfpathclose%
\pgfusepath{fill}%
\end{pgfscope}%
\begin{pgfscope}%
\pgfpathrectangle{\pgfqpoint{1.020000in}{0.880000in}}{\pgfqpoint{6.160000in}{6.160000in}}%
\pgfusepath{clip}%
\pgfsetbuttcap%
\pgfsetroundjoin%
\definecolor{currentfill}{rgb}{0.880896,0.402331,0.317115}%
\pgfsetfillcolor{currentfill}%
\pgfsetlinewidth{0.000000pt}%
\definecolor{currentstroke}{rgb}{0.000000,0.000000,0.000000}%
\pgfsetstrokecolor{currentstroke}%
\pgfsetdash{}{0pt}%
\pgfpathmoveto{\pgfqpoint{2.554295in}{4.966591in}}%
\pgfpathlineto{\pgfqpoint{2.562499in}{4.990146in}}%
\pgfpathlineto{\pgfqpoint{2.570753in}{5.011710in}}%
\pgfpathlineto{\pgfqpoint{2.603599in}{5.041166in}}%
\pgfpathlineto{\pgfqpoint{2.636476in}{5.068146in}}%
\pgfpathlineto{\pgfqpoint{2.628187in}{5.043448in}}%
\pgfpathlineto{\pgfqpoint{2.619949in}{5.016635in}}%
\pgfpathlineto{\pgfqpoint{2.587108in}{4.992737in}}%
\pgfpathlineto{\pgfqpoint{2.554295in}{4.966591in}}%
\pgfpathclose%
\pgfusepath{fill}%
\end{pgfscope}%
\begin{pgfscope}%
\pgfpathrectangle{\pgfqpoint{1.020000in}{0.880000in}}{\pgfqpoint{6.160000in}{6.160000in}}%
\pgfusepath{clip}%
\pgfsetbuttcap%
\pgfsetroundjoin%
\definecolor{currentfill}{rgb}{0.521696,0.659599,0.987736}%
\pgfsetfillcolor{currentfill}%
\pgfsetlinewidth{0.000000pt}%
\definecolor{currentstroke}{rgb}{0.000000,0.000000,0.000000}%
\pgfsetstrokecolor{currentstroke}%
\pgfsetdash{}{0pt}%
\pgfpathmoveto{\pgfqpoint{4.343884in}{3.548324in}}%
\pgfpathlineto{\pgfqpoint{4.354241in}{3.527338in}}%
\pgfpathlineto{\pgfqpoint{4.364617in}{3.506728in}}%
\pgfpathlineto{\pgfqpoint{4.397291in}{3.492102in}}%
\pgfpathlineto{\pgfqpoint{4.429934in}{3.476861in}}%
\pgfpathlineto{\pgfqpoint{4.419514in}{3.497727in}}%
\pgfpathlineto{\pgfqpoint{4.409112in}{3.519025in}}%
\pgfpathlineto{\pgfqpoint{4.376513in}{3.533819in}}%
\pgfpathlineto{\pgfqpoint{4.343884in}{3.548324in}}%
\pgfpathclose%
\pgfusepath{fill}%
\end{pgfscope}%
\begin{pgfscope}%
\pgfpathrectangle{\pgfqpoint{1.020000in}{0.880000in}}{\pgfqpoint{6.160000in}{6.160000in}}%
\pgfusepath{clip}%
\pgfsetbuttcap%
\pgfsetroundjoin%
\definecolor{currentfill}{rgb}{0.902659,0.447939,0.349721}%
\pgfsetfillcolor{currentfill}%
\pgfsetlinewidth{0.000000pt}%
\definecolor{currentstroke}{rgb}{0.000000,0.000000,0.000000}%
\pgfsetstrokecolor{currentstroke}%
\pgfsetdash{}{0pt}%
\pgfpathmoveto{\pgfqpoint{2.488735in}{4.908984in}}%
\pgfpathlineto{\pgfqpoint{2.496914in}{4.928888in}}%
\pgfpathlineto{\pgfqpoint{2.505143in}{4.946939in}}%
\pgfpathlineto{\pgfqpoint{2.537936in}{4.980167in}}%
\pgfpathlineto{\pgfqpoint{2.570753in}{5.011710in}}%
\pgfpathlineto{\pgfqpoint{2.562499in}{4.990146in}}%
\pgfpathlineto{\pgfqpoint{2.554295in}{4.966591in}}%
\pgfpathlineto{\pgfqpoint{2.521505in}{4.938551in}}%
\pgfpathlineto{\pgfqpoint{2.488735in}{4.908984in}}%
\pgfpathclose%
\pgfusepath{fill}%
\end{pgfscope}%
\begin{pgfscope}%
\pgfpathrectangle{\pgfqpoint{1.020000in}{0.880000in}}{\pgfqpoint{6.160000in}{6.160000in}}%
\pgfusepath{clip}%
\pgfsetbuttcap%
\pgfsetroundjoin%
\definecolor{currentfill}{rgb}{0.924409,0.498590,0.389059}%
\pgfsetfillcolor{currentfill}%
\pgfsetlinewidth{0.000000pt}%
\definecolor{currentstroke}{rgb}{0.000000,0.000000,0.000000}%
\pgfsetstrokecolor{currentstroke}%
\pgfsetdash{}{0pt}%
\pgfpathmoveto{\pgfqpoint{2.423236in}{4.846732in}}%
\pgfpathlineto{\pgfqpoint{2.431399in}{4.862737in}}%
\pgfpathlineto{\pgfqpoint{2.439608in}{4.877037in}}%
\pgfpathlineto{\pgfqpoint{2.472368in}{4.912428in}}%
\pgfpathlineto{\pgfqpoint{2.505143in}{4.946939in}}%
\pgfpathlineto{\pgfqpoint{2.496914in}{4.928888in}}%
\pgfpathlineto{\pgfqpoint{2.488735in}{4.908984in}}%
\pgfpathlineto{\pgfqpoint{2.455980in}{4.878256in}}%
\pgfpathlineto{\pgfqpoint{2.423236in}{4.846732in}}%
\pgfpathclose%
\pgfusepath{fill}%
\end{pgfscope}%
\begin{pgfscope}%
\pgfpathrectangle{\pgfqpoint{1.020000in}{0.880000in}}{\pgfqpoint{6.160000in}{6.160000in}}%
\pgfusepath{clip}%
\pgfsetbuttcap%
\pgfsetroundjoin%
\definecolor{currentfill}{rgb}{0.941728,0.546413,0.429707}%
\pgfsetfillcolor{currentfill}%
\pgfsetlinewidth{0.000000pt}%
\definecolor{currentstroke}{rgb}{0.000000,0.000000,0.000000}%
\pgfsetstrokecolor{currentstroke}%
\pgfsetdash{}{0pt}%
\pgfpathmoveto{\pgfqpoint{2.357759in}{4.782708in}}%
\pgfpathlineto{\pgfqpoint{2.365911in}{4.794710in}}%
\pgfpathlineto{\pgfqpoint{2.374108in}{4.805162in}}%
\pgfpathlineto{\pgfqpoint{2.406857in}{4.841157in}}%
\pgfpathlineto{\pgfqpoint{2.439608in}{4.877037in}}%
\pgfpathlineto{\pgfqpoint{2.431399in}{4.862737in}}%
\pgfpathlineto{\pgfqpoint{2.423236in}{4.846732in}}%
\pgfpathlineto{\pgfqpoint{2.390498in}{4.814769in}}%
\pgfpathlineto{\pgfqpoint{2.357759in}{4.782708in}}%
\pgfpathclose%
\pgfusepath{fill}%
\end{pgfscope}%
\begin{pgfscope}%
\pgfpathrectangle{\pgfqpoint{1.020000in}{0.880000in}}{\pgfqpoint{6.160000in}{6.160000in}}%
\pgfusepath{clip}%
\pgfsetbuttcap%
\pgfsetroundjoin%
\definecolor{currentfill}{rgb}{0.708720,0.805721,0.981117}%
\pgfsetfillcolor{currentfill}%
\pgfsetlinewidth{0.000000pt}%
\definecolor{currentstroke}{rgb}{0.000000,0.000000,0.000000}%
\pgfsetstrokecolor{currentstroke}%
\pgfsetdash{}{0pt}%
\pgfpathmoveto{\pgfqpoint{3.955354in}{3.898542in}}%
\pgfpathlineto{\pgfqpoint{3.965403in}{3.870688in}}%
\pgfpathlineto{\pgfqpoint{3.975468in}{3.842488in}}%
\pgfpathlineto{\pgfqpoint{4.008348in}{3.814207in}}%
\pgfpathlineto{\pgfqpoint{4.041188in}{3.787786in}}%
\pgfpathlineto{\pgfqpoint{4.031087in}{3.813744in}}%
\pgfpathlineto{\pgfqpoint{4.021002in}{3.839425in}}%
\pgfpathlineto{\pgfqpoint{3.988200in}{3.867958in}}%
\pgfpathlineto{\pgfqpoint{3.955354in}{3.898542in}}%
\pgfpathclose%
\pgfusepath{fill}%
\end{pgfscope}%
\begin{pgfscope}%
\pgfpathrectangle{\pgfqpoint{1.020000in}{0.880000in}}{\pgfqpoint{6.160000in}{6.160000in}}%
\pgfusepath{clip}%
\pgfsetbuttcap%
\pgfsetroundjoin%
\definecolor{currentfill}{rgb}{0.954853,0.591622,0.471337}%
\pgfsetfillcolor{currentfill}%
\pgfsetlinewidth{0.000000pt}%
\definecolor{currentstroke}{rgb}{0.000000,0.000000,0.000000}%
\pgfsetstrokecolor{currentstroke}%
\pgfsetdash{}{0pt}%
\pgfpathmoveto{\pgfqpoint{2.292260in}{4.719564in}}%
\pgfpathlineto{\pgfqpoint{2.300406in}{4.727593in}}%
\pgfpathlineto{\pgfqpoint{2.308594in}{4.734225in}}%
\pgfpathlineto{\pgfqpoint{2.341355in}{4.769408in}}%
\pgfpathlineto{\pgfqpoint{2.374108in}{4.805162in}}%
\pgfpathlineto{\pgfqpoint{2.365911in}{4.794710in}}%
\pgfpathlineto{\pgfqpoint{2.357759in}{4.782708in}}%
\pgfpathlineto{\pgfqpoint{2.325015in}{4.750872in}}%
\pgfpathlineto{\pgfqpoint{2.292260in}{4.719564in}}%
\pgfpathclose%
\pgfusepath{fill}%
\end{pgfscope}%
\begin{pgfscope}%
\pgfpathrectangle{\pgfqpoint{1.020000in}{0.880000in}}{\pgfqpoint{6.160000in}{6.160000in}}%
\pgfusepath{clip}%
\pgfsetbuttcap%
\pgfsetroundjoin%
\definecolor{currentfill}{rgb}{0.963806,0.634188,0.513721}%
\pgfsetfillcolor{currentfill}%
\pgfsetlinewidth{0.000000pt}%
\definecolor{currentstroke}{rgb}{0.000000,0.000000,0.000000}%
\pgfsetstrokecolor{currentstroke}%
\pgfsetdash{}{0pt}%
\pgfpathmoveto{\pgfqpoint{2.226695in}{4.659600in}}%
\pgfpathlineto{\pgfqpoint{2.234838in}{4.663799in}}%
\pgfpathlineto{\pgfqpoint{2.243020in}{4.666748in}}%
\pgfpathlineto{\pgfqpoint{2.275817in}{4.699915in}}%
\pgfpathlineto{\pgfqpoint{2.308594in}{4.734225in}}%
\pgfpathlineto{\pgfqpoint{2.300406in}{4.727593in}}%
\pgfpathlineto{\pgfqpoint{2.292260in}{4.719564in}}%
\pgfpathlineto{\pgfqpoint{2.259488in}{4.689058in}}%
\pgfpathlineto{\pgfqpoint{2.226695in}{4.659600in}}%
\pgfpathclose%
\pgfusepath{fill}%
\end{pgfscope}%
\begin{pgfscope}%
\pgfpathrectangle{\pgfqpoint{1.020000in}{0.880000in}}{\pgfqpoint{6.160000in}{6.160000in}}%
\pgfusepath{clip}%
\pgfsetbuttcap%
\pgfsetroundjoin%
\definecolor{currentfill}{rgb}{0.323718,0.433158,0.864722}%
\pgfsetfillcolor{currentfill}%
\pgfsetlinewidth{0.000000pt}%
\definecolor{currentstroke}{rgb}{0.000000,0.000000,0.000000}%
\pgfsetstrokecolor{currentstroke}%
\pgfsetdash{}{0pt}%
\pgfpathmoveto{\pgfqpoint{5.382147in}{3.150099in}}%
\pgfpathlineto{\pgfqpoint{5.393581in}{3.147358in}}%
\pgfpathlineto{\pgfqpoint{5.405043in}{3.145030in}}%
\pgfpathlineto{\pgfqpoint{5.437599in}{3.161135in}}%
\pgfpathlineto{\pgfqpoint{5.470131in}{3.176309in}}%
\pgfpathlineto{\pgfqpoint{5.458653in}{3.182488in}}%
\pgfpathlineto{\pgfqpoint{5.447202in}{3.188962in}}%
\pgfpathlineto{\pgfqpoint{5.414686in}{3.170123in}}%
\pgfpathlineto{\pgfqpoint{5.382147in}{3.150099in}}%
\pgfpathclose%
\pgfusepath{fill}%
\end{pgfscope}%
\begin{pgfscope}%
\pgfpathrectangle{\pgfqpoint{1.020000in}{0.880000in}}{\pgfqpoint{6.160000in}{6.160000in}}%
\pgfusepath{clip}%
\pgfsetbuttcap%
\pgfsetroundjoin%
\definecolor{currentfill}{rgb}{0.581486,0.713451,0.998314}%
\pgfsetfillcolor{currentfill}%
\pgfsetlinewidth{0.000000pt}%
\definecolor{currentstroke}{rgb}{0.000000,0.000000,0.000000}%
\pgfsetstrokecolor{currentstroke}%
\pgfsetdash{}{0pt}%
\pgfpathmoveto{\pgfqpoint{4.192586in}{3.654071in}}%
\pgfpathlineto{\pgfqpoint{4.202818in}{3.631393in}}%
\pgfpathlineto{\pgfqpoint{4.213067in}{3.608925in}}%
\pgfpathlineto{\pgfqpoint{4.245816in}{3.592884in}}%
\pgfpathlineto{\pgfqpoint{4.278535in}{3.577593in}}%
\pgfpathlineto{\pgfqpoint{4.268241in}{3.599175in}}%
\pgfpathlineto{\pgfqpoint{4.257964in}{3.620934in}}%
\pgfpathlineto{\pgfqpoint{4.225290in}{3.636999in}}%
\pgfpathlineto{\pgfqpoint{4.192586in}{3.654071in}}%
\pgfpathclose%
\pgfusepath{fill}%
\end{pgfscope}%
\begin{pgfscope}%
\pgfpathrectangle{\pgfqpoint{1.020000in}{0.880000in}}{\pgfqpoint{6.160000in}{6.160000in}}%
\pgfusepath{clip}%
\pgfsetbuttcap%
\pgfsetroundjoin%
\definecolor{currentfill}{rgb}{0.805723,0.259813,0.230562}%
\pgfsetfillcolor{currentfill}%
\pgfsetlinewidth{0.000000pt}%
\definecolor{currentstroke}{rgb}{0.000000,0.000000,0.000000}%
\pgfsetstrokecolor{currentstroke}%
\pgfsetdash{}{0pt}%
\pgfpathmoveto{\pgfqpoint{2.900823in}{5.156855in}}%
\pgfpathlineto{\pgfqpoint{2.909409in}{5.185094in}}%
\pgfpathlineto{\pgfqpoint{2.918056in}{5.210018in}}%
\pgfpathlineto{\pgfqpoint{2.951329in}{5.202078in}}%
\pgfpathlineto{\pgfqpoint{2.984629in}{5.189353in}}%
\pgfpathlineto{\pgfqpoint{2.975904in}{5.164673in}}%
\pgfpathlineto{\pgfqpoint{2.967238in}{5.136712in}}%
\pgfpathlineto{\pgfqpoint{2.934019in}{5.148994in}}%
\pgfpathlineto{\pgfqpoint{2.900823in}{5.156855in}}%
\pgfpathclose%
\pgfusepath{fill}%
\end{pgfscope}%
\begin{pgfscope}%
\pgfpathrectangle{\pgfqpoint{1.020000in}{0.880000in}}{\pgfqpoint{6.160000in}{6.160000in}}%
\pgfusepath{clip}%
\pgfsetbuttcap%
\pgfsetroundjoin%
\definecolor{currentfill}{rgb}{0.968500,0.673977,0.556649}%
\pgfsetfillcolor{currentfill}%
\pgfsetlinewidth{0.000000pt}%
\definecolor{currentstroke}{rgb}{0.000000,0.000000,0.000000}%
\pgfsetstrokecolor{currentstroke}%
\pgfsetdash{}{0pt}%
\pgfpathmoveto{\pgfqpoint{2.161026in}{4.604664in}}%
\pgfpathlineto{\pgfqpoint{2.169166in}{4.605267in}}%
\pgfpathlineto{\pgfqpoint{2.177343in}{4.604758in}}%
\pgfpathlineto{\pgfqpoint{2.210197in}{4.634961in}}%
\pgfpathlineto{\pgfqpoint{2.243020in}{4.666748in}}%
\pgfpathlineto{\pgfqpoint{2.234838in}{4.663799in}}%
\pgfpathlineto{\pgfqpoint{2.226695in}{4.659600in}}%
\pgfpathlineto{\pgfqpoint{2.193876in}{4.631407in}}%
\pgfpathlineto{\pgfqpoint{2.161026in}{4.604664in}}%
\pgfpathclose%
\pgfusepath{fill}%
\end{pgfscope}%
\begin{pgfscope}%
\pgfpathrectangle{\pgfqpoint{1.020000in}{0.880000in}}{\pgfqpoint{6.160000in}{6.160000in}}%
\pgfusepath{clip}%
\pgfsetbuttcap%
\pgfsetroundjoin%
\definecolor{currentfill}{rgb}{0.358415,0.478426,0.896795}%
\pgfsetfillcolor{currentfill}%
\pgfsetlinewidth{0.000000pt}%
\definecolor{currentstroke}{rgb}{0.000000,0.000000,0.000000}%
\pgfsetstrokecolor{currentstroke}%
\pgfsetdash{}{0pt}%
\pgfpathmoveto{\pgfqpoint{5.664681in}{3.238418in}}%
\pgfpathlineto{\pgfqpoint{5.676254in}{3.224981in}}%
\pgfpathlineto{\pgfqpoint{5.687848in}{3.211462in}}%
\pgfpathlineto{\pgfqpoint{5.720191in}{3.215995in}}%
\pgfpathlineto{\pgfqpoint{5.752506in}{3.219785in}}%
\pgfpathlineto{\pgfqpoint{5.740874in}{3.234279in}}%
\pgfpathlineto{\pgfqpoint{5.729263in}{3.248729in}}%
\pgfpathlineto{\pgfqpoint{5.696988in}{3.244077in}}%
\pgfpathlineto{\pgfqpoint{5.664681in}{3.238418in}}%
\pgfpathclose%
\pgfusepath{fill}%
\end{pgfscope}%
\begin{pgfscope}%
\pgfpathrectangle{\pgfqpoint{1.020000in}{0.880000in}}{\pgfqpoint{6.160000in}{6.160000in}}%
\pgfusepath{clip}%
\pgfsetbuttcap%
\pgfsetroundjoin%
\definecolor{currentfill}{rgb}{0.880896,0.402331,0.317115}%
\pgfsetfillcolor{currentfill}%
\pgfsetlinewidth{0.000000pt}%
\definecolor{currentstroke}{rgb}{0.000000,0.000000,0.000000}%
\pgfsetstrokecolor{currentstroke}%
\pgfsetdash{}{0pt}%
\pgfpathmoveto{\pgfqpoint{3.202785in}{5.051728in}}%
\pgfpathlineto{\pgfqpoint{3.211932in}{5.060806in}}%
\pgfpathlineto{\pgfqpoint{3.221137in}{5.065562in}}%
\pgfpathlineto{\pgfqpoint{3.254608in}{5.021398in}}%
\pgfpathlineto{\pgfqpoint{3.288055in}{4.974325in}}%
\pgfpathlineto{\pgfqpoint{3.278781in}{4.971040in}}%
\pgfpathlineto{\pgfqpoint{3.269561in}{4.963741in}}%
\pgfpathlineto{\pgfqpoint{3.236185in}{5.009125in}}%
\pgfpathlineto{\pgfqpoint{3.202785in}{5.051728in}}%
\pgfpathclose%
\pgfusepath{fill}%
\end{pgfscope}%
\begin{pgfscope}%
\pgfpathrectangle{\pgfqpoint{1.020000in}{0.880000in}}{\pgfqpoint{6.160000in}{6.160000in}}%
\pgfusepath{clip}%
\pgfsetbuttcap%
\pgfsetroundjoin%
\definecolor{currentfill}{rgb}{0.409611,0.540759,0.935545}%
\pgfsetfillcolor{currentfill}%
\pgfsetlinewidth{0.000000pt}%
\definecolor{currentstroke}{rgb}{0.000000,0.000000,0.000000}%
\pgfsetstrokecolor{currentstroke}%
\pgfsetdash{}{0pt}%
\pgfpathmoveto{\pgfqpoint{4.581227in}{3.360537in}}%
\pgfpathlineto{\pgfqpoint{4.591794in}{3.340862in}}%
\pgfpathlineto{\pgfqpoint{4.602385in}{3.322783in}}%
\pgfpathlineto{\pgfqpoint{4.634926in}{3.298576in}}%
\pgfpathlineto{\pgfqpoint{4.667422in}{3.272426in}}%
\pgfpathlineto{\pgfqpoint{4.656787in}{3.290776in}}%
\pgfpathlineto{\pgfqpoint{4.646181in}{3.311349in}}%
\pgfpathlineto{\pgfqpoint{4.613726in}{3.336863in}}%
\pgfpathlineto{\pgfqpoint{4.581227in}{3.360537in}}%
\pgfpathclose%
\pgfusepath{fill}%
\end{pgfscope}%
\begin{pgfscope}%
\pgfpathrectangle{\pgfqpoint{1.020000in}{0.880000in}}{\pgfqpoint{6.160000in}{6.160000in}}%
\pgfusepath{clip}%
\pgfsetbuttcap%
\pgfsetroundjoin%
\definecolor{currentfill}{rgb}{0.309060,0.413498,0.850128}%
\pgfsetfillcolor{currentfill}%
\pgfsetlinewidth{0.000000pt}%
\definecolor{currentstroke}{rgb}{0.000000,0.000000,0.000000}%
\pgfsetstrokecolor{currentstroke}%
\pgfsetdash{}{0pt}%
\pgfpathmoveto{\pgfqpoint{5.317027in}{3.108575in}}%
\pgfpathlineto{\pgfqpoint{5.328442in}{3.109864in}}%
\pgfpathlineto{\pgfqpoint{5.339887in}{3.111692in}}%
\pgfpathlineto{\pgfqpoint{5.372471in}{3.128387in}}%
\pgfpathlineto{\pgfqpoint{5.405043in}{3.145030in}}%
\pgfpathlineto{\pgfqpoint{5.393581in}{3.147358in}}%
\pgfpathlineto{\pgfqpoint{5.382147in}{3.150099in}}%
\pgfpathlineto{\pgfqpoint{5.349592in}{3.129380in}}%
\pgfpathlineto{\pgfqpoint{5.317027in}{3.108575in}}%
\pgfpathclose%
\pgfusepath{fill}%
\end{pgfscope}%
\begin{pgfscope}%
\pgfpathrectangle{\pgfqpoint{1.020000in}{0.880000in}}{\pgfqpoint{6.160000in}{6.160000in}}%
\pgfusepath{clip}%
\pgfsetbuttcap%
\pgfsetroundjoin%
\definecolor{currentfill}{rgb}{0.289996,0.386836,0.828926}%
\pgfsetfillcolor{currentfill}%
\pgfsetlinewidth{0.000000pt}%
\definecolor{currentstroke}{rgb}{0.000000,0.000000,0.000000}%
\pgfsetstrokecolor{currentstroke}%
\pgfsetdash{}{0pt}%
\pgfpathmoveto{\pgfqpoint{4.818531in}{3.129886in}}%
\pgfpathlineto{\pgfqpoint{4.829374in}{3.122190in}}%
\pgfpathlineto{\pgfqpoint{4.840258in}{3.117686in}}%
\pgfpathlineto{\pgfqpoint{4.872670in}{3.092457in}}%
\pgfpathlineto{\pgfqpoint{4.905059in}{3.069074in}}%
\pgfpathlineto{\pgfqpoint{4.894105in}{3.070976in}}%
\pgfpathlineto{\pgfqpoint{4.883198in}{3.076710in}}%
\pgfpathlineto{\pgfqpoint{4.850877in}{3.102302in}}%
\pgfpathlineto{\pgfqpoint{4.818531in}{3.129886in}}%
\pgfpathclose%
\pgfusepath{fill}%
\end{pgfscope}%
\begin{pgfscope}%
\pgfpathrectangle{\pgfqpoint{1.020000in}{0.880000in}}{\pgfqpoint{6.160000in}{6.160000in}}%
\pgfusepath{clip}%
\pgfsetbuttcap%
\pgfsetroundjoin%
\definecolor{currentfill}{rgb}{0.656683,0.771806,0.994914}%
\pgfsetfillcolor{currentfill}%
\pgfsetlinewidth{0.000000pt}%
\definecolor{currentstroke}{rgb}{0.000000,0.000000,0.000000}%
\pgfsetstrokecolor{currentstroke}%
\pgfsetdash{}{0pt}%
\pgfpathmoveto{\pgfqpoint{4.041188in}{3.787786in}}%
\pgfpathlineto{\pgfqpoint{4.051305in}{3.761691in}}%
\pgfpathlineto{\pgfqpoint{4.061437in}{3.735601in}}%
\pgfpathlineto{\pgfqpoint{4.094277in}{3.712987in}}%
\pgfpathlineto{\pgfqpoint{4.127080in}{3.691928in}}%
\pgfpathlineto{\pgfqpoint{4.116907in}{3.716143in}}%
\pgfpathlineto{\pgfqpoint{4.106751in}{3.740366in}}%
\pgfpathlineto{\pgfqpoint{4.073988in}{3.763192in}}%
\pgfpathlineto{\pgfqpoint{4.041188in}{3.787786in}}%
\pgfpathclose%
\pgfusepath{fill}%
\end{pgfscope}%
\begin{pgfscope}%
\pgfpathrectangle{\pgfqpoint{1.020000in}{0.880000in}}{\pgfqpoint{6.160000in}{6.160000in}}%
\pgfusepath{clip}%
\pgfsetbuttcap%
\pgfsetroundjoin%
\definecolor{currentfill}{rgb}{0.257234,0.339661,0.789661}%
\pgfsetfillcolor{currentfill}%
\pgfsetlinewidth{0.000000pt}%
\definecolor{currentstroke}{rgb}{0.000000,0.000000,0.000000}%
\pgfsetstrokecolor{currentstroke}%
\pgfsetdash{}{0pt}%
\pgfpathmoveto{\pgfqpoint{4.969795in}{3.031269in}}%
\pgfpathlineto{\pgfqpoint{4.980869in}{3.034997in}}%
\pgfpathlineto{\pgfqpoint{4.991987in}{3.041406in}}%
\pgfpathlineto{\pgfqpoint{5.024427in}{3.030303in}}%
\pgfpathlineto{\pgfqpoint{5.056873in}{3.023316in}}%
\pgfpathlineto{\pgfqpoint{5.045681in}{3.015375in}}%
\pgfpathlineto{\pgfqpoint{5.034536in}{3.010274in}}%
\pgfpathlineto{\pgfqpoint{5.002161in}{3.018376in}}%
\pgfpathlineto{\pgfqpoint{4.969795in}{3.031269in}}%
\pgfpathclose%
\pgfusepath{fill}%
\end{pgfscope}%
\begin{pgfscope}%
\pgfpathrectangle{\pgfqpoint{1.020000in}{0.880000in}}{\pgfqpoint{6.160000in}{6.160000in}}%
\pgfusepath{clip}%
\pgfsetbuttcap%
\pgfsetroundjoin%
\definecolor{currentfill}{rgb}{0.820401,0.286765,0.245160}%
\pgfsetfillcolor{currentfill}%
\pgfsetlinewidth{0.000000pt}%
\definecolor{currentstroke}{rgb}{0.000000,0.000000,0.000000}%
\pgfsetstrokecolor{currentstroke}%
\pgfsetdash{}{0pt}%
\pgfpathmoveto{\pgfqpoint{3.051287in}{5.149992in}}%
\pgfpathlineto{\pgfqpoint{3.060155in}{5.170098in}}%
\pgfpathlineto{\pgfqpoint{3.069087in}{5.186144in}}%
\pgfpathlineto{\pgfqpoint{3.102513in}{5.158675in}}%
\pgfpathlineto{\pgfqpoint{3.135943in}{5.126890in}}%
\pgfpathlineto{\pgfqpoint{3.126933in}{5.112093in}}%
\pgfpathlineto{\pgfqpoint{3.117982in}{5.093399in}}%
\pgfpathlineto{\pgfqpoint{3.084633in}{5.123724in}}%
\pgfpathlineto{\pgfqpoint{3.051287in}{5.149992in}}%
\pgfpathclose%
\pgfusepath{fill}%
\end{pgfscope}%
\begin{pgfscope}%
\pgfpathrectangle{\pgfqpoint{1.020000in}{0.880000in}}{\pgfqpoint{6.160000in}{6.160000in}}%
\pgfusepath{clip}%
\pgfsetbuttcap%
\pgfsetroundjoin%
\definecolor{currentfill}{rgb}{0.483854,0.622050,0.974808}%
\pgfsetfillcolor{currentfill}%
\pgfsetlinewidth{0.000000pt}%
\definecolor{currentstroke}{rgb}{0.000000,0.000000,0.000000}%
\pgfsetstrokecolor{currentstroke}%
\pgfsetdash{}{0pt}%
\pgfpathmoveto{\pgfqpoint{4.429934in}{3.476861in}}%
\pgfpathlineto{\pgfqpoint{4.440374in}{3.456559in}}%
\pgfpathlineto{\pgfqpoint{4.450832in}{3.436943in}}%
\pgfpathlineto{\pgfqpoint{4.483487in}{3.420280in}}%
\pgfpathlineto{\pgfqpoint{4.516106in}{3.402141in}}%
\pgfpathlineto{\pgfqpoint{4.505603in}{3.422277in}}%
\pgfpathlineto{\pgfqpoint{4.495121in}{3.443324in}}%
\pgfpathlineto{\pgfqpoint{4.462545in}{3.460696in}}%
\pgfpathlineto{\pgfqpoint{4.429934in}{3.476861in}}%
\pgfpathclose%
\pgfusepath{fill}%
\end{pgfscope}%
\begin{pgfscope}%
\pgfpathrectangle{\pgfqpoint{1.020000in}{0.880000in}}{\pgfqpoint{6.160000in}{6.160000in}}%
\pgfusepath{clip}%
\pgfsetbuttcap%
\pgfsetroundjoin%
\definecolor{currentfill}{rgb}{0.289996,0.386836,0.828926}%
\pgfsetfillcolor{currentfill}%
\pgfsetlinewidth{0.000000pt}%
\definecolor{currentstroke}{rgb}{0.000000,0.000000,0.000000}%
\pgfsetstrokecolor{currentstroke}%
\pgfsetdash{}{0pt}%
\pgfpathmoveto{\pgfqpoint{5.251900in}{3.069574in}}%
\pgfpathlineto{\pgfqpoint{5.263288in}{3.074675in}}%
\pgfpathlineto{\pgfqpoint{5.274708in}{3.080439in}}%
\pgfpathlineto{\pgfqpoint{5.307297in}{3.095507in}}%
\pgfpathlineto{\pgfqpoint{5.339887in}{3.111692in}}%
\pgfpathlineto{\pgfqpoint{5.328442in}{3.109864in}}%
\pgfpathlineto{\pgfqpoint{5.317027in}{3.108575in}}%
\pgfpathlineto{\pgfqpoint{5.284461in}{3.088387in}}%
\pgfpathlineto{\pgfqpoint{5.251900in}{3.069574in}}%
\pgfpathclose%
\pgfusepath{fill}%
\end{pgfscope}%
\begin{pgfscope}%
\pgfpathrectangle{\pgfqpoint{1.020000in}{0.880000in}}{\pgfqpoint{6.160000in}{6.160000in}}%
\pgfusepath{clip}%
\pgfsetbuttcap%
\pgfsetroundjoin%
\definecolor{currentfill}{rgb}{0.358415,0.478426,0.896795}%
\pgfsetfillcolor{currentfill}%
\pgfsetlinewidth{0.000000pt}%
\definecolor{currentstroke}{rgb}{0.000000,0.000000,0.000000}%
\pgfsetstrokecolor{currentstroke}%
\pgfsetdash{}{0pt}%
\pgfpathmoveto{\pgfqpoint{5.599965in}{3.223449in}}%
\pgfpathlineto{\pgfqpoint{5.611508in}{3.211648in}}%
\pgfpathlineto{\pgfqpoint{5.623070in}{3.199705in}}%
\pgfpathlineto{\pgfqpoint{5.655474in}{3.206068in}}%
\pgfpathlineto{\pgfqpoint{5.687848in}{3.211462in}}%
\pgfpathlineto{\pgfqpoint{5.676254in}{3.224981in}}%
\pgfpathlineto{\pgfqpoint{5.664681in}{3.238418in}}%
\pgfpathlineto{\pgfqpoint{5.632340in}{3.231590in}}%
\pgfpathlineto{\pgfqpoint{5.599965in}{3.223449in}}%
\pgfpathclose%
\pgfusepath{fill}%
\end{pgfscope}%
\begin{pgfscope}%
\pgfpathrectangle{\pgfqpoint{1.020000in}{0.880000in}}{\pgfqpoint{6.160000in}{6.160000in}}%
\pgfusepath{clip}%
\pgfsetbuttcap%
\pgfsetroundjoin%
\definecolor{currentfill}{rgb}{0.912033,0.469680,0.366565}%
\pgfsetfillcolor{currentfill}%
\pgfsetlinewidth{0.000000pt}%
\definecolor{currentstroke}{rgb}{0.000000,0.000000,0.000000}%
\pgfsetstrokecolor{currentstroke}%
\pgfsetdash{}{0pt}%
\pgfpathmoveto{\pgfqpoint{3.288055in}{4.974325in}}%
\pgfpathlineto{\pgfqpoint{3.297383in}{4.973362in}}%
\pgfpathlineto{\pgfqpoint{3.306766in}{4.967965in}}%
\pgfpathlineto{\pgfqpoint{3.340242in}{4.917371in}}%
\pgfpathlineto{\pgfqpoint{3.373682in}{4.864762in}}%
\pgfpathlineto{\pgfqpoint{3.364243in}{4.871077in}}%
\pgfpathlineto{\pgfqpoint{3.354853in}{4.873343in}}%
\pgfpathlineto{\pgfqpoint{3.321472in}{4.924813in}}%
\pgfpathlineto{\pgfqpoint{3.288055in}{4.974325in}}%
\pgfpathclose%
\pgfusepath{fill}%
\end{pgfscope}%
\begin{pgfscope}%
\pgfpathrectangle{\pgfqpoint{1.020000in}{0.880000in}}{\pgfqpoint{6.160000in}{6.160000in}}%
\pgfusepath{clip}%
\pgfsetbuttcap%
\pgfsetroundjoin%
\definecolor{currentfill}{rgb}{0.543440,0.680003,0.993051}%
\pgfsetfillcolor{currentfill}%
\pgfsetlinewidth{0.000000pt}%
\definecolor{currentstroke}{rgb}{0.000000,0.000000,0.000000}%
\pgfsetstrokecolor{currentstroke}%
\pgfsetdash{}{0pt}%
\pgfpathmoveto{\pgfqpoint{4.278535in}{3.577593in}}%
\pgfpathlineto{\pgfqpoint{4.288847in}{3.556288in}}%
\pgfpathlineto{\pgfqpoint{4.299176in}{3.535360in}}%
\pgfpathlineto{\pgfqpoint{4.331911in}{3.521048in}}%
\pgfpathlineto{\pgfqpoint{4.364617in}{3.506728in}}%
\pgfpathlineto{\pgfqpoint{4.354241in}{3.527338in}}%
\pgfpathlineto{\pgfqpoint{4.343884in}{3.548324in}}%
\pgfpathlineto{\pgfqpoint{4.311224in}{3.562824in}}%
\pgfpathlineto{\pgfqpoint{4.278535in}{3.577593in}}%
\pgfpathclose%
\pgfusepath{fill}%
\end{pgfscope}%
\begin{pgfscope}%
\pgfpathrectangle{\pgfqpoint{1.020000in}{0.880000in}}{\pgfqpoint{6.160000in}{6.160000in}}%
\pgfusepath{clip}%
\pgfsetbuttcap%
\pgfsetroundjoin%
\definecolor{currentfill}{rgb}{0.795938,0.241845,0.220830}%
\pgfsetfillcolor{currentfill}%
\pgfsetlinewidth{0.000000pt}%
\definecolor{currentstroke}{rgb}{0.000000,0.000000,0.000000}%
\pgfsetstrokecolor{currentstroke}%
\pgfsetdash{}{0pt}%
\pgfpathmoveto{\pgfqpoint{2.834519in}{5.159231in}}%
\pgfpathlineto{\pgfqpoint{2.843032in}{5.186992in}}%
\pgfpathlineto{\pgfqpoint{2.851607in}{5.211456in}}%
\pgfpathlineto{\pgfqpoint{2.884814in}{5.213135in}}%
\pgfpathlineto{\pgfqpoint{2.918056in}{5.210018in}}%
\pgfpathlineto{\pgfqpoint{2.909409in}{5.185094in}}%
\pgfpathlineto{\pgfqpoint{2.900823in}{5.156855in}}%
\pgfpathlineto{\pgfqpoint{2.867655in}{5.160259in}}%
\pgfpathlineto{\pgfqpoint{2.834519in}{5.159231in}}%
\pgfpathclose%
\pgfusepath{fill}%
\end{pgfscope}%
\begin{pgfscope}%
\pgfpathrectangle{\pgfqpoint{1.020000in}{0.880000in}}{\pgfqpoint{6.160000in}{6.160000in}}%
\pgfusepath{clip}%
\pgfsetbuttcap%
\pgfsetroundjoin%
\definecolor{currentfill}{rgb}{0.363461,0.484784,0.901019}%
\pgfsetfillcolor{currentfill}%
\pgfsetlinewidth{0.000000pt}%
\definecolor{currentstroke}{rgb}{0.000000,0.000000,0.000000}%
\pgfsetstrokecolor{currentstroke}%
\pgfsetdash{}{0pt}%
\pgfpathmoveto{\pgfqpoint{4.667422in}{3.272426in}}%
\pgfpathlineto{\pgfqpoint{4.678086in}{3.256376in}}%
\pgfpathlineto{\pgfqpoint{4.688779in}{3.242592in}}%
\pgfpathlineto{\pgfqpoint{4.721280in}{3.215548in}}%
\pgfpathlineto{\pgfqpoint{4.753738in}{3.187322in}}%
\pgfpathlineto{\pgfqpoint{4.742991in}{3.199917in}}%
\pgfpathlineto{\pgfqpoint{4.732280in}{3.215546in}}%
\pgfpathlineto{\pgfqpoint{4.699873in}{3.244602in}}%
\pgfpathlineto{\pgfqpoint{4.667422in}{3.272426in}}%
\pgfpathclose%
\pgfusepath{fill}%
\end{pgfscope}%
\begin{pgfscope}%
\pgfpathrectangle{\pgfqpoint{1.020000in}{0.880000in}}{\pgfqpoint{6.160000in}{6.160000in}}%
\pgfusepath{clip}%
\pgfsetbuttcap%
\pgfsetroundjoin%
\definecolor{currentfill}{rgb}{0.353369,0.472069,0.892570}%
\pgfsetfillcolor{currentfill}%
\pgfsetlinewidth{0.000000pt}%
\definecolor{currentstroke}{rgb}{0.000000,0.000000,0.000000}%
\pgfsetstrokecolor{currentstroke}%
\pgfsetdash{}{0pt}%
\pgfpathmoveto{\pgfqpoint{5.817052in}{3.225618in}}%
\pgfpathlineto{\pgfqpoint{5.828748in}{3.210552in}}%
\pgfpathlineto{\pgfqpoint{5.840464in}{3.195441in}}%
\pgfpathlineto{\pgfqpoint{5.872745in}{3.197441in}}%
\pgfpathlineto{\pgfqpoint{5.861005in}{3.212676in}}%
\pgfpathlineto{\pgfqpoint{5.849287in}{3.227881in}}%
\pgfpathlineto{\pgfqpoint{5.817052in}{3.225618in}}%
\pgfpathclose%
\pgfusepath{fill}%
\end{pgfscope}%
\begin{pgfscope}%
\pgfpathrectangle{\pgfqpoint{1.020000in}{0.880000in}}{\pgfqpoint{6.160000in}{6.160000in}}%
\pgfusepath{clip}%
\pgfsetbuttcap%
\pgfsetroundjoin%
\definecolor{currentfill}{rgb}{0.879622,0.858175,0.845844}%
\pgfsetfillcolor{currentfill}%
\pgfsetlinewidth{0.000000pt}%
\definecolor{currentstroke}{rgb}{0.000000,0.000000,0.000000}%
\pgfsetstrokecolor{currentstroke}%
\pgfsetdash{}{0pt}%
\pgfpathmoveto{\pgfqpoint{3.717596in}{4.256954in}}%
\pgfpathlineto{\pgfqpoint{3.727488in}{4.225713in}}%
\pgfpathlineto{\pgfqpoint{3.737401in}{4.192806in}}%
\pgfpathlineto{\pgfqpoint{3.770517in}{4.148024in}}%
\pgfpathlineto{\pgfqpoint{3.803579in}{4.104941in}}%
\pgfpathlineto{\pgfqpoint{3.793642in}{4.135245in}}%
\pgfpathlineto{\pgfqpoint{3.783725in}{4.164152in}}%
\pgfpathlineto{\pgfqpoint{3.750689in}{4.209650in}}%
\pgfpathlineto{\pgfqpoint{3.717596in}{4.256954in}}%
\pgfpathclose%
\pgfusepath{fill}%
\end{pgfscope}%
\begin{pgfscope}%
\pgfpathrectangle{\pgfqpoint{1.020000in}{0.880000in}}{\pgfqpoint{6.160000in}{6.160000in}}%
\pgfusepath{clip}%
\pgfsetbuttcap%
\pgfsetroundjoin%
\definecolor{currentfill}{rgb}{0.935774,0.812237,0.747156}%
\pgfsetfillcolor{currentfill}%
\pgfsetlinewidth{0.000000pt}%
\definecolor{currentstroke}{rgb}{0.000000,0.000000,0.000000}%
\pgfsetstrokecolor{currentstroke}%
\pgfsetdash{}{0pt}%
\pgfpathmoveto{\pgfqpoint{3.631575in}{4.416756in}}%
\pgfpathlineto{\pgfqpoint{3.641394in}{4.387595in}}%
\pgfpathlineto{\pgfqpoint{3.651240in}{4.356033in}}%
\pgfpathlineto{\pgfqpoint{3.684447in}{4.305837in}}%
\pgfpathlineto{\pgfqpoint{3.717596in}{4.256954in}}%
\pgfpathlineto{\pgfqpoint{3.707727in}{4.286324in}}%
\pgfpathlineto{\pgfqpoint{3.697882in}{4.313642in}}%
\pgfpathlineto{\pgfqpoint{3.664758in}{4.364512in}}%
\pgfpathlineto{\pgfqpoint{3.631575in}{4.416756in}}%
\pgfpathclose%
\pgfusepath{fill}%
\end{pgfscope}%
\begin{pgfscope}%
\pgfpathrectangle{\pgfqpoint{1.020000in}{0.880000in}}{\pgfqpoint{6.160000in}{6.160000in}}%
\pgfusepath{clip}%
\pgfsetbuttcap%
\pgfsetroundjoin%
\definecolor{currentfill}{rgb}{0.945854,0.559565,0.441513}%
\pgfsetfillcolor{currentfill}%
\pgfsetlinewidth{0.000000pt}%
\definecolor{currentstroke}{rgb}{0.000000,0.000000,0.000000}%
\pgfsetstrokecolor{currentstroke}%
\pgfsetdash{}{0pt}%
\pgfpathmoveto{\pgfqpoint{3.373682in}{4.864762in}}%
\pgfpathlineto{\pgfqpoint{3.383169in}{4.854277in}}%
\pgfpathlineto{\pgfqpoint{3.392705in}{4.839557in}}%
\pgfpathlineto{\pgfqpoint{3.426149in}{4.785079in}}%
\pgfpathlineto{\pgfqpoint{3.459546in}{4.729558in}}%
\pgfpathlineto{\pgfqpoint{3.449967in}{4.744421in}}%
\pgfpathlineto{\pgfqpoint{3.440429in}{4.755477in}}%
\pgfpathlineto{\pgfqpoint{3.407079in}{4.810633in}}%
\pgfpathlineto{\pgfqpoint{3.373682in}{4.864762in}}%
\pgfpathclose%
\pgfusepath{fill}%
\end{pgfscope}%
\begin{pgfscope}%
\pgfpathrectangle{\pgfqpoint{1.020000in}{0.880000in}}{\pgfqpoint{6.160000in}{6.160000in}}%
\pgfusepath{clip}%
\pgfsetbuttcap%
\pgfsetroundjoin%
\definecolor{currentfill}{rgb}{0.813693,0.854282,0.918480}%
\pgfsetfillcolor{currentfill}%
\pgfsetlinewidth{0.000000pt}%
\definecolor{currentstroke}{rgb}{0.000000,0.000000,0.000000}%
\pgfsetstrokecolor{currentstroke}%
\pgfsetdash{}{0pt}%
\pgfpathmoveto{\pgfqpoint{3.803579in}{4.104941in}}%
\pgfpathlineto{\pgfqpoint{3.813535in}{4.073433in}}%
\pgfpathlineto{\pgfqpoint{3.823509in}{4.040930in}}%
\pgfpathlineto{\pgfqpoint{3.856543in}{4.002450in}}%
\pgfpathlineto{\pgfqpoint{3.889528in}{3.965834in}}%
\pgfpathlineto{\pgfqpoint{3.879527in}{3.995595in}}%
\pgfpathlineto{\pgfqpoint{3.869544in}{4.024538in}}%
\pgfpathlineto{\pgfqpoint{3.836587in}{4.063732in}}%
\pgfpathlineto{\pgfqpoint{3.803579in}{4.104941in}}%
\pgfpathclose%
\pgfusepath{fill}%
\end{pgfscope}%
\begin{pgfscope}%
\pgfpathrectangle{\pgfqpoint{1.020000in}{0.880000in}}{\pgfqpoint{6.160000in}{6.160000in}}%
\pgfusepath{clip}%
\pgfsetbuttcap%
\pgfsetroundjoin%
\definecolor{currentfill}{rgb}{0.275827,0.366717,0.812553}%
\pgfsetfillcolor{currentfill}%
\pgfsetlinewidth{0.000000pt}%
\definecolor{currentstroke}{rgb}{0.000000,0.000000,0.000000}%
\pgfsetstrokecolor{currentstroke}%
\pgfsetdash{}{0pt}%
\pgfpathmoveto{\pgfqpoint{5.186815in}{3.039123in}}%
\pgfpathlineto{\pgfqpoint{5.198166in}{3.047223in}}%
\pgfpathlineto{\pgfqpoint{5.209549in}{3.056087in}}%
\pgfpathlineto{\pgfqpoint{5.242124in}{3.067103in}}%
\pgfpathlineto{\pgfqpoint{5.274708in}{3.080439in}}%
\pgfpathlineto{\pgfqpoint{5.263288in}{3.074675in}}%
\pgfpathlineto{\pgfqpoint{5.251900in}{3.069574in}}%
\pgfpathlineto{\pgfqpoint{5.219349in}{3.052909in}}%
\pgfpathlineto{\pgfqpoint{5.186815in}{3.039123in}}%
\pgfpathclose%
\pgfusepath{fill}%
\end{pgfscope}%
\begin{pgfscope}%
\pgfpathrectangle{\pgfqpoint{1.020000in}{0.880000in}}{\pgfqpoint{6.160000in}{6.160000in}}%
\pgfusepath{clip}%
\pgfsetbuttcap%
\pgfsetroundjoin%
\definecolor{currentfill}{rgb}{0.964835,0.744614,0.643239}%
\pgfsetfillcolor{currentfill}%
\pgfsetlinewidth{0.000000pt}%
\definecolor{currentstroke}{rgb}{0.000000,0.000000,0.000000}%
\pgfsetstrokecolor{currentstroke}%
\pgfsetdash{}{0pt}%
\pgfpathmoveto{\pgfqpoint{3.545540in}{4.577172in}}%
\pgfpathlineto{\pgfqpoint{3.555271in}{4.552185in}}%
\pgfpathlineto{\pgfqpoint{3.565035in}{4.524062in}}%
\pgfpathlineto{\pgfqpoint{3.598334in}{4.470058in}}%
\pgfpathlineto{\pgfqpoint{3.631575in}{4.416756in}}%
\pgfpathlineto{\pgfqpoint{3.621784in}{4.443337in}}%
\pgfpathlineto{\pgfqpoint{3.612023in}{4.467193in}}%
\pgfpathlineto{\pgfqpoint{3.578811in}{4.521820in}}%
\pgfpathlineto{\pgfqpoint{3.545540in}{4.577172in}}%
\pgfpathclose%
\pgfusepath{fill}%
\end{pgfscope}%
\begin{pgfscope}%
\pgfpathrectangle{\pgfqpoint{1.020000in}{0.880000in}}{\pgfqpoint{6.160000in}{6.160000in}}%
\pgfusepath{clip}%
\pgfsetbuttcap%
\pgfsetroundjoin%
\definecolor{currentfill}{rgb}{0.608547,0.735725,0.999354}%
\pgfsetfillcolor{currentfill}%
\pgfsetlinewidth{0.000000pt}%
\definecolor{currentstroke}{rgb}{0.000000,0.000000,0.000000}%
\pgfsetstrokecolor{currentstroke}%
\pgfsetdash{}{0pt}%
\pgfpathmoveto{\pgfqpoint{4.127080in}{3.691928in}}%
\pgfpathlineto{\pgfqpoint{4.137269in}{3.667842in}}%
\pgfpathlineto{\pgfqpoint{4.147475in}{3.644003in}}%
\pgfpathlineto{\pgfqpoint{4.180287in}{3.625910in}}%
\pgfpathlineto{\pgfqpoint{4.213067in}{3.608925in}}%
\pgfpathlineto{\pgfqpoint{4.202818in}{3.631393in}}%
\pgfpathlineto{\pgfqpoint{4.192586in}{3.654071in}}%
\pgfpathlineto{\pgfqpoint{4.159849in}{3.672330in}}%
\pgfpathlineto{\pgfqpoint{4.127080in}{3.691928in}}%
\pgfpathclose%
\pgfusepath{fill}%
\end{pgfscope}%
\begin{pgfscope}%
\pgfpathrectangle{\pgfqpoint{1.020000in}{0.880000in}}{\pgfqpoint{6.160000in}{6.160000in}}%
\pgfusepath{clip}%
\pgfsetbuttcap%
\pgfsetroundjoin%
\definecolor{currentfill}{rgb}{0.967317,0.657471,0.538160}%
\pgfsetfillcolor{currentfill}%
\pgfsetlinewidth{0.000000pt}%
\definecolor{currentstroke}{rgb}{0.000000,0.000000,0.000000}%
\pgfsetstrokecolor{currentstroke}%
\pgfsetdash{}{0pt}%
\pgfpathmoveto{\pgfqpoint{3.459546in}{4.729558in}}%
\pgfpathlineto{\pgfqpoint{3.469168in}{4.710880in}}%
\pgfpathlineto{\pgfqpoint{3.478831in}{4.688425in}}%
\pgfpathlineto{\pgfqpoint{3.512213in}{4.632851in}}%
\pgfpathlineto{\pgfqpoint{3.545540in}{4.577172in}}%
\pgfpathlineto{\pgfqpoint{3.535844in}{4.598902in}}%
\pgfpathlineto{\pgfqpoint{3.526183in}{4.617295in}}%
\pgfpathlineto{\pgfqpoint{3.492892in}{4.673477in}}%
\pgfpathlineto{\pgfqpoint{3.459546in}{4.729558in}}%
\pgfpathclose%
\pgfusepath{fill}%
\end{pgfscope}%
\begin{pgfscope}%
\pgfpathrectangle{\pgfqpoint{1.020000in}{0.880000in}}{\pgfqpoint{6.160000in}{6.160000in}}%
\pgfusepath{clip}%
\pgfsetbuttcap%
\pgfsetroundjoin%
\definecolor{currentfill}{rgb}{0.353369,0.472069,0.892570}%
\pgfsetfillcolor{currentfill}%
\pgfsetlinewidth{0.000000pt}%
\definecolor{currentstroke}{rgb}{0.000000,0.000000,0.000000}%
\pgfsetstrokecolor{currentstroke}%
\pgfsetdash{}{0pt}%
\pgfpathmoveto{\pgfqpoint{5.535113in}{3.202817in}}%
\pgfpathlineto{\pgfqpoint{5.546631in}{3.193410in}}%
\pgfpathlineto{\pgfqpoint{5.558169in}{3.183772in}}%
\pgfpathlineto{\pgfqpoint{5.590635in}{3.192288in}}%
\pgfpathlineto{\pgfqpoint{5.623070in}{3.199705in}}%
\pgfpathlineto{\pgfqpoint{5.611508in}{3.211648in}}%
\pgfpathlineto{\pgfqpoint{5.599965in}{3.223449in}}%
\pgfpathlineto{\pgfqpoint{5.567556in}{3.213879in}}%
\pgfpathlineto{\pgfqpoint{5.535113in}{3.202817in}}%
\pgfpathclose%
\pgfusepath{fill}%
\end{pgfscope}%
\begin{pgfscope}%
\pgfpathrectangle{\pgfqpoint{1.020000in}{0.880000in}}{\pgfqpoint{6.160000in}{6.160000in}}%
\pgfusepath{clip}%
\pgfsetbuttcap%
\pgfsetroundjoin%
\definecolor{currentfill}{rgb}{0.748682,0.827679,0.963334}%
\pgfsetfillcolor{currentfill}%
\pgfsetlinewidth{0.000000pt}%
\definecolor{currentstroke}{rgb}{0.000000,0.000000,0.000000}%
\pgfsetstrokecolor{currentstroke}%
\pgfsetdash{}{0pt}%
\pgfpathmoveto{\pgfqpoint{3.889528in}{3.965834in}}%
\pgfpathlineto{\pgfqpoint{3.899545in}{3.935442in}}%
\pgfpathlineto{\pgfqpoint{3.909579in}{3.904616in}}%
\pgfpathlineto{\pgfqpoint{3.942546in}{3.872633in}}%
\pgfpathlineto{\pgfqpoint{3.975468in}{3.842488in}}%
\pgfpathlineto{\pgfqpoint{3.965403in}{3.870688in}}%
\pgfpathlineto{\pgfqpoint{3.955354in}{3.898542in}}%
\pgfpathlineto{\pgfqpoint{3.922464in}{3.931177in}}%
\pgfpathlineto{\pgfqpoint{3.889528in}{3.965834in}}%
\pgfpathclose%
\pgfusepath{fill}%
\end{pgfscope}%
\begin{pgfscope}%
\pgfpathrectangle{\pgfqpoint{1.020000in}{0.880000in}}{\pgfqpoint{6.160000in}{6.160000in}}%
\pgfusepath{clip}%
\pgfsetbuttcap%
\pgfsetroundjoin%
\definecolor{currentfill}{rgb}{0.353369,0.472069,0.892570}%
\pgfsetfillcolor{currentfill}%
\pgfsetlinewidth{0.000000pt}%
\definecolor{currentstroke}{rgb}{0.000000,0.000000,0.000000}%
\pgfsetstrokecolor{currentstroke}%
\pgfsetdash{}{0pt}%
\pgfpathmoveto{\pgfqpoint{5.752506in}{3.219785in}}%
\pgfpathlineto{\pgfqpoint{5.764158in}{3.205217in}}%
\pgfpathlineto{\pgfqpoint{5.775830in}{3.190553in}}%
\pgfpathlineto{\pgfqpoint{5.808159in}{3.193169in}}%
\pgfpathlineto{\pgfqpoint{5.840464in}{3.195441in}}%
\pgfpathlineto{\pgfqpoint{5.828748in}{3.210552in}}%
\pgfpathlineto{\pgfqpoint{5.817052in}{3.225618in}}%
\pgfpathlineto{\pgfqpoint{5.784792in}{3.222954in}}%
\pgfpathlineto{\pgfqpoint{5.752506in}{3.219785in}}%
\pgfpathclose%
\pgfusepath{fill}%
\end{pgfscope}%
\begin{pgfscope}%
\pgfpathrectangle{\pgfqpoint{1.020000in}{0.880000in}}{\pgfqpoint{6.160000in}{6.160000in}}%
\pgfusepath{clip}%
\pgfsetbuttcap%
\pgfsetroundjoin%
\definecolor{currentfill}{rgb}{0.441123,0.576532,0.954545}%
\pgfsetfillcolor{currentfill}%
\pgfsetlinewidth{0.000000pt}%
\definecolor{currentstroke}{rgb}{0.000000,0.000000,0.000000}%
\pgfsetstrokecolor{currentstroke}%
\pgfsetdash{}{0pt}%
\pgfpathmoveto{\pgfqpoint{4.516106in}{3.402141in}}%
\pgfpathlineto{\pgfqpoint{4.526630in}{3.383053in}}%
\pgfpathlineto{\pgfqpoint{4.537176in}{3.365110in}}%
\pgfpathlineto{\pgfqpoint{4.569801in}{3.344947in}}%
\pgfpathlineto{\pgfqpoint{4.602385in}{3.322783in}}%
\pgfpathlineto{\pgfqpoint{4.591794in}{3.340862in}}%
\pgfpathlineto{\pgfqpoint{4.581227in}{3.360537in}}%
\pgfpathlineto{\pgfqpoint{4.548686in}{3.382280in}}%
\pgfpathlineto{\pgfqpoint{4.516106in}{3.402141in}}%
\pgfpathclose%
\pgfusepath{fill}%
\end{pgfscope}%
\begin{pgfscope}%
\pgfpathrectangle{\pgfqpoint{1.020000in}{0.880000in}}{\pgfqpoint{6.160000in}{6.160000in}}%
\pgfusepath{clip}%
\pgfsetbuttcap%
\pgfsetroundjoin%
\definecolor{currentfill}{rgb}{0.795938,0.241845,0.220830}%
\pgfsetfillcolor{currentfill}%
\pgfsetlinewidth{0.000000pt}%
\definecolor{currentstroke}{rgb}{0.000000,0.000000,0.000000}%
\pgfsetstrokecolor{currentstroke}%
\pgfsetdash{}{0pt}%
\pgfpathmoveto{\pgfqpoint{2.768353in}{5.144254in}}%
\pgfpathlineto{\pgfqpoint{2.776801in}{5.170800in}}%
\pgfpathlineto{\pgfqpoint{2.785313in}{5.194115in}}%
\pgfpathlineto{\pgfqpoint{2.818439in}{5.205067in}}%
\pgfpathlineto{\pgfqpoint{2.851607in}{5.211456in}}%
\pgfpathlineto{\pgfqpoint{2.843032in}{5.186992in}}%
\pgfpathlineto{\pgfqpoint{2.834519in}{5.159231in}}%
\pgfpathlineto{\pgfqpoint{2.801417in}{5.153851in}}%
\pgfpathlineto{\pgfqpoint{2.768353in}{5.144254in}}%
\pgfpathclose%
\pgfusepath{fill}%
\end{pgfscope}%
\begin{pgfscope}%
\pgfpathrectangle{\pgfqpoint{1.020000in}{0.880000in}}{\pgfqpoint{6.160000in}{6.160000in}}%
\pgfusepath{clip}%
\pgfsetbuttcap%
\pgfsetroundjoin%
\definecolor{currentfill}{rgb}{0.275827,0.366717,0.812553}%
\pgfsetfillcolor{currentfill}%
\pgfsetlinewidth{0.000000pt}%
\definecolor{currentstroke}{rgb}{0.000000,0.000000,0.000000}%
\pgfsetstrokecolor{currentstroke}%
\pgfsetdash{}{0pt}%
\pgfpathmoveto{\pgfqpoint{4.905059in}{3.069074in}}%
\pgfpathlineto{\pgfqpoint{4.916057in}{3.070333in}}%
\pgfpathlineto{\pgfqpoint{4.927095in}{3.073975in}}%
\pgfpathlineto{\pgfqpoint{4.959545in}{3.056162in}}%
\pgfpathlineto{\pgfqpoint{4.991987in}{3.041406in}}%
\pgfpathlineto{\pgfqpoint{4.980869in}{3.034997in}}%
\pgfpathlineto{\pgfqpoint{4.969795in}{3.031269in}}%
\pgfpathlineto{\pgfqpoint{4.937431in}{3.048405in}}%
\pgfpathlineto{\pgfqpoint{4.905059in}{3.069074in}}%
\pgfpathclose%
\pgfusepath{fill}%
\end{pgfscope}%
\begin{pgfscope}%
\pgfpathrectangle{\pgfqpoint{1.020000in}{0.880000in}}{\pgfqpoint{6.160000in}{6.160000in}}%
\pgfusepath{clip}%
\pgfsetbuttcap%
\pgfsetroundjoin%
\definecolor{currentfill}{rgb}{0.266381,0.353304,0.801637}%
\pgfsetfillcolor{currentfill}%
\pgfsetlinewidth{0.000000pt}%
\definecolor{currentstroke}{rgb}{0.000000,0.000000,0.000000}%
\pgfsetstrokecolor{currentstroke}%
\pgfsetdash{}{0pt}%
\pgfpathmoveto{\pgfqpoint{5.121806in}{3.022615in}}%
\pgfpathlineto{\pgfqpoint{5.133105in}{3.032377in}}%
\pgfpathlineto{\pgfqpoint{5.144436in}{3.042963in}}%
\pgfpathlineto{\pgfqpoint{5.176986in}{3.047905in}}%
\pgfpathlineto{\pgfqpoint{5.209549in}{3.056087in}}%
\pgfpathlineto{\pgfqpoint{5.198166in}{3.047223in}}%
\pgfpathlineto{\pgfqpoint{5.186815in}{3.039123in}}%
\pgfpathlineto{\pgfqpoint{5.154300in}{3.028857in}}%
\pgfpathlineto{\pgfqpoint{5.121806in}{3.022615in}}%
\pgfpathclose%
\pgfusepath{fill}%
\end{pgfscope}%
\begin{pgfscope}%
\pgfpathrectangle{\pgfqpoint{1.020000in}{0.880000in}}{\pgfqpoint{6.160000in}{6.160000in}}%
\pgfusepath{clip}%
\pgfsetbuttcap%
\pgfsetroundjoin%
\definecolor{currentfill}{rgb}{0.688188,0.793178,0.988038}%
\pgfsetfillcolor{currentfill}%
\pgfsetlinewidth{0.000000pt}%
\definecolor{currentstroke}{rgb}{0.000000,0.000000,0.000000}%
\pgfsetstrokecolor{currentstroke}%
\pgfsetdash{}{0pt}%
\pgfpathmoveto{\pgfqpoint{3.975468in}{3.842488in}}%
\pgfpathlineto{\pgfqpoint{3.985549in}{3.814109in}}%
\pgfpathlineto{\pgfqpoint{3.995645in}{3.785722in}}%
\pgfpathlineto{\pgfqpoint{4.028561in}{3.759834in}}%
\pgfpathlineto{\pgfqpoint{4.061437in}{3.735601in}}%
\pgfpathlineto{\pgfqpoint{4.051305in}{3.761691in}}%
\pgfpathlineto{\pgfqpoint{4.041188in}{3.787786in}}%
\pgfpathlineto{\pgfqpoint{4.008348in}{3.814207in}}%
\pgfpathlineto{\pgfqpoint{3.975468in}{3.842488in}}%
\pgfpathclose%
\pgfusepath{fill}%
\end{pgfscope}%
\begin{pgfscope}%
\pgfpathrectangle{\pgfqpoint{1.020000in}{0.880000in}}{\pgfqpoint{6.160000in}{6.160000in}}%
\pgfusepath{clip}%
\pgfsetbuttcap%
\pgfsetroundjoin%
\definecolor{currentfill}{rgb}{0.835027,0.313644,0.259783}%
\pgfsetfillcolor{currentfill}%
\pgfsetlinewidth{0.000000pt}%
\definecolor{currentstroke}{rgb}{0.000000,0.000000,0.000000}%
\pgfsetstrokecolor{currentstroke}%
\pgfsetdash{}{0pt}%
\pgfpathmoveto{\pgfqpoint{3.135943in}{5.126890in}}%
\pgfpathlineto{\pgfqpoint{3.145014in}{5.137429in}}%
\pgfpathlineto{\pgfqpoint{3.154150in}{5.143386in}}%
\pgfpathlineto{\pgfqpoint{3.187648in}{5.106366in}}%
\pgfpathlineto{\pgfqpoint{3.221137in}{5.065562in}}%
\pgfpathlineto{\pgfqpoint{3.211932in}{5.060806in}}%
\pgfpathlineto{\pgfqpoint{3.202785in}{5.051728in}}%
\pgfpathlineto{\pgfqpoint{3.169369in}{5.091118in}}%
\pgfpathlineto{\pgfqpoint{3.135943in}{5.126890in}}%
\pgfpathclose%
\pgfusepath{fill}%
\end{pgfscope}%
\begin{pgfscope}%
\pgfpathrectangle{\pgfqpoint{1.020000in}{0.880000in}}{\pgfqpoint{6.160000in}{6.160000in}}%
\pgfusepath{clip}%
\pgfsetbuttcap%
\pgfsetroundjoin%
\definecolor{currentfill}{rgb}{0.328604,0.439712,0.869587}%
\pgfsetfillcolor{currentfill}%
\pgfsetlinewidth{0.000000pt}%
\definecolor{currentstroke}{rgb}{0.000000,0.000000,0.000000}%
\pgfsetstrokecolor{currentstroke}%
\pgfsetdash{}{0pt}%
\pgfpathmoveto{\pgfqpoint{4.753738in}{3.187322in}}%
\pgfpathlineto{\pgfqpoint{4.764519in}{3.177566in}}%
\pgfpathlineto{\pgfqpoint{4.775336in}{3.170329in}}%
\pgfpathlineto{\pgfqpoint{4.807815in}{3.143906in}}%
\pgfpathlineto{\pgfqpoint{4.840258in}{3.117686in}}%
\pgfpathlineto{\pgfqpoint{4.829374in}{3.122190in}}%
\pgfpathlineto{\pgfqpoint{4.818531in}{3.129886in}}%
\pgfpathlineto{\pgfqpoint{4.786153in}{3.158517in}}%
\pgfpathlineto{\pgfqpoint{4.753738in}{3.187322in}}%
\pgfpathclose%
\pgfusepath{fill}%
\end{pgfscope}%
\begin{pgfscope}%
\pgfpathrectangle{\pgfqpoint{1.020000in}{0.880000in}}{\pgfqpoint{6.160000in}{6.160000in}}%
\pgfusepath{clip}%
\pgfsetbuttcap%
\pgfsetroundjoin%
\definecolor{currentfill}{rgb}{0.505423,0.643995,0.983157}%
\pgfsetfillcolor{currentfill}%
\pgfsetlinewidth{0.000000pt}%
\definecolor{currentstroke}{rgb}{0.000000,0.000000,0.000000}%
\pgfsetstrokecolor{currentstroke}%
\pgfsetdash{}{0pt}%
\pgfpathmoveto{\pgfqpoint{4.364617in}{3.506728in}}%
\pgfpathlineto{\pgfqpoint{4.375010in}{3.486599in}}%
\pgfpathlineto{\pgfqpoint{4.385423in}{3.467047in}}%
\pgfpathlineto{\pgfqpoint{4.418144in}{3.452424in}}%
\pgfpathlineto{\pgfqpoint{4.450832in}{3.436943in}}%
\pgfpathlineto{\pgfqpoint{4.440374in}{3.456559in}}%
\pgfpathlineto{\pgfqpoint{4.429934in}{3.476861in}}%
\pgfpathlineto{\pgfqpoint{4.397291in}{3.492102in}}%
\pgfpathlineto{\pgfqpoint{4.364617in}{3.506728in}}%
\pgfpathclose%
\pgfusepath{fill}%
\end{pgfscope}%
\begin{pgfscope}%
\pgfpathrectangle{\pgfqpoint{1.020000in}{0.880000in}}{\pgfqpoint{6.160000in}{6.160000in}}%
\pgfusepath{clip}%
\pgfsetbuttcap%
\pgfsetroundjoin%
\definecolor{currentfill}{rgb}{0.343278,0.459354,0.884122}%
\pgfsetfillcolor{currentfill}%
\pgfsetlinewidth{0.000000pt}%
\definecolor{currentstroke}{rgb}{0.000000,0.000000,0.000000}%
\pgfsetstrokecolor{currentstroke}%
\pgfsetdash{}{0pt}%
\pgfpathmoveto{\pgfqpoint{5.470131in}{3.176309in}}%
\pgfpathlineto{\pgfqpoint{5.481631in}{3.170084in}}%
\pgfpathlineto{\pgfqpoint{5.493150in}{3.163514in}}%
\pgfpathlineto{\pgfqpoint{5.525673in}{3.174159in}}%
\pgfpathlineto{\pgfqpoint{5.558169in}{3.183772in}}%
\pgfpathlineto{\pgfqpoint{5.546631in}{3.193410in}}%
\pgfpathlineto{\pgfqpoint{5.535113in}{3.202817in}}%
\pgfpathlineto{\pgfqpoint{5.502637in}{3.190265in}}%
\pgfpathlineto{\pgfqpoint{5.470131in}{3.176309in}}%
\pgfpathclose%
\pgfusepath{fill}%
\end{pgfscope}%
\begin{pgfscope}%
\pgfpathrectangle{\pgfqpoint{1.020000in}{0.880000in}}{\pgfqpoint{6.160000in}{6.160000in}}%
\pgfusepath{clip}%
\pgfsetbuttcap%
\pgfsetroundjoin%
\definecolor{currentfill}{rgb}{0.785153,0.220851,0.211673}%
\pgfsetfillcolor{currentfill}%
\pgfsetlinewidth{0.000000pt}%
\definecolor{currentstroke}{rgb}{0.000000,0.000000,0.000000}%
\pgfsetstrokecolor{currentstroke}%
\pgfsetdash{}{0pt}%
\pgfpathmoveto{\pgfqpoint{2.984629in}{5.189353in}}%
\pgfpathlineto{\pgfqpoint{2.993417in}{5.210294in}}%
\pgfpathlineto{\pgfqpoint{3.002273in}{5.227067in}}%
\pgfpathlineto{\pgfqpoint{3.035671in}{5.209016in}}%
\pgfpathlineto{\pgfqpoint{3.069087in}{5.186144in}}%
\pgfpathlineto{\pgfqpoint{3.060155in}{5.170098in}}%
\pgfpathlineto{\pgfqpoint{3.051287in}{5.149992in}}%
\pgfpathlineto{\pgfqpoint{3.017951in}{5.171938in}}%
\pgfpathlineto{\pgfqpoint{2.984629in}{5.189353in}}%
\pgfpathclose%
\pgfusepath{fill}%
\end{pgfscope}%
\begin{pgfscope}%
\pgfpathrectangle{\pgfqpoint{1.020000in}{0.880000in}}{\pgfqpoint{6.160000in}{6.160000in}}%
\pgfusepath{clip}%
\pgfsetbuttcap%
\pgfsetroundjoin%
\definecolor{currentfill}{rgb}{0.565182,0.699438,0.996635}%
\pgfsetfillcolor{currentfill}%
\pgfsetlinewidth{0.000000pt}%
\definecolor{currentstroke}{rgb}{0.000000,0.000000,0.000000}%
\pgfsetstrokecolor{currentstroke}%
\pgfsetdash{}{0pt}%
\pgfpathmoveto{\pgfqpoint{4.213067in}{3.608925in}}%
\pgfpathlineto{\pgfqpoint{4.223333in}{3.586770in}}%
\pgfpathlineto{\pgfqpoint{4.233617in}{3.565030in}}%
\pgfpathlineto{\pgfqpoint{4.266412in}{3.549940in}}%
\pgfpathlineto{\pgfqpoint{4.299176in}{3.535360in}}%
\pgfpathlineto{\pgfqpoint{4.288847in}{3.556288in}}%
\pgfpathlineto{\pgfqpoint{4.278535in}{3.577593in}}%
\pgfpathlineto{\pgfqpoint{4.245816in}{3.592884in}}%
\pgfpathlineto{\pgfqpoint{4.213067in}{3.608925in}}%
\pgfpathclose%
\pgfusepath{fill}%
\end{pgfscope}%
\begin{pgfscope}%
\pgfpathrectangle{\pgfqpoint{1.020000in}{0.880000in}}{\pgfqpoint{6.160000in}{6.160000in}}%
\pgfusepath{clip}%
\pgfsetbuttcap%
\pgfsetroundjoin%
\definecolor{currentfill}{rgb}{0.805723,0.259813,0.230562}%
\pgfsetfillcolor{currentfill}%
\pgfsetlinewidth{0.000000pt}%
\definecolor{currentstroke}{rgb}{0.000000,0.000000,0.000000}%
\pgfsetstrokecolor{currentstroke}%
\pgfsetdash{}{0pt}%
\pgfpathmoveto{\pgfqpoint{2.702338in}{5.113210in}}%
\pgfpathlineto{\pgfqpoint{2.710731in}{5.137857in}}%
\pgfpathlineto{\pgfqpoint{2.719188in}{5.159385in}}%
\pgfpathlineto{\pgfqpoint{2.752229in}{5.178803in}}%
\pgfpathlineto{\pgfqpoint{2.785313in}{5.194115in}}%
\pgfpathlineto{\pgfqpoint{2.776801in}{5.170800in}}%
\pgfpathlineto{\pgfqpoint{2.768353in}{5.144254in}}%
\pgfpathlineto{\pgfqpoint{2.735326in}{5.130628in}}%
\pgfpathlineto{\pgfqpoint{2.702338in}{5.113210in}}%
\pgfpathclose%
\pgfusepath{fill}%
\end{pgfscope}%
\begin{pgfscope}%
\pgfpathrectangle{\pgfqpoint{1.020000in}{0.880000in}}{\pgfqpoint{6.160000in}{6.160000in}}%
\pgfusepath{clip}%
\pgfsetbuttcap%
\pgfsetroundjoin%
\definecolor{currentfill}{rgb}{0.353369,0.472069,0.892570}%
\pgfsetfillcolor{currentfill}%
\pgfsetlinewidth{0.000000pt}%
\definecolor{currentstroke}{rgb}{0.000000,0.000000,0.000000}%
\pgfsetstrokecolor{currentstroke}%
\pgfsetdash{}{0pt}%
\pgfpathmoveto{\pgfqpoint{5.687848in}{3.211462in}}%
\pgfpathlineto{\pgfqpoint{5.699461in}{3.197803in}}%
\pgfpathlineto{\pgfqpoint{5.711093in}{3.183961in}}%
\pgfpathlineto{\pgfqpoint{5.743475in}{3.187511in}}%
\pgfpathlineto{\pgfqpoint{5.775830in}{3.190553in}}%
\pgfpathlineto{\pgfqpoint{5.764158in}{3.205217in}}%
\pgfpathlineto{\pgfqpoint{5.752506in}{3.219785in}}%
\pgfpathlineto{\pgfqpoint{5.720191in}{3.215995in}}%
\pgfpathlineto{\pgfqpoint{5.687848in}{3.211462in}}%
\pgfpathclose%
\pgfusepath{fill}%
\end{pgfscope}%
\begin{pgfscope}%
\pgfpathrectangle{\pgfqpoint{1.020000in}{0.880000in}}{\pgfqpoint{6.160000in}{6.160000in}}%
\pgfusepath{clip}%
\pgfsetbuttcap%
\pgfsetroundjoin%
\definecolor{currentfill}{rgb}{0.333490,0.446265,0.874452}%
\pgfsetfillcolor{currentfill}%
\pgfsetlinewidth{0.000000pt}%
\definecolor{currentstroke}{rgb}{0.000000,0.000000,0.000000}%
\pgfsetstrokecolor{currentstroke}%
\pgfsetdash{}{0pt}%
\pgfpathmoveto{\pgfqpoint{5.405043in}{3.145030in}}%
\pgfpathlineto{\pgfqpoint{5.416528in}{3.142637in}}%
\pgfpathlineto{\pgfqpoint{5.428031in}{3.139766in}}%
\pgfpathlineto{\pgfqpoint{5.460601in}{3.151978in}}%
\pgfpathlineto{\pgfqpoint{5.493150in}{3.163514in}}%
\pgfpathlineto{\pgfqpoint{5.481631in}{3.170084in}}%
\pgfpathlineto{\pgfqpoint{5.470131in}{3.176309in}}%
\pgfpathlineto{\pgfqpoint{5.437599in}{3.161135in}}%
\pgfpathlineto{\pgfqpoint{5.405043in}{3.145030in}}%
\pgfpathclose%
\pgfusepath{fill}%
\end{pgfscope}%
\begin{pgfscope}%
\pgfpathrectangle{\pgfqpoint{1.020000in}{0.880000in}}{\pgfqpoint{6.160000in}{6.160000in}}%
\pgfusepath{clip}%
\pgfsetbuttcap%
\pgfsetroundjoin%
\definecolor{currentfill}{rgb}{0.399231,0.528528,0.928459}%
\pgfsetfillcolor{currentfill}%
\pgfsetlinewidth{0.000000pt}%
\definecolor{currentstroke}{rgb}{0.000000,0.000000,0.000000}%
\pgfsetstrokecolor{currentstroke}%
\pgfsetdash{}{0pt}%
\pgfpathmoveto{\pgfqpoint{4.602385in}{3.322783in}}%
\pgfpathlineto{\pgfqpoint{4.613001in}{3.306361in}}%
\pgfpathlineto{\pgfqpoint{4.623644in}{3.291584in}}%
\pgfpathlineto{\pgfqpoint{4.656234in}{3.268024in}}%
\pgfpathlineto{\pgfqpoint{4.688779in}{3.242592in}}%
\pgfpathlineto{\pgfqpoint{4.678086in}{3.256376in}}%
\pgfpathlineto{\pgfqpoint{4.667422in}{3.272426in}}%
\pgfpathlineto{\pgfqpoint{4.634926in}{3.298576in}}%
\pgfpathlineto{\pgfqpoint{4.602385in}{3.322783in}}%
\pgfpathclose%
\pgfusepath{fill}%
\end{pgfscope}%
\begin{pgfscope}%
\pgfpathrectangle{\pgfqpoint{1.020000in}{0.880000in}}{\pgfqpoint{6.160000in}{6.160000in}}%
\pgfusepath{clip}%
\pgfsetbuttcap%
\pgfsetroundjoin%
\definecolor{currentfill}{rgb}{0.266381,0.353304,0.801637}%
\pgfsetfillcolor{currentfill}%
\pgfsetlinewidth{0.000000pt}%
\definecolor{currentstroke}{rgb}{0.000000,0.000000,0.000000}%
\pgfsetstrokecolor{currentstroke}%
\pgfsetdash{}{0pt}%
\pgfpathmoveto{\pgfqpoint{5.056873in}{3.023316in}}%
\pgfpathlineto{\pgfqpoint{5.068106in}{3.033092in}}%
\pgfpathlineto{\pgfqpoint{5.079371in}{3.043707in}}%
\pgfpathlineto{\pgfqpoint{5.111899in}{3.041528in}}%
\pgfpathlineto{\pgfqpoint{5.144436in}{3.042963in}}%
\pgfpathlineto{\pgfqpoint{5.133105in}{3.032377in}}%
\pgfpathlineto{\pgfqpoint{5.121806in}{3.022615in}}%
\pgfpathlineto{\pgfqpoint{5.089331in}{3.020725in}}%
\pgfpathlineto{\pgfqpoint{5.056873in}{3.023316in}}%
\pgfpathclose%
\pgfusepath{fill}%
\end{pgfscope}%
\begin{pgfscope}%
\pgfpathrectangle{\pgfqpoint{1.020000in}{0.880000in}}{\pgfqpoint{6.160000in}{6.160000in}}%
\pgfusepath{clip}%
\pgfsetbuttcap%
\pgfsetroundjoin%
\definecolor{currentfill}{rgb}{0.635474,0.756714,0.998297}%
\pgfsetfillcolor{currentfill}%
\pgfsetlinewidth{0.000000pt}%
\definecolor{currentstroke}{rgb}{0.000000,0.000000,0.000000}%
\pgfsetstrokecolor{currentstroke}%
\pgfsetdash{}{0pt}%
\pgfpathmoveto{\pgfqpoint{4.061437in}{3.735601in}}%
\pgfpathlineto{\pgfqpoint{4.071586in}{3.709659in}}%
\pgfpathlineto{\pgfqpoint{4.081750in}{3.684007in}}%
\pgfpathlineto{\pgfqpoint{4.114630in}{3.663336in}}%
\pgfpathlineto{\pgfqpoint{4.147475in}{3.644003in}}%
\pgfpathlineto{\pgfqpoint{4.137269in}{3.667842in}}%
\pgfpathlineto{\pgfqpoint{4.127080in}{3.691928in}}%
\pgfpathlineto{\pgfqpoint{4.094277in}{3.712987in}}%
\pgfpathlineto{\pgfqpoint{4.061437in}{3.735601in}}%
\pgfpathclose%
\pgfusepath{fill}%
\end{pgfscope}%
\begin{pgfscope}%
\pgfpathrectangle{\pgfqpoint{1.020000in}{0.880000in}}{\pgfqpoint{6.160000in}{6.160000in}}%
\pgfusepath{clip}%
\pgfsetbuttcap%
\pgfsetroundjoin%
\definecolor{currentfill}{rgb}{0.825294,0.295749,0.250025}%
\pgfsetfillcolor{currentfill}%
\pgfsetlinewidth{0.000000pt}%
\definecolor{currentstroke}{rgb}{0.000000,0.000000,0.000000}%
\pgfsetstrokecolor{currentstroke}%
\pgfsetdash{}{0pt}%
\pgfpathmoveto{\pgfqpoint{2.636476in}{5.068146in}}%
\pgfpathlineto{\pgfqpoint{2.644824in}{5.090298in}}%
\pgfpathlineto{\pgfqpoint{2.653234in}{5.109483in}}%
\pgfpathlineto{\pgfqpoint{2.686190in}{5.136165in}}%
\pgfpathlineto{\pgfqpoint{2.719188in}{5.159385in}}%
\pgfpathlineto{\pgfqpoint{2.710731in}{5.137857in}}%
\pgfpathlineto{\pgfqpoint{2.702338in}{5.113210in}}%
\pgfpathlineto{\pgfqpoint{2.669389in}{5.092277in}}%
\pgfpathlineto{\pgfqpoint{2.636476in}{5.068146in}}%
\pgfpathclose%
\pgfusepath{fill}%
\end{pgfscope}%
\begin{pgfscope}%
\pgfpathrectangle{\pgfqpoint{1.020000in}{0.880000in}}{\pgfqpoint{6.160000in}{6.160000in}}%
\pgfusepath{clip}%
\pgfsetbuttcap%
\pgfsetroundjoin%
\definecolor{currentfill}{rgb}{0.960490,0.616276,0.495467}%
\pgfsetfillcolor{currentfill}%
\pgfsetlinewidth{0.000000pt}%
\definecolor{currentstroke}{rgb}{0.000000,0.000000,0.000000}%
\pgfsetstrokecolor{currentstroke}%
\pgfsetdash{}{0pt}%
\pgfpathmoveto{\pgfqpoint{2.243020in}{4.666748in}}%
\pgfpathlineto{\pgfqpoint{2.251246in}{4.668237in}}%
\pgfpathlineto{\pgfqpoint{2.259518in}{4.668061in}}%
\pgfpathlineto{\pgfqpoint{2.292326in}{4.704573in}}%
\pgfpathlineto{\pgfqpoint{2.325113in}{4.742313in}}%
\pgfpathlineto{\pgfqpoint{2.316829in}{4.739212in}}%
\pgfpathlineto{\pgfqpoint{2.308594in}{4.734225in}}%
\pgfpathlineto{\pgfqpoint{2.275817in}{4.699915in}}%
\pgfpathlineto{\pgfqpoint{2.243020in}{4.666748in}}%
\pgfpathclose%
\pgfusepath{fill}%
\end{pgfscope}%
\begin{pgfscope}%
\pgfpathrectangle{\pgfqpoint{1.020000in}{0.880000in}}{\pgfqpoint{6.160000in}{6.160000in}}%
\pgfusepath{clip}%
\pgfsetbuttcap%
\pgfsetroundjoin%
\definecolor{currentfill}{rgb}{0.967711,0.662973,0.544323}%
\pgfsetfillcolor{currentfill}%
\pgfsetlinewidth{0.000000pt}%
\definecolor{currentstroke}{rgb}{0.000000,0.000000,0.000000}%
\pgfsetstrokecolor{currentstroke}%
\pgfsetdash{}{0pt}%
\pgfpathmoveto{\pgfqpoint{2.177343in}{4.604758in}}%
\pgfpathlineto{\pgfqpoint{2.185560in}{4.602961in}}%
\pgfpathlineto{\pgfqpoint{2.193820in}{4.599708in}}%
\pgfpathlineto{\pgfqpoint{2.226685in}{4.633033in}}%
\pgfpathlineto{\pgfqpoint{2.259518in}{4.668061in}}%
\pgfpathlineto{\pgfqpoint{2.251246in}{4.668237in}}%
\pgfpathlineto{\pgfqpoint{2.243020in}{4.666748in}}%
\pgfpathlineto{\pgfqpoint{2.210197in}{4.634961in}}%
\pgfpathlineto{\pgfqpoint{2.177343in}{4.604758in}}%
\pgfpathclose%
\pgfusepath{fill}%
\end{pgfscope}%
\begin{pgfscope}%
\pgfpathrectangle{\pgfqpoint{1.020000in}{0.880000in}}{\pgfqpoint{6.160000in}{6.160000in}}%
\pgfusepath{clip}%
\pgfsetbuttcap%
\pgfsetroundjoin%
\definecolor{currentfill}{rgb}{0.949454,0.572388,0.453443}%
\pgfsetfillcolor{currentfill}%
\pgfsetlinewidth{0.000000pt}%
\definecolor{currentstroke}{rgb}{0.000000,0.000000,0.000000}%
\pgfsetstrokecolor{currentstroke}%
\pgfsetdash{}{0pt}%
\pgfpathmoveto{\pgfqpoint{2.308594in}{4.734225in}}%
\pgfpathlineto{\pgfqpoint{2.316829in}{4.739212in}}%
\pgfpathlineto{\pgfqpoint{2.325113in}{4.742313in}}%
\pgfpathlineto{\pgfqpoint{2.357887in}{4.780989in}}%
\pgfpathlineto{\pgfqpoint{2.390653in}{4.820276in}}%
\pgfpathlineto{\pgfqpoint{2.382354in}{4.813777in}}%
\pgfpathlineto{\pgfqpoint{2.374108in}{4.805162in}}%
\pgfpathlineto{\pgfqpoint{2.341355in}{4.769408in}}%
\pgfpathlineto{\pgfqpoint{2.308594in}{4.734225in}}%
\pgfpathclose%
\pgfusepath{fill}%
\end{pgfscope}%
\begin{pgfscope}%
\pgfpathrectangle{\pgfqpoint{1.020000in}{0.880000in}}{\pgfqpoint{6.160000in}{6.160000in}}%
\pgfusepath{clip}%
\pgfsetbuttcap%
\pgfsetroundjoin%
\definecolor{currentfill}{rgb}{0.929357,0.512254,0.400673}%
\pgfsetfillcolor{currentfill}%
\pgfsetlinewidth{0.000000pt}%
\definecolor{currentstroke}{rgb}{0.000000,0.000000,0.000000}%
\pgfsetstrokecolor{currentstroke}%
\pgfsetdash{}{0pt}%
\pgfpathmoveto{\pgfqpoint{2.374108in}{4.805162in}}%
\pgfpathlineto{\pgfqpoint{2.382354in}{4.813777in}}%
\pgfpathlineto{\pgfqpoint{2.390653in}{4.820276in}}%
\pgfpathlineto{\pgfqpoint{2.423418in}{4.859818in}}%
\pgfpathlineto{\pgfqpoint{2.456188in}{4.899231in}}%
\pgfpathlineto{\pgfqpoint{2.447870in}{4.889308in}}%
\pgfpathlineto{\pgfqpoint{2.439608in}{4.877037in}}%
\pgfpathlineto{\pgfqpoint{2.406857in}{4.841157in}}%
\pgfpathlineto{\pgfqpoint{2.374108in}{4.805162in}}%
\pgfpathclose%
\pgfusepath{fill}%
\end{pgfscope}%
\begin{pgfscope}%
\pgfpathrectangle{\pgfqpoint{1.020000in}{0.880000in}}{\pgfqpoint{6.160000in}{6.160000in}}%
\pgfusepath{clip}%
\pgfsetbuttcap%
\pgfsetroundjoin%
\definecolor{currentfill}{rgb}{0.852378,0.346492,0.280346}%
\pgfsetfillcolor{currentfill}%
\pgfsetlinewidth{0.000000pt}%
\definecolor{currentstroke}{rgb}{0.000000,0.000000,0.000000}%
\pgfsetstrokecolor{currentstroke}%
\pgfsetdash{}{0pt}%
\pgfpathmoveto{\pgfqpoint{2.570753in}{5.011710in}}%
\pgfpathlineto{\pgfqpoint{2.579064in}{5.030882in}}%
\pgfpathlineto{\pgfqpoint{2.587436in}{5.047271in}}%
\pgfpathlineto{\pgfqpoint{2.620317in}{5.079718in}}%
\pgfpathlineto{\pgfqpoint{2.653234in}{5.109483in}}%
\pgfpathlineto{\pgfqpoint{2.644824in}{5.090298in}}%
\pgfpathlineto{\pgfqpoint{2.636476in}{5.068146in}}%
\pgfpathlineto{\pgfqpoint{2.603599in}{5.041166in}}%
\pgfpathlineto{\pgfqpoint{2.570753in}{5.011710in}}%
\pgfpathclose%
\pgfusepath{fill}%
\end{pgfscope}%
\begin{pgfscope}%
\pgfpathrectangle{\pgfqpoint{1.020000in}{0.880000in}}{\pgfqpoint{6.160000in}{6.160000in}}%
\pgfusepath{clip}%
\pgfsetbuttcap%
\pgfsetroundjoin%
\definecolor{currentfill}{rgb}{0.869655,0.379274,0.300941}%
\pgfsetfillcolor{currentfill}%
\pgfsetlinewidth{0.000000pt}%
\definecolor{currentstroke}{rgb}{0.000000,0.000000,0.000000}%
\pgfsetstrokecolor{currentstroke}%
\pgfsetdash{}{0pt}%
\pgfpathmoveto{\pgfqpoint{3.221137in}{5.065562in}}%
\pgfpathlineto{\pgfqpoint{3.230402in}{5.065742in}}%
\pgfpathlineto{\pgfqpoint{3.239727in}{5.061143in}}%
\pgfpathlineto{\pgfqpoint{3.273258in}{5.016050in}}%
\pgfpathlineto{\pgfqpoint{3.306766in}{4.967965in}}%
\pgfpathlineto{\pgfqpoint{3.297383in}{4.973362in}}%
\pgfpathlineto{\pgfqpoint{3.288055in}{4.974325in}}%
\pgfpathlineto{\pgfqpoint{3.254608in}{5.021398in}}%
\pgfpathlineto{\pgfqpoint{3.221137in}{5.065562in}}%
\pgfpathclose%
\pgfusepath{fill}%
\end{pgfscope}%
\begin{pgfscope}%
\pgfpathrectangle{\pgfqpoint{1.020000in}{0.880000in}}{\pgfqpoint{6.160000in}{6.160000in}}%
\pgfusepath{clip}%
\pgfsetbuttcap%
\pgfsetroundjoin%
\definecolor{currentfill}{rgb}{0.905783,0.455186,0.355336}%
\pgfsetfillcolor{currentfill}%
\pgfsetlinewidth{0.000000pt}%
\definecolor{currentstroke}{rgb}{0.000000,0.000000,0.000000}%
\pgfsetstrokecolor{currentstroke}%
\pgfsetdash{}{0pt}%
\pgfpathmoveto{\pgfqpoint{2.439608in}{4.877037in}}%
\pgfpathlineto{\pgfqpoint{2.447870in}{4.889308in}}%
\pgfpathlineto{\pgfqpoint{2.456188in}{4.899231in}}%
\pgfpathlineto{\pgfqpoint{2.488969in}{4.938110in}}%
\pgfpathlineto{\pgfqpoint{2.521767in}{4.976032in}}%
\pgfpathlineto{\pgfqpoint{2.513425in}{4.962772in}}%
\pgfpathlineto{\pgfqpoint{2.505143in}{4.946939in}}%
\pgfpathlineto{\pgfqpoint{2.472368in}{4.912428in}}%
\pgfpathlineto{\pgfqpoint{2.439608in}{4.877037in}}%
\pgfpathclose%
\pgfusepath{fill}%
\end{pgfscope}%
\begin{pgfscope}%
\pgfpathrectangle{\pgfqpoint{1.020000in}{0.880000in}}{\pgfqpoint{6.160000in}{6.160000in}}%
\pgfusepath{clip}%
\pgfsetbuttcap%
\pgfsetroundjoin%
\definecolor{currentfill}{rgb}{0.880896,0.402331,0.317115}%
\pgfsetfillcolor{currentfill}%
\pgfsetlinewidth{0.000000pt}%
\definecolor{currentstroke}{rgb}{0.000000,0.000000,0.000000}%
\pgfsetstrokecolor{currentstroke}%
\pgfsetdash{}{0pt}%
\pgfpathmoveto{\pgfqpoint{2.505143in}{4.946939in}}%
\pgfpathlineto{\pgfqpoint{2.513425in}{4.962772in}}%
\pgfpathlineto{\pgfqpoint{2.521767in}{4.976032in}}%
\pgfpathlineto{\pgfqpoint{2.554587in}{5.012565in}}%
\pgfpathlineto{\pgfqpoint{2.587436in}{5.047271in}}%
\pgfpathlineto{\pgfqpoint{2.579064in}{5.030882in}}%
\pgfpathlineto{\pgfqpoint{2.570753in}{5.011710in}}%
\pgfpathlineto{\pgfqpoint{2.537936in}{4.980167in}}%
\pgfpathlineto{\pgfqpoint{2.505143in}{4.946939in}}%
\pgfpathclose%
\pgfusepath{fill}%
\end{pgfscope}%
\begin{pgfscope}%
\pgfpathrectangle{\pgfqpoint{1.020000in}{0.880000in}}{\pgfqpoint{6.160000in}{6.160000in}}%
\pgfusepath{clip}%
\pgfsetbuttcap%
\pgfsetroundjoin%
\definecolor{currentfill}{rgb}{0.473070,0.611077,0.970634}%
\pgfsetfillcolor{currentfill}%
\pgfsetlinewidth{0.000000pt}%
\definecolor{currentstroke}{rgb}{0.000000,0.000000,0.000000}%
\pgfsetstrokecolor{currentstroke}%
\pgfsetdash{}{0pt}%
\pgfpathmoveto{\pgfqpoint{4.450832in}{3.436943in}}%
\pgfpathlineto{\pgfqpoint{4.461311in}{3.418114in}}%
\pgfpathlineto{\pgfqpoint{4.471810in}{3.400150in}}%
\pgfpathlineto{\pgfqpoint{4.504511in}{3.383430in}}%
\pgfpathlineto{\pgfqpoint{4.537176in}{3.365110in}}%
\pgfpathlineto{\pgfqpoint{4.526630in}{3.383053in}}%
\pgfpathlineto{\pgfqpoint{4.516106in}{3.402141in}}%
\pgfpathlineto{\pgfqpoint{4.483487in}{3.420280in}}%
\pgfpathlineto{\pgfqpoint{4.450832in}{3.436943in}}%
\pgfpathclose%
\pgfusepath{fill}%
\end{pgfscope}%
\begin{pgfscope}%
\pgfpathrectangle{\pgfqpoint{1.020000in}{0.880000in}}{\pgfqpoint{6.160000in}{6.160000in}}%
\pgfusepath{clip}%
\pgfsetbuttcap%
\pgfsetroundjoin%
\definecolor{currentfill}{rgb}{0.859385,0.864431,0.872111}%
\pgfsetfillcolor{currentfill}%
\pgfsetlinewidth{0.000000pt}%
\definecolor{currentstroke}{rgb}{0.000000,0.000000,0.000000}%
\pgfsetstrokecolor{currentstroke}%
\pgfsetdash{}{0pt}%
\pgfpathmoveto{\pgfqpoint{3.737401in}{4.192806in}}%
\pgfpathlineto{\pgfqpoint{3.747334in}{4.158461in}}%
\pgfpathlineto{\pgfqpoint{3.757286in}{4.122922in}}%
\pgfpathlineto{\pgfqpoint{3.790423in}{4.081142in}}%
\pgfpathlineto{\pgfqpoint{3.823509in}{4.040930in}}%
\pgfpathlineto{\pgfqpoint{3.813535in}{4.073433in}}%
\pgfpathlineto{\pgfqpoint{3.803579in}{4.104941in}}%
\pgfpathlineto{\pgfqpoint{3.770517in}{4.148024in}}%
\pgfpathlineto{\pgfqpoint{3.737401in}{4.192806in}}%
\pgfpathclose%
\pgfusepath{fill}%
\end{pgfscope}%
\begin{pgfscope}%
\pgfpathrectangle{\pgfqpoint{1.020000in}{0.880000in}}{\pgfqpoint{6.160000in}{6.160000in}}%
\pgfusepath{clip}%
\pgfsetbuttcap%
\pgfsetroundjoin%
\definecolor{currentfill}{rgb}{0.318832,0.426605,0.859857}%
\pgfsetfillcolor{currentfill}%
\pgfsetlinewidth{0.000000pt}%
\definecolor{currentstroke}{rgb}{0.000000,0.000000,0.000000}%
\pgfsetstrokecolor{currentstroke}%
\pgfsetdash{}{0pt}%
\pgfpathmoveto{\pgfqpoint{5.339887in}{3.111692in}}%
\pgfpathlineto{\pgfqpoint{5.351355in}{3.113440in}}%
\pgfpathlineto{\pgfqpoint{5.362840in}{3.114565in}}%
\pgfpathlineto{\pgfqpoint{5.395442in}{3.127173in}}%
\pgfpathlineto{\pgfqpoint{5.428031in}{3.139766in}}%
\pgfpathlineto{\pgfqpoint{5.416528in}{3.142637in}}%
\pgfpathlineto{\pgfqpoint{5.405043in}{3.145030in}}%
\pgfpathlineto{\pgfqpoint{5.372471in}{3.128387in}}%
\pgfpathlineto{\pgfqpoint{5.339887in}{3.111692in}}%
\pgfpathclose%
\pgfusepath{fill}%
\end{pgfscope}%
\begin{pgfscope}%
\pgfpathrectangle{\pgfqpoint{1.020000in}{0.880000in}}{\pgfqpoint{6.160000in}{6.160000in}}%
\pgfusepath{clip}%
\pgfsetbuttcap%
\pgfsetroundjoin%
\definecolor{currentfill}{rgb}{0.922681,0.828568,0.777054}%
\pgfsetfillcolor{currentfill}%
\pgfsetlinewidth{0.000000pt}%
\definecolor{currentstroke}{rgb}{0.000000,0.000000,0.000000}%
\pgfsetstrokecolor{currentstroke}%
\pgfsetdash{}{0pt}%
\pgfpathmoveto{\pgfqpoint{3.651240in}{4.356033in}}%
\pgfpathlineto{\pgfqpoint{3.661110in}{4.322277in}}%
\pgfpathlineto{\pgfqpoint{3.671003in}{4.286566in}}%
\pgfpathlineto{\pgfqpoint{3.704230in}{4.239071in}}%
\pgfpathlineto{\pgfqpoint{3.737401in}{4.192806in}}%
\pgfpathlineto{\pgfqpoint{3.727488in}{4.225713in}}%
\pgfpathlineto{\pgfqpoint{3.717596in}{4.256954in}}%
\pgfpathlineto{\pgfqpoint{3.684447in}{4.305837in}}%
\pgfpathlineto{\pgfqpoint{3.651240in}{4.356033in}}%
\pgfpathclose%
\pgfusepath{fill}%
\end{pgfscope}%
\begin{pgfscope}%
\pgfpathrectangle{\pgfqpoint{1.020000in}{0.880000in}}{\pgfqpoint{6.160000in}{6.160000in}}%
\pgfusepath{clip}%
\pgfsetbuttcap%
\pgfsetroundjoin%
\definecolor{currentfill}{rgb}{0.791392,0.846750,0.936641}%
\pgfsetfillcolor{currentfill}%
\pgfsetlinewidth{0.000000pt}%
\definecolor{currentstroke}{rgb}{0.000000,0.000000,0.000000}%
\pgfsetstrokecolor{currentstroke}%
\pgfsetdash{}{0pt}%
\pgfpathmoveto{\pgfqpoint{3.823509in}{4.040930in}}%
\pgfpathlineto{\pgfqpoint{3.833499in}{4.007654in}}%
\pgfpathlineto{\pgfqpoint{3.843506in}{3.973836in}}%
\pgfpathlineto{\pgfqpoint{3.876566in}{3.938379in}}%
\pgfpathlineto{\pgfqpoint{3.909579in}{3.904616in}}%
\pgfpathlineto{\pgfqpoint{3.899545in}{3.935442in}}%
\pgfpathlineto{\pgfqpoint{3.889528in}{3.965834in}}%
\pgfpathlineto{\pgfqpoint{3.856543in}{4.002450in}}%
\pgfpathlineto{\pgfqpoint{3.823509in}{4.040930in}}%
\pgfpathclose%
\pgfusepath{fill}%
\end{pgfscope}%
\begin{pgfscope}%
\pgfpathrectangle{\pgfqpoint{1.020000in}{0.880000in}}{\pgfqpoint{6.160000in}{6.160000in}}%
\pgfusepath{clip}%
\pgfsetbuttcap%
\pgfsetroundjoin%
\definecolor{currentfill}{rgb}{0.304174,0.406945,0.845263}%
\pgfsetfillcolor{currentfill}%
\pgfsetlinewidth{0.000000pt}%
\definecolor{currentstroke}{rgb}{0.000000,0.000000,0.000000}%
\pgfsetstrokecolor{currentstroke}%
\pgfsetdash{}{0pt}%
\pgfpathmoveto{\pgfqpoint{4.840258in}{3.117686in}}%
\pgfpathlineto{\pgfqpoint{4.851180in}{3.115822in}}%
\pgfpathlineto{\pgfqpoint{4.862138in}{3.115963in}}%
\pgfpathlineto{\pgfqpoint{4.894628in}{3.094152in}}%
\pgfpathlineto{\pgfqpoint{4.927095in}{3.073975in}}%
\pgfpathlineto{\pgfqpoint{4.916057in}{3.070333in}}%
\pgfpathlineto{\pgfqpoint{4.905059in}{3.069074in}}%
\pgfpathlineto{\pgfqpoint{4.872670in}{3.092457in}}%
\pgfpathlineto{\pgfqpoint{4.840258in}{3.117686in}}%
\pgfpathclose%
\pgfusepath{fill}%
\end{pgfscope}%
\begin{pgfscope}%
\pgfpathrectangle{\pgfqpoint{1.020000in}{0.880000in}}{\pgfqpoint{6.160000in}{6.160000in}}%
\pgfusepath{clip}%
\pgfsetbuttcap%
\pgfsetroundjoin%
\definecolor{currentfill}{rgb}{0.353369,0.472069,0.892570}%
\pgfsetfillcolor{currentfill}%
\pgfsetlinewidth{0.000000pt}%
\definecolor{currentstroke}{rgb}{0.000000,0.000000,0.000000}%
\pgfsetstrokecolor{currentstroke}%
\pgfsetdash{}{0pt}%
\pgfpathmoveto{\pgfqpoint{5.623070in}{3.199705in}}%
\pgfpathlineto{\pgfqpoint{5.634651in}{3.187518in}}%
\pgfpathlineto{\pgfqpoint{5.646249in}{3.175015in}}%
\pgfpathlineto{\pgfqpoint{5.678685in}{3.179820in}}%
\pgfpathlineto{\pgfqpoint{5.711093in}{3.183961in}}%
\pgfpathlineto{\pgfqpoint{5.699461in}{3.197803in}}%
\pgfpathlineto{\pgfqpoint{5.687848in}{3.211462in}}%
\pgfpathlineto{\pgfqpoint{5.655474in}{3.206068in}}%
\pgfpathlineto{\pgfqpoint{5.623070in}{3.199705in}}%
\pgfpathclose%
\pgfusepath{fill}%
\end{pgfscope}%
\begin{pgfscope}%
\pgfpathrectangle{\pgfqpoint{1.020000in}{0.880000in}}{\pgfqpoint{6.160000in}{6.160000in}}%
\pgfusepath{clip}%
\pgfsetbuttcap%
\pgfsetroundjoin%
\definecolor{currentfill}{rgb}{0.527132,0.664700,0.989065}%
\pgfsetfillcolor{currentfill}%
\pgfsetlinewidth{0.000000pt}%
\definecolor{currentstroke}{rgb}{0.000000,0.000000,0.000000}%
\pgfsetstrokecolor{currentstroke}%
\pgfsetdash{}{0pt}%
\pgfpathmoveto{\pgfqpoint{4.299176in}{3.535360in}}%
\pgfpathlineto{\pgfqpoint{4.309524in}{3.514903in}}%
\pgfpathlineto{\pgfqpoint{4.319889in}{3.495006in}}%
\pgfpathlineto{\pgfqpoint{4.352671in}{3.481137in}}%
\pgfpathlineto{\pgfqpoint{4.385423in}{3.467047in}}%
\pgfpathlineto{\pgfqpoint{4.375010in}{3.486599in}}%
\pgfpathlineto{\pgfqpoint{4.364617in}{3.506728in}}%
\pgfpathlineto{\pgfqpoint{4.331911in}{3.521048in}}%
\pgfpathlineto{\pgfqpoint{4.299176in}{3.535360in}}%
\pgfpathclose%
\pgfusepath{fill}%
\end{pgfscope}%
\begin{pgfscope}%
\pgfpathrectangle{\pgfqpoint{1.020000in}{0.880000in}}{\pgfqpoint{6.160000in}{6.160000in}}%
\pgfusepath{clip}%
\pgfsetbuttcap%
\pgfsetroundjoin%
\definecolor{currentfill}{rgb}{0.960581,0.762501,0.667964}%
\pgfsetfillcolor{currentfill}%
\pgfsetlinewidth{0.000000pt}%
\definecolor{currentstroke}{rgb}{0.000000,0.000000,0.000000}%
\pgfsetstrokecolor{currentstroke}%
\pgfsetdash{}{0pt}%
\pgfpathmoveto{\pgfqpoint{3.565035in}{4.524062in}}%
\pgfpathlineto{\pgfqpoint{3.574830in}{4.492968in}}%
\pgfpathlineto{\pgfqpoint{3.584656in}{4.459105in}}%
\pgfpathlineto{\pgfqpoint{3.617976in}{4.407236in}}%
\pgfpathlineto{\pgfqpoint{3.651240in}{4.356033in}}%
\pgfpathlineto{\pgfqpoint{3.641394in}{4.387595in}}%
\pgfpathlineto{\pgfqpoint{3.631575in}{4.416756in}}%
\pgfpathlineto{\pgfqpoint{3.598334in}{4.470058in}}%
\pgfpathlineto{\pgfqpoint{3.565035in}{4.524062in}}%
\pgfpathclose%
\pgfusepath{fill}%
\end{pgfscope}%
\begin{pgfscope}%
\pgfpathrectangle{\pgfqpoint{1.020000in}{0.880000in}}{\pgfqpoint{6.160000in}{6.160000in}}%
\pgfusepath{clip}%
\pgfsetbuttcap%
\pgfsetroundjoin%
\definecolor{currentfill}{rgb}{0.763520,0.178667,0.193396}%
\pgfsetfillcolor{currentfill}%
\pgfsetlinewidth{0.000000pt}%
\definecolor{currentstroke}{rgb}{0.000000,0.000000,0.000000}%
\pgfsetstrokecolor{currentstroke}%
\pgfsetdash{}{0pt}%
\pgfpathmoveto{\pgfqpoint{2.918056in}{5.210018in}}%
\pgfpathlineto{\pgfqpoint{2.926768in}{5.231165in}}%
\pgfpathlineto{\pgfqpoint{2.935551in}{5.248101in}}%
\pgfpathlineto{\pgfqpoint{2.968897in}{5.240129in}}%
\pgfpathlineto{\pgfqpoint{3.002273in}{5.227067in}}%
\pgfpathlineto{\pgfqpoint{2.993417in}{5.210294in}}%
\pgfpathlineto{\pgfqpoint{2.984629in}{5.189353in}}%
\pgfpathlineto{\pgfqpoint{2.951329in}{5.202078in}}%
\pgfpathlineto{\pgfqpoint{2.918056in}{5.210018in}}%
\pgfpathclose%
\pgfusepath{fill}%
\end{pgfscope}%
\begin{pgfscope}%
\pgfpathrectangle{\pgfqpoint{1.020000in}{0.880000in}}{\pgfqpoint{6.160000in}{6.160000in}}%
\pgfusepath{clip}%
\pgfsetbuttcap%
\pgfsetroundjoin%
\definecolor{currentfill}{rgb}{0.912033,0.469680,0.366565}%
\pgfsetfillcolor{currentfill}%
\pgfsetlinewidth{0.000000pt}%
\definecolor{currentstroke}{rgb}{0.000000,0.000000,0.000000}%
\pgfsetstrokecolor{currentstroke}%
\pgfsetdash{}{0pt}%
\pgfpathmoveto{\pgfqpoint{3.306766in}{4.967965in}}%
\pgfpathlineto{\pgfqpoint{3.316203in}{4.958002in}}%
\pgfpathlineto{\pgfqpoint{3.325694in}{4.943396in}}%
\pgfpathlineto{\pgfqpoint{3.359218in}{4.892496in}}%
\pgfpathlineto{\pgfqpoint{3.392705in}{4.839557in}}%
\pgfpathlineto{\pgfqpoint{3.383169in}{4.854277in}}%
\pgfpathlineto{\pgfqpoint{3.373682in}{4.864762in}}%
\pgfpathlineto{\pgfqpoint{3.340242in}{4.917371in}}%
\pgfpathlineto{\pgfqpoint{3.306766in}{4.967965in}}%
\pgfpathclose%
\pgfusepath{fill}%
\end{pgfscope}%
\begin{pgfscope}%
\pgfpathrectangle{\pgfqpoint{1.020000in}{0.880000in}}{\pgfqpoint{6.160000in}{6.160000in}}%
\pgfusepath{clip}%
\pgfsetbuttcap%
\pgfsetroundjoin%
\definecolor{currentfill}{rgb}{0.968500,0.673977,0.556649}%
\pgfsetfillcolor{currentfill}%
\pgfsetlinewidth{0.000000pt}%
\definecolor{currentstroke}{rgb}{0.000000,0.000000,0.000000}%
\pgfsetstrokecolor{currentstroke}%
\pgfsetdash{}{0pt}%
\pgfpathmoveto{\pgfqpoint{3.478831in}{4.688425in}}%
\pgfpathlineto{\pgfqpoint{3.488532in}{4.662281in}}%
\pgfpathlineto{\pgfqpoint{3.498272in}{4.632585in}}%
\pgfpathlineto{\pgfqpoint{3.531680in}{4.578378in}}%
\pgfpathlineto{\pgfqpoint{3.565035in}{4.524062in}}%
\pgfpathlineto{\pgfqpoint{3.555271in}{4.552185in}}%
\pgfpathlineto{\pgfqpoint{3.545540in}{4.577172in}}%
\pgfpathlineto{\pgfqpoint{3.512213in}{4.632851in}}%
\pgfpathlineto{\pgfqpoint{3.478831in}{4.688425in}}%
\pgfpathclose%
\pgfusepath{fill}%
\end{pgfscope}%
\begin{pgfscope}%
\pgfpathrectangle{\pgfqpoint{1.020000in}{0.880000in}}{\pgfqpoint{6.160000in}{6.160000in}}%
\pgfusepath{clip}%
\pgfsetbuttcap%
\pgfsetroundjoin%
\definecolor{currentfill}{rgb}{0.592356,0.722792,0.999434}%
\pgfsetfillcolor{currentfill}%
\pgfsetlinewidth{0.000000pt}%
\definecolor{currentstroke}{rgb}{0.000000,0.000000,0.000000}%
\pgfsetstrokecolor{currentstroke}%
\pgfsetdash{}{0pt}%
\pgfpathmoveto{\pgfqpoint{4.147475in}{3.644003in}}%
\pgfpathlineto{\pgfqpoint{4.157697in}{3.620532in}}%
\pgfpathlineto{\pgfqpoint{4.167936in}{3.597543in}}%
\pgfpathlineto{\pgfqpoint{4.200792in}{3.580841in}}%
\pgfpathlineto{\pgfqpoint{4.233617in}{3.565030in}}%
\pgfpathlineto{\pgfqpoint{4.223333in}{3.586770in}}%
\pgfpathlineto{\pgfqpoint{4.213067in}{3.608925in}}%
\pgfpathlineto{\pgfqpoint{4.180287in}{3.625910in}}%
\pgfpathlineto{\pgfqpoint{4.147475in}{3.644003in}}%
\pgfpathclose%
\pgfusepath{fill}%
\end{pgfscope}%
\begin{pgfscope}%
\pgfpathrectangle{\pgfqpoint{1.020000in}{0.880000in}}{\pgfqpoint{6.160000in}{6.160000in}}%
\pgfusepath{clip}%
\pgfsetbuttcap%
\pgfsetroundjoin%
\definecolor{currentfill}{rgb}{0.348323,0.465711,0.888346}%
\pgfsetfillcolor{currentfill}%
\pgfsetlinewidth{0.000000pt}%
\definecolor{currentstroke}{rgb}{0.000000,0.000000,0.000000}%
\pgfsetstrokecolor{currentstroke}%
\pgfsetdash{}{0pt}%
\pgfpathmoveto{\pgfqpoint{5.840464in}{3.195441in}}%
\pgfpathlineto{\pgfqpoint{5.852201in}{3.180280in}}%
\pgfpathlineto{\pgfqpoint{5.863958in}{3.165066in}}%
\pgfpathlineto{\pgfqpoint{5.896286in}{3.166875in}}%
\pgfpathlineto{\pgfqpoint{5.884505in}{3.182174in}}%
\pgfpathlineto{\pgfqpoint{5.872745in}{3.197441in}}%
\pgfpathlineto{\pgfqpoint{5.840464in}{3.195441in}}%
\pgfpathclose%
\pgfusepath{fill}%
\end{pgfscope}%
\begin{pgfscope}%
\pgfpathrectangle{\pgfqpoint{1.020000in}{0.880000in}}{\pgfqpoint{6.160000in}{6.160000in}}%
\pgfusepath{clip}%
\pgfsetbuttcap%
\pgfsetroundjoin%
\definecolor{currentfill}{rgb}{0.728970,0.817464,0.973188}%
\pgfsetfillcolor{currentfill}%
\pgfsetlinewidth{0.000000pt}%
\definecolor{currentstroke}{rgb}{0.000000,0.000000,0.000000}%
\pgfsetstrokecolor{currentstroke}%
\pgfsetdash{}{0pt}%
\pgfpathmoveto{\pgfqpoint{3.909579in}{3.904616in}}%
\pgfpathlineto{\pgfqpoint{3.919628in}{3.873554in}}%
\pgfpathlineto{\pgfqpoint{3.929692in}{3.842462in}}%
\pgfpathlineto{\pgfqpoint{3.962689in}{3.813272in}}%
\pgfpathlineto{\pgfqpoint{3.995645in}{3.785722in}}%
\pgfpathlineto{\pgfqpoint{3.985549in}{3.814109in}}%
\pgfpathlineto{\pgfqpoint{3.975468in}{3.842488in}}%
\pgfpathlineto{\pgfqpoint{3.942546in}{3.872633in}}%
\pgfpathlineto{\pgfqpoint{3.909579in}{3.904616in}}%
\pgfpathclose%
\pgfusepath{fill}%
\end{pgfscope}%
\begin{pgfscope}%
\pgfpathrectangle{\pgfqpoint{1.020000in}{0.880000in}}{\pgfqpoint{6.160000in}{6.160000in}}%
\pgfusepath{clip}%
\pgfsetbuttcap%
\pgfsetroundjoin%
\definecolor{currentfill}{rgb}{0.949454,0.572388,0.453443}%
\pgfsetfillcolor{currentfill}%
\pgfsetlinewidth{0.000000pt}%
\definecolor{currentstroke}{rgb}{0.000000,0.000000,0.000000}%
\pgfsetstrokecolor{currentstroke}%
\pgfsetdash{}{0pt}%
\pgfpathmoveto{\pgfqpoint{3.392705in}{4.839557in}}%
\pgfpathlineto{\pgfqpoint{3.402288in}{4.820586in}}%
\pgfpathlineto{\pgfqpoint{3.411917in}{4.797407in}}%
\pgfpathlineto{\pgfqpoint{3.445398in}{4.743437in}}%
\pgfpathlineto{\pgfqpoint{3.478831in}{4.688425in}}%
\pgfpathlineto{\pgfqpoint{3.469168in}{4.710880in}}%
\pgfpathlineto{\pgfqpoint{3.459546in}{4.729558in}}%
\pgfpathlineto{\pgfqpoint{3.426149in}{4.785079in}}%
\pgfpathlineto{\pgfqpoint{3.392705in}{4.839557in}}%
\pgfpathclose%
\pgfusepath{fill}%
\end{pgfscope}%
\begin{pgfscope}%
\pgfpathrectangle{\pgfqpoint{1.020000in}{0.880000in}}{\pgfqpoint{6.160000in}{6.160000in}}%
\pgfusepath{clip}%
\pgfsetbuttcap%
\pgfsetroundjoin%
\definecolor{currentfill}{rgb}{0.363461,0.484784,0.901019}%
\pgfsetfillcolor{currentfill}%
\pgfsetlinewidth{0.000000pt}%
\definecolor{currentstroke}{rgb}{0.000000,0.000000,0.000000}%
\pgfsetstrokecolor{currentstroke}%
\pgfsetdash{}{0pt}%
\pgfpathmoveto{\pgfqpoint{4.688779in}{3.242592in}}%
\pgfpathlineto{\pgfqpoint{4.699503in}{3.230940in}}%
\pgfpathlineto{\pgfqpoint{4.710256in}{3.221194in}}%
\pgfpathlineto{\pgfqpoint{4.742817in}{3.196272in}}%
\pgfpathlineto{\pgfqpoint{4.775336in}{3.170329in}}%
\pgfpathlineto{\pgfqpoint{4.764519in}{3.177566in}}%
\pgfpathlineto{\pgfqpoint{4.753738in}{3.187322in}}%
\pgfpathlineto{\pgfqpoint{4.721280in}{3.215548in}}%
\pgfpathlineto{\pgfqpoint{4.688779in}{3.242592in}}%
\pgfpathclose%
\pgfusepath{fill}%
\end{pgfscope}%
\begin{pgfscope}%
\pgfpathrectangle{\pgfqpoint{1.020000in}{0.880000in}}{\pgfqpoint{6.160000in}{6.160000in}}%
\pgfusepath{clip}%
\pgfsetbuttcap%
\pgfsetroundjoin%
\definecolor{currentfill}{rgb}{0.275827,0.366717,0.812553}%
\pgfsetfillcolor{currentfill}%
\pgfsetlinewidth{0.000000pt}%
\definecolor{currentstroke}{rgb}{0.000000,0.000000,0.000000}%
\pgfsetstrokecolor{currentstroke}%
\pgfsetdash{}{0pt}%
\pgfpathmoveto{\pgfqpoint{4.991987in}{3.041406in}}%
\pgfpathlineto{\pgfqpoint{5.003142in}{3.049555in}}%
\pgfpathlineto{\pgfqpoint{5.014329in}{3.058510in}}%
\pgfpathlineto{\pgfqpoint{5.046849in}{3.049441in}}%
\pgfpathlineto{\pgfqpoint{5.079371in}{3.043707in}}%
\pgfpathlineto{\pgfqpoint{5.068106in}{3.033092in}}%
\pgfpathlineto{\pgfqpoint{5.056873in}{3.023316in}}%
\pgfpathlineto{\pgfqpoint{5.024427in}{3.030303in}}%
\pgfpathlineto{\pgfqpoint{4.991987in}{3.041406in}}%
\pgfpathclose%
\pgfusepath{fill}%
\end{pgfscope}%
\begin{pgfscope}%
\pgfpathrectangle{\pgfqpoint{1.020000in}{0.880000in}}{\pgfqpoint{6.160000in}{6.160000in}}%
\pgfusepath{clip}%
\pgfsetbuttcap%
\pgfsetroundjoin%
\definecolor{currentfill}{rgb}{0.309060,0.413498,0.850128}%
\pgfsetfillcolor{currentfill}%
\pgfsetlinewidth{0.000000pt}%
\definecolor{currentstroke}{rgb}{0.000000,0.000000,0.000000}%
\pgfsetstrokecolor{currentstroke}%
\pgfsetdash{}{0pt}%
\pgfpathmoveto{\pgfqpoint{5.274708in}{3.080439in}}%
\pgfpathlineto{\pgfqpoint{5.286152in}{3.086107in}}%
\pgfpathlineto{\pgfqpoint{5.297613in}{3.091019in}}%
\pgfpathlineto{\pgfqpoint{5.330229in}{3.102362in}}%
\pgfpathlineto{\pgfqpoint{5.362840in}{3.114565in}}%
\pgfpathlineto{\pgfqpoint{5.351355in}{3.113440in}}%
\pgfpathlineto{\pgfqpoint{5.339887in}{3.111692in}}%
\pgfpathlineto{\pgfqpoint{5.307297in}{3.095507in}}%
\pgfpathlineto{\pgfqpoint{5.274708in}{3.080439in}}%
\pgfpathclose%
\pgfusepath{fill}%
\end{pgfscope}%
\begin{pgfscope}%
\pgfpathrectangle{\pgfqpoint{1.020000in}{0.880000in}}{\pgfqpoint{6.160000in}{6.160000in}}%
\pgfusepath{clip}%
\pgfsetbuttcap%
\pgfsetroundjoin%
\definecolor{currentfill}{rgb}{0.795938,0.241845,0.220830}%
\pgfsetfillcolor{currentfill}%
\pgfsetlinewidth{0.000000pt}%
\definecolor{currentstroke}{rgb}{0.000000,0.000000,0.000000}%
\pgfsetstrokecolor{currentstroke}%
\pgfsetdash{}{0pt}%
\pgfpathmoveto{\pgfqpoint{3.069087in}{5.186144in}}%
\pgfpathlineto{\pgfqpoint{3.078084in}{5.197747in}}%
\pgfpathlineto{\pgfqpoint{3.087151in}{5.204571in}}%
\pgfpathlineto{\pgfqpoint{3.120648in}{5.176236in}}%
\pgfpathlineto{\pgfqpoint{3.154150in}{5.143386in}}%
\pgfpathlineto{\pgfqpoint{3.145014in}{5.137429in}}%
\pgfpathlineto{\pgfqpoint{3.135943in}{5.126890in}}%
\pgfpathlineto{\pgfqpoint{3.102513in}{5.158675in}}%
\pgfpathlineto{\pgfqpoint{3.069087in}{5.186144in}}%
\pgfpathclose%
\pgfusepath{fill}%
\end{pgfscope}%
\begin{pgfscope}%
\pgfpathrectangle{\pgfqpoint{1.020000in}{0.880000in}}{\pgfqpoint{6.160000in}{6.160000in}}%
\pgfusepath{clip}%
\pgfsetbuttcap%
\pgfsetroundjoin%
\definecolor{currentfill}{rgb}{0.353369,0.472069,0.892570}%
\pgfsetfillcolor{currentfill}%
\pgfsetlinewidth{0.000000pt}%
\definecolor{currentstroke}{rgb}{0.000000,0.000000,0.000000}%
\pgfsetstrokecolor{currentstroke}%
\pgfsetdash{}{0pt}%
\pgfpathmoveto{\pgfqpoint{5.558169in}{3.183772in}}%
\pgfpathlineto{\pgfqpoint{5.569723in}{3.173744in}}%
\pgfpathlineto{\pgfqpoint{5.581293in}{3.163208in}}%
\pgfpathlineto{\pgfqpoint{5.613785in}{3.169488in}}%
\pgfpathlineto{\pgfqpoint{5.646249in}{3.175015in}}%
\pgfpathlineto{\pgfqpoint{5.634651in}{3.187518in}}%
\pgfpathlineto{\pgfqpoint{5.623070in}{3.199705in}}%
\pgfpathlineto{\pgfqpoint{5.590635in}{3.192288in}}%
\pgfpathlineto{\pgfqpoint{5.558169in}{3.183772in}}%
\pgfpathclose%
\pgfusepath{fill}%
\end{pgfscope}%
\begin{pgfscope}%
\pgfpathrectangle{\pgfqpoint{1.020000in}{0.880000in}}{\pgfqpoint{6.160000in}{6.160000in}}%
\pgfusepath{clip}%
\pgfsetbuttcap%
\pgfsetroundjoin%
\definecolor{currentfill}{rgb}{0.435815,0.570707,0.951717}%
\pgfsetfillcolor{currentfill}%
\pgfsetlinewidth{0.000000pt}%
\definecolor{currentstroke}{rgb}{0.000000,0.000000,0.000000}%
\pgfsetstrokecolor{currentstroke}%
\pgfsetdash{}{0pt}%
\pgfpathmoveto{\pgfqpoint{4.537176in}{3.365110in}}%
\pgfpathlineto{\pgfqpoint{4.547744in}{3.348363in}}%
\pgfpathlineto{\pgfqpoint{4.558336in}{3.332818in}}%
\pgfpathlineto{\pgfqpoint{4.591010in}{3.313169in}}%
\pgfpathlineto{\pgfqpoint{4.623644in}{3.291584in}}%
\pgfpathlineto{\pgfqpoint{4.613001in}{3.306361in}}%
\pgfpathlineto{\pgfqpoint{4.602385in}{3.322783in}}%
\pgfpathlineto{\pgfqpoint{4.569801in}{3.344947in}}%
\pgfpathlineto{\pgfqpoint{4.537176in}{3.365110in}}%
\pgfpathclose%
\pgfusepath{fill}%
\end{pgfscope}%
\begin{pgfscope}%
\pgfpathrectangle{\pgfqpoint{1.020000in}{0.880000in}}{\pgfqpoint{6.160000in}{6.160000in}}%
\pgfusepath{clip}%
\pgfsetbuttcap%
\pgfsetroundjoin%
\definecolor{currentfill}{rgb}{0.667253,0.779176,0.992959}%
\pgfsetfillcolor{currentfill}%
\pgfsetlinewidth{0.000000pt}%
\definecolor{currentstroke}{rgb}{0.000000,0.000000,0.000000}%
\pgfsetstrokecolor{currentstroke}%
\pgfsetdash{}{0pt}%
\pgfpathmoveto{\pgfqpoint{3.995645in}{3.785722in}}%
\pgfpathlineto{\pgfqpoint{4.005757in}{3.757498in}}%
\pgfpathlineto{\pgfqpoint{4.015883in}{3.729608in}}%
\pgfpathlineto{\pgfqpoint{4.048835in}{3.706085in}}%
\pgfpathlineto{\pgfqpoint{4.081750in}{3.684007in}}%
\pgfpathlineto{\pgfqpoint{4.071586in}{3.709659in}}%
\pgfpathlineto{\pgfqpoint{4.061437in}{3.735601in}}%
\pgfpathlineto{\pgfqpoint{4.028561in}{3.759834in}}%
\pgfpathlineto{\pgfqpoint{3.995645in}{3.785722in}}%
\pgfpathclose%
\pgfusepath{fill}%
\end{pgfscope}%
\begin{pgfscope}%
\pgfpathrectangle{\pgfqpoint{1.020000in}{0.880000in}}{\pgfqpoint{6.160000in}{6.160000in}}%
\pgfusepath{clip}%
\pgfsetbuttcap%
\pgfsetroundjoin%
\definecolor{currentfill}{rgb}{0.348323,0.465711,0.888346}%
\pgfsetfillcolor{currentfill}%
\pgfsetlinewidth{0.000000pt}%
\definecolor{currentstroke}{rgb}{0.000000,0.000000,0.000000}%
\pgfsetstrokecolor{currentstroke}%
\pgfsetdash{}{0pt}%
\pgfpathmoveto{\pgfqpoint{5.775830in}{3.190553in}}%
\pgfpathlineto{\pgfqpoint{5.787521in}{3.175779in}}%
\pgfpathlineto{\pgfqpoint{5.799233in}{3.160891in}}%
\pgfpathlineto{\pgfqpoint{5.831607in}{3.163087in}}%
\pgfpathlineto{\pgfqpoint{5.863958in}{3.165066in}}%
\pgfpathlineto{\pgfqpoint{5.852201in}{3.180280in}}%
\pgfpathlineto{\pgfqpoint{5.840464in}{3.195441in}}%
\pgfpathlineto{\pgfqpoint{5.808159in}{3.193169in}}%
\pgfpathlineto{\pgfqpoint{5.775830in}{3.190553in}}%
\pgfpathclose%
\pgfusepath{fill}%
\end{pgfscope}%
\begin{pgfscope}%
\pgfpathrectangle{\pgfqpoint{1.020000in}{0.880000in}}{\pgfqpoint{6.160000in}{6.160000in}}%
\pgfusepath{clip}%
\pgfsetbuttcap%
\pgfsetroundjoin%
\definecolor{currentfill}{rgb}{0.294718,0.393542,0.834384}%
\pgfsetfillcolor{currentfill}%
\pgfsetlinewidth{0.000000pt}%
\definecolor{currentstroke}{rgb}{0.000000,0.000000,0.000000}%
\pgfsetstrokecolor{currentstroke}%
\pgfsetdash{}{0pt}%
\pgfpathmoveto{\pgfqpoint{5.209549in}{3.056087in}}%
\pgfpathlineto{\pgfqpoint{5.220957in}{3.064848in}}%
\pgfpathlineto{\pgfqpoint{5.232381in}{3.072751in}}%
\pgfpathlineto{\pgfqpoint{5.264996in}{3.081002in}}%
\pgfpathlineto{\pgfqpoint{5.297613in}{3.091019in}}%
\pgfpathlineto{\pgfqpoint{5.286152in}{3.086107in}}%
\pgfpathlineto{\pgfqpoint{5.274708in}{3.080439in}}%
\pgfpathlineto{\pgfqpoint{5.242124in}{3.067103in}}%
\pgfpathlineto{\pgfqpoint{5.209549in}{3.056087in}}%
\pgfpathclose%
\pgfusepath{fill}%
\end{pgfscope}%
\begin{pgfscope}%
\pgfpathrectangle{\pgfqpoint{1.020000in}{0.880000in}}{\pgfqpoint{6.160000in}{6.160000in}}%
\pgfusepath{clip}%
\pgfsetbuttcap%
\pgfsetroundjoin%
\definecolor{currentfill}{rgb}{0.752704,0.157576,0.184258}%
\pgfsetfillcolor{currentfill}%
\pgfsetlinewidth{0.000000pt}%
\definecolor{currentstroke}{rgb}{0.000000,0.000000,0.000000}%
\pgfsetstrokecolor{currentstroke}%
\pgfsetdash{}{0pt}%
\pgfpathmoveto{\pgfqpoint{2.851607in}{5.211456in}}%
\pgfpathlineto{\pgfqpoint{2.860250in}{5.232162in}}%
\pgfpathlineto{\pgfqpoint{2.868966in}{5.248682in}}%
\pgfpathlineto{\pgfqpoint{2.902239in}{5.250943in}}%
\pgfpathlineto{\pgfqpoint{2.935551in}{5.248101in}}%
\pgfpathlineto{\pgfqpoint{2.926768in}{5.231165in}}%
\pgfpathlineto{\pgfqpoint{2.918056in}{5.210018in}}%
\pgfpathlineto{\pgfqpoint{2.884814in}{5.213135in}}%
\pgfpathlineto{\pgfqpoint{2.851607in}{5.211456in}}%
\pgfpathclose%
\pgfusepath{fill}%
\end{pgfscope}%
\begin{pgfscope}%
\pgfpathrectangle{\pgfqpoint{1.020000in}{0.880000in}}{\pgfqpoint{6.160000in}{6.160000in}}%
\pgfusepath{clip}%
\pgfsetbuttcap%
\pgfsetroundjoin%
\definecolor{currentfill}{rgb}{0.494638,0.633022,0.978983}%
\pgfsetfillcolor{currentfill}%
\pgfsetlinewidth{0.000000pt}%
\definecolor{currentstroke}{rgb}{0.000000,0.000000,0.000000}%
\pgfsetstrokecolor{currentstroke}%
\pgfsetdash{}{0pt}%
\pgfpathmoveto{\pgfqpoint{4.385423in}{3.467047in}}%
\pgfpathlineto{\pgfqpoint{4.395855in}{3.448157in}}%
\pgfpathlineto{\pgfqpoint{4.406306in}{3.429999in}}%
\pgfpathlineto{\pgfqpoint{4.439075in}{3.415566in}}%
\pgfpathlineto{\pgfqpoint{4.471810in}{3.400150in}}%
\pgfpathlineto{\pgfqpoint{4.461311in}{3.418114in}}%
\pgfpathlineto{\pgfqpoint{4.450832in}{3.436943in}}%
\pgfpathlineto{\pgfqpoint{4.418144in}{3.452424in}}%
\pgfpathlineto{\pgfqpoint{4.385423in}{3.467047in}}%
\pgfpathclose%
\pgfusepath{fill}%
\end{pgfscope}%
\begin{pgfscope}%
\pgfpathrectangle{\pgfqpoint{1.020000in}{0.880000in}}{\pgfqpoint{6.160000in}{6.160000in}}%
\pgfusepath{clip}%
\pgfsetbuttcap%
\pgfsetroundjoin%
\definecolor{currentfill}{rgb}{0.554312,0.690097,0.995516}%
\pgfsetfillcolor{currentfill}%
\pgfsetlinewidth{0.000000pt}%
\definecolor{currentstroke}{rgb}{0.000000,0.000000,0.000000}%
\pgfsetstrokecolor{currentstroke}%
\pgfsetdash{}{0pt}%
\pgfpathmoveto{\pgfqpoint{4.233617in}{3.565030in}}%
\pgfpathlineto{\pgfqpoint{4.243918in}{3.543806in}}%
\pgfpathlineto{\pgfqpoint{4.254237in}{3.523189in}}%
\pgfpathlineto{\pgfqpoint{4.287078in}{3.508940in}}%
\pgfpathlineto{\pgfqpoint{4.319889in}{3.495006in}}%
\pgfpathlineto{\pgfqpoint{4.309524in}{3.514903in}}%
\pgfpathlineto{\pgfqpoint{4.299176in}{3.535360in}}%
\pgfpathlineto{\pgfqpoint{4.266412in}{3.549940in}}%
\pgfpathlineto{\pgfqpoint{4.233617in}{3.565030in}}%
\pgfpathclose%
\pgfusepath{fill}%
\end{pgfscope}%
\begin{pgfscope}%
\pgfpathrectangle{\pgfqpoint{1.020000in}{0.880000in}}{\pgfqpoint{6.160000in}{6.160000in}}%
\pgfusepath{clip}%
\pgfsetbuttcap%
\pgfsetroundjoin%
\definecolor{currentfill}{rgb}{0.343278,0.459354,0.884122}%
\pgfsetfillcolor{currentfill}%
\pgfsetlinewidth{0.000000pt}%
\definecolor{currentstroke}{rgb}{0.000000,0.000000,0.000000}%
\pgfsetstrokecolor{currentstroke}%
\pgfsetdash{}{0pt}%
\pgfpathmoveto{\pgfqpoint{5.493150in}{3.163514in}}%
\pgfpathlineto{\pgfqpoint{5.504684in}{3.156362in}}%
\pgfpathlineto{\pgfqpoint{5.516231in}{3.148449in}}%
\pgfpathlineto{\pgfqpoint{5.548775in}{3.156179in}}%
\pgfpathlineto{\pgfqpoint{5.581293in}{3.163208in}}%
\pgfpathlineto{\pgfqpoint{5.569723in}{3.173744in}}%
\pgfpathlineto{\pgfqpoint{5.558169in}{3.183772in}}%
\pgfpathlineto{\pgfqpoint{5.525673in}{3.174159in}}%
\pgfpathlineto{\pgfqpoint{5.493150in}{3.163514in}}%
\pgfpathclose%
\pgfusepath{fill}%
\end{pgfscope}%
\begin{pgfscope}%
\pgfpathrectangle{\pgfqpoint{1.020000in}{0.880000in}}{\pgfqpoint{6.160000in}{6.160000in}}%
\pgfusepath{clip}%
\pgfsetbuttcap%
\pgfsetroundjoin%
\definecolor{currentfill}{rgb}{0.333490,0.446265,0.874452}%
\pgfsetfillcolor{currentfill}%
\pgfsetlinewidth{0.000000pt}%
\definecolor{currentstroke}{rgb}{0.000000,0.000000,0.000000}%
\pgfsetstrokecolor{currentstroke}%
\pgfsetdash{}{0pt}%
\pgfpathmoveto{\pgfqpoint{4.775336in}{3.170329in}}%
\pgfpathlineto{\pgfqpoint{4.786185in}{3.165189in}}%
\pgfpathlineto{\pgfqpoint{4.797066in}{3.161660in}}%
\pgfpathlineto{\pgfqpoint{4.829618in}{3.138687in}}%
\pgfpathlineto{\pgfqpoint{4.862138in}{3.115963in}}%
\pgfpathlineto{\pgfqpoint{4.851180in}{3.115822in}}%
\pgfpathlineto{\pgfqpoint{4.840258in}{3.117686in}}%
\pgfpathlineto{\pgfqpoint{4.807815in}{3.143906in}}%
\pgfpathlineto{\pgfqpoint{4.775336in}{3.170329in}}%
\pgfpathclose%
\pgfusepath{fill}%
\end{pgfscope}%
\begin{pgfscope}%
\pgfpathrectangle{\pgfqpoint{1.020000in}{0.880000in}}{\pgfqpoint{6.160000in}{6.160000in}}%
\pgfusepath{clip}%
\pgfsetbuttcap%
\pgfsetroundjoin%
\definecolor{currentfill}{rgb}{0.294718,0.393542,0.834384}%
\pgfsetfillcolor{currentfill}%
\pgfsetlinewidth{0.000000pt}%
\definecolor{currentstroke}{rgb}{0.000000,0.000000,0.000000}%
\pgfsetstrokecolor{currentstroke}%
\pgfsetdash{}{0pt}%
\pgfpathmoveto{\pgfqpoint{4.927095in}{3.073975in}}%
\pgfpathlineto{\pgfqpoint{4.938167in}{3.079175in}}%
\pgfpathlineto{\pgfqpoint{4.949269in}{3.085114in}}%
\pgfpathlineto{\pgfqpoint{4.981804in}{3.070558in}}%
\pgfpathlineto{\pgfqpoint{5.014329in}{3.058510in}}%
\pgfpathlineto{\pgfqpoint{5.003142in}{3.049555in}}%
\pgfpathlineto{\pgfqpoint{4.991987in}{3.041406in}}%
\pgfpathlineto{\pgfqpoint{4.959545in}{3.056162in}}%
\pgfpathlineto{\pgfqpoint{4.927095in}{3.073975in}}%
\pgfpathclose%
\pgfusepath{fill}%
\end{pgfscope}%
\begin{pgfscope}%
\pgfpathrectangle{\pgfqpoint{1.020000in}{0.880000in}}{\pgfqpoint{6.160000in}{6.160000in}}%
\pgfusepath{clip}%
\pgfsetbuttcap%
\pgfsetroundjoin%
\definecolor{currentfill}{rgb}{0.619318,0.744121,0.998931}%
\pgfsetfillcolor{currentfill}%
\pgfsetlinewidth{0.000000pt}%
\definecolor{currentstroke}{rgb}{0.000000,0.000000,0.000000}%
\pgfsetstrokecolor{currentstroke}%
\pgfsetdash{}{0pt}%
\pgfpathmoveto{\pgfqpoint{4.081750in}{3.684007in}}%
\pgfpathlineto{\pgfqpoint{4.091931in}{3.658787in}}%
\pgfpathlineto{\pgfqpoint{4.102127in}{3.634133in}}%
\pgfpathlineto{\pgfqpoint{4.135048in}{3.615273in}}%
\pgfpathlineto{\pgfqpoint{4.167936in}{3.597543in}}%
\pgfpathlineto{\pgfqpoint{4.157697in}{3.620532in}}%
\pgfpathlineto{\pgfqpoint{4.147475in}{3.644003in}}%
\pgfpathlineto{\pgfqpoint{4.114630in}{3.663336in}}%
\pgfpathlineto{\pgfqpoint{4.081750in}{3.684007in}}%
\pgfpathclose%
\pgfusepath{fill}%
\end{pgfscope}%
\begin{pgfscope}%
\pgfpathrectangle{\pgfqpoint{1.020000in}{0.880000in}}{\pgfqpoint{6.160000in}{6.160000in}}%
\pgfusepath{clip}%
\pgfsetbuttcap%
\pgfsetroundjoin%
\definecolor{currentfill}{rgb}{0.825294,0.295749,0.250025}%
\pgfsetfillcolor{currentfill}%
\pgfsetlinewidth{0.000000pt}%
\definecolor{currentstroke}{rgb}{0.000000,0.000000,0.000000}%
\pgfsetstrokecolor{currentstroke}%
\pgfsetdash{}{0pt}%
\pgfpathmoveto{\pgfqpoint{3.154150in}{5.143386in}}%
\pgfpathlineto{\pgfqpoint{3.163350in}{5.144491in}}%
\pgfpathlineto{\pgfqpoint{3.172617in}{5.140528in}}%
\pgfpathlineto{\pgfqpoint{3.206177in}{5.102781in}}%
\pgfpathlineto{\pgfqpoint{3.239727in}{5.061143in}}%
\pgfpathlineto{\pgfqpoint{3.230402in}{5.065742in}}%
\pgfpathlineto{\pgfqpoint{3.221137in}{5.065562in}}%
\pgfpathlineto{\pgfqpoint{3.187648in}{5.106366in}}%
\pgfpathlineto{\pgfqpoint{3.154150in}{5.143386in}}%
\pgfpathclose%
\pgfusepath{fill}%
\end{pgfscope}%
\begin{pgfscope}%
\pgfpathrectangle{\pgfqpoint{1.020000in}{0.880000in}}{\pgfqpoint{6.160000in}{6.160000in}}%
\pgfusepath{clip}%
\pgfsetbuttcap%
\pgfsetroundjoin%
\definecolor{currentfill}{rgb}{0.289996,0.386836,0.828926}%
\pgfsetfillcolor{currentfill}%
\pgfsetlinewidth{0.000000pt}%
\definecolor{currentstroke}{rgb}{0.000000,0.000000,0.000000}%
\pgfsetstrokecolor{currentstroke}%
\pgfsetdash{}{0pt}%
\pgfpathmoveto{\pgfqpoint{5.144436in}{3.042963in}}%
\pgfpathlineto{\pgfqpoint{5.155792in}{3.053449in}}%
\pgfpathlineto{\pgfqpoint{5.167166in}{3.063026in}}%
\pgfpathlineto{\pgfqpoint{5.199771in}{3.066656in}}%
\pgfpathlineto{\pgfqpoint{5.232381in}{3.072751in}}%
\pgfpathlineto{\pgfqpoint{5.220957in}{3.064848in}}%
\pgfpathlineto{\pgfqpoint{5.209549in}{3.056087in}}%
\pgfpathlineto{\pgfqpoint{5.176986in}{3.047905in}}%
\pgfpathlineto{\pgfqpoint{5.144436in}{3.042963in}}%
\pgfpathclose%
\pgfusepath{fill}%
\end{pgfscope}%
\begin{pgfscope}%
\pgfpathrectangle{\pgfqpoint{1.020000in}{0.880000in}}{\pgfqpoint{6.160000in}{6.160000in}}%
\pgfusepath{clip}%
\pgfsetbuttcap%
\pgfsetroundjoin%
\definecolor{currentfill}{rgb}{0.353369,0.472069,0.892570}%
\pgfsetfillcolor{currentfill}%
\pgfsetlinewidth{0.000000pt}%
\definecolor{currentstroke}{rgb}{0.000000,0.000000,0.000000}%
\pgfsetstrokecolor{currentstroke}%
\pgfsetdash{}{0pt}%
\pgfpathmoveto{\pgfqpoint{5.711093in}{3.183961in}}%
\pgfpathlineto{\pgfqpoint{5.722744in}{3.169911in}}%
\pgfpathlineto{\pgfqpoint{5.734413in}{3.155643in}}%
\pgfpathlineto{\pgfqpoint{5.766835in}{3.158428in}}%
\pgfpathlineto{\pgfqpoint{5.799233in}{3.160891in}}%
\pgfpathlineto{\pgfqpoint{5.787521in}{3.175779in}}%
\pgfpathlineto{\pgfqpoint{5.775830in}{3.190553in}}%
\pgfpathlineto{\pgfqpoint{5.743475in}{3.187511in}}%
\pgfpathlineto{\pgfqpoint{5.711093in}{3.183961in}}%
\pgfpathclose%
\pgfusepath{fill}%
\end{pgfscope}%
\begin{pgfscope}%
\pgfpathrectangle{\pgfqpoint{1.020000in}{0.880000in}}{\pgfqpoint{6.160000in}{6.160000in}}%
\pgfusepath{clip}%
\pgfsetbuttcap%
\pgfsetroundjoin%
\definecolor{currentfill}{rgb}{0.835345,0.860514,0.898970}%
\pgfsetfillcolor{currentfill}%
\pgfsetlinewidth{0.000000pt}%
\definecolor{currentstroke}{rgb}{0.000000,0.000000,0.000000}%
\pgfsetstrokecolor{currentstroke}%
\pgfsetdash{}{0pt}%
\pgfpathmoveto{\pgfqpoint{3.757286in}{4.122922in}}%
\pgfpathlineto{\pgfqpoint{3.767255in}{4.086449in}}%
\pgfpathlineto{\pgfqpoint{3.777240in}{4.049309in}}%
\pgfpathlineto{\pgfqpoint{3.810398in}{4.010863in}}%
\pgfpathlineto{\pgfqpoint{3.843506in}{3.973836in}}%
\pgfpathlineto{\pgfqpoint{3.833499in}{4.007654in}}%
\pgfpathlineto{\pgfqpoint{3.823509in}{4.040930in}}%
\pgfpathlineto{\pgfqpoint{3.790423in}{4.081142in}}%
\pgfpathlineto{\pgfqpoint{3.757286in}{4.122922in}}%
\pgfpathclose%
\pgfusepath{fill}%
\end{pgfscope}%
\begin{pgfscope}%
\pgfpathrectangle{\pgfqpoint{1.020000in}{0.880000in}}{\pgfqpoint{6.160000in}{6.160000in}}%
\pgfusepath{clip}%
\pgfsetbuttcap%
\pgfsetroundjoin%
\definecolor{currentfill}{rgb}{0.902849,0.844796,0.811970}%
\pgfsetfillcolor{currentfill}%
\pgfsetlinewidth{0.000000pt}%
\definecolor{currentstroke}{rgb}{0.000000,0.000000,0.000000}%
\pgfsetstrokecolor{currentstroke}%
\pgfsetdash{}{0pt}%
\pgfpathmoveto{\pgfqpoint{3.671003in}{4.286566in}}%
\pgfpathlineto{\pgfqpoint{3.680918in}{4.249161in}}%
\pgfpathlineto{\pgfqpoint{3.690853in}{4.210343in}}%
\pgfpathlineto{\pgfqpoint{3.724096in}{4.166068in}}%
\pgfpathlineto{\pgfqpoint{3.757286in}{4.122922in}}%
\pgfpathlineto{\pgfqpoint{3.747334in}{4.158461in}}%
\pgfpathlineto{\pgfqpoint{3.737401in}{4.192806in}}%
\pgfpathlineto{\pgfqpoint{3.704230in}{4.239071in}}%
\pgfpathlineto{\pgfqpoint{3.671003in}{4.286566in}}%
\pgfpathclose%
\pgfusepath{fill}%
\end{pgfscope}%
\begin{pgfscope}%
\pgfpathrectangle{\pgfqpoint{1.020000in}{0.880000in}}{\pgfqpoint{6.160000in}{6.160000in}}%
\pgfusepath{clip}%
\pgfsetbuttcap%
\pgfsetroundjoin%
\definecolor{currentfill}{rgb}{0.399231,0.528528,0.928459}%
\pgfsetfillcolor{currentfill}%
\pgfsetlinewidth{0.000000pt}%
\definecolor{currentstroke}{rgb}{0.000000,0.000000,0.000000}%
\pgfsetstrokecolor{currentstroke}%
\pgfsetdash{}{0pt}%
\pgfpathmoveto{\pgfqpoint{4.623644in}{3.291584in}}%
\pgfpathlineto{\pgfqpoint{4.634312in}{3.278369in}}%
\pgfpathlineto{\pgfqpoint{4.645005in}{3.266570in}}%
\pgfpathlineto{\pgfqpoint{4.677652in}{3.244705in}}%
\pgfpathlineto{\pgfqpoint{4.710256in}{3.221194in}}%
\pgfpathlineto{\pgfqpoint{4.699503in}{3.230940in}}%
\pgfpathlineto{\pgfqpoint{4.688779in}{3.242592in}}%
\pgfpathlineto{\pgfqpoint{4.656234in}{3.268024in}}%
\pgfpathlineto{\pgfqpoint{4.623644in}{3.291584in}}%
\pgfpathclose%
\pgfusepath{fill}%
\end{pgfscope}%
\begin{pgfscope}%
\pgfpathrectangle{\pgfqpoint{1.020000in}{0.880000in}}{\pgfqpoint{6.160000in}{6.160000in}}%
\pgfusepath{clip}%
\pgfsetbuttcap%
\pgfsetroundjoin%
\definecolor{currentfill}{rgb}{0.758112,0.168122,0.188827}%
\pgfsetfillcolor{currentfill}%
\pgfsetlinewidth{0.000000pt}%
\definecolor{currentstroke}{rgb}{0.000000,0.000000,0.000000}%
\pgfsetstrokecolor{currentstroke}%
\pgfsetdash{}{0pt}%
\pgfpathmoveto{\pgfqpoint{2.785313in}{5.194115in}}%
\pgfpathlineto{\pgfqpoint{2.793893in}{5.213750in}}%
\pgfpathlineto{\pgfqpoint{2.802547in}{5.229286in}}%
\pgfpathlineto{\pgfqpoint{2.835734in}{5.241411in}}%
\pgfpathlineto{\pgfqpoint{2.868966in}{5.248682in}}%
\pgfpathlineto{\pgfqpoint{2.860250in}{5.232162in}}%
\pgfpathlineto{\pgfqpoint{2.851607in}{5.211456in}}%
\pgfpathlineto{\pgfqpoint{2.818439in}{5.205067in}}%
\pgfpathlineto{\pgfqpoint{2.785313in}{5.194115in}}%
\pgfpathclose%
\pgfusepath{fill}%
\end{pgfscope}%
\begin{pgfscope}%
\pgfpathrectangle{\pgfqpoint{1.020000in}{0.880000in}}{\pgfqpoint{6.160000in}{6.160000in}}%
\pgfusepath{clip}%
\pgfsetbuttcap%
\pgfsetroundjoin%
\definecolor{currentfill}{rgb}{0.768034,0.837035,0.952488}%
\pgfsetfillcolor{currentfill}%
\pgfsetlinewidth{0.000000pt}%
\definecolor{currentstroke}{rgb}{0.000000,0.000000,0.000000}%
\pgfsetstrokecolor{currentstroke}%
\pgfsetdash{}{0pt}%
\pgfpathmoveto{\pgfqpoint{3.843506in}{3.973836in}}%
\pgfpathlineto{\pgfqpoint{3.853528in}{3.939713in}}%
\pgfpathlineto{\pgfqpoint{3.863565in}{3.905524in}}%
\pgfpathlineto{\pgfqpoint{3.896650in}{3.873239in}}%
\pgfpathlineto{\pgfqpoint{3.929692in}{3.842462in}}%
\pgfpathlineto{\pgfqpoint{3.919628in}{3.873554in}}%
\pgfpathlineto{\pgfqpoint{3.909579in}{3.904616in}}%
\pgfpathlineto{\pgfqpoint{3.876566in}{3.938379in}}%
\pgfpathlineto{\pgfqpoint{3.843506in}{3.973836in}}%
\pgfpathclose%
\pgfusepath{fill}%
\end{pgfscope}%
\begin{pgfscope}%
\pgfpathrectangle{\pgfqpoint{1.020000in}{0.880000in}}{\pgfqpoint{6.160000in}{6.160000in}}%
\pgfusepath{clip}%
\pgfsetbuttcap%
\pgfsetroundjoin%
\definecolor{currentfill}{rgb}{0.758112,0.168122,0.188827}%
\pgfsetfillcolor{currentfill}%
\pgfsetlinewidth{0.000000pt}%
\definecolor{currentstroke}{rgb}{0.000000,0.000000,0.000000}%
\pgfsetstrokecolor{currentstroke}%
\pgfsetdash{}{0pt}%
\pgfpathmoveto{\pgfqpoint{3.002273in}{5.227067in}}%
\pgfpathlineto{\pgfqpoint{3.011198in}{5.239280in}}%
\pgfpathlineto{\pgfqpoint{3.020196in}{5.246587in}}%
\pgfpathlineto{\pgfqpoint{3.053664in}{5.228100in}}%
\pgfpathlineto{\pgfqpoint{3.087151in}{5.204571in}}%
\pgfpathlineto{\pgfqpoint{3.078084in}{5.197747in}}%
\pgfpathlineto{\pgfqpoint{3.069087in}{5.186144in}}%
\pgfpathlineto{\pgfqpoint{3.035671in}{5.209016in}}%
\pgfpathlineto{\pgfqpoint{3.002273in}{5.227067in}}%
\pgfpathclose%
\pgfusepath{fill}%
\end{pgfscope}%
\begin{pgfscope}%
\pgfpathrectangle{\pgfqpoint{1.020000in}{0.880000in}}{\pgfqpoint{6.160000in}{6.160000in}}%
\pgfusepath{clip}%
\pgfsetbuttcap%
\pgfsetroundjoin%
\definecolor{currentfill}{rgb}{0.950956,0.786875,0.704761}%
\pgfsetfillcolor{currentfill}%
\pgfsetlinewidth{0.000000pt}%
\definecolor{currentstroke}{rgb}{0.000000,0.000000,0.000000}%
\pgfsetstrokecolor{currentstroke}%
\pgfsetdash{}{0pt}%
\pgfpathmoveto{\pgfqpoint{3.584656in}{4.459105in}}%
\pgfpathlineto{\pgfqpoint{3.594508in}{4.422711in}}%
\pgfpathlineto{\pgfqpoint{3.604386in}{4.384054in}}%
\pgfpathlineto{\pgfqpoint{3.637722in}{4.335002in}}%
\pgfpathlineto{\pgfqpoint{3.671003in}{4.286566in}}%
\pgfpathlineto{\pgfqpoint{3.661110in}{4.322277in}}%
\pgfpathlineto{\pgfqpoint{3.651240in}{4.356033in}}%
\pgfpathlineto{\pgfqpoint{3.617976in}{4.407236in}}%
\pgfpathlineto{\pgfqpoint{3.584656in}{4.459105in}}%
\pgfpathclose%
\pgfusepath{fill}%
\end{pgfscope}%
\begin{pgfscope}%
\pgfpathrectangle{\pgfqpoint{1.020000in}{0.880000in}}{\pgfqpoint{6.160000in}{6.160000in}}%
\pgfusepath{clip}%
\pgfsetbuttcap%
\pgfsetroundjoin%
\definecolor{currentfill}{rgb}{0.338377,0.452819,0.879317}%
\pgfsetfillcolor{currentfill}%
\pgfsetlinewidth{0.000000pt}%
\definecolor{currentstroke}{rgb}{0.000000,0.000000,0.000000}%
\pgfsetstrokecolor{currentstroke}%
\pgfsetdash{}{0pt}%
\pgfpathmoveto{\pgfqpoint{5.428031in}{3.139766in}}%
\pgfpathlineto{\pgfqpoint{5.439547in}{3.136081in}}%
\pgfpathlineto{\pgfqpoint{5.451075in}{3.131335in}}%
\pgfpathlineto{\pgfqpoint{5.483664in}{3.140116in}}%
\pgfpathlineto{\pgfqpoint{5.516231in}{3.148449in}}%
\pgfpathlineto{\pgfqpoint{5.504684in}{3.156362in}}%
\pgfpathlineto{\pgfqpoint{5.493150in}{3.163514in}}%
\pgfpathlineto{\pgfqpoint{5.460601in}{3.151978in}}%
\pgfpathlineto{\pgfqpoint{5.428031in}{3.139766in}}%
\pgfpathclose%
\pgfusepath{fill}%
\end{pgfscope}%
\begin{pgfscope}%
\pgfpathrectangle{\pgfqpoint{1.020000in}{0.880000in}}{\pgfqpoint{6.160000in}{6.160000in}}%
\pgfusepath{clip}%
\pgfsetbuttcap%
\pgfsetroundjoin%
\definecolor{currentfill}{rgb}{0.462354,0.599830,0.965857}%
\pgfsetfillcolor{currentfill}%
\pgfsetlinewidth{0.000000pt}%
\definecolor{currentstroke}{rgb}{0.000000,0.000000,0.000000}%
\pgfsetstrokecolor{currentstroke}%
\pgfsetdash{}{0pt}%
\pgfpathmoveto{\pgfqpoint{4.471810in}{3.400150in}}%
\pgfpathlineto{\pgfqpoint{4.482331in}{3.383100in}}%
\pgfpathlineto{\pgfqpoint{4.492872in}{3.366981in}}%
\pgfpathlineto{\pgfqpoint{4.525622in}{3.350679in}}%
\pgfpathlineto{\pgfqpoint{4.558336in}{3.332818in}}%
\pgfpathlineto{\pgfqpoint{4.547744in}{3.348363in}}%
\pgfpathlineto{\pgfqpoint{4.537176in}{3.365110in}}%
\pgfpathlineto{\pgfqpoint{4.504511in}{3.383430in}}%
\pgfpathlineto{\pgfqpoint{4.471810in}{3.400150in}}%
\pgfpathclose%
\pgfusepath{fill}%
\end{pgfscope}%
\begin{pgfscope}%
\pgfpathrectangle{\pgfqpoint{1.020000in}{0.880000in}}{\pgfqpoint{6.160000in}{6.160000in}}%
\pgfusepath{clip}%
\pgfsetbuttcap%
\pgfsetroundjoin%
\definecolor{currentfill}{rgb}{0.869655,0.379274,0.300941}%
\pgfsetfillcolor{currentfill}%
\pgfsetlinewidth{0.000000pt}%
\definecolor{currentstroke}{rgb}{0.000000,0.000000,0.000000}%
\pgfsetstrokecolor{currentstroke}%
\pgfsetdash{}{0pt}%
\pgfpathmoveto{\pgfqpoint{3.239727in}{5.061143in}}%
\pgfpathlineto{\pgfqpoint{3.249112in}{5.051622in}}%
\pgfpathlineto{\pgfqpoint{3.258557in}{5.037093in}}%
\pgfpathlineto{\pgfqpoint{3.292138in}{4.991758in}}%
\pgfpathlineto{\pgfqpoint{3.325694in}{4.943396in}}%
\pgfpathlineto{\pgfqpoint{3.316203in}{4.958002in}}%
\pgfpathlineto{\pgfqpoint{3.306766in}{4.967965in}}%
\pgfpathlineto{\pgfqpoint{3.273258in}{5.016050in}}%
\pgfpathlineto{\pgfqpoint{3.239727in}{5.061143in}}%
\pgfpathclose%
\pgfusepath{fill}%
\end{pgfscope}%
\begin{pgfscope}%
\pgfpathrectangle{\pgfqpoint{1.020000in}{0.880000in}}{\pgfqpoint{6.160000in}{6.160000in}}%
\pgfusepath{clip}%
\pgfsetbuttcap%
\pgfsetroundjoin%
\definecolor{currentfill}{rgb}{0.966922,0.651969,0.531997}%
\pgfsetfillcolor{currentfill}%
\pgfsetlinewidth{0.000000pt}%
\definecolor{currentstroke}{rgb}{0.000000,0.000000,0.000000}%
\pgfsetstrokecolor{currentstroke}%
\pgfsetdash{}{0pt}%
\pgfpathmoveto{\pgfqpoint{2.193820in}{4.599708in}}%
\pgfpathlineto{\pgfqpoint{2.202126in}{4.594835in}}%
\pgfpathlineto{\pgfqpoint{2.210481in}{4.588190in}}%
\pgfpathlineto{\pgfqpoint{2.243363in}{4.624168in}}%
\pgfpathlineto{\pgfqpoint{2.276214in}{4.661949in}}%
\pgfpathlineto{\pgfqpoint{2.267840in}{4.666026in}}%
\pgfpathlineto{\pgfqpoint{2.259518in}{4.668061in}}%
\pgfpathlineto{\pgfqpoint{2.226685in}{4.633033in}}%
\pgfpathlineto{\pgfqpoint{2.193820in}{4.599708in}}%
\pgfpathclose%
\pgfusepath{fill}%
\end{pgfscope}%
\begin{pgfscope}%
\pgfpathrectangle{\pgfqpoint{1.020000in}{0.880000in}}{\pgfqpoint{6.160000in}{6.160000in}}%
\pgfusepath{clip}%
\pgfsetbuttcap%
\pgfsetroundjoin%
\definecolor{currentfill}{rgb}{0.969522,0.700833,0.587508}%
\pgfsetfillcolor{currentfill}%
\pgfsetlinewidth{0.000000pt}%
\definecolor{currentstroke}{rgb}{0.000000,0.000000,0.000000}%
\pgfsetstrokecolor{currentstroke}%
\pgfsetdash{}{0pt}%
\pgfpathmoveto{\pgfqpoint{3.498272in}{4.632585in}}%
\pgfpathlineto{\pgfqpoint{3.508046in}{4.599519in}}%
\pgfpathlineto{\pgfqpoint{3.517854in}{4.563310in}}%
\pgfpathlineto{\pgfqpoint{3.551281in}{4.511265in}}%
\pgfpathlineto{\pgfqpoint{3.584656in}{4.459105in}}%
\pgfpathlineto{\pgfqpoint{3.574830in}{4.492968in}}%
\pgfpathlineto{\pgfqpoint{3.565035in}{4.524062in}}%
\pgfpathlineto{\pgfqpoint{3.531680in}{4.578378in}}%
\pgfpathlineto{\pgfqpoint{3.498272in}{4.632585in}}%
\pgfpathclose%
\pgfusepath{fill}%
\end{pgfscope}%
\begin{pgfscope}%
\pgfpathrectangle{\pgfqpoint{1.020000in}{0.880000in}}{\pgfqpoint{6.160000in}{6.160000in}}%
\pgfusepath{clip}%
\pgfsetbuttcap%
\pgfsetroundjoin%
\definecolor{currentfill}{rgb}{0.958279,0.604335,0.483297}%
\pgfsetfillcolor{currentfill}%
\pgfsetlinewidth{0.000000pt}%
\definecolor{currentstroke}{rgb}{0.000000,0.000000,0.000000}%
\pgfsetstrokecolor{currentstroke}%
\pgfsetdash{}{0pt}%
\pgfpathmoveto{\pgfqpoint{2.259518in}{4.668061in}}%
\pgfpathlineto{\pgfqpoint{2.267840in}{4.666026in}}%
\pgfpathlineto{\pgfqpoint{2.276214in}{4.661949in}}%
\pgfpathlineto{\pgfqpoint{2.309040in}{4.701300in}}%
\pgfpathlineto{\pgfqpoint{2.341847in}{4.741950in}}%
\pgfpathlineto{\pgfqpoint{2.333452in}{4.743298in}}%
\pgfpathlineto{\pgfqpoint{2.325113in}{4.742313in}}%
\pgfpathlineto{\pgfqpoint{2.292326in}{4.704573in}}%
\pgfpathlineto{\pgfqpoint{2.259518in}{4.668061in}}%
\pgfpathclose%
\pgfusepath{fill}%
\end{pgfscope}%
\begin{pgfscope}%
\pgfpathrectangle{\pgfqpoint{1.020000in}{0.880000in}}{\pgfqpoint{6.160000in}{6.160000in}}%
\pgfusepath{clip}%
\pgfsetbuttcap%
\pgfsetroundjoin%
\definecolor{currentfill}{rgb}{0.516260,0.654498,0.986407}%
\pgfsetfillcolor{currentfill}%
\pgfsetlinewidth{0.000000pt}%
\definecolor{currentstroke}{rgb}{0.000000,0.000000,0.000000}%
\pgfsetstrokecolor{currentstroke}%
\pgfsetdash{}{0pt}%
\pgfpathmoveto{\pgfqpoint{4.319889in}{3.495006in}}%
\pgfpathlineto{\pgfqpoint{4.330274in}{3.475749in}}%
\pgfpathlineto{\pgfqpoint{4.340677in}{3.457203in}}%
\pgfpathlineto{\pgfqpoint{4.373507in}{3.443774in}}%
\pgfpathlineto{\pgfqpoint{4.406306in}{3.429999in}}%
\pgfpathlineto{\pgfqpoint{4.395855in}{3.448157in}}%
\pgfpathlineto{\pgfqpoint{4.385423in}{3.467047in}}%
\pgfpathlineto{\pgfqpoint{4.352671in}{3.481137in}}%
\pgfpathlineto{\pgfqpoint{4.319889in}{3.495006in}}%
\pgfpathclose%
\pgfusepath{fill}%
\end{pgfscope}%
\begin{pgfscope}%
\pgfpathrectangle{\pgfqpoint{1.020000in}{0.880000in}}{\pgfqpoint{6.160000in}{6.160000in}}%
\pgfusepath{clip}%
\pgfsetbuttcap%
\pgfsetroundjoin%
\definecolor{currentfill}{rgb}{0.703587,0.802586,0.982847}%
\pgfsetfillcolor{currentfill}%
\pgfsetlinewidth{0.000000pt}%
\definecolor{currentstroke}{rgb}{0.000000,0.000000,0.000000}%
\pgfsetstrokecolor{currentstroke}%
\pgfsetdash{}{0pt}%
\pgfpathmoveto{\pgfqpoint{3.929692in}{3.842462in}}%
\pgfpathlineto{\pgfqpoint{3.939770in}{3.811542in}}%
\pgfpathlineto{\pgfqpoint{3.949863in}{3.780997in}}%
\pgfpathlineto{\pgfqpoint{3.982893in}{3.754585in}}%
\pgfpathlineto{\pgfqpoint{4.015883in}{3.729608in}}%
\pgfpathlineto{\pgfqpoint{4.005757in}{3.757498in}}%
\pgfpathlineto{\pgfqpoint{3.995645in}{3.785722in}}%
\pgfpathlineto{\pgfqpoint{3.962689in}{3.813272in}}%
\pgfpathlineto{\pgfqpoint{3.929692in}{3.842462in}}%
\pgfpathclose%
\pgfusepath{fill}%
\end{pgfscope}%
\begin{pgfscope}%
\pgfpathrectangle{\pgfqpoint{1.020000in}{0.880000in}}{\pgfqpoint{6.160000in}{6.160000in}}%
\pgfusepath{clip}%
\pgfsetbuttcap%
\pgfsetroundjoin%
\definecolor{currentfill}{rgb}{0.768929,0.189213,0.197965}%
\pgfsetfillcolor{currentfill}%
\pgfsetlinewidth{0.000000pt}%
\definecolor{currentstroke}{rgb}{0.000000,0.000000,0.000000}%
\pgfsetstrokecolor{currentstroke}%
\pgfsetdash{}{0pt}%
\pgfpathmoveto{\pgfqpoint{2.719188in}{5.159385in}}%
\pgfpathlineto{\pgfqpoint{2.727714in}{5.177365in}}%
\pgfpathlineto{\pgfqpoint{2.736312in}{5.191394in}}%
\pgfpathlineto{\pgfqpoint{2.769406in}{5.212523in}}%
\pgfpathlineto{\pgfqpoint{2.802547in}{5.229286in}}%
\pgfpathlineto{\pgfqpoint{2.793893in}{5.213750in}}%
\pgfpathlineto{\pgfqpoint{2.785313in}{5.194115in}}%
\pgfpathlineto{\pgfqpoint{2.752229in}{5.178803in}}%
\pgfpathlineto{\pgfqpoint{2.719188in}{5.159385in}}%
\pgfpathclose%
\pgfusepath{fill}%
\end{pgfscope}%
\begin{pgfscope}%
\pgfpathrectangle{\pgfqpoint{1.020000in}{0.880000in}}{\pgfqpoint{6.160000in}{6.160000in}}%
\pgfusepath{clip}%
\pgfsetbuttcap%
\pgfsetroundjoin%
\definecolor{currentfill}{rgb}{0.944055,0.553153,0.435548}%
\pgfsetfillcolor{currentfill}%
\pgfsetlinewidth{0.000000pt}%
\definecolor{currentstroke}{rgb}{0.000000,0.000000,0.000000}%
\pgfsetstrokecolor{currentstroke}%
\pgfsetdash{}{0pt}%
\pgfpathmoveto{\pgfqpoint{2.325113in}{4.742313in}}%
\pgfpathlineto{\pgfqpoint{2.333452in}{4.743298in}}%
\pgfpathlineto{\pgfqpoint{2.341847in}{4.741950in}}%
\pgfpathlineto{\pgfqpoint{2.374640in}{4.783591in}}%
\pgfpathlineto{\pgfqpoint{2.407428in}{4.825876in}}%
\pgfpathlineto{\pgfqpoint{2.399010in}{4.824392in}}%
\pgfpathlineto{\pgfqpoint{2.390653in}{4.820276in}}%
\pgfpathlineto{\pgfqpoint{2.357887in}{4.780989in}}%
\pgfpathlineto{\pgfqpoint{2.325113in}{4.742313in}}%
\pgfpathclose%
\pgfusepath{fill}%
\end{pgfscope}%
\begin{pgfscope}%
\pgfpathrectangle{\pgfqpoint{1.020000in}{0.880000in}}{\pgfqpoint{6.160000in}{6.160000in}}%
\pgfusepath{clip}%
\pgfsetbuttcap%
\pgfsetroundjoin%
\definecolor{currentfill}{rgb}{0.289996,0.386836,0.828926}%
\pgfsetfillcolor{currentfill}%
\pgfsetlinewidth{0.000000pt}%
\definecolor{currentstroke}{rgb}{0.000000,0.000000,0.000000}%
\pgfsetstrokecolor{currentstroke}%
\pgfsetdash{}{0pt}%
\pgfpathmoveto{\pgfqpoint{5.079371in}{3.043707in}}%
\pgfpathlineto{\pgfqpoint{5.090661in}{3.054234in}}%
\pgfpathlineto{\pgfqpoint{5.101968in}{3.063866in}}%
\pgfpathlineto{\pgfqpoint{5.134566in}{3.062067in}}%
\pgfpathlineto{\pgfqpoint{5.167166in}{3.063026in}}%
\pgfpathlineto{\pgfqpoint{5.155792in}{3.053449in}}%
\pgfpathlineto{\pgfqpoint{5.144436in}{3.042963in}}%
\pgfpathlineto{\pgfqpoint{5.111899in}{3.041528in}}%
\pgfpathlineto{\pgfqpoint{5.079371in}{3.043707in}}%
\pgfpathclose%
\pgfusepath{fill}%
\end{pgfscope}%
\begin{pgfscope}%
\pgfpathrectangle{\pgfqpoint{1.020000in}{0.880000in}}{\pgfqpoint{6.160000in}{6.160000in}}%
\pgfusepath{clip}%
\pgfsetbuttcap%
\pgfsetroundjoin%
\definecolor{currentfill}{rgb}{0.956653,0.598034,0.477302}%
\pgfsetfillcolor{currentfill}%
\pgfsetlinewidth{0.000000pt}%
\definecolor{currentstroke}{rgb}{0.000000,0.000000,0.000000}%
\pgfsetstrokecolor{currentstroke}%
\pgfsetdash{}{0pt}%
\pgfpathmoveto{\pgfqpoint{3.411917in}{4.797407in}}%
\pgfpathlineto{\pgfqpoint{3.421591in}{4.770115in}}%
\pgfpathlineto{\pgfqpoint{3.431306in}{4.738861in}}%
\pgfpathlineto{\pgfqpoint{3.464812in}{4.686235in}}%
\pgfpathlineto{\pgfqpoint{3.498272in}{4.632585in}}%
\pgfpathlineto{\pgfqpoint{3.488532in}{4.662281in}}%
\pgfpathlineto{\pgfqpoint{3.478831in}{4.688425in}}%
\pgfpathlineto{\pgfqpoint{3.445398in}{4.743437in}}%
\pgfpathlineto{\pgfqpoint{3.411917in}{4.797407in}}%
\pgfpathclose%
\pgfusepath{fill}%
\end{pgfscope}%
\begin{pgfscope}%
\pgfpathrectangle{\pgfqpoint{1.020000in}{0.880000in}}{\pgfqpoint{6.160000in}{6.160000in}}%
\pgfusepath{clip}%
\pgfsetbuttcap%
\pgfsetroundjoin%
\definecolor{currentfill}{rgb}{0.353369,0.472069,0.892570}%
\pgfsetfillcolor{currentfill}%
\pgfsetlinewidth{0.000000pt}%
\definecolor{currentstroke}{rgb}{0.000000,0.000000,0.000000}%
\pgfsetstrokecolor{currentstroke}%
\pgfsetdash{}{0pt}%
\pgfpathmoveto{\pgfqpoint{5.646249in}{3.175015in}}%
\pgfpathlineto{\pgfqpoint{5.657864in}{3.162151in}}%
\pgfpathlineto{\pgfqpoint{5.669495in}{3.148904in}}%
\pgfpathlineto{\pgfqpoint{5.701967in}{3.152484in}}%
\pgfpathlineto{\pgfqpoint{5.734413in}{3.155643in}}%
\pgfpathlineto{\pgfqpoint{5.722744in}{3.169911in}}%
\pgfpathlineto{\pgfqpoint{5.711093in}{3.183961in}}%
\pgfpathlineto{\pgfqpoint{5.678685in}{3.179820in}}%
\pgfpathlineto{\pgfqpoint{5.646249in}{3.175015in}}%
\pgfpathclose%
\pgfusepath{fill}%
\end{pgfscope}%
\begin{pgfscope}%
\pgfpathrectangle{\pgfqpoint{1.020000in}{0.880000in}}{\pgfqpoint{6.160000in}{6.160000in}}%
\pgfusepath{clip}%
\pgfsetbuttcap%
\pgfsetroundjoin%
\definecolor{currentfill}{rgb}{0.576051,0.708780,0.997755}%
\pgfsetfillcolor{currentfill}%
\pgfsetlinewidth{0.000000pt}%
\definecolor{currentstroke}{rgb}{0.000000,0.000000,0.000000}%
\pgfsetstrokecolor{currentstroke}%
\pgfsetdash{}{0pt}%
\pgfpathmoveto{\pgfqpoint{4.167936in}{3.597543in}}%
\pgfpathlineto{\pgfqpoint{4.178192in}{3.575148in}}%
\pgfpathlineto{\pgfqpoint{4.188464in}{3.553452in}}%
\pgfpathlineto{\pgfqpoint{4.221366in}{3.537969in}}%
\pgfpathlineto{\pgfqpoint{4.254237in}{3.523189in}}%
\pgfpathlineto{\pgfqpoint{4.243918in}{3.543806in}}%
\pgfpathlineto{\pgfqpoint{4.233617in}{3.565030in}}%
\pgfpathlineto{\pgfqpoint{4.200792in}{3.580841in}}%
\pgfpathlineto{\pgfqpoint{4.167936in}{3.597543in}}%
\pgfpathclose%
\pgfusepath{fill}%
\end{pgfscope}%
\begin{pgfscope}%
\pgfpathrectangle{\pgfqpoint{1.020000in}{0.880000in}}{\pgfqpoint{6.160000in}{6.160000in}}%
\pgfusepath{clip}%
\pgfsetbuttcap%
\pgfsetroundjoin%
\definecolor{currentfill}{rgb}{0.918282,0.484173,0.377794}%
\pgfsetfillcolor{currentfill}%
\pgfsetlinewidth{0.000000pt}%
\definecolor{currentstroke}{rgb}{0.000000,0.000000,0.000000}%
\pgfsetstrokecolor{currentstroke}%
\pgfsetdash{}{0pt}%
\pgfpathmoveto{\pgfqpoint{3.325694in}{4.943396in}}%
\pgfpathlineto{\pgfqpoint{3.335238in}{4.924132in}}%
\pgfpathlineto{\pgfqpoint{3.344834in}{4.900251in}}%
\pgfpathlineto{\pgfqpoint{3.378394in}{4.849844in}}%
\pgfpathlineto{\pgfqpoint{3.411917in}{4.797407in}}%
\pgfpathlineto{\pgfqpoint{3.402288in}{4.820586in}}%
\pgfpathlineto{\pgfqpoint{3.392705in}{4.839557in}}%
\pgfpathlineto{\pgfqpoint{3.359218in}{4.892496in}}%
\pgfpathlineto{\pgfqpoint{3.325694in}{4.943396in}}%
\pgfpathclose%
\pgfusepath{fill}%
\end{pgfscope}%
\begin{pgfscope}%
\pgfpathrectangle{\pgfqpoint{1.020000in}{0.880000in}}{\pgfqpoint{6.160000in}{6.160000in}}%
\pgfusepath{clip}%
\pgfsetbuttcap%
\pgfsetroundjoin%
\definecolor{currentfill}{rgb}{0.921406,0.491420,0.383408}%
\pgfsetfillcolor{currentfill}%
\pgfsetlinewidth{0.000000pt}%
\definecolor{currentstroke}{rgb}{0.000000,0.000000,0.000000}%
\pgfsetstrokecolor{currentstroke}%
\pgfsetdash{}{0pt}%
\pgfpathmoveto{\pgfqpoint{2.390653in}{4.820276in}}%
\pgfpathlineto{\pgfqpoint{2.399010in}{4.824392in}}%
\pgfpathlineto{\pgfqpoint{2.407428in}{4.825876in}}%
\pgfpathlineto{\pgfqpoint{2.440215in}{4.868427in}}%
\pgfpathlineto{\pgfqpoint{2.473010in}{4.910838in}}%
\pgfpathlineto{\pgfqpoint{2.464566in}{4.906503in}}%
\pgfpathlineto{\pgfqpoint{2.456188in}{4.899231in}}%
\pgfpathlineto{\pgfqpoint{2.423418in}{4.859818in}}%
\pgfpathlineto{\pgfqpoint{2.390653in}{4.820276in}}%
\pgfpathclose%
\pgfusepath{fill}%
\end{pgfscope}%
\begin{pgfscope}%
\pgfpathrectangle{\pgfqpoint{1.020000in}{0.880000in}}{\pgfqpoint{6.160000in}{6.160000in}}%
\pgfusepath{clip}%
\pgfsetbuttcap%
\pgfsetroundjoin%
\definecolor{currentfill}{rgb}{0.328604,0.439712,0.869587}%
\pgfsetfillcolor{currentfill}%
\pgfsetlinewidth{0.000000pt}%
\definecolor{currentstroke}{rgb}{0.000000,0.000000,0.000000}%
\pgfsetstrokecolor{currentstroke}%
\pgfsetdash{}{0pt}%
\pgfpathmoveto{\pgfqpoint{5.362840in}{3.114565in}}%
\pgfpathlineto{\pgfqpoint{5.374338in}{3.114630in}}%
\pgfpathlineto{\pgfqpoint{5.385844in}{3.113311in}}%
\pgfpathlineto{\pgfqpoint{5.418467in}{3.122314in}}%
\pgfpathlineto{\pgfqpoint{5.451075in}{3.131335in}}%
\pgfpathlineto{\pgfqpoint{5.439547in}{3.136081in}}%
\pgfpathlineto{\pgfqpoint{5.428031in}{3.139766in}}%
\pgfpathlineto{\pgfqpoint{5.395442in}{3.127173in}}%
\pgfpathlineto{\pgfqpoint{5.362840in}{3.114565in}}%
\pgfpathclose%
\pgfusepath{fill}%
\end{pgfscope}%
\begin{pgfscope}%
\pgfpathrectangle{\pgfqpoint{1.020000in}{0.880000in}}{\pgfqpoint{6.160000in}{6.160000in}}%
\pgfusepath{clip}%
\pgfsetbuttcap%
\pgfsetroundjoin%
\definecolor{currentfill}{rgb}{0.795938,0.241845,0.220830}%
\pgfsetfillcolor{currentfill}%
\pgfsetlinewidth{0.000000pt}%
\definecolor{currentstroke}{rgb}{0.000000,0.000000,0.000000}%
\pgfsetstrokecolor{currentstroke}%
\pgfsetdash{}{0pt}%
\pgfpathmoveto{\pgfqpoint{2.653234in}{5.109483in}}%
\pgfpathlineto{\pgfqpoint{2.661712in}{5.125297in}}%
\pgfpathlineto{\pgfqpoint{2.670262in}{5.137361in}}%
\pgfpathlineto{\pgfqpoint{2.703265in}{5.166218in}}%
\pgfpathlineto{\pgfqpoint{2.736312in}{5.191394in}}%
\pgfpathlineto{\pgfqpoint{2.727714in}{5.177365in}}%
\pgfpathlineto{\pgfqpoint{2.719188in}{5.159385in}}%
\pgfpathlineto{\pgfqpoint{2.686190in}{5.136165in}}%
\pgfpathlineto{\pgfqpoint{2.653234in}{5.109483in}}%
\pgfpathclose%
\pgfusepath{fill}%
\end{pgfscope}%
\begin{pgfscope}%
\pgfpathrectangle{\pgfqpoint{1.020000in}{0.880000in}}{\pgfqpoint{6.160000in}{6.160000in}}%
\pgfusepath{clip}%
\pgfsetbuttcap%
\pgfsetroundjoin%
\definecolor{currentfill}{rgb}{0.343278,0.459354,0.884122}%
\pgfsetfillcolor{currentfill}%
\pgfsetlinewidth{0.000000pt}%
\definecolor{currentstroke}{rgb}{0.000000,0.000000,0.000000}%
\pgfsetstrokecolor{currentstroke}%
\pgfsetdash{}{0pt}%
\pgfpathmoveto{\pgfqpoint{5.863958in}{3.165066in}}%
\pgfpathlineto{\pgfqpoint{5.875736in}{3.149803in}}%
\pgfpathlineto{\pgfqpoint{5.887535in}{3.134495in}}%
\pgfpathlineto{\pgfqpoint{5.919913in}{3.136192in}}%
\pgfpathlineto{\pgfqpoint{5.908089in}{3.151546in}}%
\pgfpathlineto{\pgfqpoint{5.896286in}{3.166875in}}%
\pgfpathlineto{\pgfqpoint{5.863958in}{3.165066in}}%
\pgfpathclose%
\pgfusepath{fill}%
\end{pgfscope}%
\begin{pgfscope}%
\pgfpathrectangle{\pgfqpoint{1.020000in}{0.880000in}}{\pgfqpoint{6.160000in}{6.160000in}}%
\pgfusepath{clip}%
\pgfsetbuttcap%
\pgfsetroundjoin%
\definecolor{currentfill}{rgb}{0.892138,0.425389,0.333289}%
\pgfsetfillcolor{currentfill}%
\pgfsetlinewidth{0.000000pt}%
\definecolor{currentstroke}{rgb}{0.000000,0.000000,0.000000}%
\pgfsetstrokecolor{currentstroke}%
\pgfsetdash{}{0pt}%
\pgfpathmoveto{\pgfqpoint{2.456188in}{4.899231in}}%
\pgfpathlineto{\pgfqpoint{2.464566in}{4.906503in}}%
\pgfpathlineto{\pgfqpoint{2.473010in}{4.910838in}}%
\pgfpathlineto{\pgfqpoint{2.505818in}{4.952677in}}%
\pgfpathlineto{\pgfqpoint{2.538646in}{4.993497in}}%
\pgfpathlineto{\pgfqpoint{2.530172in}{4.986380in}}%
\pgfpathlineto{\pgfqpoint{2.521767in}{4.976032in}}%
\pgfpathlineto{\pgfqpoint{2.488969in}{4.938110in}}%
\pgfpathlineto{\pgfqpoint{2.456188in}{4.899231in}}%
\pgfpathclose%
\pgfusepath{fill}%
\end{pgfscope}%
\begin{pgfscope}%
\pgfpathrectangle{\pgfqpoint{1.020000in}{0.880000in}}{\pgfqpoint{6.160000in}{6.160000in}}%
\pgfusepath{clip}%
\pgfsetbuttcap%
\pgfsetroundjoin%
\definecolor{currentfill}{rgb}{0.825294,0.295749,0.250025}%
\pgfsetfillcolor{currentfill}%
\pgfsetlinewidth{0.000000pt}%
\definecolor{currentstroke}{rgb}{0.000000,0.000000,0.000000}%
\pgfsetstrokecolor{currentstroke}%
\pgfsetdash{}{0pt}%
\pgfpathmoveto{\pgfqpoint{2.587436in}{5.047271in}}%
\pgfpathlineto{\pgfqpoint{2.595874in}{5.060504in}}%
\pgfpathlineto{\pgfqpoint{2.604383in}{5.070231in}}%
\pgfpathlineto{\pgfqpoint{2.637303in}{5.105222in}}%
\pgfpathlineto{\pgfqpoint{2.670262in}{5.137361in}}%
\pgfpathlineto{\pgfqpoint{2.661712in}{5.125297in}}%
\pgfpathlineto{\pgfqpoint{2.653234in}{5.109483in}}%
\pgfpathlineto{\pgfqpoint{2.620317in}{5.079718in}}%
\pgfpathlineto{\pgfqpoint{2.587436in}{5.047271in}}%
\pgfpathclose%
\pgfusepath{fill}%
\end{pgfscope}%
\begin{pgfscope}%
\pgfpathrectangle{\pgfqpoint{1.020000in}{0.880000in}}{\pgfqpoint{6.160000in}{6.160000in}}%
\pgfusepath{clip}%
\pgfsetbuttcap%
\pgfsetroundjoin%
\definecolor{currentfill}{rgb}{0.861054,0.362916,0.290628}%
\pgfsetfillcolor{currentfill}%
\pgfsetlinewidth{0.000000pt}%
\definecolor{currentstroke}{rgb}{0.000000,0.000000,0.000000}%
\pgfsetstrokecolor{currentstroke}%
\pgfsetdash{}{0pt}%
\pgfpathmoveto{\pgfqpoint{2.521767in}{4.976032in}}%
\pgfpathlineto{\pgfqpoint{2.530172in}{4.986380in}}%
\pgfpathlineto{\pgfqpoint{2.538646in}{4.993497in}}%
\pgfpathlineto{\pgfqpoint{2.571499in}{5.032836in}}%
\pgfpathlineto{\pgfqpoint{2.604383in}{5.070231in}}%
\pgfpathlineto{\pgfqpoint{2.595874in}{5.060504in}}%
\pgfpathlineto{\pgfqpoint{2.587436in}{5.047271in}}%
\pgfpathlineto{\pgfqpoint{2.554587in}{5.012565in}}%
\pgfpathlineto{\pgfqpoint{2.521767in}{4.976032in}}%
\pgfpathclose%
\pgfusepath{fill}%
\end{pgfscope}%
\begin{pgfscope}%
\pgfpathrectangle{\pgfqpoint{1.020000in}{0.880000in}}{\pgfqpoint{6.160000in}{6.160000in}}%
\pgfusepath{clip}%
\pgfsetbuttcap%
\pgfsetroundjoin%
\definecolor{currentfill}{rgb}{0.318832,0.426605,0.859857}%
\pgfsetfillcolor{currentfill}%
\pgfsetlinewidth{0.000000pt}%
\definecolor{currentstroke}{rgb}{0.000000,0.000000,0.000000}%
\pgfsetstrokecolor{currentstroke}%
\pgfsetdash{}{0pt}%
\pgfpathmoveto{\pgfqpoint{4.862138in}{3.115963in}}%
\pgfpathlineto{\pgfqpoint{4.873126in}{3.117431in}}%
\pgfpathlineto{\pgfqpoint{4.884143in}{3.119555in}}%
\pgfpathlineto{\pgfqpoint{4.916717in}{3.101637in}}%
\pgfpathlineto{\pgfqpoint{4.949269in}{3.085114in}}%
\pgfpathlineto{\pgfqpoint{4.938167in}{3.079175in}}%
\pgfpathlineto{\pgfqpoint{4.927095in}{3.073975in}}%
\pgfpathlineto{\pgfqpoint{4.894628in}{3.094152in}}%
\pgfpathlineto{\pgfqpoint{4.862138in}{3.115963in}}%
\pgfpathclose%
\pgfusepath{fill}%
\end{pgfscope}%
\begin{pgfscope}%
\pgfpathrectangle{\pgfqpoint{1.020000in}{0.880000in}}{\pgfqpoint{6.160000in}{6.160000in}}%
\pgfusepath{clip}%
\pgfsetbuttcap%
\pgfsetroundjoin%
\definecolor{currentfill}{rgb}{0.368507,0.491141,0.905243}%
\pgfsetfillcolor{currentfill}%
\pgfsetlinewidth{0.000000pt}%
\definecolor{currentstroke}{rgb}{0.000000,0.000000,0.000000}%
\pgfsetstrokecolor{currentstroke}%
\pgfsetdash{}{0pt}%
\pgfpathmoveto{\pgfqpoint{4.710256in}{3.221194in}}%
\pgfpathlineto{\pgfqpoint{4.721038in}{3.213056in}}%
\pgfpathlineto{\pgfqpoint{4.731847in}{3.206177in}}%
\pgfpathlineto{\pgfqpoint{4.764476in}{3.184308in}}%
\pgfpathlineto{\pgfqpoint{4.797066in}{3.161660in}}%
\pgfpathlineto{\pgfqpoint{4.786185in}{3.165189in}}%
\pgfpathlineto{\pgfqpoint{4.775336in}{3.170329in}}%
\pgfpathlineto{\pgfqpoint{4.742817in}{3.196272in}}%
\pgfpathlineto{\pgfqpoint{4.710256in}{3.221194in}}%
\pgfpathclose%
\pgfusepath{fill}%
\end{pgfscope}%
\begin{pgfscope}%
\pgfpathrectangle{\pgfqpoint{1.020000in}{0.880000in}}{\pgfqpoint{6.160000in}{6.160000in}}%
\pgfusepath{clip}%
\pgfsetbuttcap%
\pgfsetroundjoin%
\definecolor{currentfill}{rgb}{0.646113,0.764436,0.996868}%
\pgfsetfillcolor{currentfill}%
\pgfsetlinewidth{0.000000pt}%
\definecolor{currentstroke}{rgb}{0.000000,0.000000,0.000000}%
\pgfsetstrokecolor{currentstroke}%
\pgfsetdash{}{0pt}%
\pgfpathmoveto{\pgfqpoint{4.015883in}{3.729608in}}%
\pgfpathlineto{\pgfqpoint{4.026025in}{3.702219in}}%
\pgfpathlineto{\pgfqpoint{4.036181in}{3.675492in}}%
\pgfpathlineto{\pgfqpoint{4.069172in}{3.654193in}}%
\pgfpathlineto{\pgfqpoint{4.102127in}{3.634133in}}%
\pgfpathlineto{\pgfqpoint{4.091931in}{3.658787in}}%
\pgfpathlineto{\pgfqpoint{4.081750in}{3.684007in}}%
\pgfpathlineto{\pgfqpoint{4.048835in}{3.706085in}}%
\pgfpathlineto{\pgfqpoint{4.015883in}{3.729608in}}%
\pgfpathclose%
\pgfusepath{fill}%
\end{pgfscope}%
\begin{pgfscope}%
\pgfpathrectangle{\pgfqpoint{1.020000in}{0.880000in}}{\pgfqpoint{6.160000in}{6.160000in}}%
\pgfusepath{clip}%
\pgfsetbuttcap%
\pgfsetroundjoin%
\definecolor{currentfill}{rgb}{0.735077,0.104460,0.171492}%
\pgfsetfillcolor{currentfill}%
\pgfsetlinewidth{0.000000pt}%
\definecolor{currentstroke}{rgb}{0.000000,0.000000,0.000000}%
\pgfsetstrokecolor{currentstroke}%
\pgfsetdash{}{0pt}%
\pgfpathmoveto{\pgfqpoint{2.935551in}{5.248101in}}%
\pgfpathlineto{\pgfqpoint{2.944407in}{5.260431in}}%
\pgfpathlineto{\pgfqpoint{2.953340in}{5.267805in}}%
\pgfpathlineto{\pgfqpoint{2.986753in}{5.259858in}}%
\pgfpathlineto{\pgfqpoint{3.020196in}{5.246587in}}%
\pgfpathlineto{\pgfqpoint{3.011198in}{5.239280in}}%
\pgfpathlineto{\pgfqpoint{3.002273in}{5.227067in}}%
\pgfpathlineto{\pgfqpoint{2.968897in}{5.240129in}}%
\pgfpathlineto{\pgfqpoint{2.935551in}{5.248101in}}%
\pgfpathclose%
\pgfusepath{fill}%
\end{pgfscope}%
\begin{pgfscope}%
\pgfpathrectangle{\pgfqpoint{1.020000in}{0.880000in}}{\pgfqpoint{6.160000in}{6.160000in}}%
\pgfusepath{clip}%
\pgfsetbuttcap%
\pgfsetroundjoin%
\definecolor{currentfill}{rgb}{0.348323,0.465711,0.888346}%
\pgfsetfillcolor{currentfill}%
\pgfsetlinewidth{0.000000pt}%
\definecolor{currentstroke}{rgb}{0.000000,0.000000,0.000000}%
\pgfsetstrokecolor{currentstroke}%
\pgfsetdash{}{0pt}%
\pgfpathmoveto{\pgfqpoint{5.581293in}{3.163208in}}%
\pgfpathlineto{\pgfqpoint{5.592878in}{3.152090in}}%
\pgfpathlineto{\pgfqpoint{5.604476in}{3.140356in}}%
\pgfpathlineto{\pgfqpoint{5.636998in}{3.144868in}}%
\pgfpathlineto{\pgfqpoint{5.669495in}{3.148904in}}%
\pgfpathlineto{\pgfqpoint{5.657864in}{3.162151in}}%
\pgfpathlineto{\pgfqpoint{5.646249in}{3.175015in}}%
\pgfpathlineto{\pgfqpoint{5.613785in}{3.169488in}}%
\pgfpathlineto{\pgfqpoint{5.581293in}{3.163208in}}%
\pgfpathclose%
\pgfusepath{fill}%
\end{pgfscope}%
\begin{pgfscope}%
\pgfpathrectangle{\pgfqpoint{1.020000in}{0.880000in}}{\pgfqpoint{6.160000in}{6.160000in}}%
\pgfusepath{clip}%
\pgfsetbuttcap%
\pgfsetroundjoin%
\definecolor{currentfill}{rgb}{0.430507,0.564883,0.948889}%
\pgfsetfillcolor{currentfill}%
\pgfsetlinewidth{0.000000pt}%
\definecolor{currentstroke}{rgb}{0.000000,0.000000,0.000000}%
\pgfsetstrokecolor{currentstroke}%
\pgfsetdash{}{0pt}%
\pgfpathmoveto{\pgfqpoint{4.558336in}{3.332818in}}%
\pgfpathlineto{\pgfqpoint{4.568951in}{3.318432in}}%
\pgfpathlineto{\pgfqpoint{4.579589in}{3.305124in}}%
\pgfpathlineto{\pgfqpoint{4.612317in}{3.286698in}}%
\pgfpathlineto{\pgfqpoint{4.645005in}{3.266570in}}%
\pgfpathlineto{\pgfqpoint{4.634312in}{3.278369in}}%
\pgfpathlineto{\pgfqpoint{4.623644in}{3.291584in}}%
\pgfpathlineto{\pgfqpoint{4.591010in}{3.313169in}}%
\pgfpathlineto{\pgfqpoint{4.558336in}{3.332818in}}%
\pgfpathclose%
\pgfusepath{fill}%
\end{pgfscope}%
\begin{pgfscope}%
\pgfpathrectangle{\pgfqpoint{1.020000in}{0.880000in}}{\pgfqpoint{6.160000in}{6.160000in}}%
\pgfusepath{clip}%
\pgfsetbuttcap%
\pgfsetroundjoin%
\definecolor{currentfill}{rgb}{0.323718,0.433158,0.864722}%
\pgfsetfillcolor{currentfill}%
\pgfsetlinewidth{0.000000pt}%
\definecolor{currentstroke}{rgb}{0.000000,0.000000,0.000000}%
\pgfsetstrokecolor{currentstroke}%
\pgfsetdash{}{0pt}%
\pgfpathmoveto{\pgfqpoint{5.297613in}{3.091019in}}%
\pgfpathlineto{\pgfqpoint{5.309085in}{3.094640in}}%
\pgfpathlineto{\pgfqpoint{5.320564in}{3.096575in}}%
\pgfpathlineto{\pgfqpoint{5.353209in}{3.104624in}}%
\pgfpathlineto{\pgfqpoint{5.385844in}{3.113311in}}%
\pgfpathlineto{\pgfqpoint{5.374338in}{3.114630in}}%
\pgfpathlineto{\pgfqpoint{5.362840in}{3.114565in}}%
\pgfpathlineto{\pgfqpoint{5.330229in}{3.102362in}}%
\pgfpathlineto{\pgfqpoint{5.297613in}{3.091019in}}%
\pgfpathclose%
\pgfusepath{fill}%
\end{pgfscope}%
\begin{pgfscope}%
\pgfpathrectangle{\pgfqpoint{1.020000in}{0.880000in}}{\pgfqpoint{6.160000in}{6.160000in}}%
\pgfusepath{clip}%
\pgfsetbuttcap%
\pgfsetroundjoin%
\definecolor{currentfill}{rgb}{0.343278,0.459354,0.884122}%
\pgfsetfillcolor{currentfill}%
\pgfsetlinewidth{0.000000pt}%
\definecolor{currentstroke}{rgb}{0.000000,0.000000,0.000000}%
\pgfsetstrokecolor{currentstroke}%
\pgfsetdash{}{0pt}%
\pgfpathmoveto{\pgfqpoint{5.799233in}{3.160891in}}%
\pgfpathlineto{\pgfqpoint{5.810964in}{3.145892in}}%
\pgfpathlineto{\pgfqpoint{5.822716in}{3.130788in}}%
\pgfpathlineto{\pgfqpoint{5.855137in}{3.132703in}}%
\pgfpathlineto{\pgfqpoint{5.887535in}{3.134495in}}%
\pgfpathlineto{\pgfqpoint{5.875736in}{3.149803in}}%
\pgfpathlineto{\pgfqpoint{5.863958in}{3.165066in}}%
\pgfpathlineto{\pgfqpoint{5.831607in}{3.163087in}}%
\pgfpathlineto{\pgfqpoint{5.799233in}{3.160891in}}%
\pgfpathclose%
\pgfusepath{fill}%
\end{pgfscope}%
\begin{pgfscope}%
\pgfpathrectangle{\pgfqpoint{1.020000in}{0.880000in}}{\pgfqpoint{6.160000in}{6.160000in}}%
\pgfusepath{clip}%
\pgfsetbuttcap%
\pgfsetroundjoin%
\definecolor{currentfill}{rgb}{0.779745,0.210305,0.207104}%
\pgfsetfillcolor{currentfill}%
\pgfsetlinewidth{0.000000pt}%
\definecolor{currentstroke}{rgb}{0.000000,0.000000,0.000000}%
\pgfsetstrokecolor{currentstroke}%
\pgfsetdash{}{0pt}%
\pgfpathmoveto{\pgfqpoint{3.087151in}{5.204571in}}%
\pgfpathlineto{\pgfqpoint{3.096287in}{5.206333in}}%
\pgfpathlineto{\pgfqpoint{3.105495in}{5.202805in}}%
\pgfpathlineto{\pgfqpoint{3.139054in}{5.173987in}}%
\pgfpathlineto{\pgfqpoint{3.172617in}{5.140528in}}%
\pgfpathlineto{\pgfqpoint{3.163350in}{5.144491in}}%
\pgfpathlineto{\pgfqpoint{3.154150in}{5.143386in}}%
\pgfpathlineto{\pgfqpoint{3.120648in}{5.176236in}}%
\pgfpathlineto{\pgfqpoint{3.087151in}{5.204571in}}%
\pgfpathclose%
\pgfusepath{fill}%
\end{pgfscope}%
\begin{pgfscope}%
\pgfpathrectangle{\pgfqpoint{1.020000in}{0.880000in}}{\pgfqpoint{6.160000in}{6.160000in}}%
\pgfusepath{clip}%
\pgfsetbuttcap%
\pgfsetroundjoin%
\definecolor{currentfill}{rgb}{0.299441,0.400248,0.839842}%
\pgfsetfillcolor{currentfill}%
\pgfsetlinewidth{0.000000pt}%
\definecolor{currentstroke}{rgb}{0.000000,0.000000,0.000000}%
\pgfsetstrokecolor{currentstroke}%
\pgfsetdash{}{0pt}%
\pgfpathmoveto{\pgfqpoint{5.014329in}{3.058510in}}%
\pgfpathlineto{\pgfqpoint{5.025540in}{3.067404in}}%
\pgfpathlineto{\pgfqpoint{5.036770in}{3.075479in}}%
\pgfpathlineto{\pgfqpoint{5.069371in}{3.068388in}}%
\pgfpathlineto{\pgfqpoint{5.101968in}{3.063866in}}%
\pgfpathlineto{\pgfqpoint{5.090661in}{3.054234in}}%
\pgfpathlineto{\pgfqpoint{5.079371in}{3.043707in}}%
\pgfpathlineto{\pgfqpoint{5.046849in}{3.049441in}}%
\pgfpathlineto{\pgfqpoint{5.014329in}{3.058510in}}%
\pgfpathclose%
\pgfusepath{fill}%
\end{pgfscope}%
\begin{pgfscope}%
\pgfpathrectangle{\pgfqpoint{1.020000in}{0.880000in}}{\pgfqpoint{6.160000in}{6.160000in}}%
\pgfusepath{clip}%
\pgfsetbuttcap%
\pgfsetroundjoin%
\definecolor{currentfill}{rgb}{0.483854,0.622050,0.974808}%
\pgfsetfillcolor{currentfill}%
\pgfsetlinewidth{0.000000pt}%
\definecolor{currentstroke}{rgb}{0.000000,0.000000,0.000000}%
\pgfsetstrokecolor{currentstroke}%
\pgfsetdash{}{0pt}%
\pgfpathmoveto{\pgfqpoint{4.406306in}{3.429999in}}%
\pgfpathlineto{\pgfqpoint{4.416778in}{3.412621in}}%
\pgfpathlineto{\pgfqpoint{4.427270in}{3.396058in}}%
\pgfpathlineto{\pgfqpoint{4.460087in}{3.382006in}}%
\pgfpathlineto{\pgfqpoint{4.492872in}{3.366981in}}%
\pgfpathlineto{\pgfqpoint{4.482331in}{3.383100in}}%
\pgfpathlineto{\pgfqpoint{4.471810in}{3.400150in}}%
\pgfpathlineto{\pgfqpoint{4.439075in}{3.415566in}}%
\pgfpathlineto{\pgfqpoint{4.406306in}{3.429999in}}%
\pgfpathclose%
\pgfusepath{fill}%
\end{pgfscope}%
\begin{pgfscope}%
\pgfpathrectangle{\pgfqpoint{1.020000in}{0.880000in}}{\pgfqpoint{6.160000in}{6.160000in}}%
\pgfusepath{clip}%
\pgfsetbuttcap%
\pgfsetroundjoin%
\definecolor{currentfill}{rgb}{0.538004,0.674902,0.991722}%
\pgfsetfillcolor{currentfill}%
\pgfsetlinewidth{0.000000pt}%
\definecolor{currentstroke}{rgb}{0.000000,0.000000,0.000000}%
\pgfsetstrokecolor{currentstroke}%
\pgfsetdash{}{0pt}%
\pgfpathmoveto{\pgfqpoint{4.254237in}{3.523189in}}%
\pgfpathlineto{\pgfqpoint{4.264573in}{3.503266in}}%
\pgfpathlineto{\pgfqpoint{4.274929in}{3.484113in}}%
\pgfpathlineto{\pgfqpoint{4.307818in}{3.470567in}}%
\pgfpathlineto{\pgfqpoint{4.340677in}{3.457203in}}%
\pgfpathlineto{\pgfqpoint{4.330274in}{3.475749in}}%
\pgfpathlineto{\pgfqpoint{4.319889in}{3.495006in}}%
\pgfpathlineto{\pgfqpoint{4.287078in}{3.508940in}}%
\pgfpathlineto{\pgfqpoint{4.254237in}{3.523189in}}%
\pgfpathclose%
\pgfusepath{fill}%
\end{pgfscope}%
\begin{pgfscope}%
\pgfpathrectangle{\pgfqpoint{1.020000in}{0.880000in}}{\pgfqpoint{6.160000in}{6.160000in}}%
\pgfusepath{clip}%
\pgfsetbuttcap%
\pgfsetroundjoin%
\definecolor{currentfill}{rgb}{0.875557,0.860242,0.851430}%
\pgfsetfillcolor{currentfill}%
\pgfsetlinewidth{0.000000pt}%
\definecolor{currentstroke}{rgb}{0.000000,0.000000,0.000000}%
\pgfsetstrokecolor{currentstroke}%
\pgfsetdash{}{0pt}%
\pgfpathmoveto{\pgfqpoint{3.690853in}{4.210343in}}%
\pgfpathlineto{\pgfqpoint{3.700806in}{4.170412in}}%
\pgfpathlineto{\pgfqpoint{3.710775in}{4.129676in}}%
\pgfpathlineto{\pgfqpoint{3.744033in}{4.088986in}}%
\pgfpathlineto{\pgfqpoint{3.777240in}{4.049309in}}%
\pgfpathlineto{\pgfqpoint{3.767255in}{4.086449in}}%
\pgfpathlineto{\pgfqpoint{3.757286in}{4.122922in}}%
\pgfpathlineto{\pgfqpoint{3.724096in}{4.166068in}}%
\pgfpathlineto{\pgfqpoint{3.690853in}{4.210343in}}%
\pgfpathclose%
\pgfusepath{fill}%
\end{pgfscope}%
\begin{pgfscope}%
\pgfpathrectangle{\pgfqpoint{1.020000in}{0.880000in}}{\pgfqpoint{6.160000in}{6.160000in}}%
\pgfusepath{clip}%
\pgfsetbuttcap%
\pgfsetroundjoin%
\definecolor{currentfill}{rgb}{0.809329,0.852974,0.922323}%
\pgfsetfillcolor{currentfill}%
\pgfsetlinewidth{0.000000pt}%
\definecolor{currentstroke}{rgb}{0.000000,0.000000,0.000000}%
\pgfsetstrokecolor{currentstroke}%
\pgfsetdash{}{0pt}%
\pgfpathmoveto{\pgfqpoint{3.777240in}{4.049309in}}%
\pgfpathlineto{\pgfqpoint{3.787241in}{4.011780in}}%
\pgfpathlineto{\pgfqpoint{3.797255in}{3.974140in}}%
\pgfpathlineto{\pgfqpoint{3.830433in}{3.939205in}}%
\pgfpathlineto{\pgfqpoint{3.863565in}{3.905524in}}%
\pgfpathlineto{\pgfqpoint{3.853528in}{3.939713in}}%
\pgfpathlineto{\pgfqpoint{3.843506in}{3.973836in}}%
\pgfpathlineto{\pgfqpoint{3.810398in}{4.010863in}}%
\pgfpathlineto{\pgfqpoint{3.777240in}{4.049309in}}%
\pgfpathclose%
\pgfusepath{fill}%
\end{pgfscope}%
\begin{pgfscope}%
\pgfpathrectangle{\pgfqpoint{1.020000in}{0.880000in}}{\pgfqpoint{6.160000in}{6.160000in}}%
\pgfusepath{clip}%
\pgfsetbuttcap%
\pgfsetroundjoin%
\definecolor{currentfill}{rgb}{0.603162,0.731527,0.999565}%
\pgfsetfillcolor{currentfill}%
\pgfsetlinewidth{0.000000pt}%
\definecolor{currentstroke}{rgb}{0.000000,0.000000,0.000000}%
\pgfsetstrokecolor{currentstroke}%
\pgfsetdash{}{0pt}%
\pgfpathmoveto{\pgfqpoint{4.102127in}{3.634133in}}%
\pgfpathlineto{\pgfqpoint{4.112339in}{3.610177in}}%
\pgfpathlineto{\pgfqpoint{4.122567in}{3.587043in}}%
\pgfpathlineto{\pgfqpoint{4.155532in}{3.569777in}}%
\pgfpathlineto{\pgfqpoint{4.188464in}{3.553452in}}%
\pgfpathlineto{\pgfqpoint{4.178192in}{3.575148in}}%
\pgfpathlineto{\pgfqpoint{4.167936in}{3.597543in}}%
\pgfpathlineto{\pgfqpoint{4.135048in}{3.615273in}}%
\pgfpathlineto{\pgfqpoint{4.102127in}{3.634133in}}%
\pgfpathclose%
\pgfusepath{fill}%
\end{pgfscope}%
\begin{pgfscope}%
\pgfpathrectangle{\pgfqpoint{1.020000in}{0.880000in}}{\pgfqpoint{6.160000in}{6.160000in}}%
\pgfusepath{clip}%
\pgfsetbuttcap%
\pgfsetroundjoin%
\definecolor{currentfill}{rgb}{0.935774,0.812237,0.747156}%
\pgfsetfillcolor{currentfill}%
\pgfsetlinewidth{0.000000pt}%
\definecolor{currentstroke}{rgb}{0.000000,0.000000,0.000000}%
\pgfsetstrokecolor{currentstroke}%
\pgfsetdash{}{0pt}%
\pgfpathmoveto{\pgfqpoint{3.604386in}{4.384054in}}%
\pgfpathlineto{\pgfqpoint{3.614288in}{4.343433in}}%
\pgfpathlineto{\pgfqpoint{3.624211in}{4.301168in}}%
\pgfpathlineto{\pgfqpoint{3.657558in}{4.255477in}}%
\pgfpathlineto{\pgfqpoint{3.690853in}{4.210343in}}%
\pgfpathlineto{\pgfqpoint{3.680918in}{4.249161in}}%
\pgfpathlineto{\pgfqpoint{3.671003in}{4.286566in}}%
\pgfpathlineto{\pgfqpoint{3.637722in}{4.335002in}}%
\pgfpathlineto{\pgfqpoint{3.604386in}{4.384054in}}%
\pgfpathclose%
\pgfusepath{fill}%
\end{pgfscope}%
\begin{pgfscope}%
\pgfpathrectangle{\pgfqpoint{1.020000in}{0.880000in}}{\pgfqpoint{6.160000in}{6.160000in}}%
\pgfusepath{clip}%
\pgfsetbuttcap%
\pgfsetroundjoin%
\definecolor{currentfill}{rgb}{0.313946,0.420052,0.854993}%
\pgfsetfillcolor{currentfill}%
\pgfsetlinewidth{0.000000pt}%
\definecolor{currentstroke}{rgb}{0.000000,0.000000,0.000000}%
\pgfsetstrokecolor{currentstroke}%
\pgfsetdash{}{0pt}%
\pgfpathmoveto{\pgfqpoint{5.232381in}{3.072751in}}%
\pgfpathlineto{\pgfqpoint{5.243817in}{3.079184in}}%
\pgfpathlineto{\pgfqpoint{5.255258in}{3.083696in}}%
\pgfpathlineto{\pgfqpoint{5.287913in}{3.089495in}}%
\pgfpathlineto{\pgfqpoint{5.320564in}{3.096575in}}%
\pgfpathlineto{\pgfqpoint{5.309085in}{3.094640in}}%
\pgfpathlineto{\pgfqpoint{5.297613in}{3.091019in}}%
\pgfpathlineto{\pgfqpoint{5.264996in}{3.081002in}}%
\pgfpathlineto{\pgfqpoint{5.232381in}{3.072751in}}%
\pgfpathclose%
\pgfusepath{fill}%
\end{pgfscope}%
\begin{pgfscope}%
\pgfpathrectangle{\pgfqpoint{1.020000in}{0.880000in}}{\pgfqpoint{6.160000in}{6.160000in}}%
\pgfusepath{clip}%
\pgfsetbuttcap%
\pgfsetroundjoin%
\definecolor{currentfill}{rgb}{0.738826,0.822572,0.968261}%
\pgfsetfillcolor{currentfill}%
\pgfsetlinewidth{0.000000pt}%
\definecolor{currentstroke}{rgb}{0.000000,0.000000,0.000000}%
\pgfsetstrokecolor{currentstroke}%
\pgfsetdash{}{0pt}%
\pgfpathmoveto{\pgfqpoint{3.863565in}{3.905524in}}%
\pgfpathlineto{\pgfqpoint{3.873615in}{3.871511in}}%
\pgfpathlineto{\pgfqpoint{3.883678in}{3.837911in}}%
\pgfpathlineto{\pgfqpoint{3.916791in}{3.808797in}}%
\pgfpathlineto{\pgfqpoint{3.949863in}{3.780997in}}%
\pgfpathlineto{\pgfqpoint{3.939770in}{3.811542in}}%
\pgfpathlineto{\pgfqpoint{3.929692in}{3.842462in}}%
\pgfpathlineto{\pgfqpoint{3.896650in}{3.873239in}}%
\pgfpathlineto{\pgfqpoint{3.863565in}{3.905524in}}%
\pgfpathclose%
\pgfusepath{fill}%
\end{pgfscope}%
\begin{pgfscope}%
\pgfpathrectangle{\pgfqpoint{1.020000in}{0.880000in}}{\pgfqpoint{6.160000in}{6.160000in}}%
\pgfusepath{clip}%
\pgfsetbuttcap%
\pgfsetroundjoin%
\definecolor{currentfill}{rgb}{0.348323,0.465711,0.888346}%
\pgfsetfillcolor{currentfill}%
\pgfsetlinewidth{0.000000pt}%
\definecolor{currentstroke}{rgb}{0.000000,0.000000,0.000000}%
\pgfsetstrokecolor{currentstroke}%
\pgfsetdash{}{0pt}%
\pgfpathmoveto{\pgfqpoint{5.516231in}{3.148449in}}%
\pgfpathlineto{\pgfqpoint{5.527790in}{3.139664in}}%
\pgfpathlineto{\pgfqpoint{5.539360in}{3.129955in}}%
\pgfpathlineto{\pgfqpoint{5.571930in}{3.135374in}}%
\pgfpathlineto{\pgfqpoint{5.604476in}{3.140356in}}%
\pgfpathlineto{\pgfqpoint{5.592878in}{3.152090in}}%
\pgfpathlineto{\pgfqpoint{5.581293in}{3.163208in}}%
\pgfpathlineto{\pgfqpoint{5.548775in}{3.156179in}}%
\pgfpathlineto{\pgfqpoint{5.516231in}{3.148449in}}%
\pgfpathclose%
\pgfusepath{fill}%
\end{pgfscope}%
\begin{pgfscope}%
\pgfpathrectangle{\pgfqpoint{1.020000in}{0.880000in}}{\pgfqpoint{6.160000in}{6.160000in}}%
\pgfusepath{clip}%
\pgfsetbuttcap%
\pgfsetroundjoin%
\definecolor{currentfill}{rgb}{0.348323,0.465711,0.888346}%
\pgfsetfillcolor{currentfill}%
\pgfsetlinewidth{0.000000pt}%
\definecolor{currentstroke}{rgb}{0.000000,0.000000,0.000000}%
\pgfsetstrokecolor{currentstroke}%
\pgfsetdash{}{0pt}%
\pgfpathmoveto{\pgfqpoint{5.734413in}{3.155643in}}%
\pgfpathlineto{\pgfqpoint{5.746101in}{3.141157in}}%
\pgfpathlineto{\pgfqpoint{5.757807in}{3.126467in}}%
\pgfpathlineto{\pgfqpoint{5.790273in}{3.128720in}}%
\pgfpathlineto{\pgfqpoint{5.822716in}{3.130788in}}%
\pgfpathlineto{\pgfqpoint{5.810964in}{3.145892in}}%
\pgfpathlineto{\pgfqpoint{5.799233in}{3.160891in}}%
\pgfpathlineto{\pgfqpoint{5.766835in}{3.158428in}}%
\pgfpathlineto{\pgfqpoint{5.734413in}{3.155643in}}%
\pgfpathclose%
\pgfusepath{fill}%
\end{pgfscope}%
\begin{pgfscope}%
\pgfpathrectangle{\pgfqpoint{1.020000in}{0.880000in}}{\pgfqpoint{6.160000in}{6.160000in}}%
\pgfusepath{clip}%
\pgfsetbuttcap%
\pgfsetroundjoin%
\definecolor{currentfill}{rgb}{0.966962,0.735670,0.630877}%
\pgfsetfillcolor{currentfill}%
\pgfsetlinewidth{0.000000pt}%
\definecolor{currentstroke}{rgb}{0.000000,0.000000,0.000000}%
\pgfsetstrokecolor{currentstroke}%
\pgfsetdash{}{0pt}%
\pgfpathmoveto{\pgfqpoint{3.517854in}{4.563310in}}%
\pgfpathlineto{\pgfqpoint{3.527692in}{4.524223in}}%
\pgfpathlineto{\pgfqpoint{3.537559in}{4.482561in}}%
\pgfpathlineto{\pgfqpoint{3.570998in}{4.433369in}}%
\pgfpathlineto{\pgfqpoint{3.604386in}{4.384054in}}%
\pgfpathlineto{\pgfqpoint{3.594508in}{4.422711in}}%
\pgfpathlineto{\pgfqpoint{3.584656in}{4.459105in}}%
\pgfpathlineto{\pgfqpoint{3.551281in}{4.511265in}}%
\pgfpathlineto{\pgfqpoint{3.517854in}{4.563310in}}%
\pgfpathclose%
\pgfusepath{fill}%
\end{pgfscope}%
\begin{pgfscope}%
\pgfpathrectangle{\pgfqpoint{1.020000in}{0.880000in}}{\pgfqpoint{6.160000in}{6.160000in}}%
\pgfusepath{clip}%
\pgfsetbuttcap%
\pgfsetroundjoin%
\definecolor{currentfill}{rgb}{0.820401,0.286765,0.245160}%
\pgfsetfillcolor{currentfill}%
\pgfsetlinewidth{0.000000pt}%
\definecolor{currentstroke}{rgb}{0.000000,0.000000,0.000000}%
\pgfsetstrokecolor{currentstroke}%
\pgfsetdash{}{0pt}%
\pgfpathmoveto{\pgfqpoint{3.172617in}{5.140528in}}%
\pgfpathlineto{\pgfqpoint{3.181950in}{5.131341in}}%
\pgfpathlineto{\pgfqpoint{3.191349in}{5.116841in}}%
\pgfpathlineto{\pgfqpoint{3.224959in}{5.078934in}}%
\pgfpathlineto{\pgfqpoint{3.258557in}{5.037093in}}%
\pgfpathlineto{\pgfqpoint{3.249112in}{5.051622in}}%
\pgfpathlineto{\pgfqpoint{3.239727in}{5.061143in}}%
\pgfpathlineto{\pgfqpoint{3.206177in}{5.102781in}}%
\pgfpathlineto{\pgfqpoint{3.172617in}{5.140528in}}%
\pgfpathclose%
\pgfusepath{fill}%
\end{pgfscope}%
\begin{pgfscope}%
\pgfpathrectangle{\pgfqpoint{1.020000in}{0.880000in}}{\pgfqpoint{6.160000in}{6.160000in}}%
\pgfusepath{clip}%
\pgfsetbuttcap%
\pgfsetroundjoin%
\definecolor{currentfill}{rgb}{0.723315,0.068898,0.162989}%
\pgfsetfillcolor{currentfill}%
\pgfsetlinewidth{0.000000pt}%
\definecolor{currentstroke}{rgb}{0.000000,0.000000,0.000000}%
\pgfsetstrokecolor{currentstroke}%
\pgfsetdash{}{0pt}%
\pgfpathmoveto{\pgfqpoint{2.868966in}{5.248682in}}%
\pgfpathlineto{\pgfqpoint{2.877757in}{5.260622in}}%
\pgfpathlineto{\pgfqpoint{2.886626in}{5.267634in}}%
\pgfpathlineto{\pgfqpoint{2.919963in}{5.270388in}}%
\pgfpathlineto{\pgfqpoint{2.953340in}{5.267805in}}%
\pgfpathlineto{\pgfqpoint{2.944407in}{5.260431in}}%
\pgfpathlineto{\pgfqpoint{2.935551in}{5.248101in}}%
\pgfpathlineto{\pgfqpoint{2.902239in}{5.250943in}}%
\pgfpathlineto{\pgfqpoint{2.868966in}{5.248682in}}%
\pgfpathclose%
\pgfusepath{fill}%
\end{pgfscope}%
\begin{pgfscope}%
\pgfpathrectangle{\pgfqpoint{1.020000in}{0.880000in}}{\pgfqpoint{6.160000in}{6.160000in}}%
\pgfusepath{clip}%
\pgfsetbuttcap%
\pgfsetroundjoin%
\definecolor{currentfill}{rgb}{0.343278,0.459354,0.884122}%
\pgfsetfillcolor{currentfill}%
\pgfsetlinewidth{0.000000pt}%
\definecolor{currentstroke}{rgb}{0.000000,0.000000,0.000000}%
\pgfsetstrokecolor{currentstroke}%
\pgfsetdash{}{0pt}%
\pgfpathmoveto{\pgfqpoint{4.797066in}{3.161660in}}%
\pgfpathlineto{\pgfqpoint{4.807974in}{3.159222in}}%
\pgfpathlineto{\pgfqpoint{4.818909in}{3.157359in}}%
\pgfpathlineto{\pgfqpoint{4.851541in}{3.138302in}}%
\pgfpathlineto{\pgfqpoint{4.884143in}{3.119555in}}%
\pgfpathlineto{\pgfqpoint{4.873126in}{3.117431in}}%
\pgfpathlineto{\pgfqpoint{4.862138in}{3.115963in}}%
\pgfpathlineto{\pgfqpoint{4.829618in}{3.138687in}}%
\pgfpathlineto{\pgfqpoint{4.797066in}{3.161660in}}%
\pgfpathclose%
\pgfusepath{fill}%
\end{pgfscope}%
\begin{pgfscope}%
\pgfpathrectangle{\pgfqpoint{1.020000in}{0.880000in}}{\pgfqpoint{6.160000in}{6.160000in}}%
\pgfusepath{clip}%
\pgfsetbuttcap%
\pgfsetroundjoin%
\definecolor{currentfill}{rgb}{0.399231,0.528528,0.928459}%
\pgfsetfillcolor{currentfill}%
\pgfsetlinewidth{0.000000pt}%
\definecolor{currentstroke}{rgb}{0.000000,0.000000,0.000000}%
\pgfsetstrokecolor{currentstroke}%
\pgfsetdash{}{0pt}%
\pgfpathmoveto{\pgfqpoint{4.645005in}{3.266570in}}%
\pgfpathlineto{\pgfqpoint{4.655724in}{3.255993in}}%
\pgfpathlineto{\pgfqpoint{4.666468in}{3.246405in}}%
\pgfpathlineto{\pgfqpoint{4.699178in}{3.226941in}}%
\pgfpathlineto{\pgfqpoint{4.731847in}{3.206177in}}%
\pgfpathlineto{\pgfqpoint{4.721038in}{3.213056in}}%
\pgfpathlineto{\pgfqpoint{4.710256in}{3.221194in}}%
\pgfpathlineto{\pgfqpoint{4.677652in}{3.244705in}}%
\pgfpathlineto{\pgfqpoint{4.645005in}{3.266570in}}%
\pgfpathclose%
\pgfusepath{fill}%
\end{pgfscope}%
\begin{pgfscope}%
\pgfpathrectangle{\pgfqpoint{1.020000in}{0.880000in}}{\pgfqpoint{6.160000in}{6.160000in}}%
\pgfusepath{clip}%
\pgfsetbuttcap%
\pgfsetroundjoin%
\definecolor{currentfill}{rgb}{0.677823,0.786546,0.991005}%
\pgfsetfillcolor{currentfill}%
\pgfsetlinewidth{0.000000pt}%
\definecolor{currentstroke}{rgb}{0.000000,0.000000,0.000000}%
\pgfsetstrokecolor{currentstroke}%
\pgfsetdash{}{0pt}%
\pgfpathmoveto{\pgfqpoint{3.949863in}{3.780997in}}%
\pgfpathlineto{\pgfqpoint{3.959969in}{3.751025in}}%
\pgfpathlineto{\pgfqpoint{3.970090in}{3.721817in}}%
\pgfpathlineto{\pgfqpoint{4.003154in}{3.698039in}}%
\pgfpathlineto{\pgfqpoint{4.036181in}{3.675492in}}%
\pgfpathlineto{\pgfqpoint{4.026025in}{3.702219in}}%
\pgfpathlineto{\pgfqpoint{4.015883in}{3.729608in}}%
\pgfpathlineto{\pgfqpoint{3.982893in}{3.754585in}}%
\pgfpathlineto{\pgfqpoint{3.949863in}{3.780997in}}%
\pgfpathclose%
\pgfusepath{fill}%
\end{pgfscope}%
\begin{pgfscope}%
\pgfpathrectangle{\pgfqpoint{1.020000in}{0.880000in}}{\pgfqpoint{6.160000in}{6.160000in}}%
\pgfusepath{clip}%
\pgfsetbuttcap%
\pgfsetroundjoin%
\definecolor{currentfill}{rgb}{0.962701,0.628218,0.507636}%
\pgfsetfillcolor{currentfill}%
\pgfsetlinewidth{0.000000pt}%
\definecolor{currentstroke}{rgb}{0.000000,0.000000,0.000000}%
\pgfsetstrokecolor{currentstroke}%
\pgfsetdash{}{0pt}%
\pgfpathmoveto{\pgfqpoint{3.431306in}{4.738861in}}%
\pgfpathlineto{\pgfqpoint{3.441062in}{4.703845in}}%
\pgfpathlineto{\pgfqpoint{3.450854in}{4.665317in}}%
\pgfpathlineto{\pgfqpoint{3.484377in}{4.614810in}}%
\pgfpathlineto{\pgfqpoint{3.517854in}{4.563310in}}%
\pgfpathlineto{\pgfqpoint{3.508046in}{4.599519in}}%
\pgfpathlineto{\pgfqpoint{3.498272in}{4.632585in}}%
\pgfpathlineto{\pgfqpoint{3.464812in}{4.686235in}}%
\pgfpathlineto{\pgfqpoint{3.431306in}{4.738861in}}%
\pgfpathclose%
\pgfusepath{fill}%
\end{pgfscope}%
\begin{pgfscope}%
\pgfpathrectangle{\pgfqpoint{1.020000in}{0.880000in}}{\pgfqpoint{6.160000in}{6.160000in}}%
\pgfusepath{clip}%
\pgfsetbuttcap%
\pgfsetroundjoin%
\definecolor{currentfill}{rgb}{0.313946,0.420052,0.854993}%
\pgfsetfillcolor{currentfill}%
\pgfsetlinewidth{0.000000pt}%
\definecolor{currentstroke}{rgb}{0.000000,0.000000,0.000000}%
\pgfsetstrokecolor{currentstroke}%
\pgfsetdash{}{0pt}%
\pgfpathmoveto{\pgfqpoint{4.949269in}{3.085114in}}%
\pgfpathlineto{\pgfqpoint{4.960395in}{3.091031in}}%
\pgfpathlineto{\pgfqpoint{4.971540in}{3.096260in}}%
\pgfpathlineto{\pgfqpoint{5.004162in}{3.084883in}}%
\pgfpathlineto{\pgfqpoint{5.036770in}{3.075479in}}%
\pgfpathlineto{\pgfqpoint{5.025540in}{3.067404in}}%
\pgfpathlineto{\pgfqpoint{5.014329in}{3.058510in}}%
\pgfpathlineto{\pgfqpoint{4.981804in}{3.070558in}}%
\pgfpathlineto{\pgfqpoint{4.949269in}{3.085114in}}%
\pgfpathclose%
\pgfusepath{fill}%
\end{pgfscope}%
\begin{pgfscope}%
\pgfpathrectangle{\pgfqpoint{1.020000in}{0.880000in}}{\pgfqpoint{6.160000in}{6.160000in}}%
\pgfusepath{clip}%
\pgfsetbuttcap%
\pgfsetroundjoin%
\definecolor{currentfill}{rgb}{0.877149,0.394645,0.311724}%
\pgfsetfillcolor{currentfill}%
\pgfsetlinewidth{0.000000pt}%
\definecolor{currentstroke}{rgb}{0.000000,0.000000,0.000000}%
\pgfsetstrokecolor{currentstroke}%
\pgfsetdash{}{0pt}%
\pgfpathmoveto{\pgfqpoint{3.258557in}{5.037093in}}%
\pgfpathlineto{\pgfqpoint{3.268062in}{5.017538in}}%
\pgfpathlineto{\pgfqpoint{3.277623in}{4.993001in}}%
\pgfpathlineto{\pgfqpoint{3.311241in}{4.948131in}}%
\pgfpathlineto{\pgfqpoint{3.344834in}{4.900251in}}%
\pgfpathlineto{\pgfqpoint{3.335238in}{4.924132in}}%
\pgfpathlineto{\pgfqpoint{3.325694in}{4.943396in}}%
\pgfpathlineto{\pgfqpoint{3.292138in}{4.991758in}}%
\pgfpathlineto{\pgfqpoint{3.258557in}{5.037093in}}%
\pgfpathclose%
\pgfusepath{fill}%
\end{pgfscope}%
\begin{pgfscope}%
\pgfpathrectangle{\pgfqpoint{1.020000in}{0.880000in}}{\pgfqpoint{6.160000in}{6.160000in}}%
\pgfusepath{clip}%
\pgfsetbuttcap%
\pgfsetroundjoin%
\definecolor{currentfill}{rgb}{0.457046,0.594006,0.963029}%
\pgfsetfillcolor{currentfill}%
\pgfsetlinewidth{0.000000pt}%
\definecolor{currentstroke}{rgb}{0.000000,0.000000,0.000000}%
\pgfsetstrokecolor{currentstroke}%
\pgfsetdash{}{0pt}%
\pgfpathmoveto{\pgfqpoint{4.492872in}{3.366981in}}%
\pgfpathlineto{\pgfqpoint{4.503435in}{3.351785in}}%
\pgfpathlineto{\pgfqpoint{4.514020in}{3.337475in}}%
\pgfpathlineto{\pgfqpoint{4.546822in}{3.321980in}}%
\pgfpathlineto{\pgfqpoint{4.579589in}{3.305124in}}%
\pgfpathlineto{\pgfqpoint{4.568951in}{3.318432in}}%
\pgfpathlineto{\pgfqpoint{4.558336in}{3.332818in}}%
\pgfpathlineto{\pgfqpoint{4.525622in}{3.350679in}}%
\pgfpathlineto{\pgfqpoint{4.492872in}{3.366981in}}%
\pgfpathclose%
\pgfusepath{fill}%
\end{pgfscope}%
\begin{pgfscope}%
\pgfpathrectangle{\pgfqpoint{1.020000in}{0.880000in}}{\pgfqpoint{6.160000in}{6.160000in}}%
\pgfusepath{clip}%
\pgfsetbuttcap%
\pgfsetroundjoin%
\definecolor{currentfill}{rgb}{0.929357,0.512254,0.400673}%
\pgfsetfillcolor{currentfill}%
\pgfsetlinewidth{0.000000pt}%
\definecolor{currentstroke}{rgb}{0.000000,0.000000,0.000000}%
\pgfsetstrokecolor{currentstroke}%
\pgfsetdash{}{0pt}%
\pgfpathmoveto{\pgfqpoint{3.344834in}{4.900251in}}%
\pgfpathlineto{\pgfqpoint{3.354479in}{4.871858in}}%
\pgfpathlineto{\pgfqpoint{3.364172in}{4.839114in}}%
\pgfpathlineto{\pgfqpoint{3.397758in}{4.789983in}}%
\pgfpathlineto{\pgfqpoint{3.431306in}{4.738861in}}%
\pgfpathlineto{\pgfqpoint{3.421591in}{4.770115in}}%
\pgfpathlineto{\pgfqpoint{3.411917in}{4.797407in}}%
\pgfpathlineto{\pgfqpoint{3.378394in}{4.849844in}}%
\pgfpathlineto{\pgfqpoint{3.344834in}{4.900251in}}%
\pgfpathclose%
\pgfusepath{fill}%
\end{pgfscope}%
\begin{pgfscope}%
\pgfpathrectangle{\pgfqpoint{1.020000in}{0.880000in}}{\pgfqpoint{6.160000in}{6.160000in}}%
\pgfusepath{clip}%
\pgfsetbuttcap%
\pgfsetroundjoin%
\definecolor{currentfill}{rgb}{0.313946,0.420052,0.854993}%
\pgfsetfillcolor{currentfill}%
\pgfsetlinewidth{0.000000pt}%
\definecolor{currentstroke}{rgb}{0.000000,0.000000,0.000000}%
\pgfsetstrokecolor{currentstroke}%
\pgfsetdash{}{0pt}%
\pgfpathmoveto{\pgfqpoint{5.167166in}{3.063026in}}%
\pgfpathlineto{\pgfqpoint{5.178551in}{3.071040in}}%
\pgfpathlineto{\pgfqpoint{5.189941in}{3.077008in}}%
\pgfpathlineto{\pgfqpoint{5.222601in}{3.079459in}}%
\pgfpathlineto{\pgfqpoint{5.255258in}{3.083696in}}%
\pgfpathlineto{\pgfqpoint{5.243817in}{3.079184in}}%
\pgfpathlineto{\pgfqpoint{5.232381in}{3.072751in}}%
\pgfpathlineto{\pgfqpoint{5.199771in}{3.066656in}}%
\pgfpathlineto{\pgfqpoint{5.167166in}{3.063026in}}%
\pgfpathclose%
\pgfusepath{fill}%
\end{pgfscope}%
\begin{pgfscope}%
\pgfpathrectangle{\pgfqpoint{1.020000in}{0.880000in}}{\pgfqpoint{6.160000in}{6.160000in}}%
\pgfusepath{clip}%
\pgfsetbuttcap%
\pgfsetroundjoin%
\definecolor{currentfill}{rgb}{0.505423,0.643995,0.983157}%
\pgfsetfillcolor{currentfill}%
\pgfsetlinewidth{0.000000pt}%
\definecolor{currentstroke}{rgb}{0.000000,0.000000,0.000000}%
\pgfsetstrokecolor{currentstroke}%
\pgfsetdash{}{0pt}%
\pgfpathmoveto{\pgfqpoint{4.340677in}{3.457203in}}%
\pgfpathlineto{\pgfqpoint{4.351100in}{3.439423in}}%
\pgfpathlineto{\pgfqpoint{4.361542in}{3.422454in}}%
\pgfpathlineto{\pgfqpoint{4.394421in}{3.409443in}}%
\pgfpathlineto{\pgfqpoint{4.427270in}{3.396058in}}%
\pgfpathlineto{\pgfqpoint{4.416778in}{3.412621in}}%
\pgfpathlineto{\pgfqpoint{4.406306in}{3.429999in}}%
\pgfpathlineto{\pgfqpoint{4.373507in}{3.443774in}}%
\pgfpathlineto{\pgfqpoint{4.340677in}{3.457203in}}%
\pgfpathclose%
\pgfusepath{fill}%
\end{pgfscope}%
\begin{pgfscope}%
\pgfpathrectangle{\pgfqpoint{1.020000in}{0.880000in}}{\pgfqpoint{6.160000in}{6.160000in}}%
\pgfusepath{clip}%
\pgfsetbuttcap%
\pgfsetroundjoin%
\definecolor{currentfill}{rgb}{0.343278,0.459354,0.884122}%
\pgfsetfillcolor{currentfill}%
\pgfsetlinewidth{0.000000pt}%
\definecolor{currentstroke}{rgb}{0.000000,0.000000,0.000000}%
\pgfsetstrokecolor{currentstroke}%
\pgfsetdash{}{0pt}%
\pgfpathmoveto{\pgfqpoint{5.451075in}{3.131335in}}%
\pgfpathlineto{\pgfqpoint{5.462610in}{3.125371in}}%
\pgfpathlineto{\pgfqpoint{5.474154in}{3.118114in}}%
\pgfpathlineto{\pgfqpoint{5.506767in}{3.124167in}}%
\pgfpathlineto{\pgfqpoint{5.539360in}{3.129955in}}%
\pgfpathlineto{\pgfqpoint{5.527790in}{3.139664in}}%
\pgfpathlineto{\pgfqpoint{5.516231in}{3.148449in}}%
\pgfpathlineto{\pgfqpoint{5.483664in}{3.140116in}}%
\pgfpathlineto{\pgfqpoint{5.451075in}{3.131335in}}%
\pgfpathclose%
\pgfusepath{fill}%
\end{pgfscope}%
\begin{pgfscope}%
\pgfpathrectangle{\pgfqpoint{1.020000in}{0.880000in}}{\pgfqpoint{6.160000in}{6.160000in}}%
\pgfusepath{clip}%
\pgfsetbuttcap%
\pgfsetroundjoin%
\definecolor{currentfill}{rgb}{0.559747,0.694768,0.996075}%
\pgfsetfillcolor{currentfill}%
\pgfsetlinewidth{0.000000pt}%
\definecolor{currentstroke}{rgb}{0.000000,0.000000,0.000000}%
\pgfsetstrokecolor{currentstroke}%
\pgfsetdash{}{0pt}%
\pgfpathmoveto{\pgfqpoint{4.188464in}{3.553452in}}%
\pgfpathlineto{\pgfqpoint{4.198754in}{3.532555in}}%
\pgfpathlineto{\pgfqpoint{4.209062in}{3.512545in}}%
\pgfpathlineto{\pgfqpoint{4.242010in}{3.498050in}}%
\pgfpathlineto{\pgfqpoint{4.274929in}{3.484113in}}%
\pgfpathlineto{\pgfqpoint{4.264573in}{3.503266in}}%
\pgfpathlineto{\pgfqpoint{4.254237in}{3.523189in}}%
\pgfpathlineto{\pgfqpoint{4.221366in}{3.537969in}}%
\pgfpathlineto{\pgfqpoint{4.188464in}{3.553452in}}%
\pgfpathclose%
\pgfusepath{fill}%
\end{pgfscope}%
\begin{pgfscope}%
\pgfpathrectangle{\pgfqpoint{1.020000in}{0.880000in}}{\pgfqpoint{6.160000in}{6.160000in}}%
\pgfusepath{clip}%
\pgfsetbuttcap%
\pgfsetroundjoin%
\definecolor{currentfill}{rgb}{0.740957,0.122240,0.175744}%
\pgfsetfillcolor{currentfill}%
\pgfsetlinewidth{0.000000pt}%
\definecolor{currentstroke}{rgb}{0.000000,0.000000,0.000000}%
\pgfsetstrokecolor{currentstroke}%
\pgfsetdash{}{0pt}%
\pgfpathmoveto{\pgfqpoint{3.020196in}{5.246587in}}%
\pgfpathlineto{\pgfqpoint{3.029269in}{5.248696in}}%
\pgfpathlineto{\pgfqpoint{3.038418in}{5.245373in}}%
\pgfpathlineto{\pgfqpoint{3.071947in}{5.226681in}}%
\pgfpathlineto{\pgfqpoint{3.105495in}{5.202805in}}%
\pgfpathlineto{\pgfqpoint{3.096287in}{5.206333in}}%
\pgfpathlineto{\pgfqpoint{3.087151in}{5.204571in}}%
\pgfpathlineto{\pgfqpoint{3.053664in}{5.228100in}}%
\pgfpathlineto{\pgfqpoint{3.020196in}{5.246587in}}%
\pgfpathclose%
\pgfusepath{fill}%
\end{pgfscope}%
\begin{pgfscope}%
\pgfpathrectangle{\pgfqpoint{1.020000in}{0.880000in}}{\pgfqpoint{6.160000in}{6.160000in}}%
\pgfusepath{clip}%
\pgfsetbuttcap%
\pgfsetroundjoin%
\definecolor{currentfill}{rgb}{0.966922,0.651969,0.531997}%
\pgfsetfillcolor{currentfill}%
\pgfsetlinewidth{0.000000pt}%
\definecolor{currentstroke}{rgb}{0.000000,0.000000,0.000000}%
\pgfsetstrokecolor{currentstroke}%
\pgfsetdash{}{0pt}%
\pgfpathmoveto{\pgfqpoint{2.210481in}{4.588190in}}%
\pgfpathlineto{\pgfqpoint{2.218887in}{4.579636in}}%
\pgfpathlineto{\pgfqpoint{2.227344in}{4.569050in}}%
\pgfpathlineto{\pgfqpoint{2.260253in}{4.607097in}}%
\pgfpathlineto{\pgfqpoint{2.293131in}{4.647022in}}%
\pgfpathlineto{\pgfqpoint{2.284644in}{4.655663in}}%
\pgfpathlineto{\pgfqpoint{2.276214in}{4.661949in}}%
\pgfpathlineto{\pgfqpoint{2.243363in}{4.624168in}}%
\pgfpathlineto{\pgfqpoint{2.210481in}{4.588190in}}%
\pgfpathclose%
\pgfusepath{fill}%
\end{pgfscope}%
\begin{pgfscope}%
\pgfpathrectangle{\pgfqpoint{1.020000in}{0.880000in}}{\pgfqpoint{6.160000in}{6.160000in}}%
\pgfusepath{clip}%
\pgfsetbuttcap%
\pgfsetroundjoin%
\definecolor{currentfill}{rgb}{0.729196,0.086679,0.167240}%
\pgfsetfillcolor{currentfill}%
\pgfsetlinewidth{0.000000pt}%
\definecolor{currentstroke}{rgb}{0.000000,0.000000,0.000000}%
\pgfsetstrokecolor{currentstroke}%
\pgfsetdash{}{0pt}%
\pgfpathmoveto{\pgfqpoint{2.802547in}{5.229286in}}%
\pgfpathlineto{\pgfqpoint{2.811278in}{5.240341in}}%
\pgfpathlineto{\pgfqpoint{2.820088in}{5.246575in}}%
\pgfpathlineto{\pgfqpoint{2.853334in}{5.259643in}}%
\pgfpathlineto{\pgfqpoint{2.886626in}{5.267634in}}%
\pgfpathlineto{\pgfqpoint{2.877757in}{5.260622in}}%
\pgfpathlineto{\pgfqpoint{2.868966in}{5.248682in}}%
\pgfpathlineto{\pgfqpoint{2.835734in}{5.241411in}}%
\pgfpathlineto{\pgfqpoint{2.802547in}{5.229286in}}%
\pgfpathclose%
\pgfusepath{fill}%
\end{pgfscope}%
\begin{pgfscope}%
\pgfpathrectangle{\pgfqpoint{1.020000in}{0.880000in}}{\pgfqpoint{6.160000in}{6.160000in}}%
\pgfusepath{clip}%
\pgfsetbuttcap%
\pgfsetroundjoin%
\definecolor{currentfill}{rgb}{0.348323,0.465711,0.888346}%
\pgfsetfillcolor{currentfill}%
\pgfsetlinewidth{0.000000pt}%
\definecolor{currentstroke}{rgb}{0.000000,0.000000,0.000000}%
\pgfsetstrokecolor{currentstroke}%
\pgfsetdash{}{0pt}%
\pgfpathmoveto{\pgfqpoint{5.669495in}{3.148904in}}%
\pgfpathlineto{\pgfqpoint{5.681143in}{3.135276in}}%
\pgfpathlineto{\pgfqpoint{5.692807in}{3.121288in}}%
\pgfpathlineto{\pgfqpoint{5.725319in}{3.123998in}}%
\pgfpathlineto{\pgfqpoint{5.757807in}{3.126467in}}%
\pgfpathlineto{\pgfqpoint{5.746101in}{3.141157in}}%
\pgfpathlineto{\pgfqpoint{5.734413in}{3.155643in}}%
\pgfpathlineto{\pgfqpoint{5.701967in}{3.152484in}}%
\pgfpathlineto{\pgfqpoint{5.669495in}{3.148904in}}%
\pgfpathclose%
\pgfusepath{fill}%
\end{pgfscope}%
\begin{pgfscope}%
\pgfpathrectangle{\pgfqpoint{1.020000in}{0.880000in}}{\pgfqpoint{6.160000in}{6.160000in}}%
\pgfusepath{clip}%
\pgfsetbuttcap%
\pgfsetroundjoin%
\definecolor{currentfill}{rgb}{0.956653,0.598034,0.477302}%
\pgfsetfillcolor{currentfill}%
\pgfsetlinewidth{0.000000pt}%
\definecolor{currentstroke}{rgb}{0.000000,0.000000,0.000000}%
\pgfsetstrokecolor{currentstroke}%
\pgfsetdash{}{0pt}%
\pgfpathmoveto{\pgfqpoint{2.276214in}{4.661949in}}%
\pgfpathlineto{\pgfqpoint{2.284644in}{4.655663in}}%
\pgfpathlineto{\pgfqpoint{2.293131in}{4.647022in}}%
\pgfpathlineto{\pgfqpoint{2.325984in}{4.688583in}}%
\pgfpathlineto{\pgfqpoint{2.358819in}{4.731498in}}%
\pgfpathlineto{\pgfqpoint{2.350302in}{4.738075in}}%
\pgfpathlineto{\pgfqpoint{2.341847in}{4.741950in}}%
\pgfpathlineto{\pgfqpoint{2.309040in}{4.701300in}}%
\pgfpathlineto{\pgfqpoint{2.276214in}{4.661949in}}%
\pgfpathclose%
\pgfusepath{fill}%
\end{pgfscope}%
\begin{pgfscope}%
\pgfpathrectangle{\pgfqpoint{1.020000in}{0.880000in}}{\pgfqpoint{6.160000in}{6.160000in}}%
\pgfusepath{clip}%
\pgfsetbuttcap%
\pgfsetroundjoin%
\definecolor{currentfill}{rgb}{0.338377,0.452819,0.879317}%
\pgfsetfillcolor{currentfill}%
\pgfsetlinewidth{0.000000pt}%
\definecolor{currentstroke}{rgb}{0.000000,0.000000,0.000000}%
\pgfsetstrokecolor{currentstroke}%
\pgfsetdash{}{0pt}%
\pgfpathmoveto{\pgfqpoint{5.887535in}{3.134495in}}%
\pgfpathlineto{\pgfqpoint{5.899356in}{3.119150in}}%
\pgfpathlineto{\pgfqpoint{5.911197in}{3.103776in}}%
\pgfpathlineto{\pgfqpoint{5.943625in}{3.105429in}}%
\pgfpathlineto{\pgfqpoint{5.931758in}{3.120817in}}%
\pgfpathlineto{\pgfqpoint{5.919913in}{3.136192in}}%
\pgfpathlineto{\pgfqpoint{5.887535in}{3.134495in}}%
\pgfpathclose%
\pgfusepath{fill}%
\end{pgfscope}%
\begin{pgfscope}%
\pgfpathrectangle{\pgfqpoint{1.020000in}{0.880000in}}{\pgfqpoint{6.160000in}{6.160000in}}%
\pgfusepath{clip}%
\pgfsetbuttcap%
\pgfsetroundjoin%
\definecolor{currentfill}{rgb}{0.630089,0.752516,0.998508}%
\pgfsetfillcolor{currentfill}%
\pgfsetlinewidth{0.000000pt}%
\definecolor{currentstroke}{rgb}{0.000000,0.000000,0.000000}%
\pgfsetstrokecolor{currentstroke}%
\pgfsetdash{}{0pt}%
\pgfpathmoveto{\pgfqpoint{4.036181in}{3.675492in}}%
\pgfpathlineto{\pgfqpoint{4.046353in}{3.649582in}}%
\pgfpathlineto{\pgfqpoint{4.056539in}{3.624636in}}%
\pgfpathlineto{\pgfqpoint{4.089570in}{3.605317in}}%
\pgfpathlineto{\pgfqpoint{4.122567in}{3.587043in}}%
\pgfpathlineto{\pgfqpoint{4.112339in}{3.610177in}}%
\pgfpathlineto{\pgfqpoint{4.102127in}{3.634133in}}%
\pgfpathlineto{\pgfqpoint{4.069172in}{3.654193in}}%
\pgfpathlineto{\pgfqpoint{4.036181in}{3.675492in}}%
\pgfpathclose%
\pgfusepath{fill}%
\end{pgfscope}%
\begin{pgfscope}%
\pgfpathrectangle{\pgfqpoint{1.020000in}{0.880000in}}{\pgfqpoint{6.160000in}{6.160000in}}%
\pgfusepath{clip}%
\pgfsetbuttcap%
\pgfsetroundjoin%
\definecolor{currentfill}{rgb}{0.939254,0.539581,0.423900}%
\pgfsetfillcolor{currentfill}%
\pgfsetlinewidth{0.000000pt}%
\definecolor{currentstroke}{rgb}{0.000000,0.000000,0.000000}%
\pgfsetstrokecolor{currentstroke}%
\pgfsetdash{}{0pt}%
\pgfpathmoveto{\pgfqpoint{2.341847in}{4.741950in}}%
\pgfpathlineto{\pgfqpoint{2.350302in}{4.738075in}}%
\pgfpathlineto{\pgfqpoint{2.358819in}{4.731498in}}%
\pgfpathlineto{\pgfqpoint{2.391641in}{4.775444in}}%
\pgfpathlineto{\pgfqpoint{2.424458in}{4.820059in}}%
\pgfpathlineto{\pgfqpoint{2.415910in}{4.824498in}}%
\pgfpathlineto{\pgfqpoint{2.407428in}{4.825876in}}%
\pgfpathlineto{\pgfqpoint{2.374640in}{4.783591in}}%
\pgfpathlineto{\pgfqpoint{2.341847in}{4.741950in}}%
\pgfpathclose%
\pgfusepath{fill}%
\end{pgfscope}%
\begin{pgfscope}%
\pgfpathrectangle{\pgfqpoint{1.020000in}{0.880000in}}{\pgfqpoint{6.160000in}{6.160000in}}%
\pgfusepath{clip}%
\pgfsetbuttcap%
\pgfsetroundjoin%
\definecolor{currentfill}{rgb}{0.338377,0.452819,0.879317}%
\pgfsetfillcolor{currentfill}%
\pgfsetlinewidth{0.000000pt}%
\definecolor{currentstroke}{rgb}{0.000000,0.000000,0.000000}%
\pgfsetstrokecolor{currentstroke}%
\pgfsetdash{}{0pt}%
\pgfpathmoveto{\pgfqpoint{5.385844in}{3.113311in}}%
\pgfpathlineto{\pgfqpoint{5.397356in}{3.110402in}}%
\pgfpathlineto{\pgfqpoint{5.408872in}{3.105806in}}%
\pgfpathlineto{\pgfqpoint{5.441521in}{3.111936in}}%
\pgfpathlineto{\pgfqpoint{5.474154in}{3.118114in}}%
\pgfpathlineto{\pgfqpoint{5.462610in}{3.125371in}}%
\pgfpathlineto{\pgfqpoint{5.451075in}{3.131335in}}%
\pgfpathlineto{\pgfqpoint{5.418467in}{3.122314in}}%
\pgfpathlineto{\pgfqpoint{5.385844in}{3.113311in}}%
\pgfpathclose%
\pgfusepath{fill}%
\end{pgfscope}%
\begin{pgfscope}%
\pgfpathrectangle{\pgfqpoint{1.020000in}{0.880000in}}{\pgfqpoint{6.160000in}{6.160000in}}%
\pgfusepath{clip}%
\pgfsetbuttcap%
\pgfsetroundjoin%
\definecolor{currentfill}{rgb}{0.912033,0.469680,0.366565}%
\pgfsetfillcolor{currentfill}%
\pgfsetlinewidth{0.000000pt}%
\definecolor{currentstroke}{rgb}{0.000000,0.000000,0.000000}%
\pgfsetstrokecolor{currentstroke}%
\pgfsetdash{}{0pt}%
\pgfpathmoveto{\pgfqpoint{2.407428in}{4.825876in}}%
\pgfpathlineto{\pgfqpoint{2.415910in}{4.824498in}}%
\pgfpathlineto{\pgfqpoint{2.424458in}{4.820059in}}%
\pgfpathlineto{\pgfqpoint{2.457277in}{4.864950in}}%
\pgfpathlineto{\pgfqpoint{2.490104in}{4.909692in}}%
\pgfpathlineto{\pgfqpoint{2.481522in}{4.911977in}}%
\pgfpathlineto{\pgfqpoint{2.473010in}{4.910838in}}%
\pgfpathlineto{\pgfqpoint{2.440215in}{4.868427in}}%
\pgfpathlineto{\pgfqpoint{2.407428in}{4.825876in}}%
\pgfpathclose%
\pgfusepath{fill}%
\end{pgfscope}%
\begin{pgfscope}%
\pgfpathrectangle{\pgfqpoint{1.020000in}{0.880000in}}{\pgfqpoint{6.160000in}{6.160000in}}%
\pgfusepath{clip}%
\pgfsetbuttcap%
\pgfsetroundjoin%
\definecolor{currentfill}{rgb}{0.740957,0.122240,0.175744}%
\pgfsetfillcolor{currentfill}%
\pgfsetlinewidth{0.000000pt}%
\definecolor{currentstroke}{rgb}{0.000000,0.000000,0.000000}%
\pgfsetstrokecolor{currentstroke}%
\pgfsetdash{}{0pt}%
\pgfpathmoveto{\pgfqpoint{2.736312in}{5.191394in}}%
\pgfpathlineto{\pgfqpoint{2.744988in}{5.201105in}}%
\pgfpathlineto{\pgfqpoint{2.753744in}{5.206174in}}%
\pgfpathlineto{\pgfqpoint{2.786892in}{5.228658in}}%
\pgfpathlineto{\pgfqpoint{2.820088in}{5.246575in}}%
\pgfpathlineto{\pgfqpoint{2.811278in}{5.240341in}}%
\pgfpathlineto{\pgfqpoint{2.802547in}{5.229286in}}%
\pgfpathlineto{\pgfqpoint{2.769406in}{5.212523in}}%
\pgfpathlineto{\pgfqpoint{2.736312in}{5.191394in}}%
\pgfpathclose%
\pgfusepath{fill}%
\end{pgfscope}%
\begin{pgfscope}%
\pgfpathrectangle{\pgfqpoint{1.020000in}{0.880000in}}{\pgfqpoint{6.160000in}{6.160000in}}%
\pgfusepath{clip}%
\pgfsetbuttcap%
\pgfsetroundjoin%
\definecolor{currentfill}{rgb}{0.313946,0.420052,0.854993}%
\pgfsetfillcolor{currentfill}%
\pgfsetlinewidth{0.000000pt}%
\definecolor{currentstroke}{rgb}{0.000000,0.000000,0.000000}%
\pgfsetstrokecolor{currentstroke}%
\pgfsetdash{}{0pt}%
\pgfpathmoveto{\pgfqpoint{5.101968in}{3.063866in}}%
\pgfpathlineto{\pgfqpoint{5.113288in}{3.071946in}}%
\pgfpathlineto{\pgfqpoint{5.124615in}{3.077989in}}%
\pgfpathlineto{\pgfqpoint{5.157280in}{3.076494in}}%
\pgfpathlineto{\pgfqpoint{5.189941in}{3.077008in}}%
\pgfpathlineto{\pgfqpoint{5.178551in}{3.071040in}}%
\pgfpathlineto{\pgfqpoint{5.167166in}{3.063026in}}%
\pgfpathlineto{\pgfqpoint{5.134566in}{3.062067in}}%
\pgfpathlineto{\pgfqpoint{5.101968in}{3.063866in}}%
\pgfpathclose%
\pgfusepath{fill}%
\end{pgfscope}%
\begin{pgfscope}%
\pgfpathrectangle{\pgfqpoint{1.020000in}{0.880000in}}{\pgfqpoint{6.160000in}{6.160000in}}%
\pgfusepath{clip}%
\pgfsetbuttcap%
\pgfsetroundjoin%
\definecolor{currentfill}{rgb}{0.847365,0.862472,0.885540}%
\pgfsetfillcolor{currentfill}%
\pgfsetlinewidth{0.000000pt}%
\definecolor{currentstroke}{rgb}{0.000000,0.000000,0.000000}%
\pgfsetstrokecolor{currentstroke}%
\pgfsetdash{}{0pt}%
\pgfpathmoveto{\pgfqpoint{3.710775in}{4.129676in}}%
\pgfpathlineto{\pgfqpoint{3.720760in}{4.088452in}}%
\pgfpathlineto{\pgfqpoint{3.730758in}{4.047063in}}%
\pgfpathlineto{\pgfqpoint{3.764030in}{4.010159in}}%
\pgfpathlineto{\pgfqpoint{3.797255in}{3.974140in}}%
\pgfpathlineto{\pgfqpoint{3.787241in}{4.011780in}}%
\pgfpathlineto{\pgfqpoint{3.777240in}{4.049309in}}%
\pgfpathlineto{\pgfqpoint{3.744033in}{4.088986in}}%
\pgfpathlineto{\pgfqpoint{3.710775in}{4.129676in}}%
\pgfpathclose%
\pgfusepath{fill}%
\end{pgfscope}%
\begin{pgfscope}%
\pgfpathrectangle{\pgfqpoint{1.020000in}{0.880000in}}{\pgfqpoint{6.160000in}{6.160000in}}%
\pgfusepath{clip}%
\pgfsetbuttcap%
\pgfsetroundjoin%
\definecolor{currentfill}{rgb}{0.880896,0.402331,0.317115}%
\pgfsetfillcolor{currentfill}%
\pgfsetlinewidth{0.000000pt}%
\definecolor{currentstroke}{rgb}{0.000000,0.000000,0.000000}%
\pgfsetstrokecolor{currentstroke}%
\pgfsetdash{}{0pt}%
\pgfpathmoveto{\pgfqpoint{2.473010in}{4.910838in}}%
\pgfpathlineto{\pgfqpoint{2.481522in}{4.911977in}}%
\pgfpathlineto{\pgfqpoint{2.490104in}{4.909692in}}%
\pgfpathlineto{\pgfqpoint{2.522947in}{4.953834in}}%
\pgfpathlineto{\pgfqpoint{2.555811in}{4.996908in}}%
\pgfpathlineto{\pgfqpoint{2.547191in}{4.997091in}}%
\pgfpathlineto{\pgfqpoint{2.538646in}{4.993497in}}%
\pgfpathlineto{\pgfqpoint{2.505818in}{4.952677in}}%
\pgfpathlineto{\pgfqpoint{2.473010in}{4.910838in}}%
\pgfpathclose%
\pgfusepath{fill}%
\end{pgfscope}%
\begin{pgfscope}%
\pgfpathrectangle{\pgfqpoint{1.020000in}{0.880000in}}{\pgfqpoint{6.160000in}{6.160000in}}%
\pgfusepath{clip}%
\pgfsetbuttcap%
\pgfsetroundjoin%
\definecolor{currentfill}{rgb}{0.782049,0.842864,0.942980}%
\pgfsetfillcolor{currentfill}%
\pgfsetlinewidth{0.000000pt}%
\definecolor{currentstroke}{rgb}{0.000000,0.000000,0.000000}%
\pgfsetstrokecolor{currentstroke}%
\pgfsetdash{}{0pt}%
\pgfpathmoveto{\pgfqpoint{3.797255in}{3.974140in}}%
\pgfpathlineto{\pgfqpoint{3.807282in}{3.936669in}}%
\pgfpathlineto{\pgfqpoint{3.817322in}{3.899642in}}%
\pgfpathlineto{\pgfqpoint{3.850522in}{3.868235in}}%
\pgfpathlineto{\pgfqpoint{3.883678in}{3.837911in}}%
\pgfpathlineto{\pgfqpoint{3.873615in}{3.871511in}}%
\pgfpathlineto{\pgfqpoint{3.863565in}{3.905524in}}%
\pgfpathlineto{\pgfqpoint{3.830433in}{3.939205in}}%
\pgfpathlineto{\pgfqpoint{3.797255in}{3.974140in}}%
\pgfpathclose%
\pgfusepath{fill}%
\end{pgfscope}%
\begin{pgfscope}%
\pgfpathrectangle{\pgfqpoint{1.020000in}{0.880000in}}{\pgfqpoint{6.160000in}{6.160000in}}%
\pgfusepath{clip}%
\pgfsetbuttcap%
\pgfsetroundjoin%
\definecolor{currentfill}{rgb}{0.916071,0.833977,0.788693}%
\pgfsetfillcolor{currentfill}%
\pgfsetlinewidth{0.000000pt}%
\definecolor{currentstroke}{rgb}{0.000000,0.000000,0.000000}%
\pgfsetstrokecolor{currentstroke}%
\pgfsetdash{}{0pt}%
\pgfpathmoveto{\pgfqpoint{3.624211in}{4.301168in}}%
\pgfpathlineto{\pgfqpoint{3.634152in}{4.257595in}}%
\pgfpathlineto{\pgfqpoint{3.644111in}{4.213068in}}%
\pgfpathlineto{\pgfqpoint{3.677468in}{4.171129in}}%
\pgfpathlineto{\pgfqpoint{3.710775in}{4.129676in}}%
\pgfpathlineto{\pgfqpoint{3.700806in}{4.170412in}}%
\pgfpathlineto{\pgfqpoint{3.690853in}{4.210343in}}%
\pgfpathlineto{\pgfqpoint{3.657558in}{4.255477in}}%
\pgfpathlineto{\pgfqpoint{3.624211in}{4.301168in}}%
\pgfpathclose%
\pgfusepath{fill}%
\end{pgfscope}%
\begin{pgfscope}%
\pgfpathrectangle{\pgfqpoint{1.020000in}{0.880000in}}{\pgfqpoint{6.160000in}{6.160000in}}%
\pgfusepath{clip}%
\pgfsetbuttcap%
\pgfsetroundjoin%
\definecolor{currentfill}{rgb}{0.373552,0.497499,0.909467}%
\pgfsetfillcolor{currentfill}%
\pgfsetlinewidth{0.000000pt}%
\definecolor{currentstroke}{rgb}{0.000000,0.000000,0.000000}%
\pgfsetstrokecolor{currentstroke}%
\pgfsetdash{}{0pt}%
\pgfpathmoveto{\pgfqpoint{4.731847in}{3.206177in}}%
\pgfpathlineto{\pgfqpoint{4.742682in}{3.200186in}}%
\pgfpathlineto{\pgfqpoint{4.753541in}{3.194711in}}%
\pgfpathlineto{\pgfqpoint{4.786242in}{3.176280in}}%
\pgfpathlineto{\pgfqpoint{4.818909in}{3.157359in}}%
\pgfpathlineto{\pgfqpoint{4.807974in}{3.159222in}}%
\pgfpathlineto{\pgfqpoint{4.797066in}{3.161660in}}%
\pgfpathlineto{\pgfqpoint{4.764476in}{3.184308in}}%
\pgfpathlineto{\pgfqpoint{4.731847in}{3.206177in}}%
\pgfpathclose%
\pgfusepath{fill}%
\end{pgfscope}%
\begin{pgfscope}%
\pgfpathrectangle{\pgfqpoint{1.020000in}{0.880000in}}{\pgfqpoint{6.160000in}{6.160000in}}%
\pgfusepath{clip}%
\pgfsetbuttcap%
\pgfsetroundjoin%
\definecolor{currentfill}{rgb}{0.768929,0.189213,0.197965}%
\pgfsetfillcolor{currentfill}%
\pgfsetlinewidth{0.000000pt}%
\definecolor{currentstroke}{rgb}{0.000000,0.000000,0.000000}%
\pgfsetstrokecolor{currentstroke}%
\pgfsetdash{}{0pt}%
\pgfpathmoveto{\pgfqpoint{2.670262in}{5.137361in}}%
\pgfpathlineto{\pgfqpoint{2.678889in}{5.145329in}}%
\pgfpathlineto{\pgfqpoint{2.687595in}{5.148897in}}%
\pgfpathlineto{\pgfqpoint{2.720646in}{5.179460in}}%
\pgfpathlineto{\pgfqpoint{2.753744in}{5.206174in}}%
\pgfpathlineto{\pgfqpoint{2.744988in}{5.201105in}}%
\pgfpathlineto{\pgfqpoint{2.736312in}{5.191394in}}%
\pgfpathlineto{\pgfqpoint{2.703265in}{5.166218in}}%
\pgfpathlineto{\pgfqpoint{2.670262in}{5.137361in}}%
\pgfpathclose%
\pgfusepath{fill}%
\end{pgfscope}%
\begin{pgfscope}%
\pgfpathrectangle{\pgfqpoint{1.020000in}{0.880000in}}{\pgfqpoint{6.160000in}{6.160000in}}%
\pgfusepath{clip}%
\pgfsetbuttcap%
\pgfsetroundjoin%
\definecolor{currentfill}{rgb}{0.348323,0.465711,0.888346}%
\pgfsetfillcolor{currentfill}%
\pgfsetlinewidth{0.000000pt}%
\definecolor{currentstroke}{rgb}{0.000000,0.000000,0.000000}%
\pgfsetstrokecolor{currentstroke}%
\pgfsetdash{}{0pt}%
\pgfpathmoveto{\pgfqpoint{5.604476in}{3.140356in}}%
\pgfpathlineto{\pgfqpoint{5.616088in}{3.128006in}}%
\pgfpathlineto{\pgfqpoint{5.627714in}{3.115069in}}%
\pgfpathlineto{\pgfqpoint{5.660272in}{3.118314in}}%
\pgfpathlineto{\pgfqpoint{5.692807in}{3.121288in}}%
\pgfpathlineto{\pgfqpoint{5.681143in}{3.135276in}}%
\pgfpathlineto{\pgfqpoint{5.669495in}{3.148904in}}%
\pgfpathlineto{\pgfqpoint{5.636998in}{3.144868in}}%
\pgfpathlineto{\pgfqpoint{5.604476in}{3.140356in}}%
\pgfpathclose%
\pgfusepath{fill}%
\end{pgfscope}%
\begin{pgfscope}%
\pgfpathrectangle{\pgfqpoint{1.020000in}{0.880000in}}{\pgfqpoint{6.160000in}{6.160000in}}%
\pgfusepath{clip}%
\pgfsetbuttcap%
\pgfsetroundjoin%
\definecolor{currentfill}{rgb}{0.338377,0.452819,0.879317}%
\pgfsetfillcolor{currentfill}%
\pgfsetlinewidth{0.000000pt}%
\definecolor{currentstroke}{rgb}{0.000000,0.000000,0.000000}%
\pgfsetstrokecolor{currentstroke}%
\pgfsetdash{}{0pt}%
\pgfpathmoveto{\pgfqpoint{5.822716in}{3.130788in}}%
\pgfpathlineto{\pgfqpoint{5.834488in}{3.115592in}}%
\pgfpathlineto{\pgfqpoint{5.846280in}{3.100319in}}%
\pgfpathlineto{\pgfqpoint{5.878749in}{3.102077in}}%
\pgfpathlineto{\pgfqpoint{5.911197in}{3.103776in}}%
\pgfpathlineto{\pgfqpoint{5.899356in}{3.119150in}}%
\pgfpathlineto{\pgfqpoint{5.887535in}{3.134495in}}%
\pgfpathlineto{\pgfqpoint{5.855137in}{3.132703in}}%
\pgfpathlineto{\pgfqpoint{5.822716in}{3.130788in}}%
\pgfpathclose%
\pgfusepath{fill}%
\end{pgfscope}%
\begin{pgfscope}%
\pgfpathrectangle{\pgfqpoint{1.020000in}{0.880000in}}{\pgfqpoint{6.160000in}{6.160000in}}%
\pgfusepath{clip}%
\pgfsetbuttcap%
\pgfsetroundjoin%
\definecolor{currentfill}{rgb}{0.843703,0.330068,0.270065}%
\pgfsetfillcolor{currentfill}%
\pgfsetlinewidth{0.000000pt}%
\definecolor{currentstroke}{rgb}{0.000000,0.000000,0.000000}%
\pgfsetstrokecolor{currentstroke}%
\pgfsetdash{}{0pt}%
\pgfpathmoveto{\pgfqpoint{2.538646in}{4.993497in}}%
\pgfpathlineto{\pgfqpoint{2.547191in}{4.997091in}}%
\pgfpathlineto{\pgfqpoint{2.555811in}{4.996908in}}%
\pgfpathlineto{\pgfqpoint{2.588702in}{5.038433in}}%
\pgfpathlineto{\pgfqpoint{2.621627in}{5.077925in}}%
\pgfpathlineto{\pgfqpoint{2.612966in}{5.076132in}}%
\pgfpathlineto{\pgfqpoint{2.604383in}{5.070231in}}%
\pgfpathlineto{\pgfqpoint{2.571499in}{5.032836in}}%
\pgfpathlineto{\pgfqpoint{2.538646in}{4.993497in}}%
\pgfpathclose%
\pgfusepath{fill}%
\end{pgfscope}%
\begin{pgfscope}%
\pgfpathrectangle{\pgfqpoint{1.020000in}{0.880000in}}{\pgfqpoint{6.160000in}{6.160000in}}%
\pgfusepath{clip}%
\pgfsetbuttcap%
\pgfsetroundjoin%
\definecolor{currentfill}{rgb}{0.333490,0.446265,0.874452}%
\pgfsetfillcolor{currentfill}%
\pgfsetlinewidth{0.000000pt}%
\definecolor{currentstroke}{rgb}{0.000000,0.000000,0.000000}%
\pgfsetstrokecolor{currentstroke}%
\pgfsetdash{}{0pt}%
\pgfpathmoveto{\pgfqpoint{4.884143in}{3.119555in}}%
\pgfpathlineto{\pgfqpoint{4.895182in}{3.121708in}}%
\pgfpathlineto{\pgfqpoint{4.906242in}{3.123344in}}%
\pgfpathlineto{\pgfqpoint{4.938902in}{3.109219in}}%
\pgfpathlineto{\pgfqpoint{4.971540in}{3.096260in}}%
\pgfpathlineto{\pgfqpoint{4.960395in}{3.091031in}}%
\pgfpathlineto{\pgfqpoint{4.949269in}{3.085114in}}%
\pgfpathlineto{\pgfqpoint{4.916717in}{3.101637in}}%
\pgfpathlineto{\pgfqpoint{4.884143in}{3.119555in}}%
\pgfpathclose%
\pgfusepath{fill}%
\end{pgfscope}%
\begin{pgfscope}%
\pgfpathrectangle{\pgfqpoint{1.020000in}{0.880000in}}{\pgfqpoint{6.160000in}{6.160000in}}%
\pgfusepath{clip}%
\pgfsetbuttcap%
\pgfsetroundjoin%
\definecolor{currentfill}{rgb}{0.425199,0.559058,0.946061}%
\pgfsetfillcolor{currentfill}%
\pgfsetlinewidth{0.000000pt}%
\definecolor{currentstroke}{rgb}{0.000000,0.000000,0.000000}%
\pgfsetstrokecolor{currentstroke}%
\pgfsetdash{}{0pt}%
\pgfpathmoveto{\pgfqpoint{4.579589in}{3.305124in}}%
\pgfpathlineto{\pgfqpoint{4.590250in}{3.292776in}}%
\pgfpathlineto{\pgfqpoint{4.600935in}{3.281248in}}%
\pgfpathlineto{\pgfqpoint{4.633720in}{3.264497in}}%
\pgfpathlineto{\pgfqpoint{4.666468in}{3.246405in}}%
\pgfpathlineto{\pgfqpoint{4.655724in}{3.255993in}}%
\pgfpathlineto{\pgfqpoint{4.645005in}{3.266570in}}%
\pgfpathlineto{\pgfqpoint{4.612317in}{3.286698in}}%
\pgfpathlineto{\pgfqpoint{4.579589in}{3.305124in}}%
\pgfpathclose%
\pgfusepath{fill}%
\end{pgfscope}%
\begin{pgfscope}%
\pgfpathrectangle{\pgfqpoint{1.020000in}{0.880000in}}{\pgfqpoint{6.160000in}{6.160000in}}%
\pgfusepath{clip}%
\pgfsetbuttcap%
\pgfsetroundjoin%
\definecolor{currentfill}{rgb}{0.805723,0.259813,0.230562}%
\pgfsetfillcolor{currentfill}%
\pgfsetlinewidth{0.000000pt}%
\definecolor{currentstroke}{rgb}{0.000000,0.000000,0.000000}%
\pgfsetstrokecolor{currentstroke}%
\pgfsetdash{}{0pt}%
\pgfpathmoveto{\pgfqpoint{2.604383in}{5.070231in}}%
\pgfpathlineto{\pgfqpoint{2.612966in}{5.076132in}}%
\pgfpathlineto{\pgfqpoint{2.621627in}{5.077925in}}%
\pgfpathlineto{\pgfqpoint{2.654590in}{5.114902in}}%
\pgfpathlineto{\pgfqpoint{2.687595in}{5.148897in}}%
\pgfpathlineto{\pgfqpoint{2.678889in}{5.145329in}}%
\pgfpathlineto{\pgfqpoint{2.670262in}{5.137361in}}%
\pgfpathlineto{\pgfqpoint{2.637303in}{5.105222in}}%
\pgfpathlineto{\pgfqpoint{2.604383in}{5.070231in}}%
\pgfpathclose%
\pgfusepath{fill}%
\end{pgfscope}%
\begin{pgfscope}%
\pgfpathrectangle{\pgfqpoint{1.020000in}{0.880000in}}{\pgfqpoint{6.160000in}{6.160000in}}%
\pgfusepath{clip}%
\pgfsetbuttcap%
\pgfsetroundjoin%
\definecolor{currentfill}{rgb}{0.478462,0.616564,0.972721}%
\pgfsetfillcolor{currentfill}%
\pgfsetlinewidth{0.000000pt}%
\definecolor{currentstroke}{rgb}{0.000000,0.000000,0.000000}%
\pgfsetstrokecolor{currentstroke}%
\pgfsetdash{}{0pt}%
\pgfpathmoveto{\pgfqpoint{4.427270in}{3.396058in}}%
\pgfpathlineto{\pgfqpoint{4.437782in}{3.380320in}}%
\pgfpathlineto{\pgfqpoint{4.448316in}{3.365403in}}%
\pgfpathlineto{\pgfqpoint{4.481184in}{3.351858in}}%
\pgfpathlineto{\pgfqpoint{4.514020in}{3.337475in}}%
\pgfpathlineto{\pgfqpoint{4.503435in}{3.351785in}}%
\pgfpathlineto{\pgfqpoint{4.492872in}{3.366981in}}%
\pgfpathlineto{\pgfqpoint{4.460087in}{3.382006in}}%
\pgfpathlineto{\pgfqpoint{4.427270in}{3.396058in}}%
\pgfpathclose%
\pgfusepath{fill}%
\end{pgfscope}%
\begin{pgfscope}%
\pgfpathrectangle{\pgfqpoint{1.020000in}{0.880000in}}{\pgfqpoint{6.160000in}{6.160000in}}%
\pgfusepath{clip}%
\pgfsetbuttcap%
\pgfsetroundjoin%
\definecolor{currentfill}{rgb}{0.527132,0.664700,0.989065}%
\pgfsetfillcolor{currentfill}%
\pgfsetlinewidth{0.000000pt}%
\definecolor{currentstroke}{rgb}{0.000000,0.000000,0.000000}%
\pgfsetstrokecolor{currentstroke}%
\pgfsetdash{}{0pt}%
\pgfpathmoveto{\pgfqpoint{4.274929in}{3.484113in}}%
\pgfpathlineto{\pgfqpoint{4.285302in}{3.465797in}}%
\pgfpathlineto{\pgfqpoint{4.295695in}{3.448374in}}%
\pgfpathlineto{\pgfqpoint{4.328633in}{3.435355in}}%
\pgfpathlineto{\pgfqpoint{4.361542in}{3.422454in}}%
\pgfpathlineto{\pgfqpoint{4.351100in}{3.439423in}}%
\pgfpathlineto{\pgfqpoint{4.340677in}{3.457203in}}%
\pgfpathlineto{\pgfqpoint{4.307818in}{3.470567in}}%
\pgfpathlineto{\pgfqpoint{4.274929in}{3.484113in}}%
\pgfpathclose%
\pgfusepath{fill}%
\end{pgfscope}%
\begin{pgfscope}%
\pgfpathrectangle{\pgfqpoint{1.020000in}{0.880000in}}{\pgfqpoint{6.160000in}{6.160000in}}%
\pgfusepath{clip}%
\pgfsetbuttcap%
\pgfsetroundjoin%
\definecolor{currentfill}{rgb}{0.959518,0.766973,0.674145}%
\pgfsetfillcolor{currentfill}%
\pgfsetlinewidth{0.000000pt}%
\definecolor{currentstroke}{rgb}{0.000000,0.000000,0.000000}%
\pgfsetstrokecolor{currentstroke}%
\pgfsetdash{}{0pt}%
\pgfpathmoveto{\pgfqpoint{3.537559in}{4.482561in}}%
\pgfpathlineto{\pgfqpoint{3.547451in}{4.438656in}}%
\pgfpathlineto{\pgfqpoint{3.557366in}{4.392866in}}%
\pgfpathlineto{\pgfqpoint{3.590813in}{4.347083in}}%
\pgfpathlineto{\pgfqpoint{3.624211in}{4.301168in}}%
\pgfpathlineto{\pgfqpoint{3.614288in}{4.343433in}}%
\pgfpathlineto{\pgfqpoint{3.604386in}{4.384054in}}%
\pgfpathlineto{\pgfqpoint{3.570998in}{4.433369in}}%
\pgfpathlineto{\pgfqpoint{3.537559in}{4.482561in}}%
\pgfpathclose%
\pgfusepath{fill}%
\end{pgfscope}%
\begin{pgfscope}%
\pgfpathrectangle{\pgfqpoint{1.020000in}{0.880000in}}{\pgfqpoint{6.160000in}{6.160000in}}%
\pgfusepath{clip}%
\pgfsetbuttcap%
\pgfsetroundjoin%
\definecolor{currentfill}{rgb}{0.774337,0.199759,0.202535}%
\pgfsetfillcolor{currentfill}%
\pgfsetlinewidth{0.000000pt}%
\definecolor{currentstroke}{rgb}{0.000000,0.000000,0.000000}%
\pgfsetstrokecolor{currentstroke}%
\pgfsetdash{}{0pt}%
\pgfpathmoveto{\pgfqpoint{3.105495in}{5.202805in}}%
\pgfpathlineto{\pgfqpoint{3.114774in}{5.193824in}}%
\pgfpathlineto{\pgfqpoint{3.124125in}{5.179295in}}%
\pgfpathlineto{\pgfqpoint{3.157735in}{5.150414in}}%
\pgfpathlineto{\pgfqpoint{3.191349in}{5.116841in}}%
\pgfpathlineto{\pgfqpoint{3.181950in}{5.131341in}}%
\pgfpathlineto{\pgfqpoint{3.172617in}{5.140528in}}%
\pgfpathlineto{\pgfqpoint{3.139054in}{5.173987in}}%
\pgfpathlineto{\pgfqpoint{3.105495in}{5.202805in}}%
\pgfpathclose%
\pgfusepath{fill}%
\end{pgfscope}%
\begin{pgfscope}%
\pgfpathrectangle{\pgfqpoint{1.020000in}{0.880000in}}{\pgfqpoint{6.160000in}{6.160000in}}%
\pgfusepath{clip}%
\pgfsetbuttcap%
\pgfsetroundjoin%
\definecolor{currentfill}{rgb}{0.713852,0.808857,0.979386}%
\pgfsetfillcolor{currentfill}%
\pgfsetlinewidth{0.000000pt}%
\definecolor{currentstroke}{rgb}{0.000000,0.000000,0.000000}%
\pgfsetstrokecolor{currentstroke}%
\pgfsetdash{}{0pt}%
\pgfpathmoveto{\pgfqpoint{3.883678in}{3.837911in}}%
\pgfpathlineto{\pgfqpoint{3.893754in}{3.804954in}}%
\pgfpathlineto{\pgfqpoint{3.903843in}{3.772865in}}%
\pgfpathlineto{\pgfqpoint{3.936986in}{3.746782in}}%
\pgfpathlineto{\pgfqpoint{3.970090in}{3.721817in}}%
\pgfpathlineto{\pgfqpoint{3.959969in}{3.751025in}}%
\pgfpathlineto{\pgfqpoint{3.949863in}{3.780997in}}%
\pgfpathlineto{\pgfqpoint{3.916791in}{3.808797in}}%
\pgfpathlineto{\pgfqpoint{3.883678in}{3.837911in}}%
\pgfpathclose%
\pgfusepath{fill}%
\end{pgfscope}%
\begin{pgfscope}%
\pgfpathrectangle{\pgfqpoint{1.020000in}{0.880000in}}{\pgfqpoint{6.160000in}{6.160000in}}%
\pgfusepath{clip}%
\pgfsetbuttcap%
\pgfsetroundjoin%
\definecolor{currentfill}{rgb}{0.586921,0.718121,0.998874}%
\pgfsetfillcolor{currentfill}%
\pgfsetlinewidth{0.000000pt}%
\definecolor{currentstroke}{rgb}{0.000000,0.000000,0.000000}%
\pgfsetstrokecolor{currentstroke}%
\pgfsetdash{}{0pt}%
\pgfpathmoveto{\pgfqpoint{4.122567in}{3.587043in}}%
\pgfpathlineto{\pgfqpoint{4.132812in}{3.564845in}}%
\pgfpathlineto{\pgfqpoint{4.143074in}{3.543690in}}%
\pgfpathlineto{\pgfqpoint{4.176083in}{3.527727in}}%
\pgfpathlineto{\pgfqpoint{4.209062in}{3.512545in}}%
\pgfpathlineto{\pgfqpoint{4.198754in}{3.532555in}}%
\pgfpathlineto{\pgfqpoint{4.188464in}{3.553452in}}%
\pgfpathlineto{\pgfqpoint{4.155532in}{3.569777in}}%
\pgfpathlineto{\pgfqpoint{4.122567in}{3.587043in}}%
\pgfpathclose%
\pgfusepath{fill}%
\end{pgfscope}%
\begin{pgfscope}%
\pgfpathrectangle{\pgfqpoint{1.020000in}{0.880000in}}{\pgfqpoint{6.160000in}{6.160000in}}%
\pgfusepath{clip}%
\pgfsetbuttcap%
\pgfsetroundjoin%
\definecolor{currentfill}{rgb}{0.333490,0.446265,0.874452}%
\pgfsetfillcolor{currentfill}%
\pgfsetlinewidth{0.000000pt}%
\definecolor{currentstroke}{rgb}{0.000000,0.000000,0.000000}%
\pgfsetstrokecolor{currentstroke}%
\pgfsetdash{}{0pt}%
\pgfpathmoveto{\pgfqpoint{5.320564in}{3.096575in}}%
\pgfpathlineto{\pgfqpoint{5.332047in}{3.096571in}}%
\pgfpathlineto{\pgfqpoint{5.343531in}{3.094507in}}%
\pgfpathlineto{\pgfqpoint{5.376208in}{3.099924in}}%
\pgfpathlineto{\pgfqpoint{5.408872in}{3.105806in}}%
\pgfpathlineto{\pgfqpoint{5.397356in}{3.110402in}}%
\pgfpathlineto{\pgfqpoint{5.385844in}{3.113311in}}%
\pgfpathlineto{\pgfqpoint{5.353209in}{3.104624in}}%
\pgfpathlineto{\pgfqpoint{5.320564in}{3.096575in}}%
\pgfpathclose%
\pgfusepath{fill}%
\end{pgfscope}%
\begin{pgfscope}%
\pgfpathrectangle{\pgfqpoint{1.020000in}{0.880000in}}{\pgfqpoint{6.160000in}{6.160000in}}%
\pgfusepath{clip}%
\pgfsetbuttcap%
\pgfsetroundjoin%
\definecolor{currentfill}{rgb}{0.717435,0.051118,0.158737}%
\pgfsetfillcolor{currentfill}%
\pgfsetlinewidth{0.000000pt}%
\definecolor{currentstroke}{rgb}{0.000000,0.000000,0.000000}%
\pgfsetstrokecolor{currentstroke}%
\pgfsetdash{}{0pt}%
\pgfpathmoveto{\pgfqpoint{2.953340in}{5.267805in}}%
\pgfpathlineto{\pgfqpoint{2.962351in}{5.269927in}}%
\pgfpathlineto{\pgfqpoint{2.971441in}{5.266561in}}%
\pgfpathlineto{\pgfqpoint{3.004914in}{5.258704in}}%
\pgfpathlineto{\pgfqpoint{3.038418in}{5.245373in}}%
\pgfpathlineto{\pgfqpoint{3.029269in}{5.248696in}}%
\pgfpathlineto{\pgfqpoint{3.020196in}{5.246587in}}%
\pgfpathlineto{\pgfqpoint{2.986753in}{5.259858in}}%
\pgfpathlineto{\pgfqpoint{2.953340in}{5.267805in}}%
\pgfpathclose%
\pgfusepath{fill}%
\end{pgfscope}%
\begin{pgfscope}%
\pgfpathrectangle{\pgfqpoint{1.020000in}{0.880000in}}{\pgfqpoint{6.160000in}{6.160000in}}%
\pgfusepath{clip}%
\pgfsetbuttcap%
\pgfsetroundjoin%
\definecolor{currentfill}{rgb}{0.968105,0.668475,0.550486}%
\pgfsetfillcolor{currentfill}%
\pgfsetlinewidth{0.000000pt}%
\definecolor{currentstroke}{rgb}{0.000000,0.000000,0.000000}%
\pgfsetstrokecolor{currentstroke}%
\pgfsetdash{}{0pt}%
\pgfpathmoveto{\pgfqpoint{3.450854in}{4.665317in}}%
\pgfpathlineto{\pgfqpoint{3.460681in}{4.623569in}}%
\pgfpathlineto{\pgfqpoint{3.470539in}{4.578936in}}%
\pgfpathlineto{\pgfqpoint{3.504071in}{4.531225in}}%
\pgfpathlineto{\pgfqpoint{3.537559in}{4.482561in}}%
\pgfpathlineto{\pgfqpoint{3.527692in}{4.524223in}}%
\pgfpathlineto{\pgfqpoint{3.517854in}{4.563310in}}%
\pgfpathlineto{\pgfqpoint{3.484377in}{4.614810in}}%
\pgfpathlineto{\pgfqpoint{3.450854in}{4.665317in}}%
\pgfpathclose%
\pgfusepath{fill}%
\end{pgfscope}%
\begin{pgfscope}%
\pgfpathrectangle{\pgfqpoint{1.020000in}{0.880000in}}{\pgfqpoint{6.160000in}{6.160000in}}%
\pgfusepath{clip}%
\pgfsetbuttcap%
\pgfsetroundjoin%
\definecolor{currentfill}{rgb}{0.318832,0.426605,0.859857}%
\pgfsetfillcolor{currentfill}%
\pgfsetlinewidth{0.000000pt}%
\definecolor{currentstroke}{rgb}{0.000000,0.000000,0.000000}%
\pgfsetstrokecolor{currentstroke}%
\pgfsetdash{}{0pt}%
\pgfpathmoveto{\pgfqpoint{5.036770in}{3.075479in}}%
\pgfpathlineto{\pgfqpoint{5.048014in}{3.082120in}}%
\pgfpathlineto{\pgfqpoint{5.059266in}{3.086873in}}%
\pgfpathlineto{\pgfqpoint{5.091944in}{3.081481in}}%
\pgfpathlineto{\pgfqpoint{5.124615in}{3.077989in}}%
\pgfpathlineto{\pgfqpoint{5.113288in}{3.071946in}}%
\pgfpathlineto{\pgfqpoint{5.101968in}{3.063866in}}%
\pgfpathlineto{\pgfqpoint{5.069371in}{3.068388in}}%
\pgfpathlineto{\pgfqpoint{5.036770in}{3.075479in}}%
\pgfpathclose%
\pgfusepath{fill}%
\end{pgfscope}%
\begin{pgfscope}%
\pgfpathrectangle{\pgfqpoint{1.020000in}{0.880000in}}{\pgfqpoint{6.160000in}{6.160000in}}%
\pgfusepath{clip}%
\pgfsetbuttcap%
\pgfsetroundjoin%
\definecolor{currentfill}{rgb}{0.348323,0.465711,0.888346}%
\pgfsetfillcolor{currentfill}%
\pgfsetlinewidth{0.000000pt}%
\definecolor{currentstroke}{rgb}{0.000000,0.000000,0.000000}%
\pgfsetstrokecolor{currentstroke}%
\pgfsetdash{}{0pt}%
\pgfpathmoveto{\pgfqpoint{5.539360in}{3.129955in}}%
\pgfpathlineto{\pgfqpoint{5.550940in}{3.119322in}}%
\pgfpathlineto{\pgfqpoint{5.562531in}{3.107806in}}%
\pgfpathlineto{\pgfqpoint{5.595134in}{3.111559in}}%
\pgfpathlineto{\pgfqpoint{5.627714in}{3.115069in}}%
\pgfpathlineto{\pgfqpoint{5.616088in}{3.128006in}}%
\pgfpathlineto{\pgfqpoint{5.604476in}{3.140356in}}%
\pgfpathlineto{\pgfqpoint{5.571930in}{3.135374in}}%
\pgfpathlineto{\pgfqpoint{5.539360in}{3.129955in}}%
\pgfpathclose%
\pgfusepath{fill}%
\end{pgfscope}%
\begin{pgfscope}%
\pgfpathrectangle{\pgfqpoint{1.020000in}{0.880000in}}{\pgfqpoint{6.160000in}{6.160000in}}%
\pgfusepath{clip}%
\pgfsetbuttcap%
\pgfsetroundjoin%
\definecolor{currentfill}{rgb}{0.830187,0.304733,0.254891}%
\pgfsetfillcolor{currentfill}%
\pgfsetlinewidth{0.000000pt}%
\definecolor{currentstroke}{rgb}{0.000000,0.000000,0.000000}%
\pgfsetstrokecolor{currentstroke}%
\pgfsetdash{}{0pt}%
\pgfpathmoveto{\pgfqpoint{3.191349in}{5.116841in}}%
\pgfpathlineto{\pgfqpoint{3.200812in}{5.097005in}}%
\pgfpathlineto{\pgfqpoint{3.210338in}{5.071878in}}%
\pgfpathlineto{\pgfqpoint{3.243987in}{5.034395in}}%
\pgfpathlineto{\pgfqpoint{3.277623in}{4.993001in}}%
\pgfpathlineto{\pgfqpoint{3.268062in}{5.017538in}}%
\pgfpathlineto{\pgfqpoint{3.258557in}{5.037093in}}%
\pgfpathlineto{\pgfqpoint{3.224959in}{5.078934in}}%
\pgfpathlineto{\pgfqpoint{3.191349in}{5.116841in}}%
\pgfpathclose%
\pgfusepath{fill}%
\end{pgfscope}%
\begin{pgfscope}%
\pgfpathrectangle{\pgfqpoint{1.020000in}{0.880000in}}{\pgfqpoint{6.160000in}{6.160000in}}%
\pgfusepath{clip}%
\pgfsetbuttcap%
\pgfsetroundjoin%
\definecolor{currentfill}{rgb}{0.656683,0.771806,0.994914}%
\pgfsetfillcolor{currentfill}%
\pgfsetlinewidth{0.000000pt}%
\definecolor{currentstroke}{rgb}{0.000000,0.000000,0.000000}%
\pgfsetstrokecolor{currentstroke}%
\pgfsetdash{}{0pt}%
\pgfpathmoveto{\pgfqpoint{3.970090in}{3.721817in}}%
\pgfpathlineto{\pgfqpoint{3.980224in}{3.693555in}}%
\pgfpathlineto{\pgfqpoint{3.990371in}{3.666412in}}%
\pgfpathlineto{\pgfqpoint{4.023473in}{3.645007in}}%
\pgfpathlineto{\pgfqpoint{4.056539in}{3.624636in}}%
\pgfpathlineto{\pgfqpoint{4.046353in}{3.649582in}}%
\pgfpathlineto{\pgfqpoint{4.036181in}{3.675492in}}%
\pgfpathlineto{\pgfqpoint{4.003154in}{3.698039in}}%
\pgfpathlineto{\pgfqpoint{3.970090in}{3.721817in}}%
\pgfpathclose%
\pgfusepath{fill}%
\end{pgfscope}%
\begin{pgfscope}%
\pgfpathrectangle{\pgfqpoint{1.020000in}{0.880000in}}{\pgfqpoint{6.160000in}{6.160000in}}%
\pgfusepath{clip}%
\pgfsetbuttcap%
\pgfsetroundjoin%
\definecolor{currentfill}{rgb}{0.944055,0.553153,0.435548}%
\pgfsetfillcolor{currentfill}%
\pgfsetlinewidth{0.000000pt}%
\definecolor{currentstroke}{rgb}{0.000000,0.000000,0.000000}%
\pgfsetstrokecolor{currentstroke}%
\pgfsetdash{}{0pt}%
\pgfpathmoveto{\pgfqpoint{3.364172in}{4.839114in}}%
\pgfpathlineto{\pgfqpoint{3.373909in}{4.802237in}}%
\pgfpathlineto{\pgfqpoint{3.383687in}{4.761498in}}%
\pgfpathlineto{\pgfqpoint{3.417289in}{4.714368in}}%
\pgfpathlineto{\pgfqpoint{3.450854in}{4.665317in}}%
\pgfpathlineto{\pgfqpoint{3.441062in}{4.703845in}}%
\pgfpathlineto{\pgfqpoint{3.431306in}{4.738861in}}%
\pgfpathlineto{\pgfqpoint{3.397758in}{4.789983in}}%
\pgfpathlineto{\pgfqpoint{3.364172in}{4.839114in}}%
\pgfpathclose%
\pgfusepath{fill}%
\end{pgfscope}%
\begin{pgfscope}%
\pgfpathrectangle{\pgfqpoint{1.020000in}{0.880000in}}{\pgfqpoint{6.160000in}{6.160000in}}%
\pgfusepath{clip}%
\pgfsetbuttcap%
\pgfsetroundjoin%
\definecolor{currentfill}{rgb}{0.343278,0.459354,0.884122}%
\pgfsetfillcolor{currentfill}%
\pgfsetlinewidth{0.000000pt}%
\definecolor{currentstroke}{rgb}{0.000000,0.000000,0.000000}%
\pgfsetstrokecolor{currentstroke}%
\pgfsetdash{}{0pt}%
\pgfpathmoveto{\pgfqpoint{5.757807in}{3.126467in}}%
\pgfpathlineto{\pgfqpoint{5.769532in}{3.111592in}}%
\pgfpathlineto{\pgfqpoint{5.781277in}{3.096557in}}%
\pgfpathlineto{\pgfqpoint{5.813789in}{3.098484in}}%
\pgfpathlineto{\pgfqpoint{5.846280in}{3.100319in}}%
\pgfpathlineto{\pgfqpoint{5.834488in}{3.115592in}}%
\pgfpathlineto{\pgfqpoint{5.822716in}{3.130788in}}%
\pgfpathlineto{\pgfqpoint{5.790273in}{3.128720in}}%
\pgfpathlineto{\pgfqpoint{5.757807in}{3.126467in}}%
\pgfpathclose%
\pgfusepath{fill}%
\end{pgfscope}%
\begin{pgfscope}%
\pgfpathrectangle{\pgfqpoint{1.020000in}{0.880000in}}{\pgfqpoint{6.160000in}{6.160000in}}%
\pgfusepath{clip}%
\pgfsetbuttcap%
\pgfsetroundjoin%
\definecolor{currentfill}{rgb}{0.892138,0.425389,0.333289}%
\pgfsetfillcolor{currentfill}%
\pgfsetlinewidth{0.000000pt}%
\definecolor{currentstroke}{rgb}{0.000000,0.000000,0.000000}%
\pgfsetstrokecolor{currentstroke}%
\pgfsetdash{}{0pt}%
\pgfpathmoveto{\pgfqpoint{3.277623in}{4.993001in}}%
\pgfpathlineto{\pgfqpoint{3.287240in}{4.963591in}}%
\pgfpathlineto{\pgfqpoint{3.296909in}{4.929481in}}%
\pgfpathlineto{\pgfqpoint{3.330553in}{4.885771in}}%
\pgfpathlineto{\pgfqpoint{3.364172in}{4.839114in}}%
\pgfpathlineto{\pgfqpoint{3.354479in}{4.871858in}}%
\pgfpathlineto{\pgfqpoint{3.344834in}{4.900251in}}%
\pgfpathlineto{\pgfqpoint{3.311241in}{4.948131in}}%
\pgfpathlineto{\pgfqpoint{3.277623in}{4.993001in}}%
\pgfpathclose%
\pgfusepath{fill}%
\end{pgfscope}%
\begin{pgfscope}%
\pgfpathrectangle{\pgfqpoint{1.020000in}{0.880000in}}{\pgfqpoint{6.160000in}{6.160000in}}%
\pgfusepath{clip}%
\pgfsetbuttcap%
\pgfsetroundjoin%
\definecolor{currentfill}{rgb}{0.328604,0.439712,0.869587}%
\pgfsetfillcolor{currentfill}%
\pgfsetlinewidth{0.000000pt}%
\definecolor{currentstroke}{rgb}{0.000000,0.000000,0.000000}%
\pgfsetstrokecolor{currentstroke}%
\pgfsetdash{}{0pt}%
\pgfpathmoveto{\pgfqpoint{5.255258in}{3.083696in}}%
\pgfpathlineto{\pgfqpoint{5.266702in}{3.085996in}}%
\pgfpathlineto{\pgfqpoint{5.278146in}{3.085946in}}%
\pgfpathlineto{\pgfqpoint{5.310843in}{3.089776in}}%
\pgfpathlineto{\pgfqpoint{5.343531in}{3.094507in}}%
\pgfpathlineto{\pgfqpoint{5.332047in}{3.096571in}}%
\pgfpathlineto{\pgfqpoint{5.320564in}{3.096575in}}%
\pgfpathlineto{\pgfqpoint{5.287913in}{3.089495in}}%
\pgfpathlineto{\pgfqpoint{5.255258in}{3.083696in}}%
\pgfpathclose%
\pgfusepath{fill}%
\end{pgfscope}%
\begin{pgfscope}%
\pgfpathrectangle{\pgfqpoint{1.020000in}{0.880000in}}{\pgfqpoint{6.160000in}{6.160000in}}%
\pgfusepath{clip}%
\pgfsetbuttcap%
\pgfsetroundjoin%
\definecolor{currentfill}{rgb}{0.399231,0.528528,0.928459}%
\pgfsetfillcolor{currentfill}%
\pgfsetlinewidth{0.000000pt}%
\definecolor{currentstroke}{rgb}{0.000000,0.000000,0.000000}%
\pgfsetstrokecolor{currentstroke}%
\pgfsetdash{}{0pt}%
\pgfpathmoveto{\pgfqpoint{4.666468in}{3.246405in}}%
\pgfpathlineto{\pgfqpoint{4.677237in}{3.237560in}}%
\pgfpathlineto{\pgfqpoint{4.688028in}{3.229208in}}%
\pgfpathlineto{\pgfqpoint{4.720802in}{3.212403in}}%
\pgfpathlineto{\pgfqpoint{4.753541in}{3.194711in}}%
\pgfpathlineto{\pgfqpoint{4.742682in}{3.200186in}}%
\pgfpathlineto{\pgfqpoint{4.731847in}{3.206177in}}%
\pgfpathlineto{\pgfqpoint{4.699178in}{3.226941in}}%
\pgfpathlineto{\pgfqpoint{4.666468in}{3.246405in}}%
\pgfpathclose%
\pgfusepath{fill}%
\end{pgfscope}%
\begin{pgfscope}%
\pgfpathrectangle{\pgfqpoint{1.020000in}{0.880000in}}{\pgfqpoint{6.160000in}{6.160000in}}%
\pgfusepath{clip}%
\pgfsetbuttcap%
\pgfsetroundjoin%
\definecolor{currentfill}{rgb}{0.358415,0.478426,0.896795}%
\pgfsetfillcolor{currentfill}%
\pgfsetlinewidth{0.000000pt}%
\definecolor{currentstroke}{rgb}{0.000000,0.000000,0.000000}%
\pgfsetstrokecolor{currentstroke}%
\pgfsetdash{}{0pt}%
\pgfpathmoveto{\pgfqpoint{4.818909in}{3.157359in}}%
\pgfpathlineto{\pgfqpoint{4.829866in}{3.155590in}}%
\pgfpathlineto{\pgfqpoint{4.840844in}{3.153494in}}%
\pgfpathlineto{\pgfqpoint{4.873557in}{3.138227in}}%
\pgfpathlineto{\pgfqpoint{4.906242in}{3.123344in}}%
\pgfpathlineto{\pgfqpoint{4.895182in}{3.121708in}}%
\pgfpathlineto{\pgfqpoint{4.884143in}{3.119555in}}%
\pgfpathlineto{\pgfqpoint{4.851541in}{3.138302in}}%
\pgfpathlineto{\pgfqpoint{4.818909in}{3.157359in}}%
\pgfpathclose%
\pgfusepath{fill}%
\end{pgfscope}%
\begin{pgfscope}%
\pgfpathrectangle{\pgfqpoint{1.020000in}{0.880000in}}{\pgfqpoint{6.160000in}{6.160000in}}%
\pgfusepath{clip}%
\pgfsetbuttcap%
\pgfsetroundjoin%
\definecolor{currentfill}{rgb}{0.705673,0.015556,0.150233}%
\pgfsetfillcolor{currentfill}%
\pgfsetlinewidth{0.000000pt}%
\definecolor{currentstroke}{rgb}{0.000000,0.000000,0.000000}%
\pgfsetstrokecolor{currentstroke}%
\pgfsetdash{}{0pt}%
\pgfpathmoveto{\pgfqpoint{2.886626in}{5.267634in}}%
\pgfpathlineto{\pgfqpoint{2.895577in}{5.269426in}}%
\pgfpathlineto{\pgfqpoint{2.904611in}{5.265763in}}%
\pgfpathlineto{\pgfqpoint{2.938005in}{5.268904in}}%
\pgfpathlineto{\pgfqpoint{2.971441in}{5.266561in}}%
\pgfpathlineto{\pgfqpoint{2.962351in}{5.269927in}}%
\pgfpathlineto{\pgfqpoint{2.953340in}{5.267805in}}%
\pgfpathlineto{\pgfqpoint{2.919963in}{5.270388in}}%
\pgfpathlineto{\pgfqpoint{2.886626in}{5.267634in}}%
\pgfpathclose%
\pgfusepath{fill}%
\end{pgfscope}%
\begin{pgfscope}%
\pgfpathrectangle{\pgfqpoint{1.020000in}{0.880000in}}{\pgfqpoint{6.160000in}{6.160000in}}%
\pgfusepath{clip}%
\pgfsetbuttcap%
\pgfsetroundjoin%
\definecolor{currentfill}{rgb}{0.451739,0.588181,0.960201}%
\pgfsetfillcolor{currentfill}%
\pgfsetlinewidth{0.000000pt}%
\definecolor{currentstroke}{rgb}{0.000000,0.000000,0.000000}%
\pgfsetstrokecolor{currentstroke}%
\pgfsetdash{}{0pt}%
\pgfpathmoveto{\pgfqpoint{4.514020in}{3.337475in}}%
\pgfpathlineto{\pgfqpoint{4.524627in}{3.323990in}}%
\pgfpathlineto{\pgfqpoint{4.535256in}{3.311254in}}%
\pgfpathlineto{\pgfqpoint{4.568113in}{3.296775in}}%
\pgfpathlineto{\pgfqpoint{4.600935in}{3.281248in}}%
\pgfpathlineto{\pgfqpoint{4.590250in}{3.292776in}}%
\pgfpathlineto{\pgfqpoint{4.579589in}{3.305124in}}%
\pgfpathlineto{\pgfqpoint{4.546822in}{3.321980in}}%
\pgfpathlineto{\pgfqpoint{4.514020in}{3.337475in}}%
\pgfpathclose%
\pgfusepath{fill}%
\end{pgfscope}%
\begin{pgfscope}%
\pgfpathrectangle{\pgfqpoint{1.020000in}{0.880000in}}{\pgfqpoint{6.160000in}{6.160000in}}%
\pgfusepath{clip}%
\pgfsetbuttcap%
\pgfsetroundjoin%
\definecolor{currentfill}{rgb}{0.500031,0.638508,0.981070}%
\pgfsetfillcolor{currentfill}%
\pgfsetlinewidth{0.000000pt}%
\definecolor{currentstroke}{rgb}{0.000000,0.000000,0.000000}%
\pgfsetstrokecolor{currentstroke}%
\pgfsetdash{}{0pt}%
\pgfpathmoveto{\pgfqpoint{4.361542in}{3.422454in}}%
\pgfpathlineto{\pgfqpoint{4.372004in}{3.406326in}}%
\pgfpathlineto{\pgfqpoint{4.382487in}{3.391056in}}%
\pgfpathlineto{\pgfqpoint{4.415416in}{3.378383in}}%
\pgfpathlineto{\pgfqpoint{4.448316in}{3.365403in}}%
\pgfpathlineto{\pgfqpoint{4.437782in}{3.380320in}}%
\pgfpathlineto{\pgfqpoint{4.427270in}{3.396058in}}%
\pgfpathlineto{\pgfqpoint{4.394421in}{3.409443in}}%
\pgfpathlineto{\pgfqpoint{4.361542in}{3.422454in}}%
\pgfpathclose%
\pgfusepath{fill}%
\end{pgfscope}%
\begin{pgfscope}%
\pgfpathrectangle{\pgfqpoint{1.020000in}{0.880000in}}{\pgfqpoint{6.160000in}{6.160000in}}%
\pgfusepath{clip}%
\pgfsetbuttcap%
\pgfsetroundjoin%
\definecolor{currentfill}{rgb}{0.548876,0.685104,0.994379}%
\pgfsetfillcolor{currentfill}%
\pgfsetlinewidth{0.000000pt}%
\definecolor{currentstroke}{rgb}{0.000000,0.000000,0.000000}%
\pgfsetstrokecolor{currentstroke}%
\pgfsetdash{}{0pt}%
\pgfpathmoveto{\pgfqpoint{4.209062in}{3.512545in}}%
\pgfpathlineto{\pgfqpoint{4.219388in}{3.493501in}}%
\pgfpathlineto{\pgfqpoint{4.229732in}{3.475492in}}%
\pgfpathlineto{\pgfqpoint{4.262728in}{3.461703in}}%
\pgfpathlineto{\pgfqpoint{4.295695in}{3.448374in}}%
\pgfpathlineto{\pgfqpoint{4.285302in}{3.465797in}}%
\pgfpathlineto{\pgfqpoint{4.274929in}{3.484113in}}%
\pgfpathlineto{\pgfqpoint{4.242010in}{3.498050in}}%
\pgfpathlineto{\pgfqpoint{4.209062in}{3.512545in}}%
\pgfpathclose%
\pgfusepath{fill}%
\end{pgfscope}%
\begin{pgfscope}%
\pgfpathrectangle{\pgfqpoint{1.020000in}{0.880000in}}{\pgfqpoint{6.160000in}{6.160000in}}%
\pgfusepath{clip}%
\pgfsetbuttcap%
\pgfsetroundjoin%
\definecolor{currentfill}{rgb}{0.343278,0.459354,0.884122}%
\pgfsetfillcolor{currentfill}%
\pgfsetlinewidth{0.000000pt}%
\definecolor{currentstroke}{rgb}{0.000000,0.000000,0.000000}%
\pgfsetstrokecolor{currentstroke}%
\pgfsetdash{}{0pt}%
\pgfpathmoveto{\pgfqpoint{5.474154in}{3.118114in}}%
\pgfpathlineto{\pgfqpoint{5.485704in}{3.109564in}}%
\pgfpathlineto{\pgfqpoint{5.497262in}{3.099777in}}%
\pgfpathlineto{\pgfqpoint{5.529907in}{3.103856in}}%
\pgfpathlineto{\pgfqpoint{5.562531in}{3.107806in}}%
\pgfpathlineto{\pgfqpoint{5.550940in}{3.119322in}}%
\pgfpathlineto{\pgfqpoint{5.539360in}{3.129955in}}%
\pgfpathlineto{\pgfqpoint{5.506767in}{3.124167in}}%
\pgfpathlineto{\pgfqpoint{5.474154in}{3.118114in}}%
\pgfpathclose%
\pgfusepath{fill}%
\end{pgfscope}%
\begin{pgfscope}%
\pgfpathrectangle{\pgfqpoint{1.020000in}{0.880000in}}{\pgfqpoint{6.160000in}{6.160000in}}%
\pgfusepath{clip}%
\pgfsetbuttcap%
\pgfsetroundjoin%
\definecolor{currentfill}{rgb}{0.818056,0.855590,0.914638}%
\pgfsetfillcolor{currentfill}%
\pgfsetlinewidth{0.000000pt}%
\definecolor{currentstroke}{rgb}{0.000000,0.000000,0.000000}%
\pgfsetstrokecolor{currentstroke}%
\pgfsetdash{}{0pt}%
\pgfpathmoveto{\pgfqpoint{3.730758in}{4.047063in}}%
\pgfpathlineto{\pgfqpoint{3.740768in}{4.005825in}}%
\pgfpathlineto{\pgfqpoint{3.750789in}{3.965056in}}%
\pgfpathlineto{\pgfqpoint{3.784078in}{3.931976in}}%
\pgfpathlineto{\pgfqpoint{3.817322in}{3.899642in}}%
\pgfpathlineto{\pgfqpoint{3.807282in}{3.936669in}}%
\pgfpathlineto{\pgfqpoint{3.797255in}{3.974140in}}%
\pgfpathlineto{\pgfqpoint{3.764030in}{4.010159in}}%
\pgfpathlineto{\pgfqpoint{3.730758in}{4.047063in}}%
\pgfpathclose%
\pgfusepath{fill}%
\end{pgfscope}%
\begin{pgfscope}%
\pgfpathrectangle{\pgfqpoint{1.020000in}{0.880000in}}{\pgfqpoint{6.160000in}{6.160000in}}%
\pgfusepath{clip}%
\pgfsetbuttcap%
\pgfsetroundjoin%
\definecolor{currentfill}{rgb}{0.887752,0.854040,0.834671}%
\pgfsetfillcolor{currentfill}%
\pgfsetlinewidth{0.000000pt}%
\definecolor{currentstroke}{rgb}{0.000000,0.000000,0.000000}%
\pgfsetstrokecolor{currentstroke}%
\pgfsetdash{}{0pt}%
\pgfpathmoveto{\pgfqpoint{3.644111in}{4.213068in}}%
\pgfpathlineto{\pgfqpoint{3.654085in}{4.167944in}}%
\pgfpathlineto{\pgfqpoint{3.664073in}{4.122587in}}%
\pgfpathlineto{\pgfqpoint{3.697438in}{4.084623in}}%
\pgfpathlineto{\pgfqpoint{3.730758in}{4.047063in}}%
\pgfpathlineto{\pgfqpoint{3.720760in}{4.088452in}}%
\pgfpathlineto{\pgfqpoint{3.710775in}{4.129676in}}%
\pgfpathlineto{\pgfqpoint{3.677468in}{4.171129in}}%
\pgfpathlineto{\pgfqpoint{3.644111in}{4.213068in}}%
\pgfpathclose%
\pgfusepath{fill}%
\end{pgfscope}%
\begin{pgfscope}%
\pgfpathrectangle{\pgfqpoint{1.020000in}{0.880000in}}{\pgfqpoint{6.160000in}{6.160000in}}%
\pgfusepath{clip}%
\pgfsetbuttcap%
\pgfsetroundjoin%
\definecolor{currentfill}{rgb}{0.613933,0.739923,0.999142}%
\pgfsetfillcolor{currentfill}%
\pgfsetlinewidth{0.000000pt}%
\definecolor{currentstroke}{rgb}{0.000000,0.000000,0.000000}%
\pgfsetstrokecolor{currentstroke}%
\pgfsetdash{}{0pt}%
\pgfpathmoveto{\pgfqpoint{4.056539in}{3.624636in}}%
\pgfpathlineto{\pgfqpoint{4.066741in}{3.600789in}}%
\pgfpathlineto{\pgfqpoint{4.076958in}{3.578166in}}%
\pgfpathlineto{\pgfqpoint{4.110032in}{3.560493in}}%
\pgfpathlineto{\pgfqpoint{4.143074in}{3.543690in}}%
\pgfpathlineto{\pgfqpoint{4.132812in}{3.564845in}}%
\pgfpathlineto{\pgfqpoint{4.122567in}{3.587043in}}%
\pgfpathlineto{\pgfqpoint{4.089570in}{3.605317in}}%
\pgfpathlineto{\pgfqpoint{4.056539in}{3.624636in}}%
\pgfpathclose%
\pgfusepath{fill}%
\end{pgfscope}%
\begin{pgfscope}%
\pgfpathrectangle{\pgfqpoint{1.020000in}{0.880000in}}{\pgfqpoint{6.160000in}{6.160000in}}%
\pgfusepath{clip}%
\pgfsetbuttcap%
\pgfsetroundjoin%
\definecolor{currentfill}{rgb}{0.966922,0.651969,0.531997}%
\pgfsetfillcolor{currentfill}%
\pgfsetlinewidth{0.000000pt}%
\definecolor{currentstroke}{rgb}{0.000000,0.000000,0.000000}%
\pgfsetstrokecolor{currentstroke}%
\pgfsetdash{}{0pt}%
\pgfpathmoveto{\pgfqpoint{2.227344in}{4.569050in}}%
\pgfpathlineto{\pgfqpoint{2.235856in}{4.556333in}}%
\pgfpathlineto{\pgfqpoint{2.244422in}{4.541402in}}%
\pgfpathlineto{\pgfqpoint{2.277367in}{4.580843in}}%
\pgfpathlineto{\pgfqpoint{2.310282in}{4.622209in}}%
\pgfpathlineto{\pgfqpoint{2.301677in}{4.635903in}}%
\pgfpathlineto{\pgfqpoint{2.293131in}{4.647022in}}%
\pgfpathlineto{\pgfqpoint{2.260253in}{4.607097in}}%
\pgfpathlineto{\pgfqpoint{2.227344in}{4.569050in}}%
\pgfpathclose%
\pgfusepath{fill}%
\end{pgfscope}%
\begin{pgfscope}%
\pgfpathrectangle{\pgfqpoint{1.020000in}{0.880000in}}{\pgfqpoint{6.160000in}{6.160000in}}%
\pgfusepath{clip}%
\pgfsetbuttcap%
\pgfsetroundjoin%
\definecolor{currentfill}{rgb}{0.343278,0.459354,0.884122}%
\pgfsetfillcolor{currentfill}%
\pgfsetlinewidth{0.000000pt}%
\definecolor{currentstroke}{rgb}{0.000000,0.000000,0.000000}%
\pgfsetstrokecolor{currentstroke}%
\pgfsetdash{}{0pt}%
\pgfpathmoveto{\pgfqpoint{5.692807in}{3.121288in}}%
\pgfpathlineto{\pgfqpoint{5.704489in}{3.106969in}}%
\pgfpathlineto{\pgfqpoint{5.716188in}{3.092362in}}%
\pgfpathlineto{\pgfqpoint{5.748743in}{3.094521in}}%
\pgfpathlineto{\pgfqpoint{5.781277in}{3.096557in}}%
\pgfpathlineto{\pgfqpoint{5.769532in}{3.111592in}}%
\pgfpathlineto{\pgfqpoint{5.757807in}{3.126467in}}%
\pgfpathlineto{\pgfqpoint{5.725319in}{3.123998in}}%
\pgfpathlineto{\pgfqpoint{5.692807in}{3.121288in}}%
\pgfpathclose%
\pgfusepath{fill}%
\end{pgfscope}%
\begin{pgfscope}%
\pgfpathrectangle{\pgfqpoint{1.020000in}{0.880000in}}{\pgfqpoint{6.160000in}{6.160000in}}%
\pgfusepath{clip}%
\pgfsetbuttcap%
\pgfsetroundjoin%
\definecolor{currentfill}{rgb}{0.740957,0.122240,0.175744}%
\pgfsetfillcolor{currentfill}%
\pgfsetlinewidth{0.000000pt}%
\definecolor{currentstroke}{rgb}{0.000000,0.000000,0.000000}%
\pgfsetstrokecolor{currentstroke}%
\pgfsetdash{}{0pt}%
\pgfpathmoveto{\pgfqpoint{3.038418in}{5.245373in}}%
\pgfpathlineto{\pgfqpoint{3.047644in}{5.236450in}}%
\pgfpathlineto{\pgfqpoint{3.056945in}{5.221828in}}%
\pgfpathlineto{\pgfqpoint{3.090526in}{5.203181in}}%
\pgfpathlineto{\pgfqpoint{3.124125in}{5.179295in}}%
\pgfpathlineto{\pgfqpoint{3.114774in}{5.193824in}}%
\pgfpathlineto{\pgfqpoint{3.105495in}{5.202805in}}%
\pgfpathlineto{\pgfqpoint{3.071947in}{5.226681in}}%
\pgfpathlineto{\pgfqpoint{3.038418in}{5.245373in}}%
\pgfpathclose%
\pgfusepath{fill}%
\end{pgfscope}%
\begin{pgfscope}%
\pgfpathrectangle{\pgfqpoint{1.020000in}{0.880000in}}{\pgfqpoint{6.160000in}{6.160000in}}%
\pgfusepath{clip}%
\pgfsetbuttcap%
\pgfsetroundjoin%
\definecolor{currentfill}{rgb}{0.333490,0.446265,0.874452}%
\pgfsetfillcolor{currentfill}%
\pgfsetlinewidth{0.000000pt}%
\definecolor{currentstroke}{rgb}{0.000000,0.000000,0.000000}%
\pgfsetstrokecolor{currentstroke}%
\pgfsetdash{}{0pt}%
\pgfpathmoveto{\pgfqpoint{4.971540in}{3.096260in}}%
\pgfpathlineto{\pgfqpoint{4.982701in}{3.100263in}}%
\pgfpathlineto{\pgfqpoint{4.993873in}{3.102638in}}%
\pgfpathlineto{\pgfqpoint{5.026577in}{3.093999in}}%
\pgfpathlineto{\pgfqpoint{5.059266in}{3.086873in}}%
\pgfpathlineto{\pgfqpoint{5.048014in}{3.082120in}}%
\pgfpathlineto{\pgfqpoint{5.036770in}{3.075479in}}%
\pgfpathlineto{\pgfqpoint{5.004162in}{3.084883in}}%
\pgfpathlineto{\pgfqpoint{4.971540in}{3.096260in}}%
\pgfpathclose%
\pgfusepath{fill}%
\end{pgfscope}%
\begin{pgfscope}%
\pgfpathrectangle{\pgfqpoint{1.020000in}{0.880000in}}{\pgfqpoint{6.160000in}{6.160000in}}%
\pgfusepath{clip}%
\pgfsetbuttcap%
\pgfsetroundjoin%
\definecolor{currentfill}{rgb}{0.753611,0.830233,0.960871}%
\pgfsetfillcolor{currentfill}%
\pgfsetlinewidth{0.000000pt}%
\definecolor{currentstroke}{rgb}{0.000000,0.000000,0.000000}%
\pgfsetstrokecolor{currentstroke}%
\pgfsetdash{}{0pt}%
\pgfpathmoveto{\pgfqpoint{3.817322in}{3.899642in}}%
\pgfpathlineto{\pgfqpoint{3.827373in}{3.863327in}}%
\pgfpathlineto{\pgfqpoint{3.837435in}{3.827981in}}%
\pgfpathlineto{\pgfqpoint{3.870659in}{3.799970in}}%
\pgfpathlineto{\pgfqpoint{3.903843in}{3.772865in}}%
\pgfpathlineto{\pgfqpoint{3.893754in}{3.804954in}}%
\pgfpathlineto{\pgfqpoint{3.883678in}{3.837911in}}%
\pgfpathlineto{\pgfqpoint{3.850522in}{3.868235in}}%
\pgfpathlineto{\pgfqpoint{3.817322in}{3.899642in}}%
\pgfpathclose%
\pgfusepath{fill}%
\end{pgfscope}%
\begin{pgfscope}%
\pgfpathrectangle{\pgfqpoint{1.020000in}{0.880000in}}{\pgfqpoint{6.160000in}{6.160000in}}%
\pgfusepath{clip}%
\pgfsetbuttcap%
\pgfsetroundjoin%
\definecolor{currentfill}{rgb}{0.333490,0.446265,0.874452}%
\pgfsetfillcolor{currentfill}%
\pgfsetlinewidth{0.000000pt}%
\definecolor{currentstroke}{rgb}{0.000000,0.000000,0.000000}%
\pgfsetstrokecolor{currentstroke}%
\pgfsetdash{}{0pt}%
\pgfpathmoveto{\pgfqpoint{5.911197in}{3.103776in}}%
\pgfpathlineto{\pgfqpoint{5.923061in}{3.088382in}}%
\pgfpathlineto{\pgfqpoint{5.934945in}{3.072979in}}%
\pgfpathlineto{\pgfqpoint{5.967424in}{3.074642in}}%
\pgfpathlineto{\pgfqpoint{5.955513in}{3.090035in}}%
\pgfpathlineto{\pgfqpoint{5.943625in}{3.105429in}}%
\pgfpathlineto{\pgfqpoint{5.911197in}{3.103776in}}%
\pgfpathclose%
\pgfusepath{fill}%
\end{pgfscope}%
\begin{pgfscope}%
\pgfpathrectangle{\pgfqpoint{1.020000in}{0.880000in}}{\pgfqpoint{6.160000in}{6.160000in}}%
\pgfusepath{clip}%
\pgfsetbuttcap%
\pgfsetroundjoin%
\definecolor{currentfill}{rgb}{0.956653,0.598034,0.477302}%
\pgfsetfillcolor{currentfill}%
\pgfsetlinewidth{0.000000pt}%
\definecolor{currentstroke}{rgb}{0.000000,0.000000,0.000000}%
\pgfsetstrokecolor{currentstroke}%
\pgfsetdash{}{0pt}%
\pgfpathmoveto{\pgfqpoint{2.293131in}{4.647022in}}%
\pgfpathlineto{\pgfqpoint{2.301677in}{4.635903in}}%
\pgfpathlineto{\pgfqpoint{2.310282in}{4.622209in}}%
\pgfpathlineto{\pgfqpoint{2.343173in}{4.665254in}}%
\pgfpathlineto{\pgfqpoint{2.376045in}{4.709687in}}%
\pgfpathlineto{\pgfqpoint{2.367399in}{4.722074in}}%
\pgfpathlineto{\pgfqpoint{2.358819in}{4.731498in}}%
\pgfpathlineto{\pgfqpoint{2.325984in}{4.688583in}}%
\pgfpathlineto{\pgfqpoint{2.293131in}{4.647022in}}%
\pgfpathclose%
\pgfusepath{fill}%
\end{pgfscope}%
\begin{pgfscope}%
\pgfpathrectangle{\pgfqpoint{1.020000in}{0.880000in}}{\pgfqpoint{6.160000in}{6.160000in}}%
\pgfusepath{clip}%
\pgfsetbuttcap%
\pgfsetroundjoin%
\definecolor{currentfill}{rgb}{0.945540,0.798606,0.723105}%
\pgfsetfillcolor{currentfill}%
\pgfsetlinewidth{0.000000pt}%
\definecolor{currentstroke}{rgb}{0.000000,0.000000,0.000000}%
\pgfsetstrokecolor{currentstroke}%
\pgfsetdash{}{0pt}%
\pgfpathmoveto{\pgfqpoint{3.557366in}{4.392866in}}%
\pgfpathlineto{\pgfqpoint{3.567302in}{4.345570in}}%
\pgfpathlineto{\pgfqpoint{3.577255in}{4.297159in}}%
\pgfpathlineto{\pgfqpoint{3.610707in}{4.255187in}}%
\pgfpathlineto{\pgfqpoint{3.644111in}{4.213068in}}%
\pgfpathlineto{\pgfqpoint{3.634152in}{4.257595in}}%
\pgfpathlineto{\pgfqpoint{3.624211in}{4.301168in}}%
\pgfpathlineto{\pgfqpoint{3.590813in}{4.347083in}}%
\pgfpathlineto{\pgfqpoint{3.557366in}{4.392866in}}%
\pgfpathclose%
\pgfusepath{fill}%
\end{pgfscope}%
\begin{pgfscope}%
\pgfpathrectangle{\pgfqpoint{1.020000in}{0.880000in}}{\pgfqpoint{6.160000in}{6.160000in}}%
\pgfusepath{clip}%
\pgfsetbuttcap%
\pgfsetroundjoin%
\definecolor{currentfill}{rgb}{0.328604,0.439712,0.869587}%
\pgfsetfillcolor{currentfill}%
\pgfsetlinewidth{0.000000pt}%
\definecolor{currentstroke}{rgb}{0.000000,0.000000,0.000000}%
\pgfsetstrokecolor{currentstroke}%
\pgfsetdash{}{0pt}%
\pgfpathmoveto{\pgfqpoint{5.189941in}{3.077008in}}%
\pgfpathlineto{\pgfqpoint{5.201335in}{3.080617in}}%
\pgfpathlineto{\pgfqpoint{5.212728in}{3.081719in}}%
\pgfpathlineto{\pgfqpoint{5.245441in}{3.083209in}}%
\pgfpathlineto{\pgfqpoint{5.278146in}{3.085946in}}%
\pgfpathlineto{\pgfqpoint{5.266702in}{3.085996in}}%
\pgfpathlineto{\pgfqpoint{5.255258in}{3.083696in}}%
\pgfpathlineto{\pgfqpoint{5.222601in}{3.079459in}}%
\pgfpathlineto{\pgfqpoint{5.189941in}{3.077008in}}%
\pgfpathclose%
\pgfusepath{fill}%
\end{pgfscope}%
\begin{pgfscope}%
\pgfpathrectangle{\pgfqpoint{1.020000in}{0.880000in}}{\pgfqpoint{6.160000in}{6.160000in}}%
\pgfusepath{clip}%
\pgfsetbuttcap%
\pgfsetroundjoin%
\definecolor{currentfill}{rgb}{0.936780,0.532750,0.418093}%
\pgfsetfillcolor{currentfill}%
\pgfsetlinewidth{0.000000pt}%
\definecolor{currentstroke}{rgb}{0.000000,0.000000,0.000000}%
\pgfsetstrokecolor{currentstroke}%
\pgfsetdash{}{0pt}%
\pgfpathmoveto{\pgfqpoint{2.358819in}{4.731498in}}%
\pgfpathlineto{\pgfqpoint{2.367399in}{4.722074in}}%
\pgfpathlineto{\pgfqpoint{2.376045in}{4.709687in}}%
\pgfpathlineto{\pgfqpoint{2.408906in}{4.755176in}}%
\pgfpathlineto{\pgfqpoint{2.441763in}{4.801351in}}%
\pgfpathlineto{\pgfqpoint{2.433075in}{4.812388in}}%
\pgfpathlineto{\pgfqpoint{2.424458in}{4.820059in}}%
\pgfpathlineto{\pgfqpoint{2.391641in}{4.775444in}}%
\pgfpathlineto{\pgfqpoint{2.358819in}{4.731498in}}%
\pgfpathclose%
\pgfusepath{fill}%
\end{pgfscope}%
\begin{pgfscope}%
\pgfpathrectangle{\pgfqpoint{1.020000in}{0.880000in}}{\pgfqpoint{6.160000in}{6.160000in}}%
\pgfusepath{clip}%
\pgfsetbuttcap%
\pgfsetroundjoin%
\definecolor{currentfill}{rgb}{0.711554,0.033337,0.154485}%
\pgfsetfillcolor{currentfill}%
\pgfsetlinewidth{0.000000pt}%
\definecolor{currentstroke}{rgb}{0.000000,0.000000,0.000000}%
\pgfsetstrokecolor{currentstroke}%
\pgfsetdash{}{0pt}%
\pgfpathmoveto{\pgfqpoint{2.820088in}{5.246575in}}%
\pgfpathlineto{\pgfqpoint{2.828982in}{5.247702in}}%
\pgfpathlineto{\pgfqpoint{2.837960in}{5.243493in}}%
\pgfpathlineto{\pgfqpoint{2.871261in}{5.257236in}}%
\pgfpathlineto{\pgfqpoint{2.904611in}{5.265763in}}%
\pgfpathlineto{\pgfqpoint{2.895577in}{5.269426in}}%
\pgfpathlineto{\pgfqpoint{2.886626in}{5.267634in}}%
\pgfpathlineto{\pgfqpoint{2.853334in}{5.259643in}}%
\pgfpathlineto{\pgfqpoint{2.820088in}{5.246575in}}%
\pgfpathclose%
\pgfusepath{fill}%
\end{pgfscope}%
\begin{pgfscope}%
\pgfpathrectangle{\pgfqpoint{1.020000in}{0.880000in}}{\pgfqpoint{6.160000in}{6.160000in}}%
\pgfusepath{clip}%
\pgfsetbuttcap%
\pgfsetroundjoin%
\definecolor{currentfill}{rgb}{0.968863,0.710838,0.599901}%
\pgfsetfillcolor{currentfill}%
\pgfsetlinewidth{0.000000pt}%
\definecolor{currentstroke}{rgb}{0.000000,0.000000,0.000000}%
\pgfsetstrokecolor{currentstroke}%
\pgfsetdash{}{0pt}%
\pgfpathmoveto{\pgfqpoint{3.470539in}{4.578936in}}%
\pgfpathlineto{\pgfqpoint{3.480425in}{4.531783in}}%
\pgfpathlineto{\pgfqpoint{3.490337in}{4.482506in}}%
\pgfpathlineto{\pgfqpoint{3.523873in}{4.438139in}}%
\pgfpathlineto{\pgfqpoint{3.557366in}{4.392866in}}%
\pgfpathlineto{\pgfqpoint{3.547451in}{4.438656in}}%
\pgfpathlineto{\pgfqpoint{3.537559in}{4.482561in}}%
\pgfpathlineto{\pgfqpoint{3.504071in}{4.531225in}}%
\pgfpathlineto{\pgfqpoint{3.470539in}{4.578936in}}%
\pgfpathclose%
\pgfusepath{fill}%
\end{pgfscope}%
\begin{pgfscope}%
\pgfpathrectangle{\pgfqpoint{1.020000in}{0.880000in}}{\pgfqpoint{6.160000in}{6.160000in}}%
\pgfusepath{clip}%
\pgfsetbuttcap%
\pgfsetroundjoin%
\definecolor{currentfill}{rgb}{0.693321,0.796314,0.986308}%
\pgfsetfillcolor{currentfill}%
\pgfsetlinewidth{0.000000pt}%
\definecolor{currentstroke}{rgb}{0.000000,0.000000,0.000000}%
\pgfsetstrokecolor{currentstroke}%
\pgfsetdash{}{0pt}%
\pgfpathmoveto{\pgfqpoint{3.903843in}{3.772865in}}%
\pgfpathlineto{\pgfqpoint{3.913944in}{3.741856in}}%
\pgfpathlineto{\pgfqpoint{3.924056in}{3.712126in}}%
\pgfpathlineto{\pgfqpoint{3.957233in}{3.688807in}}%
\pgfpathlineto{\pgfqpoint{3.990371in}{3.666412in}}%
\pgfpathlineto{\pgfqpoint{3.980224in}{3.693555in}}%
\pgfpathlineto{\pgfqpoint{3.970090in}{3.721817in}}%
\pgfpathlineto{\pgfqpoint{3.936986in}{3.746782in}}%
\pgfpathlineto{\pgfqpoint{3.903843in}{3.772865in}}%
\pgfpathclose%
\pgfusepath{fill}%
\end{pgfscope}%
\begin{pgfscope}%
\pgfpathrectangle{\pgfqpoint{1.020000in}{0.880000in}}{\pgfqpoint{6.160000in}{6.160000in}}%
\pgfusepath{clip}%
\pgfsetbuttcap%
\pgfsetroundjoin%
\definecolor{currentfill}{rgb}{0.343278,0.459354,0.884122}%
\pgfsetfillcolor{currentfill}%
\pgfsetlinewidth{0.000000pt}%
\definecolor{currentstroke}{rgb}{0.000000,0.000000,0.000000}%
\pgfsetstrokecolor{currentstroke}%
\pgfsetdash{}{0pt}%
\pgfpathmoveto{\pgfqpoint{5.408872in}{3.105806in}}%
\pgfpathlineto{\pgfqpoint{5.420392in}{3.099519in}}%
\pgfpathlineto{\pgfqpoint{5.431915in}{3.091615in}}%
\pgfpathlineto{\pgfqpoint{5.464597in}{3.095660in}}%
\pgfpathlineto{\pgfqpoint{5.497262in}{3.099777in}}%
\pgfpathlineto{\pgfqpoint{5.485704in}{3.109564in}}%
\pgfpathlineto{\pgfqpoint{5.474154in}{3.118114in}}%
\pgfpathlineto{\pgfqpoint{5.441521in}{3.111936in}}%
\pgfpathlineto{\pgfqpoint{5.408872in}{3.105806in}}%
\pgfpathclose%
\pgfusepath{fill}%
\end{pgfscope}%
\begin{pgfscope}%
\pgfpathrectangle{\pgfqpoint{1.020000in}{0.880000in}}{\pgfqpoint{6.160000in}{6.160000in}}%
\pgfusepath{clip}%
\pgfsetbuttcap%
\pgfsetroundjoin%
\definecolor{currentfill}{rgb}{0.908908,0.462433,0.360950}%
\pgfsetfillcolor{currentfill}%
\pgfsetlinewidth{0.000000pt}%
\definecolor{currentstroke}{rgb}{0.000000,0.000000,0.000000}%
\pgfsetstrokecolor{currentstroke}%
\pgfsetdash{}{0pt}%
\pgfpathmoveto{\pgfqpoint{2.424458in}{4.820059in}}%
\pgfpathlineto{\pgfqpoint{2.433075in}{4.812388in}}%
\pgfpathlineto{\pgfqpoint{2.441763in}{4.801351in}}%
\pgfpathlineto{\pgfqpoint{2.474622in}{4.847808in}}%
\pgfpathlineto{\pgfqpoint{2.507491in}{4.894109in}}%
\pgfpathlineto{\pgfqpoint{2.498760in}{4.903787in}}%
\pgfpathlineto{\pgfqpoint{2.490104in}{4.909692in}}%
\pgfpathlineto{\pgfqpoint{2.457277in}{4.864950in}}%
\pgfpathlineto{\pgfqpoint{2.424458in}{4.820059in}}%
\pgfpathclose%
\pgfusepath{fill}%
\end{pgfscope}%
\begin{pgfscope}%
\pgfpathrectangle{\pgfqpoint{1.020000in}{0.880000in}}{\pgfqpoint{6.160000in}{6.160000in}}%
\pgfusepath{clip}%
\pgfsetbuttcap%
\pgfsetroundjoin%
\definecolor{currentfill}{rgb}{0.333490,0.446265,0.874452}%
\pgfsetfillcolor{currentfill}%
\pgfsetlinewidth{0.000000pt}%
\definecolor{currentstroke}{rgb}{0.000000,0.000000,0.000000}%
\pgfsetstrokecolor{currentstroke}%
\pgfsetdash{}{0pt}%
\pgfpathmoveto{\pgfqpoint{5.846280in}{3.100319in}}%
\pgfpathlineto{\pgfqpoint{5.858093in}{3.084984in}}%
\pgfpathlineto{\pgfqpoint{5.869927in}{3.069603in}}%
\pgfpathlineto{\pgfqpoint{5.902446in}{3.071301in}}%
\pgfpathlineto{\pgfqpoint{5.934945in}{3.072979in}}%
\pgfpathlineto{\pgfqpoint{5.923061in}{3.088382in}}%
\pgfpathlineto{\pgfqpoint{5.911197in}{3.103776in}}%
\pgfpathlineto{\pgfqpoint{5.878749in}{3.102077in}}%
\pgfpathlineto{\pgfqpoint{5.846280in}{3.100319in}}%
\pgfpathclose%
\pgfusepath{fill}%
\end{pgfscope}%
\begin{pgfscope}%
\pgfpathrectangle{\pgfqpoint{1.020000in}{0.880000in}}{\pgfqpoint{6.160000in}{6.160000in}}%
\pgfusepath{clip}%
\pgfsetbuttcap%
\pgfsetroundjoin%
\definecolor{currentfill}{rgb}{0.343278,0.459354,0.884122}%
\pgfsetfillcolor{currentfill}%
\pgfsetlinewidth{0.000000pt}%
\definecolor{currentstroke}{rgb}{0.000000,0.000000,0.000000}%
\pgfsetstrokecolor{currentstroke}%
\pgfsetdash{}{0pt}%
\pgfpathmoveto{\pgfqpoint{5.627714in}{3.115069in}}%
\pgfpathlineto{\pgfqpoint{5.639355in}{3.101596in}}%
\pgfpathlineto{\pgfqpoint{5.651011in}{3.087649in}}%
\pgfpathlineto{\pgfqpoint{5.683610in}{3.090072in}}%
\pgfpathlineto{\pgfqpoint{5.716188in}{3.092362in}}%
\pgfpathlineto{\pgfqpoint{5.704489in}{3.106969in}}%
\pgfpathlineto{\pgfqpoint{5.692807in}{3.121288in}}%
\pgfpathlineto{\pgfqpoint{5.660272in}{3.118314in}}%
\pgfpathlineto{\pgfqpoint{5.627714in}{3.115069in}}%
\pgfpathclose%
\pgfusepath{fill}%
\end{pgfscope}%
\begin{pgfscope}%
\pgfpathrectangle{\pgfqpoint{1.020000in}{0.880000in}}{\pgfqpoint{6.160000in}{6.160000in}}%
\pgfusepath{clip}%
\pgfsetbuttcap%
\pgfsetroundjoin%
\definecolor{currentfill}{rgb}{0.785153,0.220851,0.211673}%
\pgfsetfillcolor{currentfill}%
\pgfsetlinewidth{0.000000pt}%
\definecolor{currentstroke}{rgb}{0.000000,0.000000,0.000000}%
\pgfsetstrokecolor{currentstroke}%
\pgfsetdash{}{0pt}%
\pgfpathmoveto{\pgfqpoint{3.124125in}{5.179295in}}%
\pgfpathlineto{\pgfqpoint{3.133546in}{5.159192in}}%
\pgfpathlineto{\pgfqpoint{3.143035in}{5.133562in}}%
\pgfpathlineto{\pgfqpoint{3.176685in}{5.105052in}}%
\pgfpathlineto{\pgfqpoint{3.210338in}{5.071878in}}%
\pgfpathlineto{\pgfqpoint{3.200812in}{5.097005in}}%
\pgfpathlineto{\pgfqpoint{3.191349in}{5.116841in}}%
\pgfpathlineto{\pgfqpoint{3.157735in}{5.150414in}}%
\pgfpathlineto{\pgfqpoint{3.124125in}{5.179295in}}%
\pgfpathclose%
\pgfusepath{fill}%
\end{pgfscope}%
\begin{pgfscope}%
\pgfpathrectangle{\pgfqpoint{1.020000in}{0.880000in}}{\pgfqpoint{6.160000in}{6.160000in}}%
\pgfusepath{clip}%
\pgfsetbuttcap%
\pgfsetroundjoin%
\definecolor{currentfill}{rgb}{0.729196,0.086679,0.167240}%
\pgfsetfillcolor{currentfill}%
\pgfsetlinewidth{0.000000pt}%
\definecolor{currentstroke}{rgb}{0.000000,0.000000,0.000000}%
\pgfsetstrokecolor{currentstroke}%
\pgfsetdash{}{0pt}%
\pgfpathmoveto{\pgfqpoint{2.753744in}{5.206174in}}%
\pgfpathlineto{\pgfqpoint{2.762584in}{5.206326in}}%
\pgfpathlineto{\pgfqpoint{2.771507in}{5.201342in}}%
\pgfpathlineto{\pgfqpoint{2.804708in}{5.224764in}}%
\pgfpathlineto{\pgfqpoint{2.837960in}{5.243493in}}%
\pgfpathlineto{\pgfqpoint{2.828982in}{5.247702in}}%
\pgfpathlineto{\pgfqpoint{2.820088in}{5.246575in}}%
\pgfpathlineto{\pgfqpoint{2.786892in}{5.228658in}}%
\pgfpathlineto{\pgfqpoint{2.753744in}{5.206174in}}%
\pgfpathclose%
\pgfusepath{fill}%
\end{pgfscope}%
\begin{pgfscope}%
\pgfpathrectangle{\pgfqpoint{1.020000in}{0.880000in}}{\pgfqpoint{6.160000in}{6.160000in}}%
\pgfusepath{clip}%
\pgfsetbuttcap%
\pgfsetroundjoin%
\definecolor{currentfill}{rgb}{0.877149,0.394645,0.311724}%
\pgfsetfillcolor{currentfill}%
\pgfsetlinewidth{0.000000pt}%
\definecolor{currentstroke}{rgb}{0.000000,0.000000,0.000000}%
\pgfsetstrokecolor{currentstroke}%
\pgfsetdash{}{0pt}%
\pgfpathmoveto{\pgfqpoint{2.490104in}{4.909692in}}%
\pgfpathlineto{\pgfqpoint{2.498760in}{4.903787in}}%
\pgfpathlineto{\pgfqpoint{2.507491in}{4.894109in}}%
\pgfpathlineto{\pgfqpoint{2.540376in}{4.939793in}}%
\pgfpathlineto{\pgfqpoint{2.573283in}{4.984379in}}%
\pgfpathlineto{\pgfqpoint{2.564508in}{4.992728in}}%
\pgfpathlineto{\pgfqpoint{2.555811in}{4.996908in}}%
\pgfpathlineto{\pgfqpoint{2.522947in}{4.953834in}}%
\pgfpathlineto{\pgfqpoint{2.490104in}{4.909692in}}%
\pgfpathclose%
\pgfusepath{fill}%
\end{pgfscope}%
\begin{pgfscope}%
\pgfpathrectangle{\pgfqpoint{1.020000in}{0.880000in}}{\pgfqpoint{6.160000in}{6.160000in}}%
\pgfusepath{clip}%
\pgfsetbuttcap%
\pgfsetroundjoin%
\definecolor{currentfill}{rgb}{0.521696,0.659599,0.987736}%
\pgfsetfillcolor{currentfill}%
\pgfsetlinewidth{0.000000pt}%
\definecolor{currentstroke}{rgb}{0.000000,0.000000,0.000000}%
\pgfsetstrokecolor{currentstroke}%
\pgfsetdash{}{0pt}%
\pgfpathmoveto{\pgfqpoint{4.295695in}{3.448374in}}%
\pgfpathlineto{\pgfqpoint{4.306108in}{3.431889in}}%
\pgfpathlineto{\pgfqpoint{4.316540in}{3.416373in}}%
\pgfpathlineto{\pgfqpoint{4.349528in}{3.403653in}}%
\pgfpathlineto{\pgfqpoint{4.382487in}{3.391056in}}%
\pgfpathlineto{\pgfqpoint{4.372004in}{3.406326in}}%
\pgfpathlineto{\pgfqpoint{4.361542in}{3.422454in}}%
\pgfpathlineto{\pgfqpoint{4.328633in}{3.435355in}}%
\pgfpathlineto{\pgfqpoint{4.295695in}{3.448374in}}%
\pgfpathclose%
\pgfusepath{fill}%
\end{pgfscope}%
\begin{pgfscope}%
\pgfpathrectangle{\pgfqpoint{1.020000in}{0.880000in}}{\pgfqpoint{6.160000in}{6.160000in}}%
\pgfusepath{clip}%
\pgfsetbuttcap%
\pgfsetroundjoin%
\definecolor{currentfill}{rgb}{0.473070,0.611077,0.970634}%
\pgfsetfillcolor{currentfill}%
\pgfsetlinewidth{0.000000pt}%
\definecolor{currentstroke}{rgb}{0.000000,0.000000,0.000000}%
\pgfsetstrokecolor{currentstroke}%
\pgfsetdash{}{0pt}%
\pgfpathmoveto{\pgfqpoint{4.448316in}{3.365403in}}%
\pgfpathlineto{\pgfqpoint{4.458870in}{3.351286in}}%
\pgfpathlineto{\pgfqpoint{4.469446in}{3.337935in}}%
\pgfpathlineto{\pgfqpoint{4.502367in}{3.324897in}}%
\pgfpathlineto{\pgfqpoint{4.535256in}{3.311254in}}%
\pgfpathlineto{\pgfqpoint{4.524627in}{3.323990in}}%
\pgfpathlineto{\pgfqpoint{4.514020in}{3.337475in}}%
\pgfpathlineto{\pgfqpoint{4.481184in}{3.351858in}}%
\pgfpathlineto{\pgfqpoint{4.448316in}{3.365403in}}%
\pgfpathclose%
\pgfusepath{fill}%
\end{pgfscope}%
\begin{pgfscope}%
\pgfpathrectangle{\pgfqpoint{1.020000in}{0.880000in}}{\pgfqpoint{6.160000in}{6.160000in}}%
\pgfusepath{clip}%
\pgfsetbuttcap%
\pgfsetroundjoin%
\definecolor{currentfill}{rgb}{0.956653,0.598034,0.477302}%
\pgfsetfillcolor{currentfill}%
\pgfsetlinewidth{0.000000pt}%
\definecolor{currentstroke}{rgb}{0.000000,0.000000,0.000000}%
\pgfsetstrokecolor{currentstroke}%
\pgfsetdash{}{0pt}%
\pgfpathmoveto{\pgfqpoint{3.383687in}{4.761498in}}%
\pgfpathlineto{\pgfqpoint{3.393504in}{4.717216in}}%
\pgfpathlineto{\pgfqpoint{3.403355in}{4.669752in}}%
\pgfpathlineto{\pgfqpoint{3.436966in}{4.625258in}}%
\pgfpathlineto{\pgfqpoint{3.470539in}{4.578936in}}%
\pgfpathlineto{\pgfqpoint{3.460681in}{4.623569in}}%
\pgfpathlineto{\pgfqpoint{3.450854in}{4.665317in}}%
\pgfpathlineto{\pgfqpoint{3.417289in}{4.714368in}}%
\pgfpathlineto{\pgfqpoint{3.383687in}{4.761498in}}%
\pgfpathclose%
\pgfusepath{fill}%
\end{pgfscope}%
\begin{pgfscope}%
\pgfpathrectangle{\pgfqpoint{1.020000in}{0.880000in}}{\pgfqpoint{6.160000in}{6.160000in}}%
\pgfusepath{clip}%
\pgfsetbuttcap%
\pgfsetroundjoin%
\definecolor{currentfill}{rgb}{0.425199,0.559058,0.946061}%
\pgfsetfillcolor{currentfill}%
\pgfsetlinewidth{0.000000pt}%
\definecolor{currentstroke}{rgb}{0.000000,0.000000,0.000000}%
\pgfsetstrokecolor{currentstroke}%
\pgfsetdash{}{0pt}%
\pgfpathmoveto{\pgfqpoint{4.600935in}{3.281248in}}%
\pgfpathlineto{\pgfqpoint{4.611642in}{3.270388in}}%
\pgfpathlineto{\pgfqpoint{4.622372in}{3.260038in}}%
\pgfpathlineto{\pgfqpoint{4.655217in}{3.245076in}}%
\pgfpathlineto{\pgfqpoint{4.688028in}{3.229208in}}%
\pgfpathlineto{\pgfqpoint{4.677237in}{3.237560in}}%
\pgfpathlineto{\pgfqpoint{4.666468in}{3.246405in}}%
\pgfpathlineto{\pgfqpoint{4.633720in}{3.264497in}}%
\pgfpathlineto{\pgfqpoint{4.600935in}{3.281248in}}%
\pgfpathclose%
\pgfusepath{fill}%
\end{pgfscope}%
\begin{pgfscope}%
\pgfpathrectangle{\pgfqpoint{1.020000in}{0.880000in}}{\pgfqpoint{6.160000in}{6.160000in}}%
\pgfusepath{clip}%
\pgfsetbuttcap%
\pgfsetroundjoin%
\definecolor{currentfill}{rgb}{0.570616,0.704109,0.997195}%
\pgfsetfillcolor{currentfill}%
\pgfsetlinewidth{0.000000pt}%
\definecolor{currentstroke}{rgb}{0.000000,0.000000,0.000000}%
\pgfsetstrokecolor{currentstroke}%
\pgfsetdash{}{0pt}%
\pgfpathmoveto{\pgfqpoint{4.143074in}{3.543690in}}%
\pgfpathlineto{\pgfqpoint{4.153352in}{3.523672in}}%
\pgfpathlineto{\pgfqpoint{4.163648in}{3.504873in}}%
\pgfpathlineto{\pgfqpoint{4.196705in}{3.489856in}}%
\pgfpathlineto{\pgfqpoint{4.229732in}{3.475492in}}%
\pgfpathlineto{\pgfqpoint{4.219388in}{3.493501in}}%
\pgfpathlineto{\pgfqpoint{4.209062in}{3.512545in}}%
\pgfpathlineto{\pgfqpoint{4.176083in}{3.527727in}}%
\pgfpathlineto{\pgfqpoint{4.143074in}{3.543690in}}%
\pgfpathclose%
\pgfusepath{fill}%
\end{pgfscope}%
\begin{pgfscope}%
\pgfpathrectangle{\pgfqpoint{1.020000in}{0.880000in}}{\pgfqpoint{6.160000in}{6.160000in}}%
\pgfusepath{clip}%
\pgfsetbuttcap%
\pgfsetroundjoin%
\definecolor{currentfill}{rgb}{0.378598,0.503856,0.913692}%
\pgfsetfillcolor{currentfill}%
\pgfsetlinewidth{0.000000pt}%
\definecolor{currentstroke}{rgb}{0.000000,0.000000,0.000000}%
\pgfsetstrokecolor{currentstroke}%
\pgfsetdash{}{0pt}%
\pgfpathmoveto{\pgfqpoint{4.753541in}{3.194711in}}%
\pgfpathlineto{\pgfqpoint{4.764422in}{3.189407in}}%
\pgfpathlineto{\pgfqpoint{4.775325in}{3.183967in}}%
\pgfpathlineto{\pgfqpoint{4.808101in}{3.168824in}}%
\pgfpathlineto{\pgfqpoint{4.840844in}{3.153494in}}%
\pgfpathlineto{\pgfqpoint{4.829866in}{3.155590in}}%
\pgfpathlineto{\pgfqpoint{4.818909in}{3.157359in}}%
\pgfpathlineto{\pgfqpoint{4.786242in}{3.176280in}}%
\pgfpathlineto{\pgfqpoint{4.753541in}{3.194711in}}%
\pgfpathclose%
\pgfusepath{fill}%
\end{pgfscope}%
\begin{pgfscope}%
\pgfpathrectangle{\pgfqpoint{1.020000in}{0.880000in}}{\pgfqpoint{6.160000in}{6.160000in}}%
\pgfusepath{clip}%
\pgfsetbuttcap%
\pgfsetroundjoin%
\definecolor{currentfill}{rgb}{0.333490,0.446265,0.874452}%
\pgfsetfillcolor{currentfill}%
\pgfsetlinewidth{0.000000pt}%
\definecolor{currentstroke}{rgb}{0.000000,0.000000,0.000000}%
\pgfsetstrokecolor{currentstroke}%
\pgfsetdash{}{0pt}%
\pgfpathmoveto{\pgfqpoint{5.124615in}{3.077989in}}%
\pgfpathlineto{\pgfqpoint{5.135946in}{3.081684in}}%
\pgfpathlineto{\pgfqpoint{5.147279in}{3.082880in}}%
\pgfpathlineto{\pgfqpoint{5.180008in}{3.081590in}}%
\pgfpathlineto{\pgfqpoint{5.212728in}{3.081719in}}%
\pgfpathlineto{\pgfqpoint{5.201335in}{3.080617in}}%
\pgfpathlineto{\pgfqpoint{5.189941in}{3.077008in}}%
\pgfpathlineto{\pgfqpoint{5.157280in}{3.076494in}}%
\pgfpathlineto{\pgfqpoint{5.124615in}{3.077989in}}%
\pgfpathclose%
\pgfusepath{fill}%
\end{pgfscope}%
\begin{pgfscope}%
\pgfpathrectangle{\pgfqpoint{1.020000in}{0.880000in}}{\pgfqpoint{6.160000in}{6.160000in}}%
\pgfusepath{clip}%
\pgfsetbuttcap%
\pgfsetroundjoin%
\definecolor{currentfill}{rgb}{0.835027,0.313644,0.259783}%
\pgfsetfillcolor{currentfill}%
\pgfsetlinewidth{0.000000pt}%
\definecolor{currentstroke}{rgb}{0.000000,0.000000,0.000000}%
\pgfsetstrokecolor{currentstroke}%
\pgfsetdash{}{0pt}%
\pgfpathmoveto{\pgfqpoint{2.555811in}{4.996908in}}%
\pgfpathlineto{\pgfqpoint{2.564508in}{4.992728in}}%
\pgfpathlineto{\pgfqpoint{2.573283in}{4.984379in}}%
\pgfpathlineto{\pgfqpoint{2.606220in}{5.027372in}}%
\pgfpathlineto{\pgfqpoint{2.639191in}{5.068275in}}%
\pgfpathlineto{\pgfqpoint{2.630368in}{5.075369in}}%
\pgfpathlineto{\pgfqpoint{2.621627in}{5.077925in}}%
\pgfpathlineto{\pgfqpoint{2.588702in}{5.038433in}}%
\pgfpathlineto{\pgfqpoint{2.555811in}{4.996908in}}%
\pgfpathclose%
\pgfusepath{fill}%
\end{pgfscope}%
\begin{pgfscope}%
\pgfpathrectangle{\pgfqpoint{1.020000in}{0.880000in}}{\pgfqpoint{6.160000in}{6.160000in}}%
\pgfusepath{clip}%
\pgfsetbuttcap%
\pgfsetroundjoin%
\definecolor{currentfill}{rgb}{0.758112,0.168122,0.188827}%
\pgfsetfillcolor{currentfill}%
\pgfsetlinewidth{0.000000pt}%
\definecolor{currentstroke}{rgb}{0.000000,0.000000,0.000000}%
\pgfsetstrokecolor{currentstroke}%
\pgfsetdash{}{0pt}%
\pgfpathmoveto{\pgfqpoint{2.687595in}{5.148897in}}%
\pgfpathlineto{\pgfqpoint{2.696384in}{5.147805in}}%
\pgfpathlineto{\pgfqpoint{2.705256in}{5.141846in}}%
\pgfpathlineto{\pgfqpoint{2.738357in}{5.173572in}}%
\pgfpathlineto{\pgfqpoint{2.771507in}{5.201342in}}%
\pgfpathlineto{\pgfqpoint{2.762584in}{5.206326in}}%
\pgfpathlineto{\pgfqpoint{2.753744in}{5.206174in}}%
\pgfpathlineto{\pgfqpoint{2.720646in}{5.179460in}}%
\pgfpathlineto{\pgfqpoint{2.687595in}{5.148897in}}%
\pgfpathclose%
\pgfusepath{fill}%
\end{pgfscope}%
\begin{pgfscope}%
\pgfpathrectangle{\pgfqpoint{1.020000in}{0.880000in}}{\pgfqpoint{6.160000in}{6.160000in}}%
\pgfusepath{clip}%
\pgfsetbuttcap%
\pgfsetroundjoin%
\definecolor{currentfill}{rgb}{0.795938,0.241845,0.220830}%
\pgfsetfillcolor{currentfill}%
\pgfsetlinewidth{0.000000pt}%
\definecolor{currentstroke}{rgb}{0.000000,0.000000,0.000000}%
\pgfsetstrokecolor{currentstroke}%
\pgfsetdash{}{0pt}%
\pgfpathmoveto{\pgfqpoint{2.621627in}{5.077925in}}%
\pgfpathlineto{\pgfqpoint{2.630368in}{5.075369in}}%
\pgfpathlineto{\pgfqpoint{2.639191in}{5.068275in}}%
\pgfpathlineto{\pgfqpoint{2.672202in}{5.106594in}}%
\pgfpathlineto{\pgfqpoint{2.705256in}{5.141846in}}%
\pgfpathlineto{\pgfqpoint{2.696384in}{5.147805in}}%
\pgfpathlineto{\pgfqpoint{2.687595in}{5.148897in}}%
\pgfpathlineto{\pgfqpoint{2.654590in}{5.114902in}}%
\pgfpathlineto{\pgfqpoint{2.621627in}{5.077925in}}%
\pgfpathclose%
\pgfusepath{fill}%
\end{pgfscope}%
\begin{pgfscope}%
\pgfpathrectangle{\pgfqpoint{1.020000in}{0.880000in}}{\pgfqpoint{6.160000in}{6.160000in}}%
\pgfusepath{clip}%
\pgfsetbuttcap%
\pgfsetroundjoin%
\definecolor{currentfill}{rgb}{0.912033,0.469680,0.366565}%
\pgfsetfillcolor{currentfill}%
\pgfsetlinewidth{0.000000pt}%
\definecolor{currentstroke}{rgb}{0.000000,0.000000,0.000000}%
\pgfsetstrokecolor{currentstroke}%
\pgfsetdash{}{0pt}%
\pgfpathmoveto{\pgfqpoint{3.296909in}{4.929481in}}%
\pgfpathlineto{\pgfqpoint{3.306628in}{4.890903in}}%
\pgfpathlineto{\pgfqpoint{3.316393in}{4.848147in}}%
\pgfpathlineto{\pgfqpoint{3.350053in}{4.806242in}}%
\pgfpathlineto{\pgfqpoint{3.383687in}{4.761498in}}%
\pgfpathlineto{\pgfqpoint{3.373909in}{4.802237in}}%
\pgfpathlineto{\pgfqpoint{3.364172in}{4.839114in}}%
\pgfpathlineto{\pgfqpoint{3.330553in}{4.885771in}}%
\pgfpathlineto{\pgfqpoint{3.296909in}{4.929481in}}%
\pgfpathclose%
\pgfusepath{fill}%
\end{pgfscope}%
\begin{pgfscope}%
\pgfpathrectangle{\pgfqpoint{1.020000in}{0.880000in}}{\pgfqpoint{6.160000in}{6.160000in}}%
\pgfusepath{clip}%
\pgfsetbuttcap%
\pgfsetroundjoin%
\definecolor{currentfill}{rgb}{0.848040,0.338280,0.275206}%
\pgfsetfillcolor{currentfill}%
\pgfsetlinewidth{0.000000pt}%
\definecolor{currentstroke}{rgb}{0.000000,0.000000,0.000000}%
\pgfsetstrokecolor{currentstroke}%
\pgfsetdash{}{0pt}%
\pgfpathmoveto{\pgfqpoint{3.210338in}{5.071878in}}%
\pgfpathlineto{\pgfqpoint{3.219925in}{5.041575in}}%
\pgfpathlineto{\pgfqpoint{3.229570in}{5.006276in}}%
\pgfpathlineto{\pgfqpoint{3.263246in}{4.969791in}}%
\pgfpathlineto{\pgfqpoint{3.296909in}{4.929481in}}%
\pgfpathlineto{\pgfqpoint{3.287240in}{4.963591in}}%
\pgfpathlineto{\pgfqpoint{3.277623in}{4.993001in}}%
\pgfpathlineto{\pgfqpoint{3.243987in}{5.034395in}}%
\pgfpathlineto{\pgfqpoint{3.210338in}{5.071878in}}%
\pgfpathclose%
\pgfusepath{fill}%
\end{pgfscope}%
\begin{pgfscope}%
\pgfpathrectangle{\pgfqpoint{1.020000in}{0.880000in}}{\pgfqpoint{6.160000in}{6.160000in}}%
\pgfusepath{clip}%
\pgfsetbuttcap%
\pgfsetroundjoin%
\definecolor{currentfill}{rgb}{0.348323,0.465711,0.888346}%
\pgfsetfillcolor{currentfill}%
\pgfsetlinewidth{0.000000pt}%
\definecolor{currentstroke}{rgb}{0.000000,0.000000,0.000000}%
\pgfsetstrokecolor{currentstroke}%
\pgfsetdash{}{0pt}%
\pgfpathmoveto{\pgfqpoint{4.906242in}{3.123344in}}%
\pgfpathlineto{\pgfqpoint{4.917319in}{3.124019in}}%
\pgfpathlineto{\pgfqpoint{4.928410in}{3.123401in}}%
\pgfpathlineto{\pgfqpoint{4.961152in}{3.112530in}}%
\pgfpathlineto{\pgfqpoint{4.993873in}{3.102638in}}%
\pgfpathlineto{\pgfqpoint{4.982701in}{3.100263in}}%
\pgfpathlineto{\pgfqpoint{4.971540in}{3.096260in}}%
\pgfpathlineto{\pgfqpoint{4.938902in}{3.109219in}}%
\pgfpathlineto{\pgfqpoint{4.906242in}{3.123344in}}%
\pgfpathclose%
\pgfusepath{fill}%
\end{pgfscope}%
\begin{pgfscope}%
\pgfpathrectangle{\pgfqpoint{1.020000in}{0.880000in}}{\pgfqpoint{6.160000in}{6.160000in}}%
\pgfusepath{clip}%
\pgfsetbuttcap%
\pgfsetroundjoin%
\definecolor{currentfill}{rgb}{0.711554,0.033337,0.154485}%
\pgfsetfillcolor{currentfill}%
\pgfsetlinewidth{0.000000pt}%
\definecolor{currentstroke}{rgb}{0.000000,0.000000,0.000000}%
\pgfsetstrokecolor{currentstroke}%
\pgfsetdash{}{0pt}%
\pgfpathmoveto{\pgfqpoint{2.971441in}{5.266561in}}%
\pgfpathlineto{\pgfqpoint{2.980612in}{5.257536in}}%
\pgfpathlineto{\pgfqpoint{2.989864in}{5.242752in}}%
\pgfpathlineto{\pgfqpoint{3.023389in}{5.235055in}}%
\pgfpathlineto{\pgfqpoint{3.056945in}{5.221828in}}%
\pgfpathlineto{\pgfqpoint{3.047644in}{5.236450in}}%
\pgfpathlineto{\pgfqpoint{3.038418in}{5.245373in}}%
\pgfpathlineto{\pgfqpoint{3.004914in}{5.258704in}}%
\pgfpathlineto{\pgfqpoint{2.971441in}{5.266561in}}%
\pgfpathclose%
\pgfusepath{fill}%
\end{pgfscope}%
\begin{pgfscope}%
\pgfpathrectangle{\pgfqpoint{1.020000in}{0.880000in}}{\pgfqpoint{6.160000in}{6.160000in}}%
\pgfusepath{clip}%
\pgfsetbuttcap%
\pgfsetroundjoin%
\definecolor{currentfill}{rgb}{0.640828,0.760752,0.997846}%
\pgfsetfillcolor{currentfill}%
\pgfsetlinewidth{0.000000pt}%
\definecolor{currentstroke}{rgb}{0.000000,0.000000,0.000000}%
\pgfsetstrokecolor{currentstroke}%
\pgfsetdash{}{0pt}%
\pgfpathmoveto{\pgfqpoint{3.990371in}{3.666412in}}%
\pgfpathlineto{\pgfqpoint{4.000533in}{3.640545in}}%
\pgfpathlineto{\pgfqpoint{4.010708in}{3.616102in}}%
\pgfpathlineto{\pgfqpoint{4.043850in}{3.596710in}}%
\pgfpathlineto{\pgfqpoint{4.076958in}{3.578166in}}%
\pgfpathlineto{\pgfqpoint{4.066741in}{3.600789in}}%
\pgfpathlineto{\pgfqpoint{4.056539in}{3.624636in}}%
\pgfpathlineto{\pgfqpoint{4.023473in}{3.645007in}}%
\pgfpathlineto{\pgfqpoint{3.990371in}{3.666412in}}%
\pgfpathclose%
\pgfusepath{fill}%
\end{pgfscope}%
\begin{pgfscope}%
\pgfpathrectangle{\pgfqpoint{1.020000in}{0.880000in}}{\pgfqpoint{6.160000in}{6.160000in}}%
\pgfusepath{clip}%
\pgfsetbuttcap%
\pgfsetroundjoin%
\definecolor{currentfill}{rgb}{0.338377,0.452819,0.879317}%
\pgfsetfillcolor{currentfill}%
\pgfsetlinewidth{0.000000pt}%
\definecolor{currentstroke}{rgb}{0.000000,0.000000,0.000000}%
\pgfsetstrokecolor{currentstroke}%
\pgfsetdash{}{0pt}%
\pgfpathmoveto{\pgfqpoint{5.343531in}{3.094507in}}%
\pgfpathlineto{\pgfqpoint{5.355016in}{3.090378in}}%
\pgfpathlineto{\pgfqpoint{5.366503in}{3.084273in}}%
\pgfpathlineto{\pgfqpoint{5.399217in}{3.087773in}}%
\pgfpathlineto{\pgfqpoint{5.431915in}{3.091615in}}%
\pgfpathlineto{\pgfqpoint{5.420392in}{3.099519in}}%
\pgfpathlineto{\pgfqpoint{5.408872in}{3.105806in}}%
\pgfpathlineto{\pgfqpoint{5.376208in}{3.099924in}}%
\pgfpathlineto{\pgfqpoint{5.343531in}{3.094507in}}%
\pgfpathclose%
\pgfusepath{fill}%
\end{pgfscope}%
\begin{pgfscope}%
\pgfpathrectangle{\pgfqpoint{1.020000in}{0.880000in}}{\pgfqpoint{6.160000in}{6.160000in}}%
\pgfusepath{clip}%
\pgfsetbuttcap%
\pgfsetroundjoin%
\definecolor{currentfill}{rgb}{0.333490,0.446265,0.874452}%
\pgfsetfillcolor{currentfill}%
\pgfsetlinewidth{0.000000pt}%
\definecolor{currentstroke}{rgb}{0.000000,0.000000,0.000000}%
\pgfsetstrokecolor{currentstroke}%
\pgfsetdash{}{0pt}%
\pgfpathmoveto{\pgfqpoint{5.781277in}{3.096557in}}%
\pgfpathlineto{\pgfqpoint{5.793041in}{3.081389in}}%
\pgfpathlineto{\pgfqpoint{5.804826in}{3.066115in}}%
\pgfpathlineto{\pgfqpoint{5.837387in}{3.067877in}}%
\pgfpathlineto{\pgfqpoint{5.869927in}{3.069603in}}%
\pgfpathlineto{\pgfqpoint{5.858093in}{3.084984in}}%
\pgfpathlineto{\pgfqpoint{5.846280in}{3.100319in}}%
\pgfpathlineto{\pgfqpoint{5.813789in}{3.098484in}}%
\pgfpathlineto{\pgfqpoint{5.781277in}{3.096557in}}%
\pgfpathclose%
\pgfusepath{fill}%
\end{pgfscope}%
\begin{pgfscope}%
\pgfpathrectangle{\pgfqpoint{1.020000in}{0.880000in}}{\pgfqpoint{6.160000in}{6.160000in}}%
\pgfusepath{clip}%
\pgfsetbuttcap%
\pgfsetroundjoin%
\definecolor{currentfill}{rgb}{0.343278,0.459354,0.884122}%
\pgfsetfillcolor{currentfill}%
\pgfsetlinewidth{0.000000pt}%
\definecolor{currentstroke}{rgb}{0.000000,0.000000,0.000000}%
\pgfsetstrokecolor{currentstroke}%
\pgfsetdash{}{0pt}%
\pgfpathmoveto{\pgfqpoint{5.562531in}{3.107806in}}%
\pgfpathlineto{\pgfqpoint{5.574133in}{3.095480in}}%
\pgfpathlineto{\pgfqpoint{5.585748in}{3.082437in}}%
\pgfpathlineto{\pgfqpoint{5.618390in}{3.085097in}}%
\pgfpathlineto{\pgfqpoint{5.651011in}{3.087649in}}%
\pgfpathlineto{\pgfqpoint{5.639355in}{3.101596in}}%
\pgfpathlineto{\pgfqpoint{5.627714in}{3.115069in}}%
\pgfpathlineto{\pgfqpoint{5.595134in}{3.111559in}}%
\pgfpathlineto{\pgfqpoint{5.562531in}{3.107806in}}%
\pgfpathclose%
\pgfusepath{fill}%
\end{pgfscope}%
\begin{pgfscope}%
\pgfpathrectangle{\pgfqpoint{1.020000in}{0.880000in}}{\pgfqpoint{6.160000in}{6.160000in}}%
\pgfusepath{clip}%
\pgfsetbuttcap%
\pgfsetroundjoin%
\definecolor{currentfill}{rgb}{0.855378,0.863778,0.876587}%
\pgfsetfillcolor{currentfill}%
\pgfsetlinewidth{0.000000pt}%
\definecolor{currentstroke}{rgb}{0.000000,0.000000,0.000000}%
\pgfsetstrokecolor{currentstroke}%
\pgfsetdash{}{0pt}%
\pgfpathmoveto{\pgfqpoint{3.664073in}{4.122587in}}%
\pgfpathlineto{\pgfqpoint{3.674071in}{4.077357in}}%
\pgfpathlineto{\pgfqpoint{3.684080in}{4.032609in}}%
\pgfpathlineto{\pgfqpoint{3.717456in}{3.998676in}}%
\pgfpathlineto{\pgfqpoint{3.750789in}{3.965056in}}%
\pgfpathlineto{\pgfqpoint{3.740768in}{4.005825in}}%
\pgfpathlineto{\pgfqpoint{3.730758in}{4.047063in}}%
\pgfpathlineto{\pgfqpoint{3.697438in}{4.084623in}}%
\pgfpathlineto{\pgfqpoint{3.664073in}{4.122587in}}%
\pgfpathclose%
\pgfusepath{fill}%
\end{pgfscope}%
\begin{pgfscope}%
\pgfpathrectangle{\pgfqpoint{1.020000in}{0.880000in}}{\pgfqpoint{6.160000in}{6.160000in}}%
\pgfusepath{clip}%
\pgfsetbuttcap%
\pgfsetroundjoin%
\definecolor{currentfill}{rgb}{0.786721,0.844807,0.939810}%
\pgfsetfillcolor{currentfill}%
\pgfsetlinewidth{0.000000pt}%
\definecolor{currentstroke}{rgb}{0.000000,0.000000,0.000000}%
\pgfsetstrokecolor{currentstroke}%
\pgfsetdash{}{0pt}%
\pgfpathmoveto{\pgfqpoint{3.750789in}{3.965056in}}%
\pgfpathlineto{\pgfqpoint{3.760820in}{3.925059in}}%
\pgfpathlineto{\pgfqpoint{3.770861in}{3.886128in}}%
\pgfpathlineto{\pgfqpoint{3.804168in}{3.856755in}}%
\pgfpathlineto{\pgfqpoint{3.837435in}{3.827981in}}%
\pgfpathlineto{\pgfqpoint{3.827373in}{3.863327in}}%
\pgfpathlineto{\pgfqpoint{3.817322in}{3.899642in}}%
\pgfpathlineto{\pgfqpoint{3.784078in}{3.931976in}}%
\pgfpathlineto{\pgfqpoint{3.750789in}{3.965056in}}%
\pgfpathclose%
\pgfusepath{fill}%
\end{pgfscope}%
\begin{pgfscope}%
\pgfpathrectangle{\pgfqpoint{1.020000in}{0.880000in}}{\pgfqpoint{6.160000in}{6.160000in}}%
\pgfusepath{clip}%
\pgfsetbuttcap%
\pgfsetroundjoin%
\definecolor{currentfill}{rgb}{0.922681,0.828568,0.777054}%
\pgfsetfillcolor{currentfill}%
\pgfsetlinewidth{0.000000pt}%
\definecolor{currentstroke}{rgb}{0.000000,0.000000,0.000000}%
\pgfsetstrokecolor{currentstroke}%
\pgfsetdash{}{0pt}%
\pgfpathmoveto{\pgfqpoint{3.577255in}{4.297159in}}%
\pgfpathlineto{\pgfqpoint{3.587224in}{4.248035in}}%
\pgfpathlineto{\pgfqpoint{3.597206in}{4.198601in}}%
\pgfpathlineto{\pgfqpoint{3.630661in}{4.160679in}}%
\pgfpathlineto{\pgfqpoint{3.664073in}{4.122587in}}%
\pgfpathlineto{\pgfqpoint{3.654085in}{4.167944in}}%
\pgfpathlineto{\pgfqpoint{3.644111in}{4.213068in}}%
\pgfpathlineto{\pgfqpoint{3.610707in}{4.255187in}}%
\pgfpathlineto{\pgfqpoint{3.577255in}{4.297159in}}%
\pgfpathclose%
\pgfusepath{fill}%
\end{pgfscope}%
\begin{pgfscope}%
\pgfpathrectangle{\pgfqpoint{1.020000in}{0.880000in}}{\pgfqpoint{6.160000in}{6.160000in}}%
\pgfusepath{clip}%
\pgfsetbuttcap%
\pgfsetroundjoin%
\definecolor{currentfill}{rgb}{0.338377,0.452819,0.879317}%
\pgfsetfillcolor{currentfill}%
\pgfsetlinewidth{0.000000pt}%
\definecolor{currentstroke}{rgb}{0.000000,0.000000,0.000000}%
\pgfsetstrokecolor{currentstroke}%
\pgfsetdash{}{0pt}%
\pgfpathmoveto{\pgfqpoint{5.059266in}{3.086873in}}%
\pgfpathlineto{\pgfqpoint{5.070525in}{3.089442in}}%
\pgfpathlineto{\pgfqpoint{5.081789in}{3.089687in}}%
\pgfpathlineto{\pgfqpoint{5.114540in}{3.085595in}}%
\pgfpathlineto{\pgfqpoint{5.147279in}{3.082880in}}%
\pgfpathlineto{\pgfqpoint{5.135946in}{3.081684in}}%
\pgfpathlineto{\pgfqpoint{5.124615in}{3.077989in}}%
\pgfpathlineto{\pgfqpoint{5.091944in}{3.081481in}}%
\pgfpathlineto{\pgfqpoint{5.059266in}{3.086873in}}%
\pgfpathclose%
\pgfusepath{fill}%
\end{pgfscope}%
\begin{pgfscope}%
\pgfpathrectangle{\pgfqpoint{1.020000in}{0.880000in}}{\pgfqpoint{6.160000in}{6.160000in}}%
\pgfusepath{clip}%
\pgfsetbuttcap%
\pgfsetroundjoin%
\definecolor{currentfill}{rgb}{0.724041,0.814910,0.975651}%
\pgfsetfillcolor{currentfill}%
\pgfsetlinewidth{0.000000pt}%
\definecolor{currentstroke}{rgb}{0.000000,0.000000,0.000000}%
\pgfsetstrokecolor{currentstroke}%
\pgfsetdash{}{0pt}%
\pgfpathmoveto{\pgfqpoint{3.837435in}{3.827981in}}%
\pgfpathlineto{\pgfqpoint{3.847507in}{3.793849in}}%
\pgfpathlineto{\pgfqpoint{3.857589in}{3.761161in}}%
\pgfpathlineto{\pgfqpoint{3.890842in}{3.736281in}}%
\pgfpathlineto{\pgfqpoint{3.924056in}{3.712126in}}%
\pgfpathlineto{\pgfqpoint{3.913944in}{3.741856in}}%
\pgfpathlineto{\pgfqpoint{3.903843in}{3.772865in}}%
\pgfpathlineto{\pgfqpoint{3.870659in}{3.799970in}}%
\pgfpathlineto{\pgfqpoint{3.837435in}{3.827981in}}%
\pgfpathclose%
\pgfusepath{fill}%
\end{pgfscope}%
\begin{pgfscope}%
\pgfpathrectangle{\pgfqpoint{1.020000in}{0.880000in}}{\pgfqpoint{6.160000in}{6.160000in}}%
\pgfusepath{clip}%
\pgfsetbuttcap%
\pgfsetroundjoin%
\definecolor{currentfill}{rgb}{0.543440,0.680003,0.993051}%
\pgfsetfillcolor{currentfill}%
\pgfsetlinewidth{0.000000pt}%
\definecolor{currentstroke}{rgb}{0.000000,0.000000,0.000000}%
\pgfsetstrokecolor{currentstroke}%
\pgfsetdash{}{0pt}%
\pgfpathmoveto{\pgfqpoint{4.229732in}{3.475492in}}%
\pgfpathlineto{\pgfqpoint{4.240094in}{3.458576in}}%
\pgfpathlineto{\pgfqpoint{4.250476in}{3.442795in}}%
\pgfpathlineto{\pgfqpoint{4.283523in}{3.429378in}}%
\pgfpathlineto{\pgfqpoint{4.316540in}{3.416373in}}%
\pgfpathlineto{\pgfqpoint{4.306108in}{3.431889in}}%
\pgfpathlineto{\pgfqpoint{4.295695in}{3.448374in}}%
\pgfpathlineto{\pgfqpoint{4.262728in}{3.461703in}}%
\pgfpathlineto{\pgfqpoint{4.229732in}{3.475492in}}%
\pgfpathclose%
\pgfusepath{fill}%
\end{pgfscope}%
\begin{pgfscope}%
\pgfpathrectangle{\pgfqpoint{1.020000in}{0.880000in}}{\pgfqpoint{6.160000in}{6.160000in}}%
\pgfusepath{clip}%
\pgfsetbuttcap%
\pgfsetroundjoin%
\definecolor{currentfill}{rgb}{0.494638,0.633022,0.978983}%
\pgfsetfillcolor{currentfill}%
\pgfsetlinewidth{0.000000pt}%
\definecolor{currentstroke}{rgb}{0.000000,0.000000,0.000000}%
\pgfsetstrokecolor{currentstroke}%
\pgfsetdash{}{0pt}%
\pgfpathmoveto{\pgfqpoint{4.382487in}{3.391056in}}%
\pgfpathlineto{\pgfqpoint{4.392990in}{3.376646in}}%
\pgfpathlineto{\pgfqpoint{4.403514in}{3.363091in}}%
\pgfpathlineto{\pgfqpoint{4.436495in}{3.350595in}}%
\pgfpathlineto{\pgfqpoint{4.469446in}{3.337935in}}%
\pgfpathlineto{\pgfqpoint{4.458870in}{3.351286in}}%
\pgfpathlineto{\pgfqpoint{4.448316in}{3.365403in}}%
\pgfpathlineto{\pgfqpoint{4.415416in}{3.378383in}}%
\pgfpathlineto{\pgfqpoint{4.382487in}{3.391056in}}%
\pgfpathclose%
\pgfusepath{fill}%
\end{pgfscope}%
\begin{pgfscope}%
\pgfpathrectangle{\pgfqpoint{1.020000in}{0.880000in}}{\pgfqpoint{6.160000in}{6.160000in}}%
\pgfusepath{clip}%
\pgfsetbuttcap%
\pgfsetroundjoin%
\definecolor{currentfill}{rgb}{0.962708,0.753557,0.655601}%
\pgfsetfillcolor{currentfill}%
\pgfsetlinewidth{0.000000pt}%
\definecolor{currentstroke}{rgb}{0.000000,0.000000,0.000000}%
\pgfsetstrokecolor{currentstroke}%
\pgfsetdash{}{0pt}%
\pgfpathmoveto{\pgfqpoint{3.490337in}{4.482506in}}%
\pgfpathlineto{\pgfqpoint{3.500270in}{4.431521in}}%
\pgfpathlineto{\pgfqpoint{3.510222in}{4.379259in}}%
\pgfpathlineto{\pgfqpoint{3.543760in}{4.338637in}}%
\pgfpathlineto{\pgfqpoint{3.577255in}{4.297159in}}%
\pgfpathlineto{\pgfqpoint{3.567302in}{4.345570in}}%
\pgfpathlineto{\pgfqpoint{3.557366in}{4.392866in}}%
\pgfpathlineto{\pgfqpoint{3.523873in}{4.438139in}}%
\pgfpathlineto{\pgfqpoint{3.490337in}{4.482506in}}%
\pgfpathclose%
\pgfusepath{fill}%
\end{pgfscope}%
\begin{pgfscope}%
\pgfpathrectangle{\pgfqpoint{1.020000in}{0.880000in}}{\pgfqpoint{6.160000in}{6.160000in}}%
\pgfusepath{clip}%
\pgfsetbuttcap%
\pgfsetroundjoin%
\definecolor{currentfill}{rgb}{0.338377,0.452819,0.879317}%
\pgfsetfillcolor{currentfill}%
\pgfsetlinewidth{0.000000pt}%
\definecolor{currentstroke}{rgb}{0.000000,0.000000,0.000000}%
\pgfsetstrokecolor{currentstroke}%
\pgfsetdash{}{0pt}%
\pgfpathmoveto{\pgfqpoint{5.278146in}{3.085946in}}%
\pgfpathlineto{\pgfqpoint{5.289590in}{3.083540in}}%
\pgfpathlineto{\pgfqpoint{5.301034in}{3.078876in}}%
\pgfpathlineto{\pgfqpoint{5.333775in}{3.081260in}}%
\pgfpathlineto{\pgfqpoint{5.366503in}{3.084273in}}%
\pgfpathlineto{\pgfqpoint{5.355016in}{3.090378in}}%
\pgfpathlineto{\pgfqpoint{5.343531in}{3.094507in}}%
\pgfpathlineto{\pgfqpoint{5.310843in}{3.089776in}}%
\pgfpathlineto{\pgfqpoint{5.278146in}{3.085946in}}%
\pgfpathclose%
\pgfusepath{fill}%
\end{pgfscope}%
\begin{pgfscope}%
\pgfpathrectangle{\pgfqpoint{1.020000in}{0.880000in}}{\pgfqpoint{6.160000in}{6.160000in}}%
\pgfusepath{clip}%
\pgfsetbuttcap%
\pgfsetroundjoin%
\definecolor{currentfill}{rgb}{0.451739,0.588181,0.960201}%
\pgfsetfillcolor{currentfill}%
\pgfsetlinewidth{0.000000pt}%
\definecolor{currentstroke}{rgb}{0.000000,0.000000,0.000000}%
\pgfsetstrokecolor{currentstroke}%
\pgfsetdash{}{0pt}%
\pgfpathmoveto{\pgfqpoint{4.535256in}{3.311254in}}%
\pgfpathlineto{\pgfqpoint{4.545907in}{3.299181in}}%
\pgfpathlineto{\pgfqpoint{4.556580in}{3.287677in}}%
\pgfpathlineto{\pgfqpoint{4.589492in}{3.274190in}}%
\pgfpathlineto{\pgfqpoint{4.622372in}{3.260038in}}%
\pgfpathlineto{\pgfqpoint{4.611642in}{3.270388in}}%
\pgfpathlineto{\pgfqpoint{4.600935in}{3.281248in}}%
\pgfpathlineto{\pgfqpoint{4.568113in}{3.296775in}}%
\pgfpathlineto{\pgfqpoint{4.535256in}{3.311254in}}%
\pgfpathclose%
\pgfusepath{fill}%
\end{pgfscope}%
\begin{pgfscope}%
\pgfpathrectangle{\pgfqpoint{1.020000in}{0.880000in}}{\pgfqpoint{6.160000in}{6.160000in}}%
\pgfusepath{clip}%
\pgfsetbuttcap%
\pgfsetroundjoin%
\definecolor{currentfill}{rgb}{0.597777,0.727330,0.999777}%
\pgfsetfillcolor{currentfill}%
\pgfsetlinewidth{0.000000pt}%
\definecolor{currentstroke}{rgb}{0.000000,0.000000,0.000000}%
\pgfsetstrokecolor{currentstroke}%
\pgfsetdash{}{0pt}%
\pgfpathmoveto{\pgfqpoint{4.076958in}{3.578166in}}%
\pgfpathlineto{\pgfqpoint{4.087191in}{3.556878in}}%
\pgfpathlineto{\pgfqpoint{4.097440in}{3.537023in}}%
\pgfpathlineto{\pgfqpoint{4.130560in}{3.520589in}}%
\pgfpathlineto{\pgfqpoint{4.163648in}{3.504873in}}%
\pgfpathlineto{\pgfqpoint{4.153352in}{3.523672in}}%
\pgfpathlineto{\pgfqpoint{4.143074in}{3.543690in}}%
\pgfpathlineto{\pgfqpoint{4.110032in}{3.560493in}}%
\pgfpathlineto{\pgfqpoint{4.076958in}{3.578166in}}%
\pgfpathclose%
\pgfusepath{fill}%
\end{pgfscope}%
\begin{pgfscope}%
\pgfpathrectangle{\pgfqpoint{1.020000in}{0.880000in}}{\pgfqpoint{6.160000in}{6.160000in}}%
\pgfusepath{clip}%
\pgfsetbuttcap%
\pgfsetroundjoin%
\definecolor{currentfill}{rgb}{0.404421,0.534643,0.932002}%
\pgfsetfillcolor{currentfill}%
\pgfsetlinewidth{0.000000pt}%
\definecolor{currentstroke}{rgb}{0.000000,0.000000,0.000000}%
\pgfsetstrokecolor{currentstroke}%
\pgfsetdash{}{0pt}%
\pgfpathmoveto{\pgfqpoint{4.688028in}{3.229208in}}%
\pgfpathlineto{\pgfqpoint{4.698841in}{3.221114in}}%
\pgfpathlineto{\pgfqpoint{4.709676in}{3.213070in}}%
\pgfpathlineto{\pgfqpoint{4.742517in}{3.198748in}}%
\pgfpathlineto{\pgfqpoint{4.775325in}{3.183967in}}%
\pgfpathlineto{\pgfqpoint{4.764422in}{3.189407in}}%
\pgfpathlineto{\pgfqpoint{4.753541in}{3.194711in}}%
\pgfpathlineto{\pgfqpoint{4.720802in}{3.212403in}}%
\pgfpathlineto{\pgfqpoint{4.688028in}{3.229208in}}%
\pgfpathclose%
\pgfusepath{fill}%
\end{pgfscope}%
\begin{pgfscope}%
\pgfpathrectangle{\pgfqpoint{1.020000in}{0.880000in}}{\pgfqpoint{6.160000in}{6.160000in}}%
\pgfusepath{clip}%
\pgfsetbuttcap%
\pgfsetroundjoin%
\definecolor{currentfill}{rgb}{0.705673,0.015556,0.150233}%
\pgfsetfillcolor{currentfill}%
\pgfsetlinewidth{0.000000pt}%
\definecolor{currentstroke}{rgb}{0.000000,0.000000,0.000000}%
\pgfsetstrokecolor{currentstroke}%
\pgfsetdash{}{0pt}%
\pgfpathmoveto{\pgfqpoint{2.904611in}{5.265763in}}%
\pgfpathlineto{\pgfqpoint{2.913727in}{5.256474in}}%
\pgfpathlineto{\pgfqpoint{2.922927in}{5.241459in}}%
\pgfpathlineto{\pgfqpoint{2.956375in}{5.244877in}}%
\pgfpathlineto{\pgfqpoint{2.989864in}{5.242752in}}%
\pgfpathlineto{\pgfqpoint{2.980612in}{5.257536in}}%
\pgfpathlineto{\pgfqpoint{2.971441in}{5.266561in}}%
\pgfpathlineto{\pgfqpoint{2.938005in}{5.268904in}}%
\pgfpathlineto{\pgfqpoint{2.904611in}{5.265763in}}%
\pgfpathclose%
\pgfusepath{fill}%
\end{pgfscope}%
\begin{pgfscope}%
\pgfpathrectangle{\pgfqpoint{1.020000in}{0.880000in}}{\pgfqpoint{6.160000in}{6.160000in}}%
\pgfusepath{clip}%
\pgfsetbuttcap%
\pgfsetroundjoin%
\definecolor{currentfill}{rgb}{0.752704,0.157576,0.184258}%
\pgfsetfillcolor{currentfill}%
\pgfsetlinewidth{0.000000pt}%
\definecolor{currentstroke}{rgb}{0.000000,0.000000,0.000000}%
\pgfsetstrokecolor{currentstroke}%
\pgfsetdash{}{0pt}%
\pgfpathmoveto{\pgfqpoint{3.056945in}{5.221828in}}%
\pgfpathlineto{\pgfqpoint{3.066322in}{5.201479in}}%
\pgfpathlineto{\pgfqpoint{3.075771in}{5.175449in}}%
\pgfpathlineto{\pgfqpoint{3.109394in}{5.157109in}}%
\pgfpathlineto{\pgfqpoint{3.143035in}{5.133562in}}%
\pgfpathlineto{\pgfqpoint{3.133546in}{5.159192in}}%
\pgfpathlineto{\pgfqpoint{3.124125in}{5.179295in}}%
\pgfpathlineto{\pgfqpoint{3.090526in}{5.203181in}}%
\pgfpathlineto{\pgfqpoint{3.056945in}{5.221828in}}%
\pgfpathclose%
\pgfusepath{fill}%
\end{pgfscope}%
\begin{pgfscope}%
\pgfpathrectangle{\pgfqpoint{1.020000in}{0.880000in}}{\pgfqpoint{6.160000in}{6.160000in}}%
\pgfusepath{clip}%
\pgfsetbuttcap%
\pgfsetroundjoin%
\definecolor{currentfill}{rgb}{0.967317,0.657471,0.538160}%
\pgfsetfillcolor{currentfill}%
\pgfsetlinewidth{0.000000pt}%
\definecolor{currentstroke}{rgb}{0.000000,0.000000,0.000000}%
\pgfsetstrokecolor{currentstroke}%
\pgfsetdash{}{0pt}%
\pgfpathmoveto{\pgfqpoint{2.244422in}{4.541402in}}%
\pgfpathlineto{\pgfqpoint{2.253043in}{4.524202in}}%
\pgfpathlineto{\pgfqpoint{2.261720in}{4.504701in}}%
\pgfpathlineto{\pgfqpoint{2.294713in}{4.544803in}}%
\pgfpathlineto{\pgfqpoint{2.327676in}{4.586848in}}%
\pgfpathlineto{\pgfqpoint{2.318949in}{4.605871in}}%
\pgfpathlineto{\pgfqpoint{2.310282in}{4.622209in}}%
\pgfpathlineto{\pgfqpoint{2.277367in}{4.580843in}}%
\pgfpathlineto{\pgfqpoint{2.244422in}{4.541402in}}%
\pgfpathclose%
\pgfusepath{fill}%
\end{pgfscope}%
\begin{pgfscope}%
\pgfpathrectangle{\pgfqpoint{1.020000in}{0.880000in}}{\pgfqpoint{6.160000in}{6.160000in}}%
\pgfusepath{clip}%
\pgfsetbuttcap%
\pgfsetroundjoin%
\definecolor{currentfill}{rgb}{0.343278,0.459354,0.884122}%
\pgfsetfillcolor{currentfill}%
\pgfsetlinewidth{0.000000pt}%
\definecolor{currentstroke}{rgb}{0.000000,0.000000,0.000000}%
\pgfsetstrokecolor{currentstroke}%
\pgfsetdash{}{0pt}%
\pgfpathmoveto{\pgfqpoint{5.497262in}{3.099777in}}%
\pgfpathlineto{\pgfqpoint{5.508828in}{3.088853in}}%
\pgfpathlineto{\pgfqpoint{5.520403in}{3.076920in}}%
\pgfpathlineto{\pgfqpoint{5.553086in}{3.079696in}}%
\pgfpathlineto{\pgfqpoint{5.585748in}{3.082437in}}%
\pgfpathlineto{\pgfqpoint{5.574133in}{3.095480in}}%
\pgfpathlineto{\pgfqpoint{5.562531in}{3.107806in}}%
\pgfpathlineto{\pgfqpoint{5.529907in}{3.103856in}}%
\pgfpathlineto{\pgfqpoint{5.497262in}{3.099777in}}%
\pgfpathclose%
\pgfusepath{fill}%
\end{pgfscope}%
\begin{pgfscope}%
\pgfpathrectangle{\pgfqpoint{1.020000in}{0.880000in}}{\pgfqpoint{6.160000in}{6.160000in}}%
\pgfusepath{clip}%
\pgfsetbuttcap%
\pgfsetroundjoin%
\definecolor{currentfill}{rgb}{0.338377,0.452819,0.879317}%
\pgfsetfillcolor{currentfill}%
\pgfsetlinewidth{0.000000pt}%
\definecolor{currentstroke}{rgb}{0.000000,0.000000,0.000000}%
\pgfsetstrokecolor{currentstroke}%
\pgfsetdash{}{0pt}%
\pgfpathmoveto{\pgfqpoint{5.716188in}{3.092362in}}%
\pgfpathlineto{\pgfqpoint{5.727905in}{3.077511in}}%
\pgfpathlineto{\pgfqpoint{5.739641in}{3.062460in}}%
\pgfpathlineto{\pgfqpoint{5.772244in}{3.064311in}}%
\pgfpathlineto{\pgfqpoint{5.804826in}{3.066115in}}%
\pgfpathlineto{\pgfqpoint{5.793041in}{3.081389in}}%
\pgfpathlineto{\pgfqpoint{5.781277in}{3.096557in}}%
\pgfpathlineto{\pgfqpoint{5.748743in}{3.094521in}}%
\pgfpathlineto{\pgfqpoint{5.716188in}{3.092362in}}%
\pgfpathclose%
\pgfusepath{fill}%
\end{pgfscope}%
\begin{pgfscope}%
\pgfpathrectangle{\pgfqpoint{1.020000in}{0.880000in}}{\pgfqpoint{6.160000in}{6.160000in}}%
\pgfusepath{clip}%
\pgfsetbuttcap%
\pgfsetroundjoin%
\definecolor{currentfill}{rgb}{0.368507,0.491141,0.905243}%
\pgfsetfillcolor{currentfill}%
\pgfsetlinewidth{0.000000pt}%
\definecolor{currentstroke}{rgb}{0.000000,0.000000,0.000000}%
\pgfsetstrokecolor{currentstroke}%
\pgfsetdash{}{0pt}%
\pgfpathmoveto{\pgfqpoint{4.840844in}{3.153494in}}%
\pgfpathlineto{\pgfqpoint{4.851841in}{3.150726in}}%
\pgfpathlineto{\pgfqpoint{4.862854in}{3.147029in}}%
\pgfpathlineto{\pgfqpoint{4.895645in}{3.134983in}}%
\pgfpathlineto{\pgfqpoint{4.928410in}{3.123401in}}%
\pgfpathlineto{\pgfqpoint{4.917319in}{3.124019in}}%
\pgfpathlineto{\pgfqpoint{4.906242in}{3.123344in}}%
\pgfpathlineto{\pgfqpoint{4.873557in}{3.138227in}}%
\pgfpathlineto{\pgfqpoint{4.840844in}{3.153494in}}%
\pgfpathclose%
\pgfusepath{fill}%
\end{pgfscope}%
\begin{pgfscope}%
\pgfpathrectangle{\pgfqpoint{1.020000in}{0.880000in}}{\pgfqpoint{6.160000in}{6.160000in}}%
\pgfusepath{clip}%
\pgfsetbuttcap%
\pgfsetroundjoin%
\definecolor{currentfill}{rgb}{0.966922,0.651969,0.531997}%
\pgfsetfillcolor{currentfill}%
\pgfsetlinewidth{0.000000pt}%
\definecolor{currentstroke}{rgb}{0.000000,0.000000,0.000000}%
\pgfsetstrokecolor{currentstroke}%
\pgfsetdash{}{0pt}%
\pgfpathmoveto{\pgfqpoint{3.403355in}{4.669752in}}%
\pgfpathlineto{\pgfqpoint{3.413238in}{4.619504in}}%
\pgfpathlineto{\pgfqpoint{3.423148in}{4.566901in}}%
\pgfpathlineto{\pgfqpoint{3.456760in}{4.525563in}}%
\pgfpathlineto{\pgfqpoint{3.490337in}{4.482506in}}%
\pgfpathlineto{\pgfqpoint{3.480425in}{4.531783in}}%
\pgfpathlineto{\pgfqpoint{3.470539in}{4.578936in}}%
\pgfpathlineto{\pgfqpoint{3.436966in}{4.625258in}}%
\pgfpathlineto{\pgfqpoint{3.403355in}{4.669752in}}%
\pgfpathclose%
\pgfusepath{fill}%
\end{pgfscope}%
\begin{pgfscope}%
\pgfpathrectangle{\pgfqpoint{1.020000in}{0.880000in}}{\pgfqpoint{6.160000in}{6.160000in}}%
\pgfusepath{clip}%
\pgfsetbuttcap%
\pgfsetroundjoin%
\definecolor{currentfill}{rgb}{0.958279,0.604335,0.483297}%
\pgfsetfillcolor{currentfill}%
\pgfsetlinewidth{0.000000pt}%
\definecolor{currentstroke}{rgb}{0.000000,0.000000,0.000000}%
\pgfsetstrokecolor{currentstroke}%
\pgfsetdash{}{0pt}%
\pgfpathmoveto{\pgfqpoint{2.310282in}{4.622209in}}%
\pgfpathlineto{\pgfqpoint{2.318949in}{4.605871in}}%
\pgfpathlineto{\pgfqpoint{2.327676in}{4.586848in}}%
\pgfpathlineto{\pgfqpoint{2.360614in}{4.630587in}}%
\pgfpathlineto{\pgfqpoint{2.393535in}{4.675726in}}%
\pgfpathlineto{\pgfqpoint{2.384757in}{4.694254in}}%
\pgfpathlineto{\pgfqpoint{2.376045in}{4.709687in}}%
\pgfpathlineto{\pgfqpoint{2.343173in}{4.665254in}}%
\pgfpathlineto{\pgfqpoint{2.310282in}{4.622209in}}%
\pgfpathclose%
\pgfusepath{fill}%
\end{pgfscope}%
\begin{pgfscope}%
\pgfpathrectangle{\pgfqpoint{1.020000in}{0.880000in}}{\pgfqpoint{6.160000in}{6.160000in}}%
\pgfusepath{clip}%
\pgfsetbuttcap%
\pgfsetroundjoin%
\definecolor{currentfill}{rgb}{0.328604,0.439712,0.869587}%
\pgfsetfillcolor{currentfill}%
\pgfsetlinewidth{0.000000pt}%
\definecolor{currentstroke}{rgb}{0.000000,0.000000,0.000000}%
\pgfsetstrokecolor{currentstroke}%
\pgfsetdash{}{0pt}%
\pgfpathmoveto{\pgfqpoint{5.934945in}{3.072979in}}%
\pgfpathlineto{\pgfqpoint{5.946852in}{3.057574in}}%
\pgfpathlineto{\pgfqpoint{5.958781in}{3.042175in}}%
\pgfpathlineto{\pgfqpoint{5.991312in}{3.043884in}}%
\pgfpathlineto{\pgfqpoint{5.979357in}{3.059256in}}%
\pgfpathlineto{\pgfqpoint{5.967424in}{3.074642in}}%
\pgfpathlineto{\pgfqpoint{5.934945in}{3.072979in}}%
\pgfpathclose%
\pgfusepath{fill}%
\end{pgfscope}%
\begin{pgfscope}%
\pgfpathrectangle{\pgfqpoint{1.020000in}{0.880000in}}{\pgfqpoint{6.160000in}{6.160000in}}%
\pgfusepath{clip}%
\pgfsetbuttcap%
\pgfsetroundjoin%
\definecolor{currentfill}{rgb}{0.667253,0.779176,0.992959}%
\pgfsetfillcolor{currentfill}%
\pgfsetlinewidth{0.000000pt}%
\definecolor{currentstroke}{rgb}{0.000000,0.000000,0.000000}%
\pgfsetstrokecolor{currentstroke}%
\pgfsetdash{}{0pt}%
\pgfpathmoveto{\pgfqpoint{3.924056in}{3.712126in}}%
\pgfpathlineto{\pgfqpoint{3.934181in}{3.683862in}}%
\pgfpathlineto{\pgfqpoint{3.944318in}{3.657230in}}%
\pgfpathlineto{\pgfqpoint{3.977531in}{3.636298in}}%
\pgfpathlineto{\pgfqpoint{4.010708in}{3.616102in}}%
\pgfpathlineto{\pgfqpoint{4.000533in}{3.640545in}}%
\pgfpathlineto{\pgfqpoint{3.990371in}{3.666412in}}%
\pgfpathlineto{\pgfqpoint{3.957233in}{3.688807in}}%
\pgfpathlineto{\pgfqpoint{3.924056in}{3.712126in}}%
\pgfpathclose%
\pgfusepath{fill}%
\end{pgfscope}%
\begin{pgfscope}%
\pgfpathrectangle{\pgfqpoint{1.020000in}{0.880000in}}{\pgfqpoint{6.160000in}{6.160000in}}%
\pgfusepath{clip}%
\pgfsetbuttcap%
\pgfsetroundjoin%
\definecolor{currentfill}{rgb}{0.939254,0.539581,0.423900}%
\pgfsetfillcolor{currentfill}%
\pgfsetlinewidth{0.000000pt}%
\definecolor{currentstroke}{rgb}{0.000000,0.000000,0.000000}%
\pgfsetstrokecolor{currentstroke}%
\pgfsetdash{}{0pt}%
\pgfpathmoveto{\pgfqpoint{2.376045in}{4.709687in}}%
\pgfpathlineto{\pgfqpoint{2.384757in}{4.694254in}}%
\pgfpathlineto{\pgfqpoint{2.393535in}{4.675726in}}%
\pgfpathlineto{\pgfqpoint{2.426446in}{4.721932in}}%
\pgfpathlineto{\pgfqpoint{2.459352in}{4.768829in}}%
\pgfpathlineto{\pgfqpoint{2.450522in}{4.786851in}}%
\pgfpathlineto{\pgfqpoint{2.441763in}{4.801351in}}%
\pgfpathlineto{\pgfqpoint{2.408906in}{4.755176in}}%
\pgfpathlineto{\pgfqpoint{2.376045in}{4.709687in}}%
\pgfpathclose%
\pgfusepath{fill}%
\end{pgfscope}%
\begin{pgfscope}%
\pgfpathrectangle{\pgfqpoint{1.020000in}{0.880000in}}{\pgfqpoint{6.160000in}{6.160000in}}%
\pgfusepath{clip}%
\pgfsetbuttcap%
\pgfsetroundjoin%
\definecolor{currentfill}{rgb}{0.934305,0.525918,0.412286}%
\pgfsetfillcolor{currentfill}%
\pgfsetlinewidth{0.000000pt}%
\definecolor{currentstroke}{rgb}{0.000000,0.000000,0.000000}%
\pgfsetstrokecolor{currentstroke}%
\pgfsetdash{}{0pt}%
\pgfpathmoveto{\pgfqpoint{3.316393in}{4.848147in}}%
\pgfpathlineto{\pgfqpoint{3.326200in}{4.801554in}}%
\pgfpathlineto{\pgfqpoint{3.336045in}{4.751510in}}%
\pgfpathlineto{\pgfqpoint{3.369714in}{4.711978in}}%
\pgfpathlineto{\pgfqpoint{3.403355in}{4.669752in}}%
\pgfpathlineto{\pgfqpoint{3.393504in}{4.717216in}}%
\pgfpathlineto{\pgfqpoint{3.383687in}{4.761498in}}%
\pgfpathlineto{\pgfqpoint{3.350053in}{4.806242in}}%
\pgfpathlineto{\pgfqpoint{3.316393in}{4.848147in}}%
\pgfpathclose%
\pgfusepath{fill}%
\end{pgfscope}%
\begin{pgfscope}%
\pgfpathrectangle{\pgfqpoint{1.020000in}{0.880000in}}{\pgfqpoint{6.160000in}{6.160000in}}%
\pgfusepath{clip}%
\pgfsetbuttcap%
\pgfsetroundjoin%
\definecolor{currentfill}{rgb}{0.810616,0.268797,0.235428}%
\pgfsetfillcolor{currentfill}%
\pgfsetlinewidth{0.000000pt}%
\definecolor{currentstroke}{rgb}{0.000000,0.000000,0.000000}%
\pgfsetstrokecolor{currentstroke}%
\pgfsetdash{}{0pt}%
\pgfpathmoveto{\pgfqpoint{3.143035in}{5.133562in}}%
\pgfpathlineto{\pgfqpoint{3.152590in}{5.102522in}}%
\pgfpathlineto{\pgfqpoint{3.162207in}{5.066259in}}%
\pgfpathlineto{\pgfqpoint{3.195888in}{5.038547in}}%
\pgfpathlineto{\pgfqpoint{3.229570in}{5.006276in}}%
\pgfpathlineto{\pgfqpoint{3.219925in}{5.041575in}}%
\pgfpathlineto{\pgfqpoint{3.210338in}{5.071878in}}%
\pgfpathlineto{\pgfqpoint{3.176685in}{5.105052in}}%
\pgfpathlineto{\pgfqpoint{3.143035in}{5.133562in}}%
\pgfpathclose%
\pgfusepath{fill}%
\end{pgfscope}%
\begin{pgfscope}%
\pgfpathrectangle{\pgfqpoint{1.020000in}{0.880000in}}{\pgfqpoint{6.160000in}{6.160000in}}%
\pgfusepath{clip}%
\pgfsetbuttcap%
\pgfsetroundjoin%
\definecolor{currentfill}{rgb}{0.338377,0.452819,0.879317}%
\pgfsetfillcolor{currentfill}%
\pgfsetlinewidth{0.000000pt}%
\definecolor{currentstroke}{rgb}{0.000000,0.000000,0.000000}%
\pgfsetstrokecolor{currentstroke}%
\pgfsetdash{}{0pt}%
\pgfpathmoveto{\pgfqpoint{5.212728in}{3.081719in}}%
\pgfpathlineto{\pgfqpoint{5.224122in}{3.080307in}}%
\pgfpathlineto{\pgfqpoint{5.235515in}{3.076484in}}%
\pgfpathlineto{\pgfqpoint{5.268280in}{3.077247in}}%
\pgfpathlineto{\pgfqpoint{5.301034in}{3.078876in}}%
\pgfpathlineto{\pgfqpoint{5.289590in}{3.083540in}}%
\pgfpathlineto{\pgfqpoint{5.278146in}{3.085946in}}%
\pgfpathlineto{\pgfqpoint{5.245441in}{3.083209in}}%
\pgfpathlineto{\pgfqpoint{5.212728in}{3.081719in}}%
\pgfpathclose%
\pgfusepath{fill}%
\end{pgfscope}%
\begin{pgfscope}%
\pgfpathrectangle{\pgfqpoint{1.020000in}{0.880000in}}{\pgfqpoint{6.160000in}{6.160000in}}%
\pgfusepath{clip}%
\pgfsetbuttcap%
\pgfsetroundjoin%
\definecolor{currentfill}{rgb}{0.348323,0.465711,0.888346}%
\pgfsetfillcolor{currentfill}%
\pgfsetlinewidth{0.000000pt}%
\definecolor{currentstroke}{rgb}{0.000000,0.000000,0.000000}%
\pgfsetstrokecolor{currentstroke}%
\pgfsetdash{}{0pt}%
\pgfpathmoveto{\pgfqpoint{4.993873in}{3.102638in}}%
\pgfpathlineto{\pgfqpoint{5.005055in}{3.103126in}}%
\pgfpathlineto{\pgfqpoint{5.016245in}{3.101601in}}%
\pgfpathlineto{\pgfqpoint{5.049025in}{3.095065in}}%
\pgfpathlineto{\pgfqpoint{5.081789in}{3.089687in}}%
\pgfpathlineto{\pgfqpoint{5.070525in}{3.089442in}}%
\pgfpathlineto{\pgfqpoint{5.059266in}{3.086873in}}%
\pgfpathlineto{\pgfqpoint{5.026577in}{3.093999in}}%
\pgfpathlineto{\pgfqpoint{4.993873in}{3.102638in}}%
\pgfpathclose%
\pgfusepath{fill}%
\end{pgfscope}%
\begin{pgfscope}%
\pgfpathrectangle{\pgfqpoint{1.020000in}{0.880000in}}{\pgfqpoint{6.160000in}{6.160000in}}%
\pgfusepath{clip}%
\pgfsetbuttcap%
\pgfsetroundjoin%
\definecolor{currentfill}{rgb}{0.705673,0.015556,0.150233}%
\pgfsetfillcolor{currentfill}%
\pgfsetlinewidth{0.000000pt}%
\definecolor{currentstroke}{rgb}{0.000000,0.000000,0.000000}%
\pgfsetstrokecolor{currentstroke}%
\pgfsetdash{}{0pt}%
\pgfpathmoveto{\pgfqpoint{2.837960in}{5.243493in}}%
\pgfpathlineto{\pgfqpoint{2.847022in}{5.233781in}}%
\pgfpathlineto{\pgfqpoint{2.856170in}{5.218470in}}%
\pgfpathlineto{\pgfqpoint{2.889525in}{5.232600in}}%
\pgfpathlineto{\pgfqpoint{2.922927in}{5.241459in}}%
\pgfpathlineto{\pgfqpoint{2.913727in}{5.256474in}}%
\pgfpathlineto{\pgfqpoint{2.904611in}{5.265763in}}%
\pgfpathlineto{\pgfqpoint{2.871261in}{5.257236in}}%
\pgfpathlineto{\pgfqpoint{2.837960in}{5.243493in}}%
\pgfpathclose%
\pgfusepath{fill}%
\end{pgfscope}%
\begin{pgfscope}%
\pgfpathrectangle{\pgfqpoint{1.020000in}{0.880000in}}{\pgfqpoint{6.160000in}{6.160000in}}%
\pgfusepath{clip}%
\pgfsetbuttcap%
\pgfsetroundjoin%
\definecolor{currentfill}{rgb}{0.877149,0.394645,0.311724}%
\pgfsetfillcolor{currentfill}%
\pgfsetlinewidth{0.000000pt}%
\definecolor{currentstroke}{rgb}{0.000000,0.000000,0.000000}%
\pgfsetstrokecolor{currentstroke}%
\pgfsetdash{}{0pt}%
\pgfpathmoveto{\pgfqpoint{3.229570in}{5.006276in}}%
\pgfpathlineto{\pgfqpoint{3.239269in}{4.966225in}}%
\pgfpathlineto{\pgfqpoint{3.249018in}{4.921729in}}%
\pgfpathlineto{\pgfqpoint{3.282712in}{4.886779in}}%
\pgfpathlineto{\pgfqpoint{3.316393in}{4.848147in}}%
\pgfpathlineto{\pgfqpoint{3.306628in}{4.890903in}}%
\pgfpathlineto{\pgfqpoint{3.296909in}{4.929481in}}%
\pgfpathlineto{\pgfqpoint{3.263246in}{4.969791in}}%
\pgfpathlineto{\pgfqpoint{3.229570in}{5.006276in}}%
\pgfpathclose%
\pgfusepath{fill}%
\end{pgfscope}%
\begin{pgfscope}%
\pgfpathrectangle{\pgfqpoint{1.020000in}{0.880000in}}{\pgfqpoint{6.160000in}{6.160000in}}%
\pgfusepath{clip}%
\pgfsetbuttcap%
\pgfsetroundjoin%
\definecolor{currentfill}{rgb}{0.912033,0.469680,0.366565}%
\pgfsetfillcolor{currentfill}%
\pgfsetlinewidth{0.000000pt}%
\definecolor{currentstroke}{rgb}{0.000000,0.000000,0.000000}%
\pgfsetstrokecolor{currentstroke}%
\pgfsetdash{}{0pt}%
\pgfpathmoveto{\pgfqpoint{2.441763in}{4.801351in}}%
\pgfpathlineto{\pgfqpoint{2.450522in}{4.786851in}}%
\pgfpathlineto{\pgfqpoint{2.459352in}{4.768829in}}%
\pgfpathlineto{\pgfqpoint{2.492261in}{4.816010in}}%
\pgfpathlineto{\pgfqpoint{2.525179in}{4.863034in}}%
\pgfpathlineto{\pgfqpoint{2.516297in}{4.880547in}}%
\pgfpathlineto{\pgfqpoint{2.507491in}{4.894109in}}%
\pgfpathlineto{\pgfqpoint{2.474622in}{4.847808in}}%
\pgfpathlineto{\pgfqpoint{2.441763in}{4.801351in}}%
\pgfpathclose%
\pgfusepath{fill}%
\end{pgfscope}%
\begin{pgfscope}%
\pgfpathrectangle{\pgfqpoint{1.020000in}{0.880000in}}{\pgfqpoint{6.160000in}{6.160000in}}%
\pgfusepath{clip}%
\pgfsetbuttcap%
\pgfsetroundjoin%
\definecolor{currentfill}{rgb}{0.328604,0.439712,0.869587}%
\pgfsetfillcolor{currentfill}%
\pgfsetlinewidth{0.000000pt}%
\definecolor{currentstroke}{rgb}{0.000000,0.000000,0.000000}%
\pgfsetstrokecolor{currentstroke}%
\pgfsetdash{}{0pt}%
\pgfpathmoveto{\pgfqpoint{5.869927in}{3.069603in}}%
\pgfpathlineto{\pgfqpoint{5.881782in}{3.054193in}}%
\pgfpathlineto{\pgfqpoint{5.893659in}{3.038766in}}%
\pgfpathlineto{\pgfqpoint{5.926230in}{3.040469in}}%
\pgfpathlineto{\pgfqpoint{5.958781in}{3.042175in}}%
\pgfpathlineto{\pgfqpoint{5.946852in}{3.057574in}}%
\pgfpathlineto{\pgfqpoint{5.934945in}{3.072979in}}%
\pgfpathlineto{\pgfqpoint{5.902446in}{3.071301in}}%
\pgfpathlineto{\pgfqpoint{5.869927in}{3.069603in}}%
\pgfpathclose%
\pgfusepath{fill}%
\end{pgfscope}%
\begin{pgfscope}%
\pgfpathrectangle{\pgfqpoint{1.020000in}{0.880000in}}{\pgfqpoint{6.160000in}{6.160000in}}%
\pgfusepath{clip}%
\pgfsetbuttcap%
\pgfsetroundjoin%
\definecolor{currentfill}{rgb}{0.343278,0.459354,0.884122}%
\pgfsetfillcolor{currentfill}%
\pgfsetlinewidth{0.000000pt}%
\definecolor{currentstroke}{rgb}{0.000000,0.000000,0.000000}%
\pgfsetstrokecolor{currentstroke}%
\pgfsetdash{}{0pt}%
\pgfpathmoveto{\pgfqpoint{5.431915in}{3.091615in}}%
\pgfpathlineto{\pgfqpoint{5.443444in}{3.082223in}}%
\pgfpathlineto{\pgfqpoint{5.454980in}{3.071507in}}%
\pgfpathlineto{\pgfqpoint{5.487701in}{3.074167in}}%
\pgfpathlineto{\pgfqpoint{5.520403in}{3.076920in}}%
\pgfpathlineto{\pgfqpoint{5.508828in}{3.088853in}}%
\pgfpathlineto{\pgfqpoint{5.497262in}{3.099777in}}%
\pgfpathlineto{\pgfqpoint{5.464597in}{3.095660in}}%
\pgfpathlineto{\pgfqpoint{5.431915in}{3.091615in}}%
\pgfpathclose%
\pgfusepath{fill}%
\end{pgfscope}%
\begin{pgfscope}%
\pgfpathrectangle{\pgfqpoint{1.020000in}{0.880000in}}{\pgfqpoint{6.160000in}{6.160000in}}%
\pgfusepath{clip}%
\pgfsetbuttcap%
\pgfsetroundjoin%
\definecolor{currentfill}{rgb}{0.338377,0.452819,0.879317}%
\pgfsetfillcolor{currentfill}%
\pgfsetlinewidth{0.000000pt}%
\definecolor{currentstroke}{rgb}{0.000000,0.000000,0.000000}%
\pgfsetstrokecolor{currentstroke}%
\pgfsetdash{}{0pt}%
\pgfpathmoveto{\pgfqpoint{5.651011in}{3.087649in}}%
\pgfpathlineto{\pgfqpoint{5.662683in}{3.073296in}}%
\pgfpathlineto{\pgfqpoint{5.674372in}{3.058610in}}%
\pgfpathlineto{\pgfqpoint{5.707017in}{3.060559in}}%
\pgfpathlineto{\pgfqpoint{5.739641in}{3.062460in}}%
\pgfpathlineto{\pgfqpoint{5.727905in}{3.077511in}}%
\pgfpathlineto{\pgfqpoint{5.716188in}{3.092362in}}%
\pgfpathlineto{\pgfqpoint{5.683610in}{3.090072in}}%
\pgfpathlineto{\pgfqpoint{5.651011in}{3.087649in}}%
\pgfpathclose%
\pgfusepath{fill}%
\end{pgfscope}%
\begin{pgfscope}%
\pgfpathrectangle{\pgfqpoint{1.020000in}{0.880000in}}{\pgfqpoint{6.160000in}{6.160000in}}%
\pgfusepath{clip}%
\pgfsetbuttcap%
\pgfsetroundjoin%
\definecolor{currentfill}{rgb}{0.565182,0.699438,0.996635}%
\pgfsetfillcolor{currentfill}%
\pgfsetlinewidth{0.000000pt}%
\definecolor{currentstroke}{rgb}{0.000000,0.000000,0.000000}%
\pgfsetstrokecolor{currentstroke}%
\pgfsetdash{}{0pt}%
\pgfpathmoveto{\pgfqpoint{4.163648in}{3.504873in}}%
\pgfpathlineto{\pgfqpoint{4.173962in}{3.487362in}}%
\pgfpathlineto{\pgfqpoint{4.184294in}{3.471194in}}%
\pgfpathlineto{\pgfqpoint{4.217400in}{3.456715in}}%
\pgfpathlineto{\pgfqpoint{4.250476in}{3.442795in}}%
\pgfpathlineto{\pgfqpoint{4.240094in}{3.458576in}}%
\pgfpathlineto{\pgfqpoint{4.229732in}{3.475492in}}%
\pgfpathlineto{\pgfqpoint{4.196705in}{3.489856in}}%
\pgfpathlineto{\pgfqpoint{4.163648in}{3.504873in}}%
\pgfpathclose%
\pgfusepath{fill}%
\end{pgfscope}%
\begin{pgfscope}%
\pgfpathrectangle{\pgfqpoint{1.020000in}{0.880000in}}{\pgfqpoint{6.160000in}{6.160000in}}%
\pgfusepath{clip}%
\pgfsetbuttcap%
\pgfsetroundjoin%
\definecolor{currentfill}{rgb}{0.516260,0.654498,0.986407}%
\pgfsetfillcolor{currentfill}%
\pgfsetlinewidth{0.000000pt}%
\definecolor{currentstroke}{rgb}{0.000000,0.000000,0.000000}%
\pgfsetstrokecolor{currentstroke}%
\pgfsetdash{}{0pt}%
\pgfpathmoveto{\pgfqpoint{4.316540in}{3.416373in}}%
\pgfpathlineto{\pgfqpoint{4.326992in}{3.401846in}}%
\pgfpathlineto{\pgfqpoint{4.337465in}{3.388317in}}%
\pgfpathlineto{\pgfqpoint{4.370504in}{3.375611in}}%
\pgfpathlineto{\pgfqpoint{4.403514in}{3.363091in}}%
\pgfpathlineto{\pgfqpoint{4.392990in}{3.376646in}}%
\pgfpathlineto{\pgfqpoint{4.382487in}{3.391056in}}%
\pgfpathlineto{\pgfqpoint{4.349528in}{3.403653in}}%
\pgfpathlineto{\pgfqpoint{4.316540in}{3.416373in}}%
\pgfpathclose%
\pgfusepath{fill}%
\end{pgfscope}%
\begin{pgfscope}%
\pgfpathrectangle{\pgfqpoint{1.020000in}{0.880000in}}{\pgfqpoint{6.160000in}{6.160000in}}%
\pgfusepath{clip}%
\pgfsetbuttcap%
\pgfsetroundjoin%
\definecolor{currentfill}{rgb}{0.877149,0.394645,0.311724}%
\pgfsetfillcolor{currentfill}%
\pgfsetlinewidth{0.000000pt}%
\definecolor{currentstroke}{rgb}{0.000000,0.000000,0.000000}%
\pgfsetstrokecolor{currentstroke}%
\pgfsetdash{}{0pt}%
\pgfpathmoveto{\pgfqpoint{2.507491in}{4.894109in}}%
\pgfpathlineto{\pgfqpoint{2.516297in}{4.880547in}}%
\pgfpathlineto{\pgfqpoint{2.525179in}{4.863034in}}%
\pgfpathlineto{\pgfqpoint{2.558115in}{4.909434in}}%
\pgfpathlineto{\pgfqpoint{2.591074in}{4.954723in}}%
\pgfpathlineto{\pgfqpoint{2.582139in}{4.971735in}}%
\pgfpathlineto{\pgfqpoint{2.573283in}{4.984379in}}%
\pgfpathlineto{\pgfqpoint{2.540376in}{4.939793in}}%
\pgfpathlineto{\pgfqpoint{2.507491in}{4.894109in}}%
\pgfpathclose%
\pgfusepath{fill}%
\end{pgfscope}%
\begin{pgfscope}%
\pgfpathrectangle{\pgfqpoint{1.020000in}{0.880000in}}{\pgfqpoint{6.160000in}{6.160000in}}%
\pgfusepath{clip}%
\pgfsetbuttcap%
\pgfsetroundjoin%
\definecolor{currentfill}{rgb}{0.826784,0.858205,0.906953}%
\pgfsetfillcolor{currentfill}%
\pgfsetlinewidth{0.000000pt}%
\definecolor{currentstroke}{rgb}{0.000000,0.000000,0.000000}%
\pgfsetstrokecolor{currentstroke}%
\pgfsetdash{}{0pt}%
\pgfpathmoveto{\pgfqpoint{3.684080in}{4.032609in}}%
\pgfpathlineto{\pgfqpoint{3.694097in}{3.988685in}}%
\pgfpathlineto{\pgfqpoint{3.704121in}{3.945917in}}%
\pgfpathlineto{\pgfqpoint{3.737511in}{3.915917in}}%
\pgfpathlineto{\pgfqpoint{3.770861in}{3.886128in}}%
\pgfpathlineto{\pgfqpoint{3.760820in}{3.925059in}}%
\pgfpathlineto{\pgfqpoint{3.750789in}{3.965056in}}%
\pgfpathlineto{\pgfqpoint{3.717456in}{3.998676in}}%
\pgfpathlineto{\pgfqpoint{3.684080in}{4.032609in}}%
\pgfpathclose%
\pgfusepath{fill}%
\end{pgfscope}%
\begin{pgfscope}%
\pgfpathrectangle{\pgfqpoint{1.020000in}{0.880000in}}{\pgfqpoint{6.160000in}{6.160000in}}%
\pgfusepath{clip}%
\pgfsetbuttcap%
\pgfsetroundjoin%
\definecolor{currentfill}{rgb}{0.473070,0.611077,0.970634}%
\pgfsetfillcolor{currentfill}%
\pgfsetlinewidth{0.000000pt}%
\definecolor{currentstroke}{rgb}{0.000000,0.000000,0.000000}%
\pgfsetstrokecolor{currentstroke}%
\pgfsetdash{}{0pt}%
\pgfpathmoveto{\pgfqpoint{4.469446in}{3.337935in}}%
\pgfpathlineto{\pgfqpoint{4.480044in}{3.325305in}}%
\pgfpathlineto{\pgfqpoint{4.490663in}{3.313345in}}%
\pgfpathlineto{\pgfqpoint{4.523637in}{3.300668in}}%
\pgfpathlineto{\pgfqpoint{4.556580in}{3.287677in}}%
\pgfpathlineto{\pgfqpoint{4.545907in}{3.299181in}}%
\pgfpathlineto{\pgfqpoint{4.535256in}{3.311254in}}%
\pgfpathlineto{\pgfqpoint{4.502367in}{3.324897in}}%
\pgfpathlineto{\pgfqpoint{4.469446in}{3.337935in}}%
\pgfpathclose%
\pgfusepath{fill}%
\end{pgfscope}%
\begin{pgfscope}%
\pgfpathrectangle{\pgfqpoint{1.020000in}{0.880000in}}{\pgfqpoint{6.160000in}{6.160000in}}%
\pgfusepath{clip}%
\pgfsetbuttcap%
\pgfsetroundjoin%
\definecolor{currentfill}{rgb}{0.891817,0.851973,0.829085}%
\pgfsetfillcolor{currentfill}%
\pgfsetlinewidth{0.000000pt}%
\definecolor{currentstroke}{rgb}{0.000000,0.000000,0.000000}%
\pgfsetstrokecolor{currentstroke}%
\pgfsetdash{}{0pt}%
\pgfpathmoveto{\pgfqpoint{3.597206in}{4.198601in}}%
\pgfpathlineto{\pgfqpoint{3.607198in}{4.149259in}}%
\pgfpathlineto{\pgfqpoint{3.617200in}{4.100403in}}%
\pgfpathlineto{\pgfqpoint{3.650660in}{4.066606in}}%
\pgfpathlineto{\pgfqpoint{3.684080in}{4.032609in}}%
\pgfpathlineto{\pgfqpoint{3.674071in}{4.077357in}}%
\pgfpathlineto{\pgfqpoint{3.664073in}{4.122587in}}%
\pgfpathlineto{\pgfqpoint{3.630661in}{4.160679in}}%
\pgfpathlineto{\pgfqpoint{3.597206in}{4.198601in}}%
\pgfpathclose%
\pgfusepath{fill}%
\end{pgfscope}%
\begin{pgfscope}%
\pgfpathrectangle{\pgfqpoint{1.020000in}{0.880000in}}{\pgfqpoint{6.160000in}{6.160000in}}%
\pgfusepath{clip}%
\pgfsetbuttcap%
\pgfsetroundjoin%
\definecolor{currentfill}{rgb}{0.723315,0.068898,0.162989}%
\pgfsetfillcolor{currentfill}%
\pgfsetlinewidth{0.000000pt}%
\definecolor{currentstroke}{rgb}{0.000000,0.000000,0.000000}%
\pgfsetstrokecolor{currentstroke}%
\pgfsetdash{}{0pt}%
\pgfpathmoveto{\pgfqpoint{2.771507in}{5.201342in}}%
\pgfpathlineto{\pgfqpoint{2.780517in}{5.191063in}}%
\pgfpathlineto{\pgfqpoint{2.789612in}{5.175394in}}%
\pgfpathlineto{\pgfqpoint{2.822865in}{5.199303in}}%
\pgfpathlineto{\pgfqpoint{2.856170in}{5.218470in}}%
\pgfpathlineto{\pgfqpoint{2.847022in}{5.233781in}}%
\pgfpathlineto{\pgfqpoint{2.837960in}{5.243493in}}%
\pgfpathlineto{\pgfqpoint{2.804708in}{5.224764in}}%
\pgfpathlineto{\pgfqpoint{2.771507in}{5.201342in}}%
\pgfpathclose%
\pgfusepath{fill}%
\end{pgfscope}%
\begin{pgfscope}%
\pgfpathrectangle{\pgfqpoint{1.020000in}{0.880000in}}{\pgfqpoint{6.160000in}{6.160000in}}%
\pgfusepath{clip}%
\pgfsetbuttcap%
\pgfsetroundjoin%
\definecolor{currentfill}{rgb}{0.425199,0.559058,0.946061}%
\pgfsetfillcolor{currentfill}%
\pgfsetlinewidth{0.000000pt}%
\definecolor{currentstroke}{rgb}{0.000000,0.000000,0.000000}%
\pgfsetstrokecolor{currentstroke}%
\pgfsetdash{}{0pt}%
\pgfpathmoveto{\pgfqpoint{4.622372in}{3.260038in}}%
\pgfpathlineto{\pgfqpoint{4.633123in}{3.250047in}}%
\pgfpathlineto{\pgfqpoint{4.643897in}{3.240281in}}%
\pgfpathlineto{\pgfqpoint{4.676802in}{3.226903in}}%
\pgfpathlineto{\pgfqpoint{4.709676in}{3.213070in}}%
\pgfpathlineto{\pgfqpoint{4.698841in}{3.221114in}}%
\pgfpathlineto{\pgfqpoint{4.688028in}{3.229208in}}%
\pgfpathlineto{\pgfqpoint{4.655217in}{3.245076in}}%
\pgfpathlineto{\pgfqpoint{4.622372in}{3.260038in}}%
\pgfpathclose%
\pgfusepath{fill}%
\end{pgfscope}%
\begin{pgfscope}%
\pgfpathrectangle{\pgfqpoint{1.020000in}{0.880000in}}{\pgfqpoint{6.160000in}{6.160000in}}%
\pgfusepath{clip}%
\pgfsetbuttcap%
\pgfsetroundjoin%
\definecolor{currentfill}{rgb}{0.624703,0.748318,0.998719}%
\pgfsetfillcolor{currentfill}%
\pgfsetlinewidth{0.000000pt}%
\definecolor{currentstroke}{rgb}{0.000000,0.000000,0.000000}%
\pgfsetstrokecolor{currentstroke}%
\pgfsetdash{}{0pt}%
\pgfpathmoveto{\pgfqpoint{4.010708in}{3.616102in}}%
\pgfpathlineto{\pgfqpoint{4.020898in}{3.593212in}}%
\pgfpathlineto{\pgfqpoint{4.031102in}{3.571988in}}%
\pgfpathlineto{\pgfqpoint{4.064288in}{3.554166in}}%
\pgfpathlineto{\pgfqpoint{4.097440in}{3.537023in}}%
\pgfpathlineto{\pgfqpoint{4.087191in}{3.556878in}}%
\pgfpathlineto{\pgfqpoint{4.076958in}{3.578166in}}%
\pgfpathlineto{\pgfqpoint{4.043850in}{3.596710in}}%
\pgfpathlineto{\pgfqpoint{4.010708in}{3.616102in}}%
\pgfpathclose%
\pgfusepath{fill}%
\end{pgfscope}%
\begin{pgfscope}%
\pgfpathrectangle{\pgfqpoint{1.020000in}{0.880000in}}{\pgfqpoint{6.160000in}{6.160000in}}%
\pgfusepath{clip}%
\pgfsetbuttcap%
\pgfsetroundjoin%
\definecolor{currentfill}{rgb}{0.839365,0.321856,0.264924}%
\pgfsetfillcolor{currentfill}%
\pgfsetlinewidth{0.000000pt}%
\definecolor{currentstroke}{rgb}{0.000000,0.000000,0.000000}%
\pgfsetstrokecolor{currentstroke}%
\pgfsetdash{}{0pt}%
\pgfpathmoveto{\pgfqpoint{2.573283in}{4.984379in}}%
\pgfpathlineto{\pgfqpoint{2.582139in}{4.971735in}}%
\pgfpathlineto{\pgfqpoint{2.591074in}{4.954723in}}%
\pgfpathlineto{\pgfqpoint{2.624062in}{4.998404in}}%
\pgfpathlineto{\pgfqpoint{2.657085in}{5.039974in}}%
\pgfpathlineto{\pgfqpoint{2.648097in}{5.056504in}}%
\pgfpathlineto{\pgfqpoint{2.639191in}{5.068275in}}%
\pgfpathlineto{\pgfqpoint{2.606220in}{5.027372in}}%
\pgfpathlineto{\pgfqpoint{2.573283in}{4.984379in}}%
\pgfpathclose%
\pgfusepath{fill}%
\end{pgfscope}%
\begin{pgfscope}%
\pgfpathrectangle{\pgfqpoint{1.020000in}{0.880000in}}{\pgfqpoint{6.160000in}{6.160000in}}%
\pgfusepath{clip}%
\pgfsetbuttcap%
\pgfsetroundjoin%
\definecolor{currentfill}{rgb}{0.723315,0.068898,0.162989}%
\pgfsetfillcolor{currentfill}%
\pgfsetlinewidth{0.000000pt}%
\definecolor{currentstroke}{rgb}{0.000000,0.000000,0.000000}%
\pgfsetstrokecolor{currentstroke}%
\pgfsetdash{}{0pt}%
\pgfpathmoveto{\pgfqpoint{2.989864in}{5.242752in}}%
\pgfpathlineto{\pgfqpoint{2.999194in}{5.222180in}}%
\pgfpathlineto{\pgfqpoint{3.008601in}{5.195865in}}%
\pgfpathlineto{\pgfqpoint{3.042171in}{5.188405in}}%
\pgfpathlineto{\pgfqpoint{3.075771in}{5.175449in}}%
\pgfpathlineto{\pgfqpoint{3.066322in}{5.201479in}}%
\pgfpathlineto{\pgfqpoint{3.056945in}{5.221828in}}%
\pgfpathlineto{\pgfqpoint{3.023389in}{5.235055in}}%
\pgfpathlineto{\pgfqpoint{2.989864in}{5.242752in}}%
\pgfpathclose%
\pgfusepath{fill}%
\end{pgfscope}%
\begin{pgfscope}%
\pgfpathrectangle{\pgfqpoint{1.020000in}{0.880000in}}{\pgfqpoint{6.160000in}{6.160000in}}%
\pgfusepath{clip}%
\pgfsetbuttcap%
\pgfsetroundjoin%
\definecolor{currentfill}{rgb}{0.758539,0.832787,0.958408}%
\pgfsetfillcolor{currentfill}%
\pgfsetlinewidth{0.000000pt}%
\definecolor{currentstroke}{rgb}{0.000000,0.000000,0.000000}%
\pgfsetstrokecolor{currentstroke}%
\pgfsetdash{}{0pt}%
\pgfpathmoveto{\pgfqpoint{3.770861in}{3.886128in}}%
\pgfpathlineto{\pgfqpoint{3.780909in}{3.848542in}}%
\pgfpathlineto{\pgfqpoint{3.790966in}{3.812560in}}%
\pgfpathlineto{\pgfqpoint{3.824297in}{3.786637in}}%
\pgfpathlineto{\pgfqpoint{3.857589in}{3.761161in}}%
\pgfpathlineto{\pgfqpoint{3.847507in}{3.793849in}}%
\pgfpathlineto{\pgfqpoint{3.837435in}{3.827981in}}%
\pgfpathlineto{\pgfqpoint{3.804168in}{3.856755in}}%
\pgfpathlineto{\pgfqpoint{3.770861in}{3.886128in}}%
\pgfpathclose%
\pgfusepath{fill}%
\end{pgfscope}%
\begin{pgfscope}%
\pgfpathrectangle{\pgfqpoint{1.020000in}{0.880000in}}{\pgfqpoint{6.160000in}{6.160000in}}%
\pgfusepath{clip}%
\pgfsetbuttcap%
\pgfsetroundjoin%
\definecolor{currentfill}{rgb}{0.947345,0.794696,0.716991}%
\pgfsetfillcolor{currentfill}%
\pgfsetlinewidth{0.000000pt}%
\definecolor{currentstroke}{rgb}{0.000000,0.000000,0.000000}%
\pgfsetstrokecolor{currentstroke}%
\pgfsetdash{}{0pt}%
\pgfpathmoveto{\pgfqpoint{3.510222in}{4.379259in}}%
\pgfpathlineto{\pgfqpoint{3.520190in}{4.326162in}}%
\pgfpathlineto{\pgfqpoint{3.530171in}{4.272672in}}%
\pgfpathlineto{\pgfqpoint{3.563708in}{4.236041in}}%
\pgfpathlineto{\pgfqpoint{3.597206in}{4.198601in}}%
\pgfpathlineto{\pgfqpoint{3.587224in}{4.248035in}}%
\pgfpathlineto{\pgfqpoint{3.577255in}{4.297159in}}%
\pgfpathlineto{\pgfqpoint{3.543760in}{4.338637in}}%
\pgfpathlineto{\pgfqpoint{3.510222in}{4.379259in}}%
\pgfpathclose%
\pgfusepath{fill}%
\end{pgfscope}%
\begin{pgfscope}%
\pgfpathrectangle{\pgfqpoint{1.020000in}{0.880000in}}{\pgfqpoint{6.160000in}{6.160000in}}%
\pgfusepath{clip}%
\pgfsetbuttcap%
\pgfsetroundjoin%
\definecolor{currentfill}{rgb}{0.388852,0.516298,0.921373}%
\pgfsetfillcolor{currentfill}%
\pgfsetlinewidth{0.000000pt}%
\definecolor{currentstroke}{rgb}{0.000000,0.000000,0.000000}%
\pgfsetstrokecolor{currentstroke}%
\pgfsetdash{}{0pt}%
\pgfpathmoveto{\pgfqpoint{4.775325in}{3.183967in}}%
\pgfpathlineto{\pgfqpoint{4.786248in}{3.178141in}}%
\pgfpathlineto{\pgfqpoint{4.797189in}{3.171739in}}%
\pgfpathlineto{\pgfqpoint{4.830036in}{3.159334in}}%
\pgfpathlineto{\pgfqpoint{4.862854in}{3.147029in}}%
\pgfpathlineto{\pgfqpoint{4.851841in}{3.150726in}}%
\pgfpathlineto{\pgfqpoint{4.840844in}{3.153494in}}%
\pgfpathlineto{\pgfqpoint{4.808101in}{3.168824in}}%
\pgfpathlineto{\pgfqpoint{4.775325in}{3.183967in}}%
\pgfpathclose%
\pgfusepath{fill}%
\end{pgfscope}%
\begin{pgfscope}%
\pgfpathrectangle{\pgfqpoint{1.020000in}{0.880000in}}{\pgfqpoint{6.160000in}{6.160000in}}%
\pgfusepath{clip}%
\pgfsetbuttcap%
\pgfsetroundjoin%
\definecolor{currentfill}{rgb}{0.758112,0.168122,0.188827}%
\pgfsetfillcolor{currentfill}%
\pgfsetlinewidth{0.000000pt}%
\definecolor{currentstroke}{rgb}{0.000000,0.000000,0.000000}%
\pgfsetstrokecolor{currentstroke}%
\pgfsetdash{}{0pt}%
\pgfpathmoveto{\pgfqpoint{2.705256in}{5.141846in}}%
\pgfpathlineto{\pgfqpoint{2.714213in}{5.130872in}}%
\pgfpathlineto{\pgfqpoint{2.723255in}{5.114794in}}%
\pgfpathlineto{\pgfqpoint{2.756408in}{5.147092in}}%
\pgfpathlineto{\pgfqpoint{2.789612in}{5.175394in}}%
\pgfpathlineto{\pgfqpoint{2.780517in}{5.191063in}}%
\pgfpathlineto{\pgfqpoint{2.771507in}{5.201342in}}%
\pgfpathlineto{\pgfqpoint{2.738357in}{5.173572in}}%
\pgfpathlineto{\pgfqpoint{2.705256in}{5.141846in}}%
\pgfpathclose%
\pgfusepath{fill}%
\end{pgfscope}%
\begin{pgfscope}%
\pgfpathrectangle{\pgfqpoint{1.020000in}{0.880000in}}{\pgfqpoint{6.160000in}{6.160000in}}%
\pgfusepath{clip}%
\pgfsetbuttcap%
\pgfsetroundjoin%
\definecolor{currentfill}{rgb}{0.795938,0.241845,0.220830}%
\pgfsetfillcolor{currentfill}%
\pgfsetlinewidth{0.000000pt}%
\definecolor{currentstroke}{rgb}{0.000000,0.000000,0.000000}%
\pgfsetstrokecolor{currentstroke}%
\pgfsetdash{}{0pt}%
\pgfpathmoveto{\pgfqpoint{2.639191in}{5.068275in}}%
\pgfpathlineto{\pgfqpoint{2.648097in}{5.056504in}}%
\pgfpathlineto{\pgfqpoint{2.657085in}{5.039974in}}%
\pgfpathlineto{\pgfqpoint{2.690148in}{5.078933in}}%
\pgfpathlineto{\pgfqpoint{2.723255in}{5.114794in}}%
\pgfpathlineto{\pgfqpoint{2.714213in}{5.130872in}}%
\pgfpathlineto{\pgfqpoint{2.705256in}{5.141846in}}%
\pgfpathlineto{\pgfqpoint{2.672202in}{5.106594in}}%
\pgfpathlineto{\pgfqpoint{2.639191in}{5.068275in}}%
\pgfpathclose%
\pgfusepath{fill}%
\end{pgfscope}%
\begin{pgfscope}%
\pgfpathrectangle{\pgfqpoint{1.020000in}{0.880000in}}{\pgfqpoint{6.160000in}{6.160000in}}%
\pgfusepath{clip}%
\pgfsetbuttcap%
\pgfsetroundjoin%
\definecolor{currentfill}{rgb}{0.343278,0.459354,0.884122}%
\pgfsetfillcolor{currentfill}%
\pgfsetlinewidth{0.000000pt}%
\definecolor{currentstroke}{rgb}{0.000000,0.000000,0.000000}%
\pgfsetstrokecolor{currentstroke}%
\pgfsetdash{}{0pt}%
\pgfpathmoveto{\pgfqpoint{5.147279in}{3.082880in}}%
\pgfpathlineto{\pgfqpoint{5.158613in}{3.081568in}}%
\pgfpathlineto{\pgfqpoint{5.169948in}{3.077850in}}%
\pgfpathlineto{\pgfqpoint{5.202738in}{3.076668in}}%
\pgfpathlineto{\pgfqpoint{5.235515in}{3.076484in}}%
\pgfpathlineto{\pgfqpoint{5.224122in}{3.080307in}}%
\pgfpathlineto{\pgfqpoint{5.212728in}{3.081719in}}%
\pgfpathlineto{\pgfqpoint{5.180008in}{3.081590in}}%
\pgfpathlineto{\pgfqpoint{5.147279in}{3.082880in}}%
\pgfpathclose%
\pgfusepath{fill}%
\end{pgfscope}%
\begin{pgfscope}%
\pgfpathrectangle{\pgfqpoint{1.020000in}{0.880000in}}{\pgfqpoint{6.160000in}{6.160000in}}%
\pgfusepath{clip}%
\pgfsetbuttcap%
\pgfsetroundjoin%
\definecolor{currentfill}{rgb}{0.343278,0.459354,0.884122}%
\pgfsetfillcolor{currentfill}%
\pgfsetlinewidth{0.000000pt}%
\definecolor{currentstroke}{rgb}{0.000000,0.000000,0.000000}%
\pgfsetstrokecolor{currentstroke}%
\pgfsetdash{}{0pt}%
\pgfpathmoveto{\pgfqpoint{5.366503in}{3.084273in}}%
\pgfpathlineto{\pgfqpoint{5.377992in}{3.076348in}}%
\pgfpathlineto{\pgfqpoint{5.389486in}{3.066804in}}%
\pgfpathlineto{\pgfqpoint{5.422241in}{3.069023in}}%
\pgfpathlineto{\pgfqpoint{5.454980in}{3.071507in}}%
\pgfpathlineto{\pgfqpoint{5.443444in}{3.082223in}}%
\pgfpathlineto{\pgfqpoint{5.431915in}{3.091615in}}%
\pgfpathlineto{\pgfqpoint{5.399217in}{3.087773in}}%
\pgfpathlineto{\pgfqpoint{5.366503in}{3.084273in}}%
\pgfpathclose%
\pgfusepath{fill}%
\end{pgfscope}%
\begin{pgfscope}%
\pgfpathrectangle{\pgfqpoint{1.020000in}{0.880000in}}{\pgfqpoint{6.160000in}{6.160000in}}%
\pgfusepath{clip}%
\pgfsetbuttcap%
\pgfsetroundjoin%
\definecolor{currentfill}{rgb}{0.969192,0.705836,0.593704}%
\pgfsetfillcolor{currentfill}%
\pgfsetlinewidth{0.000000pt}%
\definecolor{currentstroke}{rgb}{0.000000,0.000000,0.000000}%
\pgfsetstrokecolor{currentstroke}%
\pgfsetdash{}{0pt}%
\pgfpathmoveto{\pgfqpoint{3.423148in}{4.566901in}}%
\pgfpathlineto{\pgfqpoint{3.433082in}{4.512395in}}%
\pgfpathlineto{\pgfqpoint{3.443035in}{4.456453in}}%
\pgfpathlineto{\pgfqpoint{3.476646in}{4.418656in}}%
\pgfpathlineto{\pgfqpoint{3.510222in}{4.379259in}}%
\pgfpathlineto{\pgfqpoint{3.500270in}{4.431521in}}%
\pgfpathlineto{\pgfqpoint{3.490337in}{4.482506in}}%
\pgfpathlineto{\pgfqpoint{3.456760in}{4.525563in}}%
\pgfpathlineto{\pgfqpoint{3.423148in}{4.566901in}}%
\pgfpathclose%
\pgfusepath{fill}%
\end{pgfscope}%
\begin{pgfscope}%
\pgfpathrectangle{\pgfqpoint{1.020000in}{0.880000in}}{\pgfqpoint{6.160000in}{6.160000in}}%
\pgfusepath{clip}%
\pgfsetbuttcap%
\pgfsetroundjoin%
\definecolor{currentfill}{rgb}{0.328604,0.439712,0.869587}%
\pgfsetfillcolor{currentfill}%
\pgfsetlinewidth{0.000000pt}%
\definecolor{currentstroke}{rgb}{0.000000,0.000000,0.000000}%
\pgfsetstrokecolor{currentstroke}%
\pgfsetdash{}{0pt}%
\pgfpathmoveto{\pgfqpoint{5.804826in}{3.066115in}}%
\pgfpathlineto{\pgfqpoint{5.816630in}{3.050762in}}%
\pgfpathlineto{\pgfqpoint{5.828456in}{3.035354in}}%
\pgfpathlineto{\pgfqpoint{5.861068in}{3.037061in}}%
\pgfpathlineto{\pgfqpoint{5.893659in}{3.038766in}}%
\pgfpathlineto{\pgfqpoint{5.881782in}{3.054193in}}%
\pgfpathlineto{\pgfqpoint{5.869927in}{3.069603in}}%
\pgfpathlineto{\pgfqpoint{5.837387in}{3.067877in}}%
\pgfpathlineto{\pgfqpoint{5.804826in}{3.066115in}}%
\pgfpathclose%
\pgfusepath{fill}%
\end{pgfscope}%
\begin{pgfscope}%
\pgfpathrectangle{\pgfqpoint{1.020000in}{0.880000in}}{\pgfqpoint{6.160000in}{6.160000in}}%
\pgfusepath{clip}%
\pgfsetbuttcap%
\pgfsetroundjoin%
\definecolor{currentfill}{rgb}{0.363461,0.484784,0.901019}%
\pgfsetfillcolor{currentfill}%
\pgfsetlinewidth{0.000000pt}%
\definecolor{currentstroke}{rgb}{0.000000,0.000000,0.000000}%
\pgfsetstrokecolor{currentstroke}%
\pgfsetdash{}{0pt}%
\pgfpathmoveto{\pgfqpoint{4.928410in}{3.123401in}}%
\pgfpathlineto{\pgfqpoint{4.939514in}{3.121276in}}%
\pgfpathlineto{\pgfqpoint{4.950629in}{3.117535in}}%
\pgfpathlineto{\pgfqpoint{4.983447in}{3.109145in}}%
\pgfpathlineto{\pgfqpoint{5.016245in}{3.101601in}}%
\pgfpathlineto{\pgfqpoint{5.005055in}{3.103126in}}%
\pgfpathlineto{\pgfqpoint{4.993873in}{3.102638in}}%
\pgfpathlineto{\pgfqpoint{4.961152in}{3.112530in}}%
\pgfpathlineto{\pgfqpoint{4.928410in}{3.123401in}}%
\pgfpathclose%
\pgfusepath{fill}%
\end{pgfscope}%
\begin{pgfscope}%
\pgfpathrectangle{\pgfqpoint{1.020000in}{0.880000in}}{\pgfqpoint{6.160000in}{6.160000in}}%
\pgfusepath{clip}%
\pgfsetbuttcap%
\pgfsetroundjoin%
\definecolor{currentfill}{rgb}{0.338377,0.452819,0.879317}%
\pgfsetfillcolor{currentfill}%
\pgfsetlinewidth{0.000000pt}%
\definecolor{currentstroke}{rgb}{0.000000,0.000000,0.000000}%
\pgfsetstrokecolor{currentstroke}%
\pgfsetdash{}{0pt}%
\pgfpathmoveto{\pgfqpoint{5.585748in}{3.082437in}}%
\pgfpathlineto{\pgfqpoint{5.597377in}{3.068778in}}%
\pgfpathlineto{\pgfqpoint{5.609021in}{3.054607in}}%
\pgfpathlineto{\pgfqpoint{5.641707in}{3.056622in}}%
\pgfpathlineto{\pgfqpoint{5.674372in}{3.058610in}}%
\pgfpathlineto{\pgfqpoint{5.662683in}{3.073296in}}%
\pgfpathlineto{\pgfqpoint{5.651011in}{3.087649in}}%
\pgfpathlineto{\pgfqpoint{5.618390in}{3.085097in}}%
\pgfpathlineto{\pgfqpoint{5.585748in}{3.082437in}}%
\pgfpathclose%
\pgfusepath{fill}%
\end{pgfscope}%
\begin{pgfscope}%
\pgfpathrectangle{\pgfqpoint{1.020000in}{0.880000in}}{\pgfqpoint{6.160000in}{6.160000in}}%
\pgfusepath{clip}%
\pgfsetbuttcap%
\pgfsetroundjoin%
\definecolor{currentfill}{rgb}{0.698454,0.799450,0.984577}%
\pgfsetfillcolor{currentfill}%
\pgfsetlinewidth{0.000000pt}%
\definecolor{currentstroke}{rgb}{0.000000,0.000000,0.000000}%
\pgfsetstrokecolor{currentstroke}%
\pgfsetdash{}{0pt}%
\pgfpathmoveto{\pgfqpoint{3.857589in}{3.761161in}}%
\pgfpathlineto{\pgfqpoint{3.867681in}{3.730128in}}%
\pgfpathlineto{\pgfqpoint{3.877783in}{3.700942in}}%
\pgfpathlineto{\pgfqpoint{3.911069in}{3.678813in}}%
\pgfpathlineto{\pgfqpoint{3.944318in}{3.657230in}}%
\pgfpathlineto{\pgfqpoint{3.934181in}{3.683862in}}%
\pgfpathlineto{\pgfqpoint{3.924056in}{3.712126in}}%
\pgfpathlineto{\pgfqpoint{3.890842in}{3.736281in}}%
\pgfpathlineto{\pgfqpoint{3.857589in}{3.761161in}}%
\pgfpathclose%
\pgfusepath{fill}%
\end{pgfscope}%
\begin{pgfscope}%
\pgfpathrectangle{\pgfqpoint{1.020000in}{0.880000in}}{\pgfqpoint{6.160000in}{6.160000in}}%
\pgfusepath{clip}%
\pgfsetbuttcap%
\pgfsetroundjoin%
\definecolor{currentfill}{rgb}{0.774337,0.199759,0.202535}%
\pgfsetfillcolor{currentfill}%
\pgfsetlinewidth{0.000000pt}%
\definecolor{currentstroke}{rgb}{0.000000,0.000000,0.000000}%
\pgfsetstrokecolor{currentstroke}%
\pgfsetdash{}{0pt}%
\pgfpathmoveto{\pgfqpoint{3.075771in}{5.175449in}}%
\pgfpathlineto{\pgfqpoint{3.085291in}{5.143858in}}%
\pgfpathlineto{\pgfqpoint{3.094879in}{5.106895in}}%
\pgfpathlineto{\pgfqpoint{3.128535in}{5.089121in}}%
\pgfpathlineto{\pgfqpoint{3.162207in}{5.066259in}}%
\pgfpathlineto{\pgfqpoint{3.152590in}{5.102522in}}%
\pgfpathlineto{\pgfqpoint{3.143035in}{5.133562in}}%
\pgfpathlineto{\pgfqpoint{3.109394in}{5.157109in}}%
\pgfpathlineto{\pgfqpoint{3.075771in}{5.175449in}}%
\pgfpathclose%
\pgfusepath{fill}%
\end{pgfscope}%
\begin{pgfscope}%
\pgfpathrectangle{\pgfqpoint{1.020000in}{0.880000in}}{\pgfqpoint{6.160000in}{6.160000in}}%
\pgfusepath{clip}%
\pgfsetbuttcap%
\pgfsetroundjoin%
\definecolor{currentfill}{rgb}{0.954853,0.591622,0.471337}%
\pgfsetfillcolor{currentfill}%
\pgfsetlinewidth{0.000000pt}%
\definecolor{currentstroke}{rgb}{0.000000,0.000000,0.000000}%
\pgfsetstrokecolor{currentstroke}%
\pgfsetdash{}{0pt}%
\pgfpathmoveto{\pgfqpoint{3.336045in}{4.751510in}}%
\pgfpathlineto{\pgfqpoint{3.345925in}{4.698442in}}%
\pgfpathlineto{\pgfqpoint{3.355834in}{4.642806in}}%
\pgfpathlineto{\pgfqpoint{3.389504in}{4.606114in}}%
\pgfpathlineto{\pgfqpoint{3.423148in}{4.566901in}}%
\pgfpathlineto{\pgfqpoint{3.413238in}{4.619504in}}%
\pgfpathlineto{\pgfqpoint{3.403355in}{4.669752in}}%
\pgfpathlineto{\pgfqpoint{3.369714in}{4.711978in}}%
\pgfpathlineto{\pgfqpoint{3.336045in}{4.751510in}}%
\pgfpathclose%
\pgfusepath{fill}%
\end{pgfscope}%
\begin{pgfscope}%
\pgfpathrectangle{\pgfqpoint{1.020000in}{0.880000in}}{\pgfqpoint{6.160000in}{6.160000in}}%
\pgfusepath{clip}%
\pgfsetbuttcap%
\pgfsetroundjoin%
\definecolor{currentfill}{rgb}{0.538004,0.674902,0.991722}%
\pgfsetfillcolor{currentfill}%
\pgfsetlinewidth{0.000000pt}%
\definecolor{currentstroke}{rgb}{0.000000,0.000000,0.000000}%
\pgfsetstrokecolor{currentstroke}%
\pgfsetdash{}{0pt}%
\pgfpathmoveto{\pgfqpoint{4.250476in}{3.442795in}}%
\pgfpathlineto{\pgfqpoint{4.260878in}{3.428183in}}%
\pgfpathlineto{\pgfqpoint{4.271299in}{3.414756in}}%
\pgfpathlineto{\pgfqpoint{4.304397in}{3.401334in}}%
\pgfpathlineto{\pgfqpoint{4.337465in}{3.388317in}}%
\pgfpathlineto{\pgfqpoint{4.326992in}{3.401846in}}%
\pgfpathlineto{\pgfqpoint{4.316540in}{3.416373in}}%
\pgfpathlineto{\pgfqpoint{4.283523in}{3.429378in}}%
\pgfpathlineto{\pgfqpoint{4.250476in}{3.442795in}}%
\pgfpathclose%
\pgfusepath{fill}%
\end{pgfscope}%
\begin{pgfscope}%
\pgfpathrectangle{\pgfqpoint{1.020000in}{0.880000in}}{\pgfqpoint{6.160000in}{6.160000in}}%
\pgfusepath{clip}%
\pgfsetbuttcap%
\pgfsetroundjoin%
\definecolor{currentfill}{rgb}{0.586921,0.718121,0.998874}%
\pgfsetfillcolor{currentfill}%
\pgfsetlinewidth{0.000000pt}%
\definecolor{currentstroke}{rgb}{0.000000,0.000000,0.000000}%
\pgfsetstrokecolor{currentstroke}%
\pgfsetdash{}{0pt}%
\pgfpathmoveto{\pgfqpoint{4.097440in}{3.537023in}}%
\pgfpathlineto{\pgfqpoint{4.107706in}{3.518681in}}%
\pgfpathlineto{\pgfqpoint{4.117988in}{3.501920in}}%
\pgfpathlineto{\pgfqpoint{4.151157in}{3.486262in}}%
\pgfpathlineto{\pgfqpoint{4.184294in}{3.471194in}}%
\pgfpathlineto{\pgfqpoint{4.173962in}{3.487362in}}%
\pgfpathlineto{\pgfqpoint{4.163648in}{3.504873in}}%
\pgfpathlineto{\pgfqpoint{4.130560in}{3.520589in}}%
\pgfpathlineto{\pgfqpoint{4.097440in}{3.537023in}}%
\pgfpathclose%
\pgfusepath{fill}%
\end{pgfscope}%
\begin{pgfscope}%
\pgfpathrectangle{\pgfqpoint{1.020000in}{0.880000in}}{\pgfqpoint{6.160000in}{6.160000in}}%
\pgfusepath{clip}%
\pgfsetbuttcap%
\pgfsetroundjoin%
\definecolor{currentfill}{rgb}{0.717435,0.051118,0.158737}%
\pgfsetfillcolor{currentfill}%
\pgfsetlinewidth{0.000000pt}%
\definecolor{currentstroke}{rgb}{0.000000,0.000000,0.000000}%
\pgfsetstrokecolor{currentstroke}%
\pgfsetdash{}{0pt}%
\pgfpathmoveto{\pgfqpoint{2.922927in}{5.241459in}}%
\pgfpathlineto{\pgfqpoint{2.932209in}{5.220689in}}%
\pgfpathlineto{\pgfqpoint{2.941571in}{5.194209in}}%
\pgfpathlineto{\pgfqpoint{2.975066in}{5.197790in}}%
\pgfpathlineto{\pgfqpoint{3.008601in}{5.195865in}}%
\pgfpathlineto{\pgfqpoint{2.999194in}{5.222180in}}%
\pgfpathlineto{\pgfqpoint{2.989864in}{5.242752in}}%
\pgfpathlineto{\pgfqpoint{2.956375in}{5.244877in}}%
\pgfpathlineto{\pgfqpoint{2.922927in}{5.241459in}}%
\pgfpathclose%
\pgfusepath{fill}%
\end{pgfscope}%
\begin{pgfscope}%
\pgfpathrectangle{\pgfqpoint{1.020000in}{0.880000in}}{\pgfqpoint{6.160000in}{6.160000in}}%
\pgfusepath{clip}%
\pgfsetbuttcap%
\pgfsetroundjoin%
\definecolor{currentfill}{rgb}{0.489246,0.627536,0.976896}%
\pgfsetfillcolor{currentfill}%
\pgfsetlinewidth{0.000000pt}%
\definecolor{currentstroke}{rgb}{0.000000,0.000000,0.000000}%
\pgfsetstrokecolor{currentstroke}%
\pgfsetdash{}{0pt}%
\pgfpathmoveto{\pgfqpoint{4.403514in}{3.363091in}}%
\pgfpathlineto{\pgfqpoint{4.414060in}{3.350370in}}%
\pgfpathlineto{\pgfqpoint{4.424627in}{3.338457in}}%
\pgfpathlineto{\pgfqpoint{4.457660in}{3.325887in}}%
\pgfpathlineto{\pgfqpoint{4.490663in}{3.313345in}}%
\pgfpathlineto{\pgfqpoint{4.480044in}{3.325305in}}%
\pgfpathlineto{\pgfqpoint{4.469446in}{3.337935in}}%
\pgfpathlineto{\pgfqpoint{4.436495in}{3.350595in}}%
\pgfpathlineto{\pgfqpoint{4.403514in}{3.363091in}}%
\pgfpathclose%
\pgfusepath{fill}%
\end{pgfscope}%
\begin{pgfscope}%
\pgfpathrectangle{\pgfqpoint{1.020000in}{0.880000in}}{\pgfqpoint{6.160000in}{6.160000in}}%
\pgfusepath{clip}%
\pgfsetbuttcap%
\pgfsetroundjoin%
\definecolor{currentfill}{rgb}{0.968500,0.673977,0.556649}%
\pgfsetfillcolor{currentfill}%
\pgfsetlinewidth{0.000000pt}%
\definecolor{currentstroke}{rgb}{0.000000,0.000000,0.000000}%
\pgfsetstrokecolor{currentstroke}%
\pgfsetdash{}{0pt}%
\pgfpathmoveto{\pgfqpoint{2.261720in}{4.504701in}}%
\pgfpathlineto{\pgfqpoint{2.270451in}{4.482894in}}%
\pgfpathlineto{\pgfqpoint{2.279237in}{4.458800in}}%
\pgfpathlineto{\pgfqpoint{2.312288in}{4.498808in}}%
\pgfpathlineto{\pgfqpoint{2.345309in}{4.540746in}}%
\pgfpathlineto{\pgfqpoint{2.336463in}{4.565132in}}%
\pgfpathlineto{\pgfqpoint{2.327676in}{4.586848in}}%
\pgfpathlineto{\pgfqpoint{2.294713in}{4.544803in}}%
\pgfpathlineto{\pgfqpoint{2.261720in}{4.504701in}}%
\pgfpathclose%
\pgfusepath{fill}%
\end{pgfscope}%
\begin{pgfscope}%
\pgfpathrectangle{\pgfqpoint{1.020000in}{0.880000in}}{\pgfqpoint{6.160000in}{6.160000in}}%
\pgfusepath{clip}%
\pgfsetbuttcap%
\pgfsetroundjoin%
\definecolor{currentfill}{rgb}{0.905783,0.455186,0.355336}%
\pgfsetfillcolor{currentfill}%
\pgfsetlinewidth{0.000000pt}%
\definecolor{currentstroke}{rgb}{0.000000,0.000000,0.000000}%
\pgfsetstrokecolor{currentstroke}%
\pgfsetdash{}{0pt}%
\pgfpathmoveto{\pgfqpoint{3.249018in}{4.921729in}}%
\pgfpathlineto{\pgfqpoint{3.258814in}{4.873144in}}%
\pgfpathlineto{\pgfqpoint{3.268652in}{4.820879in}}%
\pgfpathlineto{\pgfqpoint{3.302356in}{4.787939in}}%
\pgfpathlineto{\pgfqpoint{3.336045in}{4.751510in}}%
\pgfpathlineto{\pgfqpoint{3.326200in}{4.801554in}}%
\pgfpathlineto{\pgfqpoint{3.316393in}{4.848147in}}%
\pgfpathlineto{\pgfqpoint{3.282712in}{4.886779in}}%
\pgfpathlineto{\pgfqpoint{3.249018in}{4.921729in}}%
\pgfpathclose%
\pgfusepath{fill}%
\end{pgfscope}%
\begin{pgfscope}%
\pgfpathrectangle{\pgfqpoint{1.020000in}{0.880000in}}{\pgfqpoint{6.160000in}{6.160000in}}%
\pgfusepath{clip}%
\pgfsetbuttcap%
\pgfsetroundjoin%
\definecolor{currentfill}{rgb}{0.451739,0.588181,0.960201}%
\pgfsetfillcolor{currentfill}%
\pgfsetlinewidth{0.000000pt}%
\definecolor{currentstroke}{rgb}{0.000000,0.000000,0.000000}%
\pgfsetstrokecolor{currentstroke}%
\pgfsetdash{}{0pt}%
\pgfpathmoveto{\pgfqpoint{4.556580in}{3.287677in}}%
\pgfpathlineto{\pgfqpoint{4.567275in}{3.276651in}}%
\pgfpathlineto{\pgfqpoint{4.577992in}{3.266016in}}%
\pgfpathlineto{\pgfqpoint{4.610960in}{3.253281in}}%
\pgfpathlineto{\pgfqpoint{4.643897in}{3.240281in}}%
\pgfpathlineto{\pgfqpoint{4.633123in}{3.250047in}}%
\pgfpathlineto{\pgfqpoint{4.622372in}{3.260038in}}%
\pgfpathlineto{\pgfqpoint{4.589492in}{3.274190in}}%
\pgfpathlineto{\pgfqpoint{4.556580in}{3.287677in}}%
\pgfpathclose%
\pgfusepath{fill}%
\end{pgfscope}%
\begin{pgfscope}%
\pgfpathrectangle{\pgfqpoint{1.020000in}{0.880000in}}{\pgfqpoint{6.160000in}{6.160000in}}%
\pgfusepath{clip}%
\pgfsetbuttcap%
\pgfsetroundjoin%
\definecolor{currentfill}{rgb}{0.843703,0.330068,0.270065}%
\pgfsetfillcolor{currentfill}%
\pgfsetlinewidth{0.000000pt}%
\definecolor{currentstroke}{rgb}{0.000000,0.000000,0.000000}%
\pgfsetstrokecolor{currentstroke}%
\pgfsetdash{}{0pt}%
\pgfpathmoveto{\pgfqpoint{3.162207in}{5.066259in}}%
\pgfpathlineto{\pgfqpoint{3.171884in}{5.025027in}}%
\pgfpathlineto{\pgfqpoint{3.181616in}{4.979139in}}%
\pgfpathlineto{\pgfqpoint{3.215317in}{4.952626in}}%
\pgfpathlineto{\pgfqpoint{3.249018in}{4.921729in}}%
\pgfpathlineto{\pgfqpoint{3.239269in}{4.966225in}}%
\pgfpathlineto{\pgfqpoint{3.229570in}{5.006276in}}%
\pgfpathlineto{\pgfqpoint{3.195888in}{5.038547in}}%
\pgfpathlineto{\pgfqpoint{3.162207in}{5.066259in}}%
\pgfpathclose%
\pgfusepath{fill}%
\end{pgfscope}%
\begin{pgfscope}%
\pgfpathrectangle{\pgfqpoint{1.020000in}{0.880000in}}{\pgfqpoint{6.160000in}{6.160000in}}%
\pgfusepath{clip}%
\pgfsetbuttcap%
\pgfsetroundjoin%
\definecolor{currentfill}{rgb}{0.348323,0.465711,0.888346}%
\pgfsetfillcolor{currentfill}%
\pgfsetlinewidth{0.000000pt}%
\definecolor{currentstroke}{rgb}{0.000000,0.000000,0.000000}%
\pgfsetstrokecolor{currentstroke}%
\pgfsetdash{}{0pt}%
\pgfpathmoveto{\pgfqpoint{5.081789in}{3.089687in}}%
\pgfpathlineto{\pgfqpoint{5.093057in}{3.087597in}}%
\pgfpathlineto{\pgfqpoint{5.104328in}{3.083265in}}%
\pgfpathlineto{\pgfqpoint{5.137145in}{3.080053in}}%
\pgfpathlineto{\pgfqpoint{5.169948in}{3.077850in}}%
\pgfpathlineto{\pgfqpoint{5.158613in}{3.081568in}}%
\pgfpathlineto{\pgfqpoint{5.147279in}{3.082880in}}%
\pgfpathlineto{\pgfqpoint{5.114540in}{3.085595in}}%
\pgfpathlineto{\pgfqpoint{5.081789in}{3.089687in}}%
\pgfpathclose%
\pgfusepath{fill}%
\end{pgfscope}%
\begin{pgfscope}%
\pgfpathrectangle{\pgfqpoint{1.020000in}{0.880000in}}{\pgfqpoint{6.160000in}{6.160000in}}%
\pgfusepath{clip}%
\pgfsetbuttcap%
\pgfsetroundjoin%
\definecolor{currentfill}{rgb}{0.343278,0.459354,0.884122}%
\pgfsetfillcolor{currentfill}%
\pgfsetlinewidth{0.000000pt}%
\definecolor{currentstroke}{rgb}{0.000000,0.000000,0.000000}%
\pgfsetstrokecolor{currentstroke}%
\pgfsetdash{}{0pt}%
\pgfpathmoveto{\pgfqpoint{5.301034in}{3.078876in}}%
\pgfpathlineto{\pgfqpoint{5.312479in}{3.072132in}}%
\pgfpathlineto{\pgfqpoint{5.323926in}{3.063537in}}%
\pgfpathlineto{\pgfqpoint{5.356714in}{3.064944in}}%
\pgfpathlineto{\pgfqpoint{5.389486in}{3.066804in}}%
\pgfpathlineto{\pgfqpoint{5.377992in}{3.076348in}}%
\pgfpathlineto{\pgfqpoint{5.366503in}{3.084273in}}%
\pgfpathlineto{\pgfqpoint{5.333775in}{3.081260in}}%
\pgfpathlineto{\pgfqpoint{5.301034in}{3.078876in}}%
\pgfpathclose%
\pgfusepath{fill}%
\end{pgfscope}%
\begin{pgfscope}%
\pgfpathrectangle{\pgfqpoint{1.020000in}{0.880000in}}{\pgfqpoint{6.160000in}{6.160000in}}%
\pgfusepath{clip}%
\pgfsetbuttcap%
\pgfsetroundjoin%
\definecolor{currentfill}{rgb}{0.961595,0.622247,0.501551}%
\pgfsetfillcolor{currentfill}%
\pgfsetlinewidth{0.000000pt}%
\definecolor{currentstroke}{rgb}{0.000000,0.000000,0.000000}%
\pgfsetstrokecolor{currentstroke}%
\pgfsetdash{}{0pt}%
\pgfpathmoveto{\pgfqpoint{2.327676in}{4.586848in}}%
\pgfpathlineto{\pgfqpoint{2.336463in}{4.565132in}}%
\pgfpathlineto{\pgfqpoint{2.345309in}{4.540746in}}%
\pgfpathlineto{\pgfqpoint{2.378307in}{4.584367in}}%
\pgfpathlineto{\pgfqpoint{2.411287in}{4.629378in}}%
\pgfpathlineto{\pgfqpoint{2.402379in}{4.654093in}}%
\pgfpathlineto{\pgfqpoint{2.393535in}{4.675726in}}%
\pgfpathlineto{\pgfqpoint{2.360614in}{4.630587in}}%
\pgfpathlineto{\pgfqpoint{2.327676in}{4.586848in}}%
\pgfpathclose%
\pgfusepath{fill}%
\end{pgfscope}%
\begin{pgfscope}%
\pgfpathrectangle{\pgfqpoint{1.020000in}{0.880000in}}{\pgfqpoint{6.160000in}{6.160000in}}%
\pgfusepath{clip}%
\pgfsetbuttcap%
\pgfsetroundjoin%
\definecolor{currentfill}{rgb}{0.333490,0.446265,0.874452}%
\pgfsetfillcolor{currentfill}%
\pgfsetlinewidth{0.000000pt}%
\definecolor{currentstroke}{rgb}{0.000000,0.000000,0.000000}%
\pgfsetstrokecolor{currentstroke}%
\pgfsetdash{}{0pt}%
\pgfpathmoveto{\pgfqpoint{5.739641in}{3.062460in}}%
\pgfpathlineto{\pgfqpoint{5.751396in}{3.047253in}}%
\pgfpathlineto{\pgfqpoint{5.763172in}{3.031929in}}%
\pgfpathlineto{\pgfqpoint{5.795824in}{3.033643in}}%
\pgfpathlineto{\pgfqpoint{5.828456in}{3.035354in}}%
\pgfpathlineto{\pgfqpoint{5.816630in}{3.050762in}}%
\pgfpathlineto{\pgfqpoint{5.804826in}{3.066115in}}%
\pgfpathlineto{\pgfqpoint{5.772244in}{3.064311in}}%
\pgfpathlineto{\pgfqpoint{5.739641in}{3.062460in}}%
\pgfpathclose%
\pgfusepath{fill}%
\end{pgfscope}%
\begin{pgfscope}%
\pgfpathrectangle{\pgfqpoint{1.020000in}{0.880000in}}{\pgfqpoint{6.160000in}{6.160000in}}%
\pgfusepath{clip}%
\pgfsetbuttcap%
\pgfsetroundjoin%
\definecolor{currentfill}{rgb}{0.651398,0.768121,0.995891}%
\pgfsetfillcolor{currentfill}%
\pgfsetlinewidth{0.000000pt}%
\definecolor{currentstroke}{rgb}{0.000000,0.000000,0.000000}%
\pgfsetstrokecolor{currentstroke}%
\pgfsetdash{}{0pt}%
\pgfpathmoveto{\pgfqpoint{3.944318in}{3.657230in}}%
\pgfpathlineto{\pgfqpoint{3.954467in}{3.632381in}}%
\pgfpathlineto{\pgfqpoint{3.964629in}{3.609444in}}%
\pgfpathlineto{\pgfqpoint{3.997882in}{3.590437in}}%
\pgfpathlineto{\pgfqpoint{4.031102in}{3.571988in}}%
\pgfpathlineto{\pgfqpoint{4.020898in}{3.593212in}}%
\pgfpathlineto{\pgfqpoint{4.010708in}{3.616102in}}%
\pgfpathlineto{\pgfqpoint{3.977531in}{3.636298in}}%
\pgfpathlineto{\pgfqpoint{3.944318in}{3.657230in}}%
\pgfpathclose%
\pgfusepath{fill}%
\end{pgfscope}%
\begin{pgfscope}%
\pgfpathrectangle{\pgfqpoint{1.020000in}{0.880000in}}{\pgfqpoint{6.160000in}{6.160000in}}%
\pgfusepath{clip}%
\pgfsetbuttcap%
\pgfsetroundjoin%
\definecolor{currentfill}{rgb}{0.409611,0.540759,0.935545}%
\pgfsetfillcolor{currentfill}%
\pgfsetlinewidth{0.000000pt}%
\definecolor{currentstroke}{rgb}{0.000000,0.000000,0.000000}%
\pgfsetstrokecolor{currentstroke}%
\pgfsetdash{}{0pt}%
\pgfpathmoveto{\pgfqpoint{4.709676in}{3.213070in}}%
\pgfpathlineto{\pgfqpoint{4.720531in}{3.204900in}}%
\pgfpathlineto{\pgfqpoint{4.731407in}{3.196471in}}%
\pgfpathlineto{\pgfqpoint{4.764312in}{3.184136in}}%
\pgfpathlineto{\pgfqpoint{4.797189in}{3.171739in}}%
\pgfpathlineto{\pgfqpoint{4.786248in}{3.178141in}}%
\pgfpathlineto{\pgfqpoint{4.775325in}{3.183967in}}%
\pgfpathlineto{\pgfqpoint{4.742517in}{3.198748in}}%
\pgfpathlineto{\pgfqpoint{4.709676in}{3.213070in}}%
\pgfpathclose%
\pgfusepath{fill}%
\end{pgfscope}%
\begin{pgfscope}%
\pgfpathrectangle{\pgfqpoint{1.020000in}{0.880000in}}{\pgfqpoint{6.160000in}{6.160000in}}%
\pgfusepath{clip}%
\pgfsetbuttcap%
\pgfsetroundjoin%
\definecolor{currentfill}{rgb}{0.343278,0.459354,0.884122}%
\pgfsetfillcolor{currentfill}%
\pgfsetlinewidth{0.000000pt}%
\definecolor{currentstroke}{rgb}{0.000000,0.000000,0.000000}%
\pgfsetstrokecolor{currentstroke}%
\pgfsetdash{}{0pt}%
\pgfpathmoveto{\pgfqpoint{5.520403in}{3.076920in}}%
\pgfpathlineto{\pgfqpoint{5.531990in}{3.064119in}}%
\pgfpathlineto{\pgfqpoint{5.543589in}{3.050593in}}%
\pgfpathlineto{\pgfqpoint{5.576315in}{3.052587in}}%
\pgfpathlineto{\pgfqpoint{5.609021in}{3.054607in}}%
\pgfpathlineto{\pgfqpoint{5.597377in}{3.068778in}}%
\pgfpathlineto{\pgfqpoint{5.585748in}{3.082437in}}%
\pgfpathlineto{\pgfqpoint{5.553086in}{3.079696in}}%
\pgfpathlineto{\pgfqpoint{5.520403in}{3.076920in}}%
\pgfpathclose%
\pgfusepath{fill}%
\end{pgfscope}%
\begin{pgfscope}%
\pgfpathrectangle{\pgfqpoint{1.020000in}{0.880000in}}{\pgfqpoint{6.160000in}{6.160000in}}%
\pgfusepath{clip}%
\pgfsetbuttcap%
\pgfsetroundjoin%
\definecolor{currentfill}{rgb}{0.323718,0.433158,0.864722}%
\pgfsetfillcolor{currentfill}%
\pgfsetlinewidth{0.000000pt}%
\definecolor{currentstroke}{rgb}{0.000000,0.000000,0.000000}%
\pgfsetstrokecolor{currentstroke}%
\pgfsetdash{}{0pt}%
\pgfpathmoveto{\pgfqpoint{5.958781in}{3.042175in}}%
\pgfpathlineto{\pgfqpoint{5.970732in}{3.026790in}}%
\pgfpathlineto{\pgfqpoint{5.982706in}{3.011425in}}%
\pgfpathlineto{\pgfqpoint{6.015289in}{3.013203in}}%
\pgfpathlineto{\pgfqpoint{6.003289in}{3.028532in}}%
\pgfpathlineto{\pgfqpoint{5.991312in}{3.043884in}}%
\pgfpathlineto{\pgfqpoint{5.958781in}{3.042175in}}%
\pgfpathclose%
\pgfusepath{fill}%
\end{pgfscope}%
\begin{pgfscope}%
\pgfpathrectangle{\pgfqpoint{1.020000in}{0.880000in}}{\pgfqpoint{6.160000in}{6.160000in}}%
\pgfusepath{clip}%
\pgfsetbuttcap%
\pgfsetroundjoin%
\definecolor{currentfill}{rgb}{0.859385,0.864431,0.872111}%
\pgfsetfillcolor{currentfill}%
\pgfsetlinewidth{0.000000pt}%
\definecolor{currentstroke}{rgb}{0.000000,0.000000,0.000000}%
\pgfsetstrokecolor{currentstroke}%
\pgfsetdash{}{0pt}%
\pgfpathmoveto{\pgfqpoint{3.617200in}{4.100403in}}%
\pgfpathlineto{\pgfqpoint{3.627208in}{4.052413in}}%
\pgfpathlineto{\pgfqpoint{3.637222in}{4.005654in}}%
\pgfpathlineto{\pgfqpoint{3.670691in}{3.975907in}}%
\pgfpathlineto{\pgfqpoint{3.704121in}{3.945917in}}%
\pgfpathlineto{\pgfqpoint{3.694097in}{3.988685in}}%
\pgfpathlineto{\pgfqpoint{3.684080in}{4.032609in}}%
\pgfpathlineto{\pgfqpoint{3.650660in}{4.066606in}}%
\pgfpathlineto{\pgfqpoint{3.617200in}{4.100403in}}%
\pgfpathclose%
\pgfusepath{fill}%
\end{pgfscope}%
\begin{pgfscope}%
\pgfpathrectangle{\pgfqpoint{1.020000in}{0.880000in}}{\pgfqpoint{6.160000in}{6.160000in}}%
\pgfusepath{clip}%
\pgfsetbuttcap%
\pgfsetroundjoin%
\definecolor{currentfill}{rgb}{0.945854,0.559565,0.441513}%
\pgfsetfillcolor{currentfill}%
\pgfsetlinewidth{0.000000pt}%
\definecolor{currentstroke}{rgb}{0.000000,0.000000,0.000000}%
\pgfsetstrokecolor{currentstroke}%
\pgfsetdash{}{0pt}%
\pgfpathmoveto{\pgfqpoint{2.393535in}{4.675726in}}%
\pgfpathlineto{\pgfqpoint{2.402379in}{4.654093in}}%
\pgfpathlineto{\pgfqpoint{2.411287in}{4.629378in}}%
\pgfpathlineto{\pgfqpoint{2.444257in}{4.675449in}}%
\pgfpathlineto{\pgfqpoint{2.477222in}{4.722207in}}%
\pgfpathlineto{\pgfqpoint{2.468252in}{4.747273in}}%
\pgfpathlineto{\pgfqpoint{2.459352in}{4.768829in}}%
\pgfpathlineto{\pgfqpoint{2.426446in}{4.721932in}}%
\pgfpathlineto{\pgfqpoint{2.393535in}{4.675726in}}%
\pgfpathclose%
\pgfusepath{fill}%
\end{pgfscope}%
\begin{pgfscope}%
\pgfpathrectangle{\pgfqpoint{1.020000in}{0.880000in}}{\pgfqpoint{6.160000in}{6.160000in}}%
\pgfusepath{clip}%
\pgfsetbuttcap%
\pgfsetroundjoin%
\definecolor{currentfill}{rgb}{0.922681,0.828568,0.777054}%
\pgfsetfillcolor{currentfill}%
\pgfsetlinewidth{0.000000pt}%
\definecolor{currentstroke}{rgb}{0.000000,0.000000,0.000000}%
\pgfsetstrokecolor{currentstroke}%
\pgfsetdash{}{0pt}%
\pgfpathmoveto{\pgfqpoint{3.530171in}{4.272672in}}%
\pgfpathlineto{\pgfqpoint{3.540162in}{4.219230in}}%
\pgfpathlineto{\pgfqpoint{3.550161in}{4.166267in}}%
\pgfpathlineto{\pgfqpoint{3.583699in}{4.133720in}}%
\pgfpathlineto{\pgfqpoint{3.617200in}{4.100403in}}%
\pgfpathlineto{\pgfqpoint{3.607198in}{4.149259in}}%
\pgfpathlineto{\pgfqpoint{3.597206in}{4.198601in}}%
\pgfpathlineto{\pgfqpoint{3.563708in}{4.236041in}}%
\pgfpathlineto{\pgfqpoint{3.530171in}{4.272672in}}%
\pgfpathclose%
\pgfusepath{fill}%
\end{pgfscope}%
\begin{pgfscope}%
\pgfpathrectangle{\pgfqpoint{1.020000in}{0.880000in}}{\pgfqpoint{6.160000in}{6.160000in}}%
\pgfusepath{clip}%
\pgfsetbuttcap%
\pgfsetroundjoin%
\definecolor{currentfill}{rgb}{0.378598,0.503856,0.913692}%
\pgfsetfillcolor{currentfill}%
\pgfsetlinewidth{0.000000pt}%
\definecolor{currentstroke}{rgb}{0.000000,0.000000,0.000000}%
\pgfsetstrokecolor{currentstroke}%
\pgfsetdash{}{0pt}%
\pgfpathmoveto{\pgfqpoint{4.862854in}{3.147029in}}%
\pgfpathlineto{\pgfqpoint{4.873882in}{3.142234in}}%
\pgfpathlineto{\pgfqpoint{4.884925in}{3.136253in}}%
\pgfpathlineto{\pgfqpoint{4.917789in}{3.126618in}}%
\pgfpathlineto{\pgfqpoint{4.950629in}{3.117535in}}%
\pgfpathlineto{\pgfqpoint{4.939514in}{3.121276in}}%
\pgfpathlineto{\pgfqpoint{4.928410in}{3.123401in}}%
\pgfpathlineto{\pgfqpoint{4.895645in}{3.134983in}}%
\pgfpathlineto{\pgfqpoint{4.862854in}{3.147029in}}%
\pgfpathclose%
\pgfusepath{fill}%
\end{pgfscope}%
\begin{pgfscope}%
\pgfpathrectangle{\pgfqpoint{1.020000in}{0.880000in}}{\pgfqpoint{6.160000in}{6.160000in}}%
\pgfusepath{clip}%
\pgfsetbuttcap%
\pgfsetroundjoin%
\definecolor{currentfill}{rgb}{0.791392,0.846750,0.936641}%
\pgfsetfillcolor{currentfill}%
\pgfsetlinewidth{0.000000pt}%
\definecolor{currentstroke}{rgb}{0.000000,0.000000,0.000000}%
\pgfsetstrokecolor{currentstroke}%
\pgfsetdash{}{0pt}%
\pgfpathmoveto{\pgfqpoint{3.704121in}{3.945917in}}%
\pgfpathlineto{\pgfqpoint{3.714152in}{3.904614in}}%
\pgfpathlineto{\pgfqpoint{3.724188in}{3.865067in}}%
\pgfpathlineto{\pgfqpoint{3.757596in}{3.838764in}}%
\pgfpathlineto{\pgfqpoint{3.790966in}{3.812560in}}%
\pgfpathlineto{\pgfqpoint{3.780909in}{3.848542in}}%
\pgfpathlineto{\pgfqpoint{3.770861in}{3.886128in}}%
\pgfpathlineto{\pgfqpoint{3.737511in}{3.915917in}}%
\pgfpathlineto{\pgfqpoint{3.704121in}{3.945917in}}%
\pgfpathclose%
\pgfusepath{fill}%
\end{pgfscope}%
\begin{pgfscope}%
\pgfpathrectangle{\pgfqpoint{1.020000in}{0.880000in}}{\pgfqpoint{6.160000in}{6.160000in}}%
\pgfusepath{clip}%
\pgfsetbuttcap%
\pgfsetroundjoin%
\definecolor{currentfill}{rgb}{0.717435,0.051118,0.158737}%
\pgfsetfillcolor{currentfill}%
\pgfsetlinewidth{0.000000pt}%
\definecolor{currentstroke}{rgb}{0.000000,0.000000,0.000000}%
\pgfsetstrokecolor{currentstroke}%
\pgfsetdash{}{0pt}%
\pgfpathmoveto{\pgfqpoint{2.856170in}{5.218470in}}%
\pgfpathlineto{\pgfqpoint{2.865402in}{5.197529in}}%
\pgfpathlineto{\pgfqpoint{2.874715in}{5.171001in}}%
\pgfpathlineto{\pgfqpoint{2.908120in}{5.185223in}}%
\pgfpathlineto{\pgfqpoint{2.941571in}{5.194209in}}%
\pgfpathlineto{\pgfqpoint{2.932209in}{5.220689in}}%
\pgfpathlineto{\pgfqpoint{2.922927in}{5.241459in}}%
\pgfpathlineto{\pgfqpoint{2.889525in}{5.232600in}}%
\pgfpathlineto{\pgfqpoint{2.856170in}{5.218470in}}%
\pgfpathclose%
\pgfusepath{fill}%
\end{pgfscope}%
\begin{pgfscope}%
\pgfpathrectangle{\pgfqpoint{1.020000in}{0.880000in}}{\pgfqpoint{6.160000in}{6.160000in}}%
\pgfusepath{clip}%
\pgfsetbuttcap%
\pgfsetroundjoin%
\definecolor{currentfill}{rgb}{0.921406,0.491420,0.383408}%
\pgfsetfillcolor{currentfill}%
\pgfsetlinewidth{0.000000pt}%
\definecolor{currentstroke}{rgb}{0.000000,0.000000,0.000000}%
\pgfsetstrokecolor{currentstroke}%
\pgfsetdash{}{0pt}%
\pgfpathmoveto{\pgfqpoint{2.459352in}{4.768829in}}%
\pgfpathlineto{\pgfqpoint{2.468252in}{4.747273in}}%
\pgfpathlineto{\pgfqpoint{2.477222in}{4.722207in}}%
\pgfpathlineto{\pgfqpoint{2.510190in}{4.769248in}}%
\pgfpathlineto{\pgfqpoint{2.543168in}{4.816133in}}%
\pgfpathlineto{\pgfqpoint{2.534137in}{4.841553in}}%
\pgfpathlineto{\pgfqpoint{2.525179in}{4.863034in}}%
\pgfpathlineto{\pgfqpoint{2.492261in}{4.816010in}}%
\pgfpathlineto{\pgfqpoint{2.459352in}{4.768829in}}%
\pgfpathclose%
\pgfusepath{fill}%
\end{pgfscope}%
\begin{pgfscope}%
\pgfpathrectangle{\pgfqpoint{1.020000in}{0.880000in}}{\pgfqpoint{6.160000in}{6.160000in}}%
\pgfusepath{clip}%
\pgfsetbuttcap%
\pgfsetroundjoin%
\definecolor{currentfill}{rgb}{0.962708,0.753557,0.655601}%
\pgfsetfillcolor{currentfill}%
\pgfsetlinewidth{0.000000pt}%
\definecolor{currentstroke}{rgb}{0.000000,0.000000,0.000000}%
\pgfsetstrokecolor{currentstroke}%
\pgfsetdash{}{0pt}%
\pgfpathmoveto{\pgfqpoint{3.443035in}{4.456453in}}%
\pgfpathlineto{\pgfqpoint{3.453005in}{4.399554in}}%
\pgfpathlineto{\pgfqpoint{3.462989in}{4.342175in}}%
\pgfpathlineto{\pgfqpoint{3.496597in}{4.308162in}}%
\pgfpathlineto{\pgfqpoint{3.530171in}{4.272672in}}%
\pgfpathlineto{\pgfqpoint{3.520190in}{4.326162in}}%
\pgfpathlineto{\pgfqpoint{3.510222in}{4.379259in}}%
\pgfpathlineto{\pgfqpoint{3.476646in}{4.418656in}}%
\pgfpathlineto{\pgfqpoint{3.443035in}{4.456453in}}%
\pgfpathclose%
\pgfusepath{fill}%
\end{pgfscope}%
\begin{pgfscope}%
\pgfpathrectangle{\pgfqpoint{1.020000in}{0.880000in}}{\pgfqpoint{6.160000in}{6.160000in}}%
\pgfusepath{clip}%
\pgfsetbuttcap%
\pgfsetroundjoin%
\definecolor{currentfill}{rgb}{0.733898,0.820018,0.970724}%
\pgfsetfillcolor{currentfill}%
\pgfsetlinewidth{0.000000pt}%
\definecolor{currentstroke}{rgb}{0.000000,0.000000,0.000000}%
\pgfsetstrokecolor{currentstroke}%
\pgfsetdash{}{0pt}%
\pgfpathmoveto{\pgfqpoint{3.790966in}{3.812560in}}%
\pgfpathlineto{\pgfqpoint{3.801030in}{3.778421in}}%
\pgfpathlineto{\pgfqpoint{3.811102in}{3.746340in}}%
\pgfpathlineto{\pgfqpoint{3.844461in}{3.723497in}}%
\pgfpathlineto{\pgfqpoint{3.877783in}{3.700942in}}%
\pgfpathlineto{\pgfqpoint{3.867681in}{3.730128in}}%
\pgfpathlineto{\pgfqpoint{3.857589in}{3.761161in}}%
\pgfpathlineto{\pgfqpoint{3.824297in}{3.786637in}}%
\pgfpathlineto{\pgfqpoint{3.790966in}{3.812560in}}%
\pgfpathclose%
\pgfusepath{fill}%
\end{pgfscope}%
\begin{pgfscope}%
\pgfpathrectangle{\pgfqpoint{1.020000in}{0.880000in}}{\pgfqpoint{6.160000in}{6.160000in}}%
\pgfusepath{clip}%
\pgfsetbuttcap%
\pgfsetroundjoin%
\definecolor{currentfill}{rgb}{0.323718,0.433158,0.864722}%
\pgfsetfillcolor{currentfill}%
\pgfsetlinewidth{0.000000pt}%
\definecolor{currentstroke}{rgb}{0.000000,0.000000,0.000000}%
\pgfsetstrokecolor{currentstroke}%
\pgfsetdash{}{0pt}%
\pgfpathmoveto{\pgfqpoint{5.893659in}{3.038766in}}%
\pgfpathlineto{\pgfqpoint{5.905558in}{3.023335in}}%
\pgfpathlineto{\pgfqpoint{5.917480in}{3.007910in}}%
\pgfpathlineto{\pgfqpoint{5.950103in}{3.009661in}}%
\pgfpathlineto{\pgfqpoint{5.982706in}{3.011425in}}%
\pgfpathlineto{\pgfqpoint{5.970732in}{3.026790in}}%
\pgfpathlineto{\pgfqpoint{5.958781in}{3.042175in}}%
\pgfpathlineto{\pgfqpoint{5.926230in}{3.040469in}}%
\pgfpathlineto{\pgfqpoint{5.893659in}{3.038766in}}%
\pgfpathclose%
\pgfusepath{fill}%
\end{pgfscope}%
\begin{pgfscope}%
\pgfpathrectangle{\pgfqpoint{1.020000in}{0.880000in}}{\pgfqpoint{6.160000in}{6.160000in}}%
\pgfusepath{clip}%
\pgfsetbuttcap%
\pgfsetroundjoin%
\definecolor{currentfill}{rgb}{0.348323,0.465711,0.888346}%
\pgfsetfillcolor{currentfill}%
\pgfsetlinewidth{0.000000pt}%
\definecolor{currentstroke}{rgb}{0.000000,0.000000,0.000000}%
\pgfsetstrokecolor{currentstroke}%
\pgfsetdash{}{0pt}%
\pgfpathmoveto{\pgfqpoint{5.235515in}{3.076484in}}%
\pgfpathlineto{\pgfqpoint{5.246909in}{3.070440in}}%
\pgfpathlineto{\pgfqpoint{5.258307in}{3.062416in}}%
\pgfpathlineto{\pgfqpoint{5.291124in}{3.062668in}}%
\pgfpathlineto{\pgfqpoint{5.323926in}{3.063537in}}%
\pgfpathlineto{\pgfqpoint{5.312479in}{3.072132in}}%
\pgfpathlineto{\pgfqpoint{5.301034in}{3.078876in}}%
\pgfpathlineto{\pgfqpoint{5.268280in}{3.077247in}}%
\pgfpathlineto{\pgfqpoint{5.235515in}{3.076484in}}%
\pgfpathclose%
\pgfusepath{fill}%
\end{pgfscope}%
\begin{pgfscope}%
\pgfpathrectangle{\pgfqpoint{1.020000in}{0.880000in}}{\pgfqpoint{6.160000in}{6.160000in}}%
\pgfusepath{clip}%
\pgfsetbuttcap%
\pgfsetroundjoin%
\definecolor{currentfill}{rgb}{0.884643,0.410017,0.322507}%
\pgfsetfillcolor{currentfill}%
\pgfsetlinewidth{0.000000pt}%
\definecolor{currentstroke}{rgb}{0.000000,0.000000,0.000000}%
\pgfsetstrokecolor{currentstroke}%
\pgfsetdash{}{0pt}%
\pgfpathmoveto{\pgfqpoint{2.525179in}{4.863034in}}%
\pgfpathlineto{\pgfqpoint{2.534137in}{4.841553in}}%
\pgfpathlineto{\pgfqpoint{2.543168in}{4.816133in}}%
\pgfpathlineto{\pgfqpoint{2.576163in}{4.862399in}}%
\pgfpathlineto{\pgfqpoint{2.609180in}{4.907563in}}%
\pgfpathlineto{\pgfqpoint{2.600088in}{4.933322in}}%
\pgfpathlineto{\pgfqpoint{2.591074in}{4.954723in}}%
\pgfpathlineto{\pgfqpoint{2.558115in}{4.909434in}}%
\pgfpathlineto{\pgfqpoint{2.525179in}{4.863034in}}%
\pgfpathclose%
\pgfusepath{fill}%
\end{pgfscope}%
\begin{pgfscope}%
\pgfpathrectangle{\pgfqpoint{1.020000in}{0.880000in}}{\pgfqpoint{6.160000in}{6.160000in}}%
\pgfusepath{clip}%
\pgfsetbuttcap%
\pgfsetroundjoin%
\definecolor{currentfill}{rgb}{0.752704,0.157576,0.184258}%
\pgfsetfillcolor{currentfill}%
\pgfsetlinewidth{0.000000pt}%
\definecolor{currentstroke}{rgb}{0.000000,0.000000,0.000000}%
\pgfsetstrokecolor{currentstroke}%
\pgfsetdash{}{0pt}%
\pgfpathmoveto{\pgfqpoint{3.008601in}{5.195865in}}%
\pgfpathlineto{\pgfqpoint{3.018083in}{5.163926in}}%
\pgfpathlineto{\pgfqpoint{3.027636in}{5.126554in}}%
\pgfpathlineto{\pgfqpoint{3.061244in}{5.119408in}}%
\pgfpathlineto{\pgfqpoint{3.094879in}{5.106895in}}%
\pgfpathlineto{\pgfqpoint{3.085291in}{5.143858in}}%
\pgfpathlineto{\pgfqpoint{3.075771in}{5.175449in}}%
\pgfpathlineto{\pgfqpoint{3.042171in}{5.188405in}}%
\pgfpathlineto{\pgfqpoint{3.008601in}{5.195865in}}%
\pgfpathclose%
\pgfusepath{fill}%
\end{pgfscope}%
\begin{pgfscope}%
\pgfpathrectangle{\pgfqpoint{1.020000in}{0.880000in}}{\pgfqpoint{6.160000in}{6.160000in}}%
\pgfusepath{clip}%
\pgfsetbuttcap%
\pgfsetroundjoin%
\definecolor{currentfill}{rgb}{0.333490,0.446265,0.874452}%
\pgfsetfillcolor{currentfill}%
\pgfsetlinewidth{0.000000pt}%
\definecolor{currentstroke}{rgb}{0.000000,0.000000,0.000000}%
\pgfsetstrokecolor{currentstroke}%
\pgfsetdash{}{0pt}%
\pgfpathmoveto{\pgfqpoint{5.674372in}{3.058610in}}%
\pgfpathlineto{\pgfqpoint{5.686080in}{3.043657in}}%
\pgfpathlineto{\pgfqpoint{5.697806in}{3.028499in}}%
\pgfpathlineto{\pgfqpoint{5.730499in}{3.030213in}}%
\pgfpathlineto{\pgfqpoint{5.763172in}{3.031929in}}%
\pgfpathlineto{\pgfqpoint{5.751396in}{3.047253in}}%
\pgfpathlineto{\pgfqpoint{5.739641in}{3.062460in}}%
\pgfpathlineto{\pgfqpoint{5.707017in}{3.060559in}}%
\pgfpathlineto{\pgfqpoint{5.674372in}{3.058610in}}%
\pgfpathclose%
\pgfusepath{fill}%
\end{pgfscope}%
\begin{pgfscope}%
\pgfpathrectangle{\pgfqpoint{1.020000in}{0.880000in}}{\pgfqpoint{6.160000in}{6.160000in}}%
\pgfusepath{clip}%
\pgfsetbuttcap%
\pgfsetroundjoin%
\definecolor{currentfill}{rgb}{0.343278,0.459354,0.884122}%
\pgfsetfillcolor{currentfill}%
\pgfsetlinewidth{0.000000pt}%
\definecolor{currentstroke}{rgb}{0.000000,0.000000,0.000000}%
\pgfsetstrokecolor{currentstroke}%
\pgfsetdash{}{0pt}%
\pgfpathmoveto{\pgfqpoint{5.454980in}{3.071507in}}%
\pgfpathlineto{\pgfqpoint{5.466524in}{3.059651in}}%
\pgfpathlineto{\pgfqpoint{5.478079in}{3.046841in}}%
\pgfpathlineto{\pgfqpoint{5.510843in}{3.048662in}}%
\pgfpathlineto{\pgfqpoint{5.543589in}{3.050593in}}%
\pgfpathlineto{\pgfqpoint{5.531990in}{3.064119in}}%
\pgfpathlineto{\pgfqpoint{5.520403in}{3.076920in}}%
\pgfpathlineto{\pgfqpoint{5.487701in}{3.074167in}}%
\pgfpathlineto{\pgfqpoint{5.454980in}{3.071507in}}%
\pgfpathclose%
\pgfusepath{fill}%
\end{pgfscope}%
\begin{pgfscope}%
\pgfpathrectangle{\pgfqpoint{1.020000in}{0.880000in}}{\pgfqpoint{6.160000in}{6.160000in}}%
\pgfusepath{clip}%
\pgfsetbuttcap%
\pgfsetroundjoin%
\definecolor{currentfill}{rgb}{0.559747,0.694768,0.996075}%
\pgfsetfillcolor{currentfill}%
\pgfsetlinewidth{0.000000pt}%
\definecolor{currentstroke}{rgb}{0.000000,0.000000,0.000000}%
\pgfsetstrokecolor{currentstroke}%
\pgfsetdash{}{0pt}%
\pgfpathmoveto{\pgfqpoint{4.184294in}{3.471194in}}%
\pgfpathlineto{\pgfqpoint{4.194645in}{3.456410in}}%
\pgfpathlineto{\pgfqpoint{4.205014in}{3.443034in}}%
\pgfpathlineto{\pgfqpoint{4.238172in}{3.428645in}}%
\pgfpathlineto{\pgfqpoint{4.271299in}{3.414756in}}%
\pgfpathlineto{\pgfqpoint{4.260878in}{3.428183in}}%
\pgfpathlineto{\pgfqpoint{4.250476in}{3.442795in}}%
\pgfpathlineto{\pgfqpoint{4.217400in}{3.456715in}}%
\pgfpathlineto{\pgfqpoint{4.184294in}{3.471194in}}%
\pgfpathclose%
\pgfusepath{fill}%
\end{pgfscope}%
\begin{pgfscope}%
\pgfpathrectangle{\pgfqpoint{1.020000in}{0.880000in}}{\pgfqpoint{6.160000in}{6.160000in}}%
\pgfusepath{clip}%
\pgfsetbuttcap%
\pgfsetroundjoin%
\definecolor{currentfill}{rgb}{0.358415,0.478426,0.896795}%
\pgfsetfillcolor{currentfill}%
\pgfsetlinewidth{0.000000pt}%
\definecolor{currentstroke}{rgb}{0.000000,0.000000,0.000000}%
\pgfsetstrokecolor{currentstroke}%
\pgfsetdash{}{0pt}%
\pgfpathmoveto{\pgfqpoint{5.016245in}{3.101601in}}%
\pgfpathlineto{\pgfqpoint{5.027442in}{3.098050in}}%
\pgfpathlineto{\pgfqpoint{5.038646in}{3.092554in}}%
\pgfpathlineto{\pgfqpoint{5.071496in}{3.087451in}}%
\pgfpathlineto{\pgfqpoint{5.104328in}{3.083265in}}%
\pgfpathlineto{\pgfqpoint{5.093057in}{3.087597in}}%
\pgfpathlineto{\pgfqpoint{5.081789in}{3.089687in}}%
\pgfpathlineto{\pgfqpoint{5.049025in}{3.095065in}}%
\pgfpathlineto{\pgfqpoint{5.016245in}{3.101601in}}%
\pgfpathclose%
\pgfusepath{fill}%
\end{pgfscope}%
\begin{pgfscope}%
\pgfpathrectangle{\pgfqpoint{1.020000in}{0.880000in}}{\pgfqpoint{6.160000in}{6.160000in}}%
\pgfusepath{clip}%
\pgfsetbuttcap%
\pgfsetroundjoin%
\definecolor{currentfill}{rgb}{0.740957,0.122240,0.175744}%
\pgfsetfillcolor{currentfill}%
\pgfsetlinewidth{0.000000pt}%
\definecolor{currentstroke}{rgb}{0.000000,0.000000,0.000000}%
\pgfsetstrokecolor{currentstroke}%
\pgfsetdash{}{0pt}%
\pgfpathmoveto{\pgfqpoint{2.789612in}{5.175394in}}%
\pgfpathlineto{\pgfqpoint{2.798791in}{5.154307in}}%
\pgfpathlineto{\pgfqpoint{2.808053in}{5.127843in}}%
\pgfpathlineto{\pgfqpoint{2.841360in}{5.151777in}}%
\pgfpathlineto{\pgfqpoint{2.874715in}{5.171001in}}%
\pgfpathlineto{\pgfqpoint{2.865402in}{5.197529in}}%
\pgfpathlineto{\pgfqpoint{2.856170in}{5.218470in}}%
\pgfpathlineto{\pgfqpoint{2.822865in}{5.199303in}}%
\pgfpathlineto{\pgfqpoint{2.789612in}{5.175394in}}%
\pgfpathclose%
\pgfusepath{fill}%
\end{pgfscope}%
\begin{pgfscope}%
\pgfpathrectangle{\pgfqpoint{1.020000in}{0.880000in}}{\pgfqpoint{6.160000in}{6.160000in}}%
\pgfusepath{clip}%
\pgfsetbuttcap%
\pgfsetroundjoin%
\definecolor{currentfill}{rgb}{0.510824,0.649397,0.985079}%
\pgfsetfillcolor{currentfill}%
\pgfsetlinewidth{0.000000pt}%
\definecolor{currentstroke}{rgb}{0.000000,0.000000,0.000000}%
\pgfsetstrokecolor{currentstroke}%
\pgfsetdash{}{0pt}%
\pgfpathmoveto{\pgfqpoint{4.337465in}{3.388317in}}%
\pgfpathlineto{\pgfqpoint{4.347959in}{3.375782in}}%
\pgfpathlineto{\pgfqpoint{4.358474in}{3.364226in}}%
\pgfpathlineto{\pgfqpoint{4.391565in}{3.351198in}}%
\pgfpathlineto{\pgfqpoint{4.424627in}{3.338457in}}%
\pgfpathlineto{\pgfqpoint{4.414060in}{3.350370in}}%
\pgfpathlineto{\pgfqpoint{4.403514in}{3.363091in}}%
\pgfpathlineto{\pgfqpoint{4.370504in}{3.375611in}}%
\pgfpathlineto{\pgfqpoint{4.337465in}{3.388317in}}%
\pgfpathclose%
\pgfusepath{fill}%
\end{pgfscope}%
\begin{pgfscope}%
\pgfpathrectangle{\pgfqpoint{1.020000in}{0.880000in}}{\pgfqpoint{6.160000in}{6.160000in}}%
\pgfusepath{clip}%
\pgfsetbuttcap%
\pgfsetroundjoin%
\definecolor{currentfill}{rgb}{0.966922,0.651969,0.531997}%
\pgfsetfillcolor{currentfill}%
\pgfsetlinewidth{0.000000pt}%
\definecolor{currentstroke}{rgb}{0.000000,0.000000,0.000000}%
\pgfsetstrokecolor{currentstroke}%
\pgfsetdash{}{0pt}%
\pgfpathmoveto{\pgfqpoint{3.355834in}{4.642806in}}%
\pgfpathlineto{\pgfqpoint{3.365769in}{4.585085in}}%
\pgfpathlineto{\pgfqpoint{3.375726in}{4.525780in}}%
\pgfpathlineto{\pgfqpoint{3.409394in}{4.492282in}}%
\pgfpathlineto{\pgfqpoint{3.443035in}{4.456453in}}%
\pgfpathlineto{\pgfqpoint{3.433082in}{4.512395in}}%
\pgfpathlineto{\pgfqpoint{3.423148in}{4.566901in}}%
\pgfpathlineto{\pgfqpoint{3.389504in}{4.606114in}}%
\pgfpathlineto{\pgfqpoint{3.355834in}{4.642806in}}%
\pgfpathclose%
\pgfusepath{fill}%
\end{pgfscope}%
\begin{pgfscope}%
\pgfpathrectangle{\pgfqpoint{1.020000in}{0.880000in}}{\pgfqpoint{6.160000in}{6.160000in}}%
\pgfusepath{clip}%
\pgfsetbuttcap%
\pgfsetroundjoin%
\definecolor{currentfill}{rgb}{0.613933,0.739923,0.999142}%
\pgfsetfillcolor{currentfill}%
\pgfsetlinewidth{0.000000pt}%
\definecolor{currentstroke}{rgb}{0.000000,0.000000,0.000000}%
\pgfsetstrokecolor{currentstroke}%
\pgfsetdash{}{0pt}%
\pgfpathmoveto{\pgfqpoint{4.031102in}{3.571988in}}%
\pgfpathlineto{\pgfqpoint{4.041321in}{3.552524in}}%
\pgfpathlineto{\pgfqpoint{4.051555in}{3.534897in}}%
\pgfpathlineto{\pgfqpoint{4.084788in}{3.518145in}}%
\pgfpathlineto{\pgfqpoint{4.117988in}{3.501920in}}%
\pgfpathlineto{\pgfqpoint{4.107706in}{3.518681in}}%
\pgfpathlineto{\pgfqpoint{4.097440in}{3.537023in}}%
\pgfpathlineto{\pgfqpoint{4.064288in}{3.554166in}}%
\pgfpathlineto{\pgfqpoint{4.031102in}{3.571988in}}%
\pgfpathclose%
\pgfusepath{fill}%
\end{pgfscope}%
\begin{pgfscope}%
\pgfpathrectangle{\pgfqpoint{1.020000in}{0.880000in}}{\pgfqpoint{6.160000in}{6.160000in}}%
\pgfusepath{clip}%
\pgfsetbuttcap%
\pgfsetroundjoin%
\definecolor{currentfill}{rgb}{0.848040,0.338280,0.275206}%
\pgfsetfillcolor{currentfill}%
\pgfsetlinewidth{0.000000pt}%
\definecolor{currentstroke}{rgb}{0.000000,0.000000,0.000000}%
\pgfsetstrokecolor{currentstroke}%
\pgfsetdash{}{0pt}%
\pgfpathmoveto{\pgfqpoint{2.591074in}{4.954723in}}%
\pgfpathlineto{\pgfqpoint{2.600088in}{4.933322in}}%
\pgfpathlineto{\pgfqpoint{2.609180in}{4.907563in}}%
\pgfpathlineto{\pgfqpoint{2.642226in}{4.951130in}}%
\pgfpathlineto{\pgfqpoint{2.675306in}{4.992601in}}%
\pgfpathlineto{\pgfqpoint{2.666155in}{5.018661in}}%
\pgfpathlineto{\pgfqpoint{2.657085in}{5.039974in}}%
\pgfpathlineto{\pgfqpoint{2.624062in}{4.998404in}}%
\pgfpathlineto{\pgfqpoint{2.591074in}{4.954723in}}%
\pgfpathclose%
\pgfusepath{fill}%
\end{pgfscope}%
\begin{pgfscope}%
\pgfpathrectangle{\pgfqpoint{1.020000in}{0.880000in}}{\pgfqpoint{6.160000in}{6.160000in}}%
\pgfusepath{clip}%
\pgfsetbuttcap%
\pgfsetroundjoin%
\definecolor{currentfill}{rgb}{0.467678,0.605591,0.968546}%
\pgfsetfillcolor{currentfill}%
\pgfsetlinewidth{0.000000pt}%
\definecolor{currentstroke}{rgb}{0.000000,0.000000,0.000000}%
\pgfsetstrokecolor{currentstroke}%
\pgfsetdash{}{0pt}%
\pgfpathmoveto{\pgfqpoint{4.490663in}{3.313345in}}%
\pgfpathlineto{\pgfqpoint{4.501304in}{3.302000in}}%
\pgfpathlineto{\pgfqpoint{4.511967in}{3.291212in}}%
\pgfpathlineto{\pgfqpoint{4.544994in}{3.278615in}}%
\pgfpathlineto{\pgfqpoint{4.577992in}{3.266016in}}%
\pgfpathlineto{\pgfqpoint{4.567275in}{3.276651in}}%
\pgfpathlineto{\pgfqpoint{4.556580in}{3.287677in}}%
\pgfpathlineto{\pgfqpoint{4.523637in}{3.300668in}}%
\pgfpathlineto{\pgfqpoint{4.490663in}{3.313345in}}%
\pgfpathclose%
\pgfusepath{fill}%
\end{pgfscope}%
\begin{pgfscope}%
\pgfpathrectangle{\pgfqpoint{1.020000in}{0.880000in}}{\pgfqpoint{6.160000in}{6.160000in}}%
\pgfusepath{clip}%
\pgfsetbuttcap%
\pgfsetroundjoin%
\definecolor{currentfill}{rgb}{0.768929,0.189213,0.197965}%
\pgfsetfillcolor{currentfill}%
\pgfsetlinewidth{0.000000pt}%
\definecolor{currentstroke}{rgb}{0.000000,0.000000,0.000000}%
\pgfsetstrokecolor{currentstroke}%
\pgfsetdash{}{0pt}%
\pgfpathmoveto{\pgfqpoint{2.723255in}{5.114794in}}%
\pgfpathlineto{\pgfqpoint{2.732380in}{5.093584in}}%
\pgfpathlineto{\pgfqpoint{2.741588in}{5.067282in}}%
\pgfpathlineto{\pgfqpoint{2.774797in}{5.099547in}}%
\pgfpathlineto{\pgfqpoint{2.808053in}{5.127843in}}%
\pgfpathlineto{\pgfqpoint{2.798791in}{5.154307in}}%
\pgfpathlineto{\pgfqpoint{2.789612in}{5.175394in}}%
\pgfpathlineto{\pgfqpoint{2.756408in}{5.147092in}}%
\pgfpathlineto{\pgfqpoint{2.723255in}{5.114794in}}%
\pgfpathclose%
\pgfusepath{fill}%
\end{pgfscope}%
\begin{pgfscope}%
\pgfpathrectangle{\pgfqpoint{1.020000in}{0.880000in}}{\pgfqpoint{6.160000in}{6.160000in}}%
\pgfusepath{clip}%
\pgfsetbuttcap%
\pgfsetroundjoin%
\definecolor{currentfill}{rgb}{0.805723,0.259813,0.230562}%
\pgfsetfillcolor{currentfill}%
\pgfsetlinewidth{0.000000pt}%
\definecolor{currentstroke}{rgb}{0.000000,0.000000,0.000000}%
\pgfsetstrokecolor{currentstroke}%
\pgfsetdash{}{0pt}%
\pgfpathmoveto{\pgfqpoint{2.657085in}{5.039974in}}%
\pgfpathlineto{\pgfqpoint{2.666155in}{5.018661in}}%
\pgfpathlineto{\pgfqpoint{2.675306in}{4.992601in}}%
\pgfpathlineto{\pgfqpoint{2.708426in}{5.031479in}}%
\pgfpathlineto{\pgfqpoint{2.741588in}{5.067282in}}%
\pgfpathlineto{\pgfqpoint{2.732380in}{5.093584in}}%
\pgfpathlineto{\pgfqpoint{2.723255in}{5.114794in}}%
\pgfpathlineto{\pgfqpoint{2.690148in}{5.078933in}}%
\pgfpathlineto{\pgfqpoint{2.657085in}{5.039974in}}%
\pgfpathclose%
\pgfusepath{fill}%
\end{pgfscope}%
\begin{pgfscope}%
\pgfpathrectangle{\pgfqpoint{1.020000in}{0.880000in}}{\pgfqpoint{6.160000in}{6.160000in}}%
\pgfusepath{clip}%
\pgfsetbuttcap%
\pgfsetroundjoin%
\definecolor{currentfill}{rgb}{0.430507,0.564883,0.948889}%
\pgfsetfillcolor{currentfill}%
\pgfsetlinewidth{0.000000pt}%
\definecolor{currentstroke}{rgb}{0.000000,0.000000,0.000000}%
\pgfsetstrokecolor{currentstroke}%
\pgfsetdash{}{0pt}%
\pgfpathmoveto{\pgfqpoint{4.643897in}{3.240281in}}%
\pgfpathlineto{\pgfqpoint{4.654691in}{3.230621in}}%
\pgfpathlineto{\pgfqpoint{4.665506in}{3.220972in}}%
\pgfpathlineto{\pgfqpoint{4.698471in}{3.208739in}}%
\pgfpathlineto{\pgfqpoint{4.731407in}{3.196471in}}%
\pgfpathlineto{\pgfqpoint{4.720531in}{3.204900in}}%
\pgfpathlineto{\pgfqpoint{4.709676in}{3.213070in}}%
\pgfpathlineto{\pgfqpoint{4.676802in}{3.226903in}}%
\pgfpathlineto{\pgfqpoint{4.643897in}{3.240281in}}%
\pgfpathclose%
\pgfusepath{fill}%
\end{pgfscope}%
\begin{pgfscope}%
\pgfpathrectangle{\pgfqpoint{1.020000in}{0.880000in}}{\pgfqpoint{6.160000in}{6.160000in}}%
\pgfusepath{clip}%
\pgfsetbuttcap%
\pgfsetroundjoin%
\definecolor{currentfill}{rgb}{0.936780,0.532750,0.418093}%
\pgfsetfillcolor{currentfill}%
\pgfsetlinewidth{0.000000pt}%
\definecolor{currentstroke}{rgb}{0.000000,0.000000,0.000000}%
\pgfsetstrokecolor{currentstroke}%
\pgfsetdash{}{0pt}%
\pgfpathmoveto{\pgfqpoint{3.268652in}{4.820879in}}%
\pgfpathlineto{\pgfqpoint{3.278528in}{4.765381in}}%
\pgfpathlineto{\pgfqpoint{3.288435in}{4.707133in}}%
\pgfpathlineto{\pgfqpoint{3.322143in}{4.676597in}}%
\pgfpathlineto{\pgfqpoint{3.355834in}{4.642806in}}%
\pgfpathlineto{\pgfqpoint{3.345925in}{4.698442in}}%
\pgfpathlineto{\pgfqpoint{3.336045in}{4.751510in}}%
\pgfpathlineto{\pgfqpoint{3.302356in}{4.787939in}}%
\pgfpathlineto{\pgfqpoint{3.268652in}{4.820879in}}%
\pgfpathclose%
\pgfusepath{fill}%
\end{pgfscope}%
\begin{pgfscope}%
\pgfpathrectangle{\pgfqpoint{1.020000in}{0.880000in}}{\pgfqpoint{6.160000in}{6.160000in}}%
\pgfusepath{clip}%
\pgfsetbuttcap%
\pgfsetroundjoin%
\definecolor{currentfill}{rgb}{0.810616,0.268797,0.235428}%
\pgfsetfillcolor{currentfill}%
\pgfsetlinewidth{0.000000pt}%
\definecolor{currentstroke}{rgb}{0.000000,0.000000,0.000000}%
\pgfsetstrokecolor{currentstroke}%
\pgfsetdash{}{0pt}%
\pgfpathmoveto{\pgfqpoint{3.094879in}{5.106895in}}%
\pgfpathlineto{\pgfqpoint{3.104529in}{5.064818in}}%
\pgfpathlineto{\pgfqpoint{3.114239in}{5.017950in}}%
\pgfpathlineto{\pgfqpoint{3.147921in}{5.000989in}}%
\pgfpathlineto{\pgfqpoint{3.181616in}{4.979139in}}%
\pgfpathlineto{\pgfqpoint{3.171884in}{5.025027in}}%
\pgfpathlineto{\pgfqpoint{3.162207in}{5.066259in}}%
\pgfpathlineto{\pgfqpoint{3.128535in}{5.089121in}}%
\pgfpathlineto{\pgfqpoint{3.094879in}{5.106895in}}%
\pgfpathclose%
\pgfusepath{fill}%
\end{pgfscope}%
\begin{pgfscope}%
\pgfpathrectangle{\pgfqpoint{1.020000in}{0.880000in}}{\pgfqpoint{6.160000in}{6.160000in}}%
\pgfusepath{clip}%
\pgfsetbuttcap%
\pgfsetroundjoin%
\definecolor{currentfill}{rgb}{0.677823,0.786546,0.991005}%
\pgfsetfillcolor{currentfill}%
\pgfsetlinewidth{0.000000pt}%
\definecolor{currentstroke}{rgb}{0.000000,0.000000,0.000000}%
\pgfsetstrokecolor{currentstroke}%
\pgfsetdash{}{0pt}%
\pgfpathmoveto{\pgfqpoint{3.877783in}{3.700942in}}%
\pgfpathlineto{\pgfqpoint{3.887895in}{3.673773in}}%
\pgfpathlineto{\pgfqpoint{3.898017in}{3.648768in}}%
\pgfpathlineto{\pgfqpoint{3.931340in}{3.628922in}}%
\pgfpathlineto{\pgfqpoint{3.964629in}{3.609444in}}%
\pgfpathlineto{\pgfqpoint{3.954467in}{3.632381in}}%
\pgfpathlineto{\pgfqpoint{3.944318in}{3.657230in}}%
\pgfpathlineto{\pgfqpoint{3.911069in}{3.678813in}}%
\pgfpathlineto{\pgfqpoint{3.877783in}{3.700942in}}%
\pgfpathclose%
\pgfusepath{fill}%
\end{pgfscope}%
\begin{pgfscope}%
\pgfpathrectangle{\pgfqpoint{1.020000in}{0.880000in}}{\pgfqpoint{6.160000in}{6.160000in}}%
\pgfusepath{clip}%
\pgfsetbuttcap%
\pgfsetroundjoin%
\definecolor{currentfill}{rgb}{0.394042,0.522413,0.924916}%
\pgfsetfillcolor{currentfill}%
\pgfsetlinewidth{0.000000pt}%
\definecolor{currentstroke}{rgb}{0.000000,0.000000,0.000000}%
\pgfsetstrokecolor{currentstroke}%
\pgfsetdash{}{0pt}%
\pgfpathmoveto{\pgfqpoint{4.797189in}{3.171739in}}%
\pgfpathlineto{\pgfqpoint{4.808147in}{3.164632in}}%
\pgfpathlineto{\pgfqpoint{4.819123in}{3.156749in}}%
\pgfpathlineto{\pgfqpoint{4.852037in}{3.146324in}}%
\pgfpathlineto{\pgfqpoint{4.884925in}{3.136253in}}%
\pgfpathlineto{\pgfqpoint{4.873882in}{3.142234in}}%
\pgfpathlineto{\pgfqpoint{4.862854in}{3.147029in}}%
\pgfpathlineto{\pgfqpoint{4.830036in}{3.159334in}}%
\pgfpathlineto{\pgfqpoint{4.797189in}{3.171739in}}%
\pgfpathclose%
\pgfusepath{fill}%
\end{pgfscope}%
\begin{pgfscope}%
\pgfpathrectangle{\pgfqpoint{1.020000in}{0.880000in}}{\pgfqpoint{6.160000in}{6.160000in}}%
\pgfusepath{clip}%
\pgfsetbuttcap%
\pgfsetroundjoin%
\definecolor{currentfill}{rgb}{0.877149,0.394645,0.311724}%
\pgfsetfillcolor{currentfill}%
\pgfsetlinewidth{0.000000pt}%
\definecolor{currentstroke}{rgb}{0.000000,0.000000,0.000000}%
\pgfsetstrokecolor{currentstroke}%
\pgfsetdash{}{0pt}%
\pgfpathmoveto{\pgfqpoint{3.181616in}{4.979139in}}%
\pgfpathlineto{\pgfqpoint{3.191398in}{4.928969in}}%
\pgfpathlineto{\pgfqpoint{3.201226in}{4.874937in}}%
\pgfpathlineto{\pgfqpoint{3.234940in}{4.849982in}}%
\pgfpathlineto{\pgfqpoint{3.268652in}{4.820879in}}%
\pgfpathlineto{\pgfqpoint{3.258814in}{4.873144in}}%
\pgfpathlineto{\pgfqpoint{3.249018in}{4.921729in}}%
\pgfpathlineto{\pgfqpoint{3.215317in}{4.952626in}}%
\pgfpathlineto{\pgfqpoint{3.181616in}{4.979139in}}%
\pgfpathclose%
\pgfusepath{fill}%
\end{pgfscope}%
\begin{pgfscope}%
\pgfpathrectangle{\pgfqpoint{1.020000in}{0.880000in}}{\pgfqpoint{6.160000in}{6.160000in}}%
\pgfusepath{clip}%
\pgfsetbuttcap%
\pgfsetroundjoin%
\definecolor{currentfill}{rgb}{0.323718,0.433158,0.864722}%
\pgfsetfillcolor{currentfill}%
\pgfsetlinewidth{0.000000pt}%
\definecolor{currentstroke}{rgb}{0.000000,0.000000,0.000000}%
\pgfsetstrokecolor{currentstroke}%
\pgfsetdash{}{0pt}%
\pgfpathmoveto{\pgfqpoint{5.828456in}{3.035354in}}%
\pgfpathlineto{\pgfqpoint{5.840304in}{3.019911in}}%
\pgfpathlineto{\pgfqpoint{5.852173in}{3.004452in}}%
\pgfpathlineto{\pgfqpoint{5.884836in}{3.006174in}}%
\pgfpathlineto{\pgfqpoint{5.917480in}{3.007910in}}%
\pgfpathlineto{\pgfqpoint{5.905558in}{3.023335in}}%
\pgfpathlineto{\pgfqpoint{5.893659in}{3.038766in}}%
\pgfpathlineto{\pgfqpoint{5.861068in}{3.037061in}}%
\pgfpathlineto{\pgfqpoint{5.828456in}{3.035354in}}%
\pgfpathclose%
\pgfusepath{fill}%
\end{pgfscope}%
\begin{pgfscope}%
\pgfpathrectangle{\pgfqpoint{1.020000in}{0.880000in}}{\pgfqpoint{6.160000in}{6.160000in}}%
\pgfusepath{clip}%
\pgfsetbuttcap%
\pgfsetroundjoin%
\definecolor{currentfill}{rgb}{0.333490,0.446265,0.874452}%
\pgfsetfillcolor{currentfill}%
\pgfsetlinewidth{0.000000pt}%
\definecolor{currentstroke}{rgb}{0.000000,0.000000,0.000000}%
\pgfsetstrokecolor{currentstroke}%
\pgfsetdash{}{0pt}%
\pgfpathmoveto{\pgfqpoint{5.609021in}{3.054607in}}%
\pgfpathlineto{\pgfqpoint{5.620681in}{3.040024in}}%
\pgfpathlineto{\pgfqpoint{5.632358in}{3.025119in}}%
\pgfpathlineto{\pgfqpoint{5.665092in}{3.026797in}}%
\pgfpathlineto{\pgfqpoint{5.697806in}{3.028499in}}%
\pgfpathlineto{\pgfqpoint{5.686080in}{3.043657in}}%
\pgfpathlineto{\pgfqpoint{5.674372in}{3.058610in}}%
\pgfpathlineto{\pgfqpoint{5.641707in}{3.056622in}}%
\pgfpathlineto{\pgfqpoint{5.609021in}{3.054607in}}%
\pgfpathclose%
\pgfusepath{fill}%
\end{pgfscope}%
\begin{pgfscope}%
\pgfpathrectangle{\pgfqpoint{1.020000in}{0.880000in}}{\pgfqpoint{6.160000in}{6.160000in}}%
\pgfusepath{clip}%
\pgfsetbuttcap%
\pgfsetroundjoin%
\definecolor{currentfill}{rgb}{0.353369,0.472069,0.892570}%
\pgfsetfillcolor{currentfill}%
\pgfsetlinewidth{0.000000pt}%
\definecolor{currentstroke}{rgb}{0.000000,0.000000,0.000000}%
\pgfsetstrokecolor{currentstroke}%
\pgfsetdash{}{0pt}%
\pgfpathmoveto{\pgfqpoint{5.169948in}{3.077850in}}%
\pgfpathlineto{\pgfqpoint{5.181286in}{3.071915in}}%
\pgfpathlineto{\pgfqpoint{5.192628in}{3.064003in}}%
\pgfpathlineto{\pgfqpoint{5.225475in}{3.062844in}}%
\pgfpathlineto{\pgfqpoint{5.258307in}{3.062416in}}%
\pgfpathlineto{\pgfqpoint{5.246909in}{3.070440in}}%
\pgfpathlineto{\pgfqpoint{5.235515in}{3.076484in}}%
\pgfpathlineto{\pgfqpoint{5.202738in}{3.076668in}}%
\pgfpathlineto{\pgfqpoint{5.169948in}{3.077850in}}%
\pgfpathclose%
\pgfusepath{fill}%
\end{pgfscope}%
\begin{pgfscope}%
\pgfpathrectangle{\pgfqpoint{1.020000in}{0.880000in}}{\pgfqpoint{6.160000in}{6.160000in}}%
\pgfusepath{clip}%
\pgfsetbuttcap%
\pgfsetroundjoin%
\definecolor{currentfill}{rgb}{0.343278,0.459354,0.884122}%
\pgfsetfillcolor{currentfill}%
\pgfsetlinewidth{0.000000pt}%
\definecolor{currentstroke}{rgb}{0.000000,0.000000,0.000000}%
\pgfsetstrokecolor{currentstroke}%
\pgfsetdash{}{0pt}%
\pgfpathmoveto{\pgfqpoint{5.389486in}{3.066804in}}%
\pgfpathlineto{\pgfqpoint{5.400986in}{3.055863in}}%
\pgfpathlineto{\pgfqpoint{5.412494in}{3.043751in}}%
\pgfpathlineto{\pgfqpoint{5.445295in}{3.045184in}}%
\pgfpathlineto{\pgfqpoint{5.478079in}{3.046841in}}%
\pgfpathlineto{\pgfqpoint{5.466524in}{3.059651in}}%
\pgfpathlineto{\pgfqpoint{5.454980in}{3.071507in}}%
\pgfpathlineto{\pgfqpoint{5.422241in}{3.069023in}}%
\pgfpathlineto{\pgfqpoint{5.389486in}{3.066804in}}%
\pgfpathclose%
\pgfusepath{fill}%
\end{pgfscope}%
\begin{pgfscope}%
\pgfpathrectangle{\pgfqpoint{1.020000in}{0.880000in}}{\pgfqpoint{6.160000in}{6.160000in}}%
\pgfusepath{clip}%
\pgfsetbuttcap%
\pgfsetroundjoin%
\definecolor{currentfill}{rgb}{0.740957,0.122240,0.175744}%
\pgfsetfillcolor{currentfill}%
\pgfsetlinewidth{0.000000pt}%
\definecolor{currentstroke}{rgb}{0.000000,0.000000,0.000000}%
\pgfsetstrokecolor{currentstroke}%
\pgfsetdash{}{0pt}%
\pgfpathmoveto{\pgfqpoint{2.941571in}{5.194209in}}%
\pgfpathlineto{\pgfqpoint{2.951011in}{5.162134in}}%
\pgfpathlineto{\pgfqpoint{2.960524in}{5.124653in}}%
\pgfpathlineto{\pgfqpoint{2.994061in}{5.128292in}}%
\pgfpathlineto{\pgfqpoint{3.027636in}{5.126554in}}%
\pgfpathlineto{\pgfqpoint{3.018083in}{5.163926in}}%
\pgfpathlineto{\pgfqpoint{3.008601in}{5.195865in}}%
\pgfpathlineto{\pgfqpoint{2.975066in}{5.197790in}}%
\pgfpathlineto{\pgfqpoint{2.941571in}{5.194209in}}%
\pgfpathclose%
\pgfusepath{fill}%
\end{pgfscope}%
\begin{pgfscope}%
\pgfpathrectangle{\pgfqpoint{1.020000in}{0.880000in}}{\pgfqpoint{6.160000in}{6.160000in}}%
\pgfusepath{clip}%
\pgfsetbuttcap%
\pgfsetroundjoin%
\definecolor{currentfill}{rgb}{0.891817,0.851973,0.829085}%
\pgfsetfillcolor{currentfill}%
\pgfsetlinewidth{0.000000pt}%
\definecolor{currentstroke}{rgb}{0.000000,0.000000,0.000000}%
\pgfsetstrokecolor{currentstroke}%
\pgfsetdash{}{0pt}%
\pgfpathmoveto{\pgfqpoint{3.550161in}{4.166267in}}%
\pgfpathlineto{\pgfqpoint{3.560165in}{4.114199in}}%
\pgfpathlineto{\pgfqpoint{3.570173in}{4.063422in}}%
\pgfpathlineto{\pgfqpoint{3.603715in}{4.034910in}}%
\pgfpathlineto{\pgfqpoint{3.637222in}{4.005654in}}%
\pgfpathlineto{\pgfqpoint{3.627208in}{4.052413in}}%
\pgfpathlineto{\pgfqpoint{3.617200in}{4.100403in}}%
\pgfpathlineto{\pgfqpoint{3.583699in}{4.133720in}}%
\pgfpathlineto{\pgfqpoint{3.550161in}{4.166267in}}%
\pgfpathclose%
\pgfusepath{fill}%
\end{pgfscope}%
\begin{pgfscope}%
\pgfpathrectangle{\pgfqpoint{1.020000in}{0.880000in}}{\pgfqpoint{6.160000in}{6.160000in}}%
\pgfusepath{clip}%
\pgfsetbuttcap%
\pgfsetroundjoin%
\definecolor{currentfill}{rgb}{0.826784,0.858205,0.906953}%
\pgfsetfillcolor{currentfill}%
\pgfsetlinewidth{0.000000pt}%
\definecolor{currentstroke}{rgb}{0.000000,0.000000,0.000000}%
\pgfsetstrokecolor{currentstroke}%
\pgfsetdash{}{0pt}%
\pgfpathmoveto{\pgfqpoint{3.637222in}{4.005654in}}%
\pgfpathlineto{\pgfqpoint{3.647240in}{3.960469in}}%
\pgfpathlineto{\pgfqpoint{3.657261in}{3.917178in}}%
\pgfpathlineto{\pgfqpoint{3.690743in}{3.891274in}}%
\pgfpathlineto{\pgfqpoint{3.724188in}{3.865067in}}%
\pgfpathlineto{\pgfqpoint{3.714152in}{3.904614in}}%
\pgfpathlineto{\pgfqpoint{3.704121in}{3.945917in}}%
\pgfpathlineto{\pgfqpoint{3.670691in}{3.975907in}}%
\pgfpathlineto{\pgfqpoint{3.637222in}{4.005654in}}%
\pgfpathclose%
\pgfusepath{fill}%
\end{pgfscope}%
\begin{pgfscope}%
\pgfpathrectangle{\pgfqpoint{1.020000in}{0.880000in}}{\pgfqpoint{6.160000in}{6.160000in}}%
\pgfusepath{clip}%
\pgfsetbuttcap%
\pgfsetroundjoin%
\definecolor{currentfill}{rgb}{0.368507,0.491141,0.905243}%
\pgfsetfillcolor{currentfill}%
\pgfsetlinewidth{0.000000pt}%
\definecolor{currentstroke}{rgb}{0.000000,0.000000,0.000000}%
\pgfsetstrokecolor{currentstroke}%
\pgfsetdash{}{0pt}%
\pgfpathmoveto{\pgfqpoint{4.950629in}{3.117535in}}%
\pgfpathlineto{\pgfqpoint{4.961755in}{3.112165in}}%
\pgfpathlineto{\pgfqpoint{4.972891in}{3.105227in}}%
\pgfpathlineto{\pgfqpoint{5.005779in}{3.098504in}}%
\pgfpathlineto{\pgfqpoint{5.038646in}{3.092554in}}%
\pgfpathlineto{\pgfqpoint{5.027442in}{3.098050in}}%
\pgfpathlineto{\pgfqpoint{5.016245in}{3.101601in}}%
\pgfpathlineto{\pgfqpoint{4.983447in}{3.109145in}}%
\pgfpathlineto{\pgfqpoint{4.950629in}{3.117535in}}%
\pgfpathclose%
\pgfusepath{fill}%
\end{pgfscope}%
\begin{pgfscope}%
\pgfpathrectangle{\pgfqpoint{1.020000in}{0.880000in}}{\pgfqpoint{6.160000in}{6.160000in}}%
\pgfusepath{clip}%
\pgfsetbuttcap%
\pgfsetroundjoin%
\definecolor{currentfill}{rgb}{0.969851,0.695830,0.581312}%
\pgfsetfillcolor{currentfill}%
\pgfsetlinewidth{0.000000pt}%
\definecolor{currentstroke}{rgb}{0.000000,0.000000,0.000000}%
\pgfsetstrokecolor{currentstroke}%
\pgfsetdash{}{0pt}%
\pgfpathmoveto{\pgfqpoint{2.279237in}{4.458800in}}%
\pgfpathlineto{\pgfqpoint{2.288076in}{4.432465in}}%
\pgfpathlineto{\pgfqpoint{2.296966in}{4.403962in}}%
\pgfpathlineto{\pgfqpoint{2.330085in}{4.443140in}}%
\pgfpathlineto{\pgfqpoint{2.363174in}{4.484205in}}%
\pgfpathlineto{\pgfqpoint{2.354214in}{4.513742in}}%
\pgfpathlineto{\pgfqpoint{2.345309in}{4.540746in}}%
\pgfpathlineto{\pgfqpoint{2.312288in}{4.498808in}}%
\pgfpathlineto{\pgfqpoint{2.279237in}{4.458800in}}%
\pgfpathclose%
\pgfusepath{fill}%
\end{pgfscope}%
\begin{pgfscope}%
\pgfpathrectangle{\pgfqpoint{1.020000in}{0.880000in}}{\pgfqpoint{6.160000in}{6.160000in}}%
\pgfusepath{clip}%
\pgfsetbuttcap%
\pgfsetroundjoin%
\definecolor{currentfill}{rgb}{0.945540,0.798606,0.723105}%
\pgfsetfillcolor{currentfill}%
\pgfsetlinewidth{0.000000pt}%
\definecolor{currentstroke}{rgb}{0.000000,0.000000,0.000000}%
\pgfsetstrokecolor{currentstroke}%
\pgfsetdash{}{0pt}%
\pgfpathmoveto{\pgfqpoint{3.462989in}{4.342175in}}%
\pgfpathlineto{\pgfqpoint{3.472982in}{4.284792in}}%
\pgfpathlineto{\pgfqpoint{3.482981in}{4.227871in}}%
\pgfpathlineto{\pgfqpoint{3.516587in}{4.197750in}}%
\pgfpathlineto{\pgfqpoint{3.550161in}{4.166267in}}%
\pgfpathlineto{\pgfqpoint{3.540162in}{4.219230in}}%
\pgfpathlineto{\pgfqpoint{3.530171in}{4.272672in}}%
\pgfpathlineto{\pgfqpoint{3.496597in}{4.308162in}}%
\pgfpathlineto{\pgfqpoint{3.462989in}{4.342175in}}%
\pgfpathclose%
\pgfusepath{fill}%
\end{pgfscope}%
\begin{pgfscope}%
\pgfpathrectangle{\pgfqpoint{1.020000in}{0.880000in}}{\pgfqpoint{6.160000in}{6.160000in}}%
\pgfusepath{clip}%
\pgfsetbuttcap%
\pgfsetroundjoin%
\definecolor{currentfill}{rgb}{0.966017,0.646130,0.525890}%
\pgfsetfillcolor{currentfill}%
\pgfsetlinewidth{0.000000pt}%
\definecolor{currentstroke}{rgb}{0.000000,0.000000,0.000000}%
\pgfsetstrokecolor{currentstroke}%
\pgfsetdash{}{0pt}%
\pgfpathmoveto{\pgfqpoint{2.345309in}{4.540746in}}%
\pgfpathlineto{\pgfqpoint{2.354214in}{4.513742in}}%
\pgfpathlineto{\pgfqpoint{2.363174in}{4.484205in}}%
\pgfpathlineto{\pgfqpoint{2.396241in}{4.526914in}}%
\pgfpathlineto{\pgfqpoint{2.429290in}{4.570984in}}%
\pgfpathlineto{\pgfqpoint{2.420258in}{4.601643in}}%
\pgfpathlineto{\pgfqpoint{2.411287in}{4.629378in}}%
\pgfpathlineto{\pgfqpoint{2.378307in}{4.584367in}}%
\pgfpathlineto{\pgfqpoint{2.345309in}{4.540746in}}%
\pgfpathclose%
\pgfusepath{fill}%
\end{pgfscope}%
\begin{pgfscope}%
\pgfpathrectangle{\pgfqpoint{1.020000in}{0.880000in}}{\pgfqpoint{6.160000in}{6.160000in}}%
\pgfusepath{clip}%
\pgfsetbuttcap%
\pgfsetroundjoin%
\definecolor{currentfill}{rgb}{0.581486,0.713451,0.998314}%
\pgfsetfillcolor{currentfill}%
\pgfsetlinewidth{0.000000pt}%
\definecolor{currentstroke}{rgb}{0.000000,0.000000,0.000000}%
\pgfsetstrokecolor{currentstroke}%
\pgfsetdash{}{0pt}%
\pgfpathmoveto{\pgfqpoint{4.117988in}{3.501920in}}%
\pgfpathlineto{\pgfqpoint{4.128289in}{3.486785in}}%
\pgfpathlineto{\pgfqpoint{4.138607in}{3.473307in}}%
\pgfpathlineto{\pgfqpoint{4.171826in}{3.457927in}}%
\pgfpathlineto{\pgfqpoint{4.205014in}{3.443034in}}%
\pgfpathlineto{\pgfqpoint{4.194645in}{3.456410in}}%
\pgfpathlineto{\pgfqpoint{4.184294in}{3.471194in}}%
\pgfpathlineto{\pgfqpoint{4.151157in}{3.486262in}}%
\pgfpathlineto{\pgfqpoint{4.117988in}{3.501920in}}%
\pgfpathclose%
\pgfusepath{fill}%
\end{pgfscope}%
\begin{pgfscope}%
\pgfpathrectangle{\pgfqpoint{1.020000in}{0.880000in}}{\pgfqpoint{6.160000in}{6.160000in}}%
\pgfusepath{clip}%
\pgfsetbuttcap%
\pgfsetroundjoin%
\definecolor{currentfill}{rgb}{0.532568,0.669801,0.990393}%
\pgfsetfillcolor{currentfill}%
\pgfsetlinewidth{0.000000pt}%
\definecolor{currentstroke}{rgb}{0.000000,0.000000,0.000000}%
\pgfsetstrokecolor{currentstroke}%
\pgfsetdash{}{0pt}%
\pgfpathmoveto{\pgfqpoint{4.271299in}{3.414756in}}%
\pgfpathlineto{\pgfqpoint{4.281741in}{3.402519in}}%
\pgfpathlineto{\pgfqpoint{4.292203in}{3.391463in}}%
\pgfpathlineto{\pgfqpoint{4.325353in}{3.377629in}}%
\pgfpathlineto{\pgfqpoint{4.358474in}{3.364226in}}%
\pgfpathlineto{\pgfqpoint{4.347959in}{3.375782in}}%
\pgfpathlineto{\pgfqpoint{4.337465in}{3.388317in}}%
\pgfpathlineto{\pgfqpoint{4.304397in}{3.401334in}}%
\pgfpathlineto{\pgfqpoint{4.271299in}{3.414756in}}%
\pgfpathclose%
\pgfusepath{fill}%
\end{pgfscope}%
\begin{pgfscope}%
\pgfpathrectangle{\pgfqpoint{1.020000in}{0.880000in}}{\pgfqpoint{6.160000in}{6.160000in}}%
\pgfusepath{clip}%
\pgfsetbuttcap%
\pgfsetroundjoin%
\definecolor{currentfill}{rgb}{0.763363,0.835092,0.955658}%
\pgfsetfillcolor{currentfill}%
\pgfsetlinewidth{0.000000pt}%
\definecolor{currentstroke}{rgb}{0.000000,0.000000,0.000000}%
\pgfsetstrokecolor{currentstroke}%
\pgfsetdash{}{0pt}%
\pgfpathmoveto{\pgfqpoint{3.724188in}{3.865067in}}%
\pgfpathlineto{\pgfqpoint{3.734229in}{3.827541in}}%
\pgfpathlineto{\pgfqpoint{3.744275in}{3.792277in}}%
\pgfpathlineto{\pgfqpoint{3.777707in}{3.769321in}}%
\pgfpathlineto{\pgfqpoint{3.811102in}{3.746340in}}%
\pgfpathlineto{\pgfqpoint{3.801030in}{3.778421in}}%
\pgfpathlineto{\pgfqpoint{3.790966in}{3.812560in}}%
\pgfpathlineto{\pgfqpoint{3.757596in}{3.838764in}}%
\pgfpathlineto{\pgfqpoint{3.724188in}{3.865067in}}%
\pgfpathclose%
\pgfusepath{fill}%
\end{pgfscope}%
\begin{pgfscope}%
\pgfpathrectangle{\pgfqpoint{1.020000in}{0.880000in}}{\pgfqpoint{6.160000in}{6.160000in}}%
\pgfusepath{clip}%
\pgfsetbuttcap%
\pgfsetroundjoin%
\definecolor{currentfill}{rgb}{0.489246,0.627536,0.976896}%
\pgfsetfillcolor{currentfill}%
\pgfsetlinewidth{0.000000pt}%
\definecolor{currentstroke}{rgb}{0.000000,0.000000,0.000000}%
\pgfsetstrokecolor{currentstroke}%
\pgfsetdash{}{0pt}%
\pgfpathmoveto{\pgfqpoint{4.424627in}{3.338457in}}%
\pgfpathlineto{\pgfqpoint{4.435215in}{3.327316in}}%
\pgfpathlineto{\pgfqpoint{4.445826in}{3.316906in}}%
\pgfpathlineto{\pgfqpoint{4.478911in}{3.303937in}}%
\pgfpathlineto{\pgfqpoint{4.511967in}{3.291212in}}%
\pgfpathlineto{\pgfqpoint{4.501304in}{3.302000in}}%
\pgfpathlineto{\pgfqpoint{4.490663in}{3.313345in}}%
\pgfpathlineto{\pgfqpoint{4.457660in}{3.325887in}}%
\pgfpathlineto{\pgfqpoint{4.424627in}{3.338457in}}%
\pgfpathclose%
\pgfusepath{fill}%
\end{pgfscope}%
\begin{pgfscope}%
\pgfpathrectangle{\pgfqpoint{1.020000in}{0.880000in}}{\pgfqpoint{6.160000in}{6.160000in}}%
\pgfusepath{clip}%
\pgfsetbuttcap%
\pgfsetroundjoin%
\definecolor{currentfill}{rgb}{0.635474,0.756714,0.998297}%
\pgfsetfillcolor{currentfill}%
\pgfsetlinewidth{0.000000pt}%
\definecolor{currentstroke}{rgb}{0.000000,0.000000,0.000000}%
\pgfsetstrokecolor{currentstroke}%
\pgfsetdash{}{0pt}%
\pgfpathmoveto{\pgfqpoint{3.964629in}{3.609444in}}%
\pgfpathlineto{\pgfqpoint{3.974803in}{3.588528in}}%
\pgfpathlineto{\pgfqpoint{3.984991in}{3.569718in}}%
\pgfpathlineto{\pgfqpoint{4.018290in}{3.552113in}}%
\pgfpathlineto{\pgfqpoint{4.051555in}{3.534897in}}%
\pgfpathlineto{\pgfqpoint{4.041321in}{3.552524in}}%
\pgfpathlineto{\pgfqpoint{4.031102in}{3.571988in}}%
\pgfpathlineto{\pgfqpoint{3.997882in}{3.590437in}}%
\pgfpathlineto{\pgfqpoint{3.964629in}{3.609444in}}%
\pgfpathclose%
\pgfusepath{fill}%
\end{pgfscope}%
\begin{pgfscope}%
\pgfpathrectangle{\pgfqpoint{1.020000in}{0.880000in}}{\pgfqpoint{6.160000in}{6.160000in}}%
\pgfusepath{clip}%
\pgfsetbuttcap%
\pgfsetroundjoin%
\definecolor{currentfill}{rgb}{0.968533,0.715841,0.606097}%
\pgfsetfillcolor{currentfill}%
\pgfsetlinewidth{0.000000pt}%
\definecolor{currentstroke}{rgb}{0.000000,0.000000,0.000000}%
\pgfsetstrokecolor{currentstroke}%
\pgfsetdash{}{0pt}%
\pgfpathmoveto{\pgfqpoint{3.375726in}{4.525780in}}%
\pgfpathlineto{\pgfqpoint{3.385699in}{4.465398in}}%
\pgfpathlineto{\pgfqpoint{3.395686in}{4.404450in}}%
\pgfpathlineto{\pgfqpoint{3.429350in}{4.374379in}}%
\pgfpathlineto{\pgfqpoint{3.462989in}{4.342175in}}%
\pgfpathlineto{\pgfqpoint{3.453005in}{4.399554in}}%
\pgfpathlineto{\pgfqpoint{3.443035in}{4.456453in}}%
\pgfpathlineto{\pgfqpoint{3.409394in}{4.492282in}}%
\pgfpathlineto{\pgfqpoint{3.375726in}{4.525780in}}%
\pgfpathclose%
\pgfusepath{fill}%
\end{pgfscope}%
\begin{pgfscope}%
\pgfpathrectangle{\pgfqpoint{1.020000in}{0.880000in}}{\pgfqpoint{6.160000in}{6.160000in}}%
\pgfusepath{clip}%
\pgfsetbuttcap%
\pgfsetroundjoin%
\definecolor{currentfill}{rgb}{0.953054,0.585211,0.465373}%
\pgfsetfillcolor{currentfill}%
\pgfsetlinewidth{0.000000pt}%
\definecolor{currentstroke}{rgb}{0.000000,0.000000,0.000000}%
\pgfsetstrokecolor{currentstroke}%
\pgfsetdash{}{0pt}%
\pgfpathmoveto{\pgfqpoint{2.411287in}{4.629378in}}%
\pgfpathlineto{\pgfqpoint{2.420258in}{4.601643in}}%
\pgfpathlineto{\pgfqpoint{2.429290in}{4.570984in}}%
\pgfpathlineto{\pgfqpoint{2.462328in}{4.616090in}}%
\pgfpathlineto{\pgfqpoint{2.495362in}{4.661870in}}%
\pgfpathlineto{\pgfqpoint{2.486259in}{4.693703in}}%
\pgfpathlineto{\pgfqpoint{2.477222in}{4.722207in}}%
\pgfpathlineto{\pgfqpoint{2.444257in}{4.675449in}}%
\pgfpathlineto{\pgfqpoint{2.411287in}{4.629378in}}%
\pgfpathclose%
\pgfusepath{fill}%
\end{pgfscope}%
\begin{pgfscope}%
\pgfpathrectangle{\pgfqpoint{1.020000in}{0.880000in}}{\pgfqpoint{6.160000in}{6.160000in}}%
\pgfusepath{clip}%
\pgfsetbuttcap%
\pgfsetroundjoin%
\definecolor{currentfill}{rgb}{0.451739,0.588181,0.960201}%
\pgfsetfillcolor{currentfill}%
\pgfsetlinewidth{0.000000pt}%
\definecolor{currentstroke}{rgb}{0.000000,0.000000,0.000000}%
\pgfsetstrokecolor{currentstroke}%
\pgfsetdash{}{0pt}%
\pgfpathmoveto{\pgfqpoint{4.577992in}{3.266016in}}%
\pgfpathlineto{\pgfqpoint{4.588730in}{3.255693in}}%
\pgfpathlineto{\pgfqpoint{4.599490in}{3.245611in}}%
\pgfpathlineto{\pgfqpoint{4.632513in}{3.233235in}}%
\pgfpathlineto{\pgfqpoint{4.665506in}{3.220972in}}%
\pgfpathlineto{\pgfqpoint{4.654691in}{3.230621in}}%
\pgfpathlineto{\pgfqpoint{4.643897in}{3.240281in}}%
\pgfpathlineto{\pgfqpoint{4.610960in}{3.253281in}}%
\pgfpathlineto{\pgfqpoint{4.577992in}{3.266016in}}%
\pgfpathclose%
\pgfusepath{fill}%
\end{pgfscope}%
\begin{pgfscope}%
\pgfpathrectangle{\pgfqpoint{1.020000in}{0.880000in}}{\pgfqpoint{6.160000in}{6.160000in}}%
\pgfusepath{clip}%
\pgfsetbuttcap%
\pgfsetroundjoin%
\definecolor{currentfill}{rgb}{0.328604,0.439712,0.869587}%
\pgfsetfillcolor{currentfill}%
\pgfsetlinewidth{0.000000pt}%
\definecolor{currentstroke}{rgb}{0.000000,0.000000,0.000000}%
\pgfsetstrokecolor{currentstroke}%
\pgfsetdash{}{0pt}%
\pgfpathmoveto{\pgfqpoint{5.763172in}{3.031929in}}%
\pgfpathlineto{\pgfqpoint{5.774968in}{3.016523in}}%
\pgfpathlineto{\pgfqpoint{5.786785in}{3.001063in}}%
\pgfpathlineto{\pgfqpoint{5.819489in}{3.002748in}}%
\pgfpathlineto{\pgfqpoint{5.852173in}{3.004452in}}%
\pgfpathlineto{\pgfqpoint{5.840304in}{3.019911in}}%
\pgfpathlineto{\pgfqpoint{5.828456in}{3.035354in}}%
\pgfpathlineto{\pgfqpoint{5.795824in}{3.033643in}}%
\pgfpathlineto{\pgfqpoint{5.763172in}{3.031929in}}%
\pgfpathclose%
\pgfusepath{fill}%
\end{pgfscope}%
\begin{pgfscope}%
\pgfpathrectangle{\pgfqpoint{1.020000in}{0.880000in}}{\pgfqpoint{6.160000in}{6.160000in}}%
\pgfusepath{clip}%
\pgfsetbuttcap%
\pgfsetroundjoin%
\definecolor{currentfill}{rgb}{0.746838,0.140021,0.179996}%
\pgfsetfillcolor{currentfill}%
\pgfsetlinewidth{0.000000pt}%
\definecolor{currentstroke}{rgb}{0.000000,0.000000,0.000000}%
\pgfsetstrokecolor{currentstroke}%
\pgfsetdash{}{0pt}%
\pgfpathmoveto{\pgfqpoint{2.874715in}{5.171001in}}%
\pgfpathlineto{\pgfqpoint{2.884108in}{5.139000in}}%
\pgfpathlineto{\pgfqpoint{2.893578in}{5.101706in}}%
\pgfpathlineto{\pgfqpoint{2.927029in}{5.115736in}}%
\pgfpathlineto{\pgfqpoint{2.960524in}{5.124653in}}%
\pgfpathlineto{\pgfqpoint{2.951011in}{5.162134in}}%
\pgfpathlineto{\pgfqpoint{2.941571in}{5.194209in}}%
\pgfpathlineto{\pgfqpoint{2.908120in}{5.185223in}}%
\pgfpathlineto{\pgfqpoint{2.874715in}{5.171001in}}%
\pgfpathclose%
\pgfusepath{fill}%
\end{pgfscope}%
\begin{pgfscope}%
\pgfpathrectangle{\pgfqpoint{1.020000in}{0.880000in}}{\pgfqpoint{6.160000in}{6.160000in}}%
\pgfusepath{clip}%
\pgfsetbuttcap%
\pgfsetroundjoin%
\definecolor{currentfill}{rgb}{0.338377,0.452819,0.879317}%
\pgfsetfillcolor{currentfill}%
\pgfsetlinewidth{0.000000pt}%
\definecolor{currentstroke}{rgb}{0.000000,0.000000,0.000000}%
\pgfsetstrokecolor{currentstroke}%
\pgfsetdash{}{0pt}%
\pgfpathmoveto{\pgfqpoint{5.543589in}{3.050593in}}%
\pgfpathlineto{\pgfqpoint{5.555202in}{3.036480in}}%
\pgfpathlineto{\pgfqpoint{5.566831in}{3.021904in}}%
\pgfpathlineto{\pgfqpoint{5.599605in}{3.023480in}}%
\pgfpathlineto{\pgfqpoint{5.632358in}{3.025119in}}%
\pgfpathlineto{\pgfqpoint{5.620681in}{3.040024in}}%
\pgfpathlineto{\pgfqpoint{5.609021in}{3.054607in}}%
\pgfpathlineto{\pgfqpoint{5.576315in}{3.052587in}}%
\pgfpathlineto{\pgfqpoint{5.543589in}{3.050593in}}%
\pgfpathclose%
\pgfusepath{fill}%
\end{pgfscope}%
\begin{pgfscope}%
\pgfpathrectangle{\pgfqpoint{1.020000in}{0.880000in}}{\pgfqpoint{6.160000in}{6.160000in}}%
\pgfusepath{clip}%
\pgfsetbuttcap%
\pgfsetroundjoin%
\definecolor{currentfill}{rgb}{0.348323,0.465711,0.888346}%
\pgfsetfillcolor{currentfill}%
\pgfsetlinewidth{0.000000pt}%
\definecolor{currentstroke}{rgb}{0.000000,0.000000,0.000000}%
\pgfsetstrokecolor{currentstroke}%
\pgfsetdash{}{0pt}%
\pgfpathmoveto{\pgfqpoint{5.323926in}{3.063537in}}%
\pgfpathlineto{\pgfqpoint{5.335380in}{3.053342in}}%
\pgfpathlineto{\pgfqpoint{5.346840in}{3.041805in}}%
\pgfpathlineto{\pgfqpoint{5.379676in}{3.042603in}}%
\pgfpathlineto{\pgfqpoint{5.412494in}{3.043751in}}%
\pgfpathlineto{\pgfqpoint{5.400986in}{3.055863in}}%
\pgfpathlineto{\pgfqpoint{5.389486in}{3.066804in}}%
\pgfpathlineto{\pgfqpoint{5.356714in}{3.064944in}}%
\pgfpathlineto{\pgfqpoint{5.323926in}{3.063537in}}%
\pgfpathclose%
\pgfusepath{fill}%
\end{pgfscope}%
\begin{pgfscope}%
\pgfpathrectangle{\pgfqpoint{1.020000in}{0.880000in}}{\pgfqpoint{6.160000in}{6.160000in}}%
\pgfusepath{clip}%
\pgfsetbuttcap%
\pgfsetroundjoin%
\definecolor{currentfill}{rgb}{0.313946,0.420052,0.854993}%
\pgfsetfillcolor{currentfill}%
\pgfsetlinewidth{0.000000pt}%
\definecolor{currentstroke}{rgb}{0.000000,0.000000,0.000000}%
\pgfsetstrokecolor{currentstroke}%
\pgfsetdash{}{0pt}%
\pgfpathmoveto{\pgfqpoint{5.982706in}{3.011425in}}%
\pgfpathlineto{\pgfqpoint{5.994702in}{2.996086in}}%
\pgfpathlineto{\pgfqpoint{6.006722in}{2.980776in}}%
\pgfpathlineto{\pgfqpoint{6.039358in}{2.982634in}}%
\pgfpathlineto{\pgfqpoint{6.027312in}{2.997902in}}%
\pgfpathlineto{\pgfqpoint{6.015289in}{3.013203in}}%
\pgfpathlineto{\pgfqpoint{5.982706in}{3.011425in}}%
\pgfpathclose%
\pgfusepath{fill}%
\end{pgfscope}%
\begin{pgfscope}%
\pgfpathrectangle{\pgfqpoint{1.020000in}{0.880000in}}{\pgfqpoint{6.160000in}{6.160000in}}%
\pgfusepath{clip}%
\pgfsetbuttcap%
\pgfsetroundjoin%
\definecolor{currentfill}{rgb}{0.790562,0.231397,0.216242}%
\pgfsetfillcolor{currentfill}%
\pgfsetlinewidth{0.000000pt}%
\definecolor{currentstroke}{rgb}{0.000000,0.000000,0.000000}%
\pgfsetstrokecolor{currentstroke}%
\pgfsetdash{}{0pt}%
\pgfpathmoveto{\pgfqpoint{3.027636in}{5.126554in}}%
\pgfpathlineto{\pgfqpoint{3.037256in}{5.084006in}}%
\pgfpathlineto{\pgfqpoint{3.046938in}{5.036605in}}%
\pgfpathlineto{\pgfqpoint{3.080576in}{5.029855in}}%
\pgfpathlineto{\pgfqpoint{3.114239in}{5.017950in}}%
\pgfpathlineto{\pgfqpoint{3.104529in}{5.064818in}}%
\pgfpathlineto{\pgfqpoint{3.094879in}{5.106895in}}%
\pgfpathlineto{\pgfqpoint{3.061244in}{5.119408in}}%
\pgfpathlineto{\pgfqpoint{3.027636in}{5.126554in}}%
\pgfpathclose%
\pgfusepath{fill}%
\end{pgfscope}%
\begin{pgfscope}%
\pgfpathrectangle{\pgfqpoint{1.020000in}{0.880000in}}{\pgfqpoint{6.160000in}{6.160000in}}%
\pgfusepath{clip}%
\pgfsetbuttcap%
\pgfsetroundjoin%
\definecolor{currentfill}{rgb}{0.358415,0.478426,0.896795}%
\pgfsetfillcolor{currentfill}%
\pgfsetlinewidth{0.000000pt}%
\definecolor{currentstroke}{rgb}{0.000000,0.000000,0.000000}%
\pgfsetstrokecolor{currentstroke}%
\pgfsetdash{}{0pt}%
\pgfpathmoveto{\pgfqpoint{5.104328in}{3.083265in}}%
\pgfpathlineto{\pgfqpoint{5.115605in}{3.076866in}}%
\pgfpathlineto{\pgfqpoint{5.126887in}{3.068622in}}%
\pgfpathlineto{\pgfqpoint{5.159765in}{3.065923in}}%
\pgfpathlineto{\pgfqpoint{5.192628in}{3.064003in}}%
\pgfpathlineto{\pgfqpoint{5.181286in}{3.071915in}}%
\pgfpathlineto{\pgfqpoint{5.169948in}{3.077850in}}%
\pgfpathlineto{\pgfqpoint{5.137145in}{3.080053in}}%
\pgfpathlineto{\pgfqpoint{5.104328in}{3.083265in}}%
\pgfpathclose%
\pgfusepath{fill}%
\end{pgfscope}%
\begin{pgfscope}%
\pgfpathrectangle{\pgfqpoint{1.020000in}{0.880000in}}{\pgfqpoint{6.160000in}{6.160000in}}%
\pgfusepath{clip}%
\pgfsetbuttcap%
\pgfsetroundjoin%
\definecolor{currentfill}{rgb}{0.931831,0.519086,0.406480}%
\pgfsetfillcolor{currentfill}%
\pgfsetlinewidth{0.000000pt}%
\definecolor{currentstroke}{rgb}{0.000000,0.000000,0.000000}%
\pgfsetstrokecolor{currentstroke}%
\pgfsetdash{}{0pt}%
\pgfpathmoveto{\pgfqpoint{2.477222in}{4.722207in}}%
\pgfpathlineto{\pgfqpoint{2.486259in}{4.693703in}}%
\pgfpathlineto{\pgfqpoint{2.495362in}{4.661870in}}%
\pgfpathlineto{\pgfqpoint{2.528398in}{4.707927in}}%
\pgfpathlineto{\pgfqpoint{2.561444in}{4.753833in}}%
\pgfpathlineto{\pgfqpoint{2.552271in}{4.786852in}}%
\pgfpathlineto{\pgfqpoint{2.543168in}{4.816133in}}%
\pgfpathlineto{\pgfqpoint{2.510190in}{4.769248in}}%
\pgfpathlineto{\pgfqpoint{2.477222in}{4.722207in}}%
\pgfpathclose%
\pgfusepath{fill}%
\end{pgfscope}%
\begin{pgfscope}%
\pgfpathrectangle{\pgfqpoint{1.020000in}{0.880000in}}{\pgfqpoint{6.160000in}{6.160000in}}%
\pgfusepath{clip}%
\pgfsetbuttcap%
\pgfsetroundjoin%
\definecolor{currentfill}{rgb}{0.414801,0.546874,0.939088}%
\pgfsetfillcolor{currentfill}%
\pgfsetlinewidth{0.000000pt}%
\definecolor{currentstroke}{rgb}{0.000000,0.000000,0.000000}%
\pgfsetstrokecolor{currentstroke}%
\pgfsetdash{}{0pt}%
\pgfpathmoveto{\pgfqpoint{4.731407in}{3.196471in}}%
\pgfpathlineto{\pgfqpoint{4.742301in}{3.187687in}}%
\pgfpathlineto{\pgfqpoint{4.753215in}{3.178488in}}%
\pgfpathlineto{\pgfqpoint{4.786182in}{3.167476in}}%
\pgfpathlineto{\pgfqpoint{4.819123in}{3.156749in}}%
\pgfpathlineto{\pgfqpoint{4.808147in}{3.164632in}}%
\pgfpathlineto{\pgfqpoint{4.797189in}{3.171739in}}%
\pgfpathlineto{\pgfqpoint{4.764312in}{3.184136in}}%
\pgfpathlineto{\pgfqpoint{4.731407in}{3.196471in}}%
\pgfpathclose%
\pgfusepath{fill}%
\end{pgfscope}%
\begin{pgfscope}%
\pgfpathrectangle{\pgfqpoint{1.020000in}{0.880000in}}{\pgfqpoint{6.160000in}{6.160000in}}%
\pgfusepath{clip}%
\pgfsetbuttcap%
\pgfsetroundjoin%
\definecolor{currentfill}{rgb}{0.958279,0.604335,0.483297}%
\pgfsetfillcolor{currentfill}%
\pgfsetlinewidth{0.000000pt}%
\definecolor{currentstroke}{rgb}{0.000000,0.000000,0.000000}%
\pgfsetstrokecolor{currentstroke}%
\pgfsetdash{}{0pt}%
\pgfpathmoveto{\pgfqpoint{3.288435in}{4.707133in}}%
\pgfpathlineto{\pgfqpoint{3.298371in}{4.646640in}}%
\pgfpathlineto{\pgfqpoint{3.308330in}{4.584429in}}%
\pgfpathlineto{\pgfqpoint{3.342036in}{4.556603in}}%
\pgfpathlineto{\pgfqpoint{3.375726in}{4.525780in}}%
\pgfpathlineto{\pgfqpoint{3.365769in}{4.585085in}}%
\pgfpathlineto{\pgfqpoint{3.355834in}{4.642806in}}%
\pgfpathlineto{\pgfqpoint{3.322143in}{4.676597in}}%
\pgfpathlineto{\pgfqpoint{3.288435in}{4.707133in}}%
\pgfpathclose%
\pgfusepath{fill}%
\end{pgfscope}%
\begin{pgfscope}%
\pgfpathrectangle{\pgfqpoint{1.020000in}{0.880000in}}{\pgfqpoint{6.160000in}{6.160000in}}%
\pgfusepath{clip}%
\pgfsetbuttcap%
\pgfsetroundjoin%
\definecolor{currentfill}{rgb}{0.708720,0.805721,0.981117}%
\pgfsetfillcolor{currentfill}%
\pgfsetlinewidth{0.000000pt}%
\definecolor{currentstroke}{rgb}{0.000000,0.000000,0.000000}%
\pgfsetstrokecolor{currentstroke}%
\pgfsetdash{}{0pt}%
\pgfpathmoveto{\pgfqpoint{3.811102in}{3.746340in}}%
\pgfpathlineto{\pgfqpoint{3.821181in}{3.716508in}}%
\pgfpathlineto{\pgfqpoint{3.831267in}{3.689090in}}%
\pgfpathlineto{\pgfqpoint{3.864660in}{3.668867in}}%
\pgfpathlineto{\pgfqpoint{3.898017in}{3.648768in}}%
\pgfpathlineto{\pgfqpoint{3.887895in}{3.673773in}}%
\pgfpathlineto{\pgfqpoint{3.877783in}{3.700942in}}%
\pgfpathlineto{\pgfqpoint{3.844461in}{3.723497in}}%
\pgfpathlineto{\pgfqpoint{3.811102in}{3.746340in}}%
\pgfpathclose%
\pgfusepath{fill}%
\end{pgfscope}%
\begin{pgfscope}%
\pgfpathrectangle{\pgfqpoint{1.020000in}{0.880000in}}{\pgfqpoint{6.160000in}{6.160000in}}%
\pgfusepath{clip}%
\pgfsetbuttcap%
\pgfsetroundjoin%
\definecolor{currentfill}{rgb}{0.902659,0.447939,0.349721}%
\pgfsetfillcolor{currentfill}%
\pgfsetlinewidth{0.000000pt}%
\definecolor{currentstroke}{rgb}{0.000000,0.000000,0.000000}%
\pgfsetstrokecolor{currentstroke}%
\pgfsetdash{}{0pt}%
\pgfpathmoveto{\pgfqpoint{2.543168in}{4.816133in}}%
\pgfpathlineto{\pgfqpoint{2.552271in}{4.786852in}}%
\pgfpathlineto{\pgfqpoint{2.561444in}{4.753833in}}%
\pgfpathlineto{\pgfqpoint{2.594504in}{4.799138in}}%
\pgfpathlineto{\pgfqpoint{2.627587in}{4.843369in}}%
\pgfpathlineto{\pgfqpoint{2.618347in}{4.877532in}}%
\pgfpathlineto{\pgfqpoint{2.609180in}{4.907563in}}%
\pgfpathlineto{\pgfqpoint{2.576163in}{4.862399in}}%
\pgfpathlineto{\pgfqpoint{2.543168in}{4.816133in}}%
\pgfpathclose%
\pgfusepath{fill}%
\end{pgfscope}%
\begin{pgfscope}%
\pgfpathrectangle{\pgfqpoint{1.020000in}{0.880000in}}{\pgfqpoint{6.160000in}{6.160000in}}%
\pgfusepath{clip}%
\pgfsetbuttcap%
\pgfsetroundjoin%
\definecolor{currentfill}{rgb}{0.915157,0.476927,0.372179}%
\pgfsetfillcolor{currentfill}%
\pgfsetlinewidth{0.000000pt}%
\definecolor{currentstroke}{rgb}{0.000000,0.000000,0.000000}%
\pgfsetstrokecolor{currentstroke}%
\pgfsetdash{}{0pt}%
\pgfpathmoveto{\pgfqpoint{3.201226in}{4.874937in}}%
\pgfpathlineto{\pgfqpoint{3.211094in}{4.817508in}}%
\pgfpathlineto{\pgfqpoint{3.220997in}{4.757181in}}%
\pgfpathlineto{\pgfqpoint{3.254718in}{4.734089in}}%
\pgfpathlineto{\pgfqpoint{3.288435in}{4.707133in}}%
\pgfpathlineto{\pgfqpoint{3.278528in}{4.765381in}}%
\pgfpathlineto{\pgfqpoint{3.268652in}{4.820879in}}%
\pgfpathlineto{\pgfqpoint{3.234940in}{4.849982in}}%
\pgfpathlineto{\pgfqpoint{3.201226in}{4.874937in}}%
\pgfpathclose%
\pgfusepath{fill}%
\end{pgfscope}%
\begin{pgfscope}%
\pgfpathrectangle{\pgfqpoint{1.020000in}{0.880000in}}{\pgfqpoint{6.160000in}{6.160000in}}%
\pgfusepath{clip}%
\pgfsetbuttcap%
\pgfsetroundjoin%
\definecolor{currentfill}{rgb}{0.852378,0.346492,0.280346}%
\pgfsetfillcolor{currentfill}%
\pgfsetlinewidth{0.000000pt}%
\definecolor{currentstroke}{rgb}{0.000000,0.000000,0.000000}%
\pgfsetstrokecolor{currentstroke}%
\pgfsetdash{}{0pt}%
\pgfpathmoveto{\pgfqpoint{3.114239in}{5.017950in}}%
\pgfpathlineto{\pgfqpoint{3.124003in}{4.966667in}}%
\pgfpathlineto{\pgfqpoint{3.133816in}{4.911401in}}%
\pgfpathlineto{\pgfqpoint{3.167516in}{4.895480in}}%
\pgfpathlineto{\pgfqpoint{3.201226in}{4.874937in}}%
\pgfpathlineto{\pgfqpoint{3.191398in}{4.928969in}}%
\pgfpathlineto{\pgfqpoint{3.181616in}{4.979139in}}%
\pgfpathlineto{\pgfqpoint{3.147921in}{5.000989in}}%
\pgfpathlineto{\pgfqpoint{3.114239in}{5.017950in}}%
\pgfpathclose%
\pgfusepath{fill}%
\end{pgfscope}%
\begin{pgfscope}%
\pgfpathrectangle{\pgfqpoint{1.020000in}{0.880000in}}{\pgfqpoint{6.160000in}{6.160000in}}%
\pgfusepath{clip}%
\pgfsetbuttcap%
\pgfsetroundjoin%
\definecolor{currentfill}{rgb}{0.763520,0.178667,0.193396}%
\pgfsetfillcolor{currentfill}%
\pgfsetlinewidth{0.000000pt}%
\definecolor{currentstroke}{rgb}{0.000000,0.000000,0.000000}%
\pgfsetstrokecolor{currentstroke}%
\pgfsetdash{}{0pt}%
\pgfpathmoveto{\pgfqpoint{2.808053in}{5.127843in}}%
\pgfpathlineto{\pgfqpoint{2.817396in}{5.096110in}}%
\pgfpathlineto{\pgfqpoint{2.826815in}{5.059281in}}%
\pgfpathlineto{\pgfqpoint{2.860173in}{5.082792in}}%
\pgfpathlineto{\pgfqpoint{2.893578in}{5.101706in}}%
\pgfpathlineto{\pgfqpoint{2.884108in}{5.139000in}}%
\pgfpathlineto{\pgfqpoint{2.874715in}{5.171001in}}%
\pgfpathlineto{\pgfqpoint{2.841360in}{5.151777in}}%
\pgfpathlineto{\pgfqpoint{2.808053in}{5.127843in}}%
\pgfpathclose%
\pgfusepath{fill}%
\end{pgfscope}%
\begin{pgfscope}%
\pgfpathrectangle{\pgfqpoint{1.020000in}{0.880000in}}{\pgfqpoint{6.160000in}{6.160000in}}%
\pgfusepath{clip}%
\pgfsetbuttcap%
\pgfsetroundjoin%
\definecolor{currentfill}{rgb}{0.383662,0.510183,0.917831}%
\pgfsetfillcolor{currentfill}%
\pgfsetlinewidth{0.000000pt}%
\definecolor{currentstroke}{rgb}{0.000000,0.000000,0.000000}%
\pgfsetstrokecolor{currentstroke}%
\pgfsetdash{}{0pt}%
\pgfpathmoveto{\pgfqpoint{4.884925in}{3.136253in}}%
\pgfpathlineto{\pgfqpoint{4.895982in}{3.129068in}}%
\pgfpathlineto{\pgfqpoint{4.907053in}{3.120718in}}%
\pgfpathlineto{\pgfqpoint{4.939983in}{3.112652in}}%
\pgfpathlineto{\pgfqpoint{4.972891in}{3.105227in}}%
\pgfpathlineto{\pgfqpoint{4.961755in}{3.112165in}}%
\pgfpathlineto{\pgfqpoint{4.950629in}{3.117535in}}%
\pgfpathlineto{\pgfqpoint{4.917789in}{3.126618in}}%
\pgfpathlineto{\pgfqpoint{4.884925in}{3.136253in}}%
\pgfpathclose%
\pgfusepath{fill}%
\end{pgfscope}%
\begin{pgfscope}%
\pgfpathrectangle{\pgfqpoint{1.020000in}{0.880000in}}{\pgfqpoint{6.160000in}{6.160000in}}%
\pgfusepath{clip}%
\pgfsetbuttcap%
\pgfsetroundjoin%
\definecolor{currentfill}{rgb}{0.865391,0.371128,0.295769}%
\pgfsetfillcolor{currentfill}%
\pgfsetlinewidth{0.000000pt}%
\definecolor{currentstroke}{rgb}{0.000000,0.000000,0.000000}%
\pgfsetstrokecolor{currentstroke}%
\pgfsetdash{}{0pt}%
\pgfpathmoveto{\pgfqpoint{2.609180in}{4.907563in}}%
\pgfpathlineto{\pgfqpoint{2.618347in}{4.877532in}}%
\pgfpathlineto{\pgfqpoint{2.627587in}{4.843369in}}%
\pgfpathlineto{\pgfqpoint{2.660696in}{4.886042in}}%
\pgfpathlineto{\pgfqpoint{2.693839in}{4.926670in}}%
\pgfpathlineto{\pgfqpoint{2.684535in}{4.961887in}}%
\pgfpathlineto{\pgfqpoint{2.675306in}{4.992601in}}%
\pgfpathlineto{\pgfqpoint{2.642226in}{4.951130in}}%
\pgfpathlineto{\pgfqpoint{2.609180in}{4.907563in}}%
\pgfpathclose%
\pgfusepath{fill}%
\end{pgfscope}%
\begin{pgfscope}%
\pgfpathrectangle{\pgfqpoint{1.020000in}{0.880000in}}{\pgfqpoint{6.160000in}{6.160000in}}%
\pgfusepath{clip}%
\pgfsetbuttcap%
\pgfsetroundjoin%
\definecolor{currentfill}{rgb}{0.795938,0.241845,0.220830}%
\pgfsetfillcolor{currentfill}%
\pgfsetlinewidth{0.000000pt}%
\definecolor{currentstroke}{rgb}{0.000000,0.000000,0.000000}%
\pgfsetstrokecolor{currentstroke}%
\pgfsetdash{}{0pt}%
\pgfpathmoveto{\pgfqpoint{2.741588in}{5.067282in}}%
\pgfpathlineto{\pgfqpoint{2.750875in}{5.035988in}}%
\pgfpathlineto{\pgfqpoint{2.760239in}{4.999866in}}%
\pgfpathlineto{\pgfqpoint{2.793504in}{5.031511in}}%
\pgfpathlineto{\pgfqpoint{2.826815in}{5.059281in}}%
\pgfpathlineto{\pgfqpoint{2.817396in}{5.096110in}}%
\pgfpathlineto{\pgfqpoint{2.808053in}{5.127843in}}%
\pgfpathlineto{\pgfqpoint{2.774797in}{5.099547in}}%
\pgfpathlineto{\pgfqpoint{2.741588in}{5.067282in}}%
\pgfpathclose%
\pgfusepath{fill}%
\end{pgfscope}%
\begin{pgfscope}%
\pgfpathrectangle{\pgfqpoint{1.020000in}{0.880000in}}{\pgfqpoint{6.160000in}{6.160000in}}%
\pgfusepath{clip}%
\pgfsetbuttcap%
\pgfsetroundjoin%
\definecolor{currentfill}{rgb}{0.318832,0.426605,0.859857}%
\pgfsetfillcolor{currentfill}%
\pgfsetlinewidth{0.000000pt}%
\definecolor{currentstroke}{rgb}{0.000000,0.000000,0.000000}%
\pgfsetstrokecolor{currentstroke}%
\pgfsetdash{}{0pt}%
\pgfpathmoveto{\pgfqpoint{5.917480in}{3.007910in}}%
\pgfpathlineto{\pgfqpoint{5.929423in}{2.992501in}}%
\pgfpathlineto{\pgfqpoint{5.941390in}{2.977115in}}%
\pgfpathlineto{\pgfqpoint{5.974066in}{2.978936in}}%
\pgfpathlineto{\pgfqpoint{6.006722in}{2.980776in}}%
\pgfpathlineto{\pgfqpoint{5.994702in}{2.996086in}}%
\pgfpathlineto{\pgfqpoint{5.982706in}{3.011425in}}%
\pgfpathlineto{\pgfqpoint{5.950103in}{3.009661in}}%
\pgfpathlineto{\pgfqpoint{5.917480in}{3.007910in}}%
\pgfpathclose%
\pgfusepath{fill}%
\end{pgfscope}%
\begin{pgfscope}%
\pgfpathrectangle{\pgfqpoint{1.020000in}{0.880000in}}{\pgfqpoint{6.160000in}{6.160000in}}%
\pgfusepath{clip}%
\pgfsetbuttcap%
\pgfsetroundjoin%
\definecolor{currentfill}{rgb}{0.830187,0.304733,0.254891}%
\pgfsetfillcolor{currentfill}%
\pgfsetlinewidth{0.000000pt}%
\definecolor{currentstroke}{rgb}{0.000000,0.000000,0.000000}%
\pgfsetstrokecolor{currentstroke}%
\pgfsetdash{}{0pt}%
\pgfpathmoveto{\pgfqpoint{2.675306in}{4.992601in}}%
\pgfpathlineto{\pgfqpoint{2.684535in}{4.961887in}}%
\pgfpathlineto{\pgfqpoint{2.693839in}{4.926670in}}%
\pgfpathlineto{\pgfqpoint{2.727019in}{4.964769in}}%
\pgfpathlineto{\pgfqpoint{2.760239in}{4.999866in}}%
\pgfpathlineto{\pgfqpoint{2.750875in}{5.035988in}}%
\pgfpathlineto{\pgfqpoint{2.741588in}{5.067282in}}%
\pgfpathlineto{\pgfqpoint{2.708426in}{5.031479in}}%
\pgfpathlineto{\pgfqpoint{2.675306in}{4.992601in}}%
\pgfpathclose%
\pgfusepath{fill}%
\end{pgfscope}%
\begin{pgfscope}%
\pgfpathrectangle{\pgfqpoint{1.020000in}{0.880000in}}{\pgfqpoint{6.160000in}{6.160000in}}%
\pgfusepath{clip}%
\pgfsetbuttcap%
\pgfsetroundjoin%
\definecolor{currentfill}{rgb}{0.328604,0.439712,0.869587}%
\pgfsetfillcolor{currentfill}%
\pgfsetlinewidth{0.000000pt}%
\definecolor{currentstroke}{rgb}{0.000000,0.000000,0.000000}%
\pgfsetstrokecolor{currentstroke}%
\pgfsetdash{}{0pt}%
\pgfpathmoveto{\pgfqpoint{5.697806in}{3.028499in}}%
\pgfpathlineto{\pgfqpoint{5.709551in}{3.013190in}}%
\pgfpathlineto{\pgfqpoint{5.721317in}{2.997774in}}%
\pgfpathlineto{\pgfqpoint{5.754061in}{2.999404in}}%
\pgfpathlineto{\pgfqpoint{5.786785in}{3.001063in}}%
\pgfpathlineto{\pgfqpoint{5.774968in}{3.016523in}}%
\pgfpathlineto{\pgfqpoint{5.763172in}{3.031929in}}%
\pgfpathlineto{\pgfqpoint{5.730499in}{3.030213in}}%
\pgfpathlineto{\pgfqpoint{5.697806in}{3.028499in}}%
\pgfpathclose%
\pgfusepath{fill}%
\end{pgfscope}%
\begin{pgfscope}%
\pgfpathrectangle{\pgfqpoint{1.020000in}{0.880000in}}{\pgfqpoint{6.160000in}{6.160000in}}%
\pgfusepath{clip}%
\pgfsetbuttcap%
\pgfsetroundjoin%
\definecolor{currentfill}{rgb}{0.338377,0.452819,0.879317}%
\pgfsetfillcolor{currentfill}%
\pgfsetlinewidth{0.000000pt}%
\definecolor{currentstroke}{rgb}{0.000000,0.000000,0.000000}%
\pgfsetstrokecolor{currentstroke}%
\pgfsetdash{}{0pt}%
\pgfpathmoveto{\pgfqpoint{5.478079in}{3.046841in}}%
\pgfpathlineto{\pgfqpoint{5.489645in}{3.033256in}}%
\pgfpathlineto{\pgfqpoint{5.501227in}{3.019057in}}%
\pgfpathlineto{\pgfqpoint{5.534039in}{3.020419in}}%
\pgfpathlineto{\pgfqpoint{5.566831in}{3.021904in}}%
\pgfpathlineto{\pgfqpoint{5.555202in}{3.036480in}}%
\pgfpathlineto{\pgfqpoint{5.543589in}{3.050593in}}%
\pgfpathlineto{\pgfqpoint{5.510843in}{3.048662in}}%
\pgfpathlineto{\pgfqpoint{5.478079in}{3.046841in}}%
\pgfpathclose%
\pgfusepath{fill}%
\end{pgfscope}%
\begin{pgfscope}%
\pgfpathrectangle{\pgfqpoint{1.020000in}{0.880000in}}{\pgfqpoint{6.160000in}{6.160000in}}%
\pgfusepath{clip}%
\pgfsetbuttcap%
\pgfsetroundjoin%
\definecolor{currentfill}{rgb}{0.554312,0.690097,0.995516}%
\pgfsetfillcolor{currentfill}%
\pgfsetlinewidth{0.000000pt}%
\definecolor{currentstroke}{rgb}{0.000000,0.000000,0.000000}%
\pgfsetstrokecolor{currentstroke}%
\pgfsetdash{}{0pt}%
\pgfpathmoveto{\pgfqpoint{4.205014in}{3.443034in}}%
\pgfpathlineto{\pgfqpoint{4.215404in}{3.431074in}}%
\pgfpathlineto{\pgfqpoint{4.225813in}{3.420524in}}%
\pgfpathlineto{\pgfqpoint{4.259023in}{3.405761in}}%
\pgfpathlineto{\pgfqpoint{4.292203in}{3.391463in}}%
\pgfpathlineto{\pgfqpoint{4.281741in}{3.402519in}}%
\pgfpathlineto{\pgfqpoint{4.271299in}{3.414756in}}%
\pgfpathlineto{\pgfqpoint{4.238172in}{3.428645in}}%
\pgfpathlineto{\pgfqpoint{4.205014in}{3.443034in}}%
\pgfpathclose%
\pgfusepath{fill}%
\end{pgfscope}%
\begin{pgfscope}%
\pgfpathrectangle{\pgfqpoint{1.020000in}{0.880000in}}{\pgfqpoint{6.160000in}{6.160000in}}%
\pgfusepath{clip}%
\pgfsetbuttcap%
\pgfsetroundjoin%
\definecolor{currentfill}{rgb}{0.603162,0.731527,0.999565}%
\pgfsetfillcolor{currentfill}%
\pgfsetlinewidth{0.000000pt}%
\definecolor{currentstroke}{rgb}{0.000000,0.000000,0.000000}%
\pgfsetstrokecolor{currentstroke}%
\pgfsetdash{}{0pt}%
\pgfpathmoveto{\pgfqpoint{4.051555in}{3.534897in}}%
\pgfpathlineto{\pgfqpoint{4.061806in}{3.519161in}}%
\pgfpathlineto{\pgfqpoint{4.072073in}{3.505351in}}%
\pgfpathlineto{\pgfqpoint{4.105356in}{3.489135in}}%
\pgfpathlineto{\pgfqpoint{4.138607in}{3.473307in}}%
\pgfpathlineto{\pgfqpoint{4.128289in}{3.486785in}}%
\pgfpathlineto{\pgfqpoint{4.117988in}{3.501920in}}%
\pgfpathlineto{\pgfqpoint{4.084788in}{3.518145in}}%
\pgfpathlineto{\pgfqpoint{4.051555in}{3.534897in}}%
\pgfpathclose%
\pgfusepath{fill}%
\end{pgfscope}%
\begin{pgfscope}%
\pgfpathrectangle{\pgfqpoint{1.020000in}{0.880000in}}{\pgfqpoint{6.160000in}{6.160000in}}%
\pgfusepath{clip}%
\pgfsetbuttcap%
\pgfsetroundjoin%
\definecolor{currentfill}{rgb}{0.348323,0.465711,0.888346}%
\pgfsetfillcolor{currentfill}%
\pgfsetlinewidth{0.000000pt}%
\definecolor{currentstroke}{rgb}{0.000000,0.000000,0.000000}%
\pgfsetstrokecolor{currentstroke}%
\pgfsetdash{}{0pt}%
\pgfpathmoveto{\pgfqpoint{5.258307in}{3.062416in}}%
\pgfpathlineto{\pgfqpoint{5.269709in}{3.052681in}}%
\pgfpathlineto{\pgfqpoint{5.281119in}{3.041509in}}%
\pgfpathlineto{\pgfqpoint{5.313988in}{3.041421in}}%
\pgfpathlineto{\pgfqpoint{5.346840in}{3.041805in}}%
\pgfpathlineto{\pgfqpoint{5.335380in}{3.053342in}}%
\pgfpathlineto{\pgfqpoint{5.323926in}{3.063537in}}%
\pgfpathlineto{\pgfqpoint{5.291124in}{3.062668in}}%
\pgfpathlineto{\pgfqpoint{5.258307in}{3.062416in}}%
\pgfpathclose%
\pgfusepath{fill}%
\end{pgfscope}%
\begin{pgfscope}%
\pgfpathrectangle{\pgfqpoint{1.020000in}{0.880000in}}{\pgfqpoint{6.160000in}{6.160000in}}%
\pgfusepath{clip}%
\pgfsetbuttcap%
\pgfsetroundjoin%
\definecolor{currentfill}{rgb}{0.855378,0.863778,0.876587}%
\pgfsetfillcolor{currentfill}%
\pgfsetlinewidth{0.000000pt}%
\definecolor{currentstroke}{rgb}{0.000000,0.000000,0.000000}%
\pgfsetstrokecolor{currentstroke}%
\pgfsetdash{}{0pt}%
\pgfpathmoveto{\pgfqpoint{3.570173in}{4.063422in}}%
\pgfpathlineto{\pgfqpoint{3.580183in}{4.014312in}}%
\pgfpathlineto{\pgfqpoint{3.590193in}{3.967213in}}%
\pgfpathlineto{\pgfqpoint{3.623743in}{3.942565in}}%
\pgfpathlineto{\pgfqpoint{3.657261in}{3.917178in}}%
\pgfpathlineto{\pgfqpoint{3.647240in}{3.960469in}}%
\pgfpathlineto{\pgfqpoint{3.637222in}{4.005654in}}%
\pgfpathlineto{\pgfqpoint{3.603715in}{4.034910in}}%
\pgfpathlineto{\pgfqpoint{3.570173in}{4.063422in}}%
\pgfpathclose%
\pgfusepath{fill}%
\end{pgfscope}%
\begin{pgfscope}%
\pgfpathrectangle{\pgfqpoint{1.020000in}{0.880000in}}{\pgfqpoint{6.160000in}{6.160000in}}%
\pgfusepath{clip}%
\pgfsetbuttcap%
\pgfsetroundjoin%
\definecolor{currentfill}{rgb}{0.916071,0.833977,0.788693}%
\pgfsetfillcolor{currentfill}%
\pgfsetlinewidth{0.000000pt}%
\definecolor{currentstroke}{rgb}{0.000000,0.000000,0.000000}%
\pgfsetstrokecolor{currentstroke}%
\pgfsetdash{}{0pt}%
\pgfpathmoveto{\pgfqpoint{3.482981in}{4.227871in}}%
\pgfpathlineto{\pgfqpoint{3.492985in}{4.171858in}}%
\pgfpathlineto{\pgfqpoint{3.502991in}{4.117180in}}%
\pgfpathlineto{\pgfqpoint{3.536597in}{4.090932in}}%
\pgfpathlineto{\pgfqpoint{3.570173in}{4.063422in}}%
\pgfpathlineto{\pgfqpoint{3.560165in}{4.114199in}}%
\pgfpathlineto{\pgfqpoint{3.550161in}{4.166267in}}%
\pgfpathlineto{\pgfqpoint{3.516587in}{4.197750in}}%
\pgfpathlineto{\pgfqpoint{3.482981in}{4.227871in}}%
\pgfpathclose%
\pgfusepath{fill}%
\end{pgfscope}%
\begin{pgfscope}%
\pgfpathrectangle{\pgfqpoint{1.020000in}{0.880000in}}{\pgfqpoint{6.160000in}{6.160000in}}%
\pgfusepath{clip}%
\pgfsetbuttcap%
\pgfsetroundjoin%
\definecolor{currentfill}{rgb}{0.510824,0.649397,0.985079}%
\pgfsetfillcolor{currentfill}%
\pgfsetlinewidth{0.000000pt}%
\definecolor{currentstroke}{rgb}{0.000000,0.000000,0.000000}%
\pgfsetstrokecolor{currentstroke}%
\pgfsetdash{}{0pt}%
\pgfpathmoveto{\pgfqpoint{4.358474in}{3.364226in}}%
\pgfpathlineto{\pgfqpoint{4.369010in}{3.353624in}}%
\pgfpathlineto{\pgfqpoint{4.379568in}{3.343938in}}%
\pgfpathlineto{\pgfqpoint{4.412712in}{3.330216in}}%
\pgfpathlineto{\pgfqpoint{4.445826in}{3.316906in}}%
\pgfpathlineto{\pgfqpoint{4.435215in}{3.327316in}}%
\pgfpathlineto{\pgfqpoint{4.424627in}{3.338457in}}%
\pgfpathlineto{\pgfqpoint{4.391565in}{3.351198in}}%
\pgfpathlineto{\pgfqpoint{4.358474in}{3.364226in}}%
\pgfpathclose%
\pgfusepath{fill}%
\end{pgfscope}%
\begin{pgfscope}%
\pgfpathrectangle{\pgfqpoint{1.020000in}{0.880000in}}{\pgfqpoint{6.160000in}{6.160000in}}%
\pgfusepath{clip}%
\pgfsetbuttcap%
\pgfsetroundjoin%
\definecolor{currentfill}{rgb}{0.363461,0.484784,0.901019}%
\pgfsetfillcolor{currentfill}%
\pgfsetlinewidth{0.000000pt}%
\definecolor{currentstroke}{rgb}{0.000000,0.000000,0.000000}%
\pgfsetstrokecolor{currentstroke}%
\pgfsetdash{}{0pt}%
\pgfpathmoveto{\pgfqpoint{5.038646in}{3.092554in}}%
\pgfpathlineto{\pgfqpoint{5.049859in}{3.085259in}}%
\pgfpathlineto{\pgfqpoint{5.061080in}{3.076358in}}%
\pgfpathlineto{\pgfqpoint{5.093992in}{3.072102in}}%
\pgfpathlineto{\pgfqpoint{5.126887in}{3.068622in}}%
\pgfpathlineto{\pgfqpoint{5.115605in}{3.076866in}}%
\pgfpathlineto{\pgfqpoint{5.104328in}{3.083265in}}%
\pgfpathlineto{\pgfqpoint{5.071496in}{3.087451in}}%
\pgfpathlineto{\pgfqpoint{5.038646in}{3.092554in}}%
\pgfpathclose%
\pgfusepath{fill}%
\end{pgfscope}%
\begin{pgfscope}%
\pgfpathrectangle{\pgfqpoint{1.020000in}{0.880000in}}{\pgfqpoint{6.160000in}{6.160000in}}%
\pgfusepath{clip}%
\pgfsetbuttcap%
\pgfsetroundjoin%
\definecolor{currentfill}{rgb}{0.785153,0.220851,0.211673}%
\pgfsetfillcolor{currentfill}%
\pgfsetlinewidth{0.000000pt}%
\definecolor{currentstroke}{rgb}{0.000000,0.000000,0.000000}%
\pgfsetstrokecolor{currentstroke}%
\pgfsetdash{}{0pt}%
\pgfpathmoveto{\pgfqpoint{2.960524in}{5.124653in}}%
\pgfpathlineto{\pgfqpoint{2.970108in}{5.082020in}}%
\pgfpathlineto{\pgfqpoint{2.979757in}{5.034553in}}%
\pgfpathlineto{\pgfqpoint{3.013330in}{5.038160in}}%
\pgfpathlineto{\pgfqpoint{3.046938in}{5.036605in}}%
\pgfpathlineto{\pgfqpoint{3.037256in}{5.084006in}}%
\pgfpathlineto{\pgfqpoint{3.027636in}{5.126554in}}%
\pgfpathlineto{\pgfqpoint{2.994061in}{5.128292in}}%
\pgfpathlineto{\pgfqpoint{2.960524in}{5.124653in}}%
\pgfpathclose%
\pgfusepath{fill}%
\end{pgfscope}%
\begin{pgfscope}%
\pgfpathrectangle{\pgfqpoint{1.020000in}{0.880000in}}{\pgfqpoint{6.160000in}{6.160000in}}%
\pgfusepath{clip}%
\pgfsetbuttcap%
\pgfsetroundjoin%
\definecolor{currentfill}{rgb}{0.467678,0.605591,0.968546}%
\pgfsetfillcolor{currentfill}%
\pgfsetlinewidth{0.000000pt}%
\definecolor{currentstroke}{rgb}{0.000000,0.000000,0.000000}%
\pgfsetstrokecolor{currentstroke}%
\pgfsetdash{}{0pt}%
\pgfpathmoveto{\pgfqpoint{4.511967in}{3.291212in}}%
\pgfpathlineto{\pgfqpoint{4.522652in}{3.280923in}}%
\pgfpathlineto{\pgfqpoint{4.533359in}{3.271077in}}%
\pgfpathlineto{\pgfqpoint{4.566439in}{3.258193in}}%
\pgfpathlineto{\pgfqpoint{4.599490in}{3.245611in}}%
\pgfpathlineto{\pgfqpoint{4.588730in}{3.255693in}}%
\pgfpathlineto{\pgfqpoint{4.577992in}{3.266016in}}%
\pgfpathlineto{\pgfqpoint{4.544994in}{3.278615in}}%
\pgfpathlineto{\pgfqpoint{4.511967in}{3.291212in}}%
\pgfpathclose%
\pgfusepath{fill}%
\end{pgfscope}%
\begin{pgfscope}%
\pgfpathrectangle{\pgfqpoint{1.020000in}{0.880000in}}{\pgfqpoint{6.160000in}{6.160000in}}%
\pgfusepath{clip}%
\pgfsetbuttcap%
\pgfsetroundjoin%
\definecolor{currentfill}{rgb}{0.667253,0.779176,0.992959}%
\pgfsetfillcolor{currentfill}%
\pgfsetlinewidth{0.000000pt}%
\definecolor{currentstroke}{rgb}{0.000000,0.000000,0.000000}%
\pgfsetstrokecolor{currentstroke}%
\pgfsetdash{}{0pt}%
\pgfpathmoveto{\pgfqpoint{3.898017in}{3.648768in}}%
\pgfpathlineto{\pgfqpoint{3.908150in}{3.626049in}}%
\pgfpathlineto{\pgfqpoint{3.918294in}{3.605712in}}%
\pgfpathlineto{\pgfqpoint{3.951659in}{3.587618in}}%
\pgfpathlineto{\pgfqpoint{3.984991in}{3.569718in}}%
\pgfpathlineto{\pgfqpoint{3.974803in}{3.588528in}}%
\pgfpathlineto{\pgfqpoint{3.964629in}{3.609444in}}%
\pgfpathlineto{\pgfqpoint{3.931340in}{3.628922in}}%
\pgfpathlineto{\pgfqpoint{3.898017in}{3.648768in}}%
\pgfpathclose%
\pgfusepath{fill}%
\end{pgfscope}%
\begin{pgfscope}%
\pgfpathrectangle{\pgfqpoint{1.020000in}{0.880000in}}{\pgfqpoint{6.160000in}{6.160000in}}%
\pgfusepath{clip}%
\pgfsetbuttcap%
\pgfsetroundjoin%
\definecolor{currentfill}{rgb}{0.959518,0.766973,0.674145}%
\pgfsetfillcolor{currentfill}%
\pgfsetlinewidth{0.000000pt}%
\definecolor{currentstroke}{rgb}{0.000000,0.000000,0.000000}%
\pgfsetstrokecolor{currentstroke}%
\pgfsetdash{}{0pt}%
\pgfpathmoveto{\pgfqpoint{3.395686in}{4.404450in}}%
\pgfpathlineto{\pgfqpoint{3.405683in}{4.343442in}}%
\pgfpathlineto{\pgfqpoint{3.415686in}{4.282866in}}%
\pgfpathlineto{\pgfqpoint{3.449346in}{4.256337in}}%
\pgfpathlineto{\pgfqpoint{3.482981in}{4.227871in}}%
\pgfpathlineto{\pgfqpoint{3.472982in}{4.284792in}}%
\pgfpathlineto{\pgfqpoint{3.462989in}{4.342175in}}%
\pgfpathlineto{\pgfqpoint{3.429350in}{4.374379in}}%
\pgfpathlineto{\pgfqpoint{3.395686in}{4.404450in}}%
\pgfpathclose%
\pgfusepath{fill}%
\end{pgfscope}%
\begin{pgfscope}%
\pgfpathrectangle{\pgfqpoint{1.020000in}{0.880000in}}{\pgfqpoint{6.160000in}{6.160000in}}%
\pgfusepath{clip}%
\pgfsetbuttcap%
\pgfsetroundjoin%
\definecolor{currentfill}{rgb}{0.796064,0.848693,0.933471}%
\pgfsetfillcolor{currentfill}%
\pgfsetlinewidth{0.000000pt}%
\definecolor{currentstroke}{rgb}{0.000000,0.000000,0.000000}%
\pgfsetstrokecolor{currentstroke}%
\pgfsetdash{}{0pt}%
\pgfpathmoveto{\pgfqpoint{3.657261in}{3.917178in}}%
\pgfpathlineto{\pgfqpoint{3.667284in}{3.876072in}}%
\pgfpathlineto{\pgfqpoint{3.677308in}{3.837413in}}%
\pgfpathlineto{\pgfqpoint{3.710809in}{3.815035in}}%
\pgfpathlineto{\pgfqpoint{3.744275in}{3.792277in}}%
\pgfpathlineto{\pgfqpoint{3.734229in}{3.827541in}}%
\pgfpathlineto{\pgfqpoint{3.724188in}{3.865067in}}%
\pgfpathlineto{\pgfqpoint{3.690743in}{3.891274in}}%
\pgfpathlineto{\pgfqpoint{3.657261in}{3.917178in}}%
\pgfpathclose%
\pgfusepath{fill}%
\end{pgfscope}%
\begin{pgfscope}%
\pgfpathrectangle{\pgfqpoint{1.020000in}{0.880000in}}{\pgfqpoint{6.160000in}{6.160000in}}%
\pgfusepath{clip}%
\pgfsetbuttcap%
\pgfsetroundjoin%
\definecolor{currentfill}{rgb}{0.430507,0.564883,0.948889}%
\pgfsetfillcolor{currentfill}%
\pgfsetlinewidth{0.000000pt}%
\definecolor{currentstroke}{rgb}{0.000000,0.000000,0.000000}%
\pgfsetstrokecolor{currentstroke}%
\pgfsetdash{}{0pt}%
\pgfpathmoveto{\pgfqpoint{4.665506in}{3.220972in}}%
\pgfpathlineto{\pgfqpoint{4.676342in}{3.211262in}}%
\pgfpathlineto{\pgfqpoint{4.687199in}{3.201436in}}%
\pgfpathlineto{\pgfqpoint{4.720220in}{3.189796in}}%
\pgfpathlineto{\pgfqpoint{4.753215in}{3.178488in}}%
\pgfpathlineto{\pgfqpoint{4.742301in}{3.187687in}}%
\pgfpathlineto{\pgfqpoint{4.731407in}{3.196471in}}%
\pgfpathlineto{\pgfqpoint{4.698471in}{3.208739in}}%
\pgfpathlineto{\pgfqpoint{4.665506in}{3.220972in}}%
\pgfpathclose%
\pgfusepath{fill}%
\end{pgfscope}%
\begin{pgfscope}%
\pgfpathrectangle{\pgfqpoint{1.020000in}{0.880000in}}{\pgfqpoint{6.160000in}{6.160000in}}%
\pgfusepath{clip}%
\pgfsetbuttcap%
\pgfsetroundjoin%
\definecolor{currentfill}{rgb}{0.968500,0.673977,0.556649}%
\pgfsetfillcolor{currentfill}%
\pgfsetlinewidth{0.000000pt}%
\definecolor{currentstroke}{rgb}{0.000000,0.000000,0.000000}%
\pgfsetstrokecolor{currentstroke}%
\pgfsetdash{}{0pt}%
\pgfpathmoveto{\pgfqpoint{3.308330in}{4.584429in}}%
\pgfpathlineto{\pgfqpoint{3.318306in}{4.521032in}}%
\pgfpathlineto{\pgfqpoint{3.328297in}{4.456986in}}%
\pgfpathlineto{\pgfqpoint{3.362000in}{4.432082in}}%
\pgfpathlineto{\pgfqpoint{3.395686in}{4.404450in}}%
\pgfpathlineto{\pgfqpoint{3.385699in}{4.465398in}}%
\pgfpathlineto{\pgfqpoint{3.375726in}{4.525780in}}%
\pgfpathlineto{\pgfqpoint{3.342036in}{4.556603in}}%
\pgfpathlineto{\pgfqpoint{3.308330in}{4.584429in}}%
\pgfpathclose%
\pgfusepath{fill}%
\end{pgfscope}%
\begin{pgfscope}%
\pgfpathrectangle{\pgfqpoint{1.020000in}{0.880000in}}{\pgfqpoint{6.160000in}{6.160000in}}%
\pgfusepath{clip}%
\pgfsetbuttcap%
\pgfsetroundjoin%
\definecolor{currentfill}{rgb}{0.968203,0.720844,0.612293}%
\pgfsetfillcolor{currentfill}%
\pgfsetlinewidth{0.000000pt}%
\definecolor{currentstroke}{rgb}{0.000000,0.000000,0.000000}%
\pgfsetstrokecolor{currentstroke}%
\pgfsetdash{}{0pt}%
\pgfpathmoveto{\pgfqpoint{2.296966in}{4.403962in}}%
\pgfpathlineto{\pgfqpoint{2.305904in}{4.373387in}}%
\pgfpathlineto{\pgfqpoint{2.314890in}{4.340856in}}%
\pgfpathlineto{\pgfqpoint{2.348085in}{4.378523in}}%
\pgfpathlineto{\pgfqpoint{2.381253in}{4.418006in}}%
\pgfpathlineto{\pgfqpoint{2.372188in}{4.452246in}}%
\pgfpathlineto{\pgfqpoint{2.363174in}{4.484205in}}%
\pgfpathlineto{\pgfqpoint{2.330085in}{4.443140in}}%
\pgfpathlineto{\pgfqpoint{2.296966in}{4.403962in}}%
\pgfpathclose%
\pgfusepath{fill}%
\end{pgfscope}%
\begin{pgfscope}%
\pgfpathrectangle{\pgfqpoint{1.020000in}{0.880000in}}{\pgfqpoint{6.160000in}{6.160000in}}%
\pgfusepath{clip}%
\pgfsetbuttcap%
\pgfsetroundjoin%
\definecolor{currentfill}{rgb}{0.318832,0.426605,0.859857}%
\pgfsetfillcolor{currentfill}%
\pgfsetlinewidth{0.000000pt}%
\definecolor{currentstroke}{rgb}{0.000000,0.000000,0.000000}%
\pgfsetstrokecolor{currentstroke}%
\pgfsetdash{}{0pt}%
\pgfpathmoveto{\pgfqpoint{5.852173in}{3.004452in}}%
\pgfpathlineto{\pgfqpoint{5.864064in}{2.988992in}}%
\pgfpathlineto{\pgfqpoint{5.875977in}{2.973541in}}%
\pgfpathlineto{\pgfqpoint{5.908693in}{2.975316in}}%
\pgfpathlineto{\pgfqpoint{5.941390in}{2.977115in}}%
\pgfpathlineto{\pgfqpoint{5.929423in}{2.992501in}}%
\pgfpathlineto{\pgfqpoint{5.917480in}{3.007910in}}%
\pgfpathlineto{\pgfqpoint{5.884836in}{3.006174in}}%
\pgfpathlineto{\pgfqpoint{5.852173in}{3.004452in}}%
\pgfpathclose%
\pgfusepath{fill}%
\end{pgfscope}%
\begin{pgfscope}%
\pgfpathrectangle{\pgfqpoint{1.020000in}{0.880000in}}{\pgfqpoint{6.160000in}{6.160000in}}%
\pgfusepath{clip}%
\pgfsetbuttcap%
\pgfsetroundjoin%
\definecolor{currentfill}{rgb}{0.399231,0.528528,0.928459}%
\pgfsetfillcolor{currentfill}%
\pgfsetlinewidth{0.000000pt}%
\definecolor{currentstroke}{rgb}{0.000000,0.000000,0.000000}%
\pgfsetstrokecolor{currentstroke}%
\pgfsetdash{}{0pt}%
\pgfpathmoveto{\pgfqpoint{4.819123in}{3.156749in}}%
\pgfpathlineto{\pgfqpoint{4.830115in}{3.148069in}}%
\pgfpathlineto{\pgfqpoint{4.841124in}{3.138608in}}%
\pgfpathlineto{\pgfqpoint{4.874101in}{3.129379in}}%
\pgfpathlineto{\pgfqpoint{4.907053in}{3.120718in}}%
\pgfpathlineto{\pgfqpoint{4.895982in}{3.129068in}}%
\pgfpathlineto{\pgfqpoint{4.884925in}{3.136253in}}%
\pgfpathlineto{\pgfqpoint{4.852037in}{3.146324in}}%
\pgfpathlineto{\pgfqpoint{4.819123in}{3.156749in}}%
\pgfpathclose%
\pgfusepath{fill}%
\end{pgfscope}%
\begin{pgfscope}%
\pgfpathrectangle{\pgfqpoint{1.020000in}{0.880000in}}{\pgfqpoint{6.160000in}{6.160000in}}%
\pgfusepath{clip}%
\pgfsetbuttcap%
\pgfsetroundjoin%
\definecolor{currentfill}{rgb}{0.328604,0.439712,0.869587}%
\pgfsetfillcolor{currentfill}%
\pgfsetlinewidth{0.000000pt}%
\definecolor{currentstroke}{rgb}{0.000000,0.000000,0.000000}%
\pgfsetstrokecolor{currentstroke}%
\pgfsetdash{}{0pt}%
\pgfpathmoveto{\pgfqpoint{5.632358in}{3.025119in}}%
\pgfpathlineto{\pgfqpoint{5.644054in}{3.009971in}}%
\pgfpathlineto{\pgfqpoint{5.655769in}{2.994647in}}%
\pgfpathlineto{\pgfqpoint{5.688553in}{2.996185in}}%
\pgfpathlineto{\pgfqpoint{5.721317in}{2.997774in}}%
\pgfpathlineto{\pgfqpoint{5.709551in}{3.013190in}}%
\pgfpathlineto{\pgfqpoint{5.697806in}{3.028499in}}%
\pgfpathlineto{\pgfqpoint{5.665092in}{3.026797in}}%
\pgfpathlineto{\pgfqpoint{5.632358in}{3.025119in}}%
\pgfpathclose%
\pgfusepath{fill}%
\end{pgfscope}%
\begin{pgfscope}%
\pgfpathrectangle{\pgfqpoint{1.020000in}{0.880000in}}{\pgfqpoint{6.160000in}{6.160000in}}%
\pgfusepath{clip}%
\pgfsetbuttcap%
\pgfsetroundjoin%
\definecolor{currentfill}{rgb}{0.968500,0.673977,0.556649}%
\pgfsetfillcolor{currentfill}%
\pgfsetlinewidth{0.000000pt}%
\definecolor{currentstroke}{rgb}{0.000000,0.000000,0.000000}%
\pgfsetstrokecolor{currentstroke}%
\pgfsetdash{}{0pt}%
\pgfpathmoveto{\pgfqpoint{2.363174in}{4.484205in}}%
\pgfpathlineto{\pgfqpoint{2.372188in}{4.452246in}}%
\pgfpathlineto{\pgfqpoint{2.381253in}{4.418006in}}%
\pgfpathlineto{\pgfqpoint{2.414397in}{4.459071in}}%
\pgfpathlineto{\pgfqpoint{2.447523in}{4.501447in}}%
\pgfpathlineto{\pgfqpoint{2.438379in}{4.537532in}}%
\pgfpathlineto{\pgfqpoint{2.429290in}{4.570984in}}%
\pgfpathlineto{\pgfqpoint{2.396241in}{4.526914in}}%
\pgfpathlineto{\pgfqpoint{2.363174in}{4.484205in}}%
\pgfpathclose%
\pgfusepath{fill}%
\end{pgfscope}%
\begin{pgfscope}%
\pgfpathrectangle{\pgfqpoint{1.020000in}{0.880000in}}{\pgfqpoint{6.160000in}{6.160000in}}%
\pgfusepath{clip}%
\pgfsetbuttcap%
\pgfsetroundjoin%
\definecolor{currentfill}{rgb}{0.839365,0.321856,0.264924}%
\pgfsetfillcolor{currentfill}%
\pgfsetlinewidth{0.000000pt}%
\definecolor{currentstroke}{rgb}{0.000000,0.000000,0.000000}%
\pgfsetstrokecolor{currentstroke}%
\pgfsetdash{}{0pt}%
\pgfpathmoveto{\pgfqpoint{3.046938in}{5.036605in}}%
\pgfpathlineto{\pgfqpoint{3.056678in}{4.984731in}}%
\pgfpathlineto{\pgfqpoint{3.066470in}{4.928815in}}%
\pgfpathlineto{\pgfqpoint{3.100132in}{4.922544in}}%
\pgfpathlineto{\pgfqpoint{3.133816in}{4.911401in}}%
\pgfpathlineto{\pgfqpoint{3.124003in}{4.966667in}}%
\pgfpathlineto{\pgfqpoint{3.114239in}{5.017950in}}%
\pgfpathlineto{\pgfqpoint{3.080576in}{5.029855in}}%
\pgfpathlineto{\pgfqpoint{3.046938in}{5.036605in}}%
\pgfpathclose%
\pgfusepath{fill}%
\end{pgfscope}%
\begin{pgfscope}%
\pgfpathrectangle{\pgfqpoint{1.020000in}{0.880000in}}{\pgfqpoint{6.160000in}{6.160000in}}%
\pgfusepath{clip}%
\pgfsetbuttcap%
\pgfsetroundjoin%
\definecolor{currentfill}{rgb}{0.343278,0.459354,0.884122}%
\pgfsetfillcolor{currentfill}%
\pgfsetlinewidth{0.000000pt}%
\definecolor{currentstroke}{rgb}{0.000000,0.000000,0.000000}%
\pgfsetstrokecolor{currentstroke}%
\pgfsetdash{}{0pt}%
\pgfpathmoveto{\pgfqpoint{5.412494in}{3.043751in}}%
\pgfpathlineto{\pgfqpoint{5.424014in}{3.030684in}}%
\pgfpathlineto{\pgfqpoint{5.435546in}{3.016860in}}%
\pgfpathlineto{\pgfqpoint{5.468396in}{3.017856in}}%
\pgfpathlineto{\pgfqpoint{5.501227in}{3.019057in}}%
\pgfpathlineto{\pgfqpoint{5.489645in}{3.033256in}}%
\pgfpathlineto{\pgfqpoint{5.478079in}{3.046841in}}%
\pgfpathlineto{\pgfqpoint{5.445295in}{3.045184in}}%
\pgfpathlineto{\pgfqpoint{5.412494in}{3.043751in}}%
\pgfpathclose%
\pgfusepath{fill}%
\end{pgfscope}%
\begin{pgfscope}%
\pgfpathrectangle{\pgfqpoint{1.020000in}{0.880000in}}{\pgfqpoint{6.160000in}{6.160000in}}%
\pgfusepath{clip}%
\pgfsetbuttcap%
\pgfsetroundjoin%
\definecolor{currentfill}{rgb}{0.738826,0.822572,0.968261}%
\pgfsetfillcolor{currentfill}%
\pgfsetlinewidth{0.000000pt}%
\definecolor{currentstroke}{rgb}{0.000000,0.000000,0.000000}%
\pgfsetstrokecolor{currentstroke}%
\pgfsetdash{}{0pt}%
\pgfpathmoveto{\pgfqpoint{3.744275in}{3.792277in}}%
\pgfpathlineto{\pgfqpoint{3.754325in}{3.759485in}}%
\pgfpathlineto{\pgfqpoint{3.764380in}{3.729345in}}%
\pgfpathlineto{\pgfqpoint{3.797841in}{3.709299in}}%
\pgfpathlineto{\pgfqpoint{3.831267in}{3.689090in}}%
\pgfpathlineto{\pgfqpoint{3.821181in}{3.716508in}}%
\pgfpathlineto{\pgfqpoint{3.811102in}{3.746340in}}%
\pgfpathlineto{\pgfqpoint{3.777707in}{3.769321in}}%
\pgfpathlineto{\pgfqpoint{3.744275in}{3.792277in}}%
\pgfpathclose%
\pgfusepath{fill}%
\end{pgfscope}%
\begin{pgfscope}%
\pgfpathrectangle{\pgfqpoint{1.020000in}{0.880000in}}{\pgfqpoint{6.160000in}{6.160000in}}%
\pgfusepath{clip}%
\pgfsetbuttcap%
\pgfsetroundjoin%
\definecolor{currentfill}{rgb}{0.945854,0.559565,0.441513}%
\pgfsetfillcolor{currentfill}%
\pgfsetlinewidth{0.000000pt}%
\definecolor{currentstroke}{rgb}{0.000000,0.000000,0.000000}%
\pgfsetstrokecolor{currentstroke}%
\pgfsetdash{}{0pt}%
\pgfpathmoveto{\pgfqpoint{3.220997in}{4.757181in}}%
\pgfpathlineto{\pgfqpoint{3.230931in}{4.694480in}}%
\pgfpathlineto{\pgfqpoint{3.240888in}{4.629948in}}%
\pgfpathlineto{\pgfqpoint{3.274612in}{4.608963in}}%
\pgfpathlineto{\pgfqpoint{3.308330in}{4.584429in}}%
\pgfpathlineto{\pgfqpoint{3.298371in}{4.646640in}}%
\pgfpathlineto{\pgfqpoint{3.288435in}{4.707133in}}%
\pgfpathlineto{\pgfqpoint{3.254718in}{4.734089in}}%
\pgfpathlineto{\pgfqpoint{3.220997in}{4.757181in}}%
\pgfpathclose%
\pgfusepath{fill}%
\end{pgfscope}%
\begin{pgfscope}%
\pgfpathrectangle{\pgfqpoint{1.020000in}{0.880000in}}{\pgfqpoint{6.160000in}{6.160000in}}%
\pgfusepath{clip}%
\pgfsetbuttcap%
\pgfsetroundjoin%
\definecolor{currentfill}{rgb}{0.785153,0.220851,0.211673}%
\pgfsetfillcolor{currentfill}%
\pgfsetlinewidth{0.000000pt}%
\definecolor{currentstroke}{rgb}{0.000000,0.000000,0.000000}%
\pgfsetstrokecolor{currentstroke}%
\pgfsetdash{}{0pt}%
\pgfpathmoveto{\pgfqpoint{2.893578in}{5.101706in}}%
\pgfpathlineto{\pgfqpoint{2.903118in}{5.059368in}}%
\pgfpathlineto{\pgfqpoint{2.912726in}{5.012295in}}%
\pgfpathlineto{\pgfqpoint{2.946221in}{5.025878in}}%
\pgfpathlineto{\pgfqpoint{2.979757in}{5.034553in}}%
\pgfpathlineto{\pgfqpoint{2.970108in}{5.082020in}}%
\pgfpathlineto{\pgfqpoint{2.960524in}{5.124653in}}%
\pgfpathlineto{\pgfqpoint{2.927029in}{5.115736in}}%
\pgfpathlineto{\pgfqpoint{2.893578in}{5.101706in}}%
\pgfpathclose%
\pgfusepath{fill}%
\end{pgfscope}%
\begin{pgfscope}%
\pgfpathrectangle{\pgfqpoint{1.020000in}{0.880000in}}{\pgfqpoint{6.160000in}{6.160000in}}%
\pgfusepath{clip}%
\pgfsetbuttcap%
\pgfsetroundjoin%
\definecolor{currentfill}{rgb}{0.353369,0.472069,0.892570}%
\pgfsetfillcolor{currentfill}%
\pgfsetlinewidth{0.000000pt}%
\definecolor{currentstroke}{rgb}{0.000000,0.000000,0.000000}%
\pgfsetstrokecolor{currentstroke}%
\pgfsetdash{}{0pt}%
\pgfpathmoveto{\pgfqpoint{5.192628in}{3.064003in}}%
\pgfpathlineto{\pgfqpoint{5.203976in}{3.054379in}}%
\pgfpathlineto{\pgfqpoint{5.215332in}{3.043317in}}%
\pgfpathlineto{\pgfqpoint{5.248234in}{3.042125in}}%
\pgfpathlineto{\pgfqpoint{5.281119in}{3.041509in}}%
\pgfpathlineto{\pgfqpoint{5.269709in}{3.052681in}}%
\pgfpathlineto{\pgfqpoint{5.258307in}{3.062416in}}%
\pgfpathlineto{\pgfqpoint{5.225475in}{3.062844in}}%
\pgfpathlineto{\pgfqpoint{5.192628in}{3.064003in}}%
\pgfpathclose%
\pgfusepath{fill}%
\end{pgfscope}%
\begin{pgfscope}%
\pgfpathrectangle{\pgfqpoint{1.020000in}{0.880000in}}{\pgfqpoint{6.160000in}{6.160000in}}%
\pgfusepath{clip}%
\pgfsetbuttcap%
\pgfsetroundjoin%
\definecolor{currentfill}{rgb}{0.960490,0.616276,0.495467}%
\pgfsetfillcolor{currentfill}%
\pgfsetlinewidth{0.000000pt}%
\definecolor{currentstroke}{rgb}{0.000000,0.000000,0.000000}%
\pgfsetstrokecolor{currentstroke}%
\pgfsetdash{}{0pt}%
\pgfpathmoveto{\pgfqpoint{2.429290in}{4.570984in}}%
\pgfpathlineto{\pgfqpoint{2.438379in}{4.537532in}}%
\pgfpathlineto{\pgfqpoint{2.447523in}{4.501447in}}%
\pgfpathlineto{\pgfqpoint{2.480639in}{4.544823in}}%
\pgfpathlineto{\pgfqpoint{2.513749in}{4.588849in}}%
\pgfpathlineto{\pgfqpoint{2.504526in}{4.626856in}}%
\pgfpathlineto{\pgfqpoint{2.495362in}{4.661870in}}%
\pgfpathlineto{\pgfqpoint{2.462328in}{4.616090in}}%
\pgfpathlineto{\pgfqpoint{2.429290in}{4.570984in}}%
\pgfpathclose%
\pgfusepath{fill}%
\end{pgfscope}%
\begin{pgfscope}%
\pgfpathrectangle{\pgfqpoint{1.020000in}{0.880000in}}{\pgfqpoint{6.160000in}{6.160000in}}%
\pgfusepath{clip}%
\pgfsetbuttcap%
\pgfsetroundjoin%
\definecolor{currentfill}{rgb}{0.895885,0.433075,0.338681}%
\pgfsetfillcolor{currentfill}%
\pgfsetlinewidth{0.000000pt}%
\definecolor{currentstroke}{rgb}{0.000000,0.000000,0.000000}%
\pgfsetstrokecolor{currentstroke}%
\pgfsetdash{}{0pt}%
\pgfpathmoveto{\pgfqpoint{3.133816in}{4.911401in}}%
\pgfpathlineto{\pgfqpoint{3.143672in}{4.852625in}}%
\pgfpathlineto{\pgfqpoint{3.153566in}{4.790846in}}%
\pgfpathlineto{\pgfqpoint{3.187278in}{4.776164in}}%
\pgfpathlineto{\pgfqpoint{3.220997in}{4.757181in}}%
\pgfpathlineto{\pgfqpoint{3.211094in}{4.817508in}}%
\pgfpathlineto{\pgfqpoint{3.201226in}{4.874937in}}%
\pgfpathlineto{\pgfqpoint{3.167516in}{4.895480in}}%
\pgfpathlineto{\pgfqpoint{3.133816in}{4.911401in}}%
\pgfpathclose%
\pgfusepath{fill}%
\end{pgfscope}%
\begin{pgfscope}%
\pgfpathrectangle{\pgfqpoint{1.020000in}{0.880000in}}{\pgfqpoint{6.160000in}{6.160000in}}%
\pgfusepath{clip}%
\pgfsetbuttcap%
\pgfsetroundjoin%
\definecolor{currentfill}{rgb}{0.945854,0.559565,0.441513}%
\pgfsetfillcolor{currentfill}%
\pgfsetlinewidth{0.000000pt}%
\definecolor{currentstroke}{rgb}{0.000000,0.000000,0.000000}%
\pgfsetstrokecolor{currentstroke}%
\pgfsetdash{}{0pt}%
\pgfpathmoveto{\pgfqpoint{2.495362in}{4.661870in}}%
\pgfpathlineto{\pgfqpoint{2.504526in}{4.626856in}}%
\pgfpathlineto{\pgfqpoint{2.513749in}{4.588849in}}%
\pgfpathlineto{\pgfqpoint{2.546861in}{4.633145in}}%
\pgfpathlineto{\pgfqpoint{2.579981in}{4.677302in}}%
\pgfpathlineto{\pgfqpoint{2.570681in}{4.717247in}}%
\pgfpathlineto{\pgfqpoint{2.561444in}{4.753833in}}%
\pgfpathlineto{\pgfqpoint{2.528398in}{4.707927in}}%
\pgfpathlineto{\pgfqpoint{2.495362in}{4.661870in}}%
\pgfpathclose%
\pgfusepath{fill}%
\end{pgfscope}%
\begin{pgfscope}%
\pgfpathrectangle{\pgfqpoint{1.020000in}{0.880000in}}{\pgfqpoint{6.160000in}{6.160000in}}%
\pgfusepath{clip}%
\pgfsetbuttcap%
\pgfsetroundjoin%
\definecolor{currentfill}{rgb}{0.373552,0.497499,0.909467}%
\pgfsetfillcolor{currentfill}%
\pgfsetlinewidth{0.000000pt}%
\definecolor{currentstroke}{rgb}{0.000000,0.000000,0.000000}%
\pgfsetstrokecolor{currentstroke}%
\pgfsetdash{}{0pt}%
\pgfpathmoveto{\pgfqpoint{4.972891in}{3.105227in}}%
\pgfpathlineto{\pgfqpoint{4.984039in}{3.096835in}}%
\pgfpathlineto{\pgfqpoint{4.995199in}{3.087141in}}%
\pgfpathlineto{\pgfqpoint{5.028149in}{3.081376in}}%
\pgfpathlineto{\pgfqpoint{5.061080in}{3.076358in}}%
\pgfpathlineto{\pgfqpoint{5.049859in}{3.085259in}}%
\pgfpathlineto{\pgfqpoint{5.038646in}{3.092554in}}%
\pgfpathlineto{\pgfqpoint{5.005779in}{3.098504in}}%
\pgfpathlineto{\pgfqpoint{4.972891in}{3.105227in}}%
\pgfpathclose%
\pgfusepath{fill}%
\end{pgfscope}%
\begin{pgfscope}%
\pgfpathrectangle{\pgfqpoint{1.020000in}{0.880000in}}{\pgfqpoint{6.160000in}{6.160000in}}%
\pgfusepath{clip}%
\pgfsetbuttcap%
\pgfsetroundjoin%
\definecolor{currentfill}{rgb}{0.576051,0.708780,0.997755}%
\pgfsetfillcolor{currentfill}%
\pgfsetlinewidth{0.000000pt}%
\definecolor{currentstroke}{rgb}{0.000000,0.000000,0.000000}%
\pgfsetstrokecolor{currentstroke}%
\pgfsetdash{}{0pt}%
\pgfpathmoveto{\pgfqpoint{4.138607in}{3.473307in}}%
\pgfpathlineto{\pgfqpoint{4.148944in}{3.461498in}}%
\pgfpathlineto{\pgfqpoint{4.159299in}{3.451348in}}%
\pgfpathlineto{\pgfqpoint{4.192572in}{3.435733in}}%
\pgfpathlineto{\pgfqpoint{4.225813in}{3.420524in}}%
\pgfpathlineto{\pgfqpoint{4.215404in}{3.431074in}}%
\pgfpathlineto{\pgfqpoint{4.205014in}{3.443034in}}%
\pgfpathlineto{\pgfqpoint{4.171826in}{3.457927in}}%
\pgfpathlineto{\pgfqpoint{4.138607in}{3.473307in}}%
\pgfpathclose%
\pgfusepath{fill}%
\end{pgfscope}%
\begin{pgfscope}%
\pgfpathrectangle{\pgfqpoint{1.020000in}{0.880000in}}{\pgfqpoint{6.160000in}{6.160000in}}%
\pgfusepath{clip}%
\pgfsetbuttcap%
\pgfsetroundjoin%
\definecolor{currentfill}{rgb}{0.630089,0.752516,0.998508}%
\pgfsetfillcolor{currentfill}%
\pgfsetlinewidth{0.000000pt}%
\definecolor{currentstroke}{rgb}{0.000000,0.000000,0.000000}%
\pgfsetstrokecolor{currentstroke}%
\pgfsetdash{}{0pt}%
\pgfpathmoveto{\pgfqpoint{3.984991in}{3.569718in}}%
\pgfpathlineto{\pgfqpoint{3.995193in}{3.553077in}}%
\pgfpathlineto{\pgfqpoint{4.005410in}{3.538643in}}%
\pgfpathlineto{\pgfqpoint{4.038758in}{3.521884in}}%
\pgfpathlineto{\pgfqpoint{4.072073in}{3.505351in}}%
\pgfpathlineto{\pgfqpoint{4.061806in}{3.519161in}}%
\pgfpathlineto{\pgfqpoint{4.051555in}{3.534897in}}%
\pgfpathlineto{\pgfqpoint{4.018290in}{3.552113in}}%
\pgfpathlineto{\pgfqpoint{3.984991in}{3.569718in}}%
\pgfpathclose%
\pgfusepath{fill}%
\end{pgfscope}%
\begin{pgfscope}%
\pgfpathrectangle{\pgfqpoint{1.020000in}{0.880000in}}{\pgfqpoint{6.160000in}{6.160000in}}%
\pgfusepath{clip}%
\pgfsetbuttcap%
\pgfsetroundjoin%
\definecolor{currentfill}{rgb}{0.532568,0.669801,0.990393}%
\pgfsetfillcolor{currentfill}%
\pgfsetlinewidth{0.000000pt}%
\definecolor{currentstroke}{rgb}{0.000000,0.000000,0.000000}%
\pgfsetstrokecolor{currentstroke}%
\pgfsetdash{}{0pt}%
\pgfpathmoveto{\pgfqpoint{4.292203in}{3.391463in}}%
\pgfpathlineto{\pgfqpoint{4.302687in}{3.381565in}}%
\pgfpathlineto{\pgfqpoint{4.313192in}{3.372786in}}%
\pgfpathlineto{\pgfqpoint{4.346395in}{3.358121in}}%
\pgfpathlineto{\pgfqpoint{4.379568in}{3.343938in}}%
\pgfpathlineto{\pgfqpoint{4.369010in}{3.353624in}}%
\pgfpathlineto{\pgfqpoint{4.358474in}{3.364226in}}%
\pgfpathlineto{\pgfqpoint{4.325353in}{3.377629in}}%
\pgfpathlineto{\pgfqpoint{4.292203in}{3.391463in}}%
\pgfpathclose%
\pgfusepath{fill}%
\end{pgfscope}%
\begin{pgfscope}%
\pgfpathrectangle{\pgfqpoint{1.020000in}{0.880000in}}{\pgfqpoint{6.160000in}{6.160000in}}%
\pgfusepath{clip}%
\pgfsetbuttcap%
\pgfsetroundjoin%
\definecolor{currentfill}{rgb}{0.800830,0.250829,0.225696}%
\pgfsetfillcolor{currentfill}%
\pgfsetlinewidth{0.000000pt}%
\definecolor{currentstroke}{rgb}{0.000000,0.000000,0.000000}%
\pgfsetstrokecolor{currentstroke}%
\pgfsetdash{}{0pt}%
\pgfpathmoveto{\pgfqpoint{2.826815in}{5.059281in}}%
\pgfpathlineto{\pgfqpoint{2.836306in}{5.017593in}}%
\pgfpathlineto{\pgfqpoint{2.845866in}{4.971340in}}%
\pgfpathlineto{\pgfqpoint{2.879274in}{4.994023in}}%
\pgfpathlineto{\pgfqpoint{2.912726in}{5.012295in}}%
\pgfpathlineto{\pgfqpoint{2.903118in}{5.059368in}}%
\pgfpathlineto{\pgfqpoint{2.893578in}{5.101706in}}%
\pgfpathlineto{\pgfqpoint{2.860173in}{5.082792in}}%
\pgfpathlineto{\pgfqpoint{2.826815in}{5.059281in}}%
\pgfpathclose%
\pgfusepath{fill}%
\end{pgfscope}%
\begin{pgfscope}%
\pgfpathrectangle{\pgfqpoint{1.020000in}{0.880000in}}{\pgfqpoint{6.160000in}{6.160000in}}%
\pgfusepath{clip}%
\pgfsetbuttcap%
\pgfsetroundjoin%
\definecolor{currentfill}{rgb}{0.921406,0.491420,0.383408}%
\pgfsetfillcolor{currentfill}%
\pgfsetlinewidth{0.000000pt}%
\definecolor{currentstroke}{rgb}{0.000000,0.000000,0.000000}%
\pgfsetstrokecolor{currentstroke}%
\pgfsetdash{}{0pt}%
\pgfpathmoveto{\pgfqpoint{2.561444in}{4.753833in}}%
\pgfpathlineto{\pgfqpoint{2.570681in}{4.717247in}}%
\pgfpathlineto{\pgfqpoint{2.579981in}{4.677302in}}%
\pgfpathlineto{\pgfqpoint{2.613115in}{4.720884in}}%
\pgfpathlineto{\pgfqpoint{2.646268in}{4.763438in}}%
\pgfpathlineto{\pgfqpoint{2.636895in}{4.805259in}}%
\pgfpathlineto{\pgfqpoint{2.627587in}{4.843369in}}%
\pgfpathlineto{\pgfqpoint{2.594504in}{4.799138in}}%
\pgfpathlineto{\pgfqpoint{2.561444in}{4.753833in}}%
\pgfpathclose%
\pgfusepath{fill}%
\end{pgfscope}%
\begin{pgfscope}%
\pgfpathrectangle{\pgfqpoint{1.020000in}{0.880000in}}{\pgfqpoint{6.160000in}{6.160000in}}%
\pgfusepath{clip}%
\pgfsetbuttcap%
\pgfsetroundjoin%
\definecolor{currentfill}{rgb}{0.489246,0.627536,0.976896}%
\pgfsetfillcolor{currentfill}%
\pgfsetlinewidth{0.000000pt}%
\definecolor{currentstroke}{rgb}{0.000000,0.000000,0.000000}%
\pgfsetstrokecolor{currentstroke}%
\pgfsetdash{}{0pt}%
\pgfpathmoveto{\pgfqpoint{4.445826in}{3.316906in}}%
\pgfpathlineto{\pgfqpoint{4.456458in}{3.307178in}}%
\pgfpathlineto{\pgfqpoint{4.467113in}{3.298078in}}%
\pgfpathlineto{\pgfqpoint{4.500250in}{3.284347in}}%
\pgfpathlineto{\pgfqpoint{4.533359in}{3.271077in}}%
\pgfpathlineto{\pgfqpoint{4.522652in}{3.280923in}}%
\pgfpathlineto{\pgfqpoint{4.511967in}{3.291212in}}%
\pgfpathlineto{\pgfqpoint{4.478911in}{3.303937in}}%
\pgfpathlineto{\pgfqpoint{4.445826in}{3.316906in}}%
\pgfpathclose%
\pgfusepath{fill}%
\end{pgfscope}%
\begin{pgfscope}%
\pgfpathrectangle{\pgfqpoint{1.020000in}{0.880000in}}{\pgfqpoint{6.160000in}{6.160000in}}%
\pgfusepath{clip}%
\pgfsetbuttcap%
\pgfsetroundjoin%
\definecolor{currentfill}{rgb}{0.323718,0.433158,0.864722}%
\pgfsetfillcolor{currentfill}%
\pgfsetlinewidth{0.000000pt}%
\definecolor{currentstroke}{rgb}{0.000000,0.000000,0.000000}%
\pgfsetstrokecolor{currentstroke}%
\pgfsetdash{}{0pt}%
\pgfpathmoveto{\pgfqpoint{5.786785in}{3.001063in}}%
\pgfpathlineto{\pgfqpoint{5.798624in}{2.985576in}}%
\pgfpathlineto{\pgfqpoint{5.810485in}{2.970078in}}%
\pgfpathlineto{\pgfqpoint{5.843241in}{2.971794in}}%
\pgfpathlineto{\pgfqpoint{5.875977in}{2.973541in}}%
\pgfpathlineto{\pgfqpoint{5.864064in}{2.988992in}}%
\pgfpathlineto{\pgfqpoint{5.852173in}{3.004452in}}%
\pgfpathlineto{\pgfqpoint{5.819489in}{3.002748in}}%
\pgfpathlineto{\pgfqpoint{5.786785in}{3.001063in}}%
\pgfpathclose%
\pgfusepath{fill}%
\end{pgfscope}%
\begin{pgfscope}%
\pgfpathrectangle{\pgfqpoint{1.020000in}{0.880000in}}{\pgfqpoint{6.160000in}{6.160000in}}%
\pgfusepath{clip}%
\pgfsetbuttcap%
\pgfsetroundjoin%
\definecolor{currentfill}{rgb}{0.333490,0.446265,0.874452}%
\pgfsetfillcolor{currentfill}%
\pgfsetlinewidth{0.000000pt}%
\definecolor{currentstroke}{rgb}{0.000000,0.000000,0.000000}%
\pgfsetstrokecolor{currentstroke}%
\pgfsetdash{}{0pt}%
\pgfpathmoveto{\pgfqpoint{5.566831in}{3.021904in}}%
\pgfpathlineto{\pgfqpoint{5.578478in}{3.006976in}}%
\pgfpathlineto{\pgfqpoint{5.590142in}{2.991788in}}%
\pgfpathlineto{\pgfqpoint{5.622966in}{2.993175in}}%
\pgfpathlineto{\pgfqpoint{5.655769in}{2.994647in}}%
\pgfpathlineto{\pgfqpoint{5.644054in}{3.009971in}}%
\pgfpathlineto{\pgfqpoint{5.632358in}{3.025119in}}%
\pgfpathlineto{\pgfqpoint{5.599605in}{3.023480in}}%
\pgfpathlineto{\pgfqpoint{5.566831in}{3.021904in}}%
\pgfpathclose%
\pgfusepath{fill}%
\end{pgfscope}%
\begin{pgfscope}%
\pgfpathrectangle{\pgfqpoint{1.020000in}{0.880000in}}{\pgfqpoint{6.160000in}{6.160000in}}%
\pgfusepath{clip}%
\pgfsetbuttcap%
\pgfsetroundjoin%
\definecolor{currentfill}{rgb}{0.883687,0.856108,0.840258}%
\pgfsetfillcolor{currentfill}%
\pgfsetlinewidth{0.000000pt}%
\definecolor{currentstroke}{rgb}{0.000000,0.000000,0.000000}%
\pgfsetstrokecolor{currentstroke}%
\pgfsetdash{}{0pt}%
\pgfpathmoveto{\pgfqpoint{3.502991in}{4.117180in}}%
\pgfpathlineto{\pgfqpoint{3.512996in}{4.064239in}}%
\pgfpathlineto{\pgfqpoint{3.522999in}{4.013404in}}%
\pgfpathlineto{\pgfqpoint{3.556611in}{3.990899in}}%
\pgfpathlineto{\pgfqpoint{3.590193in}{3.967213in}}%
\pgfpathlineto{\pgfqpoint{3.580183in}{4.014312in}}%
\pgfpathlineto{\pgfqpoint{3.570173in}{4.063422in}}%
\pgfpathlineto{\pgfqpoint{3.536597in}{4.090932in}}%
\pgfpathlineto{\pgfqpoint{3.502991in}{4.117180in}}%
\pgfpathclose%
\pgfusepath{fill}%
\end{pgfscope}%
\begin{pgfscope}%
\pgfpathrectangle{\pgfqpoint{1.020000in}{0.880000in}}{\pgfqpoint{6.160000in}{6.160000in}}%
\pgfusepath{clip}%
\pgfsetbuttcap%
\pgfsetroundjoin%
\definecolor{currentfill}{rgb}{0.892138,0.425389,0.333289}%
\pgfsetfillcolor{currentfill}%
\pgfsetlinewidth{0.000000pt}%
\definecolor{currentstroke}{rgb}{0.000000,0.000000,0.000000}%
\pgfsetstrokecolor{currentstroke}%
\pgfsetdash{}{0pt}%
\pgfpathmoveto{\pgfqpoint{2.627587in}{4.843369in}}%
\pgfpathlineto{\pgfqpoint{2.636895in}{4.805259in}}%
\pgfpathlineto{\pgfqpoint{2.646268in}{4.763438in}}%
\pgfpathlineto{\pgfqpoint{2.679446in}{4.804501in}}%
\pgfpathlineto{\pgfqpoint{2.712655in}{4.843605in}}%
\pgfpathlineto{\pgfqpoint{2.703214in}{4.887158in}}%
\pgfpathlineto{\pgfqpoint{2.693839in}{4.926670in}}%
\pgfpathlineto{\pgfqpoint{2.660696in}{4.886042in}}%
\pgfpathlineto{\pgfqpoint{2.627587in}{4.843369in}}%
\pgfpathclose%
\pgfusepath{fill}%
\end{pgfscope}%
\begin{pgfscope}%
\pgfpathrectangle{\pgfqpoint{1.020000in}{0.880000in}}{\pgfqpoint{6.160000in}{6.160000in}}%
\pgfusepath{clip}%
\pgfsetbuttcap%
\pgfsetroundjoin%
\definecolor{currentfill}{rgb}{0.693321,0.796314,0.986308}%
\pgfsetfillcolor{currentfill}%
\pgfsetlinewidth{0.000000pt}%
\definecolor{currentstroke}{rgb}{0.000000,0.000000,0.000000}%
\pgfsetstrokecolor{currentstroke}%
\pgfsetdash{}{0pt}%
\pgfpathmoveto{\pgfqpoint{3.831267in}{3.689090in}}%
\pgfpathlineto{\pgfqpoint{3.841361in}{3.664221in}}%
\pgfpathlineto{\pgfqpoint{3.851464in}{3.642008in}}%
\pgfpathlineto{\pgfqpoint{3.884895in}{3.623883in}}%
\pgfpathlineto{\pgfqpoint{3.918294in}{3.605712in}}%
\pgfpathlineto{\pgfqpoint{3.908150in}{3.626049in}}%
\pgfpathlineto{\pgfqpoint{3.898017in}{3.648768in}}%
\pgfpathlineto{\pgfqpoint{3.864660in}{3.668867in}}%
\pgfpathlineto{\pgfqpoint{3.831267in}{3.689090in}}%
\pgfpathclose%
\pgfusepath{fill}%
\end{pgfscope}%
\begin{pgfscope}%
\pgfpathrectangle{\pgfqpoint{1.020000in}{0.880000in}}{\pgfqpoint{6.160000in}{6.160000in}}%
\pgfusepath{clip}%
\pgfsetbuttcap%
\pgfsetroundjoin%
\definecolor{currentfill}{rgb}{0.451739,0.588181,0.960201}%
\pgfsetfillcolor{currentfill}%
\pgfsetlinewidth{0.000000pt}%
\definecolor{currentstroke}{rgb}{0.000000,0.000000,0.000000}%
\pgfsetstrokecolor{currentstroke}%
\pgfsetdash{}{0pt}%
\pgfpathmoveto{\pgfqpoint{4.599490in}{3.245611in}}%
\pgfpathlineto{\pgfqpoint{4.610271in}{3.235709in}}%
\pgfpathlineto{\pgfqpoint{4.621074in}{3.225934in}}%
\pgfpathlineto{\pgfqpoint{4.654150in}{3.213461in}}%
\pgfpathlineto{\pgfqpoint{4.687199in}{3.201436in}}%
\pgfpathlineto{\pgfqpoint{4.676342in}{3.211262in}}%
\pgfpathlineto{\pgfqpoint{4.665506in}{3.220972in}}%
\pgfpathlineto{\pgfqpoint{4.632513in}{3.233235in}}%
\pgfpathlineto{\pgfqpoint{4.599490in}{3.245611in}}%
\pgfpathclose%
\pgfusepath{fill}%
\end{pgfscope}%
\begin{pgfscope}%
\pgfpathrectangle{\pgfqpoint{1.020000in}{0.880000in}}{\pgfqpoint{6.160000in}{6.160000in}}%
\pgfusepath{clip}%
\pgfsetbuttcap%
\pgfsetroundjoin%
\definecolor{currentfill}{rgb}{0.830187,0.304733,0.254891}%
\pgfsetfillcolor{currentfill}%
\pgfsetlinewidth{0.000000pt}%
\definecolor{currentstroke}{rgb}{0.000000,0.000000,0.000000}%
\pgfsetstrokecolor{currentstroke}%
\pgfsetdash{}{0pt}%
\pgfpathmoveto{\pgfqpoint{2.760239in}{4.999866in}}%
\pgfpathlineto{\pgfqpoint{2.769675in}{4.959139in}}%
\pgfpathlineto{\pgfqpoint{2.779179in}{4.914082in}}%
\pgfpathlineto{\pgfqpoint{2.812501in}{4.944570in}}%
\pgfpathlineto{\pgfqpoint{2.845866in}{4.971340in}}%
\pgfpathlineto{\pgfqpoint{2.836306in}{5.017593in}}%
\pgfpathlineto{\pgfqpoint{2.826815in}{5.059281in}}%
\pgfpathlineto{\pgfqpoint{2.793504in}{5.031511in}}%
\pgfpathlineto{\pgfqpoint{2.760239in}{4.999866in}}%
\pgfpathclose%
\pgfusepath{fill}%
\end{pgfscope}%
\begin{pgfscope}%
\pgfpathrectangle{\pgfqpoint{1.020000in}{0.880000in}}{\pgfqpoint{6.160000in}{6.160000in}}%
\pgfusepath{clip}%
\pgfsetbuttcap%
\pgfsetroundjoin%
\definecolor{currentfill}{rgb}{0.935774,0.812237,0.747156}%
\pgfsetfillcolor{currentfill}%
\pgfsetlinewidth{0.000000pt}%
\definecolor{currentstroke}{rgb}{0.000000,0.000000,0.000000}%
\pgfsetstrokecolor{currentstroke}%
\pgfsetdash{}{0pt}%
\pgfpathmoveto{\pgfqpoint{3.415686in}{4.282866in}}%
\pgfpathlineto{\pgfqpoint{3.425691in}{4.223198in}}%
\pgfpathlineto{\pgfqpoint{3.435697in}{4.164889in}}%
\pgfpathlineto{\pgfqpoint{3.469356in}{4.141915in}}%
\pgfpathlineto{\pgfqpoint{3.502991in}{4.117180in}}%
\pgfpathlineto{\pgfqpoint{3.492985in}{4.171858in}}%
\pgfpathlineto{\pgfqpoint{3.482981in}{4.227871in}}%
\pgfpathlineto{\pgfqpoint{3.449346in}{4.256337in}}%
\pgfpathlineto{\pgfqpoint{3.415686in}{4.282866in}}%
\pgfpathclose%
\pgfusepath{fill}%
\end{pgfscope}%
\begin{pgfscope}%
\pgfpathrectangle{\pgfqpoint{1.020000in}{0.880000in}}{\pgfqpoint{6.160000in}{6.160000in}}%
\pgfusepath{clip}%
\pgfsetbuttcap%
\pgfsetroundjoin%
\definecolor{currentfill}{rgb}{0.309060,0.413498,0.850128}%
\pgfsetfillcolor{currentfill}%
\pgfsetlinewidth{0.000000pt}%
\definecolor{currentstroke}{rgb}{0.000000,0.000000,0.000000}%
\pgfsetstrokecolor{currentstroke}%
\pgfsetdash{}{0pt}%
\pgfpathmoveto{\pgfqpoint{6.006722in}{2.980776in}}%
\pgfpathlineto{\pgfqpoint{6.018764in}{2.965499in}}%
\pgfpathlineto{\pgfqpoint{6.030829in}{2.950259in}}%
\pgfpathlineto{\pgfqpoint{6.063519in}{2.952202in}}%
\pgfpathlineto{\pgfqpoint{6.051427in}{2.967399in}}%
\pgfpathlineto{\pgfqpoint{6.039358in}{2.982634in}}%
\pgfpathlineto{\pgfqpoint{6.006722in}{2.980776in}}%
\pgfpathclose%
\pgfusepath{fill}%
\end{pgfscope}%
\begin{pgfscope}%
\pgfpathrectangle{\pgfqpoint{1.020000in}{0.880000in}}{\pgfqpoint{6.160000in}{6.160000in}}%
\pgfusepath{clip}%
\pgfsetbuttcap%
\pgfsetroundjoin%
\definecolor{currentfill}{rgb}{0.343278,0.459354,0.884122}%
\pgfsetfillcolor{currentfill}%
\pgfsetlinewidth{0.000000pt}%
\definecolor{currentstroke}{rgb}{0.000000,0.000000,0.000000}%
\pgfsetstrokecolor{currentstroke}%
\pgfsetdash{}{0pt}%
\pgfpathmoveto{\pgfqpoint{5.346840in}{3.041805in}}%
\pgfpathlineto{\pgfqpoint{5.358311in}{3.029173in}}%
\pgfpathlineto{\pgfqpoint{5.369794in}{3.015668in}}%
\pgfpathlineto{\pgfqpoint{5.402679in}{3.016114in}}%
\pgfpathlineto{\pgfqpoint{5.435546in}{3.016860in}}%
\pgfpathlineto{\pgfqpoint{5.424014in}{3.030684in}}%
\pgfpathlineto{\pgfqpoint{5.412494in}{3.043751in}}%
\pgfpathlineto{\pgfqpoint{5.379676in}{3.042603in}}%
\pgfpathlineto{\pgfqpoint{5.346840in}{3.041805in}}%
\pgfpathclose%
\pgfusepath{fill}%
\end{pgfscope}%
\begin{pgfscope}%
\pgfpathrectangle{\pgfqpoint{1.020000in}{0.880000in}}{\pgfqpoint{6.160000in}{6.160000in}}%
\pgfusepath{clip}%
\pgfsetbuttcap%
\pgfsetroundjoin%
\definecolor{currentfill}{rgb}{0.856716,0.354704,0.285487}%
\pgfsetfillcolor{currentfill}%
\pgfsetlinewidth{0.000000pt}%
\definecolor{currentstroke}{rgb}{0.000000,0.000000,0.000000}%
\pgfsetstrokecolor{currentstroke}%
\pgfsetdash{}{0pt}%
\pgfpathmoveto{\pgfqpoint{2.693839in}{4.926670in}}%
\pgfpathlineto{\pgfqpoint{2.703214in}{4.887158in}}%
\pgfpathlineto{\pgfqpoint{2.712655in}{4.843605in}}%
\pgfpathlineto{\pgfqpoint{2.745898in}{4.880283in}}%
\pgfpathlineto{\pgfqpoint{2.779179in}{4.914082in}}%
\pgfpathlineto{\pgfqpoint{2.769675in}{4.959139in}}%
\pgfpathlineto{\pgfqpoint{2.760239in}{4.999866in}}%
\pgfpathlineto{\pgfqpoint{2.727019in}{4.964769in}}%
\pgfpathlineto{\pgfqpoint{2.693839in}{4.926670in}}%
\pgfpathclose%
\pgfusepath{fill}%
\end{pgfscope}%
\begin{pgfscope}%
\pgfpathrectangle{\pgfqpoint{1.020000in}{0.880000in}}{\pgfqpoint{6.160000in}{6.160000in}}%
\pgfusepath{clip}%
\pgfsetbuttcap%
\pgfsetroundjoin%
\definecolor{currentfill}{rgb}{0.830187,0.304733,0.254891}%
\pgfsetfillcolor{currentfill}%
\pgfsetlinewidth{0.000000pt}%
\definecolor{currentstroke}{rgb}{0.000000,0.000000,0.000000}%
\pgfsetstrokecolor{currentstroke}%
\pgfsetdash{}{0pt}%
\pgfpathmoveto{\pgfqpoint{2.979757in}{5.034553in}}%
\pgfpathlineto{\pgfqpoint{2.989465in}{4.982628in}}%
\pgfpathlineto{\pgfqpoint{2.999228in}{4.926669in}}%
\pgfpathlineto{\pgfqpoint{3.032833in}{4.930180in}}%
\pgfpathlineto{\pgfqpoint{3.066470in}{4.928815in}}%
\pgfpathlineto{\pgfqpoint{3.056678in}{4.984731in}}%
\pgfpathlineto{\pgfqpoint{3.046938in}{5.036605in}}%
\pgfpathlineto{\pgfqpoint{3.013330in}{5.038160in}}%
\pgfpathlineto{\pgfqpoint{2.979757in}{5.034553in}}%
\pgfpathclose%
\pgfusepath{fill}%
\end{pgfscope}%
\begin{pgfscope}%
\pgfpathrectangle{\pgfqpoint{1.020000in}{0.880000in}}{\pgfqpoint{6.160000in}{6.160000in}}%
\pgfusepath{clip}%
\pgfsetbuttcap%
\pgfsetroundjoin%
\definecolor{currentfill}{rgb}{0.822420,0.856898,0.910795}%
\pgfsetfillcolor{currentfill}%
\pgfsetlinewidth{0.000000pt}%
\definecolor{currentstroke}{rgb}{0.000000,0.000000,0.000000}%
\pgfsetstrokecolor{currentstroke}%
\pgfsetdash{}{0pt}%
\pgfpathmoveto{\pgfqpoint{3.590193in}{3.967213in}}%
\pgfpathlineto{\pgfqpoint{3.600202in}{3.922441in}}%
\pgfpathlineto{\pgfqpoint{3.610210in}{3.880280in}}%
\pgfpathlineto{\pgfqpoint{3.643775in}{3.859225in}}%
\pgfpathlineto{\pgfqpoint{3.677308in}{3.837413in}}%
\pgfpathlineto{\pgfqpoint{3.667284in}{3.876072in}}%
\pgfpathlineto{\pgfqpoint{3.657261in}{3.917178in}}%
\pgfpathlineto{\pgfqpoint{3.623743in}{3.942565in}}%
\pgfpathlineto{\pgfqpoint{3.590193in}{3.967213in}}%
\pgfpathclose%
\pgfusepath{fill}%
\end{pgfscope}%
\begin{pgfscope}%
\pgfpathrectangle{\pgfqpoint{1.020000in}{0.880000in}}{\pgfqpoint{6.160000in}{6.160000in}}%
\pgfusepath{clip}%
\pgfsetbuttcap%
\pgfsetroundjoin%
\definecolor{currentfill}{rgb}{0.966962,0.735670,0.630877}%
\pgfsetfillcolor{currentfill}%
\pgfsetlinewidth{0.000000pt}%
\definecolor{currentstroke}{rgb}{0.000000,0.000000,0.000000}%
\pgfsetstrokecolor{currentstroke}%
\pgfsetdash{}{0pt}%
\pgfpathmoveto{\pgfqpoint{3.328297in}{4.456986in}}%
\pgfpathlineto{\pgfqpoint{3.338297in}{4.392819in}}%
\pgfpathlineto{\pgfqpoint{3.348303in}{4.329047in}}%
\pgfpathlineto{\pgfqpoint{3.382003in}{4.307186in}}%
\pgfpathlineto{\pgfqpoint{3.415686in}{4.282866in}}%
\pgfpathlineto{\pgfqpoint{3.405683in}{4.343442in}}%
\pgfpathlineto{\pgfqpoint{3.395686in}{4.404450in}}%
\pgfpathlineto{\pgfqpoint{3.362000in}{4.432082in}}%
\pgfpathlineto{\pgfqpoint{3.328297in}{4.456986in}}%
\pgfpathclose%
\pgfusepath{fill}%
\end{pgfscope}%
\begin{pgfscope}%
\pgfpathrectangle{\pgfqpoint{1.020000in}{0.880000in}}{\pgfqpoint{6.160000in}{6.160000in}}%
\pgfusepath{clip}%
\pgfsetbuttcap%
\pgfsetroundjoin%
\definecolor{currentfill}{rgb}{0.358415,0.478426,0.896795}%
\pgfsetfillcolor{currentfill}%
\pgfsetlinewidth{0.000000pt}%
\definecolor{currentstroke}{rgb}{0.000000,0.000000,0.000000}%
\pgfsetstrokecolor{currentstroke}%
\pgfsetdash{}{0pt}%
\pgfpathmoveto{\pgfqpoint{5.126887in}{3.068622in}}%
\pgfpathlineto{\pgfqpoint{5.138178in}{3.058778in}}%
\pgfpathlineto{\pgfqpoint{5.149478in}{3.047588in}}%
\pgfpathlineto{\pgfqpoint{5.182413in}{3.045126in}}%
\pgfpathlineto{\pgfqpoint{5.215332in}{3.043317in}}%
\pgfpathlineto{\pgfqpoint{5.203976in}{3.054379in}}%
\pgfpathlineto{\pgfqpoint{5.192628in}{3.064003in}}%
\pgfpathlineto{\pgfqpoint{5.159765in}{3.065923in}}%
\pgfpathlineto{\pgfqpoint{5.126887in}{3.068622in}}%
\pgfpathclose%
\pgfusepath{fill}%
\end{pgfscope}%
\begin{pgfscope}%
\pgfpathrectangle{\pgfqpoint{1.020000in}{0.880000in}}{\pgfqpoint{6.160000in}{6.160000in}}%
\pgfusepath{clip}%
\pgfsetbuttcap%
\pgfsetroundjoin%
\definecolor{currentfill}{rgb}{0.414801,0.546874,0.939088}%
\pgfsetfillcolor{currentfill}%
\pgfsetlinewidth{0.000000pt}%
\definecolor{currentstroke}{rgb}{0.000000,0.000000,0.000000}%
\pgfsetstrokecolor{currentstroke}%
\pgfsetdash{}{0pt}%
\pgfpathmoveto{\pgfqpoint{4.753215in}{3.178488in}}%
\pgfpathlineto{\pgfqpoint{4.764147in}{3.168846in}}%
\pgfpathlineto{\pgfqpoint{4.775099in}{3.158757in}}%
\pgfpathlineto{\pgfqpoint{4.808124in}{3.148398in}}%
\pgfpathlineto{\pgfqpoint{4.841124in}{3.138608in}}%
\pgfpathlineto{\pgfqpoint{4.830115in}{3.148069in}}%
\pgfpathlineto{\pgfqpoint{4.819123in}{3.156749in}}%
\pgfpathlineto{\pgfqpoint{4.786182in}{3.167476in}}%
\pgfpathlineto{\pgfqpoint{4.753215in}{3.178488in}}%
\pgfpathclose%
\pgfusepath{fill}%
\end{pgfscope}%
\begin{pgfscope}%
\pgfpathrectangle{\pgfqpoint{1.020000in}{0.880000in}}{\pgfqpoint{6.160000in}{6.160000in}}%
\pgfusepath{clip}%
\pgfsetbuttcap%
\pgfsetroundjoin%
\definecolor{currentfill}{rgb}{0.964911,0.640159,0.519806}%
\pgfsetfillcolor{currentfill}%
\pgfsetlinewidth{0.000000pt}%
\definecolor{currentstroke}{rgb}{0.000000,0.000000,0.000000}%
\pgfsetstrokecolor{currentstroke}%
\pgfsetdash{}{0pt}%
\pgfpathmoveto{\pgfqpoint{3.240888in}{4.629948in}}%
\pgfpathlineto{\pgfqpoint{3.250865in}{4.564137in}}%
\pgfpathlineto{\pgfqpoint{3.260857in}{4.497599in}}%
\pgfpathlineto{\pgfqpoint{3.294581in}{4.478901in}}%
\pgfpathlineto{\pgfqpoint{3.328297in}{4.456986in}}%
\pgfpathlineto{\pgfqpoint{3.318306in}{4.521032in}}%
\pgfpathlineto{\pgfqpoint{3.308330in}{4.584429in}}%
\pgfpathlineto{\pgfqpoint{3.274612in}{4.608963in}}%
\pgfpathlineto{\pgfqpoint{3.240888in}{4.629948in}}%
\pgfpathclose%
\pgfusepath{fill}%
\end{pgfscope}%
\begin{pgfscope}%
\pgfpathrectangle{\pgfqpoint{1.020000in}{0.880000in}}{\pgfqpoint{6.160000in}{6.160000in}}%
\pgfusepath{clip}%
\pgfsetbuttcap%
\pgfsetroundjoin%
\definecolor{currentfill}{rgb}{0.884643,0.410017,0.322507}%
\pgfsetfillcolor{currentfill}%
\pgfsetlinewidth{0.000000pt}%
\definecolor{currentstroke}{rgb}{0.000000,0.000000,0.000000}%
\pgfsetstrokecolor{currentstroke}%
\pgfsetdash{}{0pt}%
\pgfpathmoveto{\pgfqpoint{3.066470in}{4.928815in}}%
\pgfpathlineto{\pgfqpoint{3.076307in}{4.869332in}}%
\pgfpathlineto{\pgfqpoint{3.086184in}{4.806790in}}%
\pgfpathlineto{\pgfqpoint{3.119866in}{4.801083in}}%
\pgfpathlineto{\pgfqpoint{3.153566in}{4.790846in}}%
\pgfpathlineto{\pgfqpoint{3.143672in}{4.852625in}}%
\pgfpathlineto{\pgfqpoint{3.133816in}{4.911401in}}%
\pgfpathlineto{\pgfqpoint{3.100132in}{4.922544in}}%
\pgfpathlineto{\pgfqpoint{3.066470in}{4.928815in}}%
\pgfpathclose%
\pgfusepath{fill}%
\end{pgfscope}%
\begin{pgfscope}%
\pgfpathrectangle{\pgfqpoint{1.020000in}{0.880000in}}{\pgfqpoint{6.160000in}{6.160000in}}%
\pgfusepath{clip}%
\pgfsetbuttcap%
\pgfsetroundjoin%
\definecolor{currentfill}{rgb}{0.934305,0.525918,0.412286}%
\pgfsetfillcolor{currentfill}%
\pgfsetlinewidth{0.000000pt}%
\definecolor{currentstroke}{rgb}{0.000000,0.000000,0.000000}%
\pgfsetstrokecolor{currentstroke}%
\pgfsetdash{}{0pt}%
\pgfpathmoveto{\pgfqpoint{3.153566in}{4.790846in}}%
\pgfpathlineto{\pgfqpoint{3.163492in}{4.726598in}}%
\pgfpathlineto{\pgfqpoint{3.173443in}{4.660436in}}%
\pgfpathlineto{\pgfqpoint{3.207164in}{4.647164in}}%
\pgfpathlineto{\pgfqpoint{3.240888in}{4.629948in}}%
\pgfpathlineto{\pgfqpoint{3.230931in}{4.694480in}}%
\pgfpathlineto{\pgfqpoint{3.220997in}{4.757181in}}%
\pgfpathlineto{\pgfqpoint{3.187278in}{4.776164in}}%
\pgfpathlineto{\pgfqpoint{3.153566in}{4.790846in}}%
\pgfpathclose%
\pgfusepath{fill}%
\end{pgfscope}%
\begin{pgfscope}%
\pgfpathrectangle{\pgfqpoint{1.020000in}{0.880000in}}{\pgfqpoint{6.160000in}{6.160000in}}%
\pgfusepath{clip}%
\pgfsetbuttcap%
\pgfsetroundjoin%
\definecolor{currentfill}{rgb}{0.768034,0.837035,0.952488}%
\pgfsetfillcolor{currentfill}%
\pgfsetlinewidth{0.000000pt}%
\definecolor{currentstroke}{rgb}{0.000000,0.000000,0.000000}%
\pgfsetstrokecolor{currentstroke}%
\pgfsetdash{}{0pt}%
\pgfpathmoveto{\pgfqpoint{3.677308in}{3.837413in}}%
\pgfpathlineto{\pgfqpoint{3.687334in}{3.801432in}}%
\pgfpathlineto{\pgfqpoint{3.697361in}{3.768324in}}%
\pgfpathlineto{\pgfqpoint{3.730887in}{3.749074in}}%
\pgfpathlineto{\pgfqpoint{3.764380in}{3.729345in}}%
\pgfpathlineto{\pgfqpoint{3.754325in}{3.759485in}}%
\pgfpathlineto{\pgfqpoint{3.744275in}{3.792277in}}%
\pgfpathlineto{\pgfqpoint{3.710809in}{3.815035in}}%
\pgfpathlineto{\pgfqpoint{3.677308in}{3.837413in}}%
\pgfpathclose%
\pgfusepath{fill}%
\end{pgfscope}%
\begin{pgfscope}%
\pgfpathrectangle{\pgfqpoint{1.020000in}{0.880000in}}{\pgfqpoint{6.160000in}{6.160000in}}%
\pgfusepath{clip}%
\pgfsetbuttcap%
\pgfsetroundjoin%
\definecolor{currentfill}{rgb}{0.388852,0.516298,0.921373}%
\pgfsetfillcolor{currentfill}%
\pgfsetlinewidth{0.000000pt}%
\definecolor{currentstroke}{rgb}{0.000000,0.000000,0.000000}%
\pgfsetstrokecolor{currentstroke}%
\pgfsetdash{}{0pt}%
\pgfpathmoveto{\pgfqpoint{4.907053in}{3.120718in}}%
\pgfpathlineto{\pgfqpoint{4.918139in}{3.111285in}}%
\pgfpathlineto{\pgfqpoint{4.929239in}{3.100874in}}%
\pgfpathlineto{\pgfqpoint{4.962229in}{3.093643in}}%
\pgfpathlineto{\pgfqpoint{4.995199in}{3.087141in}}%
\pgfpathlineto{\pgfqpoint{4.984039in}{3.096835in}}%
\pgfpathlineto{\pgfqpoint{4.972891in}{3.105227in}}%
\pgfpathlineto{\pgfqpoint{4.939983in}{3.112652in}}%
\pgfpathlineto{\pgfqpoint{4.907053in}{3.120718in}}%
\pgfpathclose%
\pgfusepath{fill}%
\end{pgfscope}%
\begin{pgfscope}%
\pgfpathrectangle{\pgfqpoint{1.020000in}{0.880000in}}{\pgfqpoint{6.160000in}{6.160000in}}%
\pgfusepath{clip}%
\pgfsetbuttcap%
\pgfsetroundjoin%
\definecolor{currentfill}{rgb}{0.313946,0.420052,0.854993}%
\pgfsetfillcolor{currentfill}%
\pgfsetlinewidth{0.000000pt}%
\definecolor{currentstroke}{rgb}{0.000000,0.000000,0.000000}%
\pgfsetstrokecolor{currentstroke}%
\pgfsetdash{}{0pt}%
\pgfpathmoveto{\pgfqpoint{5.941390in}{2.977115in}}%
\pgfpathlineto{\pgfqpoint{5.953378in}{2.961758in}}%
\pgfpathlineto{\pgfqpoint{5.965390in}{2.946433in}}%
\pgfpathlineto{\pgfqpoint{5.998120in}{2.948335in}}%
\pgfpathlineto{\pgfqpoint{6.030829in}{2.950259in}}%
\pgfpathlineto{\pgfqpoint{6.018764in}{2.965499in}}%
\pgfpathlineto{\pgfqpoint{6.006722in}{2.980776in}}%
\pgfpathlineto{\pgfqpoint{5.974066in}{2.978936in}}%
\pgfpathlineto{\pgfqpoint{5.941390in}{2.977115in}}%
\pgfpathclose%
\pgfusepath{fill}%
\end{pgfscope}%
\begin{pgfscope}%
\pgfpathrectangle{\pgfqpoint{1.020000in}{0.880000in}}{\pgfqpoint{6.160000in}{6.160000in}}%
\pgfusepath{clip}%
\pgfsetbuttcap%
\pgfsetroundjoin%
\definecolor{currentfill}{rgb}{0.323718,0.433158,0.864722}%
\pgfsetfillcolor{currentfill}%
\pgfsetlinewidth{0.000000pt}%
\definecolor{currentstroke}{rgb}{0.000000,0.000000,0.000000}%
\pgfsetstrokecolor{currentstroke}%
\pgfsetdash{}{0pt}%
\pgfpathmoveto{\pgfqpoint{5.721317in}{2.997774in}}%
\pgfpathlineto{\pgfqpoint{5.733104in}{2.982291in}}%
\pgfpathlineto{\pgfqpoint{5.744912in}{2.966768in}}%
\pgfpathlineto{\pgfqpoint{5.777708in}{2.968401in}}%
\pgfpathlineto{\pgfqpoint{5.810485in}{2.970078in}}%
\pgfpathlineto{\pgfqpoint{5.798624in}{2.985576in}}%
\pgfpathlineto{\pgfqpoint{5.786785in}{3.001063in}}%
\pgfpathlineto{\pgfqpoint{5.754061in}{2.999404in}}%
\pgfpathlineto{\pgfqpoint{5.721317in}{2.997774in}}%
\pgfpathclose%
\pgfusepath{fill}%
\end{pgfscope}%
\begin{pgfscope}%
\pgfpathrectangle{\pgfqpoint{1.020000in}{0.880000in}}{\pgfqpoint{6.160000in}{6.160000in}}%
\pgfusepath{clip}%
\pgfsetbuttcap%
\pgfsetroundjoin%
\definecolor{currentfill}{rgb}{0.333490,0.446265,0.874452}%
\pgfsetfillcolor{currentfill}%
\pgfsetlinewidth{0.000000pt}%
\definecolor{currentstroke}{rgb}{0.000000,0.000000,0.000000}%
\pgfsetstrokecolor{currentstroke}%
\pgfsetdash{}{0pt}%
\pgfpathmoveto{\pgfqpoint{5.501227in}{3.019057in}}%
\pgfpathlineto{\pgfqpoint{5.512824in}{3.004386in}}%
\pgfpathlineto{\pgfqpoint{5.524438in}{2.989362in}}%
\pgfpathlineto{\pgfqpoint{5.557300in}{2.990508in}}%
\pgfpathlineto{\pgfqpoint{5.590142in}{2.991788in}}%
\pgfpathlineto{\pgfqpoint{5.578478in}{3.006976in}}%
\pgfpathlineto{\pgfqpoint{5.566831in}{3.021904in}}%
\pgfpathlineto{\pgfqpoint{5.534039in}{3.020419in}}%
\pgfpathlineto{\pgfqpoint{5.501227in}{3.019057in}}%
\pgfpathclose%
\pgfusepath{fill}%
\end{pgfscope}%
\begin{pgfscope}%
\pgfpathrectangle{\pgfqpoint{1.020000in}{0.880000in}}{\pgfqpoint{6.160000in}{6.160000in}}%
\pgfusepath{clip}%
\pgfsetbuttcap%
\pgfsetroundjoin%
\definecolor{currentfill}{rgb}{0.835027,0.313644,0.259783}%
\pgfsetfillcolor{currentfill}%
\pgfsetlinewidth{0.000000pt}%
\definecolor{currentstroke}{rgb}{0.000000,0.000000,0.000000}%
\pgfsetstrokecolor{currentstroke}%
\pgfsetdash{}{0pt}%
\pgfpathmoveto{\pgfqpoint{2.912726in}{5.012295in}}%
\pgfpathlineto{\pgfqpoint{2.922395in}{4.960849in}}%
\pgfpathlineto{\pgfqpoint{2.932120in}{4.905445in}}%
\pgfpathlineto{\pgfqpoint{2.965656in}{4.918374in}}%
\pgfpathlineto{\pgfqpoint{2.999228in}{4.926669in}}%
\pgfpathlineto{\pgfqpoint{2.989465in}{4.982628in}}%
\pgfpathlineto{\pgfqpoint{2.979757in}{5.034553in}}%
\pgfpathlineto{\pgfqpoint{2.946221in}{5.025878in}}%
\pgfpathlineto{\pgfqpoint{2.912726in}{5.012295in}}%
\pgfpathclose%
\pgfusepath{fill}%
\end{pgfscope}%
\begin{pgfscope}%
\pgfpathrectangle{\pgfqpoint{1.020000in}{0.880000in}}{\pgfqpoint{6.160000in}{6.160000in}}%
\pgfusepath{clip}%
\pgfsetbuttcap%
\pgfsetroundjoin%
\definecolor{currentfill}{rgb}{0.603162,0.731527,0.999565}%
\pgfsetfillcolor{currentfill}%
\pgfsetlinewidth{0.000000pt}%
\definecolor{currentstroke}{rgb}{0.000000,0.000000,0.000000}%
\pgfsetstrokecolor{currentstroke}%
\pgfsetdash{}{0pt}%
\pgfpathmoveto{\pgfqpoint{4.072073in}{3.505351in}}%
\pgfpathlineto{\pgfqpoint{4.082357in}{3.493480in}}%
\pgfpathlineto{\pgfqpoint{4.092659in}{3.483535in}}%
\pgfpathlineto{\pgfqpoint{4.125995in}{3.467307in}}%
\pgfpathlineto{\pgfqpoint{4.159299in}{3.451348in}}%
\pgfpathlineto{\pgfqpoint{4.148944in}{3.461498in}}%
\pgfpathlineto{\pgfqpoint{4.138607in}{3.473307in}}%
\pgfpathlineto{\pgfqpoint{4.105356in}{3.489135in}}%
\pgfpathlineto{\pgfqpoint{4.072073in}{3.505351in}}%
\pgfpathclose%
\pgfusepath{fill}%
\end{pgfscope}%
\begin{pgfscope}%
\pgfpathrectangle{\pgfqpoint{1.020000in}{0.880000in}}{\pgfqpoint{6.160000in}{6.160000in}}%
\pgfusepath{clip}%
\pgfsetbuttcap%
\pgfsetroundjoin%
\definecolor{currentfill}{rgb}{0.964835,0.744614,0.643239}%
\pgfsetfillcolor{currentfill}%
\pgfsetlinewidth{0.000000pt}%
\definecolor{currentstroke}{rgb}{0.000000,0.000000,0.000000}%
\pgfsetstrokecolor{currentstroke}%
\pgfsetdash{}{0pt}%
\pgfpathmoveto{\pgfqpoint{2.314890in}{4.340856in}}%
\pgfpathlineto{\pgfqpoint{2.323918in}{4.306510in}}%
\pgfpathlineto{\pgfqpoint{2.332988in}{4.270503in}}%
\pgfpathlineto{\pgfqpoint{2.366268in}{4.306067in}}%
\pgfpathlineto{\pgfqpoint{2.399520in}{4.343352in}}%
\pgfpathlineto{\pgfqpoint{2.390364in}{4.381646in}}%
\pgfpathlineto{\pgfqpoint{2.381253in}{4.418006in}}%
\pgfpathlineto{\pgfqpoint{2.348085in}{4.378523in}}%
\pgfpathlineto{\pgfqpoint{2.314890in}{4.340856in}}%
\pgfpathclose%
\pgfusepath{fill}%
\end{pgfscope}%
\begin{pgfscope}%
\pgfpathrectangle{\pgfqpoint{1.020000in}{0.880000in}}{\pgfqpoint{6.160000in}{6.160000in}}%
\pgfusepath{clip}%
\pgfsetbuttcap%
\pgfsetroundjoin%
\definecolor{currentfill}{rgb}{0.348323,0.465711,0.888346}%
\pgfsetfillcolor{currentfill}%
\pgfsetlinewidth{0.000000pt}%
\definecolor{currentstroke}{rgb}{0.000000,0.000000,0.000000}%
\pgfsetstrokecolor{currentstroke}%
\pgfsetdash{}{0pt}%
\pgfpathmoveto{\pgfqpoint{5.281119in}{3.041509in}}%
\pgfpathlineto{\pgfqpoint{5.292539in}{3.029162in}}%
\pgfpathlineto{\pgfqpoint{5.303970in}{3.015876in}}%
\pgfpathlineto{\pgfqpoint{5.336891in}{3.015572in}}%
\pgfpathlineto{\pgfqpoint{5.369794in}{3.015668in}}%
\pgfpathlineto{\pgfqpoint{5.358311in}{3.029173in}}%
\pgfpathlineto{\pgfqpoint{5.346840in}{3.041805in}}%
\pgfpathlineto{\pgfqpoint{5.313988in}{3.041421in}}%
\pgfpathlineto{\pgfqpoint{5.281119in}{3.041509in}}%
\pgfpathclose%
\pgfusepath{fill}%
\end{pgfscope}%
\begin{pgfscope}%
\pgfpathrectangle{\pgfqpoint{1.020000in}{0.880000in}}{\pgfqpoint{6.160000in}{6.160000in}}%
\pgfusepath{clip}%
\pgfsetbuttcap%
\pgfsetroundjoin%
\definecolor{currentfill}{rgb}{0.554312,0.690097,0.995516}%
\pgfsetfillcolor{currentfill}%
\pgfsetlinewidth{0.000000pt}%
\definecolor{currentstroke}{rgb}{0.000000,0.000000,0.000000}%
\pgfsetstrokecolor{currentstroke}%
\pgfsetdash{}{0pt}%
\pgfpathmoveto{\pgfqpoint{4.225813in}{3.420524in}}%
\pgfpathlineto{\pgfqpoint{4.236243in}{3.411358in}}%
\pgfpathlineto{\pgfqpoint{4.246694in}{3.403533in}}%
\pgfpathlineto{\pgfqpoint{4.279959in}{3.387932in}}%
\pgfpathlineto{\pgfqpoint{4.313192in}{3.372786in}}%
\pgfpathlineto{\pgfqpoint{4.302687in}{3.381565in}}%
\pgfpathlineto{\pgfqpoint{4.292203in}{3.391463in}}%
\pgfpathlineto{\pgfqpoint{4.259023in}{3.405761in}}%
\pgfpathlineto{\pgfqpoint{4.225813in}{3.420524in}}%
\pgfpathclose%
\pgfusepath{fill}%
\end{pgfscope}%
\begin{pgfscope}%
\pgfpathrectangle{\pgfqpoint{1.020000in}{0.880000in}}{\pgfqpoint{6.160000in}{6.160000in}}%
\pgfusepath{clip}%
\pgfsetbuttcap%
\pgfsetroundjoin%
\definecolor{currentfill}{rgb}{0.656683,0.771806,0.994914}%
\pgfsetfillcolor{currentfill}%
\pgfsetlinewidth{0.000000pt}%
\definecolor{currentstroke}{rgb}{0.000000,0.000000,0.000000}%
\pgfsetstrokecolor{currentstroke}%
\pgfsetdash{}{0pt}%
\pgfpathmoveto{\pgfqpoint{3.918294in}{3.605712in}}%
\pgfpathlineto{\pgfqpoint{3.928449in}{3.587827in}}%
\pgfpathlineto{\pgfqpoint{3.938617in}{3.572436in}}%
\pgfpathlineto{\pgfqpoint{3.972030in}{3.555530in}}%
\pgfpathlineto{\pgfqpoint{4.005410in}{3.538643in}}%
\pgfpathlineto{\pgfqpoint{3.995193in}{3.553077in}}%
\pgfpathlineto{\pgfqpoint{3.984991in}{3.569718in}}%
\pgfpathlineto{\pgfqpoint{3.951659in}{3.587618in}}%
\pgfpathlineto{\pgfqpoint{3.918294in}{3.605712in}}%
\pgfpathclose%
\pgfusepath{fill}%
\end{pgfscope}%
\begin{pgfscope}%
\pgfpathrectangle{\pgfqpoint{1.020000in}{0.880000in}}{\pgfqpoint{6.160000in}{6.160000in}}%
\pgfusepath{clip}%
\pgfsetbuttcap%
\pgfsetroundjoin%
\definecolor{currentfill}{rgb}{0.969192,0.705836,0.593704}%
\pgfsetfillcolor{currentfill}%
\pgfsetlinewidth{0.000000pt}%
\definecolor{currentstroke}{rgb}{0.000000,0.000000,0.000000}%
\pgfsetstrokecolor{currentstroke}%
\pgfsetdash{}{0pt}%
\pgfpathmoveto{\pgfqpoint{2.381253in}{4.418006in}}%
\pgfpathlineto{\pgfqpoint{2.390364in}{4.381646in}}%
\pgfpathlineto{\pgfqpoint{2.399520in}{4.343352in}}%
\pgfpathlineto{\pgfqpoint{2.432749in}{4.382138in}}%
\pgfpathlineto{\pgfqpoint{2.465961in}{4.422169in}}%
\pgfpathlineto{\pgfqpoint{2.456719in}{4.462922in}}%
\pgfpathlineto{\pgfqpoint{2.447523in}{4.501447in}}%
\pgfpathlineto{\pgfqpoint{2.414397in}{4.459071in}}%
\pgfpathlineto{\pgfqpoint{2.381253in}{4.418006in}}%
\pgfpathclose%
\pgfusepath{fill}%
\end{pgfscope}%
\begin{pgfscope}%
\pgfpathrectangle{\pgfqpoint{1.020000in}{0.880000in}}{\pgfqpoint{6.160000in}{6.160000in}}%
\pgfusepath{clip}%
\pgfsetbuttcap%
\pgfsetroundjoin%
\definecolor{currentfill}{rgb}{0.510824,0.649397,0.985079}%
\pgfsetfillcolor{currentfill}%
\pgfsetlinewidth{0.000000pt}%
\definecolor{currentstroke}{rgb}{0.000000,0.000000,0.000000}%
\pgfsetstrokecolor{currentstroke}%
\pgfsetdash{}{0pt}%
\pgfpathmoveto{\pgfqpoint{4.379568in}{3.343938in}}%
\pgfpathlineto{\pgfqpoint{4.390148in}{3.335121in}}%
\pgfpathlineto{\pgfqpoint{4.400751in}{3.327115in}}%
\pgfpathlineto{\pgfqpoint{4.433947in}{3.312322in}}%
\pgfpathlineto{\pgfqpoint{4.467113in}{3.298078in}}%
\pgfpathlineto{\pgfqpoint{4.456458in}{3.307178in}}%
\pgfpathlineto{\pgfqpoint{4.445826in}{3.316906in}}%
\pgfpathlineto{\pgfqpoint{4.412712in}{3.330216in}}%
\pgfpathlineto{\pgfqpoint{4.379568in}{3.343938in}}%
\pgfpathclose%
\pgfusepath{fill}%
\end{pgfscope}%
\begin{pgfscope}%
\pgfpathrectangle{\pgfqpoint{1.020000in}{0.880000in}}{\pgfqpoint{6.160000in}{6.160000in}}%
\pgfusepath{clip}%
\pgfsetbuttcap%
\pgfsetroundjoin%
\definecolor{currentfill}{rgb}{0.368507,0.491141,0.905243}%
\pgfsetfillcolor{currentfill}%
\pgfsetlinewidth{0.000000pt}%
\definecolor{currentstroke}{rgb}{0.000000,0.000000,0.000000}%
\pgfsetstrokecolor{currentstroke}%
\pgfsetdash{}{0pt}%
\pgfpathmoveto{\pgfqpoint{5.061080in}{3.076358in}}%
\pgfpathlineto{\pgfqpoint{5.072311in}{3.066061in}}%
\pgfpathlineto{\pgfqpoint{5.083555in}{3.054585in}}%
\pgfpathlineto{\pgfqpoint{5.116525in}{3.050732in}}%
\pgfpathlineto{\pgfqpoint{5.149478in}{3.047588in}}%
\pgfpathlineto{\pgfqpoint{5.138178in}{3.058778in}}%
\pgfpathlineto{\pgfqpoint{5.126887in}{3.068622in}}%
\pgfpathlineto{\pgfqpoint{5.093992in}{3.072102in}}%
\pgfpathlineto{\pgfqpoint{5.061080in}{3.076358in}}%
\pgfpathclose%
\pgfusepath{fill}%
\end{pgfscope}%
\begin{pgfscope}%
\pgfpathrectangle{\pgfqpoint{1.020000in}{0.880000in}}{\pgfqpoint{6.160000in}{6.160000in}}%
\pgfusepath{clip}%
\pgfsetbuttcap%
\pgfsetroundjoin%
\definecolor{currentfill}{rgb}{0.967317,0.657471,0.538160}%
\pgfsetfillcolor{currentfill}%
\pgfsetlinewidth{0.000000pt}%
\definecolor{currentstroke}{rgb}{0.000000,0.000000,0.000000}%
\pgfsetstrokecolor{currentstroke}%
\pgfsetdash{}{0pt}%
\pgfpathmoveto{\pgfqpoint{2.447523in}{4.501447in}}%
\pgfpathlineto{\pgfqpoint{2.456719in}{4.462922in}}%
\pgfpathlineto{\pgfqpoint{2.465961in}{4.422169in}}%
\pgfpathlineto{\pgfqpoint{2.499160in}{4.463150in}}%
\pgfpathlineto{\pgfqpoint{2.532354in}{4.504753in}}%
\pgfpathlineto{\pgfqpoint{2.523027in}{4.548065in}}%
\pgfpathlineto{\pgfqpoint{2.513749in}{4.588849in}}%
\pgfpathlineto{\pgfqpoint{2.480639in}{4.544823in}}%
\pgfpathlineto{\pgfqpoint{2.447523in}{4.501447in}}%
\pgfpathclose%
\pgfusepath{fill}%
\end{pgfscope}%
\begin{pgfscope}%
\pgfpathrectangle{\pgfqpoint{1.020000in}{0.880000in}}{\pgfqpoint{6.160000in}{6.160000in}}%
\pgfusepath{clip}%
\pgfsetbuttcap%
\pgfsetroundjoin%
\definecolor{currentfill}{rgb}{0.467678,0.605591,0.968546}%
\pgfsetfillcolor{currentfill}%
\pgfsetlinewidth{0.000000pt}%
\definecolor{currentstroke}{rgb}{0.000000,0.000000,0.000000}%
\pgfsetstrokecolor{currentstroke}%
\pgfsetdash{}{0pt}%
\pgfpathmoveto{\pgfqpoint{4.533359in}{3.271077in}}%
\pgfpathlineto{\pgfqpoint{4.544088in}{3.261614in}}%
\pgfpathlineto{\pgfqpoint{4.554838in}{3.252479in}}%
\pgfpathlineto{\pgfqpoint{4.587970in}{3.238919in}}%
\pgfpathlineto{\pgfqpoint{4.621074in}{3.225934in}}%
\pgfpathlineto{\pgfqpoint{4.610271in}{3.235709in}}%
\pgfpathlineto{\pgfqpoint{4.599490in}{3.245611in}}%
\pgfpathlineto{\pgfqpoint{4.566439in}{3.258193in}}%
\pgfpathlineto{\pgfqpoint{4.533359in}{3.271077in}}%
\pgfpathclose%
\pgfusepath{fill}%
\end{pgfscope}%
\begin{pgfscope}%
\pgfpathrectangle{\pgfqpoint{1.020000in}{0.880000in}}{\pgfqpoint{6.160000in}{6.160000in}}%
\pgfusepath{clip}%
\pgfsetbuttcap%
\pgfsetroundjoin%
\definecolor{currentfill}{rgb}{0.848040,0.338280,0.275206}%
\pgfsetfillcolor{currentfill}%
\pgfsetlinewidth{0.000000pt}%
\definecolor{currentstroke}{rgb}{0.000000,0.000000,0.000000}%
\pgfsetstrokecolor{currentstroke}%
\pgfsetdash{}{0pt}%
\pgfpathmoveto{\pgfqpoint{2.845866in}{4.971340in}}%
\pgfpathlineto{\pgfqpoint{2.855488in}{4.920870in}}%
\pgfpathlineto{\pgfqpoint{2.865166in}{4.866579in}}%
\pgfpathlineto{\pgfqpoint{2.898624in}{4.888093in}}%
\pgfpathlineto{\pgfqpoint{2.932120in}{4.905445in}}%
\pgfpathlineto{\pgfqpoint{2.922395in}{4.960849in}}%
\pgfpathlineto{\pgfqpoint{2.912726in}{5.012295in}}%
\pgfpathlineto{\pgfqpoint{2.879274in}{4.994023in}}%
\pgfpathlineto{\pgfqpoint{2.845866in}{4.971340in}}%
\pgfpathclose%
\pgfusepath{fill}%
\end{pgfscope}%
\begin{pgfscope}%
\pgfpathrectangle{\pgfqpoint{1.020000in}{0.880000in}}{\pgfqpoint{6.160000in}{6.160000in}}%
\pgfusepath{clip}%
\pgfsetbuttcap%
\pgfsetroundjoin%
\definecolor{currentfill}{rgb}{0.718985,0.811993,0.977656}%
\pgfsetfillcolor{currentfill}%
\pgfsetlinewidth{0.000000pt}%
\definecolor{currentstroke}{rgb}{0.000000,0.000000,0.000000}%
\pgfsetstrokecolor{currentstroke}%
\pgfsetdash{}{0pt}%
\pgfpathmoveto{\pgfqpoint{3.764380in}{3.729345in}}%
\pgfpathlineto{\pgfqpoint{3.774439in}{3.702007in}}%
\pgfpathlineto{\pgfqpoint{3.784504in}{3.677587in}}%
\pgfpathlineto{\pgfqpoint{3.818000in}{3.659955in}}%
\pgfpathlineto{\pgfqpoint{3.851464in}{3.642008in}}%
\pgfpathlineto{\pgfqpoint{3.841361in}{3.664221in}}%
\pgfpathlineto{\pgfqpoint{3.831267in}{3.689090in}}%
\pgfpathlineto{\pgfqpoint{3.797841in}{3.709299in}}%
\pgfpathlineto{\pgfqpoint{3.764380in}{3.729345in}}%
\pgfpathclose%
\pgfusepath{fill}%
\end{pgfscope}%
\begin{pgfscope}%
\pgfpathrectangle{\pgfqpoint{1.020000in}{0.880000in}}{\pgfqpoint{6.160000in}{6.160000in}}%
\pgfusepath{clip}%
\pgfsetbuttcap%
\pgfsetroundjoin%
\definecolor{currentfill}{rgb}{0.958279,0.604335,0.483297}%
\pgfsetfillcolor{currentfill}%
\pgfsetlinewidth{0.000000pt}%
\definecolor{currentstroke}{rgb}{0.000000,0.000000,0.000000}%
\pgfsetstrokecolor{currentstroke}%
\pgfsetdash{}{0pt}%
\pgfpathmoveto{\pgfqpoint{2.513749in}{4.588849in}}%
\pgfpathlineto{\pgfqpoint{2.523027in}{4.548065in}}%
\pgfpathlineto{\pgfqpoint{2.532354in}{4.504753in}}%
\pgfpathlineto{\pgfqpoint{2.565548in}{4.546617in}}%
\pgfpathlineto{\pgfqpoint{2.598748in}{4.588356in}}%
\pgfpathlineto{\pgfqpoint{2.589338in}{4.634245in}}%
\pgfpathlineto{\pgfqpoint{2.579981in}{4.677302in}}%
\pgfpathlineto{\pgfqpoint{2.546861in}{4.633145in}}%
\pgfpathlineto{\pgfqpoint{2.513749in}{4.588849in}}%
\pgfpathclose%
\pgfusepath{fill}%
\end{pgfscope}%
\begin{pgfscope}%
\pgfpathrectangle{\pgfqpoint{1.020000in}{0.880000in}}{\pgfqpoint{6.160000in}{6.160000in}}%
\pgfusepath{clip}%
\pgfsetbuttcap%
\pgfsetroundjoin%
\definecolor{currentfill}{rgb}{0.430507,0.564883,0.948889}%
\pgfsetfillcolor{currentfill}%
\pgfsetlinewidth{0.000000pt}%
\definecolor{currentstroke}{rgb}{0.000000,0.000000,0.000000}%
\pgfsetstrokecolor{currentstroke}%
\pgfsetdash{}{0pt}%
\pgfpathmoveto{\pgfqpoint{4.687199in}{3.201436in}}%
\pgfpathlineto{\pgfqpoint{4.698076in}{3.191459in}}%
\pgfpathlineto{\pgfqpoint{4.708972in}{3.181308in}}%
\pgfpathlineto{\pgfqpoint{4.742048in}{3.169715in}}%
\pgfpathlineto{\pgfqpoint{4.775099in}{3.158757in}}%
\pgfpathlineto{\pgfqpoint{4.764147in}{3.168846in}}%
\pgfpathlineto{\pgfqpoint{4.753215in}{3.178488in}}%
\pgfpathlineto{\pgfqpoint{4.720220in}{3.189796in}}%
\pgfpathlineto{\pgfqpoint{4.687199in}{3.201436in}}%
\pgfpathclose%
\pgfusepath{fill}%
\end{pgfscope}%
\begin{pgfscope}%
\pgfpathrectangle{\pgfqpoint{1.020000in}{0.880000in}}{\pgfqpoint{6.160000in}{6.160000in}}%
\pgfusepath{clip}%
\pgfsetbuttcap%
\pgfsetroundjoin%
\definecolor{currentfill}{rgb}{0.906154,0.842091,0.806151}%
\pgfsetfillcolor{currentfill}%
\pgfsetlinewidth{0.000000pt}%
\definecolor{currentstroke}{rgb}{0.000000,0.000000,0.000000}%
\pgfsetstrokecolor{currentstroke}%
\pgfsetdash{}{0pt}%
\pgfpathmoveto{\pgfqpoint{3.435697in}{4.164889in}}%
\pgfpathlineto{\pgfqpoint{3.445700in}{4.108363in}}%
\pgfpathlineto{\pgfqpoint{3.455698in}{4.054009in}}%
\pgfpathlineto{\pgfqpoint{3.489361in}{4.034509in}}%
\pgfpathlineto{\pgfqpoint{3.522999in}{4.013404in}}%
\pgfpathlineto{\pgfqpoint{3.512996in}{4.064239in}}%
\pgfpathlineto{\pgfqpoint{3.502991in}{4.117180in}}%
\pgfpathlineto{\pgfqpoint{3.469356in}{4.141915in}}%
\pgfpathlineto{\pgfqpoint{3.435697in}{4.164889in}}%
\pgfpathclose%
\pgfusepath{fill}%
\end{pgfscope}%
\begin{pgfscope}%
\pgfpathrectangle{\pgfqpoint{1.020000in}{0.880000in}}{\pgfqpoint{6.160000in}{6.160000in}}%
\pgfusepath{clip}%
\pgfsetbuttcap%
\pgfsetroundjoin%
\definecolor{currentfill}{rgb}{0.950956,0.786875,0.704761}%
\pgfsetfillcolor{currentfill}%
\pgfsetlinewidth{0.000000pt}%
\definecolor{currentstroke}{rgb}{0.000000,0.000000,0.000000}%
\pgfsetstrokecolor{currentstroke}%
\pgfsetdash{}{0pt}%
\pgfpathmoveto{\pgfqpoint{3.348303in}{4.329047in}}%
\pgfpathlineto{\pgfqpoint{3.358310in}{4.266167in}}%
\pgfpathlineto{\pgfqpoint{3.368316in}{4.204649in}}%
\pgfpathlineto{\pgfqpoint{3.402015in}{4.185872in}}%
\pgfpathlineto{\pgfqpoint{3.435697in}{4.164889in}}%
\pgfpathlineto{\pgfqpoint{3.425691in}{4.223198in}}%
\pgfpathlineto{\pgfqpoint{3.415686in}{4.282866in}}%
\pgfpathlineto{\pgfqpoint{3.382003in}{4.307186in}}%
\pgfpathlineto{\pgfqpoint{3.348303in}{4.329047in}}%
\pgfpathclose%
\pgfusepath{fill}%
\end{pgfscope}%
\begin{pgfscope}%
\pgfpathrectangle{\pgfqpoint{1.020000in}{0.880000in}}{\pgfqpoint{6.160000in}{6.160000in}}%
\pgfusepath{clip}%
\pgfsetbuttcap%
\pgfsetroundjoin%
\definecolor{currentfill}{rgb}{0.877149,0.394645,0.311724}%
\pgfsetfillcolor{currentfill}%
\pgfsetlinewidth{0.000000pt}%
\definecolor{currentstroke}{rgb}{0.000000,0.000000,0.000000}%
\pgfsetstrokecolor{currentstroke}%
\pgfsetdash{}{0pt}%
\pgfpathmoveto{\pgfqpoint{2.999228in}{4.926669in}}%
\pgfpathlineto{\pgfqpoint{3.009038in}{4.867145in}}%
\pgfpathlineto{\pgfqpoint{3.018891in}{4.804558in}}%
\pgfpathlineto{\pgfqpoint{3.052525in}{4.807939in}}%
\pgfpathlineto{\pgfqpoint{3.086184in}{4.806790in}}%
\pgfpathlineto{\pgfqpoint{3.076307in}{4.869332in}}%
\pgfpathlineto{\pgfqpoint{3.066470in}{4.928815in}}%
\pgfpathlineto{\pgfqpoint{3.032833in}{4.930180in}}%
\pgfpathlineto{\pgfqpoint{2.999228in}{4.926669in}}%
\pgfpathclose%
\pgfusepath{fill}%
\end{pgfscope}%
\begin{pgfscope}%
\pgfpathrectangle{\pgfqpoint{1.020000in}{0.880000in}}{\pgfqpoint{6.160000in}{6.160000in}}%
\pgfusepath{clip}%
\pgfsetbuttcap%
\pgfsetroundjoin%
\definecolor{currentfill}{rgb}{0.313946,0.420052,0.854993}%
\pgfsetfillcolor{currentfill}%
\pgfsetlinewidth{0.000000pt}%
\definecolor{currentstroke}{rgb}{0.000000,0.000000,0.000000}%
\pgfsetstrokecolor{currentstroke}%
\pgfsetdash{}{0pt}%
\pgfpathmoveto{\pgfqpoint{5.875977in}{2.973541in}}%
\pgfpathlineto{\pgfqpoint{5.887913in}{2.958111in}}%
\pgfpathlineto{\pgfqpoint{5.899871in}{2.942707in}}%
\pgfpathlineto{\pgfqpoint{5.932641in}{2.944556in}}%
\pgfpathlineto{\pgfqpoint{5.965390in}{2.946433in}}%
\pgfpathlineto{\pgfqpoint{5.953378in}{2.961758in}}%
\pgfpathlineto{\pgfqpoint{5.941390in}{2.977115in}}%
\pgfpathlineto{\pgfqpoint{5.908693in}{2.975316in}}%
\pgfpathlineto{\pgfqpoint{5.875977in}{2.973541in}}%
\pgfpathclose%
\pgfusepath{fill}%
\end{pgfscope}%
\begin{pgfscope}%
\pgfpathrectangle{\pgfqpoint{1.020000in}{0.880000in}}{\pgfqpoint{6.160000in}{6.160000in}}%
\pgfusepath{clip}%
\pgfsetbuttcap%
\pgfsetroundjoin%
\definecolor{currentfill}{rgb}{0.941728,0.546413,0.429707}%
\pgfsetfillcolor{currentfill}%
\pgfsetlinewidth{0.000000pt}%
\definecolor{currentstroke}{rgb}{0.000000,0.000000,0.000000}%
\pgfsetstrokecolor{currentstroke}%
\pgfsetdash{}{0pt}%
\pgfpathmoveto{\pgfqpoint{2.579981in}{4.677302in}}%
\pgfpathlineto{\pgfqpoint{2.589338in}{4.634245in}}%
\pgfpathlineto{\pgfqpoint{2.598748in}{4.588356in}}%
\pgfpathlineto{\pgfqpoint{2.631959in}{4.629559in}}%
\pgfpathlineto{\pgfqpoint{2.665188in}{4.669799in}}%
\pgfpathlineto{\pgfqpoint{2.655701in}{4.718181in}}%
\pgfpathlineto{\pgfqpoint{2.646268in}{4.763438in}}%
\pgfpathlineto{\pgfqpoint{2.613115in}{4.720884in}}%
\pgfpathlineto{\pgfqpoint{2.579981in}{4.677302in}}%
\pgfpathclose%
\pgfusepath{fill}%
\end{pgfscope}%
\begin{pgfscope}%
\pgfpathrectangle{\pgfqpoint{1.020000in}{0.880000in}}{\pgfqpoint{6.160000in}{6.160000in}}%
\pgfusepath{clip}%
\pgfsetbuttcap%
\pgfsetroundjoin%
\definecolor{currentfill}{rgb}{0.323718,0.433158,0.864722}%
\pgfsetfillcolor{currentfill}%
\pgfsetlinewidth{0.000000pt}%
\definecolor{currentstroke}{rgb}{0.000000,0.000000,0.000000}%
\pgfsetstrokecolor{currentstroke}%
\pgfsetdash{}{0pt}%
\pgfpathmoveto{\pgfqpoint{5.655769in}{2.994647in}}%
\pgfpathlineto{\pgfqpoint{5.667505in}{2.979202in}}%
\pgfpathlineto{\pgfqpoint{5.679260in}{2.963680in}}%
\pgfpathlineto{\pgfqpoint{5.712096in}{2.965191in}}%
\pgfpathlineto{\pgfqpoint{5.744912in}{2.966768in}}%
\pgfpathlineto{\pgfqpoint{5.733104in}{2.982291in}}%
\pgfpathlineto{\pgfqpoint{5.721317in}{2.997774in}}%
\pgfpathlineto{\pgfqpoint{5.688553in}{2.996185in}}%
\pgfpathlineto{\pgfqpoint{5.655769in}{2.994647in}}%
\pgfpathclose%
\pgfusepath{fill}%
\end{pgfscope}%
\begin{pgfscope}%
\pgfpathrectangle{\pgfqpoint{1.020000in}{0.880000in}}{\pgfqpoint{6.160000in}{6.160000in}}%
\pgfusepath{clip}%
\pgfsetbuttcap%
\pgfsetroundjoin%
\definecolor{currentfill}{rgb}{0.847365,0.862472,0.885540}%
\pgfsetfillcolor{currentfill}%
\pgfsetlinewidth{0.000000pt}%
\definecolor{currentstroke}{rgb}{0.000000,0.000000,0.000000}%
\pgfsetstrokecolor{currentstroke}%
\pgfsetdash{}{0pt}%
\pgfpathmoveto{\pgfqpoint{3.522999in}{4.013404in}}%
\pgfpathlineto{\pgfqpoint{3.532999in}{3.965011in}}%
\pgfpathlineto{\pgfqpoint{3.542994in}{3.919363in}}%
\pgfpathlineto{\pgfqpoint{3.576616in}{3.900389in}}%
\pgfpathlineto{\pgfqpoint{3.610210in}{3.880280in}}%
\pgfpathlineto{\pgfqpoint{3.600202in}{3.922441in}}%
\pgfpathlineto{\pgfqpoint{3.590193in}{3.967213in}}%
\pgfpathlineto{\pgfqpoint{3.556611in}{3.990899in}}%
\pgfpathlineto{\pgfqpoint{3.522999in}{4.013404in}}%
\pgfpathclose%
\pgfusepath{fill}%
\end{pgfscope}%
\begin{pgfscope}%
\pgfpathrectangle{\pgfqpoint{1.020000in}{0.880000in}}{\pgfqpoint{6.160000in}{6.160000in}}%
\pgfusepath{clip}%
\pgfsetbuttcap%
\pgfsetroundjoin%
\definecolor{currentfill}{rgb}{0.869655,0.379274,0.300941}%
\pgfsetfillcolor{currentfill}%
\pgfsetlinewidth{0.000000pt}%
\definecolor{currentstroke}{rgb}{0.000000,0.000000,0.000000}%
\pgfsetstrokecolor{currentstroke}%
\pgfsetdash{}{0pt}%
\pgfpathmoveto{\pgfqpoint{2.779179in}{4.914082in}}%
\pgfpathlineto{\pgfqpoint{2.788745in}{4.865024in}}%
\pgfpathlineto{\pgfqpoint{2.798368in}{4.812335in}}%
\pgfpathlineto{\pgfqpoint{2.831748in}{4.841209in}}%
\pgfpathlineto{\pgfqpoint{2.865166in}{4.866579in}}%
\pgfpathlineto{\pgfqpoint{2.855488in}{4.920870in}}%
\pgfpathlineto{\pgfqpoint{2.845866in}{4.971340in}}%
\pgfpathlineto{\pgfqpoint{2.812501in}{4.944570in}}%
\pgfpathlineto{\pgfqpoint{2.779179in}{4.914082in}}%
\pgfpathclose%
\pgfusepath{fill}%
\end{pgfscope}%
\begin{pgfscope}%
\pgfpathrectangle{\pgfqpoint{1.020000in}{0.880000in}}{\pgfqpoint{6.160000in}{6.160000in}}%
\pgfusepath{clip}%
\pgfsetbuttcap%
\pgfsetroundjoin%
\definecolor{currentfill}{rgb}{0.969192,0.705836,0.593704}%
\pgfsetfillcolor{currentfill}%
\pgfsetlinewidth{0.000000pt}%
\definecolor{currentstroke}{rgb}{0.000000,0.000000,0.000000}%
\pgfsetstrokecolor{currentstroke}%
\pgfsetdash{}{0pt}%
\pgfpathmoveto{\pgfqpoint{3.260857in}{4.497599in}}%
\pgfpathlineto{\pgfqpoint{3.270858in}{4.430880in}}%
\pgfpathlineto{\pgfqpoint{3.280864in}{4.364512in}}%
\pgfpathlineto{\pgfqpoint{3.314588in}{4.348222in}}%
\pgfpathlineto{\pgfqpoint{3.348303in}{4.329047in}}%
\pgfpathlineto{\pgfqpoint{3.338297in}{4.392819in}}%
\pgfpathlineto{\pgfqpoint{3.328297in}{4.456986in}}%
\pgfpathlineto{\pgfqpoint{3.294581in}{4.478901in}}%
\pgfpathlineto{\pgfqpoint{3.260857in}{4.497599in}}%
\pgfpathclose%
\pgfusepath{fill}%
\end{pgfscope}%
\begin{pgfscope}%
\pgfpathrectangle{\pgfqpoint{1.020000in}{0.880000in}}{\pgfqpoint{6.160000in}{6.160000in}}%
\pgfusepath{clip}%
\pgfsetbuttcap%
\pgfsetroundjoin%
\definecolor{currentfill}{rgb}{0.338377,0.452819,0.879317}%
\pgfsetfillcolor{currentfill}%
\pgfsetlinewidth{0.000000pt}%
\definecolor{currentstroke}{rgb}{0.000000,0.000000,0.000000}%
\pgfsetstrokecolor{currentstroke}%
\pgfsetdash{}{0pt}%
\pgfpathmoveto{\pgfqpoint{5.435546in}{3.016860in}}%
\pgfpathlineto{\pgfqpoint{5.447094in}{3.002450in}}%
\pgfpathlineto{\pgfqpoint{5.458657in}{2.987601in}}%
\pgfpathlineto{\pgfqpoint{5.491557in}{2.988382in}}%
\pgfpathlineto{\pgfqpoint{5.524438in}{2.989362in}}%
\pgfpathlineto{\pgfqpoint{5.512824in}{3.004386in}}%
\pgfpathlineto{\pgfqpoint{5.501227in}{3.019057in}}%
\pgfpathlineto{\pgfqpoint{5.468396in}{3.017856in}}%
\pgfpathlineto{\pgfqpoint{5.435546in}{3.016860in}}%
\pgfpathclose%
\pgfusepath{fill}%
\end{pgfscope}%
\begin{pgfscope}%
\pgfpathrectangle{\pgfqpoint{1.020000in}{0.880000in}}{\pgfqpoint{6.160000in}{6.160000in}}%
\pgfusepath{clip}%
\pgfsetbuttcap%
\pgfsetroundjoin%
\definecolor{currentfill}{rgb}{0.918282,0.484173,0.377794}%
\pgfsetfillcolor{currentfill}%
\pgfsetlinewidth{0.000000pt}%
\definecolor{currentstroke}{rgb}{0.000000,0.000000,0.000000}%
\pgfsetstrokecolor{currentstroke}%
\pgfsetdash{}{0pt}%
\pgfpathmoveto{\pgfqpoint{2.646268in}{4.763438in}}%
\pgfpathlineto{\pgfqpoint{2.655701in}{4.718181in}}%
\pgfpathlineto{\pgfqpoint{2.665188in}{4.669799in}}%
\pgfpathlineto{\pgfqpoint{2.698439in}{4.708636in}}%
\pgfpathlineto{\pgfqpoint{2.731717in}{4.745629in}}%
\pgfpathlineto{\pgfqpoint{2.722158in}{4.796314in}}%
\pgfpathlineto{\pgfqpoint{2.712655in}{4.843605in}}%
\pgfpathlineto{\pgfqpoint{2.679446in}{4.804501in}}%
\pgfpathlineto{\pgfqpoint{2.646268in}{4.763438in}}%
\pgfpathclose%
\pgfusepath{fill}%
\end{pgfscope}%
\begin{pgfscope}%
\pgfpathrectangle{\pgfqpoint{1.020000in}{0.880000in}}{\pgfqpoint{6.160000in}{6.160000in}}%
\pgfusepath{clip}%
\pgfsetbuttcap%
\pgfsetroundjoin%
\definecolor{currentfill}{rgb}{0.399231,0.528528,0.928459}%
\pgfsetfillcolor{currentfill}%
\pgfsetlinewidth{0.000000pt}%
\definecolor{currentstroke}{rgb}{0.000000,0.000000,0.000000}%
\pgfsetstrokecolor{currentstroke}%
\pgfsetdash{}{0pt}%
\pgfpathmoveto{\pgfqpoint{4.841124in}{3.138608in}}%
\pgfpathlineto{\pgfqpoint{4.852150in}{3.128414in}}%
\pgfpathlineto{\pgfqpoint{4.863193in}{3.117549in}}%
\pgfpathlineto{\pgfqpoint{4.896227in}{3.108839in}}%
\pgfpathlineto{\pgfqpoint{4.929239in}{3.100874in}}%
\pgfpathlineto{\pgfqpoint{4.918139in}{3.111285in}}%
\pgfpathlineto{\pgfqpoint{4.907053in}{3.120718in}}%
\pgfpathlineto{\pgfqpoint{4.874101in}{3.129379in}}%
\pgfpathlineto{\pgfqpoint{4.841124in}{3.138608in}}%
\pgfpathclose%
\pgfusepath{fill}%
\end{pgfscope}%
\begin{pgfscope}%
\pgfpathrectangle{\pgfqpoint{1.020000in}{0.880000in}}{\pgfqpoint{6.160000in}{6.160000in}}%
\pgfusepath{clip}%
\pgfsetbuttcap%
\pgfsetroundjoin%
\definecolor{currentfill}{rgb}{0.892138,0.425389,0.333289}%
\pgfsetfillcolor{currentfill}%
\pgfsetlinewidth{0.000000pt}%
\definecolor{currentstroke}{rgb}{0.000000,0.000000,0.000000}%
\pgfsetstrokecolor{currentstroke}%
\pgfsetdash{}{0pt}%
\pgfpathmoveto{\pgfqpoint{2.712655in}{4.843605in}}%
\pgfpathlineto{\pgfqpoint{2.722158in}{4.796314in}}%
\pgfpathlineto{\pgfqpoint{2.731717in}{4.745629in}}%
\pgfpathlineto{\pgfqpoint{2.765025in}{4.780338in}}%
\pgfpathlineto{\pgfqpoint{2.798368in}{4.812335in}}%
\pgfpathlineto{\pgfqpoint{2.788745in}{4.865024in}}%
\pgfpathlineto{\pgfqpoint{2.779179in}{4.914082in}}%
\pgfpathlineto{\pgfqpoint{2.745898in}{4.880283in}}%
\pgfpathlineto{\pgfqpoint{2.712655in}{4.843605in}}%
\pgfpathclose%
\pgfusepath{fill}%
\end{pgfscope}%
\begin{pgfscope}%
\pgfpathrectangle{\pgfqpoint{1.020000in}{0.880000in}}{\pgfqpoint{6.160000in}{6.160000in}}%
\pgfusepath{clip}%
\pgfsetbuttcap%
\pgfsetroundjoin%
\definecolor{currentfill}{rgb}{0.924409,0.498590,0.389059}%
\pgfsetfillcolor{currentfill}%
\pgfsetlinewidth{0.000000pt}%
\definecolor{currentstroke}{rgb}{0.000000,0.000000,0.000000}%
\pgfsetstrokecolor{currentstroke}%
\pgfsetdash{}{0pt}%
\pgfpathmoveto{\pgfqpoint{3.086184in}{4.806790in}}%
\pgfpathlineto{\pgfqpoint{3.096096in}{4.741726in}}%
\pgfpathlineto{\pgfqpoint{3.106034in}{4.674693in}}%
\pgfpathlineto{\pgfqpoint{3.139732in}{4.669637in}}%
\pgfpathlineto{\pgfqpoint{3.173443in}{4.660436in}}%
\pgfpathlineto{\pgfqpoint{3.163492in}{4.726598in}}%
\pgfpathlineto{\pgfqpoint{3.153566in}{4.790846in}}%
\pgfpathlineto{\pgfqpoint{3.119866in}{4.801083in}}%
\pgfpathlineto{\pgfqpoint{3.086184in}{4.806790in}}%
\pgfpathclose%
\pgfusepath{fill}%
\end{pgfscope}%
\begin{pgfscope}%
\pgfpathrectangle{\pgfqpoint{1.020000in}{0.880000in}}{\pgfqpoint{6.160000in}{6.160000in}}%
\pgfusepath{clip}%
\pgfsetbuttcap%
\pgfsetroundjoin%
\definecolor{currentfill}{rgb}{0.959385,0.610306,0.489382}%
\pgfsetfillcolor{currentfill}%
\pgfsetlinewidth{0.000000pt}%
\definecolor{currentstroke}{rgb}{0.000000,0.000000,0.000000}%
\pgfsetstrokecolor{currentstroke}%
\pgfsetdash{}{0pt}%
\pgfpathmoveto{\pgfqpoint{3.173443in}{4.660436in}}%
\pgfpathlineto{\pgfqpoint{3.183416in}{4.592918in}}%
\pgfpathlineto{\pgfqpoint{3.193403in}{4.524607in}}%
\pgfpathlineto{\pgfqpoint{3.227130in}{4.512885in}}%
\pgfpathlineto{\pgfqpoint{3.260857in}{4.497599in}}%
\pgfpathlineto{\pgfqpoint{3.250865in}{4.564137in}}%
\pgfpathlineto{\pgfqpoint{3.240888in}{4.629948in}}%
\pgfpathlineto{\pgfqpoint{3.207164in}{4.647164in}}%
\pgfpathlineto{\pgfqpoint{3.173443in}{4.660436in}}%
\pgfpathclose%
\pgfusepath{fill}%
\end{pgfscope}%
\begin{pgfscope}%
\pgfpathrectangle{\pgfqpoint{1.020000in}{0.880000in}}{\pgfqpoint{6.160000in}{6.160000in}}%
\pgfusepath{clip}%
\pgfsetbuttcap%
\pgfsetroundjoin%
\definecolor{currentfill}{rgb}{0.353369,0.472069,0.892570}%
\pgfsetfillcolor{currentfill}%
\pgfsetlinewidth{0.000000pt}%
\definecolor{currentstroke}{rgb}{0.000000,0.000000,0.000000}%
\pgfsetstrokecolor{currentstroke}%
\pgfsetdash{}{0pt}%
\pgfpathmoveto{\pgfqpoint{5.215332in}{3.043317in}}%
\pgfpathlineto{\pgfqpoint{5.226698in}{3.031075in}}%
\pgfpathlineto{\pgfqpoint{5.238077in}{3.017889in}}%
\pgfpathlineto{\pgfqpoint{5.271032in}{3.016632in}}%
\pgfpathlineto{\pgfqpoint{5.303970in}{3.015876in}}%
\pgfpathlineto{\pgfqpoint{5.292539in}{3.029162in}}%
\pgfpathlineto{\pgfqpoint{5.281119in}{3.041509in}}%
\pgfpathlineto{\pgfqpoint{5.248234in}{3.042125in}}%
\pgfpathlineto{\pgfqpoint{5.215332in}{3.043317in}}%
\pgfpathclose%
\pgfusepath{fill}%
\end{pgfscope}%
\begin{pgfscope}%
\pgfpathrectangle{\pgfqpoint{1.020000in}{0.880000in}}{\pgfqpoint{6.160000in}{6.160000in}}%
\pgfusepath{clip}%
\pgfsetbuttcap%
\pgfsetroundjoin%
\definecolor{currentfill}{rgb}{0.791392,0.846750,0.936641}%
\pgfsetfillcolor{currentfill}%
\pgfsetlinewidth{0.000000pt}%
\definecolor{currentstroke}{rgb}{0.000000,0.000000,0.000000}%
\pgfsetstrokecolor{currentstroke}%
\pgfsetdash{}{0pt}%
\pgfpathmoveto{\pgfqpoint{3.610210in}{3.880280in}}%
\pgfpathlineto{\pgfqpoint{3.620216in}{3.840977in}}%
\pgfpathlineto{\pgfqpoint{3.630219in}{3.804741in}}%
\pgfpathlineto{\pgfqpoint{3.663805in}{3.786935in}}%
\pgfpathlineto{\pgfqpoint{3.697361in}{3.768324in}}%
\pgfpathlineto{\pgfqpoint{3.687334in}{3.801432in}}%
\pgfpathlineto{\pgfqpoint{3.677308in}{3.837413in}}%
\pgfpathlineto{\pgfqpoint{3.643775in}{3.859225in}}%
\pgfpathlineto{\pgfqpoint{3.610210in}{3.880280in}}%
\pgfpathclose%
\pgfusepath{fill}%
\end{pgfscope}%
\begin{pgfscope}%
\pgfpathrectangle{\pgfqpoint{1.020000in}{0.880000in}}{\pgfqpoint{6.160000in}{6.160000in}}%
\pgfusepath{clip}%
\pgfsetbuttcap%
\pgfsetroundjoin%
\definecolor{currentfill}{rgb}{0.373552,0.497499,0.909467}%
\pgfsetfillcolor{currentfill}%
\pgfsetlinewidth{0.000000pt}%
\definecolor{currentstroke}{rgb}{0.000000,0.000000,0.000000}%
\pgfsetstrokecolor{currentstroke}%
\pgfsetdash{}{0pt}%
\pgfpathmoveto{\pgfqpoint{4.995199in}{3.087141in}}%
\pgfpathlineto{\pgfqpoint{5.006372in}{3.076312in}}%
\pgfpathlineto{\pgfqpoint{5.017559in}{3.064517in}}%
\pgfpathlineto{\pgfqpoint{5.050566in}{3.059172in}}%
\pgfpathlineto{\pgfqpoint{5.083555in}{3.054585in}}%
\pgfpathlineto{\pgfqpoint{5.072311in}{3.066061in}}%
\pgfpathlineto{\pgfqpoint{5.061080in}{3.076358in}}%
\pgfpathlineto{\pgfqpoint{5.028149in}{3.081376in}}%
\pgfpathlineto{\pgfqpoint{4.995199in}{3.087141in}}%
\pgfpathclose%
\pgfusepath{fill}%
\end{pgfscope}%
\begin{pgfscope}%
\pgfpathrectangle{\pgfqpoint{1.020000in}{0.880000in}}{\pgfqpoint{6.160000in}{6.160000in}}%
\pgfusepath{clip}%
\pgfsetbuttcap%
\pgfsetroundjoin%
\definecolor{currentfill}{rgb}{0.624703,0.748318,0.998719}%
\pgfsetfillcolor{currentfill}%
\pgfsetlinewidth{0.000000pt}%
\definecolor{currentstroke}{rgb}{0.000000,0.000000,0.000000}%
\pgfsetstrokecolor{currentstroke}%
\pgfsetdash{}{0pt}%
\pgfpathmoveto{\pgfqpoint{4.005410in}{3.538643in}}%
\pgfpathlineto{\pgfqpoint{4.015642in}{3.526429in}}%
\pgfpathlineto{\pgfqpoint{4.025889in}{3.516423in}}%
\pgfpathlineto{\pgfqpoint{4.059290in}{3.499941in}}%
\pgfpathlineto{\pgfqpoint{4.092659in}{3.483535in}}%
\pgfpathlineto{\pgfqpoint{4.082357in}{3.493480in}}%
\pgfpathlineto{\pgfqpoint{4.072073in}{3.505351in}}%
\pgfpathlineto{\pgfqpoint{4.038758in}{3.521884in}}%
\pgfpathlineto{\pgfqpoint{4.005410in}{3.538643in}}%
\pgfpathclose%
\pgfusepath{fill}%
\end{pgfscope}%
\begin{pgfscope}%
\pgfpathrectangle{\pgfqpoint{1.020000in}{0.880000in}}{\pgfqpoint{6.160000in}{6.160000in}}%
\pgfusepath{clip}%
\pgfsetbuttcap%
\pgfsetroundjoin%
\definecolor{currentfill}{rgb}{0.576051,0.708780,0.997755}%
\pgfsetfillcolor{currentfill}%
\pgfsetlinewidth{0.000000pt}%
\definecolor{currentstroke}{rgb}{0.000000,0.000000,0.000000}%
\pgfsetstrokecolor{currentstroke}%
\pgfsetdash{}{0pt}%
\pgfpathmoveto{\pgfqpoint{4.159299in}{3.451348in}}%
\pgfpathlineto{\pgfqpoint{4.169675in}{3.442829in}}%
\pgfpathlineto{\pgfqpoint{4.180070in}{3.435892in}}%
\pgfpathlineto{\pgfqpoint{4.213398in}{3.419542in}}%
\pgfpathlineto{\pgfqpoint{4.246694in}{3.403533in}}%
\pgfpathlineto{\pgfqpoint{4.236243in}{3.411358in}}%
\pgfpathlineto{\pgfqpoint{4.225813in}{3.420524in}}%
\pgfpathlineto{\pgfqpoint{4.192572in}{3.435733in}}%
\pgfpathlineto{\pgfqpoint{4.159299in}{3.451348in}}%
\pgfpathclose%
\pgfusepath{fill}%
\end{pgfscope}%
\begin{pgfscope}%
\pgfpathrectangle{\pgfqpoint{1.020000in}{0.880000in}}{\pgfqpoint{6.160000in}{6.160000in}}%
\pgfusepath{clip}%
\pgfsetbuttcap%
\pgfsetroundjoin%
\definecolor{currentfill}{rgb}{0.880896,0.402331,0.317115}%
\pgfsetfillcolor{currentfill}%
\pgfsetlinewidth{0.000000pt}%
\definecolor{currentstroke}{rgb}{0.000000,0.000000,0.000000}%
\pgfsetstrokecolor{currentstroke}%
\pgfsetdash{}{0pt}%
\pgfpathmoveto{\pgfqpoint{2.932120in}{4.905445in}}%
\pgfpathlineto{\pgfqpoint{2.941895in}{4.846537in}}%
\pgfpathlineto{\pgfqpoint{2.951713in}{4.784612in}}%
\pgfpathlineto{\pgfqpoint{2.985286in}{4.796735in}}%
\pgfpathlineto{\pgfqpoint{3.018891in}{4.804558in}}%
\pgfpathlineto{\pgfqpoint{3.009038in}{4.867145in}}%
\pgfpathlineto{\pgfqpoint{2.999228in}{4.926669in}}%
\pgfpathlineto{\pgfqpoint{2.965656in}{4.918374in}}%
\pgfpathlineto{\pgfqpoint{2.932120in}{4.905445in}}%
\pgfpathclose%
\pgfusepath{fill}%
\end{pgfscope}%
\begin{pgfscope}%
\pgfpathrectangle{\pgfqpoint{1.020000in}{0.880000in}}{\pgfqpoint{6.160000in}{6.160000in}}%
\pgfusepath{clip}%
\pgfsetbuttcap%
\pgfsetroundjoin%
\definecolor{currentfill}{rgb}{0.532568,0.669801,0.990393}%
\pgfsetfillcolor{currentfill}%
\pgfsetlinewidth{0.000000pt}%
\definecolor{currentstroke}{rgb}{0.000000,0.000000,0.000000}%
\pgfsetstrokecolor{currentstroke}%
\pgfsetdash{}{0pt}%
\pgfpathmoveto{\pgfqpoint{4.313192in}{3.372786in}}%
\pgfpathlineto{\pgfqpoint{4.323719in}{3.365076in}}%
\pgfpathlineto{\pgfqpoint{4.334268in}{3.358367in}}%
\pgfpathlineto{\pgfqpoint{4.367525in}{3.342467in}}%
\pgfpathlineto{\pgfqpoint{4.400751in}{3.327115in}}%
\pgfpathlineto{\pgfqpoint{4.390148in}{3.335121in}}%
\pgfpathlineto{\pgfqpoint{4.379568in}{3.343938in}}%
\pgfpathlineto{\pgfqpoint{4.346395in}{3.358121in}}%
\pgfpathlineto{\pgfqpoint{4.313192in}{3.372786in}}%
\pgfpathclose%
\pgfusepath{fill}%
\end{pgfscope}%
\begin{pgfscope}%
\pgfpathrectangle{\pgfqpoint{1.020000in}{0.880000in}}{\pgfqpoint{6.160000in}{6.160000in}}%
\pgfusepath{clip}%
\pgfsetbuttcap%
\pgfsetroundjoin%
\definecolor{currentfill}{rgb}{0.677823,0.786546,0.991005}%
\pgfsetfillcolor{currentfill}%
\pgfsetlinewidth{0.000000pt}%
\definecolor{currentstroke}{rgb}{0.000000,0.000000,0.000000}%
\pgfsetstrokecolor{currentstroke}%
\pgfsetdash{}{0pt}%
\pgfpathmoveto{\pgfqpoint{3.851464in}{3.642008in}}%
\pgfpathlineto{\pgfqpoint{3.861575in}{3.622526in}}%
\pgfpathlineto{\pgfqpoint{3.871696in}{3.605822in}}%
\pgfpathlineto{\pgfqpoint{3.905172in}{3.589241in}}%
\pgfpathlineto{\pgfqpoint{3.938617in}{3.572436in}}%
\pgfpathlineto{\pgfqpoint{3.928449in}{3.587827in}}%
\pgfpathlineto{\pgfqpoint{3.918294in}{3.605712in}}%
\pgfpathlineto{\pgfqpoint{3.884895in}{3.623883in}}%
\pgfpathlineto{\pgfqpoint{3.851464in}{3.642008in}}%
\pgfpathclose%
\pgfusepath{fill}%
\end{pgfscope}%
\begin{pgfscope}%
\pgfpathrectangle{\pgfqpoint{1.020000in}{0.880000in}}{\pgfqpoint{6.160000in}{6.160000in}}%
\pgfusepath{clip}%
\pgfsetbuttcap%
\pgfsetroundjoin%
\definecolor{currentfill}{rgb}{0.313946,0.420052,0.854993}%
\pgfsetfillcolor{currentfill}%
\pgfsetlinewidth{0.000000pt}%
\definecolor{currentstroke}{rgb}{0.000000,0.000000,0.000000}%
\pgfsetstrokecolor{currentstroke}%
\pgfsetdash{}{0pt}%
\pgfpathmoveto{\pgfqpoint{5.810485in}{2.970078in}}%
\pgfpathlineto{\pgfqpoint{5.822367in}{2.954586in}}%
\pgfpathlineto{\pgfqpoint{5.834273in}{2.939112in}}%
\pgfpathlineto{\pgfqpoint{5.867082in}{2.940891in}}%
\pgfpathlineto{\pgfqpoint{5.899871in}{2.942707in}}%
\pgfpathlineto{\pgfqpoint{5.887913in}{2.958111in}}%
\pgfpathlineto{\pgfqpoint{5.875977in}{2.973541in}}%
\pgfpathlineto{\pgfqpoint{5.843241in}{2.971794in}}%
\pgfpathlineto{\pgfqpoint{5.810485in}{2.970078in}}%
\pgfpathclose%
\pgfusepath{fill}%
\end{pgfscope}%
\begin{pgfscope}%
\pgfpathrectangle{\pgfqpoint{1.020000in}{0.880000in}}{\pgfqpoint{6.160000in}{6.160000in}}%
\pgfusepath{clip}%
\pgfsetbuttcap%
\pgfsetroundjoin%
\definecolor{currentfill}{rgb}{0.328604,0.439712,0.869587}%
\pgfsetfillcolor{currentfill}%
\pgfsetlinewidth{0.000000pt}%
\definecolor{currentstroke}{rgb}{0.000000,0.000000,0.000000}%
\pgfsetstrokecolor{currentstroke}%
\pgfsetdash{}{0pt}%
\pgfpathmoveto{\pgfqpoint{5.590142in}{2.991788in}}%
\pgfpathlineto{\pgfqpoint{5.601826in}{2.976414in}}%
\pgfpathlineto{\pgfqpoint{5.613530in}{2.960917in}}%
\pgfpathlineto{\pgfqpoint{5.646405in}{2.962249in}}%
\pgfpathlineto{\pgfqpoint{5.679260in}{2.963680in}}%
\pgfpathlineto{\pgfqpoint{5.667505in}{2.979202in}}%
\pgfpathlineto{\pgfqpoint{5.655769in}{2.994647in}}%
\pgfpathlineto{\pgfqpoint{5.622966in}{2.993175in}}%
\pgfpathlineto{\pgfqpoint{5.590142in}{2.991788in}}%
\pgfpathclose%
\pgfusepath{fill}%
\end{pgfscope}%
\begin{pgfscope}%
\pgfpathrectangle{\pgfqpoint{1.020000in}{0.880000in}}{\pgfqpoint{6.160000in}{6.160000in}}%
\pgfusepath{clip}%
\pgfsetbuttcap%
\pgfsetroundjoin%
\definecolor{currentfill}{rgb}{0.489246,0.627536,0.976896}%
\pgfsetfillcolor{currentfill}%
\pgfsetlinewidth{0.000000pt}%
\definecolor{currentstroke}{rgb}{0.000000,0.000000,0.000000}%
\pgfsetstrokecolor{currentstroke}%
\pgfsetdash{}{0pt}%
\pgfpathmoveto{\pgfqpoint{4.467113in}{3.298078in}}%
\pgfpathlineto{\pgfqpoint{4.477790in}{3.289544in}}%
\pgfpathlineto{\pgfqpoint{4.488490in}{3.281511in}}%
\pgfpathlineto{\pgfqpoint{4.521678in}{3.266664in}}%
\pgfpathlineto{\pgfqpoint{4.554838in}{3.252479in}}%
\pgfpathlineto{\pgfqpoint{4.544088in}{3.261614in}}%
\pgfpathlineto{\pgfqpoint{4.533359in}{3.271077in}}%
\pgfpathlineto{\pgfqpoint{4.500250in}{3.284347in}}%
\pgfpathlineto{\pgfqpoint{4.467113in}{3.298078in}}%
\pgfpathclose%
\pgfusepath{fill}%
\end{pgfscope}%
\begin{pgfscope}%
\pgfpathrectangle{\pgfqpoint{1.020000in}{0.880000in}}{\pgfqpoint{6.160000in}{6.160000in}}%
\pgfusepath{clip}%
\pgfsetbuttcap%
\pgfsetroundjoin%
\definecolor{currentfill}{rgb}{0.304174,0.406945,0.845263}%
\pgfsetfillcolor{currentfill}%
\pgfsetlinewidth{0.000000pt}%
\definecolor{currentstroke}{rgb}{0.000000,0.000000,0.000000}%
\pgfsetstrokecolor{currentstroke}%
\pgfsetdash{}{0pt}%
\pgfpathmoveto{\pgfqpoint{6.030829in}{2.950259in}}%
\pgfpathlineto{\pgfqpoint{6.042918in}{2.935057in}}%
\pgfpathlineto{\pgfqpoint{6.055030in}{2.919896in}}%
\pgfpathlineto{\pgfqpoint{6.087773in}{2.921927in}}%
\pgfpathlineto{\pgfqpoint{6.075634in}{2.937044in}}%
\pgfpathlineto{\pgfqpoint{6.063519in}{2.952202in}}%
\pgfpathlineto{\pgfqpoint{6.030829in}{2.950259in}}%
\pgfpathclose%
\pgfusepath{fill}%
\end{pgfscope}%
\begin{pgfscope}%
\pgfpathrectangle{\pgfqpoint{1.020000in}{0.880000in}}{\pgfqpoint{6.160000in}{6.160000in}}%
\pgfusepath{clip}%
\pgfsetbuttcap%
\pgfsetroundjoin%
\definecolor{currentfill}{rgb}{0.338377,0.452819,0.879317}%
\pgfsetfillcolor{currentfill}%
\pgfsetlinewidth{0.000000pt}%
\definecolor{currentstroke}{rgb}{0.000000,0.000000,0.000000}%
\pgfsetstrokecolor{currentstroke}%
\pgfsetdash{}{0pt}%
\pgfpathmoveto{\pgfqpoint{5.369794in}{3.015668in}}%
\pgfpathlineto{\pgfqpoint{5.381290in}{3.001487in}}%
\pgfpathlineto{\pgfqpoint{5.392803in}{2.986794in}}%
\pgfpathlineto{\pgfqpoint{5.425739in}{2.987057in}}%
\pgfpathlineto{\pgfqpoint{5.458657in}{2.987601in}}%
\pgfpathlineto{\pgfqpoint{5.447094in}{3.002450in}}%
\pgfpathlineto{\pgfqpoint{5.435546in}{3.016860in}}%
\pgfpathlineto{\pgfqpoint{5.402679in}{3.016114in}}%
\pgfpathlineto{\pgfqpoint{5.369794in}{3.015668in}}%
\pgfpathclose%
\pgfusepath{fill}%
\end{pgfscope}%
\begin{pgfscope}%
\pgfpathrectangle{\pgfqpoint{1.020000in}{0.880000in}}{\pgfqpoint{6.160000in}{6.160000in}}%
\pgfusepath{clip}%
\pgfsetbuttcap%
\pgfsetroundjoin%
\definecolor{currentfill}{rgb}{0.451739,0.588181,0.960201}%
\pgfsetfillcolor{currentfill}%
\pgfsetlinewidth{0.000000pt}%
\definecolor{currentstroke}{rgb}{0.000000,0.000000,0.000000}%
\pgfsetstrokecolor{currentstroke}%
\pgfsetdash{}{0pt}%
\pgfpathmoveto{\pgfqpoint{4.621074in}{3.225934in}}%
\pgfpathlineto{\pgfqpoint{4.631898in}{3.216241in}}%
\pgfpathlineto{\pgfqpoint{4.642743in}{3.206586in}}%
\pgfpathlineto{\pgfqpoint{4.675871in}{3.193582in}}%
\pgfpathlineto{\pgfqpoint{4.708972in}{3.181308in}}%
\pgfpathlineto{\pgfqpoint{4.698076in}{3.191459in}}%
\pgfpathlineto{\pgfqpoint{4.687199in}{3.201436in}}%
\pgfpathlineto{\pgfqpoint{4.654150in}{3.213461in}}%
\pgfpathlineto{\pgfqpoint{4.621074in}{3.225934in}}%
\pgfpathclose%
\pgfusepath{fill}%
\end{pgfscope}%
\begin{pgfscope}%
\pgfpathrectangle{\pgfqpoint{1.020000in}{0.880000in}}{\pgfqpoint{6.160000in}{6.160000in}}%
\pgfusepath{clip}%
\pgfsetbuttcap%
\pgfsetroundjoin%
\definecolor{currentfill}{rgb}{0.743754,0.825125,0.965798}%
\pgfsetfillcolor{currentfill}%
\pgfsetlinewidth{0.000000pt}%
\definecolor{currentstroke}{rgb}{0.000000,0.000000,0.000000}%
\pgfsetstrokecolor{currentstroke}%
\pgfsetdash{}{0pt}%
\pgfpathmoveto{\pgfqpoint{3.697361in}{3.768324in}}%
\pgfpathlineto{\pgfqpoint{3.707389in}{3.738251in}}%
\pgfpathlineto{\pgfqpoint{3.717419in}{3.711337in}}%
\pgfpathlineto{\pgfqpoint{3.750976in}{3.694762in}}%
\pgfpathlineto{\pgfqpoint{3.784504in}{3.677587in}}%
\pgfpathlineto{\pgfqpoint{3.774439in}{3.702007in}}%
\pgfpathlineto{\pgfqpoint{3.764380in}{3.729345in}}%
\pgfpathlineto{\pgfqpoint{3.730887in}{3.749074in}}%
\pgfpathlineto{\pgfqpoint{3.697361in}{3.768324in}}%
\pgfpathclose%
\pgfusepath{fill}%
\end{pgfscope}%
\begin{pgfscope}%
\pgfpathrectangle{\pgfqpoint{1.020000in}{0.880000in}}{\pgfqpoint{6.160000in}{6.160000in}}%
\pgfusepath{clip}%
\pgfsetbuttcap%
\pgfsetroundjoin%
\definecolor{currentfill}{rgb}{0.921406,0.491420,0.383408}%
\pgfsetfillcolor{currentfill}%
\pgfsetlinewidth{0.000000pt}%
\definecolor{currentstroke}{rgb}{0.000000,0.000000,0.000000}%
\pgfsetstrokecolor{currentstroke}%
\pgfsetdash{}{0pt}%
\pgfpathmoveto{\pgfqpoint{3.018891in}{4.804558in}}%
\pgfpathlineto{\pgfqpoint{3.028779in}{4.739437in}}%
\pgfpathlineto{\pgfqpoint{3.038696in}{4.672325in}}%
\pgfpathlineto{\pgfqpoint{3.072354in}{4.675579in}}%
\pgfpathlineto{\pgfqpoint{3.106034in}{4.674693in}}%
\pgfpathlineto{\pgfqpoint{3.096096in}{4.741726in}}%
\pgfpathlineto{\pgfqpoint{3.086184in}{4.806790in}}%
\pgfpathlineto{\pgfqpoint{3.052525in}{4.807939in}}%
\pgfpathlineto{\pgfqpoint{3.018891in}{4.804558in}}%
\pgfpathclose%
\pgfusepath{fill}%
\end{pgfscope}%
\begin{pgfscope}%
\pgfpathrectangle{\pgfqpoint{1.020000in}{0.880000in}}{\pgfqpoint{6.160000in}{6.160000in}}%
\pgfusepath{clip}%
\pgfsetbuttcap%
\pgfsetroundjoin%
\definecolor{currentfill}{rgb}{0.892138,0.425389,0.333289}%
\pgfsetfillcolor{currentfill}%
\pgfsetlinewidth{0.000000pt}%
\definecolor{currentstroke}{rgb}{0.000000,0.000000,0.000000}%
\pgfsetstrokecolor{currentstroke}%
\pgfsetdash{}{0pt}%
\pgfpathmoveto{\pgfqpoint{2.865166in}{4.866579in}}%
\pgfpathlineto{\pgfqpoint{2.874895in}{4.808899in}}%
\pgfpathlineto{\pgfqpoint{2.884668in}{4.748296in}}%
\pgfpathlineto{\pgfqpoint{2.918174in}{4.768384in}}%
\pgfpathlineto{\pgfqpoint{2.951713in}{4.784612in}}%
\pgfpathlineto{\pgfqpoint{2.941895in}{4.846537in}}%
\pgfpathlineto{\pgfqpoint{2.932120in}{4.905445in}}%
\pgfpathlineto{\pgfqpoint{2.898624in}{4.888093in}}%
\pgfpathlineto{\pgfqpoint{2.865166in}{4.866579in}}%
\pgfpathclose%
\pgfusepath{fill}%
\end{pgfscope}%
\begin{pgfscope}%
\pgfpathrectangle{\pgfqpoint{1.020000in}{0.880000in}}{\pgfqpoint{6.160000in}{6.160000in}}%
\pgfusepath{clip}%
\pgfsetbuttcap%
\pgfsetroundjoin%
\definecolor{currentfill}{rgb}{0.358415,0.478426,0.896795}%
\pgfsetfillcolor{currentfill}%
\pgfsetlinewidth{0.000000pt}%
\definecolor{currentstroke}{rgb}{0.000000,0.000000,0.000000}%
\pgfsetstrokecolor{currentstroke}%
\pgfsetdash{}{0pt}%
\pgfpathmoveto{\pgfqpoint{5.149478in}{3.047588in}}%
\pgfpathlineto{\pgfqpoint{5.160790in}{3.035290in}}%
\pgfpathlineto{\pgfqpoint{5.172115in}{3.022102in}}%
\pgfpathlineto{\pgfqpoint{5.205105in}{3.019697in}}%
\pgfpathlineto{\pgfqpoint{5.238077in}{3.017889in}}%
\pgfpathlineto{\pgfqpoint{5.226698in}{3.031075in}}%
\pgfpathlineto{\pgfqpoint{5.215332in}{3.043317in}}%
\pgfpathlineto{\pgfqpoint{5.182413in}{3.045126in}}%
\pgfpathlineto{\pgfqpoint{5.149478in}{3.047588in}}%
\pgfpathclose%
\pgfusepath{fill}%
\end{pgfscope}%
\begin{pgfscope}%
\pgfpathrectangle{\pgfqpoint{1.020000in}{0.880000in}}{\pgfqpoint{6.160000in}{6.160000in}}%
\pgfusepath{clip}%
\pgfsetbuttcap%
\pgfsetroundjoin%
\definecolor{currentfill}{rgb}{0.956371,0.775144,0.686416}%
\pgfsetfillcolor{currentfill}%
\pgfsetlinewidth{0.000000pt}%
\definecolor{currentstroke}{rgb}{0.000000,0.000000,0.000000}%
\pgfsetstrokecolor{currentstroke}%
\pgfsetdash{}{0pt}%
\pgfpathmoveto{\pgfqpoint{2.332988in}{4.270503in}}%
\pgfpathlineto{\pgfqpoint{2.342095in}{4.233008in}}%
\pgfpathlineto{\pgfqpoint{2.351236in}{4.194207in}}%
\pgfpathlineto{\pgfqpoint{2.384606in}{4.227191in}}%
\pgfpathlineto{\pgfqpoint{2.417948in}{4.261784in}}%
\pgfpathlineto{\pgfqpoint{2.408716in}{4.303326in}}%
\pgfpathlineto{\pgfqpoint{2.399520in}{4.343352in}}%
\pgfpathlineto{\pgfqpoint{2.366268in}{4.306067in}}%
\pgfpathlineto{\pgfqpoint{2.332988in}{4.270503in}}%
\pgfpathclose%
\pgfusepath{fill}%
\end{pgfscope}%
\begin{pgfscope}%
\pgfpathrectangle{\pgfqpoint{1.020000in}{0.880000in}}{\pgfqpoint{6.160000in}{6.160000in}}%
\pgfusepath{clip}%
\pgfsetbuttcap%
\pgfsetroundjoin%
\definecolor{currentfill}{rgb}{0.922681,0.828568,0.777054}%
\pgfsetfillcolor{currentfill}%
\pgfsetlinewidth{0.000000pt}%
\definecolor{currentstroke}{rgb}{0.000000,0.000000,0.000000}%
\pgfsetstrokecolor{currentstroke}%
\pgfsetdash{}{0pt}%
\pgfpathmoveto{\pgfqpoint{3.368316in}{4.204649in}}%
\pgfpathlineto{\pgfqpoint{3.378317in}{4.144934in}}%
\pgfpathlineto{\pgfqpoint{3.388312in}{4.087426in}}%
\pgfpathlineto{\pgfqpoint{3.422014in}{4.071707in}}%
\pgfpathlineto{\pgfqpoint{3.455698in}{4.054009in}}%
\pgfpathlineto{\pgfqpoint{3.445700in}{4.108363in}}%
\pgfpathlineto{\pgfqpoint{3.435697in}{4.164889in}}%
\pgfpathlineto{\pgfqpoint{3.402015in}{4.185872in}}%
\pgfpathlineto{\pgfqpoint{3.368316in}{4.204649in}}%
\pgfpathclose%
\pgfusepath{fill}%
\end{pgfscope}%
\begin{pgfscope}%
\pgfpathrectangle{\pgfqpoint{1.020000in}{0.880000in}}{\pgfqpoint{6.160000in}{6.160000in}}%
\pgfusepath{clip}%
\pgfsetbuttcap%
\pgfsetroundjoin%
\definecolor{currentfill}{rgb}{0.965899,0.740142,0.637058}%
\pgfsetfillcolor{currentfill}%
\pgfsetlinewidth{0.000000pt}%
\definecolor{currentstroke}{rgb}{0.000000,0.000000,0.000000}%
\pgfsetstrokecolor{currentstroke}%
\pgfsetdash{}{0pt}%
\pgfpathmoveto{\pgfqpoint{2.399520in}{4.343352in}}%
\pgfpathlineto{\pgfqpoint{2.408716in}{4.303326in}}%
\pgfpathlineto{\pgfqpoint{2.417948in}{4.261784in}}%
\pgfpathlineto{\pgfqpoint{2.451268in}{4.297782in}}%
\pgfpathlineto{\pgfqpoint{2.484569in}{4.334946in}}%
\pgfpathlineto{\pgfqpoint{2.475246in}{4.379426in}}%
\pgfpathlineto{\pgfqpoint{2.465961in}{4.422169in}}%
\pgfpathlineto{\pgfqpoint{2.432749in}{4.382138in}}%
\pgfpathlineto{\pgfqpoint{2.399520in}{4.343352in}}%
\pgfpathclose%
\pgfusepath{fill}%
\end{pgfscope}%
\begin{pgfscope}%
\pgfpathrectangle{\pgfqpoint{1.020000in}{0.880000in}}{\pgfqpoint{6.160000in}{6.160000in}}%
\pgfusepath{clip}%
\pgfsetbuttcap%
\pgfsetroundjoin%
\definecolor{currentfill}{rgb}{0.959518,0.766973,0.674145}%
\pgfsetfillcolor{currentfill}%
\pgfsetlinewidth{0.000000pt}%
\definecolor{currentstroke}{rgb}{0.000000,0.000000,0.000000}%
\pgfsetstrokecolor{currentstroke}%
\pgfsetdash{}{0pt}%
\pgfpathmoveto{\pgfqpoint{3.280864in}{4.364512in}}%
\pgfpathlineto{\pgfqpoint{3.290871in}{4.299004in}}%
\pgfpathlineto{\pgfqpoint{3.300876in}{4.234840in}}%
\pgfpathlineto{\pgfqpoint{3.334602in}{4.221028in}}%
\pgfpathlineto{\pgfqpoint{3.368316in}{4.204649in}}%
\pgfpathlineto{\pgfqpoint{3.358310in}{4.266167in}}%
\pgfpathlineto{\pgfqpoint{3.348303in}{4.329047in}}%
\pgfpathlineto{\pgfqpoint{3.314588in}{4.348222in}}%
\pgfpathlineto{\pgfqpoint{3.280864in}{4.364512in}}%
\pgfpathclose%
\pgfusepath{fill}%
\end{pgfscope}%
\begin{pgfscope}%
\pgfpathrectangle{\pgfqpoint{1.020000in}{0.880000in}}{\pgfqpoint{6.160000in}{6.160000in}}%
\pgfusepath{clip}%
\pgfsetbuttcap%
\pgfsetroundjoin%
\definecolor{currentfill}{rgb}{0.414801,0.546874,0.939088}%
\pgfsetfillcolor{currentfill}%
\pgfsetlinewidth{0.000000pt}%
\definecolor{currentstroke}{rgb}{0.000000,0.000000,0.000000}%
\pgfsetstrokecolor{currentstroke}%
\pgfsetdash{}{0pt}%
\pgfpathmoveto{\pgfqpoint{4.775099in}{3.158757in}}%
\pgfpathlineto{\pgfqpoint{4.786068in}{3.148237in}}%
\pgfpathlineto{\pgfqpoint{4.797057in}{3.137309in}}%
\pgfpathlineto{\pgfqpoint{4.830137in}{3.127028in}}%
\pgfpathlineto{\pgfqpoint{4.863193in}{3.117549in}}%
\pgfpathlineto{\pgfqpoint{4.852150in}{3.128414in}}%
\pgfpathlineto{\pgfqpoint{4.841124in}{3.138608in}}%
\pgfpathlineto{\pgfqpoint{4.808124in}{3.148398in}}%
\pgfpathlineto{\pgfqpoint{4.775099in}{3.158757in}}%
\pgfpathclose%
\pgfusepath{fill}%
\end{pgfscope}%
\begin{pgfscope}%
\pgfpathrectangle{\pgfqpoint{1.020000in}{0.880000in}}{\pgfqpoint{6.160000in}{6.160000in}}%
\pgfusepath{clip}%
\pgfsetbuttcap%
\pgfsetroundjoin%
\definecolor{currentfill}{rgb}{0.969289,0.684982,0.568975}%
\pgfsetfillcolor{currentfill}%
\pgfsetlinewidth{0.000000pt}%
\definecolor{currentstroke}{rgb}{0.000000,0.000000,0.000000}%
\pgfsetstrokecolor{currentstroke}%
\pgfsetdash{}{0pt}%
\pgfpathmoveto{\pgfqpoint{3.193403in}{4.524607in}}%
\pgfpathlineto{\pgfqpoint{3.203401in}{4.456057in}}%
\pgfpathlineto{\pgfqpoint{3.213403in}{4.387807in}}%
\pgfpathlineto{\pgfqpoint{3.247134in}{4.377750in}}%
\pgfpathlineto{\pgfqpoint{3.280864in}{4.364512in}}%
\pgfpathlineto{\pgfqpoint{3.270858in}{4.430880in}}%
\pgfpathlineto{\pgfqpoint{3.260857in}{4.497599in}}%
\pgfpathlineto{\pgfqpoint{3.227130in}{4.512885in}}%
\pgfpathlineto{\pgfqpoint{3.193403in}{4.524607in}}%
\pgfpathclose%
\pgfusepath{fill}%
\end{pgfscope}%
\begin{pgfscope}%
\pgfpathrectangle{\pgfqpoint{1.020000in}{0.880000in}}{\pgfqpoint{6.160000in}{6.160000in}}%
\pgfusepath{clip}%
\pgfsetbuttcap%
\pgfsetroundjoin%
\definecolor{currentfill}{rgb}{0.954853,0.591622,0.471337}%
\pgfsetfillcolor{currentfill}%
\pgfsetlinewidth{0.000000pt}%
\definecolor{currentstroke}{rgb}{0.000000,0.000000,0.000000}%
\pgfsetstrokecolor{currentstroke}%
\pgfsetdash{}{0pt}%
\pgfpathmoveto{\pgfqpoint{3.106034in}{4.674693in}}%
\pgfpathlineto{\pgfqpoint{3.115995in}{4.606251in}}%
\pgfpathlineto{\pgfqpoint{3.125972in}{4.536964in}}%
\pgfpathlineto{\pgfqpoint{3.159683in}{4.532655in}}%
\pgfpathlineto{\pgfqpoint{3.193403in}{4.524607in}}%
\pgfpathlineto{\pgfqpoint{3.183416in}{4.592918in}}%
\pgfpathlineto{\pgfqpoint{3.173443in}{4.660436in}}%
\pgfpathlineto{\pgfqpoint{3.139732in}{4.669637in}}%
\pgfpathlineto{\pgfqpoint{3.106034in}{4.674693in}}%
\pgfpathclose%
\pgfusepath{fill}%
\end{pgfscope}%
\begin{pgfscope}%
\pgfpathrectangle{\pgfqpoint{1.020000in}{0.880000in}}{\pgfqpoint{6.160000in}{6.160000in}}%
\pgfusepath{clip}%
\pgfsetbuttcap%
\pgfsetroundjoin%
\definecolor{currentfill}{rgb}{0.969522,0.700833,0.587508}%
\pgfsetfillcolor{currentfill}%
\pgfsetlinewidth{0.000000pt}%
\definecolor{currentstroke}{rgb}{0.000000,0.000000,0.000000}%
\pgfsetstrokecolor{currentstroke}%
\pgfsetdash{}{0pt}%
\pgfpathmoveto{\pgfqpoint{2.465961in}{4.422169in}}%
\pgfpathlineto{\pgfqpoint{2.475246in}{4.379426in}}%
\pgfpathlineto{\pgfqpoint{2.484569in}{4.334946in}}%
\pgfpathlineto{\pgfqpoint{2.517858in}{4.373003in}}%
\pgfpathlineto{\pgfqpoint{2.551140in}{4.411647in}}%
\pgfpathlineto{\pgfqpoint{2.541727in}{4.459183in}}%
\pgfpathlineto{\pgfqpoint{2.532354in}{4.504753in}}%
\pgfpathlineto{\pgfqpoint{2.499160in}{4.463150in}}%
\pgfpathlineto{\pgfqpoint{2.465961in}{4.422169in}}%
\pgfpathclose%
\pgfusepath{fill}%
\end{pgfscope}%
\begin{pgfscope}%
\pgfpathrectangle{\pgfqpoint{1.020000in}{0.880000in}}{\pgfqpoint{6.160000in}{6.160000in}}%
\pgfusepath{clip}%
\pgfsetbuttcap%
\pgfsetroundjoin%
\definecolor{currentfill}{rgb}{0.867428,0.864377,0.862602}%
\pgfsetfillcolor{currentfill}%
\pgfsetlinewidth{0.000000pt}%
\definecolor{currentstroke}{rgb}{0.000000,0.000000,0.000000}%
\pgfsetstrokecolor{currentstroke}%
\pgfsetdash{}{0pt}%
\pgfpathmoveto{\pgfqpoint{3.455698in}{4.054009in}}%
\pgfpathlineto{\pgfqpoint{3.465691in}{4.002181in}}%
\pgfpathlineto{\pgfqpoint{3.475675in}{3.953195in}}%
\pgfpathlineto{\pgfqpoint{3.509346in}{3.937022in}}%
\pgfpathlineto{\pgfqpoint{3.542994in}{3.919363in}}%
\pgfpathlineto{\pgfqpoint{3.532999in}{3.965011in}}%
\pgfpathlineto{\pgfqpoint{3.522999in}{4.013404in}}%
\pgfpathlineto{\pgfqpoint{3.489361in}{4.034509in}}%
\pgfpathlineto{\pgfqpoint{3.455698in}{4.054009in}}%
\pgfpathclose%
\pgfusepath{fill}%
\end{pgfscope}%
\begin{pgfscope}%
\pgfpathrectangle{\pgfqpoint{1.020000in}{0.880000in}}{\pgfqpoint{6.160000in}{6.160000in}}%
\pgfusepath{clip}%
\pgfsetbuttcap%
\pgfsetroundjoin%
\definecolor{currentfill}{rgb}{0.966922,0.651969,0.531997}%
\pgfsetfillcolor{currentfill}%
\pgfsetlinewidth{0.000000pt}%
\definecolor{currentstroke}{rgb}{0.000000,0.000000,0.000000}%
\pgfsetstrokecolor{currentstroke}%
\pgfsetdash{}{0pt}%
\pgfpathmoveto{\pgfqpoint{2.532354in}{4.504753in}}%
\pgfpathlineto{\pgfqpoint{2.541727in}{4.459183in}}%
\pgfpathlineto{\pgfqpoint{2.551140in}{4.411647in}}%
\pgfpathlineto{\pgfqpoint{2.584420in}{4.450545in}}%
\pgfpathlineto{\pgfqpoint{2.617704in}{4.489336in}}%
\pgfpathlineto{\pgfqpoint{2.608205in}{4.539943in}}%
\pgfpathlineto{\pgfqpoint{2.598748in}{4.588356in}}%
\pgfpathlineto{\pgfqpoint{2.565548in}{4.546617in}}%
\pgfpathlineto{\pgfqpoint{2.532354in}{4.504753in}}%
\pgfpathclose%
\pgfusepath{fill}%
\end{pgfscope}%
\begin{pgfscope}%
\pgfpathrectangle{\pgfqpoint{1.020000in}{0.880000in}}{\pgfqpoint{6.160000in}{6.160000in}}%
\pgfusepath{clip}%
\pgfsetbuttcap%
\pgfsetroundjoin%
\definecolor{currentfill}{rgb}{0.905783,0.455186,0.355336}%
\pgfsetfillcolor{currentfill}%
\pgfsetlinewidth{0.000000pt}%
\definecolor{currentstroke}{rgb}{0.000000,0.000000,0.000000}%
\pgfsetstrokecolor{currentstroke}%
\pgfsetdash{}{0pt}%
\pgfpathmoveto{\pgfqpoint{2.798368in}{4.812335in}}%
\pgfpathlineto{\pgfqpoint{2.808042in}{4.756422in}}%
\pgfpathlineto{\pgfqpoint{2.817760in}{4.697724in}}%
\pgfpathlineto{\pgfqpoint{2.851197in}{4.724634in}}%
\pgfpathlineto{\pgfqpoint{2.884668in}{4.748296in}}%
\pgfpathlineto{\pgfqpoint{2.874895in}{4.808899in}}%
\pgfpathlineto{\pgfqpoint{2.865166in}{4.866579in}}%
\pgfpathlineto{\pgfqpoint{2.831748in}{4.841209in}}%
\pgfpathlineto{\pgfqpoint{2.798368in}{4.812335in}}%
\pgfpathclose%
\pgfusepath{fill}%
\end{pgfscope}%
\begin{pgfscope}%
\pgfpathrectangle{\pgfqpoint{1.020000in}{0.880000in}}{\pgfqpoint{6.160000in}{6.160000in}}%
\pgfusepath{clip}%
\pgfsetbuttcap%
\pgfsetroundjoin%
\definecolor{currentfill}{rgb}{0.309060,0.413498,0.850128}%
\pgfsetfillcolor{currentfill}%
\pgfsetlinewidth{0.000000pt}%
\definecolor{currentstroke}{rgb}{0.000000,0.000000,0.000000}%
\pgfsetstrokecolor{currentstroke}%
\pgfsetdash{}{0pt}%
\pgfpathmoveto{\pgfqpoint{5.965390in}{2.946433in}}%
\pgfpathlineto{\pgfqpoint{5.977425in}{2.931146in}}%
\pgfpathlineto{\pgfqpoint{5.989483in}{2.915897in}}%
\pgfpathlineto{\pgfqpoint{6.022267in}{2.917885in}}%
\pgfpathlineto{\pgfqpoint{6.055030in}{2.919896in}}%
\pgfpathlineto{\pgfqpoint{6.042918in}{2.935057in}}%
\pgfpathlineto{\pgfqpoint{6.030829in}{2.950259in}}%
\pgfpathlineto{\pgfqpoint{5.998120in}{2.948335in}}%
\pgfpathlineto{\pgfqpoint{5.965390in}{2.946433in}}%
\pgfpathclose%
\pgfusepath{fill}%
\end{pgfscope}%
\begin{pgfscope}%
\pgfpathrectangle{\pgfqpoint{1.020000in}{0.880000in}}{\pgfqpoint{6.160000in}{6.160000in}}%
\pgfusepath{clip}%
\pgfsetbuttcap%
\pgfsetroundjoin%
\definecolor{currentfill}{rgb}{0.958279,0.604335,0.483297}%
\pgfsetfillcolor{currentfill}%
\pgfsetlinewidth{0.000000pt}%
\definecolor{currentstroke}{rgb}{0.000000,0.000000,0.000000}%
\pgfsetstrokecolor{currentstroke}%
\pgfsetdash{}{0pt}%
\pgfpathmoveto{\pgfqpoint{2.598748in}{4.588356in}}%
\pgfpathlineto{\pgfqpoint{2.608205in}{4.539943in}}%
\pgfpathlineto{\pgfqpoint{2.617704in}{4.489336in}}%
\pgfpathlineto{\pgfqpoint{2.650997in}{4.527640in}}%
\pgfpathlineto{\pgfqpoint{2.684304in}{4.565060in}}%
\pgfpathlineto{\pgfqpoint{2.674725in}{4.618636in}}%
\pgfpathlineto{\pgfqpoint{2.665188in}{4.669799in}}%
\pgfpathlineto{\pgfqpoint{2.631959in}{4.629559in}}%
\pgfpathlineto{\pgfqpoint{2.598748in}{4.588356in}}%
\pgfpathclose%
\pgfusepath{fill}%
\end{pgfscope}%
\begin{pgfscope}%
\pgfpathrectangle{\pgfqpoint{1.020000in}{0.880000in}}{\pgfqpoint{6.160000in}{6.160000in}}%
\pgfusepath{clip}%
\pgfsetbuttcap%
\pgfsetroundjoin%
\definecolor{currentfill}{rgb}{0.318832,0.426605,0.859857}%
\pgfsetfillcolor{currentfill}%
\pgfsetlinewidth{0.000000pt}%
\definecolor{currentstroke}{rgb}{0.000000,0.000000,0.000000}%
\pgfsetstrokecolor{currentstroke}%
\pgfsetdash{}{0pt}%
\pgfpathmoveto{\pgfqpoint{5.744912in}{2.966768in}}%
\pgfpathlineto{\pgfqpoint{5.756742in}{2.951231in}}%
\pgfpathlineto{\pgfqpoint{5.768594in}{2.935696in}}%
\pgfpathlineto{\pgfqpoint{5.801443in}{2.937378in}}%
\pgfpathlineto{\pgfqpoint{5.834273in}{2.939112in}}%
\pgfpathlineto{\pgfqpoint{5.822367in}{2.954586in}}%
\pgfpathlineto{\pgfqpoint{5.810485in}{2.970078in}}%
\pgfpathlineto{\pgfqpoint{5.777708in}{2.968401in}}%
\pgfpathlineto{\pgfqpoint{5.744912in}{2.966768in}}%
\pgfpathclose%
\pgfusepath{fill}%
\end{pgfscope}%
\begin{pgfscope}%
\pgfpathrectangle{\pgfqpoint{1.020000in}{0.880000in}}{\pgfqpoint{6.160000in}{6.160000in}}%
\pgfusepath{clip}%
\pgfsetbuttcap%
\pgfsetroundjoin%
\definecolor{currentfill}{rgb}{0.924409,0.498590,0.389059}%
\pgfsetfillcolor{currentfill}%
\pgfsetlinewidth{0.000000pt}%
\definecolor{currentstroke}{rgb}{0.000000,0.000000,0.000000}%
\pgfsetstrokecolor{currentstroke}%
\pgfsetdash{}{0pt}%
\pgfpathmoveto{\pgfqpoint{2.731717in}{4.745629in}}%
\pgfpathlineto{\pgfqpoint{2.741326in}{4.691927in}}%
\pgfpathlineto{\pgfqpoint{2.750979in}{4.635611in}}%
\pgfpathlineto{\pgfqpoint{2.784355in}{4.667922in}}%
\pgfpathlineto{\pgfqpoint{2.817760in}{4.697724in}}%
\pgfpathlineto{\pgfqpoint{2.808042in}{4.756422in}}%
\pgfpathlineto{\pgfqpoint{2.798368in}{4.812335in}}%
\pgfpathlineto{\pgfqpoint{2.765025in}{4.780338in}}%
\pgfpathlineto{\pgfqpoint{2.731717in}{4.745629in}}%
\pgfpathclose%
\pgfusepath{fill}%
\end{pgfscope}%
\begin{pgfscope}%
\pgfpathrectangle{\pgfqpoint{1.020000in}{0.880000in}}{\pgfqpoint{6.160000in}{6.160000in}}%
\pgfusepath{clip}%
\pgfsetbuttcap%
\pgfsetroundjoin%
\definecolor{currentfill}{rgb}{0.388852,0.516298,0.921373}%
\pgfsetfillcolor{currentfill}%
\pgfsetlinewidth{0.000000pt}%
\definecolor{currentstroke}{rgb}{0.000000,0.000000,0.000000}%
\pgfsetstrokecolor{currentstroke}%
\pgfsetdash{}{0pt}%
\pgfpathmoveto{\pgfqpoint{4.929239in}{3.100874in}}%
\pgfpathlineto{\pgfqpoint{4.940354in}{3.089605in}}%
\pgfpathlineto{\pgfqpoint{4.951486in}{3.077596in}}%
\pgfpathlineto{\pgfqpoint{4.984532in}{3.070648in}}%
\pgfpathlineto{\pgfqpoint{5.017559in}{3.064517in}}%
\pgfpathlineto{\pgfqpoint{5.006372in}{3.076312in}}%
\pgfpathlineto{\pgfqpoint{4.995199in}{3.087141in}}%
\pgfpathlineto{\pgfqpoint{4.962229in}{3.093643in}}%
\pgfpathlineto{\pgfqpoint{4.929239in}{3.100874in}}%
\pgfpathclose%
\pgfusepath{fill}%
\end{pgfscope}%
\begin{pgfscope}%
\pgfpathrectangle{\pgfqpoint{1.020000in}{0.880000in}}{\pgfqpoint{6.160000in}{6.160000in}}%
\pgfusepath{clip}%
\pgfsetbuttcap%
\pgfsetroundjoin%
\definecolor{currentfill}{rgb}{0.944055,0.553153,0.435548}%
\pgfsetfillcolor{currentfill}%
\pgfsetlinewidth{0.000000pt}%
\definecolor{currentstroke}{rgb}{0.000000,0.000000,0.000000}%
\pgfsetstrokecolor{currentstroke}%
\pgfsetdash{}{0pt}%
\pgfpathmoveto{\pgfqpoint{2.665188in}{4.669799in}}%
\pgfpathlineto{\pgfqpoint{2.674725in}{4.618636in}}%
\pgfpathlineto{\pgfqpoint{2.684304in}{4.565060in}}%
\pgfpathlineto{\pgfqpoint{2.717630in}{4.601187in}}%
\pgfpathlineto{\pgfqpoint{2.750979in}{4.635611in}}%
\pgfpathlineto{\pgfqpoint{2.741326in}{4.691927in}}%
\pgfpathlineto{\pgfqpoint{2.731717in}{4.745629in}}%
\pgfpathlineto{\pgfqpoint{2.698439in}{4.708636in}}%
\pgfpathlineto{\pgfqpoint{2.665188in}{4.669799in}}%
\pgfpathclose%
\pgfusepath{fill}%
\end{pgfscope}%
\begin{pgfscope}%
\pgfpathrectangle{\pgfqpoint{1.020000in}{0.880000in}}{\pgfqpoint{6.160000in}{6.160000in}}%
\pgfusepath{clip}%
\pgfsetbuttcap%
\pgfsetroundjoin%
\definecolor{currentfill}{rgb}{0.328604,0.439712,0.869587}%
\pgfsetfillcolor{currentfill}%
\pgfsetlinewidth{0.000000pt}%
\definecolor{currentstroke}{rgb}{0.000000,0.000000,0.000000}%
\pgfsetstrokecolor{currentstroke}%
\pgfsetdash{}{0pt}%
\pgfpathmoveto{\pgfqpoint{5.524438in}{2.989362in}}%
\pgfpathlineto{\pgfqpoint{5.536070in}{2.974085in}}%
\pgfpathlineto{\pgfqpoint{5.547722in}{2.958632in}}%
\pgfpathlineto{\pgfqpoint{5.580636in}{2.959703in}}%
\pgfpathlineto{\pgfqpoint{5.613530in}{2.960917in}}%
\pgfpathlineto{\pgfqpoint{5.601826in}{2.976414in}}%
\pgfpathlineto{\pgfqpoint{5.590142in}{2.991788in}}%
\pgfpathlineto{\pgfqpoint{5.557300in}{2.990508in}}%
\pgfpathlineto{\pgfqpoint{5.524438in}{2.989362in}}%
\pgfpathclose%
\pgfusepath{fill}%
\end{pgfscope}%
\begin{pgfscope}%
\pgfpathrectangle{\pgfqpoint{1.020000in}{0.880000in}}{\pgfqpoint{6.160000in}{6.160000in}}%
\pgfusepath{clip}%
\pgfsetbuttcap%
\pgfsetroundjoin%
\definecolor{currentfill}{rgb}{0.813693,0.854282,0.918480}%
\pgfsetfillcolor{currentfill}%
\pgfsetlinewidth{0.000000pt}%
\definecolor{currentstroke}{rgb}{0.000000,0.000000,0.000000}%
\pgfsetstrokecolor{currentstroke}%
\pgfsetdash{}{0pt}%
\pgfpathmoveto{\pgfqpoint{3.542994in}{3.919363in}}%
\pgfpathlineto{\pgfqpoint{3.552983in}{3.876720in}}%
\pgfpathlineto{\pgfqpoint{3.562967in}{3.837305in}}%
\pgfpathlineto{\pgfqpoint{3.596606in}{3.821584in}}%
\pgfpathlineto{\pgfqpoint{3.630219in}{3.804741in}}%
\pgfpathlineto{\pgfqpoint{3.620216in}{3.840977in}}%
\pgfpathlineto{\pgfqpoint{3.610210in}{3.880280in}}%
\pgfpathlineto{\pgfqpoint{3.576616in}{3.900389in}}%
\pgfpathlineto{\pgfqpoint{3.542994in}{3.919363in}}%
\pgfpathclose%
\pgfusepath{fill}%
\end{pgfscope}%
\begin{pgfscope}%
\pgfpathrectangle{\pgfqpoint{1.020000in}{0.880000in}}{\pgfqpoint{6.160000in}{6.160000in}}%
\pgfusepath{clip}%
\pgfsetbuttcap%
\pgfsetroundjoin%
\definecolor{currentfill}{rgb}{0.343278,0.459354,0.884122}%
\pgfsetfillcolor{currentfill}%
\pgfsetlinewidth{0.000000pt}%
\definecolor{currentstroke}{rgb}{0.000000,0.000000,0.000000}%
\pgfsetstrokecolor{currentstroke}%
\pgfsetdash{}{0pt}%
\pgfpathmoveto{\pgfqpoint{5.303970in}{3.015876in}}%
\pgfpathlineto{\pgfqpoint{5.315415in}{3.001860in}}%
\pgfpathlineto{\pgfqpoint{5.326876in}{2.987289in}}%
\pgfpathlineto{\pgfqpoint{5.359849in}{2.986855in}}%
\pgfpathlineto{\pgfqpoint{5.392803in}{2.986794in}}%
\pgfpathlineto{\pgfqpoint{5.381290in}{3.001487in}}%
\pgfpathlineto{\pgfqpoint{5.369794in}{3.015668in}}%
\pgfpathlineto{\pgfqpoint{5.336891in}{3.015572in}}%
\pgfpathlineto{\pgfqpoint{5.303970in}{3.015876in}}%
\pgfpathclose%
\pgfusepath{fill}%
\end{pgfscope}%
\begin{pgfscope}%
\pgfpathrectangle{\pgfqpoint{1.020000in}{0.880000in}}{\pgfqpoint{6.160000in}{6.160000in}}%
\pgfusepath{clip}%
\pgfsetbuttcap%
\pgfsetroundjoin%
\definecolor{currentfill}{rgb}{0.921406,0.491420,0.383408}%
\pgfsetfillcolor{currentfill}%
\pgfsetlinewidth{0.000000pt}%
\definecolor{currentstroke}{rgb}{0.000000,0.000000,0.000000}%
\pgfsetstrokecolor{currentstroke}%
\pgfsetdash{}{0pt}%
\pgfpathmoveto{\pgfqpoint{2.951713in}{4.784612in}}%
\pgfpathlineto{\pgfqpoint{2.961568in}{4.720182in}}%
\pgfpathlineto{\pgfqpoint{2.971453in}{4.653776in}}%
\pgfpathlineto{\pgfqpoint{3.005061in}{4.665013in}}%
\pgfpathlineto{\pgfqpoint{3.038696in}{4.672325in}}%
\pgfpathlineto{\pgfqpoint{3.028779in}{4.739437in}}%
\pgfpathlineto{\pgfqpoint{3.018891in}{4.804558in}}%
\pgfpathlineto{\pgfqpoint{2.985286in}{4.796735in}}%
\pgfpathlineto{\pgfqpoint{2.951713in}{4.784612in}}%
\pgfpathclose%
\pgfusepath{fill}%
\end{pgfscope}%
\begin{pgfscope}%
\pgfpathrectangle{\pgfqpoint{1.020000in}{0.880000in}}{\pgfqpoint{6.160000in}{6.160000in}}%
\pgfusepath{clip}%
\pgfsetbuttcap%
\pgfsetroundjoin%
\definecolor{currentfill}{rgb}{0.651398,0.768121,0.995891}%
\pgfsetfillcolor{currentfill}%
\pgfsetlinewidth{0.000000pt}%
\definecolor{currentstroke}{rgb}{0.000000,0.000000,0.000000}%
\pgfsetstrokecolor{currentstroke}%
\pgfsetdash{}{0pt}%
\pgfpathmoveto{\pgfqpoint{3.938617in}{3.572436in}}%
\pgfpathlineto{\pgfqpoint{3.948798in}{3.559551in}}%
\pgfpathlineto{\pgfqpoint{3.958992in}{3.549158in}}%
\pgfpathlineto{\pgfqpoint{3.992457in}{3.532868in}}%
\pgfpathlineto{\pgfqpoint{4.025889in}{3.516423in}}%
\pgfpathlineto{\pgfqpoint{4.015642in}{3.526429in}}%
\pgfpathlineto{\pgfqpoint{4.005410in}{3.538643in}}%
\pgfpathlineto{\pgfqpoint{3.972030in}{3.555530in}}%
\pgfpathlineto{\pgfqpoint{3.938617in}{3.572436in}}%
\pgfpathclose%
\pgfusepath{fill}%
\end{pgfscope}%
\begin{pgfscope}%
\pgfpathrectangle{\pgfqpoint{1.020000in}{0.880000in}}{\pgfqpoint{6.160000in}{6.160000in}}%
\pgfusepath{clip}%
\pgfsetbuttcap%
\pgfsetroundjoin%
\definecolor{currentfill}{rgb}{0.603162,0.731527,0.999565}%
\pgfsetfillcolor{currentfill}%
\pgfsetlinewidth{0.000000pt}%
\definecolor{currentstroke}{rgb}{0.000000,0.000000,0.000000}%
\pgfsetstrokecolor{currentstroke}%
\pgfsetdash{}{0pt}%
\pgfpathmoveto{\pgfqpoint{4.092659in}{3.483535in}}%
\pgfpathlineto{\pgfqpoint{4.102979in}{3.475485in}}%
\pgfpathlineto{\pgfqpoint{4.113318in}{3.469271in}}%
\pgfpathlineto{\pgfqpoint{4.146710in}{3.452501in}}%
\pgfpathlineto{\pgfqpoint{4.180070in}{3.435892in}}%
\pgfpathlineto{\pgfqpoint{4.169675in}{3.442829in}}%
\pgfpathlineto{\pgfqpoint{4.159299in}{3.451348in}}%
\pgfpathlineto{\pgfqpoint{4.125995in}{3.467307in}}%
\pgfpathlineto{\pgfqpoint{4.092659in}{3.483535in}}%
\pgfpathclose%
\pgfusepath{fill}%
\end{pgfscope}%
\begin{pgfscope}%
\pgfpathrectangle{\pgfqpoint{1.020000in}{0.880000in}}{\pgfqpoint{6.160000in}{6.160000in}}%
\pgfusepath{clip}%
\pgfsetbuttcap%
\pgfsetroundjoin%
\definecolor{currentfill}{rgb}{0.559747,0.694768,0.996075}%
\pgfsetfillcolor{currentfill}%
\pgfsetlinewidth{0.000000pt}%
\definecolor{currentstroke}{rgb}{0.000000,0.000000,0.000000}%
\pgfsetstrokecolor{currentstroke}%
\pgfsetdash{}{0pt}%
\pgfpathmoveto{\pgfqpoint{4.246694in}{3.403533in}}%
\pgfpathlineto{\pgfqpoint{4.257167in}{3.396991in}}%
\pgfpathlineto{\pgfqpoint{4.267661in}{3.391654in}}%
\pgfpathlineto{\pgfqpoint{4.300980in}{3.374781in}}%
\pgfpathlineto{\pgfqpoint{4.334268in}{3.358367in}}%
\pgfpathlineto{\pgfqpoint{4.323719in}{3.365076in}}%
\pgfpathlineto{\pgfqpoint{4.313192in}{3.372786in}}%
\pgfpathlineto{\pgfqpoint{4.279959in}{3.387932in}}%
\pgfpathlineto{\pgfqpoint{4.246694in}{3.403533in}}%
\pgfpathclose%
\pgfusepath{fill}%
\end{pgfscope}%
\begin{pgfscope}%
\pgfpathrectangle{\pgfqpoint{1.020000in}{0.880000in}}{\pgfqpoint{6.160000in}{6.160000in}}%
\pgfusepath{clip}%
\pgfsetbuttcap%
\pgfsetroundjoin%
\definecolor{currentfill}{rgb}{0.703587,0.802586,0.982847}%
\pgfsetfillcolor{currentfill}%
\pgfsetlinewidth{0.000000pt}%
\definecolor{currentstroke}{rgb}{0.000000,0.000000,0.000000}%
\pgfsetstrokecolor{currentstroke}%
\pgfsetdash{}{0pt}%
\pgfpathmoveto{\pgfqpoint{3.784504in}{3.677587in}}%
\pgfpathlineto{\pgfqpoint{3.794573in}{3.656166in}}%
\pgfpathlineto{\pgfqpoint{3.804650in}{3.637794in}}%
\pgfpathlineto{\pgfqpoint{3.838188in}{3.622050in}}%
\pgfpathlineto{\pgfqpoint{3.871696in}{3.605822in}}%
\pgfpathlineto{\pgfqpoint{3.861575in}{3.622526in}}%
\pgfpathlineto{\pgfqpoint{3.851464in}{3.642008in}}%
\pgfpathlineto{\pgfqpoint{3.818000in}{3.659955in}}%
\pgfpathlineto{\pgfqpoint{3.784504in}{3.677587in}}%
\pgfpathclose%
\pgfusepath{fill}%
\end{pgfscope}%
\begin{pgfscope}%
\pgfpathrectangle{\pgfqpoint{1.020000in}{0.880000in}}{\pgfqpoint{6.160000in}{6.160000in}}%
\pgfusepath{clip}%
\pgfsetbuttcap%
\pgfsetroundjoin%
\definecolor{currentfill}{rgb}{0.516260,0.654498,0.986407}%
\pgfsetfillcolor{currentfill}%
\pgfsetlinewidth{0.000000pt}%
\definecolor{currentstroke}{rgb}{0.000000,0.000000,0.000000}%
\pgfsetstrokecolor{currentstroke}%
\pgfsetdash{}{0pt}%
\pgfpathmoveto{\pgfqpoint{4.400751in}{3.327115in}}%
\pgfpathlineto{\pgfqpoint{4.411376in}{3.319852in}}%
\pgfpathlineto{\pgfqpoint{4.422023in}{3.313252in}}%
\pgfpathlineto{\pgfqpoint{4.455271in}{3.297040in}}%
\pgfpathlineto{\pgfqpoint{4.488490in}{3.281511in}}%
\pgfpathlineto{\pgfqpoint{4.477790in}{3.289544in}}%
\pgfpathlineto{\pgfqpoint{4.467113in}{3.298078in}}%
\pgfpathlineto{\pgfqpoint{4.433947in}{3.312322in}}%
\pgfpathlineto{\pgfqpoint{4.400751in}{3.327115in}}%
\pgfpathclose%
\pgfusepath{fill}%
\end{pgfscope}%
\begin{pgfscope}%
\pgfpathrectangle{\pgfqpoint{1.020000in}{0.880000in}}{\pgfqpoint{6.160000in}{6.160000in}}%
\pgfusepath{clip}%
\pgfsetbuttcap%
\pgfsetroundjoin%
\definecolor{currentfill}{rgb}{0.363461,0.484784,0.901019}%
\pgfsetfillcolor{currentfill}%
\pgfsetlinewidth{0.000000pt}%
\definecolor{currentstroke}{rgb}{0.000000,0.000000,0.000000}%
\pgfsetstrokecolor{currentstroke}%
\pgfsetdash{}{0pt}%
\pgfpathmoveto{\pgfqpoint{5.083555in}{3.054585in}}%
\pgfpathlineto{\pgfqpoint{5.094812in}{3.042135in}}%
\pgfpathlineto{\pgfqpoint{5.106083in}{3.028896in}}%
\pgfpathlineto{\pgfqpoint{5.139108in}{3.025152in}}%
\pgfpathlineto{\pgfqpoint{5.172115in}{3.022102in}}%
\pgfpathlineto{\pgfqpoint{5.160790in}{3.035290in}}%
\pgfpathlineto{\pgfqpoint{5.149478in}{3.047588in}}%
\pgfpathlineto{\pgfqpoint{5.116525in}{3.050732in}}%
\pgfpathlineto{\pgfqpoint{5.083555in}{3.054585in}}%
\pgfpathclose%
\pgfusepath{fill}%
\end{pgfscope}%
\begin{pgfscope}%
\pgfpathrectangle{\pgfqpoint{1.020000in}{0.880000in}}{\pgfqpoint{6.160000in}{6.160000in}}%
\pgfusepath{clip}%
\pgfsetbuttcap%
\pgfsetroundjoin%
\definecolor{currentfill}{rgb}{0.473070,0.611077,0.970634}%
\pgfsetfillcolor{currentfill}%
\pgfsetlinewidth{0.000000pt}%
\definecolor{currentstroke}{rgb}{0.000000,0.000000,0.000000}%
\pgfsetstrokecolor{currentstroke}%
\pgfsetdash{}{0pt}%
\pgfpathmoveto{\pgfqpoint{4.554838in}{3.252479in}}%
\pgfpathlineto{\pgfqpoint{4.565611in}{3.243612in}}%
\pgfpathlineto{\pgfqpoint{4.576405in}{3.234953in}}%
\pgfpathlineto{\pgfqpoint{4.609588in}{3.220364in}}%
\pgfpathlineto{\pgfqpoint{4.642743in}{3.206586in}}%
\pgfpathlineto{\pgfqpoint{4.631898in}{3.216241in}}%
\pgfpathlineto{\pgfqpoint{4.621074in}{3.225934in}}%
\pgfpathlineto{\pgfqpoint{4.587970in}{3.238919in}}%
\pgfpathlineto{\pgfqpoint{4.554838in}{3.252479in}}%
\pgfpathclose%
\pgfusepath{fill}%
\end{pgfscope}%
\begin{pgfscope}%
\pgfpathrectangle{\pgfqpoint{1.020000in}{0.880000in}}{\pgfqpoint{6.160000in}{6.160000in}}%
\pgfusepath{clip}%
\pgfsetbuttcap%
\pgfsetroundjoin%
\definecolor{currentfill}{rgb}{0.951254,0.578799,0.459408}%
\pgfsetfillcolor{currentfill}%
\pgfsetlinewidth{0.000000pt}%
\definecolor{currentstroke}{rgb}{0.000000,0.000000,0.000000}%
\pgfsetstrokecolor{currentstroke}%
\pgfsetdash{}{0pt}%
\pgfpathmoveto{\pgfqpoint{3.038696in}{4.672325in}}%
\pgfpathlineto{\pgfqpoint{3.048635in}{4.603777in}}%
\pgfpathlineto{\pgfqpoint{3.058592in}{4.534345in}}%
\pgfpathlineto{\pgfqpoint{3.092274in}{4.537517in}}%
\pgfpathlineto{\pgfqpoint{3.125972in}{4.536964in}}%
\pgfpathlineto{\pgfqpoint{3.115995in}{4.606251in}}%
\pgfpathlineto{\pgfqpoint{3.106034in}{4.674693in}}%
\pgfpathlineto{\pgfqpoint{3.072354in}{4.675579in}}%
\pgfpathlineto{\pgfqpoint{3.038696in}{4.672325in}}%
\pgfpathclose%
\pgfusepath{fill}%
\end{pgfscope}%
\begin{pgfscope}%
\pgfpathrectangle{\pgfqpoint{1.020000in}{0.880000in}}{\pgfqpoint{6.160000in}{6.160000in}}%
\pgfusepath{clip}%
\pgfsetbuttcap%
\pgfsetroundjoin%
\definecolor{currentfill}{rgb}{0.309060,0.413498,0.850128}%
\pgfsetfillcolor{currentfill}%
\pgfsetlinewidth{0.000000pt}%
\definecolor{currentstroke}{rgb}{0.000000,0.000000,0.000000}%
\pgfsetstrokecolor{currentstroke}%
\pgfsetdash{}{0pt}%
\pgfpathmoveto{\pgfqpoint{5.899871in}{2.942707in}}%
\pgfpathlineto{\pgfqpoint{5.911853in}{2.927337in}}%
\pgfpathlineto{\pgfqpoint{5.923857in}{2.912003in}}%
\pgfpathlineto{\pgfqpoint{5.956680in}{2.913936in}}%
\pgfpathlineto{\pgfqpoint{5.989483in}{2.915897in}}%
\pgfpathlineto{\pgfqpoint{5.977425in}{2.931146in}}%
\pgfpathlineto{\pgfqpoint{5.965390in}{2.946433in}}%
\pgfpathlineto{\pgfqpoint{5.932641in}{2.944556in}}%
\pgfpathlineto{\pgfqpoint{5.899871in}{2.942707in}}%
\pgfpathclose%
\pgfusepath{fill}%
\end{pgfscope}%
\begin{pgfscope}%
\pgfpathrectangle{\pgfqpoint{1.020000in}{0.880000in}}{\pgfqpoint{6.160000in}{6.160000in}}%
\pgfusepath{clip}%
\pgfsetbuttcap%
\pgfsetroundjoin%
\definecolor{currentfill}{rgb}{0.430507,0.564883,0.948889}%
\pgfsetfillcolor{currentfill}%
\pgfsetlinewidth{0.000000pt}%
\definecolor{currentstroke}{rgb}{0.000000,0.000000,0.000000}%
\pgfsetstrokecolor{currentstroke}%
\pgfsetdash{}{0pt}%
\pgfpathmoveto{\pgfqpoint{4.708972in}{3.181308in}}%
\pgfpathlineto{\pgfqpoint{4.719889in}{3.170967in}}%
\pgfpathlineto{\pgfqpoint{4.730826in}{3.160428in}}%
\pgfpathlineto{\pgfqpoint{4.763954in}{3.148428in}}%
\pgfpathlineto{\pgfqpoint{4.797057in}{3.137309in}}%
\pgfpathlineto{\pgfqpoint{4.786068in}{3.148237in}}%
\pgfpathlineto{\pgfqpoint{4.775099in}{3.158757in}}%
\pgfpathlineto{\pgfqpoint{4.742048in}{3.169715in}}%
\pgfpathlineto{\pgfqpoint{4.708972in}{3.181308in}}%
\pgfpathclose%
\pgfusepath{fill}%
\end{pgfscope}%
\begin{pgfscope}%
\pgfpathrectangle{\pgfqpoint{1.020000in}{0.880000in}}{\pgfqpoint{6.160000in}{6.160000in}}%
\pgfusepath{clip}%
\pgfsetbuttcap%
\pgfsetroundjoin%
\definecolor{currentfill}{rgb}{0.929357,0.512254,0.400673}%
\pgfsetfillcolor{currentfill}%
\pgfsetlinewidth{0.000000pt}%
\definecolor{currentstroke}{rgb}{0.000000,0.000000,0.000000}%
\pgfsetstrokecolor{currentstroke}%
\pgfsetdash{}{0pt}%
\pgfpathmoveto{\pgfqpoint{2.884668in}{4.748296in}}%
\pgfpathlineto{\pgfqpoint{2.894479in}{4.685259in}}%
\pgfpathlineto{\pgfqpoint{2.904321in}{4.620293in}}%
\pgfpathlineto{\pgfqpoint{2.937872in}{4.638794in}}%
\pgfpathlineto{\pgfqpoint{2.971453in}{4.653776in}}%
\pgfpathlineto{\pgfqpoint{2.961568in}{4.720182in}}%
\pgfpathlineto{\pgfqpoint{2.951713in}{4.784612in}}%
\pgfpathlineto{\pgfqpoint{2.918174in}{4.768384in}}%
\pgfpathlineto{\pgfqpoint{2.884668in}{4.748296in}}%
\pgfpathclose%
\pgfusepath{fill}%
\end{pgfscope}%
\begin{pgfscope}%
\pgfpathrectangle{\pgfqpoint{1.020000in}{0.880000in}}{\pgfqpoint{6.160000in}{6.160000in}}%
\pgfusepath{clip}%
\pgfsetbuttcap%
\pgfsetroundjoin%
\definecolor{currentfill}{rgb}{0.968105,0.668475,0.550486}%
\pgfsetfillcolor{currentfill}%
\pgfsetlinewidth{0.000000pt}%
\definecolor{currentstroke}{rgb}{0.000000,0.000000,0.000000}%
\pgfsetstrokecolor{currentstroke}%
\pgfsetdash{}{0pt}%
\pgfpathmoveto{\pgfqpoint{3.125972in}{4.536964in}}%
\pgfpathlineto{\pgfqpoint{3.135958in}{4.467383in}}%
\pgfpathlineto{\pgfqpoint{3.145950in}{4.398048in}}%
\pgfpathlineto{\pgfqpoint{3.179674in}{4.394590in}}%
\pgfpathlineto{\pgfqpoint{3.213403in}{4.387807in}}%
\pgfpathlineto{\pgfqpoint{3.203401in}{4.456057in}}%
\pgfpathlineto{\pgfqpoint{3.193403in}{4.524607in}}%
\pgfpathlineto{\pgfqpoint{3.159683in}{4.532655in}}%
\pgfpathlineto{\pgfqpoint{3.125972in}{4.536964in}}%
\pgfpathclose%
\pgfusepath{fill}%
\end{pgfscope}%
\begin{pgfscope}%
\pgfpathrectangle{\pgfqpoint{1.020000in}{0.880000in}}{\pgfqpoint{6.160000in}{6.160000in}}%
\pgfusepath{clip}%
\pgfsetbuttcap%
\pgfsetroundjoin%
\definecolor{currentfill}{rgb}{0.768034,0.837035,0.952488}%
\pgfsetfillcolor{currentfill}%
\pgfsetlinewidth{0.000000pt}%
\definecolor{currentstroke}{rgb}{0.000000,0.000000,0.000000}%
\pgfsetstrokecolor{currentstroke}%
\pgfsetdash{}{0pt}%
\pgfpathmoveto{\pgfqpoint{3.630219in}{3.804741in}}%
\pgfpathlineto{\pgfqpoint{3.640221in}{3.771745in}}%
\pgfpathlineto{\pgfqpoint{3.650220in}{3.742121in}}%
\pgfpathlineto{\pgfqpoint{3.683833in}{3.727171in}}%
\pgfpathlineto{\pgfqpoint{3.717419in}{3.711337in}}%
\pgfpathlineto{\pgfqpoint{3.707389in}{3.738251in}}%
\pgfpathlineto{\pgfqpoint{3.697361in}{3.768324in}}%
\pgfpathlineto{\pgfqpoint{3.663805in}{3.786935in}}%
\pgfpathlineto{\pgfqpoint{3.630219in}{3.804741in}}%
\pgfpathclose%
\pgfusepath{fill}%
\end{pgfscope}%
\begin{pgfscope}%
\pgfpathrectangle{\pgfqpoint{1.020000in}{0.880000in}}{\pgfqpoint{6.160000in}{6.160000in}}%
\pgfusepath{clip}%
\pgfsetbuttcap%
\pgfsetroundjoin%
\definecolor{currentfill}{rgb}{0.318832,0.426605,0.859857}%
\pgfsetfillcolor{currentfill}%
\pgfsetlinewidth{0.000000pt}%
\definecolor{currentstroke}{rgb}{0.000000,0.000000,0.000000}%
\pgfsetstrokecolor{currentstroke}%
\pgfsetdash{}{0pt}%
\pgfpathmoveto{\pgfqpoint{5.679260in}{2.963680in}}%
\pgfpathlineto{\pgfqpoint{5.691038in}{2.948114in}}%
\pgfpathlineto{\pgfqpoint{5.702837in}{2.932531in}}%
\pgfpathlineto{\pgfqpoint{5.735725in}{2.934076in}}%
\pgfpathlineto{\pgfqpoint{5.768594in}{2.935696in}}%
\pgfpathlineto{\pgfqpoint{5.756742in}{2.951231in}}%
\pgfpathlineto{\pgfqpoint{5.744912in}{2.966768in}}%
\pgfpathlineto{\pgfqpoint{5.712096in}{2.965191in}}%
\pgfpathlineto{\pgfqpoint{5.679260in}{2.963680in}}%
\pgfpathclose%
\pgfusepath{fill}%
\end{pgfscope}%
\begin{pgfscope}%
\pgfpathrectangle{\pgfqpoint{1.020000in}{0.880000in}}{\pgfqpoint{6.160000in}{6.160000in}}%
\pgfusepath{clip}%
\pgfsetbuttcap%
\pgfsetroundjoin%
\definecolor{currentfill}{rgb}{0.963772,0.749086,0.649420}%
\pgfsetfillcolor{currentfill}%
\pgfsetlinewidth{0.000000pt}%
\definecolor{currentstroke}{rgb}{0.000000,0.000000,0.000000}%
\pgfsetstrokecolor{currentstroke}%
\pgfsetdash{}{0pt}%
\pgfpathmoveto{\pgfqpoint{3.213403in}{4.387807in}}%
\pgfpathlineto{\pgfqpoint{3.223406in}{4.320373in}}%
\pgfpathlineto{\pgfqpoint{3.233406in}{4.254244in}}%
\pgfpathlineto{\pgfqpoint{3.267143in}{4.245948in}}%
\pgfpathlineto{\pgfqpoint{3.300876in}{4.234840in}}%
\pgfpathlineto{\pgfqpoint{3.290871in}{4.299004in}}%
\pgfpathlineto{\pgfqpoint{3.280864in}{4.364512in}}%
\pgfpathlineto{\pgfqpoint{3.247134in}{4.377750in}}%
\pgfpathlineto{\pgfqpoint{3.213403in}{4.387807in}}%
\pgfpathclose%
\pgfusepath{fill}%
\end{pgfscope}%
\begin{pgfscope}%
\pgfpathrectangle{\pgfqpoint{1.020000in}{0.880000in}}{\pgfqpoint{6.160000in}{6.160000in}}%
\pgfusepath{clip}%
\pgfsetbuttcap%
\pgfsetroundjoin%
\definecolor{currentfill}{rgb}{0.935774,0.812237,0.747156}%
\pgfsetfillcolor{currentfill}%
\pgfsetlinewidth{0.000000pt}%
\definecolor{currentstroke}{rgb}{0.000000,0.000000,0.000000}%
\pgfsetstrokecolor{currentstroke}%
\pgfsetdash{}{0pt}%
\pgfpathmoveto{\pgfqpoint{3.300876in}{4.234840in}}%
\pgfpathlineto{\pgfqpoint{3.310875in}{4.172471in}}%
\pgfpathlineto{\pgfqpoint{3.320865in}{4.112313in}}%
\pgfpathlineto{\pgfqpoint{3.354595in}{4.101007in}}%
\pgfpathlineto{\pgfqpoint{3.388312in}{4.087426in}}%
\pgfpathlineto{\pgfqpoint{3.378317in}{4.144934in}}%
\pgfpathlineto{\pgfqpoint{3.368316in}{4.204649in}}%
\pgfpathlineto{\pgfqpoint{3.334602in}{4.221028in}}%
\pgfpathlineto{\pgfqpoint{3.300876in}{4.234840in}}%
\pgfpathclose%
\pgfusepath{fill}%
\end{pgfscope}%
\begin{pgfscope}%
\pgfpathrectangle{\pgfqpoint{1.020000in}{0.880000in}}{\pgfqpoint{6.160000in}{6.160000in}}%
\pgfusepath{clip}%
\pgfsetbuttcap%
\pgfsetroundjoin%
\definecolor{currentfill}{rgb}{0.333490,0.446265,0.874452}%
\pgfsetfillcolor{currentfill}%
\pgfsetlinewidth{0.000000pt}%
\definecolor{currentstroke}{rgb}{0.000000,0.000000,0.000000}%
\pgfsetstrokecolor{currentstroke}%
\pgfsetdash{}{0pt}%
\pgfpathmoveto{\pgfqpoint{5.458657in}{2.987601in}}%
\pgfpathlineto{\pgfqpoint{5.470239in}{2.972429in}}%
\pgfpathlineto{\pgfqpoint{5.481839in}{2.957031in}}%
\pgfpathlineto{\pgfqpoint{5.514790in}{2.957731in}}%
\pgfpathlineto{\pgfqpoint{5.547722in}{2.958632in}}%
\pgfpathlineto{\pgfqpoint{5.536070in}{2.974085in}}%
\pgfpathlineto{\pgfqpoint{5.524438in}{2.989362in}}%
\pgfpathlineto{\pgfqpoint{5.491557in}{2.988382in}}%
\pgfpathlineto{\pgfqpoint{5.458657in}{2.987601in}}%
\pgfpathclose%
\pgfusepath{fill}%
\end{pgfscope}%
\begin{pgfscope}%
\pgfpathrectangle{\pgfqpoint{1.020000in}{0.880000in}}{\pgfqpoint{6.160000in}{6.160000in}}%
\pgfusepath{clip}%
\pgfsetbuttcap%
\pgfsetroundjoin%
\definecolor{currentfill}{rgb}{0.399231,0.528528,0.928459}%
\pgfsetfillcolor{currentfill}%
\pgfsetlinewidth{0.000000pt}%
\definecolor{currentstroke}{rgb}{0.000000,0.000000,0.000000}%
\pgfsetstrokecolor{currentstroke}%
\pgfsetdash{}{0pt}%
\pgfpathmoveto{\pgfqpoint{4.863193in}{3.117549in}}%
\pgfpathlineto{\pgfqpoint{4.874254in}{3.106084in}}%
\pgfpathlineto{\pgfqpoint{4.885332in}{3.094091in}}%
\pgfpathlineto{\pgfqpoint{4.918419in}{3.085397in}}%
\pgfpathlineto{\pgfqpoint{4.951486in}{3.077596in}}%
\pgfpathlineto{\pgfqpoint{4.940354in}{3.089605in}}%
\pgfpathlineto{\pgfqpoint{4.929239in}{3.100874in}}%
\pgfpathlineto{\pgfqpoint{4.896227in}{3.108839in}}%
\pgfpathlineto{\pgfqpoint{4.863193in}{3.117549in}}%
\pgfpathclose%
\pgfusepath{fill}%
\end{pgfscope}%
\begin{pgfscope}%
\pgfpathrectangle{\pgfqpoint{1.020000in}{0.880000in}}{\pgfqpoint{6.160000in}{6.160000in}}%
\pgfusepath{clip}%
\pgfsetbuttcap%
\pgfsetroundjoin%
\definecolor{currentfill}{rgb}{0.887752,0.854040,0.834671}%
\pgfsetfillcolor{currentfill}%
\pgfsetlinewidth{0.000000pt}%
\definecolor{currentstroke}{rgb}{0.000000,0.000000,0.000000}%
\pgfsetstrokecolor{currentstroke}%
\pgfsetdash{}{0pt}%
\pgfpathmoveto{\pgfqpoint{3.388312in}{4.087426in}}%
\pgfpathlineto{\pgfqpoint{3.398298in}{4.032493in}}%
\pgfpathlineto{\pgfqpoint{3.408274in}{3.980459in}}%
\pgfpathlineto{\pgfqpoint{3.441983in}{3.967722in}}%
\pgfpathlineto{\pgfqpoint{3.475675in}{3.953195in}}%
\pgfpathlineto{\pgfqpoint{3.465691in}{4.002181in}}%
\pgfpathlineto{\pgfqpoint{3.455698in}{4.054009in}}%
\pgfpathlineto{\pgfqpoint{3.422014in}{4.071707in}}%
\pgfpathlineto{\pgfqpoint{3.388312in}{4.087426in}}%
\pgfpathclose%
\pgfusepath{fill}%
\end{pgfscope}%
\begin{pgfscope}%
\pgfpathrectangle{\pgfqpoint{1.020000in}{0.880000in}}{\pgfqpoint{6.160000in}{6.160000in}}%
\pgfusepath{clip}%
\pgfsetbuttcap%
\pgfsetroundjoin%
\definecolor{currentfill}{rgb}{0.348323,0.465711,0.888346}%
\pgfsetfillcolor{currentfill}%
\pgfsetlinewidth{0.000000pt}%
\definecolor{currentstroke}{rgb}{0.000000,0.000000,0.000000}%
\pgfsetstrokecolor{currentstroke}%
\pgfsetdash{}{0pt}%
\pgfpathmoveto{\pgfqpoint{5.238077in}{3.017889in}}%
\pgfpathlineto{\pgfqpoint{5.249470in}{3.003965in}}%
\pgfpathlineto{\pgfqpoint{5.260879in}{2.989475in}}%
\pgfpathlineto{\pgfqpoint{5.293887in}{2.988144in}}%
\pgfpathlineto{\pgfqpoint{5.326876in}{2.987289in}}%
\pgfpathlineto{\pgfqpoint{5.315415in}{3.001860in}}%
\pgfpathlineto{\pgfqpoint{5.303970in}{3.015876in}}%
\pgfpathlineto{\pgfqpoint{5.271032in}{3.016632in}}%
\pgfpathlineto{\pgfqpoint{5.238077in}{3.017889in}}%
\pgfpathclose%
\pgfusepath{fill}%
\end{pgfscope}%
\begin{pgfscope}%
\pgfpathrectangle{\pgfqpoint{1.020000in}{0.880000in}}{\pgfqpoint{6.160000in}{6.160000in}}%
\pgfusepath{clip}%
\pgfsetbuttcap%
\pgfsetroundjoin%
\definecolor{currentfill}{rgb}{0.939254,0.539581,0.423900}%
\pgfsetfillcolor{currentfill}%
\pgfsetlinewidth{0.000000pt}%
\definecolor{currentstroke}{rgb}{0.000000,0.000000,0.000000}%
\pgfsetstrokecolor{currentstroke}%
\pgfsetdash{}{0pt}%
\pgfpathmoveto{\pgfqpoint{2.817760in}{4.697724in}}%
\pgfpathlineto{\pgfqpoint{2.827516in}{4.636699in}}%
\pgfpathlineto{\pgfqpoint{2.837304in}{4.573820in}}%
\pgfpathlineto{\pgfqpoint{2.870798in}{4.598535in}}%
\pgfpathlineto{\pgfqpoint{2.904321in}{4.620293in}}%
\pgfpathlineto{\pgfqpoint{2.894479in}{4.685259in}}%
\pgfpathlineto{\pgfqpoint{2.884668in}{4.748296in}}%
\pgfpathlineto{\pgfqpoint{2.851197in}{4.724634in}}%
\pgfpathlineto{\pgfqpoint{2.817760in}{4.697724in}}%
\pgfpathclose%
\pgfusepath{fill}%
\end{pgfscope}%
\begin{pgfscope}%
\pgfpathrectangle{\pgfqpoint{1.020000in}{0.880000in}}{\pgfqpoint{6.160000in}{6.160000in}}%
\pgfusepath{clip}%
\pgfsetbuttcap%
\pgfsetroundjoin%
\definecolor{currentfill}{rgb}{0.951254,0.578799,0.459408}%
\pgfsetfillcolor{currentfill}%
\pgfsetlinewidth{0.000000pt}%
\definecolor{currentstroke}{rgb}{0.000000,0.000000,0.000000}%
\pgfsetstrokecolor{currentstroke}%
\pgfsetdash{}{0pt}%
\pgfpathmoveto{\pgfqpoint{2.750979in}{4.635611in}}%
\pgfpathlineto{\pgfqpoint{2.760671in}{4.577106in}}%
\pgfpathlineto{\pgfqpoint{2.770394in}{4.516851in}}%
\pgfpathlineto{\pgfqpoint{2.803837in}{4.546476in}}%
\pgfpathlineto{\pgfqpoint{2.837304in}{4.573820in}}%
\pgfpathlineto{\pgfqpoint{2.827516in}{4.636699in}}%
\pgfpathlineto{\pgfqpoint{2.817760in}{4.697724in}}%
\pgfpathlineto{\pgfqpoint{2.784355in}{4.667922in}}%
\pgfpathlineto{\pgfqpoint{2.750979in}{4.635611in}}%
\pgfpathclose%
\pgfusepath{fill}%
\end{pgfscope}%
\begin{pgfscope}%
\pgfpathrectangle{\pgfqpoint{1.020000in}{0.880000in}}{\pgfqpoint{6.160000in}{6.160000in}}%
\pgfusepath{clip}%
\pgfsetbuttcap%
\pgfsetroundjoin%
\definecolor{currentfill}{rgb}{0.956371,0.775144,0.686416}%
\pgfsetfillcolor{currentfill}%
\pgfsetlinewidth{0.000000pt}%
\definecolor{currentstroke}{rgb}{0.000000,0.000000,0.000000}%
\pgfsetstrokecolor{currentstroke}%
\pgfsetdash{}{0pt}%
\pgfpathmoveto{\pgfqpoint{2.417948in}{4.261784in}}%
\pgfpathlineto{\pgfqpoint{2.427213in}{4.218955in}}%
\pgfpathlineto{\pgfqpoint{2.436505in}{4.175072in}}%
\pgfpathlineto{\pgfqpoint{2.469919in}{4.207918in}}%
\pgfpathlineto{\pgfqpoint{2.503314in}{4.241843in}}%
\pgfpathlineto{\pgfqpoint{2.493927in}{4.288994in}}%
\pgfpathlineto{\pgfqpoint{2.484569in}{4.334946in}}%
\pgfpathlineto{\pgfqpoint{2.451268in}{4.297782in}}%
\pgfpathlineto{\pgfqpoint{2.417948in}{4.261784in}}%
\pgfpathclose%
\pgfusepath{fill}%
\end{pgfscope}%
\begin{pgfscope}%
\pgfpathrectangle{\pgfqpoint{1.020000in}{0.880000in}}{\pgfqpoint{6.160000in}{6.160000in}}%
\pgfusepath{clip}%
\pgfsetbuttcap%
\pgfsetroundjoin%
\definecolor{currentfill}{rgb}{0.965899,0.740142,0.637058}%
\pgfsetfillcolor{currentfill}%
\pgfsetlinewidth{0.000000pt}%
\definecolor{currentstroke}{rgb}{0.000000,0.000000,0.000000}%
\pgfsetstrokecolor{currentstroke}%
\pgfsetdash{}{0pt}%
\pgfpathmoveto{\pgfqpoint{2.484569in}{4.334946in}}%
\pgfpathlineto{\pgfqpoint{2.493927in}{4.288994in}}%
\pgfpathlineto{\pgfqpoint{2.503314in}{4.241843in}}%
\pgfpathlineto{\pgfqpoint{2.536695in}{4.276598in}}%
\pgfpathlineto{\pgfqpoint{2.570068in}{4.311906in}}%
\pgfpathlineto{\pgfqpoint{2.560589in}{4.362450in}}%
\pgfpathlineto{\pgfqpoint{2.551140in}{4.411647in}}%
\pgfpathlineto{\pgfqpoint{2.517858in}{4.373003in}}%
\pgfpathlineto{\pgfqpoint{2.484569in}{4.334946in}}%
\pgfpathclose%
\pgfusepath{fill}%
\end{pgfscope}%
\begin{pgfscope}%
\pgfpathrectangle{\pgfqpoint{1.020000in}{0.880000in}}{\pgfqpoint{6.160000in}{6.160000in}}%
\pgfusepath{clip}%
\pgfsetbuttcap%
\pgfsetroundjoin%
\definecolor{currentfill}{rgb}{0.953054,0.585211,0.465373}%
\pgfsetfillcolor{currentfill}%
\pgfsetlinewidth{0.000000pt}%
\definecolor{currentstroke}{rgb}{0.000000,0.000000,0.000000}%
\pgfsetstrokecolor{currentstroke}%
\pgfsetdash{}{0pt}%
\pgfpathmoveto{\pgfqpoint{2.971453in}{4.653776in}}%
\pgfpathlineto{\pgfqpoint{2.981361in}{4.585930in}}%
\pgfpathlineto{\pgfqpoint{2.991288in}{4.517180in}}%
\pgfpathlineto{\pgfqpoint{3.024929in}{4.527525in}}%
\pgfpathlineto{\pgfqpoint{3.058592in}{4.534345in}}%
\pgfpathlineto{\pgfqpoint{3.048635in}{4.603777in}}%
\pgfpathlineto{\pgfqpoint{3.038696in}{4.672325in}}%
\pgfpathlineto{\pgfqpoint{3.005061in}{4.665013in}}%
\pgfpathlineto{\pgfqpoint{2.971453in}{4.653776in}}%
\pgfpathclose%
\pgfusepath{fill}%
\end{pgfscope}%
\begin{pgfscope}%
\pgfpathrectangle{\pgfqpoint{1.020000in}{0.880000in}}{\pgfqpoint{6.160000in}{6.160000in}}%
\pgfusepath{clip}%
\pgfsetbuttcap%
\pgfsetroundjoin%
\definecolor{currentfill}{rgb}{0.943432,0.802276,0.729172}%
\pgfsetfillcolor{currentfill}%
\pgfsetlinewidth{0.000000pt}%
\definecolor{currentstroke}{rgb}{0.000000,0.000000,0.000000}%
\pgfsetstrokecolor{currentstroke}%
\pgfsetdash{}{0pt}%
\pgfpathmoveto{\pgfqpoint{2.351236in}{4.194207in}}%
\pgfpathlineto{\pgfqpoint{2.360407in}{4.154293in}}%
\pgfpathlineto{\pgfqpoint{2.369606in}{4.113463in}}%
\pgfpathlineto{\pgfqpoint{2.403069in}{4.143525in}}%
\pgfpathlineto{\pgfqpoint{2.436505in}{4.175072in}}%
\pgfpathlineto{\pgfqpoint{2.427213in}{4.218955in}}%
\pgfpathlineto{\pgfqpoint{2.417948in}{4.261784in}}%
\pgfpathlineto{\pgfqpoint{2.384606in}{4.227191in}}%
\pgfpathlineto{\pgfqpoint{2.351236in}{4.194207in}}%
\pgfpathclose%
\pgfusepath{fill}%
\end{pgfscope}%
\begin{pgfscope}%
\pgfpathrectangle{\pgfqpoint{1.020000in}{0.880000in}}{\pgfqpoint{6.160000in}{6.160000in}}%
\pgfusepath{clip}%
\pgfsetbuttcap%
\pgfsetroundjoin%
\definecolor{currentfill}{rgb}{0.969192,0.705836,0.593704}%
\pgfsetfillcolor{currentfill}%
\pgfsetlinewidth{0.000000pt}%
\definecolor{currentstroke}{rgb}{0.000000,0.000000,0.000000}%
\pgfsetstrokecolor{currentstroke}%
\pgfsetdash{}{0pt}%
\pgfpathmoveto{\pgfqpoint{2.551140in}{4.411647in}}%
\pgfpathlineto{\pgfqpoint{2.560589in}{4.362450in}}%
\pgfpathlineto{\pgfqpoint{2.570068in}{4.311906in}}%
\pgfpathlineto{\pgfqpoint{2.603437in}{4.347461in}}%
\pgfpathlineto{\pgfqpoint{2.636807in}{4.382934in}}%
\pgfpathlineto{\pgfqpoint{2.627240in}{4.436881in}}%
\pgfpathlineto{\pgfqpoint{2.617704in}{4.489336in}}%
\pgfpathlineto{\pgfqpoint{2.584420in}{4.450545in}}%
\pgfpathlineto{\pgfqpoint{2.551140in}{4.411647in}}%
\pgfpathclose%
\pgfusepath{fill}%
\end{pgfscope}%
\begin{pgfscope}%
\pgfpathrectangle{\pgfqpoint{1.020000in}{0.880000in}}{\pgfqpoint{6.160000in}{6.160000in}}%
\pgfusepath{clip}%
\pgfsetbuttcap%
\pgfsetroundjoin%
\definecolor{currentfill}{rgb}{0.960490,0.616276,0.495467}%
\pgfsetfillcolor{currentfill}%
\pgfsetlinewidth{0.000000pt}%
\definecolor{currentstroke}{rgb}{0.000000,0.000000,0.000000}%
\pgfsetstrokecolor{currentstroke}%
\pgfsetdash{}{0pt}%
\pgfpathmoveto{\pgfqpoint{2.684304in}{4.565060in}}%
\pgfpathlineto{\pgfqpoint{2.693922in}{4.509457in}}%
\pgfpathlineto{\pgfqpoint{2.703571in}{4.452226in}}%
\pgfpathlineto{\pgfqpoint{2.736973in}{4.485309in}}%
\pgfpathlineto{\pgfqpoint{2.770394in}{4.516851in}}%
\pgfpathlineto{\pgfqpoint{2.760671in}{4.577106in}}%
\pgfpathlineto{\pgfqpoint{2.750979in}{4.635611in}}%
\pgfpathlineto{\pgfqpoint{2.717630in}{4.601187in}}%
\pgfpathlineto{\pgfqpoint{2.684304in}{4.565060in}}%
\pgfpathclose%
\pgfusepath{fill}%
\end{pgfscope}%
\begin{pgfscope}%
\pgfpathrectangle{\pgfqpoint{1.020000in}{0.880000in}}{\pgfqpoint{6.160000in}{6.160000in}}%
\pgfusepath{clip}%
\pgfsetbuttcap%
\pgfsetroundjoin%
\definecolor{currentfill}{rgb}{0.967711,0.662973,0.544323}%
\pgfsetfillcolor{currentfill}%
\pgfsetlinewidth{0.000000pt}%
\definecolor{currentstroke}{rgb}{0.000000,0.000000,0.000000}%
\pgfsetstrokecolor{currentstroke}%
\pgfsetdash{}{0pt}%
\pgfpathmoveto{\pgfqpoint{2.617704in}{4.489336in}}%
\pgfpathlineto{\pgfqpoint{2.627240in}{4.436881in}}%
\pgfpathlineto{\pgfqpoint{2.636807in}{4.382934in}}%
\pgfpathlineto{\pgfqpoint{2.670184in}{4.417976in}}%
\pgfpathlineto{\pgfqpoint{2.703571in}{4.452226in}}%
\pgfpathlineto{\pgfqpoint{2.693922in}{4.509457in}}%
\pgfpathlineto{\pgfqpoint{2.684304in}{4.565060in}}%
\pgfpathlineto{\pgfqpoint{2.650997in}{4.527640in}}%
\pgfpathlineto{\pgfqpoint{2.617704in}{4.489336in}}%
\pgfpathclose%
\pgfusepath{fill}%
\end{pgfscope}%
\begin{pgfscope}%
\pgfpathrectangle{\pgfqpoint{1.020000in}{0.880000in}}{\pgfqpoint{6.160000in}{6.160000in}}%
\pgfusepath{clip}%
\pgfsetbuttcap%
\pgfsetroundjoin%
\definecolor{currentfill}{rgb}{0.835345,0.860514,0.898970}%
\pgfsetfillcolor{currentfill}%
\pgfsetlinewidth{0.000000pt}%
\definecolor{currentstroke}{rgb}{0.000000,0.000000,0.000000}%
\pgfsetstrokecolor{currentstroke}%
\pgfsetdash{}{0pt}%
\pgfpathmoveto{\pgfqpoint{3.475675in}{3.953195in}}%
\pgfpathlineto{\pgfqpoint{3.485651in}{3.907325in}}%
\pgfpathlineto{\pgfqpoint{3.495617in}{3.864802in}}%
\pgfpathlineto{\pgfqpoint{3.529303in}{3.851757in}}%
\pgfpathlineto{\pgfqpoint{3.562967in}{3.837305in}}%
\pgfpathlineto{\pgfqpoint{3.552983in}{3.876720in}}%
\pgfpathlineto{\pgfqpoint{3.542994in}{3.919363in}}%
\pgfpathlineto{\pgfqpoint{3.509346in}{3.937022in}}%
\pgfpathlineto{\pgfqpoint{3.475675in}{3.953195in}}%
\pgfpathclose%
\pgfusepath{fill}%
\end{pgfscope}%
\begin{pgfscope}%
\pgfpathrectangle{\pgfqpoint{1.020000in}{0.880000in}}{\pgfqpoint{6.160000in}{6.160000in}}%
\pgfusepath{clip}%
\pgfsetbuttcap%
\pgfsetroundjoin%
\definecolor{currentfill}{rgb}{0.373552,0.497499,0.909467}%
\pgfsetfillcolor{currentfill}%
\pgfsetlinewidth{0.000000pt}%
\definecolor{currentstroke}{rgb}{0.000000,0.000000,0.000000}%
\pgfsetstrokecolor{currentstroke}%
\pgfsetdash{}{0pt}%
\pgfpathmoveto{\pgfqpoint{5.017559in}{3.064517in}}%
\pgfpathlineto{\pgfqpoint{5.028761in}{3.051918in}}%
\pgfpathlineto{\pgfqpoint{5.039979in}{3.038658in}}%
\pgfpathlineto{\pgfqpoint{5.073040in}{3.033381in}}%
\pgfpathlineto{\pgfqpoint{5.106083in}{3.028896in}}%
\pgfpathlineto{\pgfqpoint{5.094812in}{3.042135in}}%
\pgfpathlineto{\pgfqpoint{5.083555in}{3.054585in}}%
\pgfpathlineto{\pgfqpoint{5.050566in}{3.059172in}}%
\pgfpathlineto{\pgfqpoint{5.017559in}{3.064517in}}%
\pgfpathclose%
\pgfusepath{fill}%
\end{pgfscope}%
\begin{pgfscope}%
\pgfpathrectangle{\pgfqpoint{1.020000in}{0.880000in}}{\pgfqpoint{6.160000in}{6.160000in}}%
\pgfusepath{clip}%
\pgfsetbuttcap%
\pgfsetroundjoin%
\definecolor{currentfill}{rgb}{0.672538,0.782861,0.991982}%
\pgfsetfillcolor{currentfill}%
\pgfsetlinewidth{0.000000pt}%
\definecolor{currentstroke}{rgb}{0.000000,0.000000,0.000000}%
\pgfsetstrokecolor{currentstroke}%
\pgfsetdash{}{0pt}%
\pgfpathmoveto{\pgfqpoint{3.871696in}{3.605822in}}%
\pgfpathlineto{\pgfqpoint{3.881826in}{3.591908in}}%
\pgfpathlineto{\pgfqpoint{3.891968in}{3.580765in}}%
\pgfpathlineto{\pgfqpoint{3.925496in}{3.565166in}}%
\pgfpathlineto{\pgfqpoint{3.958992in}{3.549158in}}%
\pgfpathlineto{\pgfqpoint{3.948798in}{3.559551in}}%
\pgfpathlineto{\pgfqpoint{3.938617in}{3.572436in}}%
\pgfpathlineto{\pgfqpoint{3.905172in}{3.589241in}}%
\pgfpathlineto{\pgfqpoint{3.871696in}{3.605822in}}%
\pgfpathclose%
\pgfusepath{fill}%
\end{pgfscope}%
\begin{pgfscope}%
\pgfpathrectangle{\pgfqpoint{1.020000in}{0.880000in}}{\pgfqpoint{6.160000in}{6.160000in}}%
\pgfusepath{clip}%
\pgfsetbuttcap%
\pgfsetroundjoin%
\definecolor{currentfill}{rgb}{0.630089,0.752516,0.998508}%
\pgfsetfillcolor{currentfill}%
\pgfsetlinewidth{0.000000pt}%
\definecolor{currentstroke}{rgb}{0.000000,0.000000,0.000000}%
\pgfsetstrokecolor{currentstroke}%
\pgfsetdash{}{0pt}%
\pgfpathmoveto{\pgfqpoint{4.025889in}{3.516423in}}%
\pgfpathlineto{\pgfqpoint{4.036154in}{3.508584in}}%
\pgfpathlineto{\pgfqpoint{4.046436in}{3.502847in}}%
\pgfpathlineto{\pgfqpoint{4.079893in}{3.486093in}}%
\pgfpathlineto{\pgfqpoint{4.113318in}{3.469271in}}%
\pgfpathlineto{\pgfqpoint{4.102979in}{3.475485in}}%
\pgfpathlineto{\pgfqpoint{4.092659in}{3.483535in}}%
\pgfpathlineto{\pgfqpoint{4.059290in}{3.499941in}}%
\pgfpathlineto{\pgfqpoint{4.025889in}{3.516423in}}%
\pgfpathclose%
\pgfusepath{fill}%
\end{pgfscope}%
\begin{pgfscope}%
\pgfpathrectangle{\pgfqpoint{1.020000in}{0.880000in}}{\pgfqpoint{6.160000in}{6.160000in}}%
\pgfusepath{clip}%
\pgfsetbuttcap%
\pgfsetroundjoin%
\definecolor{currentfill}{rgb}{0.309060,0.413498,0.850128}%
\pgfsetfillcolor{currentfill}%
\pgfsetlinewidth{0.000000pt}%
\definecolor{currentstroke}{rgb}{0.000000,0.000000,0.000000}%
\pgfsetstrokecolor{currentstroke}%
\pgfsetdash{}{0pt}%
\pgfpathmoveto{\pgfqpoint{5.834273in}{2.939112in}}%
\pgfpathlineto{\pgfqpoint{5.846200in}{2.923664in}}%
\pgfpathlineto{\pgfqpoint{5.858151in}{2.908248in}}%
\pgfpathlineto{\pgfqpoint{5.891014in}{2.910105in}}%
\pgfpathlineto{\pgfqpoint{5.923857in}{2.912003in}}%
\pgfpathlineto{\pgfqpoint{5.911853in}{2.927337in}}%
\pgfpathlineto{\pgfqpoint{5.899871in}{2.942707in}}%
\pgfpathlineto{\pgfqpoint{5.867082in}{2.940891in}}%
\pgfpathlineto{\pgfqpoint{5.834273in}{2.939112in}}%
\pgfpathclose%
\pgfusepath{fill}%
\end{pgfscope}%
\begin{pgfscope}%
\pgfpathrectangle{\pgfqpoint{1.020000in}{0.880000in}}{\pgfqpoint{6.160000in}{6.160000in}}%
\pgfusepath{clip}%
\pgfsetbuttcap%
\pgfsetroundjoin%
\definecolor{currentfill}{rgb}{0.581486,0.713451,0.998314}%
\pgfsetfillcolor{currentfill}%
\pgfsetlinewidth{0.000000pt}%
\definecolor{currentstroke}{rgb}{0.000000,0.000000,0.000000}%
\pgfsetstrokecolor{currentstroke}%
\pgfsetdash{}{0pt}%
\pgfpathmoveto{\pgfqpoint{4.180070in}{3.435892in}}%
\pgfpathlineto{\pgfqpoint{4.190487in}{3.430467in}}%
\pgfpathlineto{\pgfqpoint{4.200924in}{3.426465in}}%
\pgfpathlineto{\pgfqpoint{4.234309in}{3.408912in}}%
\pgfpathlineto{\pgfqpoint{4.267661in}{3.391654in}}%
\pgfpathlineto{\pgfqpoint{4.257167in}{3.396991in}}%
\pgfpathlineto{\pgfqpoint{4.246694in}{3.403533in}}%
\pgfpathlineto{\pgfqpoint{4.213398in}{3.419542in}}%
\pgfpathlineto{\pgfqpoint{4.180070in}{3.435892in}}%
\pgfpathclose%
\pgfusepath{fill}%
\end{pgfscope}%
\begin{pgfscope}%
\pgfpathrectangle{\pgfqpoint{1.020000in}{0.880000in}}{\pgfqpoint{6.160000in}{6.160000in}}%
\pgfusepath{clip}%
\pgfsetbuttcap%
\pgfsetroundjoin%
\definecolor{currentfill}{rgb}{0.323718,0.433158,0.864722}%
\pgfsetfillcolor{currentfill}%
\pgfsetlinewidth{0.000000pt}%
\definecolor{currentstroke}{rgb}{0.000000,0.000000,0.000000}%
\pgfsetstrokecolor{currentstroke}%
\pgfsetdash{}{0pt}%
\pgfpathmoveto{\pgfqpoint{5.613530in}{2.960917in}}%
\pgfpathlineto{\pgfqpoint{5.625255in}{2.945341in}}%
\pgfpathlineto{\pgfqpoint{5.637001in}{2.929724in}}%
\pgfpathlineto{\pgfqpoint{5.669928in}{2.931074in}}%
\pgfpathlineto{\pgfqpoint{5.702837in}{2.932531in}}%
\pgfpathlineto{\pgfqpoint{5.691038in}{2.948114in}}%
\pgfpathlineto{\pgfqpoint{5.679260in}{2.963680in}}%
\pgfpathlineto{\pgfqpoint{5.646405in}{2.962249in}}%
\pgfpathlineto{\pgfqpoint{5.613530in}{2.960917in}}%
\pgfpathclose%
\pgfusepath{fill}%
\end{pgfscope}%
\begin{pgfscope}%
\pgfpathrectangle{\pgfqpoint{1.020000in}{0.880000in}}{\pgfqpoint{6.160000in}{6.160000in}}%
\pgfusepath{clip}%
\pgfsetbuttcap%
\pgfsetroundjoin%
\definecolor{currentfill}{rgb}{0.728970,0.817464,0.973188}%
\pgfsetfillcolor{currentfill}%
\pgfsetlinewidth{0.000000pt}%
\definecolor{currentstroke}{rgb}{0.000000,0.000000,0.000000}%
\pgfsetstrokecolor{currentstroke}%
\pgfsetdash{}{0pt}%
\pgfpathmoveto{\pgfqpoint{3.717419in}{3.711337in}}%
\pgfpathlineto{\pgfqpoint{3.727451in}{3.687671in}}%
\pgfpathlineto{\pgfqpoint{3.737486in}{3.667302in}}%
\pgfpathlineto{\pgfqpoint{3.771082in}{3.652921in}}%
\pgfpathlineto{\pgfqpoint{3.804650in}{3.637794in}}%
\pgfpathlineto{\pgfqpoint{3.794573in}{3.656166in}}%
\pgfpathlineto{\pgfqpoint{3.784504in}{3.677587in}}%
\pgfpathlineto{\pgfqpoint{3.750976in}{3.694762in}}%
\pgfpathlineto{\pgfqpoint{3.717419in}{3.711337in}}%
\pgfpathclose%
\pgfusepath{fill}%
\end{pgfscope}%
\begin{pgfscope}%
\pgfpathrectangle{\pgfqpoint{1.020000in}{0.880000in}}{\pgfqpoint{6.160000in}{6.160000in}}%
\pgfusepath{clip}%
\pgfsetbuttcap%
\pgfsetroundjoin%
\definecolor{currentfill}{rgb}{0.967711,0.662973,0.544323}%
\pgfsetfillcolor{currentfill}%
\pgfsetlinewidth{0.000000pt}%
\definecolor{currentstroke}{rgb}{0.000000,0.000000,0.000000}%
\pgfsetstrokecolor{currentstroke}%
\pgfsetdash{}{0pt}%
\pgfpathmoveto{\pgfqpoint{3.058592in}{4.534345in}}%
\pgfpathlineto{\pgfqpoint{3.068560in}{4.464574in}}%
\pgfpathlineto{\pgfqpoint{3.078533in}{4.394992in}}%
\pgfpathlineto{\pgfqpoint{3.112236in}{4.398171in}}%
\pgfpathlineto{\pgfqpoint{3.145950in}{4.398048in}}%
\pgfpathlineto{\pgfqpoint{3.135958in}{4.467383in}}%
\pgfpathlineto{\pgfqpoint{3.125972in}{4.536964in}}%
\pgfpathlineto{\pgfqpoint{3.092274in}{4.537517in}}%
\pgfpathlineto{\pgfqpoint{3.058592in}{4.534345in}}%
\pgfpathclose%
\pgfusepath{fill}%
\end{pgfscope}%
\begin{pgfscope}%
\pgfpathrectangle{\pgfqpoint{1.020000in}{0.880000in}}{\pgfqpoint{6.160000in}{6.160000in}}%
\pgfusepath{clip}%
\pgfsetbuttcap%
\pgfsetroundjoin%
\definecolor{currentfill}{rgb}{0.299441,0.400248,0.839842}%
\pgfsetfillcolor{currentfill}%
\pgfsetlinewidth{0.000000pt}%
\definecolor{currentstroke}{rgb}{0.000000,0.000000,0.000000}%
\pgfsetstrokecolor{currentstroke}%
\pgfsetdash{}{0pt}%
\pgfpathmoveto{\pgfqpoint{6.055030in}{2.919896in}}%
\pgfpathlineto{\pgfqpoint{6.067165in}{2.904777in}}%
\pgfpathlineto{\pgfqpoint{6.079324in}{2.889701in}}%
\pgfpathlineto{\pgfqpoint{6.112121in}{2.891819in}}%
\pgfpathlineto{\pgfqpoint{6.099935in}{2.906851in}}%
\pgfpathlineto{\pgfqpoint{6.087773in}{2.921927in}}%
\pgfpathlineto{\pgfqpoint{6.055030in}{2.919896in}}%
\pgfpathclose%
\pgfusepath{fill}%
\end{pgfscope}%
\begin{pgfscope}%
\pgfpathrectangle{\pgfqpoint{1.020000in}{0.880000in}}{\pgfqpoint{6.160000in}{6.160000in}}%
\pgfusepath{clip}%
\pgfsetbuttcap%
\pgfsetroundjoin%
\definecolor{currentfill}{rgb}{0.538004,0.674902,0.991722}%
\pgfsetfillcolor{currentfill}%
\pgfsetlinewidth{0.000000pt}%
\definecolor{currentstroke}{rgb}{0.000000,0.000000,0.000000}%
\pgfsetstrokecolor{currentstroke}%
\pgfsetdash{}{0pt}%
\pgfpathmoveto{\pgfqpoint{4.334268in}{3.358367in}}%
\pgfpathlineto{\pgfqpoint{4.344840in}{3.352580in}}%
\pgfpathlineto{\pgfqpoint{4.355434in}{3.347620in}}%
\pgfpathlineto{\pgfqpoint{4.388744in}{3.330126in}}%
\pgfpathlineto{\pgfqpoint{4.422023in}{3.313252in}}%
\pgfpathlineto{\pgfqpoint{4.411376in}{3.319852in}}%
\pgfpathlineto{\pgfqpoint{4.400751in}{3.327115in}}%
\pgfpathlineto{\pgfqpoint{4.367525in}{3.342467in}}%
\pgfpathlineto{\pgfqpoint{4.334268in}{3.358367in}}%
\pgfpathclose%
\pgfusepath{fill}%
\end{pgfscope}%
\begin{pgfscope}%
\pgfpathrectangle{\pgfqpoint{1.020000in}{0.880000in}}{\pgfqpoint{6.160000in}{6.160000in}}%
\pgfusepath{clip}%
\pgfsetbuttcap%
\pgfsetroundjoin%
\definecolor{currentfill}{rgb}{0.333490,0.446265,0.874452}%
\pgfsetfillcolor{currentfill}%
\pgfsetlinewidth{0.000000pt}%
\definecolor{currentstroke}{rgb}{0.000000,0.000000,0.000000}%
\pgfsetstrokecolor{currentstroke}%
\pgfsetdash{}{0pt}%
\pgfpathmoveto{\pgfqpoint{5.392803in}{2.986794in}}%
\pgfpathlineto{\pgfqpoint{5.404333in}{2.971724in}}%
\pgfpathlineto{\pgfqpoint{5.415881in}{2.956383in}}%
\pgfpathlineto{\pgfqpoint{5.448869in}{2.956569in}}%
\pgfpathlineto{\pgfqpoint{5.481839in}{2.957031in}}%
\pgfpathlineto{\pgfqpoint{5.470239in}{2.972429in}}%
\pgfpathlineto{\pgfqpoint{5.458657in}{2.987601in}}%
\pgfpathlineto{\pgfqpoint{5.425739in}{2.987057in}}%
\pgfpathlineto{\pgfqpoint{5.392803in}{2.986794in}}%
\pgfpathclose%
\pgfusepath{fill}%
\end{pgfscope}%
\begin{pgfscope}%
\pgfpathrectangle{\pgfqpoint{1.020000in}{0.880000in}}{\pgfqpoint{6.160000in}{6.160000in}}%
\pgfusepath{clip}%
\pgfsetbuttcap%
\pgfsetroundjoin%
\definecolor{currentfill}{rgb}{0.494638,0.633022,0.978983}%
\pgfsetfillcolor{currentfill}%
\pgfsetlinewidth{0.000000pt}%
\definecolor{currentstroke}{rgb}{0.000000,0.000000,0.000000}%
\pgfsetstrokecolor{currentstroke}%
\pgfsetdash{}{0pt}%
\pgfpathmoveto{\pgfqpoint{4.488490in}{3.281511in}}%
\pgfpathlineto{\pgfqpoint{4.499211in}{3.273905in}}%
\pgfpathlineto{\pgfqpoint{4.509955in}{3.266648in}}%
\pgfpathlineto{\pgfqpoint{4.543195in}{3.250378in}}%
\pgfpathlineto{\pgfqpoint{4.576405in}{3.234953in}}%
\pgfpathlineto{\pgfqpoint{4.565611in}{3.243612in}}%
\pgfpathlineto{\pgfqpoint{4.554838in}{3.252479in}}%
\pgfpathlineto{\pgfqpoint{4.521678in}{3.266664in}}%
\pgfpathlineto{\pgfqpoint{4.488490in}{3.281511in}}%
\pgfpathclose%
\pgfusepath{fill}%
\end{pgfscope}%
\begin{pgfscope}%
\pgfpathrectangle{\pgfqpoint{1.020000in}{0.880000in}}{\pgfqpoint{6.160000in}{6.160000in}}%
\pgfusepath{clip}%
\pgfsetbuttcap%
\pgfsetroundjoin%
\definecolor{currentfill}{rgb}{0.965899,0.740142,0.637058}%
\pgfsetfillcolor{currentfill}%
\pgfsetlinewidth{0.000000pt}%
\definecolor{currentstroke}{rgb}{0.000000,0.000000,0.000000}%
\pgfsetstrokecolor{currentstroke}%
\pgfsetdash{}{0pt}%
\pgfpathmoveto{\pgfqpoint{3.145950in}{4.398048in}}%
\pgfpathlineto{\pgfqpoint{3.155943in}{4.329473in}}%
\pgfpathlineto{\pgfqpoint{3.165931in}{4.262146in}}%
\pgfpathlineto{\pgfqpoint{3.199667in}{4.259656in}}%
\pgfpathlineto{\pgfqpoint{3.233406in}{4.254244in}}%
\pgfpathlineto{\pgfqpoint{3.223406in}{4.320373in}}%
\pgfpathlineto{\pgfqpoint{3.213403in}{4.387807in}}%
\pgfpathlineto{\pgfqpoint{3.179674in}{4.394590in}}%
\pgfpathlineto{\pgfqpoint{3.145950in}{4.398048in}}%
\pgfpathclose%
\pgfusepath{fill}%
\end{pgfscope}%
\begin{pgfscope}%
\pgfpathrectangle{\pgfqpoint{1.020000in}{0.880000in}}{\pgfqpoint{6.160000in}{6.160000in}}%
\pgfusepath{clip}%
\pgfsetbuttcap%
\pgfsetroundjoin%
\definecolor{currentfill}{rgb}{0.956653,0.598034,0.477302}%
\pgfsetfillcolor{currentfill}%
\pgfsetlinewidth{0.000000pt}%
\definecolor{currentstroke}{rgb}{0.000000,0.000000,0.000000}%
\pgfsetstrokecolor{currentstroke}%
\pgfsetdash{}{0pt}%
\pgfpathmoveto{\pgfqpoint{2.904321in}{4.620293in}}%
\pgfpathlineto{\pgfqpoint{2.914188in}{4.553908in}}%
\pgfpathlineto{\pgfqpoint{2.924072in}{4.486616in}}%
\pgfpathlineto{\pgfqpoint{2.957669in}{4.503476in}}%
\pgfpathlineto{\pgfqpoint{2.991288in}{4.517180in}}%
\pgfpathlineto{\pgfqpoint{2.981361in}{4.585930in}}%
\pgfpathlineto{\pgfqpoint{2.971453in}{4.653776in}}%
\pgfpathlineto{\pgfqpoint{2.937872in}{4.638794in}}%
\pgfpathlineto{\pgfqpoint{2.904321in}{4.620293in}}%
\pgfpathclose%
\pgfusepath{fill}%
\end{pgfscope}%
\begin{pgfscope}%
\pgfpathrectangle{\pgfqpoint{1.020000in}{0.880000in}}{\pgfqpoint{6.160000in}{6.160000in}}%
\pgfusepath{clip}%
\pgfsetbuttcap%
\pgfsetroundjoin%
\definecolor{currentfill}{rgb}{0.451739,0.588181,0.960201}%
\pgfsetfillcolor{currentfill}%
\pgfsetlinewidth{0.000000pt}%
\definecolor{currentstroke}{rgb}{0.000000,0.000000,0.000000}%
\pgfsetstrokecolor{currentstroke}%
\pgfsetdash{}{0pt}%
\pgfpathmoveto{\pgfqpoint{4.642743in}{3.206586in}}%
\pgfpathlineto{\pgfqpoint{4.653609in}{3.196932in}}%
\pgfpathlineto{\pgfqpoint{4.664496in}{3.187236in}}%
\pgfpathlineto{\pgfqpoint{4.697674in}{3.173351in}}%
\pgfpathlineto{\pgfqpoint{4.730826in}{3.160428in}}%
\pgfpathlineto{\pgfqpoint{4.719889in}{3.170967in}}%
\pgfpathlineto{\pgfqpoint{4.708972in}{3.181308in}}%
\pgfpathlineto{\pgfqpoint{4.675871in}{3.193582in}}%
\pgfpathlineto{\pgfqpoint{4.642743in}{3.206586in}}%
\pgfpathclose%
\pgfusepath{fill}%
\end{pgfscope}%
\begin{pgfscope}%
\pgfpathrectangle{\pgfqpoint{1.020000in}{0.880000in}}{\pgfqpoint{6.160000in}{6.160000in}}%
\pgfusepath{clip}%
\pgfsetbuttcap%
\pgfsetroundjoin%
\definecolor{currentfill}{rgb}{0.353369,0.472069,0.892570}%
\pgfsetfillcolor{currentfill}%
\pgfsetlinewidth{0.000000pt}%
\definecolor{currentstroke}{rgb}{0.000000,0.000000,0.000000}%
\pgfsetstrokecolor{currentstroke}%
\pgfsetdash{}{0pt}%
\pgfpathmoveto{\pgfqpoint{5.172115in}{3.022102in}}%
\pgfpathlineto{\pgfqpoint{5.183455in}{3.008212in}}%
\pgfpathlineto{\pgfqpoint{5.194812in}{2.993780in}}%
\pgfpathlineto{\pgfqpoint{5.227854in}{2.991335in}}%
\pgfpathlineto{\pgfqpoint{5.260879in}{2.989475in}}%
\pgfpathlineto{\pgfqpoint{5.249470in}{3.003965in}}%
\pgfpathlineto{\pgfqpoint{5.238077in}{3.017889in}}%
\pgfpathlineto{\pgfqpoint{5.205105in}{3.019697in}}%
\pgfpathlineto{\pgfqpoint{5.172115in}{3.022102in}}%
\pgfpathclose%
\pgfusepath{fill}%
\end{pgfscope}%
\begin{pgfscope}%
\pgfpathrectangle{\pgfqpoint{1.020000in}{0.880000in}}{\pgfqpoint{6.160000in}{6.160000in}}%
\pgfusepath{clip}%
\pgfsetbuttcap%
\pgfsetroundjoin%
\definecolor{currentfill}{rgb}{0.943432,0.802276,0.729172}%
\pgfsetfillcolor{currentfill}%
\pgfsetlinewidth{0.000000pt}%
\definecolor{currentstroke}{rgb}{0.000000,0.000000,0.000000}%
\pgfsetstrokecolor{currentstroke}%
\pgfsetdash{}{0pt}%
\pgfpathmoveto{\pgfqpoint{3.233406in}{4.254244in}}%
\pgfpathlineto{\pgfqpoint{3.243398in}{4.189876in}}%
\pgfpathlineto{\pgfqpoint{3.253380in}{4.127687in}}%
\pgfpathlineto{\pgfqpoint{3.287126in}{4.121234in}}%
\pgfpathlineto{\pgfqpoint{3.320865in}{4.112313in}}%
\pgfpathlineto{\pgfqpoint{3.310875in}{4.172471in}}%
\pgfpathlineto{\pgfqpoint{3.300876in}{4.234840in}}%
\pgfpathlineto{\pgfqpoint{3.267143in}{4.245948in}}%
\pgfpathlineto{\pgfqpoint{3.233406in}{4.254244in}}%
\pgfpathclose%
\pgfusepath{fill}%
\end{pgfscope}%
\begin{pgfscope}%
\pgfpathrectangle{\pgfqpoint{1.020000in}{0.880000in}}{\pgfqpoint{6.160000in}{6.160000in}}%
\pgfusepath{clip}%
\pgfsetbuttcap%
\pgfsetroundjoin%
\definecolor{currentfill}{rgb}{0.786721,0.844807,0.939810}%
\pgfsetfillcolor{currentfill}%
\pgfsetlinewidth{0.000000pt}%
\definecolor{currentstroke}{rgb}{0.000000,0.000000,0.000000}%
\pgfsetstrokecolor{currentstroke}%
\pgfsetdash{}{0pt}%
\pgfpathmoveto{\pgfqpoint{3.562967in}{3.837305in}}%
\pgfpathlineto{\pgfqpoint{3.572944in}{3.801298in}}%
\pgfpathlineto{\pgfqpoint{3.582916in}{3.768838in}}%
\pgfpathlineto{\pgfqpoint{3.616580in}{3.756053in}}%
\pgfpathlineto{\pgfqpoint{3.650220in}{3.742121in}}%
\pgfpathlineto{\pgfqpoint{3.640221in}{3.771745in}}%
\pgfpathlineto{\pgfqpoint{3.630219in}{3.804741in}}%
\pgfpathlineto{\pgfqpoint{3.596606in}{3.821584in}}%
\pgfpathlineto{\pgfqpoint{3.562967in}{3.837305in}}%
\pgfpathclose%
\pgfusepath{fill}%
\end{pgfscope}%
\begin{pgfscope}%
\pgfpathrectangle{\pgfqpoint{1.020000in}{0.880000in}}{\pgfqpoint{6.160000in}{6.160000in}}%
\pgfusepath{clip}%
\pgfsetbuttcap%
\pgfsetroundjoin%
\definecolor{currentfill}{rgb}{0.414801,0.546874,0.939088}%
\pgfsetfillcolor{currentfill}%
\pgfsetlinewidth{0.000000pt}%
\definecolor{currentstroke}{rgb}{0.000000,0.000000,0.000000}%
\pgfsetstrokecolor{currentstroke}%
\pgfsetdash{}{0pt}%
\pgfpathmoveto{\pgfqpoint{4.797057in}{3.137309in}}%
\pgfpathlineto{\pgfqpoint{4.808065in}{3.126000in}}%
\pgfpathlineto{\pgfqpoint{4.819092in}{3.114337in}}%
\pgfpathlineto{\pgfqpoint{4.852223in}{3.103722in}}%
\pgfpathlineto{\pgfqpoint{4.885332in}{3.094091in}}%
\pgfpathlineto{\pgfqpoint{4.874254in}{3.106084in}}%
\pgfpathlineto{\pgfqpoint{4.863193in}{3.117549in}}%
\pgfpathlineto{\pgfqpoint{4.830137in}{3.127028in}}%
\pgfpathlineto{\pgfqpoint{4.797057in}{3.137309in}}%
\pgfpathclose%
\pgfusepath{fill}%
\end{pgfscope}%
\begin{pgfscope}%
\pgfpathrectangle{\pgfqpoint{1.020000in}{0.880000in}}{\pgfqpoint{6.160000in}{6.160000in}}%
\pgfusepath{clip}%
\pgfsetbuttcap%
\pgfsetroundjoin%
\definecolor{currentfill}{rgb}{0.902849,0.844796,0.811970}%
\pgfsetfillcolor{currentfill}%
\pgfsetlinewidth{0.000000pt}%
\definecolor{currentstroke}{rgb}{0.000000,0.000000,0.000000}%
\pgfsetstrokecolor{currentstroke}%
\pgfsetdash{}{0pt}%
\pgfpathmoveto{\pgfqpoint{3.320865in}{4.112313in}}%
\pgfpathlineto{\pgfqpoint{3.330844in}{4.054738in}}%
\pgfpathlineto{\pgfqpoint{3.340810in}{4.000080in}}%
\pgfpathlineto{\pgfqpoint{3.374548in}{3.991280in}}%
\pgfpathlineto{\pgfqpoint{3.408274in}{3.980459in}}%
\pgfpathlineto{\pgfqpoint{3.398298in}{4.032493in}}%
\pgfpathlineto{\pgfqpoint{3.388312in}{4.087426in}}%
\pgfpathlineto{\pgfqpoint{3.354595in}{4.101007in}}%
\pgfpathlineto{\pgfqpoint{3.320865in}{4.112313in}}%
\pgfpathclose%
\pgfusepath{fill}%
\end{pgfscope}%
\begin{pgfscope}%
\pgfpathrectangle{\pgfqpoint{1.020000in}{0.880000in}}{\pgfqpoint{6.160000in}{6.160000in}}%
\pgfusepath{clip}%
\pgfsetbuttcap%
\pgfsetroundjoin%
\definecolor{currentfill}{rgb}{0.961595,0.622247,0.501551}%
\pgfsetfillcolor{currentfill}%
\pgfsetlinewidth{0.000000pt}%
\definecolor{currentstroke}{rgb}{0.000000,0.000000,0.000000}%
\pgfsetstrokecolor{currentstroke}%
\pgfsetdash{}{0pt}%
\pgfpathmoveto{\pgfqpoint{2.837304in}{4.573820in}}%
\pgfpathlineto{\pgfqpoint{2.847117in}{4.509569in}}%
\pgfpathlineto{\pgfqpoint{2.856949in}{4.444424in}}%
\pgfpathlineto{\pgfqpoint{2.890500in}{4.466842in}}%
\pgfpathlineto{\pgfqpoint{2.924072in}{4.486616in}}%
\pgfpathlineto{\pgfqpoint{2.914188in}{4.553908in}}%
\pgfpathlineto{\pgfqpoint{2.904321in}{4.620293in}}%
\pgfpathlineto{\pgfqpoint{2.870798in}{4.598535in}}%
\pgfpathlineto{\pgfqpoint{2.837304in}{4.573820in}}%
\pgfpathclose%
\pgfusepath{fill}%
\end{pgfscope}%
\begin{pgfscope}%
\pgfpathrectangle{\pgfqpoint{1.020000in}{0.880000in}}{\pgfqpoint{6.160000in}{6.160000in}}%
\pgfusepath{clip}%
\pgfsetbuttcap%
\pgfsetroundjoin%
\definecolor{currentfill}{rgb}{0.304174,0.406945,0.845263}%
\pgfsetfillcolor{currentfill}%
\pgfsetlinewidth{0.000000pt}%
\definecolor{currentstroke}{rgb}{0.000000,0.000000,0.000000}%
\pgfsetstrokecolor{currentstroke}%
\pgfsetdash{}{0pt}%
\pgfpathmoveto{\pgfqpoint{5.989483in}{2.915897in}}%
\pgfpathlineto{\pgfqpoint{6.001565in}{2.900691in}}%
\pgfpathlineto{\pgfqpoint{6.013670in}{2.885528in}}%
\pgfpathlineto{\pgfqpoint{6.046507in}{2.887603in}}%
\pgfpathlineto{\pgfqpoint{6.079324in}{2.889701in}}%
\pgfpathlineto{\pgfqpoint{6.067165in}{2.904777in}}%
\pgfpathlineto{\pgfqpoint{6.055030in}{2.919896in}}%
\pgfpathlineto{\pgfqpoint{6.022267in}{2.917885in}}%
\pgfpathlineto{\pgfqpoint{5.989483in}{2.915897in}}%
\pgfpathclose%
\pgfusepath{fill}%
\end{pgfscope}%
\begin{pgfscope}%
\pgfpathrectangle{\pgfqpoint{1.020000in}{0.880000in}}{\pgfqpoint{6.160000in}{6.160000in}}%
\pgfusepath{clip}%
\pgfsetbuttcap%
\pgfsetroundjoin%
\definecolor{currentfill}{rgb}{0.313946,0.420052,0.854993}%
\pgfsetfillcolor{currentfill}%
\pgfsetlinewidth{0.000000pt}%
\definecolor{currentstroke}{rgb}{0.000000,0.000000,0.000000}%
\pgfsetstrokecolor{currentstroke}%
\pgfsetdash{}{0pt}%
\pgfpathmoveto{\pgfqpoint{5.768594in}{2.935696in}}%
\pgfpathlineto{\pgfqpoint{5.780468in}{2.920176in}}%
\pgfpathlineto{\pgfqpoint{5.792365in}{2.904683in}}%
\pgfpathlineto{\pgfqpoint{5.825268in}{2.906437in}}%
\pgfpathlineto{\pgfqpoint{5.858151in}{2.908248in}}%
\pgfpathlineto{\pgfqpoint{5.846200in}{2.923664in}}%
\pgfpathlineto{\pgfqpoint{5.834273in}{2.939112in}}%
\pgfpathlineto{\pgfqpoint{5.801443in}{2.937378in}}%
\pgfpathlineto{\pgfqpoint{5.768594in}{2.935696in}}%
\pgfpathclose%
\pgfusepath{fill}%
\end{pgfscope}%
\begin{pgfscope}%
\pgfpathrectangle{\pgfqpoint{1.020000in}{0.880000in}}{\pgfqpoint{6.160000in}{6.160000in}}%
\pgfusepath{clip}%
\pgfsetbuttcap%
\pgfsetroundjoin%
\definecolor{currentfill}{rgb}{0.968105,0.668475,0.550486}%
\pgfsetfillcolor{currentfill}%
\pgfsetlinewidth{0.000000pt}%
\definecolor{currentstroke}{rgb}{0.000000,0.000000,0.000000}%
\pgfsetstrokecolor{currentstroke}%
\pgfsetdash{}{0pt}%
\pgfpathmoveto{\pgfqpoint{2.991288in}{4.517180in}}%
\pgfpathlineto{\pgfqpoint{3.001226in}{4.448053in}}%
\pgfpathlineto{\pgfqpoint{3.011170in}{4.379060in}}%
\pgfpathlineto{\pgfqpoint{3.044843in}{4.388584in}}%
\pgfpathlineto{\pgfqpoint{3.078533in}{4.394992in}}%
\pgfpathlineto{\pgfqpoint{3.068560in}{4.464574in}}%
\pgfpathlineto{\pgfqpoint{3.058592in}{4.534345in}}%
\pgfpathlineto{\pgfqpoint{3.024929in}{4.527525in}}%
\pgfpathlineto{\pgfqpoint{2.991288in}{4.517180in}}%
\pgfpathclose%
\pgfusepath{fill}%
\end{pgfscope}%
\begin{pgfscope}%
\pgfpathrectangle{\pgfqpoint{1.020000in}{0.880000in}}{\pgfqpoint{6.160000in}{6.160000in}}%
\pgfusepath{clip}%
\pgfsetbuttcap%
\pgfsetroundjoin%
\definecolor{currentfill}{rgb}{0.323718,0.433158,0.864722}%
\pgfsetfillcolor{currentfill}%
\pgfsetlinewidth{0.000000pt}%
\definecolor{currentstroke}{rgb}{0.000000,0.000000,0.000000}%
\pgfsetstrokecolor{currentstroke}%
\pgfsetdash{}{0pt}%
\pgfpathmoveto{\pgfqpoint{5.547722in}{2.958632in}}%
\pgfpathlineto{\pgfqpoint{5.559395in}{2.943062in}}%
\pgfpathlineto{\pgfqpoint{5.571088in}{2.927421in}}%
\pgfpathlineto{\pgfqpoint{5.604054in}{2.928498in}}%
\pgfpathlineto{\pgfqpoint{5.637001in}{2.929724in}}%
\pgfpathlineto{\pgfqpoint{5.625255in}{2.945341in}}%
\pgfpathlineto{\pgfqpoint{5.613530in}{2.960917in}}%
\pgfpathlineto{\pgfqpoint{5.580636in}{2.959703in}}%
\pgfpathlineto{\pgfqpoint{5.547722in}{2.958632in}}%
\pgfpathclose%
\pgfusepath{fill}%
\end{pgfscope}%
\begin{pgfscope}%
\pgfpathrectangle{\pgfqpoint{1.020000in}{0.880000in}}{\pgfqpoint{6.160000in}{6.160000in}}%
\pgfusepath{clip}%
\pgfsetbuttcap%
\pgfsetroundjoin%
\definecolor{currentfill}{rgb}{0.383662,0.510183,0.917831}%
\pgfsetfillcolor{currentfill}%
\pgfsetlinewidth{0.000000pt}%
\definecolor{currentstroke}{rgb}{0.000000,0.000000,0.000000}%
\pgfsetstrokecolor{currentstroke}%
\pgfsetdash{}{0pt}%
\pgfpathmoveto{\pgfqpoint{4.951486in}{3.077596in}}%
\pgfpathlineto{\pgfqpoint{4.962634in}{3.064962in}}%
\pgfpathlineto{\pgfqpoint{4.973800in}{3.051801in}}%
\pgfpathlineto{\pgfqpoint{5.006899in}{3.044780in}}%
\pgfpathlineto{\pgfqpoint{5.039979in}{3.038658in}}%
\pgfpathlineto{\pgfqpoint{5.028761in}{3.051918in}}%
\pgfpathlineto{\pgfqpoint{5.017559in}{3.064517in}}%
\pgfpathlineto{\pgfqpoint{4.984532in}{3.070648in}}%
\pgfpathlineto{\pgfqpoint{4.951486in}{3.077596in}}%
\pgfpathclose%
\pgfusepath{fill}%
\end{pgfscope}%
\begin{pgfscope}%
\pgfpathrectangle{\pgfqpoint{1.020000in}{0.880000in}}{\pgfqpoint{6.160000in}{6.160000in}}%
\pgfusepath{clip}%
\pgfsetbuttcap%
\pgfsetroundjoin%
\definecolor{currentfill}{rgb}{0.966922,0.651969,0.531997}%
\pgfsetfillcolor{currentfill}%
\pgfsetlinewidth{0.000000pt}%
\definecolor{currentstroke}{rgb}{0.000000,0.000000,0.000000}%
\pgfsetstrokecolor{currentstroke}%
\pgfsetdash{}{0pt}%
\pgfpathmoveto{\pgfqpoint{2.770394in}{4.516851in}}%
\pgfpathlineto{\pgfqpoint{2.780143in}{4.455288in}}%
\pgfpathlineto{\pgfqpoint{2.789911in}{4.392860in}}%
\pgfpathlineto{\pgfqpoint{2.823421in}{4.419659in}}%
\pgfpathlineto{\pgfqpoint{2.856949in}{4.444424in}}%
\pgfpathlineto{\pgfqpoint{2.847117in}{4.509569in}}%
\pgfpathlineto{\pgfqpoint{2.837304in}{4.573820in}}%
\pgfpathlineto{\pgfqpoint{2.803837in}{4.546476in}}%
\pgfpathlineto{\pgfqpoint{2.770394in}{4.516851in}}%
\pgfpathclose%
\pgfusepath{fill}%
\end{pgfscope}%
\begin{pgfscope}%
\pgfpathrectangle{\pgfqpoint{1.020000in}{0.880000in}}{\pgfqpoint{6.160000in}{6.160000in}}%
\pgfusepath{clip}%
\pgfsetbuttcap%
\pgfsetroundjoin%
\definecolor{currentfill}{rgb}{0.851372,0.863125,0.881064}%
\pgfsetfillcolor{currentfill}%
\pgfsetlinewidth{0.000000pt}%
\definecolor{currentstroke}{rgb}{0.000000,0.000000,0.000000}%
\pgfsetstrokecolor{currentstroke}%
\pgfsetdash{}{0pt}%
\pgfpathmoveto{\pgfqpoint{3.408274in}{3.980459in}}%
\pgfpathlineto{\pgfqpoint{3.418237in}{3.931608in}}%
\pgfpathlineto{\pgfqpoint{3.428189in}{3.886176in}}%
\pgfpathlineto{\pgfqpoint{3.461912in}{3.876312in}}%
\pgfpathlineto{\pgfqpoint{3.495617in}{3.864802in}}%
\pgfpathlineto{\pgfqpoint{3.485651in}{3.907325in}}%
\pgfpathlineto{\pgfqpoint{3.475675in}{3.953195in}}%
\pgfpathlineto{\pgfqpoint{3.441983in}{3.967722in}}%
\pgfpathlineto{\pgfqpoint{3.408274in}{3.980459in}}%
\pgfpathclose%
\pgfusepath{fill}%
\end{pgfscope}%
\begin{pgfscope}%
\pgfpathrectangle{\pgfqpoint{1.020000in}{0.880000in}}{\pgfqpoint{6.160000in}{6.160000in}}%
\pgfusepath{clip}%
\pgfsetbuttcap%
\pgfsetroundjoin%
\definecolor{currentfill}{rgb}{0.338377,0.452819,0.879317}%
\pgfsetfillcolor{currentfill}%
\pgfsetlinewidth{0.000000pt}%
\definecolor{currentstroke}{rgb}{0.000000,0.000000,0.000000}%
\pgfsetstrokecolor{currentstroke}%
\pgfsetdash{}{0pt}%
\pgfpathmoveto{\pgfqpoint{5.326876in}{2.987289in}}%
\pgfpathlineto{\pgfqpoint{5.338354in}{2.972304in}}%
\pgfpathlineto{\pgfqpoint{5.349850in}{2.957018in}}%
\pgfpathlineto{\pgfqpoint{5.382875in}{2.956517in}}%
\pgfpathlineto{\pgfqpoint{5.415881in}{2.956383in}}%
\pgfpathlineto{\pgfqpoint{5.404333in}{2.971724in}}%
\pgfpathlineto{\pgfqpoint{5.392803in}{2.986794in}}%
\pgfpathlineto{\pgfqpoint{5.359849in}{2.986855in}}%
\pgfpathlineto{\pgfqpoint{5.326876in}{2.987289in}}%
\pgfpathclose%
\pgfusepath{fill}%
\end{pgfscope}%
\begin{pgfscope}%
\pgfpathrectangle{\pgfqpoint{1.020000in}{0.880000in}}{\pgfqpoint{6.160000in}{6.160000in}}%
\pgfusepath{clip}%
\pgfsetbuttcap%
\pgfsetroundjoin%
\definecolor{currentfill}{rgb}{0.969289,0.684982,0.568975}%
\pgfsetfillcolor{currentfill}%
\pgfsetlinewidth{0.000000pt}%
\definecolor{currentstroke}{rgb}{0.000000,0.000000,0.000000}%
\pgfsetstrokecolor{currentstroke}%
\pgfsetdash{}{0pt}%
\pgfpathmoveto{\pgfqpoint{2.703571in}{4.452226in}}%
\pgfpathlineto{\pgfqpoint{2.713246in}{4.393770in}}%
\pgfpathlineto{\pgfqpoint{2.722941in}{4.334491in}}%
\pgfpathlineto{\pgfqpoint{2.756419in}{4.364359in}}%
\pgfpathlineto{\pgfqpoint{2.789911in}{4.392860in}}%
\pgfpathlineto{\pgfqpoint{2.780143in}{4.455288in}}%
\pgfpathlineto{\pgfqpoint{2.770394in}{4.516851in}}%
\pgfpathlineto{\pgfqpoint{2.736973in}{4.485309in}}%
\pgfpathlineto{\pgfqpoint{2.703571in}{4.452226in}}%
\pgfpathclose%
\pgfusepath{fill}%
\end{pgfscope}%
\begin{pgfscope}%
\pgfpathrectangle{\pgfqpoint{1.020000in}{0.880000in}}{\pgfqpoint{6.160000in}{6.160000in}}%
\pgfusepath{clip}%
\pgfsetbuttcap%
\pgfsetroundjoin%
\definecolor{currentfill}{rgb}{0.966962,0.735670,0.630877}%
\pgfsetfillcolor{currentfill}%
\pgfsetlinewidth{0.000000pt}%
\definecolor{currentstroke}{rgb}{0.000000,0.000000,0.000000}%
\pgfsetstrokecolor{currentstroke}%
\pgfsetdash{}{0pt}%
\pgfpathmoveto{\pgfqpoint{3.078533in}{4.394992in}}%
\pgfpathlineto{\pgfqpoint{3.088506in}{4.326106in}}%
\pgfpathlineto{\pgfqpoint{3.098475in}{4.258392in}}%
\pgfpathlineto{\pgfqpoint{3.132200in}{4.261713in}}%
\pgfpathlineto{\pgfqpoint{3.165931in}{4.262146in}}%
\pgfpathlineto{\pgfqpoint{3.155943in}{4.329473in}}%
\pgfpathlineto{\pgfqpoint{3.145950in}{4.398048in}}%
\pgfpathlineto{\pgfqpoint{3.112236in}{4.398171in}}%
\pgfpathlineto{\pgfqpoint{3.078533in}{4.394992in}}%
\pgfpathclose%
\pgfusepath{fill}%
\end{pgfscope}%
\begin{pgfscope}%
\pgfpathrectangle{\pgfqpoint{1.020000in}{0.880000in}}{\pgfqpoint{6.160000in}{6.160000in}}%
\pgfusepath{clip}%
\pgfsetbuttcap%
\pgfsetroundjoin%
\definecolor{currentfill}{rgb}{0.698454,0.799450,0.984577}%
\pgfsetfillcolor{currentfill}%
\pgfsetlinewidth{0.000000pt}%
\definecolor{currentstroke}{rgb}{0.000000,0.000000,0.000000}%
\pgfsetstrokecolor{currentstroke}%
\pgfsetdash{}{0pt}%
\pgfpathmoveto{\pgfqpoint{3.804650in}{3.637794in}}%
\pgfpathlineto{\pgfqpoint{3.814733in}{3.622483in}}%
\pgfpathlineto{\pgfqpoint{3.824825in}{3.610210in}}%
\pgfpathlineto{\pgfqpoint{3.858411in}{3.595822in}}%
\pgfpathlineto{\pgfqpoint{3.891968in}{3.580765in}}%
\pgfpathlineto{\pgfqpoint{3.881826in}{3.591908in}}%
\pgfpathlineto{\pgfqpoint{3.871696in}{3.605822in}}%
\pgfpathlineto{\pgfqpoint{3.838188in}{3.622050in}}%
\pgfpathlineto{\pgfqpoint{3.804650in}{3.637794in}}%
\pgfpathclose%
\pgfusepath{fill}%
\end{pgfscope}%
\begin{pgfscope}%
\pgfpathrectangle{\pgfqpoint{1.020000in}{0.880000in}}{\pgfqpoint{6.160000in}{6.160000in}}%
\pgfusepath{clip}%
\pgfsetbuttcap%
\pgfsetroundjoin%
\definecolor{currentfill}{rgb}{0.968533,0.715841,0.606097}%
\pgfsetfillcolor{currentfill}%
\pgfsetlinewidth{0.000000pt}%
\definecolor{currentstroke}{rgb}{0.000000,0.000000,0.000000}%
\pgfsetstrokecolor{currentstroke}%
\pgfsetdash{}{0pt}%
\pgfpathmoveto{\pgfqpoint{2.636807in}{4.382934in}}%
\pgfpathlineto{\pgfqpoint{2.646400in}{4.327856in}}%
\pgfpathlineto{\pgfqpoint{2.656013in}{4.272009in}}%
\pgfpathlineto{\pgfqpoint{2.689474in}{4.303596in}}%
\pgfpathlineto{\pgfqpoint{2.722941in}{4.334491in}}%
\pgfpathlineto{\pgfqpoint{2.713246in}{4.393770in}}%
\pgfpathlineto{\pgfqpoint{2.703571in}{4.452226in}}%
\pgfpathlineto{\pgfqpoint{2.670184in}{4.417976in}}%
\pgfpathlineto{\pgfqpoint{2.636807in}{4.382934in}}%
\pgfpathclose%
\pgfusepath{fill}%
\end{pgfscope}%
\begin{pgfscope}%
\pgfpathrectangle{\pgfqpoint{1.020000in}{0.880000in}}{\pgfqpoint{6.160000in}{6.160000in}}%
\pgfusepath{clip}%
\pgfsetbuttcap%
\pgfsetroundjoin%
\definecolor{currentfill}{rgb}{0.651398,0.768121,0.995891}%
\pgfsetfillcolor{currentfill}%
\pgfsetlinewidth{0.000000pt}%
\definecolor{currentstroke}{rgb}{0.000000,0.000000,0.000000}%
\pgfsetstrokecolor{currentstroke}%
\pgfsetdash{}{0pt}%
\pgfpathmoveto{\pgfqpoint{3.958992in}{3.549158in}}%
\pgfpathlineto{\pgfqpoint{3.969201in}{3.541210in}}%
\pgfpathlineto{\pgfqpoint{3.979425in}{3.535633in}}%
\pgfpathlineto{\pgfqpoint{4.012946in}{3.519405in}}%
\pgfpathlineto{\pgfqpoint{4.046436in}{3.502847in}}%
\pgfpathlineto{\pgfqpoint{4.036154in}{3.508584in}}%
\pgfpathlineto{\pgfqpoint{4.025889in}{3.516423in}}%
\pgfpathlineto{\pgfqpoint{3.992457in}{3.532868in}}%
\pgfpathlineto{\pgfqpoint{3.958992in}{3.549158in}}%
\pgfpathclose%
\pgfusepath{fill}%
\end{pgfscope}%
\begin{pgfscope}%
\pgfpathrectangle{\pgfqpoint{1.020000in}{0.880000in}}{\pgfqpoint{6.160000in}{6.160000in}}%
\pgfusepath{clip}%
\pgfsetbuttcap%
\pgfsetroundjoin%
\definecolor{currentfill}{rgb}{0.962708,0.753557,0.655601}%
\pgfsetfillcolor{currentfill}%
\pgfsetlinewidth{0.000000pt}%
\definecolor{currentstroke}{rgb}{0.000000,0.000000,0.000000}%
\pgfsetstrokecolor{currentstroke}%
\pgfsetdash{}{0pt}%
\pgfpathmoveto{\pgfqpoint{2.570068in}{4.311906in}}%
\pgfpathlineto{\pgfqpoint{2.579572in}{4.260335in}}%
\pgfpathlineto{\pgfqpoint{2.589097in}{4.208054in}}%
\pgfpathlineto{\pgfqpoint{2.622556in}{4.240058in}}%
\pgfpathlineto{\pgfqpoint{2.656013in}{4.272009in}}%
\pgfpathlineto{\pgfqpoint{2.646400in}{4.327856in}}%
\pgfpathlineto{\pgfqpoint{2.636807in}{4.382934in}}%
\pgfpathlineto{\pgfqpoint{2.603437in}{4.347461in}}%
\pgfpathlineto{\pgfqpoint{2.570068in}{4.311906in}}%
\pgfpathclose%
\pgfusepath{fill}%
\end{pgfscope}%
\begin{pgfscope}%
\pgfpathrectangle{\pgfqpoint{1.020000in}{0.880000in}}{\pgfqpoint{6.160000in}{6.160000in}}%
\pgfusepath{clip}%
\pgfsetbuttcap%
\pgfsetroundjoin%
\definecolor{currentfill}{rgb}{0.952761,0.782965,0.698646}%
\pgfsetfillcolor{currentfill}%
\pgfsetlinewidth{0.000000pt}%
\definecolor{currentstroke}{rgb}{0.000000,0.000000,0.000000}%
\pgfsetstrokecolor{currentstroke}%
\pgfsetdash{}{0pt}%
\pgfpathmoveto{\pgfqpoint{2.503314in}{4.241843in}}%
\pgfpathlineto{\pgfqpoint{2.512725in}{4.193770in}}%
\pgfpathlineto{\pgfqpoint{2.522157in}{4.145052in}}%
\pgfpathlineto{\pgfqpoint{2.555632in}{4.176294in}}%
\pgfpathlineto{\pgfqpoint{2.589097in}{4.208054in}}%
\pgfpathlineto{\pgfqpoint{2.579572in}{4.260335in}}%
\pgfpathlineto{\pgfqpoint{2.570068in}{4.311906in}}%
\pgfpathlineto{\pgfqpoint{2.536695in}{4.276598in}}%
\pgfpathlineto{\pgfqpoint{2.503314in}{4.241843in}}%
\pgfpathclose%
\pgfusepath{fill}%
\end{pgfscope}%
\begin{pgfscope}%
\pgfpathrectangle{\pgfqpoint{1.020000in}{0.880000in}}{\pgfqpoint{6.160000in}{6.160000in}}%
\pgfusepath{clip}%
\pgfsetbuttcap%
\pgfsetroundjoin%
\definecolor{currentfill}{rgb}{0.748682,0.827679,0.963334}%
\pgfsetfillcolor{currentfill}%
\pgfsetlinewidth{0.000000pt}%
\definecolor{currentstroke}{rgb}{0.000000,0.000000,0.000000}%
\pgfsetstrokecolor{currentstroke}%
\pgfsetdash{}{0pt}%
\pgfpathmoveto{\pgfqpoint{3.650220in}{3.742121in}}%
\pgfpathlineto{\pgfqpoint{3.660217in}{3.715961in}}%
\pgfpathlineto{\pgfqpoint{3.670214in}{3.693317in}}%
\pgfpathlineto{\pgfqpoint{3.703863in}{3.680807in}}%
\pgfpathlineto{\pgfqpoint{3.737486in}{3.667302in}}%
\pgfpathlineto{\pgfqpoint{3.727451in}{3.687671in}}%
\pgfpathlineto{\pgfqpoint{3.717419in}{3.711337in}}%
\pgfpathlineto{\pgfqpoint{3.683833in}{3.727171in}}%
\pgfpathlineto{\pgfqpoint{3.650220in}{3.742121in}}%
\pgfpathclose%
\pgfusepath{fill}%
\end{pgfscope}%
\begin{pgfscope}%
\pgfpathrectangle{\pgfqpoint{1.020000in}{0.880000in}}{\pgfqpoint{6.160000in}{6.160000in}}%
\pgfusepath{clip}%
\pgfsetbuttcap%
\pgfsetroundjoin%
\definecolor{currentfill}{rgb}{0.608547,0.735725,0.999354}%
\pgfsetfillcolor{currentfill}%
\pgfsetlinewidth{0.000000pt}%
\definecolor{currentstroke}{rgb}{0.000000,0.000000,0.000000}%
\pgfsetstrokecolor{currentstroke}%
\pgfsetdash{}{0pt}%
\pgfpathmoveto{\pgfqpoint{4.113318in}{3.469271in}}%
\pgfpathlineto{\pgfqpoint{4.123677in}{3.464814in}}%
\pgfpathlineto{\pgfqpoint{4.134056in}{3.462009in}}%
\pgfpathlineto{\pgfqpoint{4.167507in}{3.444204in}}%
\pgfpathlineto{\pgfqpoint{4.200924in}{3.426465in}}%
\pgfpathlineto{\pgfqpoint{4.190487in}{3.430467in}}%
\pgfpathlineto{\pgfqpoint{4.180070in}{3.435892in}}%
\pgfpathlineto{\pgfqpoint{4.146710in}{3.452501in}}%
\pgfpathlineto{\pgfqpoint{4.113318in}{3.469271in}}%
\pgfpathclose%
\pgfusepath{fill}%
\end{pgfscope}%
\begin{pgfscope}%
\pgfpathrectangle{\pgfqpoint{1.020000in}{0.880000in}}{\pgfqpoint{6.160000in}{6.160000in}}%
\pgfusepath{clip}%
\pgfsetbuttcap%
\pgfsetroundjoin%
\definecolor{currentfill}{rgb}{0.363461,0.484784,0.901019}%
\pgfsetfillcolor{currentfill}%
\pgfsetlinewidth{0.000000pt}%
\definecolor{currentstroke}{rgb}{0.000000,0.000000,0.000000}%
\pgfsetstrokecolor{currentstroke}%
\pgfsetdash{}{0pt}%
\pgfpathmoveto{\pgfqpoint{5.106083in}{3.028896in}}%
\pgfpathlineto{\pgfqpoint{5.117370in}{3.015029in}}%
\pgfpathlineto{\pgfqpoint{5.128674in}{3.000666in}}%
\pgfpathlineto{\pgfqpoint{5.161752in}{2.996870in}}%
\pgfpathlineto{\pgfqpoint{5.194812in}{2.993780in}}%
\pgfpathlineto{\pgfqpoint{5.183455in}{3.008212in}}%
\pgfpathlineto{\pgfqpoint{5.172115in}{3.022102in}}%
\pgfpathlineto{\pgfqpoint{5.139108in}{3.025152in}}%
\pgfpathlineto{\pgfqpoint{5.106083in}{3.028896in}}%
\pgfpathclose%
\pgfusepath{fill}%
\end{pgfscope}%
\begin{pgfscope}%
\pgfpathrectangle{\pgfqpoint{1.020000in}{0.880000in}}{\pgfqpoint{6.160000in}{6.160000in}}%
\pgfusepath{clip}%
\pgfsetbuttcap%
\pgfsetroundjoin%
\definecolor{currentfill}{rgb}{0.940879,0.805596,0.735167}%
\pgfsetfillcolor{currentfill}%
\pgfsetlinewidth{0.000000pt}%
\definecolor{currentstroke}{rgb}{0.000000,0.000000,0.000000}%
\pgfsetstrokecolor{currentstroke}%
\pgfsetdash{}{0pt}%
\pgfpathmoveto{\pgfqpoint{2.436505in}{4.175072in}}%
\pgfpathlineto{\pgfqpoint{2.445822in}{4.130373in}}%
\pgfpathlineto{\pgfqpoint{2.455159in}{4.085094in}}%
\pgfpathlineto{\pgfqpoint{2.488668in}{4.114577in}}%
\pgfpathlineto{\pgfqpoint{2.522157in}{4.145052in}}%
\pgfpathlineto{\pgfqpoint{2.512725in}{4.193770in}}%
\pgfpathlineto{\pgfqpoint{2.503314in}{4.241843in}}%
\pgfpathlineto{\pgfqpoint{2.469919in}{4.207918in}}%
\pgfpathlineto{\pgfqpoint{2.436505in}{4.175072in}}%
\pgfpathclose%
\pgfusepath{fill}%
\end{pgfscope}%
\begin{pgfscope}%
\pgfpathrectangle{\pgfqpoint{1.020000in}{0.880000in}}{\pgfqpoint{6.160000in}{6.160000in}}%
\pgfusepath{clip}%
\pgfsetbuttcap%
\pgfsetroundjoin%
\definecolor{currentfill}{rgb}{0.565182,0.699438,0.996635}%
\pgfsetfillcolor{currentfill}%
\pgfsetlinewidth{0.000000pt}%
\definecolor{currentstroke}{rgb}{0.000000,0.000000,0.000000}%
\pgfsetstrokecolor{currentstroke}%
\pgfsetdash{}{0pt}%
\pgfpathmoveto{\pgfqpoint{4.267661in}{3.391654in}}%
\pgfpathlineto{\pgfqpoint{4.278177in}{3.387427in}}%
\pgfpathlineto{\pgfqpoint{4.288716in}{3.384200in}}%
\pgfpathlineto{\pgfqpoint{4.322091in}{3.365672in}}%
\pgfpathlineto{\pgfqpoint{4.355434in}{3.347620in}}%
\pgfpathlineto{\pgfqpoint{4.344840in}{3.352580in}}%
\pgfpathlineto{\pgfqpoint{4.334268in}{3.358367in}}%
\pgfpathlineto{\pgfqpoint{4.300980in}{3.374781in}}%
\pgfpathlineto{\pgfqpoint{4.267661in}{3.391654in}}%
\pgfpathclose%
\pgfusepath{fill}%
\end{pgfscope}%
\begin{pgfscope}%
\pgfpathrectangle{\pgfqpoint{1.020000in}{0.880000in}}{\pgfqpoint{6.160000in}{6.160000in}}%
\pgfusepath{clip}%
\pgfsetbuttcap%
\pgfsetroundjoin%
\definecolor{currentfill}{rgb}{0.968894,0.679480,0.562812}%
\pgfsetfillcolor{currentfill}%
\pgfsetlinewidth{0.000000pt}%
\definecolor{currentstroke}{rgb}{0.000000,0.000000,0.000000}%
\pgfsetstrokecolor{currentstroke}%
\pgfsetdash{}{0pt}%
\pgfpathmoveto{\pgfqpoint{2.924072in}{4.486616in}}%
\pgfpathlineto{\pgfqpoint{2.933970in}{4.418919in}}%
\pgfpathlineto{\pgfqpoint{2.943874in}{4.351305in}}%
\pgfpathlineto{\pgfqpoint{2.977513in}{4.366572in}}%
\pgfpathlineto{\pgfqpoint{3.011170in}{4.379060in}}%
\pgfpathlineto{\pgfqpoint{3.001226in}{4.448053in}}%
\pgfpathlineto{\pgfqpoint{2.991288in}{4.517180in}}%
\pgfpathlineto{\pgfqpoint{2.957669in}{4.503476in}}%
\pgfpathlineto{\pgfqpoint{2.924072in}{4.486616in}}%
\pgfpathclose%
\pgfusepath{fill}%
\end{pgfscope}%
\begin{pgfscope}%
\pgfpathrectangle{\pgfqpoint{1.020000in}{0.880000in}}{\pgfqpoint{6.160000in}{6.160000in}}%
\pgfusepath{clip}%
\pgfsetbuttcap%
\pgfsetroundjoin%
\definecolor{currentfill}{rgb}{0.947345,0.794696,0.716991}%
\pgfsetfillcolor{currentfill}%
\pgfsetlinewidth{0.000000pt}%
\definecolor{currentstroke}{rgb}{0.000000,0.000000,0.000000}%
\pgfsetstrokecolor{currentstroke}%
\pgfsetdash{}{0pt}%
\pgfpathmoveto{\pgfqpoint{3.165931in}{4.262146in}}%
\pgfpathlineto{\pgfqpoint{3.175911in}{4.196519in}}%
\pgfpathlineto{\pgfqpoint{3.185881in}{4.133008in}}%
\pgfpathlineto{\pgfqpoint{3.219631in}{4.131619in}}%
\pgfpathlineto{\pgfqpoint{3.253380in}{4.127687in}}%
\pgfpathlineto{\pgfqpoint{3.243398in}{4.189876in}}%
\pgfpathlineto{\pgfqpoint{3.233406in}{4.254244in}}%
\pgfpathlineto{\pgfqpoint{3.199667in}{4.259656in}}%
\pgfpathlineto{\pgfqpoint{3.165931in}{4.262146in}}%
\pgfpathclose%
\pgfusepath{fill}%
\end{pgfscope}%
\begin{pgfscope}%
\pgfpathrectangle{\pgfqpoint{1.020000in}{0.880000in}}{\pgfqpoint{6.160000in}{6.160000in}}%
\pgfusepath{clip}%
\pgfsetbuttcap%
\pgfsetroundjoin%
\definecolor{currentfill}{rgb}{0.516260,0.654498,0.986407}%
\pgfsetfillcolor{currentfill}%
\pgfsetlinewidth{0.000000pt}%
\definecolor{currentstroke}{rgb}{0.000000,0.000000,0.000000}%
\pgfsetstrokecolor{currentstroke}%
\pgfsetdash{}{0pt}%
\pgfpathmoveto{\pgfqpoint{4.422023in}{3.313252in}}%
\pgfpathlineto{\pgfqpoint{4.432693in}{3.307226in}}%
\pgfpathlineto{\pgfqpoint{4.443386in}{3.301675in}}%
\pgfpathlineto{\pgfqpoint{4.476686in}{3.283756in}}%
\pgfpathlineto{\pgfqpoint{4.509955in}{3.266648in}}%
\pgfpathlineto{\pgfqpoint{4.499211in}{3.273905in}}%
\pgfpathlineto{\pgfqpoint{4.488490in}{3.281511in}}%
\pgfpathlineto{\pgfqpoint{4.455271in}{3.297040in}}%
\pgfpathlineto{\pgfqpoint{4.422023in}{3.313252in}}%
\pgfpathclose%
\pgfusepath{fill}%
\end{pgfscope}%
\begin{pgfscope}%
\pgfpathrectangle{\pgfqpoint{1.020000in}{0.880000in}}{\pgfqpoint{6.160000in}{6.160000in}}%
\pgfusepath{clip}%
\pgfsetbuttcap%
\pgfsetroundjoin%
\definecolor{currentfill}{rgb}{0.922681,0.828568,0.777054}%
\pgfsetfillcolor{currentfill}%
\pgfsetlinewidth{0.000000pt}%
\definecolor{currentstroke}{rgb}{0.000000,0.000000,0.000000}%
\pgfsetstrokecolor{currentstroke}%
\pgfsetdash{}{0pt}%
\pgfpathmoveto{\pgfqpoint{2.369606in}{4.113463in}}%
\pgfpathlineto{\pgfqpoint{2.378827in}{4.071917in}}%
\pgfpathlineto{\pgfqpoint{2.388069in}{4.029856in}}%
\pgfpathlineto{\pgfqpoint{2.421627in}{4.056798in}}%
\pgfpathlineto{\pgfqpoint{2.455159in}{4.085094in}}%
\pgfpathlineto{\pgfqpoint{2.445822in}{4.130373in}}%
\pgfpathlineto{\pgfqpoint{2.436505in}{4.175072in}}%
\pgfpathlineto{\pgfqpoint{2.403069in}{4.143525in}}%
\pgfpathlineto{\pgfqpoint{2.369606in}{4.113463in}}%
\pgfpathclose%
\pgfusepath{fill}%
\end{pgfscope}%
\begin{pgfscope}%
\pgfpathrectangle{\pgfqpoint{1.020000in}{0.880000in}}{\pgfqpoint{6.160000in}{6.160000in}}%
\pgfusepath{clip}%
\pgfsetbuttcap%
\pgfsetroundjoin%
\definecolor{currentfill}{rgb}{0.473070,0.611077,0.970634}%
\pgfsetfillcolor{currentfill}%
\pgfsetlinewidth{0.000000pt}%
\definecolor{currentstroke}{rgb}{0.000000,0.000000,0.000000}%
\pgfsetstrokecolor{currentstroke}%
\pgfsetdash{}{0pt}%
\pgfpathmoveto{\pgfqpoint{4.576405in}{3.234953in}}%
\pgfpathlineto{\pgfqpoint{4.587221in}{3.226439in}}%
\pgfpathlineto{\pgfqpoint{4.598059in}{3.218002in}}%
\pgfpathlineto{\pgfqpoint{4.631291in}{3.202113in}}%
\pgfpathlineto{\pgfqpoint{4.664496in}{3.187236in}}%
\pgfpathlineto{\pgfqpoint{4.653609in}{3.196932in}}%
\pgfpathlineto{\pgfqpoint{4.642743in}{3.206586in}}%
\pgfpathlineto{\pgfqpoint{4.609588in}{3.220364in}}%
\pgfpathlineto{\pgfqpoint{4.576405in}{3.234953in}}%
\pgfpathclose%
\pgfusepath{fill}%
\end{pgfscope}%
\begin{pgfscope}%
\pgfpathrectangle{\pgfqpoint{1.020000in}{0.880000in}}{\pgfqpoint{6.160000in}{6.160000in}}%
\pgfusepath{clip}%
\pgfsetbuttcap%
\pgfsetroundjoin%
\definecolor{currentfill}{rgb}{0.304174,0.406945,0.845263}%
\pgfsetfillcolor{currentfill}%
\pgfsetlinewidth{0.000000pt}%
\definecolor{currentstroke}{rgb}{0.000000,0.000000,0.000000}%
\pgfsetstrokecolor{currentstroke}%
\pgfsetdash{}{0pt}%
\pgfpathmoveto{\pgfqpoint{5.923857in}{2.912003in}}%
\pgfpathlineto{\pgfqpoint{5.935885in}{2.896710in}}%
\pgfpathlineto{\pgfqpoint{5.947935in}{2.881460in}}%
\pgfpathlineto{\pgfqpoint{5.980813in}{2.883479in}}%
\pgfpathlineto{\pgfqpoint{6.013670in}{2.885528in}}%
\pgfpathlineto{\pgfqpoint{6.001565in}{2.900691in}}%
\pgfpathlineto{\pgfqpoint{5.989483in}{2.915897in}}%
\pgfpathlineto{\pgfqpoint{5.956680in}{2.913936in}}%
\pgfpathlineto{\pgfqpoint{5.923857in}{2.912003in}}%
\pgfpathclose%
\pgfusepath{fill}%
\end{pgfscope}%
\begin{pgfscope}%
\pgfpathrectangle{\pgfqpoint{1.020000in}{0.880000in}}{\pgfqpoint{6.160000in}{6.160000in}}%
\pgfusepath{clip}%
\pgfsetbuttcap%
\pgfsetroundjoin%
\definecolor{currentfill}{rgb}{0.313946,0.420052,0.854993}%
\pgfsetfillcolor{currentfill}%
\pgfsetlinewidth{0.000000pt}%
\definecolor{currentstroke}{rgb}{0.000000,0.000000,0.000000}%
\pgfsetstrokecolor{currentstroke}%
\pgfsetdash{}{0pt}%
\pgfpathmoveto{\pgfqpoint{5.702837in}{2.932531in}}%
\pgfpathlineto{\pgfqpoint{5.714657in}{2.916949in}}%
\pgfpathlineto{\pgfqpoint{5.726501in}{2.901383in}}%
\pgfpathlineto{\pgfqpoint{5.759443in}{2.902994in}}%
\pgfpathlineto{\pgfqpoint{5.792365in}{2.904683in}}%
\pgfpathlineto{\pgfqpoint{5.780468in}{2.920176in}}%
\pgfpathlineto{\pgfqpoint{5.768594in}{2.935696in}}%
\pgfpathlineto{\pgfqpoint{5.735725in}{2.934076in}}%
\pgfpathlineto{\pgfqpoint{5.702837in}{2.932531in}}%
\pgfpathclose%
\pgfusepath{fill}%
\end{pgfscope}%
\begin{pgfscope}%
\pgfpathrectangle{\pgfqpoint{1.020000in}{0.880000in}}{\pgfqpoint{6.160000in}{6.160000in}}%
\pgfusepath{clip}%
\pgfsetbuttcap%
\pgfsetroundjoin%
\definecolor{currentfill}{rgb}{0.804965,0.851666,0.926165}%
\pgfsetfillcolor{currentfill}%
\pgfsetlinewidth{0.000000pt}%
\definecolor{currentstroke}{rgb}{0.000000,0.000000,0.000000}%
\pgfsetstrokecolor{currentstroke}%
\pgfsetdash{}{0pt}%
\pgfpathmoveto{\pgfqpoint{3.495617in}{3.864802in}}%
\pgfpathlineto{\pgfqpoint{3.505575in}{3.825812in}}%
\pgfpathlineto{\pgfqpoint{3.515522in}{3.790498in}}%
\pgfpathlineto{\pgfqpoint{3.549230in}{3.780356in}}%
\pgfpathlineto{\pgfqpoint{3.582916in}{3.768838in}}%
\pgfpathlineto{\pgfqpoint{3.572944in}{3.801298in}}%
\pgfpathlineto{\pgfqpoint{3.562967in}{3.837305in}}%
\pgfpathlineto{\pgfqpoint{3.529303in}{3.851757in}}%
\pgfpathlineto{\pgfqpoint{3.495617in}{3.864802in}}%
\pgfpathclose%
\pgfusepath{fill}%
\end{pgfscope}%
\begin{pgfscope}%
\pgfpathrectangle{\pgfqpoint{1.020000in}{0.880000in}}{\pgfqpoint{6.160000in}{6.160000in}}%
\pgfusepath{clip}%
\pgfsetbuttcap%
\pgfsetroundjoin%
\definecolor{currentfill}{rgb}{0.430507,0.564883,0.948889}%
\pgfsetfillcolor{currentfill}%
\pgfsetlinewidth{0.000000pt}%
\definecolor{currentstroke}{rgb}{0.000000,0.000000,0.000000}%
\pgfsetstrokecolor{currentstroke}%
\pgfsetdash{}{0pt}%
\pgfpathmoveto{\pgfqpoint{4.730826in}{3.160428in}}%
\pgfpathlineto{\pgfqpoint{4.741783in}{3.149679in}}%
\pgfpathlineto{\pgfqpoint{4.752760in}{3.138710in}}%
\pgfpathlineto{\pgfqpoint{4.785938in}{3.125985in}}%
\pgfpathlineto{\pgfqpoint{4.819092in}{3.114337in}}%
\pgfpathlineto{\pgfqpoint{4.808065in}{3.126000in}}%
\pgfpathlineto{\pgfqpoint{4.797057in}{3.137309in}}%
\pgfpathlineto{\pgfqpoint{4.763954in}{3.148428in}}%
\pgfpathlineto{\pgfqpoint{4.730826in}{3.160428in}}%
\pgfpathclose%
\pgfusepath{fill}%
\end{pgfscope}%
\begin{pgfscope}%
\pgfpathrectangle{\pgfqpoint{1.020000in}{0.880000in}}{\pgfqpoint{6.160000in}{6.160000in}}%
\pgfusepath{clip}%
\pgfsetbuttcap%
\pgfsetroundjoin%
\definecolor{currentfill}{rgb}{0.909460,0.839386,0.800331}%
\pgfsetfillcolor{currentfill}%
\pgfsetlinewidth{0.000000pt}%
\definecolor{currentstroke}{rgb}{0.000000,0.000000,0.000000}%
\pgfsetstrokecolor{currentstroke}%
\pgfsetdash{}{0pt}%
\pgfpathmoveto{\pgfqpoint{3.253380in}{4.127687in}}%
\pgfpathlineto{\pgfqpoint{3.263349in}{4.068053in}}%
\pgfpathlineto{\pgfqpoint{3.273304in}{4.011308in}}%
\pgfpathlineto{\pgfqpoint{3.307061in}{4.006775in}}%
\pgfpathlineto{\pgfqpoint{3.340810in}{4.000080in}}%
\pgfpathlineto{\pgfqpoint{3.330844in}{4.054738in}}%
\pgfpathlineto{\pgfqpoint{3.320865in}{4.112313in}}%
\pgfpathlineto{\pgfqpoint{3.287126in}{4.121234in}}%
\pgfpathlineto{\pgfqpoint{3.253380in}{4.127687in}}%
\pgfpathclose%
\pgfusepath{fill}%
\end{pgfscope}%
\begin{pgfscope}%
\pgfpathrectangle{\pgfqpoint{1.020000in}{0.880000in}}{\pgfqpoint{6.160000in}{6.160000in}}%
\pgfusepath{clip}%
\pgfsetbuttcap%
\pgfsetroundjoin%
\definecolor{currentfill}{rgb}{0.328604,0.439712,0.869587}%
\pgfsetfillcolor{currentfill}%
\pgfsetlinewidth{0.000000pt}%
\definecolor{currentstroke}{rgb}{0.000000,0.000000,0.000000}%
\pgfsetstrokecolor{currentstroke}%
\pgfsetdash{}{0pt}%
\pgfpathmoveto{\pgfqpoint{5.481839in}{2.957031in}}%
\pgfpathlineto{\pgfqpoint{5.493459in}{2.941478in}}%
\pgfpathlineto{\pgfqpoint{5.505099in}{2.925826in}}%
\pgfpathlineto{\pgfqpoint{5.538103in}{2.926520in}}%
\pgfpathlineto{\pgfqpoint{5.571088in}{2.927421in}}%
\pgfpathlineto{\pgfqpoint{5.559395in}{2.943062in}}%
\pgfpathlineto{\pgfqpoint{5.547722in}{2.958632in}}%
\pgfpathlineto{\pgfqpoint{5.514790in}{2.957731in}}%
\pgfpathlineto{\pgfqpoint{5.481839in}{2.957031in}}%
\pgfpathclose%
\pgfusepath{fill}%
\end{pgfscope}%
\begin{pgfscope}%
\pgfpathrectangle{\pgfqpoint{1.020000in}{0.880000in}}{\pgfqpoint{6.160000in}{6.160000in}}%
\pgfusepath{clip}%
\pgfsetbuttcap%
\pgfsetroundjoin%
\definecolor{currentfill}{rgb}{0.969851,0.695830,0.581312}%
\pgfsetfillcolor{currentfill}%
\pgfsetlinewidth{0.000000pt}%
\definecolor{currentstroke}{rgb}{0.000000,0.000000,0.000000}%
\pgfsetstrokecolor{currentstroke}%
\pgfsetdash{}{0pt}%
\pgfpathmoveto{\pgfqpoint{2.856949in}{4.444424in}}%
\pgfpathlineto{\pgfqpoint{2.866795in}{4.378857in}}%
\pgfpathlineto{\pgfqpoint{2.876648in}{4.313324in}}%
\pgfpathlineto{\pgfqpoint{2.910252in}{4.333475in}}%
\pgfpathlineto{\pgfqpoint{2.943874in}{4.351305in}}%
\pgfpathlineto{\pgfqpoint{2.933970in}{4.418919in}}%
\pgfpathlineto{\pgfqpoint{2.924072in}{4.486616in}}%
\pgfpathlineto{\pgfqpoint{2.890500in}{4.466842in}}%
\pgfpathlineto{\pgfqpoint{2.856949in}{4.444424in}}%
\pgfpathclose%
\pgfusepath{fill}%
\end{pgfscope}%
\begin{pgfscope}%
\pgfpathrectangle{\pgfqpoint{1.020000in}{0.880000in}}{\pgfqpoint{6.160000in}{6.160000in}}%
\pgfusepath{clip}%
\pgfsetbuttcap%
\pgfsetroundjoin%
\definecolor{currentfill}{rgb}{0.966962,0.735670,0.630877}%
\pgfsetfillcolor{currentfill}%
\pgfsetlinewidth{0.000000pt}%
\definecolor{currentstroke}{rgb}{0.000000,0.000000,0.000000}%
\pgfsetstrokecolor{currentstroke}%
\pgfsetdash{}{0pt}%
\pgfpathmoveto{\pgfqpoint{3.011170in}{4.379060in}}%
\pgfpathlineto{\pgfqpoint{3.021115in}{4.310691in}}%
\pgfpathlineto{\pgfqpoint{3.031056in}{4.243407in}}%
\pgfpathlineto{\pgfqpoint{3.064760in}{4.252255in}}%
\pgfpathlineto{\pgfqpoint{3.098475in}{4.258392in}}%
\pgfpathlineto{\pgfqpoint{3.088506in}{4.326106in}}%
\pgfpathlineto{\pgfqpoint{3.078533in}{4.394992in}}%
\pgfpathlineto{\pgfqpoint{3.044843in}{4.388584in}}%
\pgfpathlineto{\pgfqpoint{3.011170in}{4.379060in}}%
\pgfpathclose%
\pgfusepath{fill}%
\end{pgfscope}%
\begin{pgfscope}%
\pgfpathrectangle{\pgfqpoint{1.020000in}{0.880000in}}{\pgfqpoint{6.160000in}{6.160000in}}%
\pgfusepath{clip}%
\pgfsetbuttcap%
\pgfsetroundjoin%
\definecolor{currentfill}{rgb}{0.343278,0.459354,0.884122}%
\pgfsetfillcolor{currentfill}%
\pgfsetlinewidth{0.000000pt}%
\definecolor{currentstroke}{rgb}{0.000000,0.000000,0.000000}%
\pgfsetstrokecolor{currentstroke}%
\pgfsetdash{}{0pt}%
\pgfpathmoveto{\pgfqpoint{5.260879in}{2.989475in}}%
\pgfpathlineto{\pgfqpoint{5.272305in}{2.974559in}}%
\pgfpathlineto{\pgfqpoint{5.283749in}{2.959329in}}%
\pgfpathlineto{\pgfqpoint{5.316808in}{2.957937in}}%
\pgfpathlineto{\pgfqpoint{5.349850in}{2.957018in}}%
\pgfpathlineto{\pgfqpoint{5.338354in}{2.972304in}}%
\pgfpathlineto{\pgfqpoint{5.326876in}{2.987289in}}%
\pgfpathlineto{\pgfqpoint{5.293887in}{2.988144in}}%
\pgfpathlineto{\pgfqpoint{5.260879in}{2.989475in}}%
\pgfpathclose%
\pgfusepath{fill}%
\end{pgfscope}%
\begin{pgfscope}%
\pgfpathrectangle{\pgfqpoint{1.020000in}{0.880000in}}{\pgfqpoint{6.160000in}{6.160000in}}%
\pgfusepath{clip}%
\pgfsetbuttcap%
\pgfsetroundjoin%
\definecolor{currentfill}{rgb}{0.399231,0.528528,0.928459}%
\pgfsetfillcolor{currentfill}%
\pgfsetlinewidth{0.000000pt}%
\definecolor{currentstroke}{rgb}{0.000000,0.000000,0.000000}%
\pgfsetstrokecolor{currentstroke}%
\pgfsetdash{}{0pt}%
\pgfpathmoveto{\pgfqpoint{4.885332in}{3.094091in}}%
\pgfpathlineto{\pgfqpoint{4.896428in}{3.081634in}}%
\pgfpathlineto{\pgfqpoint{4.907542in}{3.068769in}}%
\pgfpathlineto{\pgfqpoint{4.940681in}{3.059778in}}%
\pgfpathlineto{\pgfqpoint{4.973800in}{3.051801in}}%
\pgfpathlineto{\pgfqpoint{4.962634in}{3.064962in}}%
\pgfpathlineto{\pgfqpoint{4.951486in}{3.077596in}}%
\pgfpathlineto{\pgfqpoint{4.918419in}{3.085397in}}%
\pgfpathlineto{\pgfqpoint{4.885332in}{3.094091in}}%
\pgfpathclose%
\pgfusepath{fill}%
\end{pgfscope}%
\begin{pgfscope}%
\pgfpathrectangle{\pgfqpoint{1.020000in}{0.880000in}}{\pgfqpoint{6.160000in}{6.160000in}}%
\pgfusepath{clip}%
\pgfsetbuttcap%
\pgfsetroundjoin%
\definecolor{currentfill}{rgb}{0.863392,0.865084,0.867634}%
\pgfsetfillcolor{currentfill}%
\pgfsetlinewidth{0.000000pt}%
\definecolor{currentstroke}{rgb}{0.000000,0.000000,0.000000}%
\pgfsetstrokecolor{currentstroke}%
\pgfsetdash{}{0pt}%
\pgfpathmoveto{\pgfqpoint{3.340810in}{4.000080in}}%
\pgfpathlineto{\pgfqpoint{3.350761in}{3.948624in}}%
\pgfpathlineto{\pgfqpoint{3.360698in}{3.900610in}}%
\pgfpathlineto{\pgfqpoint{3.394450in}{3.894301in}}%
\pgfpathlineto{\pgfqpoint{3.428189in}{3.886176in}}%
\pgfpathlineto{\pgfqpoint{3.418237in}{3.931608in}}%
\pgfpathlineto{\pgfqpoint{3.408274in}{3.980459in}}%
\pgfpathlineto{\pgfqpoint{3.374548in}{3.991280in}}%
\pgfpathlineto{\pgfqpoint{3.340810in}{4.000080in}}%
\pgfpathclose%
\pgfusepath{fill}%
\end{pgfscope}%
\begin{pgfscope}%
\pgfpathrectangle{\pgfqpoint{1.020000in}{0.880000in}}{\pgfqpoint{6.160000in}{6.160000in}}%
\pgfusepath{clip}%
\pgfsetbuttcap%
\pgfsetroundjoin%
\definecolor{currentfill}{rgb}{0.968533,0.715841,0.606097}%
\pgfsetfillcolor{currentfill}%
\pgfsetlinewidth{0.000000pt}%
\definecolor{currentstroke}{rgb}{0.000000,0.000000,0.000000}%
\pgfsetstrokecolor{currentstroke}%
\pgfsetdash{}{0pt}%
\pgfpathmoveto{\pgfqpoint{2.789911in}{4.392860in}}%
\pgfpathlineto{\pgfqpoint{2.799694in}{4.330003in}}%
\pgfpathlineto{\pgfqpoint{2.809485in}{4.267139in}}%
\pgfpathlineto{\pgfqpoint{2.843059in}{4.291118in}}%
\pgfpathlineto{\pgfqpoint{2.876648in}{4.313324in}}%
\pgfpathlineto{\pgfqpoint{2.866795in}{4.378857in}}%
\pgfpathlineto{\pgfqpoint{2.856949in}{4.444424in}}%
\pgfpathlineto{\pgfqpoint{2.823421in}{4.419659in}}%
\pgfpathlineto{\pgfqpoint{2.789911in}{4.392860in}}%
\pgfpathclose%
\pgfusepath{fill}%
\end{pgfscope}%
\begin{pgfscope}%
\pgfpathrectangle{\pgfqpoint{1.020000in}{0.880000in}}{\pgfqpoint{6.160000in}{6.160000in}}%
\pgfusepath{clip}%
\pgfsetbuttcap%
\pgfsetroundjoin%
\definecolor{currentfill}{rgb}{0.949151,0.790785,0.710876}%
\pgfsetfillcolor{currentfill}%
\pgfsetlinewidth{0.000000pt}%
\definecolor{currentstroke}{rgb}{0.000000,0.000000,0.000000}%
\pgfsetstrokecolor{currentstroke}%
\pgfsetdash{}{0pt}%
\pgfpathmoveto{\pgfqpoint{3.098475in}{4.258392in}}%
\pgfpathlineto{\pgfqpoint{3.108436in}{4.192296in}}%
\pgfpathlineto{\pgfqpoint{3.118386in}{4.128222in}}%
\pgfpathlineto{\pgfqpoint{3.152132in}{4.131862in}}%
\pgfpathlineto{\pgfqpoint{3.185881in}{4.133008in}}%
\pgfpathlineto{\pgfqpoint{3.175911in}{4.196519in}}%
\pgfpathlineto{\pgfqpoint{3.165931in}{4.262146in}}%
\pgfpathlineto{\pgfqpoint{3.132200in}{4.261713in}}%
\pgfpathlineto{\pgfqpoint{3.098475in}{4.258392in}}%
\pgfpathclose%
\pgfusepath{fill}%
\end{pgfscope}%
\begin{pgfscope}%
\pgfpathrectangle{\pgfqpoint{1.020000in}{0.880000in}}{\pgfqpoint{6.160000in}{6.160000in}}%
\pgfusepath{clip}%
\pgfsetbuttcap%
\pgfsetroundjoin%
\definecolor{currentfill}{rgb}{0.373552,0.497499,0.909467}%
\pgfsetfillcolor{currentfill}%
\pgfsetlinewidth{0.000000pt}%
\definecolor{currentstroke}{rgb}{0.000000,0.000000,0.000000}%
\pgfsetstrokecolor{currentstroke}%
\pgfsetdash{}{0pt}%
\pgfpathmoveto{\pgfqpoint{5.039979in}{3.038658in}}%
\pgfpathlineto{\pgfqpoint{5.051213in}{3.024863in}}%
\pgfpathlineto{\pgfqpoint{5.062466in}{3.010633in}}%
\pgfpathlineto{\pgfqpoint{5.095579in}{3.005232in}}%
\pgfpathlineto{\pgfqpoint{5.128674in}{3.000666in}}%
\pgfpathlineto{\pgfqpoint{5.117370in}{3.015029in}}%
\pgfpathlineto{\pgfqpoint{5.106083in}{3.028896in}}%
\pgfpathlineto{\pgfqpoint{5.073040in}{3.033381in}}%
\pgfpathlineto{\pgfqpoint{5.039979in}{3.038658in}}%
\pgfpathclose%
\pgfusepath{fill}%
\end{pgfscope}%
\begin{pgfscope}%
\pgfpathrectangle{\pgfqpoint{1.020000in}{0.880000in}}{\pgfqpoint{6.160000in}{6.160000in}}%
\pgfusepath{clip}%
\pgfsetbuttcap%
\pgfsetroundjoin%
\definecolor{currentfill}{rgb}{0.304174,0.406945,0.845263}%
\pgfsetfillcolor{currentfill}%
\pgfsetlinewidth{0.000000pt}%
\definecolor{currentstroke}{rgb}{0.000000,0.000000,0.000000}%
\pgfsetstrokecolor{currentstroke}%
\pgfsetdash{}{0pt}%
\pgfpathmoveto{\pgfqpoint{5.858151in}{2.908248in}}%
\pgfpathlineto{\pgfqpoint{5.870125in}{2.892869in}}%
\pgfpathlineto{\pgfqpoint{5.882121in}{2.877532in}}%
\pgfpathlineto{\pgfqpoint{5.915038in}{2.879475in}}%
\pgfpathlineto{\pgfqpoint{5.947935in}{2.881460in}}%
\pgfpathlineto{\pgfqpoint{5.935885in}{2.896710in}}%
\pgfpathlineto{\pgfqpoint{5.923857in}{2.912003in}}%
\pgfpathlineto{\pgfqpoint{5.891014in}{2.910105in}}%
\pgfpathlineto{\pgfqpoint{5.858151in}{2.908248in}}%
\pgfpathclose%
\pgfusepath{fill}%
\end{pgfscope}%
\begin{pgfscope}%
\pgfpathrectangle{\pgfqpoint{1.020000in}{0.880000in}}{\pgfqpoint{6.160000in}{6.160000in}}%
\pgfusepath{clip}%
\pgfsetbuttcap%
\pgfsetroundjoin%
\definecolor{currentfill}{rgb}{0.964835,0.744614,0.643239}%
\pgfsetfillcolor{currentfill}%
\pgfsetlinewidth{0.000000pt}%
\definecolor{currentstroke}{rgb}{0.000000,0.000000,0.000000}%
\pgfsetstrokecolor{currentstroke}%
\pgfsetdash{}{0pt}%
\pgfpathmoveto{\pgfqpoint{2.722941in}{4.334491in}}%
\pgfpathlineto{\pgfqpoint{2.732651in}{4.274786in}}%
\pgfpathlineto{\pgfqpoint{2.742370in}{4.215037in}}%
\pgfpathlineto{\pgfqpoint{2.775922in}{4.241678in}}%
\pgfpathlineto{\pgfqpoint{2.809485in}{4.267139in}}%
\pgfpathlineto{\pgfqpoint{2.799694in}{4.330003in}}%
\pgfpathlineto{\pgfqpoint{2.789911in}{4.392860in}}%
\pgfpathlineto{\pgfqpoint{2.756419in}{4.364359in}}%
\pgfpathlineto{\pgfqpoint{2.722941in}{4.334491in}}%
\pgfpathclose%
\pgfusepath{fill}%
\end{pgfscope}%
\begin{pgfscope}%
\pgfpathrectangle{\pgfqpoint{1.020000in}{0.880000in}}{\pgfqpoint{6.160000in}{6.160000in}}%
\pgfusepath{clip}%
\pgfsetbuttcap%
\pgfsetroundjoin%
\definecolor{currentfill}{rgb}{0.718985,0.811993,0.977656}%
\pgfsetfillcolor{currentfill}%
\pgfsetlinewidth{0.000000pt}%
\definecolor{currentstroke}{rgb}{0.000000,0.000000,0.000000}%
\pgfsetstrokecolor{currentstroke}%
\pgfsetdash{}{0pt}%
\pgfpathmoveto{\pgfqpoint{3.737486in}{3.667302in}}%
\pgfpathlineto{\pgfqpoint{3.747525in}{3.650242in}}%
\pgfpathlineto{\pgfqpoint{3.757568in}{3.636466in}}%
\pgfpathlineto{\pgfqpoint{3.791210in}{3.623799in}}%
\pgfpathlineto{\pgfqpoint{3.824825in}{3.610210in}}%
\pgfpathlineto{\pgfqpoint{3.814733in}{3.622483in}}%
\pgfpathlineto{\pgfqpoint{3.804650in}{3.637794in}}%
\pgfpathlineto{\pgfqpoint{3.771082in}{3.652921in}}%
\pgfpathlineto{\pgfqpoint{3.737486in}{3.667302in}}%
\pgfpathclose%
\pgfusepath{fill}%
\end{pgfscope}%
\begin{pgfscope}%
\pgfpathrectangle{\pgfqpoint{1.020000in}{0.880000in}}{\pgfqpoint{6.160000in}{6.160000in}}%
\pgfusepath{clip}%
\pgfsetbuttcap%
\pgfsetroundjoin%
\definecolor{currentfill}{rgb}{0.318832,0.426605,0.859857}%
\pgfsetfillcolor{currentfill}%
\pgfsetlinewidth{0.000000pt}%
\definecolor{currentstroke}{rgb}{0.000000,0.000000,0.000000}%
\pgfsetstrokecolor{currentstroke}%
\pgfsetdash{}{0pt}%
\pgfpathmoveto{\pgfqpoint{5.637001in}{2.929724in}}%
\pgfpathlineto{\pgfqpoint{5.648768in}{2.914089in}}%
\pgfpathlineto{\pgfqpoint{5.660558in}{2.898457in}}%
\pgfpathlineto{\pgfqpoint{5.693539in}{2.899865in}}%
\pgfpathlineto{\pgfqpoint{5.726501in}{2.901383in}}%
\pgfpathlineto{\pgfqpoint{5.714657in}{2.916949in}}%
\pgfpathlineto{\pgfqpoint{5.702837in}{2.932531in}}%
\pgfpathlineto{\pgfqpoint{5.669928in}{2.931074in}}%
\pgfpathlineto{\pgfqpoint{5.637001in}{2.929724in}}%
\pgfpathclose%
\pgfusepath{fill}%
\end{pgfscope}%
\begin{pgfscope}%
\pgfpathrectangle{\pgfqpoint{1.020000in}{0.880000in}}{\pgfqpoint{6.160000in}{6.160000in}}%
\pgfusepath{clip}%
\pgfsetbuttcap%
\pgfsetroundjoin%
\definecolor{currentfill}{rgb}{0.677823,0.786546,0.991005}%
\pgfsetfillcolor{currentfill}%
\pgfsetlinewidth{0.000000pt}%
\definecolor{currentstroke}{rgb}{0.000000,0.000000,0.000000}%
\pgfsetstrokecolor{currentstroke}%
\pgfsetdash{}{0pt}%
\pgfpathmoveto{\pgfqpoint{3.891968in}{3.580765in}}%
\pgfpathlineto{\pgfqpoint{3.902123in}{3.572341in}}%
\pgfpathlineto{\pgfqpoint{3.912290in}{3.566553in}}%
\pgfpathlineto{\pgfqpoint{3.945873in}{3.551395in}}%
\pgfpathlineto{\pgfqpoint{3.979425in}{3.535633in}}%
\pgfpathlineto{\pgfqpoint{3.969201in}{3.541210in}}%
\pgfpathlineto{\pgfqpoint{3.958992in}{3.549158in}}%
\pgfpathlineto{\pgfqpoint{3.925496in}{3.565166in}}%
\pgfpathlineto{\pgfqpoint{3.891968in}{3.580765in}}%
\pgfpathclose%
\pgfusepath{fill}%
\end{pgfscope}%
\begin{pgfscope}%
\pgfpathrectangle{\pgfqpoint{1.020000in}{0.880000in}}{\pgfqpoint{6.160000in}{6.160000in}}%
\pgfusepath{clip}%
\pgfsetbuttcap%
\pgfsetroundjoin%
\definecolor{currentfill}{rgb}{0.768034,0.837035,0.952488}%
\pgfsetfillcolor{currentfill}%
\pgfsetlinewidth{0.000000pt}%
\definecolor{currentstroke}{rgb}{0.000000,0.000000,0.000000}%
\pgfsetstrokecolor{currentstroke}%
\pgfsetdash{}{0pt}%
\pgfpathmoveto{\pgfqpoint{3.582916in}{3.768838in}}%
\pgfpathlineto{\pgfqpoint{3.592883in}{3.740020in}}%
\pgfpathlineto{\pgfqpoint{3.602845in}{3.714897in}}%
\pgfpathlineto{\pgfqpoint{3.636541in}{3.704715in}}%
\pgfpathlineto{\pgfqpoint{3.670214in}{3.693317in}}%
\pgfpathlineto{\pgfqpoint{3.660217in}{3.715961in}}%
\pgfpathlineto{\pgfqpoint{3.650220in}{3.742121in}}%
\pgfpathlineto{\pgfqpoint{3.616580in}{3.756053in}}%
\pgfpathlineto{\pgfqpoint{3.582916in}{3.768838in}}%
\pgfpathclose%
\pgfusepath{fill}%
\end{pgfscope}%
\begin{pgfscope}%
\pgfpathrectangle{\pgfqpoint{1.020000in}{0.880000in}}{\pgfqpoint{6.160000in}{6.160000in}}%
\pgfusepath{clip}%
\pgfsetbuttcap%
\pgfsetroundjoin%
\definecolor{currentfill}{rgb}{0.964835,0.744614,0.643239}%
\pgfsetfillcolor{currentfill}%
\pgfsetlinewidth{0.000000pt}%
\definecolor{currentstroke}{rgb}{0.000000,0.000000,0.000000}%
\pgfsetstrokecolor{currentstroke}%
\pgfsetdash{}{0pt}%
\pgfpathmoveto{\pgfqpoint{2.943874in}{4.351305in}}%
\pgfpathlineto{\pgfqpoint{2.953780in}{4.284238in}}%
\pgfpathlineto{\pgfqpoint{2.963682in}{4.218157in}}%
\pgfpathlineto{\pgfqpoint{2.997363in}{4.231985in}}%
\pgfpathlineto{\pgfqpoint{3.031056in}{4.243407in}}%
\pgfpathlineto{\pgfqpoint{3.021115in}{4.310691in}}%
\pgfpathlineto{\pgfqpoint{3.011170in}{4.379060in}}%
\pgfpathlineto{\pgfqpoint{2.977513in}{4.366572in}}%
\pgfpathlineto{\pgfqpoint{2.943874in}{4.351305in}}%
\pgfpathclose%
\pgfusepath{fill}%
\end{pgfscope}%
\begin{pgfscope}%
\pgfpathrectangle{\pgfqpoint{1.020000in}{0.880000in}}{\pgfqpoint{6.160000in}{6.160000in}}%
\pgfusepath{clip}%
\pgfsetbuttcap%
\pgfsetroundjoin%
\definecolor{currentfill}{rgb}{0.294718,0.393542,0.834384}%
\pgfsetfillcolor{currentfill}%
\pgfsetlinewidth{0.000000pt}%
\definecolor{currentstroke}{rgb}{0.000000,0.000000,0.000000}%
\pgfsetstrokecolor{currentstroke}%
\pgfsetdash{}{0pt}%
\pgfpathmoveto{\pgfqpoint{6.079324in}{2.889701in}}%
\pgfpathlineto{\pgfqpoint{6.091506in}{2.874670in}}%
\pgfpathlineto{\pgfqpoint{6.103713in}{2.859684in}}%
\pgfpathlineto{\pgfqpoint{6.136564in}{2.861888in}}%
\pgfpathlineto{\pgfqpoint{6.124330in}{2.876831in}}%
\pgfpathlineto{\pgfqpoint{6.112121in}{2.891819in}}%
\pgfpathlineto{\pgfqpoint{6.079324in}{2.889701in}}%
\pgfpathclose%
\pgfusepath{fill}%
\end{pgfscope}%
\begin{pgfscope}%
\pgfpathrectangle{\pgfqpoint{1.020000in}{0.880000in}}{\pgfqpoint{6.160000in}{6.160000in}}%
\pgfusepath{clip}%
\pgfsetbuttcap%
\pgfsetroundjoin%
\definecolor{currentfill}{rgb}{0.958176,0.771234,0.680301}%
\pgfsetfillcolor{currentfill}%
\pgfsetlinewidth{0.000000pt}%
\definecolor{currentstroke}{rgb}{0.000000,0.000000,0.000000}%
\pgfsetstrokecolor{currentstroke}%
\pgfsetdash{}{0pt}%
\pgfpathmoveto{\pgfqpoint{2.656013in}{4.272009in}}%
\pgfpathlineto{\pgfqpoint{2.665642in}{4.215747in}}%
\pgfpathlineto{\pgfqpoint{2.675280in}{4.159411in}}%
\pgfpathlineto{\pgfqpoint{2.708823in}{4.187515in}}%
\pgfpathlineto{\pgfqpoint{2.742370in}{4.215037in}}%
\pgfpathlineto{\pgfqpoint{2.732651in}{4.274786in}}%
\pgfpathlineto{\pgfqpoint{2.722941in}{4.334491in}}%
\pgfpathlineto{\pgfqpoint{2.689474in}{4.303596in}}%
\pgfpathlineto{\pgfqpoint{2.656013in}{4.272009in}}%
\pgfpathclose%
\pgfusepath{fill}%
\end{pgfscope}%
\begin{pgfscope}%
\pgfpathrectangle{\pgfqpoint{1.020000in}{0.880000in}}{\pgfqpoint{6.160000in}{6.160000in}}%
\pgfusepath{clip}%
\pgfsetbuttcap%
\pgfsetroundjoin%
\definecolor{currentfill}{rgb}{0.328604,0.439712,0.869587}%
\pgfsetfillcolor{currentfill}%
\pgfsetlinewidth{0.000000pt}%
\definecolor{currentstroke}{rgb}{0.000000,0.000000,0.000000}%
\pgfsetstrokecolor{currentstroke}%
\pgfsetdash{}{0pt}%
\pgfpathmoveto{\pgfqpoint{5.415881in}{2.956383in}}%
\pgfpathlineto{\pgfqpoint{5.427448in}{2.940854in}}%
\pgfpathlineto{\pgfqpoint{5.439036in}{2.925198in}}%
\pgfpathlineto{\pgfqpoint{5.472077in}{2.925372in}}%
\pgfpathlineto{\pgfqpoint{5.505099in}{2.925826in}}%
\pgfpathlineto{\pgfqpoint{5.493459in}{2.941478in}}%
\pgfpathlineto{\pgfqpoint{5.481839in}{2.957031in}}%
\pgfpathlineto{\pgfqpoint{5.448869in}{2.956569in}}%
\pgfpathlineto{\pgfqpoint{5.415881in}{2.956383in}}%
\pgfpathclose%
\pgfusepath{fill}%
\end{pgfscope}%
\begin{pgfscope}%
\pgfpathrectangle{\pgfqpoint{1.020000in}{0.880000in}}{\pgfqpoint{6.160000in}{6.160000in}}%
\pgfusepath{clip}%
\pgfsetbuttcap%
\pgfsetroundjoin%
\definecolor{currentfill}{rgb}{0.635474,0.756714,0.998297}%
\pgfsetfillcolor{currentfill}%
\pgfsetlinewidth{0.000000pt}%
\definecolor{currentstroke}{rgb}{0.000000,0.000000,0.000000}%
\pgfsetstrokecolor{currentstroke}%
\pgfsetdash{}{0pt}%
\pgfpathmoveto{\pgfqpoint{4.046436in}{3.502847in}}%
\pgfpathlineto{\pgfqpoint{4.056736in}{3.499120in}}%
\pgfpathlineto{\pgfqpoint{4.067055in}{3.497284in}}%
\pgfpathlineto{\pgfqpoint{4.100572in}{3.479749in}}%
\pgfpathlineto{\pgfqpoint{4.134056in}{3.462009in}}%
\pgfpathlineto{\pgfqpoint{4.123677in}{3.464814in}}%
\pgfpathlineto{\pgfqpoint{4.113318in}{3.469271in}}%
\pgfpathlineto{\pgfqpoint{4.079893in}{3.486093in}}%
\pgfpathlineto{\pgfqpoint{4.046436in}{3.502847in}}%
\pgfpathclose%
\pgfusepath{fill}%
\end{pgfscope}%
\begin{pgfscope}%
\pgfpathrectangle{\pgfqpoint{1.020000in}{0.880000in}}{\pgfqpoint{6.160000in}{6.160000in}}%
\pgfusepath{clip}%
\pgfsetbuttcap%
\pgfsetroundjoin%
\definecolor{currentfill}{rgb}{0.916071,0.833977,0.788693}%
\pgfsetfillcolor{currentfill}%
\pgfsetlinewidth{0.000000pt}%
\definecolor{currentstroke}{rgb}{0.000000,0.000000,0.000000}%
\pgfsetstrokecolor{currentstroke}%
\pgfsetdash{}{0pt}%
\pgfpathmoveto{\pgfqpoint{3.185881in}{4.133008in}}%
\pgfpathlineto{\pgfqpoint{3.195836in}{4.071986in}}%
\pgfpathlineto{\pgfqpoint{3.205774in}{4.013782in}}%
\pgfpathlineto{\pgfqpoint{3.239541in}{4.013646in}}%
\pgfpathlineto{\pgfqpoint{3.273304in}{4.011308in}}%
\pgfpathlineto{\pgfqpoint{3.263349in}{4.068053in}}%
\pgfpathlineto{\pgfqpoint{3.253380in}{4.127687in}}%
\pgfpathlineto{\pgfqpoint{3.219631in}{4.131619in}}%
\pgfpathlineto{\pgfqpoint{3.185881in}{4.133008in}}%
\pgfpathclose%
\pgfusepath{fill}%
\end{pgfscope}%
\begin{pgfscope}%
\pgfpathrectangle{\pgfqpoint{1.020000in}{0.880000in}}{\pgfqpoint{6.160000in}{6.160000in}}%
\pgfusepath{clip}%
\pgfsetbuttcap%
\pgfsetroundjoin%
\definecolor{currentfill}{rgb}{0.947345,0.794696,0.716991}%
\pgfsetfillcolor{currentfill}%
\pgfsetlinewidth{0.000000pt}%
\definecolor{currentstroke}{rgb}{0.000000,0.000000,0.000000}%
\pgfsetstrokecolor{currentstroke}%
\pgfsetdash{}{0pt}%
\pgfpathmoveto{\pgfqpoint{2.589097in}{4.208054in}}%
\pgfpathlineto{\pgfqpoint{2.598637in}{4.155376in}}%
\pgfpathlineto{\pgfqpoint{2.608189in}{4.102602in}}%
\pgfpathlineto{\pgfqpoint{2.641737in}{4.131015in}}%
\pgfpathlineto{\pgfqpoint{2.675280in}{4.159411in}}%
\pgfpathlineto{\pgfqpoint{2.665642in}{4.215747in}}%
\pgfpathlineto{\pgfqpoint{2.656013in}{4.272009in}}%
\pgfpathlineto{\pgfqpoint{2.622556in}{4.240058in}}%
\pgfpathlineto{\pgfqpoint{2.589097in}{4.208054in}}%
\pgfpathclose%
\pgfusepath{fill}%
\end{pgfscope}%
\begin{pgfscope}%
\pgfpathrectangle{\pgfqpoint{1.020000in}{0.880000in}}{\pgfqpoint{6.160000in}{6.160000in}}%
\pgfusepath{clip}%
\pgfsetbuttcap%
\pgfsetroundjoin%
\definecolor{currentfill}{rgb}{0.592356,0.722792,0.999434}%
\pgfsetfillcolor{currentfill}%
\pgfsetlinewidth{0.000000pt}%
\definecolor{currentstroke}{rgb}{0.000000,0.000000,0.000000}%
\pgfsetstrokecolor{currentstroke}%
\pgfsetdash{}{0pt}%
\pgfpathmoveto{\pgfqpoint{4.200924in}{3.426465in}}%
\pgfpathlineto{\pgfqpoint{4.211383in}{3.423774in}}%
\pgfpathlineto{\pgfqpoint{4.221864in}{3.422265in}}%
\pgfpathlineto{\pgfqpoint{4.255307in}{3.403103in}}%
\pgfpathlineto{\pgfqpoint{4.288716in}{3.384200in}}%
\pgfpathlineto{\pgfqpoint{4.278177in}{3.387427in}}%
\pgfpathlineto{\pgfqpoint{4.267661in}{3.391654in}}%
\pgfpathlineto{\pgfqpoint{4.234309in}{3.408912in}}%
\pgfpathlineto{\pgfqpoint{4.200924in}{3.426465in}}%
\pgfpathclose%
\pgfusepath{fill}%
\end{pgfscope}%
\begin{pgfscope}%
\pgfpathrectangle{\pgfqpoint{1.020000in}{0.880000in}}{\pgfqpoint{6.160000in}{6.160000in}}%
\pgfusepath{clip}%
\pgfsetbuttcap%
\pgfsetroundjoin%
\definecolor{currentfill}{rgb}{0.543440,0.680003,0.993051}%
\pgfsetfillcolor{currentfill}%
\pgfsetlinewidth{0.000000pt}%
\definecolor{currentstroke}{rgb}{0.000000,0.000000,0.000000}%
\pgfsetstrokecolor{currentstroke}%
\pgfsetdash{}{0pt}%
\pgfpathmoveto{\pgfqpoint{4.355434in}{3.347620in}}%
\pgfpathlineto{\pgfqpoint{4.366051in}{3.343379in}}%
\pgfpathlineto{\pgfqpoint{4.376690in}{3.339738in}}%
\pgfpathlineto{\pgfqpoint{4.410054in}{3.320358in}}%
\pgfpathlineto{\pgfqpoint{4.443386in}{3.301675in}}%
\pgfpathlineto{\pgfqpoint{4.432693in}{3.307226in}}%
\pgfpathlineto{\pgfqpoint{4.422023in}{3.313252in}}%
\pgfpathlineto{\pgfqpoint{4.388744in}{3.330126in}}%
\pgfpathlineto{\pgfqpoint{4.355434in}{3.347620in}}%
\pgfpathclose%
\pgfusepath{fill}%
\end{pgfscope}%
\begin{pgfscope}%
\pgfpathrectangle{\pgfqpoint{1.020000in}{0.880000in}}{\pgfqpoint{6.160000in}{6.160000in}}%
\pgfusepath{clip}%
\pgfsetbuttcap%
\pgfsetroundjoin%
\definecolor{currentfill}{rgb}{0.818056,0.855590,0.914638}%
\pgfsetfillcolor{currentfill}%
\pgfsetlinewidth{0.000000pt}%
\definecolor{currentstroke}{rgb}{0.000000,0.000000,0.000000}%
\pgfsetstrokecolor{currentstroke}%
\pgfsetdash{}{0pt}%
\pgfpathmoveto{\pgfqpoint{3.428189in}{3.886176in}}%
\pgfpathlineto{\pgfqpoint{3.438127in}{3.844355in}}%
\pgfpathlineto{\pgfqpoint{3.448053in}{3.806290in}}%
\pgfpathlineto{\pgfqpoint{3.481796in}{3.799170in}}%
\pgfpathlineto{\pgfqpoint{3.515522in}{3.790498in}}%
\pgfpathlineto{\pgfqpoint{3.505575in}{3.825812in}}%
\pgfpathlineto{\pgfqpoint{3.495617in}{3.864802in}}%
\pgfpathlineto{\pgfqpoint{3.461912in}{3.876312in}}%
\pgfpathlineto{\pgfqpoint{3.428189in}{3.886176in}}%
\pgfpathclose%
\pgfusepath{fill}%
\end{pgfscope}%
\begin{pgfscope}%
\pgfpathrectangle{\pgfqpoint{1.020000in}{0.880000in}}{\pgfqpoint{6.160000in}{6.160000in}}%
\pgfusepath{clip}%
\pgfsetbuttcap%
\pgfsetroundjoin%
\definecolor{currentfill}{rgb}{0.500031,0.638508,0.981070}%
\pgfsetfillcolor{currentfill}%
\pgfsetlinewidth{0.000000pt}%
\definecolor{currentstroke}{rgb}{0.000000,0.000000,0.000000}%
\pgfsetstrokecolor{currentstroke}%
\pgfsetdash{}{0pt}%
\pgfpathmoveto{\pgfqpoint{4.509955in}{3.266648in}}%
\pgfpathlineto{\pgfqpoint{4.520721in}{3.259652in}}%
\pgfpathlineto{\pgfqpoint{4.531509in}{3.252827in}}%
\pgfpathlineto{\pgfqpoint{4.564799in}{3.234910in}}%
\pgfpathlineto{\pgfqpoint{4.598059in}{3.218002in}}%
\pgfpathlineto{\pgfqpoint{4.587221in}{3.226439in}}%
\pgfpathlineto{\pgfqpoint{4.576405in}{3.234953in}}%
\pgfpathlineto{\pgfqpoint{4.543195in}{3.250378in}}%
\pgfpathlineto{\pgfqpoint{4.509955in}{3.266648in}}%
\pgfpathclose%
\pgfusepath{fill}%
\end{pgfscope}%
\begin{pgfscope}%
\pgfpathrectangle{\pgfqpoint{1.020000in}{0.880000in}}{\pgfqpoint{6.160000in}{6.160000in}}%
\pgfusepath{clip}%
\pgfsetbuttcap%
\pgfsetroundjoin%
\definecolor{currentfill}{rgb}{0.933221,0.815557,0.753151}%
\pgfsetfillcolor{currentfill}%
\pgfsetlinewidth{0.000000pt}%
\definecolor{currentstroke}{rgb}{0.000000,0.000000,0.000000}%
\pgfsetstrokecolor{currentstroke}%
\pgfsetdash{}{0pt}%
\pgfpathmoveto{\pgfqpoint{2.522157in}{4.145052in}}%
\pgfpathlineto{\pgfqpoint{2.531605in}{4.095959in}}%
\pgfpathlineto{\pgfqpoint{2.541066in}{4.046755in}}%
\pgfpathlineto{\pgfqpoint{2.574633in}{4.074435in}}%
\pgfpathlineto{\pgfqpoint{2.608189in}{4.102602in}}%
\pgfpathlineto{\pgfqpoint{2.598637in}{4.155376in}}%
\pgfpathlineto{\pgfqpoint{2.589097in}{4.208054in}}%
\pgfpathlineto{\pgfqpoint{2.555632in}{4.176294in}}%
\pgfpathlineto{\pgfqpoint{2.522157in}{4.145052in}}%
\pgfpathclose%
\pgfusepath{fill}%
\end{pgfscope}%
\begin{pgfscope}%
\pgfpathrectangle{\pgfqpoint{1.020000in}{0.880000in}}{\pgfqpoint{6.160000in}{6.160000in}}%
\pgfusepath{clip}%
\pgfsetbuttcap%
\pgfsetroundjoin%
\definecolor{currentfill}{rgb}{0.348323,0.465711,0.888346}%
\pgfsetfillcolor{currentfill}%
\pgfsetlinewidth{0.000000pt}%
\definecolor{currentstroke}{rgb}{0.000000,0.000000,0.000000}%
\pgfsetstrokecolor{currentstroke}%
\pgfsetdash{}{0pt}%
\pgfpathmoveto{\pgfqpoint{5.194812in}{2.993780in}}%
\pgfpathlineto{\pgfqpoint{5.206185in}{2.978933in}}%
\pgfpathlineto{\pgfqpoint{5.217577in}{2.963771in}}%
\pgfpathlineto{\pgfqpoint{5.250672in}{2.961253in}}%
\pgfpathlineto{\pgfqpoint{5.283749in}{2.959329in}}%
\pgfpathlineto{\pgfqpoint{5.272305in}{2.974559in}}%
\pgfpathlineto{\pgfqpoint{5.260879in}{2.989475in}}%
\pgfpathlineto{\pgfqpoint{5.227854in}{2.991335in}}%
\pgfpathlineto{\pgfqpoint{5.194812in}{2.993780in}}%
\pgfpathclose%
\pgfusepath{fill}%
\end{pgfscope}%
\begin{pgfscope}%
\pgfpathrectangle{\pgfqpoint{1.020000in}{0.880000in}}{\pgfqpoint{6.160000in}{6.160000in}}%
\pgfusepath{clip}%
\pgfsetbuttcap%
\pgfsetroundjoin%
\definecolor{currentfill}{rgb}{0.451739,0.588181,0.960201}%
\pgfsetfillcolor{currentfill}%
\pgfsetlinewidth{0.000000pt}%
\definecolor{currentstroke}{rgb}{0.000000,0.000000,0.000000}%
\pgfsetstrokecolor{currentstroke}%
\pgfsetdash{}{0pt}%
\pgfpathmoveto{\pgfqpoint{4.664496in}{3.187236in}}%
\pgfpathlineto{\pgfqpoint{4.675403in}{3.177456in}}%
\pgfpathlineto{\pgfqpoint{4.686332in}{3.167548in}}%
\pgfpathlineto{\pgfqpoint{4.719559in}{3.152553in}}%
\pgfpathlineto{\pgfqpoint{4.752760in}{3.138710in}}%
\pgfpathlineto{\pgfqpoint{4.741783in}{3.149679in}}%
\pgfpathlineto{\pgfqpoint{4.730826in}{3.160428in}}%
\pgfpathlineto{\pgfqpoint{4.697674in}{3.173351in}}%
\pgfpathlineto{\pgfqpoint{4.664496in}{3.187236in}}%
\pgfpathclose%
\pgfusepath{fill}%
\end{pgfscope}%
\begin{pgfscope}%
\pgfpathrectangle{\pgfqpoint{1.020000in}{0.880000in}}{\pgfqpoint{6.160000in}{6.160000in}}%
\pgfusepath{clip}%
\pgfsetbuttcap%
\pgfsetroundjoin%
\definecolor{currentfill}{rgb}{0.916071,0.833977,0.788693}%
\pgfsetfillcolor{currentfill}%
\pgfsetlinewidth{0.000000pt}%
\definecolor{currentstroke}{rgb}{0.000000,0.000000,0.000000}%
\pgfsetstrokecolor{currentstroke}%
\pgfsetdash{}{0pt}%
\pgfpathmoveto{\pgfqpoint{2.455159in}{4.085094in}}%
\pgfpathlineto{\pgfqpoint{2.464512in}{4.039468in}}%
\pgfpathlineto{\pgfqpoint{2.473879in}{3.993718in}}%
\pgfpathlineto{\pgfqpoint{2.507482in}{4.019784in}}%
\pgfpathlineto{\pgfqpoint{2.541066in}{4.046755in}}%
\pgfpathlineto{\pgfqpoint{2.531605in}{4.095959in}}%
\pgfpathlineto{\pgfqpoint{2.522157in}{4.145052in}}%
\pgfpathlineto{\pgfqpoint{2.488668in}{4.114577in}}%
\pgfpathlineto{\pgfqpoint{2.455159in}{4.085094in}}%
\pgfpathclose%
\pgfusepath{fill}%
\end{pgfscope}%
\begin{pgfscope}%
\pgfpathrectangle{\pgfqpoint{1.020000in}{0.880000in}}{\pgfqpoint{6.160000in}{6.160000in}}%
\pgfusepath{clip}%
\pgfsetbuttcap%
\pgfsetroundjoin%
\definecolor{currentfill}{rgb}{0.961645,0.758029,0.661782}%
\pgfsetfillcolor{currentfill}%
\pgfsetlinewidth{0.000000pt}%
\definecolor{currentstroke}{rgb}{0.000000,0.000000,0.000000}%
\pgfsetstrokecolor{currentstroke}%
\pgfsetdash{}{0pt}%
\pgfpathmoveto{\pgfqpoint{2.876648in}{4.313324in}}%
\pgfpathlineto{\pgfqpoint{2.886503in}{4.248262in}}%
\pgfpathlineto{\pgfqpoint{2.896356in}{4.184081in}}%
\pgfpathlineto{\pgfqpoint{2.930013in}{4.202117in}}%
\pgfpathlineto{\pgfqpoint{2.963682in}{4.218157in}}%
\pgfpathlineto{\pgfqpoint{2.953780in}{4.284238in}}%
\pgfpathlineto{\pgfqpoint{2.943874in}{4.351305in}}%
\pgfpathlineto{\pgfqpoint{2.910252in}{4.333475in}}%
\pgfpathlineto{\pgfqpoint{2.876648in}{4.313324in}}%
\pgfpathclose%
\pgfusepath{fill}%
\end{pgfscope}%
\begin{pgfscope}%
\pgfpathrectangle{\pgfqpoint{1.020000in}{0.880000in}}{\pgfqpoint{6.160000in}{6.160000in}}%
\pgfusepath{clip}%
\pgfsetbuttcap%
\pgfsetroundjoin%
\definecolor{currentfill}{rgb}{0.414801,0.546874,0.939088}%
\pgfsetfillcolor{currentfill}%
\pgfsetlinewidth{0.000000pt}%
\definecolor{currentstroke}{rgb}{0.000000,0.000000,0.000000}%
\pgfsetstrokecolor{currentstroke}%
\pgfsetdash{}{0pt}%
\pgfpathmoveto{\pgfqpoint{4.819092in}{3.114337in}}%
\pgfpathlineto{\pgfqpoint{4.830137in}{3.102343in}}%
\pgfpathlineto{\pgfqpoint{4.841202in}{3.090033in}}%
\pgfpathlineto{\pgfqpoint{4.874383in}{3.078835in}}%
\pgfpathlineto{\pgfqpoint{4.907542in}{3.068769in}}%
\pgfpathlineto{\pgfqpoint{4.896428in}{3.081634in}}%
\pgfpathlineto{\pgfqpoint{4.885332in}{3.094091in}}%
\pgfpathlineto{\pgfqpoint{4.852223in}{3.103722in}}%
\pgfpathlineto{\pgfqpoint{4.819092in}{3.114337in}}%
\pgfpathclose%
\pgfusepath{fill}%
\end{pgfscope}%
\begin{pgfscope}%
\pgfpathrectangle{\pgfqpoint{1.020000in}{0.880000in}}{\pgfqpoint{6.160000in}{6.160000in}}%
\pgfusepath{clip}%
\pgfsetbuttcap%
\pgfsetroundjoin%
\definecolor{currentfill}{rgb}{0.949151,0.790785,0.710876}%
\pgfsetfillcolor{currentfill}%
\pgfsetlinewidth{0.000000pt}%
\definecolor{currentstroke}{rgb}{0.000000,0.000000,0.000000}%
\pgfsetstrokecolor{currentstroke}%
\pgfsetdash{}{0pt}%
\pgfpathmoveto{\pgfqpoint{3.031056in}{4.243407in}}%
\pgfpathlineto{\pgfqpoint{3.040988in}{4.177635in}}%
\pgfpathlineto{\pgfqpoint{3.050909in}{4.113768in}}%
\pgfpathlineto{\pgfqpoint{3.084645in}{4.122157in}}%
\pgfpathlineto{\pgfqpoint{3.118386in}{4.128222in}}%
\pgfpathlineto{\pgfqpoint{3.108436in}{4.192296in}}%
\pgfpathlineto{\pgfqpoint{3.098475in}{4.258392in}}%
\pgfpathlineto{\pgfqpoint{3.064760in}{4.252255in}}%
\pgfpathlineto{\pgfqpoint{3.031056in}{4.243407in}}%
\pgfpathclose%
\pgfusepath{fill}%
\end{pgfscope}%
\begin{pgfscope}%
\pgfpathrectangle{\pgfqpoint{1.020000in}{0.880000in}}{\pgfqpoint{6.160000in}{6.160000in}}%
\pgfusepath{clip}%
\pgfsetbuttcap%
\pgfsetroundjoin%
\definecolor{currentfill}{rgb}{0.299441,0.400248,0.839842}%
\pgfsetfillcolor{currentfill}%
\pgfsetlinewidth{0.000000pt}%
\definecolor{currentstroke}{rgb}{0.000000,0.000000,0.000000}%
\pgfsetstrokecolor{currentstroke}%
\pgfsetdash{}{0pt}%
\pgfpathmoveto{\pgfqpoint{6.013670in}{2.885528in}}%
\pgfpathlineto{\pgfqpoint{6.025798in}{2.870410in}}%
\pgfpathlineto{\pgfqpoint{6.037950in}{2.855338in}}%
\pgfpathlineto{\pgfqpoint{6.070841in}{2.857500in}}%
\pgfpathlineto{\pgfqpoint{6.103713in}{2.859684in}}%
\pgfpathlineto{\pgfqpoint{6.091506in}{2.874670in}}%
\pgfpathlineto{\pgfqpoint{6.079324in}{2.889701in}}%
\pgfpathlineto{\pgfqpoint{6.046507in}{2.887603in}}%
\pgfpathlineto{\pgfqpoint{6.013670in}{2.885528in}}%
\pgfpathclose%
\pgfusepath{fill}%
\end{pgfscope}%
\begin{pgfscope}%
\pgfpathrectangle{\pgfqpoint{1.020000in}{0.880000in}}{\pgfqpoint{6.160000in}{6.160000in}}%
\pgfusepath{clip}%
\pgfsetbuttcap%
\pgfsetroundjoin%
\definecolor{currentfill}{rgb}{0.871493,0.862309,0.857016}%
\pgfsetfillcolor{currentfill}%
\pgfsetlinewidth{0.000000pt}%
\definecolor{currentstroke}{rgb}{0.000000,0.000000,0.000000}%
\pgfsetstrokecolor{currentstroke}%
\pgfsetdash{}{0pt}%
\pgfpathmoveto{\pgfqpoint{3.273304in}{4.011308in}}%
\pgfpathlineto{\pgfqpoint{3.283241in}{3.957738in}}%
\pgfpathlineto{\pgfqpoint{3.293162in}{3.907583in}}%
\pgfpathlineto{\pgfqpoint{3.326934in}{3.905049in}}%
\pgfpathlineto{\pgfqpoint{3.360698in}{3.900610in}}%
\pgfpathlineto{\pgfqpoint{3.350761in}{3.948624in}}%
\pgfpathlineto{\pgfqpoint{3.340810in}{4.000080in}}%
\pgfpathlineto{\pgfqpoint{3.307061in}{4.006775in}}%
\pgfpathlineto{\pgfqpoint{3.273304in}{4.011308in}}%
\pgfpathclose%
\pgfusepath{fill}%
\end{pgfscope}%
\begin{pgfscope}%
\pgfpathrectangle{\pgfqpoint{1.020000in}{0.880000in}}{\pgfqpoint{6.160000in}{6.160000in}}%
\pgfusepath{clip}%
\pgfsetbuttcap%
\pgfsetroundjoin%
\definecolor{currentfill}{rgb}{0.895882,0.849906,0.823499}%
\pgfsetfillcolor{currentfill}%
\pgfsetlinewidth{0.000000pt}%
\definecolor{currentstroke}{rgb}{0.000000,0.000000,0.000000}%
\pgfsetstrokecolor{currentstroke}%
\pgfsetdash{}{0pt}%
\pgfpathmoveto{\pgfqpoint{2.388069in}{4.029856in}}%
\pgfpathlineto{\pgfqpoint{2.397327in}{3.987476in}}%
\pgfpathlineto{\pgfqpoint{2.406599in}{3.944965in}}%
\pgfpathlineto{\pgfqpoint{2.440252in}{3.968730in}}%
\pgfpathlineto{\pgfqpoint{2.473879in}{3.993718in}}%
\pgfpathlineto{\pgfqpoint{2.464512in}{4.039468in}}%
\pgfpathlineto{\pgfqpoint{2.455159in}{4.085094in}}%
\pgfpathlineto{\pgfqpoint{2.421627in}{4.056798in}}%
\pgfpathlineto{\pgfqpoint{2.388069in}{4.029856in}}%
\pgfpathclose%
\pgfusepath{fill}%
\end{pgfscope}%
\begin{pgfscope}%
\pgfpathrectangle{\pgfqpoint{1.020000in}{0.880000in}}{\pgfqpoint{6.160000in}{6.160000in}}%
\pgfusepath{clip}%
\pgfsetbuttcap%
\pgfsetroundjoin%
\definecolor{currentfill}{rgb}{0.309060,0.413498,0.850128}%
\pgfsetfillcolor{currentfill}%
\pgfsetlinewidth{0.000000pt}%
\definecolor{currentstroke}{rgb}{0.000000,0.000000,0.000000}%
\pgfsetstrokecolor{currentstroke}%
\pgfsetdash{}{0pt}%
\pgfpathmoveto{\pgfqpoint{5.792365in}{2.904683in}}%
\pgfpathlineto{\pgfqpoint{5.804285in}{2.889221in}}%
\pgfpathlineto{\pgfqpoint{5.816227in}{2.873798in}}%
\pgfpathlineto{\pgfqpoint{5.849184in}{2.875636in}}%
\pgfpathlineto{\pgfqpoint{5.882121in}{2.877532in}}%
\pgfpathlineto{\pgfqpoint{5.870125in}{2.892869in}}%
\pgfpathlineto{\pgfqpoint{5.858151in}{2.908248in}}%
\pgfpathlineto{\pgfqpoint{5.825268in}{2.906437in}}%
\pgfpathlineto{\pgfqpoint{5.792365in}{2.904683in}}%
\pgfpathclose%
\pgfusepath{fill}%
\end{pgfscope}%
\begin{pgfscope}%
\pgfpathrectangle{\pgfqpoint{1.020000in}{0.880000in}}{\pgfqpoint{6.160000in}{6.160000in}}%
\pgfusepath{clip}%
\pgfsetbuttcap%
\pgfsetroundjoin%
\definecolor{currentfill}{rgb}{0.318832,0.426605,0.859857}%
\pgfsetfillcolor{currentfill}%
\pgfsetlinewidth{0.000000pt}%
\definecolor{currentstroke}{rgb}{0.000000,0.000000,0.000000}%
\pgfsetstrokecolor{currentstroke}%
\pgfsetdash{}{0pt}%
\pgfpathmoveto{\pgfqpoint{5.571088in}{2.927421in}}%
\pgfpathlineto{\pgfqpoint{5.582802in}{2.911744in}}%
\pgfpathlineto{\pgfqpoint{5.594538in}{2.896053in}}%
\pgfpathlineto{\pgfqpoint{5.627558in}{2.897178in}}%
\pgfpathlineto{\pgfqpoint{5.660558in}{2.898457in}}%
\pgfpathlineto{\pgfqpoint{5.648768in}{2.914089in}}%
\pgfpathlineto{\pgfqpoint{5.637001in}{2.929724in}}%
\pgfpathlineto{\pgfqpoint{5.604054in}{2.928498in}}%
\pgfpathlineto{\pgfqpoint{5.571088in}{2.927421in}}%
\pgfpathclose%
\pgfusepath{fill}%
\end{pgfscope}%
\begin{pgfscope}%
\pgfpathrectangle{\pgfqpoint{1.020000in}{0.880000in}}{\pgfqpoint{6.160000in}{6.160000in}}%
\pgfusepath{clip}%
\pgfsetbuttcap%
\pgfsetroundjoin%
\definecolor{currentfill}{rgb}{0.383662,0.510183,0.917831}%
\pgfsetfillcolor{currentfill}%
\pgfsetlinewidth{0.000000pt}%
\definecolor{currentstroke}{rgb}{0.000000,0.000000,0.000000}%
\pgfsetstrokecolor{currentstroke}%
\pgfsetdash{}{0pt}%
\pgfpathmoveto{\pgfqpoint{4.973800in}{3.051801in}}%
\pgfpathlineto{\pgfqpoint{4.984983in}{3.038197in}}%
\pgfpathlineto{\pgfqpoint{4.996184in}{3.024217in}}%
\pgfpathlineto{\pgfqpoint{5.029334in}{3.016938in}}%
\pgfpathlineto{\pgfqpoint{5.062466in}{3.010633in}}%
\pgfpathlineto{\pgfqpoint{5.051213in}{3.024863in}}%
\pgfpathlineto{\pgfqpoint{5.039979in}{3.038658in}}%
\pgfpathlineto{\pgfqpoint{5.006899in}{3.044780in}}%
\pgfpathlineto{\pgfqpoint{4.973800in}{3.051801in}}%
\pgfpathclose%
\pgfusepath{fill}%
\end{pgfscope}%
\begin{pgfscope}%
\pgfpathrectangle{\pgfqpoint{1.020000in}{0.880000in}}{\pgfqpoint{6.160000in}{6.160000in}}%
\pgfusepath{clip}%
\pgfsetbuttcap%
\pgfsetroundjoin%
\definecolor{currentfill}{rgb}{0.956371,0.775144,0.686416}%
\pgfsetfillcolor{currentfill}%
\pgfsetlinewidth{0.000000pt}%
\definecolor{currentstroke}{rgb}{0.000000,0.000000,0.000000}%
\pgfsetstrokecolor{currentstroke}%
\pgfsetdash{}{0pt}%
\pgfpathmoveto{\pgfqpoint{2.809485in}{4.267139in}}%
\pgfpathlineto{\pgfqpoint{2.819279in}{4.204669in}}%
\pgfpathlineto{\pgfqpoint{2.829072in}{4.142972in}}%
\pgfpathlineto{\pgfqpoint{2.862710in}{4.164283in}}%
\pgfpathlineto{\pgfqpoint{2.896356in}{4.184081in}}%
\pgfpathlineto{\pgfqpoint{2.886503in}{4.248262in}}%
\pgfpathlineto{\pgfqpoint{2.876648in}{4.313324in}}%
\pgfpathlineto{\pgfqpoint{2.843059in}{4.291118in}}%
\pgfpathlineto{\pgfqpoint{2.809485in}{4.267139in}}%
\pgfpathclose%
\pgfusepath{fill}%
\end{pgfscope}%
\begin{pgfscope}%
\pgfpathrectangle{\pgfqpoint{1.020000in}{0.880000in}}{\pgfqpoint{6.160000in}{6.160000in}}%
\pgfusepath{clip}%
\pgfsetbuttcap%
\pgfsetroundjoin%
\definecolor{currentfill}{rgb}{0.333490,0.446265,0.874452}%
\pgfsetfillcolor{currentfill}%
\pgfsetlinewidth{0.000000pt}%
\definecolor{currentstroke}{rgb}{0.000000,0.000000,0.000000}%
\pgfsetstrokecolor{currentstroke}%
\pgfsetdash{}{0pt}%
\pgfpathmoveto{\pgfqpoint{5.349850in}{2.957018in}}%
\pgfpathlineto{\pgfqpoint{5.361365in}{2.941518in}}%
\pgfpathlineto{\pgfqpoint{5.372900in}{2.925868in}}%
\pgfpathlineto{\pgfqpoint{5.405977in}{2.925347in}}%
\pgfpathlineto{\pgfqpoint{5.439036in}{2.925198in}}%
\pgfpathlineto{\pgfqpoint{5.427448in}{2.940854in}}%
\pgfpathlineto{\pgfqpoint{5.415881in}{2.956383in}}%
\pgfpathlineto{\pgfqpoint{5.382875in}{2.956517in}}%
\pgfpathlineto{\pgfqpoint{5.349850in}{2.957018in}}%
\pgfpathclose%
\pgfusepath{fill}%
\end{pgfscope}%
\begin{pgfscope}%
\pgfpathrectangle{\pgfqpoint{1.020000in}{0.880000in}}{\pgfqpoint{6.160000in}{6.160000in}}%
\pgfusepath{clip}%
\pgfsetbuttcap%
\pgfsetroundjoin%
\definecolor{currentfill}{rgb}{0.919376,0.831273,0.782874}%
\pgfsetfillcolor{currentfill}%
\pgfsetlinewidth{0.000000pt}%
\definecolor{currentstroke}{rgb}{0.000000,0.000000,0.000000}%
\pgfsetstrokecolor{currentstroke}%
\pgfsetdash{}{0pt}%
\pgfpathmoveto{\pgfqpoint{3.118386in}{4.128222in}}%
\pgfpathlineto{\pgfqpoint{3.128320in}{4.066536in}}%
\pgfpathlineto{\pgfqpoint{3.138237in}{4.007558in}}%
\pgfpathlineto{\pgfqpoint{3.172006in}{4.011737in}}%
\pgfpathlineto{\pgfqpoint{3.205774in}{4.013782in}}%
\pgfpathlineto{\pgfqpoint{3.195836in}{4.071986in}}%
\pgfpathlineto{\pgfqpoint{3.185881in}{4.133008in}}%
\pgfpathlineto{\pgfqpoint{3.152132in}{4.131862in}}%
\pgfpathlineto{\pgfqpoint{3.118386in}{4.128222in}}%
\pgfpathclose%
\pgfusepath{fill}%
\end{pgfscope}%
\begin{pgfscope}%
\pgfpathrectangle{\pgfqpoint{1.020000in}{0.880000in}}{\pgfqpoint{6.160000in}{6.160000in}}%
\pgfusepath{clip}%
\pgfsetbuttcap%
\pgfsetroundjoin%
\definecolor{currentfill}{rgb}{0.738826,0.822572,0.968261}%
\pgfsetfillcolor{currentfill}%
\pgfsetlinewidth{0.000000pt}%
\definecolor{currentstroke}{rgb}{0.000000,0.000000,0.000000}%
\pgfsetstrokecolor{currentstroke}%
\pgfsetdash{}{0pt}%
\pgfpathmoveto{\pgfqpoint{3.670214in}{3.693317in}}%
\pgfpathlineto{\pgfqpoint{3.680211in}{3.674200in}}%
\pgfpathlineto{\pgfqpoint{3.690210in}{3.658581in}}%
\pgfpathlineto{\pgfqpoint{3.723901in}{3.648096in}}%
\pgfpathlineto{\pgfqpoint{3.757568in}{3.636466in}}%
\pgfpathlineto{\pgfqpoint{3.747525in}{3.650242in}}%
\pgfpathlineto{\pgfqpoint{3.737486in}{3.667302in}}%
\pgfpathlineto{\pgfqpoint{3.703863in}{3.680807in}}%
\pgfpathlineto{\pgfqpoint{3.670214in}{3.693317in}}%
\pgfpathclose%
\pgfusepath{fill}%
\end{pgfscope}%
\begin{pgfscope}%
\pgfpathrectangle{\pgfqpoint{1.020000in}{0.880000in}}{\pgfqpoint{6.160000in}{6.160000in}}%
\pgfusepath{clip}%
\pgfsetbuttcap%
\pgfsetroundjoin%
\definecolor{currentfill}{rgb}{0.782049,0.842864,0.942980}%
\pgfsetfillcolor{currentfill}%
\pgfsetlinewidth{0.000000pt}%
\definecolor{currentstroke}{rgb}{0.000000,0.000000,0.000000}%
\pgfsetstrokecolor{currentstroke}%
\pgfsetdash{}{0pt}%
\pgfpathmoveto{\pgfqpoint{3.515522in}{3.790498in}}%
\pgfpathlineto{\pgfqpoint{3.525461in}{3.758959in}}%
\pgfpathlineto{\pgfqpoint{3.535392in}{3.731247in}}%
\pgfpathlineto{\pgfqpoint{3.569128in}{3.723768in}}%
\pgfpathlineto{\pgfqpoint{3.602845in}{3.714897in}}%
\pgfpathlineto{\pgfqpoint{3.592883in}{3.740020in}}%
\pgfpathlineto{\pgfqpoint{3.582916in}{3.768838in}}%
\pgfpathlineto{\pgfqpoint{3.549230in}{3.780356in}}%
\pgfpathlineto{\pgfqpoint{3.515522in}{3.790498in}}%
\pgfpathclose%
\pgfusepath{fill}%
\end{pgfscope}%
\begin{pgfscope}%
\pgfpathrectangle{\pgfqpoint{1.020000in}{0.880000in}}{\pgfqpoint{6.160000in}{6.160000in}}%
\pgfusepath{clip}%
\pgfsetbuttcap%
\pgfsetroundjoin%
\definecolor{currentfill}{rgb}{0.947345,0.794696,0.716991}%
\pgfsetfillcolor{currentfill}%
\pgfsetlinewidth{0.000000pt}%
\definecolor{currentstroke}{rgb}{0.000000,0.000000,0.000000}%
\pgfsetstrokecolor{currentstroke}%
\pgfsetdash{}{0pt}%
\pgfpathmoveto{\pgfqpoint{2.742370in}{4.215037in}}%
\pgfpathlineto{\pgfqpoint{2.752093in}{4.155610in}}%
\pgfpathlineto{\pgfqpoint{2.761817in}{4.096849in}}%
\pgfpathlineto{\pgfqpoint{2.795442in}{4.120406in}}%
\pgfpathlineto{\pgfqpoint{2.829072in}{4.142972in}}%
\pgfpathlineto{\pgfqpoint{2.819279in}{4.204669in}}%
\pgfpathlineto{\pgfqpoint{2.809485in}{4.267139in}}%
\pgfpathlineto{\pgfqpoint{2.775922in}{4.241678in}}%
\pgfpathlineto{\pgfqpoint{2.742370in}{4.215037in}}%
\pgfpathclose%
\pgfusepath{fill}%
\end{pgfscope}%
\begin{pgfscope}%
\pgfpathrectangle{\pgfqpoint{1.020000in}{0.880000in}}{\pgfqpoint{6.160000in}{6.160000in}}%
\pgfusepath{clip}%
\pgfsetbuttcap%
\pgfsetroundjoin%
\definecolor{currentfill}{rgb}{0.945540,0.798606,0.723105}%
\pgfsetfillcolor{currentfill}%
\pgfsetlinewidth{0.000000pt}%
\definecolor{currentstroke}{rgb}{0.000000,0.000000,0.000000}%
\pgfsetstrokecolor{currentstroke}%
\pgfsetdash{}{0pt}%
\pgfpathmoveto{\pgfqpoint{2.963682in}{4.218157in}}%
\pgfpathlineto{\pgfqpoint{2.973577in}{4.153470in}}%
\pgfpathlineto{\pgfqpoint{2.983460in}{4.090547in}}%
\pgfpathlineto{\pgfqpoint{3.017181in}{4.103181in}}%
\pgfpathlineto{\pgfqpoint{3.050909in}{4.113768in}}%
\pgfpathlineto{\pgfqpoint{3.040988in}{4.177635in}}%
\pgfpathlineto{\pgfqpoint{3.031056in}{4.243407in}}%
\pgfpathlineto{\pgfqpoint{2.997363in}{4.231985in}}%
\pgfpathlineto{\pgfqpoint{2.963682in}{4.218157in}}%
\pgfpathclose%
\pgfusepath{fill}%
\end{pgfscope}%
\begin{pgfscope}%
\pgfpathrectangle{\pgfqpoint{1.020000in}{0.880000in}}{\pgfqpoint{6.160000in}{6.160000in}}%
\pgfusepath{clip}%
\pgfsetbuttcap%
\pgfsetroundjoin%
\definecolor{currentfill}{rgb}{0.698454,0.799450,0.984577}%
\pgfsetfillcolor{currentfill}%
\pgfsetlinewidth{0.000000pt}%
\definecolor{currentstroke}{rgb}{0.000000,0.000000,0.000000}%
\pgfsetstrokecolor{currentstroke}%
\pgfsetdash{}{0pt}%
\pgfpathmoveto{\pgfqpoint{3.824825in}{3.610210in}}%
\pgfpathlineto{\pgfqpoint{3.834926in}{3.600918in}}%
\pgfpathlineto{\pgfqpoint{3.845037in}{3.594514in}}%
\pgfpathlineto{\pgfqpoint{3.878678in}{3.580971in}}%
\pgfpathlineto{\pgfqpoint{3.912290in}{3.566553in}}%
\pgfpathlineto{\pgfqpoint{3.902123in}{3.572341in}}%
\pgfpathlineto{\pgfqpoint{3.891968in}{3.580765in}}%
\pgfpathlineto{\pgfqpoint{3.858411in}{3.595822in}}%
\pgfpathlineto{\pgfqpoint{3.824825in}{3.610210in}}%
\pgfpathclose%
\pgfusepath{fill}%
\end{pgfscope}%
\begin{pgfscope}%
\pgfpathrectangle{\pgfqpoint{1.020000in}{0.880000in}}{\pgfqpoint{6.160000in}{6.160000in}}%
\pgfusepath{clip}%
\pgfsetbuttcap%
\pgfsetroundjoin%
\definecolor{currentfill}{rgb}{0.358415,0.478426,0.896795}%
\pgfsetfillcolor{currentfill}%
\pgfsetlinewidth{0.000000pt}%
\definecolor{currentstroke}{rgb}{0.000000,0.000000,0.000000}%
\pgfsetstrokecolor{currentstroke}%
\pgfsetdash{}{0pt}%
\pgfpathmoveto{\pgfqpoint{5.128674in}{3.000666in}}%
\pgfpathlineto{\pgfqpoint{5.139996in}{2.985916in}}%
\pgfpathlineto{\pgfqpoint{5.151336in}{2.970861in}}%
\pgfpathlineto{\pgfqpoint{5.184465in}{2.966951in}}%
\pgfpathlineto{\pgfqpoint{5.217577in}{2.963771in}}%
\pgfpathlineto{\pgfqpoint{5.206185in}{2.978933in}}%
\pgfpathlineto{\pgfqpoint{5.194812in}{2.993780in}}%
\pgfpathlineto{\pgfqpoint{5.161752in}{2.996870in}}%
\pgfpathlineto{\pgfqpoint{5.128674in}{3.000666in}}%
\pgfpathclose%
\pgfusepath{fill}%
\end{pgfscope}%
\begin{pgfscope}%
\pgfpathrectangle{\pgfqpoint{1.020000in}{0.880000in}}{\pgfqpoint{6.160000in}{6.160000in}}%
\pgfusepath{clip}%
\pgfsetbuttcap%
\pgfsetroundjoin%
\definecolor{currentfill}{rgb}{0.831148,0.859513,0.903110}%
\pgfsetfillcolor{currentfill}%
\pgfsetlinewidth{0.000000pt}%
\definecolor{currentstroke}{rgb}{0.000000,0.000000,0.000000}%
\pgfsetstrokecolor{currentstroke}%
\pgfsetdash{}{0pt}%
\pgfpathmoveto{\pgfqpoint{3.360698in}{3.900610in}}%
\pgfpathlineto{\pgfqpoint{3.370618in}{3.856232in}}%
\pgfpathlineto{\pgfqpoint{3.380523in}{3.815635in}}%
\pgfpathlineto{\pgfqpoint{3.414295in}{3.811793in}}%
\pgfpathlineto{\pgfqpoint{3.448053in}{3.806290in}}%
\pgfpathlineto{\pgfqpoint{3.438127in}{3.844355in}}%
\pgfpathlineto{\pgfqpoint{3.428189in}{3.886176in}}%
\pgfpathlineto{\pgfqpoint{3.394450in}{3.894301in}}%
\pgfpathlineto{\pgfqpoint{3.360698in}{3.900610in}}%
\pgfpathclose%
\pgfusepath{fill}%
\end{pgfscope}%
\begin{pgfscope}%
\pgfpathrectangle{\pgfqpoint{1.020000in}{0.880000in}}{\pgfqpoint{6.160000in}{6.160000in}}%
\pgfusepath{clip}%
\pgfsetbuttcap%
\pgfsetroundjoin%
\definecolor{currentfill}{rgb}{0.656683,0.771806,0.994914}%
\pgfsetfillcolor{currentfill}%
\pgfsetlinewidth{0.000000pt}%
\definecolor{currentstroke}{rgb}{0.000000,0.000000,0.000000}%
\pgfsetstrokecolor{currentstroke}%
\pgfsetdash{}{0pt}%
\pgfpathmoveto{\pgfqpoint{3.979425in}{3.535633in}}%
\pgfpathlineto{\pgfqpoint{3.989666in}{3.532323in}}%
\pgfpathlineto{\pgfqpoint{3.999924in}{3.531145in}}%
\pgfpathlineto{\pgfqpoint{4.033506in}{3.514466in}}%
\pgfpathlineto{\pgfqpoint{4.067055in}{3.497284in}}%
\pgfpathlineto{\pgfqpoint{4.056736in}{3.499120in}}%
\pgfpathlineto{\pgfqpoint{4.046436in}{3.502847in}}%
\pgfpathlineto{\pgfqpoint{4.012946in}{3.519405in}}%
\pgfpathlineto{\pgfqpoint{3.979425in}{3.535633in}}%
\pgfpathclose%
\pgfusepath{fill}%
\end{pgfscope}%
\begin{pgfscope}%
\pgfpathrectangle{\pgfqpoint{1.020000in}{0.880000in}}{\pgfqpoint{6.160000in}{6.160000in}}%
\pgfusepath{clip}%
\pgfsetbuttcap%
\pgfsetroundjoin%
\definecolor{currentfill}{rgb}{0.935774,0.812237,0.747156}%
\pgfsetfillcolor{currentfill}%
\pgfsetlinewidth{0.000000pt}%
\definecolor{currentstroke}{rgb}{0.000000,0.000000,0.000000}%
\pgfsetstrokecolor{currentstroke}%
\pgfsetdash{}{0pt}%
\pgfpathmoveto{\pgfqpoint{2.675280in}{4.159411in}}%
\pgfpathlineto{\pgfqpoint{2.684925in}{4.103331in}}%
\pgfpathlineto{\pgfqpoint{2.694571in}{4.047812in}}%
\pgfpathlineto{\pgfqpoint{2.728194in}{4.072565in}}%
\pgfpathlineto{\pgfqpoint{2.761817in}{4.096849in}}%
\pgfpathlineto{\pgfqpoint{2.752093in}{4.155610in}}%
\pgfpathlineto{\pgfqpoint{2.742370in}{4.215037in}}%
\pgfpathlineto{\pgfqpoint{2.708823in}{4.187515in}}%
\pgfpathlineto{\pgfqpoint{2.675280in}{4.159411in}}%
\pgfpathclose%
\pgfusepath{fill}%
\end{pgfscope}%
\begin{pgfscope}%
\pgfpathrectangle{\pgfqpoint{1.020000in}{0.880000in}}{\pgfqpoint{6.160000in}{6.160000in}}%
\pgfusepath{clip}%
\pgfsetbuttcap%
\pgfsetroundjoin%
\definecolor{currentfill}{rgb}{0.299441,0.400248,0.839842}%
\pgfsetfillcolor{currentfill}%
\pgfsetlinewidth{0.000000pt}%
\definecolor{currentstroke}{rgb}{0.000000,0.000000,0.000000}%
\pgfsetstrokecolor{currentstroke}%
\pgfsetdash{}{0pt}%
\pgfpathmoveto{\pgfqpoint{5.947935in}{2.881460in}}%
\pgfpathlineto{\pgfqpoint{5.960010in}{2.866254in}}%
\pgfpathlineto{\pgfqpoint{5.972107in}{2.851094in}}%
\pgfpathlineto{\pgfqpoint{6.005039in}{2.853201in}}%
\pgfpathlineto{\pgfqpoint{6.037950in}{2.855338in}}%
\pgfpathlineto{\pgfqpoint{6.025798in}{2.870410in}}%
\pgfpathlineto{\pgfqpoint{6.013670in}{2.885528in}}%
\pgfpathlineto{\pgfqpoint{5.980813in}{2.883479in}}%
\pgfpathlineto{\pgfqpoint{5.947935in}{2.881460in}}%
\pgfpathclose%
\pgfusepath{fill}%
\end{pgfscope}%
\begin{pgfscope}%
\pgfpathrectangle{\pgfqpoint{1.020000in}{0.880000in}}{\pgfqpoint{6.160000in}{6.160000in}}%
\pgfusepath{clip}%
\pgfsetbuttcap%
\pgfsetroundjoin%
\definecolor{currentfill}{rgb}{0.309060,0.413498,0.850128}%
\pgfsetfillcolor{currentfill}%
\pgfsetlinewidth{0.000000pt}%
\definecolor{currentstroke}{rgb}{0.000000,0.000000,0.000000}%
\pgfsetstrokecolor{currentstroke}%
\pgfsetdash{}{0pt}%
\pgfpathmoveto{\pgfqpoint{5.726501in}{2.901383in}}%
\pgfpathlineto{\pgfqpoint{5.738366in}{2.885843in}}%
\pgfpathlineto{\pgfqpoint{5.750255in}{2.870334in}}%
\pgfpathlineto{\pgfqpoint{5.783251in}{2.872026in}}%
\pgfpathlineto{\pgfqpoint{5.816227in}{2.873798in}}%
\pgfpathlineto{\pgfqpoint{5.804285in}{2.889221in}}%
\pgfpathlineto{\pgfqpoint{5.792365in}{2.904683in}}%
\pgfpathlineto{\pgfqpoint{5.759443in}{2.902994in}}%
\pgfpathlineto{\pgfqpoint{5.726501in}{2.901383in}}%
\pgfpathclose%
\pgfusepath{fill}%
\end{pgfscope}%
\begin{pgfscope}%
\pgfpathrectangle{\pgfqpoint{1.020000in}{0.880000in}}{\pgfqpoint{6.160000in}{6.160000in}}%
\pgfusepath{clip}%
\pgfsetbuttcap%
\pgfsetroundjoin%
\definecolor{currentfill}{rgb}{0.619318,0.744121,0.998931}%
\pgfsetfillcolor{currentfill}%
\pgfsetlinewidth{0.000000pt}%
\definecolor{currentstroke}{rgb}{0.000000,0.000000,0.000000}%
\pgfsetstrokecolor{currentstroke}%
\pgfsetdash{}{0pt}%
\pgfpathmoveto{\pgfqpoint{4.134056in}{3.462009in}}%
\pgfpathlineto{\pgfqpoint{4.144456in}{3.460729in}}%
\pgfpathlineto{\pgfqpoint{4.154877in}{3.460824in}}%
\pgfpathlineto{\pgfqpoint{4.188388in}{3.441554in}}%
\pgfpathlineto{\pgfqpoint{4.221864in}{3.422265in}}%
\pgfpathlineto{\pgfqpoint{4.211383in}{3.423774in}}%
\pgfpathlineto{\pgfqpoint{4.200924in}{3.426465in}}%
\pgfpathlineto{\pgfqpoint{4.167507in}{3.444204in}}%
\pgfpathlineto{\pgfqpoint{4.134056in}{3.462009in}}%
\pgfpathclose%
\pgfusepath{fill}%
\end{pgfscope}%
\begin{pgfscope}%
\pgfpathrectangle{\pgfqpoint{1.020000in}{0.880000in}}{\pgfqpoint{6.160000in}{6.160000in}}%
\pgfusepath{clip}%
\pgfsetbuttcap%
\pgfsetroundjoin%
\definecolor{currentfill}{rgb}{0.875557,0.860242,0.851430}%
\pgfsetfillcolor{currentfill}%
\pgfsetlinewidth{0.000000pt}%
\definecolor{currentstroke}{rgb}{0.000000,0.000000,0.000000}%
\pgfsetstrokecolor{currentstroke}%
\pgfsetdash{}{0pt}%
\pgfpathmoveto{\pgfqpoint{3.205774in}{4.013782in}}%
\pgfpathlineto{\pgfqpoint{3.215695in}{3.958680in}}%
\pgfpathlineto{\pgfqpoint{3.225596in}{3.906914in}}%
\pgfpathlineto{\pgfqpoint{3.259382in}{3.908201in}}%
\pgfpathlineto{\pgfqpoint{3.293162in}{3.907583in}}%
\pgfpathlineto{\pgfqpoint{3.283241in}{3.957738in}}%
\pgfpathlineto{\pgfqpoint{3.273304in}{4.011308in}}%
\pgfpathlineto{\pgfqpoint{3.239541in}{4.013646in}}%
\pgfpathlineto{\pgfqpoint{3.205774in}{4.013782in}}%
\pgfpathclose%
\pgfusepath{fill}%
\end{pgfscope}%
\begin{pgfscope}%
\pgfpathrectangle{\pgfqpoint{1.020000in}{0.880000in}}{\pgfqpoint{6.160000in}{6.160000in}}%
\pgfusepath{clip}%
\pgfsetbuttcap%
\pgfsetroundjoin%
\definecolor{currentfill}{rgb}{0.570616,0.704109,0.997195}%
\pgfsetfillcolor{currentfill}%
\pgfsetlinewidth{0.000000pt}%
\definecolor{currentstroke}{rgb}{0.000000,0.000000,0.000000}%
\pgfsetstrokecolor{currentstroke}%
\pgfsetdash{}{0pt}%
\pgfpathmoveto{\pgfqpoint{4.288716in}{3.384200in}}%
\pgfpathlineto{\pgfqpoint{4.299277in}{3.381844in}}%
\pgfpathlineto{\pgfqpoint{4.309861in}{3.380216in}}%
\pgfpathlineto{\pgfqpoint{4.343293in}{3.359726in}}%
\pgfpathlineto{\pgfqpoint{4.376690in}{3.339738in}}%
\pgfpathlineto{\pgfqpoint{4.366051in}{3.343379in}}%
\pgfpathlineto{\pgfqpoint{4.355434in}{3.347620in}}%
\pgfpathlineto{\pgfqpoint{4.322091in}{3.365672in}}%
\pgfpathlineto{\pgfqpoint{4.288716in}{3.384200in}}%
\pgfpathclose%
\pgfusepath{fill}%
\end{pgfscope}%
\begin{pgfscope}%
\pgfpathrectangle{\pgfqpoint{1.020000in}{0.880000in}}{\pgfqpoint{6.160000in}{6.160000in}}%
\pgfusepath{clip}%
\pgfsetbuttcap%
\pgfsetroundjoin%
\definecolor{currentfill}{rgb}{0.521696,0.659599,0.987736}%
\pgfsetfillcolor{currentfill}%
\pgfsetlinewidth{0.000000pt}%
\definecolor{currentstroke}{rgb}{0.000000,0.000000,0.000000}%
\pgfsetstrokecolor{currentstroke}%
\pgfsetdash{}{0pt}%
\pgfpathmoveto{\pgfqpoint{4.443386in}{3.301675in}}%
\pgfpathlineto{\pgfqpoint{4.454101in}{3.296490in}}%
\pgfpathlineto{\pgfqpoint{4.464838in}{3.291552in}}%
\pgfpathlineto{\pgfqpoint{4.498189in}{3.271724in}}%
\pgfpathlineto{\pgfqpoint{4.531509in}{3.252827in}}%
\pgfpathlineto{\pgfqpoint{4.520721in}{3.259652in}}%
\pgfpathlineto{\pgfqpoint{4.509955in}{3.266648in}}%
\pgfpathlineto{\pgfqpoint{4.476686in}{3.283756in}}%
\pgfpathlineto{\pgfqpoint{4.443386in}{3.301675in}}%
\pgfpathclose%
\pgfusepath{fill}%
\end{pgfscope}%
\begin{pgfscope}%
\pgfpathrectangle{\pgfqpoint{1.020000in}{0.880000in}}{\pgfqpoint{6.160000in}{6.160000in}}%
\pgfusepath{clip}%
\pgfsetbuttcap%
\pgfsetroundjoin%
\definecolor{currentfill}{rgb}{0.473070,0.611077,0.970634}%
\pgfsetfillcolor{currentfill}%
\pgfsetlinewidth{0.000000pt}%
\definecolor{currentstroke}{rgb}{0.000000,0.000000,0.000000}%
\pgfsetstrokecolor{currentstroke}%
\pgfsetdash{}{0pt}%
\pgfpathmoveto{\pgfqpoint{4.598059in}{3.218002in}}%
\pgfpathlineto{\pgfqpoint{4.608918in}{3.209571in}}%
\pgfpathlineto{\pgfqpoint{4.619798in}{3.201070in}}%
\pgfpathlineto{\pgfqpoint{4.653079in}{3.183717in}}%
\pgfpathlineto{\pgfqpoint{4.686332in}{3.167548in}}%
\pgfpathlineto{\pgfqpoint{4.675403in}{3.177456in}}%
\pgfpathlineto{\pgfqpoint{4.664496in}{3.187236in}}%
\pgfpathlineto{\pgfqpoint{4.631291in}{3.202113in}}%
\pgfpathlineto{\pgfqpoint{4.598059in}{3.218002in}}%
\pgfpathclose%
\pgfusepath{fill}%
\end{pgfscope}%
\begin{pgfscope}%
\pgfpathrectangle{\pgfqpoint{1.020000in}{0.880000in}}{\pgfqpoint{6.160000in}{6.160000in}}%
\pgfusepath{clip}%
\pgfsetbuttcap%
\pgfsetroundjoin%
\definecolor{currentfill}{rgb}{0.922681,0.828568,0.777054}%
\pgfsetfillcolor{currentfill}%
\pgfsetlinewidth{0.000000pt}%
\definecolor{currentstroke}{rgb}{0.000000,0.000000,0.000000}%
\pgfsetstrokecolor{currentstroke}%
\pgfsetdash{}{0pt}%
\pgfpathmoveto{\pgfqpoint{2.608189in}{4.102602in}}%
\pgfpathlineto{\pgfqpoint{2.617748in}{4.050022in}}%
\pgfpathlineto{\pgfqpoint{2.627310in}{3.997906in}}%
\pgfpathlineto{\pgfqpoint{2.660944in}{4.022846in}}%
\pgfpathlineto{\pgfqpoint{2.694571in}{4.047812in}}%
\pgfpathlineto{\pgfqpoint{2.684925in}{4.103331in}}%
\pgfpathlineto{\pgfqpoint{2.675280in}{4.159411in}}%
\pgfpathlineto{\pgfqpoint{2.641737in}{4.131015in}}%
\pgfpathlineto{\pgfqpoint{2.608189in}{4.102602in}}%
\pgfpathclose%
\pgfusepath{fill}%
\end{pgfscope}%
\begin{pgfscope}%
\pgfpathrectangle{\pgfqpoint{1.020000in}{0.880000in}}{\pgfqpoint{6.160000in}{6.160000in}}%
\pgfusepath{clip}%
\pgfsetbuttcap%
\pgfsetroundjoin%
\definecolor{currentfill}{rgb}{0.323718,0.433158,0.864722}%
\pgfsetfillcolor{currentfill}%
\pgfsetlinewidth{0.000000pt}%
\definecolor{currentstroke}{rgb}{0.000000,0.000000,0.000000}%
\pgfsetstrokecolor{currentstroke}%
\pgfsetdash{}{0pt}%
\pgfpathmoveto{\pgfqpoint{5.505099in}{2.925826in}}%
\pgfpathlineto{\pgfqpoint{5.516760in}{2.910114in}}%
\pgfpathlineto{\pgfqpoint{5.528443in}{2.894371in}}%
\pgfpathlineto{\pgfqpoint{5.561500in}{2.895106in}}%
\pgfpathlineto{\pgfqpoint{5.594538in}{2.896053in}}%
\pgfpathlineto{\pgfqpoint{5.582802in}{2.911744in}}%
\pgfpathlineto{\pgfqpoint{5.571088in}{2.927421in}}%
\pgfpathlineto{\pgfqpoint{5.538103in}{2.926520in}}%
\pgfpathlineto{\pgfqpoint{5.505099in}{2.925826in}}%
\pgfpathclose%
\pgfusepath{fill}%
\end{pgfscope}%
\begin{pgfscope}%
\pgfpathrectangle{\pgfqpoint{1.020000in}{0.880000in}}{\pgfqpoint{6.160000in}{6.160000in}}%
\pgfusepath{clip}%
\pgfsetbuttcap%
\pgfsetroundjoin%
\definecolor{currentfill}{rgb}{0.938326,0.808917,0.741162}%
\pgfsetfillcolor{currentfill}%
\pgfsetlinewidth{0.000000pt}%
\definecolor{currentstroke}{rgb}{0.000000,0.000000,0.000000}%
\pgfsetstrokecolor{currentstroke}%
\pgfsetdash{}{0pt}%
\pgfpathmoveto{\pgfqpoint{2.896356in}{4.184081in}}%
\pgfpathlineto{\pgfqpoint{2.906202in}{4.121162in}}%
\pgfpathlineto{\pgfqpoint{2.916038in}{4.059852in}}%
\pgfpathlineto{\pgfqpoint{2.949746in}{4.076040in}}%
\pgfpathlineto{\pgfqpoint{2.983460in}{4.090547in}}%
\pgfpathlineto{\pgfqpoint{2.973577in}{4.153470in}}%
\pgfpathlineto{\pgfqpoint{2.963682in}{4.218157in}}%
\pgfpathlineto{\pgfqpoint{2.930013in}{4.202117in}}%
\pgfpathlineto{\pgfqpoint{2.896356in}{4.184081in}}%
\pgfpathclose%
\pgfusepath{fill}%
\end{pgfscope}%
\begin{pgfscope}%
\pgfpathrectangle{\pgfqpoint{1.020000in}{0.880000in}}{\pgfqpoint{6.160000in}{6.160000in}}%
\pgfusepath{clip}%
\pgfsetbuttcap%
\pgfsetroundjoin%
\definecolor{currentfill}{rgb}{0.430507,0.564883,0.948889}%
\pgfsetfillcolor{currentfill}%
\pgfsetlinewidth{0.000000pt}%
\definecolor{currentstroke}{rgb}{0.000000,0.000000,0.000000}%
\pgfsetstrokecolor{currentstroke}%
\pgfsetdash{}{0pt}%
\pgfpathmoveto{\pgfqpoint{4.752760in}{3.138710in}}%
\pgfpathlineto{\pgfqpoint{4.763757in}{3.127502in}}%
\pgfpathlineto{\pgfqpoint{4.774774in}{3.116036in}}%
\pgfpathlineto{\pgfqpoint{4.808000in}{3.102417in}}%
\pgfpathlineto{\pgfqpoint{4.841202in}{3.090033in}}%
\pgfpathlineto{\pgfqpoint{4.830137in}{3.102343in}}%
\pgfpathlineto{\pgfqpoint{4.819092in}{3.114337in}}%
\pgfpathlineto{\pgfqpoint{4.785938in}{3.125985in}}%
\pgfpathlineto{\pgfqpoint{4.752760in}{3.138710in}}%
\pgfpathclose%
\pgfusepath{fill}%
\end{pgfscope}%
\begin{pgfscope}%
\pgfpathrectangle{\pgfqpoint{1.020000in}{0.880000in}}{\pgfqpoint{6.160000in}{6.160000in}}%
\pgfusepath{clip}%
\pgfsetbuttcap%
\pgfsetroundjoin%
\definecolor{currentfill}{rgb}{0.916071,0.833977,0.788693}%
\pgfsetfillcolor{currentfill}%
\pgfsetlinewidth{0.000000pt}%
\definecolor{currentstroke}{rgb}{0.000000,0.000000,0.000000}%
\pgfsetstrokecolor{currentstroke}%
\pgfsetdash{}{0pt}%
\pgfpathmoveto{\pgfqpoint{3.050909in}{4.113768in}}%
\pgfpathlineto{\pgfqpoint{3.060816in}{4.052155in}}%
\pgfpathlineto{\pgfqpoint{3.070704in}{3.993106in}}%
\pgfpathlineto{\pgfqpoint{3.104470in}{4.001315in}}%
\pgfpathlineto{\pgfqpoint{3.138237in}{4.007558in}}%
\pgfpathlineto{\pgfqpoint{3.128320in}{4.066536in}}%
\pgfpathlineto{\pgfqpoint{3.118386in}{4.128222in}}%
\pgfpathlineto{\pgfqpoint{3.084645in}{4.122157in}}%
\pgfpathlineto{\pgfqpoint{3.050909in}{4.113768in}}%
\pgfpathclose%
\pgfusepath{fill}%
\end{pgfscope}%
\begin{pgfscope}%
\pgfpathrectangle{\pgfqpoint{1.020000in}{0.880000in}}{\pgfqpoint{6.160000in}{6.160000in}}%
\pgfusepath{clip}%
\pgfsetbuttcap%
\pgfsetroundjoin%
\definecolor{currentfill}{rgb}{0.338377,0.452819,0.879317}%
\pgfsetfillcolor{currentfill}%
\pgfsetlinewidth{0.000000pt}%
\definecolor{currentstroke}{rgb}{0.000000,0.000000,0.000000}%
\pgfsetstrokecolor{currentstroke}%
\pgfsetdash{}{0pt}%
\pgfpathmoveto{\pgfqpoint{5.283749in}{2.959329in}}%
\pgfpathlineto{\pgfqpoint{5.295212in}{2.943867in}}%
\pgfpathlineto{\pgfqpoint{5.306694in}{2.928239in}}%
\pgfpathlineto{\pgfqpoint{5.339806in}{2.926813in}}%
\pgfpathlineto{\pgfqpoint{5.372900in}{2.925868in}}%
\pgfpathlineto{\pgfqpoint{5.361365in}{2.941518in}}%
\pgfpathlineto{\pgfqpoint{5.349850in}{2.957018in}}%
\pgfpathlineto{\pgfqpoint{5.316808in}{2.957937in}}%
\pgfpathlineto{\pgfqpoint{5.283749in}{2.959329in}}%
\pgfpathclose%
\pgfusepath{fill}%
\end{pgfscope}%
\begin{pgfscope}%
\pgfpathrectangle{\pgfqpoint{1.020000in}{0.880000in}}{\pgfqpoint{6.160000in}{6.160000in}}%
\pgfusepath{clip}%
\pgfsetbuttcap%
\pgfsetroundjoin%
\definecolor{currentfill}{rgb}{0.902849,0.844796,0.811970}%
\pgfsetfillcolor{currentfill}%
\pgfsetlinewidth{0.000000pt}%
\definecolor{currentstroke}{rgb}{0.000000,0.000000,0.000000}%
\pgfsetstrokecolor{currentstroke}%
\pgfsetdash{}{0pt}%
\pgfpathmoveto{\pgfqpoint{2.541066in}{4.046755in}}%
\pgfpathlineto{\pgfqpoint{2.550534in}{3.997690in}}%
\pgfpathlineto{\pgfqpoint{2.560007in}{3.949000in}}%
\pgfpathlineto{\pgfqpoint{2.593666in}{3.973220in}}%
\pgfpathlineto{\pgfqpoint{2.627310in}{3.997906in}}%
\pgfpathlineto{\pgfqpoint{2.617748in}{4.050022in}}%
\pgfpathlineto{\pgfqpoint{2.608189in}{4.102602in}}%
\pgfpathlineto{\pgfqpoint{2.574633in}{4.074435in}}%
\pgfpathlineto{\pgfqpoint{2.541066in}{4.046755in}}%
\pgfpathclose%
\pgfusepath{fill}%
\end{pgfscope}%
\begin{pgfscope}%
\pgfpathrectangle{\pgfqpoint{1.020000in}{0.880000in}}{\pgfqpoint{6.160000in}{6.160000in}}%
\pgfusepath{clip}%
\pgfsetbuttcap%
\pgfsetroundjoin%
\definecolor{currentfill}{rgb}{0.394042,0.522413,0.924916}%
\pgfsetfillcolor{currentfill}%
\pgfsetlinewidth{0.000000pt}%
\definecolor{currentstroke}{rgb}{0.000000,0.000000,0.000000}%
\pgfsetstrokecolor{currentstroke}%
\pgfsetdash{}{0pt}%
\pgfpathmoveto{\pgfqpoint{4.907542in}{3.068769in}}%
\pgfpathlineto{\pgfqpoint{4.918675in}{3.055541in}}%
\pgfpathlineto{\pgfqpoint{4.929827in}{3.041981in}}%
\pgfpathlineto{\pgfqpoint{4.963015in}{3.032541in}}%
\pgfpathlineto{\pgfqpoint{4.996184in}{3.024217in}}%
\pgfpathlineto{\pgfqpoint{4.984983in}{3.038197in}}%
\pgfpathlineto{\pgfqpoint{4.973800in}{3.051801in}}%
\pgfpathlineto{\pgfqpoint{4.940681in}{3.059778in}}%
\pgfpathlineto{\pgfqpoint{4.907542in}{3.068769in}}%
\pgfpathclose%
\pgfusepath{fill}%
\end{pgfscope}%
\begin{pgfscope}%
\pgfpathrectangle{\pgfqpoint{1.020000in}{0.880000in}}{\pgfqpoint{6.160000in}{6.160000in}}%
\pgfusepath{clip}%
\pgfsetbuttcap%
\pgfsetroundjoin%
\definecolor{currentfill}{rgb}{0.883687,0.856108,0.840258}%
\pgfsetfillcolor{currentfill}%
\pgfsetlinewidth{0.000000pt}%
\definecolor{currentstroke}{rgb}{0.000000,0.000000,0.000000}%
\pgfsetstrokecolor{currentstroke}%
\pgfsetdash{}{0pt}%
\pgfpathmoveto{\pgfqpoint{2.473879in}{3.993718in}}%
\pgfpathlineto{\pgfqpoint{2.483254in}{3.948060in}}%
\pgfpathlineto{\pgfqpoint{2.492636in}{3.902696in}}%
\pgfpathlineto{\pgfqpoint{2.526332in}{3.925435in}}%
\pgfpathlineto{\pgfqpoint{2.560007in}{3.949000in}}%
\pgfpathlineto{\pgfqpoint{2.550534in}{3.997690in}}%
\pgfpathlineto{\pgfqpoint{2.541066in}{4.046755in}}%
\pgfpathlineto{\pgfqpoint{2.507482in}{4.019784in}}%
\pgfpathlineto{\pgfqpoint{2.473879in}{3.993718in}}%
\pgfpathclose%
\pgfusepath{fill}%
\end{pgfscope}%
\begin{pgfscope}%
\pgfpathrectangle{\pgfqpoint{1.020000in}{0.880000in}}{\pgfqpoint{6.160000in}{6.160000in}}%
\pgfusepath{clip}%
\pgfsetbuttcap%
\pgfsetroundjoin%
\definecolor{currentfill}{rgb}{0.930669,0.818877,0.759146}%
\pgfsetfillcolor{currentfill}%
\pgfsetlinewidth{0.000000pt}%
\definecolor{currentstroke}{rgb}{0.000000,0.000000,0.000000}%
\pgfsetstrokecolor{currentstroke}%
\pgfsetdash{}{0pt}%
\pgfpathmoveto{\pgfqpoint{2.829072in}{4.142972in}}%
\pgfpathlineto{\pgfqpoint{2.838860in}{4.082400in}}%
\pgfpathlineto{\pgfqpoint{2.848639in}{4.023272in}}%
\pgfpathlineto{\pgfqpoint{2.882337in}{4.042190in}}%
\pgfpathlineto{\pgfqpoint{2.916038in}{4.059852in}}%
\pgfpathlineto{\pgfqpoint{2.906202in}{4.121162in}}%
\pgfpathlineto{\pgfqpoint{2.896356in}{4.184081in}}%
\pgfpathlineto{\pgfqpoint{2.862710in}{4.164283in}}%
\pgfpathlineto{\pgfqpoint{2.829072in}{4.142972in}}%
\pgfpathclose%
\pgfusepath{fill}%
\end{pgfscope}%
\begin{pgfscope}%
\pgfpathrectangle{\pgfqpoint{1.020000in}{0.880000in}}{\pgfqpoint{6.160000in}{6.160000in}}%
\pgfusepath{clip}%
\pgfsetbuttcap%
\pgfsetroundjoin%
\definecolor{currentfill}{rgb}{0.796064,0.848693,0.933471}%
\pgfsetfillcolor{currentfill}%
\pgfsetlinewidth{0.000000pt}%
\definecolor{currentstroke}{rgb}{0.000000,0.000000,0.000000}%
\pgfsetstrokecolor{currentstroke}%
\pgfsetdash{}{0pt}%
\pgfpathmoveto{\pgfqpoint{3.448053in}{3.806290in}}%
\pgfpathlineto{\pgfqpoint{3.457966in}{3.772079in}}%
\pgfpathlineto{\pgfqpoint{3.467868in}{3.741774in}}%
\pgfpathlineto{\pgfqpoint{3.501638in}{3.737267in}}%
\pgfpathlineto{\pgfqpoint{3.535392in}{3.731247in}}%
\pgfpathlineto{\pgfqpoint{3.525461in}{3.758959in}}%
\pgfpathlineto{\pgfqpoint{3.515522in}{3.790498in}}%
\pgfpathlineto{\pgfqpoint{3.481796in}{3.799170in}}%
\pgfpathlineto{\pgfqpoint{3.448053in}{3.806290in}}%
\pgfpathclose%
\pgfusepath{fill}%
\end{pgfscope}%
\begin{pgfscope}%
\pgfpathrectangle{\pgfqpoint{1.020000in}{0.880000in}}{\pgfqpoint{6.160000in}{6.160000in}}%
\pgfusepath{clip}%
\pgfsetbuttcap%
\pgfsetroundjoin%
\definecolor{currentfill}{rgb}{0.753611,0.830233,0.960871}%
\pgfsetfillcolor{currentfill}%
\pgfsetlinewidth{0.000000pt}%
\definecolor{currentstroke}{rgb}{0.000000,0.000000,0.000000}%
\pgfsetstrokecolor{currentstroke}%
\pgfsetdash{}{0pt}%
\pgfpathmoveto{\pgfqpoint{3.602845in}{3.714897in}}%
\pgfpathlineto{\pgfqpoint{3.612804in}{3.693479in}}%
\pgfpathlineto{\pgfqpoint{3.622760in}{3.675734in}}%
\pgfpathlineto{\pgfqpoint{3.656495in}{3.667822in}}%
\pgfpathlineto{\pgfqpoint{3.690210in}{3.658581in}}%
\pgfpathlineto{\pgfqpoint{3.680211in}{3.674200in}}%
\pgfpathlineto{\pgfqpoint{3.670214in}{3.693317in}}%
\pgfpathlineto{\pgfqpoint{3.636541in}{3.704715in}}%
\pgfpathlineto{\pgfqpoint{3.602845in}{3.714897in}}%
\pgfpathclose%
\pgfusepath{fill}%
\end{pgfscope}%
\begin{pgfscope}%
\pgfpathrectangle{\pgfqpoint{1.020000in}{0.880000in}}{\pgfqpoint{6.160000in}{6.160000in}}%
\pgfusepath{clip}%
\pgfsetbuttcap%
\pgfsetroundjoin%
\definecolor{currentfill}{rgb}{0.368507,0.491141,0.905243}%
\pgfsetfillcolor{currentfill}%
\pgfsetlinewidth{0.000000pt}%
\definecolor{currentstroke}{rgb}{0.000000,0.000000,0.000000}%
\pgfsetstrokecolor{currentstroke}%
\pgfsetdash{}{0pt}%
\pgfpathmoveto{\pgfqpoint{5.062466in}{3.010633in}}%
\pgfpathlineto{\pgfqpoint{5.073736in}{2.996049in}}%
\pgfpathlineto{\pgfqpoint{5.085025in}{2.981172in}}%
\pgfpathlineto{\pgfqpoint{5.118189in}{2.975576in}}%
\pgfpathlineto{\pgfqpoint{5.151336in}{2.970861in}}%
\pgfpathlineto{\pgfqpoint{5.139996in}{2.985916in}}%
\pgfpathlineto{\pgfqpoint{5.128674in}{3.000666in}}%
\pgfpathlineto{\pgfqpoint{5.095579in}{3.005232in}}%
\pgfpathlineto{\pgfqpoint{5.062466in}{3.010633in}}%
\pgfpathclose%
\pgfusepath{fill}%
\end{pgfscope}%
\begin{pgfscope}%
\pgfpathrectangle{\pgfqpoint{1.020000in}{0.880000in}}{\pgfqpoint{6.160000in}{6.160000in}}%
\pgfusepath{clip}%
\pgfsetbuttcap%
\pgfsetroundjoin%
\definecolor{currentfill}{rgb}{0.839351,0.861167,0.894494}%
\pgfsetfillcolor{currentfill}%
\pgfsetlinewidth{0.000000pt}%
\definecolor{currentstroke}{rgb}{0.000000,0.000000,0.000000}%
\pgfsetstrokecolor{currentstroke}%
\pgfsetdash{}{0pt}%
\pgfpathmoveto{\pgfqpoint{3.293162in}{3.907583in}}%
\pgfpathlineto{\pgfqpoint{3.303063in}{3.861034in}}%
\pgfpathlineto{\pgfqpoint{3.312947in}{3.818236in}}%
\pgfpathlineto{\pgfqpoint{3.346740in}{3.817785in}}%
\pgfpathlineto{\pgfqpoint{3.380523in}{3.815635in}}%
\pgfpathlineto{\pgfqpoint{3.370618in}{3.856232in}}%
\pgfpathlineto{\pgfqpoint{3.360698in}{3.900610in}}%
\pgfpathlineto{\pgfqpoint{3.326934in}{3.905049in}}%
\pgfpathlineto{\pgfqpoint{3.293162in}{3.907583in}}%
\pgfpathclose%
\pgfusepath{fill}%
\end{pgfscope}%
\begin{pgfscope}%
\pgfpathrectangle{\pgfqpoint{1.020000in}{0.880000in}}{\pgfqpoint{6.160000in}{6.160000in}}%
\pgfusepath{clip}%
\pgfsetbuttcap%
\pgfsetroundjoin%
\definecolor{currentfill}{rgb}{0.299441,0.400248,0.839842}%
\pgfsetfillcolor{currentfill}%
\pgfsetlinewidth{0.000000pt}%
\definecolor{currentstroke}{rgb}{0.000000,0.000000,0.000000}%
\pgfsetstrokecolor{currentstroke}%
\pgfsetdash{}{0pt}%
\pgfpathmoveto{\pgfqpoint{5.882121in}{2.877532in}}%
\pgfpathlineto{\pgfqpoint{5.894141in}{2.862238in}}%
\pgfpathlineto{\pgfqpoint{5.906184in}{2.846989in}}%
\pgfpathlineto{\pgfqpoint{5.939156in}{2.849021in}}%
\pgfpathlineto{\pgfqpoint{5.972107in}{2.851094in}}%
\pgfpathlineto{\pgfqpoint{5.960010in}{2.866254in}}%
\pgfpathlineto{\pgfqpoint{5.947935in}{2.881460in}}%
\pgfpathlineto{\pgfqpoint{5.915038in}{2.879475in}}%
\pgfpathlineto{\pgfqpoint{5.882121in}{2.877532in}}%
\pgfpathclose%
\pgfusepath{fill}%
\end{pgfscope}%
\begin{pgfscope}%
\pgfpathrectangle{\pgfqpoint{1.020000in}{0.880000in}}{\pgfqpoint{6.160000in}{6.160000in}}%
\pgfusepath{clip}%
\pgfsetbuttcap%
\pgfsetroundjoin%
\definecolor{currentfill}{rgb}{0.313946,0.420052,0.854993}%
\pgfsetfillcolor{currentfill}%
\pgfsetlinewidth{0.000000pt}%
\definecolor{currentstroke}{rgb}{0.000000,0.000000,0.000000}%
\pgfsetstrokecolor{currentstroke}%
\pgfsetdash{}{0pt}%
\pgfpathmoveto{\pgfqpoint{5.660558in}{2.898457in}}%
\pgfpathlineto{\pgfqpoint{5.672370in}{2.882841in}}%
\pgfpathlineto{\pgfqpoint{5.684204in}{2.867250in}}%
\pgfpathlineto{\pgfqpoint{5.717239in}{2.868736in}}%
\pgfpathlineto{\pgfqpoint{5.750255in}{2.870334in}}%
\pgfpathlineto{\pgfqpoint{5.738366in}{2.885843in}}%
\pgfpathlineto{\pgfqpoint{5.726501in}{2.901383in}}%
\pgfpathlineto{\pgfqpoint{5.693539in}{2.899865in}}%
\pgfpathlineto{\pgfqpoint{5.660558in}{2.898457in}}%
\pgfpathclose%
\pgfusepath{fill}%
\end{pgfscope}%
\begin{pgfscope}%
\pgfpathrectangle{\pgfqpoint{1.020000in}{0.880000in}}{\pgfqpoint{6.160000in}{6.160000in}}%
\pgfusepath{clip}%
\pgfsetbuttcap%
\pgfsetroundjoin%
\definecolor{currentfill}{rgb}{0.863392,0.865084,0.867634}%
\pgfsetfillcolor{currentfill}%
\pgfsetlinewidth{0.000000pt}%
\definecolor{currentstroke}{rgb}{0.000000,0.000000,0.000000}%
\pgfsetstrokecolor{currentstroke}%
\pgfsetdash{}{0pt}%
\pgfpathmoveto{\pgfqpoint{2.406599in}{3.944965in}}%
\pgfpathlineto{\pgfqpoint{2.415881in}{3.902507in}}%
\pgfpathlineto{\pgfqpoint{2.425170in}{3.860270in}}%
\pgfpathlineto{\pgfqpoint{2.458917in}{3.880932in}}%
\pgfpathlineto{\pgfqpoint{2.492636in}{3.902696in}}%
\pgfpathlineto{\pgfqpoint{2.483254in}{3.948060in}}%
\pgfpathlineto{\pgfqpoint{2.473879in}{3.993718in}}%
\pgfpathlineto{\pgfqpoint{2.440252in}{3.968730in}}%
\pgfpathlineto{\pgfqpoint{2.406599in}{3.944965in}}%
\pgfpathclose%
\pgfusepath{fill}%
\end{pgfscope}%
\begin{pgfscope}%
\pgfpathrectangle{\pgfqpoint{1.020000in}{0.880000in}}{\pgfqpoint{6.160000in}{6.160000in}}%
\pgfusepath{clip}%
\pgfsetbuttcap%
\pgfsetroundjoin%
\definecolor{currentfill}{rgb}{0.718985,0.811993,0.977656}%
\pgfsetfillcolor{currentfill}%
\pgfsetlinewidth{0.000000pt}%
\definecolor{currentstroke}{rgb}{0.000000,0.000000,0.000000}%
\pgfsetstrokecolor{currentstroke}%
\pgfsetdash{}{0pt}%
\pgfpathmoveto{\pgfqpoint{3.757568in}{3.636466in}}%
\pgfpathlineto{\pgfqpoint{3.767618in}{3.625911in}}%
\pgfpathlineto{\pgfqpoint{3.777675in}{3.618475in}}%
\pgfpathlineto{\pgfqpoint{3.811369in}{3.607055in}}%
\pgfpathlineto{\pgfqpoint{3.845037in}{3.594514in}}%
\pgfpathlineto{\pgfqpoint{3.834926in}{3.600918in}}%
\pgfpathlineto{\pgfqpoint{3.824825in}{3.610210in}}%
\pgfpathlineto{\pgfqpoint{3.791210in}{3.623799in}}%
\pgfpathlineto{\pgfqpoint{3.757568in}{3.636466in}}%
\pgfpathclose%
\pgfusepath{fill}%
\end{pgfscope}%
\begin{pgfscope}%
\pgfpathrectangle{\pgfqpoint{1.020000in}{0.880000in}}{\pgfqpoint{6.160000in}{6.160000in}}%
\pgfusepath{clip}%
\pgfsetbuttcap%
\pgfsetroundjoin%
\definecolor{currentfill}{rgb}{0.879622,0.858175,0.845844}%
\pgfsetfillcolor{currentfill}%
\pgfsetlinewidth{0.000000pt}%
\definecolor{currentstroke}{rgb}{0.000000,0.000000,0.000000}%
\pgfsetstrokecolor{currentstroke}%
\pgfsetdash{}{0pt}%
\pgfpathmoveto{\pgfqpoint{3.138237in}{4.007558in}}%
\pgfpathlineto{\pgfqpoint{3.148135in}{3.951563in}}%
\pgfpathlineto{\pgfqpoint{3.158013in}{3.898782in}}%
\pgfpathlineto{\pgfqpoint{3.191806in}{3.903756in}}%
\pgfpathlineto{\pgfqpoint{3.225596in}{3.906914in}}%
\pgfpathlineto{\pgfqpoint{3.215695in}{3.958680in}}%
\pgfpathlineto{\pgfqpoint{3.205774in}{4.013782in}}%
\pgfpathlineto{\pgfqpoint{3.172006in}{4.011737in}}%
\pgfpathlineto{\pgfqpoint{3.138237in}{4.007558in}}%
\pgfpathclose%
\pgfusepath{fill}%
\end{pgfscope}%
\begin{pgfscope}%
\pgfpathrectangle{\pgfqpoint{1.020000in}{0.880000in}}{\pgfqpoint{6.160000in}{6.160000in}}%
\pgfusepath{clip}%
\pgfsetbuttcap%
\pgfsetroundjoin%
\definecolor{currentfill}{rgb}{0.912765,0.836682,0.794512}%
\pgfsetfillcolor{currentfill}%
\pgfsetlinewidth{0.000000pt}%
\definecolor{currentstroke}{rgb}{0.000000,0.000000,0.000000}%
\pgfsetstrokecolor{currentstroke}%
\pgfsetdash{}{0pt}%
\pgfpathmoveto{\pgfqpoint{2.983460in}{4.090547in}}%
\pgfpathlineto{\pgfqpoint{2.993329in}{4.029721in}}%
\pgfpathlineto{\pgfqpoint{3.003181in}{3.971284in}}%
\pgfpathlineto{\pgfqpoint{3.036941in}{3.983050in}}%
\pgfpathlineto{\pgfqpoint{3.070704in}{3.993106in}}%
\pgfpathlineto{\pgfqpoint{3.060816in}{4.052155in}}%
\pgfpathlineto{\pgfqpoint{3.050909in}{4.113768in}}%
\pgfpathlineto{\pgfqpoint{3.017181in}{4.103181in}}%
\pgfpathlineto{\pgfqpoint{2.983460in}{4.090547in}}%
\pgfpathclose%
\pgfusepath{fill}%
\end{pgfscope}%
\begin{pgfscope}%
\pgfpathrectangle{\pgfqpoint{1.020000in}{0.880000in}}{\pgfqpoint{6.160000in}{6.160000in}}%
\pgfusepath{clip}%
\pgfsetbuttcap%
\pgfsetroundjoin%
\definecolor{currentfill}{rgb}{0.294718,0.393542,0.834384}%
\pgfsetfillcolor{currentfill}%
\pgfsetlinewidth{0.000000pt}%
\definecolor{currentstroke}{rgb}{0.000000,0.000000,0.000000}%
\pgfsetstrokecolor{currentstroke}%
\pgfsetdash{}{0pt}%
\pgfpathmoveto{\pgfqpoint{6.103713in}{2.859684in}}%
\pgfpathlineto{\pgfqpoint{6.115943in}{2.844744in}}%
\pgfpathlineto{\pgfqpoint{6.128197in}{2.829851in}}%
\pgfpathlineto{\pgfqpoint{6.161103in}{2.832139in}}%
\pgfpathlineto{\pgfqpoint{6.148821in}{2.846990in}}%
\pgfpathlineto{\pgfqpoint{6.136564in}{2.861888in}}%
\pgfpathlineto{\pgfqpoint{6.103713in}{2.859684in}}%
\pgfpathclose%
\pgfusepath{fill}%
\end{pgfscope}%
\begin{pgfscope}%
\pgfpathrectangle{\pgfqpoint{1.020000in}{0.880000in}}{\pgfqpoint{6.160000in}{6.160000in}}%
\pgfusepath{clip}%
\pgfsetbuttcap%
\pgfsetroundjoin%
\definecolor{currentfill}{rgb}{0.919376,0.831273,0.782874}%
\pgfsetfillcolor{currentfill}%
\pgfsetlinewidth{0.000000pt}%
\definecolor{currentstroke}{rgb}{0.000000,0.000000,0.000000}%
\pgfsetstrokecolor{currentstroke}%
\pgfsetdash{}{0pt}%
\pgfpathmoveto{\pgfqpoint{2.761817in}{4.096849in}}%
\pgfpathlineto{\pgfqpoint{2.771537in}{4.039073in}}%
\pgfpathlineto{\pgfqpoint{2.781250in}{3.982572in}}%
\pgfpathlineto{\pgfqpoint{2.814944in}{4.003323in}}%
\pgfpathlineto{\pgfqpoint{2.848639in}{4.023272in}}%
\pgfpathlineto{\pgfqpoint{2.838860in}{4.082400in}}%
\pgfpathlineto{\pgfqpoint{2.829072in}{4.142972in}}%
\pgfpathlineto{\pgfqpoint{2.795442in}{4.120406in}}%
\pgfpathlineto{\pgfqpoint{2.761817in}{4.096849in}}%
\pgfpathclose%
\pgfusepath{fill}%
\end{pgfscope}%
\begin{pgfscope}%
\pgfpathrectangle{\pgfqpoint{1.020000in}{0.880000in}}{\pgfqpoint{6.160000in}{6.160000in}}%
\pgfusepath{clip}%
\pgfsetbuttcap%
\pgfsetroundjoin%
\definecolor{currentfill}{rgb}{0.323718,0.433158,0.864722}%
\pgfsetfillcolor{currentfill}%
\pgfsetlinewidth{0.000000pt}%
\definecolor{currentstroke}{rgb}{0.000000,0.000000,0.000000}%
\pgfsetstrokecolor{currentstroke}%
\pgfsetdash{}{0pt}%
\pgfpathmoveto{\pgfqpoint{5.439036in}{2.925198in}}%
\pgfpathlineto{\pgfqpoint{5.450644in}{2.909460in}}%
\pgfpathlineto{\pgfqpoint{5.462273in}{2.893674in}}%
\pgfpathlineto{\pgfqpoint{5.495367in}{2.893880in}}%
\pgfpathlineto{\pgfqpoint{5.528443in}{2.894371in}}%
\pgfpathlineto{\pgfqpoint{5.516760in}{2.910114in}}%
\pgfpathlineto{\pgfqpoint{5.505099in}{2.925826in}}%
\pgfpathlineto{\pgfqpoint{5.472077in}{2.925372in}}%
\pgfpathlineto{\pgfqpoint{5.439036in}{2.925198in}}%
\pgfpathclose%
\pgfusepath{fill}%
\end{pgfscope}%
\begin{pgfscope}%
\pgfpathrectangle{\pgfqpoint{1.020000in}{0.880000in}}{\pgfqpoint{6.160000in}{6.160000in}}%
\pgfusepath{clip}%
\pgfsetbuttcap%
\pgfsetroundjoin%
\definecolor{currentfill}{rgb}{0.683056,0.790043,0.989768}%
\pgfsetfillcolor{currentfill}%
\pgfsetlinewidth{0.000000pt}%
\definecolor{currentstroke}{rgb}{0.000000,0.000000,0.000000}%
\pgfsetstrokecolor{currentstroke}%
\pgfsetdash{}{0pt}%
\pgfpathmoveto{\pgfqpoint{3.912290in}{3.566553in}}%
\pgfpathlineto{\pgfqpoint{3.922472in}{3.563285in}}%
\pgfpathlineto{\pgfqpoint{3.932668in}{3.562388in}}%
\pgfpathlineto{\pgfqpoint{3.966312in}{3.547170in}}%
\pgfpathlineto{\pgfqpoint{3.999924in}{3.531145in}}%
\pgfpathlineto{\pgfqpoint{3.989666in}{3.532323in}}%
\pgfpathlineto{\pgfqpoint{3.979425in}{3.535633in}}%
\pgfpathlineto{\pgfqpoint{3.945873in}{3.551395in}}%
\pgfpathlineto{\pgfqpoint{3.912290in}{3.566553in}}%
\pgfpathclose%
\pgfusepath{fill}%
\end{pgfscope}%
\begin{pgfscope}%
\pgfpathrectangle{\pgfqpoint{1.020000in}{0.880000in}}{\pgfqpoint{6.160000in}{6.160000in}}%
\pgfusepath{clip}%
\pgfsetbuttcap%
\pgfsetroundjoin%
\definecolor{currentfill}{rgb}{0.343278,0.459354,0.884122}%
\pgfsetfillcolor{currentfill}%
\pgfsetlinewidth{0.000000pt}%
\definecolor{currentstroke}{rgb}{0.000000,0.000000,0.000000}%
\pgfsetstrokecolor{currentstroke}%
\pgfsetdash{}{0pt}%
\pgfpathmoveto{\pgfqpoint{5.217577in}{2.963771in}}%
\pgfpathlineto{\pgfqpoint{5.228988in}{2.948371in}}%
\pgfpathlineto{\pgfqpoint{5.240418in}{2.932789in}}%
\pgfpathlineto{\pgfqpoint{5.273565in}{2.930208in}}%
\pgfpathlineto{\pgfqpoint{5.306694in}{2.928239in}}%
\pgfpathlineto{\pgfqpoint{5.295212in}{2.943867in}}%
\pgfpathlineto{\pgfqpoint{5.283749in}{2.959329in}}%
\pgfpathlineto{\pgfqpoint{5.250672in}{2.961253in}}%
\pgfpathlineto{\pgfqpoint{5.217577in}{2.963771in}}%
\pgfpathclose%
\pgfusepath{fill}%
\end{pgfscope}%
\begin{pgfscope}%
\pgfpathrectangle{\pgfqpoint{1.020000in}{0.880000in}}{\pgfqpoint{6.160000in}{6.160000in}}%
\pgfusepath{clip}%
\pgfsetbuttcap%
\pgfsetroundjoin%
\definecolor{currentfill}{rgb}{0.902849,0.844796,0.811970}%
\pgfsetfillcolor{currentfill}%
\pgfsetlinewidth{0.000000pt}%
\definecolor{currentstroke}{rgb}{0.000000,0.000000,0.000000}%
\pgfsetstrokecolor{currentstroke}%
\pgfsetdash{}{0pt}%
\pgfpathmoveto{\pgfqpoint{2.694571in}{4.047812in}}%
\pgfpathlineto{\pgfqpoint{2.704215in}{3.993141in}}%
\pgfpathlineto{\pgfqpoint{2.713854in}{3.939577in}}%
\pgfpathlineto{\pgfqpoint{2.747554in}{3.961248in}}%
\pgfpathlineto{\pgfqpoint{2.781250in}{3.982572in}}%
\pgfpathlineto{\pgfqpoint{2.771537in}{4.039073in}}%
\pgfpathlineto{\pgfqpoint{2.761817in}{4.096849in}}%
\pgfpathlineto{\pgfqpoint{2.728194in}{4.072565in}}%
\pgfpathlineto{\pgfqpoint{2.694571in}{4.047812in}}%
\pgfpathclose%
\pgfusepath{fill}%
\end{pgfscope}%
\begin{pgfscope}%
\pgfpathrectangle{\pgfqpoint{1.020000in}{0.880000in}}{\pgfqpoint{6.160000in}{6.160000in}}%
\pgfusepath{clip}%
\pgfsetbuttcap%
\pgfsetroundjoin%
\definecolor{currentfill}{rgb}{0.646113,0.764436,0.996868}%
\pgfsetfillcolor{currentfill}%
\pgfsetlinewidth{0.000000pt}%
\definecolor{currentstroke}{rgb}{0.000000,0.000000,0.000000}%
\pgfsetstrokecolor{currentstroke}%
\pgfsetdash{}{0pt}%
\pgfpathmoveto{\pgfqpoint{4.067055in}{3.497284in}}%
\pgfpathlineto{\pgfqpoint{4.077394in}{3.497194in}}%
\pgfpathlineto{\pgfqpoint{4.087753in}{3.498680in}}%
\pgfpathlineto{\pgfqpoint{4.121332in}{3.479920in}}%
\pgfpathlineto{\pgfqpoint{4.154877in}{3.460824in}}%
\pgfpathlineto{\pgfqpoint{4.144456in}{3.460729in}}%
\pgfpathlineto{\pgfqpoint{4.134056in}{3.462009in}}%
\pgfpathlineto{\pgfqpoint{4.100572in}{3.479749in}}%
\pgfpathlineto{\pgfqpoint{4.067055in}{3.497284in}}%
\pgfpathclose%
\pgfusepath{fill}%
\end{pgfscope}%
\begin{pgfscope}%
\pgfpathrectangle{\pgfqpoint{1.020000in}{0.880000in}}{\pgfqpoint{6.160000in}{6.160000in}}%
\pgfusepath{clip}%
\pgfsetbuttcap%
\pgfsetroundjoin%
\definecolor{currentfill}{rgb}{0.451739,0.588181,0.960201}%
\pgfsetfillcolor{currentfill}%
\pgfsetlinewidth{0.000000pt}%
\definecolor{currentstroke}{rgb}{0.000000,0.000000,0.000000}%
\pgfsetstrokecolor{currentstroke}%
\pgfsetdash{}{0pt}%
\pgfpathmoveto{\pgfqpoint{4.686332in}{3.167548in}}%
\pgfpathlineto{\pgfqpoint{4.697281in}{3.157460in}}%
\pgfpathlineto{\pgfqpoint{4.708250in}{3.147139in}}%
\pgfpathlineto{\pgfqpoint{4.741524in}{3.130934in}}%
\pgfpathlineto{\pgfqpoint{4.774774in}{3.116036in}}%
\pgfpathlineto{\pgfqpoint{4.763757in}{3.127502in}}%
\pgfpathlineto{\pgfqpoint{4.752760in}{3.138710in}}%
\pgfpathlineto{\pgfqpoint{4.719559in}{3.152553in}}%
\pgfpathlineto{\pgfqpoint{4.686332in}{3.167548in}}%
\pgfpathclose%
\pgfusepath{fill}%
\end{pgfscope}%
\begin{pgfscope}%
\pgfpathrectangle{\pgfqpoint{1.020000in}{0.880000in}}{\pgfqpoint{6.160000in}{6.160000in}}%
\pgfusepath{clip}%
\pgfsetbuttcap%
\pgfsetroundjoin%
\definecolor{currentfill}{rgb}{0.500031,0.638508,0.981070}%
\pgfsetfillcolor{currentfill}%
\pgfsetlinewidth{0.000000pt}%
\definecolor{currentstroke}{rgb}{0.000000,0.000000,0.000000}%
\pgfsetstrokecolor{currentstroke}%
\pgfsetdash{}{0pt}%
\pgfpathmoveto{\pgfqpoint{4.531509in}{3.252827in}}%
\pgfpathlineto{\pgfqpoint{4.542319in}{3.246072in}}%
\pgfpathlineto{\pgfqpoint{4.553151in}{3.239283in}}%
\pgfpathlineto{\pgfqpoint{4.586489in}{3.219600in}}%
\pgfpathlineto{\pgfqpoint{4.619798in}{3.201070in}}%
\pgfpathlineto{\pgfqpoint{4.608918in}{3.209571in}}%
\pgfpathlineto{\pgfqpoint{4.598059in}{3.218002in}}%
\pgfpathlineto{\pgfqpoint{4.564799in}{3.234910in}}%
\pgfpathlineto{\pgfqpoint{4.531509in}{3.252827in}}%
\pgfpathclose%
\pgfusepath{fill}%
\end{pgfscope}%
\begin{pgfscope}%
\pgfpathrectangle{\pgfqpoint{1.020000in}{0.880000in}}{\pgfqpoint{6.160000in}{6.160000in}}%
\pgfusepath{clip}%
\pgfsetbuttcap%
\pgfsetroundjoin%
\definecolor{currentfill}{rgb}{0.554312,0.690097,0.995516}%
\pgfsetfillcolor{currentfill}%
\pgfsetlinewidth{0.000000pt}%
\definecolor{currentstroke}{rgb}{0.000000,0.000000,0.000000}%
\pgfsetstrokecolor{currentstroke}%
\pgfsetdash{}{0pt}%
\pgfpathmoveto{\pgfqpoint{4.376690in}{3.339738in}}%
\pgfpathlineto{\pgfqpoint{4.387352in}{3.336563in}}%
\pgfpathlineto{\pgfqpoint{4.398037in}{3.333710in}}%
\pgfpathlineto{\pgfqpoint{4.431454in}{3.312245in}}%
\pgfpathlineto{\pgfqpoint{4.464838in}{3.291552in}}%
\pgfpathlineto{\pgfqpoint{4.454101in}{3.296490in}}%
\pgfpathlineto{\pgfqpoint{4.443386in}{3.301675in}}%
\pgfpathlineto{\pgfqpoint{4.410054in}{3.320358in}}%
\pgfpathlineto{\pgfqpoint{4.376690in}{3.339738in}}%
\pgfpathclose%
\pgfusepath{fill}%
\end{pgfscope}%
\begin{pgfscope}%
\pgfpathrectangle{\pgfqpoint{1.020000in}{0.880000in}}{\pgfqpoint{6.160000in}{6.160000in}}%
\pgfusepath{clip}%
\pgfsetbuttcap%
\pgfsetroundjoin%
\definecolor{currentfill}{rgb}{0.603162,0.731527,0.999565}%
\pgfsetfillcolor{currentfill}%
\pgfsetlinewidth{0.000000pt}%
\definecolor{currentstroke}{rgb}{0.000000,0.000000,0.000000}%
\pgfsetstrokecolor{currentstroke}%
\pgfsetdash{}{0pt}%
\pgfpathmoveto{\pgfqpoint{4.221864in}{3.422265in}}%
\pgfpathlineto{\pgfqpoint{4.232368in}{3.421789in}}%
\pgfpathlineto{\pgfqpoint{4.242894in}{3.422178in}}%
\pgfpathlineto{\pgfqpoint{4.276395in}{3.401080in}}%
\pgfpathlineto{\pgfqpoint{4.309861in}{3.380216in}}%
\pgfpathlineto{\pgfqpoint{4.299277in}{3.381844in}}%
\pgfpathlineto{\pgfqpoint{4.288716in}{3.384200in}}%
\pgfpathlineto{\pgfqpoint{4.255307in}{3.403103in}}%
\pgfpathlineto{\pgfqpoint{4.221864in}{3.422265in}}%
\pgfpathclose%
\pgfusepath{fill}%
\end{pgfscope}%
\begin{pgfscope}%
\pgfpathrectangle{\pgfqpoint{1.020000in}{0.880000in}}{\pgfqpoint{6.160000in}{6.160000in}}%
\pgfusepath{clip}%
\pgfsetbuttcap%
\pgfsetroundjoin%
\definecolor{currentfill}{rgb}{0.906154,0.842091,0.806151}%
\pgfsetfillcolor{currentfill}%
\pgfsetlinewidth{0.000000pt}%
\definecolor{currentstroke}{rgb}{0.000000,0.000000,0.000000}%
\pgfsetstrokecolor{currentstroke}%
\pgfsetdash{}{0pt}%
\pgfpathmoveto{\pgfqpoint{2.916038in}{4.059852in}}%
\pgfpathlineto{\pgfqpoint{2.925861in}{4.000462in}}%
\pgfpathlineto{\pgfqpoint{2.935667in}{3.943264in}}%
\pgfpathlineto{\pgfqpoint{2.969423in}{3.957965in}}%
\pgfpathlineto{\pgfqpoint{3.003181in}{3.971284in}}%
\pgfpathlineto{\pgfqpoint{2.993329in}{4.029721in}}%
\pgfpathlineto{\pgfqpoint{2.983460in}{4.090547in}}%
\pgfpathlineto{\pgfqpoint{2.949746in}{4.076040in}}%
\pgfpathlineto{\pgfqpoint{2.916038in}{4.059852in}}%
\pgfpathclose%
\pgfusepath{fill}%
\end{pgfscope}%
\begin{pgfscope}%
\pgfpathrectangle{\pgfqpoint{1.020000in}{0.880000in}}{\pgfqpoint{6.160000in}{6.160000in}}%
\pgfusepath{clip}%
\pgfsetbuttcap%
\pgfsetroundjoin%
\definecolor{currentfill}{rgb}{0.409611,0.540759,0.935545}%
\pgfsetfillcolor{currentfill}%
\pgfsetlinewidth{0.000000pt}%
\definecolor{currentstroke}{rgb}{0.000000,0.000000,0.000000}%
\pgfsetstrokecolor{currentstroke}%
\pgfsetdash{}{0pt}%
\pgfpathmoveto{\pgfqpoint{4.841202in}{3.090033in}}%
\pgfpathlineto{\pgfqpoint{4.852286in}{3.077412in}}%
\pgfpathlineto{\pgfqpoint{4.863390in}{3.064481in}}%
\pgfpathlineto{\pgfqpoint{4.896619in}{3.052605in}}%
\pgfpathlineto{\pgfqpoint{4.929827in}{3.041981in}}%
\pgfpathlineto{\pgfqpoint{4.918675in}{3.055541in}}%
\pgfpathlineto{\pgfqpoint{4.907542in}{3.068769in}}%
\pgfpathlineto{\pgfqpoint{4.874383in}{3.078835in}}%
\pgfpathlineto{\pgfqpoint{4.841202in}{3.090033in}}%
\pgfpathclose%
\pgfusepath{fill}%
\end{pgfscope}%
\begin{pgfscope}%
\pgfpathrectangle{\pgfqpoint{1.020000in}{0.880000in}}{\pgfqpoint{6.160000in}{6.160000in}}%
\pgfusepath{clip}%
\pgfsetbuttcap%
\pgfsetroundjoin%
\definecolor{currentfill}{rgb}{0.294718,0.393542,0.834384}%
\pgfsetfillcolor{currentfill}%
\pgfsetlinewidth{0.000000pt}%
\definecolor{currentstroke}{rgb}{0.000000,0.000000,0.000000}%
\pgfsetstrokecolor{currentstroke}%
\pgfsetdash{}{0pt}%
\pgfpathmoveto{\pgfqpoint{6.037950in}{2.855338in}}%
\pgfpathlineto{\pgfqpoint{6.050126in}{2.840312in}}%
\pgfpathlineto{\pgfqpoint{6.062326in}{2.825334in}}%
\pgfpathlineto{\pgfqpoint{6.095271in}{2.827582in}}%
\pgfpathlineto{\pgfqpoint{6.128197in}{2.829851in}}%
\pgfpathlineto{\pgfqpoint{6.115943in}{2.844744in}}%
\pgfpathlineto{\pgfqpoint{6.103713in}{2.859684in}}%
\pgfpathlineto{\pgfqpoint{6.070841in}{2.857500in}}%
\pgfpathlineto{\pgfqpoint{6.037950in}{2.855338in}}%
\pgfpathclose%
\pgfusepath{fill}%
\end{pgfscope}%
\begin{pgfscope}%
\pgfpathrectangle{\pgfqpoint{1.020000in}{0.880000in}}{\pgfqpoint{6.160000in}{6.160000in}}%
\pgfusepath{clip}%
\pgfsetbuttcap%
\pgfsetroundjoin%
\definecolor{currentfill}{rgb}{0.887752,0.854040,0.834671}%
\pgfsetfillcolor{currentfill}%
\pgfsetlinewidth{0.000000pt}%
\definecolor{currentstroke}{rgb}{0.000000,0.000000,0.000000}%
\pgfsetstrokecolor{currentstroke}%
\pgfsetdash{}{0pt}%
\pgfpathmoveto{\pgfqpoint{2.627310in}{3.997906in}}%
\pgfpathlineto{\pgfqpoint{2.636872in}{3.946506in}}%
\pgfpathlineto{\pgfqpoint{2.646430in}{3.896050in}}%
\pgfpathlineto{\pgfqpoint{2.680147in}{3.917775in}}%
\pgfpathlineto{\pgfqpoint{2.713854in}{3.939577in}}%
\pgfpathlineto{\pgfqpoint{2.704215in}{3.993141in}}%
\pgfpathlineto{\pgfqpoint{2.694571in}{4.047812in}}%
\pgfpathlineto{\pgfqpoint{2.660944in}{4.022846in}}%
\pgfpathlineto{\pgfqpoint{2.627310in}{3.997906in}}%
\pgfpathclose%
\pgfusepath{fill}%
\end{pgfscope}%
\begin{pgfscope}%
\pgfpathrectangle{\pgfqpoint{1.020000in}{0.880000in}}{\pgfqpoint{6.160000in}{6.160000in}}%
\pgfusepath{clip}%
\pgfsetbuttcap%
\pgfsetroundjoin%
\definecolor{currentfill}{rgb}{0.304174,0.406945,0.845263}%
\pgfsetfillcolor{currentfill}%
\pgfsetlinewidth{0.000000pt}%
\definecolor{currentstroke}{rgb}{0.000000,0.000000,0.000000}%
\pgfsetstrokecolor{currentstroke}%
\pgfsetdash{}{0pt}%
\pgfpathmoveto{\pgfqpoint{5.816227in}{2.873798in}}%
\pgfpathlineto{\pgfqpoint{5.828193in}{2.858415in}}%
\pgfpathlineto{\pgfqpoint{5.840182in}{2.843076in}}%
\pgfpathlineto{\pgfqpoint{5.873193in}{2.845004in}}%
\pgfpathlineto{\pgfqpoint{5.906184in}{2.846989in}}%
\pgfpathlineto{\pgfqpoint{5.894141in}{2.862238in}}%
\pgfpathlineto{\pgfqpoint{5.882121in}{2.877532in}}%
\pgfpathlineto{\pgfqpoint{5.849184in}{2.875636in}}%
\pgfpathlineto{\pgfqpoint{5.816227in}{2.873798in}}%
\pgfpathclose%
\pgfusepath{fill}%
\end{pgfscope}%
\begin{pgfscope}%
\pgfpathrectangle{\pgfqpoint{1.020000in}{0.880000in}}{\pgfqpoint{6.160000in}{6.160000in}}%
\pgfusepath{clip}%
\pgfsetbuttcap%
\pgfsetroundjoin%
\definecolor{currentfill}{rgb}{0.875557,0.860242,0.851430}%
\pgfsetfillcolor{currentfill}%
\pgfsetlinewidth{0.000000pt}%
\definecolor{currentstroke}{rgb}{0.000000,0.000000,0.000000}%
\pgfsetstrokecolor{currentstroke}%
\pgfsetdash{}{0pt}%
\pgfpathmoveto{\pgfqpoint{3.070704in}{3.993106in}}%
\pgfpathlineto{\pgfqpoint{3.080573in}{3.936885in}}%
\pgfpathlineto{\pgfqpoint{3.090421in}{3.883709in}}%
\pgfpathlineto{\pgfqpoint{3.124218in}{3.892067in}}%
\pgfpathlineto{\pgfqpoint{3.158013in}{3.898782in}}%
\pgfpathlineto{\pgfqpoint{3.148135in}{3.951563in}}%
\pgfpathlineto{\pgfqpoint{3.138237in}{4.007558in}}%
\pgfpathlineto{\pgfqpoint{3.104470in}{4.001315in}}%
\pgfpathlineto{\pgfqpoint{3.070704in}{3.993106in}}%
\pgfpathclose%
\pgfusepath{fill}%
\end{pgfscope}%
\begin{pgfscope}%
\pgfpathrectangle{\pgfqpoint{1.020000in}{0.880000in}}{\pgfqpoint{6.160000in}{6.160000in}}%
\pgfusepath{clip}%
\pgfsetbuttcap%
\pgfsetroundjoin%
\definecolor{currentfill}{rgb}{0.839351,0.861167,0.894494}%
\pgfsetfillcolor{currentfill}%
\pgfsetlinewidth{0.000000pt}%
\definecolor{currentstroke}{rgb}{0.000000,0.000000,0.000000}%
\pgfsetstrokecolor{currentstroke}%
\pgfsetdash{}{0pt}%
\pgfpathmoveto{\pgfqpoint{3.225596in}{3.906914in}}%
\pgfpathlineto{\pgfqpoint{3.235477in}{3.858674in}}%
\pgfpathlineto{\pgfqpoint{3.245337in}{3.814099in}}%
\pgfpathlineto{\pgfqpoint{3.279146in}{3.816997in}}%
\pgfpathlineto{\pgfqpoint{3.312947in}{3.818236in}}%
\pgfpathlineto{\pgfqpoint{3.303063in}{3.861034in}}%
\pgfpathlineto{\pgfqpoint{3.293162in}{3.907583in}}%
\pgfpathlineto{\pgfqpoint{3.259382in}{3.908201in}}%
\pgfpathlineto{\pgfqpoint{3.225596in}{3.906914in}}%
\pgfpathclose%
\pgfusepath{fill}%
\end{pgfscope}%
\begin{pgfscope}%
\pgfpathrectangle{\pgfqpoint{1.020000in}{0.880000in}}{\pgfqpoint{6.160000in}{6.160000in}}%
\pgfusepath{clip}%
\pgfsetbuttcap%
\pgfsetroundjoin%
\definecolor{currentfill}{rgb}{0.804965,0.851666,0.926165}%
\pgfsetfillcolor{currentfill}%
\pgfsetlinewidth{0.000000pt}%
\definecolor{currentstroke}{rgb}{0.000000,0.000000,0.000000}%
\pgfsetstrokecolor{currentstroke}%
\pgfsetdash{}{0pt}%
\pgfpathmoveto{\pgfqpoint{3.380523in}{3.815635in}}%
\pgfpathlineto{\pgfqpoint{3.390412in}{3.778916in}}%
\pgfpathlineto{\pgfqpoint{3.400287in}{3.746129in}}%
\pgfpathlineto{\pgfqpoint{3.434084in}{3.744734in}}%
\pgfpathlineto{\pgfqpoint{3.467868in}{3.741774in}}%
\pgfpathlineto{\pgfqpoint{3.457966in}{3.772079in}}%
\pgfpathlineto{\pgfqpoint{3.448053in}{3.806290in}}%
\pgfpathlineto{\pgfqpoint{3.414295in}{3.811793in}}%
\pgfpathlineto{\pgfqpoint{3.380523in}{3.815635in}}%
\pgfpathclose%
\pgfusepath{fill}%
\end{pgfscope}%
\begin{pgfscope}%
\pgfpathrectangle{\pgfqpoint{1.020000in}{0.880000in}}{\pgfqpoint{6.160000in}{6.160000in}}%
\pgfusepath{clip}%
\pgfsetbuttcap%
\pgfsetroundjoin%
\definecolor{currentfill}{rgb}{0.313946,0.420052,0.854993}%
\pgfsetfillcolor{currentfill}%
\pgfsetlinewidth{0.000000pt}%
\definecolor{currentstroke}{rgb}{0.000000,0.000000,0.000000}%
\pgfsetstrokecolor{currentstroke}%
\pgfsetdash{}{0pt}%
\pgfpathmoveto{\pgfqpoint{5.594538in}{2.896053in}}%
\pgfpathlineto{\pgfqpoint{5.606296in}{2.880365in}}%
\pgfpathlineto{\pgfqpoint{5.618076in}{2.864694in}}%
\pgfpathlineto{\pgfqpoint{5.651150in}{2.865895in}}%
\pgfpathlineto{\pgfqpoint{5.684204in}{2.867250in}}%
\pgfpathlineto{\pgfqpoint{5.672370in}{2.882841in}}%
\pgfpathlineto{\pgfqpoint{5.660558in}{2.898457in}}%
\pgfpathlineto{\pgfqpoint{5.627558in}{2.897178in}}%
\pgfpathlineto{\pgfqpoint{5.594538in}{2.896053in}}%
\pgfpathclose%
\pgfusepath{fill}%
\end{pgfscope}%
\begin{pgfscope}%
\pgfpathrectangle{\pgfqpoint{1.020000in}{0.880000in}}{\pgfqpoint{6.160000in}{6.160000in}}%
\pgfusepath{clip}%
\pgfsetbuttcap%
\pgfsetroundjoin%
\definecolor{currentfill}{rgb}{0.378598,0.503856,0.913692}%
\pgfsetfillcolor{currentfill}%
\pgfsetlinewidth{0.000000pt}%
\definecolor{currentstroke}{rgb}{0.000000,0.000000,0.000000}%
\pgfsetstrokecolor{currentstroke}%
\pgfsetdash{}{0pt}%
\pgfpathmoveto{\pgfqpoint{4.996184in}{3.024217in}}%
\pgfpathlineto{\pgfqpoint{5.007404in}{3.009912in}}%
\pgfpathlineto{\pgfqpoint{5.018643in}{2.995318in}}%
\pgfpathlineto{\pgfqpoint{5.051843in}{2.987726in}}%
\pgfpathlineto{\pgfqpoint{5.085025in}{2.981172in}}%
\pgfpathlineto{\pgfqpoint{5.073736in}{2.996049in}}%
\pgfpathlineto{\pgfqpoint{5.062466in}{3.010633in}}%
\pgfpathlineto{\pgfqpoint{5.029334in}{3.016938in}}%
\pgfpathlineto{\pgfqpoint{4.996184in}{3.024217in}}%
\pgfpathclose%
\pgfusepath{fill}%
\end{pgfscope}%
\begin{pgfscope}%
\pgfpathrectangle{\pgfqpoint{1.020000in}{0.880000in}}{\pgfqpoint{6.160000in}{6.160000in}}%
\pgfusepath{clip}%
\pgfsetbuttcap%
\pgfsetroundjoin%
\definecolor{currentfill}{rgb}{0.768034,0.837035,0.952488}%
\pgfsetfillcolor{currentfill}%
\pgfsetlinewidth{0.000000pt}%
\definecolor{currentstroke}{rgb}{0.000000,0.000000,0.000000}%
\pgfsetstrokecolor{currentstroke}%
\pgfsetdash{}{0pt}%
\pgfpathmoveto{\pgfqpoint{3.535392in}{3.731247in}}%
\pgfpathlineto{\pgfqpoint{3.545315in}{3.707371in}}%
\pgfpathlineto{\pgfqpoint{3.555233in}{3.687296in}}%
\pgfpathlineto{\pgfqpoint{3.589005in}{3.682244in}}%
\pgfpathlineto{\pgfqpoint{3.622760in}{3.675734in}}%
\pgfpathlineto{\pgfqpoint{3.612804in}{3.693479in}}%
\pgfpathlineto{\pgfqpoint{3.602845in}{3.714897in}}%
\pgfpathlineto{\pgfqpoint{3.569128in}{3.723768in}}%
\pgfpathlineto{\pgfqpoint{3.535392in}{3.731247in}}%
\pgfpathclose%
\pgfusepath{fill}%
\end{pgfscope}%
\begin{pgfscope}%
\pgfpathrectangle{\pgfqpoint{1.020000in}{0.880000in}}{\pgfqpoint{6.160000in}{6.160000in}}%
\pgfusepath{clip}%
\pgfsetbuttcap%
\pgfsetroundjoin%
\definecolor{currentfill}{rgb}{0.328604,0.439712,0.869587}%
\pgfsetfillcolor{currentfill}%
\pgfsetlinewidth{0.000000pt}%
\definecolor{currentstroke}{rgb}{0.000000,0.000000,0.000000}%
\pgfsetstrokecolor{currentstroke}%
\pgfsetdash{}{0pt}%
\pgfpathmoveto{\pgfqpoint{5.372900in}{2.925868in}}%
\pgfpathlineto{\pgfqpoint{5.384456in}{2.910116in}}%
\pgfpathlineto{\pgfqpoint{5.396032in}{2.894295in}}%
\pgfpathlineto{\pgfqpoint{5.429161in}{2.893796in}}%
\pgfpathlineto{\pgfqpoint{5.462273in}{2.893674in}}%
\pgfpathlineto{\pgfqpoint{5.450644in}{2.909460in}}%
\pgfpathlineto{\pgfqpoint{5.439036in}{2.925198in}}%
\pgfpathlineto{\pgfqpoint{5.405977in}{2.925347in}}%
\pgfpathlineto{\pgfqpoint{5.372900in}{2.925868in}}%
\pgfpathclose%
\pgfusepath{fill}%
\end{pgfscope}%
\begin{pgfscope}%
\pgfpathrectangle{\pgfqpoint{1.020000in}{0.880000in}}{\pgfqpoint{6.160000in}{6.160000in}}%
\pgfusepath{clip}%
\pgfsetbuttcap%
\pgfsetroundjoin%
\definecolor{currentfill}{rgb}{0.867428,0.864377,0.862602}%
\pgfsetfillcolor{currentfill}%
\pgfsetlinewidth{0.000000pt}%
\definecolor{currentstroke}{rgb}{0.000000,0.000000,0.000000}%
\pgfsetstrokecolor{currentstroke}%
\pgfsetdash{}{0pt}%
\pgfpathmoveto{\pgfqpoint{2.560007in}{3.949000in}}%
\pgfpathlineto{\pgfqpoint{2.569482in}{3.900902in}}%
\pgfpathlineto{\pgfqpoint{2.578956in}{3.853595in}}%
\pgfpathlineto{\pgfqpoint{2.612701in}{3.874596in}}%
\pgfpathlineto{\pgfqpoint{2.646430in}{3.896050in}}%
\pgfpathlineto{\pgfqpoint{2.636872in}{3.946506in}}%
\pgfpathlineto{\pgfqpoint{2.627310in}{3.997906in}}%
\pgfpathlineto{\pgfqpoint{2.593666in}{3.973220in}}%
\pgfpathlineto{\pgfqpoint{2.560007in}{3.949000in}}%
\pgfpathclose%
\pgfusepath{fill}%
\end{pgfscope}%
\begin{pgfscope}%
\pgfpathrectangle{\pgfqpoint{1.020000in}{0.880000in}}{\pgfqpoint{6.160000in}{6.160000in}}%
\pgfusepath{clip}%
\pgfsetbuttcap%
\pgfsetroundjoin%
\definecolor{currentfill}{rgb}{0.895882,0.849906,0.823499}%
\pgfsetfillcolor{currentfill}%
\pgfsetlinewidth{0.000000pt}%
\definecolor{currentstroke}{rgb}{0.000000,0.000000,0.000000}%
\pgfsetstrokecolor{currentstroke}%
\pgfsetdash{}{0pt}%
\pgfpathmoveto{\pgfqpoint{2.848639in}{4.023272in}}%
\pgfpathlineto{\pgfqpoint{2.858406in}{3.965873in}}%
\pgfpathlineto{\pgfqpoint{2.868158in}{3.910455in}}%
\pgfpathlineto{\pgfqpoint{2.901913in}{3.927364in}}%
\pgfpathlineto{\pgfqpoint{2.935667in}{3.943264in}}%
\pgfpathlineto{\pgfqpoint{2.925861in}{4.000462in}}%
\pgfpathlineto{\pgfqpoint{2.916038in}{4.059852in}}%
\pgfpathlineto{\pgfqpoint{2.882337in}{4.042190in}}%
\pgfpathlineto{\pgfqpoint{2.848639in}{4.023272in}}%
\pgfpathclose%
\pgfusepath{fill}%
\end{pgfscope}%
\begin{pgfscope}%
\pgfpathrectangle{\pgfqpoint{1.020000in}{0.880000in}}{\pgfqpoint{6.160000in}{6.160000in}}%
\pgfusepath{clip}%
\pgfsetbuttcap%
\pgfsetroundjoin%
\definecolor{currentfill}{rgb}{0.733898,0.820018,0.970724}%
\pgfsetfillcolor{currentfill}%
\pgfsetlinewidth{0.000000pt}%
\definecolor{currentstroke}{rgb}{0.000000,0.000000,0.000000}%
\pgfsetstrokecolor{currentstroke}%
\pgfsetdash{}{0pt}%
\pgfpathmoveto{\pgfqpoint{3.690210in}{3.658581in}}%
\pgfpathlineto{\pgfqpoint{3.700211in}{3.646391in}}%
\pgfpathlineto{\pgfqpoint{3.710216in}{3.637522in}}%
\pgfpathlineto{\pgfqpoint{3.743957in}{3.628663in}}%
\pgfpathlineto{\pgfqpoint{3.777675in}{3.618475in}}%
\pgfpathlineto{\pgfqpoint{3.767618in}{3.625911in}}%
\pgfpathlineto{\pgfqpoint{3.757568in}{3.636466in}}%
\pgfpathlineto{\pgfqpoint{3.723901in}{3.648096in}}%
\pgfpathlineto{\pgfqpoint{3.690210in}{3.658581in}}%
\pgfpathclose%
\pgfusepath{fill}%
\end{pgfscope}%
\begin{pgfscope}%
\pgfpathrectangle{\pgfqpoint{1.020000in}{0.880000in}}{\pgfqpoint{6.160000in}{6.160000in}}%
\pgfusepath{clip}%
\pgfsetbuttcap%
\pgfsetroundjoin%
\definecolor{currentfill}{rgb}{0.353369,0.472069,0.892570}%
\pgfsetfillcolor{currentfill}%
\pgfsetlinewidth{0.000000pt}%
\definecolor{currentstroke}{rgb}{0.000000,0.000000,0.000000}%
\pgfsetstrokecolor{currentstroke}%
\pgfsetdash{}{0pt}%
\pgfpathmoveto{\pgfqpoint{5.151336in}{2.970861in}}%
\pgfpathlineto{\pgfqpoint{5.162695in}{2.955564in}}%
\pgfpathlineto{\pgfqpoint{5.174074in}{2.940070in}}%
\pgfpathlineto{\pgfqpoint{5.207255in}{2.936051in}}%
\pgfpathlineto{\pgfqpoint{5.240418in}{2.932789in}}%
\pgfpathlineto{\pgfqpoint{5.228988in}{2.948371in}}%
\pgfpathlineto{\pgfqpoint{5.217577in}{2.963771in}}%
\pgfpathlineto{\pgfqpoint{5.184465in}{2.966951in}}%
\pgfpathlineto{\pgfqpoint{5.151336in}{2.970861in}}%
\pgfpathclose%
\pgfusepath{fill}%
\end{pgfscope}%
\begin{pgfscope}%
\pgfpathrectangle{\pgfqpoint{1.020000in}{0.880000in}}{\pgfqpoint{6.160000in}{6.160000in}}%
\pgfusepath{clip}%
\pgfsetbuttcap%
\pgfsetroundjoin%
\definecolor{currentfill}{rgb}{0.851372,0.863125,0.881064}%
\pgfsetfillcolor{currentfill}%
\pgfsetlinewidth{0.000000pt}%
\definecolor{currentstroke}{rgb}{0.000000,0.000000,0.000000}%
\pgfsetstrokecolor{currentstroke}%
\pgfsetdash{}{0pt}%
\pgfpathmoveto{\pgfqpoint{2.492636in}{3.902696in}}%
\pgfpathlineto{\pgfqpoint{2.502021in}{3.857812in}}%
\pgfpathlineto{\pgfqpoint{2.511408in}{3.813578in}}%
\pgfpathlineto{\pgfqpoint{2.545193in}{3.833207in}}%
\pgfpathlineto{\pgfqpoint{2.578956in}{3.853595in}}%
\pgfpathlineto{\pgfqpoint{2.569482in}{3.900902in}}%
\pgfpathlineto{\pgfqpoint{2.560007in}{3.949000in}}%
\pgfpathlineto{\pgfqpoint{2.526332in}{3.925435in}}%
\pgfpathlineto{\pgfqpoint{2.492636in}{3.902696in}}%
\pgfpathclose%
\pgfusepath{fill}%
\end{pgfscope}%
\begin{pgfscope}%
\pgfpathrectangle{\pgfqpoint{1.020000in}{0.880000in}}{\pgfqpoint{6.160000in}{6.160000in}}%
\pgfusepath{clip}%
\pgfsetbuttcap%
\pgfsetroundjoin%
\definecolor{currentfill}{rgb}{0.871493,0.862309,0.857016}%
\pgfsetfillcolor{currentfill}%
\pgfsetlinewidth{0.000000pt}%
\definecolor{currentstroke}{rgb}{0.000000,0.000000,0.000000}%
\pgfsetstrokecolor{currentstroke}%
\pgfsetdash{}{0pt}%
\pgfpathmoveto{\pgfqpoint{3.003181in}{3.971284in}}%
\pgfpathlineto{\pgfqpoint{3.013013in}{3.915485in}}%
\pgfpathlineto{\pgfqpoint{3.022825in}{3.862532in}}%
\pgfpathlineto{\pgfqpoint{3.056624in}{3.873821in}}%
\pgfpathlineto{\pgfqpoint{3.090421in}{3.883709in}}%
\pgfpathlineto{\pgfqpoint{3.080573in}{3.936885in}}%
\pgfpathlineto{\pgfqpoint{3.070704in}{3.993106in}}%
\pgfpathlineto{\pgfqpoint{3.036941in}{3.983050in}}%
\pgfpathlineto{\pgfqpoint{3.003181in}{3.971284in}}%
\pgfpathclose%
\pgfusepath{fill}%
\end{pgfscope}%
\begin{pgfscope}%
\pgfpathrectangle{\pgfqpoint{1.020000in}{0.880000in}}{\pgfqpoint{6.160000in}{6.160000in}}%
\pgfusepath{clip}%
\pgfsetbuttcap%
\pgfsetroundjoin%
\definecolor{currentfill}{rgb}{0.879622,0.858175,0.845844}%
\pgfsetfillcolor{currentfill}%
\pgfsetlinewidth{0.000000pt}%
\definecolor{currentstroke}{rgb}{0.000000,0.000000,0.000000}%
\pgfsetstrokecolor{currentstroke}%
\pgfsetdash{}{0pt}%
\pgfpathmoveto{\pgfqpoint{2.781250in}{3.982572in}}%
\pgfpathlineto{\pgfqpoint{2.790952in}{3.927605in}}%
\pgfpathlineto{\pgfqpoint{2.800642in}{3.874398in}}%
\pgfpathlineto{\pgfqpoint{2.834402in}{3.892734in}}%
\pgfpathlineto{\pgfqpoint{2.868158in}{3.910455in}}%
\pgfpathlineto{\pgfqpoint{2.858406in}{3.965873in}}%
\pgfpathlineto{\pgfqpoint{2.848639in}{4.023272in}}%
\pgfpathlineto{\pgfqpoint{2.814944in}{4.003323in}}%
\pgfpathlineto{\pgfqpoint{2.781250in}{3.982572in}}%
\pgfpathclose%
\pgfusepath{fill}%
\end{pgfscope}%
\begin{pgfscope}%
\pgfpathrectangle{\pgfqpoint{1.020000in}{0.880000in}}{\pgfqpoint{6.160000in}{6.160000in}}%
\pgfusepath{clip}%
\pgfsetbuttcap%
\pgfsetroundjoin%
\definecolor{currentfill}{rgb}{0.703587,0.802586,0.982847}%
\pgfsetfillcolor{currentfill}%
\pgfsetlinewidth{0.000000pt}%
\definecolor{currentstroke}{rgb}{0.000000,0.000000,0.000000}%
\pgfsetstrokecolor{currentstroke}%
\pgfsetdash{}{0pt}%
\pgfpathmoveto{\pgfqpoint{3.845037in}{3.594514in}}%
\pgfpathlineto{\pgfqpoint{3.855160in}{3.590870in}}%
\pgfpathlineto{\pgfqpoint{3.865296in}{3.589826in}}%
\pgfpathlineto{\pgfqpoint{3.898996in}{3.576654in}}%
\pgfpathlineto{\pgfqpoint{3.932668in}{3.562388in}}%
\pgfpathlineto{\pgfqpoint{3.922472in}{3.563285in}}%
\pgfpathlineto{\pgfqpoint{3.912290in}{3.566553in}}%
\pgfpathlineto{\pgfqpoint{3.878678in}{3.580971in}}%
\pgfpathlineto{\pgfqpoint{3.845037in}{3.594514in}}%
\pgfpathclose%
\pgfusepath{fill}%
\end{pgfscope}%
\begin{pgfscope}%
\pgfpathrectangle{\pgfqpoint{1.020000in}{0.880000in}}{\pgfqpoint{6.160000in}{6.160000in}}%
\pgfusepath{clip}%
\pgfsetbuttcap%
\pgfsetroundjoin%
\definecolor{currentfill}{rgb}{0.294718,0.393542,0.834384}%
\pgfsetfillcolor{currentfill}%
\pgfsetlinewidth{0.000000pt}%
\definecolor{currentstroke}{rgb}{0.000000,0.000000,0.000000}%
\pgfsetstrokecolor{currentstroke}%
\pgfsetdash{}{0pt}%
\pgfpathmoveto{\pgfqpoint{5.972107in}{2.851094in}}%
\pgfpathlineto{\pgfqpoint{5.984228in}{2.835981in}}%
\pgfpathlineto{\pgfqpoint{5.996373in}{2.820915in}}%
\pgfpathlineto{\pgfqpoint{6.029360in}{2.823110in}}%
\pgfpathlineto{\pgfqpoint{6.062326in}{2.825334in}}%
\pgfpathlineto{\pgfqpoint{6.050126in}{2.840312in}}%
\pgfpathlineto{\pgfqpoint{6.037950in}{2.855338in}}%
\pgfpathlineto{\pgfqpoint{6.005039in}{2.853201in}}%
\pgfpathlineto{\pgfqpoint{5.972107in}{2.851094in}}%
\pgfpathclose%
\pgfusepath{fill}%
\end{pgfscope}%
\begin{pgfscope}%
\pgfpathrectangle{\pgfqpoint{1.020000in}{0.880000in}}{\pgfqpoint{6.160000in}{6.160000in}}%
\pgfusepath{clip}%
\pgfsetbuttcap%
\pgfsetroundjoin%
\definecolor{currentfill}{rgb}{0.304174,0.406945,0.845263}%
\pgfsetfillcolor{currentfill}%
\pgfsetlinewidth{0.000000pt}%
\definecolor{currentstroke}{rgb}{0.000000,0.000000,0.000000}%
\pgfsetstrokecolor{currentstroke}%
\pgfsetdash{}{0pt}%
\pgfpathmoveto{\pgfqpoint{5.750255in}{2.870334in}}%
\pgfpathlineto{\pgfqpoint{5.762166in}{2.854863in}}%
\pgfpathlineto{\pgfqpoint{5.774100in}{2.839433in}}%
\pgfpathlineto{\pgfqpoint{5.807151in}{2.841215in}}%
\pgfpathlineto{\pgfqpoint{5.840182in}{2.843076in}}%
\pgfpathlineto{\pgfqpoint{5.828193in}{2.858415in}}%
\pgfpathlineto{\pgfqpoint{5.816227in}{2.873798in}}%
\pgfpathlineto{\pgfqpoint{5.783251in}{2.872026in}}%
\pgfpathlineto{\pgfqpoint{5.750255in}{2.870334in}}%
\pgfpathclose%
\pgfusepath{fill}%
\end{pgfscope}%
\begin{pgfscope}%
\pgfpathrectangle{\pgfqpoint{1.020000in}{0.880000in}}{\pgfqpoint{6.160000in}{6.160000in}}%
\pgfusepath{clip}%
\pgfsetbuttcap%
\pgfsetroundjoin%
\definecolor{currentfill}{rgb}{0.839351,0.861167,0.894494}%
\pgfsetfillcolor{currentfill}%
\pgfsetlinewidth{0.000000pt}%
\definecolor{currentstroke}{rgb}{0.000000,0.000000,0.000000}%
\pgfsetstrokecolor{currentstroke}%
\pgfsetdash{}{0pt}%
\pgfpathmoveto{\pgfqpoint{3.158013in}{3.898782in}}%
\pgfpathlineto{\pgfqpoint{3.167869in}{3.849394in}}%
\pgfpathlineto{\pgfqpoint{3.177703in}{3.803537in}}%
\pgfpathlineto{\pgfqpoint{3.211523in}{3.809590in}}%
\pgfpathlineto{\pgfqpoint{3.245337in}{3.814099in}}%
\pgfpathlineto{\pgfqpoint{3.235477in}{3.858674in}}%
\pgfpathlineto{\pgfqpoint{3.225596in}{3.906914in}}%
\pgfpathlineto{\pgfqpoint{3.191806in}{3.903756in}}%
\pgfpathlineto{\pgfqpoint{3.158013in}{3.898782in}}%
\pgfpathclose%
\pgfusepath{fill}%
\end{pgfscope}%
\begin{pgfscope}%
\pgfpathrectangle{\pgfqpoint{1.020000in}{0.880000in}}{\pgfqpoint{6.160000in}{6.160000in}}%
\pgfusepath{clip}%
\pgfsetbuttcap%
\pgfsetroundjoin%
\definecolor{currentfill}{rgb}{0.318832,0.426605,0.859857}%
\pgfsetfillcolor{currentfill}%
\pgfsetlinewidth{0.000000pt}%
\definecolor{currentstroke}{rgb}{0.000000,0.000000,0.000000}%
\pgfsetstrokecolor{currentstroke}%
\pgfsetdash{}{0pt}%
\pgfpathmoveto{\pgfqpoint{5.528443in}{2.894371in}}%
\pgfpathlineto{\pgfqpoint{5.540147in}{2.878618in}}%
\pgfpathlineto{\pgfqpoint{5.551873in}{2.862868in}}%
\pgfpathlineto{\pgfqpoint{5.584984in}{2.863675in}}%
\pgfpathlineto{\pgfqpoint{5.618076in}{2.864694in}}%
\pgfpathlineto{\pgfqpoint{5.606296in}{2.880365in}}%
\pgfpathlineto{\pgfqpoint{5.594538in}{2.896053in}}%
\pgfpathlineto{\pgfqpoint{5.561500in}{2.895106in}}%
\pgfpathlineto{\pgfqpoint{5.528443in}{2.894371in}}%
\pgfpathclose%
\pgfusepath{fill}%
\end{pgfscope}%
\begin{pgfscope}%
\pgfpathrectangle{\pgfqpoint{1.020000in}{0.880000in}}{\pgfqpoint{6.160000in}{6.160000in}}%
\pgfusepath{clip}%
\pgfsetbuttcap%
\pgfsetroundjoin%
\definecolor{currentfill}{rgb}{0.835345,0.860514,0.898970}%
\pgfsetfillcolor{currentfill}%
\pgfsetlinewidth{0.000000pt}%
\definecolor{currentstroke}{rgb}{0.000000,0.000000,0.000000}%
\pgfsetstrokecolor{currentstroke}%
\pgfsetdash{}{0pt}%
\pgfpathmoveto{\pgfqpoint{2.425170in}{3.860270in}}%
\pgfpathlineto{\pgfqpoint{2.434465in}{3.818413in}}%
\pgfpathlineto{\pgfqpoint{2.443763in}{3.777080in}}%
\pgfpathlineto{\pgfqpoint{2.477599in}{3.794834in}}%
\pgfpathlineto{\pgfqpoint{2.511408in}{3.813578in}}%
\pgfpathlineto{\pgfqpoint{2.502021in}{3.857812in}}%
\pgfpathlineto{\pgfqpoint{2.492636in}{3.902696in}}%
\pgfpathlineto{\pgfqpoint{2.458917in}{3.880932in}}%
\pgfpathlineto{\pgfqpoint{2.425170in}{3.860270in}}%
\pgfpathclose%
\pgfusepath{fill}%
\end{pgfscope}%
\begin{pgfscope}%
\pgfpathrectangle{\pgfqpoint{1.020000in}{0.880000in}}{\pgfqpoint{6.160000in}{6.160000in}}%
\pgfusepath{clip}%
\pgfsetbuttcap%
\pgfsetroundjoin%
\definecolor{currentfill}{rgb}{0.809329,0.852974,0.922323}%
\pgfsetfillcolor{currentfill}%
\pgfsetlinewidth{0.000000pt}%
\definecolor{currentstroke}{rgb}{0.000000,0.000000,0.000000}%
\pgfsetstrokecolor{currentstroke}%
\pgfsetdash{}{0pt}%
\pgfpathmoveto{\pgfqpoint{3.312947in}{3.818236in}}%
\pgfpathlineto{\pgfqpoint{3.322812in}{3.779285in}}%
\pgfpathlineto{\pgfqpoint{3.332660in}{3.744230in}}%
\pgfpathlineto{\pgfqpoint{3.366478in}{3.745955in}}%
\pgfpathlineto{\pgfqpoint{3.400287in}{3.746129in}}%
\pgfpathlineto{\pgfqpoint{3.390412in}{3.778916in}}%
\pgfpathlineto{\pgfqpoint{3.380523in}{3.815635in}}%
\pgfpathlineto{\pgfqpoint{3.346740in}{3.817785in}}%
\pgfpathlineto{\pgfqpoint{3.312947in}{3.818236in}}%
\pgfpathclose%
\pgfusepath{fill}%
\end{pgfscope}%
\begin{pgfscope}%
\pgfpathrectangle{\pgfqpoint{1.020000in}{0.880000in}}{\pgfqpoint{6.160000in}{6.160000in}}%
\pgfusepath{clip}%
\pgfsetbuttcap%
\pgfsetroundjoin%
\definecolor{currentfill}{rgb}{0.672538,0.782861,0.991982}%
\pgfsetfillcolor{currentfill}%
\pgfsetlinewidth{0.000000pt}%
\definecolor{currentstroke}{rgb}{0.000000,0.000000,0.000000}%
\pgfsetstrokecolor{currentstroke}%
\pgfsetdash{}{0pt}%
\pgfpathmoveto{\pgfqpoint{3.999924in}{3.531145in}}%
\pgfpathlineto{\pgfqpoint{4.010201in}{3.531938in}}%
\pgfpathlineto{\pgfqpoint{4.020497in}{3.534512in}}%
\pgfpathlineto{\pgfqpoint{4.054141in}{3.516934in}}%
\pgfpathlineto{\pgfqpoint{4.087753in}{3.498680in}}%
\pgfpathlineto{\pgfqpoint{4.077394in}{3.497194in}}%
\pgfpathlineto{\pgfqpoint{4.067055in}{3.497284in}}%
\pgfpathlineto{\pgfqpoint{4.033506in}{3.514466in}}%
\pgfpathlineto{\pgfqpoint{3.999924in}{3.531145in}}%
\pgfpathclose%
\pgfusepath{fill}%
\end{pgfscope}%
\begin{pgfscope}%
\pgfpathrectangle{\pgfqpoint{1.020000in}{0.880000in}}{\pgfqpoint{6.160000in}{6.160000in}}%
\pgfusepath{clip}%
\pgfsetbuttcap%
\pgfsetroundjoin%
\definecolor{currentfill}{rgb}{0.478462,0.616564,0.972721}%
\pgfsetfillcolor{currentfill}%
\pgfsetlinewidth{0.000000pt}%
\definecolor{currentstroke}{rgb}{0.000000,0.000000,0.000000}%
\pgfsetstrokecolor{currentstroke}%
\pgfsetdash{}{0pt}%
\pgfpathmoveto{\pgfqpoint{4.619798in}{3.201070in}}%
\pgfpathlineto{\pgfqpoint{4.630699in}{3.192418in}}%
\pgfpathlineto{\pgfqpoint{4.641621in}{3.183534in}}%
\pgfpathlineto{\pgfqpoint{4.674949in}{3.164672in}}%
\pgfpathlineto{\pgfqpoint{4.708250in}{3.147139in}}%
\pgfpathlineto{\pgfqpoint{4.697281in}{3.157460in}}%
\pgfpathlineto{\pgfqpoint{4.686332in}{3.167548in}}%
\pgfpathlineto{\pgfqpoint{4.653079in}{3.183717in}}%
\pgfpathlineto{\pgfqpoint{4.619798in}{3.201070in}}%
\pgfpathclose%
\pgfusepath{fill}%
\end{pgfscope}%
\begin{pgfscope}%
\pgfpathrectangle{\pgfqpoint{1.020000in}{0.880000in}}{\pgfqpoint{6.160000in}{6.160000in}}%
\pgfusepath{clip}%
\pgfsetbuttcap%
\pgfsetroundjoin%
\definecolor{currentfill}{rgb}{0.430507,0.564883,0.948889}%
\pgfsetfillcolor{currentfill}%
\pgfsetlinewidth{0.000000pt}%
\definecolor{currentstroke}{rgb}{0.000000,0.000000,0.000000}%
\pgfsetstrokecolor{currentstroke}%
\pgfsetdash{}{0pt}%
\pgfpathmoveto{\pgfqpoint{4.774774in}{3.116036in}}%
\pgfpathlineto{\pgfqpoint{4.785810in}{3.104285in}}%
\pgfpathlineto{\pgfqpoint{4.796866in}{3.092218in}}%
\pgfpathlineto{\pgfqpoint{4.830139in}{3.077667in}}%
\pgfpathlineto{\pgfqpoint{4.863390in}{3.064481in}}%
\pgfpathlineto{\pgfqpoint{4.852286in}{3.077412in}}%
\pgfpathlineto{\pgfqpoint{4.841202in}{3.090033in}}%
\pgfpathlineto{\pgfqpoint{4.808000in}{3.102417in}}%
\pgfpathlineto{\pgfqpoint{4.774774in}{3.116036in}}%
\pgfpathclose%
\pgfusepath{fill}%
\end{pgfscope}%
\begin{pgfscope}%
\pgfpathrectangle{\pgfqpoint{1.020000in}{0.880000in}}{\pgfqpoint{6.160000in}{6.160000in}}%
\pgfusepath{clip}%
\pgfsetbuttcap%
\pgfsetroundjoin%
\definecolor{currentfill}{rgb}{0.532568,0.669801,0.990393}%
\pgfsetfillcolor{currentfill}%
\pgfsetlinewidth{0.000000pt}%
\definecolor{currentstroke}{rgb}{0.000000,0.000000,0.000000}%
\pgfsetstrokecolor{currentstroke}%
\pgfsetdash{}{0pt}%
\pgfpathmoveto{\pgfqpoint{4.464838in}{3.291552in}}%
\pgfpathlineto{\pgfqpoint{4.475598in}{3.286738in}}%
\pgfpathlineto{\pgfqpoint{4.486379in}{3.281913in}}%
\pgfpathlineto{\pgfqpoint{4.519781in}{3.260076in}}%
\pgfpathlineto{\pgfqpoint{4.553151in}{3.239283in}}%
\pgfpathlineto{\pgfqpoint{4.542319in}{3.246072in}}%
\pgfpathlineto{\pgfqpoint{4.531509in}{3.252827in}}%
\pgfpathlineto{\pgfqpoint{4.498189in}{3.271724in}}%
\pgfpathlineto{\pgfqpoint{4.464838in}{3.291552in}}%
\pgfpathclose%
\pgfusepath{fill}%
\end{pgfscope}%
\begin{pgfscope}%
\pgfpathrectangle{\pgfqpoint{1.020000in}{0.880000in}}{\pgfqpoint{6.160000in}{6.160000in}}%
\pgfusepath{clip}%
\pgfsetbuttcap%
\pgfsetroundjoin%
\definecolor{currentfill}{rgb}{0.863392,0.865084,0.867634}%
\pgfsetfillcolor{currentfill}%
\pgfsetlinewidth{0.000000pt}%
\definecolor{currentstroke}{rgb}{0.000000,0.000000,0.000000}%
\pgfsetstrokecolor{currentstroke}%
\pgfsetdash{}{0pt}%
\pgfpathmoveto{\pgfqpoint{2.713854in}{3.939577in}}%
\pgfpathlineto{\pgfqpoint{2.723485in}{3.887350in}}%
\pgfpathlineto{\pgfqpoint{2.733105in}{3.836665in}}%
\pgfpathlineto{\pgfqpoint{2.766877in}{3.855644in}}%
\pgfpathlineto{\pgfqpoint{2.800642in}{3.874398in}}%
\pgfpathlineto{\pgfqpoint{2.790952in}{3.927605in}}%
\pgfpathlineto{\pgfqpoint{2.781250in}{3.982572in}}%
\pgfpathlineto{\pgfqpoint{2.747554in}{3.961248in}}%
\pgfpathlineto{\pgfqpoint{2.713854in}{3.939577in}}%
\pgfpathclose%
\pgfusepath{fill}%
\end{pgfscope}%
\begin{pgfscope}%
\pgfpathrectangle{\pgfqpoint{1.020000in}{0.880000in}}{\pgfqpoint{6.160000in}{6.160000in}}%
\pgfusepath{clip}%
\pgfsetbuttcap%
\pgfsetroundjoin%
\definecolor{currentfill}{rgb}{0.333490,0.446265,0.874452}%
\pgfsetfillcolor{currentfill}%
\pgfsetlinewidth{0.000000pt}%
\definecolor{currentstroke}{rgb}{0.000000,0.000000,0.000000}%
\pgfsetstrokecolor{currentstroke}%
\pgfsetdash{}{0pt}%
\pgfpathmoveto{\pgfqpoint{5.306694in}{2.928239in}}%
\pgfpathlineto{\pgfqpoint{5.318197in}{2.912488in}}%
\pgfpathlineto{\pgfqpoint{5.329720in}{2.896648in}}%
\pgfpathlineto{\pgfqpoint{5.362884in}{2.895227in}}%
\pgfpathlineto{\pgfqpoint{5.396032in}{2.894295in}}%
\pgfpathlineto{\pgfqpoint{5.384456in}{2.910116in}}%
\pgfpathlineto{\pgfqpoint{5.372900in}{2.925868in}}%
\pgfpathlineto{\pgfqpoint{5.339806in}{2.926813in}}%
\pgfpathlineto{\pgfqpoint{5.306694in}{2.928239in}}%
\pgfpathclose%
\pgfusepath{fill}%
\end{pgfscope}%
\begin{pgfscope}%
\pgfpathrectangle{\pgfqpoint{1.020000in}{0.880000in}}{\pgfqpoint{6.160000in}{6.160000in}}%
\pgfusepath{clip}%
\pgfsetbuttcap%
\pgfsetroundjoin%
\definecolor{currentfill}{rgb}{0.777378,0.840921,0.946149}%
\pgfsetfillcolor{currentfill}%
\pgfsetlinewidth{0.000000pt}%
\definecolor{currentstroke}{rgb}{0.000000,0.000000,0.000000}%
\pgfsetstrokecolor{currentstroke}%
\pgfsetdash{}{0pt}%
\pgfpathmoveto{\pgfqpoint{3.467868in}{3.741774in}}%
\pgfpathlineto{\pgfqpoint{3.477759in}{3.715384in}}%
\pgfpathlineto{\pgfqpoint{3.487641in}{3.692871in}}%
\pgfpathlineto{\pgfqpoint{3.521444in}{3.690846in}}%
\pgfpathlineto{\pgfqpoint{3.555233in}{3.687296in}}%
\pgfpathlineto{\pgfqpoint{3.545315in}{3.707371in}}%
\pgfpathlineto{\pgfqpoint{3.535392in}{3.731247in}}%
\pgfpathlineto{\pgfqpoint{3.501638in}{3.737267in}}%
\pgfpathlineto{\pgfqpoint{3.467868in}{3.741774in}}%
\pgfpathclose%
\pgfusepath{fill}%
\end{pgfscope}%
\begin{pgfscope}%
\pgfpathrectangle{\pgfqpoint{1.020000in}{0.880000in}}{\pgfqpoint{6.160000in}{6.160000in}}%
\pgfusepath{clip}%
\pgfsetbuttcap%
\pgfsetroundjoin%
\definecolor{currentfill}{rgb}{0.630089,0.752516,0.998508}%
\pgfsetfillcolor{currentfill}%
\pgfsetlinewidth{0.000000pt}%
\definecolor{currentstroke}{rgb}{0.000000,0.000000,0.000000}%
\pgfsetstrokecolor{currentstroke}%
\pgfsetdash{}{0pt}%
\pgfpathmoveto{\pgfqpoint{4.154877in}{3.460824in}}%
\pgfpathlineto{\pgfqpoint{4.165320in}{3.462122in}}%
\pgfpathlineto{\pgfqpoint{4.175786in}{3.464432in}}%
\pgfpathlineto{\pgfqpoint{4.209358in}{3.443352in}}%
\pgfpathlineto{\pgfqpoint{4.242894in}{3.422178in}}%
\pgfpathlineto{\pgfqpoint{4.232368in}{3.421789in}}%
\pgfpathlineto{\pgfqpoint{4.221864in}{3.422265in}}%
\pgfpathlineto{\pgfqpoint{4.188388in}{3.441554in}}%
\pgfpathlineto{\pgfqpoint{4.154877in}{3.460824in}}%
\pgfpathclose%
\pgfusepath{fill}%
\end{pgfscope}%
\begin{pgfscope}%
\pgfpathrectangle{\pgfqpoint{1.020000in}{0.880000in}}{\pgfqpoint{6.160000in}{6.160000in}}%
\pgfusepath{clip}%
\pgfsetbuttcap%
\pgfsetroundjoin%
\definecolor{currentfill}{rgb}{0.581486,0.713451,0.998314}%
\pgfsetfillcolor{currentfill}%
\pgfsetlinewidth{0.000000pt}%
\definecolor{currentstroke}{rgb}{0.000000,0.000000,0.000000}%
\pgfsetstrokecolor{currentstroke}%
\pgfsetdash{}{0pt}%
\pgfpathmoveto{\pgfqpoint{4.309861in}{3.380216in}}%
\pgfpathlineto{\pgfqpoint{4.320468in}{3.379158in}}%
\pgfpathlineto{\pgfqpoint{4.331098in}{3.378499in}}%
\pgfpathlineto{\pgfqpoint{4.364585in}{3.355838in}}%
\pgfpathlineto{\pgfqpoint{4.398037in}{3.333710in}}%
\pgfpathlineto{\pgfqpoint{4.387352in}{3.336563in}}%
\pgfpathlineto{\pgfqpoint{4.376690in}{3.339738in}}%
\pgfpathlineto{\pgfqpoint{4.343293in}{3.359726in}}%
\pgfpathlineto{\pgfqpoint{4.309861in}{3.380216in}}%
\pgfpathclose%
\pgfusepath{fill}%
\end{pgfscope}%
\begin{pgfscope}%
\pgfpathrectangle{\pgfqpoint{1.020000in}{0.880000in}}{\pgfqpoint{6.160000in}{6.160000in}}%
\pgfusepath{clip}%
\pgfsetbuttcap%
\pgfsetroundjoin%
\definecolor{currentfill}{rgb}{0.394042,0.522413,0.924916}%
\pgfsetfillcolor{currentfill}%
\pgfsetlinewidth{0.000000pt}%
\definecolor{currentstroke}{rgb}{0.000000,0.000000,0.000000}%
\pgfsetstrokecolor{currentstroke}%
\pgfsetdash{}{0pt}%
\pgfpathmoveto{\pgfqpoint{4.929827in}{3.041981in}}%
\pgfpathlineto{\pgfqpoint{4.940998in}{3.028109in}}%
\pgfpathlineto{\pgfqpoint{4.952188in}{3.013938in}}%
\pgfpathlineto{\pgfqpoint{4.985425in}{3.004029in}}%
\pgfpathlineto{\pgfqpoint{5.018643in}{2.995318in}}%
\pgfpathlineto{\pgfqpoint{5.007404in}{3.009912in}}%
\pgfpathlineto{\pgfqpoint{4.996184in}{3.024217in}}%
\pgfpathlineto{\pgfqpoint{4.963015in}{3.032541in}}%
\pgfpathlineto{\pgfqpoint{4.929827in}{3.041981in}}%
\pgfpathclose%
\pgfusepath{fill}%
\end{pgfscope}%
\begin{pgfscope}%
\pgfpathrectangle{\pgfqpoint{1.020000in}{0.880000in}}{\pgfqpoint{6.160000in}{6.160000in}}%
\pgfusepath{clip}%
\pgfsetbuttcap%
\pgfsetroundjoin%
\definecolor{currentfill}{rgb}{0.863392,0.865084,0.867634}%
\pgfsetfillcolor{currentfill}%
\pgfsetlinewidth{0.000000pt}%
\definecolor{currentstroke}{rgb}{0.000000,0.000000,0.000000}%
\pgfsetstrokecolor{currentstroke}%
\pgfsetdash{}{0pt}%
\pgfpathmoveto{\pgfqpoint{2.935667in}{3.943264in}}%
\pgfpathlineto{\pgfqpoint{2.945455in}{3.888491in}}%
\pgfpathlineto{\pgfqpoint{2.955223in}{3.836335in}}%
\pgfpathlineto{\pgfqpoint{2.989025in}{3.849985in}}%
\pgfpathlineto{\pgfqpoint{3.022825in}{3.862532in}}%
\pgfpathlineto{\pgfqpoint{3.013013in}{3.915485in}}%
\pgfpathlineto{\pgfqpoint{3.003181in}{3.971284in}}%
\pgfpathlineto{\pgfqpoint{2.969423in}{3.957965in}}%
\pgfpathlineto{\pgfqpoint{2.935667in}{3.943264in}}%
\pgfpathclose%
\pgfusepath{fill}%
\end{pgfscope}%
\begin{pgfscope}%
\pgfpathrectangle{\pgfqpoint{1.020000in}{0.880000in}}{\pgfqpoint{6.160000in}{6.160000in}}%
\pgfusepath{clip}%
\pgfsetbuttcap%
\pgfsetroundjoin%
\definecolor{currentfill}{rgb}{0.847365,0.862472,0.885540}%
\pgfsetfillcolor{currentfill}%
\pgfsetlinewidth{0.000000pt}%
\definecolor{currentstroke}{rgb}{0.000000,0.000000,0.000000}%
\pgfsetstrokecolor{currentstroke}%
\pgfsetdash{}{0pt}%
\pgfpathmoveto{\pgfqpoint{2.646430in}{3.896050in}}%
\pgfpathlineto{\pgfqpoint{2.655983in}{3.846742in}}%
\pgfpathlineto{\pgfqpoint{2.665528in}{3.798761in}}%
\pgfpathlineto{\pgfqpoint{2.699322in}{3.817645in}}%
\pgfpathlineto{\pgfqpoint{2.733105in}{3.836665in}}%
\pgfpathlineto{\pgfqpoint{2.723485in}{3.887350in}}%
\pgfpathlineto{\pgfqpoint{2.713854in}{3.939577in}}%
\pgfpathlineto{\pgfqpoint{2.680147in}{3.917775in}}%
\pgfpathlineto{\pgfqpoint{2.646430in}{3.896050in}}%
\pgfpathclose%
\pgfusepath{fill}%
\end{pgfscope}%
\begin{pgfscope}%
\pgfpathrectangle{\pgfqpoint{1.020000in}{0.880000in}}{\pgfqpoint{6.160000in}{6.160000in}}%
\pgfusepath{clip}%
\pgfsetbuttcap%
\pgfsetroundjoin%
\definecolor{currentfill}{rgb}{0.748682,0.827679,0.963334}%
\pgfsetfillcolor{currentfill}%
\pgfsetlinewidth{0.000000pt}%
\definecolor{currentstroke}{rgb}{0.000000,0.000000,0.000000}%
\pgfsetstrokecolor{currentstroke}%
\pgfsetdash{}{0pt}%
\pgfpathmoveto{\pgfqpoint{3.622760in}{3.675734in}}%
\pgfpathlineto{\pgfqpoint{3.632716in}{3.661588in}}%
\pgfpathlineto{\pgfqpoint{3.642672in}{3.650926in}}%
\pgfpathlineto{\pgfqpoint{3.676454in}{3.644966in}}%
\pgfpathlineto{\pgfqpoint{3.710216in}{3.637522in}}%
\pgfpathlineto{\pgfqpoint{3.700211in}{3.646391in}}%
\pgfpathlineto{\pgfqpoint{3.690210in}{3.658581in}}%
\pgfpathlineto{\pgfqpoint{3.656495in}{3.667822in}}%
\pgfpathlineto{\pgfqpoint{3.622760in}{3.675734in}}%
\pgfpathclose%
\pgfusepath{fill}%
\end{pgfscope}%
\begin{pgfscope}%
\pgfpathrectangle{\pgfqpoint{1.020000in}{0.880000in}}{\pgfqpoint{6.160000in}{6.160000in}}%
\pgfusepath{clip}%
\pgfsetbuttcap%
\pgfsetroundjoin%
\definecolor{currentfill}{rgb}{0.839351,0.861167,0.894494}%
\pgfsetfillcolor{currentfill}%
\pgfsetlinewidth{0.000000pt}%
\definecolor{currentstroke}{rgb}{0.000000,0.000000,0.000000}%
\pgfsetstrokecolor{currentstroke}%
\pgfsetdash{}{0pt}%
\pgfpathmoveto{\pgfqpoint{3.090421in}{3.883709in}}%
\pgfpathlineto{\pgfqpoint{3.100248in}{3.833754in}}%
\pgfpathlineto{\pgfqpoint{3.110051in}{3.787147in}}%
\pgfpathlineto{\pgfqpoint{3.143879in}{3.796023in}}%
\pgfpathlineto{\pgfqpoint{3.177703in}{3.803537in}}%
\pgfpathlineto{\pgfqpoint{3.167869in}{3.849394in}}%
\pgfpathlineto{\pgfqpoint{3.158013in}{3.898782in}}%
\pgfpathlineto{\pgfqpoint{3.124218in}{3.892067in}}%
\pgfpathlineto{\pgfqpoint{3.090421in}{3.883709in}}%
\pgfpathclose%
\pgfusepath{fill}%
\end{pgfscope}%
\begin{pgfscope}%
\pgfpathrectangle{\pgfqpoint{1.020000in}{0.880000in}}{\pgfqpoint{6.160000in}{6.160000in}}%
\pgfusepath{clip}%
\pgfsetbuttcap%
\pgfsetroundjoin%
\definecolor{currentfill}{rgb}{0.363461,0.484784,0.901019}%
\pgfsetfillcolor{currentfill}%
\pgfsetlinewidth{0.000000pt}%
\definecolor{currentstroke}{rgb}{0.000000,0.000000,0.000000}%
\pgfsetstrokecolor{currentstroke}%
\pgfsetdash{}{0pt}%
\pgfpathmoveto{\pgfqpoint{5.085025in}{2.981172in}}%
\pgfpathlineto{\pgfqpoint{5.096333in}{2.966046in}}%
\pgfpathlineto{\pgfqpoint{5.107661in}{2.950702in}}%
\pgfpathlineto{\pgfqpoint{5.140876in}{2.944926in}}%
\pgfpathlineto{\pgfqpoint{5.174074in}{2.940070in}}%
\pgfpathlineto{\pgfqpoint{5.162695in}{2.955564in}}%
\pgfpathlineto{\pgfqpoint{5.151336in}{2.970861in}}%
\pgfpathlineto{\pgfqpoint{5.118189in}{2.975576in}}%
\pgfpathlineto{\pgfqpoint{5.085025in}{2.981172in}}%
\pgfpathclose%
\pgfusepath{fill}%
\end{pgfscope}%
\begin{pgfscope}%
\pgfpathrectangle{\pgfqpoint{1.020000in}{0.880000in}}{\pgfqpoint{6.160000in}{6.160000in}}%
\pgfusepath{clip}%
\pgfsetbuttcap%
\pgfsetroundjoin%
\definecolor{currentfill}{rgb}{0.294718,0.393542,0.834384}%
\pgfsetfillcolor{currentfill}%
\pgfsetlinewidth{0.000000pt}%
\definecolor{currentstroke}{rgb}{0.000000,0.000000,0.000000}%
\pgfsetstrokecolor{currentstroke}%
\pgfsetdash{}{0pt}%
\pgfpathmoveto{\pgfqpoint{5.906184in}{2.846989in}}%
\pgfpathlineto{\pgfqpoint{5.918251in}{2.831787in}}%
\pgfpathlineto{\pgfqpoint{5.930341in}{2.816632in}}%
\pgfpathlineto{\pgfqpoint{5.963367in}{2.818754in}}%
\pgfpathlineto{\pgfqpoint{5.996373in}{2.820915in}}%
\pgfpathlineto{\pgfqpoint{5.984228in}{2.835981in}}%
\pgfpathlineto{\pgfqpoint{5.972107in}{2.851094in}}%
\pgfpathlineto{\pgfqpoint{5.939156in}{2.849021in}}%
\pgfpathlineto{\pgfqpoint{5.906184in}{2.846989in}}%
\pgfpathclose%
\pgfusepath{fill}%
\end{pgfscope}%
\begin{pgfscope}%
\pgfpathrectangle{\pgfqpoint{1.020000in}{0.880000in}}{\pgfqpoint{6.160000in}{6.160000in}}%
\pgfusepath{clip}%
\pgfsetbuttcap%
\pgfsetroundjoin%
\definecolor{currentfill}{rgb}{0.304174,0.406945,0.845263}%
\pgfsetfillcolor{currentfill}%
\pgfsetlinewidth{0.000000pt}%
\definecolor{currentstroke}{rgb}{0.000000,0.000000,0.000000}%
\pgfsetstrokecolor{currentstroke}%
\pgfsetdash{}{0pt}%
\pgfpathmoveto{\pgfqpoint{5.684204in}{2.867250in}}%
\pgfpathlineto{\pgfqpoint{5.696061in}{2.851691in}}%
\pgfpathlineto{\pgfqpoint{5.707941in}{2.836168in}}%
\pgfpathlineto{\pgfqpoint{5.741030in}{2.837745in}}%
\pgfpathlineto{\pgfqpoint{5.774100in}{2.839433in}}%
\pgfpathlineto{\pgfqpoint{5.762166in}{2.854863in}}%
\pgfpathlineto{\pgfqpoint{5.750255in}{2.870334in}}%
\pgfpathlineto{\pgfqpoint{5.717239in}{2.868736in}}%
\pgfpathlineto{\pgfqpoint{5.684204in}{2.867250in}}%
\pgfpathclose%
\pgfusepath{fill}%
\end{pgfscope}%
\begin{pgfscope}%
\pgfpathrectangle{\pgfqpoint{1.020000in}{0.880000in}}{\pgfqpoint{6.160000in}{6.160000in}}%
\pgfusepath{clip}%
\pgfsetbuttcap%
\pgfsetroundjoin%
\definecolor{currentfill}{rgb}{0.851372,0.863125,0.881064}%
\pgfsetfillcolor{currentfill}%
\pgfsetlinewidth{0.000000pt}%
\definecolor{currentstroke}{rgb}{0.000000,0.000000,0.000000}%
\pgfsetstrokecolor{currentstroke}%
\pgfsetdash{}{0pt}%
\pgfpathmoveto{\pgfqpoint{2.868158in}{3.910455in}}%
\pgfpathlineto{\pgfqpoint{2.877893in}{3.857231in}}%
\pgfpathlineto{\pgfqpoint{2.887610in}{3.806377in}}%
\pgfpathlineto{\pgfqpoint{2.921419in}{3.821743in}}%
\pgfpathlineto{\pgfqpoint{2.955223in}{3.836335in}}%
\pgfpathlineto{\pgfqpoint{2.945455in}{3.888491in}}%
\pgfpathlineto{\pgfqpoint{2.935667in}{3.943264in}}%
\pgfpathlineto{\pgfqpoint{2.901913in}{3.927364in}}%
\pgfpathlineto{\pgfqpoint{2.868158in}{3.910455in}}%
\pgfpathclose%
\pgfusepath{fill}%
\end{pgfscope}%
\begin{pgfscope}%
\pgfpathrectangle{\pgfqpoint{1.020000in}{0.880000in}}{\pgfqpoint{6.160000in}{6.160000in}}%
\pgfusepath{clip}%
\pgfsetbuttcap%
\pgfsetroundjoin%
\definecolor{currentfill}{rgb}{0.289996,0.386836,0.828926}%
\pgfsetfillcolor{currentfill}%
\pgfsetlinewidth{0.000000pt}%
\definecolor{currentstroke}{rgb}{0.000000,0.000000,0.000000}%
\pgfsetstrokecolor{currentstroke}%
\pgfsetdash{}{0pt}%
\pgfpathmoveto{\pgfqpoint{6.128197in}{2.829851in}}%
\pgfpathlineto{\pgfqpoint{6.140476in}{2.815005in}}%
\pgfpathlineto{\pgfqpoint{6.152778in}{2.800206in}}%
\pgfpathlineto{\pgfqpoint{6.185738in}{2.802577in}}%
\pgfpathlineto{\pgfqpoint{6.173409in}{2.817334in}}%
\pgfpathlineto{\pgfqpoint{6.161103in}{2.832139in}}%
\pgfpathlineto{\pgfqpoint{6.128197in}{2.829851in}}%
\pgfpathclose%
\pgfusepath{fill}%
\end{pgfscope}%
\begin{pgfscope}%
\pgfpathrectangle{\pgfqpoint{1.020000in}{0.880000in}}{\pgfqpoint{6.160000in}{6.160000in}}%
\pgfusepath{clip}%
\pgfsetbuttcap%
\pgfsetroundjoin%
\definecolor{currentfill}{rgb}{0.318832,0.426605,0.859857}%
\pgfsetfillcolor{currentfill}%
\pgfsetlinewidth{0.000000pt}%
\definecolor{currentstroke}{rgb}{0.000000,0.000000,0.000000}%
\pgfsetstrokecolor{currentstroke}%
\pgfsetdash{}{0pt}%
\pgfpathmoveto{\pgfqpoint{5.462273in}{2.893674in}}%
\pgfpathlineto{\pgfqpoint{5.473924in}{2.877860in}}%
\pgfpathlineto{\pgfqpoint{5.485596in}{2.862037in}}%
\pgfpathlineto{\pgfqpoint{5.518744in}{2.862309in}}%
\pgfpathlineto{\pgfqpoint{5.551873in}{2.862868in}}%
\pgfpathlineto{\pgfqpoint{5.540147in}{2.878618in}}%
\pgfpathlineto{\pgfqpoint{5.528443in}{2.894371in}}%
\pgfpathlineto{\pgfqpoint{5.495367in}{2.893880in}}%
\pgfpathlineto{\pgfqpoint{5.462273in}{2.893674in}}%
\pgfpathclose%
\pgfusepath{fill}%
\end{pgfscope}%
\begin{pgfscope}%
\pgfpathrectangle{\pgfqpoint{1.020000in}{0.880000in}}{\pgfqpoint{6.160000in}{6.160000in}}%
\pgfusepath{clip}%
\pgfsetbuttcap%
\pgfsetroundjoin%
\definecolor{currentfill}{rgb}{0.809329,0.852974,0.922323}%
\pgfsetfillcolor{currentfill}%
\pgfsetlinewidth{0.000000pt}%
\definecolor{currentstroke}{rgb}{0.000000,0.000000,0.000000}%
\pgfsetstrokecolor{currentstroke}%
\pgfsetdash{}{0pt}%
\pgfpathmoveto{\pgfqpoint{3.245337in}{3.814099in}}%
\pgfpathlineto{\pgfqpoint{3.255177in}{3.773284in}}%
\pgfpathlineto{\pgfqpoint{3.264998in}{3.736276in}}%
\pgfpathlineto{\pgfqpoint{3.298833in}{3.740987in}}%
\pgfpathlineto{\pgfqpoint{3.332660in}{3.744230in}}%
\pgfpathlineto{\pgfqpoint{3.322812in}{3.779285in}}%
\pgfpathlineto{\pgfqpoint{3.312947in}{3.818236in}}%
\pgfpathlineto{\pgfqpoint{3.279146in}{3.816997in}}%
\pgfpathlineto{\pgfqpoint{3.245337in}{3.814099in}}%
\pgfpathclose%
\pgfusepath{fill}%
\end{pgfscope}%
\begin{pgfscope}%
\pgfpathrectangle{\pgfqpoint{1.020000in}{0.880000in}}{\pgfqpoint{6.160000in}{6.160000in}}%
\pgfusepath{clip}%
\pgfsetbuttcap%
\pgfsetroundjoin%
\definecolor{currentfill}{rgb}{0.831148,0.859513,0.903110}%
\pgfsetfillcolor{currentfill}%
\pgfsetlinewidth{0.000000pt}%
\definecolor{currentstroke}{rgb}{0.000000,0.000000,0.000000}%
\pgfsetstrokecolor{currentstroke}%
\pgfsetdash{}{0pt}%
\pgfpathmoveto{\pgfqpoint{2.578956in}{3.853595in}}%
\pgfpathlineto{\pgfqpoint{2.588426in}{3.807256in}}%
\pgfpathlineto{\pgfqpoint{2.597891in}{3.762041in}}%
\pgfpathlineto{\pgfqpoint{2.631718in}{3.780176in}}%
\pgfpathlineto{\pgfqpoint{2.665528in}{3.798761in}}%
\pgfpathlineto{\pgfqpoint{2.655983in}{3.846742in}}%
\pgfpathlineto{\pgfqpoint{2.646430in}{3.896050in}}%
\pgfpathlineto{\pgfqpoint{2.612701in}{3.874596in}}%
\pgfpathlineto{\pgfqpoint{2.578956in}{3.853595in}}%
\pgfpathclose%
\pgfusepath{fill}%
\end{pgfscope}%
\begin{pgfscope}%
\pgfpathrectangle{\pgfqpoint{1.020000in}{0.880000in}}{\pgfqpoint{6.160000in}{6.160000in}}%
\pgfusepath{clip}%
\pgfsetbuttcap%
\pgfsetroundjoin%
\definecolor{currentfill}{rgb}{0.724041,0.814910,0.975651}%
\pgfsetfillcolor{currentfill}%
\pgfsetlinewidth{0.000000pt}%
\definecolor{currentstroke}{rgb}{0.000000,0.000000,0.000000}%
\pgfsetstrokecolor{currentstroke}%
\pgfsetdash{}{0pt}%
\pgfpathmoveto{\pgfqpoint{3.777675in}{3.618475in}}%
\pgfpathlineto{\pgfqpoint{3.787741in}{3.614021in}}%
\pgfpathlineto{\pgfqpoint{3.797817in}{3.612374in}}%
\pgfpathlineto{\pgfqpoint{3.831569in}{3.601773in}}%
\pgfpathlineto{\pgfqpoint{3.865296in}{3.589826in}}%
\pgfpathlineto{\pgfqpoint{3.855160in}{3.590870in}}%
\pgfpathlineto{\pgfqpoint{3.845037in}{3.594514in}}%
\pgfpathlineto{\pgfqpoint{3.811369in}{3.607055in}}%
\pgfpathlineto{\pgfqpoint{3.777675in}{3.618475in}}%
\pgfpathclose%
\pgfusepath{fill}%
\end{pgfscope}%
\begin{pgfscope}%
\pgfpathrectangle{\pgfqpoint{1.020000in}{0.880000in}}{\pgfqpoint{6.160000in}{6.160000in}}%
\pgfusepath{clip}%
\pgfsetbuttcap%
\pgfsetroundjoin%
\definecolor{currentfill}{rgb}{0.338377,0.452819,0.879317}%
\pgfsetfillcolor{currentfill}%
\pgfsetlinewidth{0.000000pt}%
\definecolor{currentstroke}{rgb}{0.000000,0.000000,0.000000}%
\pgfsetstrokecolor{currentstroke}%
\pgfsetdash{}{0pt}%
\pgfpathmoveto{\pgfqpoint{5.240418in}{2.932789in}}%
\pgfpathlineto{\pgfqpoint{5.251868in}{2.917065in}}%
\pgfpathlineto{\pgfqpoint{5.263339in}{2.901227in}}%
\pgfpathlineto{\pgfqpoint{5.296538in}{2.898625in}}%
\pgfpathlineto{\pgfqpoint{5.329720in}{2.896648in}}%
\pgfpathlineto{\pgfqpoint{5.318197in}{2.912488in}}%
\pgfpathlineto{\pgfqpoint{5.306694in}{2.928239in}}%
\pgfpathlineto{\pgfqpoint{5.273565in}{2.930208in}}%
\pgfpathlineto{\pgfqpoint{5.240418in}{2.932789in}}%
\pgfpathclose%
\pgfusepath{fill}%
\end{pgfscope}%
\begin{pgfscope}%
\pgfpathrectangle{\pgfqpoint{1.020000in}{0.880000in}}{\pgfqpoint{6.160000in}{6.160000in}}%
\pgfusepath{clip}%
\pgfsetbuttcap%
\pgfsetroundjoin%
\definecolor{currentfill}{rgb}{0.782049,0.842864,0.942980}%
\pgfsetfillcolor{currentfill}%
\pgfsetlinewidth{0.000000pt}%
\definecolor{currentstroke}{rgb}{0.000000,0.000000,0.000000}%
\pgfsetstrokecolor{currentstroke}%
\pgfsetdash{}{0pt}%
\pgfpathmoveto{\pgfqpoint{3.400287in}{3.746129in}}%
\pgfpathlineto{\pgfqpoint{3.410147in}{3.717277in}}%
\pgfpathlineto{\pgfqpoint{3.419995in}{3.692322in}}%
\pgfpathlineto{\pgfqpoint{3.453824in}{3.693360in}}%
\pgfpathlineto{\pgfqpoint{3.487641in}{3.692871in}}%
\pgfpathlineto{\pgfqpoint{3.477759in}{3.715384in}}%
\pgfpathlineto{\pgfqpoint{3.467868in}{3.741774in}}%
\pgfpathlineto{\pgfqpoint{3.434084in}{3.744734in}}%
\pgfpathlineto{\pgfqpoint{3.400287in}{3.746129in}}%
\pgfpathclose%
\pgfusepath{fill}%
\end{pgfscope}%
\begin{pgfscope}%
\pgfpathrectangle{\pgfqpoint{1.020000in}{0.880000in}}{\pgfqpoint{6.160000in}{6.160000in}}%
\pgfusepath{clip}%
\pgfsetbuttcap%
\pgfsetroundjoin%
\definecolor{currentfill}{rgb}{0.831148,0.859513,0.903110}%
\pgfsetfillcolor{currentfill}%
\pgfsetlinewidth{0.000000pt}%
\definecolor{currentstroke}{rgb}{0.000000,0.000000,0.000000}%
\pgfsetstrokecolor{currentstroke}%
\pgfsetdash{}{0pt}%
\pgfpathmoveto{\pgfqpoint{3.022825in}{3.862532in}}%
\pgfpathlineto{\pgfqpoint{3.032615in}{3.812587in}}%
\pgfpathlineto{\pgfqpoint{3.042383in}{3.765772in}}%
\pgfpathlineto{\pgfqpoint{3.076219in}{3.777022in}}%
\pgfpathlineto{\pgfqpoint{3.110051in}{3.787147in}}%
\pgfpathlineto{\pgfqpoint{3.100248in}{3.833754in}}%
\pgfpathlineto{\pgfqpoint{3.090421in}{3.883709in}}%
\pgfpathlineto{\pgfqpoint{3.056624in}{3.873821in}}%
\pgfpathlineto{\pgfqpoint{3.022825in}{3.862532in}}%
\pgfpathclose%
\pgfusepath{fill}%
\end{pgfscope}%
\begin{pgfscope}%
\pgfpathrectangle{\pgfqpoint{1.020000in}{0.880000in}}{\pgfqpoint{6.160000in}{6.160000in}}%
\pgfusepath{clip}%
\pgfsetbuttcap%
\pgfsetroundjoin%
\definecolor{currentfill}{rgb}{0.839351,0.861167,0.894494}%
\pgfsetfillcolor{currentfill}%
\pgfsetlinewidth{0.000000pt}%
\definecolor{currentstroke}{rgb}{0.000000,0.000000,0.000000}%
\pgfsetstrokecolor{currentstroke}%
\pgfsetdash{}{0pt}%
\pgfpathmoveto{\pgfqpoint{2.800642in}{3.874398in}}%
\pgfpathlineto{\pgfqpoint{2.810317in}{3.823146in}}%
\pgfpathlineto{\pgfqpoint{2.819976in}{3.774008in}}%
\pgfpathlineto{\pgfqpoint{2.853796in}{3.790409in}}%
\pgfpathlineto{\pgfqpoint{2.887610in}{3.806377in}}%
\pgfpathlineto{\pgfqpoint{2.877893in}{3.857231in}}%
\pgfpathlineto{\pgfqpoint{2.868158in}{3.910455in}}%
\pgfpathlineto{\pgfqpoint{2.834402in}{3.892734in}}%
\pgfpathlineto{\pgfqpoint{2.800642in}{3.874398in}}%
\pgfpathclose%
\pgfusepath{fill}%
\end{pgfscope}%
\begin{pgfscope}%
\pgfpathrectangle{\pgfqpoint{1.020000in}{0.880000in}}{\pgfqpoint{6.160000in}{6.160000in}}%
\pgfusepath{clip}%
\pgfsetbuttcap%
\pgfsetroundjoin%
\definecolor{currentfill}{rgb}{0.818056,0.855590,0.914638}%
\pgfsetfillcolor{currentfill}%
\pgfsetlinewidth{0.000000pt}%
\definecolor{currentstroke}{rgb}{0.000000,0.000000,0.000000}%
\pgfsetstrokecolor{currentstroke}%
\pgfsetdash{}{0pt}%
\pgfpathmoveto{\pgfqpoint{2.511408in}{3.813578in}}%
\pgfpathlineto{\pgfqpoint{2.520793in}{3.770147in}}%
\pgfpathlineto{\pgfqpoint{2.530175in}{3.727651in}}%
\pgfpathlineto{\pgfqpoint{2.564044in}{3.744493in}}%
\pgfpathlineto{\pgfqpoint{2.597891in}{3.762041in}}%
\pgfpathlineto{\pgfqpoint{2.588426in}{3.807256in}}%
\pgfpathlineto{\pgfqpoint{2.578956in}{3.853595in}}%
\pgfpathlineto{\pgfqpoint{2.545193in}{3.833207in}}%
\pgfpathlineto{\pgfqpoint{2.511408in}{3.813578in}}%
\pgfpathclose%
\pgfusepath{fill}%
\end{pgfscope}%
\begin{pgfscope}%
\pgfpathrectangle{\pgfqpoint{1.020000in}{0.880000in}}{\pgfqpoint{6.160000in}{6.160000in}}%
\pgfusepath{clip}%
\pgfsetbuttcap%
\pgfsetroundjoin%
\definecolor{currentfill}{rgb}{0.451739,0.588181,0.960201}%
\pgfsetfillcolor{currentfill}%
\pgfsetlinewidth{0.000000pt}%
\definecolor{currentstroke}{rgb}{0.000000,0.000000,0.000000}%
\pgfsetstrokecolor{currentstroke}%
\pgfsetdash{}{0pt}%
\pgfpathmoveto{\pgfqpoint{4.708250in}{3.147139in}}%
\pgfpathlineto{\pgfqpoint{4.719239in}{3.136530in}}%
\pgfpathlineto{\pgfqpoint{4.730247in}{3.125571in}}%
\pgfpathlineto{\pgfqpoint{4.763569in}{3.108175in}}%
\pgfpathlineto{\pgfqpoint{4.796866in}{3.092218in}}%
\pgfpathlineto{\pgfqpoint{4.785810in}{3.104285in}}%
\pgfpathlineto{\pgfqpoint{4.774774in}{3.116036in}}%
\pgfpathlineto{\pgfqpoint{4.741524in}{3.130934in}}%
\pgfpathlineto{\pgfqpoint{4.708250in}{3.147139in}}%
\pgfpathclose%
\pgfusepath{fill}%
\end{pgfscope}%
\begin{pgfscope}%
\pgfpathrectangle{\pgfqpoint{1.020000in}{0.880000in}}{\pgfqpoint{6.160000in}{6.160000in}}%
\pgfusepath{clip}%
\pgfsetbuttcap%
\pgfsetroundjoin%
\definecolor{currentfill}{rgb}{0.693321,0.796314,0.986308}%
\pgfsetfillcolor{currentfill}%
\pgfsetlinewidth{0.000000pt}%
\definecolor{currentstroke}{rgb}{0.000000,0.000000,0.000000}%
\pgfsetstrokecolor{currentstroke}%
\pgfsetdash{}{0pt}%
\pgfpathmoveto{\pgfqpoint{3.932668in}{3.562388in}}%
\pgfpathlineto{\pgfqpoint{3.942882in}{3.563684in}}%
\pgfpathlineto{\pgfqpoint{3.953113in}{3.566963in}}%
\pgfpathlineto{\pgfqpoint{3.986820in}{3.551244in}}%
\pgfpathlineto{\pgfqpoint{4.020497in}{3.534512in}}%
\pgfpathlineto{\pgfqpoint{4.010201in}{3.531938in}}%
\pgfpathlineto{\pgfqpoint{3.999924in}{3.531145in}}%
\pgfpathlineto{\pgfqpoint{3.966312in}{3.547170in}}%
\pgfpathlineto{\pgfqpoint{3.932668in}{3.562388in}}%
\pgfpathclose%
\pgfusepath{fill}%
\end{pgfscope}%
\begin{pgfscope}%
\pgfpathrectangle{\pgfqpoint{1.020000in}{0.880000in}}{\pgfqpoint{6.160000in}{6.160000in}}%
\pgfusepath{clip}%
\pgfsetbuttcap%
\pgfsetroundjoin%
\definecolor{currentfill}{rgb}{0.409611,0.540759,0.935545}%
\pgfsetfillcolor{currentfill}%
\pgfsetlinewidth{0.000000pt}%
\definecolor{currentstroke}{rgb}{0.000000,0.000000,0.000000}%
\pgfsetstrokecolor{currentstroke}%
\pgfsetdash{}{0pt}%
\pgfpathmoveto{\pgfqpoint{4.863390in}{3.064481in}}%
\pgfpathlineto{\pgfqpoint{4.874512in}{3.051230in}}%
\pgfpathlineto{\pgfqpoint{4.885653in}{3.037648in}}%
\pgfpathlineto{\pgfqpoint{4.918931in}{3.025120in}}%
\pgfpathlineto{\pgfqpoint{4.952188in}{3.013938in}}%
\pgfpathlineto{\pgfqpoint{4.940998in}{3.028109in}}%
\pgfpathlineto{\pgfqpoint{4.929827in}{3.041981in}}%
\pgfpathlineto{\pgfqpoint{4.896619in}{3.052605in}}%
\pgfpathlineto{\pgfqpoint{4.863390in}{3.064481in}}%
\pgfpathclose%
\pgfusepath{fill}%
\end{pgfscope}%
\begin{pgfscope}%
\pgfpathrectangle{\pgfqpoint{1.020000in}{0.880000in}}{\pgfqpoint{6.160000in}{6.160000in}}%
\pgfusepath{clip}%
\pgfsetbuttcap%
\pgfsetroundjoin%
\definecolor{currentfill}{rgb}{0.505423,0.643995,0.983157}%
\pgfsetfillcolor{currentfill}%
\pgfsetlinewidth{0.000000pt}%
\definecolor{currentstroke}{rgb}{0.000000,0.000000,0.000000}%
\pgfsetstrokecolor{currentstroke}%
\pgfsetdash{}{0pt}%
\pgfpathmoveto{\pgfqpoint{4.553151in}{3.239283in}}%
\pgfpathlineto{\pgfqpoint{4.564003in}{3.232353in}}%
\pgfpathlineto{\pgfqpoint{4.574877in}{3.225169in}}%
\pgfpathlineto{\pgfqpoint{4.608264in}{3.203711in}}%
\pgfpathlineto{\pgfqpoint{4.641621in}{3.183534in}}%
\pgfpathlineto{\pgfqpoint{4.630699in}{3.192418in}}%
\pgfpathlineto{\pgfqpoint{4.619798in}{3.201070in}}%
\pgfpathlineto{\pgfqpoint{4.586489in}{3.219600in}}%
\pgfpathlineto{\pgfqpoint{4.553151in}{3.239283in}}%
\pgfpathclose%
\pgfusepath{fill}%
\end{pgfscope}%
\begin{pgfscope}%
\pgfpathrectangle{\pgfqpoint{1.020000in}{0.880000in}}{\pgfqpoint{6.160000in}{6.160000in}}%
\pgfusepath{clip}%
\pgfsetbuttcap%
\pgfsetroundjoin%
\definecolor{currentfill}{rgb}{0.289996,0.386836,0.828926}%
\pgfsetfillcolor{currentfill}%
\pgfsetlinewidth{0.000000pt}%
\definecolor{currentstroke}{rgb}{0.000000,0.000000,0.000000}%
\pgfsetstrokecolor{currentstroke}%
\pgfsetdash{}{0pt}%
\pgfpathmoveto{\pgfqpoint{6.062326in}{2.825334in}}%
\pgfpathlineto{\pgfqpoint{6.074549in}{2.810403in}}%
\pgfpathlineto{\pgfqpoint{6.086797in}{2.795521in}}%
\pgfpathlineto{\pgfqpoint{6.119797in}{2.797854in}}%
\pgfpathlineto{\pgfqpoint{6.152778in}{2.800206in}}%
\pgfpathlineto{\pgfqpoint{6.140476in}{2.815005in}}%
\pgfpathlineto{\pgfqpoint{6.128197in}{2.829851in}}%
\pgfpathlineto{\pgfqpoint{6.095271in}{2.827582in}}%
\pgfpathlineto{\pgfqpoint{6.062326in}{2.825334in}}%
\pgfpathclose%
\pgfusepath{fill}%
\end{pgfscope}%
\begin{pgfscope}%
\pgfpathrectangle{\pgfqpoint{1.020000in}{0.880000in}}{\pgfqpoint{6.160000in}{6.160000in}}%
\pgfusepath{clip}%
\pgfsetbuttcap%
\pgfsetroundjoin%
\definecolor{currentfill}{rgb}{0.299441,0.400248,0.839842}%
\pgfsetfillcolor{currentfill}%
\pgfsetlinewidth{0.000000pt}%
\definecolor{currentstroke}{rgb}{0.000000,0.000000,0.000000}%
\pgfsetstrokecolor{currentstroke}%
\pgfsetdash{}{0pt}%
\pgfpathmoveto{\pgfqpoint{5.840182in}{2.843076in}}%
\pgfpathlineto{\pgfqpoint{5.852194in}{2.827783in}}%
\pgfpathlineto{\pgfqpoint{5.864229in}{2.812537in}}%
\pgfpathlineto{\pgfqpoint{5.897295in}{2.814557in}}%
\pgfpathlineto{\pgfqpoint{5.930341in}{2.816632in}}%
\pgfpathlineto{\pgfqpoint{5.918251in}{2.831787in}}%
\pgfpathlineto{\pgfqpoint{5.906184in}{2.846989in}}%
\pgfpathlineto{\pgfqpoint{5.873193in}{2.845004in}}%
\pgfpathlineto{\pgfqpoint{5.840182in}{2.843076in}}%
\pgfpathclose%
\pgfusepath{fill}%
\end{pgfscope}%
\begin{pgfscope}%
\pgfpathrectangle{\pgfqpoint{1.020000in}{0.880000in}}{\pgfqpoint{6.160000in}{6.160000in}}%
\pgfusepath{clip}%
\pgfsetbuttcap%
\pgfsetroundjoin%
\definecolor{currentfill}{rgb}{0.309060,0.413498,0.850128}%
\pgfsetfillcolor{currentfill}%
\pgfsetlinewidth{0.000000pt}%
\definecolor{currentstroke}{rgb}{0.000000,0.000000,0.000000}%
\pgfsetstrokecolor{currentstroke}%
\pgfsetdash{}{0pt}%
\pgfpathmoveto{\pgfqpoint{5.618076in}{2.864694in}}%
\pgfpathlineto{\pgfqpoint{5.629879in}{2.849047in}}%
\pgfpathlineto{\pgfqpoint{5.641704in}{2.833429in}}%
\pgfpathlineto{\pgfqpoint{5.674832in}{2.834721in}}%
\pgfpathlineto{\pgfqpoint{5.707941in}{2.836168in}}%
\pgfpathlineto{\pgfqpoint{5.696061in}{2.851691in}}%
\pgfpathlineto{\pgfqpoint{5.684204in}{2.867250in}}%
\pgfpathlineto{\pgfqpoint{5.651150in}{2.865895in}}%
\pgfpathlineto{\pgfqpoint{5.618076in}{2.864694in}}%
\pgfpathclose%
\pgfusepath{fill}%
\end{pgfscope}%
\begin{pgfscope}%
\pgfpathrectangle{\pgfqpoint{1.020000in}{0.880000in}}{\pgfqpoint{6.160000in}{6.160000in}}%
\pgfusepath{clip}%
\pgfsetbuttcap%
\pgfsetroundjoin%
\definecolor{currentfill}{rgb}{0.559747,0.694768,0.996075}%
\pgfsetfillcolor{currentfill}%
\pgfsetlinewidth{0.000000pt}%
\definecolor{currentstroke}{rgb}{0.000000,0.000000,0.000000}%
\pgfsetstrokecolor{currentstroke}%
\pgfsetdash{}{0pt}%
\pgfpathmoveto{\pgfqpoint{4.398037in}{3.333710in}}%
\pgfpathlineto{\pgfqpoint{4.408744in}{3.331027in}}%
\pgfpathlineto{\pgfqpoint{4.419474in}{3.328352in}}%
\pgfpathlineto{\pgfqpoint{4.452944in}{3.304708in}}%
\pgfpathlineto{\pgfqpoint{4.486379in}{3.281913in}}%
\pgfpathlineto{\pgfqpoint{4.475598in}{3.286738in}}%
\pgfpathlineto{\pgfqpoint{4.464838in}{3.291552in}}%
\pgfpathlineto{\pgfqpoint{4.431454in}{3.312245in}}%
\pgfpathlineto{\pgfqpoint{4.398037in}{3.333710in}}%
\pgfpathclose%
\pgfusepath{fill}%
\end{pgfscope}%
\begin{pgfscope}%
\pgfpathrectangle{\pgfqpoint{1.020000in}{0.880000in}}{\pgfqpoint{6.160000in}{6.160000in}}%
\pgfusepath{clip}%
\pgfsetbuttcap%
\pgfsetroundjoin%
\definecolor{currentfill}{rgb}{0.758539,0.832787,0.958408}%
\pgfsetfillcolor{currentfill}%
\pgfsetlinewidth{0.000000pt}%
\definecolor{currentstroke}{rgb}{0.000000,0.000000,0.000000}%
\pgfsetstrokecolor{currentstroke}%
\pgfsetdash{}{0pt}%
\pgfpathmoveto{\pgfqpoint{3.555233in}{3.687296in}}%
\pgfpathlineto{\pgfqpoint{3.565146in}{3.670944in}}%
\pgfpathlineto{\pgfqpoint{3.575057in}{3.658196in}}%
\pgfpathlineto{\pgfqpoint{3.608873in}{3.655348in}}%
\pgfpathlineto{\pgfqpoint{3.642672in}{3.650926in}}%
\pgfpathlineto{\pgfqpoint{3.632716in}{3.661588in}}%
\pgfpathlineto{\pgfqpoint{3.622760in}{3.675734in}}%
\pgfpathlineto{\pgfqpoint{3.589005in}{3.682244in}}%
\pgfpathlineto{\pgfqpoint{3.555233in}{3.687296in}}%
\pgfpathclose%
\pgfusepath{fill}%
\end{pgfscope}%
\begin{pgfscope}%
\pgfpathrectangle{\pgfqpoint{1.020000in}{0.880000in}}{\pgfqpoint{6.160000in}{6.160000in}}%
\pgfusepath{clip}%
\pgfsetbuttcap%
\pgfsetroundjoin%
\definecolor{currentfill}{rgb}{0.661968,0.775491,0.993937}%
\pgfsetfillcolor{currentfill}%
\pgfsetlinewidth{0.000000pt}%
\definecolor{currentstroke}{rgb}{0.000000,0.000000,0.000000}%
\pgfsetstrokecolor{currentstroke}%
\pgfsetdash{}{0pt}%
\pgfpathmoveto{\pgfqpoint{4.087753in}{3.498680in}}%
\pgfpathlineto{\pgfqpoint{4.098134in}{3.501548in}}%
\pgfpathlineto{\pgfqpoint{4.108536in}{3.505581in}}%
\pgfpathlineto{\pgfqpoint{4.142179in}{3.485238in}}%
\pgfpathlineto{\pgfqpoint{4.175786in}{3.464432in}}%
\pgfpathlineto{\pgfqpoint{4.165320in}{3.462122in}}%
\pgfpathlineto{\pgfqpoint{4.154877in}{3.460824in}}%
\pgfpathlineto{\pgfqpoint{4.121332in}{3.479920in}}%
\pgfpathlineto{\pgfqpoint{4.087753in}{3.498680in}}%
\pgfpathclose%
\pgfusepath{fill}%
\end{pgfscope}%
\begin{pgfscope}%
\pgfpathrectangle{\pgfqpoint{1.020000in}{0.880000in}}{\pgfqpoint{6.160000in}{6.160000in}}%
\pgfusepath{clip}%
\pgfsetbuttcap%
\pgfsetroundjoin%
\definecolor{currentfill}{rgb}{0.809329,0.852974,0.922323}%
\pgfsetfillcolor{currentfill}%
\pgfsetlinewidth{0.000000pt}%
\definecolor{currentstroke}{rgb}{0.000000,0.000000,0.000000}%
\pgfsetstrokecolor{currentstroke}%
\pgfsetdash{}{0pt}%
\pgfpathmoveto{\pgfqpoint{3.177703in}{3.803537in}}%
\pgfpathlineto{\pgfqpoint{3.187516in}{3.761300in}}%
\pgfpathlineto{\pgfqpoint{3.197308in}{3.722727in}}%
\pgfpathlineto{\pgfqpoint{3.231156in}{3.730163in}}%
\pgfpathlineto{\pgfqpoint{3.264998in}{3.736276in}}%
\pgfpathlineto{\pgfqpoint{3.255177in}{3.773284in}}%
\pgfpathlineto{\pgfqpoint{3.245337in}{3.814099in}}%
\pgfpathlineto{\pgfqpoint{3.211523in}{3.809590in}}%
\pgfpathlineto{\pgfqpoint{3.177703in}{3.803537in}}%
\pgfpathclose%
\pgfusepath{fill}%
\end{pgfscope}%
\begin{pgfscope}%
\pgfpathrectangle{\pgfqpoint{1.020000in}{0.880000in}}{\pgfqpoint{6.160000in}{6.160000in}}%
\pgfusepath{clip}%
\pgfsetbuttcap%
\pgfsetroundjoin%
\definecolor{currentfill}{rgb}{0.373552,0.497499,0.909467}%
\pgfsetfillcolor{currentfill}%
\pgfsetlinewidth{0.000000pt}%
\definecolor{currentstroke}{rgb}{0.000000,0.000000,0.000000}%
\pgfsetstrokecolor{currentstroke}%
\pgfsetdash{}{0pt}%
\pgfpathmoveto{\pgfqpoint{5.018643in}{2.995318in}}%
\pgfpathlineto{\pgfqpoint{5.029901in}{2.980460in}}%
\pgfpathlineto{\pgfqpoint{5.041178in}{2.965351in}}%
\pgfpathlineto{\pgfqpoint{5.074428in}{2.957481in}}%
\pgfpathlineto{\pgfqpoint{5.107661in}{2.950702in}}%
\pgfpathlineto{\pgfqpoint{5.096333in}{2.966046in}}%
\pgfpathlineto{\pgfqpoint{5.085025in}{2.981172in}}%
\pgfpathlineto{\pgfqpoint{5.051843in}{2.987726in}}%
\pgfpathlineto{\pgfqpoint{5.018643in}{2.995318in}}%
\pgfpathclose%
\pgfusepath{fill}%
\end{pgfscope}%
\begin{pgfscope}%
\pgfpathrectangle{\pgfqpoint{1.020000in}{0.880000in}}{\pgfqpoint{6.160000in}{6.160000in}}%
\pgfusepath{clip}%
\pgfsetbuttcap%
\pgfsetroundjoin%
\definecolor{currentfill}{rgb}{0.613933,0.739923,0.999142}%
\pgfsetfillcolor{currentfill}%
\pgfsetlinewidth{0.000000pt}%
\definecolor{currentstroke}{rgb}{0.000000,0.000000,0.000000}%
\pgfsetstrokecolor{currentstroke}%
\pgfsetdash{}{0pt}%
\pgfpathmoveto{\pgfqpoint{4.242894in}{3.422178in}}%
\pgfpathlineto{\pgfqpoint{4.253443in}{3.423249in}}%
\pgfpathlineto{\pgfqpoint{4.264015in}{3.424803in}}%
\pgfpathlineto{\pgfqpoint{4.297575in}{3.401543in}}%
\pgfpathlineto{\pgfqpoint{4.331098in}{3.378499in}}%
\pgfpathlineto{\pgfqpoint{4.320468in}{3.379158in}}%
\pgfpathlineto{\pgfqpoint{4.309861in}{3.380216in}}%
\pgfpathlineto{\pgfqpoint{4.276395in}{3.401080in}}%
\pgfpathlineto{\pgfqpoint{4.242894in}{3.422178in}}%
\pgfpathclose%
\pgfusepath{fill}%
\end{pgfscope}%
\begin{pgfscope}%
\pgfpathrectangle{\pgfqpoint{1.020000in}{0.880000in}}{\pgfqpoint{6.160000in}{6.160000in}}%
\pgfusepath{clip}%
\pgfsetbuttcap%
\pgfsetroundjoin%
\definecolor{currentfill}{rgb}{0.323718,0.433158,0.864722}%
\pgfsetfillcolor{currentfill}%
\pgfsetlinewidth{0.000000pt}%
\definecolor{currentstroke}{rgb}{0.000000,0.000000,0.000000}%
\pgfsetstrokecolor{currentstroke}%
\pgfsetdash{}{0pt}%
\pgfpathmoveto{\pgfqpoint{5.396032in}{2.894295in}}%
\pgfpathlineto{\pgfqpoint{5.407629in}{2.878430in}}%
\pgfpathlineto{\pgfqpoint{5.419247in}{2.862536in}}%
\pgfpathlineto{\pgfqpoint{5.452430in}{2.862096in}}%
\pgfpathlineto{\pgfqpoint{5.485596in}{2.862037in}}%
\pgfpathlineto{\pgfqpoint{5.473924in}{2.877860in}}%
\pgfpathlineto{\pgfqpoint{5.462273in}{2.893674in}}%
\pgfpathlineto{\pgfqpoint{5.429161in}{2.893796in}}%
\pgfpathlineto{\pgfqpoint{5.396032in}{2.894295in}}%
\pgfpathclose%
\pgfusepath{fill}%
\end{pgfscope}%
\begin{pgfscope}%
\pgfpathrectangle{\pgfqpoint{1.020000in}{0.880000in}}{\pgfqpoint{6.160000in}{6.160000in}}%
\pgfusepath{clip}%
\pgfsetbuttcap%
\pgfsetroundjoin%
\definecolor{currentfill}{rgb}{0.826784,0.858205,0.906953}%
\pgfsetfillcolor{currentfill}%
\pgfsetlinewidth{0.000000pt}%
\definecolor{currentstroke}{rgb}{0.000000,0.000000,0.000000}%
\pgfsetstrokecolor{currentstroke}%
\pgfsetdash{}{0pt}%
\pgfpathmoveto{\pgfqpoint{2.733105in}{3.836665in}}%
\pgfpathlineto{\pgfqpoint{2.742712in}{3.787693in}}%
\pgfpathlineto{\pgfqpoint{2.752306in}{3.740578in}}%
\pgfpathlineto{\pgfqpoint{2.786146in}{3.757343in}}%
\pgfpathlineto{\pgfqpoint{2.819976in}{3.774008in}}%
\pgfpathlineto{\pgfqpoint{2.810317in}{3.823146in}}%
\pgfpathlineto{\pgfqpoint{2.800642in}{3.874398in}}%
\pgfpathlineto{\pgfqpoint{2.766877in}{3.855644in}}%
\pgfpathlineto{\pgfqpoint{2.733105in}{3.836665in}}%
\pgfpathclose%
\pgfusepath{fill}%
\end{pgfscope}%
\begin{pgfscope}%
\pgfpathrectangle{\pgfqpoint{1.020000in}{0.880000in}}{\pgfqpoint{6.160000in}{6.160000in}}%
\pgfusepath{clip}%
\pgfsetbuttcap%
\pgfsetroundjoin%
\definecolor{currentfill}{rgb}{0.800601,0.850358,0.930008}%
\pgfsetfillcolor{currentfill}%
\pgfsetlinewidth{0.000000pt}%
\definecolor{currentstroke}{rgb}{0.000000,0.000000,0.000000}%
\pgfsetstrokecolor{currentstroke}%
\pgfsetdash{}{0pt}%
\pgfpathmoveto{\pgfqpoint{2.443763in}{3.777080in}}%
\pgfpathlineto{\pgfqpoint{2.453063in}{3.736399in}}%
\pgfpathlineto{\pgfqpoint{2.462361in}{3.696483in}}%
\pgfpathlineto{\pgfqpoint{2.496282in}{3.711619in}}%
\pgfpathlineto{\pgfqpoint{2.530175in}{3.727651in}}%
\pgfpathlineto{\pgfqpoint{2.520793in}{3.770147in}}%
\pgfpathlineto{\pgfqpoint{2.511408in}{3.813578in}}%
\pgfpathlineto{\pgfqpoint{2.477599in}{3.794834in}}%
\pgfpathlineto{\pgfqpoint{2.443763in}{3.777080in}}%
\pgfpathclose%
\pgfusepath{fill}%
\end{pgfscope}%
\begin{pgfscope}%
\pgfpathrectangle{\pgfqpoint{1.020000in}{0.880000in}}{\pgfqpoint{6.160000in}{6.160000in}}%
\pgfusepath{clip}%
\pgfsetbuttcap%
\pgfsetroundjoin%
\definecolor{currentfill}{rgb}{0.822420,0.856898,0.910795}%
\pgfsetfillcolor{currentfill}%
\pgfsetlinewidth{0.000000pt}%
\definecolor{currentstroke}{rgb}{0.000000,0.000000,0.000000}%
\pgfsetstrokecolor{currentstroke}%
\pgfsetdash{}{0pt}%
\pgfpathmoveto{\pgfqpoint{2.955223in}{3.836335in}}%
\pgfpathlineto{\pgfqpoint{2.964971in}{3.786947in}}%
\pgfpathlineto{\pgfqpoint{2.974697in}{3.740441in}}%
\pgfpathlineto{\pgfqpoint{3.008543in}{3.753531in}}%
\pgfpathlineto{\pgfqpoint{3.042383in}{3.765772in}}%
\pgfpathlineto{\pgfqpoint{3.032615in}{3.812587in}}%
\pgfpathlineto{\pgfqpoint{3.022825in}{3.862532in}}%
\pgfpathlineto{\pgfqpoint{2.989025in}{3.849985in}}%
\pgfpathlineto{\pgfqpoint{2.955223in}{3.836335in}}%
\pgfpathclose%
\pgfusepath{fill}%
\end{pgfscope}%
\begin{pgfscope}%
\pgfpathrectangle{\pgfqpoint{1.020000in}{0.880000in}}{\pgfqpoint{6.160000in}{6.160000in}}%
\pgfusepath{clip}%
\pgfsetbuttcap%
\pgfsetroundjoin%
\definecolor{currentfill}{rgb}{0.786721,0.844807,0.939810}%
\pgfsetfillcolor{currentfill}%
\pgfsetlinewidth{0.000000pt}%
\definecolor{currentstroke}{rgb}{0.000000,0.000000,0.000000}%
\pgfsetstrokecolor{currentstroke}%
\pgfsetdash{}{0pt}%
\pgfpathmoveto{\pgfqpoint{3.332660in}{3.744230in}}%
\pgfpathlineto{\pgfqpoint{3.342491in}{3.713077in}}%
\pgfpathlineto{\pgfqpoint{3.352308in}{3.685784in}}%
\pgfpathlineto{\pgfqpoint{3.386156in}{3.689783in}}%
\pgfpathlineto{\pgfqpoint{3.419995in}{3.692322in}}%
\pgfpathlineto{\pgfqpoint{3.410147in}{3.717277in}}%
\pgfpathlineto{\pgfqpoint{3.400287in}{3.746129in}}%
\pgfpathlineto{\pgfqpoint{3.366478in}{3.745955in}}%
\pgfpathlineto{\pgfqpoint{3.332660in}{3.744230in}}%
\pgfpathclose%
\pgfusepath{fill}%
\end{pgfscope}%
\begin{pgfscope}%
\pgfpathrectangle{\pgfqpoint{1.020000in}{0.880000in}}{\pgfqpoint{6.160000in}{6.160000in}}%
\pgfusepath{clip}%
\pgfsetbuttcap%
\pgfsetroundjoin%
\definecolor{currentfill}{rgb}{0.348323,0.465711,0.888346}%
\pgfsetfillcolor{currentfill}%
\pgfsetlinewidth{0.000000pt}%
\definecolor{currentstroke}{rgb}{0.000000,0.000000,0.000000}%
\pgfsetstrokecolor{currentstroke}%
\pgfsetdash{}{0pt}%
\pgfpathmoveto{\pgfqpoint{5.174074in}{2.940070in}}%
\pgfpathlineto{\pgfqpoint{5.185472in}{2.924411in}}%
\pgfpathlineto{\pgfqpoint{5.196890in}{2.908607in}}%
\pgfpathlineto{\pgfqpoint{5.230123in}{2.904528in}}%
\pgfpathlineto{\pgfqpoint{5.263339in}{2.901227in}}%
\pgfpathlineto{\pgfqpoint{5.251868in}{2.917065in}}%
\pgfpathlineto{\pgfqpoint{5.240418in}{2.932789in}}%
\pgfpathlineto{\pgfqpoint{5.207255in}{2.936051in}}%
\pgfpathlineto{\pgfqpoint{5.174074in}{2.940070in}}%
\pgfpathclose%
\pgfusepath{fill}%
\end{pgfscope}%
\begin{pgfscope}%
\pgfpathrectangle{\pgfqpoint{1.020000in}{0.880000in}}{\pgfqpoint{6.160000in}{6.160000in}}%
\pgfusepath{clip}%
\pgfsetbuttcap%
\pgfsetroundjoin%
\definecolor{currentfill}{rgb}{0.738826,0.822572,0.968261}%
\pgfsetfillcolor{currentfill}%
\pgfsetlinewidth{0.000000pt}%
\definecolor{currentstroke}{rgb}{0.000000,0.000000,0.000000}%
\pgfsetstrokecolor{currentstroke}%
\pgfsetdash{}{0pt}%
\pgfpathmoveto{\pgfqpoint{3.710216in}{3.637522in}}%
\pgfpathlineto{\pgfqpoint{3.720227in}{3.631827in}}%
\pgfpathlineto{\pgfqpoint{3.730246in}{3.629122in}}%
\pgfpathlineto{\pgfqpoint{3.764042in}{3.621521in}}%
\pgfpathlineto{\pgfqpoint{3.797817in}{3.612374in}}%
\pgfpathlineto{\pgfqpoint{3.787741in}{3.614021in}}%
\pgfpathlineto{\pgfqpoint{3.777675in}{3.618475in}}%
\pgfpathlineto{\pgfqpoint{3.743957in}{3.628663in}}%
\pgfpathlineto{\pgfqpoint{3.710216in}{3.637522in}}%
\pgfpathclose%
\pgfusepath{fill}%
\end{pgfscope}%
\begin{pgfscope}%
\pgfpathrectangle{\pgfqpoint{1.020000in}{0.880000in}}{\pgfqpoint{6.160000in}{6.160000in}}%
\pgfusepath{clip}%
\pgfsetbuttcap%
\pgfsetroundjoin%
\definecolor{currentfill}{rgb}{0.809329,0.852974,0.922323}%
\pgfsetfillcolor{currentfill}%
\pgfsetlinewidth{0.000000pt}%
\definecolor{currentstroke}{rgb}{0.000000,0.000000,0.000000}%
\pgfsetstrokecolor{currentstroke}%
\pgfsetdash{}{0pt}%
\pgfpathmoveto{\pgfqpoint{2.665528in}{3.798761in}}%
\pgfpathlineto{\pgfqpoint{2.675062in}{3.752259in}}%
\pgfpathlineto{\pgfqpoint{2.684586in}{3.707361in}}%
\pgfpathlineto{\pgfqpoint{2.718454in}{3.723868in}}%
\pgfpathlineto{\pgfqpoint{2.752306in}{3.740578in}}%
\pgfpathlineto{\pgfqpoint{2.742712in}{3.787693in}}%
\pgfpathlineto{\pgfqpoint{2.733105in}{3.836665in}}%
\pgfpathlineto{\pgfqpoint{2.699322in}{3.817645in}}%
\pgfpathlineto{\pgfqpoint{2.665528in}{3.798761in}}%
\pgfpathclose%
\pgfusepath{fill}%
\end{pgfscope}%
\begin{pgfscope}%
\pgfpathrectangle{\pgfqpoint{1.020000in}{0.880000in}}{\pgfqpoint{6.160000in}{6.160000in}}%
\pgfusepath{clip}%
\pgfsetbuttcap%
\pgfsetroundjoin%
\definecolor{currentfill}{rgb}{0.289996,0.386836,0.828926}%
\pgfsetfillcolor{currentfill}%
\pgfsetlinewidth{0.000000pt}%
\definecolor{currentstroke}{rgb}{0.000000,0.000000,0.000000}%
\pgfsetstrokecolor{currentstroke}%
\pgfsetdash{}{0pt}%
\pgfpathmoveto{\pgfqpoint{5.996373in}{2.820915in}}%
\pgfpathlineto{\pgfqpoint{6.008542in}{2.805898in}}%
\pgfpathlineto{\pgfqpoint{6.020735in}{2.790930in}}%
\pgfpathlineto{\pgfqpoint{6.053776in}{2.793212in}}%
\pgfpathlineto{\pgfqpoint{6.086797in}{2.795521in}}%
\pgfpathlineto{\pgfqpoint{6.074549in}{2.810403in}}%
\pgfpathlineto{\pgfqpoint{6.062326in}{2.825334in}}%
\pgfpathlineto{\pgfqpoint{6.029360in}{2.823110in}}%
\pgfpathlineto{\pgfqpoint{5.996373in}{2.820915in}}%
\pgfpathclose%
\pgfusepath{fill}%
\end{pgfscope}%
\begin{pgfscope}%
\pgfpathrectangle{\pgfqpoint{1.020000in}{0.880000in}}{\pgfqpoint{6.160000in}{6.160000in}}%
\pgfusepath{clip}%
\pgfsetbuttcap%
\pgfsetroundjoin%
\definecolor{currentfill}{rgb}{0.804965,0.851666,0.926165}%
\pgfsetfillcolor{currentfill}%
\pgfsetlinewidth{0.000000pt}%
\definecolor{currentstroke}{rgb}{0.000000,0.000000,0.000000}%
\pgfsetstrokecolor{currentstroke}%
\pgfsetdash{}{0pt}%
\pgfpathmoveto{\pgfqpoint{3.110051in}{3.787147in}}%
\pgfpathlineto{\pgfqpoint{3.119833in}{3.743975in}}%
\pgfpathlineto{\pgfqpoint{3.129593in}{3.704278in}}%
\pgfpathlineto{\pgfqpoint{3.163453in}{3.714064in}}%
\pgfpathlineto{\pgfqpoint{3.197308in}{3.722727in}}%
\pgfpathlineto{\pgfqpoint{3.187516in}{3.761300in}}%
\pgfpathlineto{\pgfqpoint{3.177703in}{3.803537in}}%
\pgfpathlineto{\pgfqpoint{3.143879in}{3.796023in}}%
\pgfpathlineto{\pgfqpoint{3.110051in}{3.787147in}}%
\pgfpathclose%
\pgfusepath{fill}%
\end{pgfscope}%
\begin{pgfscope}%
\pgfpathrectangle{\pgfqpoint{1.020000in}{0.880000in}}{\pgfqpoint{6.160000in}{6.160000in}}%
\pgfusepath{clip}%
\pgfsetbuttcap%
\pgfsetroundjoin%
\definecolor{currentfill}{rgb}{0.299441,0.400248,0.839842}%
\pgfsetfillcolor{currentfill}%
\pgfsetlinewidth{0.000000pt}%
\definecolor{currentstroke}{rgb}{0.000000,0.000000,0.000000}%
\pgfsetstrokecolor{currentstroke}%
\pgfsetdash{}{0pt}%
\pgfpathmoveto{\pgfqpoint{5.774100in}{2.839433in}}%
\pgfpathlineto{\pgfqpoint{5.786058in}{2.824046in}}%
\pgfpathlineto{\pgfqpoint{5.798038in}{2.808705in}}%
\pgfpathlineto{\pgfqpoint{5.831144in}{2.810582in}}%
\pgfpathlineto{\pgfqpoint{5.864229in}{2.812537in}}%
\pgfpathlineto{\pgfqpoint{5.852194in}{2.827783in}}%
\pgfpathlineto{\pgfqpoint{5.840182in}{2.843076in}}%
\pgfpathlineto{\pgfqpoint{5.807151in}{2.841215in}}%
\pgfpathlineto{\pgfqpoint{5.774100in}{2.839433in}}%
\pgfpathclose%
\pgfusepath{fill}%
\end{pgfscope}%
\begin{pgfscope}%
\pgfpathrectangle{\pgfqpoint{1.020000in}{0.880000in}}{\pgfqpoint{6.160000in}{6.160000in}}%
\pgfusepath{clip}%
\pgfsetbuttcap%
\pgfsetroundjoin%
\definecolor{currentfill}{rgb}{0.813693,0.854282,0.918480}%
\pgfsetfillcolor{currentfill}%
\pgfsetlinewidth{0.000000pt}%
\definecolor{currentstroke}{rgb}{0.000000,0.000000,0.000000}%
\pgfsetstrokecolor{currentstroke}%
\pgfsetdash{}{0pt}%
\pgfpathmoveto{\pgfqpoint{2.887610in}{3.806377in}}%
\pgfpathlineto{\pgfqpoint{2.897308in}{3.758033in}}%
\pgfpathlineto{\pgfqpoint{2.906986in}{3.712301in}}%
\pgfpathlineto{\pgfqpoint{2.940845in}{3.726648in}}%
\pgfpathlineto{\pgfqpoint{2.974697in}{3.740441in}}%
\pgfpathlineto{\pgfqpoint{2.964971in}{3.786947in}}%
\pgfpathlineto{\pgfqpoint{2.955223in}{3.836335in}}%
\pgfpathlineto{\pgfqpoint{2.921419in}{3.821743in}}%
\pgfpathlineto{\pgfqpoint{2.887610in}{3.806377in}}%
\pgfpathclose%
\pgfusepath{fill}%
\end{pgfscope}%
\begin{pgfscope}%
\pgfpathrectangle{\pgfqpoint{1.020000in}{0.880000in}}{\pgfqpoint{6.160000in}{6.160000in}}%
\pgfusepath{clip}%
\pgfsetbuttcap%
\pgfsetroundjoin%
\definecolor{currentfill}{rgb}{0.309060,0.413498,0.850128}%
\pgfsetfillcolor{currentfill}%
\pgfsetlinewidth{0.000000pt}%
\definecolor{currentstroke}{rgb}{0.000000,0.000000,0.000000}%
\pgfsetstrokecolor{currentstroke}%
\pgfsetdash{}{0pt}%
\pgfpathmoveto{\pgfqpoint{5.551873in}{2.862868in}}%
\pgfpathlineto{\pgfqpoint{5.563621in}{2.847133in}}%
\pgfpathlineto{\pgfqpoint{5.575392in}{2.831418in}}%
\pgfpathlineto{\pgfqpoint{5.608557in}{2.832318in}}%
\pgfpathlineto{\pgfqpoint{5.641704in}{2.833429in}}%
\pgfpathlineto{\pgfqpoint{5.629879in}{2.849047in}}%
\pgfpathlineto{\pgfqpoint{5.618076in}{2.864694in}}%
\pgfpathlineto{\pgfqpoint{5.584984in}{2.863675in}}%
\pgfpathlineto{\pgfqpoint{5.551873in}{2.862868in}}%
\pgfpathclose%
\pgfusepath{fill}%
\end{pgfscope}%
\begin{pgfscope}%
\pgfpathrectangle{\pgfqpoint{1.020000in}{0.880000in}}{\pgfqpoint{6.160000in}{6.160000in}}%
\pgfusepath{clip}%
\pgfsetbuttcap%
\pgfsetroundjoin%
\definecolor{currentfill}{rgb}{0.768034,0.837035,0.952488}%
\pgfsetfillcolor{currentfill}%
\pgfsetlinewidth{0.000000pt}%
\definecolor{currentstroke}{rgb}{0.000000,0.000000,0.000000}%
\pgfsetstrokecolor{currentstroke}%
\pgfsetdash{}{0pt}%
\pgfpathmoveto{\pgfqpoint{3.487641in}{3.692871in}}%
\pgfpathlineto{\pgfqpoint{3.497515in}{3.674154in}}%
\pgfpathlineto{\pgfqpoint{3.507383in}{3.659111in}}%
\pgfpathlineto{\pgfqpoint{3.541227in}{3.659450in}}%
\pgfpathlineto{\pgfqpoint{3.575057in}{3.658196in}}%
\pgfpathlineto{\pgfqpoint{3.565146in}{3.670944in}}%
\pgfpathlineto{\pgfqpoint{3.555233in}{3.687296in}}%
\pgfpathlineto{\pgfqpoint{3.521444in}{3.690846in}}%
\pgfpathlineto{\pgfqpoint{3.487641in}{3.692871in}}%
\pgfpathclose%
\pgfusepath{fill}%
\end{pgfscope}%
\begin{pgfscope}%
\pgfpathrectangle{\pgfqpoint{1.020000in}{0.880000in}}{\pgfqpoint{6.160000in}{6.160000in}}%
\pgfusepath{clip}%
\pgfsetbuttcap%
\pgfsetroundjoin%
\definecolor{currentfill}{rgb}{0.796064,0.848693,0.933471}%
\pgfsetfillcolor{currentfill}%
\pgfsetlinewidth{0.000000pt}%
\definecolor{currentstroke}{rgb}{0.000000,0.000000,0.000000}%
\pgfsetstrokecolor{currentstroke}%
\pgfsetdash{}{0pt}%
\pgfpathmoveto{\pgfqpoint{2.597891in}{3.762041in}}%
\pgfpathlineto{\pgfqpoint{2.607349in}{3.718082in}}%
\pgfpathlineto{\pgfqpoint{2.616798in}{3.675489in}}%
\pgfpathlineto{\pgfqpoint{2.650702in}{3.691193in}}%
\pgfpathlineto{\pgfqpoint{2.684586in}{3.707361in}}%
\pgfpathlineto{\pgfqpoint{2.675062in}{3.752259in}}%
\pgfpathlineto{\pgfqpoint{2.665528in}{3.798761in}}%
\pgfpathlineto{\pgfqpoint{2.631718in}{3.780176in}}%
\pgfpathlineto{\pgfqpoint{2.597891in}{3.762041in}}%
\pgfpathclose%
\pgfusepath{fill}%
\end{pgfscope}%
\begin{pgfscope}%
\pgfpathrectangle{\pgfqpoint{1.020000in}{0.880000in}}{\pgfqpoint{6.160000in}{6.160000in}}%
\pgfusepath{clip}%
\pgfsetbuttcap%
\pgfsetroundjoin%
\definecolor{currentfill}{rgb}{0.718985,0.811993,0.977656}%
\pgfsetfillcolor{currentfill}%
\pgfsetlinewidth{0.000000pt}%
\definecolor{currentstroke}{rgb}{0.000000,0.000000,0.000000}%
\pgfsetstrokecolor{currentstroke}%
\pgfsetdash{}{0pt}%
\pgfpathmoveto{\pgfqpoint{3.865296in}{3.589826in}}%
\pgfpathlineto{\pgfqpoint{3.875446in}{3.591186in}}%
\pgfpathlineto{\pgfqpoint{3.885613in}{3.594725in}}%
\pgfpathlineto{\pgfqpoint{3.919377in}{3.581507in}}%
\pgfpathlineto{\pgfqpoint{3.953113in}{3.566963in}}%
\pgfpathlineto{\pgfqpoint{3.942882in}{3.563684in}}%
\pgfpathlineto{\pgfqpoint{3.932668in}{3.562388in}}%
\pgfpathlineto{\pgfqpoint{3.898996in}{3.576654in}}%
\pgfpathlineto{\pgfqpoint{3.865296in}{3.589826in}}%
\pgfpathclose%
\pgfusepath{fill}%
\end{pgfscope}%
\begin{pgfscope}%
\pgfpathrectangle{\pgfqpoint{1.020000in}{0.880000in}}{\pgfqpoint{6.160000in}{6.160000in}}%
\pgfusepath{clip}%
\pgfsetbuttcap%
\pgfsetroundjoin%
\definecolor{currentfill}{rgb}{0.328604,0.439712,0.869587}%
\pgfsetfillcolor{currentfill}%
\pgfsetlinewidth{0.000000pt}%
\definecolor{currentstroke}{rgb}{0.000000,0.000000,0.000000}%
\pgfsetstrokecolor{currentstroke}%
\pgfsetdash{}{0pt}%
\pgfpathmoveto{\pgfqpoint{5.329720in}{2.896648in}}%
\pgfpathlineto{\pgfqpoint{5.341263in}{2.880742in}}%
\pgfpathlineto{\pgfqpoint{5.352828in}{2.864783in}}%
\pgfpathlineto{\pgfqpoint{5.386046in}{2.863411in}}%
\pgfpathlineto{\pgfqpoint{5.419247in}{2.862536in}}%
\pgfpathlineto{\pgfqpoint{5.407629in}{2.878430in}}%
\pgfpathlineto{\pgfqpoint{5.396032in}{2.894295in}}%
\pgfpathlineto{\pgfqpoint{5.362884in}{2.895227in}}%
\pgfpathlineto{\pgfqpoint{5.329720in}{2.896648in}}%
\pgfpathclose%
\pgfusepath{fill}%
\end{pgfscope}%
\begin{pgfscope}%
\pgfpathrectangle{\pgfqpoint{1.020000in}{0.880000in}}{\pgfqpoint{6.160000in}{6.160000in}}%
\pgfusepath{clip}%
\pgfsetbuttcap%
\pgfsetroundjoin%
\definecolor{currentfill}{rgb}{0.430507,0.564883,0.948889}%
\pgfsetfillcolor{currentfill}%
\pgfsetlinewidth{0.000000pt}%
\definecolor{currentstroke}{rgb}{0.000000,0.000000,0.000000}%
\pgfsetstrokecolor{currentstroke}%
\pgfsetdash{}{0pt}%
\pgfpathmoveto{\pgfqpoint{4.796866in}{3.092218in}}%
\pgfpathlineto{\pgfqpoint{4.807940in}{3.079799in}}%
\pgfpathlineto{\pgfqpoint{4.819034in}{3.066991in}}%
\pgfpathlineto{\pgfqpoint{4.852355in}{3.051586in}}%
\pgfpathlineto{\pgfqpoint{4.885653in}{3.037648in}}%
\pgfpathlineto{\pgfqpoint{4.874512in}{3.051230in}}%
\pgfpathlineto{\pgfqpoint{4.863390in}{3.064481in}}%
\pgfpathlineto{\pgfqpoint{4.830139in}{3.077667in}}%
\pgfpathlineto{\pgfqpoint{4.796866in}{3.092218in}}%
\pgfpathclose%
\pgfusepath{fill}%
\end{pgfscope}%
\begin{pgfscope}%
\pgfpathrectangle{\pgfqpoint{1.020000in}{0.880000in}}{\pgfqpoint{6.160000in}{6.160000in}}%
\pgfusepath{clip}%
\pgfsetbuttcap%
\pgfsetroundjoin%
\definecolor{currentfill}{rgb}{0.478462,0.616564,0.972721}%
\pgfsetfillcolor{currentfill}%
\pgfsetlinewidth{0.000000pt}%
\definecolor{currentstroke}{rgb}{0.000000,0.000000,0.000000}%
\pgfsetstrokecolor{currentstroke}%
\pgfsetdash{}{0pt}%
\pgfpathmoveto{\pgfqpoint{4.641621in}{3.183534in}}%
\pgfpathlineto{\pgfqpoint{4.652562in}{3.174331in}}%
\pgfpathlineto{\pgfqpoint{4.663524in}{3.164724in}}%
\pgfpathlineto{\pgfqpoint{4.696899in}{3.144421in}}%
\pgfpathlineto{\pgfqpoint{4.730247in}{3.125571in}}%
\pgfpathlineto{\pgfqpoint{4.719239in}{3.136530in}}%
\pgfpathlineto{\pgfqpoint{4.708250in}{3.147139in}}%
\pgfpathlineto{\pgfqpoint{4.674949in}{3.164672in}}%
\pgfpathlineto{\pgfqpoint{4.641621in}{3.183534in}}%
\pgfpathclose%
\pgfusepath{fill}%
\end{pgfscope}%
\begin{pgfscope}%
\pgfpathrectangle{\pgfqpoint{1.020000in}{0.880000in}}{\pgfqpoint{6.160000in}{6.160000in}}%
\pgfusepath{clip}%
\pgfsetbuttcap%
\pgfsetroundjoin%
\definecolor{currentfill}{rgb}{0.388852,0.516298,0.921373}%
\pgfsetfillcolor{currentfill}%
\pgfsetlinewidth{0.000000pt}%
\definecolor{currentstroke}{rgb}{0.000000,0.000000,0.000000}%
\pgfsetstrokecolor{currentstroke}%
\pgfsetdash{}{0pt}%
\pgfpathmoveto{\pgfqpoint{4.952188in}{3.013938in}}%
\pgfpathlineto{\pgfqpoint{4.963396in}{2.999470in}}%
\pgfpathlineto{\pgfqpoint{4.974624in}{2.984702in}}%
\pgfpathlineto{\pgfqpoint{5.007910in}{2.974396in}}%
\pgfpathlineto{\pgfqpoint{5.041178in}{2.965351in}}%
\pgfpathlineto{\pgfqpoint{5.029901in}{2.980460in}}%
\pgfpathlineto{\pgfqpoint{5.018643in}{2.995318in}}%
\pgfpathlineto{\pgfqpoint{4.985425in}{3.004029in}}%
\pgfpathlineto{\pgfqpoint{4.952188in}{3.013938in}}%
\pgfpathclose%
\pgfusepath{fill}%
\end{pgfscope}%
\begin{pgfscope}%
\pgfpathrectangle{\pgfqpoint{1.020000in}{0.880000in}}{\pgfqpoint{6.160000in}{6.160000in}}%
\pgfusepath{clip}%
\pgfsetbuttcap%
\pgfsetroundjoin%
\definecolor{currentfill}{rgb}{0.786721,0.844807,0.939810}%
\pgfsetfillcolor{currentfill}%
\pgfsetlinewidth{0.000000pt}%
\definecolor{currentstroke}{rgb}{0.000000,0.000000,0.000000}%
\pgfsetstrokecolor{currentstroke}%
\pgfsetdash{}{0pt}%
\pgfpathmoveto{\pgfqpoint{3.264998in}{3.736276in}}%
\pgfpathlineto{\pgfqpoint{3.274800in}{3.703078in}}%
\pgfpathlineto{\pgfqpoint{3.284585in}{3.673649in}}%
\pgfpathlineto{\pgfqpoint{3.318450in}{3.680382in}}%
\pgfpathlineto{\pgfqpoint{3.352308in}{3.685784in}}%
\pgfpathlineto{\pgfqpoint{3.342491in}{3.713077in}}%
\pgfpathlineto{\pgfqpoint{3.332660in}{3.744230in}}%
\pgfpathlineto{\pgfqpoint{3.298833in}{3.740987in}}%
\pgfpathlineto{\pgfqpoint{3.264998in}{3.736276in}}%
\pgfpathclose%
\pgfusepath{fill}%
\end{pgfscope}%
\begin{pgfscope}%
\pgfpathrectangle{\pgfqpoint{1.020000in}{0.880000in}}{\pgfqpoint{6.160000in}{6.160000in}}%
\pgfusepath{clip}%
\pgfsetbuttcap%
\pgfsetroundjoin%
\definecolor{currentfill}{rgb}{0.538004,0.674902,0.991722}%
\pgfsetfillcolor{currentfill}%
\pgfsetlinewidth{0.000000pt}%
\definecolor{currentstroke}{rgb}{0.000000,0.000000,0.000000}%
\pgfsetstrokecolor{currentstroke}%
\pgfsetdash{}{0pt}%
\pgfpathmoveto{\pgfqpoint{4.486379in}{3.281913in}}%
\pgfpathlineto{\pgfqpoint{4.497182in}{3.276941in}}%
\pgfpathlineto{\pgfqpoint{4.508006in}{3.271683in}}%
\pgfpathlineto{\pgfqpoint{4.541458in}{3.247852in}}%
\pgfpathlineto{\pgfqpoint{4.574877in}{3.225169in}}%
\pgfpathlineto{\pgfqpoint{4.564003in}{3.232353in}}%
\pgfpathlineto{\pgfqpoint{4.553151in}{3.239283in}}%
\pgfpathlineto{\pgfqpoint{4.519781in}{3.260076in}}%
\pgfpathlineto{\pgfqpoint{4.486379in}{3.281913in}}%
\pgfpathclose%
\pgfusepath{fill}%
\end{pgfscope}%
\begin{pgfscope}%
\pgfpathrectangle{\pgfqpoint{1.020000in}{0.880000in}}{\pgfqpoint{6.160000in}{6.160000in}}%
\pgfusepath{clip}%
\pgfsetbuttcap%
\pgfsetroundjoin%
\definecolor{currentfill}{rgb}{0.800601,0.850358,0.930008}%
\pgfsetfillcolor{currentfill}%
\pgfsetlinewidth{0.000000pt}%
\definecolor{currentstroke}{rgb}{0.000000,0.000000,0.000000}%
\pgfsetstrokecolor{currentstroke}%
\pgfsetdash{}{0pt}%
\pgfpathmoveto{\pgfqpoint{2.819976in}{3.774008in}}%
\pgfpathlineto{\pgfqpoint{2.829617in}{3.727110in}}%
\pgfpathlineto{\pgfqpoint{2.839241in}{3.682545in}}%
\pgfpathlineto{\pgfqpoint{2.873118in}{3.697551in}}%
\pgfpathlineto{\pgfqpoint{2.906986in}{3.712301in}}%
\pgfpathlineto{\pgfqpoint{2.897308in}{3.758033in}}%
\pgfpathlineto{\pgfqpoint{2.887610in}{3.806377in}}%
\pgfpathlineto{\pgfqpoint{2.853796in}{3.790409in}}%
\pgfpathlineto{\pgfqpoint{2.819976in}{3.774008in}}%
\pgfpathclose%
\pgfusepath{fill}%
\end{pgfscope}%
\begin{pgfscope}%
\pgfpathrectangle{\pgfqpoint{1.020000in}{0.880000in}}{\pgfqpoint{6.160000in}{6.160000in}}%
\pgfusepath{clip}%
\pgfsetbuttcap%
\pgfsetroundjoin%
\definecolor{currentfill}{rgb}{0.796064,0.848693,0.933471}%
\pgfsetfillcolor{currentfill}%
\pgfsetlinewidth{0.000000pt}%
\definecolor{currentstroke}{rgb}{0.000000,0.000000,0.000000}%
\pgfsetstrokecolor{currentstroke}%
\pgfsetdash{}{0pt}%
\pgfpathmoveto{\pgfqpoint{3.042383in}{3.765772in}}%
\pgfpathlineto{\pgfqpoint{3.052129in}{3.722166in}}%
\pgfpathlineto{\pgfqpoint{3.061853in}{3.681809in}}%
\pgfpathlineto{\pgfqpoint{3.095726in}{3.693485in}}%
\pgfpathlineto{\pgfqpoint{3.129593in}{3.704278in}}%
\pgfpathlineto{\pgfqpoint{3.119833in}{3.743975in}}%
\pgfpathlineto{\pgfqpoint{3.110051in}{3.787147in}}%
\pgfpathlineto{\pgfqpoint{3.076219in}{3.777022in}}%
\pgfpathlineto{\pgfqpoint{3.042383in}{3.765772in}}%
\pgfpathclose%
\pgfusepath{fill}%
\end{pgfscope}%
\begin{pgfscope}%
\pgfpathrectangle{\pgfqpoint{1.020000in}{0.880000in}}{\pgfqpoint{6.160000in}{6.160000in}}%
\pgfusepath{clip}%
\pgfsetbuttcap%
\pgfsetroundjoin%
\definecolor{currentfill}{rgb}{0.688188,0.793178,0.988038}%
\pgfsetfillcolor{currentfill}%
\pgfsetlinewidth{0.000000pt}%
\definecolor{currentstroke}{rgb}{0.000000,0.000000,0.000000}%
\pgfsetstrokecolor{currentstroke}%
\pgfsetdash{}{0pt}%
\pgfpathmoveto{\pgfqpoint{4.020497in}{3.534512in}}%
\pgfpathlineto{\pgfqpoint{4.030813in}{3.538651in}}%
\pgfpathlineto{\pgfqpoint{4.041150in}{3.544114in}}%
\pgfpathlineto{\pgfqpoint{4.074860in}{3.525271in}}%
\pgfpathlineto{\pgfqpoint{4.108536in}{3.505581in}}%
\pgfpathlineto{\pgfqpoint{4.098134in}{3.501548in}}%
\pgfpathlineto{\pgfqpoint{4.087753in}{3.498680in}}%
\pgfpathlineto{\pgfqpoint{4.054141in}{3.516934in}}%
\pgfpathlineto{\pgfqpoint{4.020497in}{3.534512in}}%
\pgfpathclose%
\pgfusepath{fill}%
\end{pgfscope}%
\begin{pgfscope}%
\pgfpathrectangle{\pgfqpoint{1.020000in}{0.880000in}}{\pgfqpoint{6.160000in}{6.160000in}}%
\pgfusepath{clip}%
\pgfsetbuttcap%
\pgfsetroundjoin%
\definecolor{currentfill}{rgb}{0.592356,0.722792,0.999434}%
\pgfsetfillcolor{currentfill}%
\pgfsetlinewidth{0.000000pt}%
\definecolor{currentstroke}{rgb}{0.000000,0.000000,0.000000}%
\pgfsetstrokecolor{currentstroke}%
\pgfsetdash{}{0pt}%
\pgfpathmoveto{\pgfqpoint{4.331098in}{3.378499in}}%
\pgfpathlineto{\pgfqpoint{4.341751in}{3.378059in}}%
\pgfpathlineto{\pgfqpoint{4.352426in}{3.377646in}}%
\pgfpathlineto{\pgfqpoint{4.385968in}{3.352715in}}%
\pgfpathlineto{\pgfqpoint{4.419474in}{3.328352in}}%
\pgfpathlineto{\pgfqpoint{4.408744in}{3.331027in}}%
\pgfpathlineto{\pgfqpoint{4.398037in}{3.333710in}}%
\pgfpathlineto{\pgfqpoint{4.364585in}{3.355838in}}%
\pgfpathlineto{\pgfqpoint{4.331098in}{3.378499in}}%
\pgfpathclose%
\pgfusepath{fill}%
\end{pgfscope}%
\begin{pgfscope}%
\pgfpathrectangle{\pgfqpoint{1.020000in}{0.880000in}}{\pgfqpoint{6.160000in}{6.160000in}}%
\pgfusepath{clip}%
\pgfsetbuttcap%
\pgfsetroundjoin%
\definecolor{currentfill}{rgb}{0.358415,0.478426,0.896795}%
\pgfsetfillcolor{currentfill}%
\pgfsetlinewidth{0.000000pt}%
\definecolor{currentstroke}{rgb}{0.000000,0.000000,0.000000}%
\pgfsetstrokecolor{currentstroke}%
\pgfsetdash{}{0pt}%
\pgfpathmoveto{\pgfqpoint{5.107661in}{2.950702in}}%
\pgfpathlineto{\pgfqpoint{5.119008in}{2.935161in}}%
\pgfpathlineto{\pgfqpoint{5.130374in}{2.919435in}}%
\pgfpathlineto{\pgfqpoint{5.163641in}{2.913548in}}%
\pgfpathlineto{\pgfqpoint{5.196890in}{2.908607in}}%
\pgfpathlineto{\pgfqpoint{5.185472in}{2.924411in}}%
\pgfpathlineto{\pgfqpoint{5.174074in}{2.940070in}}%
\pgfpathlineto{\pgfqpoint{5.140876in}{2.944926in}}%
\pgfpathlineto{\pgfqpoint{5.107661in}{2.950702in}}%
\pgfpathclose%
\pgfusepath{fill}%
\end{pgfscope}%
\begin{pgfscope}%
\pgfpathrectangle{\pgfqpoint{1.020000in}{0.880000in}}{\pgfqpoint{6.160000in}{6.160000in}}%
\pgfusepath{clip}%
\pgfsetbuttcap%
\pgfsetroundjoin%
\definecolor{currentfill}{rgb}{0.782049,0.842864,0.942980}%
\pgfsetfillcolor{currentfill}%
\pgfsetlinewidth{0.000000pt}%
\definecolor{currentstroke}{rgb}{0.000000,0.000000,0.000000}%
\pgfsetstrokecolor{currentstroke}%
\pgfsetdash{}{0pt}%
\pgfpathmoveto{\pgfqpoint{2.530175in}{3.727651in}}%
\pgfpathlineto{\pgfqpoint{2.539553in}{3.686204in}}%
\pgfpathlineto{\pgfqpoint{2.548926in}{3.645900in}}%
\pgfpathlineto{\pgfqpoint{2.582873in}{3.660358in}}%
\pgfpathlineto{\pgfqpoint{2.616798in}{3.675489in}}%
\pgfpathlineto{\pgfqpoint{2.607349in}{3.718082in}}%
\pgfpathlineto{\pgfqpoint{2.597891in}{3.762041in}}%
\pgfpathlineto{\pgfqpoint{2.564044in}{3.744493in}}%
\pgfpathlineto{\pgfqpoint{2.530175in}{3.727651in}}%
\pgfpathclose%
\pgfusepath{fill}%
\end{pgfscope}%
\begin{pgfscope}%
\pgfpathrectangle{\pgfqpoint{1.020000in}{0.880000in}}{\pgfqpoint{6.160000in}{6.160000in}}%
\pgfusepath{clip}%
\pgfsetbuttcap%
\pgfsetroundjoin%
\definecolor{currentfill}{rgb}{0.646113,0.764436,0.996868}%
\pgfsetfillcolor{currentfill}%
\pgfsetlinewidth{0.000000pt}%
\definecolor{currentstroke}{rgb}{0.000000,0.000000,0.000000}%
\pgfsetstrokecolor{currentstroke}%
\pgfsetdash{}{0pt}%
\pgfpathmoveto{\pgfqpoint{4.175786in}{3.464432in}}%
\pgfpathlineto{\pgfqpoint{4.186274in}{3.467543in}}%
\pgfpathlineto{\pgfqpoint{4.196785in}{3.471228in}}%
\pgfpathlineto{\pgfqpoint{4.230419in}{3.448096in}}%
\pgfpathlineto{\pgfqpoint{4.264015in}{3.424803in}}%
\pgfpathlineto{\pgfqpoint{4.253443in}{3.423249in}}%
\pgfpathlineto{\pgfqpoint{4.242894in}{3.422178in}}%
\pgfpathlineto{\pgfqpoint{4.209358in}{3.443352in}}%
\pgfpathlineto{\pgfqpoint{4.175786in}{3.464432in}}%
\pgfpathclose%
\pgfusepath{fill}%
\end{pgfscope}%
\begin{pgfscope}%
\pgfpathrectangle{\pgfqpoint{1.020000in}{0.880000in}}{\pgfqpoint{6.160000in}{6.160000in}}%
\pgfusepath{clip}%
\pgfsetbuttcap%
\pgfsetroundjoin%
\definecolor{currentfill}{rgb}{0.289996,0.386836,0.828926}%
\pgfsetfillcolor{currentfill}%
\pgfsetlinewidth{0.000000pt}%
\definecolor{currentstroke}{rgb}{0.000000,0.000000,0.000000}%
\pgfsetstrokecolor{currentstroke}%
\pgfsetdash{}{0pt}%
\pgfpathmoveto{\pgfqpoint{5.930341in}{2.816632in}}%
\pgfpathlineto{\pgfqpoint{5.942455in}{2.801526in}}%
\pgfpathlineto{\pgfqpoint{5.954593in}{2.786469in}}%
\pgfpathlineto{\pgfqpoint{5.987674in}{2.788681in}}%
\pgfpathlineto{\pgfqpoint{6.020735in}{2.790930in}}%
\pgfpathlineto{\pgfqpoint{6.008542in}{2.805898in}}%
\pgfpathlineto{\pgfqpoint{5.996373in}{2.820915in}}%
\pgfpathlineto{\pgfqpoint{5.963367in}{2.818754in}}%
\pgfpathlineto{\pgfqpoint{5.930341in}{2.816632in}}%
\pgfpathclose%
\pgfusepath{fill}%
\end{pgfscope}%
\begin{pgfscope}%
\pgfpathrectangle{\pgfqpoint{1.020000in}{0.880000in}}{\pgfqpoint{6.160000in}{6.160000in}}%
\pgfusepath{clip}%
\pgfsetbuttcap%
\pgfsetroundjoin%
\definecolor{currentfill}{rgb}{0.299441,0.400248,0.839842}%
\pgfsetfillcolor{currentfill}%
\pgfsetlinewidth{0.000000pt}%
\definecolor{currentstroke}{rgb}{0.000000,0.000000,0.000000}%
\pgfsetstrokecolor{currentstroke}%
\pgfsetdash{}{0pt}%
\pgfpathmoveto{\pgfqpoint{5.707941in}{2.836168in}}%
\pgfpathlineto{\pgfqpoint{5.719843in}{2.820684in}}%
\pgfpathlineto{\pgfqpoint{5.731769in}{2.805243in}}%
\pgfpathlineto{\pgfqpoint{5.764913in}{2.806920in}}%
\pgfpathlineto{\pgfqpoint{5.798038in}{2.808705in}}%
\pgfpathlineto{\pgfqpoint{5.786058in}{2.824046in}}%
\pgfpathlineto{\pgfqpoint{5.774100in}{2.839433in}}%
\pgfpathlineto{\pgfqpoint{5.741030in}{2.837745in}}%
\pgfpathlineto{\pgfqpoint{5.707941in}{2.836168in}}%
\pgfpathclose%
\pgfusepath{fill}%
\end{pgfscope}%
\begin{pgfscope}%
\pgfpathrectangle{\pgfqpoint{1.020000in}{0.880000in}}{\pgfqpoint{6.160000in}{6.160000in}}%
\pgfusepath{clip}%
\pgfsetbuttcap%
\pgfsetroundjoin%
\definecolor{currentfill}{rgb}{0.748682,0.827679,0.963334}%
\pgfsetfillcolor{currentfill}%
\pgfsetlinewidth{0.000000pt}%
\definecolor{currentstroke}{rgb}{0.000000,0.000000,0.000000}%
\pgfsetstrokecolor{currentstroke}%
\pgfsetdash{}{0pt}%
\pgfpathmoveto{\pgfqpoint{3.642672in}{3.650926in}}%
\pgfpathlineto{\pgfqpoint{3.652631in}{3.643595in}}%
\pgfpathlineto{\pgfqpoint{3.662595in}{3.639403in}}%
\pgfpathlineto{\pgfqpoint{3.696429in}{3.635102in}}%
\pgfpathlineto{\pgfqpoint{3.730246in}{3.629122in}}%
\pgfpathlineto{\pgfqpoint{3.720227in}{3.631827in}}%
\pgfpathlineto{\pgfqpoint{3.710216in}{3.637522in}}%
\pgfpathlineto{\pgfqpoint{3.676454in}{3.644966in}}%
\pgfpathlineto{\pgfqpoint{3.642672in}{3.650926in}}%
\pgfpathclose%
\pgfusepath{fill}%
\end{pgfscope}%
\begin{pgfscope}%
\pgfpathrectangle{\pgfqpoint{1.020000in}{0.880000in}}{\pgfqpoint{6.160000in}{6.160000in}}%
\pgfusepath{clip}%
\pgfsetbuttcap%
\pgfsetroundjoin%
\definecolor{currentfill}{rgb}{0.313946,0.420052,0.854993}%
\pgfsetfillcolor{currentfill}%
\pgfsetlinewidth{0.000000pt}%
\definecolor{currentstroke}{rgb}{0.000000,0.000000,0.000000}%
\pgfsetstrokecolor{currentstroke}%
\pgfsetdash{}{0pt}%
\pgfpathmoveto{\pgfqpoint{5.485596in}{2.862037in}}%
\pgfpathlineto{\pgfqpoint{5.497290in}{2.846213in}}%
\pgfpathlineto{\pgfqpoint{5.509006in}{2.830397in}}%
\pgfpathlineto{\pgfqpoint{5.542208in}{2.830764in}}%
\pgfpathlineto{\pgfqpoint{5.575392in}{2.831418in}}%
\pgfpathlineto{\pgfqpoint{5.563621in}{2.847133in}}%
\pgfpathlineto{\pgfqpoint{5.551873in}{2.862868in}}%
\pgfpathlineto{\pgfqpoint{5.518744in}{2.862309in}}%
\pgfpathlineto{\pgfqpoint{5.485596in}{2.862037in}}%
\pgfpathclose%
\pgfusepath{fill}%
\end{pgfscope}%
\begin{pgfscope}%
\pgfpathrectangle{\pgfqpoint{1.020000in}{0.880000in}}{\pgfqpoint{6.160000in}{6.160000in}}%
\pgfusepath{clip}%
\pgfsetbuttcap%
\pgfsetroundjoin%
\definecolor{currentfill}{rgb}{0.285273,0.380129,0.823469}%
\pgfsetfillcolor{currentfill}%
\pgfsetlinewidth{0.000000pt}%
\definecolor{currentstroke}{rgb}{0.000000,0.000000,0.000000}%
\pgfsetstrokecolor{currentstroke}%
\pgfsetdash{}{0pt}%
\pgfpathmoveto{\pgfqpoint{6.152778in}{2.800206in}}%
\pgfpathlineto{\pgfqpoint{6.165105in}{2.785455in}}%
\pgfpathlineto{\pgfqpoint{6.177456in}{2.770753in}}%
\pgfpathlineto{\pgfqpoint{6.210471in}{2.773203in}}%
\pgfpathlineto{\pgfqpoint{6.198093in}{2.787866in}}%
\pgfpathlineto{\pgfqpoint{6.185738in}{2.802577in}}%
\pgfpathlineto{\pgfqpoint{6.152778in}{2.800206in}}%
\pgfpathclose%
\pgfusepath{fill}%
\end{pgfscope}%
\begin{pgfscope}%
\pgfpathrectangle{\pgfqpoint{1.020000in}{0.880000in}}{\pgfqpoint{6.160000in}{6.160000in}}%
\pgfusepath{clip}%
\pgfsetbuttcap%
\pgfsetroundjoin%
\definecolor{currentfill}{rgb}{0.786721,0.844807,0.939810}%
\pgfsetfillcolor{currentfill}%
\pgfsetlinewidth{0.000000pt}%
\definecolor{currentstroke}{rgb}{0.000000,0.000000,0.000000}%
\pgfsetstrokecolor{currentstroke}%
\pgfsetdash{}{0pt}%
\pgfpathmoveto{\pgfqpoint{2.752306in}{3.740578in}}%
\pgfpathlineto{\pgfqpoint{2.761885in}{3.695432in}}%
\pgfpathlineto{\pgfqpoint{2.771449in}{3.652337in}}%
\pgfpathlineto{\pgfqpoint{2.805351in}{3.667428in}}%
\pgfpathlineto{\pgfqpoint{2.839241in}{3.682545in}}%
\pgfpathlineto{\pgfqpoint{2.829617in}{3.727110in}}%
\pgfpathlineto{\pgfqpoint{2.819976in}{3.774008in}}%
\pgfpathlineto{\pgfqpoint{2.786146in}{3.757343in}}%
\pgfpathlineto{\pgfqpoint{2.752306in}{3.740578in}}%
\pgfpathclose%
\pgfusepath{fill}%
\end{pgfscope}%
\begin{pgfscope}%
\pgfpathrectangle{\pgfqpoint{1.020000in}{0.880000in}}{\pgfqpoint{6.160000in}{6.160000in}}%
\pgfusepath{clip}%
\pgfsetbuttcap%
\pgfsetroundjoin%
\definecolor{currentfill}{rgb}{0.768034,0.837035,0.952488}%
\pgfsetfillcolor{currentfill}%
\pgfsetlinewidth{0.000000pt}%
\definecolor{currentstroke}{rgb}{0.000000,0.000000,0.000000}%
\pgfsetstrokecolor{currentstroke}%
\pgfsetdash{}{0pt}%
\pgfpathmoveto{\pgfqpoint{3.419995in}{3.692322in}}%
\pgfpathlineto{\pgfqpoint{3.429833in}{3.671183in}}%
\pgfpathlineto{\pgfqpoint{3.439662in}{3.653735in}}%
\pgfpathlineto{\pgfqpoint{3.473528in}{3.657194in}}%
\pgfpathlineto{\pgfqpoint{3.507383in}{3.659111in}}%
\pgfpathlineto{\pgfqpoint{3.497515in}{3.674154in}}%
\pgfpathlineto{\pgfqpoint{3.487641in}{3.692871in}}%
\pgfpathlineto{\pgfqpoint{3.453824in}{3.693360in}}%
\pgfpathlineto{\pgfqpoint{3.419995in}{3.692322in}}%
\pgfpathclose%
\pgfusepath{fill}%
\end{pgfscope}%
\begin{pgfscope}%
\pgfpathrectangle{\pgfqpoint{1.020000in}{0.880000in}}{\pgfqpoint{6.160000in}{6.160000in}}%
\pgfusepath{clip}%
\pgfsetbuttcap%
\pgfsetroundjoin%
\definecolor{currentfill}{rgb}{0.786721,0.844807,0.939810}%
\pgfsetfillcolor{currentfill}%
\pgfsetlinewidth{0.000000pt}%
\definecolor{currentstroke}{rgb}{0.000000,0.000000,0.000000}%
\pgfsetstrokecolor{currentstroke}%
\pgfsetdash{}{0pt}%
\pgfpathmoveto{\pgfqpoint{2.974697in}{3.740441in}}%
\pgfpathlineto{\pgfqpoint{2.984402in}{3.696889in}}%
\pgfpathlineto{\pgfqpoint{2.994085in}{3.656328in}}%
\pgfpathlineto{\pgfqpoint{3.027973in}{3.669378in}}%
\pgfpathlineto{\pgfqpoint{3.061853in}{3.681809in}}%
\pgfpathlineto{\pgfqpoint{3.052129in}{3.722166in}}%
\pgfpathlineto{\pgfqpoint{3.042383in}{3.765772in}}%
\pgfpathlineto{\pgfqpoint{3.008543in}{3.753531in}}%
\pgfpathlineto{\pgfqpoint{2.974697in}{3.740441in}}%
\pgfpathclose%
\pgfusepath{fill}%
\end{pgfscope}%
\begin{pgfscope}%
\pgfpathrectangle{\pgfqpoint{1.020000in}{0.880000in}}{\pgfqpoint{6.160000in}{6.160000in}}%
\pgfusepath{clip}%
\pgfsetbuttcap%
\pgfsetroundjoin%
\definecolor{currentfill}{rgb}{0.782049,0.842864,0.942980}%
\pgfsetfillcolor{currentfill}%
\pgfsetlinewidth{0.000000pt}%
\definecolor{currentstroke}{rgb}{0.000000,0.000000,0.000000}%
\pgfsetstrokecolor{currentstroke}%
\pgfsetdash{}{0pt}%
\pgfpathmoveto{\pgfqpoint{3.197308in}{3.722727in}}%
\pgfpathlineto{\pgfqpoint{3.207080in}{3.687821in}}%
\pgfpathlineto{\pgfqpoint{3.216833in}{3.656542in}}%
\pgfpathlineto{\pgfqpoint{3.250712in}{3.665670in}}%
\pgfpathlineto{\pgfqpoint{3.284585in}{3.673649in}}%
\pgfpathlineto{\pgfqpoint{3.274800in}{3.703078in}}%
\pgfpathlineto{\pgfqpoint{3.264998in}{3.736276in}}%
\pgfpathlineto{\pgfqpoint{3.231156in}{3.730163in}}%
\pgfpathlineto{\pgfqpoint{3.197308in}{3.722727in}}%
\pgfpathclose%
\pgfusepath{fill}%
\end{pgfscope}%
\begin{pgfscope}%
\pgfpathrectangle{\pgfqpoint{1.020000in}{0.880000in}}{\pgfqpoint{6.160000in}{6.160000in}}%
\pgfusepath{clip}%
\pgfsetbuttcap%
\pgfsetroundjoin%
\definecolor{currentfill}{rgb}{0.333490,0.446265,0.874452}%
\pgfsetfillcolor{currentfill}%
\pgfsetlinewidth{0.000000pt}%
\definecolor{currentstroke}{rgb}{0.000000,0.000000,0.000000}%
\pgfsetstrokecolor{currentstroke}%
\pgfsetdash{}{0pt}%
\pgfpathmoveto{\pgfqpoint{5.263339in}{2.901227in}}%
\pgfpathlineto{\pgfqpoint{5.274829in}{2.885295in}}%
\pgfpathlineto{\pgfqpoint{5.286341in}{2.869283in}}%
\pgfpathlineto{\pgfqpoint{5.319593in}{2.866717in}}%
\pgfpathlineto{\pgfqpoint{5.352828in}{2.864783in}}%
\pgfpathlineto{\pgfqpoint{5.341263in}{2.880742in}}%
\pgfpathlineto{\pgfqpoint{5.329720in}{2.896648in}}%
\pgfpathlineto{\pgfqpoint{5.296538in}{2.898625in}}%
\pgfpathlineto{\pgfqpoint{5.263339in}{2.901227in}}%
\pgfpathclose%
\pgfusepath{fill}%
\end{pgfscope}%
\begin{pgfscope}%
\pgfpathrectangle{\pgfqpoint{1.020000in}{0.880000in}}{\pgfqpoint{6.160000in}{6.160000in}}%
\pgfusepath{clip}%
\pgfsetbuttcap%
\pgfsetroundjoin%
\definecolor{currentfill}{rgb}{0.768034,0.837035,0.952488}%
\pgfsetfillcolor{currentfill}%
\pgfsetlinewidth{0.000000pt}%
\definecolor{currentstroke}{rgb}{0.000000,0.000000,0.000000}%
\pgfsetstrokecolor{currentstroke}%
\pgfsetdash{}{0pt}%
\pgfpathmoveto{\pgfqpoint{2.462361in}{3.696483in}}%
\pgfpathlineto{\pgfqpoint{2.471658in}{3.657429in}}%
\pgfpathlineto{\pgfqpoint{2.480952in}{3.619316in}}%
\pgfpathlineto{\pgfqpoint{2.514952in}{3.632196in}}%
\pgfpathlineto{\pgfqpoint{2.548926in}{3.645900in}}%
\pgfpathlineto{\pgfqpoint{2.539553in}{3.686204in}}%
\pgfpathlineto{\pgfqpoint{2.530175in}{3.727651in}}%
\pgfpathlineto{\pgfqpoint{2.496282in}{3.711619in}}%
\pgfpathlineto{\pgfqpoint{2.462361in}{3.696483in}}%
\pgfpathclose%
\pgfusepath{fill}%
\end{pgfscope}%
\begin{pgfscope}%
\pgfpathrectangle{\pgfqpoint{1.020000in}{0.880000in}}{\pgfqpoint{6.160000in}{6.160000in}}%
\pgfusepath{clip}%
\pgfsetbuttcap%
\pgfsetroundjoin%
\definecolor{currentfill}{rgb}{0.733898,0.820018,0.970724}%
\pgfsetfillcolor{currentfill}%
\pgfsetlinewidth{0.000000pt}%
\definecolor{currentstroke}{rgb}{0.000000,0.000000,0.000000}%
\pgfsetstrokecolor{currentstroke}%
\pgfsetdash{}{0pt}%
\pgfpathmoveto{\pgfqpoint{3.797817in}{3.612374in}}%
\pgfpathlineto{\pgfqpoint{3.807906in}{3.613327in}}%
\pgfpathlineto{\pgfqpoint{3.818008in}{3.616637in}}%
\pgfpathlineto{\pgfqpoint{3.851823in}{3.606477in}}%
\pgfpathlineto{\pgfqpoint{3.885613in}{3.594725in}}%
\pgfpathlineto{\pgfqpoint{3.875446in}{3.591186in}}%
\pgfpathlineto{\pgfqpoint{3.865296in}{3.589826in}}%
\pgfpathlineto{\pgfqpoint{3.831569in}{3.601773in}}%
\pgfpathlineto{\pgfqpoint{3.797817in}{3.612374in}}%
\pgfpathclose%
\pgfusepath{fill}%
\end{pgfscope}%
\begin{pgfscope}%
\pgfpathrectangle{\pgfqpoint{1.020000in}{0.880000in}}{\pgfqpoint{6.160000in}{6.160000in}}%
\pgfusepath{clip}%
\pgfsetbuttcap%
\pgfsetroundjoin%
\definecolor{currentfill}{rgb}{0.772706,0.838978,0.949319}%
\pgfsetfillcolor{currentfill}%
\pgfsetlinewidth{0.000000pt}%
\definecolor{currentstroke}{rgb}{0.000000,0.000000,0.000000}%
\pgfsetstrokecolor{currentstroke}%
\pgfsetdash{}{0pt}%
\pgfpathmoveto{\pgfqpoint{2.684586in}{3.707361in}}%
\pgfpathlineto{\pgfqpoint{2.694097in}{3.664168in}}%
\pgfpathlineto{\pgfqpoint{2.703596in}{3.622752in}}%
\pgfpathlineto{\pgfqpoint{2.737531in}{3.637405in}}%
\pgfpathlineto{\pgfqpoint{2.771449in}{3.652337in}}%
\pgfpathlineto{\pgfqpoint{2.761885in}{3.695432in}}%
\pgfpathlineto{\pgfqpoint{2.752306in}{3.740578in}}%
\pgfpathlineto{\pgfqpoint{2.718454in}{3.723868in}}%
\pgfpathlineto{\pgfqpoint{2.684586in}{3.707361in}}%
\pgfpathclose%
\pgfusepath{fill}%
\end{pgfscope}%
\begin{pgfscope}%
\pgfpathrectangle{\pgfqpoint{1.020000in}{0.880000in}}{\pgfqpoint{6.160000in}{6.160000in}}%
\pgfusepath{clip}%
\pgfsetbuttcap%
\pgfsetroundjoin%
\definecolor{currentfill}{rgb}{0.451739,0.588181,0.960201}%
\pgfsetfillcolor{currentfill}%
\pgfsetlinewidth{0.000000pt}%
\definecolor{currentstroke}{rgb}{0.000000,0.000000,0.000000}%
\pgfsetstrokecolor{currentstroke}%
\pgfsetdash{}{0pt}%
\pgfpathmoveto{\pgfqpoint{4.730247in}{3.125571in}}%
\pgfpathlineto{\pgfqpoint{4.741275in}{3.114204in}}%
\pgfpathlineto{\pgfqpoint{4.752321in}{3.102367in}}%
\pgfpathlineto{\pgfqpoint{4.785690in}{3.083908in}}%
\pgfpathlineto{\pgfqpoint{4.819034in}{3.066991in}}%
\pgfpathlineto{\pgfqpoint{4.807940in}{3.079799in}}%
\pgfpathlineto{\pgfqpoint{4.796866in}{3.092218in}}%
\pgfpathlineto{\pgfqpoint{4.763569in}{3.108175in}}%
\pgfpathlineto{\pgfqpoint{4.730247in}{3.125571in}}%
\pgfpathclose%
\pgfusepath{fill}%
\end{pgfscope}%
\begin{pgfscope}%
\pgfpathrectangle{\pgfqpoint{1.020000in}{0.880000in}}{\pgfqpoint{6.160000in}{6.160000in}}%
\pgfusepath{clip}%
\pgfsetbuttcap%
\pgfsetroundjoin%
\definecolor{currentfill}{rgb}{0.404421,0.534643,0.932002}%
\pgfsetfillcolor{currentfill}%
\pgfsetlinewidth{0.000000pt}%
\definecolor{currentstroke}{rgb}{0.000000,0.000000,0.000000}%
\pgfsetstrokecolor{currentstroke}%
\pgfsetdash{}{0pt}%
\pgfpathmoveto{\pgfqpoint{4.885653in}{3.037648in}}%
\pgfpathlineto{\pgfqpoint{4.896813in}{3.023716in}}%
\pgfpathlineto{\pgfqpoint{4.907992in}{3.009414in}}%
\pgfpathlineto{\pgfqpoint{4.941318in}{2.996349in}}%
\pgfpathlineto{\pgfqpoint{4.974624in}{2.984702in}}%
\pgfpathlineto{\pgfqpoint{4.963396in}{2.999470in}}%
\pgfpathlineto{\pgfqpoint{4.952188in}{3.013938in}}%
\pgfpathlineto{\pgfqpoint{4.918931in}{3.025120in}}%
\pgfpathlineto{\pgfqpoint{4.885653in}{3.037648in}}%
\pgfpathclose%
\pgfusepath{fill}%
\end{pgfscope}%
\begin{pgfscope}%
\pgfpathrectangle{\pgfqpoint{1.020000in}{0.880000in}}{\pgfqpoint{6.160000in}{6.160000in}}%
\pgfusepath{clip}%
\pgfsetbuttcap%
\pgfsetroundjoin%
\definecolor{currentfill}{rgb}{0.285273,0.380129,0.823469}%
\pgfsetfillcolor{currentfill}%
\pgfsetlinewidth{0.000000pt}%
\definecolor{currentstroke}{rgb}{0.000000,0.000000,0.000000}%
\pgfsetstrokecolor{currentstroke}%
\pgfsetdash{}{0pt}%
\pgfpathmoveto{\pgfqpoint{6.086797in}{2.795521in}}%
\pgfpathlineto{\pgfqpoint{6.099069in}{2.780688in}}%
\pgfpathlineto{\pgfqpoint{6.111365in}{2.765903in}}%
\pgfpathlineto{\pgfqpoint{6.144420in}{2.768318in}}%
\pgfpathlineto{\pgfqpoint{6.177456in}{2.770753in}}%
\pgfpathlineto{\pgfqpoint{6.165105in}{2.785455in}}%
\pgfpathlineto{\pgfqpoint{6.152778in}{2.800206in}}%
\pgfpathlineto{\pgfqpoint{6.119797in}{2.797854in}}%
\pgfpathlineto{\pgfqpoint{6.086797in}{2.795521in}}%
\pgfpathclose%
\pgfusepath{fill}%
\end{pgfscope}%
\begin{pgfscope}%
\pgfpathrectangle{\pgfqpoint{1.020000in}{0.880000in}}{\pgfqpoint{6.160000in}{6.160000in}}%
\pgfusepath{clip}%
\pgfsetbuttcap%
\pgfsetroundjoin%
\definecolor{currentfill}{rgb}{0.294718,0.393542,0.834384}%
\pgfsetfillcolor{currentfill}%
\pgfsetlinewidth{0.000000pt}%
\definecolor{currentstroke}{rgb}{0.000000,0.000000,0.000000}%
\pgfsetstrokecolor{currentstroke}%
\pgfsetdash{}{0pt}%
\pgfpathmoveto{\pgfqpoint{5.864229in}{2.812537in}}%
\pgfpathlineto{\pgfqpoint{5.876288in}{2.797338in}}%
\pgfpathlineto{\pgfqpoint{5.888371in}{2.782189in}}%
\pgfpathlineto{\pgfqpoint{5.921492in}{2.784303in}}%
\pgfpathlineto{\pgfqpoint{5.954593in}{2.786469in}}%
\pgfpathlineto{\pgfqpoint{5.942455in}{2.801526in}}%
\pgfpathlineto{\pgfqpoint{5.930341in}{2.816632in}}%
\pgfpathlineto{\pgfqpoint{5.897295in}{2.814557in}}%
\pgfpathlineto{\pgfqpoint{5.864229in}{2.812537in}}%
\pgfpathclose%
\pgfusepath{fill}%
\end{pgfscope}%
\begin{pgfscope}%
\pgfpathrectangle{\pgfqpoint{1.020000in}{0.880000in}}{\pgfqpoint{6.160000in}{6.160000in}}%
\pgfusepath{clip}%
\pgfsetbuttcap%
\pgfsetroundjoin%
\definecolor{currentfill}{rgb}{0.304174,0.406945,0.845263}%
\pgfsetfillcolor{currentfill}%
\pgfsetlinewidth{0.000000pt}%
\definecolor{currentstroke}{rgb}{0.000000,0.000000,0.000000}%
\pgfsetstrokecolor{currentstroke}%
\pgfsetdash{}{0pt}%
\pgfpathmoveto{\pgfqpoint{5.641704in}{2.833429in}}%
\pgfpathlineto{\pgfqpoint{5.653552in}{2.817845in}}%
\pgfpathlineto{\pgfqpoint{5.665423in}{2.802298in}}%
\pgfpathlineto{\pgfqpoint{5.698605in}{2.803695in}}%
\pgfpathlineto{\pgfqpoint{5.731769in}{2.805243in}}%
\pgfpathlineto{\pgfqpoint{5.719843in}{2.820684in}}%
\pgfpathlineto{\pgfqpoint{5.707941in}{2.836168in}}%
\pgfpathlineto{\pgfqpoint{5.674832in}{2.834721in}}%
\pgfpathlineto{\pgfqpoint{5.641704in}{2.833429in}}%
\pgfpathclose%
\pgfusepath{fill}%
\end{pgfscope}%
\begin{pgfscope}%
\pgfpathrectangle{\pgfqpoint{1.020000in}{0.880000in}}{\pgfqpoint{6.160000in}{6.160000in}}%
\pgfusepath{clip}%
\pgfsetbuttcap%
\pgfsetroundjoin%
\definecolor{currentfill}{rgb}{0.510824,0.649397,0.985079}%
\pgfsetfillcolor{currentfill}%
\pgfsetlinewidth{0.000000pt}%
\definecolor{currentstroke}{rgb}{0.000000,0.000000,0.000000}%
\pgfsetstrokecolor{currentstroke}%
\pgfsetdash{}{0pt}%
\pgfpathmoveto{\pgfqpoint{4.574877in}{3.225169in}}%
\pgfpathlineto{\pgfqpoint{4.585771in}{3.217618in}}%
\pgfpathlineto{\pgfqpoint{4.596684in}{3.209588in}}%
\pgfpathlineto{\pgfqpoint{4.630119in}{3.186460in}}%
\pgfpathlineto{\pgfqpoint{4.663524in}{3.164724in}}%
\pgfpathlineto{\pgfqpoint{4.652562in}{3.174331in}}%
\pgfpathlineto{\pgfqpoint{4.641621in}{3.183534in}}%
\pgfpathlineto{\pgfqpoint{4.608264in}{3.203711in}}%
\pgfpathlineto{\pgfqpoint{4.574877in}{3.225169in}}%
\pgfpathclose%
\pgfusepath{fill}%
\end{pgfscope}%
\begin{pgfscope}%
\pgfpathrectangle{\pgfqpoint{1.020000in}{0.880000in}}{\pgfqpoint{6.160000in}{6.160000in}}%
\pgfusepath{clip}%
\pgfsetbuttcap%
\pgfsetroundjoin%
\definecolor{currentfill}{rgb}{0.777378,0.840921,0.946149}%
\pgfsetfillcolor{currentfill}%
\pgfsetlinewidth{0.000000pt}%
\definecolor{currentstroke}{rgb}{0.000000,0.000000,0.000000}%
\pgfsetstrokecolor{currentstroke}%
\pgfsetdash{}{0pt}%
\pgfpathmoveto{\pgfqpoint{2.906986in}{3.712301in}}%
\pgfpathlineto{\pgfqpoint{2.916644in}{3.669249in}}%
\pgfpathlineto{\pgfqpoint{2.926282in}{3.628910in}}%
\pgfpathlineto{\pgfqpoint{2.960189in}{3.642793in}}%
\pgfpathlineto{\pgfqpoint{2.994085in}{3.656328in}}%
\pgfpathlineto{\pgfqpoint{2.984402in}{3.696889in}}%
\pgfpathlineto{\pgfqpoint{2.974697in}{3.740441in}}%
\pgfpathlineto{\pgfqpoint{2.940845in}{3.726648in}}%
\pgfpathlineto{\pgfqpoint{2.906986in}{3.712301in}}%
\pgfpathclose%
\pgfusepath{fill}%
\end{pgfscope}%
\begin{pgfscope}%
\pgfpathrectangle{\pgfqpoint{1.020000in}{0.880000in}}{\pgfqpoint{6.160000in}{6.160000in}}%
\pgfusepath{clip}%
\pgfsetbuttcap%
\pgfsetroundjoin%
\definecolor{currentfill}{rgb}{0.368507,0.491141,0.905243}%
\pgfsetfillcolor{currentfill}%
\pgfsetlinewidth{0.000000pt}%
\definecolor{currentstroke}{rgb}{0.000000,0.000000,0.000000}%
\pgfsetstrokecolor{currentstroke}%
\pgfsetdash{}{0pt}%
\pgfpathmoveto{\pgfqpoint{5.041178in}{2.965351in}}%
\pgfpathlineto{\pgfqpoint{5.052475in}{2.950000in}}%
\pgfpathlineto{\pgfqpoint{5.063790in}{2.934409in}}%
\pgfpathlineto{\pgfqpoint{5.097091in}{2.926359in}}%
\pgfpathlineto{\pgfqpoint{5.130374in}{2.919435in}}%
\pgfpathlineto{\pgfqpoint{5.119008in}{2.935161in}}%
\pgfpathlineto{\pgfqpoint{5.107661in}{2.950702in}}%
\pgfpathlineto{\pgfqpoint{5.074428in}{2.957481in}}%
\pgfpathlineto{\pgfqpoint{5.041178in}{2.965351in}}%
\pgfpathclose%
\pgfusepath{fill}%
\end{pgfscope}%
\begin{pgfscope}%
\pgfpathrectangle{\pgfqpoint{1.020000in}{0.880000in}}{\pgfqpoint{6.160000in}{6.160000in}}%
\pgfusepath{clip}%
\pgfsetbuttcap%
\pgfsetroundjoin%
\definecolor{currentfill}{rgb}{0.318832,0.426605,0.859857}%
\pgfsetfillcolor{currentfill}%
\pgfsetlinewidth{0.000000pt}%
\definecolor{currentstroke}{rgb}{0.000000,0.000000,0.000000}%
\pgfsetstrokecolor{currentstroke}%
\pgfsetdash{}{0pt}%
\pgfpathmoveto{\pgfqpoint{5.419247in}{2.862536in}}%
\pgfpathlineto{\pgfqpoint{5.430887in}{2.846624in}}%
\pgfpathlineto{\pgfqpoint{5.442549in}{2.830702in}}%
\pgfpathlineto{\pgfqpoint{5.475786in}{2.830360in}}%
\pgfpathlineto{\pgfqpoint{5.509006in}{2.830397in}}%
\pgfpathlineto{\pgfqpoint{5.497290in}{2.846213in}}%
\pgfpathlineto{\pgfqpoint{5.485596in}{2.862037in}}%
\pgfpathlineto{\pgfqpoint{5.452430in}{2.862096in}}%
\pgfpathlineto{\pgfqpoint{5.419247in}{2.862536in}}%
\pgfpathclose%
\pgfusepath{fill}%
\end{pgfscope}%
\begin{pgfscope}%
\pgfpathrectangle{\pgfqpoint{1.020000in}{0.880000in}}{\pgfqpoint{6.160000in}{6.160000in}}%
\pgfusepath{clip}%
\pgfsetbuttcap%
\pgfsetroundjoin%
\definecolor{currentfill}{rgb}{0.758539,0.832787,0.958408}%
\pgfsetfillcolor{currentfill}%
\pgfsetlinewidth{0.000000pt}%
\definecolor{currentstroke}{rgb}{0.000000,0.000000,0.000000}%
\pgfsetstrokecolor{currentstroke}%
\pgfsetdash{}{0pt}%
\pgfpathmoveto{\pgfqpoint{3.575057in}{3.658196in}}%
\pgfpathlineto{\pgfqpoint{3.584967in}{3.648893in}}%
\pgfpathlineto{\pgfqpoint{3.594879in}{3.642837in}}%
\pgfpathlineto{\pgfqpoint{3.628744in}{3.641987in}}%
\pgfpathlineto{\pgfqpoint{3.662595in}{3.639403in}}%
\pgfpathlineto{\pgfqpoint{3.652631in}{3.643595in}}%
\pgfpathlineto{\pgfqpoint{3.642672in}{3.650926in}}%
\pgfpathlineto{\pgfqpoint{3.608873in}{3.655348in}}%
\pgfpathlineto{\pgfqpoint{3.575057in}{3.658196in}}%
\pgfpathclose%
\pgfusepath{fill}%
\end{pgfscope}%
\begin{pgfscope}%
\pgfpathrectangle{\pgfqpoint{1.020000in}{0.880000in}}{\pgfqpoint{6.160000in}{6.160000in}}%
\pgfusepath{clip}%
\pgfsetbuttcap%
\pgfsetroundjoin%
\definecolor{currentfill}{rgb}{0.777378,0.840921,0.946149}%
\pgfsetfillcolor{currentfill}%
\pgfsetlinewidth{0.000000pt}%
\definecolor{currentstroke}{rgb}{0.000000,0.000000,0.000000}%
\pgfsetstrokecolor{currentstroke}%
\pgfsetdash{}{0pt}%
\pgfpathmoveto{\pgfqpoint{3.129593in}{3.704278in}}%
\pgfpathlineto{\pgfqpoint{3.139332in}{3.668058in}}%
\pgfpathlineto{\pgfqpoint{3.149052in}{3.635278in}}%
\pgfpathlineto{\pgfqpoint{3.182946in}{3.646373in}}%
\pgfpathlineto{\pgfqpoint{3.216833in}{3.656542in}}%
\pgfpathlineto{\pgfqpoint{3.207080in}{3.687821in}}%
\pgfpathlineto{\pgfqpoint{3.197308in}{3.722727in}}%
\pgfpathlineto{\pgfqpoint{3.163453in}{3.714064in}}%
\pgfpathlineto{\pgfqpoint{3.129593in}{3.704278in}}%
\pgfpathclose%
\pgfusepath{fill}%
\end{pgfscope}%
\begin{pgfscope}%
\pgfpathrectangle{\pgfqpoint{1.020000in}{0.880000in}}{\pgfqpoint{6.160000in}{6.160000in}}%
\pgfusepath{clip}%
\pgfsetbuttcap%
\pgfsetroundjoin%
\definecolor{currentfill}{rgb}{0.768034,0.837035,0.952488}%
\pgfsetfillcolor{currentfill}%
\pgfsetlinewidth{0.000000pt}%
\definecolor{currentstroke}{rgb}{0.000000,0.000000,0.000000}%
\pgfsetstrokecolor{currentstroke}%
\pgfsetdash{}{0pt}%
\pgfpathmoveto{\pgfqpoint{3.352308in}{3.685784in}}%
\pgfpathlineto{\pgfqpoint{3.362111in}{3.662270in}}%
\pgfpathlineto{\pgfqpoint{3.371902in}{3.642413in}}%
\pgfpathlineto{\pgfqpoint{3.405786in}{3.648786in}}%
\pgfpathlineto{\pgfqpoint{3.439662in}{3.653735in}}%
\pgfpathlineto{\pgfqpoint{3.429833in}{3.671183in}}%
\pgfpathlineto{\pgfqpoint{3.419995in}{3.692322in}}%
\pgfpathlineto{\pgfqpoint{3.386156in}{3.689783in}}%
\pgfpathlineto{\pgfqpoint{3.352308in}{3.685784in}}%
\pgfpathclose%
\pgfusepath{fill}%
\end{pgfscope}%
\begin{pgfscope}%
\pgfpathrectangle{\pgfqpoint{1.020000in}{0.880000in}}{\pgfqpoint{6.160000in}{6.160000in}}%
\pgfusepath{clip}%
\pgfsetbuttcap%
\pgfsetroundjoin%
\definecolor{currentfill}{rgb}{0.713852,0.808857,0.979386}%
\pgfsetfillcolor{currentfill}%
\pgfsetlinewidth{0.000000pt}%
\definecolor{currentstroke}{rgb}{0.000000,0.000000,0.000000}%
\pgfsetstrokecolor{currentstroke}%
\pgfsetdash{}{0pt}%
\pgfpathmoveto{\pgfqpoint{3.953113in}{3.566963in}}%
\pgfpathlineto{\pgfqpoint{3.963363in}{3.571989in}}%
\pgfpathlineto{\pgfqpoint{3.973633in}{3.578498in}}%
\pgfpathlineto{\pgfqpoint{4.007407in}{3.561918in}}%
\pgfpathlineto{\pgfqpoint{4.041150in}{3.544114in}}%
\pgfpathlineto{\pgfqpoint{4.030813in}{3.538651in}}%
\pgfpathlineto{\pgfqpoint{4.020497in}{3.534512in}}%
\pgfpathlineto{\pgfqpoint{3.986820in}{3.551244in}}%
\pgfpathlineto{\pgfqpoint{3.953113in}{3.566963in}}%
\pgfpathclose%
\pgfusepath{fill}%
\end{pgfscope}%
\begin{pgfscope}%
\pgfpathrectangle{\pgfqpoint{1.020000in}{0.880000in}}{\pgfqpoint{6.160000in}{6.160000in}}%
\pgfusepath{clip}%
\pgfsetbuttcap%
\pgfsetroundjoin%
\definecolor{currentfill}{rgb}{0.570616,0.704109,0.997195}%
\pgfsetfillcolor{currentfill}%
\pgfsetlinewidth{0.000000pt}%
\definecolor{currentstroke}{rgb}{0.000000,0.000000,0.000000}%
\pgfsetstrokecolor{currentstroke}%
\pgfsetdash{}{0pt}%
\pgfpathmoveto{\pgfqpoint{4.419474in}{3.328352in}}%
\pgfpathlineto{\pgfqpoint{4.430226in}{3.325520in}}%
\pgfpathlineto{\pgfqpoint{4.440999in}{3.322359in}}%
\pgfpathlineto{\pgfqpoint{4.474520in}{3.296560in}}%
\pgfpathlineto{\pgfqpoint{4.508006in}{3.271683in}}%
\pgfpathlineto{\pgfqpoint{4.497182in}{3.276941in}}%
\pgfpathlineto{\pgfqpoint{4.486379in}{3.281913in}}%
\pgfpathlineto{\pgfqpoint{4.452944in}{3.304708in}}%
\pgfpathlineto{\pgfqpoint{4.419474in}{3.328352in}}%
\pgfpathclose%
\pgfusepath{fill}%
\end{pgfscope}%
\begin{pgfscope}%
\pgfpathrectangle{\pgfqpoint{1.020000in}{0.880000in}}{\pgfqpoint{6.160000in}{6.160000in}}%
\pgfusepath{clip}%
\pgfsetbuttcap%
\pgfsetroundjoin%
\definecolor{currentfill}{rgb}{0.763363,0.835092,0.955658}%
\pgfsetfillcolor{currentfill}%
\pgfsetlinewidth{0.000000pt}%
\definecolor{currentstroke}{rgb}{0.000000,0.000000,0.000000}%
\pgfsetstrokecolor{currentstroke}%
\pgfsetdash{}{0pt}%
\pgfpathmoveto{\pgfqpoint{2.616798in}{3.675489in}}%
\pgfpathlineto{\pgfqpoint{2.626238in}{3.634347in}}%
\pgfpathlineto{\pgfqpoint{2.635668in}{3.594721in}}%
\pgfpathlineto{\pgfqpoint{2.669642in}{3.608490in}}%
\pgfpathlineto{\pgfqpoint{2.703596in}{3.622752in}}%
\pgfpathlineto{\pgfqpoint{2.694097in}{3.664168in}}%
\pgfpathlineto{\pgfqpoint{2.684586in}{3.707361in}}%
\pgfpathlineto{\pgfqpoint{2.650702in}{3.691193in}}%
\pgfpathlineto{\pgfqpoint{2.616798in}{3.675489in}}%
\pgfpathclose%
\pgfusepath{fill}%
\end{pgfscope}%
\begin{pgfscope}%
\pgfpathrectangle{\pgfqpoint{1.020000in}{0.880000in}}{\pgfqpoint{6.160000in}{6.160000in}}%
\pgfusepath{clip}%
\pgfsetbuttcap%
\pgfsetroundjoin%
\definecolor{currentfill}{rgb}{0.630089,0.752516,0.998508}%
\pgfsetfillcolor{currentfill}%
\pgfsetlinewidth{0.000000pt}%
\definecolor{currentstroke}{rgb}{0.000000,0.000000,0.000000}%
\pgfsetstrokecolor{currentstroke}%
\pgfsetdash{}{0pt}%
\pgfpathmoveto{\pgfqpoint{4.264015in}{3.424803in}}%
\pgfpathlineto{\pgfqpoint{4.274610in}{3.426628in}}%
\pgfpathlineto{\pgfqpoint{4.285229in}{3.428504in}}%
\pgfpathlineto{\pgfqpoint{4.318846in}{3.402972in}}%
\pgfpathlineto{\pgfqpoint{4.352426in}{3.377646in}}%
\pgfpathlineto{\pgfqpoint{4.341751in}{3.378059in}}%
\pgfpathlineto{\pgfqpoint{4.331098in}{3.378499in}}%
\pgfpathlineto{\pgfqpoint{4.297575in}{3.401543in}}%
\pgfpathlineto{\pgfqpoint{4.264015in}{3.424803in}}%
\pgfpathclose%
\pgfusepath{fill}%
\end{pgfscope}%
\begin{pgfscope}%
\pgfpathrectangle{\pgfqpoint{1.020000in}{0.880000in}}{\pgfqpoint{6.160000in}{6.160000in}}%
\pgfusepath{clip}%
\pgfsetbuttcap%
\pgfsetroundjoin%
\definecolor{currentfill}{rgb}{0.343278,0.459354,0.884122}%
\pgfsetfillcolor{currentfill}%
\pgfsetlinewidth{0.000000pt}%
\definecolor{currentstroke}{rgb}{0.000000,0.000000,0.000000}%
\pgfsetstrokecolor{currentstroke}%
\pgfsetdash{}{0pt}%
\pgfpathmoveto{\pgfqpoint{5.196890in}{2.908607in}}%
\pgfpathlineto{\pgfqpoint{5.208328in}{2.892674in}}%
\pgfpathlineto{\pgfqpoint{5.219787in}{2.876622in}}%
\pgfpathlineto{\pgfqpoint{5.253072in}{2.872558in}}%
\pgfpathlineto{\pgfqpoint{5.286341in}{2.869283in}}%
\pgfpathlineto{\pgfqpoint{5.274829in}{2.885295in}}%
\pgfpathlineto{\pgfqpoint{5.263339in}{2.901227in}}%
\pgfpathlineto{\pgfqpoint{5.230123in}{2.904528in}}%
\pgfpathlineto{\pgfqpoint{5.196890in}{2.908607in}}%
\pgfpathclose%
\pgfusepath{fill}%
\end{pgfscope}%
\begin{pgfscope}%
\pgfpathrectangle{\pgfqpoint{1.020000in}{0.880000in}}{\pgfqpoint{6.160000in}{6.160000in}}%
\pgfusepath{clip}%
\pgfsetbuttcap%
\pgfsetroundjoin%
\definecolor{currentfill}{rgb}{0.677823,0.786546,0.991005}%
\pgfsetfillcolor{currentfill}%
\pgfsetlinewidth{0.000000pt}%
\definecolor{currentstroke}{rgb}{0.000000,0.000000,0.000000}%
\pgfsetstrokecolor{currentstroke}%
\pgfsetdash{}{0pt}%
\pgfpathmoveto{\pgfqpoint{4.108536in}{3.505581in}}%
\pgfpathlineto{\pgfqpoint{4.118961in}{3.510543in}}%
\pgfpathlineto{\pgfqpoint{4.129409in}{3.516178in}}%
\pgfpathlineto{\pgfqpoint{4.163115in}{3.493993in}}%
\pgfpathlineto{\pgfqpoint{4.196785in}{3.471228in}}%
\pgfpathlineto{\pgfqpoint{4.186274in}{3.467543in}}%
\pgfpathlineto{\pgfqpoint{4.175786in}{3.464432in}}%
\pgfpathlineto{\pgfqpoint{4.142179in}{3.485238in}}%
\pgfpathlineto{\pgfqpoint{4.108536in}{3.505581in}}%
\pgfpathclose%
\pgfusepath{fill}%
\end{pgfscope}%
\begin{pgfscope}%
\pgfpathrectangle{\pgfqpoint{1.020000in}{0.880000in}}{\pgfqpoint{6.160000in}{6.160000in}}%
\pgfusepath{clip}%
\pgfsetbuttcap%
\pgfsetroundjoin%
\definecolor{currentfill}{rgb}{0.763363,0.835092,0.955658}%
\pgfsetfillcolor{currentfill}%
\pgfsetlinewidth{0.000000pt}%
\definecolor{currentstroke}{rgb}{0.000000,0.000000,0.000000}%
\pgfsetstrokecolor{currentstroke}%
\pgfsetdash{}{0pt}%
\pgfpathmoveto{\pgfqpoint{2.839241in}{3.682545in}}%
\pgfpathlineto{\pgfqpoint{2.848846in}{3.640374in}}%
\pgfpathlineto{\pgfqpoint{2.858434in}{3.600627in}}%
\pgfpathlineto{\pgfqpoint{2.892364in}{3.614811in}}%
\pgfpathlineto{\pgfqpoint{2.926282in}{3.628910in}}%
\pgfpathlineto{\pgfqpoint{2.916644in}{3.669249in}}%
\pgfpathlineto{\pgfqpoint{2.906986in}{3.712301in}}%
\pgfpathlineto{\pgfqpoint{2.873118in}{3.697551in}}%
\pgfpathlineto{\pgfqpoint{2.839241in}{3.682545in}}%
\pgfpathclose%
\pgfusepath{fill}%
\end{pgfscope}%
\begin{pgfscope}%
\pgfpathrectangle{\pgfqpoint{1.020000in}{0.880000in}}{\pgfqpoint{6.160000in}{6.160000in}}%
\pgfusepath{clip}%
\pgfsetbuttcap%
\pgfsetroundjoin%
\definecolor{currentfill}{rgb}{0.285273,0.380129,0.823469}%
\pgfsetfillcolor{currentfill}%
\pgfsetlinewidth{0.000000pt}%
\definecolor{currentstroke}{rgb}{0.000000,0.000000,0.000000}%
\pgfsetstrokecolor{currentstroke}%
\pgfsetdash{}{0pt}%
\pgfpathmoveto{\pgfqpoint{6.020735in}{2.790930in}}%
\pgfpathlineto{\pgfqpoint{6.032952in}{2.776012in}}%
\pgfpathlineto{\pgfqpoint{6.045192in}{2.761143in}}%
\pgfpathlineto{\pgfqpoint{6.078289in}{2.763510in}}%
\pgfpathlineto{\pgfqpoint{6.111365in}{2.765903in}}%
\pgfpathlineto{\pgfqpoint{6.099069in}{2.780688in}}%
\pgfpathlineto{\pgfqpoint{6.086797in}{2.795521in}}%
\pgfpathlineto{\pgfqpoint{6.053776in}{2.793212in}}%
\pgfpathlineto{\pgfqpoint{6.020735in}{2.790930in}}%
\pgfpathclose%
\pgfusepath{fill}%
\end{pgfscope}%
\begin{pgfscope}%
\pgfpathrectangle{\pgfqpoint{1.020000in}{0.880000in}}{\pgfqpoint{6.160000in}{6.160000in}}%
\pgfusepath{clip}%
\pgfsetbuttcap%
\pgfsetroundjoin%
\definecolor{currentfill}{rgb}{0.294718,0.393542,0.834384}%
\pgfsetfillcolor{currentfill}%
\pgfsetlinewidth{0.000000pt}%
\definecolor{currentstroke}{rgb}{0.000000,0.000000,0.000000}%
\pgfsetstrokecolor{currentstroke}%
\pgfsetdash{}{0pt}%
\pgfpathmoveto{\pgfqpoint{5.798038in}{2.808705in}}%
\pgfpathlineto{\pgfqpoint{5.810042in}{2.793410in}}%
\pgfpathlineto{\pgfqpoint{5.822069in}{2.778162in}}%
\pgfpathlineto{\pgfqpoint{5.855230in}{2.780138in}}%
\pgfpathlineto{\pgfqpoint{5.888371in}{2.782189in}}%
\pgfpathlineto{\pgfqpoint{5.876288in}{2.797338in}}%
\pgfpathlineto{\pgfqpoint{5.864229in}{2.812537in}}%
\pgfpathlineto{\pgfqpoint{5.831144in}{2.810582in}}%
\pgfpathlineto{\pgfqpoint{5.798038in}{2.808705in}}%
\pgfpathclose%
\pgfusepath{fill}%
\end{pgfscope}%
\begin{pgfscope}%
\pgfpathrectangle{\pgfqpoint{1.020000in}{0.880000in}}{\pgfqpoint{6.160000in}{6.160000in}}%
\pgfusepath{clip}%
\pgfsetbuttcap%
\pgfsetroundjoin%
\definecolor{currentfill}{rgb}{0.768034,0.837035,0.952488}%
\pgfsetfillcolor{currentfill}%
\pgfsetlinewidth{0.000000pt}%
\definecolor{currentstroke}{rgb}{0.000000,0.000000,0.000000}%
\pgfsetstrokecolor{currentstroke}%
\pgfsetdash{}{0pt}%
\pgfpathmoveto{\pgfqpoint{3.061853in}{3.681809in}}%
\pgfpathlineto{\pgfqpoint{3.071557in}{3.644700in}}%
\pgfpathlineto{\pgfqpoint{3.081241in}{3.610806in}}%
\pgfpathlineto{\pgfqpoint{3.115150in}{3.623380in}}%
\pgfpathlineto{\pgfqpoint{3.149052in}{3.635278in}}%
\pgfpathlineto{\pgfqpoint{3.139332in}{3.668058in}}%
\pgfpathlineto{\pgfqpoint{3.129593in}{3.704278in}}%
\pgfpathlineto{\pgfqpoint{3.095726in}{3.693485in}}%
\pgfpathlineto{\pgfqpoint{3.061853in}{3.681809in}}%
\pgfpathclose%
\pgfusepath{fill}%
\end{pgfscope}%
\begin{pgfscope}%
\pgfpathrectangle{\pgfqpoint{1.020000in}{0.880000in}}{\pgfqpoint{6.160000in}{6.160000in}}%
\pgfusepath{clip}%
\pgfsetbuttcap%
\pgfsetroundjoin%
\definecolor{currentfill}{rgb}{0.304174,0.406945,0.845263}%
\pgfsetfillcolor{currentfill}%
\pgfsetlinewidth{0.000000pt}%
\definecolor{currentstroke}{rgb}{0.000000,0.000000,0.000000}%
\pgfsetstrokecolor{currentstroke}%
\pgfsetdash{}{0pt}%
\pgfpathmoveto{\pgfqpoint{5.575392in}{2.831418in}}%
\pgfpathlineto{\pgfqpoint{5.587185in}{2.815728in}}%
\pgfpathlineto{\pgfqpoint{5.599001in}{2.800067in}}%
\pgfpathlineto{\pgfqpoint{5.632221in}{2.801078in}}%
\pgfpathlineto{\pgfqpoint{5.665423in}{2.802298in}}%
\pgfpathlineto{\pgfqpoint{5.653552in}{2.817845in}}%
\pgfpathlineto{\pgfqpoint{5.641704in}{2.833429in}}%
\pgfpathlineto{\pgfqpoint{5.608557in}{2.832318in}}%
\pgfpathlineto{\pgfqpoint{5.575392in}{2.831418in}}%
\pgfpathclose%
\pgfusepath{fill}%
\end{pgfscope}%
\begin{pgfscope}%
\pgfpathrectangle{\pgfqpoint{1.020000in}{0.880000in}}{\pgfqpoint{6.160000in}{6.160000in}}%
\pgfusepath{clip}%
\pgfsetbuttcap%
\pgfsetroundjoin%
\definecolor{currentfill}{rgb}{0.748682,0.827679,0.963334}%
\pgfsetfillcolor{currentfill}%
\pgfsetlinewidth{0.000000pt}%
\definecolor{currentstroke}{rgb}{0.000000,0.000000,0.000000}%
\pgfsetstrokecolor{currentstroke}%
\pgfsetdash{}{0pt}%
\pgfpathmoveto{\pgfqpoint{2.548926in}{3.645900in}}%
\pgfpathlineto{\pgfqpoint{2.558291in}{3.606814in}}%
\pgfpathlineto{\pgfqpoint{2.567650in}{3.569001in}}%
\pgfpathlineto{\pgfqpoint{2.601671in}{3.581532in}}%
\pgfpathlineto{\pgfqpoint{2.635668in}{3.594721in}}%
\pgfpathlineto{\pgfqpoint{2.626238in}{3.634347in}}%
\pgfpathlineto{\pgfqpoint{2.616798in}{3.675489in}}%
\pgfpathlineto{\pgfqpoint{2.582873in}{3.660358in}}%
\pgfpathlineto{\pgfqpoint{2.548926in}{3.645900in}}%
\pgfpathclose%
\pgfusepath{fill}%
\end{pgfscope}%
\begin{pgfscope}%
\pgfpathrectangle{\pgfqpoint{1.020000in}{0.880000in}}{\pgfqpoint{6.160000in}{6.160000in}}%
\pgfusepath{clip}%
\pgfsetbuttcap%
\pgfsetroundjoin%
\definecolor{currentfill}{rgb}{0.748682,0.827679,0.963334}%
\pgfsetfillcolor{currentfill}%
\pgfsetlinewidth{0.000000pt}%
\definecolor{currentstroke}{rgb}{0.000000,0.000000,0.000000}%
\pgfsetstrokecolor{currentstroke}%
\pgfsetdash{}{0pt}%
\pgfpathmoveto{\pgfqpoint{3.730246in}{3.629122in}}%
\pgfpathlineto{\pgfqpoint{3.740274in}{3.629187in}}%
\pgfpathlineto{\pgfqpoint{3.750313in}{3.631768in}}%
\pgfpathlineto{\pgfqpoint{3.784171in}{3.625098in}}%
\pgfpathlineto{\pgfqpoint{3.818008in}{3.616637in}}%
\pgfpathlineto{\pgfqpoint{3.807906in}{3.613327in}}%
\pgfpathlineto{\pgfqpoint{3.797817in}{3.612374in}}%
\pgfpathlineto{\pgfqpoint{3.764042in}{3.621521in}}%
\pgfpathlineto{\pgfqpoint{3.730246in}{3.629122in}}%
\pgfpathclose%
\pgfusepath{fill}%
\end{pgfscope}%
\begin{pgfscope}%
\pgfpathrectangle{\pgfqpoint{1.020000in}{0.880000in}}{\pgfqpoint{6.160000in}{6.160000in}}%
\pgfusepath{clip}%
\pgfsetbuttcap%
\pgfsetroundjoin%
\definecolor{currentfill}{rgb}{0.768034,0.837035,0.952488}%
\pgfsetfillcolor{currentfill}%
\pgfsetlinewidth{0.000000pt}%
\definecolor{currentstroke}{rgb}{0.000000,0.000000,0.000000}%
\pgfsetstrokecolor{currentstroke}%
\pgfsetdash{}{0pt}%
\pgfpathmoveto{\pgfqpoint{3.284585in}{3.673649in}}%
\pgfpathlineto{\pgfqpoint{3.294355in}{3.647909in}}%
\pgfpathlineto{\pgfqpoint{3.304111in}{3.625738in}}%
\pgfpathlineto{\pgfqpoint{3.338010in}{3.634698in}}%
\pgfpathlineto{\pgfqpoint{3.371902in}{3.642413in}}%
\pgfpathlineto{\pgfqpoint{3.362111in}{3.662270in}}%
\pgfpathlineto{\pgfqpoint{3.352308in}{3.685784in}}%
\pgfpathlineto{\pgfqpoint{3.318450in}{3.680382in}}%
\pgfpathlineto{\pgfqpoint{3.284585in}{3.673649in}}%
\pgfpathclose%
\pgfusepath{fill}%
\end{pgfscope}%
\begin{pgfscope}%
\pgfpathrectangle{\pgfqpoint{1.020000in}{0.880000in}}{\pgfqpoint{6.160000in}{6.160000in}}%
\pgfusepath{clip}%
\pgfsetbuttcap%
\pgfsetroundjoin%
\definecolor{currentfill}{rgb}{0.323718,0.433158,0.864722}%
\pgfsetfillcolor{currentfill}%
\pgfsetlinewidth{0.000000pt}%
\definecolor{currentstroke}{rgb}{0.000000,0.000000,0.000000}%
\pgfsetstrokecolor{currentstroke}%
\pgfsetdash{}{0pt}%
\pgfpathmoveto{\pgfqpoint{5.352828in}{2.864783in}}%
\pgfpathlineto{\pgfqpoint{5.364414in}{2.848784in}}%
\pgfpathlineto{\pgfqpoint{5.376021in}{2.832752in}}%
\pgfpathlineto{\pgfqpoint{5.409294in}{2.831480in}}%
\pgfpathlineto{\pgfqpoint{5.442549in}{2.830702in}}%
\pgfpathlineto{\pgfqpoint{5.430887in}{2.846624in}}%
\pgfpathlineto{\pgfqpoint{5.419247in}{2.862536in}}%
\pgfpathlineto{\pgfqpoint{5.386046in}{2.863411in}}%
\pgfpathlineto{\pgfqpoint{5.352828in}{2.864783in}}%
\pgfpathclose%
\pgfusepath{fill}%
\end{pgfscope}%
\begin{pgfscope}%
\pgfpathrectangle{\pgfqpoint{1.020000in}{0.880000in}}{\pgfqpoint{6.160000in}{6.160000in}}%
\pgfusepath{clip}%
\pgfsetbuttcap%
\pgfsetroundjoin%
\definecolor{currentfill}{rgb}{0.763363,0.835092,0.955658}%
\pgfsetfillcolor{currentfill}%
\pgfsetlinewidth{0.000000pt}%
\definecolor{currentstroke}{rgb}{0.000000,0.000000,0.000000}%
\pgfsetstrokecolor{currentstroke}%
\pgfsetdash{}{0pt}%
\pgfpathmoveto{\pgfqpoint{3.507383in}{3.659111in}}%
\pgfpathlineto{\pgfqpoint{3.517248in}{3.647580in}}%
\pgfpathlineto{\pgfqpoint{3.527111in}{3.639360in}}%
\pgfpathlineto{\pgfqpoint{3.561001in}{3.641953in}}%
\pgfpathlineto{\pgfqpoint{3.594879in}{3.642837in}}%
\pgfpathlineto{\pgfqpoint{3.584967in}{3.648893in}}%
\pgfpathlineto{\pgfqpoint{3.575057in}{3.658196in}}%
\pgfpathlineto{\pgfqpoint{3.541227in}{3.659450in}}%
\pgfpathlineto{\pgfqpoint{3.507383in}{3.659111in}}%
\pgfpathclose%
\pgfusepath{fill}%
\end{pgfscope}%
\begin{pgfscope}%
\pgfpathrectangle{\pgfqpoint{1.020000in}{0.880000in}}{\pgfqpoint{6.160000in}{6.160000in}}%
\pgfusepath{clip}%
\pgfsetbuttcap%
\pgfsetroundjoin%
\definecolor{currentfill}{rgb}{0.425199,0.559058,0.946061}%
\pgfsetfillcolor{currentfill}%
\pgfsetlinewidth{0.000000pt}%
\definecolor{currentstroke}{rgb}{0.000000,0.000000,0.000000}%
\pgfsetstrokecolor{currentstroke}%
\pgfsetdash{}{0pt}%
\pgfpathmoveto{\pgfqpoint{4.819034in}{3.066991in}}%
\pgfpathlineto{\pgfqpoint{4.830146in}{3.053757in}}%
\pgfpathlineto{\pgfqpoint{4.841277in}{3.040061in}}%
\pgfpathlineto{\pgfqpoint{4.874645in}{3.023965in}}%
\pgfpathlineto{\pgfqpoint{4.907992in}{3.009414in}}%
\pgfpathlineto{\pgfqpoint{4.896813in}{3.023716in}}%
\pgfpathlineto{\pgfqpoint{4.885653in}{3.037648in}}%
\pgfpathlineto{\pgfqpoint{4.852355in}{3.051586in}}%
\pgfpathlineto{\pgfqpoint{4.819034in}{3.066991in}}%
\pgfpathclose%
\pgfusepath{fill}%
\end{pgfscope}%
\begin{pgfscope}%
\pgfpathrectangle{\pgfqpoint{1.020000in}{0.880000in}}{\pgfqpoint{6.160000in}{6.160000in}}%
\pgfusepath{clip}%
\pgfsetbuttcap%
\pgfsetroundjoin%
\definecolor{currentfill}{rgb}{0.753611,0.830233,0.960871}%
\pgfsetfillcolor{currentfill}%
\pgfsetlinewidth{0.000000pt}%
\definecolor{currentstroke}{rgb}{0.000000,0.000000,0.000000}%
\pgfsetstrokecolor{currentstroke}%
\pgfsetdash{}{0pt}%
\pgfpathmoveto{\pgfqpoint{2.771449in}{3.652337in}}%
\pgfpathlineto{\pgfqpoint{2.780997in}{3.611350in}}%
\pgfpathlineto{\pgfqpoint{2.790529in}{3.572496in}}%
\pgfpathlineto{\pgfqpoint{2.824489in}{3.586483in}}%
\pgfpathlineto{\pgfqpoint{2.858434in}{3.600627in}}%
\pgfpathlineto{\pgfqpoint{2.848846in}{3.640374in}}%
\pgfpathlineto{\pgfqpoint{2.839241in}{3.682545in}}%
\pgfpathlineto{\pgfqpoint{2.805351in}{3.667428in}}%
\pgfpathlineto{\pgfqpoint{2.771449in}{3.652337in}}%
\pgfpathclose%
\pgfusepath{fill}%
\end{pgfscope}%
\begin{pgfscope}%
\pgfpathrectangle{\pgfqpoint{1.020000in}{0.880000in}}{\pgfqpoint{6.160000in}{6.160000in}}%
\pgfusepath{clip}%
\pgfsetbuttcap%
\pgfsetroundjoin%
\definecolor{currentfill}{rgb}{0.383662,0.510183,0.917831}%
\pgfsetfillcolor{currentfill}%
\pgfsetlinewidth{0.000000pt}%
\definecolor{currentstroke}{rgb}{0.000000,0.000000,0.000000}%
\pgfsetstrokecolor{currentstroke}%
\pgfsetdash{}{0pt}%
\pgfpathmoveto{\pgfqpoint{4.974624in}{2.984702in}}%
\pgfpathlineto{\pgfqpoint{4.985870in}{2.969629in}}%
\pgfpathlineto{\pgfqpoint{4.997135in}{2.954244in}}%
\pgfpathlineto{\pgfqpoint{5.030471in}{2.943675in}}%
\pgfpathlineto{\pgfqpoint{5.063790in}{2.934409in}}%
\pgfpathlineto{\pgfqpoint{5.052475in}{2.950000in}}%
\pgfpathlineto{\pgfqpoint{5.041178in}{2.965351in}}%
\pgfpathlineto{\pgfqpoint{5.007910in}{2.974396in}}%
\pgfpathlineto{\pgfqpoint{4.974624in}{2.984702in}}%
\pgfpathclose%
\pgfusepath{fill}%
\end{pgfscope}%
\begin{pgfscope}%
\pgfpathrectangle{\pgfqpoint{1.020000in}{0.880000in}}{\pgfqpoint{6.160000in}{6.160000in}}%
\pgfusepath{clip}%
\pgfsetbuttcap%
\pgfsetroundjoin%
\definecolor{currentfill}{rgb}{0.478462,0.616564,0.972721}%
\pgfsetfillcolor{currentfill}%
\pgfsetlinewidth{0.000000pt}%
\definecolor{currentstroke}{rgb}{0.000000,0.000000,0.000000}%
\pgfsetstrokecolor{currentstroke}%
\pgfsetdash{}{0pt}%
\pgfpathmoveto{\pgfqpoint{4.663524in}{3.164724in}}%
\pgfpathlineto{\pgfqpoint{4.674504in}{3.154627in}}%
\pgfpathlineto{\pgfqpoint{4.685504in}{3.143957in}}%
\pgfpathlineto{\pgfqpoint{4.718926in}{3.122384in}}%
\pgfpathlineto{\pgfqpoint{4.752321in}{3.102367in}}%
\pgfpathlineto{\pgfqpoint{4.741275in}{3.114204in}}%
\pgfpathlineto{\pgfqpoint{4.730247in}{3.125571in}}%
\pgfpathlineto{\pgfqpoint{4.696899in}{3.144421in}}%
\pgfpathlineto{\pgfqpoint{4.663524in}{3.164724in}}%
\pgfpathclose%
\pgfusepath{fill}%
\end{pgfscope}%
\begin{pgfscope}%
\pgfpathrectangle{\pgfqpoint{1.020000in}{0.880000in}}{\pgfqpoint{6.160000in}{6.160000in}}%
\pgfusepath{clip}%
\pgfsetbuttcap%
\pgfsetroundjoin%
\definecolor{currentfill}{rgb}{0.758539,0.832787,0.958408}%
\pgfsetfillcolor{currentfill}%
\pgfsetlinewidth{0.000000pt}%
\definecolor{currentstroke}{rgb}{0.000000,0.000000,0.000000}%
\pgfsetstrokecolor{currentstroke}%
\pgfsetdash{}{0pt}%
\pgfpathmoveto{\pgfqpoint{2.994085in}{3.656328in}}%
\pgfpathlineto{\pgfqpoint{3.003749in}{3.618759in}}%
\pgfpathlineto{\pgfqpoint{3.013394in}{3.584148in}}%
\pgfpathlineto{\pgfqpoint{3.047322in}{3.597685in}}%
\pgfpathlineto{\pgfqpoint{3.081241in}{3.610806in}}%
\pgfpathlineto{\pgfqpoint{3.071557in}{3.644700in}}%
\pgfpathlineto{\pgfqpoint{3.061853in}{3.681809in}}%
\pgfpathlineto{\pgfqpoint{3.027973in}{3.669378in}}%
\pgfpathlineto{\pgfqpoint{2.994085in}{3.656328in}}%
\pgfpathclose%
\pgfusepath{fill}%
\end{pgfscope}%
\begin{pgfscope}%
\pgfpathrectangle{\pgfqpoint{1.020000in}{0.880000in}}{\pgfqpoint{6.160000in}{6.160000in}}%
\pgfusepath{clip}%
\pgfsetbuttcap%
\pgfsetroundjoin%
\definecolor{currentfill}{rgb}{0.738826,0.822572,0.968261}%
\pgfsetfillcolor{currentfill}%
\pgfsetlinewidth{0.000000pt}%
\definecolor{currentstroke}{rgb}{0.000000,0.000000,0.000000}%
\pgfsetstrokecolor{currentstroke}%
\pgfsetdash{}{0pt}%
\pgfpathmoveto{\pgfqpoint{2.480952in}{3.619316in}}%
\pgfpathlineto{\pgfqpoint{2.490243in}{3.582208in}}%
\pgfpathlineto{\pgfqpoint{2.499530in}{3.546152in}}%
\pgfpathlineto{\pgfqpoint{2.533604in}{3.557190in}}%
\pgfpathlineto{\pgfqpoint{2.567650in}{3.569001in}}%
\pgfpathlineto{\pgfqpoint{2.558291in}{3.606814in}}%
\pgfpathlineto{\pgfqpoint{2.548926in}{3.645900in}}%
\pgfpathlineto{\pgfqpoint{2.514952in}{3.632196in}}%
\pgfpathlineto{\pgfqpoint{2.480952in}{3.619316in}}%
\pgfpathclose%
\pgfusepath{fill}%
\end{pgfscope}%
\begin{pgfscope}%
\pgfpathrectangle{\pgfqpoint{1.020000in}{0.880000in}}{\pgfqpoint{6.160000in}{6.160000in}}%
\pgfusepath{clip}%
\pgfsetbuttcap%
\pgfsetroundjoin%
\definecolor{currentfill}{rgb}{0.353369,0.472069,0.892570}%
\pgfsetfillcolor{currentfill}%
\pgfsetlinewidth{0.000000pt}%
\definecolor{currentstroke}{rgb}{0.000000,0.000000,0.000000}%
\pgfsetstrokecolor{currentstroke}%
\pgfsetdash{}{0pt}%
\pgfpathmoveto{\pgfqpoint{5.130374in}{2.919435in}}%
\pgfpathlineto{\pgfqpoint{5.141760in}{2.903534in}}%
\pgfpathlineto{\pgfqpoint{5.153166in}{2.887461in}}%
\pgfpathlineto{\pgfqpoint{5.186485in}{2.881560in}}%
\pgfpathlineto{\pgfqpoint{5.219787in}{2.876622in}}%
\pgfpathlineto{\pgfqpoint{5.208328in}{2.892674in}}%
\pgfpathlineto{\pgfqpoint{5.196890in}{2.908607in}}%
\pgfpathlineto{\pgfqpoint{5.163641in}{2.913548in}}%
\pgfpathlineto{\pgfqpoint{5.130374in}{2.919435in}}%
\pgfpathclose%
\pgfusepath{fill}%
\end{pgfscope}%
\begin{pgfscope}%
\pgfpathrectangle{\pgfqpoint{1.020000in}{0.880000in}}{\pgfqpoint{6.160000in}{6.160000in}}%
\pgfusepath{clip}%
\pgfsetbuttcap%
\pgfsetroundjoin%
\definecolor{currentfill}{rgb}{0.543440,0.680003,0.993051}%
\pgfsetfillcolor{currentfill}%
\pgfsetlinewidth{0.000000pt}%
\definecolor{currentstroke}{rgb}{0.000000,0.000000,0.000000}%
\pgfsetstrokecolor{currentstroke}%
\pgfsetdash{}{0pt}%
\pgfpathmoveto{\pgfqpoint{4.508006in}{3.271683in}}%
\pgfpathlineto{\pgfqpoint{4.518851in}{3.265995in}}%
\pgfpathlineto{\pgfqpoint{4.529716in}{3.259739in}}%
\pgfpathlineto{\pgfqpoint{4.563217in}{3.234043in}}%
\pgfpathlineto{\pgfqpoint{4.596684in}{3.209588in}}%
\pgfpathlineto{\pgfqpoint{4.585771in}{3.217618in}}%
\pgfpathlineto{\pgfqpoint{4.574877in}{3.225169in}}%
\pgfpathlineto{\pgfqpoint{4.541458in}{3.247852in}}%
\pgfpathlineto{\pgfqpoint{4.508006in}{3.271683in}}%
\pgfpathclose%
\pgfusepath{fill}%
\end{pgfscope}%
\begin{pgfscope}%
\pgfpathrectangle{\pgfqpoint{1.020000in}{0.880000in}}{\pgfqpoint{6.160000in}{6.160000in}}%
\pgfusepath{clip}%
\pgfsetbuttcap%
\pgfsetroundjoin%
\definecolor{currentfill}{rgb}{0.733898,0.820018,0.970724}%
\pgfsetfillcolor{currentfill}%
\pgfsetlinewidth{0.000000pt}%
\definecolor{currentstroke}{rgb}{0.000000,0.000000,0.000000}%
\pgfsetstrokecolor{currentstroke}%
\pgfsetdash{}{0pt}%
\pgfpathmoveto{\pgfqpoint{3.885613in}{3.594725in}}%
\pgfpathlineto{\pgfqpoint{3.895796in}{3.600185in}}%
\pgfpathlineto{\pgfqpoint{3.905998in}{3.607284in}}%
\pgfpathlineto{\pgfqpoint{3.939830in}{3.593675in}}%
\pgfpathlineto{\pgfqpoint{3.973633in}{3.578498in}}%
\pgfpathlineto{\pgfqpoint{3.963363in}{3.571989in}}%
\pgfpathlineto{\pgfqpoint{3.953113in}{3.566963in}}%
\pgfpathlineto{\pgfqpoint{3.919377in}{3.581507in}}%
\pgfpathlineto{\pgfqpoint{3.885613in}{3.594725in}}%
\pgfpathclose%
\pgfusepath{fill}%
\end{pgfscope}%
\begin{pgfscope}%
\pgfpathrectangle{\pgfqpoint{1.020000in}{0.880000in}}{\pgfqpoint{6.160000in}{6.160000in}}%
\pgfusepath{clip}%
\pgfsetbuttcap%
\pgfsetroundjoin%
\definecolor{currentfill}{rgb}{0.289996,0.386836,0.828926}%
\pgfsetfillcolor{currentfill}%
\pgfsetlinewidth{0.000000pt}%
\definecolor{currentstroke}{rgb}{0.000000,0.000000,0.000000}%
\pgfsetstrokecolor{currentstroke}%
\pgfsetdash{}{0pt}%
\pgfpathmoveto{\pgfqpoint{5.954593in}{2.786469in}}%
\pgfpathlineto{\pgfqpoint{5.966754in}{2.771462in}}%
\pgfpathlineto{\pgfqpoint{5.978940in}{2.756505in}}%
\pgfpathlineto{\pgfqpoint{6.012076in}{2.758806in}}%
\pgfpathlineto{\pgfqpoint{6.045192in}{2.761143in}}%
\pgfpathlineto{\pgfqpoint{6.032952in}{2.776012in}}%
\pgfpathlineto{\pgfqpoint{6.020735in}{2.790930in}}%
\pgfpathlineto{\pgfqpoint{5.987674in}{2.788681in}}%
\pgfpathlineto{\pgfqpoint{5.954593in}{2.786469in}}%
\pgfpathclose%
\pgfusepath{fill}%
\end{pgfscope}%
\begin{pgfscope}%
\pgfpathrectangle{\pgfqpoint{1.020000in}{0.880000in}}{\pgfqpoint{6.160000in}{6.160000in}}%
\pgfusepath{clip}%
\pgfsetbuttcap%
\pgfsetroundjoin%
\definecolor{currentfill}{rgb}{0.294718,0.393542,0.834384}%
\pgfsetfillcolor{currentfill}%
\pgfsetlinewidth{0.000000pt}%
\definecolor{currentstroke}{rgb}{0.000000,0.000000,0.000000}%
\pgfsetstrokecolor{currentstroke}%
\pgfsetdash{}{0pt}%
\pgfpathmoveto{\pgfqpoint{5.731769in}{2.805243in}}%
\pgfpathlineto{\pgfqpoint{5.743718in}{2.789845in}}%
\pgfpathlineto{\pgfqpoint{5.755690in}{2.774492in}}%
\pgfpathlineto{\pgfqpoint{5.788889in}{2.776275in}}%
\pgfpathlineto{\pgfqpoint{5.822069in}{2.778162in}}%
\pgfpathlineto{\pgfqpoint{5.810042in}{2.793410in}}%
\pgfpathlineto{\pgfqpoint{5.798038in}{2.808705in}}%
\pgfpathlineto{\pgfqpoint{5.764913in}{2.806920in}}%
\pgfpathlineto{\pgfqpoint{5.731769in}{2.805243in}}%
\pgfpathclose%
\pgfusepath{fill}%
\end{pgfscope}%
\begin{pgfscope}%
\pgfpathrectangle{\pgfqpoint{1.020000in}{0.880000in}}{\pgfqpoint{6.160000in}{6.160000in}}%
\pgfusepath{clip}%
\pgfsetbuttcap%
\pgfsetroundjoin%
\definecolor{currentfill}{rgb}{0.763363,0.835092,0.955658}%
\pgfsetfillcolor{currentfill}%
\pgfsetlinewidth{0.000000pt}%
\definecolor{currentstroke}{rgb}{0.000000,0.000000,0.000000}%
\pgfsetstrokecolor{currentstroke}%
\pgfsetdash{}{0pt}%
\pgfpathmoveto{\pgfqpoint{3.216833in}{3.656542in}}%
\pgfpathlineto{\pgfqpoint{3.226569in}{3.628812in}}%
\pgfpathlineto{\pgfqpoint{3.236290in}{3.604515in}}%
\pgfpathlineto{\pgfqpoint{3.270204in}{3.615638in}}%
\pgfpathlineto{\pgfqpoint{3.304111in}{3.625738in}}%
\pgfpathlineto{\pgfqpoint{3.294355in}{3.647909in}}%
\pgfpathlineto{\pgfqpoint{3.284585in}{3.673649in}}%
\pgfpathlineto{\pgfqpoint{3.250712in}{3.665670in}}%
\pgfpathlineto{\pgfqpoint{3.216833in}{3.656542in}}%
\pgfpathclose%
\pgfusepath{fill}%
\end{pgfscope}%
\begin{pgfscope}%
\pgfpathrectangle{\pgfqpoint{1.020000in}{0.880000in}}{\pgfqpoint{6.160000in}{6.160000in}}%
\pgfusepath{clip}%
\pgfsetbuttcap%
\pgfsetroundjoin%
\definecolor{currentfill}{rgb}{0.738826,0.822572,0.968261}%
\pgfsetfillcolor{currentfill}%
\pgfsetlinewidth{0.000000pt}%
\definecolor{currentstroke}{rgb}{0.000000,0.000000,0.000000}%
\pgfsetstrokecolor{currentstroke}%
\pgfsetdash{}{0pt}%
\pgfpathmoveto{\pgfqpoint{2.703596in}{3.622752in}}%
\pgfpathlineto{\pgfqpoint{2.713082in}{3.583162in}}%
\pgfpathlineto{\pgfqpoint{2.722555in}{3.545422in}}%
\pgfpathlineto{\pgfqpoint{2.756552in}{3.558776in}}%
\pgfpathlineto{\pgfqpoint{2.790529in}{3.572496in}}%
\pgfpathlineto{\pgfqpoint{2.780997in}{3.611350in}}%
\pgfpathlineto{\pgfqpoint{2.771449in}{3.652337in}}%
\pgfpathlineto{\pgfqpoint{2.737531in}{3.637405in}}%
\pgfpathlineto{\pgfqpoint{2.703596in}{3.622752in}}%
\pgfpathclose%
\pgfusepath{fill}%
\end{pgfscope}%
\begin{pgfscope}%
\pgfpathrectangle{\pgfqpoint{1.020000in}{0.880000in}}{\pgfqpoint{6.160000in}{6.160000in}}%
\pgfusepath{clip}%
\pgfsetbuttcap%
\pgfsetroundjoin%
\definecolor{currentfill}{rgb}{0.309060,0.413498,0.850128}%
\pgfsetfillcolor{currentfill}%
\pgfsetlinewidth{0.000000pt}%
\definecolor{currentstroke}{rgb}{0.000000,0.000000,0.000000}%
\pgfsetstrokecolor{currentstroke}%
\pgfsetdash{}{0pt}%
\pgfpathmoveto{\pgfqpoint{5.509006in}{2.830397in}}%
\pgfpathlineto{\pgfqpoint{5.520744in}{2.814594in}}%
\pgfpathlineto{\pgfqpoint{5.532505in}{2.798808in}}%
\pgfpathlineto{\pgfqpoint{5.565762in}{2.799297in}}%
\pgfpathlineto{\pgfqpoint{5.599001in}{2.800067in}}%
\pgfpathlineto{\pgfqpoint{5.587185in}{2.815728in}}%
\pgfpathlineto{\pgfqpoint{5.575392in}{2.831418in}}%
\pgfpathlineto{\pgfqpoint{5.542208in}{2.830764in}}%
\pgfpathlineto{\pgfqpoint{5.509006in}{2.830397in}}%
\pgfpathclose%
\pgfusepath{fill}%
\end{pgfscope}%
\begin{pgfscope}%
\pgfpathrectangle{\pgfqpoint{1.020000in}{0.880000in}}{\pgfqpoint{6.160000in}{6.160000in}}%
\pgfusepath{clip}%
\pgfsetbuttcap%
\pgfsetroundjoin%
\definecolor{currentfill}{rgb}{0.280550,0.373423,0.818011}%
\pgfsetfillcolor{currentfill}%
\pgfsetlinewidth{0.000000pt}%
\definecolor{currentstroke}{rgb}{0.000000,0.000000,0.000000}%
\pgfsetstrokecolor{currentstroke}%
\pgfsetdash{}{0pt}%
\pgfpathmoveto{\pgfqpoint{6.177456in}{2.770753in}}%
\pgfpathlineto{\pgfqpoint{6.189832in}{2.756098in}}%
\pgfpathlineto{\pgfqpoint{6.202232in}{2.741491in}}%
\pgfpathlineto{\pgfqpoint{6.235302in}{2.744020in}}%
\pgfpathlineto{\pgfqpoint{6.222874in}{2.758588in}}%
\pgfpathlineto{\pgfqpoint{6.210471in}{2.773203in}}%
\pgfpathlineto{\pgfqpoint{6.177456in}{2.770753in}}%
\pgfpathclose%
\pgfusepath{fill}%
\end{pgfscope}%
\begin{pgfscope}%
\pgfpathrectangle{\pgfqpoint{1.020000in}{0.880000in}}{\pgfqpoint{6.160000in}{6.160000in}}%
\pgfusepath{clip}%
\pgfsetbuttcap%
\pgfsetroundjoin%
\definecolor{currentfill}{rgb}{0.603162,0.731527,0.999565}%
\pgfsetfillcolor{currentfill}%
\pgfsetlinewidth{0.000000pt}%
\definecolor{currentstroke}{rgb}{0.000000,0.000000,0.000000}%
\pgfsetstrokecolor{currentstroke}%
\pgfsetdash{}{0pt}%
\pgfpathmoveto{\pgfqpoint{4.352426in}{3.377646in}}%
\pgfpathlineto{\pgfqpoint{4.363124in}{3.377063in}}%
\pgfpathlineto{\pgfqpoint{4.373843in}{3.376111in}}%
\pgfpathlineto{\pgfqpoint{4.407440in}{3.348933in}}%
\pgfpathlineto{\pgfqpoint{4.440999in}{3.322359in}}%
\pgfpathlineto{\pgfqpoint{4.430226in}{3.325520in}}%
\pgfpathlineto{\pgfqpoint{4.419474in}{3.328352in}}%
\pgfpathlineto{\pgfqpoint{4.385968in}{3.352715in}}%
\pgfpathlineto{\pgfqpoint{4.352426in}{3.377646in}}%
\pgfpathclose%
\pgfusepath{fill}%
\end{pgfscope}%
\begin{pgfscope}%
\pgfpathrectangle{\pgfqpoint{1.020000in}{0.880000in}}{\pgfqpoint{6.160000in}{6.160000in}}%
\pgfusepath{clip}%
\pgfsetbuttcap%
\pgfsetroundjoin%
\definecolor{currentfill}{rgb}{0.763363,0.835092,0.955658}%
\pgfsetfillcolor{currentfill}%
\pgfsetlinewidth{0.000000pt}%
\definecolor{currentstroke}{rgb}{0.000000,0.000000,0.000000}%
\pgfsetstrokecolor{currentstroke}%
\pgfsetdash{}{0pt}%
\pgfpathmoveto{\pgfqpoint{3.439662in}{3.653735in}}%
\pgfpathlineto{\pgfqpoint{3.449484in}{3.639816in}}%
\pgfpathlineto{\pgfqpoint{3.459302in}{3.629225in}}%
\pgfpathlineto{\pgfqpoint{3.493211in}{3.635097in}}%
\pgfpathlineto{\pgfqpoint{3.527111in}{3.639360in}}%
\pgfpathlineto{\pgfqpoint{3.517248in}{3.647580in}}%
\pgfpathlineto{\pgfqpoint{3.507383in}{3.659111in}}%
\pgfpathlineto{\pgfqpoint{3.473528in}{3.657194in}}%
\pgfpathlineto{\pgfqpoint{3.439662in}{3.653735in}}%
\pgfpathclose%
\pgfusepath{fill}%
\end{pgfscope}%
\begin{pgfscope}%
\pgfpathrectangle{\pgfqpoint{1.020000in}{0.880000in}}{\pgfqpoint{6.160000in}{6.160000in}}%
\pgfusepath{clip}%
\pgfsetbuttcap%
\pgfsetroundjoin%
\definecolor{currentfill}{rgb}{0.703587,0.802586,0.982847}%
\pgfsetfillcolor{currentfill}%
\pgfsetlinewidth{0.000000pt}%
\definecolor{currentstroke}{rgb}{0.000000,0.000000,0.000000}%
\pgfsetstrokecolor{currentstroke}%
\pgfsetdash{}{0pt}%
\pgfpathmoveto{\pgfqpoint{4.041150in}{3.544114in}}%
\pgfpathlineto{\pgfqpoint{4.051508in}{3.550637in}}%
\pgfpathlineto{\pgfqpoint{4.061890in}{3.557939in}}%
\pgfpathlineto{\pgfqpoint{4.095666in}{3.537565in}}%
\pgfpathlineto{\pgfqpoint{4.129409in}{3.516178in}}%
\pgfpathlineto{\pgfqpoint{4.118961in}{3.510543in}}%
\pgfpathlineto{\pgfqpoint{4.108536in}{3.505581in}}%
\pgfpathlineto{\pgfqpoint{4.074860in}{3.525271in}}%
\pgfpathlineto{\pgfqpoint{4.041150in}{3.544114in}}%
\pgfpathclose%
\pgfusepath{fill}%
\end{pgfscope}%
\begin{pgfscope}%
\pgfpathrectangle{\pgfqpoint{1.020000in}{0.880000in}}{\pgfqpoint{6.160000in}{6.160000in}}%
\pgfusepath{clip}%
\pgfsetbuttcap%
\pgfsetroundjoin%
\definecolor{currentfill}{rgb}{0.743754,0.825125,0.965798}%
\pgfsetfillcolor{currentfill}%
\pgfsetlinewidth{0.000000pt}%
\definecolor{currentstroke}{rgb}{0.000000,0.000000,0.000000}%
\pgfsetstrokecolor{currentstroke}%
\pgfsetdash{}{0pt}%
\pgfpathmoveto{\pgfqpoint{2.926282in}{3.628910in}}%
\pgfpathlineto{\pgfqpoint{2.935902in}{3.591284in}}%
\pgfpathlineto{\pgfqpoint{2.945503in}{3.556343in}}%
\pgfpathlineto{\pgfqpoint{2.979454in}{3.570325in}}%
\pgfpathlineto{\pgfqpoint{3.013394in}{3.584148in}}%
\pgfpathlineto{\pgfqpoint{3.003749in}{3.618759in}}%
\pgfpathlineto{\pgfqpoint{2.994085in}{3.656328in}}%
\pgfpathlineto{\pgfqpoint{2.960189in}{3.642793in}}%
\pgfpathlineto{\pgfqpoint{2.926282in}{3.628910in}}%
\pgfpathclose%
\pgfusepath{fill}%
\end{pgfscope}%
\begin{pgfscope}%
\pgfpathrectangle{\pgfqpoint{1.020000in}{0.880000in}}{\pgfqpoint{6.160000in}{6.160000in}}%
\pgfusepath{clip}%
\pgfsetbuttcap%
\pgfsetroundjoin%
\definecolor{currentfill}{rgb}{0.758539,0.832787,0.958408}%
\pgfsetfillcolor{currentfill}%
\pgfsetlinewidth{0.000000pt}%
\definecolor{currentstroke}{rgb}{0.000000,0.000000,0.000000}%
\pgfsetstrokecolor{currentstroke}%
\pgfsetdash{}{0pt}%
\pgfpathmoveto{\pgfqpoint{3.662595in}{3.639403in}}%
\pgfpathlineto{\pgfqpoint{3.672565in}{3.638120in}}%
\pgfpathlineto{\pgfqpoint{3.682544in}{3.639483in}}%
\pgfpathlineto{\pgfqpoint{3.716437in}{3.636579in}}%
\pgfpathlineto{\pgfqpoint{3.750313in}{3.631768in}}%
\pgfpathlineto{\pgfqpoint{3.740274in}{3.629187in}}%
\pgfpathlineto{\pgfqpoint{3.730246in}{3.629122in}}%
\pgfpathlineto{\pgfqpoint{3.696429in}{3.635102in}}%
\pgfpathlineto{\pgfqpoint{3.662595in}{3.639403in}}%
\pgfpathclose%
\pgfusepath{fill}%
\end{pgfscope}%
\begin{pgfscope}%
\pgfpathrectangle{\pgfqpoint{1.020000in}{0.880000in}}{\pgfqpoint{6.160000in}{6.160000in}}%
\pgfusepath{clip}%
\pgfsetbuttcap%
\pgfsetroundjoin%
\definecolor{currentfill}{rgb}{0.661968,0.775491,0.993937}%
\pgfsetfillcolor{currentfill}%
\pgfsetlinewidth{0.000000pt}%
\definecolor{currentstroke}{rgb}{0.000000,0.000000,0.000000}%
\pgfsetstrokecolor{currentstroke}%
\pgfsetdash{}{0pt}%
\pgfpathmoveto{\pgfqpoint{4.196785in}{3.471228in}}%
\pgfpathlineto{\pgfqpoint{4.207320in}{3.475246in}}%
\pgfpathlineto{\pgfqpoint{4.217878in}{3.479344in}}%
\pgfpathlineto{\pgfqpoint{4.251572in}{3.454035in}}%
\pgfpathlineto{\pgfqpoint{4.285229in}{3.428504in}}%
\pgfpathlineto{\pgfqpoint{4.274610in}{3.426628in}}%
\pgfpathlineto{\pgfqpoint{4.264015in}{3.424803in}}%
\pgfpathlineto{\pgfqpoint{4.230419in}{3.448096in}}%
\pgfpathlineto{\pgfqpoint{4.196785in}{3.471228in}}%
\pgfpathclose%
\pgfusepath{fill}%
\end{pgfscope}%
\begin{pgfscope}%
\pgfpathrectangle{\pgfqpoint{1.020000in}{0.880000in}}{\pgfqpoint{6.160000in}{6.160000in}}%
\pgfusepath{clip}%
\pgfsetbuttcap%
\pgfsetroundjoin%
\definecolor{currentfill}{rgb}{0.328604,0.439712,0.869587}%
\pgfsetfillcolor{currentfill}%
\pgfsetlinewidth{0.000000pt}%
\definecolor{currentstroke}{rgb}{0.000000,0.000000,0.000000}%
\pgfsetstrokecolor{currentstroke}%
\pgfsetdash{}{0pt}%
\pgfpathmoveto{\pgfqpoint{5.286341in}{2.869283in}}%
\pgfpathlineto{\pgfqpoint{5.297873in}{2.853199in}}%
\pgfpathlineto{\pgfqpoint{5.309426in}{2.837050in}}%
\pgfpathlineto{\pgfqpoint{5.342732in}{2.834585in}}%
\pgfpathlineto{\pgfqpoint{5.376021in}{2.832752in}}%
\pgfpathlineto{\pgfqpoint{5.364414in}{2.848784in}}%
\pgfpathlineto{\pgfqpoint{5.352828in}{2.864783in}}%
\pgfpathlineto{\pgfqpoint{5.319593in}{2.866717in}}%
\pgfpathlineto{\pgfqpoint{5.286341in}{2.869283in}}%
\pgfpathclose%
\pgfusepath{fill}%
\end{pgfscope}%
\begin{pgfscope}%
\pgfpathrectangle{\pgfqpoint{1.020000in}{0.880000in}}{\pgfqpoint{6.160000in}{6.160000in}}%
\pgfusepath{clip}%
\pgfsetbuttcap%
\pgfsetroundjoin%
\definecolor{currentfill}{rgb}{0.728970,0.817464,0.973188}%
\pgfsetfillcolor{currentfill}%
\pgfsetlinewidth{0.000000pt}%
\definecolor{currentstroke}{rgb}{0.000000,0.000000,0.000000}%
\pgfsetstrokecolor{currentstroke}%
\pgfsetdash{}{0pt}%
\pgfpathmoveto{\pgfqpoint{2.635668in}{3.594721in}}%
\pgfpathlineto{\pgfqpoint{2.645088in}{3.556654in}}%
\pgfpathlineto{\pgfqpoint{2.654499in}{3.520168in}}%
\pgfpathlineto{\pgfqpoint{2.688538in}{3.532527in}}%
\pgfpathlineto{\pgfqpoint{2.722555in}{3.545422in}}%
\pgfpathlineto{\pgfqpoint{2.713082in}{3.583162in}}%
\pgfpathlineto{\pgfqpoint{2.703596in}{3.622752in}}%
\pgfpathlineto{\pgfqpoint{2.669642in}{3.608490in}}%
\pgfpathlineto{\pgfqpoint{2.635668in}{3.594721in}}%
\pgfpathclose%
\pgfusepath{fill}%
\end{pgfscope}%
\begin{pgfscope}%
\pgfpathrectangle{\pgfqpoint{1.020000in}{0.880000in}}{\pgfqpoint{6.160000in}{6.160000in}}%
\pgfusepath{clip}%
\pgfsetbuttcap%
\pgfsetroundjoin%
\definecolor{currentfill}{rgb}{0.753611,0.830233,0.960871}%
\pgfsetfillcolor{currentfill}%
\pgfsetlinewidth{0.000000pt}%
\definecolor{currentstroke}{rgb}{0.000000,0.000000,0.000000}%
\pgfsetstrokecolor{currentstroke}%
\pgfsetdash{}{0pt}%
\pgfpathmoveto{\pgfqpoint{3.149052in}{3.635278in}}%
\pgfpathlineto{\pgfqpoint{3.158754in}{3.605863in}}%
\pgfpathlineto{\pgfqpoint{3.168440in}{3.579701in}}%
\pgfpathlineto{\pgfqpoint{3.202369in}{3.592493in}}%
\pgfpathlineto{\pgfqpoint{3.236290in}{3.604515in}}%
\pgfpathlineto{\pgfqpoint{3.226569in}{3.628812in}}%
\pgfpathlineto{\pgfqpoint{3.216833in}{3.656542in}}%
\pgfpathlineto{\pgfqpoint{3.182946in}{3.646373in}}%
\pgfpathlineto{\pgfqpoint{3.149052in}{3.635278in}}%
\pgfpathclose%
\pgfusepath{fill}%
\end{pgfscope}%
\begin{pgfscope}%
\pgfpathrectangle{\pgfqpoint{1.020000in}{0.880000in}}{\pgfqpoint{6.160000in}{6.160000in}}%
\pgfusepath{clip}%
\pgfsetbuttcap%
\pgfsetroundjoin%
\definecolor{currentfill}{rgb}{0.399231,0.528528,0.928459}%
\pgfsetfillcolor{currentfill}%
\pgfsetlinewidth{0.000000pt}%
\definecolor{currentstroke}{rgb}{0.000000,0.000000,0.000000}%
\pgfsetstrokecolor{currentstroke}%
\pgfsetdash{}{0pt}%
\pgfpathmoveto{\pgfqpoint{4.907992in}{3.009414in}}%
\pgfpathlineto{\pgfqpoint{4.919189in}{2.994724in}}%
\pgfpathlineto{\pgfqpoint{4.930404in}{2.979625in}}%
\pgfpathlineto{\pgfqpoint{4.963779in}{2.966201in}}%
\pgfpathlineto{\pgfqpoint{4.997135in}{2.954244in}}%
\pgfpathlineto{\pgfqpoint{4.985870in}{2.969629in}}%
\pgfpathlineto{\pgfqpoint{4.974624in}{2.984702in}}%
\pgfpathlineto{\pgfqpoint{4.941318in}{2.996349in}}%
\pgfpathlineto{\pgfqpoint{4.907992in}{3.009414in}}%
\pgfpathclose%
\pgfusepath{fill}%
\end{pgfscope}%
\begin{pgfscope}%
\pgfpathrectangle{\pgfqpoint{1.020000in}{0.880000in}}{\pgfqpoint{6.160000in}{6.160000in}}%
\pgfusepath{clip}%
\pgfsetbuttcap%
\pgfsetroundjoin%
\definecolor{currentfill}{rgb}{0.280550,0.373423,0.818011}%
\pgfsetfillcolor{currentfill}%
\pgfsetlinewidth{0.000000pt}%
\definecolor{currentstroke}{rgb}{0.000000,0.000000,0.000000}%
\pgfsetstrokecolor{currentstroke}%
\pgfsetdash{}{0pt}%
\pgfpathmoveto{\pgfqpoint{6.111365in}{2.765903in}}%
\pgfpathlineto{\pgfqpoint{6.123685in}{2.751168in}}%
\pgfpathlineto{\pgfqpoint{6.136030in}{2.736482in}}%
\pgfpathlineto{\pgfqpoint{6.169141in}{2.738977in}}%
\pgfpathlineto{\pgfqpoint{6.202232in}{2.741491in}}%
\pgfpathlineto{\pgfqpoint{6.189832in}{2.756098in}}%
\pgfpathlineto{\pgfqpoint{6.177456in}{2.770753in}}%
\pgfpathlineto{\pgfqpoint{6.144420in}{2.768318in}}%
\pgfpathlineto{\pgfqpoint{6.111365in}{2.765903in}}%
\pgfpathclose%
\pgfusepath{fill}%
\end{pgfscope}%
\begin{pgfscope}%
\pgfpathrectangle{\pgfqpoint{1.020000in}{0.880000in}}{\pgfqpoint{6.160000in}{6.160000in}}%
\pgfusepath{clip}%
\pgfsetbuttcap%
\pgfsetroundjoin%
\definecolor{currentfill}{rgb}{0.289996,0.386836,0.828926}%
\pgfsetfillcolor{currentfill}%
\pgfsetlinewidth{0.000000pt}%
\definecolor{currentstroke}{rgb}{0.000000,0.000000,0.000000}%
\pgfsetstrokecolor{currentstroke}%
\pgfsetdash{}{0pt}%
\pgfpathmoveto{\pgfqpoint{5.888371in}{2.782189in}}%
\pgfpathlineto{\pgfqpoint{5.900477in}{2.767089in}}%
\pgfpathlineto{\pgfqpoint{5.912607in}{2.752039in}}%
\pgfpathlineto{\pgfqpoint{5.945783in}{2.754247in}}%
\pgfpathlineto{\pgfqpoint{5.978940in}{2.756505in}}%
\pgfpathlineto{\pgfqpoint{5.966754in}{2.771462in}}%
\pgfpathlineto{\pgfqpoint{5.954593in}{2.786469in}}%
\pgfpathlineto{\pgfqpoint{5.921492in}{2.784303in}}%
\pgfpathlineto{\pgfqpoint{5.888371in}{2.782189in}}%
\pgfpathclose%
\pgfusepath{fill}%
\end{pgfscope}%
\begin{pgfscope}%
\pgfpathrectangle{\pgfqpoint{1.020000in}{0.880000in}}{\pgfqpoint{6.160000in}{6.160000in}}%
\pgfusepath{clip}%
\pgfsetbuttcap%
\pgfsetroundjoin%
\definecolor{currentfill}{rgb}{0.451739,0.588181,0.960201}%
\pgfsetfillcolor{currentfill}%
\pgfsetlinewidth{0.000000pt}%
\definecolor{currentstroke}{rgb}{0.000000,0.000000,0.000000}%
\pgfsetstrokecolor{currentstroke}%
\pgfsetdash{}{0pt}%
\pgfpathmoveto{\pgfqpoint{4.752321in}{3.102367in}}%
\pgfpathlineto{\pgfqpoint{4.763386in}{3.090003in}}%
\pgfpathlineto{\pgfqpoint{4.774469in}{3.077058in}}%
\pgfpathlineto{\pgfqpoint{4.807885in}{3.057748in}}%
\pgfpathlineto{\pgfqpoint{4.841277in}{3.040061in}}%
\pgfpathlineto{\pgfqpoint{4.830146in}{3.053757in}}%
\pgfpathlineto{\pgfqpoint{4.819034in}{3.066991in}}%
\pgfpathlineto{\pgfqpoint{4.785690in}{3.083908in}}%
\pgfpathlineto{\pgfqpoint{4.752321in}{3.102367in}}%
\pgfpathclose%
\pgfusepath{fill}%
\end{pgfscope}%
\begin{pgfscope}%
\pgfpathrectangle{\pgfqpoint{1.020000in}{0.880000in}}{\pgfqpoint{6.160000in}{6.160000in}}%
\pgfusepath{clip}%
\pgfsetbuttcap%
\pgfsetroundjoin%
\definecolor{currentfill}{rgb}{0.299441,0.400248,0.839842}%
\pgfsetfillcolor{currentfill}%
\pgfsetlinewidth{0.000000pt}%
\definecolor{currentstroke}{rgb}{0.000000,0.000000,0.000000}%
\pgfsetstrokecolor{currentstroke}%
\pgfsetdash{}{0pt}%
\pgfpathmoveto{\pgfqpoint{5.665423in}{2.802298in}}%
\pgfpathlineto{\pgfqpoint{5.677316in}{2.786789in}}%
\pgfpathlineto{\pgfqpoint{5.689233in}{2.771322in}}%
\pgfpathlineto{\pgfqpoint{5.722471in}{2.772834in}}%
\pgfpathlineto{\pgfqpoint{5.755690in}{2.774492in}}%
\pgfpathlineto{\pgfqpoint{5.743718in}{2.789845in}}%
\pgfpathlineto{\pgfqpoint{5.731769in}{2.805243in}}%
\pgfpathlineto{\pgfqpoint{5.698605in}{2.803695in}}%
\pgfpathlineto{\pgfqpoint{5.665423in}{2.802298in}}%
\pgfpathclose%
\pgfusepath{fill}%
\end{pgfscope}%
\begin{pgfscope}%
\pgfpathrectangle{\pgfqpoint{1.020000in}{0.880000in}}{\pgfqpoint{6.160000in}{6.160000in}}%
\pgfusepath{clip}%
\pgfsetbuttcap%
\pgfsetroundjoin%
\definecolor{currentfill}{rgb}{0.733898,0.820018,0.970724}%
\pgfsetfillcolor{currentfill}%
\pgfsetlinewidth{0.000000pt}%
\definecolor{currentstroke}{rgb}{0.000000,0.000000,0.000000}%
\pgfsetstrokecolor{currentstroke}%
\pgfsetdash{}{0pt}%
\pgfpathmoveto{\pgfqpoint{2.858434in}{3.600627in}}%
\pgfpathlineto{\pgfqpoint{2.868005in}{3.563306in}}%
\pgfpathlineto{\pgfqpoint{2.877559in}{3.528384in}}%
\pgfpathlineto{\pgfqpoint{2.911539in}{3.542324in}}%
\pgfpathlineto{\pgfqpoint{2.945503in}{3.556343in}}%
\pgfpathlineto{\pgfqpoint{2.935902in}{3.591284in}}%
\pgfpathlineto{\pgfqpoint{2.926282in}{3.628910in}}%
\pgfpathlineto{\pgfqpoint{2.892364in}{3.614811in}}%
\pgfpathlineto{\pgfqpoint{2.858434in}{3.600627in}}%
\pgfpathclose%
\pgfusepath{fill}%
\end{pgfscope}%
\begin{pgfscope}%
\pgfpathrectangle{\pgfqpoint{1.020000in}{0.880000in}}{\pgfqpoint{6.160000in}{6.160000in}}%
\pgfusepath{clip}%
\pgfsetbuttcap%
\pgfsetroundjoin%
\definecolor{currentfill}{rgb}{0.363461,0.484784,0.901019}%
\pgfsetfillcolor{currentfill}%
\pgfsetlinewidth{0.000000pt}%
\definecolor{currentstroke}{rgb}{0.000000,0.000000,0.000000}%
\pgfsetstrokecolor{currentstroke}%
\pgfsetdash{}{0pt}%
\pgfpathmoveto{\pgfqpoint{5.063790in}{2.934409in}}%
\pgfpathlineto{\pgfqpoint{5.075124in}{2.918581in}}%
\pgfpathlineto{\pgfqpoint{5.086478in}{2.902513in}}%
\pgfpathlineto{\pgfqpoint{5.119831in}{2.894415in}}%
\pgfpathlineto{\pgfqpoint{5.153166in}{2.887461in}}%
\pgfpathlineto{\pgfqpoint{5.141760in}{2.903534in}}%
\pgfpathlineto{\pgfqpoint{5.130374in}{2.919435in}}%
\pgfpathlineto{\pgfqpoint{5.097091in}{2.926359in}}%
\pgfpathlineto{\pgfqpoint{5.063790in}{2.934409in}}%
\pgfpathclose%
\pgfusepath{fill}%
\end{pgfscope}%
\begin{pgfscope}%
\pgfpathrectangle{\pgfqpoint{1.020000in}{0.880000in}}{\pgfqpoint{6.160000in}{6.160000in}}%
\pgfusepath{clip}%
\pgfsetbuttcap%
\pgfsetroundjoin%
\definecolor{currentfill}{rgb}{0.313946,0.420052,0.854993}%
\pgfsetfillcolor{currentfill}%
\pgfsetlinewidth{0.000000pt}%
\definecolor{currentstroke}{rgb}{0.000000,0.000000,0.000000}%
\pgfsetstrokecolor{currentstroke}%
\pgfsetdash{}{0pt}%
\pgfpathmoveto{\pgfqpoint{5.442549in}{2.830702in}}%
\pgfpathlineto{\pgfqpoint{5.454232in}{2.814777in}}%
\pgfpathlineto{\pgfqpoint{5.465937in}{2.798852in}}%
\pgfpathlineto{\pgfqpoint{5.499230in}{2.798644in}}%
\pgfpathlineto{\pgfqpoint{5.532505in}{2.798808in}}%
\pgfpathlineto{\pgfqpoint{5.520744in}{2.814594in}}%
\pgfpathlineto{\pgfqpoint{5.509006in}{2.830397in}}%
\pgfpathlineto{\pgfqpoint{5.475786in}{2.830360in}}%
\pgfpathlineto{\pgfqpoint{5.442549in}{2.830702in}}%
\pgfpathclose%
\pgfusepath{fill}%
\end{pgfscope}%
\begin{pgfscope}%
\pgfpathrectangle{\pgfqpoint{1.020000in}{0.880000in}}{\pgfqpoint{6.160000in}{6.160000in}}%
\pgfusepath{clip}%
\pgfsetbuttcap%
\pgfsetroundjoin%
\definecolor{currentfill}{rgb}{0.763363,0.835092,0.955658}%
\pgfsetfillcolor{currentfill}%
\pgfsetlinewidth{0.000000pt}%
\definecolor{currentstroke}{rgb}{0.000000,0.000000,0.000000}%
\pgfsetstrokecolor{currentstroke}%
\pgfsetdash{}{0pt}%
\pgfpathmoveto{\pgfqpoint{3.371902in}{3.642413in}}%
\pgfpathlineto{\pgfqpoint{3.381685in}{3.626049in}}%
\pgfpathlineto{\pgfqpoint{3.391460in}{3.612984in}}%
\pgfpathlineto{\pgfqpoint{3.425385in}{3.621823in}}%
\pgfpathlineto{\pgfqpoint{3.459302in}{3.629225in}}%
\pgfpathlineto{\pgfqpoint{3.449484in}{3.639816in}}%
\pgfpathlineto{\pgfqpoint{3.439662in}{3.653735in}}%
\pgfpathlineto{\pgfqpoint{3.405786in}{3.648786in}}%
\pgfpathlineto{\pgfqpoint{3.371902in}{3.642413in}}%
\pgfpathclose%
\pgfusepath{fill}%
\end{pgfscope}%
\begin{pgfscope}%
\pgfpathrectangle{\pgfqpoint{1.020000in}{0.880000in}}{\pgfqpoint{6.160000in}{6.160000in}}%
\pgfusepath{clip}%
\pgfsetbuttcap%
\pgfsetroundjoin%
\definecolor{currentfill}{rgb}{0.510824,0.649397,0.985079}%
\pgfsetfillcolor{currentfill}%
\pgfsetlinewidth{0.000000pt}%
\definecolor{currentstroke}{rgb}{0.000000,0.000000,0.000000}%
\pgfsetstrokecolor{currentstroke}%
\pgfsetdash{}{0pt}%
\pgfpathmoveto{\pgfqpoint{4.596684in}{3.209588in}}%
\pgfpathlineto{\pgfqpoint{4.607617in}{3.200967in}}%
\pgfpathlineto{\pgfqpoint{4.618569in}{3.191650in}}%
\pgfpathlineto{\pgfqpoint{4.652052in}{3.167061in}}%
\pgfpathlineto{\pgfqpoint{4.685504in}{3.143957in}}%
\pgfpathlineto{\pgfqpoint{4.674504in}{3.154627in}}%
\pgfpathlineto{\pgfqpoint{4.663524in}{3.164724in}}%
\pgfpathlineto{\pgfqpoint{4.630119in}{3.186460in}}%
\pgfpathlineto{\pgfqpoint{4.596684in}{3.209588in}}%
\pgfpathclose%
\pgfusepath{fill}%
\end{pgfscope}%
\begin{pgfscope}%
\pgfpathrectangle{\pgfqpoint{1.020000in}{0.880000in}}{\pgfqpoint{6.160000in}{6.160000in}}%
\pgfusepath{clip}%
\pgfsetbuttcap%
\pgfsetroundjoin%
\definecolor{currentfill}{rgb}{0.753611,0.830233,0.960871}%
\pgfsetfillcolor{currentfill}%
\pgfsetlinewidth{0.000000pt}%
\definecolor{currentstroke}{rgb}{0.000000,0.000000,0.000000}%
\pgfsetstrokecolor{currentstroke}%
\pgfsetdash{}{0pt}%
\pgfpathmoveto{\pgfqpoint{3.818008in}{3.616637in}}%
\pgfpathlineto{\pgfqpoint{3.828125in}{3.622033in}}%
\pgfpathlineto{\pgfqpoint{3.838259in}{3.629213in}}%
\pgfpathlineto{\pgfqpoint{3.872141in}{3.619174in}}%
\pgfpathlineto{\pgfqpoint{3.905998in}{3.607284in}}%
\pgfpathlineto{\pgfqpoint{3.895796in}{3.600185in}}%
\pgfpathlineto{\pgfqpoint{3.885613in}{3.594725in}}%
\pgfpathlineto{\pgfqpoint{3.851823in}{3.606477in}}%
\pgfpathlineto{\pgfqpoint{3.818008in}{3.616637in}}%
\pgfpathclose%
\pgfusepath{fill}%
\end{pgfscope}%
\begin{pgfscope}%
\pgfpathrectangle{\pgfqpoint{1.020000in}{0.880000in}}{\pgfqpoint{6.160000in}{6.160000in}}%
\pgfusepath{clip}%
\pgfsetbuttcap%
\pgfsetroundjoin%
\definecolor{currentfill}{rgb}{0.718985,0.811993,0.977656}%
\pgfsetfillcolor{currentfill}%
\pgfsetlinewidth{0.000000pt}%
\definecolor{currentstroke}{rgb}{0.000000,0.000000,0.000000}%
\pgfsetstrokecolor{currentstroke}%
\pgfsetdash{}{0pt}%
\pgfpathmoveto{\pgfqpoint{2.567650in}{3.569001in}}%
\pgfpathlineto{\pgfqpoint{2.577002in}{3.532498in}}%
\pgfpathlineto{\pgfqpoint{2.586347in}{3.497327in}}%
\pgfpathlineto{\pgfqpoint{2.620436in}{3.508416in}}%
\pgfpathlineto{\pgfqpoint{2.654499in}{3.520168in}}%
\pgfpathlineto{\pgfqpoint{2.645088in}{3.556654in}}%
\pgfpathlineto{\pgfqpoint{2.635668in}{3.594721in}}%
\pgfpathlineto{\pgfqpoint{2.601671in}{3.581532in}}%
\pgfpathlineto{\pgfqpoint{2.567650in}{3.569001in}}%
\pgfpathclose%
\pgfusepath{fill}%
\end{pgfscope}%
\begin{pgfscope}%
\pgfpathrectangle{\pgfqpoint{1.020000in}{0.880000in}}{\pgfqpoint{6.160000in}{6.160000in}}%
\pgfusepath{clip}%
\pgfsetbuttcap%
\pgfsetroundjoin%
\definecolor{currentfill}{rgb}{0.743754,0.825125,0.965798}%
\pgfsetfillcolor{currentfill}%
\pgfsetlinewidth{0.000000pt}%
\definecolor{currentstroke}{rgb}{0.000000,0.000000,0.000000}%
\pgfsetstrokecolor{currentstroke}%
\pgfsetdash{}{0pt}%
\pgfpathmoveto{\pgfqpoint{3.081241in}{3.610806in}}%
\pgfpathlineto{\pgfqpoint{3.090906in}{3.580054in}}%
\pgfpathlineto{\pgfqpoint{3.100556in}{3.552344in}}%
\pgfpathlineto{\pgfqpoint{3.134502in}{3.566273in}}%
\pgfpathlineto{\pgfqpoint{3.168440in}{3.579701in}}%
\pgfpathlineto{\pgfqpoint{3.158754in}{3.605863in}}%
\pgfpathlineto{\pgfqpoint{3.149052in}{3.635278in}}%
\pgfpathlineto{\pgfqpoint{3.115150in}{3.623380in}}%
\pgfpathlineto{\pgfqpoint{3.081241in}{3.610806in}}%
\pgfpathclose%
\pgfusepath{fill}%
\end{pgfscope}%
\begin{pgfscope}%
\pgfpathrectangle{\pgfqpoint{1.020000in}{0.880000in}}{\pgfqpoint{6.160000in}{6.160000in}}%
\pgfusepath{clip}%
\pgfsetbuttcap%
\pgfsetroundjoin%
\definecolor{currentfill}{rgb}{0.768034,0.837035,0.952488}%
\pgfsetfillcolor{currentfill}%
\pgfsetlinewidth{0.000000pt}%
\definecolor{currentstroke}{rgb}{0.000000,0.000000,0.000000}%
\pgfsetstrokecolor{currentstroke}%
\pgfsetdash{}{0pt}%
\pgfpathmoveto{\pgfqpoint{3.594879in}{3.642837in}}%
\pgfpathlineto{\pgfqpoint{3.604794in}{3.639793in}}%
\pgfpathlineto{\pgfqpoint{3.614716in}{3.639492in}}%
\pgfpathlineto{\pgfqpoint{3.648636in}{3.640455in}}%
\pgfpathlineto{\pgfqpoint{3.682544in}{3.639483in}}%
\pgfpathlineto{\pgfqpoint{3.672565in}{3.638120in}}%
\pgfpathlineto{\pgfqpoint{3.662595in}{3.639403in}}%
\pgfpathlineto{\pgfqpoint{3.628744in}{3.641987in}}%
\pgfpathlineto{\pgfqpoint{3.594879in}{3.642837in}}%
\pgfpathclose%
\pgfusepath{fill}%
\end{pgfscope}%
\begin{pgfscope}%
\pgfpathrectangle{\pgfqpoint{1.020000in}{0.880000in}}{\pgfqpoint{6.160000in}{6.160000in}}%
\pgfusepath{clip}%
\pgfsetbuttcap%
\pgfsetroundjoin%
\definecolor{currentfill}{rgb}{0.333490,0.446265,0.874452}%
\pgfsetfillcolor{currentfill}%
\pgfsetlinewidth{0.000000pt}%
\definecolor{currentstroke}{rgb}{0.000000,0.000000,0.000000}%
\pgfsetstrokecolor{currentstroke}%
\pgfsetdash{}{0pt}%
\pgfpathmoveto{\pgfqpoint{5.219787in}{2.876622in}}%
\pgfpathlineto{\pgfqpoint{5.231265in}{2.860457in}}%
\pgfpathlineto{\pgfqpoint{5.242764in}{2.844185in}}%
\pgfpathlineto{\pgfqpoint{5.276103in}{2.840223in}}%
\pgfpathlineto{\pgfqpoint{5.309426in}{2.837050in}}%
\pgfpathlineto{\pgfqpoint{5.297873in}{2.853199in}}%
\pgfpathlineto{\pgfqpoint{5.286341in}{2.869283in}}%
\pgfpathlineto{\pgfqpoint{5.253072in}{2.872558in}}%
\pgfpathlineto{\pgfqpoint{5.219787in}{2.876622in}}%
\pgfpathclose%
\pgfusepath{fill}%
\end{pgfscope}%
\begin{pgfscope}%
\pgfpathrectangle{\pgfqpoint{1.020000in}{0.880000in}}{\pgfqpoint{6.160000in}{6.160000in}}%
\pgfusepath{clip}%
\pgfsetbuttcap%
\pgfsetroundjoin%
\definecolor{currentfill}{rgb}{0.718985,0.811993,0.977656}%
\pgfsetfillcolor{currentfill}%
\pgfsetlinewidth{0.000000pt}%
\definecolor{currentstroke}{rgb}{0.000000,0.000000,0.000000}%
\pgfsetstrokecolor{currentstroke}%
\pgfsetdash{}{0pt}%
\pgfpathmoveto{\pgfqpoint{2.790529in}{3.572496in}}%
\pgfpathlineto{\pgfqpoint{2.800047in}{3.535778in}}%
\pgfpathlineto{\pgfqpoint{2.809551in}{3.501174in}}%
\pgfpathlineto{\pgfqpoint{2.843564in}{3.514634in}}%
\pgfpathlineto{\pgfqpoint{2.877559in}{3.528384in}}%
\pgfpathlineto{\pgfqpoint{2.868005in}{3.563306in}}%
\pgfpathlineto{\pgfqpoint{2.858434in}{3.600627in}}%
\pgfpathlineto{\pgfqpoint{2.824489in}{3.586483in}}%
\pgfpathlineto{\pgfqpoint{2.790529in}{3.572496in}}%
\pgfpathclose%
\pgfusepath{fill}%
\end{pgfscope}%
\begin{pgfscope}%
\pgfpathrectangle{\pgfqpoint{1.020000in}{0.880000in}}{\pgfqpoint{6.160000in}{6.160000in}}%
\pgfusepath{clip}%
\pgfsetbuttcap%
\pgfsetroundjoin%
\definecolor{currentfill}{rgb}{0.576051,0.708780,0.997755}%
\pgfsetfillcolor{currentfill}%
\pgfsetlinewidth{0.000000pt}%
\definecolor{currentstroke}{rgb}{0.000000,0.000000,0.000000}%
\pgfsetstrokecolor{currentstroke}%
\pgfsetdash{}{0pt}%
\pgfpathmoveto{\pgfqpoint{4.440999in}{3.322359in}}%
\pgfpathlineto{\pgfqpoint{4.451793in}{3.318700in}}%
\pgfpathlineto{\pgfqpoint{4.462608in}{3.314373in}}%
\pgfpathlineto{\pgfqpoint{4.496181in}{3.286561in}}%
\pgfpathlineto{\pgfqpoint{4.529716in}{3.259739in}}%
\pgfpathlineto{\pgfqpoint{4.518851in}{3.265995in}}%
\pgfpathlineto{\pgfqpoint{4.508006in}{3.271683in}}%
\pgfpathlineto{\pgfqpoint{4.474520in}{3.296560in}}%
\pgfpathlineto{\pgfqpoint{4.440999in}{3.322359in}}%
\pgfpathclose%
\pgfusepath{fill}%
\end{pgfscope}%
\begin{pgfscope}%
\pgfpathrectangle{\pgfqpoint{1.020000in}{0.880000in}}{\pgfqpoint{6.160000in}{6.160000in}}%
\pgfusepath{clip}%
\pgfsetbuttcap%
\pgfsetroundjoin%
\definecolor{currentfill}{rgb}{0.280550,0.373423,0.818011}%
\pgfsetfillcolor{currentfill}%
\pgfsetlinewidth{0.000000pt}%
\definecolor{currentstroke}{rgb}{0.000000,0.000000,0.000000}%
\pgfsetstrokecolor{currentstroke}%
\pgfsetdash{}{0pt}%
\pgfpathmoveto{\pgfqpoint{6.045192in}{2.761143in}}%
\pgfpathlineto{\pgfqpoint{6.057457in}{2.746324in}}%
\pgfpathlineto{\pgfqpoint{6.069747in}{2.731555in}}%
\pgfpathlineto{\pgfqpoint{6.102898in}{2.734006in}}%
\pgfpathlineto{\pgfqpoint{6.136030in}{2.736482in}}%
\pgfpathlineto{\pgfqpoint{6.123685in}{2.751168in}}%
\pgfpathlineto{\pgfqpoint{6.111365in}{2.765903in}}%
\pgfpathlineto{\pgfqpoint{6.078289in}{2.763510in}}%
\pgfpathlineto{\pgfqpoint{6.045192in}{2.761143in}}%
\pgfpathclose%
\pgfusepath{fill}%
\end{pgfscope}%
\begin{pgfscope}%
\pgfpathrectangle{\pgfqpoint{1.020000in}{0.880000in}}{\pgfqpoint{6.160000in}{6.160000in}}%
\pgfusepath{clip}%
\pgfsetbuttcap%
\pgfsetroundjoin%
\definecolor{currentfill}{rgb}{0.289996,0.386836,0.828926}%
\pgfsetfillcolor{currentfill}%
\pgfsetlinewidth{0.000000pt}%
\definecolor{currentstroke}{rgb}{0.000000,0.000000,0.000000}%
\pgfsetstrokecolor{currentstroke}%
\pgfsetdash{}{0pt}%
\pgfpathmoveto{\pgfqpoint{5.822069in}{2.778162in}}%
\pgfpathlineto{\pgfqpoint{5.834120in}{2.762963in}}%
\pgfpathlineto{\pgfqpoint{5.846195in}{2.747813in}}%
\pgfpathlineto{\pgfqpoint{5.879411in}{2.749890in}}%
\pgfpathlineto{\pgfqpoint{5.912607in}{2.752039in}}%
\pgfpathlineto{\pgfqpoint{5.900477in}{2.767089in}}%
\pgfpathlineto{\pgfqpoint{5.888371in}{2.782189in}}%
\pgfpathlineto{\pgfqpoint{5.855230in}{2.780138in}}%
\pgfpathlineto{\pgfqpoint{5.822069in}{2.778162in}}%
\pgfpathclose%
\pgfusepath{fill}%
\end{pgfscope}%
\begin{pgfscope}%
\pgfpathrectangle{\pgfqpoint{1.020000in}{0.880000in}}{\pgfqpoint{6.160000in}{6.160000in}}%
\pgfusepath{clip}%
\pgfsetbuttcap%
\pgfsetroundjoin%
\definecolor{currentfill}{rgb}{0.728970,0.817464,0.973188}%
\pgfsetfillcolor{currentfill}%
\pgfsetlinewidth{0.000000pt}%
\definecolor{currentstroke}{rgb}{0.000000,0.000000,0.000000}%
\pgfsetstrokecolor{currentstroke}%
\pgfsetdash{}{0pt}%
\pgfpathmoveto{\pgfqpoint{3.973633in}{3.578498in}}%
\pgfpathlineto{\pgfqpoint{3.983924in}{3.586202in}}%
\pgfpathlineto{\pgfqpoint{3.994237in}{3.594794in}}%
\pgfpathlineto{\pgfqpoint{4.028079in}{3.577084in}}%
\pgfpathlineto{\pgfqpoint{4.061890in}{3.557939in}}%
\pgfpathlineto{\pgfqpoint{4.051508in}{3.550637in}}%
\pgfpathlineto{\pgfqpoint{4.041150in}{3.544114in}}%
\pgfpathlineto{\pgfqpoint{4.007407in}{3.561918in}}%
\pgfpathlineto{\pgfqpoint{3.973633in}{3.578498in}}%
\pgfpathclose%
\pgfusepath{fill}%
\end{pgfscope}%
\begin{pgfscope}%
\pgfpathrectangle{\pgfqpoint{1.020000in}{0.880000in}}{\pgfqpoint{6.160000in}{6.160000in}}%
\pgfusepath{clip}%
\pgfsetbuttcap%
\pgfsetroundjoin%
\definecolor{currentfill}{rgb}{0.299441,0.400248,0.839842}%
\pgfsetfillcolor{currentfill}%
\pgfsetlinewidth{0.000000pt}%
\definecolor{currentstroke}{rgb}{0.000000,0.000000,0.000000}%
\pgfsetstrokecolor{currentstroke}%
\pgfsetdash{}{0pt}%
\pgfpathmoveto{\pgfqpoint{5.599001in}{2.800067in}}%
\pgfpathlineto{\pgfqpoint{5.610839in}{2.784437in}}%
\pgfpathlineto{\pgfqpoint{5.622700in}{2.768841in}}%
\pgfpathlineto{\pgfqpoint{5.655976in}{2.769981in}}%
\pgfpathlineto{\pgfqpoint{5.689233in}{2.771322in}}%
\pgfpathlineto{\pgfqpoint{5.677316in}{2.786789in}}%
\pgfpathlineto{\pgfqpoint{5.665423in}{2.802298in}}%
\pgfpathlineto{\pgfqpoint{5.632221in}{2.801078in}}%
\pgfpathlineto{\pgfqpoint{5.599001in}{2.800067in}}%
\pgfpathclose%
\pgfusepath{fill}%
\end{pgfscope}%
\begin{pgfscope}%
\pgfpathrectangle{\pgfqpoint{1.020000in}{0.880000in}}{\pgfqpoint{6.160000in}{6.160000in}}%
\pgfusepath{clip}%
\pgfsetbuttcap%
\pgfsetroundjoin%
\definecolor{currentfill}{rgb}{0.758539,0.832787,0.958408}%
\pgfsetfillcolor{currentfill}%
\pgfsetlinewidth{0.000000pt}%
\definecolor{currentstroke}{rgb}{0.000000,0.000000,0.000000}%
\pgfsetstrokecolor{currentstroke}%
\pgfsetdash{}{0pt}%
\pgfpathmoveto{\pgfqpoint{3.304111in}{3.625738in}}%
\pgfpathlineto{\pgfqpoint{3.313855in}{3.606978in}}%
\pgfpathlineto{\pgfqpoint{3.323592in}{3.591438in}}%
\pgfpathlineto{\pgfqpoint{3.357529in}{3.602815in}}%
\pgfpathlineto{\pgfqpoint{3.391460in}{3.612984in}}%
\pgfpathlineto{\pgfqpoint{3.381685in}{3.626049in}}%
\pgfpathlineto{\pgfqpoint{3.371902in}{3.642413in}}%
\pgfpathlineto{\pgfqpoint{3.338010in}{3.634698in}}%
\pgfpathlineto{\pgfqpoint{3.304111in}{3.625738in}}%
\pgfpathclose%
\pgfusepath{fill}%
\end{pgfscope}%
\begin{pgfscope}%
\pgfpathrectangle{\pgfqpoint{1.020000in}{0.880000in}}{\pgfqpoint{6.160000in}{6.160000in}}%
\pgfusepath{clip}%
\pgfsetbuttcap%
\pgfsetroundjoin%
\definecolor{currentfill}{rgb}{0.640828,0.760752,0.997846}%
\pgfsetfillcolor{currentfill}%
\pgfsetlinewidth{0.000000pt}%
\definecolor{currentstroke}{rgb}{0.000000,0.000000,0.000000}%
\pgfsetstrokecolor{currentstroke}%
\pgfsetdash{}{0pt}%
\pgfpathmoveto{\pgfqpoint{4.285229in}{3.428504in}}%
\pgfpathlineto{\pgfqpoint{4.295869in}{3.430202in}}%
\pgfpathlineto{\pgfqpoint{4.306533in}{3.431490in}}%
\pgfpathlineto{\pgfqpoint{4.340208in}{3.403700in}}%
\pgfpathlineto{\pgfqpoint{4.373843in}{3.376111in}}%
\pgfpathlineto{\pgfqpoint{4.363124in}{3.377063in}}%
\pgfpathlineto{\pgfqpoint{4.352426in}{3.377646in}}%
\pgfpathlineto{\pgfqpoint{4.318846in}{3.402972in}}%
\pgfpathlineto{\pgfqpoint{4.285229in}{3.428504in}}%
\pgfpathclose%
\pgfusepath{fill}%
\end{pgfscope}%
\begin{pgfscope}%
\pgfpathrectangle{\pgfqpoint{1.020000in}{0.880000in}}{\pgfqpoint{6.160000in}{6.160000in}}%
\pgfusepath{clip}%
\pgfsetbuttcap%
\pgfsetroundjoin%
\definecolor{currentfill}{rgb}{0.708720,0.805721,0.981117}%
\pgfsetfillcolor{currentfill}%
\pgfsetlinewidth{0.000000pt}%
\definecolor{currentstroke}{rgb}{0.000000,0.000000,0.000000}%
\pgfsetstrokecolor{currentstroke}%
\pgfsetdash{}{0pt}%
\pgfpathmoveto{\pgfqpoint{2.499530in}{3.546152in}}%
\pgfpathlineto{\pgfqpoint{2.508812in}{3.511183in}}%
\pgfpathlineto{\pgfqpoint{2.518090in}{3.477317in}}%
\pgfpathlineto{\pgfqpoint{2.552233in}{3.486949in}}%
\pgfpathlineto{\pgfqpoint{2.586347in}{3.497327in}}%
\pgfpathlineto{\pgfqpoint{2.577002in}{3.532498in}}%
\pgfpathlineto{\pgfqpoint{2.567650in}{3.569001in}}%
\pgfpathlineto{\pgfqpoint{2.533604in}{3.557190in}}%
\pgfpathlineto{\pgfqpoint{2.499530in}{3.546152in}}%
\pgfpathclose%
\pgfusepath{fill}%
\end{pgfscope}%
\begin{pgfscope}%
\pgfpathrectangle{\pgfqpoint{1.020000in}{0.880000in}}{\pgfqpoint{6.160000in}{6.160000in}}%
\pgfusepath{clip}%
\pgfsetbuttcap%
\pgfsetroundjoin%
\definecolor{currentfill}{rgb}{0.728970,0.817464,0.973188}%
\pgfsetfillcolor{currentfill}%
\pgfsetlinewidth{0.000000pt}%
\definecolor{currentstroke}{rgb}{0.000000,0.000000,0.000000}%
\pgfsetstrokecolor{currentstroke}%
\pgfsetdash{}{0pt}%
\pgfpathmoveto{\pgfqpoint{3.013394in}{3.584148in}}%
\pgfpathlineto{\pgfqpoint{3.023021in}{3.552431in}}%
\pgfpathlineto{\pgfqpoint{3.032632in}{3.523515in}}%
\pgfpathlineto{\pgfqpoint{3.066600in}{3.538047in}}%
\pgfpathlineto{\pgfqpoint{3.100556in}{3.552344in}}%
\pgfpathlineto{\pgfqpoint{3.090906in}{3.580054in}}%
\pgfpathlineto{\pgfqpoint{3.081241in}{3.610806in}}%
\pgfpathlineto{\pgfqpoint{3.047322in}{3.597685in}}%
\pgfpathlineto{\pgfqpoint{3.013394in}{3.584148in}}%
\pgfpathclose%
\pgfusepath{fill}%
\end{pgfscope}%
\begin{pgfscope}%
\pgfpathrectangle{\pgfqpoint{1.020000in}{0.880000in}}{\pgfqpoint{6.160000in}{6.160000in}}%
\pgfusepath{clip}%
\pgfsetbuttcap%
\pgfsetroundjoin%
\definecolor{currentfill}{rgb}{0.693321,0.796314,0.986308}%
\pgfsetfillcolor{currentfill}%
\pgfsetlinewidth{0.000000pt}%
\definecolor{currentstroke}{rgb}{0.000000,0.000000,0.000000}%
\pgfsetstrokecolor{currentstroke}%
\pgfsetdash{}{0pt}%
\pgfpathmoveto{\pgfqpoint{4.129409in}{3.516178in}}%
\pgfpathlineto{\pgfqpoint{4.139879in}{3.522214in}}%
\pgfpathlineto{\pgfqpoint{4.150374in}{3.528370in}}%
\pgfpathlineto{\pgfqpoint{4.184145in}{3.504202in}}%
\pgfpathlineto{\pgfqpoint{4.217878in}{3.479344in}}%
\pgfpathlineto{\pgfqpoint{4.207320in}{3.475246in}}%
\pgfpathlineto{\pgfqpoint{4.196785in}{3.471228in}}%
\pgfpathlineto{\pgfqpoint{4.163115in}{3.493993in}}%
\pgfpathlineto{\pgfqpoint{4.129409in}{3.516178in}}%
\pgfpathclose%
\pgfusepath{fill}%
\end{pgfscope}%
\begin{pgfscope}%
\pgfpathrectangle{\pgfqpoint{1.020000in}{0.880000in}}{\pgfqpoint{6.160000in}{6.160000in}}%
\pgfusepath{clip}%
\pgfsetbuttcap%
\pgfsetroundjoin%
\definecolor{currentfill}{rgb}{0.318832,0.426605,0.859857}%
\pgfsetfillcolor{currentfill}%
\pgfsetlinewidth{0.000000pt}%
\definecolor{currentstroke}{rgb}{0.000000,0.000000,0.000000}%
\pgfsetstrokecolor{currentstroke}%
\pgfsetdash{}{0pt}%
\pgfpathmoveto{\pgfqpoint{5.376021in}{2.832752in}}%
\pgfpathlineto{\pgfqpoint{5.387650in}{2.816693in}}%
\pgfpathlineto{\pgfqpoint{5.399300in}{2.800611in}}%
\pgfpathlineto{\pgfqpoint{5.432627in}{2.799489in}}%
\pgfpathlineto{\pgfqpoint{5.465937in}{2.798852in}}%
\pgfpathlineto{\pgfqpoint{5.454232in}{2.814777in}}%
\pgfpathlineto{\pgfqpoint{5.442549in}{2.830702in}}%
\pgfpathlineto{\pgfqpoint{5.409294in}{2.831480in}}%
\pgfpathlineto{\pgfqpoint{5.376021in}{2.832752in}}%
\pgfpathclose%
\pgfusepath{fill}%
\end{pgfscope}%
\begin{pgfscope}%
\pgfpathrectangle{\pgfqpoint{1.020000in}{0.880000in}}{\pgfqpoint{6.160000in}{6.160000in}}%
\pgfusepath{clip}%
\pgfsetbuttcap%
\pgfsetroundjoin%
\definecolor{currentfill}{rgb}{0.708720,0.805721,0.981117}%
\pgfsetfillcolor{currentfill}%
\pgfsetlinewidth{0.000000pt}%
\definecolor{currentstroke}{rgb}{0.000000,0.000000,0.000000}%
\pgfsetstrokecolor{currentstroke}%
\pgfsetdash{}{0pt}%
\pgfpathmoveto{\pgfqpoint{2.722555in}{3.545422in}}%
\pgfpathlineto{\pgfqpoint{2.732016in}{3.509536in}}%
\pgfpathlineto{\pgfqpoint{2.741465in}{3.475485in}}%
\pgfpathlineto{\pgfqpoint{2.775518in}{3.488097in}}%
\pgfpathlineto{\pgfqpoint{2.809551in}{3.501174in}}%
\pgfpathlineto{\pgfqpoint{2.800047in}{3.535778in}}%
\pgfpathlineto{\pgfqpoint{2.790529in}{3.572496in}}%
\pgfpathlineto{\pgfqpoint{2.756552in}{3.558776in}}%
\pgfpathlineto{\pgfqpoint{2.722555in}{3.545422in}}%
\pgfpathclose%
\pgfusepath{fill}%
\end{pgfscope}%
\begin{pgfscope}%
\pgfpathrectangle{\pgfqpoint{1.020000in}{0.880000in}}{\pgfqpoint{6.160000in}{6.160000in}}%
\pgfusepath{clip}%
\pgfsetbuttcap%
\pgfsetroundjoin%
\definecolor{currentfill}{rgb}{0.419991,0.552989,0.942630}%
\pgfsetfillcolor{currentfill}%
\pgfsetlinewidth{0.000000pt}%
\definecolor{currentstroke}{rgb}{0.000000,0.000000,0.000000}%
\pgfsetstrokecolor{currentstroke}%
\pgfsetdash{}{0pt}%
\pgfpathmoveto{\pgfqpoint{4.841277in}{3.040061in}}%
\pgfpathlineto{\pgfqpoint{4.852425in}{3.025867in}}%
\pgfpathlineto{\pgfqpoint{4.863591in}{3.011146in}}%
\pgfpathlineto{\pgfqpoint{4.897009in}{2.994586in}}%
\pgfpathlineto{\pgfqpoint{4.930404in}{2.979625in}}%
\pgfpathlineto{\pgfqpoint{4.919189in}{2.994724in}}%
\pgfpathlineto{\pgfqpoint{4.907992in}{3.009414in}}%
\pgfpathlineto{\pgfqpoint{4.874645in}{3.023965in}}%
\pgfpathlineto{\pgfqpoint{4.841277in}{3.040061in}}%
\pgfpathclose%
\pgfusepath{fill}%
\end{pgfscope}%
\begin{pgfscope}%
\pgfpathrectangle{\pgfqpoint{1.020000in}{0.880000in}}{\pgfqpoint{6.160000in}{6.160000in}}%
\pgfusepath{clip}%
\pgfsetbuttcap%
\pgfsetroundjoin%
\definecolor{currentfill}{rgb}{0.378598,0.503856,0.913692}%
\pgfsetfillcolor{currentfill}%
\pgfsetlinewidth{0.000000pt}%
\definecolor{currentstroke}{rgb}{0.000000,0.000000,0.000000}%
\pgfsetstrokecolor{currentstroke}%
\pgfsetdash{}{0pt}%
\pgfpathmoveto{\pgfqpoint{4.997135in}{2.954244in}}%
\pgfpathlineto{\pgfqpoint{5.008418in}{2.938539in}}%
\pgfpathlineto{\pgfqpoint{5.019720in}{2.922508in}}%
\pgfpathlineto{\pgfqpoint{5.053108in}{2.911848in}}%
\pgfpathlineto{\pgfqpoint{5.086478in}{2.902513in}}%
\pgfpathlineto{\pgfqpoint{5.075124in}{2.918581in}}%
\pgfpathlineto{\pgfqpoint{5.063790in}{2.934409in}}%
\pgfpathlineto{\pgfqpoint{5.030471in}{2.943675in}}%
\pgfpathlineto{\pgfqpoint{4.997135in}{2.954244in}}%
\pgfpathclose%
\pgfusepath{fill}%
\end{pgfscope}%
\begin{pgfscope}%
\pgfpathrectangle{\pgfqpoint{1.020000in}{0.880000in}}{\pgfqpoint{6.160000in}{6.160000in}}%
\pgfusepath{clip}%
\pgfsetbuttcap%
\pgfsetroundjoin%
\definecolor{currentfill}{rgb}{0.768034,0.837035,0.952488}%
\pgfsetfillcolor{currentfill}%
\pgfsetlinewidth{0.000000pt}%
\definecolor{currentstroke}{rgb}{0.000000,0.000000,0.000000}%
\pgfsetstrokecolor{currentstroke}%
\pgfsetdash{}{0pt}%
\pgfpathmoveto{\pgfqpoint{3.527111in}{3.639360in}}%
\pgfpathlineto{\pgfqpoint{3.536975in}{3.634214in}}%
\pgfpathlineto{\pgfqpoint{3.546842in}{3.631871in}}%
\pgfpathlineto{\pgfqpoint{3.580784in}{3.636616in}}%
\pgfpathlineto{\pgfqpoint{3.614716in}{3.639492in}}%
\pgfpathlineto{\pgfqpoint{3.604794in}{3.639793in}}%
\pgfpathlineto{\pgfqpoint{3.594879in}{3.642837in}}%
\pgfpathlineto{\pgfqpoint{3.561001in}{3.641953in}}%
\pgfpathlineto{\pgfqpoint{3.527111in}{3.639360in}}%
\pgfpathclose%
\pgfusepath{fill}%
\end{pgfscope}%
\begin{pgfscope}%
\pgfpathrectangle{\pgfqpoint{1.020000in}{0.880000in}}{\pgfqpoint{6.160000in}{6.160000in}}%
\pgfusepath{clip}%
\pgfsetbuttcap%
\pgfsetroundjoin%
\definecolor{currentfill}{rgb}{0.478462,0.616564,0.972721}%
\pgfsetfillcolor{currentfill}%
\pgfsetlinewidth{0.000000pt}%
\definecolor{currentstroke}{rgb}{0.000000,0.000000,0.000000}%
\pgfsetstrokecolor{currentstroke}%
\pgfsetdash{}{0pt}%
\pgfpathmoveto{\pgfqpoint{4.685504in}{3.143957in}}%
\pgfpathlineto{\pgfqpoint{4.696521in}{3.132636in}}%
\pgfpathlineto{\pgfqpoint{4.707557in}{3.120591in}}%
\pgfpathlineto{\pgfqpoint{4.741026in}{3.098008in}}%
\pgfpathlineto{\pgfqpoint{4.774469in}{3.077058in}}%
\pgfpathlineto{\pgfqpoint{4.763386in}{3.090003in}}%
\pgfpathlineto{\pgfqpoint{4.752321in}{3.102367in}}%
\pgfpathlineto{\pgfqpoint{4.718926in}{3.122384in}}%
\pgfpathlineto{\pgfqpoint{4.685504in}{3.143957in}}%
\pgfpathclose%
\pgfusepath{fill}%
\end{pgfscope}%
\begin{pgfscope}%
\pgfpathrectangle{\pgfqpoint{1.020000in}{0.880000in}}{\pgfqpoint{6.160000in}{6.160000in}}%
\pgfusepath{clip}%
\pgfsetbuttcap%
\pgfsetroundjoin%
\definecolor{currentfill}{rgb}{0.763363,0.835092,0.955658}%
\pgfsetfillcolor{currentfill}%
\pgfsetlinewidth{0.000000pt}%
\definecolor{currentstroke}{rgb}{0.000000,0.000000,0.000000}%
\pgfsetstrokecolor{currentstroke}%
\pgfsetdash{}{0pt}%
\pgfpathmoveto{\pgfqpoint{3.750313in}{3.631768in}}%
\pgfpathlineto{\pgfqpoint{3.760365in}{3.636579in}}%
\pgfpathlineto{\pgfqpoint{3.770433in}{3.643304in}}%
\pgfpathlineto{\pgfqpoint{3.804356in}{3.637286in}}%
\pgfpathlineto{\pgfqpoint{3.838259in}{3.629213in}}%
\pgfpathlineto{\pgfqpoint{3.828125in}{3.622033in}}%
\pgfpathlineto{\pgfqpoint{3.818008in}{3.616637in}}%
\pgfpathlineto{\pgfqpoint{3.784171in}{3.625098in}}%
\pgfpathlineto{\pgfqpoint{3.750313in}{3.631768in}}%
\pgfpathclose%
\pgfusepath{fill}%
\end{pgfscope}%
\begin{pgfscope}%
\pgfpathrectangle{\pgfqpoint{1.020000in}{0.880000in}}{\pgfqpoint{6.160000in}{6.160000in}}%
\pgfusepath{clip}%
\pgfsetbuttcap%
\pgfsetroundjoin%
\definecolor{currentfill}{rgb}{0.718985,0.811993,0.977656}%
\pgfsetfillcolor{currentfill}%
\pgfsetlinewidth{0.000000pt}%
\definecolor{currentstroke}{rgb}{0.000000,0.000000,0.000000}%
\pgfsetstrokecolor{currentstroke}%
\pgfsetdash{}{0pt}%
\pgfpathmoveto{\pgfqpoint{2.945503in}{3.556343in}}%
\pgfpathlineto{\pgfqpoint{2.955088in}{3.524028in}}%
\pgfpathlineto{\pgfqpoint{2.964659in}{3.494254in}}%
\pgfpathlineto{\pgfqpoint{2.998652in}{3.508877in}}%
\pgfpathlineto{\pgfqpoint{3.032632in}{3.523515in}}%
\pgfpathlineto{\pgfqpoint{3.023021in}{3.552431in}}%
\pgfpathlineto{\pgfqpoint{3.013394in}{3.584148in}}%
\pgfpathlineto{\pgfqpoint{2.979454in}{3.570325in}}%
\pgfpathlineto{\pgfqpoint{2.945503in}{3.556343in}}%
\pgfpathclose%
\pgfusepath{fill}%
\end{pgfscope}%
\begin{pgfscope}%
\pgfpathrectangle{\pgfqpoint{1.020000in}{0.880000in}}{\pgfqpoint{6.160000in}{6.160000in}}%
\pgfusepath{clip}%
\pgfsetbuttcap%
\pgfsetroundjoin%
\definecolor{currentfill}{rgb}{0.748682,0.827679,0.963334}%
\pgfsetfillcolor{currentfill}%
\pgfsetlinewidth{0.000000pt}%
\definecolor{currentstroke}{rgb}{0.000000,0.000000,0.000000}%
\pgfsetstrokecolor{currentstroke}%
\pgfsetdash{}{0pt}%
\pgfpathmoveto{\pgfqpoint{3.236290in}{3.604515in}}%
\pgfpathlineto{\pgfqpoint{3.245998in}{3.583500in}}%
\pgfpathlineto{\pgfqpoint{3.255697in}{3.565584in}}%
\pgfpathlineto{\pgfqpoint{3.289647in}{3.578981in}}%
\pgfpathlineto{\pgfqpoint{3.323592in}{3.591438in}}%
\pgfpathlineto{\pgfqpoint{3.313855in}{3.606978in}}%
\pgfpathlineto{\pgfqpoint{3.304111in}{3.625738in}}%
\pgfpathlineto{\pgfqpoint{3.270204in}{3.615638in}}%
\pgfpathlineto{\pgfqpoint{3.236290in}{3.604515in}}%
\pgfpathclose%
\pgfusepath{fill}%
\end{pgfscope}%
\begin{pgfscope}%
\pgfpathrectangle{\pgfqpoint{1.020000in}{0.880000in}}{\pgfqpoint{6.160000in}{6.160000in}}%
\pgfusepath{clip}%
\pgfsetbuttcap%
\pgfsetroundjoin%
\definecolor{currentfill}{rgb}{0.343278,0.459354,0.884122}%
\pgfsetfillcolor{currentfill}%
\pgfsetlinewidth{0.000000pt}%
\definecolor{currentstroke}{rgb}{0.000000,0.000000,0.000000}%
\pgfsetstrokecolor{currentstroke}%
\pgfsetdash{}{0pt}%
\pgfpathmoveto{\pgfqpoint{5.153166in}{2.887461in}}%
\pgfpathlineto{\pgfqpoint{5.164591in}{2.871221in}}%
\pgfpathlineto{\pgfqpoint{5.176037in}{2.854819in}}%
\pgfpathlineto{\pgfqpoint{5.209409in}{2.849021in}}%
\pgfpathlineto{\pgfqpoint{5.242764in}{2.844185in}}%
\pgfpathlineto{\pgfqpoint{5.231265in}{2.860457in}}%
\pgfpathlineto{\pgfqpoint{5.219787in}{2.876622in}}%
\pgfpathlineto{\pgfqpoint{5.186485in}{2.881560in}}%
\pgfpathlineto{\pgfqpoint{5.153166in}{2.887461in}}%
\pgfpathclose%
\pgfusepath{fill}%
\end{pgfscope}%
\begin{pgfscope}%
\pgfpathrectangle{\pgfqpoint{1.020000in}{0.880000in}}{\pgfqpoint{6.160000in}{6.160000in}}%
\pgfusepath{clip}%
\pgfsetbuttcap%
\pgfsetroundjoin%
\definecolor{currentfill}{rgb}{0.285273,0.380129,0.823469}%
\pgfsetfillcolor{currentfill}%
\pgfsetlinewidth{0.000000pt}%
\definecolor{currentstroke}{rgb}{0.000000,0.000000,0.000000}%
\pgfsetstrokecolor{currentstroke}%
\pgfsetdash{}{0pt}%
\pgfpathmoveto{\pgfqpoint{5.978940in}{2.756505in}}%
\pgfpathlineto{\pgfqpoint{5.991149in}{2.741599in}}%
\pgfpathlineto{\pgfqpoint{6.003383in}{2.726743in}}%
\pgfpathlineto{\pgfqpoint{6.036575in}{2.729132in}}%
\pgfpathlineto{\pgfqpoint{6.069747in}{2.731555in}}%
\pgfpathlineto{\pgfqpoint{6.057457in}{2.746324in}}%
\pgfpathlineto{\pgfqpoint{6.045192in}{2.761143in}}%
\pgfpathlineto{\pgfqpoint{6.012076in}{2.758806in}}%
\pgfpathlineto{\pgfqpoint{5.978940in}{2.756505in}}%
\pgfpathclose%
\pgfusepath{fill}%
\end{pgfscope}%
\begin{pgfscope}%
\pgfpathrectangle{\pgfqpoint{1.020000in}{0.880000in}}{\pgfqpoint{6.160000in}{6.160000in}}%
\pgfusepath{clip}%
\pgfsetbuttcap%
\pgfsetroundjoin%
\definecolor{currentfill}{rgb}{0.289996,0.386836,0.828926}%
\pgfsetfillcolor{currentfill}%
\pgfsetlinewidth{0.000000pt}%
\definecolor{currentstroke}{rgb}{0.000000,0.000000,0.000000}%
\pgfsetstrokecolor{currentstroke}%
\pgfsetdash{}{0pt}%
\pgfpathmoveto{\pgfqpoint{5.755690in}{2.774492in}}%
\pgfpathlineto{\pgfqpoint{5.767685in}{2.759186in}}%
\pgfpathlineto{\pgfqpoint{5.779704in}{2.743927in}}%
\pgfpathlineto{\pgfqpoint{5.812959in}{2.745820in}}%
\pgfpathlineto{\pgfqpoint{5.846195in}{2.747813in}}%
\pgfpathlineto{\pgfqpoint{5.834120in}{2.762963in}}%
\pgfpathlineto{\pgfqpoint{5.822069in}{2.778162in}}%
\pgfpathlineto{\pgfqpoint{5.788889in}{2.776275in}}%
\pgfpathlineto{\pgfqpoint{5.755690in}{2.774492in}}%
\pgfpathclose%
\pgfusepath{fill}%
\end{pgfscope}%
\begin{pgfscope}%
\pgfpathrectangle{\pgfqpoint{1.020000in}{0.880000in}}{\pgfqpoint{6.160000in}{6.160000in}}%
\pgfusepath{clip}%
\pgfsetbuttcap%
\pgfsetroundjoin%
\definecolor{currentfill}{rgb}{0.698454,0.799450,0.984577}%
\pgfsetfillcolor{currentfill}%
\pgfsetlinewidth{0.000000pt}%
\definecolor{currentstroke}{rgb}{0.000000,0.000000,0.000000}%
\pgfsetstrokecolor{currentstroke}%
\pgfsetdash{}{0pt}%
\pgfpathmoveto{\pgfqpoint{2.654499in}{3.520168in}}%
\pgfpathlineto{\pgfqpoint{2.663899in}{3.485267in}}%
\pgfpathlineto{\pgfqpoint{2.673291in}{3.451935in}}%
\pgfpathlineto{\pgfqpoint{2.707390in}{3.463410in}}%
\pgfpathlineto{\pgfqpoint{2.741465in}{3.475485in}}%
\pgfpathlineto{\pgfqpoint{2.732016in}{3.509536in}}%
\pgfpathlineto{\pgfqpoint{2.722555in}{3.545422in}}%
\pgfpathlineto{\pgfqpoint{2.688538in}{3.532527in}}%
\pgfpathlineto{\pgfqpoint{2.654499in}{3.520168in}}%
\pgfpathclose%
\pgfusepath{fill}%
\end{pgfscope}%
\begin{pgfscope}%
\pgfpathrectangle{\pgfqpoint{1.020000in}{0.880000in}}{\pgfqpoint{6.160000in}{6.160000in}}%
\pgfusepath{clip}%
\pgfsetbuttcap%
\pgfsetroundjoin%
\definecolor{currentfill}{rgb}{0.543440,0.680003,0.993051}%
\pgfsetfillcolor{currentfill}%
\pgfsetlinewidth{0.000000pt}%
\definecolor{currentstroke}{rgb}{0.000000,0.000000,0.000000}%
\pgfsetstrokecolor{currentstroke}%
\pgfsetdash{}{0pt}%
\pgfpathmoveto{\pgfqpoint{4.529716in}{3.259739in}}%
\pgfpathlineto{\pgfqpoint{4.540601in}{3.252776in}}%
\pgfpathlineto{\pgfqpoint{4.551505in}{3.244976in}}%
\pgfpathlineto{\pgfqpoint{4.585054in}{3.217654in}}%
\pgfpathlineto{\pgfqpoint{4.618569in}{3.191650in}}%
\pgfpathlineto{\pgfqpoint{4.607617in}{3.200967in}}%
\pgfpathlineto{\pgfqpoint{4.596684in}{3.209588in}}%
\pgfpathlineto{\pgfqpoint{4.563217in}{3.234043in}}%
\pgfpathlineto{\pgfqpoint{4.529716in}{3.259739in}}%
\pgfpathclose%
\pgfusepath{fill}%
\end{pgfscope}%
\begin{pgfscope}%
\pgfpathrectangle{\pgfqpoint{1.020000in}{0.880000in}}{\pgfqpoint{6.160000in}{6.160000in}}%
\pgfusepath{clip}%
\pgfsetbuttcap%
\pgfsetroundjoin%
\definecolor{currentfill}{rgb}{0.304174,0.406945,0.845263}%
\pgfsetfillcolor{currentfill}%
\pgfsetlinewidth{0.000000pt}%
\definecolor{currentstroke}{rgb}{0.000000,0.000000,0.000000}%
\pgfsetstrokecolor{currentstroke}%
\pgfsetdash{}{0pt}%
\pgfpathmoveto{\pgfqpoint{5.532505in}{2.798808in}}%
\pgfpathlineto{\pgfqpoint{5.544288in}{2.783043in}}%
\pgfpathlineto{\pgfqpoint{5.556093in}{2.767301in}}%
\pgfpathlineto{\pgfqpoint{5.589406in}{2.767935in}}%
\pgfpathlineto{\pgfqpoint{5.622700in}{2.768841in}}%
\pgfpathlineto{\pgfqpoint{5.610839in}{2.784437in}}%
\pgfpathlineto{\pgfqpoint{5.599001in}{2.800067in}}%
\pgfpathlineto{\pgfqpoint{5.565762in}{2.799297in}}%
\pgfpathlineto{\pgfqpoint{5.532505in}{2.798808in}}%
\pgfpathclose%
\pgfusepath{fill}%
\end{pgfscope}%
\begin{pgfscope}%
\pgfpathrectangle{\pgfqpoint{1.020000in}{0.880000in}}{\pgfqpoint{6.160000in}{6.160000in}}%
\pgfusepath{clip}%
\pgfsetbuttcap%
\pgfsetroundjoin%
\definecolor{currentfill}{rgb}{0.275827,0.366717,0.812553}%
\pgfsetfillcolor{currentfill}%
\pgfsetlinewidth{0.000000pt}%
\definecolor{currentstroke}{rgb}{0.000000,0.000000,0.000000}%
\pgfsetstrokecolor{currentstroke}%
\pgfsetdash{}{0pt}%
\pgfpathmoveto{\pgfqpoint{6.202232in}{2.741491in}}%
\pgfpathlineto{\pgfqpoint{6.214657in}{2.726933in}}%
\pgfpathlineto{\pgfqpoint{6.227106in}{2.712422in}}%
\pgfpathlineto{\pgfqpoint{6.260232in}{2.715028in}}%
\pgfpathlineto{\pgfqpoint{6.247755in}{2.729500in}}%
\pgfpathlineto{\pgfqpoint{6.235302in}{2.744020in}}%
\pgfpathlineto{\pgfqpoint{6.202232in}{2.741491in}}%
\pgfpathclose%
\pgfusepath{fill}%
\end{pgfscope}%
\begin{pgfscope}%
\pgfpathrectangle{\pgfqpoint{1.020000in}{0.880000in}}{\pgfqpoint{6.160000in}{6.160000in}}%
\pgfusepath{clip}%
\pgfsetbuttcap%
\pgfsetroundjoin%
\definecolor{currentfill}{rgb}{0.323718,0.433158,0.864722}%
\pgfsetfillcolor{currentfill}%
\pgfsetlinewidth{0.000000pt}%
\definecolor{currentstroke}{rgb}{0.000000,0.000000,0.000000}%
\pgfsetstrokecolor{currentstroke}%
\pgfsetdash{}{0pt}%
\pgfpathmoveto{\pgfqpoint{5.309426in}{2.837050in}}%
\pgfpathlineto{\pgfqpoint{5.321000in}{2.820842in}}%
\pgfpathlineto{\pgfqpoint{5.332595in}{2.804581in}}%
\pgfpathlineto{\pgfqpoint{5.365956in}{2.802286in}}%
\pgfpathlineto{\pgfqpoint{5.399300in}{2.800611in}}%
\pgfpathlineto{\pgfqpoint{5.387650in}{2.816693in}}%
\pgfpathlineto{\pgfqpoint{5.376021in}{2.832752in}}%
\pgfpathlineto{\pgfqpoint{5.342732in}{2.834585in}}%
\pgfpathlineto{\pgfqpoint{5.309426in}{2.837050in}}%
\pgfpathclose%
\pgfusepath{fill}%
\end{pgfscope}%
\begin{pgfscope}%
\pgfpathrectangle{\pgfqpoint{1.020000in}{0.880000in}}{\pgfqpoint{6.160000in}{6.160000in}}%
\pgfusepath{clip}%
\pgfsetbuttcap%
\pgfsetroundjoin%
\definecolor{currentfill}{rgb}{0.703587,0.802586,0.982847}%
\pgfsetfillcolor{currentfill}%
\pgfsetlinewidth{0.000000pt}%
\definecolor{currentstroke}{rgb}{0.000000,0.000000,0.000000}%
\pgfsetstrokecolor{currentstroke}%
\pgfsetdash{}{0pt}%
\pgfpathmoveto{\pgfqpoint{2.877559in}{3.528384in}}%
\pgfpathlineto{\pgfqpoint{2.887099in}{3.495810in}}%
\pgfpathlineto{\pgfqpoint{2.896626in}{3.465510in}}%
\pgfpathlineto{\pgfqpoint{2.930651in}{3.479762in}}%
\pgfpathlineto{\pgfqpoint{2.964659in}{3.494254in}}%
\pgfpathlineto{\pgfqpoint{2.955088in}{3.524028in}}%
\pgfpathlineto{\pgfqpoint{2.945503in}{3.556343in}}%
\pgfpathlineto{\pgfqpoint{2.911539in}{3.542324in}}%
\pgfpathlineto{\pgfqpoint{2.877559in}{3.528384in}}%
\pgfpathclose%
\pgfusepath{fill}%
\end{pgfscope}%
\begin{pgfscope}%
\pgfpathrectangle{\pgfqpoint{1.020000in}{0.880000in}}{\pgfqpoint{6.160000in}{6.160000in}}%
\pgfusepath{clip}%
\pgfsetbuttcap%
\pgfsetroundjoin%
\definecolor{currentfill}{rgb}{0.738826,0.822572,0.968261}%
\pgfsetfillcolor{currentfill}%
\pgfsetlinewidth{0.000000pt}%
\definecolor{currentstroke}{rgb}{0.000000,0.000000,0.000000}%
\pgfsetstrokecolor{currentstroke}%
\pgfsetdash{}{0pt}%
\pgfpathmoveto{\pgfqpoint{3.168440in}{3.579701in}}%
\pgfpathlineto{\pgfqpoint{3.178112in}{3.556653in}}%
\pgfpathlineto{\pgfqpoint{3.187773in}{3.536544in}}%
\pgfpathlineto{\pgfqpoint{3.221739in}{3.551389in}}%
\pgfpathlineto{\pgfqpoint{3.255697in}{3.565584in}}%
\pgfpathlineto{\pgfqpoint{3.245998in}{3.583500in}}%
\pgfpathlineto{\pgfqpoint{3.236290in}{3.604515in}}%
\pgfpathlineto{\pgfqpoint{3.202369in}{3.592493in}}%
\pgfpathlineto{\pgfqpoint{3.168440in}{3.579701in}}%
\pgfpathclose%
\pgfusepath{fill}%
\end{pgfscope}%
\begin{pgfscope}%
\pgfpathrectangle{\pgfqpoint{1.020000in}{0.880000in}}{\pgfqpoint{6.160000in}{6.160000in}}%
\pgfusepath{clip}%
\pgfsetbuttcap%
\pgfsetroundjoin%
\definecolor{currentfill}{rgb}{0.768034,0.837035,0.952488}%
\pgfsetfillcolor{currentfill}%
\pgfsetlinewidth{0.000000pt}%
\definecolor{currentstroke}{rgb}{0.000000,0.000000,0.000000}%
\pgfsetstrokecolor{currentstroke}%
\pgfsetdash{}{0pt}%
\pgfpathmoveto{\pgfqpoint{3.459302in}{3.629225in}}%
\pgfpathlineto{\pgfqpoint{3.469118in}{3.621727in}}%
\pgfpathlineto{\pgfqpoint{3.478935in}{3.617052in}}%
\pgfpathlineto{\pgfqpoint{3.512892in}{3.625321in}}%
\pgfpathlineto{\pgfqpoint{3.546842in}{3.631871in}}%
\pgfpathlineto{\pgfqpoint{3.536975in}{3.634214in}}%
\pgfpathlineto{\pgfqpoint{3.527111in}{3.639360in}}%
\pgfpathlineto{\pgfqpoint{3.493211in}{3.635097in}}%
\pgfpathlineto{\pgfqpoint{3.459302in}{3.629225in}}%
\pgfpathclose%
\pgfusepath{fill}%
\end{pgfscope}%
\begin{pgfscope}%
\pgfpathrectangle{\pgfqpoint{1.020000in}{0.880000in}}{\pgfqpoint{6.160000in}{6.160000in}}%
\pgfusepath{clip}%
\pgfsetbuttcap%
\pgfsetroundjoin%
\definecolor{currentfill}{rgb}{0.753611,0.830233,0.960871}%
\pgfsetfillcolor{currentfill}%
\pgfsetlinewidth{0.000000pt}%
\definecolor{currentstroke}{rgb}{0.000000,0.000000,0.000000}%
\pgfsetstrokecolor{currentstroke}%
\pgfsetdash{}{0pt}%
\pgfpathmoveto{\pgfqpoint{3.905998in}{3.607284in}}%
\pgfpathlineto{\pgfqpoint{3.916220in}{3.615712in}}%
\pgfpathlineto{\pgfqpoint{3.926463in}{3.625137in}}%
\pgfpathlineto{\pgfqpoint{3.960364in}{3.610873in}}%
\pgfpathlineto{\pgfqpoint{3.994237in}{3.594794in}}%
\pgfpathlineto{\pgfqpoint{3.983924in}{3.586202in}}%
\pgfpathlineto{\pgfqpoint{3.973633in}{3.578498in}}%
\pgfpathlineto{\pgfqpoint{3.939830in}{3.593675in}}%
\pgfpathlineto{\pgfqpoint{3.905998in}{3.607284in}}%
\pgfpathclose%
\pgfusepath{fill}%
\end{pgfscope}%
\begin{pgfscope}%
\pgfpathrectangle{\pgfqpoint{1.020000in}{0.880000in}}{\pgfqpoint{6.160000in}{6.160000in}}%
\pgfusepath{clip}%
\pgfsetbuttcap%
\pgfsetroundjoin%
\definecolor{currentfill}{rgb}{0.613933,0.739923,0.999142}%
\pgfsetfillcolor{currentfill}%
\pgfsetlinewidth{0.000000pt}%
\definecolor{currentstroke}{rgb}{0.000000,0.000000,0.000000}%
\pgfsetstrokecolor{currentstroke}%
\pgfsetdash{}{0pt}%
\pgfpathmoveto{\pgfqpoint{4.373843in}{3.376111in}}%
\pgfpathlineto{\pgfqpoint{4.384585in}{3.374586in}}%
\pgfpathlineto{\pgfqpoint{4.395347in}{3.372290in}}%
\pgfpathlineto{\pgfqpoint{4.428997in}{3.343012in}}%
\pgfpathlineto{\pgfqpoint{4.462608in}{3.314373in}}%
\pgfpathlineto{\pgfqpoint{4.451793in}{3.318700in}}%
\pgfpathlineto{\pgfqpoint{4.440999in}{3.322359in}}%
\pgfpathlineto{\pgfqpoint{4.407440in}{3.348933in}}%
\pgfpathlineto{\pgfqpoint{4.373843in}{3.376111in}}%
\pgfpathclose%
\pgfusepath{fill}%
\end{pgfscope}%
\begin{pgfscope}%
\pgfpathrectangle{\pgfqpoint{1.020000in}{0.880000in}}{\pgfqpoint{6.160000in}{6.160000in}}%
\pgfusepath{clip}%
\pgfsetbuttcap%
\pgfsetroundjoin%
\definecolor{currentfill}{rgb}{0.688188,0.793178,0.988038}%
\pgfsetfillcolor{currentfill}%
\pgfsetlinewidth{0.000000pt}%
\definecolor{currentstroke}{rgb}{0.000000,0.000000,0.000000}%
\pgfsetstrokecolor{currentstroke}%
\pgfsetdash{}{0pt}%
\pgfpathmoveto{\pgfqpoint{2.586347in}{3.497327in}}%
\pgfpathlineto{\pgfqpoint{2.595686in}{3.463490in}}%
\pgfpathlineto{\pgfqpoint{2.605018in}{3.430977in}}%
\pgfpathlineto{\pgfqpoint{2.639167in}{3.441110in}}%
\pgfpathlineto{\pgfqpoint{2.673291in}{3.451935in}}%
\pgfpathlineto{\pgfqpoint{2.663899in}{3.485267in}}%
\pgfpathlineto{\pgfqpoint{2.654499in}{3.520168in}}%
\pgfpathlineto{\pgfqpoint{2.620436in}{3.508416in}}%
\pgfpathlineto{\pgfqpoint{2.586347in}{3.497327in}}%
\pgfpathclose%
\pgfusepath{fill}%
\end{pgfscope}%
\begin{pgfscope}%
\pgfpathrectangle{\pgfqpoint{1.020000in}{0.880000in}}{\pgfqpoint{6.160000in}{6.160000in}}%
\pgfusepath{clip}%
\pgfsetbuttcap%
\pgfsetroundjoin%
\definecolor{currentfill}{rgb}{0.394042,0.522413,0.924916}%
\pgfsetfillcolor{currentfill}%
\pgfsetlinewidth{0.000000pt}%
\definecolor{currentstroke}{rgb}{0.000000,0.000000,0.000000}%
\pgfsetstrokecolor{currentstroke}%
\pgfsetdash{}{0pt}%
\pgfpathmoveto{\pgfqpoint{4.930404in}{2.979625in}}%
\pgfpathlineto{\pgfqpoint{4.941638in}{2.964103in}}%
\pgfpathlineto{\pgfqpoint{4.952889in}{2.948145in}}%
\pgfpathlineto{\pgfqpoint{4.986314in}{2.934580in}}%
\pgfpathlineto{\pgfqpoint{5.019720in}{2.922508in}}%
\pgfpathlineto{\pgfqpoint{5.008418in}{2.938539in}}%
\pgfpathlineto{\pgfqpoint{4.997135in}{2.954244in}}%
\pgfpathlineto{\pgfqpoint{4.963779in}{2.966201in}}%
\pgfpathlineto{\pgfqpoint{4.930404in}{2.979625in}}%
\pgfpathclose%
\pgfusepath{fill}%
\end{pgfscope}%
\begin{pgfscope}%
\pgfpathrectangle{\pgfqpoint{1.020000in}{0.880000in}}{\pgfqpoint{6.160000in}{6.160000in}}%
\pgfusepath{clip}%
\pgfsetbuttcap%
\pgfsetroundjoin%
\definecolor{currentfill}{rgb}{0.772706,0.838978,0.949319}%
\pgfsetfillcolor{currentfill}%
\pgfsetlinewidth{0.000000pt}%
\definecolor{currentstroke}{rgb}{0.000000,0.000000,0.000000}%
\pgfsetstrokecolor{currentstroke}%
\pgfsetdash{}{0pt}%
\pgfpathmoveto{\pgfqpoint{3.682544in}{3.639483in}}%
\pgfpathlineto{\pgfqpoint{3.692533in}{3.643197in}}%
\pgfpathlineto{\pgfqpoint{3.702536in}{3.648936in}}%
\pgfpathlineto{\pgfqpoint{3.736492in}{3.647201in}}%
\pgfpathlineto{\pgfqpoint{3.770433in}{3.643304in}}%
\pgfpathlineto{\pgfqpoint{3.760365in}{3.636579in}}%
\pgfpathlineto{\pgfqpoint{3.750313in}{3.631768in}}%
\pgfpathlineto{\pgfqpoint{3.716437in}{3.636579in}}%
\pgfpathlineto{\pgfqpoint{3.682544in}{3.639483in}}%
\pgfpathclose%
\pgfusepath{fill}%
\end{pgfscope}%
\begin{pgfscope}%
\pgfpathrectangle{\pgfqpoint{1.020000in}{0.880000in}}{\pgfqpoint{6.160000in}{6.160000in}}%
\pgfusepath{clip}%
\pgfsetbuttcap%
\pgfsetroundjoin%
\definecolor{currentfill}{rgb}{0.275827,0.366717,0.812553}%
\pgfsetfillcolor{currentfill}%
\pgfsetlinewidth{0.000000pt}%
\definecolor{currentstroke}{rgb}{0.000000,0.000000,0.000000}%
\pgfsetstrokecolor{currentstroke}%
\pgfsetdash{}{0pt}%
\pgfpathmoveto{\pgfqpoint{6.136030in}{2.736482in}}%
\pgfpathlineto{\pgfqpoint{6.148399in}{2.721845in}}%
\pgfpathlineto{\pgfqpoint{6.160793in}{2.707257in}}%
\pgfpathlineto{\pgfqpoint{6.193960in}{2.709831in}}%
\pgfpathlineto{\pgfqpoint{6.227106in}{2.712422in}}%
\pgfpathlineto{\pgfqpoint{6.214657in}{2.726933in}}%
\pgfpathlineto{\pgfqpoint{6.202232in}{2.741491in}}%
\pgfpathlineto{\pgfqpoint{6.169141in}{2.738977in}}%
\pgfpathlineto{\pgfqpoint{6.136030in}{2.736482in}}%
\pgfpathclose%
\pgfusepath{fill}%
\end{pgfscope}%
\begin{pgfscope}%
\pgfpathrectangle{\pgfqpoint{1.020000in}{0.880000in}}{\pgfqpoint{6.160000in}{6.160000in}}%
\pgfusepath{clip}%
\pgfsetbuttcap%
\pgfsetroundjoin%
\definecolor{currentfill}{rgb}{0.285273,0.380129,0.823469}%
\pgfsetfillcolor{currentfill}%
\pgfsetlinewidth{0.000000pt}%
\definecolor{currentstroke}{rgb}{0.000000,0.000000,0.000000}%
\pgfsetstrokecolor{currentstroke}%
\pgfsetdash{}{0pt}%
\pgfpathmoveto{\pgfqpoint{5.912607in}{2.752039in}}%
\pgfpathlineto{\pgfqpoint{5.924761in}{2.737039in}}%
\pgfpathlineto{\pgfqpoint{5.936939in}{2.722091in}}%
\pgfpathlineto{\pgfqpoint{5.970171in}{2.724393in}}%
\pgfpathlineto{\pgfqpoint{6.003383in}{2.726743in}}%
\pgfpathlineto{\pgfqpoint{5.991149in}{2.741599in}}%
\pgfpathlineto{\pgfqpoint{5.978940in}{2.756505in}}%
\pgfpathlineto{\pgfqpoint{5.945783in}{2.754247in}}%
\pgfpathlineto{\pgfqpoint{5.912607in}{2.752039in}}%
\pgfpathclose%
\pgfusepath{fill}%
\end{pgfscope}%
\begin{pgfscope}%
\pgfpathrectangle{\pgfqpoint{1.020000in}{0.880000in}}{\pgfqpoint{6.160000in}{6.160000in}}%
\pgfusepath{clip}%
\pgfsetbuttcap%
\pgfsetroundjoin%
\definecolor{currentfill}{rgb}{0.677823,0.786546,0.991005}%
\pgfsetfillcolor{currentfill}%
\pgfsetlinewidth{0.000000pt}%
\definecolor{currentstroke}{rgb}{0.000000,0.000000,0.000000}%
\pgfsetstrokecolor{currentstroke}%
\pgfsetdash{}{0pt}%
\pgfpathmoveto{\pgfqpoint{4.217878in}{3.479344in}}%
\pgfpathlineto{\pgfqpoint{4.228459in}{3.483263in}}%
\pgfpathlineto{\pgfqpoint{4.239063in}{3.486738in}}%
\pgfpathlineto{\pgfqpoint{4.272818in}{3.459250in}}%
\pgfpathlineto{\pgfqpoint{4.306533in}{3.431490in}}%
\pgfpathlineto{\pgfqpoint{4.295869in}{3.430202in}}%
\pgfpathlineto{\pgfqpoint{4.285229in}{3.428504in}}%
\pgfpathlineto{\pgfqpoint{4.251572in}{3.454035in}}%
\pgfpathlineto{\pgfqpoint{4.217878in}{3.479344in}}%
\pgfpathclose%
\pgfusepath{fill}%
\end{pgfscope}%
\begin{pgfscope}%
\pgfpathrectangle{\pgfqpoint{1.020000in}{0.880000in}}{\pgfqpoint{6.160000in}{6.160000in}}%
\pgfusepath{clip}%
\pgfsetbuttcap%
\pgfsetroundjoin%
\definecolor{currentfill}{rgb}{0.724041,0.814910,0.975651}%
\pgfsetfillcolor{currentfill}%
\pgfsetlinewidth{0.000000pt}%
\definecolor{currentstroke}{rgb}{0.000000,0.000000,0.000000}%
\pgfsetstrokecolor{currentstroke}%
\pgfsetdash{}{0pt}%
\pgfpathmoveto{\pgfqpoint{4.061890in}{3.557939in}}%
\pgfpathlineto{\pgfqpoint{4.072294in}{3.565719in}}%
\pgfpathlineto{\pgfqpoint{4.082722in}{3.573665in}}%
\pgfpathlineto{\pgfqpoint{4.116566in}{3.551605in}}%
\pgfpathlineto{\pgfqpoint{4.150374in}{3.528370in}}%
\pgfpathlineto{\pgfqpoint{4.139879in}{3.522214in}}%
\pgfpathlineto{\pgfqpoint{4.129409in}{3.516178in}}%
\pgfpathlineto{\pgfqpoint{4.095666in}{3.537565in}}%
\pgfpathlineto{\pgfqpoint{4.061890in}{3.557939in}}%
\pgfpathclose%
\pgfusepath{fill}%
\end{pgfscope}%
\begin{pgfscope}%
\pgfpathrectangle{\pgfqpoint{1.020000in}{0.880000in}}{\pgfqpoint{6.160000in}{6.160000in}}%
\pgfusepath{clip}%
\pgfsetbuttcap%
\pgfsetroundjoin%
\definecolor{currentfill}{rgb}{0.446431,0.582356,0.957373}%
\pgfsetfillcolor{currentfill}%
\pgfsetlinewidth{0.000000pt}%
\definecolor{currentstroke}{rgb}{0.000000,0.000000,0.000000}%
\pgfsetstrokecolor{currentstroke}%
\pgfsetdash{}{0pt}%
\pgfpathmoveto{\pgfqpoint{4.774469in}{3.077058in}}%
\pgfpathlineto{\pgfqpoint{4.785569in}{3.063483in}}%
\pgfpathlineto{\pgfqpoint{4.796687in}{3.049235in}}%
\pgfpathlineto{\pgfqpoint{4.830151in}{3.029351in}}%
\pgfpathlineto{\pgfqpoint{4.863591in}{3.011146in}}%
\pgfpathlineto{\pgfqpoint{4.852425in}{3.025867in}}%
\pgfpathlineto{\pgfqpoint{4.841277in}{3.040061in}}%
\pgfpathlineto{\pgfqpoint{4.807885in}{3.057748in}}%
\pgfpathlineto{\pgfqpoint{4.774469in}{3.077058in}}%
\pgfpathclose%
\pgfusepath{fill}%
\end{pgfscope}%
\begin{pgfscope}%
\pgfpathrectangle{\pgfqpoint{1.020000in}{0.880000in}}{\pgfqpoint{6.160000in}{6.160000in}}%
\pgfusepath{clip}%
\pgfsetbuttcap%
\pgfsetroundjoin%
\definecolor{currentfill}{rgb}{0.294718,0.393542,0.834384}%
\pgfsetfillcolor{currentfill}%
\pgfsetlinewidth{0.000000pt}%
\definecolor{currentstroke}{rgb}{0.000000,0.000000,0.000000}%
\pgfsetstrokecolor{currentstroke}%
\pgfsetdash{}{0pt}%
\pgfpathmoveto{\pgfqpoint{5.689233in}{2.771322in}}%
\pgfpathlineto{\pgfqpoint{5.701172in}{2.755896in}}%
\pgfpathlineto{\pgfqpoint{5.713135in}{2.740515in}}%
\pgfpathlineto{\pgfqpoint{5.746429in}{2.742151in}}%
\pgfpathlineto{\pgfqpoint{5.779704in}{2.743927in}}%
\pgfpathlineto{\pgfqpoint{5.767685in}{2.759186in}}%
\pgfpathlineto{\pgfqpoint{5.755690in}{2.774492in}}%
\pgfpathlineto{\pgfqpoint{5.722471in}{2.772834in}}%
\pgfpathlineto{\pgfqpoint{5.689233in}{2.771322in}}%
\pgfpathclose%
\pgfusepath{fill}%
\end{pgfscope}%
\begin{pgfscope}%
\pgfpathrectangle{\pgfqpoint{1.020000in}{0.880000in}}{\pgfqpoint{6.160000in}{6.160000in}}%
\pgfusepath{clip}%
\pgfsetbuttcap%
\pgfsetroundjoin%
\definecolor{currentfill}{rgb}{0.358415,0.478426,0.896795}%
\pgfsetfillcolor{currentfill}%
\pgfsetlinewidth{0.000000pt}%
\definecolor{currentstroke}{rgb}{0.000000,0.000000,0.000000}%
\pgfsetstrokecolor{currentstroke}%
\pgfsetdash{}{0pt}%
\pgfpathmoveto{\pgfqpoint{5.086478in}{2.902513in}}%
\pgfpathlineto{\pgfqpoint{5.097851in}{2.886208in}}%
\pgfpathlineto{\pgfqpoint{5.109243in}{2.869666in}}%
\pgfpathlineto{\pgfqpoint{5.142648in}{2.861670in}}%
\pgfpathlineto{\pgfqpoint{5.176037in}{2.854819in}}%
\pgfpathlineto{\pgfqpoint{5.164591in}{2.871221in}}%
\pgfpathlineto{\pgfqpoint{5.153166in}{2.887461in}}%
\pgfpathlineto{\pgfqpoint{5.119831in}{2.894415in}}%
\pgfpathlineto{\pgfqpoint{5.086478in}{2.902513in}}%
\pgfpathclose%
\pgfusepath{fill}%
\end{pgfscope}%
\begin{pgfscope}%
\pgfpathrectangle{\pgfqpoint{1.020000in}{0.880000in}}{\pgfqpoint{6.160000in}{6.160000in}}%
\pgfusepath{clip}%
\pgfsetbuttcap%
\pgfsetroundjoin%
\definecolor{currentfill}{rgb}{0.693321,0.796314,0.986308}%
\pgfsetfillcolor{currentfill}%
\pgfsetlinewidth{0.000000pt}%
\definecolor{currentstroke}{rgb}{0.000000,0.000000,0.000000}%
\pgfsetstrokecolor{currentstroke}%
\pgfsetdash{}{0pt}%
\pgfpathmoveto{\pgfqpoint{2.809551in}{3.501174in}}%
\pgfpathlineto{\pgfqpoint{2.819042in}{3.468639in}}%
\pgfpathlineto{\pgfqpoint{2.828521in}{3.438108in}}%
\pgfpathlineto{\pgfqpoint{2.862583in}{3.451596in}}%
\pgfpathlineto{\pgfqpoint{2.896626in}{3.465510in}}%
\pgfpathlineto{\pgfqpoint{2.887099in}{3.495810in}}%
\pgfpathlineto{\pgfqpoint{2.877559in}{3.528384in}}%
\pgfpathlineto{\pgfqpoint{2.843564in}{3.514634in}}%
\pgfpathlineto{\pgfqpoint{2.809551in}{3.501174in}}%
\pgfpathclose%
\pgfusepath{fill}%
\end{pgfscope}%
\begin{pgfscope}%
\pgfpathrectangle{\pgfqpoint{1.020000in}{0.880000in}}{\pgfqpoint{6.160000in}{6.160000in}}%
\pgfusepath{clip}%
\pgfsetbuttcap%
\pgfsetroundjoin%
\definecolor{currentfill}{rgb}{0.724041,0.814910,0.975651}%
\pgfsetfillcolor{currentfill}%
\pgfsetlinewidth{0.000000pt}%
\definecolor{currentstroke}{rgb}{0.000000,0.000000,0.000000}%
\pgfsetstrokecolor{currentstroke}%
\pgfsetdash{}{0pt}%
\pgfpathmoveto{\pgfqpoint{3.100556in}{3.552344in}}%
\pgfpathlineto{\pgfqpoint{3.110192in}{3.527542in}}%
\pgfpathlineto{\pgfqpoint{3.119816in}{3.505491in}}%
\pgfpathlineto{\pgfqpoint{3.153800in}{3.521196in}}%
\pgfpathlineto{\pgfqpoint{3.187773in}{3.536544in}}%
\pgfpathlineto{\pgfqpoint{3.178112in}{3.556653in}}%
\pgfpathlineto{\pgfqpoint{3.168440in}{3.579701in}}%
\pgfpathlineto{\pgfqpoint{3.134502in}{3.566273in}}%
\pgfpathlineto{\pgfqpoint{3.100556in}{3.552344in}}%
\pgfpathclose%
\pgfusepath{fill}%
\end{pgfscope}%
\begin{pgfscope}%
\pgfpathrectangle{\pgfqpoint{1.020000in}{0.880000in}}{\pgfqpoint{6.160000in}{6.160000in}}%
\pgfusepath{clip}%
\pgfsetbuttcap%
\pgfsetroundjoin%
\definecolor{currentfill}{rgb}{0.309060,0.413498,0.850128}%
\pgfsetfillcolor{currentfill}%
\pgfsetlinewidth{0.000000pt}%
\definecolor{currentstroke}{rgb}{0.000000,0.000000,0.000000}%
\pgfsetstrokecolor{currentstroke}%
\pgfsetdash{}{0pt}%
\pgfpathmoveto{\pgfqpoint{5.465937in}{2.798852in}}%
\pgfpathlineto{\pgfqpoint{5.477665in}{2.782933in}}%
\pgfpathlineto{\pgfqpoint{5.489414in}{2.767021in}}%
\pgfpathlineto{\pgfqpoint{5.522763in}{2.766981in}}%
\pgfpathlineto{\pgfqpoint{5.556093in}{2.767301in}}%
\pgfpathlineto{\pgfqpoint{5.544288in}{2.783043in}}%
\pgfpathlineto{\pgfqpoint{5.532505in}{2.798808in}}%
\pgfpathlineto{\pgfqpoint{5.499230in}{2.798644in}}%
\pgfpathlineto{\pgfqpoint{5.465937in}{2.798852in}}%
\pgfpathclose%
\pgfusepath{fill}%
\end{pgfscope}%
\begin{pgfscope}%
\pgfpathrectangle{\pgfqpoint{1.020000in}{0.880000in}}{\pgfqpoint{6.160000in}{6.160000in}}%
\pgfusepath{clip}%
\pgfsetbuttcap%
\pgfsetroundjoin%
\definecolor{currentfill}{rgb}{0.763363,0.835092,0.955658}%
\pgfsetfillcolor{currentfill}%
\pgfsetlinewidth{0.000000pt}%
\definecolor{currentstroke}{rgb}{0.000000,0.000000,0.000000}%
\pgfsetstrokecolor{currentstroke}%
\pgfsetdash{}{0pt}%
\pgfpathmoveto{\pgfqpoint{3.391460in}{3.612984in}}%
\pgfpathlineto{\pgfqpoint{3.401232in}{3.602984in}}%
\pgfpathlineto{\pgfqpoint{3.411003in}{3.595789in}}%
\pgfpathlineto{\pgfqpoint{3.444972in}{3.607168in}}%
\pgfpathlineto{\pgfqpoint{3.478935in}{3.617052in}}%
\pgfpathlineto{\pgfqpoint{3.469118in}{3.621727in}}%
\pgfpathlineto{\pgfqpoint{3.459302in}{3.629225in}}%
\pgfpathlineto{\pgfqpoint{3.425385in}{3.621823in}}%
\pgfpathlineto{\pgfqpoint{3.391460in}{3.612984in}}%
\pgfpathclose%
\pgfusepath{fill}%
\end{pgfscope}%
\begin{pgfscope}%
\pgfpathrectangle{\pgfqpoint{1.020000in}{0.880000in}}{\pgfqpoint{6.160000in}{6.160000in}}%
\pgfusepath{clip}%
\pgfsetbuttcap%
\pgfsetroundjoin%
\definecolor{currentfill}{rgb}{0.510824,0.649397,0.985079}%
\pgfsetfillcolor{currentfill}%
\pgfsetlinewidth{0.000000pt}%
\definecolor{currentstroke}{rgb}{0.000000,0.000000,0.000000}%
\pgfsetstrokecolor{currentstroke}%
\pgfsetdash{}{0pt}%
\pgfpathmoveto{\pgfqpoint{4.618569in}{3.191650in}}%
\pgfpathlineto{\pgfqpoint{4.629540in}{3.181536in}}%
\pgfpathlineto{\pgfqpoint{4.640528in}{3.170533in}}%
\pgfpathlineto{\pgfqpoint{4.674058in}{3.144783in}}%
\pgfpathlineto{\pgfqpoint{4.707557in}{3.120591in}}%
\pgfpathlineto{\pgfqpoint{4.696521in}{3.132636in}}%
\pgfpathlineto{\pgfqpoint{4.685504in}{3.143957in}}%
\pgfpathlineto{\pgfqpoint{4.652052in}{3.167061in}}%
\pgfpathlineto{\pgfqpoint{4.618569in}{3.191650in}}%
\pgfpathclose%
\pgfusepath{fill}%
\end{pgfscope}%
\begin{pgfscope}%
\pgfpathrectangle{\pgfqpoint{1.020000in}{0.880000in}}{\pgfqpoint{6.160000in}{6.160000in}}%
\pgfusepath{clip}%
\pgfsetbuttcap%
\pgfsetroundjoin%
\definecolor{currentfill}{rgb}{0.677823,0.786546,0.991005}%
\pgfsetfillcolor{currentfill}%
\pgfsetlinewidth{0.000000pt}%
\definecolor{currentstroke}{rgb}{0.000000,0.000000,0.000000}%
\pgfsetstrokecolor{currentstroke}%
\pgfsetdash{}{0pt}%
\pgfpathmoveto{\pgfqpoint{2.518090in}{3.477317in}}%
\pgfpathlineto{\pgfqpoint{2.527365in}{3.444559in}}%
\pgfpathlineto{\pgfqpoint{2.536635in}{3.412901in}}%
\pgfpathlineto{\pgfqpoint{2.570841in}{3.421567in}}%
\pgfpathlineto{\pgfqpoint{2.605018in}{3.430977in}}%
\pgfpathlineto{\pgfqpoint{2.595686in}{3.463490in}}%
\pgfpathlineto{\pgfqpoint{2.586347in}{3.497327in}}%
\pgfpathlineto{\pgfqpoint{2.552233in}{3.486949in}}%
\pgfpathlineto{\pgfqpoint{2.518090in}{3.477317in}}%
\pgfpathclose%
\pgfusepath{fill}%
\end{pgfscope}%
\begin{pgfscope}%
\pgfpathrectangle{\pgfqpoint{1.020000in}{0.880000in}}{\pgfqpoint{6.160000in}{6.160000in}}%
\pgfusepath{clip}%
\pgfsetbuttcap%
\pgfsetroundjoin%
\definecolor{currentfill}{rgb}{0.328604,0.439712,0.869587}%
\pgfsetfillcolor{currentfill}%
\pgfsetlinewidth{0.000000pt}%
\definecolor{currentstroke}{rgb}{0.000000,0.000000,0.000000}%
\pgfsetstrokecolor{currentstroke}%
\pgfsetdash{}{0pt}%
\pgfpathmoveto{\pgfqpoint{5.242764in}{2.844185in}}%
\pgfpathlineto{\pgfqpoint{5.254284in}{2.827811in}}%
\pgfpathlineto{\pgfqpoint{5.265824in}{2.811341in}}%
\pgfpathlineto{\pgfqpoint{5.299218in}{2.807573in}}%
\pgfpathlineto{\pgfqpoint{5.332595in}{2.804581in}}%
\pgfpathlineto{\pgfqpoint{5.321000in}{2.820842in}}%
\pgfpathlineto{\pgfqpoint{5.309426in}{2.837050in}}%
\pgfpathlineto{\pgfqpoint{5.276103in}{2.840223in}}%
\pgfpathlineto{\pgfqpoint{5.242764in}{2.844185in}}%
\pgfpathclose%
\pgfusepath{fill}%
\end{pgfscope}%
\begin{pgfscope}%
\pgfpathrectangle{\pgfqpoint{1.020000in}{0.880000in}}{\pgfqpoint{6.160000in}{6.160000in}}%
\pgfusepath{clip}%
\pgfsetbuttcap%
\pgfsetroundjoin%
\definecolor{currentfill}{rgb}{0.713852,0.808857,0.979386}%
\pgfsetfillcolor{currentfill}%
\pgfsetlinewidth{0.000000pt}%
\definecolor{currentstroke}{rgb}{0.000000,0.000000,0.000000}%
\pgfsetstrokecolor{currentstroke}%
\pgfsetdash{}{0pt}%
\pgfpathmoveto{\pgfqpoint{3.032632in}{3.523515in}}%
\pgfpathlineto{\pgfqpoint{3.042230in}{3.497279in}}%
\pgfpathlineto{\pgfqpoint{3.051816in}{3.473577in}}%
\pgfpathlineto{\pgfqpoint{3.085822in}{3.489572in}}%
\pgfpathlineto{\pgfqpoint{3.119816in}{3.505491in}}%
\pgfpathlineto{\pgfqpoint{3.110192in}{3.527542in}}%
\pgfpathlineto{\pgfqpoint{3.100556in}{3.552344in}}%
\pgfpathlineto{\pgfqpoint{3.066600in}{3.538047in}}%
\pgfpathlineto{\pgfqpoint{3.032632in}{3.523515in}}%
\pgfpathclose%
\pgfusepath{fill}%
\end{pgfscope}%
\begin{pgfscope}%
\pgfpathrectangle{\pgfqpoint{1.020000in}{0.880000in}}{\pgfqpoint{6.160000in}{6.160000in}}%
\pgfusepath{clip}%
\pgfsetbuttcap%
\pgfsetroundjoin%
\definecolor{currentfill}{rgb}{0.683056,0.790043,0.989768}%
\pgfsetfillcolor{currentfill}%
\pgfsetlinewidth{0.000000pt}%
\definecolor{currentstroke}{rgb}{0.000000,0.000000,0.000000}%
\pgfsetstrokecolor{currentstroke}%
\pgfsetdash{}{0pt}%
\pgfpathmoveto{\pgfqpoint{2.741465in}{3.475485in}}%
\pgfpathlineto{\pgfqpoint{2.750904in}{3.443230in}}%
\pgfpathlineto{\pgfqpoint{2.760333in}{3.412718in}}%
\pgfpathlineto{\pgfqpoint{2.794438in}{3.425126in}}%
\pgfpathlineto{\pgfqpoint{2.828521in}{3.438108in}}%
\pgfpathlineto{\pgfqpoint{2.819042in}{3.468639in}}%
\pgfpathlineto{\pgfqpoint{2.809551in}{3.501174in}}%
\pgfpathlineto{\pgfqpoint{2.775518in}{3.488097in}}%
\pgfpathlineto{\pgfqpoint{2.741465in}{3.475485in}}%
\pgfpathclose%
\pgfusepath{fill}%
\end{pgfscope}%
\begin{pgfscope}%
\pgfpathrectangle{\pgfqpoint{1.020000in}{0.880000in}}{\pgfqpoint{6.160000in}{6.160000in}}%
\pgfusepath{clip}%
\pgfsetbuttcap%
\pgfsetroundjoin%
\definecolor{currentfill}{rgb}{0.280550,0.373423,0.818011}%
\pgfsetfillcolor{currentfill}%
\pgfsetlinewidth{0.000000pt}%
\definecolor{currentstroke}{rgb}{0.000000,0.000000,0.000000}%
\pgfsetstrokecolor{currentstroke}%
\pgfsetdash{}{0pt}%
\pgfpathmoveto{\pgfqpoint{6.069747in}{2.731555in}}%
\pgfpathlineto{\pgfqpoint{6.082060in}{2.716836in}}%
\pgfpathlineto{\pgfqpoint{6.094399in}{2.702168in}}%
\pgfpathlineto{\pgfqpoint{6.127606in}{2.704701in}}%
\pgfpathlineto{\pgfqpoint{6.160793in}{2.707257in}}%
\pgfpathlineto{\pgfqpoint{6.148399in}{2.721845in}}%
\pgfpathlineto{\pgfqpoint{6.136030in}{2.736482in}}%
\pgfpathlineto{\pgfqpoint{6.102898in}{2.734006in}}%
\pgfpathlineto{\pgfqpoint{6.069747in}{2.731555in}}%
\pgfpathclose%
\pgfusepath{fill}%
\end{pgfscope}%
\begin{pgfscope}%
\pgfpathrectangle{\pgfqpoint{1.020000in}{0.880000in}}{\pgfqpoint{6.160000in}{6.160000in}}%
\pgfusepath{clip}%
\pgfsetbuttcap%
\pgfsetroundjoin%
\definecolor{currentfill}{rgb}{0.285273,0.380129,0.823469}%
\pgfsetfillcolor{currentfill}%
\pgfsetlinewidth{0.000000pt}%
\definecolor{currentstroke}{rgb}{0.000000,0.000000,0.000000}%
\pgfsetstrokecolor{currentstroke}%
\pgfsetdash{}{0pt}%
\pgfpathmoveto{\pgfqpoint{5.846195in}{2.747813in}}%
\pgfpathlineto{\pgfqpoint{5.858293in}{2.732713in}}%
\pgfpathlineto{\pgfqpoint{5.870415in}{2.717664in}}%
\pgfpathlineto{\pgfqpoint{5.903687in}{2.719844in}}%
\pgfpathlineto{\pgfqpoint{5.936939in}{2.722091in}}%
\pgfpathlineto{\pgfqpoint{5.924761in}{2.737039in}}%
\pgfpathlineto{\pgfqpoint{5.912607in}{2.752039in}}%
\pgfpathlineto{\pgfqpoint{5.879411in}{2.749890in}}%
\pgfpathlineto{\pgfqpoint{5.846195in}{2.747813in}}%
\pgfpathclose%
\pgfusepath{fill}%
\end{pgfscope}%
\begin{pgfscope}%
\pgfpathrectangle{\pgfqpoint{1.020000in}{0.880000in}}{\pgfqpoint{6.160000in}{6.160000in}}%
\pgfusepath{clip}%
\pgfsetbuttcap%
\pgfsetroundjoin%
\definecolor{currentfill}{rgb}{0.581486,0.713451,0.998314}%
\pgfsetfillcolor{currentfill}%
\pgfsetlinewidth{0.000000pt}%
\definecolor{currentstroke}{rgb}{0.000000,0.000000,0.000000}%
\pgfsetstrokecolor{currentstroke}%
\pgfsetdash{}{0pt}%
\pgfpathmoveto{\pgfqpoint{4.462608in}{3.314373in}}%
\pgfpathlineto{\pgfqpoint{4.473443in}{3.309214in}}%
\pgfpathlineto{\pgfqpoint{4.484297in}{3.303064in}}%
\pgfpathlineto{\pgfqpoint{4.517920in}{3.273496in}}%
\pgfpathlineto{\pgfqpoint{4.551505in}{3.244976in}}%
\pgfpathlineto{\pgfqpoint{4.540601in}{3.252776in}}%
\pgfpathlineto{\pgfqpoint{4.529716in}{3.259739in}}%
\pgfpathlineto{\pgfqpoint{4.496181in}{3.286561in}}%
\pgfpathlineto{\pgfqpoint{4.462608in}{3.314373in}}%
\pgfpathclose%
\pgfusepath{fill}%
\end{pgfscope}%
\begin{pgfscope}%
\pgfpathrectangle{\pgfqpoint{1.020000in}{0.880000in}}{\pgfqpoint{6.160000in}{6.160000in}}%
\pgfusepath{clip}%
\pgfsetbuttcap%
\pgfsetroundjoin%
\definecolor{currentfill}{rgb}{0.772706,0.838978,0.949319}%
\pgfsetfillcolor{currentfill}%
\pgfsetlinewidth{0.000000pt}%
\definecolor{currentstroke}{rgb}{0.000000,0.000000,0.000000}%
\pgfsetstrokecolor{currentstroke}%
\pgfsetdash{}{0pt}%
\pgfpathmoveto{\pgfqpoint{3.838259in}{3.629213in}}%
\pgfpathlineto{\pgfqpoint{3.848412in}{3.637848in}}%
\pgfpathlineto{\pgfqpoint{3.858585in}{3.647591in}}%
\pgfpathlineto{\pgfqpoint{3.892536in}{3.637424in}}%
\pgfpathlineto{\pgfqpoint{3.926463in}{3.625137in}}%
\pgfpathlineto{\pgfqpoint{3.916220in}{3.615712in}}%
\pgfpathlineto{\pgfqpoint{3.905998in}{3.607284in}}%
\pgfpathlineto{\pgfqpoint{3.872141in}{3.619174in}}%
\pgfpathlineto{\pgfqpoint{3.838259in}{3.629213in}}%
\pgfpathclose%
\pgfusepath{fill}%
\end{pgfscope}%
\begin{pgfscope}%
\pgfpathrectangle{\pgfqpoint{1.020000in}{0.880000in}}{\pgfqpoint{6.160000in}{6.160000in}}%
\pgfusepath{clip}%
\pgfsetbuttcap%
\pgfsetroundjoin%
\definecolor{currentfill}{rgb}{0.777378,0.840921,0.946149}%
\pgfsetfillcolor{currentfill}%
\pgfsetlinewidth{0.000000pt}%
\definecolor{currentstroke}{rgb}{0.000000,0.000000,0.000000}%
\pgfsetstrokecolor{currentstroke}%
\pgfsetdash{}{0pt}%
\pgfpathmoveto{\pgfqpoint{3.614716in}{3.639492in}}%
\pgfpathlineto{\pgfqpoint{3.624645in}{3.641634in}}%
\pgfpathlineto{\pgfqpoint{3.634586in}{3.645889in}}%
\pgfpathlineto{\pgfqpoint{3.668567in}{3.648495in}}%
\pgfpathlineto{\pgfqpoint{3.702536in}{3.648936in}}%
\pgfpathlineto{\pgfqpoint{3.692533in}{3.643197in}}%
\pgfpathlineto{\pgfqpoint{3.682544in}{3.639483in}}%
\pgfpathlineto{\pgfqpoint{3.648636in}{3.640455in}}%
\pgfpathlineto{\pgfqpoint{3.614716in}{3.639492in}}%
\pgfpathclose%
\pgfusepath{fill}%
\end{pgfscope}%
\begin{pgfscope}%
\pgfpathrectangle{\pgfqpoint{1.020000in}{0.880000in}}{\pgfqpoint{6.160000in}{6.160000in}}%
\pgfusepath{clip}%
\pgfsetbuttcap%
\pgfsetroundjoin%
\definecolor{currentfill}{rgb}{0.294718,0.393542,0.834384}%
\pgfsetfillcolor{currentfill}%
\pgfsetlinewidth{0.000000pt}%
\definecolor{currentstroke}{rgb}{0.000000,0.000000,0.000000}%
\pgfsetstrokecolor{currentstroke}%
\pgfsetdash{}{0pt}%
\pgfpathmoveto{\pgfqpoint{5.622700in}{2.768841in}}%
\pgfpathlineto{\pgfqpoint{5.634584in}{2.753281in}}%
\pgfpathlineto{\pgfqpoint{5.646491in}{2.737760in}}%
\pgfpathlineto{\pgfqpoint{5.679822in}{2.739042in}}%
\pgfpathlineto{\pgfqpoint{5.713135in}{2.740515in}}%
\pgfpathlineto{\pgfqpoint{5.701172in}{2.755896in}}%
\pgfpathlineto{\pgfqpoint{5.689233in}{2.771322in}}%
\pgfpathlineto{\pgfqpoint{5.655976in}{2.769981in}}%
\pgfpathlineto{\pgfqpoint{5.622700in}{2.768841in}}%
\pgfpathclose%
\pgfusepath{fill}%
\end{pgfscope}%
\begin{pgfscope}%
\pgfpathrectangle{\pgfqpoint{1.020000in}{0.880000in}}{\pgfqpoint{6.160000in}{6.160000in}}%
\pgfusepath{clip}%
\pgfsetbuttcap%
\pgfsetroundjoin%
\definecolor{currentfill}{rgb}{0.753611,0.830233,0.960871}%
\pgfsetfillcolor{currentfill}%
\pgfsetlinewidth{0.000000pt}%
\definecolor{currentstroke}{rgb}{0.000000,0.000000,0.000000}%
\pgfsetstrokecolor{currentstroke}%
\pgfsetdash{}{0pt}%
\pgfpathmoveto{\pgfqpoint{3.323592in}{3.591438in}}%
\pgfpathlineto{\pgfqpoint{3.333322in}{3.578894in}}%
\pgfpathlineto{\pgfqpoint{3.343049in}{3.569095in}}%
\pgfpathlineto{\pgfqpoint{3.377028in}{3.583049in}}%
\pgfpathlineto{\pgfqpoint{3.411003in}{3.595789in}}%
\pgfpathlineto{\pgfqpoint{3.401232in}{3.602984in}}%
\pgfpathlineto{\pgfqpoint{3.391460in}{3.612984in}}%
\pgfpathlineto{\pgfqpoint{3.357529in}{3.602815in}}%
\pgfpathlineto{\pgfqpoint{3.323592in}{3.591438in}}%
\pgfpathclose%
\pgfusepath{fill}%
\end{pgfscope}%
\begin{pgfscope}%
\pgfpathrectangle{\pgfqpoint{1.020000in}{0.880000in}}{\pgfqpoint{6.160000in}{6.160000in}}%
\pgfusepath{clip}%
\pgfsetbuttcap%
\pgfsetroundjoin%
\definecolor{currentfill}{rgb}{0.313946,0.420052,0.854993}%
\pgfsetfillcolor{currentfill}%
\pgfsetlinewidth{0.000000pt}%
\definecolor{currentstroke}{rgb}{0.000000,0.000000,0.000000}%
\pgfsetstrokecolor{currentstroke}%
\pgfsetdash{}{0pt}%
\pgfpathmoveto{\pgfqpoint{5.399300in}{2.800611in}}%
\pgfpathlineto{\pgfqpoint{5.410972in}{2.784512in}}%
\pgfpathlineto{\pgfqpoint{5.422666in}{2.768401in}}%
\pgfpathlineto{\pgfqpoint{5.456049in}{2.767476in}}%
\pgfpathlineto{\pgfqpoint{5.489414in}{2.767021in}}%
\pgfpathlineto{\pgfqpoint{5.477665in}{2.782933in}}%
\pgfpathlineto{\pgfqpoint{5.465937in}{2.798852in}}%
\pgfpathlineto{\pgfqpoint{5.432627in}{2.799489in}}%
\pgfpathlineto{\pgfqpoint{5.399300in}{2.800611in}}%
\pgfpathclose%
\pgfusepath{fill}%
\end{pgfscope}%
\begin{pgfscope}%
\pgfpathrectangle{\pgfqpoint{1.020000in}{0.880000in}}{\pgfqpoint{6.160000in}{6.160000in}}%
\pgfusepath{clip}%
\pgfsetbuttcap%
\pgfsetroundjoin%
\definecolor{currentfill}{rgb}{0.414801,0.546874,0.939088}%
\pgfsetfillcolor{currentfill}%
\pgfsetlinewidth{0.000000pt}%
\definecolor{currentstroke}{rgb}{0.000000,0.000000,0.000000}%
\pgfsetstrokecolor{currentstroke}%
\pgfsetdash{}{0pt}%
\pgfpathmoveto{\pgfqpoint{4.863591in}{3.011146in}}%
\pgfpathlineto{\pgfqpoint{4.874775in}{2.995872in}}%
\pgfpathlineto{\pgfqpoint{4.885976in}{2.980027in}}%
\pgfpathlineto{\pgfqpoint{4.919443in}{2.963273in}}%
\pgfpathlineto{\pgfqpoint{4.952889in}{2.948145in}}%
\pgfpathlineto{\pgfqpoint{4.941638in}{2.964103in}}%
\pgfpathlineto{\pgfqpoint{4.930404in}{2.979625in}}%
\pgfpathlineto{\pgfqpoint{4.897009in}{2.994586in}}%
\pgfpathlineto{\pgfqpoint{4.863591in}{3.011146in}}%
\pgfpathclose%
\pgfusepath{fill}%
\end{pgfscope}%
\begin{pgfscope}%
\pgfpathrectangle{\pgfqpoint{1.020000in}{0.880000in}}{\pgfqpoint{6.160000in}{6.160000in}}%
\pgfusepath{clip}%
\pgfsetbuttcap%
\pgfsetroundjoin%
\definecolor{currentfill}{rgb}{0.698454,0.799450,0.984577}%
\pgfsetfillcolor{currentfill}%
\pgfsetlinewidth{0.000000pt}%
\definecolor{currentstroke}{rgb}{0.000000,0.000000,0.000000}%
\pgfsetstrokecolor{currentstroke}%
\pgfsetdash{}{0pt}%
\pgfpathmoveto{\pgfqpoint{2.964659in}{3.494254in}}%
\pgfpathlineto{\pgfqpoint{2.974217in}{3.466913in}}%
\pgfpathlineto{\pgfqpoint{2.983764in}{3.441875in}}%
\pgfpathlineto{\pgfqpoint{3.017797in}{3.457637in}}%
\pgfpathlineto{\pgfqpoint{3.051816in}{3.473577in}}%
\pgfpathlineto{\pgfqpoint{3.042230in}{3.497279in}}%
\pgfpathlineto{\pgfqpoint{3.032632in}{3.523515in}}%
\pgfpathlineto{\pgfqpoint{2.998652in}{3.508877in}}%
\pgfpathlineto{\pgfqpoint{2.964659in}{3.494254in}}%
\pgfpathclose%
\pgfusepath{fill}%
\end{pgfscope}%
\begin{pgfscope}%
\pgfpathrectangle{\pgfqpoint{1.020000in}{0.880000in}}{\pgfqpoint{6.160000in}{6.160000in}}%
\pgfusepath{clip}%
\pgfsetbuttcap%
\pgfsetroundjoin%
\definecolor{currentfill}{rgb}{0.368507,0.491141,0.905243}%
\pgfsetfillcolor{currentfill}%
\pgfsetlinewidth{0.000000pt}%
\definecolor{currentstroke}{rgb}{0.000000,0.000000,0.000000}%
\pgfsetstrokecolor{currentstroke}%
\pgfsetdash{}{0pt}%
\pgfpathmoveto{\pgfqpoint{5.019720in}{2.922508in}}%
\pgfpathlineto{\pgfqpoint{5.031041in}{2.906148in}}%
\pgfpathlineto{\pgfqpoint{5.042380in}{2.889458in}}%
\pgfpathlineto{\pgfqpoint{5.075820in}{2.878899in}}%
\pgfpathlineto{\pgfqpoint{5.109243in}{2.869666in}}%
\pgfpathlineto{\pgfqpoint{5.097851in}{2.886208in}}%
\pgfpathlineto{\pgfqpoint{5.086478in}{2.902513in}}%
\pgfpathlineto{\pgfqpoint{5.053108in}{2.911848in}}%
\pgfpathlineto{\pgfqpoint{5.019720in}{2.922508in}}%
\pgfpathclose%
\pgfusepath{fill}%
\end{pgfscope}%
\begin{pgfscope}%
\pgfpathrectangle{\pgfqpoint{1.020000in}{0.880000in}}{\pgfqpoint{6.160000in}{6.160000in}}%
\pgfusepath{clip}%
\pgfsetbuttcap%
\pgfsetroundjoin%
\definecolor{currentfill}{rgb}{0.672538,0.782861,0.991982}%
\pgfsetfillcolor{currentfill}%
\pgfsetlinewidth{0.000000pt}%
\definecolor{currentstroke}{rgb}{0.000000,0.000000,0.000000}%
\pgfsetstrokecolor{currentstroke}%
\pgfsetdash{}{0pt}%
\pgfpathmoveto{\pgfqpoint{2.673291in}{3.451935in}}%
\pgfpathlineto{\pgfqpoint{2.682675in}{3.420141in}}%
\pgfpathlineto{\pgfqpoint{2.692052in}{3.389841in}}%
\pgfpathlineto{\pgfqpoint{2.726205in}{3.400940in}}%
\pgfpathlineto{\pgfqpoint{2.760333in}{3.412718in}}%
\pgfpathlineto{\pgfqpoint{2.750904in}{3.443230in}}%
\pgfpathlineto{\pgfqpoint{2.741465in}{3.475485in}}%
\pgfpathlineto{\pgfqpoint{2.707390in}{3.463410in}}%
\pgfpathlineto{\pgfqpoint{2.673291in}{3.451935in}}%
\pgfpathclose%
\pgfusepath{fill}%
\end{pgfscope}%
\begin{pgfscope}%
\pgfpathrectangle{\pgfqpoint{1.020000in}{0.880000in}}{\pgfqpoint{6.160000in}{6.160000in}}%
\pgfusepath{clip}%
\pgfsetbuttcap%
\pgfsetroundjoin%
\definecolor{currentfill}{rgb}{0.656683,0.771806,0.994914}%
\pgfsetfillcolor{currentfill}%
\pgfsetlinewidth{0.000000pt}%
\definecolor{currentstroke}{rgb}{0.000000,0.000000,0.000000}%
\pgfsetstrokecolor{currentstroke}%
\pgfsetdash{}{0pt}%
\pgfpathmoveto{\pgfqpoint{4.306533in}{3.431490in}}%
\pgfpathlineto{\pgfqpoint{4.317219in}{3.432134in}}%
\pgfpathlineto{\pgfqpoint{4.327926in}{3.431903in}}%
\pgfpathlineto{\pgfqpoint{4.361657in}{3.401997in}}%
\pgfpathlineto{\pgfqpoint{4.395347in}{3.372290in}}%
\pgfpathlineto{\pgfqpoint{4.384585in}{3.374586in}}%
\pgfpathlineto{\pgfqpoint{4.373843in}{3.376111in}}%
\pgfpathlineto{\pgfqpoint{4.340208in}{3.403700in}}%
\pgfpathlineto{\pgfqpoint{4.306533in}{3.431490in}}%
\pgfpathclose%
\pgfusepath{fill}%
\end{pgfscope}%
\begin{pgfscope}%
\pgfpathrectangle{\pgfqpoint{1.020000in}{0.880000in}}{\pgfqpoint{6.160000in}{6.160000in}}%
\pgfusepath{clip}%
\pgfsetbuttcap%
\pgfsetroundjoin%
\definecolor{currentfill}{rgb}{0.753611,0.830233,0.960871}%
\pgfsetfillcolor{currentfill}%
\pgfsetlinewidth{0.000000pt}%
\definecolor{currentstroke}{rgb}{0.000000,0.000000,0.000000}%
\pgfsetstrokecolor{currentstroke}%
\pgfsetdash{}{0pt}%
\pgfpathmoveto{\pgfqpoint{3.994237in}{3.594794in}}%
\pgfpathlineto{\pgfqpoint{4.004572in}{3.603948in}}%
\pgfpathlineto{\pgfqpoint{4.014931in}{3.613323in}}%
\pgfpathlineto{\pgfqpoint{4.048843in}{3.594315in}}%
\pgfpathlineto{\pgfqpoint{4.082722in}{3.573665in}}%
\pgfpathlineto{\pgfqpoint{4.072294in}{3.565719in}}%
\pgfpathlineto{\pgfqpoint{4.061890in}{3.557939in}}%
\pgfpathlineto{\pgfqpoint{4.028079in}{3.577084in}}%
\pgfpathlineto{\pgfqpoint{3.994237in}{3.594794in}}%
\pgfpathclose%
\pgfusepath{fill}%
\end{pgfscope}%
\begin{pgfscope}%
\pgfpathrectangle{\pgfqpoint{1.020000in}{0.880000in}}{\pgfqpoint{6.160000in}{6.160000in}}%
\pgfusepath{clip}%
\pgfsetbuttcap%
\pgfsetroundjoin%
\definecolor{currentfill}{rgb}{0.473070,0.611077,0.970634}%
\pgfsetfillcolor{currentfill}%
\pgfsetlinewidth{0.000000pt}%
\definecolor{currentstroke}{rgb}{0.000000,0.000000,0.000000}%
\pgfsetstrokecolor{currentstroke}%
\pgfsetdash{}{0pt}%
\pgfpathmoveto{\pgfqpoint{4.707557in}{3.120591in}}%
\pgfpathlineto{\pgfqpoint{4.718610in}{3.107758in}}%
\pgfpathlineto{\pgfqpoint{4.729679in}{3.094081in}}%
\pgfpathlineto{\pgfqpoint{4.763197in}{3.070813in}}%
\pgfpathlineto{\pgfqpoint{4.796687in}{3.049235in}}%
\pgfpathlineto{\pgfqpoint{4.785569in}{3.063483in}}%
\pgfpathlineto{\pgfqpoint{4.774469in}{3.077058in}}%
\pgfpathlineto{\pgfqpoint{4.741026in}{3.098008in}}%
\pgfpathlineto{\pgfqpoint{4.707557in}{3.120591in}}%
\pgfpathclose%
\pgfusepath{fill}%
\end{pgfscope}%
\begin{pgfscope}%
\pgfpathrectangle{\pgfqpoint{1.020000in}{0.880000in}}{\pgfqpoint{6.160000in}{6.160000in}}%
\pgfusepath{clip}%
\pgfsetbuttcap%
\pgfsetroundjoin%
\definecolor{currentfill}{rgb}{0.338377,0.452819,0.879317}%
\pgfsetfillcolor{currentfill}%
\pgfsetlinewidth{0.000000pt}%
\definecolor{currentstroke}{rgb}{0.000000,0.000000,0.000000}%
\pgfsetstrokecolor{currentstroke}%
\pgfsetdash{}{0pt}%
\pgfpathmoveto{\pgfqpoint{5.176037in}{2.854819in}}%
\pgfpathlineto{\pgfqpoint{5.187502in}{2.838257in}}%
\pgfpathlineto{\pgfqpoint{5.198987in}{2.821544in}}%
\pgfpathlineto{\pgfqpoint{5.232414in}{2.815969in}}%
\pgfpathlineto{\pgfqpoint{5.265824in}{2.811341in}}%
\pgfpathlineto{\pgfqpoint{5.254284in}{2.827811in}}%
\pgfpathlineto{\pgfqpoint{5.242764in}{2.844185in}}%
\pgfpathlineto{\pgfqpoint{5.209409in}{2.849021in}}%
\pgfpathlineto{\pgfqpoint{5.176037in}{2.854819in}}%
\pgfpathclose%
\pgfusepath{fill}%
\end{pgfscope}%
\begin{pgfscope}%
\pgfpathrectangle{\pgfqpoint{1.020000in}{0.880000in}}{\pgfqpoint{6.160000in}{6.160000in}}%
\pgfusepath{clip}%
\pgfsetbuttcap%
\pgfsetroundjoin%
\definecolor{currentfill}{rgb}{0.743754,0.825125,0.965798}%
\pgfsetfillcolor{currentfill}%
\pgfsetlinewidth{0.000000pt}%
\definecolor{currentstroke}{rgb}{0.000000,0.000000,0.000000}%
\pgfsetstrokecolor{currentstroke}%
\pgfsetdash{}{0pt}%
\pgfpathmoveto{\pgfqpoint{3.255697in}{3.565584in}}%
\pgfpathlineto{\pgfqpoint{3.265388in}{3.550555in}}%
\pgfpathlineto{\pgfqpoint{3.275074in}{3.538174in}}%
\pgfpathlineto{\pgfqpoint{3.309064in}{3.554083in}}%
\pgfpathlineto{\pgfqpoint{3.343049in}{3.569095in}}%
\pgfpathlineto{\pgfqpoint{3.333322in}{3.578894in}}%
\pgfpathlineto{\pgfqpoint{3.323592in}{3.591438in}}%
\pgfpathlineto{\pgfqpoint{3.289647in}{3.578981in}}%
\pgfpathlineto{\pgfqpoint{3.255697in}{3.565584in}}%
\pgfpathclose%
\pgfusepath{fill}%
\end{pgfscope}%
\begin{pgfscope}%
\pgfpathrectangle{\pgfqpoint{1.020000in}{0.880000in}}{\pgfqpoint{6.160000in}{6.160000in}}%
\pgfusepath{clip}%
\pgfsetbuttcap%
\pgfsetroundjoin%
\definecolor{currentfill}{rgb}{0.713852,0.808857,0.979386}%
\pgfsetfillcolor{currentfill}%
\pgfsetlinewidth{0.000000pt}%
\definecolor{currentstroke}{rgb}{0.000000,0.000000,0.000000}%
\pgfsetstrokecolor{currentstroke}%
\pgfsetdash{}{0pt}%
\pgfpathmoveto{\pgfqpoint{4.150374in}{3.528370in}}%
\pgfpathlineto{\pgfqpoint{4.160892in}{3.534352in}}%
\pgfpathlineto{\pgfqpoint{4.171433in}{3.539865in}}%
\pgfpathlineto{\pgfqpoint{4.205268in}{3.513697in}}%
\pgfpathlineto{\pgfqpoint{4.239063in}{3.486738in}}%
\pgfpathlineto{\pgfqpoint{4.228459in}{3.483263in}}%
\pgfpathlineto{\pgfqpoint{4.217878in}{3.479344in}}%
\pgfpathlineto{\pgfqpoint{4.184145in}{3.504202in}}%
\pgfpathlineto{\pgfqpoint{4.150374in}{3.528370in}}%
\pgfpathclose%
\pgfusepath{fill}%
\end{pgfscope}%
\begin{pgfscope}%
\pgfpathrectangle{\pgfqpoint{1.020000in}{0.880000in}}{\pgfqpoint{6.160000in}{6.160000in}}%
\pgfusepath{clip}%
\pgfsetbuttcap%
\pgfsetroundjoin%
\definecolor{currentfill}{rgb}{0.280550,0.373423,0.818011}%
\pgfsetfillcolor{currentfill}%
\pgfsetlinewidth{0.000000pt}%
\definecolor{currentstroke}{rgb}{0.000000,0.000000,0.000000}%
\pgfsetstrokecolor{currentstroke}%
\pgfsetdash{}{0pt}%
\pgfpathmoveto{\pgfqpoint{6.003383in}{2.726743in}}%
\pgfpathlineto{\pgfqpoint{6.015641in}{2.711938in}}%
\pgfpathlineto{\pgfqpoint{6.027923in}{2.697184in}}%
\pgfpathlineto{\pgfqpoint{6.061171in}{2.699660in}}%
\pgfpathlineto{\pgfqpoint{6.094399in}{2.702168in}}%
\pgfpathlineto{\pgfqpoint{6.082060in}{2.716836in}}%
\pgfpathlineto{\pgfqpoint{6.069747in}{2.731555in}}%
\pgfpathlineto{\pgfqpoint{6.036575in}{2.729132in}}%
\pgfpathlineto{\pgfqpoint{6.003383in}{2.726743in}}%
\pgfpathclose%
\pgfusepath{fill}%
\end{pgfscope}%
\begin{pgfscope}%
\pgfpathrectangle{\pgfqpoint{1.020000in}{0.880000in}}{\pgfqpoint{6.160000in}{6.160000in}}%
\pgfusepath{clip}%
\pgfsetbuttcap%
\pgfsetroundjoin%
\definecolor{currentfill}{rgb}{0.289996,0.386836,0.828926}%
\pgfsetfillcolor{currentfill}%
\pgfsetlinewidth{0.000000pt}%
\definecolor{currentstroke}{rgb}{0.000000,0.000000,0.000000}%
\pgfsetstrokecolor{currentstroke}%
\pgfsetdash{}{0pt}%
\pgfpathmoveto{\pgfqpoint{5.779704in}{2.743927in}}%
\pgfpathlineto{\pgfqpoint{5.791746in}{2.728716in}}%
\pgfpathlineto{\pgfqpoint{5.803812in}{2.713554in}}%
\pgfpathlineto{\pgfqpoint{5.837123in}{2.715562in}}%
\pgfpathlineto{\pgfqpoint{5.870415in}{2.717664in}}%
\pgfpathlineto{\pgfqpoint{5.858293in}{2.732713in}}%
\pgfpathlineto{\pgfqpoint{5.846195in}{2.747813in}}%
\pgfpathlineto{\pgfqpoint{5.812959in}{2.745820in}}%
\pgfpathlineto{\pgfqpoint{5.779704in}{2.743927in}}%
\pgfpathclose%
\pgfusepath{fill}%
\end{pgfscope}%
\begin{pgfscope}%
\pgfpathrectangle{\pgfqpoint{1.020000in}{0.880000in}}{\pgfqpoint{6.160000in}{6.160000in}}%
\pgfusepath{clip}%
\pgfsetbuttcap%
\pgfsetroundjoin%
\definecolor{currentfill}{rgb}{0.782049,0.842864,0.942980}%
\pgfsetfillcolor{currentfill}%
\pgfsetlinewidth{0.000000pt}%
\definecolor{currentstroke}{rgb}{0.000000,0.000000,0.000000}%
\pgfsetstrokecolor{currentstroke}%
\pgfsetdash{}{0pt}%
\pgfpathmoveto{\pgfqpoint{3.546842in}{3.631871in}}%
\pgfpathlineto{\pgfqpoint{3.556716in}{3.632030in}}%
\pgfpathlineto{\pgfqpoint{3.566598in}{3.634362in}}%
\pgfpathlineto{\pgfqpoint{3.600596in}{3.641157in}}%
\pgfpathlineto{\pgfqpoint{3.634586in}{3.645889in}}%
\pgfpathlineto{\pgfqpoint{3.624645in}{3.641634in}}%
\pgfpathlineto{\pgfqpoint{3.614716in}{3.639492in}}%
\pgfpathlineto{\pgfqpoint{3.580784in}{3.636616in}}%
\pgfpathlineto{\pgfqpoint{3.546842in}{3.631871in}}%
\pgfpathclose%
\pgfusepath{fill}%
\end{pgfscope}%
\begin{pgfscope}%
\pgfpathrectangle{\pgfqpoint{1.020000in}{0.880000in}}{\pgfqpoint{6.160000in}{6.160000in}}%
\pgfusepath{clip}%
\pgfsetbuttcap%
\pgfsetroundjoin%
\definecolor{currentfill}{rgb}{0.683056,0.790043,0.989768}%
\pgfsetfillcolor{currentfill}%
\pgfsetlinewidth{0.000000pt}%
\definecolor{currentstroke}{rgb}{0.000000,0.000000,0.000000}%
\pgfsetstrokecolor{currentstroke}%
\pgfsetdash{}{0pt}%
\pgfpathmoveto{\pgfqpoint{2.896626in}{3.465510in}}%
\pgfpathlineto{\pgfqpoint{2.906141in}{3.437387in}}%
\pgfpathlineto{\pgfqpoint{2.915646in}{3.411327in}}%
\pgfpathlineto{\pgfqpoint{2.949714in}{3.426404in}}%
\pgfpathlineto{\pgfqpoint{2.983764in}{3.441875in}}%
\pgfpathlineto{\pgfqpoint{2.974217in}{3.466913in}}%
\pgfpathlineto{\pgfqpoint{2.964659in}{3.494254in}}%
\pgfpathlineto{\pgfqpoint{2.930651in}{3.479762in}}%
\pgfpathlineto{\pgfqpoint{2.896626in}{3.465510in}}%
\pgfpathclose%
\pgfusepath{fill}%
\end{pgfscope}%
\begin{pgfscope}%
\pgfpathrectangle{\pgfqpoint{1.020000in}{0.880000in}}{\pgfqpoint{6.160000in}{6.160000in}}%
\pgfusepath{clip}%
\pgfsetbuttcap%
\pgfsetroundjoin%
\definecolor{currentfill}{rgb}{0.299441,0.400248,0.839842}%
\pgfsetfillcolor{currentfill}%
\pgfsetlinewidth{0.000000pt}%
\definecolor{currentstroke}{rgb}{0.000000,0.000000,0.000000}%
\pgfsetstrokecolor{currentstroke}%
\pgfsetdash{}{0pt}%
\pgfpathmoveto{\pgfqpoint{5.556093in}{2.767301in}}%
\pgfpathlineto{\pgfqpoint{5.567921in}{2.751586in}}%
\pgfpathlineto{\pgfqpoint{5.579772in}{2.735900in}}%
\pgfpathlineto{\pgfqpoint{5.613140in}{2.736700in}}%
\pgfpathlineto{\pgfqpoint{5.646491in}{2.737760in}}%
\pgfpathlineto{\pgfqpoint{5.634584in}{2.753281in}}%
\pgfpathlineto{\pgfqpoint{5.622700in}{2.768841in}}%
\pgfpathlineto{\pgfqpoint{5.589406in}{2.767935in}}%
\pgfpathlineto{\pgfqpoint{5.556093in}{2.767301in}}%
\pgfpathclose%
\pgfusepath{fill}%
\end{pgfscope}%
\begin{pgfscope}%
\pgfpathrectangle{\pgfqpoint{1.020000in}{0.880000in}}{\pgfqpoint{6.160000in}{6.160000in}}%
\pgfusepath{clip}%
\pgfsetbuttcap%
\pgfsetroundjoin%
\definecolor{currentfill}{rgb}{0.271104,0.360011,0.807095}%
\pgfsetfillcolor{currentfill}%
\pgfsetlinewidth{0.000000pt}%
\definecolor{currentstroke}{rgb}{0.000000,0.000000,0.000000}%
\pgfsetstrokecolor{currentstroke}%
\pgfsetdash{}{0pt}%
\pgfpathmoveto{\pgfqpoint{6.227106in}{2.712422in}}%
\pgfpathlineto{\pgfqpoint{6.239580in}{2.697960in}}%
\pgfpathlineto{\pgfqpoint{6.252080in}{2.683545in}}%
\pgfpathlineto{\pgfqpoint{6.285262in}{2.686224in}}%
\pgfpathlineto{\pgfqpoint{6.272734in}{2.700602in}}%
\pgfpathlineto{\pgfqpoint{6.260232in}{2.715028in}}%
\pgfpathlineto{\pgfqpoint{6.227106in}{2.712422in}}%
\pgfpathclose%
\pgfusepath{fill}%
\end{pgfscope}%
\begin{pgfscope}%
\pgfpathrectangle{\pgfqpoint{1.020000in}{0.880000in}}{\pgfqpoint{6.160000in}{6.160000in}}%
\pgfusepath{clip}%
\pgfsetbuttcap%
\pgfsetroundjoin%
\definecolor{currentfill}{rgb}{0.661968,0.775491,0.993937}%
\pgfsetfillcolor{currentfill}%
\pgfsetlinewidth{0.000000pt}%
\definecolor{currentstroke}{rgb}{0.000000,0.000000,0.000000}%
\pgfsetstrokecolor{currentstroke}%
\pgfsetdash{}{0pt}%
\pgfpathmoveto{\pgfqpoint{2.605018in}{3.430977in}}%
\pgfpathlineto{\pgfqpoint{2.614344in}{3.399763in}}%
\pgfpathlineto{\pgfqpoint{2.623666in}{3.369810in}}%
\pgfpathlineto{\pgfqpoint{2.657873in}{3.379455in}}%
\pgfpathlineto{\pgfqpoint{2.692052in}{3.389841in}}%
\pgfpathlineto{\pgfqpoint{2.682675in}{3.420141in}}%
\pgfpathlineto{\pgfqpoint{2.673291in}{3.451935in}}%
\pgfpathlineto{\pgfqpoint{2.639167in}{3.441110in}}%
\pgfpathlineto{\pgfqpoint{2.605018in}{3.430977in}}%
\pgfpathclose%
\pgfusepath{fill}%
\end{pgfscope}%
\begin{pgfscope}%
\pgfpathrectangle{\pgfqpoint{1.020000in}{0.880000in}}{\pgfqpoint{6.160000in}{6.160000in}}%
\pgfusepath{clip}%
\pgfsetbuttcap%
\pgfsetroundjoin%
\definecolor{currentfill}{rgb}{0.548876,0.685104,0.994379}%
\pgfsetfillcolor{currentfill}%
\pgfsetlinewidth{0.000000pt}%
\definecolor{currentstroke}{rgb}{0.000000,0.000000,0.000000}%
\pgfsetstrokecolor{currentstroke}%
\pgfsetdash{}{0pt}%
\pgfpathmoveto{\pgfqpoint{4.551505in}{3.244976in}}%
\pgfpathlineto{\pgfqpoint{4.562427in}{3.236216in}}%
\pgfpathlineto{\pgfqpoint{4.573367in}{3.226383in}}%
\pgfpathlineto{\pgfqpoint{4.606965in}{3.197767in}}%
\pgfpathlineto{\pgfqpoint{4.640528in}{3.170533in}}%
\pgfpathlineto{\pgfqpoint{4.629540in}{3.181536in}}%
\pgfpathlineto{\pgfqpoint{4.618569in}{3.191650in}}%
\pgfpathlineto{\pgfqpoint{4.585054in}{3.217654in}}%
\pgfpathlineto{\pgfqpoint{4.551505in}{3.244976in}}%
\pgfpathclose%
\pgfusepath{fill}%
\end{pgfscope}%
\begin{pgfscope}%
\pgfpathrectangle{\pgfqpoint{1.020000in}{0.880000in}}{\pgfqpoint{6.160000in}{6.160000in}}%
\pgfusepath{clip}%
\pgfsetbuttcap%
\pgfsetroundjoin%
\definecolor{currentfill}{rgb}{0.786721,0.844807,0.939810}%
\pgfsetfillcolor{currentfill}%
\pgfsetlinewidth{0.000000pt}%
\definecolor{currentstroke}{rgb}{0.000000,0.000000,0.000000}%
\pgfsetstrokecolor{currentstroke}%
\pgfsetdash{}{0pt}%
\pgfpathmoveto{\pgfqpoint{3.770433in}{3.643304in}}%
\pgfpathlineto{\pgfqpoint{3.780517in}{3.651603in}}%
\pgfpathlineto{\pgfqpoint{3.790620in}{3.661111in}}%
\pgfpathlineto{\pgfqpoint{3.824612in}{3.655517in}}%
\pgfpathlineto{\pgfqpoint{3.858585in}{3.647591in}}%
\pgfpathlineto{\pgfqpoint{3.848412in}{3.637848in}}%
\pgfpathlineto{\pgfqpoint{3.838259in}{3.629213in}}%
\pgfpathlineto{\pgfqpoint{3.804356in}{3.637286in}}%
\pgfpathlineto{\pgfqpoint{3.770433in}{3.643304in}}%
\pgfpathclose%
\pgfusepath{fill}%
\end{pgfscope}%
\begin{pgfscope}%
\pgfpathrectangle{\pgfqpoint{1.020000in}{0.880000in}}{\pgfqpoint{6.160000in}{6.160000in}}%
\pgfusepath{clip}%
\pgfsetbuttcap%
\pgfsetroundjoin%
\definecolor{currentfill}{rgb}{0.318832,0.426605,0.859857}%
\pgfsetfillcolor{currentfill}%
\pgfsetlinewidth{0.000000pt}%
\definecolor{currentstroke}{rgb}{0.000000,0.000000,0.000000}%
\pgfsetstrokecolor{currentstroke}%
\pgfsetdash{}{0pt}%
\pgfpathmoveto{\pgfqpoint{5.332595in}{2.804581in}}%
\pgfpathlineto{\pgfqpoint{5.344211in}{2.788271in}}%
\pgfpathlineto{\pgfqpoint{5.355849in}{2.771920in}}%
\pgfpathlineto{\pgfqpoint{5.389266in}{2.769859in}}%
\pgfpathlineto{\pgfqpoint{5.422666in}{2.768401in}}%
\pgfpathlineto{\pgfqpoint{5.410972in}{2.784512in}}%
\pgfpathlineto{\pgfqpoint{5.399300in}{2.800611in}}%
\pgfpathlineto{\pgfqpoint{5.365956in}{2.802286in}}%
\pgfpathlineto{\pgfqpoint{5.332595in}{2.804581in}}%
\pgfpathclose%
\pgfusepath{fill}%
\end{pgfscope}%
\begin{pgfscope}%
\pgfpathrectangle{\pgfqpoint{1.020000in}{0.880000in}}{\pgfqpoint{6.160000in}{6.160000in}}%
\pgfusepath{clip}%
\pgfsetbuttcap%
\pgfsetroundjoin%
\definecolor{currentfill}{rgb}{0.728970,0.817464,0.973188}%
\pgfsetfillcolor{currentfill}%
\pgfsetlinewidth{0.000000pt}%
\definecolor{currentstroke}{rgb}{0.000000,0.000000,0.000000}%
\pgfsetstrokecolor{currentstroke}%
\pgfsetdash{}{0pt}%
\pgfpathmoveto{\pgfqpoint{3.187773in}{3.536544in}}%
\pgfpathlineto{\pgfqpoint{3.197426in}{3.519177in}}%
\pgfpathlineto{\pgfqpoint{3.207073in}{3.504328in}}%
\pgfpathlineto{\pgfqpoint{3.241077in}{3.521533in}}%
\pgfpathlineto{\pgfqpoint{3.275074in}{3.538174in}}%
\pgfpathlineto{\pgfqpoint{3.265388in}{3.550555in}}%
\pgfpathlineto{\pgfqpoint{3.255697in}{3.565584in}}%
\pgfpathlineto{\pgfqpoint{3.221739in}{3.551389in}}%
\pgfpathlineto{\pgfqpoint{3.187773in}{3.536544in}}%
\pgfpathclose%
\pgfusepath{fill}%
\end{pgfscope}%
\begin{pgfscope}%
\pgfpathrectangle{\pgfqpoint{1.020000in}{0.880000in}}{\pgfqpoint{6.160000in}{6.160000in}}%
\pgfusepath{clip}%
\pgfsetbuttcap%
\pgfsetroundjoin%
\definecolor{currentfill}{rgb}{0.672538,0.782861,0.991982}%
\pgfsetfillcolor{currentfill}%
\pgfsetlinewidth{0.000000pt}%
\definecolor{currentstroke}{rgb}{0.000000,0.000000,0.000000}%
\pgfsetstrokecolor{currentstroke}%
\pgfsetdash{}{0pt}%
\pgfpathmoveto{\pgfqpoint{2.828521in}{3.438108in}}%
\pgfpathlineto{\pgfqpoint{2.837991in}{3.409499in}}%
\pgfpathlineto{\pgfqpoint{2.847453in}{3.382711in}}%
\pgfpathlineto{\pgfqpoint{2.881560in}{3.396736in}}%
\pgfpathlineto{\pgfqpoint{2.915646in}{3.411327in}}%
\pgfpathlineto{\pgfqpoint{2.906141in}{3.437387in}}%
\pgfpathlineto{\pgfqpoint{2.896626in}{3.465510in}}%
\pgfpathlineto{\pgfqpoint{2.862583in}{3.451596in}}%
\pgfpathlineto{\pgfqpoint{2.828521in}{3.438108in}}%
\pgfpathclose%
\pgfusepath{fill}%
\end{pgfscope}%
\begin{pgfscope}%
\pgfpathrectangle{\pgfqpoint{1.020000in}{0.880000in}}{\pgfqpoint{6.160000in}{6.160000in}}%
\pgfusepath{clip}%
\pgfsetbuttcap%
\pgfsetroundjoin%
\definecolor{currentfill}{rgb}{0.388852,0.516298,0.921373}%
\pgfsetfillcolor{currentfill}%
\pgfsetlinewidth{0.000000pt}%
\definecolor{currentstroke}{rgb}{0.000000,0.000000,0.000000}%
\pgfsetstrokecolor{currentstroke}%
\pgfsetdash{}{0pt}%
\pgfpathmoveto{\pgfqpoint{4.952889in}{2.948145in}}%
\pgfpathlineto{\pgfqpoint{4.964158in}{2.931743in}}%
\pgfpathlineto{\pgfqpoint{4.975445in}{2.914896in}}%
\pgfpathlineto{\pgfqpoint{5.008922in}{2.901431in}}%
\pgfpathlineto{\pgfqpoint{5.042380in}{2.889458in}}%
\pgfpathlineto{\pgfqpoint{5.031041in}{2.906148in}}%
\pgfpathlineto{\pgfqpoint{5.019720in}{2.922508in}}%
\pgfpathlineto{\pgfqpoint{4.986314in}{2.934580in}}%
\pgfpathlineto{\pgfqpoint{4.952889in}{2.948145in}}%
\pgfpathclose%
\pgfusepath{fill}%
\end{pgfscope}%
\begin{pgfscope}%
\pgfpathrectangle{\pgfqpoint{1.020000in}{0.880000in}}{\pgfqpoint{6.160000in}{6.160000in}}%
\pgfusepath{clip}%
\pgfsetbuttcap%
\pgfsetroundjoin%
\definecolor{currentfill}{rgb}{0.777378,0.840921,0.946149}%
\pgfsetfillcolor{currentfill}%
\pgfsetlinewidth{0.000000pt}%
\definecolor{currentstroke}{rgb}{0.000000,0.000000,0.000000}%
\pgfsetstrokecolor{currentstroke}%
\pgfsetdash{}{0pt}%
\pgfpathmoveto{\pgfqpoint{3.478935in}{3.617052in}}%
\pgfpathlineto{\pgfqpoint{3.488756in}{3.614904in}}%
\pgfpathlineto{\pgfqpoint{3.498584in}{3.614956in}}%
\pgfpathlineto{\pgfqpoint{3.532593in}{3.625592in}}%
\pgfpathlineto{\pgfqpoint{3.566598in}{3.634362in}}%
\pgfpathlineto{\pgfqpoint{3.556716in}{3.632030in}}%
\pgfpathlineto{\pgfqpoint{3.546842in}{3.631871in}}%
\pgfpathlineto{\pgfqpoint{3.512892in}{3.625321in}}%
\pgfpathlineto{\pgfqpoint{3.478935in}{3.617052in}}%
\pgfpathclose%
\pgfusepath{fill}%
\end{pgfscope}%
\begin{pgfscope}%
\pgfpathrectangle{\pgfqpoint{1.020000in}{0.880000in}}{\pgfqpoint{6.160000in}{6.160000in}}%
\pgfusepath{clip}%
\pgfsetbuttcap%
\pgfsetroundjoin%
\definecolor{currentfill}{rgb}{0.624703,0.748318,0.998719}%
\pgfsetfillcolor{currentfill}%
\pgfsetlinewidth{0.000000pt}%
\definecolor{currentstroke}{rgb}{0.000000,0.000000,0.000000}%
\pgfsetstrokecolor{currentstroke}%
\pgfsetdash{}{0pt}%
\pgfpathmoveto{\pgfqpoint{4.395347in}{3.372290in}}%
\pgfpathlineto{\pgfqpoint{4.406130in}{3.369029in}}%
\pgfpathlineto{\pgfqpoint{4.416933in}{3.364617in}}%
\pgfpathlineto{\pgfqpoint{4.450635in}{3.333504in}}%
\pgfpathlineto{\pgfqpoint{4.484297in}{3.303064in}}%
\pgfpathlineto{\pgfqpoint{4.473443in}{3.309214in}}%
\pgfpathlineto{\pgfqpoint{4.462608in}{3.314373in}}%
\pgfpathlineto{\pgfqpoint{4.428997in}{3.343012in}}%
\pgfpathlineto{\pgfqpoint{4.395347in}{3.372290in}}%
\pgfpathclose%
\pgfusepath{fill}%
\end{pgfscope}%
\begin{pgfscope}%
\pgfpathrectangle{\pgfqpoint{1.020000in}{0.880000in}}{\pgfqpoint{6.160000in}{6.160000in}}%
\pgfusepath{clip}%
\pgfsetbuttcap%
\pgfsetroundjoin%
\definecolor{currentfill}{rgb}{0.280550,0.373423,0.818011}%
\pgfsetfillcolor{currentfill}%
\pgfsetlinewidth{0.000000pt}%
\definecolor{currentstroke}{rgb}{0.000000,0.000000,0.000000}%
\pgfsetstrokecolor{currentstroke}%
\pgfsetdash{}{0pt}%
\pgfpathmoveto{\pgfqpoint{5.936939in}{2.722091in}}%
\pgfpathlineto{\pgfqpoint{5.949141in}{2.707193in}}%
\pgfpathlineto{\pgfqpoint{5.961367in}{2.692347in}}%
\pgfpathlineto{\pgfqpoint{5.994655in}{2.694744in}}%
\pgfpathlineto{\pgfqpoint{6.027923in}{2.697184in}}%
\pgfpathlineto{\pgfqpoint{6.015641in}{2.711938in}}%
\pgfpathlineto{\pgfqpoint{6.003383in}{2.726743in}}%
\pgfpathlineto{\pgfqpoint{5.970171in}{2.724393in}}%
\pgfpathlineto{\pgfqpoint{5.936939in}{2.722091in}}%
\pgfpathclose%
\pgfusepath{fill}%
\end{pgfscope}%
\begin{pgfscope}%
\pgfpathrectangle{\pgfqpoint{1.020000in}{0.880000in}}{\pgfqpoint{6.160000in}{6.160000in}}%
\pgfusepath{clip}%
\pgfsetbuttcap%
\pgfsetroundjoin%
\definecolor{currentfill}{rgb}{0.275827,0.366717,0.812553}%
\pgfsetfillcolor{currentfill}%
\pgfsetlinewidth{0.000000pt}%
\definecolor{currentstroke}{rgb}{0.000000,0.000000,0.000000}%
\pgfsetstrokecolor{currentstroke}%
\pgfsetdash{}{0pt}%
\pgfpathmoveto{\pgfqpoint{6.160793in}{2.707257in}}%
\pgfpathlineto{\pgfqpoint{6.173212in}{2.692718in}}%
\pgfpathlineto{\pgfqpoint{6.185655in}{2.678228in}}%
\pgfpathlineto{\pgfqpoint{6.218878in}{2.680879in}}%
\pgfpathlineto{\pgfqpoint{6.252080in}{2.683545in}}%
\pgfpathlineto{\pgfqpoint{6.239580in}{2.697960in}}%
\pgfpathlineto{\pgfqpoint{6.227106in}{2.712422in}}%
\pgfpathlineto{\pgfqpoint{6.193960in}{2.709831in}}%
\pgfpathlineto{\pgfqpoint{6.160793in}{2.707257in}}%
\pgfpathclose%
\pgfusepath{fill}%
\end{pgfscope}%
\begin{pgfscope}%
\pgfpathrectangle{\pgfqpoint{1.020000in}{0.880000in}}{\pgfqpoint{6.160000in}{6.160000in}}%
\pgfusepath{clip}%
\pgfsetbuttcap%
\pgfsetroundjoin%
\definecolor{currentfill}{rgb}{0.713852,0.808857,0.979386}%
\pgfsetfillcolor{currentfill}%
\pgfsetlinewidth{0.000000pt}%
\definecolor{currentstroke}{rgb}{0.000000,0.000000,0.000000}%
\pgfsetstrokecolor{currentstroke}%
\pgfsetdash{}{0pt}%
\pgfpathmoveto{\pgfqpoint{3.119816in}{3.505491in}}%
\pgfpathlineto{\pgfqpoint{3.129431in}{3.486005in}}%
\pgfpathlineto{\pgfqpoint{3.139041in}{3.468880in}}%
\pgfpathlineto{\pgfqpoint{3.173062in}{3.486724in}}%
\pgfpathlineto{\pgfqpoint{3.207073in}{3.504328in}}%
\pgfpathlineto{\pgfqpoint{3.197426in}{3.519177in}}%
\pgfpathlineto{\pgfqpoint{3.187773in}{3.536544in}}%
\pgfpathlineto{\pgfqpoint{3.153800in}{3.521196in}}%
\pgfpathlineto{\pgfqpoint{3.119816in}{3.505491in}}%
\pgfpathclose%
\pgfusepath{fill}%
\end{pgfscope}%
\begin{pgfscope}%
\pgfpathrectangle{\pgfqpoint{1.020000in}{0.880000in}}{\pgfqpoint{6.160000in}{6.160000in}}%
\pgfusepath{clip}%
\pgfsetbuttcap%
\pgfsetroundjoin%
\definecolor{currentfill}{rgb}{0.289996,0.386836,0.828926}%
\pgfsetfillcolor{currentfill}%
\pgfsetlinewidth{0.000000pt}%
\definecolor{currentstroke}{rgb}{0.000000,0.000000,0.000000}%
\pgfsetstrokecolor{currentstroke}%
\pgfsetdash{}{0pt}%
\pgfpathmoveto{\pgfqpoint{5.713135in}{2.740515in}}%
\pgfpathlineto{\pgfqpoint{5.725122in}{2.725179in}}%
\pgfpathlineto{\pgfqpoint{5.737131in}{2.709889in}}%
\pgfpathlineto{\pgfqpoint{5.770482in}{2.711657in}}%
\pgfpathlineto{\pgfqpoint{5.803812in}{2.713554in}}%
\pgfpathlineto{\pgfqpoint{5.791746in}{2.728716in}}%
\pgfpathlineto{\pgfqpoint{5.779704in}{2.743927in}}%
\pgfpathlineto{\pgfqpoint{5.746429in}{2.742151in}}%
\pgfpathlineto{\pgfqpoint{5.713135in}{2.740515in}}%
\pgfpathclose%
\pgfusepath{fill}%
\end{pgfscope}%
\begin{pgfscope}%
\pgfpathrectangle{\pgfqpoint{1.020000in}{0.880000in}}{\pgfqpoint{6.160000in}{6.160000in}}%
\pgfusepath{clip}%
\pgfsetbuttcap%
\pgfsetroundjoin%
\definecolor{currentfill}{rgb}{0.441123,0.576532,0.954545}%
\pgfsetfillcolor{currentfill}%
\pgfsetlinewidth{0.000000pt}%
\definecolor{currentstroke}{rgb}{0.000000,0.000000,0.000000}%
\pgfsetstrokecolor{currentstroke}%
\pgfsetdash{}{0pt}%
\pgfpathmoveto{\pgfqpoint{4.796687in}{3.049235in}}%
\pgfpathlineto{\pgfqpoint{4.807822in}{3.034279in}}%
\pgfpathlineto{\pgfqpoint{4.818973in}{3.018590in}}%
\pgfpathlineto{\pgfqpoint{4.852487in}{2.998455in}}%
\pgfpathlineto{\pgfqpoint{4.885976in}{2.980027in}}%
\pgfpathlineto{\pgfqpoint{4.874775in}{2.995872in}}%
\pgfpathlineto{\pgfqpoint{4.863591in}{3.011146in}}%
\pgfpathlineto{\pgfqpoint{4.830151in}{3.029351in}}%
\pgfpathlineto{\pgfqpoint{4.796687in}{3.049235in}}%
\pgfpathclose%
\pgfusepath{fill}%
\end{pgfscope}%
\begin{pgfscope}%
\pgfpathrectangle{\pgfqpoint{1.020000in}{0.880000in}}{\pgfqpoint{6.160000in}{6.160000in}}%
\pgfusepath{clip}%
\pgfsetbuttcap%
\pgfsetroundjoin%
\definecolor{currentfill}{rgb}{0.651398,0.768121,0.995891}%
\pgfsetfillcolor{currentfill}%
\pgfsetlinewidth{0.000000pt}%
\definecolor{currentstroke}{rgb}{0.000000,0.000000,0.000000}%
\pgfsetstrokecolor{currentstroke}%
\pgfsetdash{}{0pt}%
\pgfpathmoveto{\pgfqpoint{2.536635in}{3.412901in}}%
\pgfpathlineto{\pgfqpoint{2.545903in}{3.382325in}}%
\pgfpathlineto{\pgfqpoint{2.555169in}{3.352801in}}%
\pgfpathlineto{\pgfqpoint{2.589432in}{3.360922in}}%
\pgfpathlineto{\pgfqpoint{2.623666in}{3.369810in}}%
\pgfpathlineto{\pgfqpoint{2.614344in}{3.399763in}}%
\pgfpathlineto{\pgfqpoint{2.605018in}{3.430977in}}%
\pgfpathlineto{\pgfqpoint{2.570841in}{3.421567in}}%
\pgfpathlineto{\pgfqpoint{2.536635in}{3.412901in}}%
\pgfpathclose%
\pgfusepath{fill}%
\end{pgfscope}%
\begin{pgfscope}%
\pgfpathrectangle{\pgfqpoint{1.020000in}{0.880000in}}{\pgfqpoint{6.160000in}{6.160000in}}%
\pgfusepath{clip}%
\pgfsetbuttcap%
\pgfsetroundjoin%
\definecolor{currentfill}{rgb}{0.348323,0.465711,0.888346}%
\pgfsetfillcolor{currentfill}%
\pgfsetlinewidth{0.000000pt}%
\definecolor{currentstroke}{rgb}{0.000000,0.000000,0.000000}%
\pgfsetstrokecolor{currentstroke}%
\pgfsetdash{}{0pt}%
\pgfpathmoveto{\pgfqpoint{5.109243in}{2.869666in}}%
\pgfpathlineto{\pgfqpoint{5.120654in}{2.852892in}}%
\pgfpathlineto{\pgfqpoint{5.132084in}{2.835893in}}%
\pgfpathlineto{\pgfqpoint{5.165544in}{2.828155in}}%
\pgfpathlineto{\pgfqpoint{5.198987in}{2.821544in}}%
\pgfpathlineto{\pgfqpoint{5.187502in}{2.838257in}}%
\pgfpathlineto{\pgfqpoint{5.176037in}{2.854819in}}%
\pgfpathlineto{\pgfqpoint{5.142648in}{2.861670in}}%
\pgfpathlineto{\pgfqpoint{5.109243in}{2.869666in}}%
\pgfpathclose%
\pgfusepath{fill}%
\end{pgfscope}%
\begin{pgfscope}%
\pgfpathrectangle{\pgfqpoint{1.020000in}{0.880000in}}{\pgfqpoint{6.160000in}{6.160000in}}%
\pgfusepath{clip}%
\pgfsetbuttcap%
\pgfsetroundjoin%
\definecolor{currentfill}{rgb}{0.304174,0.406945,0.845263}%
\pgfsetfillcolor{currentfill}%
\pgfsetlinewidth{0.000000pt}%
\definecolor{currentstroke}{rgb}{0.000000,0.000000,0.000000}%
\pgfsetstrokecolor{currentstroke}%
\pgfsetdash{}{0pt}%
\pgfpathmoveto{\pgfqpoint{5.489414in}{2.767021in}}%
\pgfpathlineto{\pgfqpoint{5.501186in}{2.751122in}}%
\pgfpathlineto{\pgfqpoint{5.512980in}{2.735240in}}%
\pgfpathlineto{\pgfqpoint{5.546385in}{2.735398in}}%
\pgfpathlineto{\pgfqpoint{5.579772in}{2.735900in}}%
\pgfpathlineto{\pgfqpoint{5.567921in}{2.751586in}}%
\pgfpathlineto{\pgfqpoint{5.556093in}{2.767301in}}%
\pgfpathlineto{\pgfqpoint{5.522763in}{2.766981in}}%
\pgfpathlineto{\pgfqpoint{5.489414in}{2.767021in}}%
\pgfpathclose%
\pgfusepath{fill}%
\end{pgfscope}%
\begin{pgfscope}%
\pgfpathrectangle{\pgfqpoint{1.020000in}{0.880000in}}{\pgfqpoint{6.160000in}{6.160000in}}%
\pgfusepath{clip}%
\pgfsetbuttcap%
\pgfsetroundjoin%
\definecolor{currentfill}{rgb}{0.772706,0.838978,0.949319}%
\pgfsetfillcolor{currentfill}%
\pgfsetlinewidth{0.000000pt}%
\definecolor{currentstroke}{rgb}{0.000000,0.000000,0.000000}%
\pgfsetstrokecolor{currentstroke}%
\pgfsetdash{}{0pt}%
\pgfpathmoveto{\pgfqpoint{3.926463in}{3.625137in}}%
\pgfpathlineto{\pgfqpoint{3.936728in}{3.635212in}}%
\pgfpathlineto{\pgfqpoint{3.947016in}{3.645571in}}%
\pgfpathlineto{\pgfqpoint{3.980988in}{3.630475in}}%
\pgfpathlineto{\pgfqpoint{4.014931in}{3.613323in}}%
\pgfpathlineto{\pgfqpoint{4.004572in}{3.603948in}}%
\pgfpathlineto{\pgfqpoint{3.994237in}{3.594794in}}%
\pgfpathlineto{\pgfqpoint{3.960364in}{3.610873in}}%
\pgfpathlineto{\pgfqpoint{3.926463in}{3.625137in}}%
\pgfpathclose%
\pgfusepath{fill}%
\end{pgfscope}%
\begin{pgfscope}%
\pgfpathrectangle{\pgfqpoint{1.020000in}{0.880000in}}{\pgfqpoint{6.160000in}{6.160000in}}%
\pgfusepath{clip}%
\pgfsetbuttcap%
\pgfsetroundjoin%
\definecolor{currentfill}{rgb}{0.656683,0.771806,0.994914}%
\pgfsetfillcolor{currentfill}%
\pgfsetlinewidth{0.000000pt}%
\definecolor{currentstroke}{rgb}{0.000000,0.000000,0.000000}%
\pgfsetstrokecolor{currentstroke}%
\pgfsetdash{}{0pt}%
\pgfpathmoveto{\pgfqpoint{2.760333in}{3.412718in}}%
\pgfpathlineto{\pgfqpoint{2.769755in}{3.383877in}}%
\pgfpathlineto{\pgfqpoint{2.779171in}{3.356622in}}%
\pgfpathlineto{\pgfqpoint{2.813323in}{3.369321in}}%
\pgfpathlineto{\pgfqpoint{2.847453in}{3.382711in}}%
\pgfpathlineto{\pgfqpoint{2.837991in}{3.409499in}}%
\pgfpathlineto{\pgfqpoint{2.828521in}{3.438108in}}%
\pgfpathlineto{\pgfqpoint{2.794438in}{3.425126in}}%
\pgfpathlineto{\pgfqpoint{2.760333in}{3.412718in}}%
\pgfpathclose%
\pgfusepath{fill}%
\end{pgfscope}%
\begin{pgfscope}%
\pgfpathrectangle{\pgfqpoint{1.020000in}{0.880000in}}{\pgfqpoint{6.160000in}{6.160000in}}%
\pgfusepath{clip}%
\pgfsetbuttcap%
\pgfsetroundjoin%
\definecolor{currentfill}{rgb}{0.693321,0.796314,0.986308}%
\pgfsetfillcolor{currentfill}%
\pgfsetlinewidth{0.000000pt}%
\definecolor{currentstroke}{rgb}{0.000000,0.000000,0.000000}%
\pgfsetstrokecolor{currentstroke}%
\pgfsetdash{}{0pt}%
\pgfpathmoveto{\pgfqpoint{4.239063in}{3.486738in}}%
\pgfpathlineto{\pgfqpoint{4.249690in}{3.489502in}}%
\pgfpathlineto{\pgfqpoint{4.260339in}{3.491294in}}%
\pgfpathlineto{\pgfqpoint{4.294154in}{3.461757in}}%
\pgfpathlineto{\pgfqpoint{4.327926in}{3.431903in}}%
\pgfpathlineto{\pgfqpoint{4.317219in}{3.432134in}}%
\pgfpathlineto{\pgfqpoint{4.306533in}{3.431490in}}%
\pgfpathlineto{\pgfqpoint{4.272818in}{3.459250in}}%
\pgfpathlineto{\pgfqpoint{4.239063in}{3.486738in}}%
\pgfpathclose%
\pgfusepath{fill}%
\end{pgfscope}%
\begin{pgfscope}%
\pgfpathrectangle{\pgfqpoint{1.020000in}{0.880000in}}{\pgfqpoint{6.160000in}{6.160000in}}%
\pgfusepath{clip}%
\pgfsetbuttcap%
\pgfsetroundjoin%
\definecolor{currentfill}{rgb}{0.510824,0.649397,0.985079}%
\pgfsetfillcolor{currentfill}%
\pgfsetlinewidth{0.000000pt}%
\definecolor{currentstroke}{rgb}{0.000000,0.000000,0.000000}%
\pgfsetstrokecolor{currentstroke}%
\pgfsetdash{}{0pt}%
\pgfpathmoveto{\pgfqpoint{4.640528in}{3.170533in}}%
\pgfpathlineto{\pgfqpoint{4.651533in}{3.158561in}}%
\pgfpathlineto{\pgfqpoint{4.662555in}{3.145547in}}%
\pgfpathlineto{\pgfqpoint{4.696132in}{3.119009in}}%
\pgfpathlineto{\pgfqpoint{4.729679in}{3.094081in}}%
\pgfpathlineto{\pgfqpoint{4.718610in}{3.107758in}}%
\pgfpathlineto{\pgfqpoint{4.707557in}{3.120591in}}%
\pgfpathlineto{\pgfqpoint{4.674058in}{3.144783in}}%
\pgfpathlineto{\pgfqpoint{4.640528in}{3.170533in}}%
\pgfpathclose%
\pgfusepath{fill}%
\end{pgfscope}%
\begin{pgfscope}%
\pgfpathrectangle{\pgfqpoint{1.020000in}{0.880000in}}{\pgfqpoint{6.160000in}{6.160000in}}%
\pgfusepath{clip}%
\pgfsetbuttcap%
\pgfsetroundjoin%
\definecolor{currentfill}{rgb}{0.323718,0.433158,0.864722}%
\pgfsetfillcolor{currentfill}%
\pgfsetlinewidth{0.000000pt}%
\definecolor{currentstroke}{rgb}{0.000000,0.000000,0.000000}%
\pgfsetstrokecolor{currentstroke}%
\pgfsetdash{}{0pt}%
\pgfpathmoveto{\pgfqpoint{5.265824in}{2.811341in}}%
\pgfpathlineto{\pgfqpoint{5.277385in}{2.794781in}}%
\pgfpathlineto{\pgfqpoint{5.288966in}{2.778140in}}%
\pgfpathlineto{\pgfqpoint{5.322416in}{2.774654in}}%
\pgfpathlineto{\pgfqpoint{5.355849in}{2.771920in}}%
\pgfpathlineto{\pgfqpoint{5.344211in}{2.788271in}}%
\pgfpathlineto{\pgfqpoint{5.332595in}{2.804581in}}%
\pgfpathlineto{\pgfqpoint{5.299218in}{2.807573in}}%
\pgfpathlineto{\pgfqpoint{5.265824in}{2.811341in}}%
\pgfpathclose%
\pgfusepath{fill}%
\end{pgfscope}%
\begin{pgfscope}%
\pgfpathrectangle{\pgfqpoint{1.020000in}{0.880000in}}{\pgfqpoint{6.160000in}{6.160000in}}%
\pgfusepath{clip}%
\pgfsetbuttcap%
\pgfsetroundjoin%
\definecolor{currentfill}{rgb}{0.791392,0.846750,0.936641}%
\pgfsetfillcolor{currentfill}%
\pgfsetlinewidth{0.000000pt}%
\definecolor{currentstroke}{rgb}{0.000000,0.000000,0.000000}%
\pgfsetstrokecolor{currentstroke}%
\pgfsetdash{}{0pt}%
\pgfpathmoveto{\pgfqpoint{3.702536in}{3.648936in}}%
\pgfpathlineto{\pgfqpoint{3.712554in}{3.656351in}}%
\pgfpathlineto{\pgfqpoint{3.722589in}{3.665067in}}%
\pgfpathlineto{\pgfqpoint{3.756611in}{3.664306in}}%
\pgfpathlineto{\pgfqpoint{3.790620in}{3.661111in}}%
\pgfpathlineto{\pgfqpoint{3.780517in}{3.651603in}}%
\pgfpathlineto{\pgfqpoint{3.770433in}{3.643304in}}%
\pgfpathlineto{\pgfqpoint{3.736492in}{3.647201in}}%
\pgfpathlineto{\pgfqpoint{3.702536in}{3.648936in}}%
\pgfpathclose%
\pgfusepath{fill}%
\end{pgfscope}%
\begin{pgfscope}%
\pgfpathrectangle{\pgfqpoint{1.020000in}{0.880000in}}{\pgfqpoint{6.160000in}{6.160000in}}%
\pgfusepath{clip}%
\pgfsetbuttcap%
\pgfsetroundjoin%
\definecolor{currentfill}{rgb}{0.698454,0.799450,0.984577}%
\pgfsetfillcolor{currentfill}%
\pgfsetlinewidth{0.000000pt}%
\definecolor{currentstroke}{rgb}{0.000000,0.000000,0.000000}%
\pgfsetstrokecolor{currentstroke}%
\pgfsetdash{}{0pt}%
\pgfpathmoveto{\pgfqpoint{3.051816in}{3.473577in}}%
\pgfpathlineto{\pgfqpoint{3.061394in}{3.452243in}}%
\pgfpathlineto{\pgfqpoint{3.070965in}{3.433090in}}%
\pgfpathlineto{\pgfqpoint{3.105009in}{3.450953in}}%
\pgfpathlineto{\pgfqpoint{3.139041in}{3.468880in}}%
\pgfpathlineto{\pgfqpoint{3.129431in}{3.486005in}}%
\pgfpathlineto{\pgfqpoint{3.119816in}{3.505491in}}%
\pgfpathlineto{\pgfqpoint{3.085822in}{3.489572in}}%
\pgfpathlineto{\pgfqpoint{3.051816in}{3.473577in}}%
\pgfpathclose%
\pgfusepath{fill}%
\end{pgfscope}%
\begin{pgfscope}%
\pgfpathrectangle{\pgfqpoint{1.020000in}{0.880000in}}{\pgfqpoint{6.160000in}{6.160000in}}%
\pgfusepath{clip}%
\pgfsetbuttcap%
\pgfsetroundjoin%
\definecolor{currentfill}{rgb}{0.743754,0.825125,0.965798}%
\pgfsetfillcolor{currentfill}%
\pgfsetlinewidth{0.000000pt}%
\definecolor{currentstroke}{rgb}{0.000000,0.000000,0.000000}%
\pgfsetstrokecolor{currentstroke}%
\pgfsetdash{}{0pt}%
\pgfpathmoveto{\pgfqpoint{4.082722in}{3.573665in}}%
\pgfpathlineto{\pgfqpoint{4.093173in}{3.581455in}}%
\pgfpathlineto{\pgfqpoint{4.103649in}{3.588763in}}%
\pgfpathlineto{\pgfqpoint{4.137560in}{3.564976in}}%
\pgfpathlineto{\pgfqpoint{4.171433in}{3.539865in}}%
\pgfpathlineto{\pgfqpoint{4.160892in}{3.534352in}}%
\pgfpathlineto{\pgfqpoint{4.150374in}{3.528370in}}%
\pgfpathlineto{\pgfqpoint{4.116566in}{3.551605in}}%
\pgfpathlineto{\pgfqpoint{4.082722in}{3.573665in}}%
\pgfpathclose%
\pgfusepath{fill}%
\end{pgfscope}%
\begin{pgfscope}%
\pgfpathrectangle{\pgfqpoint{1.020000in}{0.880000in}}{\pgfqpoint{6.160000in}{6.160000in}}%
\pgfusepath{clip}%
\pgfsetbuttcap%
\pgfsetroundjoin%
\definecolor{currentfill}{rgb}{0.768034,0.837035,0.952488}%
\pgfsetfillcolor{currentfill}%
\pgfsetlinewidth{0.000000pt}%
\definecolor{currentstroke}{rgb}{0.000000,0.000000,0.000000}%
\pgfsetstrokecolor{currentstroke}%
\pgfsetdash{}{0pt}%
\pgfpathmoveto{\pgfqpoint{3.411003in}{3.595789in}}%
\pgfpathlineto{\pgfqpoint{3.420775in}{3.591107in}}%
\pgfpathlineto{\pgfqpoint{3.430552in}{3.588623in}}%
\pgfpathlineto{\pgfqpoint{3.464569in}{3.602584in}}%
\pgfpathlineto{\pgfqpoint{3.498584in}{3.614956in}}%
\pgfpathlineto{\pgfqpoint{3.488756in}{3.614904in}}%
\pgfpathlineto{\pgfqpoint{3.478935in}{3.617052in}}%
\pgfpathlineto{\pgfqpoint{3.444972in}{3.607168in}}%
\pgfpathlineto{\pgfqpoint{3.411003in}{3.595789in}}%
\pgfpathclose%
\pgfusepath{fill}%
\end{pgfscope}%
\begin{pgfscope}%
\pgfpathrectangle{\pgfqpoint{1.020000in}{0.880000in}}{\pgfqpoint{6.160000in}{6.160000in}}%
\pgfusepath{clip}%
\pgfsetbuttcap%
\pgfsetroundjoin%
\definecolor{currentfill}{rgb}{0.275827,0.366717,0.812553}%
\pgfsetfillcolor{currentfill}%
\pgfsetlinewidth{0.000000pt}%
\definecolor{currentstroke}{rgb}{0.000000,0.000000,0.000000}%
\pgfsetstrokecolor{currentstroke}%
\pgfsetdash{}{0pt}%
\pgfpathmoveto{\pgfqpoint{6.094399in}{2.702168in}}%
\pgfpathlineto{\pgfqpoint{6.106761in}{2.687549in}}%
\pgfpathlineto{\pgfqpoint{6.119149in}{2.672980in}}%
\pgfpathlineto{\pgfqpoint{6.152412in}{2.675594in}}%
\pgfpathlineto{\pgfqpoint{6.185655in}{2.678228in}}%
\pgfpathlineto{\pgfqpoint{6.173212in}{2.692718in}}%
\pgfpathlineto{\pgfqpoint{6.160793in}{2.707257in}}%
\pgfpathlineto{\pgfqpoint{6.127606in}{2.704701in}}%
\pgfpathlineto{\pgfqpoint{6.094399in}{2.702168in}}%
\pgfpathclose%
\pgfusepath{fill}%
\end{pgfscope}%
\begin{pgfscope}%
\pgfpathrectangle{\pgfqpoint{1.020000in}{0.880000in}}{\pgfqpoint{6.160000in}{6.160000in}}%
\pgfusepath{clip}%
\pgfsetbuttcap%
\pgfsetroundjoin%
\definecolor{currentfill}{rgb}{0.280550,0.373423,0.818011}%
\pgfsetfillcolor{currentfill}%
\pgfsetlinewidth{0.000000pt}%
\definecolor{currentstroke}{rgb}{0.000000,0.000000,0.000000}%
\pgfsetstrokecolor{currentstroke}%
\pgfsetdash{}{0pt}%
\pgfpathmoveto{\pgfqpoint{5.870415in}{2.717664in}}%
\pgfpathlineto{\pgfqpoint{5.882561in}{2.702665in}}%
\pgfpathlineto{\pgfqpoint{5.894731in}{2.687718in}}%
\pgfpathlineto{\pgfqpoint{5.928059in}{2.690002in}}%
\pgfpathlineto{\pgfqpoint{5.961367in}{2.692347in}}%
\pgfpathlineto{\pgfqpoint{5.949141in}{2.707193in}}%
\pgfpathlineto{\pgfqpoint{5.936939in}{2.722091in}}%
\pgfpathlineto{\pgfqpoint{5.903687in}{2.719844in}}%
\pgfpathlineto{\pgfqpoint{5.870415in}{2.717664in}}%
\pgfpathclose%
\pgfusepath{fill}%
\end{pgfscope}%
\begin{pgfscope}%
\pgfpathrectangle{\pgfqpoint{1.020000in}{0.880000in}}{\pgfqpoint{6.160000in}{6.160000in}}%
\pgfusepath{clip}%
\pgfsetbuttcap%
\pgfsetroundjoin%
\definecolor{currentfill}{rgb}{0.289996,0.386836,0.828926}%
\pgfsetfillcolor{currentfill}%
\pgfsetlinewidth{0.000000pt}%
\definecolor{currentstroke}{rgb}{0.000000,0.000000,0.000000}%
\pgfsetstrokecolor{currentstroke}%
\pgfsetdash{}{0pt}%
\pgfpathmoveto{\pgfqpoint{5.646491in}{2.737760in}}%
\pgfpathlineto{\pgfqpoint{5.658421in}{2.722278in}}%
\pgfpathlineto{\pgfqpoint{5.670374in}{2.706839in}}%
\pgfpathlineto{\pgfqpoint{5.703762in}{2.708275in}}%
\pgfpathlineto{\pgfqpoint{5.737131in}{2.709889in}}%
\pgfpathlineto{\pgfqpoint{5.725122in}{2.725179in}}%
\pgfpathlineto{\pgfqpoint{5.713135in}{2.740515in}}%
\pgfpathlineto{\pgfqpoint{5.679822in}{2.739042in}}%
\pgfpathlineto{\pgfqpoint{5.646491in}{2.737760in}}%
\pgfpathclose%
\pgfusepath{fill}%
\end{pgfscope}%
\begin{pgfscope}%
\pgfpathrectangle{\pgfqpoint{1.020000in}{0.880000in}}{\pgfqpoint{6.160000in}{6.160000in}}%
\pgfusepath{clip}%
\pgfsetbuttcap%
\pgfsetroundjoin%
\definecolor{currentfill}{rgb}{0.646113,0.764436,0.996868}%
\pgfsetfillcolor{currentfill}%
\pgfsetlinewidth{0.000000pt}%
\definecolor{currentstroke}{rgb}{0.000000,0.000000,0.000000}%
\pgfsetstrokecolor{currentstroke}%
\pgfsetdash{}{0pt}%
\pgfpathmoveto{\pgfqpoint{2.692052in}{3.389841in}}%
\pgfpathlineto{\pgfqpoint{2.701423in}{3.360974in}}%
\pgfpathlineto{\pgfqpoint{2.710790in}{3.333470in}}%
\pgfpathlineto{\pgfqpoint{2.744993in}{3.344661in}}%
\pgfpathlineto{\pgfqpoint{2.779171in}{3.356622in}}%
\pgfpathlineto{\pgfqpoint{2.769755in}{3.383877in}}%
\pgfpathlineto{\pgfqpoint{2.760333in}{3.412718in}}%
\pgfpathlineto{\pgfqpoint{2.726205in}{3.400940in}}%
\pgfpathlineto{\pgfqpoint{2.692052in}{3.389841in}}%
\pgfpathclose%
\pgfusepath{fill}%
\end{pgfscope}%
\begin{pgfscope}%
\pgfpathrectangle{\pgfqpoint{1.020000in}{0.880000in}}{\pgfqpoint{6.160000in}{6.160000in}}%
\pgfusepath{clip}%
\pgfsetbuttcap%
\pgfsetroundjoin%
\definecolor{currentfill}{rgb}{0.683056,0.790043,0.989768}%
\pgfsetfillcolor{currentfill}%
\pgfsetlinewidth{0.000000pt}%
\definecolor{currentstroke}{rgb}{0.000000,0.000000,0.000000}%
\pgfsetstrokecolor{currentstroke}%
\pgfsetdash{}{0pt}%
\pgfpathmoveto{\pgfqpoint{2.983764in}{3.441875in}}%
\pgfpathlineto{\pgfqpoint{2.993302in}{3.418991in}}%
\pgfpathlineto{\pgfqpoint{3.002835in}{3.398095in}}%
\pgfpathlineto{\pgfqpoint{3.036908in}{3.415429in}}%
\pgfpathlineto{\pgfqpoint{3.070965in}{3.433090in}}%
\pgfpathlineto{\pgfqpoint{3.061394in}{3.452243in}}%
\pgfpathlineto{\pgfqpoint{3.051816in}{3.473577in}}%
\pgfpathlineto{\pgfqpoint{3.017797in}{3.457637in}}%
\pgfpathlineto{\pgfqpoint{2.983764in}{3.441875in}}%
\pgfpathclose%
\pgfusepath{fill}%
\end{pgfscope}%
\begin{pgfscope}%
\pgfpathrectangle{\pgfqpoint{1.020000in}{0.880000in}}{\pgfqpoint{6.160000in}{6.160000in}}%
\pgfusepath{clip}%
\pgfsetbuttcap%
\pgfsetroundjoin%
\definecolor{currentfill}{rgb}{0.586921,0.718121,0.998874}%
\pgfsetfillcolor{currentfill}%
\pgfsetlinewidth{0.000000pt}%
\definecolor{currentstroke}{rgb}{0.000000,0.000000,0.000000}%
\pgfsetstrokecolor{currentstroke}%
\pgfsetdash{}{0pt}%
\pgfpathmoveto{\pgfqpoint{4.484297in}{3.303064in}}%
\pgfpathlineto{\pgfqpoint{4.495170in}{3.295776in}}%
\pgfpathlineto{\pgfqpoint{4.506061in}{3.287215in}}%
\pgfpathlineto{\pgfqpoint{4.539733in}{3.256252in}}%
\pgfpathlineto{\pgfqpoint{4.573367in}{3.226383in}}%
\pgfpathlineto{\pgfqpoint{4.562427in}{3.236216in}}%
\pgfpathlineto{\pgfqpoint{4.551505in}{3.244976in}}%
\pgfpathlineto{\pgfqpoint{4.517920in}{3.273496in}}%
\pgfpathlineto{\pgfqpoint{4.484297in}{3.303064in}}%
\pgfpathclose%
\pgfusepath{fill}%
\end{pgfscope}%
\begin{pgfscope}%
\pgfpathrectangle{\pgfqpoint{1.020000in}{0.880000in}}{\pgfqpoint{6.160000in}{6.160000in}}%
\pgfusepath{clip}%
\pgfsetbuttcap%
\pgfsetroundjoin%
\definecolor{currentfill}{rgb}{0.304174,0.406945,0.845263}%
\pgfsetfillcolor{currentfill}%
\pgfsetlinewidth{0.000000pt}%
\definecolor{currentstroke}{rgb}{0.000000,0.000000,0.000000}%
\pgfsetstrokecolor{currentstroke}%
\pgfsetdash{}{0pt}%
\pgfpathmoveto{\pgfqpoint{5.422666in}{2.768401in}}%
\pgfpathlineto{\pgfqpoint{5.434381in}{2.752281in}}%
\pgfpathlineto{\pgfqpoint{5.446119in}{2.736160in}}%
\pgfpathlineto{\pgfqpoint{5.479558in}{2.735476in}}%
\pgfpathlineto{\pgfqpoint{5.512980in}{2.735240in}}%
\pgfpathlineto{\pgfqpoint{5.501186in}{2.751122in}}%
\pgfpathlineto{\pgfqpoint{5.489414in}{2.767021in}}%
\pgfpathlineto{\pgfqpoint{5.456049in}{2.767476in}}%
\pgfpathlineto{\pgfqpoint{5.422666in}{2.768401in}}%
\pgfpathclose%
\pgfusepath{fill}%
\end{pgfscope}%
\begin{pgfscope}%
\pgfpathrectangle{\pgfqpoint{1.020000in}{0.880000in}}{\pgfqpoint{6.160000in}{6.160000in}}%
\pgfusepath{clip}%
\pgfsetbuttcap%
\pgfsetroundjoin%
\definecolor{currentfill}{rgb}{0.758539,0.832787,0.958408}%
\pgfsetfillcolor{currentfill}%
\pgfsetlinewidth{0.000000pt}%
\definecolor{currentstroke}{rgb}{0.000000,0.000000,0.000000}%
\pgfsetstrokecolor{currentstroke}%
\pgfsetdash{}{0pt}%
\pgfpathmoveto{\pgfqpoint{3.343049in}{3.569095in}}%
\pgfpathlineto{\pgfqpoint{3.352776in}{3.561762in}}%
\pgfpathlineto{\pgfqpoint{3.362505in}{3.556592in}}%
\pgfpathlineto{\pgfqpoint{3.396530in}{3.573234in}}%
\pgfpathlineto{\pgfqpoint{3.430552in}{3.588623in}}%
\pgfpathlineto{\pgfqpoint{3.420775in}{3.591107in}}%
\pgfpathlineto{\pgfqpoint{3.411003in}{3.595789in}}%
\pgfpathlineto{\pgfqpoint{3.377028in}{3.583049in}}%
\pgfpathlineto{\pgfqpoint{3.343049in}{3.569095in}}%
\pgfpathclose%
\pgfusepath{fill}%
\end{pgfscope}%
\begin{pgfscope}%
\pgfpathrectangle{\pgfqpoint{1.020000in}{0.880000in}}{\pgfqpoint{6.160000in}{6.160000in}}%
\pgfusepath{clip}%
\pgfsetbuttcap%
\pgfsetroundjoin%
\definecolor{currentfill}{rgb}{0.409611,0.540759,0.935545}%
\pgfsetfillcolor{currentfill}%
\pgfsetlinewidth{0.000000pt}%
\definecolor{currentstroke}{rgb}{0.000000,0.000000,0.000000}%
\pgfsetstrokecolor{currentstroke}%
\pgfsetdash{}{0pt}%
\pgfpathmoveto{\pgfqpoint{4.885976in}{2.980027in}}%
\pgfpathlineto{\pgfqpoint{4.897194in}{2.963598in}}%
\pgfpathlineto{\pgfqpoint{4.908429in}{2.946582in}}%
\pgfpathlineto{\pgfqpoint{4.941947in}{2.929925in}}%
\pgfpathlineto{\pgfqpoint{4.975445in}{2.914896in}}%
\pgfpathlineto{\pgfqpoint{4.964158in}{2.931743in}}%
\pgfpathlineto{\pgfqpoint{4.952889in}{2.948145in}}%
\pgfpathlineto{\pgfqpoint{4.919443in}{2.963273in}}%
\pgfpathlineto{\pgfqpoint{4.885976in}{2.980027in}}%
\pgfpathclose%
\pgfusepath{fill}%
\end{pgfscope}%
\begin{pgfscope}%
\pgfpathrectangle{\pgfqpoint{1.020000in}{0.880000in}}{\pgfqpoint{6.160000in}{6.160000in}}%
\pgfusepath{clip}%
\pgfsetbuttcap%
\pgfsetroundjoin%
\definecolor{currentfill}{rgb}{0.363461,0.484784,0.901019}%
\pgfsetfillcolor{currentfill}%
\pgfsetlinewidth{0.000000pt}%
\definecolor{currentstroke}{rgb}{0.000000,0.000000,0.000000}%
\pgfsetstrokecolor{currentstroke}%
\pgfsetdash{}{0pt}%
\pgfpathmoveto{\pgfqpoint{5.042380in}{2.889458in}}%
\pgfpathlineto{\pgfqpoint{5.053738in}{2.872443in}}%
\pgfpathlineto{\pgfqpoint{5.065114in}{2.855111in}}%
\pgfpathlineto{\pgfqpoint{5.098608in}{2.844849in}}%
\pgfpathlineto{\pgfqpoint{5.132084in}{2.835893in}}%
\pgfpathlineto{\pgfqpoint{5.120654in}{2.852892in}}%
\pgfpathlineto{\pgfqpoint{5.109243in}{2.869666in}}%
\pgfpathlineto{\pgfqpoint{5.075820in}{2.878899in}}%
\pgfpathlineto{\pgfqpoint{5.042380in}{2.889458in}}%
\pgfpathclose%
\pgfusepath{fill}%
\end{pgfscope}%
\begin{pgfscope}%
\pgfpathrectangle{\pgfqpoint{1.020000in}{0.880000in}}{\pgfqpoint{6.160000in}{6.160000in}}%
\pgfusepath{clip}%
\pgfsetbuttcap%
\pgfsetroundjoin%
\definecolor{currentfill}{rgb}{0.796064,0.848693,0.933471}%
\pgfsetfillcolor{currentfill}%
\pgfsetlinewidth{0.000000pt}%
\definecolor{currentstroke}{rgb}{0.000000,0.000000,0.000000}%
\pgfsetstrokecolor{currentstroke}%
\pgfsetdash{}{0pt}%
\pgfpathmoveto{\pgfqpoint{3.634586in}{3.645889in}}%
\pgfpathlineto{\pgfqpoint{3.644540in}{3.651902in}}%
\pgfpathlineto{\pgfqpoint{3.654509in}{3.659295in}}%
\pgfpathlineto{\pgfqpoint{3.688554in}{3.663389in}}%
\pgfpathlineto{\pgfqpoint{3.722589in}{3.665067in}}%
\pgfpathlineto{\pgfqpoint{3.712554in}{3.656351in}}%
\pgfpathlineto{\pgfqpoint{3.702536in}{3.648936in}}%
\pgfpathlineto{\pgfqpoint{3.668567in}{3.648495in}}%
\pgfpathlineto{\pgfqpoint{3.634586in}{3.645889in}}%
\pgfpathclose%
\pgfusepath{fill}%
\end{pgfscope}%
\begin{pgfscope}%
\pgfpathrectangle{\pgfqpoint{1.020000in}{0.880000in}}{\pgfqpoint{6.160000in}{6.160000in}}%
\pgfusepath{clip}%
\pgfsetbuttcap%
\pgfsetroundjoin%
\definecolor{currentfill}{rgb}{0.473070,0.611077,0.970634}%
\pgfsetfillcolor{currentfill}%
\pgfsetlinewidth{0.000000pt}%
\definecolor{currentstroke}{rgb}{0.000000,0.000000,0.000000}%
\pgfsetstrokecolor{currentstroke}%
\pgfsetdash{}{0pt}%
\pgfpathmoveto{\pgfqpoint{4.729679in}{3.094081in}}%
\pgfpathlineto{\pgfqpoint{4.740765in}{3.079513in}}%
\pgfpathlineto{\pgfqpoint{4.751868in}{3.064022in}}%
\pgfpathlineto{\pgfqpoint{4.785434in}{3.040448in}}%
\pgfpathlineto{\pgfqpoint{4.818973in}{3.018590in}}%
\pgfpathlineto{\pgfqpoint{4.807822in}{3.034279in}}%
\pgfpathlineto{\pgfqpoint{4.796687in}{3.049235in}}%
\pgfpathlineto{\pgfqpoint{4.763197in}{3.070813in}}%
\pgfpathlineto{\pgfqpoint{4.729679in}{3.094081in}}%
\pgfpathclose%
\pgfusepath{fill}%
\end{pgfscope}%
\begin{pgfscope}%
\pgfpathrectangle{\pgfqpoint{1.020000in}{0.880000in}}{\pgfqpoint{6.160000in}{6.160000in}}%
\pgfusepath{clip}%
\pgfsetbuttcap%
\pgfsetroundjoin%
\definecolor{currentfill}{rgb}{0.667253,0.779176,0.992959}%
\pgfsetfillcolor{currentfill}%
\pgfsetlinewidth{0.000000pt}%
\definecolor{currentstroke}{rgb}{0.000000,0.000000,0.000000}%
\pgfsetstrokecolor{currentstroke}%
\pgfsetdash{}{0pt}%
\pgfpathmoveto{\pgfqpoint{2.915646in}{3.411327in}}%
\pgfpathlineto{\pgfqpoint{2.925144in}{3.387199in}}%
\pgfpathlineto{\pgfqpoint{2.934637in}{3.364856in}}%
\pgfpathlineto{\pgfqpoint{2.968745in}{3.381203in}}%
\pgfpathlineto{\pgfqpoint{3.002835in}{3.398095in}}%
\pgfpathlineto{\pgfqpoint{2.993302in}{3.418991in}}%
\pgfpathlineto{\pgfqpoint{2.983764in}{3.441875in}}%
\pgfpathlineto{\pgfqpoint{2.949714in}{3.426404in}}%
\pgfpathlineto{\pgfqpoint{2.915646in}{3.411327in}}%
\pgfpathclose%
\pgfusepath{fill}%
\end{pgfscope}%
\begin{pgfscope}%
\pgfpathrectangle{\pgfqpoint{1.020000in}{0.880000in}}{\pgfqpoint{6.160000in}{6.160000in}}%
\pgfusepath{clip}%
\pgfsetbuttcap%
\pgfsetroundjoin%
\definecolor{currentfill}{rgb}{0.791392,0.846750,0.936641}%
\pgfsetfillcolor{currentfill}%
\pgfsetlinewidth{0.000000pt}%
\definecolor{currentstroke}{rgb}{0.000000,0.000000,0.000000}%
\pgfsetstrokecolor{currentstroke}%
\pgfsetdash{}{0pt}%
\pgfpathmoveto{\pgfqpoint{3.858585in}{3.647591in}}%
\pgfpathlineto{\pgfqpoint{3.868779in}{3.658071in}}%
\pgfpathlineto{\pgfqpoint{3.878996in}{3.668906in}}%
\pgfpathlineto{\pgfqpoint{3.913018in}{3.658433in}}%
\pgfpathlineto{\pgfqpoint{3.947016in}{3.645571in}}%
\pgfpathlineto{\pgfqpoint{3.936728in}{3.635212in}}%
\pgfpathlineto{\pgfqpoint{3.926463in}{3.625137in}}%
\pgfpathlineto{\pgfqpoint{3.892536in}{3.637424in}}%
\pgfpathlineto{\pgfqpoint{3.858585in}{3.647591in}}%
\pgfpathclose%
\pgfusepath{fill}%
\end{pgfscope}%
\begin{pgfscope}%
\pgfpathrectangle{\pgfqpoint{1.020000in}{0.880000in}}{\pgfqpoint{6.160000in}{6.160000in}}%
\pgfusepath{clip}%
\pgfsetbuttcap%
\pgfsetroundjoin%
\definecolor{currentfill}{rgb}{0.333490,0.446265,0.874452}%
\pgfsetfillcolor{currentfill}%
\pgfsetlinewidth{0.000000pt}%
\definecolor{currentstroke}{rgb}{0.000000,0.000000,0.000000}%
\pgfsetstrokecolor{currentstroke}%
\pgfsetdash{}{0pt}%
\pgfpathmoveto{\pgfqpoint{5.198987in}{2.821544in}}%
\pgfpathlineto{\pgfqpoint{5.210492in}{2.804686in}}%
\pgfpathlineto{\pgfqpoint{5.222017in}{2.787694in}}%
\pgfpathlineto{\pgfqpoint{5.255500in}{2.782458in}}%
\pgfpathlineto{\pgfqpoint{5.288966in}{2.778140in}}%
\pgfpathlineto{\pgfqpoint{5.277385in}{2.794781in}}%
\pgfpathlineto{\pgfqpoint{5.265824in}{2.811341in}}%
\pgfpathlineto{\pgfqpoint{5.232414in}{2.815969in}}%
\pgfpathlineto{\pgfqpoint{5.198987in}{2.821544in}}%
\pgfpathclose%
\pgfusepath{fill}%
\end{pgfscope}%
\begin{pgfscope}%
\pgfpathrectangle{\pgfqpoint{1.020000in}{0.880000in}}{\pgfqpoint{6.160000in}{6.160000in}}%
\pgfusepath{clip}%
\pgfsetbuttcap%
\pgfsetroundjoin%
\definecolor{currentfill}{rgb}{0.640828,0.760752,0.997846}%
\pgfsetfillcolor{currentfill}%
\pgfsetlinewidth{0.000000pt}%
\definecolor{currentstroke}{rgb}{0.000000,0.000000,0.000000}%
\pgfsetstrokecolor{currentstroke}%
\pgfsetdash{}{0pt}%
\pgfpathmoveto{\pgfqpoint{2.623666in}{3.369810in}}%
\pgfpathlineto{\pgfqpoint{2.632985in}{3.341070in}}%
\pgfpathlineto{\pgfqpoint{2.642301in}{3.313486in}}%
\pgfpathlineto{\pgfqpoint{2.676560in}{3.323074in}}%
\pgfpathlineto{\pgfqpoint{2.710790in}{3.333470in}}%
\pgfpathlineto{\pgfqpoint{2.701423in}{3.360974in}}%
\pgfpathlineto{\pgfqpoint{2.692052in}{3.389841in}}%
\pgfpathlineto{\pgfqpoint{2.657873in}{3.379455in}}%
\pgfpathlineto{\pgfqpoint{2.623666in}{3.369810in}}%
\pgfpathclose%
\pgfusepath{fill}%
\end{pgfscope}%
\begin{pgfscope}%
\pgfpathrectangle{\pgfqpoint{1.020000in}{0.880000in}}{\pgfqpoint{6.160000in}{6.160000in}}%
\pgfusepath{clip}%
\pgfsetbuttcap%
\pgfsetroundjoin%
\definecolor{currentfill}{rgb}{0.667253,0.779176,0.992959}%
\pgfsetfillcolor{currentfill}%
\pgfsetlinewidth{0.000000pt}%
\definecolor{currentstroke}{rgb}{0.000000,0.000000,0.000000}%
\pgfsetstrokecolor{currentstroke}%
\pgfsetdash{}{0pt}%
\pgfpathmoveto{\pgfqpoint{4.327926in}{3.431903in}}%
\pgfpathlineto{\pgfqpoint{4.338655in}{3.430574in}}%
\pgfpathlineto{\pgfqpoint{4.349404in}{3.427932in}}%
\pgfpathlineto{\pgfqpoint{4.383190in}{3.396175in}}%
\pgfpathlineto{\pgfqpoint{4.416933in}{3.364617in}}%
\pgfpathlineto{\pgfqpoint{4.406130in}{3.369029in}}%
\pgfpathlineto{\pgfqpoint{4.395347in}{3.372290in}}%
\pgfpathlineto{\pgfqpoint{4.361657in}{3.401997in}}%
\pgfpathlineto{\pgfqpoint{4.327926in}{3.431903in}}%
\pgfpathclose%
\pgfusepath{fill}%
\end{pgfscope}%
\begin{pgfscope}%
\pgfpathrectangle{\pgfqpoint{1.020000in}{0.880000in}}{\pgfqpoint{6.160000in}{6.160000in}}%
\pgfusepath{clip}%
\pgfsetbuttcap%
\pgfsetroundjoin%
\definecolor{currentfill}{rgb}{0.743754,0.825125,0.965798}%
\pgfsetfillcolor{currentfill}%
\pgfsetlinewidth{0.000000pt}%
\definecolor{currentstroke}{rgb}{0.000000,0.000000,0.000000}%
\pgfsetstrokecolor{currentstroke}%
\pgfsetdash{}{0pt}%
\pgfpathmoveto{\pgfqpoint{3.275074in}{3.538174in}}%
\pgfpathlineto{\pgfqpoint{3.284758in}{3.528177in}}%
\pgfpathlineto{\pgfqpoint{3.294444in}{3.520279in}}%
\pgfpathlineto{\pgfqpoint{3.328477in}{3.538878in}}%
\pgfpathlineto{\pgfqpoint{3.362505in}{3.556592in}}%
\pgfpathlineto{\pgfqpoint{3.352776in}{3.561762in}}%
\pgfpathlineto{\pgfqpoint{3.343049in}{3.569095in}}%
\pgfpathlineto{\pgfqpoint{3.309064in}{3.554083in}}%
\pgfpathlineto{\pgfqpoint{3.275074in}{3.538174in}}%
\pgfpathclose%
\pgfusepath{fill}%
\end{pgfscope}%
\begin{pgfscope}%
\pgfpathrectangle{\pgfqpoint{1.020000in}{0.880000in}}{\pgfqpoint{6.160000in}{6.160000in}}%
\pgfusepath{clip}%
\pgfsetbuttcap%
\pgfsetroundjoin%
\definecolor{currentfill}{rgb}{0.275827,0.366717,0.812553}%
\pgfsetfillcolor{currentfill}%
\pgfsetlinewidth{0.000000pt}%
\definecolor{currentstroke}{rgb}{0.000000,0.000000,0.000000}%
\pgfsetstrokecolor{currentstroke}%
\pgfsetdash{}{0pt}%
\pgfpathmoveto{\pgfqpoint{6.027923in}{2.697184in}}%
\pgfpathlineto{\pgfqpoint{6.040230in}{2.682481in}}%
\pgfpathlineto{\pgfqpoint{6.052561in}{2.667828in}}%
\pgfpathlineto{\pgfqpoint{6.085865in}{2.670390in}}%
\pgfpathlineto{\pgfqpoint{6.119149in}{2.672980in}}%
\pgfpathlineto{\pgfqpoint{6.106761in}{2.687549in}}%
\pgfpathlineto{\pgfqpoint{6.094399in}{2.702168in}}%
\pgfpathlineto{\pgfqpoint{6.061171in}{2.699660in}}%
\pgfpathlineto{\pgfqpoint{6.027923in}{2.697184in}}%
\pgfpathclose%
\pgfusepath{fill}%
\end{pgfscope}%
\begin{pgfscope}%
\pgfpathrectangle{\pgfqpoint{1.020000in}{0.880000in}}{\pgfqpoint{6.160000in}{6.160000in}}%
\pgfusepath{clip}%
\pgfsetbuttcap%
\pgfsetroundjoin%
\definecolor{currentfill}{rgb}{0.285273,0.380129,0.823469}%
\pgfsetfillcolor{currentfill}%
\pgfsetlinewidth{0.000000pt}%
\definecolor{currentstroke}{rgb}{0.000000,0.000000,0.000000}%
\pgfsetstrokecolor{currentstroke}%
\pgfsetdash{}{0pt}%
\pgfpathmoveto{\pgfqpoint{5.803812in}{2.713554in}}%
\pgfpathlineto{\pgfqpoint{5.815902in}{2.698443in}}%
\pgfpathlineto{\pgfqpoint{5.828016in}{2.683382in}}%
\pgfpathlineto{\pgfqpoint{5.861383in}{2.685507in}}%
\pgfpathlineto{\pgfqpoint{5.894731in}{2.687718in}}%
\pgfpathlineto{\pgfqpoint{5.882561in}{2.702665in}}%
\pgfpathlineto{\pgfqpoint{5.870415in}{2.717664in}}%
\pgfpathlineto{\pgfqpoint{5.837123in}{2.715562in}}%
\pgfpathlineto{\pgfqpoint{5.803812in}{2.713554in}}%
\pgfpathclose%
\pgfusepath{fill}%
\end{pgfscope}%
\begin{pgfscope}%
\pgfpathrectangle{\pgfqpoint{1.020000in}{0.880000in}}{\pgfqpoint{6.160000in}{6.160000in}}%
\pgfusepath{clip}%
\pgfsetbuttcap%
\pgfsetroundjoin%
\definecolor{currentfill}{rgb}{0.294718,0.393542,0.834384}%
\pgfsetfillcolor{currentfill}%
\pgfsetlinewidth{0.000000pt}%
\definecolor{currentstroke}{rgb}{0.000000,0.000000,0.000000}%
\pgfsetstrokecolor{currentstroke}%
\pgfsetdash{}{0pt}%
\pgfpathmoveto{\pgfqpoint{5.579772in}{2.735900in}}%
\pgfpathlineto{\pgfqpoint{5.591645in}{2.720246in}}%
\pgfpathlineto{\pgfqpoint{5.603542in}{2.704628in}}%
\pgfpathlineto{\pgfqpoint{5.636967in}{2.705612in}}%
\pgfpathlineto{\pgfqpoint{5.670374in}{2.706839in}}%
\pgfpathlineto{\pgfqpoint{5.658421in}{2.722278in}}%
\pgfpathlineto{\pgfqpoint{5.646491in}{2.737760in}}%
\pgfpathlineto{\pgfqpoint{5.613140in}{2.736700in}}%
\pgfpathlineto{\pgfqpoint{5.579772in}{2.735900in}}%
\pgfpathclose%
\pgfusepath{fill}%
\end{pgfscope}%
\begin{pgfscope}%
\pgfpathrectangle{\pgfqpoint{1.020000in}{0.880000in}}{\pgfqpoint{6.160000in}{6.160000in}}%
\pgfusepath{clip}%
\pgfsetbuttcap%
\pgfsetroundjoin%
\definecolor{currentfill}{rgb}{0.271104,0.360011,0.807095}%
\pgfsetfillcolor{currentfill}%
\pgfsetlinewidth{0.000000pt}%
\definecolor{currentstroke}{rgb}{0.000000,0.000000,0.000000}%
\pgfsetstrokecolor{currentstroke}%
\pgfsetdash{}{0pt}%
\pgfpathmoveto{\pgfqpoint{6.252080in}{2.683545in}}%
\pgfpathlineto{\pgfqpoint{6.264604in}{2.669178in}}%
\pgfpathlineto{\pgfqpoint{6.277154in}{2.654858in}}%
\pgfpathlineto{\pgfqpoint{6.310391in}{2.657609in}}%
\pgfpathlineto{\pgfqpoint{6.297814in}{2.671893in}}%
\pgfpathlineto{\pgfqpoint{6.285262in}{2.686224in}}%
\pgfpathlineto{\pgfqpoint{6.252080in}{2.683545in}}%
\pgfpathclose%
\pgfusepath{fill}%
\end{pgfscope}%
\begin{pgfscope}%
\pgfpathrectangle{\pgfqpoint{1.020000in}{0.880000in}}{\pgfqpoint{6.160000in}{6.160000in}}%
\pgfusepath{clip}%
\pgfsetbuttcap%
\pgfsetroundjoin%
\definecolor{currentfill}{rgb}{0.651398,0.768121,0.995891}%
\pgfsetfillcolor{currentfill}%
\pgfsetlinewidth{0.000000pt}%
\definecolor{currentstroke}{rgb}{0.000000,0.000000,0.000000}%
\pgfsetstrokecolor{currentstroke}%
\pgfsetdash{}{0pt}%
\pgfpathmoveto{\pgfqpoint{2.847453in}{3.382711in}}%
\pgfpathlineto{\pgfqpoint{2.856908in}{3.357632in}}%
\pgfpathlineto{\pgfqpoint{2.866360in}{3.334134in}}%
\pgfpathlineto{\pgfqpoint{2.900509in}{3.349141in}}%
\pgfpathlineto{\pgfqpoint{2.934637in}{3.364856in}}%
\pgfpathlineto{\pgfqpoint{2.925144in}{3.387199in}}%
\pgfpathlineto{\pgfqpoint{2.915646in}{3.411327in}}%
\pgfpathlineto{\pgfqpoint{2.881560in}{3.396736in}}%
\pgfpathlineto{\pgfqpoint{2.847453in}{3.382711in}}%
\pgfpathclose%
\pgfusepath{fill}%
\end{pgfscope}%
\begin{pgfscope}%
\pgfpathrectangle{\pgfqpoint{1.020000in}{0.880000in}}{\pgfqpoint{6.160000in}{6.160000in}}%
\pgfusepath{clip}%
\pgfsetbuttcap%
\pgfsetroundjoin%
\definecolor{currentfill}{rgb}{0.772706,0.838978,0.949319}%
\pgfsetfillcolor{currentfill}%
\pgfsetlinewidth{0.000000pt}%
\definecolor{currentstroke}{rgb}{0.000000,0.000000,0.000000}%
\pgfsetstrokecolor{currentstroke}%
\pgfsetdash{}{0pt}%
\pgfpathmoveto{\pgfqpoint{4.014931in}{3.613323in}}%
\pgfpathlineto{\pgfqpoint{4.025314in}{3.622570in}}%
\pgfpathlineto{\pgfqpoint{4.035722in}{3.631335in}}%
\pgfpathlineto{\pgfqpoint{4.069703in}{3.610966in}}%
\pgfpathlineto{\pgfqpoint{4.103649in}{3.588763in}}%
\pgfpathlineto{\pgfqpoint{4.093173in}{3.581455in}}%
\pgfpathlineto{\pgfqpoint{4.082722in}{3.573665in}}%
\pgfpathlineto{\pgfqpoint{4.048843in}{3.594315in}}%
\pgfpathlineto{\pgfqpoint{4.014931in}{3.613323in}}%
\pgfpathclose%
\pgfusepath{fill}%
\end{pgfscope}%
\begin{pgfscope}%
\pgfpathrectangle{\pgfqpoint{1.020000in}{0.880000in}}{\pgfqpoint{6.160000in}{6.160000in}}%
\pgfusepath{clip}%
\pgfsetbuttcap%
\pgfsetroundjoin%
\definecolor{currentfill}{rgb}{0.313946,0.420052,0.854993}%
\pgfsetfillcolor{currentfill}%
\pgfsetlinewidth{0.000000pt}%
\definecolor{currentstroke}{rgb}{0.000000,0.000000,0.000000}%
\pgfsetstrokecolor{currentstroke}%
\pgfsetdash{}{0pt}%
\pgfpathmoveto{\pgfqpoint{5.355849in}{2.771920in}}%
\pgfpathlineto{\pgfqpoint{5.367508in}{2.755533in}}%
\pgfpathlineto{\pgfqpoint{5.379189in}{2.739118in}}%
\pgfpathlineto{\pgfqpoint{5.412662in}{2.737352in}}%
\pgfpathlineto{\pgfqpoint{5.446119in}{2.736160in}}%
\pgfpathlineto{\pgfqpoint{5.434381in}{2.752281in}}%
\pgfpathlineto{\pgfqpoint{5.422666in}{2.768401in}}%
\pgfpathlineto{\pgfqpoint{5.389266in}{2.769859in}}%
\pgfpathlineto{\pgfqpoint{5.355849in}{2.771920in}}%
\pgfpathclose%
\pgfusepath{fill}%
\end{pgfscope}%
\begin{pgfscope}%
\pgfpathrectangle{\pgfqpoint{1.020000in}{0.880000in}}{\pgfqpoint{6.160000in}{6.160000in}}%
\pgfusepath{clip}%
\pgfsetbuttcap%
\pgfsetroundjoin%
\definecolor{currentfill}{rgb}{0.728970,0.817464,0.973188}%
\pgfsetfillcolor{currentfill}%
\pgfsetlinewidth{0.000000pt}%
\definecolor{currentstroke}{rgb}{0.000000,0.000000,0.000000}%
\pgfsetstrokecolor{currentstroke}%
\pgfsetdash{}{0pt}%
\pgfpathmoveto{\pgfqpoint{4.171433in}{3.539865in}}%
\pgfpathlineto{\pgfqpoint{4.181998in}{3.544611in}}%
\pgfpathlineto{\pgfqpoint{4.192587in}{3.548297in}}%
\pgfpathlineto{\pgfqpoint{4.226484in}{3.520236in}}%
\pgfpathlineto{\pgfqpoint{4.260339in}{3.491294in}}%
\pgfpathlineto{\pgfqpoint{4.249690in}{3.489502in}}%
\pgfpathlineto{\pgfqpoint{4.239063in}{3.486738in}}%
\pgfpathlineto{\pgfqpoint{4.205268in}{3.513697in}}%
\pgfpathlineto{\pgfqpoint{4.171433in}{3.539865in}}%
\pgfpathclose%
\pgfusepath{fill}%
\end{pgfscope}%
\begin{pgfscope}%
\pgfpathrectangle{\pgfqpoint{1.020000in}{0.880000in}}{\pgfqpoint{6.160000in}{6.160000in}}%
\pgfusepath{clip}%
\pgfsetbuttcap%
\pgfsetroundjoin%
\definecolor{currentfill}{rgb}{0.548876,0.685104,0.994379}%
\pgfsetfillcolor{currentfill}%
\pgfsetlinewidth{0.000000pt}%
\definecolor{currentstroke}{rgb}{0.000000,0.000000,0.000000}%
\pgfsetstrokecolor{currentstroke}%
\pgfsetdash{}{0pt}%
\pgfpathmoveto{\pgfqpoint{4.573367in}{3.226383in}}%
\pgfpathlineto{\pgfqpoint{4.584324in}{3.215377in}}%
\pgfpathlineto{\pgfqpoint{4.595298in}{3.203112in}}%
\pgfpathlineto{\pgfqpoint{4.628944in}{3.173617in}}%
\pgfpathlineto{\pgfqpoint{4.662555in}{3.145547in}}%
\pgfpathlineto{\pgfqpoint{4.651533in}{3.158561in}}%
\pgfpathlineto{\pgfqpoint{4.640528in}{3.170533in}}%
\pgfpathlineto{\pgfqpoint{4.606965in}{3.197767in}}%
\pgfpathlineto{\pgfqpoint{4.573367in}{3.226383in}}%
\pgfpathclose%
\pgfusepath{fill}%
\end{pgfscope}%
\begin{pgfscope}%
\pgfpathrectangle{\pgfqpoint{1.020000in}{0.880000in}}{\pgfqpoint{6.160000in}{6.160000in}}%
\pgfusepath{clip}%
\pgfsetbuttcap%
\pgfsetroundjoin%
\definecolor{currentfill}{rgb}{0.728970,0.817464,0.973188}%
\pgfsetfillcolor{currentfill}%
\pgfsetlinewidth{0.000000pt}%
\definecolor{currentstroke}{rgb}{0.000000,0.000000,0.000000}%
\pgfsetstrokecolor{currentstroke}%
\pgfsetdash{}{0pt}%
\pgfpathmoveto{\pgfqpoint{3.207073in}{3.504328in}}%
\pgfpathlineto{\pgfqpoint{3.216718in}{3.491752in}}%
\pgfpathlineto{\pgfqpoint{3.226363in}{3.481183in}}%
\pgfpathlineto{\pgfqpoint{3.260406in}{3.500984in}}%
\pgfpathlineto{\pgfqpoint{3.294444in}{3.520279in}}%
\pgfpathlineto{\pgfqpoint{3.284758in}{3.528177in}}%
\pgfpathlineto{\pgfqpoint{3.275074in}{3.538174in}}%
\pgfpathlineto{\pgfqpoint{3.241077in}{3.521533in}}%
\pgfpathlineto{\pgfqpoint{3.207073in}{3.504328in}}%
\pgfpathclose%
\pgfusepath{fill}%
\end{pgfscope}%
\begin{pgfscope}%
\pgfpathrectangle{\pgfqpoint{1.020000in}{0.880000in}}{\pgfqpoint{6.160000in}{6.160000in}}%
\pgfusepath{clip}%
\pgfsetbuttcap%
\pgfsetroundjoin%
\definecolor{currentfill}{rgb}{0.796064,0.848693,0.933471}%
\pgfsetfillcolor{currentfill}%
\pgfsetlinewidth{0.000000pt}%
\definecolor{currentstroke}{rgb}{0.000000,0.000000,0.000000}%
\pgfsetstrokecolor{currentstroke}%
\pgfsetdash{}{0pt}%
\pgfpathmoveto{\pgfqpoint{3.566598in}{3.634362in}}%
\pgfpathlineto{\pgfqpoint{3.576491in}{3.638511in}}%
\pgfpathlineto{\pgfqpoint{3.586399in}{3.644103in}}%
\pgfpathlineto{\pgfqpoint{3.620457in}{3.652839in}}%
\pgfpathlineto{\pgfqpoint{3.654509in}{3.659295in}}%
\pgfpathlineto{\pgfqpoint{3.644540in}{3.651902in}}%
\pgfpathlineto{\pgfqpoint{3.634586in}{3.645889in}}%
\pgfpathlineto{\pgfqpoint{3.600596in}{3.641157in}}%
\pgfpathlineto{\pgfqpoint{3.566598in}{3.634362in}}%
\pgfpathclose%
\pgfusepath{fill}%
\end{pgfscope}%
\begin{pgfscope}%
\pgfpathrectangle{\pgfqpoint{1.020000in}{0.880000in}}{\pgfqpoint{6.160000in}{6.160000in}}%
\pgfusepath{clip}%
\pgfsetbuttcap%
\pgfsetroundjoin%
\definecolor{currentfill}{rgb}{0.630089,0.752516,0.998508}%
\pgfsetfillcolor{currentfill}%
\pgfsetlinewidth{0.000000pt}%
\definecolor{currentstroke}{rgb}{0.000000,0.000000,0.000000}%
\pgfsetstrokecolor{currentstroke}%
\pgfsetdash{}{0pt}%
\pgfpathmoveto{\pgfqpoint{2.555169in}{3.352801in}}%
\pgfpathlineto{\pgfqpoint{2.564434in}{3.324290in}}%
\pgfpathlineto{\pgfqpoint{2.573698in}{3.296746in}}%
\pgfpathlineto{\pgfqpoint{2.608014in}{3.304711in}}%
\pgfpathlineto{\pgfqpoint{2.642301in}{3.313486in}}%
\pgfpathlineto{\pgfqpoint{2.632985in}{3.341070in}}%
\pgfpathlineto{\pgfqpoint{2.623666in}{3.369810in}}%
\pgfpathlineto{\pgfqpoint{2.589432in}{3.360922in}}%
\pgfpathlineto{\pgfqpoint{2.555169in}{3.352801in}}%
\pgfpathclose%
\pgfusepath{fill}%
\end{pgfscope}%
\begin{pgfscope}%
\pgfpathrectangle{\pgfqpoint{1.020000in}{0.880000in}}{\pgfqpoint{6.160000in}{6.160000in}}%
\pgfusepath{clip}%
\pgfsetbuttcap%
\pgfsetroundjoin%
\definecolor{currentfill}{rgb}{0.383662,0.510183,0.917831}%
\pgfsetfillcolor{currentfill}%
\pgfsetlinewidth{0.000000pt}%
\definecolor{currentstroke}{rgb}{0.000000,0.000000,0.000000}%
\pgfsetstrokecolor{currentstroke}%
\pgfsetdash{}{0pt}%
\pgfpathmoveto{\pgfqpoint{4.975445in}{2.914896in}}%
\pgfpathlineto{\pgfqpoint{4.986749in}{2.897606in}}%
\pgfpathlineto{\pgfqpoint{4.998071in}{2.879884in}}%
\pgfpathlineto{\pgfqpoint{5.031602in}{2.866762in}}%
\pgfpathlineto{\pgfqpoint{5.065114in}{2.855111in}}%
\pgfpathlineto{\pgfqpoint{5.053738in}{2.872443in}}%
\pgfpathlineto{\pgfqpoint{5.042380in}{2.889458in}}%
\pgfpathlineto{\pgfqpoint{5.008922in}{2.901431in}}%
\pgfpathlineto{\pgfqpoint{4.975445in}{2.914896in}}%
\pgfpathclose%
\pgfusepath{fill}%
\end{pgfscope}%
\begin{pgfscope}%
\pgfpathrectangle{\pgfqpoint{1.020000in}{0.880000in}}{\pgfqpoint{6.160000in}{6.160000in}}%
\pgfusepath{clip}%
\pgfsetbuttcap%
\pgfsetroundjoin%
\definecolor{currentfill}{rgb}{0.435815,0.570707,0.951717}%
\pgfsetfillcolor{currentfill}%
\pgfsetlinewidth{0.000000pt}%
\definecolor{currentstroke}{rgb}{0.000000,0.000000,0.000000}%
\pgfsetstrokecolor{currentstroke}%
\pgfsetdash{}{0pt}%
\pgfpathmoveto{\pgfqpoint{4.818973in}{3.018590in}}%
\pgfpathlineto{\pgfqpoint{4.830141in}{3.002150in}}%
\pgfpathlineto{\pgfqpoint{4.841324in}{2.984953in}}%
\pgfpathlineto{\pgfqpoint{4.874889in}{2.964913in}}%
\pgfpathlineto{\pgfqpoint{4.908429in}{2.946582in}}%
\pgfpathlineto{\pgfqpoint{4.897194in}{2.963598in}}%
\pgfpathlineto{\pgfqpoint{4.885976in}{2.980027in}}%
\pgfpathlineto{\pgfqpoint{4.852487in}{2.998455in}}%
\pgfpathlineto{\pgfqpoint{4.818973in}{3.018590in}}%
\pgfpathclose%
\pgfusepath{fill}%
\end{pgfscope}%
\begin{pgfscope}%
\pgfpathrectangle{\pgfqpoint{1.020000in}{0.880000in}}{\pgfqpoint{6.160000in}{6.160000in}}%
\pgfusepath{clip}%
\pgfsetbuttcap%
\pgfsetroundjoin%
\definecolor{currentfill}{rgb}{0.275827,0.366717,0.812553}%
\pgfsetfillcolor{currentfill}%
\pgfsetlinewidth{0.000000pt}%
\definecolor{currentstroke}{rgb}{0.000000,0.000000,0.000000}%
\pgfsetstrokecolor{currentstroke}%
\pgfsetdash{}{0pt}%
\pgfpathmoveto{\pgfqpoint{5.961367in}{2.692347in}}%
\pgfpathlineto{\pgfqpoint{5.973618in}{2.677553in}}%
\pgfpathlineto{\pgfqpoint{5.985893in}{2.662810in}}%
\pgfpathlineto{\pgfqpoint{6.019237in}{2.665300in}}%
\pgfpathlineto{\pgfqpoint{6.052561in}{2.667828in}}%
\pgfpathlineto{\pgfqpoint{6.040230in}{2.682481in}}%
\pgfpathlineto{\pgfqpoint{6.027923in}{2.697184in}}%
\pgfpathlineto{\pgfqpoint{5.994655in}{2.694744in}}%
\pgfpathlineto{\pgfqpoint{5.961367in}{2.692347in}}%
\pgfpathclose%
\pgfusepath{fill}%
\end{pgfscope}%
\begin{pgfscope}%
\pgfpathrectangle{\pgfqpoint{1.020000in}{0.880000in}}{\pgfqpoint{6.160000in}{6.160000in}}%
\pgfusepath{clip}%
\pgfsetbuttcap%
\pgfsetroundjoin%
\definecolor{currentfill}{rgb}{0.708720,0.805721,0.981117}%
\pgfsetfillcolor{currentfill}%
\pgfsetlinewidth{0.000000pt}%
\definecolor{currentstroke}{rgb}{0.000000,0.000000,0.000000}%
\pgfsetstrokecolor{currentstroke}%
\pgfsetdash{}{0pt}%
\pgfpathmoveto{\pgfqpoint{3.139041in}{3.468880in}}%
\pgfpathlineto{\pgfqpoint{3.148646in}{3.453890in}}%
\pgfpathlineto{\pgfqpoint{3.158252in}{3.440793in}}%
\pgfpathlineto{\pgfqpoint{3.192311in}{3.461059in}}%
\pgfpathlineto{\pgfqpoint{3.226363in}{3.481183in}}%
\pgfpathlineto{\pgfqpoint{3.216718in}{3.491752in}}%
\pgfpathlineto{\pgfqpoint{3.207073in}{3.504328in}}%
\pgfpathlineto{\pgfqpoint{3.173062in}{3.486724in}}%
\pgfpathlineto{\pgfqpoint{3.139041in}{3.468880in}}%
\pgfpathclose%
\pgfusepath{fill}%
\end{pgfscope}%
\begin{pgfscope}%
\pgfpathrectangle{\pgfqpoint{1.020000in}{0.880000in}}{\pgfqpoint{6.160000in}{6.160000in}}%
\pgfusepath{clip}%
\pgfsetbuttcap%
\pgfsetroundjoin%
\definecolor{currentfill}{rgb}{0.640828,0.760752,0.997846}%
\pgfsetfillcolor{currentfill}%
\pgfsetlinewidth{0.000000pt}%
\definecolor{currentstroke}{rgb}{0.000000,0.000000,0.000000}%
\pgfsetstrokecolor{currentstroke}%
\pgfsetdash{}{0pt}%
\pgfpathmoveto{\pgfqpoint{2.779171in}{3.356622in}}%
\pgfpathlineto{\pgfqpoint{2.788582in}{3.330858in}}%
\pgfpathlineto{\pgfqpoint{2.797990in}{3.306476in}}%
\pgfpathlineto{\pgfqpoint{2.832187in}{3.319896in}}%
\pgfpathlineto{\pgfqpoint{2.866360in}{3.334134in}}%
\pgfpathlineto{\pgfqpoint{2.856908in}{3.357632in}}%
\pgfpathlineto{\pgfqpoint{2.847453in}{3.382711in}}%
\pgfpathlineto{\pgfqpoint{2.813323in}{3.369321in}}%
\pgfpathlineto{\pgfqpoint{2.779171in}{3.356622in}}%
\pgfpathclose%
\pgfusepath{fill}%
\end{pgfscope}%
\begin{pgfscope}%
\pgfpathrectangle{\pgfqpoint{1.020000in}{0.880000in}}{\pgfqpoint{6.160000in}{6.160000in}}%
\pgfusepath{clip}%
\pgfsetbuttcap%
\pgfsetroundjoin%
\definecolor{currentfill}{rgb}{0.271104,0.360011,0.807095}%
\pgfsetfillcolor{currentfill}%
\pgfsetlinewidth{0.000000pt}%
\definecolor{currentstroke}{rgb}{0.000000,0.000000,0.000000}%
\pgfsetstrokecolor{currentstroke}%
\pgfsetdash{}{0pt}%
\pgfpathmoveto{\pgfqpoint{6.185655in}{2.678228in}}%
\pgfpathlineto{\pgfqpoint{6.198123in}{2.663787in}}%
\pgfpathlineto{\pgfqpoint{6.210617in}{2.649394in}}%
\pgfpathlineto{\pgfqpoint{6.243895in}{2.652120in}}%
\pgfpathlineto{\pgfqpoint{6.277154in}{2.654858in}}%
\pgfpathlineto{\pgfqpoint{6.264604in}{2.669178in}}%
\pgfpathlineto{\pgfqpoint{6.252080in}{2.683545in}}%
\pgfpathlineto{\pgfqpoint{6.218878in}{2.680879in}}%
\pgfpathlineto{\pgfqpoint{6.185655in}{2.678228in}}%
\pgfpathclose%
\pgfusepath{fill}%
\end{pgfscope}%
\begin{pgfscope}%
\pgfpathrectangle{\pgfqpoint{1.020000in}{0.880000in}}{\pgfqpoint{6.160000in}{6.160000in}}%
\pgfusepath{clip}%
\pgfsetbuttcap%
\pgfsetroundjoin%
\definecolor{currentfill}{rgb}{0.343278,0.459354,0.884122}%
\pgfsetfillcolor{currentfill}%
\pgfsetlinewidth{0.000000pt}%
\definecolor{currentstroke}{rgb}{0.000000,0.000000,0.000000}%
\pgfsetstrokecolor{currentstroke}%
\pgfsetdash{}{0pt}%
\pgfpathmoveto{\pgfqpoint{5.132084in}{2.835893in}}%
\pgfpathlineto{\pgfqpoint{5.143534in}{2.818679in}}%
\pgfpathlineto{\pgfqpoint{5.155003in}{2.801263in}}%
\pgfpathlineto{\pgfqpoint{5.188519in}{2.793932in}}%
\pgfpathlineto{\pgfqpoint{5.222017in}{2.787694in}}%
\pgfpathlineto{\pgfqpoint{5.210492in}{2.804686in}}%
\pgfpathlineto{\pgfqpoint{5.198987in}{2.821544in}}%
\pgfpathlineto{\pgfqpoint{5.165544in}{2.828155in}}%
\pgfpathlineto{\pgfqpoint{5.132084in}{2.835893in}}%
\pgfpathclose%
\pgfusepath{fill}%
\end{pgfscope}%
\begin{pgfscope}%
\pgfpathrectangle{\pgfqpoint{1.020000in}{0.880000in}}{\pgfqpoint{6.160000in}{6.160000in}}%
\pgfusepath{clip}%
\pgfsetbuttcap%
\pgfsetroundjoin%
\definecolor{currentfill}{rgb}{0.285273,0.380129,0.823469}%
\pgfsetfillcolor{currentfill}%
\pgfsetlinewidth{0.000000pt}%
\definecolor{currentstroke}{rgb}{0.000000,0.000000,0.000000}%
\pgfsetstrokecolor{currentstroke}%
\pgfsetdash{}{0pt}%
\pgfpathmoveto{\pgfqpoint{5.737131in}{2.709889in}}%
\pgfpathlineto{\pgfqpoint{5.749165in}{2.694648in}}%
\pgfpathlineto{\pgfqpoint{5.761222in}{2.679456in}}%
\pgfpathlineto{\pgfqpoint{5.794629in}{2.681359in}}%
\pgfpathlineto{\pgfqpoint{5.828016in}{2.683382in}}%
\pgfpathlineto{\pgfqpoint{5.815902in}{2.698443in}}%
\pgfpathlineto{\pgfqpoint{5.803812in}{2.713554in}}%
\pgfpathlineto{\pgfqpoint{5.770482in}{2.711657in}}%
\pgfpathlineto{\pgfqpoint{5.737131in}{2.709889in}}%
\pgfpathclose%
\pgfusepath{fill}%
\end{pgfscope}%
\begin{pgfscope}%
\pgfpathrectangle{\pgfqpoint{1.020000in}{0.880000in}}{\pgfqpoint{6.160000in}{6.160000in}}%
\pgfusepath{clip}%
\pgfsetbuttcap%
\pgfsetroundjoin%
\definecolor{currentfill}{rgb}{0.804965,0.851666,0.926165}%
\pgfsetfillcolor{currentfill}%
\pgfsetlinewidth{0.000000pt}%
\definecolor{currentstroke}{rgb}{0.000000,0.000000,0.000000}%
\pgfsetstrokecolor{currentstroke}%
\pgfsetdash{}{0pt}%
\pgfpathmoveto{\pgfqpoint{3.790620in}{3.661111in}}%
\pgfpathlineto{\pgfqpoint{3.800743in}{3.671444in}}%
\pgfpathlineto{\pgfqpoint{3.810889in}{3.682206in}}%
\pgfpathlineto{\pgfqpoint{3.844952in}{3.676863in}}%
\pgfpathlineto{\pgfqpoint{3.878996in}{3.668906in}}%
\pgfpathlineto{\pgfqpoint{3.868779in}{3.658071in}}%
\pgfpathlineto{\pgfqpoint{3.858585in}{3.647591in}}%
\pgfpathlineto{\pgfqpoint{3.824612in}{3.655517in}}%
\pgfpathlineto{\pgfqpoint{3.790620in}{3.661111in}}%
\pgfpathclose%
\pgfusepath{fill}%
\end{pgfscope}%
\begin{pgfscope}%
\pgfpathrectangle{\pgfqpoint{1.020000in}{0.880000in}}{\pgfqpoint{6.160000in}{6.160000in}}%
\pgfusepath{clip}%
\pgfsetbuttcap%
\pgfsetroundjoin%
\definecolor{currentfill}{rgb}{0.299441,0.400248,0.839842}%
\pgfsetfillcolor{currentfill}%
\pgfsetlinewidth{0.000000pt}%
\definecolor{currentstroke}{rgb}{0.000000,0.000000,0.000000}%
\pgfsetstrokecolor{currentstroke}%
\pgfsetdash{}{0pt}%
\pgfpathmoveto{\pgfqpoint{5.512980in}{2.735240in}}%
\pgfpathlineto{\pgfqpoint{5.524797in}{2.719378in}}%
\pgfpathlineto{\pgfqpoint{5.536637in}{2.703541in}}%
\pgfpathlineto{\pgfqpoint{5.570098in}{2.703923in}}%
\pgfpathlineto{\pgfqpoint{5.603542in}{2.704628in}}%
\pgfpathlineto{\pgfqpoint{5.591645in}{2.720246in}}%
\pgfpathlineto{\pgfqpoint{5.579772in}{2.735900in}}%
\pgfpathlineto{\pgfqpoint{5.546385in}{2.735398in}}%
\pgfpathlineto{\pgfqpoint{5.512980in}{2.735240in}}%
\pgfpathclose%
\pgfusepath{fill}%
\end{pgfscope}%
\begin{pgfscope}%
\pgfpathrectangle{\pgfqpoint{1.020000in}{0.880000in}}{\pgfqpoint{6.160000in}{6.160000in}}%
\pgfusepath{clip}%
\pgfsetbuttcap%
\pgfsetroundjoin%
\definecolor{currentfill}{rgb}{0.630089,0.752516,0.998508}%
\pgfsetfillcolor{currentfill}%
\pgfsetlinewidth{0.000000pt}%
\definecolor{currentstroke}{rgb}{0.000000,0.000000,0.000000}%
\pgfsetstrokecolor{currentstroke}%
\pgfsetdash{}{0pt}%
\pgfpathmoveto{\pgfqpoint{4.416933in}{3.364617in}}%
\pgfpathlineto{\pgfqpoint{4.427756in}{3.358880in}}%
\pgfpathlineto{\pgfqpoint{4.438597in}{3.351661in}}%
\pgfpathlineto{\pgfqpoint{4.472349in}{3.319089in}}%
\pgfpathlineto{\pgfqpoint{4.506061in}{3.287215in}}%
\pgfpathlineto{\pgfqpoint{4.495170in}{3.295776in}}%
\pgfpathlineto{\pgfqpoint{4.484297in}{3.303064in}}%
\pgfpathlineto{\pgfqpoint{4.450635in}{3.333504in}}%
\pgfpathlineto{\pgfqpoint{4.416933in}{3.364617in}}%
\pgfpathclose%
\pgfusepath{fill}%
\end{pgfscope}%
\begin{pgfscope}%
\pgfpathrectangle{\pgfqpoint{1.020000in}{0.880000in}}{\pgfqpoint{6.160000in}{6.160000in}}%
\pgfusepath{clip}%
\pgfsetbuttcap%
\pgfsetroundjoin%
\definecolor{currentfill}{rgb}{0.791392,0.846750,0.936641}%
\pgfsetfillcolor{currentfill}%
\pgfsetlinewidth{0.000000pt}%
\definecolor{currentstroke}{rgb}{0.000000,0.000000,0.000000}%
\pgfsetstrokecolor{currentstroke}%
\pgfsetdash{}{0pt}%
\pgfpathmoveto{\pgfqpoint{3.498584in}{3.614956in}}%
\pgfpathlineto{\pgfqpoint{3.508420in}{3.616861in}}%
\pgfpathlineto{\pgfqpoint{3.518270in}{3.620250in}}%
\pgfpathlineto{\pgfqpoint{3.552336in}{3.633195in}}%
\pgfpathlineto{\pgfqpoint{3.586399in}{3.644103in}}%
\pgfpathlineto{\pgfqpoint{3.576491in}{3.638511in}}%
\pgfpathlineto{\pgfqpoint{3.566598in}{3.634362in}}%
\pgfpathlineto{\pgfqpoint{3.532593in}{3.625592in}}%
\pgfpathlineto{\pgfqpoint{3.498584in}{3.614956in}}%
\pgfpathclose%
\pgfusepath{fill}%
\end{pgfscope}%
\begin{pgfscope}%
\pgfpathrectangle{\pgfqpoint{1.020000in}{0.880000in}}{\pgfqpoint{6.160000in}{6.160000in}}%
\pgfusepath{clip}%
\pgfsetbuttcap%
\pgfsetroundjoin%
\definecolor{currentfill}{rgb}{0.693321,0.796314,0.986308}%
\pgfsetfillcolor{currentfill}%
\pgfsetlinewidth{0.000000pt}%
\definecolor{currentstroke}{rgb}{0.000000,0.000000,0.000000}%
\pgfsetstrokecolor{currentstroke}%
\pgfsetdash{}{0pt}%
\pgfpathmoveto{\pgfqpoint{3.070965in}{3.433090in}}%
\pgfpathlineto{\pgfqpoint{3.080533in}{3.415915in}}%
\pgfpathlineto{\pgfqpoint{3.090099in}{3.400500in}}%
\pgfpathlineto{\pgfqpoint{3.124181in}{3.420553in}}%
\pgfpathlineto{\pgfqpoint{3.158252in}{3.440793in}}%
\pgfpathlineto{\pgfqpoint{3.148646in}{3.453890in}}%
\pgfpathlineto{\pgfqpoint{3.139041in}{3.468880in}}%
\pgfpathlineto{\pgfqpoint{3.105009in}{3.450953in}}%
\pgfpathlineto{\pgfqpoint{3.070965in}{3.433090in}}%
\pgfpathclose%
\pgfusepath{fill}%
\end{pgfscope}%
\begin{pgfscope}%
\pgfpathrectangle{\pgfqpoint{1.020000in}{0.880000in}}{\pgfqpoint{6.160000in}{6.160000in}}%
\pgfusepath{clip}%
\pgfsetbuttcap%
\pgfsetroundjoin%
\definecolor{currentfill}{rgb}{0.318832,0.426605,0.859857}%
\pgfsetfillcolor{currentfill}%
\pgfsetlinewidth{0.000000pt}%
\definecolor{currentstroke}{rgb}{0.000000,0.000000,0.000000}%
\pgfsetstrokecolor{currentstroke}%
\pgfsetdash{}{0pt}%
\pgfpathmoveto{\pgfqpoint{5.288966in}{2.778140in}}%
\pgfpathlineto{\pgfqpoint{5.300568in}{2.761426in}}%
\pgfpathlineto{\pgfqpoint{5.312192in}{2.744649in}}%
\pgfpathlineto{\pgfqpoint{5.345698in}{2.741525in}}%
\pgfpathlineto{\pgfqpoint{5.379189in}{2.739118in}}%
\pgfpathlineto{\pgfqpoint{5.367508in}{2.755533in}}%
\pgfpathlineto{\pgfqpoint{5.355849in}{2.771920in}}%
\pgfpathlineto{\pgfqpoint{5.322416in}{2.774654in}}%
\pgfpathlineto{\pgfqpoint{5.288966in}{2.778140in}}%
\pgfpathclose%
\pgfusepath{fill}%
\end{pgfscope}%
\begin{pgfscope}%
\pgfpathrectangle{\pgfqpoint{1.020000in}{0.880000in}}{\pgfqpoint{6.160000in}{6.160000in}}%
\pgfusepath{clip}%
\pgfsetbuttcap%
\pgfsetroundjoin%
\definecolor{currentfill}{rgb}{0.505423,0.643995,0.983157}%
\pgfsetfillcolor{currentfill}%
\pgfsetlinewidth{0.000000pt}%
\definecolor{currentstroke}{rgb}{0.000000,0.000000,0.000000}%
\pgfsetstrokecolor{currentstroke}%
\pgfsetdash{}{0pt}%
\pgfpathmoveto{\pgfqpoint{4.662555in}{3.145547in}}%
\pgfpathlineto{\pgfqpoint{4.673592in}{3.131436in}}%
\pgfpathlineto{\pgfqpoint{4.684646in}{3.116185in}}%
\pgfpathlineto{\pgfqpoint{4.718272in}{3.089286in}}%
\pgfpathlineto{\pgfqpoint{4.751868in}{3.064022in}}%
\pgfpathlineto{\pgfqpoint{4.740765in}{3.079513in}}%
\pgfpathlineto{\pgfqpoint{4.729679in}{3.094081in}}%
\pgfpathlineto{\pgfqpoint{4.696132in}{3.119009in}}%
\pgfpathlineto{\pgfqpoint{4.662555in}{3.145547in}}%
\pgfpathclose%
\pgfusepath{fill}%
\end{pgfscope}%
\begin{pgfscope}%
\pgfpathrectangle{\pgfqpoint{1.020000in}{0.880000in}}{\pgfqpoint{6.160000in}{6.160000in}}%
\pgfusepath{clip}%
\pgfsetbuttcap%
\pgfsetroundjoin%
\definecolor{currentfill}{rgb}{0.630089,0.752516,0.998508}%
\pgfsetfillcolor{currentfill}%
\pgfsetlinewidth{0.000000pt}%
\definecolor{currentstroke}{rgb}{0.000000,0.000000,0.000000}%
\pgfsetstrokecolor{currentstroke}%
\pgfsetdash{}{0pt}%
\pgfpathmoveto{\pgfqpoint{2.710790in}{3.333470in}}%
\pgfpathlineto{\pgfqpoint{2.720154in}{3.307248in}}%
\pgfpathlineto{\pgfqpoint{2.729517in}{3.282219in}}%
\pgfpathlineto{\pgfqpoint{2.763767in}{3.293909in}}%
\pgfpathlineto{\pgfqpoint{2.797990in}{3.306476in}}%
\pgfpathlineto{\pgfqpoint{2.788582in}{3.330858in}}%
\pgfpathlineto{\pgfqpoint{2.779171in}{3.356622in}}%
\pgfpathlineto{\pgfqpoint{2.744993in}{3.344661in}}%
\pgfpathlineto{\pgfqpoint{2.710790in}{3.333470in}}%
\pgfpathclose%
\pgfusepath{fill}%
\end{pgfscope}%
\begin{pgfscope}%
\pgfpathrectangle{\pgfqpoint{1.020000in}{0.880000in}}{\pgfqpoint{6.160000in}{6.160000in}}%
\pgfusepath{clip}%
\pgfsetbuttcap%
\pgfsetroundjoin%
\definecolor{currentfill}{rgb}{0.703587,0.802586,0.982847}%
\pgfsetfillcolor{currentfill}%
\pgfsetlinewidth{0.000000pt}%
\definecolor{currentstroke}{rgb}{0.000000,0.000000,0.000000}%
\pgfsetstrokecolor{currentstroke}%
\pgfsetdash{}{0pt}%
\pgfpathmoveto{\pgfqpoint{4.260339in}{3.491294in}}%
\pgfpathlineto{\pgfqpoint{4.271011in}{3.491858in}}%
\pgfpathlineto{\pgfqpoint{4.281705in}{3.490953in}}%
\pgfpathlineto{\pgfqpoint{4.315576in}{3.459619in}}%
\pgfpathlineto{\pgfqpoint{4.349404in}{3.427932in}}%
\pgfpathlineto{\pgfqpoint{4.338655in}{3.430574in}}%
\pgfpathlineto{\pgfqpoint{4.327926in}{3.431903in}}%
\pgfpathlineto{\pgfqpoint{4.294154in}{3.461757in}}%
\pgfpathlineto{\pgfqpoint{4.260339in}{3.491294in}}%
\pgfpathclose%
\pgfusepath{fill}%
\end{pgfscope}%
\begin{pgfscope}%
\pgfpathrectangle{\pgfqpoint{1.020000in}{0.880000in}}{\pgfqpoint{6.160000in}{6.160000in}}%
\pgfusepath{clip}%
\pgfsetbuttcap%
\pgfsetroundjoin%
\definecolor{currentfill}{rgb}{0.796064,0.848693,0.933471}%
\pgfsetfillcolor{currentfill}%
\pgfsetlinewidth{0.000000pt}%
\definecolor{currentstroke}{rgb}{0.000000,0.000000,0.000000}%
\pgfsetstrokecolor{currentstroke}%
\pgfsetdash{}{0pt}%
\pgfpathmoveto{\pgfqpoint{3.947016in}{3.645571in}}%
\pgfpathlineto{\pgfqpoint{3.957329in}{3.655843in}}%
\pgfpathlineto{\pgfqpoint{3.967667in}{3.665647in}}%
\pgfpathlineto{\pgfqpoint{4.001709in}{3.649635in}}%
\pgfpathlineto{\pgfqpoint{4.035722in}{3.631335in}}%
\pgfpathlineto{\pgfqpoint{4.025314in}{3.622570in}}%
\pgfpathlineto{\pgfqpoint{4.014931in}{3.613323in}}%
\pgfpathlineto{\pgfqpoint{3.980988in}{3.630475in}}%
\pgfpathlineto{\pgfqpoint{3.947016in}{3.645571in}}%
\pgfpathclose%
\pgfusepath{fill}%
\end{pgfscope}%
\begin{pgfscope}%
\pgfpathrectangle{\pgfqpoint{1.020000in}{0.880000in}}{\pgfqpoint{6.160000in}{6.160000in}}%
\pgfusepath{clip}%
\pgfsetbuttcap%
\pgfsetroundjoin%
\definecolor{currentfill}{rgb}{0.280550,0.373423,0.818011}%
\pgfsetfillcolor{currentfill}%
\pgfsetlinewidth{0.000000pt}%
\definecolor{currentstroke}{rgb}{0.000000,0.000000,0.000000}%
\pgfsetstrokecolor{currentstroke}%
\pgfsetdash{}{0pt}%
\pgfpathmoveto{\pgfqpoint{5.894731in}{2.687718in}}%
\pgfpathlineto{\pgfqpoint{5.906925in}{2.672823in}}%
\pgfpathlineto{\pgfqpoint{5.919144in}{2.657981in}}%
\pgfpathlineto{\pgfqpoint{5.952528in}{2.660368in}}%
\pgfpathlineto{\pgfqpoint{5.985893in}{2.662810in}}%
\pgfpathlineto{\pgfqpoint{5.973618in}{2.677553in}}%
\pgfpathlineto{\pgfqpoint{5.961367in}{2.692347in}}%
\pgfpathlineto{\pgfqpoint{5.928059in}{2.690002in}}%
\pgfpathlineto{\pgfqpoint{5.894731in}{2.687718in}}%
\pgfpathclose%
\pgfusepath{fill}%
\end{pgfscope}%
\begin{pgfscope}%
\pgfpathrectangle{\pgfqpoint{1.020000in}{0.880000in}}{\pgfqpoint{6.160000in}{6.160000in}}%
\pgfusepath{clip}%
\pgfsetbuttcap%
\pgfsetroundjoin%
\definecolor{currentfill}{rgb}{0.271104,0.360011,0.807095}%
\pgfsetfillcolor{currentfill}%
\pgfsetlinewidth{0.000000pt}%
\definecolor{currentstroke}{rgb}{0.000000,0.000000,0.000000}%
\pgfsetstrokecolor{currentstroke}%
\pgfsetdash{}{0pt}%
\pgfpathmoveto{\pgfqpoint{6.119149in}{2.672980in}}%
\pgfpathlineto{\pgfqpoint{6.131561in}{2.658462in}}%
\pgfpathlineto{\pgfqpoint{6.143998in}{2.643993in}}%
\pgfpathlineto{\pgfqpoint{6.177318in}{2.646684in}}%
\pgfpathlineto{\pgfqpoint{6.210617in}{2.649394in}}%
\pgfpathlineto{\pgfqpoint{6.198123in}{2.663787in}}%
\pgfpathlineto{\pgfqpoint{6.185655in}{2.678228in}}%
\pgfpathlineto{\pgfqpoint{6.152412in}{2.675594in}}%
\pgfpathlineto{\pgfqpoint{6.119149in}{2.672980in}}%
\pgfpathclose%
\pgfusepath{fill}%
\end{pgfscope}%
\begin{pgfscope}%
\pgfpathrectangle{\pgfqpoint{1.020000in}{0.880000in}}{\pgfqpoint{6.160000in}{6.160000in}}%
\pgfusepath{clip}%
\pgfsetbuttcap%
\pgfsetroundjoin%
\definecolor{currentfill}{rgb}{0.672538,0.782861,0.991982}%
\pgfsetfillcolor{currentfill}%
\pgfsetlinewidth{0.000000pt}%
\definecolor{currentstroke}{rgb}{0.000000,0.000000,0.000000}%
\pgfsetstrokecolor{currentstroke}%
\pgfsetdash{}{0pt}%
\pgfpathmoveto{\pgfqpoint{3.002835in}{3.398095in}}%
\pgfpathlineto{\pgfqpoint{3.012364in}{3.379005in}}%
\pgfpathlineto{\pgfqpoint{3.021893in}{3.361527in}}%
\pgfpathlineto{\pgfqpoint{3.056004in}{3.380780in}}%
\pgfpathlineto{\pgfqpoint{3.090099in}{3.400500in}}%
\pgfpathlineto{\pgfqpoint{3.080533in}{3.415915in}}%
\pgfpathlineto{\pgfqpoint{3.070965in}{3.433090in}}%
\pgfpathlineto{\pgfqpoint{3.036908in}{3.415429in}}%
\pgfpathlineto{\pgfqpoint{3.002835in}{3.398095in}}%
\pgfpathclose%
\pgfusepath{fill}%
\end{pgfscope}%
\begin{pgfscope}%
\pgfpathrectangle{\pgfqpoint{1.020000in}{0.880000in}}{\pgfqpoint{6.160000in}{6.160000in}}%
\pgfusepath{clip}%
\pgfsetbuttcap%
\pgfsetroundjoin%
\definecolor{currentfill}{rgb}{0.285273,0.380129,0.823469}%
\pgfsetfillcolor{currentfill}%
\pgfsetlinewidth{0.000000pt}%
\definecolor{currentstroke}{rgb}{0.000000,0.000000,0.000000}%
\pgfsetstrokecolor{currentstroke}%
\pgfsetdash{}{0pt}%
\pgfpathmoveto{\pgfqpoint{5.670374in}{2.706839in}}%
\pgfpathlineto{\pgfqpoint{5.682351in}{2.691444in}}%
\pgfpathlineto{\pgfqpoint{5.694351in}{2.676096in}}%
\pgfpathlineto{\pgfqpoint{5.727796in}{2.677693in}}%
\pgfpathlineto{\pgfqpoint{5.761222in}{2.679456in}}%
\pgfpathlineto{\pgfqpoint{5.749165in}{2.694648in}}%
\pgfpathlineto{\pgfqpoint{5.737131in}{2.709889in}}%
\pgfpathlineto{\pgfqpoint{5.703762in}{2.708275in}}%
\pgfpathlineto{\pgfqpoint{5.670374in}{2.706839in}}%
\pgfpathclose%
\pgfusepath{fill}%
\end{pgfscope}%
\begin{pgfscope}%
\pgfpathrectangle{\pgfqpoint{1.020000in}{0.880000in}}{\pgfqpoint{6.160000in}{6.160000in}}%
\pgfusepath{clip}%
\pgfsetbuttcap%
\pgfsetroundjoin%
\definecolor{currentfill}{rgb}{0.777378,0.840921,0.946149}%
\pgfsetfillcolor{currentfill}%
\pgfsetlinewidth{0.000000pt}%
\definecolor{currentstroke}{rgb}{0.000000,0.000000,0.000000}%
\pgfsetstrokecolor{currentstroke}%
\pgfsetdash{}{0pt}%
\pgfpathmoveto{\pgfqpoint{3.430552in}{3.588623in}}%
\pgfpathlineto{\pgfqpoint{3.440336in}{3.588000in}}%
\pgfpathlineto{\pgfqpoint{3.450131in}{3.588882in}}%
\pgfpathlineto{\pgfqpoint{3.484201in}{3.605421in}}%
\pgfpathlineto{\pgfqpoint{3.518270in}{3.620250in}}%
\pgfpathlineto{\pgfqpoint{3.508420in}{3.616861in}}%
\pgfpathlineto{\pgfqpoint{3.498584in}{3.614956in}}%
\pgfpathlineto{\pgfqpoint{3.464569in}{3.602584in}}%
\pgfpathlineto{\pgfqpoint{3.430552in}{3.588623in}}%
\pgfpathclose%
\pgfusepath{fill}%
\end{pgfscope}%
\begin{pgfscope}%
\pgfpathrectangle{\pgfqpoint{1.020000in}{0.880000in}}{\pgfqpoint{6.160000in}{6.160000in}}%
\pgfusepath{clip}%
\pgfsetbuttcap%
\pgfsetroundjoin%
\definecolor{currentfill}{rgb}{0.299441,0.400248,0.839842}%
\pgfsetfillcolor{currentfill}%
\pgfsetlinewidth{0.000000pt}%
\definecolor{currentstroke}{rgb}{0.000000,0.000000,0.000000}%
\pgfsetstrokecolor{currentstroke}%
\pgfsetdash{}{0pt}%
\pgfpathmoveto{\pgfqpoint{5.446119in}{2.736160in}}%
\pgfpathlineto{\pgfqpoint{5.457878in}{2.720042in}}%
\pgfpathlineto{\pgfqpoint{5.469660in}{2.703934in}}%
\pgfpathlineto{\pgfqpoint{5.503157in}{2.703527in}}%
\pgfpathlineto{\pgfqpoint{5.536637in}{2.703541in}}%
\pgfpathlineto{\pgfqpoint{5.524797in}{2.719378in}}%
\pgfpathlineto{\pgfqpoint{5.512980in}{2.735240in}}%
\pgfpathlineto{\pgfqpoint{5.479558in}{2.735476in}}%
\pgfpathlineto{\pgfqpoint{5.446119in}{2.736160in}}%
\pgfpathclose%
\pgfusepath{fill}%
\end{pgfscope}%
\begin{pgfscope}%
\pgfpathrectangle{\pgfqpoint{1.020000in}{0.880000in}}{\pgfqpoint{6.160000in}{6.160000in}}%
\pgfusepath{clip}%
\pgfsetbuttcap%
\pgfsetroundjoin%
\definecolor{currentfill}{rgb}{0.763363,0.835092,0.955658}%
\pgfsetfillcolor{currentfill}%
\pgfsetlinewidth{0.000000pt}%
\definecolor{currentstroke}{rgb}{0.000000,0.000000,0.000000}%
\pgfsetstrokecolor{currentstroke}%
\pgfsetdash{}{0pt}%
\pgfpathmoveto{\pgfqpoint{4.103649in}{3.588763in}}%
\pgfpathlineto{\pgfqpoint{4.114150in}{3.595260in}}%
\pgfpathlineto{\pgfqpoint{4.124675in}{3.600624in}}%
\pgfpathlineto{\pgfqpoint{4.158650in}{3.575190in}}%
\pgfpathlineto{\pgfqpoint{4.192587in}{3.548297in}}%
\pgfpathlineto{\pgfqpoint{4.181998in}{3.544611in}}%
\pgfpathlineto{\pgfqpoint{4.171433in}{3.539865in}}%
\pgfpathlineto{\pgfqpoint{4.137560in}{3.564976in}}%
\pgfpathlineto{\pgfqpoint{4.103649in}{3.588763in}}%
\pgfpathclose%
\pgfusepath{fill}%
\end{pgfscope}%
\begin{pgfscope}%
\pgfpathrectangle{\pgfqpoint{1.020000in}{0.880000in}}{\pgfqpoint{6.160000in}{6.160000in}}%
\pgfusepath{clip}%
\pgfsetbuttcap%
\pgfsetroundjoin%
\definecolor{currentfill}{rgb}{0.813693,0.854282,0.918480}%
\pgfsetfillcolor{currentfill}%
\pgfsetlinewidth{0.000000pt}%
\definecolor{currentstroke}{rgb}{0.000000,0.000000,0.000000}%
\pgfsetstrokecolor{currentstroke}%
\pgfsetdash{}{0pt}%
\pgfpathmoveto{\pgfqpoint{3.722589in}{3.665067in}}%
\pgfpathlineto{\pgfqpoint{3.732643in}{3.674693in}}%
\pgfpathlineto{\pgfqpoint{3.742719in}{3.684822in}}%
\pgfpathlineto{\pgfqpoint{3.776811in}{3.684870in}}%
\pgfpathlineto{\pgfqpoint{3.810889in}{3.682206in}}%
\pgfpathlineto{\pgfqpoint{3.800743in}{3.671444in}}%
\pgfpathlineto{\pgfqpoint{3.790620in}{3.661111in}}%
\pgfpathlineto{\pgfqpoint{3.756611in}{3.664306in}}%
\pgfpathlineto{\pgfqpoint{3.722589in}{3.665067in}}%
\pgfpathclose%
\pgfusepath{fill}%
\end{pgfscope}%
\begin{pgfscope}%
\pgfpathrectangle{\pgfqpoint{1.020000in}{0.880000in}}{\pgfqpoint{6.160000in}{6.160000in}}%
\pgfusepath{clip}%
\pgfsetbuttcap%
\pgfsetroundjoin%
\definecolor{currentfill}{rgb}{0.404421,0.534643,0.932002}%
\pgfsetfillcolor{currentfill}%
\pgfsetlinewidth{0.000000pt}%
\definecolor{currentstroke}{rgb}{0.000000,0.000000,0.000000}%
\pgfsetstrokecolor{currentstroke}%
\pgfsetdash{}{0pt}%
\pgfpathmoveto{\pgfqpoint{4.908429in}{2.946582in}}%
\pgfpathlineto{\pgfqpoint{4.919681in}{2.928981in}}%
\pgfpathlineto{\pgfqpoint{4.930949in}{2.910808in}}%
\pgfpathlineto{\pgfqpoint{4.964520in}{2.894545in}}%
\pgfpathlineto{\pgfqpoint{4.998071in}{2.879884in}}%
\pgfpathlineto{\pgfqpoint{4.986749in}{2.897606in}}%
\pgfpathlineto{\pgfqpoint{4.975445in}{2.914896in}}%
\pgfpathlineto{\pgfqpoint{4.941947in}{2.929925in}}%
\pgfpathlineto{\pgfqpoint{4.908429in}{2.946582in}}%
\pgfpathclose%
\pgfusepath{fill}%
\end{pgfscope}%
\begin{pgfscope}%
\pgfpathrectangle{\pgfqpoint{1.020000in}{0.880000in}}{\pgfqpoint{6.160000in}{6.160000in}}%
\pgfusepath{clip}%
\pgfsetbuttcap%
\pgfsetroundjoin%
\definecolor{currentfill}{rgb}{0.358415,0.478426,0.896795}%
\pgfsetfillcolor{currentfill}%
\pgfsetlinewidth{0.000000pt}%
\definecolor{currentstroke}{rgb}{0.000000,0.000000,0.000000}%
\pgfsetstrokecolor{currentstroke}%
\pgfsetdash{}{0pt}%
\pgfpathmoveto{\pgfqpoint{5.065114in}{2.855111in}}%
\pgfpathlineto{\pgfqpoint{5.076508in}{2.837472in}}%
\pgfpathlineto{\pgfqpoint{5.087921in}{2.819545in}}%
\pgfpathlineto{\pgfqpoint{5.121471in}{2.809772in}}%
\pgfpathlineto{\pgfqpoint{5.155003in}{2.801263in}}%
\pgfpathlineto{\pgfqpoint{5.143534in}{2.818679in}}%
\pgfpathlineto{\pgfqpoint{5.132084in}{2.835893in}}%
\pgfpathlineto{\pgfqpoint{5.098608in}{2.844849in}}%
\pgfpathlineto{\pgfqpoint{5.065114in}{2.855111in}}%
\pgfpathclose%
\pgfusepath{fill}%
\end{pgfscope}%
\begin{pgfscope}%
\pgfpathrectangle{\pgfqpoint{1.020000in}{0.880000in}}{\pgfqpoint{6.160000in}{6.160000in}}%
\pgfusepath{clip}%
\pgfsetbuttcap%
\pgfsetroundjoin%
\definecolor{currentfill}{rgb}{0.586921,0.718121,0.998874}%
\pgfsetfillcolor{currentfill}%
\pgfsetlinewidth{0.000000pt}%
\definecolor{currentstroke}{rgb}{0.000000,0.000000,0.000000}%
\pgfsetstrokecolor{currentstroke}%
\pgfsetdash{}{0pt}%
\pgfpathmoveto{\pgfqpoint{4.506061in}{3.287215in}}%
\pgfpathlineto{\pgfqpoint{4.516970in}{3.277262in}}%
\pgfpathlineto{\pgfqpoint{4.527895in}{3.265813in}}%
\pgfpathlineto{\pgfqpoint{4.561616in}{3.233898in}}%
\pgfpathlineto{\pgfqpoint{4.595298in}{3.203112in}}%
\pgfpathlineto{\pgfqpoint{4.584324in}{3.215377in}}%
\pgfpathlineto{\pgfqpoint{4.573367in}{3.226383in}}%
\pgfpathlineto{\pgfqpoint{4.539733in}{3.256252in}}%
\pgfpathlineto{\pgfqpoint{4.506061in}{3.287215in}}%
\pgfpathclose%
\pgfusepath{fill}%
\end{pgfscope}%
\begin{pgfscope}%
\pgfpathrectangle{\pgfqpoint{1.020000in}{0.880000in}}{\pgfqpoint{6.160000in}{6.160000in}}%
\pgfusepath{clip}%
\pgfsetbuttcap%
\pgfsetroundjoin%
\definecolor{currentfill}{rgb}{0.619318,0.744121,0.998931}%
\pgfsetfillcolor{currentfill}%
\pgfsetlinewidth{0.000000pt}%
\definecolor{currentstroke}{rgb}{0.000000,0.000000,0.000000}%
\pgfsetstrokecolor{currentstroke}%
\pgfsetdash{}{0pt}%
\pgfpathmoveto{\pgfqpoint{2.642301in}{3.313486in}}%
\pgfpathlineto{\pgfqpoint{2.651617in}{3.286991in}}%
\pgfpathlineto{\pgfqpoint{2.660933in}{3.261512in}}%
\pgfpathlineto{\pgfqpoint{2.695239in}{3.271419in}}%
\pgfpathlineto{\pgfqpoint{2.729517in}{3.282219in}}%
\pgfpathlineto{\pgfqpoint{2.720154in}{3.307248in}}%
\pgfpathlineto{\pgfqpoint{2.710790in}{3.333470in}}%
\pgfpathlineto{\pgfqpoint{2.676560in}{3.323074in}}%
\pgfpathlineto{\pgfqpoint{2.642301in}{3.313486in}}%
\pgfpathclose%
\pgfusepath{fill}%
\end{pgfscope}%
\begin{pgfscope}%
\pgfpathrectangle{\pgfqpoint{1.020000in}{0.880000in}}{\pgfqpoint{6.160000in}{6.160000in}}%
\pgfusepath{clip}%
\pgfsetbuttcap%
\pgfsetroundjoin%
\definecolor{currentfill}{rgb}{0.656683,0.771806,0.994914}%
\pgfsetfillcolor{currentfill}%
\pgfsetlinewidth{0.000000pt}%
\definecolor{currentstroke}{rgb}{0.000000,0.000000,0.000000}%
\pgfsetstrokecolor{currentstroke}%
\pgfsetdash{}{0pt}%
\pgfpathmoveto{\pgfqpoint{2.934637in}{3.364856in}}%
\pgfpathlineto{\pgfqpoint{2.944127in}{3.344140in}}%
\pgfpathlineto{\pgfqpoint{2.953617in}{3.324882in}}%
\pgfpathlineto{\pgfqpoint{2.987764in}{3.342860in}}%
\pgfpathlineto{\pgfqpoint{3.021893in}{3.361527in}}%
\pgfpathlineto{\pgfqpoint{3.012364in}{3.379005in}}%
\pgfpathlineto{\pgfqpoint{3.002835in}{3.398095in}}%
\pgfpathlineto{\pgfqpoint{2.968745in}{3.381203in}}%
\pgfpathlineto{\pgfqpoint{2.934637in}{3.364856in}}%
\pgfpathclose%
\pgfusepath{fill}%
\end{pgfscope}%
\begin{pgfscope}%
\pgfpathrectangle{\pgfqpoint{1.020000in}{0.880000in}}{\pgfqpoint{6.160000in}{6.160000in}}%
\pgfusepath{clip}%
\pgfsetbuttcap%
\pgfsetroundjoin%
\definecolor{currentfill}{rgb}{0.768034,0.837035,0.952488}%
\pgfsetfillcolor{currentfill}%
\pgfsetlinewidth{0.000000pt}%
\definecolor{currentstroke}{rgb}{0.000000,0.000000,0.000000}%
\pgfsetstrokecolor{currentstroke}%
\pgfsetdash{}{0pt}%
\pgfpathmoveto{\pgfqpoint{3.362505in}{3.556592in}}%
\pgfpathlineto{\pgfqpoint{3.372241in}{3.553264in}}%
\pgfpathlineto{\pgfqpoint{3.381986in}{3.551440in}}%
\pgfpathlineto{\pgfqpoint{3.416059in}{3.570821in}}%
\pgfpathlineto{\pgfqpoint{3.450131in}{3.588882in}}%
\pgfpathlineto{\pgfqpoint{3.440336in}{3.588000in}}%
\pgfpathlineto{\pgfqpoint{3.430552in}{3.588623in}}%
\pgfpathlineto{\pgfqpoint{3.396530in}{3.573234in}}%
\pgfpathlineto{\pgfqpoint{3.362505in}{3.556592in}}%
\pgfpathclose%
\pgfusepath{fill}%
\end{pgfscope}%
\begin{pgfscope}%
\pgfpathrectangle{\pgfqpoint{1.020000in}{0.880000in}}{\pgfqpoint{6.160000in}{6.160000in}}%
\pgfusepath{clip}%
\pgfsetbuttcap%
\pgfsetroundjoin%
\definecolor{currentfill}{rgb}{0.462354,0.599830,0.965857}%
\pgfsetfillcolor{currentfill}%
\pgfsetlinewidth{0.000000pt}%
\definecolor{currentstroke}{rgb}{0.000000,0.000000,0.000000}%
\pgfsetstrokecolor{currentstroke}%
\pgfsetdash{}{0pt}%
\pgfpathmoveto{\pgfqpoint{4.751868in}{3.064022in}}%
\pgfpathlineto{\pgfqpoint{4.762985in}{3.047586in}}%
\pgfpathlineto{\pgfqpoint{4.774119in}{3.030195in}}%
\pgfpathlineto{\pgfqpoint{4.807735in}{3.006715in}}%
\pgfpathlineto{\pgfqpoint{4.841324in}{2.984953in}}%
\pgfpathlineto{\pgfqpoint{4.830141in}{3.002150in}}%
\pgfpathlineto{\pgfqpoint{4.818973in}{3.018590in}}%
\pgfpathlineto{\pgfqpoint{4.785434in}{3.040448in}}%
\pgfpathlineto{\pgfqpoint{4.751868in}{3.064022in}}%
\pgfpathclose%
\pgfusepath{fill}%
\end{pgfscope}%
\begin{pgfscope}%
\pgfpathrectangle{\pgfqpoint{1.020000in}{0.880000in}}{\pgfqpoint{6.160000in}{6.160000in}}%
\pgfusepath{clip}%
\pgfsetbuttcap%
\pgfsetroundjoin%
\definecolor{currentfill}{rgb}{0.328604,0.439712,0.869587}%
\pgfsetfillcolor{currentfill}%
\pgfsetlinewidth{0.000000pt}%
\definecolor{currentstroke}{rgb}{0.000000,0.000000,0.000000}%
\pgfsetstrokecolor{currentstroke}%
\pgfsetdash{}{0pt}%
\pgfpathmoveto{\pgfqpoint{5.222017in}{2.787694in}}%
\pgfpathlineto{\pgfqpoint{5.233563in}{2.770579in}}%
\pgfpathlineto{\pgfqpoint{5.245129in}{2.753355in}}%
\pgfpathlineto{\pgfqpoint{5.278669in}{2.748565in}}%
\pgfpathlineto{\pgfqpoint{5.312192in}{2.744649in}}%
\pgfpathlineto{\pgfqpoint{5.300568in}{2.761426in}}%
\pgfpathlineto{\pgfqpoint{5.288966in}{2.778140in}}%
\pgfpathlineto{\pgfqpoint{5.255500in}{2.782458in}}%
\pgfpathlineto{\pgfqpoint{5.222017in}{2.787694in}}%
\pgfpathclose%
\pgfusepath{fill}%
\end{pgfscope}%
\begin{pgfscope}%
\pgfpathrectangle{\pgfqpoint{1.020000in}{0.880000in}}{\pgfqpoint{6.160000in}{6.160000in}}%
\pgfusepath{clip}%
\pgfsetbuttcap%
\pgfsetroundjoin%
\definecolor{currentfill}{rgb}{0.271104,0.360011,0.807095}%
\pgfsetfillcolor{currentfill}%
\pgfsetlinewidth{0.000000pt}%
\definecolor{currentstroke}{rgb}{0.000000,0.000000,0.000000}%
\pgfsetstrokecolor{currentstroke}%
\pgfsetdash{}{0pt}%
\pgfpathmoveto{\pgfqpoint{6.052561in}{2.667828in}}%
\pgfpathlineto{\pgfqpoint{6.064917in}{2.653227in}}%
\pgfpathlineto{\pgfqpoint{6.077298in}{2.638677in}}%
\pgfpathlineto{\pgfqpoint{6.110658in}{2.641322in}}%
\pgfpathlineto{\pgfqpoint{6.143998in}{2.643993in}}%
\pgfpathlineto{\pgfqpoint{6.131561in}{2.658462in}}%
\pgfpathlineto{\pgfqpoint{6.119149in}{2.672980in}}%
\pgfpathlineto{\pgfqpoint{6.085865in}{2.670390in}}%
\pgfpathlineto{\pgfqpoint{6.052561in}{2.667828in}}%
\pgfpathclose%
\pgfusepath{fill}%
\end{pgfscope}%
\begin{pgfscope}%
\pgfpathrectangle{\pgfqpoint{1.020000in}{0.880000in}}{\pgfqpoint{6.160000in}{6.160000in}}%
\pgfusepath{clip}%
\pgfsetbuttcap%
\pgfsetroundjoin%
\definecolor{currentfill}{rgb}{0.280550,0.373423,0.818011}%
\pgfsetfillcolor{currentfill}%
\pgfsetlinewidth{0.000000pt}%
\definecolor{currentstroke}{rgb}{0.000000,0.000000,0.000000}%
\pgfsetstrokecolor{currentstroke}%
\pgfsetdash{}{0pt}%
\pgfpathmoveto{\pgfqpoint{5.828016in}{2.683382in}}%
\pgfpathlineto{\pgfqpoint{5.840153in}{2.668373in}}%
\pgfpathlineto{\pgfqpoint{5.852315in}{2.653417in}}%
\pgfpathlineto{\pgfqpoint{5.885739in}{2.655660in}}%
\pgfpathlineto{\pgfqpoint{5.919144in}{2.657981in}}%
\pgfpathlineto{\pgfqpoint{5.906925in}{2.672823in}}%
\pgfpathlineto{\pgfqpoint{5.894731in}{2.687718in}}%
\pgfpathlineto{\pgfqpoint{5.861383in}{2.685507in}}%
\pgfpathlineto{\pgfqpoint{5.828016in}{2.683382in}}%
\pgfpathclose%
\pgfusepath{fill}%
\end{pgfscope}%
\begin{pgfscope}%
\pgfpathrectangle{\pgfqpoint{1.020000in}{0.880000in}}{\pgfqpoint{6.160000in}{6.160000in}}%
\pgfusepath{clip}%
\pgfsetbuttcap%
\pgfsetroundjoin%
\definecolor{currentfill}{rgb}{0.289996,0.386836,0.828926}%
\pgfsetfillcolor{currentfill}%
\pgfsetlinewidth{0.000000pt}%
\definecolor{currentstroke}{rgb}{0.000000,0.000000,0.000000}%
\pgfsetstrokecolor{currentstroke}%
\pgfsetdash{}{0pt}%
\pgfpathmoveto{\pgfqpoint{5.603542in}{2.704628in}}%
\pgfpathlineto{\pgfqpoint{5.615462in}{2.689048in}}%
\pgfpathlineto{\pgfqpoint{5.627404in}{2.673510in}}%
\pgfpathlineto{\pgfqpoint{5.660887in}{2.674691in}}%
\pgfpathlineto{\pgfqpoint{5.694351in}{2.676096in}}%
\pgfpathlineto{\pgfqpoint{5.682351in}{2.691444in}}%
\pgfpathlineto{\pgfqpoint{5.670374in}{2.706839in}}%
\pgfpathlineto{\pgfqpoint{5.636967in}{2.705612in}}%
\pgfpathlineto{\pgfqpoint{5.603542in}{2.704628in}}%
\pgfpathclose%
\pgfusepath{fill}%
\end{pgfscope}%
\begin{pgfscope}%
\pgfpathrectangle{\pgfqpoint{1.020000in}{0.880000in}}{\pgfqpoint{6.160000in}{6.160000in}}%
\pgfusepath{clip}%
\pgfsetbuttcap%
\pgfsetroundjoin%
\definecolor{currentfill}{rgb}{0.640828,0.760752,0.997846}%
\pgfsetfillcolor{currentfill}%
\pgfsetlinewidth{0.000000pt}%
\definecolor{currentstroke}{rgb}{0.000000,0.000000,0.000000}%
\pgfsetstrokecolor{currentstroke}%
\pgfsetdash{}{0pt}%
\pgfpathmoveto{\pgfqpoint{2.866360in}{3.334134in}}%
\pgfpathlineto{\pgfqpoint{2.875809in}{3.312082in}}%
\pgfpathlineto{\pgfqpoint{2.885259in}{3.291329in}}%
\pgfpathlineto{\pgfqpoint{2.919449in}{3.307681in}}%
\pgfpathlineto{\pgfqpoint{2.953617in}{3.324882in}}%
\pgfpathlineto{\pgfqpoint{2.944127in}{3.344140in}}%
\pgfpathlineto{\pgfqpoint{2.934637in}{3.364856in}}%
\pgfpathlineto{\pgfqpoint{2.900509in}{3.349141in}}%
\pgfpathlineto{\pgfqpoint{2.866360in}{3.334134in}}%
\pgfpathclose%
\pgfusepath{fill}%
\end{pgfscope}%
\begin{pgfscope}%
\pgfpathrectangle{\pgfqpoint{1.020000in}{0.880000in}}{\pgfqpoint{6.160000in}{6.160000in}}%
\pgfusepath{clip}%
\pgfsetbuttcap%
\pgfsetroundjoin%
\definecolor{currentfill}{rgb}{0.748682,0.827679,0.963334}%
\pgfsetfillcolor{currentfill}%
\pgfsetlinewidth{0.000000pt}%
\definecolor{currentstroke}{rgb}{0.000000,0.000000,0.000000}%
\pgfsetstrokecolor{currentstroke}%
\pgfsetdash{}{0pt}%
\pgfpathmoveto{\pgfqpoint{3.294444in}{3.520279in}}%
\pgfpathlineto{\pgfqpoint{3.304135in}{3.514178in}}%
\pgfpathlineto{\pgfqpoint{3.313833in}{3.509557in}}%
\pgfpathlineto{\pgfqpoint{3.347910in}{3.530948in}}%
\pgfpathlineto{\pgfqpoint{3.381986in}{3.551440in}}%
\pgfpathlineto{\pgfqpoint{3.372241in}{3.553264in}}%
\pgfpathlineto{\pgfqpoint{3.362505in}{3.556592in}}%
\pgfpathlineto{\pgfqpoint{3.328477in}{3.538878in}}%
\pgfpathlineto{\pgfqpoint{3.294444in}{3.520279in}}%
\pgfpathclose%
\pgfusepath{fill}%
\end{pgfscope}%
\begin{pgfscope}%
\pgfpathrectangle{\pgfqpoint{1.020000in}{0.880000in}}{\pgfqpoint{6.160000in}{6.160000in}}%
\pgfusepath{clip}%
\pgfsetbuttcap%
\pgfsetroundjoin%
\definecolor{currentfill}{rgb}{0.672538,0.782861,0.991982}%
\pgfsetfillcolor{currentfill}%
\pgfsetlinewidth{0.000000pt}%
\definecolor{currentstroke}{rgb}{0.000000,0.000000,0.000000}%
\pgfsetstrokecolor{currentstroke}%
\pgfsetdash{}{0pt}%
\pgfpathmoveto{\pgfqpoint{4.349404in}{3.427932in}}%
\pgfpathlineto{\pgfqpoint{4.360174in}{3.423776in}}%
\pgfpathlineto{\pgfqpoint{4.370963in}{3.417925in}}%
\pgfpathlineto{\pgfqpoint{4.404801in}{3.384693in}}%
\pgfpathlineto{\pgfqpoint{4.438597in}{3.351661in}}%
\pgfpathlineto{\pgfqpoint{4.427756in}{3.358880in}}%
\pgfpathlineto{\pgfqpoint{4.416933in}{3.364617in}}%
\pgfpathlineto{\pgfqpoint{4.383190in}{3.396175in}}%
\pgfpathlineto{\pgfqpoint{4.349404in}{3.427932in}}%
\pgfpathclose%
\pgfusepath{fill}%
\end{pgfscope}%
\begin{pgfscope}%
\pgfpathrectangle{\pgfqpoint{1.020000in}{0.880000in}}{\pgfqpoint{6.160000in}{6.160000in}}%
\pgfusepath{clip}%
\pgfsetbuttcap%
\pgfsetroundjoin%
\definecolor{currentfill}{rgb}{0.608547,0.735725,0.999354}%
\pgfsetfillcolor{currentfill}%
\pgfsetlinewidth{0.000000pt}%
\definecolor{currentstroke}{rgb}{0.000000,0.000000,0.000000}%
\pgfsetstrokecolor{currentstroke}%
\pgfsetdash{}{0pt}%
\pgfpathmoveto{\pgfqpoint{2.573698in}{3.296746in}}%
\pgfpathlineto{\pgfqpoint{2.582963in}{3.270116in}}%
\pgfpathlineto{\pgfqpoint{2.592230in}{3.244341in}}%
\pgfpathlineto{\pgfqpoint{2.626596in}{3.252491in}}%
\pgfpathlineto{\pgfqpoint{2.660933in}{3.261512in}}%
\pgfpathlineto{\pgfqpoint{2.651617in}{3.286991in}}%
\pgfpathlineto{\pgfqpoint{2.642301in}{3.313486in}}%
\pgfpathlineto{\pgfqpoint{2.608014in}{3.304711in}}%
\pgfpathlineto{\pgfqpoint{2.573698in}{3.296746in}}%
\pgfpathclose%
\pgfusepath{fill}%
\end{pgfscope}%
\begin{pgfscope}%
\pgfpathrectangle{\pgfqpoint{1.020000in}{0.880000in}}{\pgfqpoint{6.160000in}{6.160000in}}%
\pgfusepath{clip}%
\pgfsetbuttcap%
\pgfsetroundjoin%
\definecolor{currentfill}{rgb}{0.266381,0.353304,0.801637}%
\pgfsetfillcolor{currentfill}%
\pgfsetlinewidth{0.000000pt}%
\definecolor{currentstroke}{rgb}{0.000000,0.000000,0.000000}%
\pgfsetstrokecolor{currentstroke}%
\pgfsetdash{}{0pt}%
\pgfpathmoveto{\pgfqpoint{6.277154in}{2.654858in}}%
\pgfpathlineto{\pgfqpoint{6.289728in}{2.640585in}}%
\pgfpathlineto{\pgfqpoint{6.302328in}{2.626358in}}%
\pgfpathlineto{\pgfqpoint{6.335622in}{2.629177in}}%
\pgfpathlineto{\pgfqpoint{6.322994in}{2.643370in}}%
\pgfpathlineto{\pgfqpoint{6.310391in}{2.657609in}}%
\pgfpathlineto{\pgfqpoint{6.277154in}{2.654858in}}%
\pgfpathclose%
\pgfusepath{fill}%
\end{pgfscope}%
\begin{pgfscope}%
\pgfpathrectangle{\pgfqpoint{1.020000in}{0.880000in}}{\pgfqpoint{6.160000in}{6.160000in}}%
\pgfusepath{clip}%
\pgfsetbuttcap%
\pgfsetroundjoin%
\definecolor{currentfill}{rgb}{0.813693,0.854282,0.918480}%
\pgfsetfillcolor{currentfill}%
\pgfsetlinewidth{0.000000pt}%
\definecolor{currentstroke}{rgb}{0.000000,0.000000,0.000000}%
\pgfsetstrokecolor{currentstroke}%
\pgfsetdash{}{0pt}%
\pgfpathmoveto{\pgfqpoint{3.654509in}{3.659295in}}%
\pgfpathlineto{\pgfqpoint{3.664497in}{3.667673in}}%
\pgfpathlineto{\pgfqpoint{3.674505in}{3.676628in}}%
\pgfpathlineto{\pgfqpoint{3.708616in}{3.682063in}}%
\pgfpathlineto{\pgfqpoint{3.742719in}{3.684822in}}%
\pgfpathlineto{\pgfqpoint{3.732643in}{3.674693in}}%
\pgfpathlineto{\pgfqpoint{3.722589in}{3.665067in}}%
\pgfpathlineto{\pgfqpoint{3.688554in}{3.663389in}}%
\pgfpathlineto{\pgfqpoint{3.654509in}{3.659295in}}%
\pgfpathclose%
\pgfusepath{fill}%
\end{pgfscope}%
\begin{pgfscope}%
\pgfpathrectangle{\pgfqpoint{1.020000in}{0.880000in}}{\pgfqpoint{6.160000in}{6.160000in}}%
\pgfusepath{clip}%
\pgfsetbuttcap%
\pgfsetroundjoin%
\definecolor{currentfill}{rgb}{0.304174,0.406945,0.845263}%
\pgfsetfillcolor{currentfill}%
\pgfsetlinewidth{0.000000pt}%
\definecolor{currentstroke}{rgb}{0.000000,0.000000,0.000000}%
\pgfsetstrokecolor{currentstroke}%
\pgfsetdash{}{0pt}%
\pgfpathmoveto{\pgfqpoint{5.379189in}{2.739118in}}%
\pgfpathlineto{\pgfqpoint{5.390891in}{2.722683in}}%
\pgfpathlineto{\pgfqpoint{5.402615in}{2.706238in}}%
\pgfpathlineto{\pgfqpoint{5.436146in}{2.704818in}}%
\pgfpathlineto{\pgfqpoint{5.469660in}{2.703934in}}%
\pgfpathlineto{\pgfqpoint{5.457878in}{2.720042in}}%
\pgfpathlineto{\pgfqpoint{5.446119in}{2.736160in}}%
\pgfpathlineto{\pgfqpoint{5.412662in}{2.737352in}}%
\pgfpathlineto{\pgfqpoint{5.379189in}{2.739118in}}%
\pgfpathclose%
\pgfusepath{fill}%
\end{pgfscope}%
\begin{pgfscope}%
\pgfpathrectangle{\pgfqpoint{1.020000in}{0.880000in}}{\pgfqpoint{6.160000in}{6.160000in}}%
\pgfusepath{clip}%
\pgfsetbuttcap%
\pgfsetroundjoin%
\definecolor{currentfill}{rgb}{0.543440,0.680003,0.993051}%
\pgfsetfillcolor{currentfill}%
\pgfsetlinewidth{0.000000pt}%
\definecolor{currentstroke}{rgb}{0.000000,0.000000,0.000000}%
\pgfsetstrokecolor{currentstroke}%
\pgfsetdash{}{0pt}%
\pgfpathmoveto{\pgfqpoint{4.595298in}{3.203112in}}%
\pgfpathlineto{\pgfqpoint{4.606288in}{3.189518in}}%
\pgfpathlineto{\pgfqpoint{4.617293in}{3.174544in}}%
\pgfpathlineto{\pgfqpoint{4.650987in}{3.144641in}}%
\pgfpathlineto{\pgfqpoint{4.684646in}{3.116185in}}%
\pgfpathlineto{\pgfqpoint{4.673592in}{3.131436in}}%
\pgfpathlineto{\pgfqpoint{4.662555in}{3.145547in}}%
\pgfpathlineto{\pgfqpoint{4.628944in}{3.173617in}}%
\pgfpathlineto{\pgfqpoint{4.595298in}{3.203112in}}%
\pgfpathclose%
\pgfusepath{fill}%
\end{pgfscope}%
\begin{pgfscope}%
\pgfpathrectangle{\pgfqpoint{1.020000in}{0.880000in}}{\pgfqpoint{6.160000in}{6.160000in}}%
\pgfusepath{clip}%
\pgfsetbuttcap%
\pgfsetroundjoin%
\definecolor{currentfill}{rgb}{0.813693,0.854282,0.918480}%
\pgfsetfillcolor{currentfill}%
\pgfsetlinewidth{0.000000pt}%
\definecolor{currentstroke}{rgb}{0.000000,0.000000,0.000000}%
\pgfsetstrokecolor{currentstroke}%
\pgfsetdash{}{0pt}%
\pgfpathmoveto{\pgfqpoint{3.878996in}{3.668906in}}%
\pgfpathlineto{\pgfqpoint{3.889237in}{3.679703in}}%
\pgfpathlineto{\pgfqpoint{3.899503in}{3.690064in}}%
\pgfpathlineto{\pgfqpoint{3.933597in}{3.679179in}}%
\pgfpathlineto{\pgfqpoint{3.967667in}{3.665647in}}%
\pgfpathlineto{\pgfqpoint{3.957329in}{3.655843in}}%
\pgfpathlineto{\pgfqpoint{3.947016in}{3.645571in}}%
\pgfpathlineto{\pgfqpoint{3.913018in}{3.658433in}}%
\pgfpathlineto{\pgfqpoint{3.878996in}{3.668906in}}%
\pgfpathclose%
\pgfusepath{fill}%
\end{pgfscope}%
\begin{pgfscope}%
\pgfpathrectangle{\pgfqpoint{1.020000in}{0.880000in}}{\pgfqpoint{6.160000in}{6.160000in}}%
\pgfusepath{clip}%
\pgfsetbuttcap%
\pgfsetroundjoin%
\definecolor{currentfill}{rgb}{0.728970,0.817464,0.973188}%
\pgfsetfillcolor{currentfill}%
\pgfsetlinewidth{0.000000pt}%
\definecolor{currentstroke}{rgb}{0.000000,0.000000,0.000000}%
\pgfsetstrokecolor{currentstroke}%
\pgfsetdash{}{0pt}%
\pgfpathmoveto{\pgfqpoint{3.226363in}{3.481183in}}%
\pgfpathlineto{\pgfqpoint{3.236011in}{3.472342in}}%
\pgfpathlineto{\pgfqpoint{3.245665in}{3.464936in}}%
\pgfpathlineto{\pgfqpoint{3.279751in}{3.487482in}}%
\pgfpathlineto{\pgfqpoint{3.313833in}{3.509557in}}%
\pgfpathlineto{\pgfqpoint{3.304135in}{3.514178in}}%
\pgfpathlineto{\pgfqpoint{3.294444in}{3.520279in}}%
\pgfpathlineto{\pgfqpoint{3.260406in}{3.500984in}}%
\pgfpathlineto{\pgfqpoint{3.226363in}{3.481183in}}%
\pgfpathclose%
\pgfusepath{fill}%
\end{pgfscope}%
\begin{pgfscope}%
\pgfpathrectangle{\pgfqpoint{1.020000in}{0.880000in}}{\pgfqpoint{6.160000in}{6.160000in}}%
\pgfusepath{clip}%
\pgfsetbuttcap%
\pgfsetroundjoin%
\definecolor{currentfill}{rgb}{0.373552,0.497499,0.909467}%
\pgfsetfillcolor{currentfill}%
\pgfsetlinewidth{0.000000pt}%
\definecolor{currentstroke}{rgb}{0.000000,0.000000,0.000000}%
\pgfsetstrokecolor{currentstroke}%
\pgfsetdash{}{0pt}%
\pgfpathmoveto{\pgfqpoint{4.998071in}{2.879884in}}%
\pgfpathlineto{\pgfqpoint{5.009411in}{2.861744in}}%
\pgfpathlineto{\pgfqpoint{5.020768in}{2.843207in}}%
\pgfpathlineto{\pgfqpoint{5.054354in}{2.830665in}}%
\pgfpathlineto{\pgfqpoint{5.087921in}{2.819545in}}%
\pgfpathlineto{\pgfqpoint{5.076508in}{2.837472in}}%
\pgfpathlineto{\pgfqpoint{5.065114in}{2.855111in}}%
\pgfpathlineto{\pgfqpoint{5.031602in}{2.866762in}}%
\pgfpathlineto{\pgfqpoint{4.998071in}{2.879884in}}%
\pgfpathclose%
\pgfusepath{fill}%
\end{pgfscope}%
\begin{pgfscope}%
\pgfpathrectangle{\pgfqpoint{1.020000in}{0.880000in}}{\pgfqpoint{6.160000in}{6.160000in}}%
\pgfusepath{clip}%
\pgfsetbuttcap%
\pgfsetroundjoin%
\definecolor{currentfill}{rgb}{0.624703,0.748318,0.998719}%
\pgfsetfillcolor{currentfill}%
\pgfsetlinewidth{0.000000pt}%
\definecolor{currentstroke}{rgb}{0.000000,0.000000,0.000000}%
\pgfsetstrokecolor{currentstroke}%
\pgfsetdash{}{0pt}%
\pgfpathmoveto{\pgfqpoint{2.797990in}{3.306476in}}%
\pgfpathlineto{\pgfqpoint{2.807397in}{3.283360in}}%
\pgfpathlineto{\pgfqpoint{2.816805in}{3.261389in}}%
\pgfpathlineto{\pgfqpoint{2.851045in}{3.275885in}}%
\pgfpathlineto{\pgfqpoint{2.885259in}{3.291329in}}%
\pgfpathlineto{\pgfqpoint{2.875809in}{3.312082in}}%
\pgfpathlineto{\pgfqpoint{2.866360in}{3.334134in}}%
\pgfpathlineto{\pgfqpoint{2.832187in}{3.319896in}}%
\pgfpathlineto{\pgfqpoint{2.797990in}{3.306476in}}%
\pgfpathclose%
\pgfusepath{fill}%
\end{pgfscope}%
\begin{pgfscope}%
\pgfpathrectangle{\pgfqpoint{1.020000in}{0.880000in}}{\pgfqpoint{6.160000in}{6.160000in}}%
\pgfusepath{clip}%
\pgfsetbuttcap%
\pgfsetroundjoin%
\definecolor{currentfill}{rgb}{0.743754,0.825125,0.965798}%
\pgfsetfillcolor{currentfill}%
\pgfsetlinewidth{0.000000pt}%
\definecolor{currentstroke}{rgb}{0.000000,0.000000,0.000000}%
\pgfsetstrokecolor{currentstroke}%
\pgfsetdash{}{0pt}%
\pgfpathmoveto{\pgfqpoint{4.192587in}{3.548297in}}%
\pgfpathlineto{\pgfqpoint{4.203199in}{3.550639in}}%
\pgfpathlineto{\pgfqpoint{4.213834in}{3.551363in}}%
\pgfpathlineto{\pgfqpoint{4.247791in}{3.521636in}}%
\pgfpathlineto{\pgfqpoint{4.281705in}{3.490953in}}%
\pgfpathlineto{\pgfqpoint{4.271011in}{3.491858in}}%
\pgfpathlineto{\pgfqpoint{4.260339in}{3.491294in}}%
\pgfpathlineto{\pgfqpoint{4.226484in}{3.520236in}}%
\pgfpathlineto{\pgfqpoint{4.192587in}{3.548297in}}%
\pgfpathclose%
\pgfusepath{fill}%
\end{pgfscope}%
\begin{pgfscope}%
\pgfpathrectangle{\pgfqpoint{1.020000in}{0.880000in}}{\pgfqpoint{6.160000in}{6.160000in}}%
\pgfusepath{clip}%
\pgfsetbuttcap%
\pgfsetroundjoin%
\definecolor{currentfill}{rgb}{0.425199,0.559058,0.946061}%
\pgfsetfillcolor{currentfill}%
\pgfsetlinewidth{0.000000pt}%
\definecolor{currentstroke}{rgb}{0.000000,0.000000,0.000000}%
\pgfsetstrokecolor{currentstroke}%
\pgfsetdash{}{0pt}%
\pgfpathmoveto{\pgfqpoint{4.841324in}{2.984953in}}%
\pgfpathlineto{\pgfqpoint{4.852524in}{2.967002in}}%
\pgfpathlineto{\pgfqpoint{4.863740in}{2.948311in}}%
\pgfpathlineto{\pgfqpoint{4.897356in}{2.928719in}}%
\pgfpathlineto{\pgfqpoint{4.930949in}{2.910808in}}%
\pgfpathlineto{\pgfqpoint{4.919681in}{2.928981in}}%
\pgfpathlineto{\pgfqpoint{4.908429in}{2.946582in}}%
\pgfpathlineto{\pgfqpoint{4.874889in}{2.964913in}}%
\pgfpathlineto{\pgfqpoint{4.841324in}{2.984953in}}%
\pgfpathclose%
\pgfusepath{fill}%
\end{pgfscope}%
\begin{pgfscope}%
\pgfpathrectangle{\pgfqpoint{1.020000in}{0.880000in}}{\pgfqpoint{6.160000in}{6.160000in}}%
\pgfusepath{clip}%
\pgfsetbuttcap%
\pgfsetroundjoin%
\definecolor{currentfill}{rgb}{0.338377,0.452819,0.879317}%
\pgfsetfillcolor{currentfill}%
\pgfsetlinewidth{0.000000pt}%
\definecolor{currentstroke}{rgb}{0.000000,0.000000,0.000000}%
\pgfsetstrokecolor{currentstroke}%
\pgfsetdash{}{0pt}%
\pgfpathmoveto{\pgfqpoint{5.155003in}{2.801263in}}%
\pgfpathlineto{\pgfqpoint{5.166492in}{2.783659in}}%
\pgfpathlineto{\pgfqpoint{5.178000in}{2.765886in}}%
\pgfpathlineto{\pgfqpoint{5.211573in}{2.759101in}}%
\pgfpathlineto{\pgfqpoint{5.245129in}{2.753355in}}%
\pgfpathlineto{\pgfqpoint{5.233563in}{2.770579in}}%
\pgfpathlineto{\pgfqpoint{5.222017in}{2.787694in}}%
\pgfpathlineto{\pgfqpoint{5.188519in}{2.793932in}}%
\pgfpathlineto{\pgfqpoint{5.155003in}{2.801263in}}%
\pgfpathclose%
\pgfusepath{fill}%
\end{pgfscope}%
\begin{pgfscope}%
\pgfpathrectangle{\pgfqpoint{1.020000in}{0.880000in}}{\pgfqpoint{6.160000in}{6.160000in}}%
\pgfusepath{clip}%
\pgfsetbuttcap%
\pgfsetroundjoin%
\definecolor{currentfill}{rgb}{0.275827,0.366717,0.812553}%
\pgfsetfillcolor{currentfill}%
\pgfsetlinewidth{0.000000pt}%
\definecolor{currentstroke}{rgb}{0.000000,0.000000,0.000000}%
\pgfsetstrokecolor{currentstroke}%
\pgfsetdash{}{0pt}%
\pgfpathmoveto{\pgfqpoint{5.985893in}{2.662810in}}%
\pgfpathlineto{\pgfqpoint{5.998192in}{2.648120in}}%
\pgfpathlineto{\pgfqpoint{6.010516in}{2.633481in}}%
\pgfpathlineto{\pgfqpoint{6.043917in}{2.636061in}}%
\pgfpathlineto{\pgfqpoint{6.077298in}{2.638677in}}%
\pgfpathlineto{\pgfqpoint{6.064917in}{2.653227in}}%
\pgfpathlineto{\pgfqpoint{6.052561in}{2.667828in}}%
\pgfpathlineto{\pgfqpoint{6.019237in}{2.665300in}}%
\pgfpathlineto{\pgfqpoint{5.985893in}{2.662810in}}%
\pgfpathclose%
\pgfusepath{fill}%
\end{pgfscope}%
\begin{pgfscope}%
\pgfpathrectangle{\pgfqpoint{1.020000in}{0.880000in}}{\pgfqpoint{6.160000in}{6.160000in}}%
\pgfusepath{clip}%
\pgfsetbuttcap%
\pgfsetroundjoin%
\definecolor{currentfill}{rgb}{0.791392,0.846750,0.936641}%
\pgfsetfillcolor{currentfill}%
\pgfsetlinewidth{0.000000pt}%
\definecolor{currentstroke}{rgb}{0.000000,0.000000,0.000000}%
\pgfsetstrokecolor{currentstroke}%
\pgfsetdash{}{0pt}%
\pgfpathmoveto{\pgfqpoint{4.035722in}{3.631335in}}%
\pgfpathlineto{\pgfqpoint{4.046155in}{3.639263in}}%
\pgfpathlineto{\pgfqpoint{4.056613in}{3.646005in}}%
\pgfpathlineto{\pgfqpoint{4.090662in}{3.624320in}}%
\pgfpathlineto{\pgfqpoint{4.124675in}{3.600624in}}%
\pgfpathlineto{\pgfqpoint{4.114150in}{3.595260in}}%
\pgfpathlineto{\pgfqpoint{4.103649in}{3.588763in}}%
\pgfpathlineto{\pgfqpoint{4.069703in}{3.610966in}}%
\pgfpathlineto{\pgfqpoint{4.035722in}{3.631335in}}%
\pgfpathclose%
\pgfusepath{fill}%
\end{pgfscope}%
\begin{pgfscope}%
\pgfpathrectangle{\pgfqpoint{1.020000in}{0.880000in}}{\pgfqpoint{6.160000in}{6.160000in}}%
\pgfusepath{clip}%
\pgfsetbuttcap%
\pgfsetroundjoin%
\definecolor{currentfill}{rgb}{0.280550,0.373423,0.818011}%
\pgfsetfillcolor{currentfill}%
\pgfsetlinewidth{0.000000pt}%
\definecolor{currentstroke}{rgb}{0.000000,0.000000,0.000000}%
\pgfsetstrokecolor{currentstroke}%
\pgfsetdash{}{0pt}%
\pgfpathmoveto{\pgfqpoint{5.761222in}{2.679456in}}%
\pgfpathlineto{\pgfqpoint{5.773303in}{2.664314in}}%
\pgfpathlineto{\pgfqpoint{5.785408in}{2.649225in}}%
\pgfpathlineto{\pgfqpoint{5.818871in}{2.651266in}}%
\pgfpathlineto{\pgfqpoint{5.852315in}{2.653417in}}%
\pgfpathlineto{\pgfqpoint{5.840153in}{2.668373in}}%
\pgfpathlineto{\pgfqpoint{5.828016in}{2.683382in}}%
\pgfpathlineto{\pgfqpoint{5.794629in}{2.681359in}}%
\pgfpathlineto{\pgfqpoint{5.761222in}{2.679456in}}%
\pgfpathclose%
\pgfusepath{fill}%
\end{pgfscope}%
\begin{pgfscope}%
\pgfpathrectangle{\pgfqpoint{1.020000in}{0.880000in}}{\pgfqpoint{6.160000in}{6.160000in}}%
\pgfusepath{clip}%
\pgfsetbuttcap%
\pgfsetroundjoin%
\definecolor{currentfill}{rgb}{0.266381,0.353304,0.801637}%
\pgfsetfillcolor{currentfill}%
\pgfsetlinewidth{0.000000pt}%
\definecolor{currentstroke}{rgb}{0.000000,0.000000,0.000000}%
\pgfsetstrokecolor{currentstroke}%
\pgfsetdash{}{0pt}%
\pgfpathmoveto{\pgfqpoint{6.210617in}{2.649394in}}%
\pgfpathlineto{\pgfqpoint{6.223135in}{2.635049in}}%
\pgfpathlineto{\pgfqpoint{6.235679in}{2.620752in}}%
\pgfpathlineto{\pgfqpoint{6.269014in}{2.623549in}}%
\pgfpathlineto{\pgfqpoint{6.302328in}{2.626358in}}%
\pgfpathlineto{\pgfqpoint{6.289728in}{2.640585in}}%
\pgfpathlineto{\pgfqpoint{6.277154in}{2.654858in}}%
\pgfpathlineto{\pgfqpoint{6.243895in}{2.652120in}}%
\pgfpathlineto{\pgfqpoint{6.210617in}{2.649394in}}%
\pgfpathclose%
\pgfusepath{fill}%
\end{pgfscope}%
\begin{pgfscope}%
\pgfpathrectangle{\pgfqpoint{1.020000in}{0.880000in}}{\pgfqpoint{6.160000in}{6.160000in}}%
\pgfusepath{clip}%
\pgfsetbuttcap%
\pgfsetroundjoin%
\definecolor{currentfill}{rgb}{0.708720,0.805721,0.981117}%
\pgfsetfillcolor{currentfill}%
\pgfsetlinewidth{0.000000pt}%
\definecolor{currentstroke}{rgb}{0.000000,0.000000,0.000000}%
\pgfsetstrokecolor{currentstroke}%
\pgfsetdash{}{0pt}%
\pgfpathmoveto{\pgfqpoint{3.158252in}{3.440793in}}%
\pgfpathlineto{\pgfqpoint{3.167859in}{3.429334in}}%
\pgfpathlineto{\pgfqpoint{3.177473in}{3.419246in}}%
\pgfpathlineto{\pgfqpoint{3.211573in}{3.442124in}}%
\pgfpathlineto{\pgfqpoint{3.245665in}{3.464936in}}%
\pgfpathlineto{\pgfqpoint{3.236011in}{3.472342in}}%
\pgfpathlineto{\pgfqpoint{3.226363in}{3.481183in}}%
\pgfpathlineto{\pgfqpoint{3.192311in}{3.461059in}}%
\pgfpathlineto{\pgfqpoint{3.158252in}{3.440793in}}%
\pgfpathclose%
\pgfusepath{fill}%
\end{pgfscope}%
\begin{pgfscope}%
\pgfpathrectangle{\pgfqpoint{1.020000in}{0.880000in}}{\pgfqpoint{6.160000in}{6.160000in}}%
\pgfusepath{clip}%
\pgfsetbuttcap%
\pgfsetroundjoin%
\definecolor{currentfill}{rgb}{0.294718,0.393542,0.834384}%
\pgfsetfillcolor{currentfill}%
\pgfsetlinewidth{0.000000pt}%
\definecolor{currentstroke}{rgb}{0.000000,0.000000,0.000000}%
\pgfsetstrokecolor{currentstroke}%
\pgfsetdash{}{0pt}%
\pgfpathmoveto{\pgfqpoint{5.536637in}{2.703541in}}%
\pgfpathlineto{\pgfqpoint{5.548499in}{2.687733in}}%
\pgfpathlineto{\pgfqpoint{5.560384in}{2.671960in}}%
\pgfpathlineto{\pgfqpoint{5.593903in}{2.672586in}}%
\pgfpathlineto{\pgfqpoint{5.627404in}{2.673510in}}%
\pgfpathlineto{\pgfqpoint{5.615462in}{2.689048in}}%
\pgfpathlineto{\pgfqpoint{5.603542in}{2.704628in}}%
\pgfpathlineto{\pgfqpoint{5.570098in}{2.703923in}}%
\pgfpathlineto{\pgfqpoint{5.536637in}{2.703541in}}%
\pgfpathclose%
\pgfusepath{fill}%
\end{pgfscope}%
\begin{pgfscope}%
\pgfpathrectangle{\pgfqpoint{1.020000in}{0.880000in}}{\pgfqpoint{6.160000in}{6.160000in}}%
\pgfusepath{clip}%
\pgfsetbuttcap%
\pgfsetroundjoin%
\definecolor{currentfill}{rgb}{0.813693,0.854282,0.918480}%
\pgfsetfillcolor{currentfill}%
\pgfsetlinewidth{0.000000pt}%
\definecolor{currentstroke}{rgb}{0.000000,0.000000,0.000000}%
\pgfsetstrokecolor{currentstroke}%
\pgfsetdash{}{0pt}%
\pgfpathmoveto{\pgfqpoint{3.586399in}{3.644103in}}%
\pgfpathlineto{\pgfqpoint{3.596323in}{3.650744in}}%
\pgfpathlineto{\pgfqpoint{3.606266in}{3.658031in}}%
\pgfpathlineto{\pgfqpoint{3.640387in}{3.668584in}}%
\pgfpathlineto{\pgfqpoint{3.674505in}{3.676628in}}%
\pgfpathlineto{\pgfqpoint{3.664497in}{3.667673in}}%
\pgfpathlineto{\pgfqpoint{3.654509in}{3.659295in}}%
\pgfpathlineto{\pgfqpoint{3.620457in}{3.652839in}}%
\pgfpathlineto{\pgfqpoint{3.586399in}{3.644103in}}%
\pgfpathclose%
\pgfusepath{fill}%
\end{pgfscope}%
\begin{pgfscope}%
\pgfpathrectangle{\pgfqpoint{1.020000in}{0.880000in}}{\pgfqpoint{6.160000in}{6.160000in}}%
\pgfusepath{clip}%
\pgfsetbuttcap%
\pgfsetroundjoin%
\definecolor{currentfill}{rgb}{0.630089,0.752516,0.998508}%
\pgfsetfillcolor{currentfill}%
\pgfsetlinewidth{0.000000pt}%
\definecolor{currentstroke}{rgb}{0.000000,0.000000,0.000000}%
\pgfsetstrokecolor{currentstroke}%
\pgfsetdash{}{0pt}%
\pgfpathmoveto{\pgfqpoint{4.438597in}{3.351661in}}%
\pgfpathlineto{\pgfqpoint{4.449455in}{3.342818in}}%
\pgfpathlineto{\pgfqpoint{4.460331in}{3.332231in}}%
\pgfpathlineto{\pgfqpoint{4.494134in}{3.298664in}}%
\pgfpathlineto{\pgfqpoint{4.527895in}{3.265813in}}%
\pgfpathlineto{\pgfqpoint{4.516970in}{3.277262in}}%
\pgfpathlineto{\pgfqpoint{4.506061in}{3.287215in}}%
\pgfpathlineto{\pgfqpoint{4.472349in}{3.319089in}}%
\pgfpathlineto{\pgfqpoint{4.438597in}{3.351661in}}%
\pgfpathclose%
\pgfusepath{fill}%
\end{pgfscope}%
\begin{pgfscope}%
\pgfpathrectangle{\pgfqpoint{1.020000in}{0.880000in}}{\pgfqpoint{6.160000in}{6.160000in}}%
\pgfusepath{clip}%
\pgfsetbuttcap%
\pgfsetroundjoin%
\definecolor{currentfill}{rgb}{0.608547,0.735725,0.999354}%
\pgfsetfillcolor{currentfill}%
\pgfsetlinewidth{0.000000pt}%
\definecolor{currentstroke}{rgb}{0.000000,0.000000,0.000000}%
\pgfsetstrokecolor{currentstroke}%
\pgfsetdash{}{0pt}%
\pgfpathmoveto{\pgfqpoint{2.729517in}{3.282219in}}%
\pgfpathlineto{\pgfqpoint{2.738880in}{3.258286in}}%
\pgfpathlineto{\pgfqpoint{2.748245in}{3.235346in}}%
\pgfpathlineto{\pgfqpoint{2.782539in}{3.247871in}}%
\pgfpathlineto{\pgfqpoint{2.816805in}{3.261389in}}%
\pgfpathlineto{\pgfqpoint{2.807397in}{3.283360in}}%
\pgfpathlineto{\pgfqpoint{2.797990in}{3.306476in}}%
\pgfpathlineto{\pgfqpoint{2.763767in}{3.293909in}}%
\pgfpathlineto{\pgfqpoint{2.729517in}{3.282219in}}%
\pgfpathclose%
\pgfusepath{fill}%
\end{pgfscope}%
\begin{pgfscope}%
\pgfpathrectangle{\pgfqpoint{1.020000in}{0.880000in}}{\pgfqpoint{6.160000in}{6.160000in}}%
\pgfusepath{clip}%
\pgfsetbuttcap%
\pgfsetroundjoin%
\definecolor{currentfill}{rgb}{0.313946,0.420052,0.854993}%
\pgfsetfillcolor{currentfill}%
\pgfsetlinewidth{0.000000pt}%
\definecolor{currentstroke}{rgb}{0.000000,0.000000,0.000000}%
\pgfsetstrokecolor{currentstroke}%
\pgfsetdash{}{0pt}%
\pgfpathmoveto{\pgfqpoint{5.312192in}{2.744649in}}%
\pgfpathlineto{\pgfqpoint{5.323836in}{2.727820in}}%
\pgfpathlineto{\pgfqpoint{5.335502in}{2.710952in}}%
\pgfpathlineto{\pgfqpoint{5.369067in}{2.708260in}}%
\pgfpathlineto{\pgfqpoint{5.402615in}{2.706238in}}%
\pgfpathlineto{\pgfqpoint{5.390891in}{2.722683in}}%
\pgfpathlineto{\pgfqpoint{5.379189in}{2.739118in}}%
\pgfpathlineto{\pgfqpoint{5.345698in}{2.741525in}}%
\pgfpathlineto{\pgfqpoint{5.312192in}{2.744649in}}%
\pgfpathclose%
\pgfusepath{fill}%
\end{pgfscope}%
\begin{pgfscope}%
\pgfpathrectangle{\pgfqpoint{1.020000in}{0.880000in}}{\pgfqpoint{6.160000in}{6.160000in}}%
\pgfusepath{clip}%
\pgfsetbuttcap%
\pgfsetroundjoin%
\definecolor{currentfill}{rgb}{0.500031,0.638508,0.981070}%
\pgfsetfillcolor{currentfill}%
\pgfsetlinewidth{0.000000pt}%
\definecolor{currentstroke}{rgb}{0.000000,0.000000,0.000000}%
\pgfsetstrokecolor{currentstroke}%
\pgfsetdash{}{0pt}%
\pgfpathmoveto{\pgfqpoint{4.684646in}{3.116185in}}%
\pgfpathlineto{\pgfqpoint{4.695714in}{3.099767in}}%
\pgfpathlineto{\pgfqpoint{4.706798in}{3.082168in}}%
\pgfpathlineto{\pgfqpoint{4.740474in}{3.055364in}}%
\pgfpathlineto{\pgfqpoint{4.774119in}{3.030195in}}%
\pgfpathlineto{\pgfqpoint{4.762985in}{3.047586in}}%
\pgfpathlineto{\pgfqpoint{4.751868in}{3.064022in}}%
\pgfpathlineto{\pgfqpoint{4.718272in}{3.089286in}}%
\pgfpathlineto{\pgfqpoint{4.684646in}{3.116185in}}%
\pgfpathclose%
\pgfusepath{fill}%
\end{pgfscope}%
\begin{pgfscope}%
\pgfpathrectangle{\pgfqpoint{1.020000in}{0.880000in}}{\pgfqpoint{6.160000in}{6.160000in}}%
\pgfusepath{clip}%
\pgfsetbuttcap%
\pgfsetroundjoin%
\definecolor{currentfill}{rgb}{0.688188,0.793178,0.988038}%
\pgfsetfillcolor{currentfill}%
\pgfsetlinewidth{0.000000pt}%
\definecolor{currentstroke}{rgb}{0.000000,0.000000,0.000000}%
\pgfsetstrokecolor{currentstroke}%
\pgfsetdash{}{0pt}%
\pgfpathmoveto{\pgfqpoint{3.090099in}{3.400500in}}%
\pgfpathlineto{\pgfqpoint{3.099668in}{3.386615in}}%
\pgfpathlineto{\pgfqpoint{3.109242in}{3.374021in}}%
\pgfpathlineto{\pgfqpoint{3.143363in}{3.396487in}}%
\pgfpathlineto{\pgfqpoint{3.177473in}{3.419246in}}%
\pgfpathlineto{\pgfqpoint{3.167859in}{3.429334in}}%
\pgfpathlineto{\pgfqpoint{3.158252in}{3.440793in}}%
\pgfpathlineto{\pgfqpoint{3.124181in}{3.420553in}}%
\pgfpathlineto{\pgfqpoint{3.090099in}{3.400500in}}%
\pgfpathclose%
\pgfusepath{fill}%
\end{pgfscope}%
\begin{pgfscope}%
\pgfpathrectangle{\pgfqpoint{1.020000in}{0.880000in}}{\pgfqpoint{6.160000in}{6.160000in}}%
\pgfusepath{clip}%
\pgfsetbuttcap%
\pgfsetroundjoin%
\definecolor{currentfill}{rgb}{0.826784,0.858205,0.906953}%
\pgfsetfillcolor{currentfill}%
\pgfsetlinewidth{0.000000pt}%
\definecolor{currentstroke}{rgb}{0.000000,0.000000,0.000000}%
\pgfsetstrokecolor{currentstroke}%
\pgfsetdash{}{0pt}%
\pgfpathmoveto{\pgfqpoint{3.810889in}{3.682206in}}%
\pgfpathlineto{\pgfqpoint{3.821059in}{3.692989in}}%
\pgfpathlineto{\pgfqpoint{3.831254in}{3.703383in}}%
\pgfpathlineto{\pgfqpoint{3.865388in}{3.698166in}}%
\pgfpathlineto{\pgfqpoint{3.899503in}{3.690064in}}%
\pgfpathlineto{\pgfqpoint{3.889237in}{3.679703in}}%
\pgfpathlineto{\pgfqpoint{3.878996in}{3.668906in}}%
\pgfpathlineto{\pgfqpoint{3.844952in}{3.676863in}}%
\pgfpathlineto{\pgfqpoint{3.810889in}{3.682206in}}%
\pgfpathclose%
\pgfusepath{fill}%
\end{pgfscope}%
\begin{pgfscope}%
\pgfpathrectangle{\pgfqpoint{1.020000in}{0.880000in}}{\pgfqpoint{6.160000in}{6.160000in}}%
\pgfusepath{clip}%
\pgfsetbuttcap%
\pgfsetroundjoin%
\definecolor{currentfill}{rgb}{0.804965,0.851666,0.926165}%
\pgfsetfillcolor{currentfill}%
\pgfsetlinewidth{0.000000pt}%
\definecolor{currentstroke}{rgb}{0.000000,0.000000,0.000000}%
\pgfsetstrokecolor{currentstroke}%
\pgfsetdash{}{0pt}%
\pgfpathmoveto{\pgfqpoint{3.518270in}{3.620250in}}%
\pgfpathlineto{\pgfqpoint{3.528134in}{3.624740in}}%
\pgfpathlineto{\pgfqpoint{3.538016in}{3.629936in}}%
\pgfpathlineto{\pgfqpoint{3.572141in}{3.645097in}}%
\pgfpathlineto{\pgfqpoint{3.606266in}{3.658031in}}%
\pgfpathlineto{\pgfqpoint{3.596323in}{3.650744in}}%
\pgfpathlineto{\pgfqpoint{3.586399in}{3.644103in}}%
\pgfpathlineto{\pgfqpoint{3.552336in}{3.633195in}}%
\pgfpathlineto{\pgfqpoint{3.518270in}{3.620250in}}%
\pgfpathclose%
\pgfusepath{fill}%
\end{pgfscope}%
\begin{pgfscope}%
\pgfpathrectangle{\pgfqpoint{1.020000in}{0.880000in}}{\pgfqpoint{6.160000in}{6.160000in}}%
\pgfusepath{clip}%
\pgfsetbuttcap%
\pgfsetroundjoin%
\definecolor{currentfill}{rgb}{0.667253,0.779176,0.992959}%
\pgfsetfillcolor{currentfill}%
\pgfsetlinewidth{0.000000pt}%
\definecolor{currentstroke}{rgb}{0.000000,0.000000,0.000000}%
\pgfsetstrokecolor{currentstroke}%
\pgfsetdash{}{0pt}%
\pgfpathmoveto{\pgfqpoint{3.021893in}{3.361527in}}%
\pgfpathlineto{\pgfqpoint{3.031423in}{3.345459in}}%
\pgfpathlineto{\pgfqpoint{3.040958in}{3.330590in}}%
\pgfpathlineto{\pgfqpoint{3.075108in}{3.352008in}}%
\pgfpathlineto{\pgfqpoint{3.109242in}{3.374021in}}%
\pgfpathlineto{\pgfqpoint{3.099668in}{3.386615in}}%
\pgfpathlineto{\pgfqpoint{3.090099in}{3.400500in}}%
\pgfpathlineto{\pgfqpoint{3.056004in}{3.380780in}}%
\pgfpathlineto{\pgfqpoint{3.021893in}{3.361527in}}%
\pgfpathclose%
\pgfusepath{fill}%
\end{pgfscope}%
\begin{pgfscope}%
\pgfpathrectangle{\pgfqpoint{1.020000in}{0.880000in}}{\pgfqpoint{6.160000in}{6.160000in}}%
\pgfusepath{clip}%
\pgfsetbuttcap%
\pgfsetroundjoin%
\definecolor{currentfill}{rgb}{0.275827,0.366717,0.812553}%
\pgfsetfillcolor{currentfill}%
\pgfsetlinewidth{0.000000pt}%
\definecolor{currentstroke}{rgb}{0.000000,0.000000,0.000000}%
\pgfsetstrokecolor{currentstroke}%
\pgfsetdash{}{0pt}%
\pgfpathmoveto{\pgfqpoint{5.919144in}{2.657981in}}%
\pgfpathlineto{\pgfqpoint{5.931387in}{2.643191in}}%
\pgfpathlineto{\pgfqpoint{5.943654in}{2.628453in}}%
\pgfpathlineto{\pgfqpoint{5.977095in}{2.630942in}}%
\pgfpathlineto{\pgfqpoint{6.010516in}{2.633481in}}%
\pgfpathlineto{\pgfqpoint{5.998192in}{2.648120in}}%
\pgfpathlineto{\pgfqpoint{5.985893in}{2.662810in}}%
\pgfpathlineto{\pgfqpoint{5.952528in}{2.660368in}}%
\pgfpathlineto{\pgfqpoint{5.919144in}{2.657981in}}%
\pgfpathclose%
\pgfusepath{fill}%
\end{pgfscope}%
\begin{pgfscope}%
\pgfpathrectangle{\pgfqpoint{1.020000in}{0.880000in}}{\pgfqpoint{6.160000in}{6.160000in}}%
\pgfusepath{clip}%
\pgfsetbuttcap%
\pgfsetroundjoin%
\definecolor{currentfill}{rgb}{0.266381,0.353304,0.801637}%
\pgfsetfillcolor{currentfill}%
\pgfsetlinewidth{0.000000pt}%
\definecolor{currentstroke}{rgb}{0.000000,0.000000,0.000000}%
\pgfsetstrokecolor{currentstroke}%
\pgfsetdash{}{0pt}%
\pgfpathmoveto{\pgfqpoint{6.143998in}{2.643993in}}%
\pgfpathlineto{\pgfqpoint{6.156460in}{2.629573in}}%
\pgfpathlineto{\pgfqpoint{6.168947in}{2.615202in}}%
\pgfpathlineto{\pgfqpoint{6.202323in}{2.617969in}}%
\pgfpathlineto{\pgfqpoint{6.235679in}{2.620752in}}%
\pgfpathlineto{\pgfqpoint{6.223135in}{2.635049in}}%
\pgfpathlineto{\pgfqpoint{6.210617in}{2.649394in}}%
\pgfpathlineto{\pgfqpoint{6.177318in}{2.646684in}}%
\pgfpathlineto{\pgfqpoint{6.143998in}{2.643993in}}%
\pgfpathclose%
\pgfusepath{fill}%
\end{pgfscope}%
\begin{pgfscope}%
\pgfpathrectangle{\pgfqpoint{1.020000in}{0.880000in}}{\pgfqpoint{6.160000in}{6.160000in}}%
\pgfusepath{clip}%
\pgfsetbuttcap%
\pgfsetroundjoin%
\definecolor{currentfill}{rgb}{0.285273,0.380129,0.823469}%
\pgfsetfillcolor{currentfill}%
\pgfsetlinewidth{0.000000pt}%
\definecolor{currentstroke}{rgb}{0.000000,0.000000,0.000000}%
\pgfsetstrokecolor{currentstroke}%
\pgfsetdash{}{0pt}%
\pgfpathmoveto{\pgfqpoint{5.694351in}{2.676096in}}%
\pgfpathlineto{\pgfqpoint{5.706375in}{2.660796in}}%
\pgfpathlineto{\pgfqpoint{5.718422in}{2.645547in}}%
\pgfpathlineto{\pgfqpoint{5.751924in}{2.647311in}}%
\pgfpathlineto{\pgfqpoint{5.785408in}{2.649225in}}%
\pgfpathlineto{\pgfqpoint{5.773303in}{2.664314in}}%
\pgfpathlineto{\pgfqpoint{5.761222in}{2.679456in}}%
\pgfpathlineto{\pgfqpoint{5.727796in}{2.677693in}}%
\pgfpathlineto{\pgfqpoint{5.694351in}{2.676096in}}%
\pgfpathclose%
\pgfusepath{fill}%
\end{pgfscope}%
\begin{pgfscope}%
\pgfpathrectangle{\pgfqpoint{1.020000in}{0.880000in}}{\pgfqpoint{6.160000in}{6.160000in}}%
\pgfusepath{clip}%
\pgfsetbuttcap%
\pgfsetroundjoin%
\definecolor{currentfill}{rgb}{0.603162,0.731527,0.999565}%
\pgfsetfillcolor{currentfill}%
\pgfsetlinewidth{0.000000pt}%
\definecolor{currentstroke}{rgb}{0.000000,0.000000,0.000000}%
\pgfsetstrokecolor{currentstroke}%
\pgfsetdash{}{0pt}%
\pgfpathmoveto{\pgfqpoint{2.660933in}{3.261512in}}%
\pgfpathlineto{\pgfqpoint{2.670250in}{3.236969in}}%
\pgfpathlineto{\pgfqpoint{2.679571in}{3.213280in}}%
\pgfpathlineto{\pgfqpoint{2.713923in}{3.223818in}}%
\pgfpathlineto{\pgfqpoint{2.748245in}{3.235346in}}%
\pgfpathlineto{\pgfqpoint{2.738880in}{3.258286in}}%
\pgfpathlineto{\pgfqpoint{2.729517in}{3.282219in}}%
\pgfpathlineto{\pgfqpoint{2.695239in}{3.271419in}}%
\pgfpathlineto{\pgfqpoint{2.660933in}{3.261512in}}%
\pgfpathclose%
\pgfusepath{fill}%
\end{pgfscope}%
\begin{pgfscope}%
\pgfpathrectangle{\pgfqpoint{1.020000in}{0.880000in}}{\pgfqpoint{6.160000in}{6.160000in}}%
\pgfusepath{clip}%
\pgfsetbuttcap%
\pgfsetroundjoin%
\definecolor{currentfill}{rgb}{0.294718,0.393542,0.834384}%
\pgfsetfillcolor{currentfill}%
\pgfsetlinewidth{0.000000pt}%
\definecolor{currentstroke}{rgb}{0.000000,0.000000,0.000000}%
\pgfsetstrokecolor{currentstroke}%
\pgfsetdash{}{0pt}%
\pgfpathmoveto{\pgfqpoint{5.469660in}{2.703934in}}%
\pgfpathlineto{\pgfqpoint{5.481465in}{2.687843in}}%
\pgfpathlineto{\pgfqpoint{5.493292in}{2.671776in}}%
\pgfpathlineto{\pgfqpoint{5.526847in}{2.671674in}}%
\pgfpathlineto{\pgfqpoint{5.560384in}{2.671960in}}%
\pgfpathlineto{\pgfqpoint{5.548499in}{2.687733in}}%
\pgfpathlineto{\pgfqpoint{5.536637in}{2.703541in}}%
\pgfpathlineto{\pgfqpoint{5.503157in}{2.703527in}}%
\pgfpathlineto{\pgfqpoint{5.469660in}{2.703934in}}%
\pgfpathclose%
\pgfusepath{fill}%
\end{pgfscope}%
\begin{pgfscope}%
\pgfpathrectangle{\pgfqpoint{1.020000in}{0.880000in}}{\pgfqpoint{6.160000in}{6.160000in}}%
\pgfusepath{clip}%
\pgfsetbuttcap%
\pgfsetroundjoin%
\definecolor{currentfill}{rgb}{0.394042,0.522413,0.924916}%
\pgfsetfillcolor{currentfill}%
\pgfsetlinewidth{0.000000pt}%
\definecolor{currentstroke}{rgb}{0.000000,0.000000,0.000000}%
\pgfsetstrokecolor{currentstroke}%
\pgfsetdash{}{0pt}%
\pgfpathmoveto{\pgfqpoint{4.930949in}{2.910808in}}%
\pgfpathlineto{\pgfqpoint{4.942235in}{2.892081in}}%
\pgfpathlineto{\pgfqpoint{4.953537in}{2.872826in}}%
\pgfpathlineto{\pgfqpoint{4.987163in}{2.857242in}}%
\pgfpathlineto{\pgfqpoint{5.020768in}{2.843207in}}%
\pgfpathlineto{\pgfqpoint{5.009411in}{2.861744in}}%
\pgfpathlineto{\pgfqpoint{4.998071in}{2.879884in}}%
\pgfpathlineto{\pgfqpoint{4.964520in}{2.894545in}}%
\pgfpathlineto{\pgfqpoint{4.930949in}{2.910808in}}%
\pgfpathclose%
\pgfusepath{fill}%
\end{pgfscope}%
\begin{pgfscope}%
\pgfpathrectangle{\pgfqpoint{1.020000in}{0.880000in}}{\pgfqpoint{6.160000in}{6.160000in}}%
\pgfusepath{clip}%
\pgfsetbuttcap%
\pgfsetroundjoin%
\definecolor{currentfill}{rgb}{0.713852,0.808857,0.979386}%
\pgfsetfillcolor{currentfill}%
\pgfsetlinewidth{0.000000pt}%
\definecolor{currentstroke}{rgb}{0.000000,0.000000,0.000000}%
\pgfsetstrokecolor{currentstroke}%
\pgfsetdash{}{0pt}%
\pgfpathmoveto{\pgfqpoint{4.281705in}{3.490953in}}%
\pgfpathlineto{\pgfqpoint{4.292419in}{3.488349in}}%
\pgfpathlineto{\pgfqpoint{4.303154in}{3.483840in}}%
\pgfpathlineto{\pgfqpoint{4.337081in}{3.451073in}}%
\pgfpathlineto{\pgfqpoint{4.370963in}{3.417925in}}%
\pgfpathlineto{\pgfqpoint{4.360174in}{3.423776in}}%
\pgfpathlineto{\pgfqpoint{4.349404in}{3.427932in}}%
\pgfpathlineto{\pgfqpoint{4.315576in}{3.459619in}}%
\pgfpathlineto{\pgfqpoint{4.281705in}{3.490953in}}%
\pgfpathclose%
\pgfusepath{fill}%
\end{pgfscope}%
\begin{pgfscope}%
\pgfpathrectangle{\pgfqpoint{1.020000in}{0.880000in}}{\pgfqpoint{6.160000in}{6.160000in}}%
\pgfusepath{clip}%
\pgfsetbuttcap%
\pgfsetroundjoin%
\definecolor{currentfill}{rgb}{0.348323,0.465711,0.888346}%
\pgfsetfillcolor{currentfill}%
\pgfsetlinewidth{0.000000pt}%
\definecolor{currentstroke}{rgb}{0.000000,0.000000,0.000000}%
\pgfsetstrokecolor{currentstroke}%
\pgfsetdash{}{0pt}%
\pgfpathmoveto{\pgfqpoint{5.087921in}{2.819545in}}%
\pgfpathlineto{\pgfqpoint{5.099353in}{2.801349in}}%
\pgfpathlineto{\pgfqpoint{5.110804in}{2.782908in}}%
\pgfpathlineto{\pgfqpoint{5.144411in}{2.773795in}}%
\pgfpathlineto{\pgfqpoint{5.178000in}{2.765886in}}%
\pgfpathlineto{\pgfqpoint{5.166492in}{2.783659in}}%
\pgfpathlineto{\pgfqpoint{5.155003in}{2.801263in}}%
\pgfpathlineto{\pgfqpoint{5.121471in}{2.809772in}}%
\pgfpathlineto{\pgfqpoint{5.087921in}{2.819545in}}%
\pgfpathclose%
\pgfusepath{fill}%
\end{pgfscope}%
\begin{pgfscope}%
\pgfpathrectangle{\pgfqpoint{1.020000in}{0.880000in}}{\pgfqpoint{6.160000in}{6.160000in}}%
\pgfusepath{clip}%
\pgfsetbuttcap%
\pgfsetroundjoin%
\definecolor{currentfill}{rgb}{0.586921,0.718121,0.998874}%
\pgfsetfillcolor{currentfill}%
\pgfsetlinewidth{0.000000pt}%
\definecolor{currentstroke}{rgb}{0.000000,0.000000,0.000000}%
\pgfsetstrokecolor{currentstroke}%
\pgfsetdash{}{0pt}%
\pgfpathmoveto{\pgfqpoint{4.527895in}{3.265813in}}%
\pgfpathlineto{\pgfqpoint{4.538836in}{3.252785in}}%
\pgfpathlineto{\pgfqpoint{4.549792in}{3.238116in}}%
\pgfpathlineto{\pgfqpoint{4.583562in}{3.205757in}}%
\pgfpathlineto{\pgfqpoint{4.617293in}{3.174544in}}%
\pgfpathlineto{\pgfqpoint{4.606288in}{3.189518in}}%
\pgfpathlineto{\pgfqpoint{4.595298in}{3.203112in}}%
\pgfpathlineto{\pgfqpoint{4.561616in}{3.233898in}}%
\pgfpathlineto{\pgfqpoint{4.527895in}{3.265813in}}%
\pgfpathclose%
\pgfusepath{fill}%
\end{pgfscope}%
\begin{pgfscope}%
\pgfpathrectangle{\pgfqpoint{1.020000in}{0.880000in}}{\pgfqpoint{6.160000in}{6.160000in}}%
\pgfusepath{clip}%
\pgfsetbuttcap%
\pgfsetroundjoin%
\definecolor{currentfill}{rgb}{0.646113,0.764436,0.996868}%
\pgfsetfillcolor{currentfill}%
\pgfsetlinewidth{0.000000pt}%
\definecolor{currentstroke}{rgb}{0.000000,0.000000,0.000000}%
\pgfsetstrokecolor{currentstroke}%
\pgfsetdash{}{0pt}%
\pgfpathmoveto{\pgfqpoint{2.953617in}{3.324882in}}%
\pgfpathlineto{\pgfqpoint{2.963109in}{3.306904in}}%
\pgfpathlineto{\pgfqpoint{2.972605in}{3.290024in}}%
\pgfpathlineto{\pgfqpoint{3.006791in}{3.309893in}}%
\pgfpathlineto{\pgfqpoint{3.040958in}{3.330590in}}%
\pgfpathlineto{\pgfqpoint{3.031423in}{3.345459in}}%
\pgfpathlineto{\pgfqpoint{3.021893in}{3.361527in}}%
\pgfpathlineto{\pgfqpoint{2.987764in}{3.342860in}}%
\pgfpathlineto{\pgfqpoint{2.953617in}{3.324882in}}%
\pgfpathclose%
\pgfusepath{fill}%
\end{pgfscope}%
\begin{pgfscope}%
\pgfpathrectangle{\pgfqpoint{1.020000in}{0.880000in}}{\pgfqpoint{6.160000in}{6.160000in}}%
\pgfusepath{clip}%
\pgfsetbuttcap%
\pgfsetroundjoin%
\definecolor{currentfill}{rgb}{0.791392,0.846750,0.936641}%
\pgfsetfillcolor{currentfill}%
\pgfsetlinewidth{0.000000pt}%
\definecolor{currentstroke}{rgb}{0.000000,0.000000,0.000000}%
\pgfsetstrokecolor{currentstroke}%
\pgfsetdash{}{0pt}%
\pgfpathmoveto{\pgfqpoint{3.450131in}{3.588882in}}%
\pgfpathlineto{\pgfqpoint{3.459939in}{3.590900in}}%
\pgfpathlineto{\pgfqpoint{3.469764in}{3.593674in}}%
\pgfpathlineto{\pgfqpoint{3.503890in}{3.612729in}}%
\pgfpathlineto{\pgfqpoint{3.538016in}{3.629936in}}%
\pgfpathlineto{\pgfqpoint{3.528134in}{3.624740in}}%
\pgfpathlineto{\pgfqpoint{3.518270in}{3.620250in}}%
\pgfpathlineto{\pgfqpoint{3.484201in}{3.605421in}}%
\pgfpathlineto{\pgfqpoint{3.450131in}{3.588882in}}%
\pgfpathclose%
\pgfusepath{fill}%
\end{pgfscope}%
\begin{pgfscope}%
\pgfpathrectangle{\pgfqpoint{1.020000in}{0.880000in}}{\pgfqpoint{6.160000in}{6.160000in}}%
\pgfusepath{clip}%
\pgfsetbuttcap%
\pgfsetroundjoin%
\definecolor{currentfill}{rgb}{0.457046,0.594006,0.963029}%
\pgfsetfillcolor{currentfill}%
\pgfsetlinewidth{0.000000pt}%
\definecolor{currentstroke}{rgb}{0.000000,0.000000,0.000000}%
\pgfsetstrokecolor{currentstroke}%
\pgfsetdash{}{0pt}%
\pgfpathmoveto{\pgfqpoint{4.774119in}{3.030195in}}%
\pgfpathlineto{\pgfqpoint{4.785267in}{3.011853in}}%
\pgfpathlineto{\pgfqpoint{4.796431in}{2.992577in}}%
\pgfpathlineto{\pgfqpoint{4.830099in}{2.969599in}}%
\pgfpathlineto{\pgfqpoint{4.863740in}{2.948311in}}%
\pgfpathlineto{\pgfqpoint{4.852524in}{2.967002in}}%
\pgfpathlineto{\pgfqpoint{4.841324in}{2.984953in}}%
\pgfpathlineto{\pgfqpoint{4.807735in}{3.006715in}}%
\pgfpathlineto{\pgfqpoint{4.774119in}{3.030195in}}%
\pgfpathclose%
\pgfusepath{fill}%
\end{pgfscope}%
\begin{pgfscope}%
\pgfpathrectangle{\pgfqpoint{1.020000in}{0.880000in}}{\pgfqpoint{6.160000in}{6.160000in}}%
\pgfusepath{clip}%
\pgfsetbuttcap%
\pgfsetroundjoin%
\definecolor{currentfill}{rgb}{0.318832,0.426605,0.859857}%
\pgfsetfillcolor{currentfill}%
\pgfsetlinewidth{0.000000pt}%
\definecolor{currentstroke}{rgb}{0.000000,0.000000,0.000000}%
\pgfsetstrokecolor{currentstroke}%
\pgfsetdash{}{0pt}%
\pgfpathmoveto{\pgfqpoint{5.245129in}{2.753355in}}%
\pgfpathlineto{\pgfqpoint{5.256716in}{2.736037in}}%
\pgfpathlineto{\pgfqpoint{5.268323in}{2.718643in}}%
\pgfpathlineto{\pgfqpoint{5.301921in}{2.714388in}}%
\pgfpathlineto{\pgfqpoint{5.335502in}{2.710952in}}%
\pgfpathlineto{\pgfqpoint{5.323836in}{2.727820in}}%
\pgfpathlineto{\pgfqpoint{5.312192in}{2.744649in}}%
\pgfpathlineto{\pgfqpoint{5.278669in}{2.748565in}}%
\pgfpathlineto{\pgfqpoint{5.245129in}{2.753355in}}%
\pgfpathclose%
\pgfusepath{fill}%
\end{pgfscope}%
\begin{pgfscope}%
\pgfpathrectangle{\pgfqpoint{1.020000in}{0.880000in}}{\pgfqpoint{6.160000in}{6.160000in}}%
\pgfusepath{clip}%
\pgfsetbuttcap%
\pgfsetroundjoin%
\definecolor{currentfill}{rgb}{0.813693,0.854282,0.918480}%
\pgfsetfillcolor{currentfill}%
\pgfsetlinewidth{0.000000pt}%
\definecolor{currentstroke}{rgb}{0.000000,0.000000,0.000000}%
\pgfsetstrokecolor{currentstroke}%
\pgfsetdash{}{0pt}%
\pgfpathmoveto{\pgfqpoint{3.967667in}{3.665647in}}%
\pgfpathlineto{\pgfqpoint{3.978030in}{3.674606in}}%
\pgfpathlineto{\pgfqpoint{3.988420in}{3.682348in}}%
\pgfpathlineto{\pgfqpoint{4.022532in}{3.665425in}}%
\pgfpathlineto{\pgfqpoint{4.056613in}{3.646005in}}%
\pgfpathlineto{\pgfqpoint{4.046155in}{3.639263in}}%
\pgfpathlineto{\pgfqpoint{4.035722in}{3.631335in}}%
\pgfpathlineto{\pgfqpoint{4.001709in}{3.649635in}}%
\pgfpathlineto{\pgfqpoint{3.967667in}{3.665647in}}%
\pgfpathclose%
\pgfusepath{fill}%
\end{pgfscope}%
\begin{pgfscope}%
\pgfpathrectangle{\pgfqpoint{1.020000in}{0.880000in}}{\pgfqpoint{6.160000in}{6.160000in}}%
\pgfusepath{clip}%
\pgfsetbuttcap%
\pgfsetroundjoin%
\definecolor{currentfill}{rgb}{0.777378,0.840921,0.946149}%
\pgfsetfillcolor{currentfill}%
\pgfsetlinewidth{0.000000pt}%
\definecolor{currentstroke}{rgb}{0.000000,0.000000,0.000000}%
\pgfsetstrokecolor{currentstroke}%
\pgfsetdash{}{0pt}%
\pgfpathmoveto{\pgfqpoint{4.124675in}{3.600624in}}%
\pgfpathlineto{\pgfqpoint{4.135224in}{3.604543in}}%
\pgfpathlineto{\pgfqpoint{4.145797in}{3.606716in}}%
\pgfpathlineto{\pgfqpoint{4.179836in}{3.579826in}}%
\pgfpathlineto{\pgfqpoint{4.213834in}{3.551363in}}%
\pgfpathlineto{\pgfqpoint{4.203199in}{3.550639in}}%
\pgfpathlineto{\pgfqpoint{4.192587in}{3.548297in}}%
\pgfpathlineto{\pgfqpoint{4.158650in}{3.575190in}}%
\pgfpathlineto{\pgfqpoint{4.124675in}{3.600624in}}%
\pgfpathclose%
\pgfusepath{fill}%
\end{pgfscope}%
\begin{pgfscope}%
\pgfpathrectangle{\pgfqpoint{1.020000in}{0.880000in}}{\pgfqpoint{6.160000in}{6.160000in}}%
\pgfusepath{clip}%
\pgfsetbuttcap%
\pgfsetroundjoin%
\definecolor{currentfill}{rgb}{0.592356,0.722792,0.999434}%
\pgfsetfillcolor{currentfill}%
\pgfsetlinewidth{0.000000pt}%
\definecolor{currentstroke}{rgb}{0.000000,0.000000,0.000000}%
\pgfsetstrokecolor{currentstroke}%
\pgfsetdash{}{0pt}%
\pgfpathmoveto{\pgfqpoint{2.592230in}{3.244341in}}%
\pgfpathlineto{\pgfqpoint{2.601500in}{3.219358in}}%
\pgfpathlineto{\pgfqpoint{2.610774in}{3.195099in}}%
\pgfpathlineto{\pgfqpoint{2.645188in}{3.203715in}}%
\pgfpathlineto{\pgfqpoint{2.679571in}{3.213280in}}%
\pgfpathlineto{\pgfqpoint{2.670250in}{3.236969in}}%
\pgfpathlineto{\pgfqpoint{2.660933in}{3.261512in}}%
\pgfpathlineto{\pgfqpoint{2.626596in}{3.252491in}}%
\pgfpathlineto{\pgfqpoint{2.592230in}{3.244341in}}%
\pgfpathclose%
\pgfusepath{fill}%
\end{pgfscope}%
\begin{pgfscope}%
\pgfpathrectangle{\pgfqpoint{1.020000in}{0.880000in}}{\pgfqpoint{6.160000in}{6.160000in}}%
\pgfusepath{clip}%
\pgfsetbuttcap%
\pgfsetroundjoin%
\definecolor{currentfill}{rgb}{0.624703,0.748318,0.998719}%
\pgfsetfillcolor{currentfill}%
\pgfsetlinewidth{0.000000pt}%
\definecolor{currentstroke}{rgb}{0.000000,0.000000,0.000000}%
\pgfsetstrokecolor{currentstroke}%
\pgfsetdash{}{0pt}%
\pgfpathmoveto{\pgfqpoint{2.885259in}{3.291329in}}%
\pgfpathlineto{\pgfqpoint{2.894711in}{3.271725in}}%
\pgfpathlineto{\pgfqpoint{2.904168in}{3.253111in}}%
\pgfpathlineto{\pgfqpoint{2.938398in}{3.271072in}}%
\pgfpathlineto{\pgfqpoint{2.972605in}{3.290024in}}%
\pgfpathlineto{\pgfqpoint{2.963109in}{3.306904in}}%
\pgfpathlineto{\pgfqpoint{2.953617in}{3.324882in}}%
\pgfpathlineto{\pgfqpoint{2.919449in}{3.307681in}}%
\pgfpathlineto{\pgfqpoint{2.885259in}{3.291329in}}%
\pgfpathclose%
\pgfusepath{fill}%
\end{pgfscope}%
\begin{pgfscope}%
\pgfpathrectangle{\pgfqpoint{1.020000in}{0.880000in}}{\pgfqpoint{6.160000in}{6.160000in}}%
\pgfusepath{clip}%
\pgfsetbuttcap%
\pgfsetroundjoin%
\definecolor{currentfill}{rgb}{0.835345,0.860514,0.898970}%
\pgfsetfillcolor{currentfill}%
\pgfsetlinewidth{0.000000pt}%
\definecolor{currentstroke}{rgb}{0.000000,0.000000,0.000000}%
\pgfsetstrokecolor{currentstroke}%
\pgfsetdash{}{0pt}%
\pgfpathmoveto{\pgfqpoint{3.742719in}{3.684822in}}%
\pgfpathlineto{\pgfqpoint{3.752818in}{3.695039in}}%
\pgfpathlineto{\pgfqpoint{3.762942in}{3.704926in}}%
\pgfpathlineto{\pgfqpoint{3.797104in}{3.705646in}}%
\pgfpathlineto{\pgfqpoint{3.831254in}{3.703383in}}%
\pgfpathlineto{\pgfqpoint{3.821059in}{3.692989in}}%
\pgfpathlineto{\pgfqpoint{3.810889in}{3.682206in}}%
\pgfpathlineto{\pgfqpoint{3.776811in}{3.684870in}}%
\pgfpathlineto{\pgfqpoint{3.742719in}{3.684822in}}%
\pgfpathclose%
\pgfusepath{fill}%
\end{pgfscope}%
\begin{pgfscope}%
\pgfpathrectangle{\pgfqpoint{1.020000in}{0.880000in}}{\pgfqpoint{6.160000in}{6.160000in}}%
\pgfusepath{clip}%
\pgfsetbuttcap%
\pgfsetroundjoin%
\definecolor{currentfill}{rgb}{0.275827,0.366717,0.812553}%
\pgfsetfillcolor{currentfill}%
\pgfsetlinewidth{0.000000pt}%
\definecolor{currentstroke}{rgb}{0.000000,0.000000,0.000000}%
\pgfsetstrokecolor{currentstroke}%
\pgfsetdash{}{0pt}%
\pgfpathmoveto{\pgfqpoint{5.852315in}{2.653417in}}%
\pgfpathlineto{\pgfqpoint{5.864501in}{2.638514in}}%
\pgfpathlineto{\pgfqpoint{5.876711in}{2.623664in}}%
\pgfpathlineto{\pgfqpoint{5.910193in}{2.626024in}}%
\pgfpathlineto{\pgfqpoint{5.943654in}{2.628453in}}%
\pgfpathlineto{\pgfqpoint{5.931387in}{2.643191in}}%
\pgfpathlineto{\pgfqpoint{5.919144in}{2.657981in}}%
\pgfpathlineto{\pgfqpoint{5.885739in}{2.655660in}}%
\pgfpathlineto{\pgfqpoint{5.852315in}{2.653417in}}%
\pgfpathclose%
\pgfusepath{fill}%
\end{pgfscope}%
\begin{pgfscope}%
\pgfpathrectangle{\pgfqpoint{1.020000in}{0.880000in}}{\pgfqpoint{6.160000in}{6.160000in}}%
\pgfusepath{clip}%
\pgfsetbuttcap%
\pgfsetroundjoin%
\definecolor{currentfill}{rgb}{0.271104,0.360011,0.807095}%
\pgfsetfillcolor{currentfill}%
\pgfsetlinewidth{0.000000pt}%
\definecolor{currentstroke}{rgb}{0.000000,0.000000,0.000000}%
\pgfsetstrokecolor{currentstroke}%
\pgfsetdash{}{0pt}%
\pgfpathmoveto{\pgfqpoint{6.077298in}{2.638677in}}%
\pgfpathlineto{\pgfqpoint{6.089703in}{2.624177in}}%
\pgfpathlineto{\pgfqpoint{6.102134in}{2.609727in}}%
\pgfpathlineto{\pgfqpoint{6.135551in}{2.612453in}}%
\pgfpathlineto{\pgfqpoint{6.168947in}{2.615202in}}%
\pgfpathlineto{\pgfqpoint{6.156460in}{2.629573in}}%
\pgfpathlineto{\pgfqpoint{6.143998in}{2.643993in}}%
\pgfpathlineto{\pgfqpoint{6.110658in}{2.641322in}}%
\pgfpathlineto{\pgfqpoint{6.077298in}{2.638677in}}%
\pgfpathclose%
\pgfusepath{fill}%
\end{pgfscope}%
\begin{pgfscope}%
\pgfpathrectangle{\pgfqpoint{1.020000in}{0.880000in}}{\pgfqpoint{6.160000in}{6.160000in}}%
\pgfusepath{clip}%
\pgfsetbuttcap%
\pgfsetroundjoin%
\definecolor{currentfill}{rgb}{0.285273,0.380129,0.823469}%
\pgfsetfillcolor{currentfill}%
\pgfsetlinewidth{0.000000pt}%
\definecolor{currentstroke}{rgb}{0.000000,0.000000,0.000000}%
\pgfsetstrokecolor{currentstroke}%
\pgfsetdash{}{0pt}%
\pgfpathmoveto{\pgfqpoint{5.627404in}{2.673510in}}%
\pgfpathlineto{\pgfqpoint{5.639371in}{2.658017in}}%
\pgfpathlineto{\pgfqpoint{5.651361in}{2.642571in}}%
\pgfpathlineto{\pgfqpoint{5.684901in}{2.643958in}}%
\pgfpathlineto{\pgfqpoint{5.718422in}{2.645547in}}%
\pgfpathlineto{\pgfqpoint{5.706375in}{2.660796in}}%
\pgfpathlineto{\pgfqpoint{5.694351in}{2.676096in}}%
\pgfpathlineto{\pgfqpoint{5.660887in}{2.674691in}}%
\pgfpathlineto{\pgfqpoint{5.627404in}{2.673510in}}%
\pgfpathclose%
\pgfusepath{fill}%
\end{pgfscope}%
\begin{pgfscope}%
\pgfpathrectangle{\pgfqpoint{1.020000in}{0.880000in}}{\pgfqpoint{6.160000in}{6.160000in}}%
\pgfusepath{clip}%
\pgfsetbuttcap%
\pgfsetroundjoin%
\definecolor{currentfill}{rgb}{0.777378,0.840921,0.946149}%
\pgfsetfillcolor{currentfill}%
\pgfsetlinewidth{0.000000pt}%
\definecolor{currentstroke}{rgb}{0.000000,0.000000,0.000000}%
\pgfsetstrokecolor{currentstroke}%
\pgfsetdash{}{0pt}%
\pgfpathmoveto{\pgfqpoint{3.381986in}{3.551440in}}%
\pgfpathlineto{\pgfqpoint{3.391742in}{3.550770in}}%
\pgfpathlineto{\pgfqpoint{3.401514in}{3.550892in}}%
\pgfpathlineto{\pgfqpoint{3.435639in}{3.572986in}}%
\pgfpathlineto{\pgfqpoint{3.469764in}{3.593674in}}%
\pgfpathlineto{\pgfqpoint{3.459939in}{3.590900in}}%
\pgfpathlineto{\pgfqpoint{3.450131in}{3.588882in}}%
\pgfpathlineto{\pgfqpoint{3.416059in}{3.570821in}}%
\pgfpathlineto{\pgfqpoint{3.381986in}{3.551440in}}%
\pgfpathclose%
\pgfusepath{fill}%
\end{pgfscope}%
\begin{pgfscope}%
\pgfpathrectangle{\pgfqpoint{1.020000in}{0.880000in}}{\pgfqpoint{6.160000in}{6.160000in}}%
\pgfusepath{clip}%
\pgfsetbuttcap%
\pgfsetroundjoin%
\definecolor{currentfill}{rgb}{0.299441,0.400248,0.839842}%
\pgfsetfillcolor{currentfill}%
\pgfsetlinewidth{0.000000pt}%
\definecolor{currentstroke}{rgb}{0.000000,0.000000,0.000000}%
\pgfsetstrokecolor{currentstroke}%
\pgfsetdash{}{0pt}%
\pgfpathmoveto{\pgfqpoint{5.402615in}{2.706238in}}%
\pgfpathlineto{\pgfqpoint{5.414362in}{2.689791in}}%
\pgfpathlineto{\pgfqpoint{5.426130in}{2.673352in}}%
\pgfpathlineto{\pgfqpoint{5.459720in}{2.672317in}}%
\pgfpathlineto{\pgfqpoint{5.493292in}{2.671776in}}%
\pgfpathlineto{\pgfqpoint{5.481465in}{2.687843in}}%
\pgfpathlineto{\pgfqpoint{5.469660in}{2.703934in}}%
\pgfpathlineto{\pgfqpoint{5.436146in}{2.704818in}}%
\pgfpathlineto{\pgfqpoint{5.402615in}{2.706238in}}%
\pgfpathclose%
\pgfusepath{fill}%
\end{pgfscope}%
\begin{pgfscope}%
\pgfpathrectangle{\pgfqpoint{1.020000in}{0.880000in}}{\pgfqpoint{6.160000in}{6.160000in}}%
\pgfusepath{clip}%
\pgfsetbuttcap%
\pgfsetroundjoin%
\definecolor{currentfill}{rgb}{0.261805,0.346484,0.795658}%
\pgfsetfillcolor{currentfill}%
\pgfsetlinewidth{0.000000pt}%
\definecolor{currentstroke}{rgb}{0.000000,0.000000,0.000000}%
\pgfsetstrokecolor{currentstroke}%
\pgfsetdash{}{0pt}%
\pgfpathmoveto{\pgfqpoint{6.302328in}{2.626358in}}%
\pgfpathlineto{\pgfqpoint{6.314954in}{2.612177in}}%
\pgfpathlineto{\pgfqpoint{6.327604in}{2.598041in}}%
\pgfpathlineto{\pgfqpoint{6.360955in}{2.600927in}}%
\pgfpathlineto{\pgfqpoint{6.348276in}{2.615030in}}%
\pgfpathlineto{\pgfqpoint{6.335622in}{2.629177in}}%
\pgfpathlineto{\pgfqpoint{6.302328in}{2.626358in}}%
\pgfpathclose%
\pgfusepath{fill}%
\end{pgfscope}%
\begin{pgfscope}%
\pgfpathrectangle{\pgfqpoint{1.020000in}{0.880000in}}{\pgfqpoint{6.160000in}{6.160000in}}%
\pgfusepath{clip}%
\pgfsetbuttcap%
\pgfsetroundjoin%
\definecolor{currentfill}{rgb}{0.538004,0.674902,0.991722}%
\pgfsetfillcolor{currentfill}%
\pgfsetlinewidth{0.000000pt}%
\definecolor{currentstroke}{rgb}{0.000000,0.000000,0.000000}%
\pgfsetstrokecolor{currentstroke}%
\pgfsetdash{}{0pt}%
\pgfpathmoveto{\pgfqpoint{4.617293in}{3.174544in}}%
\pgfpathlineto{\pgfqpoint{4.628312in}{3.158154in}}%
\pgfpathlineto{\pgfqpoint{4.639347in}{3.140335in}}%
\pgfpathlineto{\pgfqpoint{4.673090in}{3.110528in}}%
\pgfpathlineto{\pgfqpoint{4.706798in}{3.082168in}}%
\pgfpathlineto{\pgfqpoint{4.695714in}{3.099767in}}%
\pgfpathlineto{\pgfqpoint{4.684646in}{3.116185in}}%
\pgfpathlineto{\pgfqpoint{4.650987in}{3.144641in}}%
\pgfpathlineto{\pgfqpoint{4.617293in}{3.174544in}}%
\pgfpathclose%
\pgfusepath{fill}%
\end{pgfscope}%
\begin{pgfscope}%
\pgfpathrectangle{\pgfqpoint{1.020000in}{0.880000in}}{\pgfqpoint{6.160000in}{6.160000in}}%
\pgfusepath{clip}%
\pgfsetbuttcap%
\pgfsetroundjoin%
\definecolor{currentfill}{rgb}{0.677823,0.786546,0.991005}%
\pgfsetfillcolor{currentfill}%
\pgfsetlinewidth{0.000000pt}%
\definecolor{currentstroke}{rgb}{0.000000,0.000000,0.000000}%
\pgfsetstrokecolor{currentstroke}%
\pgfsetdash{}{0pt}%
\pgfpathmoveto{\pgfqpoint{4.370963in}{3.417925in}}%
\pgfpathlineto{\pgfqpoint{4.381771in}{3.410214in}}%
\pgfpathlineto{\pgfqpoint{4.392596in}{3.400505in}}%
\pgfpathlineto{\pgfqpoint{4.426485in}{3.366268in}}%
\pgfpathlineto{\pgfqpoint{4.460331in}{3.332231in}}%
\pgfpathlineto{\pgfqpoint{4.449455in}{3.342818in}}%
\pgfpathlineto{\pgfqpoint{4.438597in}{3.351661in}}%
\pgfpathlineto{\pgfqpoint{4.404801in}{3.384693in}}%
\pgfpathlineto{\pgfqpoint{4.370963in}{3.417925in}}%
\pgfpathclose%
\pgfusepath{fill}%
\end{pgfscope}%
\begin{pgfscope}%
\pgfpathrectangle{\pgfqpoint{1.020000in}{0.880000in}}{\pgfqpoint{6.160000in}{6.160000in}}%
\pgfusepath{clip}%
\pgfsetbuttcap%
\pgfsetroundjoin%
\definecolor{currentfill}{rgb}{0.758539,0.832787,0.958408}%
\pgfsetfillcolor{currentfill}%
\pgfsetlinewidth{0.000000pt}%
\definecolor{currentstroke}{rgb}{0.000000,0.000000,0.000000}%
\pgfsetstrokecolor{currentstroke}%
\pgfsetdash{}{0pt}%
\pgfpathmoveto{\pgfqpoint{3.313833in}{3.509557in}}%
\pgfpathlineto{\pgfqpoint{3.323541in}{3.506088in}}%
\pgfpathlineto{\pgfqpoint{3.333264in}{3.503436in}}%
\pgfpathlineto{\pgfqpoint{3.367389in}{3.527629in}}%
\pgfpathlineto{\pgfqpoint{3.401514in}{3.550892in}}%
\pgfpathlineto{\pgfqpoint{3.391742in}{3.550770in}}%
\pgfpathlineto{\pgfqpoint{3.381986in}{3.551440in}}%
\pgfpathlineto{\pgfqpoint{3.347910in}{3.530948in}}%
\pgfpathlineto{\pgfqpoint{3.313833in}{3.509557in}}%
\pgfpathclose%
\pgfusepath{fill}%
\end{pgfscope}%
\begin{pgfscope}%
\pgfpathrectangle{\pgfqpoint{1.020000in}{0.880000in}}{\pgfqpoint{6.160000in}{6.160000in}}%
\pgfusepath{clip}%
\pgfsetbuttcap%
\pgfsetroundjoin%
\definecolor{currentfill}{rgb}{0.608547,0.735725,0.999354}%
\pgfsetfillcolor{currentfill}%
\pgfsetlinewidth{0.000000pt}%
\definecolor{currentstroke}{rgb}{0.000000,0.000000,0.000000}%
\pgfsetstrokecolor{currentstroke}%
\pgfsetdash{}{0pt}%
\pgfpathmoveto{\pgfqpoint{2.816805in}{3.261389in}}%
\pgfpathlineto{\pgfqpoint{2.826217in}{3.240433in}}%
\pgfpathlineto{\pgfqpoint{2.835633in}{3.220360in}}%
\pgfpathlineto{\pgfqpoint{2.869913in}{3.236194in}}%
\pgfpathlineto{\pgfqpoint{2.904168in}{3.253111in}}%
\pgfpathlineto{\pgfqpoint{2.894711in}{3.271725in}}%
\pgfpathlineto{\pgfqpoint{2.885259in}{3.291329in}}%
\pgfpathlineto{\pgfqpoint{2.851045in}{3.275885in}}%
\pgfpathlineto{\pgfqpoint{2.816805in}{3.261389in}}%
\pgfpathclose%
\pgfusepath{fill}%
\end{pgfscope}%
\begin{pgfscope}%
\pgfpathrectangle{\pgfqpoint{1.020000in}{0.880000in}}{\pgfqpoint{6.160000in}{6.160000in}}%
\pgfusepath{clip}%
\pgfsetbuttcap%
\pgfsetroundjoin%
\definecolor{currentfill}{rgb}{0.363461,0.484784,0.901019}%
\pgfsetfillcolor{currentfill}%
\pgfsetlinewidth{0.000000pt}%
\definecolor{currentstroke}{rgb}{0.000000,0.000000,0.000000}%
\pgfsetstrokecolor{currentstroke}%
\pgfsetdash{}{0pt}%
\pgfpathmoveto{\pgfqpoint{5.020768in}{2.843207in}}%
\pgfpathlineto{\pgfqpoint{5.032143in}{2.824300in}}%
\pgfpathlineto{\pgfqpoint{5.043537in}{2.805054in}}%
\pgfpathlineto{\pgfqpoint{5.077180in}{2.793303in}}%
\pgfpathlineto{\pgfqpoint{5.110804in}{2.782908in}}%
\pgfpathlineto{\pgfqpoint{5.099353in}{2.801349in}}%
\pgfpathlineto{\pgfqpoint{5.087921in}{2.819545in}}%
\pgfpathlineto{\pgfqpoint{5.054354in}{2.830665in}}%
\pgfpathlineto{\pgfqpoint{5.020768in}{2.843207in}}%
\pgfpathclose%
\pgfusepath{fill}%
\end{pgfscope}%
\begin{pgfscope}%
\pgfpathrectangle{\pgfqpoint{1.020000in}{0.880000in}}{\pgfqpoint{6.160000in}{6.160000in}}%
\pgfusepath{clip}%
\pgfsetbuttcap%
\pgfsetroundjoin%
\definecolor{currentfill}{rgb}{0.419991,0.552989,0.942630}%
\pgfsetfillcolor{currentfill}%
\pgfsetlinewidth{0.000000pt}%
\definecolor{currentstroke}{rgb}{0.000000,0.000000,0.000000}%
\pgfsetstrokecolor{currentstroke}%
\pgfsetdash{}{0pt}%
\pgfpathmoveto{\pgfqpoint{4.863740in}{2.948311in}}%
\pgfpathlineto{\pgfqpoint{4.874971in}{2.928904in}}%
\pgfpathlineto{\pgfqpoint{4.886219in}{2.908814in}}%
\pgfpathlineto{\pgfqpoint{4.919889in}{2.890006in}}%
\pgfpathlineto{\pgfqpoint{4.953537in}{2.872826in}}%
\pgfpathlineto{\pgfqpoint{4.942235in}{2.892081in}}%
\pgfpathlineto{\pgfqpoint{4.930949in}{2.910808in}}%
\pgfpathlineto{\pgfqpoint{4.897356in}{2.928719in}}%
\pgfpathlineto{\pgfqpoint{4.863740in}{2.948311in}}%
\pgfpathclose%
\pgfusepath{fill}%
\end{pgfscope}%
\begin{pgfscope}%
\pgfpathrectangle{\pgfqpoint{1.020000in}{0.880000in}}{\pgfqpoint{6.160000in}{6.160000in}}%
\pgfusepath{clip}%
\pgfsetbuttcap%
\pgfsetroundjoin%
\definecolor{currentfill}{rgb}{0.328604,0.439712,0.869587}%
\pgfsetfillcolor{currentfill}%
\pgfsetlinewidth{0.000000pt}%
\definecolor{currentstroke}{rgb}{0.000000,0.000000,0.000000}%
\pgfsetstrokecolor{currentstroke}%
\pgfsetdash{}{0pt}%
\pgfpathmoveto{\pgfqpoint{5.178000in}{2.765886in}}%
\pgfpathlineto{\pgfqpoint{5.189529in}{2.747965in}}%
\pgfpathlineto{\pgfqpoint{5.201078in}{2.729918in}}%
\pgfpathlineto{\pgfqpoint{5.234709in}{2.723794in}}%
\pgfpathlineto{\pgfqpoint{5.268323in}{2.718643in}}%
\pgfpathlineto{\pgfqpoint{5.256716in}{2.736037in}}%
\pgfpathlineto{\pgfqpoint{5.245129in}{2.753355in}}%
\pgfpathlineto{\pgfqpoint{5.211573in}{2.759101in}}%
\pgfpathlineto{\pgfqpoint{5.178000in}{2.765886in}}%
\pgfpathclose%
\pgfusepath{fill}%
\end{pgfscope}%
\begin{pgfscope}%
\pgfpathrectangle{\pgfqpoint{1.020000in}{0.880000in}}{\pgfqpoint{6.160000in}{6.160000in}}%
\pgfusepath{clip}%
\pgfsetbuttcap%
\pgfsetroundjoin%
\definecolor{currentfill}{rgb}{0.733898,0.820018,0.970724}%
\pgfsetfillcolor{currentfill}%
\pgfsetlinewidth{0.000000pt}%
\definecolor{currentstroke}{rgb}{0.000000,0.000000,0.000000}%
\pgfsetstrokecolor{currentstroke}%
\pgfsetdash{}{0pt}%
\pgfpathmoveto{\pgfqpoint{3.245665in}{3.464936in}}%
\pgfpathlineto{\pgfqpoint{3.255329in}{3.458663in}}%
\pgfpathlineto{\pgfqpoint{3.265006in}{3.453215in}}%
\pgfpathlineto{\pgfqpoint{3.299136in}{3.478553in}}%
\pgfpathlineto{\pgfqpoint{3.333264in}{3.503436in}}%
\pgfpathlineto{\pgfqpoint{3.323541in}{3.506088in}}%
\pgfpathlineto{\pgfqpoint{3.313833in}{3.509557in}}%
\pgfpathlineto{\pgfqpoint{3.279751in}{3.487482in}}%
\pgfpathlineto{\pgfqpoint{3.245665in}{3.464936in}}%
\pgfpathclose%
\pgfusepath{fill}%
\end{pgfscope}%
\begin{pgfscope}%
\pgfpathrectangle{\pgfqpoint{1.020000in}{0.880000in}}{\pgfqpoint{6.160000in}{6.160000in}}%
\pgfusepath{clip}%
\pgfsetbuttcap%
\pgfsetroundjoin%
\definecolor{currentfill}{rgb}{0.835345,0.860514,0.898970}%
\pgfsetfillcolor{currentfill}%
\pgfsetlinewidth{0.000000pt}%
\definecolor{currentstroke}{rgb}{0.000000,0.000000,0.000000}%
\pgfsetstrokecolor{currentstroke}%
\pgfsetdash{}{0pt}%
\pgfpathmoveto{\pgfqpoint{3.674505in}{3.676628in}}%
\pgfpathlineto{\pgfqpoint{3.684535in}{3.685743in}}%
\pgfpathlineto{\pgfqpoint{3.694590in}{3.694597in}}%
\pgfpathlineto{\pgfqpoint{3.728770in}{3.701228in}}%
\pgfpathlineto{\pgfqpoint{3.762942in}{3.704926in}}%
\pgfpathlineto{\pgfqpoint{3.752818in}{3.695039in}}%
\pgfpathlineto{\pgfqpoint{3.742719in}{3.684822in}}%
\pgfpathlineto{\pgfqpoint{3.708616in}{3.682063in}}%
\pgfpathlineto{\pgfqpoint{3.674505in}{3.676628in}}%
\pgfpathclose%
\pgfusepath{fill}%
\end{pgfscope}%
\begin{pgfscope}%
\pgfpathrectangle{\pgfqpoint{1.020000in}{0.880000in}}{\pgfqpoint{6.160000in}{6.160000in}}%
\pgfusepath{clip}%
\pgfsetbuttcap%
\pgfsetroundjoin%
\definecolor{currentfill}{rgb}{0.271104,0.360011,0.807095}%
\pgfsetfillcolor{currentfill}%
\pgfsetlinewidth{0.000000pt}%
\definecolor{currentstroke}{rgb}{0.000000,0.000000,0.000000}%
\pgfsetstrokecolor{currentstroke}%
\pgfsetdash{}{0pt}%
\pgfpathmoveto{\pgfqpoint{6.010516in}{2.633481in}}%
\pgfpathlineto{\pgfqpoint{6.022865in}{2.618894in}}%
\pgfpathlineto{\pgfqpoint{6.035239in}{2.604358in}}%
\pgfpathlineto{\pgfqpoint{6.068696in}{2.607027in}}%
\pgfpathlineto{\pgfqpoint{6.102134in}{2.609727in}}%
\pgfpathlineto{\pgfqpoint{6.089703in}{2.624177in}}%
\pgfpathlineto{\pgfqpoint{6.077298in}{2.638677in}}%
\pgfpathlineto{\pgfqpoint{6.043917in}{2.636061in}}%
\pgfpathlineto{\pgfqpoint{6.010516in}{2.633481in}}%
\pgfpathclose%
\pgfusepath{fill}%
\end{pgfscope}%
\begin{pgfscope}%
\pgfpathrectangle{\pgfqpoint{1.020000in}{0.880000in}}{\pgfqpoint{6.160000in}{6.160000in}}%
\pgfusepath{clip}%
\pgfsetbuttcap%
\pgfsetroundjoin%
\definecolor{currentfill}{rgb}{0.275827,0.366717,0.812553}%
\pgfsetfillcolor{currentfill}%
\pgfsetlinewidth{0.000000pt}%
\definecolor{currentstroke}{rgb}{0.000000,0.000000,0.000000}%
\pgfsetstrokecolor{currentstroke}%
\pgfsetdash{}{0pt}%
\pgfpathmoveto{\pgfqpoint{5.785408in}{2.649225in}}%
\pgfpathlineto{\pgfqpoint{5.797536in}{2.634189in}}%
\pgfpathlineto{\pgfqpoint{5.809689in}{2.619207in}}%
\pgfpathlineto{\pgfqpoint{5.843210in}{2.621387in}}%
\pgfpathlineto{\pgfqpoint{5.876711in}{2.623664in}}%
\pgfpathlineto{\pgfqpoint{5.864501in}{2.638514in}}%
\pgfpathlineto{\pgfqpoint{5.852315in}{2.653417in}}%
\pgfpathlineto{\pgfqpoint{5.818871in}{2.651266in}}%
\pgfpathlineto{\pgfqpoint{5.785408in}{2.649225in}}%
\pgfpathclose%
\pgfusepath{fill}%
\end{pgfscope}%
\begin{pgfscope}%
\pgfpathrectangle{\pgfqpoint{1.020000in}{0.880000in}}{\pgfqpoint{6.160000in}{6.160000in}}%
\pgfusepath{clip}%
\pgfsetbuttcap%
\pgfsetroundjoin%
\definecolor{currentfill}{rgb}{0.266381,0.353304,0.801637}%
\pgfsetfillcolor{currentfill}%
\pgfsetlinewidth{0.000000pt}%
\definecolor{currentstroke}{rgb}{0.000000,0.000000,0.000000}%
\pgfsetstrokecolor{currentstroke}%
\pgfsetdash{}{0pt}%
\pgfpathmoveto{\pgfqpoint{6.235679in}{2.620752in}}%
\pgfpathlineto{\pgfqpoint{6.248247in}{2.606502in}}%
\pgfpathlineto{\pgfqpoint{6.260842in}{2.592298in}}%
\pgfpathlineto{\pgfqpoint{6.294233in}{2.595164in}}%
\pgfpathlineto{\pgfqpoint{6.327604in}{2.598041in}}%
\pgfpathlineto{\pgfqpoint{6.314954in}{2.612177in}}%
\pgfpathlineto{\pgfqpoint{6.302328in}{2.626358in}}%
\pgfpathlineto{\pgfqpoint{6.269014in}{2.623549in}}%
\pgfpathlineto{\pgfqpoint{6.235679in}{2.620752in}}%
\pgfpathclose%
\pgfusepath{fill}%
\end{pgfscope}%
\begin{pgfscope}%
\pgfpathrectangle{\pgfqpoint{1.020000in}{0.880000in}}{\pgfqpoint{6.160000in}{6.160000in}}%
\pgfusepath{clip}%
\pgfsetbuttcap%
\pgfsetroundjoin%
\definecolor{currentfill}{rgb}{0.289996,0.386836,0.828926}%
\pgfsetfillcolor{currentfill}%
\pgfsetlinewidth{0.000000pt}%
\definecolor{currentstroke}{rgb}{0.000000,0.000000,0.000000}%
\pgfsetstrokecolor{currentstroke}%
\pgfsetdash{}{0pt}%
\pgfpathmoveto{\pgfqpoint{5.560384in}{2.671960in}}%
\pgfpathlineto{\pgfqpoint{5.572293in}{2.656226in}}%
\pgfpathlineto{\pgfqpoint{5.584224in}{2.640536in}}%
\pgfpathlineto{\pgfqpoint{5.617802in}{2.641419in}}%
\pgfpathlineto{\pgfqpoint{5.651361in}{2.642571in}}%
\pgfpathlineto{\pgfqpoint{5.639371in}{2.658017in}}%
\pgfpathlineto{\pgfqpoint{5.627404in}{2.673510in}}%
\pgfpathlineto{\pgfqpoint{5.593903in}{2.672586in}}%
\pgfpathlineto{\pgfqpoint{5.560384in}{2.671960in}}%
\pgfpathclose%
\pgfusepath{fill}%
\end{pgfscope}%
\begin{pgfscope}%
\pgfpathrectangle{\pgfqpoint{1.020000in}{0.880000in}}{\pgfqpoint{6.160000in}{6.160000in}}%
\pgfusepath{clip}%
\pgfsetbuttcap%
\pgfsetroundjoin%
\definecolor{currentfill}{rgb}{0.835345,0.860514,0.898970}%
\pgfsetfillcolor{currentfill}%
\pgfsetlinewidth{0.000000pt}%
\definecolor{currentstroke}{rgb}{0.000000,0.000000,0.000000}%
\pgfsetstrokecolor{currentstroke}%
\pgfsetdash{}{0pt}%
\pgfpathmoveto{\pgfqpoint{3.899503in}{3.690064in}}%
\pgfpathlineto{\pgfqpoint{3.909796in}{3.699592in}}%
\pgfpathlineto{\pgfqpoint{3.920115in}{3.707896in}}%
\pgfpathlineto{\pgfqpoint{3.954280in}{3.696564in}}%
\pgfpathlineto{\pgfqpoint{3.988420in}{3.682348in}}%
\pgfpathlineto{\pgfqpoint{3.978030in}{3.674606in}}%
\pgfpathlineto{\pgfqpoint{3.967667in}{3.665647in}}%
\pgfpathlineto{\pgfqpoint{3.933597in}{3.679179in}}%
\pgfpathlineto{\pgfqpoint{3.899503in}{3.690064in}}%
\pgfpathclose%
\pgfusepath{fill}%
\end{pgfscope}%
\begin{pgfscope}%
\pgfpathrectangle{\pgfqpoint{1.020000in}{0.880000in}}{\pgfqpoint{6.160000in}{6.160000in}}%
\pgfusepath{clip}%
\pgfsetbuttcap%
\pgfsetroundjoin%
\definecolor{currentfill}{rgb}{0.597777,0.727330,0.999777}%
\pgfsetfillcolor{currentfill}%
\pgfsetlinewidth{0.000000pt}%
\definecolor{currentstroke}{rgb}{0.000000,0.000000,0.000000}%
\pgfsetstrokecolor{currentstroke}%
\pgfsetdash{}{0pt}%
\pgfpathmoveto{\pgfqpoint{2.748245in}{3.235346in}}%
\pgfpathlineto{\pgfqpoint{2.757615in}{3.213294in}}%
\pgfpathlineto{\pgfqpoint{2.766989in}{3.192020in}}%
\pgfpathlineto{\pgfqpoint{2.801325in}{3.205632in}}%
\pgfpathlineto{\pgfqpoint{2.835633in}{3.220360in}}%
\pgfpathlineto{\pgfqpoint{2.826217in}{3.240433in}}%
\pgfpathlineto{\pgfqpoint{2.816805in}{3.261389in}}%
\pgfpathlineto{\pgfqpoint{2.782539in}{3.247871in}}%
\pgfpathlineto{\pgfqpoint{2.748245in}{3.235346in}}%
\pgfpathclose%
\pgfusepath{fill}%
\end{pgfscope}%
\begin{pgfscope}%
\pgfpathrectangle{\pgfqpoint{1.020000in}{0.880000in}}{\pgfqpoint{6.160000in}{6.160000in}}%
\pgfusepath{clip}%
\pgfsetbuttcap%
\pgfsetroundjoin%
\definecolor{currentfill}{rgb}{0.753611,0.830233,0.960871}%
\pgfsetfillcolor{currentfill}%
\pgfsetlinewidth{0.000000pt}%
\definecolor{currentstroke}{rgb}{0.000000,0.000000,0.000000}%
\pgfsetstrokecolor{currentstroke}%
\pgfsetdash{}{0pt}%
\pgfpathmoveto{\pgfqpoint{4.213834in}{3.551363in}}%
\pgfpathlineto{\pgfqpoint{4.224491in}{3.550217in}}%
\pgfpathlineto{\pgfqpoint{4.235169in}{3.546968in}}%
\pgfpathlineto{\pgfqpoint{4.269184in}{3.515912in}}%
\pgfpathlineto{\pgfqpoint{4.303154in}{3.483840in}}%
\pgfpathlineto{\pgfqpoint{4.292419in}{3.488349in}}%
\pgfpathlineto{\pgfqpoint{4.281705in}{3.490953in}}%
\pgfpathlineto{\pgfqpoint{4.247791in}{3.521636in}}%
\pgfpathlineto{\pgfqpoint{4.213834in}{3.551363in}}%
\pgfpathclose%
\pgfusepath{fill}%
\end{pgfscope}%
\begin{pgfscope}%
\pgfpathrectangle{\pgfqpoint{1.020000in}{0.880000in}}{\pgfqpoint{6.160000in}{6.160000in}}%
\pgfusepath{clip}%
\pgfsetbuttcap%
\pgfsetroundjoin%
\definecolor{currentfill}{rgb}{0.713852,0.808857,0.979386}%
\pgfsetfillcolor{currentfill}%
\pgfsetlinewidth{0.000000pt}%
\definecolor{currentstroke}{rgb}{0.000000,0.000000,0.000000}%
\pgfsetstrokecolor{currentstroke}%
\pgfsetdash{}{0pt}%
\pgfpathmoveto{\pgfqpoint{3.177473in}{3.419246in}}%
\pgfpathlineto{\pgfqpoint{3.187095in}{3.410255in}}%
\pgfpathlineto{\pgfqpoint{3.196728in}{3.402082in}}%
\pgfpathlineto{\pgfqpoint{3.230870in}{3.427652in}}%
\pgfpathlineto{\pgfqpoint{3.265006in}{3.453215in}}%
\pgfpathlineto{\pgfqpoint{3.255329in}{3.458663in}}%
\pgfpathlineto{\pgfqpoint{3.245665in}{3.464936in}}%
\pgfpathlineto{\pgfqpoint{3.211573in}{3.442124in}}%
\pgfpathlineto{\pgfqpoint{3.177473in}{3.419246in}}%
\pgfpathclose%
\pgfusepath{fill}%
\end{pgfscope}%
\begin{pgfscope}%
\pgfpathrectangle{\pgfqpoint{1.020000in}{0.880000in}}{\pgfqpoint{6.160000in}{6.160000in}}%
\pgfusepath{clip}%
\pgfsetbuttcap%
\pgfsetroundjoin%
\definecolor{currentfill}{rgb}{0.489246,0.627536,0.976896}%
\pgfsetfillcolor{currentfill}%
\pgfsetlinewidth{0.000000pt}%
\definecolor{currentstroke}{rgb}{0.000000,0.000000,0.000000}%
\pgfsetstrokecolor{currentstroke}%
\pgfsetdash{}{0pt}%
\pgfpathmoveto{\pgfqpoint{4.706798in}{3.082168in}}%
\pgfpathlineto{\pgfqpoint{4.717896in}{3.063394in}}%
\pgfpathlineto{\pgfqpoint{4.729008in}{3.043464in}}%
\pgfpathlineto{\pgfqpoint{4.762735in}{3.017216in}}%
\pgfpathlineto{\pgfqpoint{4.796431in}{2.992577in}}%
\pgfpathlineto{\pgfqpoint{4.785267in}{3.011853in}}%
\pgfpathlineto{\pgfqpoint{4.774119in}{3.030195in}}%
\pgfpathlineto{\pgfqpoint{4.740474in}{3.055364in}}%
\pgfpathlineto{\pgfqpoint{4.706798in}{3.082168in}}%
\pgfpathclose%
\pgfusepath{fill}%
\end{pgfscope}%
\begin{pgfscope}%
\pgfpathrectangle{\pgfqpoint{1.020000in}{0.880000in}}{\pgfqpoint{6.160000in}{6.160000in}}%
\pgfusepath{clip}%
\pgfsetbuttcap%
\pgfsetroundjoin%
\definecolor{currentfill}{rgb}{0.304174,0.406945,0.845263}%
\pgfsetfillcolor{currentfill}%
\pgfsetlinewidth{0.000000pt}%
\definecolor{currentstroke}{rgb}{0.000000,0.000000,0.000000}%
\pgfsetstrokecolor{currentstroke}%
\pgfsetdash{}{0pt}%
\pgfpathmoveto{\pgfqpoint{5.335502in}{2.710952in}}%
\pgfpathlineto{\pgfqpoint{5.347190in}{2.694059in}}%
\pgfpathlineto{\pgfqpoint{5.358900in}{2.677152in}}%
\pgfpathlineto{\pgfqpoint{5.392523in}{2.674943in}}%
\pgfpathlineto{\pgfqpoint{5.426130in}{2.673352in}}%
\pgfpathlineto{\pgfqpoint{5.414362in}{2.689791in}}%
\pgfpathlineto{\pgfqpoint{5.402615in}{2.706238in}}%
\pgfpathlineto{\pgfqpoint{5.369067in}{2.708260in}}%
\pgfpathlineto{\pgfqpoint{5.335502in}{2.710952in}}%
\pgfpathclose%
\pgfusepath{fill}%
\end{pgfscope}%
\begin{pgfscope}%
\pgfpathrectangle{\pgfqpoint{1.020000in}{0.880000in}}{\pgfqpoint{6.160000in}{6.160000in}}%
\pgfusepath{clip}%
\pgfsetbuttcap%
\pgfsetroundjoin%
\definecolor{currentfill}{rgb}{0.630089,0.752516,0.998508}%
\pgfsetfillcolor{currentfill}%
\pgfsetlinewidth{0.000000pt}%
\definecolor{currentstroke}{rgb}{0.000000,0.000000,0.000000}%
\pgfsetstrokecolor{currentstroke}%
\pgfsetdash{}{0pt}%
\pgfpathmoveto{\pgfqpoint{4.460331in}{3.332231in}}%
\pgfpathlineto{\pgfqpoint{4.471223in}{3.319801in}}%
\pgfpathlineto{\pgfqpoint{4.482131in}{3.305455in}}%
\pgfpathlineto{\pgfqpoint{4.515983in}{3.271423in}}%
\pgfpathlineto{\pgfqpoint{4.549792in}{3.238116in}}%
\pgfpathlineto{\pgfqpoint{4.538836in}{3.252785in}}%
\pgfpathlineto{\pgfqpoint{4.527895in}{3.265813in}}%
\pgfpathlineto{\pgfqpoint{4.494134in}{3.298664in}}%
\pgfpathlineto{\pgfqpoint{4.460331in}{3.332231in}}%
\pgfpathclose%
\pgfusepath{fill}%
\end{pgfscope}%
\begin{pgfscope}%
\pgfpathrectangle{\pgfqpoint{1.020000in}{0.880000in}}{\pgfqpoint{6.160000in}{6.160000in}}%
\pgfusepath{clip}%
\pgfsetbuttcap%
\pgfsetroundjoin%
\definecolor{currentfill}{rgb}{0.809329,0.852974,0.922323}%
\pgfsetfillcolor{currentfill}%
\pgfsetlinewidth{0.000000pt}%
\definecolor{currentstroke}{rgb}{0.000000,0.000000,0.000000}%
\pgfsetstrokecolor{currentstroke}%
\pgfsetdash{}{0pt}%
\pgfpathmoveto{\pgfqpoint{4.056613in}{3.646005in}}%
\pgfpathlineto{\pgfqpoint{4.067097in}{3.651221in}}%
\pgfpathlineto{\pgfqpoint{4.077606in}{3.654589in}}%
\pgfpathlineto{\pgfqpoint{4.111720in}{3.631733in}}%
\pgfpathlineto{\pgfqpoint{4.145797in}{3.606716in}}%
\pgfpathlineto{\pgfqpoint{4.135224in}{3.604543in}}%
\pgfpathlineto{\pgfqpoint{4.124675in}{3.600624in}}%
\pgfpathlineto{\pgfqpoint{4.090662in}{3.624320in}}%
\pgfpathlineto{\pgfqpoint{4.056613in}{3.646005in}}%
\pgfpathclose%
\pgfusepath{fill}%
\end{pgfscope}%
\begin{pgfscope}%
\pgfpathrectangle{\pgfqpoint{1.020000in}{0.880000in}}{\pgfqpoint{6.160000in}{6.160000in}}%
\pgfusepath{clip}%
\pgfsetbuttcap%
\pgfsetroundjoin%
\definecolor{currentfill}{rgb}{0.688188,0.793178,0.988038}%
\pgfsetfillcolor{currentfill}%
\pgfsetlinewidth{0.000000pt}%
\definecolor{currentstroke}{rgb}{0.000000,0.000000,0.000000}%
\pgfsetstrokecolor{currentstroke}%
\pgfsetdash{}{0pt}%
\pgfpathmoveto{\pgfqpoint{3.109242in}{3.374021in}}%
\pgfpathlineto{\pgfqpoint{3.118823in}{3.362474in}}%
\pgfpathlineto{\pgfqpoint{3.128415in}{3.351723in}}%
\pgfpathlineto{\pgfqpoint{3.162577in}{3.376709in}}%
\pgfpathlineto{\pgfqpoint{3.196728in}{3.402082in}}%
\pgfpathlineto{\pgfqpoint{3.187095in}{3.410255in}}%
\pgfpathlineto{\pgfqpoint{3.177473in}{3.419246in}}%
\pgfpathlineto{\pgfqpoint{3.143363in}{3.396487in}}%
\pgfpathlineto{\pgfqpoint{3.109242in}{3.374021in}}%
\pgfpathclose%
\pgfusepath{fill}%
\end{pgfscope}%
\begin{pgfscope}%
\pgfpathrectangle{\pgfqpoint{1.020000in}{0.880000in}}{\pgfqpoint{6.160000in}{6.160000in}}%
\pgfusepath{clip}%
\pgfsetbuttcap%
\pgfsetroundjoin%
\definecolor{currentfill}{rgb}{0.831148,0.859513,0.903110}%
\pgfsetfillcolor{currentfill}%
\pgfsetlinewidth{0.000000pt}%
\definecolor{currentstroke}{rgb}{0.000000,0.000000,0.000000}%
\pgfsetstrokecolor{currentstroke}%
\pgfsetdash{}{0pt}%
\pgfpathmoveto{\pgfqpoint{3.606266in}{3.658031in}}%
\pgfpathlineto{\pgfqpoint{3.616230in}{3.665551in}}%
\pgfpathlineto{\pgfqpoint{3.626219in}{3.672889in}}%
\pgfpathlineto{\pgfqpoint{3.660406in}{3.685113in}}%
\pgfpathlineto{\pgfqpoint{3.694590in}{3.694597in}}%
\pgfpathlineto{\pgfqpoint{3.684535in}{3.685743in}}%
\pgfpathlineto{\pgfqpoint{3.674505in}{3.676628in}}%
\pgfpathlineto{\pgfqpoint{3.640387in}{3.668584in}}%
\pgfpathlineto{\pgfqpoint{3.606266in}{3.658031in}}%
\pgfpathclose%
\pgfusepath{fill}%
\end{pgfscope}%
\begin{pgfscope}%
\pgfpathrectangle{\pgfqpoint{1.020000in}{0.880000in}}{\pgfqpoint{6.160000in}{6.160000in}}%
\pgfusepath{clip}%
\pgfsetbuttcap%
\pgfsetroundjoin%
\definecolor{currentfill}{rgb}{0.586921,0.718121,0.998874}%
\pgfsetfillcolor{currentfill}%
\pgfsetlinewidth{0.000000pt}%
\definecolor{currentstroke}{rgb}{0.000000,0.000000,0.000000}%
\pgfsetstrokecolor{currentstroke}%
\pgfsetdash{}{0pt}%
\pgfpathmoveto{\pgfqpoint{2.679571in}{3.213280in}}%
\pgfpathlineto{\pgfqpoint{2.688896in}{3.190357in}}%
\pgfpathlineto{\pgfqpoint{2.698228in}{3.168112in}}%
\pgfpathlineto{\pgfqpoint{2.732624in}{3.179518in}}%
\pgfpathlineto{\pgfqpoint{2.766989in}{3.192020in}}%
\pgfpathlineto{\pgfqpoint{2.757615in}{3.213294in}}%
\pgfpathlineto{\pgfqpoint{2.748245in}{3.235346in}}%
\pgfpathlineto{\pgfqpoint{2.713923in}{3.223818in}}%
\pgfpathlineto{\pgfqpoint{2.679571in}{3.213280in}}%
\pgfpathclose%
\pgfusepath{fill}%
\end{pgfscope}%
\begin{pgfscope}%
\pgfpathrectangle{\pgfqpoint{1.020000in}{0.880000in}}{\pgfqpoint{6.160000in}{6.160000in}}%
\pgfusepath{clip}%
\pgfsetbuttcap%
\pgfsetroundjoin%
\definecolor{currentfill}{rgb}{0.661968,0.775491,0.993937}%
\pgfsetfillcolor{currentfill}%
\pgfsetlinewidth{0.000000pt}%
\definecolor{currentstroke}{rgb}{0.000000,0.000000,0.000000}%
\pgfsetstrokecolor{currentstroke}%
\pgfsetdash{}{0pt}%
\pgfpathmoveto{\pgfqpoint{3.040958in}{3.330590in}}%
\pgfpathlineto{\pgfqpoint{3.050500in}{3.316704in}}%
\pgfpathlineto{\pgfqpoint{3.060052in}{3.303582in}}%
\pgfpathlineto{\pgfqpoint{3.094241in}{3.327297in}}%
\pgfpathlineto{\pgfqpoint{3.128415in}{3.351723in}}%
\pgfpathlineto{\pgfqpoint{3.118823in}{3.362474in}}%
\pgfpathlineto{\pgfqpoint{3.109242in}{3.374021in}}%
\pgfpathlineto{\pgfqpoint{3.075108in}{3.352008in}}%
\pgfpathlineto{\pgfqpoint{3.040958in}{3.330590in}}%
\pgfpathclose%
\pgfusepath{fill}%
\end{pgfscope}%
\begin{pgfscope}%
\pgfpathrectangle{\pgfqpoint{1.020000in}{0.880000in}}{\pgfqpoint{6.160000in}{6.160000in}}%
\pgfusepath{clip}%
\pgfsetbuttcap%
\pgfsetroundjoin%
\definecolor{currentfill}{rgb}{0.271104,0.360011,0.807095}%
\pgfsetfillcolor{currentfill}%
\pgfsetlinewidth{0.000000pt}%
\definecolor{currentstroke}{rgb}{0.000000,0.000000,0.000000}%
\pgfsetstrokecolor{currentstroke}%
\pgfsetdash{}{0pt}%
\pgfpathmoveto{\pgfqpoint{5.943654in}{2.628453in}}%
\pgfpathlineto{\pgfqpoint{5.955946in}{2.613769in}}%
\pgfpathlineto{\pgfqpoint{5.968262in}{2.599138in}}%
\pgfpathlineto{\pgfqpoint{6.001761in}{2.601726in}}%
\pgfpathlineto{\pgfqpoint{6.035239in}{2.604358in}}%
\pgfpathlineto{\pgfqpoint{6.022865in}{2.618894in}}%
\pgfpathlineto{\pgfqpoint{6.010516in}{2.633481in}}%
\pgfpathlineto{\pgfqpoint{5.977095in}{2.630942in}}%
\pgfpathlineto{\pgfqpoint{5.943654in}{2.628453in}}%
\pgfpathclose%
\pgfusepath{fill}%
\end{pgfscope}%
\begin{pgfscope}%
\pgfpathrectangle{\pgfqpoint{1.020000in}{0.880000in}}{\pgfqpoint{6.160000in}{6.160000in}}%
\pgfusepath{clip}%
\pgfsetbuttcap%
\pgfsetroundjoin%
\definecolor{currentfill}{rgb}{0.280550,0.373423,0.818011}%
\pgfsetfillcolor{currentfill}%
\pgfsetlinewidth{0.000000pt}%
\definecolor{currentstroke}{rgb}{0.000000,0.000000,0.000000}%
\pgfsetstrokecolor{currentstroke}%
\pgfsetdash{}{0pt}%
\pgfpathmoveto{\pgfqpoint{5.718422in}{2.645547in}}%
\pgfpathlineto{\pgfqpoint{5.730493in}{2.630351in}}%
\pgfpathlineto{\pgfqpoint{5.742589in}{2.615209in}}%
\pgfpathlineto{\pgfqpoint{5.776149in}{2.617141in}}%
\pgfpathlineto{\pgfqpoint{5.809689in}{2.619207in}}%
\pgfpathlineto{\pgfqpoint{5.797536in}{2.634189in}}%
\pgfpathlineto{\pgfqpoint{5.785408in}{2.649225in}}%
\pgfpathlineto{\pgfqpoint{5.751924in}{2.647311in}}%
\pgfpathlineto{\pgfqpoint{5.718422in}{2.645547in}}%
\pgfpathclose%
\pgfusepath{fill}%
\end{pgfscope}%
\begin{pgfscope}%
\pgfpathrectangle{\pgfqpoint{1.020000in}{0.880000in}}{\pgfqpoint{6.160000in}{6.160000in}}%
\pgfusepath{clip}%
\pgfsetbuttcap%
\pgfsetroundjoin%
\definecolor{currentfill}{rgb}{0.266381,0.353304,0.801637}%
\pgfsetfillcolor{currentfill}%
\pgfsetlinewidth{0.000000pt}%
\definecolor{currentstroke}{rgb}{0.000000,0.000000,0.000000}%
\pgfsetstrokecolor{currentstroke}%
\pgfsetdash{}{0pt}%
\pgfpathmoveto{\pgfqpoint{6.168947in}{2.615202in}}%
\pgfpathlineto{\pgfqpoint{6.181459in}{2.600879in}}%
\pgfpathlineto{\pgfqpoint{6.193997in}{2.586604in}}%
\pgfpathlineto{\pgfqpoint{6.227429in}{2.589444in}}%
\pgfpathlineto{\pgfqpoint{6.260842in}{2.592298in}}%
\pgfpathlineto{\pgfqpoint{6.248247in}{2.606502in}}%
\pgfpathlineto{\pgfqpoint{6.235679in}{2.620752in}}%
\pgfpathlineto{\pgfqpoint{6.202323in}{2.617969in}}%
\pgfpathlineto{\pgfqpoint{6.168947in}{2.615202in}}%
\pgfpathclose%
\pgfusepath{fill}%
\end{pgfscope}%
\begin{pgfscope}%
\pgfpathrectangle{\pgfqpoint{1.020000in}{0.880000in}}{\pgfqpoint{6.160000in}{6.160000in}}%
\pgfusepath{clip}%
\pgfsetbuttcap%
\pgfsetroundjoin%
\definecolor{currentfill}{rgb}{0.383662,0.510183,0.917831}%
\pgfsetfillcolor{currentfill}%
\pgfsetlinewidth{0.000000pt}%
\definecolor{currentstroke}{rgb}{0.000000,0.000000,0.000000}%
\pgfsetstrokecolor{currentstroke}%
\pgfsetdash{}{0pt}%
\pgfpathmoveto{\pgfqpoint{4.953537in}{2.872826in}}%
\pgfpathlineto{\pgfqpoint{4.964856in}{2.853077in}}%
\pgfpathlineto{\pgfqpoint{4.976192in}{2.832873in}}%
\pgfpathlineto{\pgfqpoint{5.009875in}{2.818225in}}%
\pgfpathlineto{\pgfqpoint{5.043537in}{2.805054in}}%
\pgfpathlineto{\pgfqpoint{5.032143in}{2.824300in}}%
\pgfpathlineto{\pgfqpoint{5.020768in}{2.843207in}}%
\pgfpathlineto{\pgfqpoint{4.987163in}{2.857242in}}%
\pgfpathlineto{\pgfqpoint{4.953537in}{2.872826in}}%
\pgfpathclose%
\pgfusepath{fill}%
\end{pgfscope}%
\begin{pgfscope}%
\pgfpathrectangle{\pgfqpoint{1.020000in}{0.880000in}}{\pgfqpoint{6.160000in}{6.160000in}}%
\pgfusepath{clip}%
\pgfsetbuttcap%
\pgfsetroundjoin%
\definecolor{currentfill}{rgb}{0.343278,0.459354,0.884122}%
\pgfsetfillcolor{currentfill}%
\pgfsetlinewidth{0.000000pt}%
\definecolor{currentstroke}{rgb}{0.000000,0.000000,0.000000}%
\pgfsetstrokecolor{currentstroke}%
\pgfsetdash{}{0pt}%
\pgfpathmoveto{\pgfqpoint{5.110804in}{2.782908in}}%
\pgfpathlineto{\pgfqpoint{5.122274in}{2.764249in}}%
\pgfpathlineto{\pgfqpoint{5.133764in}{2.745401in}}%
\pgfpathlineto{\pgfqpoint{5.167429in}{2.737095in}}%
\pgfpathlineto{\pgfqpoint{5.201078in}{2.729918in}}%
\pgfpathlineto{\pgfqpoint{5.189529in}{2.747965in}}%
\pgfpathlineto{\pgfqpoint{5.178000in}{2.765886in}}%
\pgfpathlineto{\pgfqpoint{5.144411in}{2.773795in}}%
\pgfpathlineto{\pgfqpoint{5.110804in}{2.782908in}}%
\pgfpathclose%
\pgfusepath{fill}%
\end{pgfscope}%
\begin{pgfscope}%
\pgfpathrectangle{\pgfqpoint{1.020000in}{0.880000in}}{\pgfqpoint{6.160000in}{6.160000in}}%
\pgfusepath{clip}%
\pgfsetbuttcap%
\pgfsetroundjoin%
\definecolor{currentfill}{rgb}{0.289996,0.386836,0.828926}%
\pgfsetfillcolor{currentfill}%
\pgfsetlinewidth{0.000000pt}%
\definecolor{currentstroke}{rgb}{0.000000,0.000000,0.000000}%
\pgfsetstrokecolor{currentstroke}%
\pgfsetdash{}{0pt}%
\pgfpathmoveto{\pgfqpoint{5.493292in}{2.671776in}}%
\pgfpathlineto{\pgfqpoint{5.505142in}{2.655739in}}%
\pgfpathlineto{\pgfqpoint{5.517015in}{2.639740in}}%
\pgfpathlineto{\pgfqpoint{5.550629in}{2.639962in}}%
\pgfpathlineto{\pgfqpoint{5.584224in}{2.640536in}}%
\pgfpathlineto{\pgfqpoint{5.572293in}{2.656226in}}%
\pgfpathlineto{\pgfqpoint{5.560384in}{2.671960in}}%
\pgfpathlineto{\pgfqpoint{5.526847in}{2.671674in}}%
\pgfpathlineto{\pgfqpoint{5.493292in}{2.671776in}}%
\pgfpathclose%
\pgfusepath{fill}%
\end{pgfscope}%
\begin{pgfscope}%
\pgfpathrectangle{\pgfqpoint{1.020000in}{0.880000in}}{\pgfqpoint{6.160000in}{6.160000in}}%
\pgfusepath{clip}%
\pgfsetbuttcap%
\pgfsetroundjoin%
\definecolor{currentfill}{rgb}{0.581486,0.713451,0.998314}%
\pgfsetfillcolor{currentfill}%
\pgfsetlinewidth{0.000000pt}%
\definecolor{currentstroke}{rgb}{0.000000,0.000000,0.000000}%
\pgfsetstrokecolor{currentstroke}%
\pgfsetdash{}{0pt}%
\pgfpathmoveto{\pgfqpoint{4.549792in}{3.238116in}}%
\pgfpathlineto{\pgfqpoint{4.560764in}{3.221764in}}%
\pgfpathlineto{\pgfqpoint{4.571749in}{3.203712in}}%
\pgfpathlineto{\pgfqpoint{4.605567in}{3.171452in}}%
\pgfpathlineto{\pgfqpoint{4.639347in}{3.140335in}}%
\pgfpathlineto{\pgfqpoint{4.628312in}{3.158154in}}%
\pgfpathlineto{\pgfqpoint{4.617293in}{3.174544in}}%
\pgfpathlineto{\pgfqpoint{4.583562in}{3.205757in}}%
\pgfpathlineto{\pgfqpoint{4.549792in}{3.238116in}}%
\pgfpathclose%
\pgfusepath{fill}%
\end{pgfscope}%
\begin{pgfscope}%
\pgfpathrectangle{\pgfqpoint{1.020000in}{0.880000in}}{\pgfqpoint{6.160000in}{6.160000in}}%
\pgfusepath{clip}%
\pgfsetbuttcap%
\pgfsetroundjoin%
\definecolor{currentfill}{rgb}{0.640828,0.760752,0.997846}%
\pgfsetfillcolor{currentfill}%
\pgfsetlinewidth{0.000000pt}%
\definecolor{currentstroke}{rgb}{0.000000,0.000000,0.000000}%
\pgfsetstrokecolor{currentstroke}%
\pgfsetdash{}{0pt}%
\pgfpathmoveto{\pgfqpoint{2.972605in}{3.290024in}}%
\pgfpathlineto{\pgfqpoint{2.982108in}{3.274054in}}%
\pgfpathlineto{\pgfqpoint{2.991620in}{3.258804in}}%
\pgfpathlineto{\pgfqpoint{3.025846in}{3.280714in}}%
\pgfpathlineto{\pgfqpoint{3.060052in}{3.303582in}}%
\pgfpathlineto{\pgfqpoint{3.050500in}{3.316704in}}%
\pgfpathlineto{\pgfqpoint{3.040958in}{3.330590in}}%
\pgfpathlineto{\pgfqpoint{3.006791in}{3.309893in}}%
\pgfpathlineto{\pgfqpoint{2.972605in}{3.290024in}}%
\pgfpathclose%
\pgfusepath{fill}%
\end{pgfscope}%
\begin{pgfscope}%
\pgfpathrectangle{\pgfqpoint{1.020000in}{0.880000in}}{\pgfqpoint{6.160000in}{6.160000in}}%
\pgfusepath{clip}%
\pgfsetbuttcap%
\pgfsetroundjoin%
\definecolor{currentfill}{rgb}{0.446431,0.582356,0.957373}%
\pgfsetfillcolor{currentfill}%
\pgfsetlinewidth{0.000000pt}%
\definecolor{currentstroke}{rgb}{0.000000,0.000000,0.000000}%
\pgfsetstrokecolor{currentstroke}%
\pgfsetdash{}{0pt}%
\pgfpathmoveto{\pgfqpoint{4.796431in}{2.992577in}}%
\pgfpathlineto{\pgfqpoint{4.807609in}{2.972395in}}%
\pgfpathlineto{\pgfqpoint{4.818803in}{2.951348in}}%
\pgfpathlineto{\pgfqpoint{4.852524in}{2.929263in}}%
\pgfpathlineto{\pgfqpoint{4.886219in}{2.908814in}}%
\pgfpathlineto{\pgfqpoint{4.874971in}{2.928904in}}%
\pgfpathlineto{\pgfqpoint{4.863740in}{2.948311in}}%
\pgfpathlineto{\pgfqpoint{4.830099in}{2.969599in}}%
\pgfpathlineto{\pgfqpoint{4.796431in}{2.992577in}}%
\pgfpathclose%
\pgfusepath{fill}%
\end{pgfscope}%
\begin{pgfscope}%
\pgfpathrectangle{\pgfqpoint{1.020000in}{0.880000in}}{\pgfqpoint{6.160000in}{6.160000in}}%
\pgfusepath{clip}%
\pgfsetbuttcap%
\pgfsetroundjoin%
\definecolor{currentfill}{rgb}{0.718985,0.811993,0.977656}%
\pgfsetfillcolor{currentfill}%
\pgfsetlinewidth{0.000000pt}%
\definecolor{currentstroke}{rgb}{0.000000,0.000000,0.000000}%
\pgfsetstrokecolor{currentstroke}%
\pgfsetdash{}{0pt}%
\pgfpathmoveto{\pgfqpoint{4.303154in}{3.483840in}}%
\pgfpathlineto{\pgfqpoint{4.313909in}{3.477240in}}%
\pgfpathlineto{\pgfqpoint{4.324682in}{3.468391in}}%
\pgfpathlineto{\pgfqpoint{4.358662in}{3.434649in}}%
\pgfpathlineto{\pgfqpoint{4.392596in}{3.400505in}}%
\pgfpathlineto{\pgfqpoint{4.381771in}{3.410214in}}%
\pgfpathlineto{\pgfqpoint{4.370963in}{3.417925in}}%
\pgfpathlineto{\pgfqpoint{4.337081in}{3.451073in}}%
\pgfpathlineto{\pgfqpoint{4.303154in}{3.483840in}}%
\pgfpathclose%
\pgfusepath{fill}%
\end{pgfscope}%
\begin{pgfscope}%
\pgfpathrectangle{\pgfqpoint{1.020000in}{0.880000in}}{\pgfqpoint{6.160000in}{6.160000in}}%
\pgfusepath{clip}%
\pgfsetbuttcap%
\pgfsetroundjoin%
\definecolor{currentfill}{rgb}{0.847365,0.862472,0.885540}%
\pgfsetfillcolor{currentfill}%
\pgfsetlinewidth{0.000000pt}%
\definecolor{currentstroke}{rgb}{0.000000,0.000000,0.000000}%
\pgfsetstrokecolor{currentstroke}%
\pgfsetdash{}{0pt}%
\pgfpathmoveto{\pgfqpoint{3.831254in}{3.703383in}}%
\pgfpathlineto{\pgfqpoint{3.841475in}{3.712976in}}%
\pgfpathlineto{\pgfqpoint{3.851724in}{3.721365in}}%
\pgfpathlineto{\pgfqpoint{3.885929in}{3.716199in}}%
\pgfpathlineto{\pgfqpoint{3.920115in}{3.707896in}}%
\pgfpathlineto{\pgfqpoint{3.909796in}{3.699592in}}%
\pgfpathlineto{\pgfqpoint{3.899503in}{3.690064in}}%
\pgfpathlineto{\pgfqpoint{3.865388in}{3.698166in}}%
\pgfpathlineto{\pgfqpoint{3.831254in}{3.703383in}}%
\pgfpathclose%
\pgfusepath{fill}%
\end{pgfscope}%
\begin{pgfscope}%
\pgfpathrectangle{\pgfqpoint{1.020000in}{0.880000in}}{\pgfqpoint{6.160000in}{6.160000in}}%
\pgfusepath{clip}%
\pgfsetbuttcap%
\pgfsetroundjoin%
\definecolor{currentfill}{rgb}{0.822420,0.856898,0.910795}%
\pgfsetfillcolor{currentfill}%
\pgfsetlinewidth{0.000000pt}%
\definecolor{currentstroke}{rgb}{0.000000,0.000000,0.000000}%
\pgfsetstrokecolor{currentstroke}%
\pgfsetdash{}{0pt}%
\pgfpathmoveto{\pgfqpoint{3.538016in}{3.629936in}}%
\pgfpathlineto{\pgfqpoint{3.547918in}{3.635437in}}%
\pgfpathlineto{\pgfqpoint{3.557844in}{3.640839in}}%
\pgfpathlineto{\pgfqpoint{3.592031in}{3.658072in}}%
\pgfpathlineto{\pgfqpoint{3.626219in}{3.672889in}}%
\pgfpathlineto{\pgfqpoint{3.616230in}{3.665551in}}%
\pgfpathlineto{\pgfqpoint{3.606266in}{3.658031in}}%
\pgfpathlineto{\pgfqpoint{3.572141in}{3.645097in}}%
\pgfpathlineto{\pgfqpoint{3.538016in}{3.629936in}}%
\pgfpathclose%
\pgfusepath{fill}%
\end{pgfscope}%
\begin{pgfscope}%
\pgfpathrectangle{\pgfqpoint{1.020000in}{0.880000in}}{\pgfqpoint{6.160000in}{6.160000in}}%
\pgfusepath{clip}%
\pgfsetbuttcap%
\pgfsetroundjoin%
\definecolor{currentfill}{rgb}{0.313946,0.420052,0.854993}%
\pgfsetfillcolor{currentfill}%
\pgfsetlinewidth{0.000000pt}%
\definecolor{currentstroke}{rgb}{0.000000,0.000000,0.000000}%
\pgfsetstrokecolor{currentstroke}%
\pgfsetdash{}{0pt}%
\pgfpathmoveto{\pgfqpoint{5.268323in}{2.718643in}}%
\pgfpathlineto{\pgfqpoint{5.279952in}{2.701190in}}%
\pgfpathlineto{\pgfqpoint{5.291602in}{2.683696in}}%
\pgfpathlineto{\pgfqpoint{5.325259in}{2.680047in}}%
\pgfpathlineto{\pgfqpoint{5.358900in}{2.677152in}}%
\pgfpathlineto{\pgfqpoint{5.347190in}{2.694059in}}%
\pgfpathlineto{\pgfqpoint{5.335502in}{2.710952in}}%
\pgfpathlineto{\pgfqpoint{5.301921in}{2.714388in}}%
\pgfpathlineto{\pgfqpoint{5.268323in}{2.718643in}}%
\pgfpathclose%
\pgfusepath{fill}%
\end{pgfscope}%
\begin{pgfscope}%
\pgfpathrectangle{\pgfqpoint{1.020000in}{0.880000in}}{\pgfqpoint{6.160000in}{6.160000in}}%
\pgfusepath{clip}%
\pgfsetbuttcap%
\pgfsetroundjoin%
\definecolor{currentfill}{rgb}{0.576051,0.708780,0.997755}%
\pgfsetfillcolor{currentfill}%
\pgfsetlinewidth{0.000000pt}%
\definecolor{currentstroke}{rgb}{0.000000,0.000000,0.000000}%
\pgfsetstrokecolor{currentstroke}%
\pgfsetdash{}{0pt}%
\pgfpathmoveto{\pgfqpoint{2.610774in}{3.195099in}}%
\pgfpathlineto{\pgfqpoint{2.620055in}{3.171496in}}%
\pgfpathlineto{\pgfqpoint{2.629342in}{3.148477in}}%
\pgfpathlineto{\pgfqpoint{2.663801in}{3.157776in}}%
\pgfpathlineto{\pgfqpoint{2.698228in}{3.168112in}}%
\pgfpathlineto{\pgfqpoint{2.688896in}{3.190357in}}%
\pgfpathlineto{\pgfqpoint{2.679571in}{3.213280in}}%
\pgfpathlineto{\pgfqpoint{2.645188in}{3.203715in}}%
\pgfpathlineto{\pgfqpoint{2.610774in}{3.195099in}}%
\pgfpathclose%
\pgfusepath{fill}%
\end{pgfscope}%
\begin{pgfscope}%
\pgfpathrectangle{\pgfqpoint{1.020000in}{0.880000in}}{\pgfqpoint{6.160000in}{6.160000in}}%
\pgfusepath{clip}%
\pgfsetbuttcap%
\pgfsetroundjoin%
\definecolor{currentfill}{rgb}{0.619318,0.744121,0.998931}%
\pgfsetfillcolor{currentfill}%
\pgfsetlinewidth{0.000000pt}%
\definecolor{currentstroke}{rgb}{0.000000,0.000000,0.000000}%
\pgfsetstrokecolor{currentstroke}%
\pgfsetdash{}{0pt}%
\pgfpathmoveto{\pgfqpoint{2.904168in}{3.253111in}}%
\pgfpathlineto{\pgfqpoint{2.913631in}{3.235328in}}%
\pgfpathlineto{\pgfqpoint{2.923104in}{3.218215in}}%
\pgfpathlineto{\pgfqpoint{2.957374in}{3.237947in}}%
\pgfpathlineto{\pgfqpoint{2.991620in}{3.258804in}}%
\pgfpathlineto{\pgfqpoint{2.982108in}{3.274054in}}%
\pgfpathlineto{\pgfqpoint{2.972605in}{3.290024in}}%
\pgfpathlineto{\pgfqpoint{2.938398in}{3.271072in}}%
\pgfpathlineto{\pgfqpoint{2.904168in}{3.253111in}}%
\pgfpathclose%
\pgfusepath{fill}%
\end{pgfscope}%
\begin{pgfscope}%
\pgfpathrectangle{\pgfqpoint{1.020000in}{0.880000in}}{\pgfqpoint{6.160000in}{6.160000in}}%
\pgfusepath{clip}%
\pgfsetbuttcap%
\pgfsetroundjoin%
\definecolor{currentfill}{rgb}{0.271104,0.360011,0.807095}%
\pgfsetfillcolor{currentfill}%
\pgfsetlinewidth{0.000000pt}%
\definecolor{currentstroke}{rgb}{0.000000,0.000000,0.000000}%
\pgfsetstrokecolor{currentstroke}%
\pgfsetdash{}{0pt}%
\pgfpathmoveto{\pgfqpoint{5.876711in}{2.623664in}}%
\pgfpathlineto{\pgfqpoint{5.888946in}{2.608868in}}%
\pgfpathlineto{\pgfqpoint{5.901205in}{2.594127in}}%
\pgfpathlineto{\pgfqpoint{5.934744in}{2.596602in}}%
\pgfpathlineto{\pgfqpoint{5.968262in}{2.599138in}}%
\pgfpathlineto{\pgfqpoint{5.955946in}{2.613769in}}%
\pgfpathlineto{\pgfqpoint{5.943654in}{2.628453in}}%
\pgfpathlineto{\pgfqpoint{5.910193in}{2.626024in}}%
\pgfpathlineto{\pgfqpoint{5.876711in}{2.623664in}}%
\pgfpathclose%
\pgfusepath{fill}%
\end{pgfscope}%
\begin{pgfscope}%
\pgfpathrectangle{\pgfqpoint{1.020000in}{0.880000in}}{\pgfqpoint{6.160000in}{6.160000in}}%
\pgfusepath{clip}%
\pgfsetbuttcap%
\pgfsetroundjoin%
\definecolor{currentfill}{rgb}{0.266381,0.353304,0.801637}%
\pgfsetfillcolor{currentfill}%
\pgfsetlinewidth{0.000000pt}%
\definecolor{currentstroke}{rgb}{0.000000,0.000000,0.000000}%
\pgfsetstrokecolor{currentstroke}%
\pgfsetdash{}{0pt}%
\pgfpathmoveto{\pgfqpoint{6.102134in}{2.609727in}}%
\pgfpathlineto{\pgfqpoint{6.114589in}{2.595327in}}%
\pgfpathlineto{\pgfqpoint{6.127069in}{2.580976in}}%
\pgfpathlineto{\pgfqpoint{6.160543in}{2.583780in}}%
\pgfpathlineto{\pgfqpoint{6.193997in}{2.586604in}}%
\pgfpathlineto{\pgfqpoint{6.181459in}{2.600879in}}%
\pgfpathlineto{\pgfqpoint{6.168947in}{2.615202in}}%
\pgfpathlineto{\pgfqpoint{6.135551in}{2.612453in}}%
\pgfpathlineto{\pgfqpoint{6.102134in}{2.609727in}}%
\pgfpathclose%
\pgfusepath{fill}%
\end{pgfscope}%
\begin{pgfscope}%
\pgfpathrectangle{\pgfqpoint{1.020000in}{0.880000in}}{\pgfqpoint{6.160000in}{6.160000in}}%
\pgfusepath{clip}%
\pgfsetbuttcap%
\pgfsetroundjoin%
\definecolor{currentfill}{rgb}{0.280550,0.373423,0.818011}%
\pgfsetfillcolor{currentfill}%
\pgfsetlinewidth{0.000000pt}%
\definecolor{currentstroke}{rgb}{0.000000,0.000000,0.000000}%
\pgfsetstrokecolor{currentstroke}%
\pgfsetdash{}{0pt}%
\pgfpathmoveto{\pgfqpoint{5.651361in}{2.642571in}}%
\pgfpathlineto{\pgfqpoint{5.663374in}{2.627177in}}%
\pgfpathlineto{\pgfqpoint{5.675411in}{2.611837in}}%
\pgfpathlineto{\pgfqpoint{5.709009in}{2.613433in}}%
\pgfpathlineto{\pgfqpoint{5.742589in}{2.615209in}}%
\pgfpathlineto{\pgfqpoint{5.730493in}{2.630351in}}%
\pgfpathlineto{\pgfqpoint{5.718422in}{2.645547in}}%
\pgfpathlineto{\pgfqpoint{5.684901in}{2.643958in}}%
\pgfpathlineto{\pgfqpoint{5.651361in}{2.642571in}}%
\pgfpathclose%
\pgfusepath{fill}%
\end{pgfscope}%
\begin{pgfscope}%
\pgfpathrectangle{\pgfqpoint{1.020000in}{0.880000in}}{\pgfqpoint{6.160000in}{6.160000in}}%
\pgfusepath{clip}%
\pgfsetbuttcap%
\pgfsetroundjoin%
\definecolor{currentfill}{rgb}{0.809329,0.852974,0.922323}%
\pgfsetfillcolor{currentfill}%
\pgfsetlinewidth{0.000000pt}%
\definecolor{currentstroke}{rgb}{0.000000,0.000000,0.000000}%
\pgfsetstrokecolor{currentstroke}%
\pgfsetdash{}{0pt}%
\pgfpathmoveto{\pgfqpoint{3.469764in}{3.593674in}}%
\pgfpathlineto{\pgfqpoint{3.479608in}{3.596819in}}%
\pgfpathlineto{\pgfqpoint{3.489474in}{3.599948in}}%
\pgfpathlineto{\pgfqpoint{3.523658in}{3.621390in}}%
\pgfpathlineto{\pgfqpoint{3.557844in}{3.640839in}}%
\pgfpathlineto{\pgfqpoint{3.547918in}{3.635437in}}%
\pgfpathlineto{\pgfqpoint{3.538016in}{3.629936in}}%
\pgfpathlineto{\pgfqpoint{3.503890in}{3.612729in}}%
\pgfpathlineto{\pgfqpoint{3.469764in}{3.593674in}}%
\pgfpathclose%
\pgfusepath{fill}%
\end{pgfscope}%
\begin{pgfscope}%
\pgfpathrectangle{\pgfqpoint{1.020000in}{0.880000in}}{\pgfqpoint{6.160000in}{6.160000in}}%
\pgfusepath{clip}%
\pgfsetbuttcap%
\pgfsetroundjoin%
\definecolor{currentfill}{rgb}{0.791392,0.846750,0.936641}%
\pgfsetfillcolor{currentfill}%
\pgfsetlinewidth{0.000000pt}%
\definecolor{currentstroke}{rgb}{0.000000,0.000000,0.000000}%
\pgfsetstrokecolor{currentstroke}%
\pgfsetdash{}{0pt}%
\pgfpathmoveto{\pgfqpoint{4.145797in}{3.606716in}}%
\pgfpathlineto{\pgfqpoint{4.156394in}{3.606866in}}%
\pgfpathlineto{\pgfqpoint{4.167014in}{3.604736in}}%
\pgfpathlineto{\pgfqpoint{4.201113in}{3.576684in}}%
\pgfpathlineto{\pgfqpoint{4.235169in}{3.546968in}}%
\pgfpathlineto{\pgfqpoint{4.224491in}{3.550217in}}%
\pgfpathlineto{\pgfqpoint{4.213834in}{3.551363in}}%
\pgfpathlineto{\pgfqpoint{4.179836in}{3.579826in}}%
\pgfpathlineto{\pgfqpoint{4.145797in}{3.606716in}}%
\pgfpathclose%
\pgfusepath{fill}%
\end{pgfscope}%
\begin{pgfscope}%
\pgfpathrectangle{\pgfqpoint{1.020000in}{0.880000in}}{\pgfqpoint{6.160000in}{6.160000in}}%
\pgfusepath{clip}%
\pgfsetbuttcap%
\pgfsetroundjoin%
\definecolor{currentfill}{rgb}{0.294718,0.393542,0.834384}%
\pgfsetfillcolor{currentfill}%
\pgfsetlinewidth{0.000000pt}%
\definecolor{currentstroke}{rgb}{0.000000,0.000000,0.000000}%
\pgfsetstrokecolor{currentstroke}%
\pgfsetdash{}{0pt}%
\pgfpathmoveto{\pgfqpoint{5.426130in}{2.673352in}}%
\pgfpathlineto{\pgfqpoint{5.437921in}{2.656932in}}%
\pgfpathlineto{\pgfqpoint{5.449735in}{2.640540in}}%
\pgfpathlineto{\pgfqpoint{5.483384in}{2.639915in}}%
\pgfpathlineto{\pgfqpoint{5.517015in}{2.639740in}}%
\pgfpathlineto{\pgfqpoint{5.505142in}{2.655739in}}%
\pgfpathlineto{\pgfqpoint{5.493292in}{2.671776in}}%
\pgfpathlineto{\pgfqpoint{5.459720in}{2.672317in}}%
\pgfpathlineto{\pgfqpoint{5.426130in}{2.673352in}}%
\pgfpathclose%
\pgfusepath{fill}%
\end{pgfscope}%
\begin{pgfscope}%
\pgfpathrectangle{\pgfqpoint{1.020000in}{0.880000in}}{\pgfqpoint{6.160000in}{6.160000in}}%
\pgfusepath{clip}%
\pgfsetbuttcap%
\pgfsetroundjoin%
\definecolor{currentfill}{rgb}{0.831148,0.859513,0.903110}%
\pgfsetfillcolor{currentfill}%
\pgfsetlinewidth{0.000000pt}%
\definecolor{currentstroke}{rgb}{0.000000,0.000000,0.000000}%
\pgfsetstrokecolor{currentstroke}%
\pgfsetdash{}{0pt}%
\pgfpathmoveto{\pgfqpoint{3.988420in}{3.682348in}}%
\pgfpathlineto{\pgfqpoint{3.998836in}{3.688511in}}%
\pgfpathlineto{\pgfqpoint{4.009279in}{3.692752in}}%
\pgfpathlineto{\pgfqpoint{4.043458in}{3.675011in}}%
\pgfpathlineto{\pgfqpoint{4.077606in}{3.654589in}}%
\pgfpathlineto{\pgfqpoint{4.067097in}{3.651221in}}%
\pgfpathlineto{\pgfqpoint{4.056613in}{3.646005in}}%
\pgfpathlineto{\pgfqpoint{4.022532in}{3.665425in}}%
\pgfpathlineto{\pgfqpoint{3.988420in}{3.682348in}}%
\pgfpathclose%
\pgfusepath{fill}%
\end{pgfscope}%
\begin{pgfscope}%
\pgfpathrectangle{\pgfqpoint{1.020000in}{0.880000in}}{\pgfqpoint{6.160000in}{6.160000in}}%
\pgfusepath{clip}%
\pgfsetbuttcap%
\pgfsetroundjoin%
\definecolor{currentfill}{rgb}{0.527132,0.664700,0.989065}%
\pgfsetfillcolor{currentfill}%
\pgfsetlinewidth{0.000000pt}%
\definecolor{currentstroke}{rgb}{0.000000,0.000000,0.000000}%
\pgfsetstrokecolor{currentstroke}%
\pgfsetdash{}{0pt}%
\pgfpathmoveto{\pgfqpoint{4.639347in}{3.140335in}}%
\pgfpathlineto{\pgfqpoint{4.650395in}{3.121091in}}%
\pgfpathlineto{\pgfqpoint{4.661457in}{3.100444in}}%
\pgfpathlineto{\pgfqpoint{4.695250in}{3.071243in}}%
\pgfpathlineto{\pgfqpoint{4.729008in}{3.043464in}}%
\pgfpathlineto{\pgfqpoint{4.717896in}{3.063394in}}%
\pgfpathlineto{\pgfqpoint{4.706798in}{3.082168in}}%
\pgfpathlineto{\pgfqpoint{4.673090in}{3.110528in}}%
\pgfpathlineto{\pgfqpoint{4.639347in}{3.140335in}}%
\pgfpathclose%
\pgfusepath{fill}%
\end{pgfscope}%
\begin{pgfscope}%
\pgfpathrectangle{\pgfqpoint{1.020000in}{0.880000in}}{\pgfqpoint{6.160000in}{6.160000in}}%
\pgfusepath{clip}%
\pgfsetbuttcap%
\pgfsetroundjoin%
\definecolor{currentfill}{rgb}{0.261805,0.346484,0.795658}%
\pgfsetfillcolor{currentfill}%
\pgfsetlinewidth{0.000000pt}%
\definecolor{currentstroke}{rgb}{0.000000,0.000000,0.000000}%
\pgfsetstrokecolor{currentstroke}%
\pgfsetdash{}{0pt}%
\pgfpathmoveto{\pgfqpoint{6.327604in}{2.598041in}}%
\pgfpathlineto{\pgfqpoint{6.340281in}{2.583950in}}%
\pgfpathlineto{\pgfqpoint{6.352983in}{2.569902in}}%
\pgfpathlineto{\pgfqpoint{6.386390in}{2.572852in}}%
\pgfpathlineto{\pgfqpoint{6.373659in}{2.586868in}}%
\pgfpathlineto{\pgfqpoint{6.360955in}{2.600927in}}%
\pgfpathlineto{\pgfqpoint{6.327604in}{2.598041in}}%
\pgfpathclose%
\pgfusepath{fill}%
\end{pgfscope}%
\begin{pgfscope}%
\pgfpathrectangle{\pgfqpoint{1.020000in}{0.880000in}}{\pgfqpoint{6.160000in}{6.160000in}}%
\pgfusepath{clip}%
\pgfsetbuttcap%
\pgfsetroundjoin%
\definecolor{currentfill}{rgb}{0.603162,0.731527,0.999565}%
\pgfsetfillcolor{currentfill}%
\pgfsetlinewidth{0.000000pt}%
\definecolor{currentstroke}{rgb}{0.000000,0.000000,0.000000}%
\pgfsetstrokecolor{currentstroke}%
\pgfsetdash{}{0pt}%
\pgfpathmoveto{\pgfqpoint{2.835633in}{3.220360in}}%
\pgfpathlineto{\pgfqpoint{2.845056in}{3.201036in}}%
\pgfpathlineto{\pgfqpoint{2.854489in}{3.182326in}}%
\pgfpathlineto{\pgfqpoint{2.888809in}{3.199663in}}%
\pgfpathlineto{\pgfqpoint{2.923104in}{3.218215in}}%
\pgfpathlineto{\pgfqpoint{2.913631in}{3.235328in}}%
\pgfpathlineto{\pgfqpoint{2.904168in}{3.253111in}}%
\pgfpathlineto{\pgfqpoint{2.869913in}{3.236194in}}%
\pgfpathlineto{\pgfqpoint{2.835633in}{3.220360in}}%
\pgfpathclose%
\pgfusepath{fill}%
\end{pgfscope}%
\begin{pgfscope}%
\pgfpathrectangle{\pgfqpoint{1.020000in}{0.880000in}}{\pgfqpoint{6.160000in}{6.160000in}}%
\pgfusepath{clip}%
\pgfsetbuttcap%
\pgfsetroundjoin%
\definecolor{currentfill}{rgb}{0.358415,0.478426,0.896795}%
\pgfsetfillcolor{currentfill}%
\pgfsetlinewidth{0.000000pt}%
\definecolor{currentstroke}{rgb}{0.000000,0.000000,0.000000}%
\pgfsetstrokecolor{currentstroke}%
\pgfsetdash{}{0pt}%
\pgfpathmoveto{\pgfqpoint{5.043537in}{2.805054in}}%
\pgfpathlineto{\pgfqpoint{5.054949in}{2.785503in}}%
\pgfpathlineto{\pgfqpoint{5.066379in}{2.765687in}}%
\pgfpathlineto{\pgfqpoint{5.100081in}{2.754909in}}%
\pgfpathlineto{\pgfqpoint{5.133764in}{2.745401in}}%
\pgfpathlineto{\pgfqpoint{5.122274in}{2.764249in}}%
\pgfpathlineto{\pgfqpoint{5.110804in}{2.782908in}}%
\pgfpathlineto{\pgfqpoint{5.077180in}{2.793303in}}%
\pgfpathlineto{\pgfqpoint{5.043537in}{2.805054in}}%
\pgfpathclose%
\pgfusepath{fill}%
\end{pgfscope}%
\begin{pgfscope}%
\pgfpathrectangle{\pgfqpoint{1.020000in}{0.880000in}}{\pgfqpoint{6.160000in}{6.160000in}}%
\pgfusepath{clip}%
\pgfsetbuttcap%
\pgfsetroundjoin%
\definecolor{currentfill}{rgb}{0.677823,0.786546,0.991005}%
\pgfsetfillcolor{currentfill}%
\pgfsetlinewidth{0.000000pt}%
\definecolor{currentstroke}{rgb}{0.000000,0.000000,0.000000}%
\pgfsetstrokecolor{currentstroke}%
\pgfsetdash{}{0pt}%
\pgfpathmoveto{\pgfqpoint{4.392596in}{3.400505in}}%
\pgfpathlineto{\pgfqpoint{4.403438in}{3.388684in}}%
\pgfpathlineto{\pgfqpoint{4.414297in}{3.374666in}}%
\pgfpathlineto{\pgfqpoint{4.448236in}{3.339961in}}%
\pgfpathlineto{\pgfqpoint{4.482131in}{3.305455in}}%
\pgfpathlineto{\pgfqpoint{4.471223in}{3.319801in}}%
\pgfpathlineto{\pgfqpoint{4.460331in}{3.332231in}}%
\pgfpathlineto{\pgfqpoint{4.426485in}{3.366268in}}%
\pgfpathlineto{\pgfqpoint{4.392596in}{3.400505in}}%
\pgfpathclose%
\pgfusepath{fill}%
\end{pgfscope}%
\begin{pgfscope}%
\pgfpathrectangle{\pgfqpoint{1.020000in}{0.880000in}}{\pgfqpoint{6.160000in}{6.160000in}}%
\pgfusepath{clip}%
\pgfsetbuttcap%
\pgfsetroundjoin%
\definecolor{currentfill}{rgb}{0.791392,0.846750,0.936641}%
\pgfsetfillcolor{currentfill}%
\pgfsetlinewidth{0.000000pt}%
\definecolor{currentstroke}{rgb}{0.000000,0.000000,0.000000}%
\pgfsetstrokecolor{currentstroke}%
\pgfsetdash{}{0pt}%
\pgfpathmoveto{\pgfqpoint{3.401514in}{3.550892in}}%
\pgfpathlineto{\pgfqpoint{3.411304in}{3.551444in}}%
\pgfpathlineto{\pgfqpoint{3.421114in}{3.552061in}}%
\pgfpathlineto{\pgfqpoint{3.455293in}{3.576754in}}%
\pgfpathlineto{\pgfqpoint{3.489474in}{3.599948in}}%
\pgfpathlineto{\pgfqpoint{3.479608in}{3.596819in}}%
\pgfpathlineto{\pgfqpoint{3.469764in}{3.593674in}}%
\pgfpathlineto{\pgfqpoint{3.435639in}{3.572986in}}%
\pgfpathlineto{\pgfqpoint{3.401514in}{3.550892in}}%
\pgfpathclose%
\pgfusepath{fill}%
\end{pgfscope}%
\begin{pgfscope}%
\pgfpathrectangle{\pgfqpoint{1.020000in}{0.880000in}}{\pgfqpoint{6.160000in}{6.160000in}}%
\pgfusepath{clip}%
\pgfsetbuttcap%
\pgfsetroundjoin%
\definecolor{currentfill}{rgb}{0.409611,0.540759,0.935545}%
\pgfsetfillcolor{currentfill}%
\pgfsetlinewidth{0.000000pt}%
\definecolor{currentstroke}{rgb}{0.000000,0.000000,0.000000}%
\pgfsetstrokecolor{currentstroke}%
\pgfsetdash{}{0pt}%
\pgfpathmoveto{\pgfqpoint{4.886219in}{2.908814in}}%
\pgfpathlineto{\pgfqpoint{4.897482in}{2.888082in}}%
\pgfpathlineto{\pgfqpoint{4.908762in}{2.866759in}}%
\pgfpathlineto{\pgfqpoint{4.942489in}{2.849041in}}%
\pgfpathlineto{\pgfqpoint{4.976192in}{2.832873in}}%
\pgfpathlineto{\pgfqpoint{4.964856in}{2.853077in}}%
\pgfpathlineto{\pgfqpoint{4.953537in}{2.872826in}}%
\pgfpathlineto{\pgfqpoint{4.919889in}{2.890006in}}%
\pgfpathlineto{\pgfqpoint{4.886219in}{2.908814in}}%
\pgfpathclose%
\pgfusepath{fill}%
\end{pgfscope}%
\begin{pgfscope}%
\pgfpathrectangle{\pgfqpoint{1.020000in}{0.880000in}}{\pgfqpoint{6.160000in}{6.160000in}}%
\pgfusepath{clip}%
\pgfsetbuttcap%
\pgfsetroundjoin%
\definecolor{currentfill}{rgb}{0.851372,0.863125,0.881064}%
\pgfsetfillcolor{currentfill}%
\pgfsetlinewidth{0.000000pt}%
\definecolor{currentstroke}{rgb}{0.000000,0.000000,0.000000}%
\pgfsetstrokecolor{currentstroke}%
\pgfsetdash{}{0pt}%
\pgfpathmoveto{\pgfqpoint{3.762942in}{3.704926in}}%
\pgfpathlineto{\pgfqpoint{3.773092in}{3.714065in}}%
\pgfpathlineto{\pgfqpoint{3.783271in}{3.722046in}}%
\pgfpathlineto{\pgfqpoint{3.817503in}{3.723324in}}%
\pgfpathlineto{\pgfqpoint{3.851724in}{3.721365in}}%
\pgfpathlineto{\pgfqpoint{3.841475in}{3.712976in}}%
\pgfpathlineto{\pgfqpoint{3.831254in}{3.703383in}}%
\pgfpathlineto{\pgfqpoint{3.797104in}{3.705646in}}%
\pgfpathlineto{\pgfqpoint{3.762942in}{3.704926in}}%
\pgfpathclose%
\pgfusepath{fill}%
\end{pgfscope}%
\begin{pgfscope}%
\pgfpathrectangle{\pgfqpoint{1.020000in}{0.880000in}}{\pgfqpoint{6.160000in}{6.160000in}}%
\pgfusepath{clip}%
\pgfsetbuttcap%
\pgfsetroundjoin%
\definecolor{currentfill}{rgb}{0.323718,0.433158,0.864722}%
\pgfsetfillcolor{currentfill}%
\pgfsetlinewidth{0.000000pt}%
\definecolor{currentstroke}{rgb}{0.000000,0.000000,0.000000}%
\pgfsetstrokecolor{currentstroke}%
\pgfsetdash{}{0pt}%
\pgfpathmoveto{\pgfqpoint{5.201078in}{2.729918in}}%
\pgfpathlineto{\pgfqpoint{5.212647in}{2.711770in}}%
\pgfpathlineto{\pgfqpoint{5.224237in}{2.693544in}}%
\pgfpathlineto{\pgfqpoint{5.257928in}{2.688171in}}%
\pgfpathlineto{\pgfqpoint{5.291602in}{2.683696in}}%
\pgfpathlineto{\pgfqpoint{5.279952in}{2.701190in}}%
\pgfpathlineto{\pgfqpoint{5.268323in}{2.718643in}}%
\pgfpathlineto{\pgfqpoint{5.234709in}{2.723794in}}%
\pgfpathlineto{\pgfqpoint{5.201078in}{2.729918in}}%
\pgfpathclose%
\pgfusepath{fill}%
\end{pgfscope}%
\begin{pgfscope}%
\pgfpathrectangle{\pgfqpoint{1.020000in}{0.880000in}}{\pgfqpoint{6.160000in}{6.160000in}}%
\pgfusepath{clip}%
\pgfsetbuttcap%
\pgfsetroundjoin%
\definecolor{currentfill}{rgb}{0.768034,0.837035,0.952488}%
\pgfsetfillcolor{currentfill}%
\pgfsetlinewidth{0.000000pt}%
\definecolor{currentstroke}{rgb}{0.000000,0.000000,0.000000}%
\pgfsetstrokecolor{currentstroke}%
\pgfsetdash{}{0pt}%
\pgfpathmoveto{\pgfqpoint{3.333264in}{3.503436in}}%
\pgfpathlineto{\pgfqpoint{3.343003in}{3.501261in}}%
\pgfpathlineto{\pgfqpoint{3.352762in}{3.499226in}}%
\pgfpathlineto{\pgfqpoint{3.386937in}{3.526130in}}%
\pgfpathlineto{\pgfqpoint{3.421114in}{3.552061in}}%
\pgfpathlineto{\pgfqpoint{3.411304in}{3.551444in}}%
\pgfpathlineto{\pgfqpoint{3.401514in}{3.550892in}}%
\pgfpathlineto{\pgfqpoint{3.367389in}{3.527629in}}%
\pgfpathlineto{\pgfqpoint{3.333264in}{3.503436in}}%
\pgfpathclose%
\pgfusepath{fill}%
\end{pgfscope}%
\begin{pgfscope}%
\pgfpathrectangle{\pgfqpoint{1.020000in}{0.880000in}}{\pgfqpoint{6.160000in}{6.160000in}}%
\pgfusepath{clip}%
\pgfsetbuttcap%
\pgfsetroundjoin%
\definecolor{currentfill}{rgb}{0.586921,0.718121,0.998874}%
\pgfsetfillcolor{currentfill}%
\pgfsetlinewidth{0.000000pt}%
\definecolor{currentstroke}{rgb}{0.000000,0.000000,0.000000}%
\pgfsetstrokecolor{currentstroke}%
\pgfsetdash{}{0pt}%
\pgfpathmoveto{\pgfqpoint{2.766989in}{3.192020in}}%
\pgfpathlineto{\pgfqpoint{2.776371in}{3.171412in}}%
\pgfpathlineto{\pgfqpoint{2.785762in}{3.151359in}}%
\pgfpathlineto{\pgfqpoint{2.820140in}{3.166223in}}%
\pgfpathlineto{\pgfqpoint{2.854489in}{3.182326in}}%
\pgfpathlineto{\pgfqpoint{2.845056in}{3.201036in}}%
\pgfpathlineto{\pgfqpoint{2.835633in}{3.220360in}}%
\pgfpathlineto{\pgfqpoint{2.801325in}{3.205632in}}%
\pgfpathlineto{\pgfqpoint{2.766989in}{3.192020in}}%
\pgfpathclose%
\pgfusepath{fill}%
\end{pgfscope}%
\begin{pgfscope}%
\pgfpathrectangle{\pgfqpoint{1.020000in}{0.880000in}}{\pgfqpoint{6.160000in}{6.160000in}}%
\pgfusepath{clip}%
\pgfsetbuttcap%
\pgfsetroundjoin%
\definecolor{currentfill}{rgb}{0.275827,0.366717,0.812553}%
\pgfsetfillcolor{currentfill}%
\pgfsetlinewidth{0.000000pt}%
\definecolor{currentstroke}{rgb}{0.000000,0.000000,0.000000}%
\pgfsetstrokecolor{currentstroke}%
\pgfsetdash{}{0pt}%
\pgfpathmoveto{\pgfqpoint{5.809689in}{2.619207in}}%
\pgfpathlineto{\pgfqpoint{5.821866in}{2.604280in}}%
\pgfpathlineto{\pgfqpoint{5.834068in}{2.589409in}}%
\pgfpathlineto{\pgfqpoint{5.867646in}{2.591725in}}%
\pgfpathlineto{\pgfqpoint{5.901205in}{2.594127in}}%
\pgfpathlineto{\pgfqpoint{5.888946in}{2.608868in}}%
\pgfpathlineto{\pgfqpoint{5.876711in}{2.623664in}}%
\pgfpathlineto{\pgfqpoint{5.843210in}{2.621387in}}%
\pgfpathlineto{\pgfqpoint{5.809689in}{2.619207in}}%
\pgfpathclose%
\pgfusepath{fill}%
\end{pgfscope}%
\begin{pgfscope}%
\pgfpathrectangle{\pgfqpoint{1.020000in}{0.880000in}}{\pgfqpoint{6.160000in}{6.160000in}}%
\pgfusepath{clip}%
\pgfsetbuttcap%
\pgfsetroundjoin%
\definecolor{currentfill}{rgb}{0.266381,0.353304,0.801637}%
\pgfsetfillcolor{currentfill}%
\pgfsetlinewidth{0.000000pt}%
\definecolor{currentstroke}{rgb}{0.000000,0.000000,0.000000}%
\pgfsetstrokecolor{currentstroke}%
\pgfsetdash{}{0pt}%
\pgfpathmoveto{\pgfqpoint{6.035239in}{2.604358in}}%
\pgfpathlineto{\pgfqpoint{6.047637in}{2.589874in}}%
\pgfpathlineto{\pgfqpoint{6.060060in}{2.575440in}}%
\pgfpathlineto{\pgfqpoint{6.093575in}{2.578195in}}%
\pgfpathlineto{\pgfqpoint{6.127069in}{2.580976in}}%
\pgfpathlineto{\pgfqpoint{6.114589in}{2.595327in}}%
\pgfpathlineto{\pgfqpoint{6.102134in}{2.609727in}}%
\pgfpathlineto{\pgfqpoint{6.068696in}{2.607027in}}%
\pgfpathlineto{\pgfqpoint{6.035239in}{2.604358in}}%
\pgfpathclose%
\pgfusepath{fill}%
\end{pgfscope}%
\begin{pgfscope}%
\pgfpathrectangle{\pgfqpoint{1.020000in}{0.880000in}}{\pgfqpoint{6.160000in}{6.160000in}}%
\pgfusepath{clip}%
\pgfsetbuttcap%
\pgfsetroundjoin%
\definecolor{currentfill}{rgb}{0.261805,0.346484,0.795658}%
\pgfsetfillcolor{currentfill}%
\pgfsetlinewidth{0.000000pt}%
\definecolor{currentstroke}{rgb}{0.000000,0.000000,0.000000}%
\pgfsetstrokecolor{currentstroke}%
\pgfsetdash{}{0pt}%
\pgfpathmoveto{\pgfqpoint{6.260842in}{2.592298in}}%
\pgfpathlineto{\pgfqpoint{6.273461in}{2.578141in}}%
\pgfpathlineto{\pgfqpoint{6.286106in}{2.564028in}}%
\pgfpathlineto{\pgfqpoint{6.319555in}{2.566961in}}%
\pgfpathlineto{\pgfqpoint{6.352983in}{2.569902in}}%
\pgfpathlineto{\pgfqpoint{6.340281in}{2.583950in}}%
\pgfpathlineto{\pgfqpoint{6.327604in}{2.598041in}}%
\pgfpathlineto{\pgfqpoint{6.294233in}{2.595164in}}%
\pgfpathlineto{\pgfqpoint{6.260842in}{2.592298in}}%
\pgfpathclose%
\pgfusepath{fill}%
\end{pgfscope}%
\begin{pgfscope}%
\pgfpathrectangle{\pgfqpoint{1.020000in}{0.880000in}}{\pgfqpoint{6.160000in}{6.160000in}}%
\pgfusepath{clip}%
\pgfsetbuttcap%
\pgfsetroundjoin%
\definecolor{currentfill}{rgb}{0.285273,0.380129,0.823469}%
\pgfsetfillcolor{currentfill}%
\pgfsetlinewidth{0.000000pt}%
\definecolor{currentstroke}{rgb}{0.000000,0.000000,0.000000}%
\pgfsetstrokecolor{currentstroke}%
\pgfsetdash{}{0pt}%
\pgfpathmoveto{\pgfqpoint{5.584224in}{2.640536in}}%
\pgfpathlineto{\pgfqpoint{5.596179in}{2.624895in}}%
\pgfpathlineto{\pgfqpoint{5.608158in}{2.609306in}}%
\pgfpathlineto{\pgfqpoint{5.641794in}{2.610452in}}%
\pgfpathlineto{\pgfqpoint{5.675411in}{2.611837in}}%
\pgfpathlineto{\pgfqpoint{5.663374in}{2.627177in}}%
\pgfpathlineto{\pgfqpoint{5.651361in}{2.642571in}}%
\pgfpathlineto{\pgfqpoint{5.617802in}{2.641419in}}%
\pgfpathlineto{\pgfqpoint{5.584224in}{2.640536in}}%
\pgfpathclose%
\pgfusepath{fill}%
\end{pgfscope}%
\begin{pgfscope}%
\pgfpathrectangle{\pgfqpoint{1.020000in}{0.880000in}}{\pgfqpoint{6.160000in}{6.160000in}}%
\pgfusepath{clip}%
\pgfsetbuttcap%
\pgfsetroundjoin%
\definecolor{currentfill}{rgb}{0.743754,0.825125,0.965798}%
\pgfsetfillcolor{currentfill}%
\pgfsetlinewidth{0.000000pt}%
\definecolor{currentstroke}{rgb}{0.000000,0.000000,0.000000}%
\pgfsetstrokecolor{currentstroke}%
\pgfsetdash{}{0pt}%
\pgfpathmoveto{\pgfqpoint{3.265006in}{3.453215in}}%
\pgfpathlineto{\pgfqpoint{3.274698in}{3.448281in}}%
\pgfpathlineto{\pgfqpoint{3.284408in}{3.443551in}}%
\pgfpathlineto{\pgfqpoint{3.318585in}{3.471613in}}%
\pgfpathlineto{\pgfqpoint{3.352762in}{3.499226in}}%
\pgfpathlineto{\pgfqpoint{3.343003in}{3.501261in}}%
\pgfpathlineto{\pgfqpoint{3.333264in}{3.503436in}}%
\pgfpathlineto{\pgfqpoint{3.299136in}{3.478553in}}%
\pgfpathlineto{\pgfqpoint{3.265006in}{3.453215in}}%
\pgfpathclose%
\pgfusepath{fill}%
\end{pgfscope}%
\begin{pgfscope}%
\pgfpathrectangle{\pgfqpoint{1.020000in}{0.880000in}}{\pgfqpoint{6.160000in}{6.160000in}}%
\pgfusepath{clip}%
\pgfsetbuttcap%
\pgfsetroundjoin%
\definecolor{currentfill}{rgb}{0.478462,0.616564,0.972721}%
\pgfsetfillcolor{currentfill}%
\pgfsetlinewidth{0.000000pt}%
\definecolor{currentstroke}{rgb}{0.000000,0.000000,0.000000}%
\pgfsetstrokecolor{currentstroke}%
\pgfsetdash{}{0pt}%
\pgfpathmoveto{\pgfqpoint{4.729008in}{3.043464in}}%
\pgfpathlineto{\pgfqpoint{4.740135in}{3.022414in}}%
\pgfpathlineto{\pgfqpoint{4.751276in}{3.000293in}}%
\pgfpathlineto{\pgfqpoint{4.785054in}{2.975042in}}%
\pgfpathlineto{\pgfqpoint{4.818803in}{2.951348in}}%
\pgfpathlineto{\pgfqpoint{4.807609in}{2.972395in}}%
\pgfpathlineto{\pgfqpoint{4.796431in}{2.992577in}}%
\pgfpathlineto{\pgfqpoint{4.762735in}{3.017216in}}%
\pgfpathlineto{\pgfqpoint{4.729008in}{3.043464in}}%
\pgfpathclose%
\pgfusepath{fill}%
\end{pgfscope}%
\begin{pgfscope}%
\pgfpathrectangle{\pgfqpoint{1.020000in}{0.880000in}}{\pgfqpoint{6.160000in}{6.160000in}}%
\pgfusepath{clip}%
\pgfsetbuttcap%
\pgfsetroundjoin%
\definecolor{currentfill}{rgb}{0.299441,0.400248,0.839842}%
\pgfsetfillcolor{currentfill}%
\pgfsetlinewidth{0.000000pt}%
\definecolor{currentstroke}{rgb}{0.000000,0.000000,0.000000}%
\pgfsetstrokecolor{currentstroke}%
\pgfsetdash{}{0pt}%
\pgfpathmoveto{\pgfqpoint{5.358900in}{2.677152in}}%
\pgfpathlineto{\pgfqpoint{5.370631in}{2.660247in}}%
\pgfpathlineto{\pgfqpoint{5.382385in}{2.643358in}}%
\pgfpathlineto{\pgfqpoint{5.416069in}{2.641668in}}%
\pgfpathlineto{\pgfqpoint{5.449735in}{2.640540in}}%
\pgfpathlineto{\pgfqpoint{5.437921in}{2.656932in}}%
\pgfpathlineto{\pgfqpoint{5.426130in}{2.673352in}}%
\pgfpathlineto{\pgfqpoint{5.392523in}{2.674943in}}%
\pgfpathlineto{\pgfqpoint{5.358900in}{2.677152in}}%
\pgfpathclose%
\pgfusepath{fill}%
\end{pgfscope}%
\begin{pgfscope}%
\pgfpathrectangle{\pgfqpoint{1.020000in}{0.880000in}}{\pgfqpoint{6.160000in}{6.160000in}}%
\pgfusepath{clip}%
\pgfsetbuttcap%
\pgfsetroundjoin%
\definecolor{currentfill}{rgb}{0.624703,0.748318,0.998719}%
\pgfsetfillcolor{currentfill}%
\pgfsetlinewidth{0.000000pt}%
\definecolor{currentstroke}{rgb}{0.000000,0.000000,0.000000}%
\pgfsetstrokecolor{currentstroke}%
\pgfsetdash{}{0pt}%
\pgfpathmoveto{\pgfqpoint{4.482131in}{3.305455in}}%
\pgfpathlineto{\pgfqpoint{4.493054in}{3.289145in}}%
\pgfpathlineto{\pgfqpoint{4.503991in}{3.270848in}}%
\pgfpathlineto{\pgfqpoint{4.537891in}{3.236918in}}%
\pgfpathlineto{\pgfqpoint{4.571749in}{3.203712in}}%
\pgfpathlineto{\pgfqpoint{4.560764in}{3.221764in}}%
\pgfpathlineto{\pgfqpoint{4.549792in}{3.238116in}}%
\pgfpathlineto{\pgfqpoint{4.515983in}{3.271423in}}%
\pgfpathlineto{\pgfqpoint{4.482131in}{3.305455in}}%
\pgfpathclose%
\pgfusepath{fill}%
\end{pgfscope}%
\begin{pgfscope}%
\pgfpathrectangle{\pgfqpoint{1.020000in}{0.880000in}}{\pgfqpoint{6.160000in}{6.160000in}}%
\pgfusepath{clip}%
\pgfsetbuttcap%
\pgfsetroundjoin%
\definecolor{currentfill}{rgb}{0.713852,0.808857,0.979386}%
\pgfsetfillcolor{currentfill}%
\pgfsetlinewidth{0.000000pt}%
\definecolor{currentstroke}{rgb}{0.000000,0.000000,0.000000}%
\pgfsetstrokecolor{currentstroke}%
\pgfsetdash{}{0pt}%
\pgfpathmoveto{\pgfqpoint{3.196728in}{3.402082in}}%
\pgfpathlineto{\pgfqpoint{3.206375in}{3.394446in}}%
\pgfpathlineto{\pgfqpoint{3.216040in}{3.387068in}}%
\pgfpathlineto{\pgfqpoint{3.250226in}{3.415290in}}%
\pgfpathlineto{\pgfqpoint{3.284408in}{3.443551in}}%
\pgfpathlineto{\pgfqpoint{3.274698in}{3.448281in}}%
\pgfpathlineto{\pgfqpoint{3.265006in}{3.453215in}}%
\pgfpathlineto{\pgfqpoint{3.230870in}{3.427652in}}%
\pgfpathlineto{\pgfqpoint{3.196728in}{3.402082in}}%
\pgfpathclose%
\pgfusepath{fill}%
\end{pgfscope}%
\begin{pgfscope}%
\pgfpathrectangle{\pgfqpoint{1.020000in}{0.880000in}}{\pgfqpoint{6.160000in}{6.160000in}}%
\pgfusepath{clip}%
\pgfsetbuttcap%
\pgfsetroundjoin%
\definecolor{currentfill}{rgb}{0.758539,0.832787,0.958408}%
\pgfsetfillcolor{currentfill}%
\pgfsetlinewidth{0.000000pt}%
\definecolor{currentstroke}{rgb}{0.000000,0.000000,0.000000}%
\pgfsetstrokecolor{currentstroke}%
\pgfsetdash{}{0pt}%
\pgfpathmoveto{\pgfqpoint{4.235169in}{3.546968in}}%
\pgfpathlineto{\pgfqpoint{4.245869in}{3.541410in}}%
\pgfpathlineto{\pgfqpoint{4.256589in}{3.533365in}}%
\pgfpathlineto{\pgfqpoint{4.290658in}{3.501407in}}%
\pgfpathlineto{\pgfqpoint{4.324682in}{3.468391in}}%
\pgfpathlineto{\pgfqpoint{4.313909in}{3.477240in}}%
\pgfpathlineto{\pgfqpoint{4.303154in}{3.483840in}}%
\pgfpathlineto{\pgfqpoint{4.269184in}{3.515912in}}%
\pgfpathlineto{\pgfqpoint{4.235169in}{3.546968in}}%
\pgfpathclose%
\pgfusepath{fill}%
\end{pgfscope}%
\begin{pgfscope}%
\pgfpathrectangle{\pgfqpoint{1.020000in}{0.880000in}}{\pgfqpoint{6.160000in}{6.160000in}}%
\pgfusepath{clip}%
\pgfsetbuttcap%
\pgfsetroundjoin%
\definecolor{currentfill}{rgb}{0.570616,0.704109,0.997195}%
\pgfsetfillcolor{currentfill}%
\pgfsetlinewidth{0.000000pt}%
\definecolor{currentstroke}{rgb}{0.000000,0.000000,0.000000}%
\pgfsetstrokecolor{currentstroke}%
\pgfsetdash{}{0pt}%
\pgfpathmoveto{\pgfqpoint{2.698228in}{3.168112in}}%
\pgfpathlineto{\pgfqpoint{2.707567in}{3.146453in}}%
\pgfpathlineto{\pgfqpoint{2.716915in}{3.125291in}}%
\pgfpathlineto{\pgfqpoint{2.751354in}{3.137722in}}%
\pgfpathlineto{\pgfqpoint{2.785762in}{3.151359in}}%
\pgfpathlineto{\pgfqpoint{2.776371in}{3.171412in}}%
\pgfpathlineto{\pgfqpoint{2.766989in}{3.192020in}}%
\pgfpathlineto{\pgfqpoint{2.732624in}{3.179518in}}%
\pgfpathlineto{\pgfqpoint{2.698228in}{3.168112in}}%
\pgfpathclose%
\pgfusepath{fill}%
\end{pgfscope}%
\begin{pgfscope}%
\pgfpathrectangle{\pgfqpoint{1.020000in}{0.880000in}}{\pgfqpoint{6.160000in}{6.160000in}}%
\pgfusepath{clip}%
\pgfsetbuttcap%
\pgfsetroundjoin%
\definecolor{currentfill}{rgb}{0.851372,0.863125,0.881064}%
\pgfsetfillcolor{currentfill}%
\pgfsetlinewidth{0.000000pt}%
\definecolor{currentstroke}{rgb}{0.000000,0.000000,0.000000}%
\pgfsetstrokecolor{currentstroke}%
\pgfsetdash{}{0pt}%
\pgfpathmoveto{\pgfqpoint{3.694590in}{3.694597in}}%
\pgfpathlineto{\pgfqpoint{3.704672in}{3.702774in}}%
\pgfpathlineto{\pgfqpoint{3.714781in}{3.709863in}}%
\pgfpathlineto{\pgfqpoint{3.749029in}{3.717542in}}%
\pgfpathlineto{\pgfqpoint{3.783271in}{3.722046in}}%
\pgfpathlineto{\pgfqpoint{3.773092in}{3.714065in}}%
\pgfpathlineto{\pgfqpoint{3.762942in}{3.704926in}}%
\pgfpathlineto{\pgfqpoint{3.728770in}{3.701228in}}%
\pgfpathlineto{\pgfqpoint{3.694590in}{3.694597in}}%
\pgfpathclose%
\pgfusepath{fill}%
\end{pgfscope}%
\begin{pgfscope}%
\pgfpathrectangle{\pgfqpoint{1.020000in}{0.880000in}}{\pgfqpoint{6.160000in}{6.160000in}}%
\pgfusepath{clip}%
\pgfsetbuttcap%
\pgfsetroundjoin%
\definecolor{currentfill}{rgb}{0.688188,0.793178,0.988038}%
\pgfsetfillcolor{currentfill}%
\pgfsetlinewidth{0.000000pt}%
\definecolor{currentstroke}{rgb}{0.000000,0.000000,0.000000}%
\pgfsetstrokecolor{currentstroke}%
\pgfsetdash{}{0pt}%
\pgfpathmoveto{\pgfqpoint{3.128415in}{3.351723in}}%
\pgfpathlineto{\pgfqpoint{3.138021in}{3.341520in}}%
\pgfpathlineto{\pgfqpoint{3.147642in}{3.331617in}}%
\pgfpathlineto{\pgfqpoint{3.181846in}{3.359109in}}%
\pgfpathlineto{\pgfqpoint{3.216040in}{3.387068in}}%
\pgfpathlineto{\pgfqpoint{3.206375in}{3.394446in}}%
\pgfpathlineto{\pgfqpoint{3.196728in}{3.402082in}}%
\pgfpathlineto{\pgfqpoint{3.162577in}{3.376709in}}%
\pgfpathlineto{\pgfqpoint{3.128415in}{3.351723in}}%
\pgfpathclose%
\pgfusepath{fill}%
\end{pgfscope}%
\begin{pgfscope}%
\pgfpathrectangle{\pgfqpoint{1.020000in}{0.880000in}}{\pgfqpoint{6.160000in}{6.160000in}}%
\pgfusepath{clip}%
\pgfsetbuttcap%
\pgfsetroundjoin%
\definecolor{currentfill}{rgb}{0.851372,0.863125,0.881064}%
\pgfsetfillcolor{currentfill}%
\pgfsetlinewidth{0.000000pt}%
\definecolor{currentstroke}{rgb}{0.000000,0.000000,0.000000}%
\pgfsetstrokecolor{currentstroke}%
\pgfsetdash{}{0pt}%
\pgfpathmoveto{\pgfqpoint{3.920115in}{3.707896in}}%
\pgfpathlineto{\pgfqpoint{3.930462in}{3.714599in}}%
\pgfpathlineto{\pgfqpoint{3.940837in}{3.719340in}}%
\pgfpathlineto{\pgfqpoint{3.975071in}{3.707591in}}%
\pgfpathlineto{\pgfqpoint{4.009279in}{3.692752in}}%
\pgfpathlineto{\pgfqpoint{3.998836in}{3.688511in}}%
\pgfpathlineto{\pgfqpoint{3.988420in}{3.682348in}}%
\pgfpathlineto{\pgfqpoint{3.954280in}{3.696564in}}%
\pgfpathlineto{\pgfqpoint{3.920115in}{3.707896in}}%
\pgfpathclose%
\pgfusepath{fill}%
\end{pgfscope}%
\begin{pgfscope}%
\pgfpathrectangle{\pgfqpoint{1.020000in}{0.880000in}}{\pgfqpoint{6.160000in}{6.160000in}}%
\pgfusepath{clip}%
\pgfsetbuttcap%
\pgfsetroundjoin%
\definecolor{currentfill}{rgb}{0.373552,0.497499,0.909467}%
\pgfsetfillcolor{currentfill}%
\pgfsetlinewidth{0.000000pt}%
\definecolor{currentstroke}{rgb}{0.000000,0.000000,0.000000}%
\pgfsetstrokecolor{currentstroke}%
\pgfsetdash{}{0pt}%
\pgfpathmoveto{\pgfqpoint{4.976192in}{2.832873in}}%
\pgfpathlineto{\pgfqpoint{4.987546in}{2.812260in}}%
\pgfpathlineto{\pgfqpoint{4.998918in}{2.791288in}}%
\pgfpathlineto{\pgfqpoint{5.032658in}{2.777796in}}%
\pgfpathlineto{\pgfqpoint{5.066379in}{2.765687in}}%
\pgfpathlineto{\pgfqpoint{5.054949in}{2.785503in}}%
\pgfpathlineto{\pgfqpoint{5.043537in}{2.805054in}}%
\pgfpathlineto{\pgfqpoint{5.009875in}{2.818225in}}%
\pgfpathlineto{\pgfqpoint{4.976192in}{2.832873in}}%
\pgfpathclose%
\pgfusepath{fill}%
\end{pgfscope}%
\begin{pgfscope}%
\pgfpathrectangle{\pgfqpoint{1.020000in}{0.880000in}}{\pgfqpoint{6.160000in}{6.160000in}}%
\pgfusepath{clip}%
\pgfsetbuttcap%
\pgfsetroundjoin%
\definecolor{currentfill}{rgb}{0.266381,0.353304,0.801637}%
\pgfsetfillcolor{currentfill}%
\pgfsetlinewidth{0.000000pt}%
\definecolor{currentstroke}{rgb}{0.000000,0.000000,0.000000}%
\pgfsetstrokecolor{currentstroke}%
\pgfsetdash{}{0pt}%
\pgfpathmoveto{\pgfqpoint{5.968262in}{2.599138in}}%
\pgfpathlineto{\pgfqpoint{5.980603in}{2.584560in}}%
\pgfpathlineto{\pgfqpoint{5.992969in}{2.570034in}}%
\pgfpathlineto{\pgfqpoint{6.026525in}{2.572718in}}%
\pgfpathlineto{\pgfqpoint{6.060060in}{2.575440in}}%
\pgfpathlineto{\pgfqpoint{6.047637in}{2.589874in}}%
\pgfpathlineto{\pgfqpoint{6.035239in}{2.604358in}}%
\pgfpathlineto{\pgfqpoint{6.001761in}{2.601726in}}%
\pgfpathlineto{\pgfqpoint{5.968262in}{2.599138in}}%
\pgfpathclose%
\pgfusepath{fill}%
\end{pgfscope}%
\begin{pgfscope}%
\pgfpathrectangle{\pgfqpoint{1.020000in}{0.880000in}}{\pgfqpoint{6.160000in}{6.160000in}}%
\pgfusepath{clip}%
\pgfsetbuttcap%
\pgfsetroundjoin%
\definecolor{currentfill}{rgb}{0.275827,0.366717,0.812553}%
\pgfsetfillcolor{currentfill}%
\pgfsetlinewidth{0.000000pt}%
\definecolor{currentstroke}{rgb}{0.000000,0.000000,0.000000}%
\pgfsetstrokecolor{currentstroke}%
\pgfsetdash{}{0pt}%
\pgfpathmoveto{\pgfqpoint{5.742589in}{2.615209in}}%
\pgfpathlineto{\pgfqpoint{5.754708in}{2.600123in}}%
\pgfpathlineto{\pgfqpoint{5.766852in}{2.585094in}}%
\pgfpathlineto{\pgfqpoint{5.800470in}{2.587193in}}%
\pgfpathlineto{\pgfqpoint{5.834068in}{2.589409in}}%
\pgfpathlineto{\pgfqpoint{5.821866in}{2.604280in}}%
\pgfpathlineto{\pgfqpoint{5.809689in}{2.619207in}}%
\pgfpathlineto{\pgfqpoint{5.776149in}{2.617141in}}%
\pgfpathlineto{\pgfqpoint{5.742589in}{2.615209in}}%
\pgfpathclose%
\pgfusepath{fill}%
\end{pgfscope}%
\begin{pgfscope}%
\pgfpathrectangle{\pgfqpoint{1.020000in}{0.880000in}}{\pgfqpoint{6.160000in}{6.160000in}}%
\pgfusepath{clip}%
\pgfsetbuttcap%
\pgfsetroundjoin%
\definecolor{currentfill}{rgb}{0.333490,0.446265,0.874452}%
\pgfsetfillcolor{currentfill}%
\pgfsetlinewidth{0.000000pt}%
\definecolor{currentstroke}{rgb}{0.000000,0.000000,0.000000}%
\pgfsetstrokecolor{currentstroke}%
\pgfsetdash{}{0pt}%
\pgfpathmoveto{\pgfqpoint{5.133764in}{2.745401in}}%
\pgfpathlineto{\pgfqpoint{5.145273in}{2.726398in}}%
\pgfpathlineto{\pgfqpoint{5.156803in}{2.707272in}}%
\pgfpathlineto{\pgfqpoint{5.190528in}{2.699888in}}%
\pgfpathlineto{\pgfqpoint{5.224237in}{2.693544in}}%
\pgfpathlineto{\pgfqpoint{5.212647in}{2.711770in}}%
\pgfpathlineto{\pgfqpoint{5.201078in}{2.729918in}}%
\pgfpathlineto{\pgfqpoint{5.167429in}{2.737095in}}%
\pgfpathlineto{\pgfqpoint{5.133764in}{2.745401in}}%
\pgfpathclose%
\pgfusepath{fill}%
\end{pgfscope}%
\begin{pgfscope}%
\pgfpathrectangle{\pgfqpoint{1.020000in}{0.880000in}}{\pgfqpoint{6.160000in}{6.160000in}}%
\pgfusepath{clip}%
\pgfsetbuttcap%
\pgfsetroundjoin%
\definecolor{currentfill}{rgb}{0.822420,0.856898,0.910795}%
\pgfsetfillcolor{currentfill}%
\pgfsetlinewidth{0.000000pt}%
\definecolor{currentstroke}{rgb}{0.000000,0.000000,0.000000}%
\pgfsetstrokecolor{currentstroke}%
\pgfsetdash{}{0pt}%
\pgfpathmoveto{\pgfqpoint{4.077606in}{3.654589in}}%
\pgfpathlineto{\pgfqpoint{4.088141in}{3.655806in}}%
\pgfpathlineto{\pgfqpoint{4.098700in}{3.654596in}}%
\pgfpathlineto{\pgfqpoint{4.132876in}{3.630807in}}%
\pgfpathlineto{\pgfqpoint{4.167014in}{3.604736in}}%
\pgfpathlineto{\pgfqpoint{4.156394in}{3.606866in}}%
\pgfpathlineto{\pgfqpoint{4.145797in}{3.606716in}}%
\pgfpathlineto{\pgfqpoint{4.111720in}{3.631733in}}%
\pgfpathlineto{\pgfqpoint{4.077606in}{3.654589in}}%
\pgfpathclose%
\pgfusepath{fill}%
\end{pgfscope}%
\begin{pgfscope}%
\pgfpathrectangle{\pgfqpoint{1.020000in}{0.880000in}}{\pgfqpoint{6.160000in}{6.160000in}}%
\pgfusepath{clip}%
\pgfsetbuttcap%
\pgfsetroundjoin%
\definecolor{currentfill}{rgb}{0.261805,0.346484,0.795658}%
\pgfsetfillcolor{currentfill}%
\pgfsetlinewidth{0.000000pt}%
\definecolor{currentstroke}{rgb}{0.000000,0.000000,0.000000}%
\pgfsetstrokecolor{currentstroke}%
\pgfsetdash{}{0pt}%
\pgfpathmoveto{\pgfqpoint{6.193997in}{2.586604in}}%
\pgfpathlineto{\pgfqpoint{6.206559in}{2.572377in}}%
\pgfpathlineto{\pgfqpoint{6.219147in}{2.558196in}}%
\pgfpathlineto{\pgfqpoint{6.252637in}{2.561106in}}%
\pgfpathlineto{\pgfqpoint{6.286106in}{2.564028in}}%
\pgfpathlineto{\pgfqpoint{6.273461in}{2.578141in}}%
\pgfpathlineto{\pgfqpoint{6.260842in}{2.592298in}}%
\pgfpathlineto{\pgfqpoint{6.227429in}{2.589444in}}%
\pgfpathlineto{\pgfqpoint{6.193997in}{2.586604in}}%
\pgfpathclose%
\pgfusepath{fill}%
\end{pgfscope}%
\begin{pgfscope}%
\pgfpathrectangle{\pgfqpoint{1.020000in}{0.880000in}}{\pgfqpoint{6.160000in}{6.160000in}}%
\pgfusepath{clip}%
\pgfsetbuttcap%
\pgfsetroundjoin%
\definecolor{currentfill}{rgb}{0.661968,0.775491,0.993937}%
\pgfsetfillcolor{currentfill}%
\pgfsetlinewidth{0.000000pt}%
\definecolor{currentstroke}{rgb}{0.000000,0.000000,0.000000}%
\pgfsetstrokecolor{currentstroke}%
\pgfsetdash{}{0pt}%
\pgfpathmoveto{\pgfqpoint{3.060052in}{3.303582in}}%
\pgfpathlineto{\pgfqpoint{3.069616in}{3.291006in}}%
\pgfpathlineto{\pgfqpoint{3.079195in}{3.278757in}}%
\pgfpathlineto{\pgfqpoint{3.113426in}{3.304778in}}%
\pgfpathlineto{\pgfqpoint{3.147642in}{3.331617in}}%
\pgfpathlineto{\pgfqpoint{3.138021in}{3.341520in}}%
\pgfpathlineto{\pgfqpoint{3.128415in}{3.351723in}}%
\pgfpathlineto{\pgfqpoint{3.094241in}{3.327297in}}%
\pgfpathlineto{\pgfqpoint{3.060052in}{3.303582in}}%
\pgfpathclose%
\pgfusepath{fill}%
\end{pgfscope}%
\begin{pgfscope}%
\pgfpathrectangle{\pgfqpoint{1.020000in}{0.880000in}}{\pgfqpoint{6.160000in}{6.160000in}}%
\pgfusepath{clip}%
\pgfsetbuttcap%
\pgfsetroundjoin%
\definecolor{currentfill}{rgb}{0.285273,0.380129,0.823469}%
\pgfsetfillcolor{currentfill}%
\pgfsetlinewidth{0.000000pt}%
\definecolor{currentstroke}{rgb}{0.000000,0.000000,0.000000}%
\pgfsetstrokecolor{currentstroke}%
\pgfsetdash{}{0pt}%
\pgfpathmoveto{\pgfqpoint{5.517015in}{2.639740in}}%
\pgfpathlineto{\pgfqpoint{5.528911in}{2.623785in}}%
\pgfpathlineto{\pgfqpoint{5.540831in}{2.607880in}}%
\pgfpathlineto{\pgfqpoint{5.574504in}{2.608436in}}%
\pgfpathlineto{\pgfqpoint{5.608158in}{2.609306in}}%
\pgfpathlineto{\pgfqpoint{5.596179in}{2.624895in}}%
\pgfpathlineto{\pgfqpoint{5.584224in}{2.640536in}}%
\pgfpathlineto{\pgfqpoint{5.550629in}{2.639962in}}%
\pgfpathlineto{\pgfqpoint{5.517015in}{2.639740in}}%
\pgfpathclose%
\pgfusepath{fill}%
\end{pgfscope}%
\begin{pgfscope}%
\pgfpathrectangle{\pgfqpoint{1.020000in}{0.880000in}}{\pgfqpoint{6.160000in}{6.160000in}}%
\pgfusepath{clip}%
\pgfsetbuttcap%
\pgfsetroundjoin%
\definecolor{currentfill}{rgb}{0.570616,0.704109,0.997195}%
\pgfsetfillcolor{currentfill}%
\pgfsetlinewidth{0.000000pt}%
\definecolor{currentstroke}{rgb}{0.000000,0.000000,0.000000}%
\pgfsetstrokecolor{currentstroke}%
\pgfsetdash{}{0pt}%
\pgfpathmoveto{\pgfqpoint{4.571749in}{3.203712in}}%
\pgfpathlineto{\pgfqpoint{4.582749in}{3.183963in}}%
\pgfpathlineto{\pgfqpoint{4.593762in}{3.162547in}}%
\pgfpathlineto{\pgfqpoint{4.627628in}{3.130934in}}%
\pgfpathlineto{\pgfqpoint{4.661457in}{3.100444in}}%
\pgfpathlineto{\pgfqpoint{4.650395in}{3.121091in}}%
\pgfpathlineto{\pgfqpoint{4.639347in}{3.140335in}}%
\pgfpathlineto{\pgfqpoint{4.605567in}{3.171452in}}%
\pgfpathlineto{\pgfqpoint{4.571749in}{3.203712in}}%
\pgfpathclose%
\pgfusepath{fill}%
\end{pgfscope}%
\begin{pgfscope}%
\pgfpathrectangle{\pgfqpoint{1.020000in}{0.880000in}}{\pgfqpoint{6.160000in}{6.160000in}}%
\pgfusepath{clip}%
\pgfsetbuttcap%
\pgfsetroundjoin%
\definecolor{currentfill}{rgb}{0.435815,0.570707,0.951717}%
\pgfsetfillcolor{currentfill}%
\pgfsetlinewidth{0.000000pt}%
\definecolor{currentstroke}{rgb}{0.000000,0.000000,0.000000}%
\pgfsetstrokecolor{currentstroke}%
\pgfsetdash{}{0pt}%
\pgfpathmoveto{\pgfqpoint{4.818803in}{2.951348in}}%
\pgfpathlineto{\pgfqpoint{4.830012in}{2.929489in}}%
\pgfpathlineto{\pgfqpoint{4.841236in}{2.906880in}}%
\pgfpathlineto{\pgfqpoint{4.875012in}{2.886041in}}%
\pgfpathlineto{\pgfqpoint{4.908762in}{2.866759in}}%
\pgfpathlineto{\pgfqpoint{4.897482in}{2.888082in}}%
\pgfpathlineto{\pgfqpoint{4.886219in}{2.908814in}}%
\pgfpathlineto{\pgfqpoint{4.852524in}{2.929263in}}%
\pgfpathlineto{\pgfqpoint{4.818803in}{2.951348in}}%
\pgfpathclose%
\pgfusepath{fill}%
\end{pgfscope}%
\begin{pgfscope}%
\pgfpathrectangle{\pgfqpoint{1.020000in}{0.880000in}}{\pgfqpoint{6.160000in}{6.160000in}}%
\pgfusepath{clip}%
\pgfsetbuttcap%
\pgfsetroundjoin%
\definecolor{currentfill}{rgb}{0.565182,0.699438,0.996635}%
\pgfsetfillcolor{currentfill}%
\pgfsetlinewidth{0.000000pt}%
\definecolor{currentstroke}{rgb}{0.000000,0.000000,0.000000}%
\pgfsetstrokecolor{currentstroke}%
\pgfsetdash{}{0pt}%
\pgfpathmoveto{\pgfqpoint{2.629342in}{3.148477in}}%
\pgfpathlineto{\pgfqpoint{2.638637in}{3.125969in}}%
\pgfpathlineto{\pgfqpoint{2.647941in}{3.103902in}}%
\pgfpathlineto{\pgfqpoint{2.682444in}{3.114031in}}%
\pgfpathlineto{\pgfqpoint{2.716915in}{3.125291in}}%
\pgfpathlineto{\pgfqpoint{2.707567in}{3.146453in}}%
\pgfpathlineto{\pgfqpoint{2.698228in}{3.168112in}}%
\pgfpathlineto{\pgfqpoint{2.663801in}{3.157776in}}%
\pgfpathlineto{\pgfqpoint{2.629342in}{3.148477in}}%
\pgfpathclose%
\pgfusepath{fill}%
\end{pgfscope}%
\begin{pgfscope}%
\pgfpathrectangle{\pgfqpoint{1.020000in}{0.880000in}}{\pgfqpoint{6.160000in}{6.160000in}}%
\pgfusepath{clip}%
\pgfsetbuttcap%
\pgfsetroundjoin%
\definecolor{currentfill}{rgb}{0.635474,0.756714,0.998297}%
\pgfsetfillcolor{currentfill}%
\pgfsetlinewidth{0.000000pt}%
\definecolor{currentstroke}{rgb}{0.000000,0.000000,0.000000}%
\pgfsetstrokecolor{currentstroke}%
\pgfsetdash{}{0pt}%
\pgfpathmoveto{\pgfqpoint{2.991620in}{3.258804in}}%
\pgfpathlineto{\pgfqpoint{3.001144in}{3.244087in}}%
\pgfpathlineto{\pgfqpoint{3.010682in}{3.229714in}}%
\pgfpathlineto{\pgfqpoint{3.044948in}{3.253696in}}%
\pgfpathlineto{\pgfqpoint{3.079195in}{3.278757in}}%
\pgfpathlineto{\pgfqpoint{3.069616in}{3.291006in}}%
\pgfpathlineto{\pgfqpoint{3.060052in}{3.303582in}}%
\pgfpathlineto{\pgfqpoint{3.025846in}{3.280714in}}%
\pgfpathlineto{\pgfqpoint{2.991620in}{3.258804in}}%
\pgfpathclose%
\pgfusepath{fill}%
\end{pgfscope}%
\begin{pgfscope}%
\pgfpathrectangle{\pgfqpoint{1.020000in}{0.880000in}}{\pgfqpoint{6.160000in}{6.160000in}}%
\pgfusepath{clip}%
\pgfsetbuttcap%
\pgfsetroundjoin%
\definecolor{currentfill}{rgb}{0.847365,0.862472,0.885540}%
\pgfsetfillcolor{currentfill}%
\pgfsetlinewidth{0.000000pt}%
\definecolor{currentstroke}{rgb}{0.000000,0.000000,0.000000}%
\pgfsetstrokecolor{currentstroke}%
\pgfsetdash{}{0pt}%
\pgfpathmoveto{\pgfqpoint{3.626219in}{3.672889in}}%
\pgfpathlineto{\pgfqpoint{3.636233in}{3.679633in}}%
\pgfpathlineto{\pgfqpoint{3.646276in}{3.685380in}}%
\pgfpathlineto{\pgfqpoint{3.680529in}{3.699100in}}%
\pgfpathlineto{\pgfqpoint{3.714781in}{3.709863in}}%
\pgfpathlineto{\pgfqpoint{3.704672in}{3.702774in}}%
\pgfpathlineto{\pgfqpoint{3.694590in}{3.694597in}}%
\pgfpathlineto{\pgfqpoint{3.660406in}{3.685113in}}%
\pgfpathlineto{\pgfqpoint{3.626219in}{3.672889in}}%
\pgfpathclose%
\pgfusepath{fill}%
\end{pgfscope}%
\begin{pgfscope}%
\pgfpathrectangle{\pgfqpoint{1.020000in}{0.880000in}}{\pgfqpoint{6.160000in}{6.160000in}}%
\pgfusepath{clip}%
\pgfsetbuttcap%
\pgfsetroundjoin%
\definecolor{currentfill}{rgb}{0.309060,0.413498,0.850128}%
\pgfsetfillcolor{currentfill}%
\pgfsetlinewidth{0.000000pt}%
\definecolor{currentstroke}{rgb}{0.000000,0.000000,0.000000}%
\pgfsetstrokecolor{currentstroke}%
\pgfsetdash{}{0pt}%
\pgfpathmoveto{\pgfqpoint{5.291602in}{2.683696in}}%
\pgfpathlineto{\pgfqpoint{5.303273in}{2.666181in}}%
\pgfpathlineto{\pgfqpoint{5.314967in}{2.648664in}}%
\pgfpathlineto{\pgfqpoint{5.348684in}{2.645668in}}%
\pgfpathlineto{\pgfqpoint{5.382385in}{2.643358in}}%
\pgfpathlineto{\pgfqpoint{5.370631in}{2.660247in}}%
\pgfpathlineto{\pgfqpoint{5.358900in}{2.677152in}}%
\pgfpathlineto{\pgfqpoint{5.325259in}{2.680047in}}%
\pgfpathlineto{\pgfqpoint{5.291602in}{2.683696in}}%
\pgfpathclose%
\pgfusepath{fill}%
\end{pgfscope}%
\begin{pgfscope}%
\pgfpathrectangle{\pgfqpoint{1.020000in}{0.880000in}}{\pgfqpoint{6.160000in}{6.160000in}}%
\pgfusepath{clip}%
\pgfsetbuttcap%
\pgfsetroundjoin%
\definecolor{currentfill}{rgb}{0.718985,0.811993,0.977656}%
\pgfsetfillcolor{currentfill}%
\pgfsetlinewidth{0.000000pt}%
\definecolor{currentstroke}{rgb}{0.000000,0.000000,0.000000}%
\pgfsetstrokecolor{currentstroke}%
\pgfsetdash{}{0pt}%
\pgfpathmoveto{\pgfqpoint{4.324682in}{3.468391in}}%
\pgfpathlineto{\pgfqpoint{4.335474in}{3.457164in}}%
\pgfpathlineto{\pgfqpoint{4.346283in}{3.443462in}}%
\pgfpathlineto{\pgfqpoint{4.380313in}{3.409271in}}%
\pgfpathlineto{\pgfqpoint{4.414297in}{3.374666in}}%
\pgfpathlineto{\pgfqpoint{4.403438in}{3.388684in}}%
\pgfpathlineto{\pgfqpoint{4.392596in}{3.400505in}}%
\pgfpathlineto{\pgfqpoint{4.358662in}{3.434649in}}%
\pgfpathlineto{\pgfqpoint{4.324682in}{3.468391in}}%
\pgfpathclose%
\pgfusepath{fill}%
\end{pgfscope}%
\begin{pgfscope}%
\pgfpathrectangle{\pgfqpoint{1.020000in}{0.880000in}}{\pgfqpoint{6.160000in}{6.160000in}}%
\pgfusepath{clip}%
\pgfsetbuttcap%
\pgfsetroundjoin%
\definecolor{currentfill}{rgb}{0.257234,0.339661,0.789661}%
\pgfsetfillcolor{currentfill}%
\pgfsetlinewidth{0.000000pt}%
\definecolor{currentstroke}{rgb}{0.000000,0.000000,0.000000}%
\pgfsetstrokecolor{currentstroke}%
\pgfsetdash{}{0pt}%
\pgfpathmoveto{\pgfqpoint{6.352983in}{2.569902in}}%
\pgfpathlineto{\pgfqpoint{6.365710in}{2.555898in}}%
\pgfpathlineto{\pgfqpoint{6.399146in}{2.558879in}}%
\pgfpathlineto{\pgfqpoint{6.386390in}{2.572852in}}%
\pgfpathlineto{\pgfqpoint{6.352983in}{2.569902in}}%
\pgfpathclose%
\pgfusepath{fill}%
\end{pgfscope}%
\begin{pgfscope}%
\pgfpathrectangle{\pgfqpoint{1.020000in}{0.880000in}}{\pgfqpoint{6.160000in}{6.160000in}}%
\pgfusepath{clip}%
\pgfsetbuttcap%
\pgfsetroundjoin%
\definecolor{currentfill}{rgb}{0.613933,0.739923,0.999142}%
\pgfsetfillcolor{currentfill}%
\pgfsetlinewidth{0.000000pt}%
\definecolor{currentstroke}{rgb}{0.000000,0.000000,0.000000}%
\pgfsetstrokecolor{currentstroke}%
\pgfsetdash{}{0pt}%
\pgfpathmoveto{\pgfqpoint{2.923104in}{3.218215in}}%
\pgfpathlineto{\pgfqpoint{2.932588in}{3.201611in}}%
\pgfpathlineto{\pgfqpoint{2.942085in}{3.185358in}}%
\pgfpathlineto{\pgfqpoint{2.976395in}{3.206910in}}%
\pgfpathlineto{\pgfqpoint{3.010682in}{3.229714in}}%
\pgfpathlineto{\pgfqpoint{3.001144in}{3.244087in}}%
\pgfpathlineto{\pgfqpoint{2.991620in}{3.258804in}}%
\pgfpathlineto{\pgfqpoint{2.957374in}{3.237947in}}%
\pgfpathlineto{\pgfqpoint{2.923104in}{3.218215in}}%
\pgfpathclose%
\pgfusepath{fill}%
\end{pgfscope}%
\begin{pgfscope}%
\pgfpathrectangle{\pgfqpoint{1.020000in}{0.880000in}}{\pgfqpoint{6.160000in}{6.160000in}}%
\pgfusepath{clip}%
\pgfsetbuttcap%
\pgfsetroundjoin%
\definecolor{currentfill}{rgb}{0.271104,0.360011,0.807095}%
\pgfsetfillcolor{currentfill}%
\pgfsetlinewidth{0.000000pt}%
\definecolor{currentstroke}{rgb}{0.000000,0.000000,0.000000}%
\pgfsetstrokecolor{currentstroke}%
\pgfsetdash{}{0pt}%
\pgfpathmoveto{\pgfqpoint{5.901205in}{2.594127in}}%
\pgfpathlineto{\pgfqpoint{5.913489in}{2.579441in}}%
\pgfpathlineto{\pgfqpoint{5.925797in}{2.564809in}}%
\pgfpathlineto{\pgfqpoint{5.959393in}{2.567395in}}%
\pgfpathlineto{\pgfqpoint{5.992969in}{2.570034in}}%
\pgfpathlineto{\pgfqpoint{5.980603in}{2.584560in}}%
\pgfpathlineto{\pgfqpoint{5.968262in}{2.599138in}}%
\pgfpathlineto{\pgfqpoint{5.934744in}{2.596602in}}%
\pgfpathlineto{\pgfqpoint{5.901205in}{2.594127in}}%
\pgfpathclose%
\pgfusepath{fill}%
\end{pgfscope}%
\begin{pgfscope}%
\pgfpathrectangle{\pgfqpoint{1.020000in}{0.880000in}}{\pgfqpoint{6.160000in}{6.160000in}}%
\pgfusepath{clip}%
\pgfsetbuttcap%
\pgfsetroundjoin%
\definecolor{currentfill}{rgb}{0.275827,0.366717,0.812553}%
\pgfsetfillcolor{currentfill}%
\pgfsetlinewidth{0.000000pt}%
\definecolor{currentstroke}{rgb}{0.000000,0.000000,0.000000}%
\pgfsetstrokecolor{currentstroke}%
\pgfsetdash{}{0pt}%
\pgfpathmoveto{\pgfqpoint{5.675411in}{2.611837in}}%
\pgfpathlineto{\pgfqpoint{5.687472in}{2.596554in}}%
\pgfpathlineto{\pgfqpoint{5.699557in}{2.581330in}}%
\pgfpathlineto{\pgfqpoint{5.733214in}{2.583133in}}%
\pgfpathlineto{\pgfqpoint{5.766852in}{2.585094in}}%
\pgfpathlineto{\pgfqpoint{5.754708in}{2.600123in}}%
\pgfpathlineto{\pgfqpoint{5.742589in}{2.615209in}}%
\pgfpathlineto{\pgfqpoint{5.709009in}{2.613433in}}%
\pgfpathlineto{\pgfqpoint{5.675411in}{2.611837in}}%
\pgfpathclose%
\pgfusepath{fill}%
\end{pgfscope}%
\begin{pgfscope}%
\pgfpathrectangle{\pgfqpoint{1.020000in}{0.880000in}}{\pgfqpoint{6.160000in}{6.160000in}}%
\pgfusepath{clip}%
\pgfsetbuttcap%
\pgfsetroundjoin%
\definecolor{currentfill}{rgb}{0.261805,0.346484,0.795658}%
\pgfsetfillcolor{currentfill}%
\pgfsetlinewidth{0.000000pt}%
\definecolor{currentstroke}{rgb}{0.000000,0.000000,0.000000}%
\pgfsetstrokecolor{currentstroke}%
\pgfsetdash{}{0pt}%
\pgfpathmoveto{\pgfqpoint{6.127069in}{2.580976in}}%
\pgfpathlineto{\pgfqpoint{6.139575in}{2.566674in}}%
\pgfpathlineto{\pgfqpoint{6.152106in}{2.552421in}}%
\pgfpathlineto{\pgfqpoint{6.185637in}{2.555300in}}%
\pgfpathlineto{\pgfqpoint{6.219147in}{2.558196in}}%
\pgfpathlineto{\pgfqpoint{6.206559in}{2.572377in}}%
\pgfpathlineto{\pgfqpoint{6.193997in}{2.586604in}}%
\pgfpathlineto{\pgfqpoint{6.160543in}{2.583780in}}%
\pgfpathlineto{\pgfqpoint{6.127069in}{2.580976in}}%
\pgfpathclose%
\pgfusepath{fill}%
\end{pgfscope}%
\begin{pgfscope}%
\pgfpathrectangle{\pgfqpoint{1.020000in}{0.880000in}}{\pgfqpoint{6.160000in}{6.160000in}}%
\pgfusepath{clip}%
\pgfsetbuttcap%
\pgfsetroundjoin%
\definecolor{currentfill}{rgb}{0.516260,0.654498,0.986407}%
\pgfsetfillcolor{currentfill}%
\pgfsetlinewidth{0.000000pt}%
\definecolor{currentstroke}{rgb}{0.000000,0.000000,0.000000}%
\pgfsetstrokecolor{currentstroke}%
\pgfsetdash{}{0pt}%
\pgfpathmoveto{\pgfqpoint{4.661457in}{3.100444in}}%
\pgfpathlineto{\pgfqpoint{4.672533in}{3.078438in}}%
\pgfpathlineto{\pgfqpoint{4.683622in}{3.055132in}}%
\pgfpathlineto{\pgfqpoint{4.717466in}{3.027024in}}%
\pgfpathlineto{\pgfqpoint{4.751276in}{3.000293in}}%
\pgfpathlineto{\pgfqpoint{4.740135in}{3.022414in}}%
\pgfpathlineto{\pgfqpoint{4.729008in}{3.043464in}}%
\pgfpathlineto{\pgfqpoint{4.695250in}{3.071243in}}%
\pgfpathlineto{\pgfqpoint{4.661457in}{3.100444in}}%
\pgfpathclose%
\pgfusepath{fill}%
\end{pgfscope}%
\begin{pgfscope}%
\pgfpathrectangle{\pgfqpoint{1.020000in}{0.880000in}}{\pgfqpoint{6.160000in}{6.160000in}}%
\pgfusepath{clip}%
\pgfsetbuttcap%
\pgfsetroundjoin%
\definecolor{currentfill}{rgb}{0.839351,0.861167,0.894494}%
\pgfsetfillcolor{currentfill}%
\pgfsetlinewidth{0.000000pt}%
\definecolor{currentstroke}{rgb}{0.000000,0.000000,0.000000}%
\pgfsetstrokecolor{currentstroke}%
\pgfsetdash{}{0pt}%
\pgfpathmoveto{\pgfqpoint{3.557844in}{3.640839in}}%
\pgfpathlineto{\pgfqpoint{3.567794in}{3.645742in}}%
\pgfpathlineto{\pgfqpoint{3.577773in}{3.649755in}}%
\pgfpathlineto{\pgfqpoint{3.612023in}{3.668867in}}%
\pgfpathlineto{\pgfqpoint{3.646276in}{3.685380in}}%
\pgfpathlineto{\pgfqpoint{3.636233in}{3.679633in}}%
\pgfpathlineto{\pgfqpoint{3.626219in}{3.672889in}}%
\pgfpathlineto{\pgfqpoint{3.592031in}{3.658072in}}%
\pgfpathlineto{\pgfqpoint{3.557844in}{3.640839in}}%
\pgfpathclose%
\pgfusepath{fill}%
\end{pgfscope}%
\begin{pgfscope}%
\pgfpathrectangle{\pgfqpoint{1.020000in}{0.880000in}}{\pgfqpoint{6.160000in}{6.160000in}}%
\pgfusepath{clip}%
\pgfsetbuttcap%
\pgfsetroundjoin%
\definecolor{currentfill}{rgb}{0.863392,0.865084,0.867634}%
\pgfsetfillcolor{currentfill}%
\pgfsetlinewidth{0.000000pt}%
\definecolor{currentstroke}{rgb}{0.000000,0.000000,0.000000}%
\pgfsetstrokecolor{currentstroke}%
\pgfsetdash{}{0pt}%
\pgfpathmoveto{\pgfqpoint{3.851724in}{3.721365in}}%
\pgfpathlineto{\pgfqpoint{3.862001in}{3.728162in}}%
\pgfpathlineto{\pgfqpoint{3.872307in}{3.732993in}}%
\pgfpathlineto{\pgfqpoint{3.906582in}{3.727845in}}%
\pgfpathlineto{\pgfqpoint{3.940837in}{3.719340in}}%
\pgfpathlineto{\pgfqpoint{3.930462in}{3.714599in}}%
\pgfpathlineto{\pgfqpoint{3.920115in}{3.707896in}}%
\pgfpathlineto{\pgfqpoint{3.885929in}{3.716199in}}%
\pgfpathlineto{\pgfqpoint{3.851724in}{3.721365in}}%
\pgfpathclose%
\pgfusepath{fill}%
\end{pgfscope}%
\begin{pgfscope}%
\pgfpathrectangle{\pgfqpoint{1.020000in}{0.880000in}}{\pgfqpoint{6.160000in}{6.160000in}}%
\pgfusepath{clip}%
\pgfsetbuttcap%
\pgfsetroundjoin%
\definecolor{currentfill}{rgb}{0.289996,0.386836,0.828926}%
\pgfsetfillcolor{currentfill}%
\pgfsetlinewidth{0.000000pt}%
\definecolor{currentstroke}{rgb}{0.000000,0.000000,0.000000}%
\pgfsetstrokecolor{currentstroke}%
\pgfsetdash{}{0pt}%
\pgfpathmoveto{\pgfqpoint{5.449735in}{2.640540in}}%
\pgfpathlineto{\pgfqpoint{5.461572in}{2.624186in}}%
\pgfpathlineto{\pgfqpoint{5.473431in}{2.607880in}}%
\pgfpathlineto{\pgfqpoint{5.507140in}{2.607680in}}%
\pgfpathlineto{\pgfqpoint{5.540831in}{2.607880in}}%
\pgfpathlineto{\pgfqpoint{5.528911in}{2.623785in}}%
\pgfpathlineto{\pgfqpoint{5.517015in}{2.639740in}}%
\pgfpathlineto{\pgfqpoint{5.483384in}{2.639915in}}%
\pgfpathlineto{\pgfqpoint{5.449735in}{2.640540in}}%
\pgfpathclose%
\pgfusepath{fill}%
\end{pgfscope}%
\begin{pgfscope}%
\pgfpathrectangle{\pgfqpoint{1.020000in}{0.880000in}}{\pgfqpoint{6.160000in}{6.160000in}}%
\pgfusepath{clip}%
\pgfsetbuttcap%
\pgfsetroundjoin%
\definecolor{currentfill}{rgb}{0.592356,0.722792,0.999434}%
\pgfsetfillcolor{currentfill}%
\pgfsetlinewidth{0.000000pt}%
\definecolor{currentstroke}{rgb}{0.000000,0.000000,0.000000}%
\pgfsetstrokecolor{currentstroke}%
\pgfsetdash{}{0pt}%
\pgfpathmoveto{\pgfqpoint{2.854489in}{3.182326in}}%
\pgfpathlineto{\pgfqpoint{2.863932in}{3.164095in}}%
\pgfpathlineto{\pgfqpoint{2.873387in}{3.146210in}}%
\pgfpathlineto{\pgfqpoint{2.907749in}{3.165113in}}%
\pgfpathlineto{\pgfqpoint{2.942085in}{3.185358in}}%
\pgfpathlineto{\pgfqpoint{2.932588in}{3.201611in}}%
\pgfpathlineto{\pgfqpoint{2.923104in}{3.218215in}}%
\pgfpathlineto{\pgfqpoint{2.888809in}{3.199663in}}%
\pgfpathlineto{\pgfqpoint{2.854489in}{3.182326in}}%
\pgfpathclose%
\pgfusepath{fill}%
\end{pgfscope}%
\begin{pgfscope}%
\pgfpathrectangle{\pgfqpoint{1.020000in}{0.880000in}}{\pgfqpoint{6.160000in}{6.160000in}}%
\pgfusepath{clip}%
\pgfsetbuttcap%
\pgfsetroundjoin%
\definecolor{currentfill}{rgb}{0.257234,0.339661,0.789661}%
\pgfsetfillcolor{currentfill}%
\pgfsetlinewidth{0.000000pt}%
\definecolor{currentstroke}{rgb}{0.000000,0.000000,0.000000}%
\pgfsetstrokecolor{currentstroke}%
\pgfsetdash{}{0pt}%
\pgfpathmoveto{\pgfqpoint{6.286106in}{2.564028in}}%
\pgfpathlineto{\pgfqpoint{6.298777in}{2.549960in}}%
\pgfpathlineto{\pgfqpoint{6.332254in}{2.552925in}}%
\pgfpathlineto{\pgfqpoint{6.365710in}{2.555898in}}%
\pgfpathlineto{\pgfqpoint{6.352983in}{2.569902in}}%
\pgfpathlineto{\pgfqpoint{6.319555in}{2.566961in}}%
\pgfpathlineto{\pgfqpoint{6.286106in}{2.564028in}}%
\pgfpathclose%
\pgfusepath{fill}%
\end{pgfscope}%
\begin{pgfscope}%
\pgfpathrectangle{\pgfqpoint{1.020000in}{0.880000in}}{\pgfqpoint{6.160000in}{6.160000in}}%
\pgfusepath{clip}%
\pgfsetbuttcap%
\pgfsetroundjoin%
\definecolor{currentfill}{rgb}{0.348323,0.465711,0.888346}%
\pgfsetfillcolor{currentfill}%
\pgfsetlinewidth{0.000000pt}%
\definecolor{currentstroke}{rgb}{0.000000,0.000000,0.000000}%
\pgfsetstrokecolor{currentstroke}%
\pgfsetdash{}{0pt}%
\pgfpathmoveto{\pgfqpoint{5.066379in}{2.765687in}}%
\pgfpathlineto{\pgfqpoint{5.077829in}{2.745647in}}%
\pgfpathlineto{\pgfqpoint{5.089297in}{2.725426in}}%
\pgfpathlineto{\pgfqpoint{5.123059in}{2.715764in}}%
\pgfpathlineto{\pgfqpoint{5.156803in}{2.707272in}}%
\pgfpathlineto{\pgfqpoint{5.145273in}{2.726398in}}%
\pgfpathlineto{\pgfqpoint{5.133764in}{2.745401in}}%
\pgfpathlineto{\pgfqpoint{5.100081in}{2.754909in}}%
\pgfpathlineto{\pgfqpoint{5.066379in}{2.765687in}}%
\pgfpathclose%
\pgfusepath{fill}%
\end{pgfscope}%
\begin{pgfscope}%
\pgfpathrectangle{\pgfqpoint{1.020000in}{0.880000in}}{\pgfqpoint{6.160000in}{6.160000in}}%
\pgfusepath{clip}%
\pgfsetbuttcap%
\pgfsetroundjoin%
\definecolor{currentfill}{rgb}{0.394042,0.522413,0.924916}%
\pgfsetfillcolor{currentfill}%
\pgfsetlinewidth{0.000000pt}%
\definecolor{currentstroke}{rgb}{0.000000,0.000000,0.000000}%
\pgfsetstrokecolor{currentstroke}%
\pgfsetdash{}{0pt}%
\pgfpathmoveto{\pgfqpoint{4.908762in}{2.866759in}}%
\pgfpathlineto{\pgfqpoint{4.920059in}{2.844902in}}%
\pgfpathlineto{\pgfqpoint{4.931372in}{2.822573in}}%
\pgfpathlineto{\pgfqpoint{4.965156in}{2.806204in}}%
\pgfpathlineto{\pgfqpoint{4.998918in}{2.791288in}}%
\pgfpathlineto{\pgfqpoint{4.987546in}{2.812260in}}%
\pgfpathlineto{\pgfqpoint{4.976192in}{2.832873in}}%
\pgfpathlineto{\pgfqpoint{4.942489in}{2.849041in}}%
\pgfpathlineto{\pgfqpoint{4.908762in}{2.866759in}}%
\pgfpathclose%
\pgfusepath{fill}%
\end{pgfscope}%
\begin{pgfscope}%
\pgfpathrectangle{\pgfqpoint{1.020000in}{0.880000in}}{\pgfqpoint{6.160000in}{6.160000in}}%
\pgfusepath{clip}%
\pgfsetbuttcap%
\pgfsetroundjoin%
\definecolor{currentfill}{rgb}{0.672538,0.782861,0.991982}%
\pgfsetfillcolor{currentfill}%
\pgfsetlinewidth{0.000000pt}%
\definecolor{currentstroke}{rgb}{0.000000,0.000000,0.000000}%
\pgfsetstrokecolor{currentstroke}%
\pgfsetdash{}{0pt}%
\pgfpathmoveto{\pgfqpoint{4.414297in}{3.374666in}}%
\pgfpathlineto{\pgfqpoint{4.425171in}{3.358394in}}%
\pgfpathlineto{\pgfqpoint{4.436061in}{3.339844in}}%
\pgfpathlineto{\pgfqpoint{4.470048in}{3.305248in}}%
\pgfpathlineto{\pgfqpoint{4.503991in}{3.270848in}}%
\pgfpathlineto{\pgfqpoint{4.493054in}{3.289145in}}%
\pgfpathlineto{\pgfqpoint{4.482131in}{3.305455in}}%
\pgfpathlineto{\pgfqpoint{4.448236in}{3.339961in}}%
\pgfpathlineto{\pgfqpoint{4.414297in}{3.374666in}}%
\pgfpathclose%
\pgfusepath{fill}%
\end{pgfscope}%
\begin{pgfscope}%
\pgfpathrectangle{\pgfqpoint{1.020000in}{0.880000in}}{\pgfqpoint{6.160000in}{6.160000in}}%
\pgfusepath{clip}%
\pgfsetbuttcap%
\pgfsetroundjoin%
\definecolor{currentfill}{rgb}{0.796064,0.848693,0.933471}%
\pgfsetfillcolor{currentfill}%
\pgfsetlinewidth{0.000000pt}%
\definecolor{currentstroke}{rgb}{0.000000,0.000000,0.000000}%
\pgfsetstrokecolor{currentstroke}%
\pgfsetdash{}{0pt}%
\pgfpathmoveto{\pgfqpoint{4.167014in}{3.604736in}}%
\pgfpathlineto{\pgfqpoint{4.177657in}{3.600099in}}%
\pgfpathlineto{\pgfqpoint{4.188321in}{3.592762in}}%
\pgfpathlineto{\pgfqpoint{4.222476in}{3.563928in}}%
\pgfpathlineto{\pgfqpoint{4.256589in}{3.533365in}}%
\pgfpathlineto{\pgfqpoint{4.245869in}{3.541410in}}%
\pgfpathlineto{\pgfqpoint{4.235169in}{3.546968in}}%
\pgfpathlineto{\pgfqpoint{4.201113in}{3.576684in}}%
\pgfpathlineto{\pgfqpoint{4.167014in}{3.604736in}}%
\pgfpathclose%
\pgfusepath{fill}%
\end{pgfscope}%
\begin{pgfscope}%
\pgfpathrectangle{\pgfqpoint{1.020000in}{0.880000in}}{\pgfqpoint{6.160000in}{6.160000in}}%
\pgfusepath{clip}%
\pgfsetbuttcap%
\pgfsetroundjoin%
\definecolor{currentfill}{rgb}{0.822420,0.856898,0.910795}%
\pgfsetfillcolor{currentfill}%
\pgfsetlinewidth{0.000000pt}%
\definecolor{currentstroke}{rgb}{0.000000,0.000000,0.000000}%
\pgfsetstrokecolor{currentstroke}%
\pgfsetdash{}{0pt}%
\pgfpathmoveto{\pgfqpoint{3.489474in}{3.599948in}}%
\pgfpathlineto{\pgfqpoint{3.499365in}{3.602680in}}%
\pgfpathlineto{\pgfqpoint{3.509282in}{3.604641in}}%
\pgfpathlineto{\pgfqpoint{3.543525in}{3.628265in}}%
\pgfpathlineto{\pgfqpoint{3.577773in}{3.649755in}}%
\pgfpathlineto{\pgfqpoint{3.567794in}{3.645742in}}%
\pgfpathlineto{\pgfqpoint{3.557844in}{3.640839in}}%
\pgfpathlineto{\pgfqpoint{3.523658in}{3.621390in}}%
\pgfpathlineto{\pgfqpoint{3.489474in}{3.599948in}}%
\pgfpathclose%
\pgfusepath{fill}%
\end{pgfscope}%
\begin{pgfscope}%
\pgfpathrectangle{\pgfqpoint{1.020000in}{0.880000in}}{\pgfqpoint{6.160000in}{6.160000in}}%
\pgfusepath{clip}%
\pgfsetbuttcap%
\pgfsetroundjoin%
\definecolor{currentfill}{rgb}{0.313946,0.420052,0.854993}%
\pgfsetfillcolor{currentfill}%
\pgfsetlinewidth{0.000000pt}%
\definecolor{currentstroke}{rgb}{0.000000,0.000000,0.000000}%
\pgfsetstrokecolor{currentstroke}%
\pgfsetdash{}{0pt}%
\pgfpathmoveto{\pgfqpoint{5.224237in}{2.693544in}}%
\pgfpathlineto{\pgfqpoint{5.235848in}{2.675268in}}%
\pgfpathlineto{\pgfqpoint{5.247480in}{2.656967in}}%
\pgfpathlineto{\pgfqpoint{5.281232in}{2.652408in}}%
\pgfpathlineto{\pgfqpoint{5.314967in}{2.648664in}}%
\pgfpathlineto{\pgfqpoint{5.303273in}{2.666181in}}%
\pgfpathlineto{\pgfqpoint{5.291602in}{2.683696in}}%
\pgfpathlineto{\pgfqpoint{5.257928in}{2.688171in}}%
\pgfpathlineto{\pgfqpoint{5.224237in}{2.693544in}}%
\pgfpathclose%
\pgfusepath{fill}%
\end{pgfscope}%
\begin{pgfscope}%
\pgfpathrectangle{\pgfqpoint{1.020000in}{0.880000in}}{\pgfqpoint{6.160000in}{6.160000in}}%
\pgfusepath{clip}%
\pgfsetbuttcap%
\pgfsetroundjoin%
\definecolor{currentfill}{rgb}{0.843358,0.861820,0.890017}%
\pgfsetfillcolor{currentfill}%
\pgfsetlinewidth{0.000000pt}%
\definecolor{currentstroke}{rgb}{0.000000,0.000000,0.000000}%
\pgfsetstrokecolor{currentstroke}%
\pgfsetdash{}{0pt}%
\pgfpathmoveto{\pgfqpoint{4.009279in}{3.692752in}}%
\pgfpathlineto{\pgfqpoint{4.019749in}{3.694749in}}%
\pgfpathlineto{\pgfqpoint{4.030245in}{3.694208in}}%
\pgfpathlineto{\pgfqpoint{4.064489in}{3.675816in}}%
\pgfpathlineto{\pgfqpoint{4.098700in}{3.654596in}}%
\pgfpathlineto{\pgfqpoint{4.088141in}{3.655806in}}%
\pgfpathlineto{\pgfqpoint{4.077606in}{3.654589in}}%
\pgfpathlineto{\pgfqpoint{4.043458in}{3.675011in}}%
\pgfpathlineto{\pgfqpoint{4.009279in}{3.692752in}}%
\pgfpathclose%
\pgfusepath{fill}%
\end{pgfscope}%
\begin{pgfscope}%
\pgfpathrectangle{\pgfqpoint{1.020000in}{0.880000in}}{\pgfqpoint{6.160000in}{6.160000in}}%
\pgfusepath{clip}%
\pgfsetbuttcap%
\pgfsetroundjoin%
\definecolor{currentfill}{rgb}{0.576051,0.708780,0.997755}%
\pgfsetfillcolor{currentfill}%
\pgfsetlinewidth{0.000000pt}%
\definecolor{currentstroke}{rgb}{0.000000,0.000000,0.000000}%
\pgfsetstrokecolor{currentstroke}%
\pgfsetdash{}{0pt}%
\pgfpathmoveto{\pgfqpoint{2.785762in}{3.151359in}}%
\pgfpathlineto{\pgfqpoint{2.795164in}{3.131751in}}%
\pgfpathlineto{\pgfqpoint{2.804577in}{3.112478in}}%
\pgfpathlineto{\pgfqpoint{2.838997in}{3.128665in}}%
\pgfpathlineto{\pgfqpoint{2.873387in}{3.146210in}}%
\pgfpathlineto{\pgfqpoint{2.863932in}{3.164095in}}%
\pgfpathlineto{\pgfqpoint{2.854489in}{3.182326in}}%
\pgfpathlineto{\pgfqpoint{2.820140in}{3.166223in}}%
\pgfpathlineto{\pgfqpoint{2.785762in}{3.151359in}}%
\pgfpathclose%
\pgfusepath{fill}%
\end{pgfscope}%
\begin{pgfscope}%
\pgfpathrectangle{\pgfqpoint{1.020000in}{0.880000in}}{\pgfqpoint{6.160000in}{6.160000in}}%
\pgfusepath{clip}%
\pgfsetbuttcap%
\pgfsetroundjoin%
\definecolor{currentfill}{rgb}{0.467678,0.605591,0.968546}%
\pgfsetfillcolor{currentfill}%
\pgfsetlinewidth{0.000000pt}%
\definecolor{currentstroke}{rgb}{0.000000,0.000000,0.000000}%
\pgfsetstrokecolor{currentstroke}%
\pgfsetdash{}{0pt}%
\pgfpathmoveto{\pgfqpoint{4.751276in}{3.000293in}}%
\pgfpathlineto{\pgfqpoint{4.762431in}{2.977164in}}%
\pgfpathlineto{\pgfqpoint{4.773601in}{2.953103in}}%
\pgfpathlineto{\pgfqpoint{4.807433in}{2.929250in}}%
\pgfpathlineto{\pgfqpoint{4.841236in}{2.906880in}}%
\pgfpathlineto{\pgfqpoint{4.830012in}{2.929489in}}%
\pgfpathlineto{\pgfqpoint{4.818803in}{2.951348in}}%
\pgfpathlineto{\pgfqpoint{4.785054in}{2.975042in}}%
\pgfpathlineto{\pgfqpoint{4.751276in}{3.000293in}}%
\pgfpathclose%
\pgfusepath{fill}%
\end{pgfscope}%
\begin{pgfscope}%
\pgfpathrectangle{\pgfqpoint{1.020000in}{0.880000in}}{\pgfqpoint{6.160000in}{6.160000in}}%
\pgfusepath{clip}%
\pgfsetbuttcap%
\pgfsetroundjoin%
\definecolor{currentfill}{rgb}{0.800601,0.850358,0.930008}%
\pgfsetfillcolor{currentfill}%
\pgfsetlinewidth{0.000000pt}%
\definecolor{currentstroke}{rgb}{0.000000,0.000000,0.000000}%
\pgfsetstrokecolor{currentstroke}%
\pgfsetdash{}{0pt}%
\pgfpathmoveto{\pgfqpoint{3.421114in}{3.552061in}}%
\pgfpathlineto{\pgfqpoint{3.430948in}{3.552384in}}%
\pgfpathlineto{\pgfqpoint{3.440808in}{3.552060in}}%
\pgfpathlineto{\pgfqpoint{3.475043in}{3.579148in}}%
\pgfpathlineto{\pgfqpoint{3.509282in}{3.604641in}}%
\pgfpathlineto{\pgfqpoint{3.499365in}{3.602680in}}%
\pgfpathlineto{\pgfqpoint{3.489474in}{3.599948in}}%
\pgfpathlineto{\pgfqpoint{3.455293in}{3.576754in}}%
\pgfpathlineto{\pgfqpoint{3.421114in}{3.552061in}}%
\pgfpathclose%
\pgfusepath{fill}%
\end{pgfscope}%
\begin{pgfscope}%
\pgfpathrectangle{\pgfqpoint{1.020000in}{0.880000in}}{\pgfqpoint{6.160000in}{6.160000in}}%
\pgfusepath{clip}%
\pgfsetbuttcap%
\pgfsetroundjoin%
\definecolor{currentfill}{rgb}{0.271104,0.360011,0.807095}%
\pgfsetfillcolor{currentfill}%
\pgfsetlinewidth{0.000000pt}%
\definecolor{currentstroke}{rgb}{0.000000,0.000000,0.000000}%
\pgfsetstrokecolor{currentstroke}%
\pgfsetdash{}{0pt}%
\pgfpathmoveto{\pgfqpoint{5.834068in}{2.589409in}}%
\pgfpathlineto{\pgfqpoint{5.846294in}{2.574594in}}%
\pgfpathlineto{\pgfqpoint{5.858544in}{2.559837in}}%
\pgfpathlineto{\pgfqpoint{5.892181in}{2.562286in}}%
\pgfpathlineto{\pgfqpoint{5.925797in}{2.564809in}}%
\pgfpathlineto{\pgfqpoint{5.913489in}{2.579441in}}%
\pgfpathlineto{\pgfqpoint{5.901205in}{2.594127in}}%
\pgfpathlineto{\pgfqpoint{5.867646in}{2.591725in}}%
\pgfpathlineto{\pgfqpoint{5.834068in}{2.589409in}}%
\pgfpathclose%
\pgfusepath{fill}%
\end{pgfscope}%
\begin{pgfscope}%
\pgfpathrectangle{\pgfqpoint{1.020000in}{0.880000in}}{\pgfqpoint{6.160000in}{6.160000in}}%
\pgfusepath{clip}%
\pgfsetbuttcap%
\pgfsetroundjoin%
\definecolor{currentfill}{rgb}{0.261805,0.346484,0.795658}%
\pgfsetfillcolor{currentfill}%
\pgfsetlinewidth{0.000000pt}%
\definecolor{currentstroke}{rgb}{0.000000,0.000000,0.000000}%
\pgfsetstrokecolor{currentstroke}%
\pgfsetdash{}{0pt}%
\pgfpathmoveto{\pgfqpoint{6.060060in}{2.575440in}}%
\pgfpathlineto{\pgfqpoint{6.072509in}{2.561058in}}%
\pgfpathlineto{\pgfqpoint{6.084982in}{2.546725in}}%
\pgfpathlineto{\pgfqpoint{6.118554in}{2.549561in}}%
\pgfpathlineto{\pgfqpoint{6.152106in}{2.552421in}}%
\pgfpathlineto{\pgfqpoint{6.139575in}{2.566674in}}%
\pgfpathlineto{\pgfqpoint{6.127069in}{2.580976in}}%
\pgfpathlineto{\pgfqpoint{6.093575in}{2.578195in}}%
\pgfpathlineto{\pgfqpoint{6.060060in}{2.575440in}}%
\pgfpathclose%
\pgfusepath{fill}%
\end{pgfscope}%
\begin{pgfscope}%
\pgfpathrectangle{\pgfqpoint{1.020000in}{0.880000in}}{\pgfqpoint{6.160000in}{6.160000in}}%
\pgfusepath{clip}%
\pgfsetbuttcap%
\pgfsetroundjoin%
\definecolor{currentfill}{rgb}{0.280550,0.373423,0.818011}%
\pgfsetfillcolor{currentfill}%
\pgfsetlinewidth{0.000000pt}%
\definecolor{currentstroke}{rgb}{0.000000,0.000000,0.000000}%
\pgfsetstrokecolor{currentstroke}%
\pgfsetdash{}{0pt}%
\pgfpathmoveto{\pgfqpoint{5.608158in}{2.609306in}}%
\pgfpathlineto{\pgfqpoint{5.620160in}{2.593775in}}%
\pgfpathlineto{\pgfqpoint{5.632186in}{2.578305in}}%
\pgfpathlineto{\pgfqpoint{5.665881in}{2.579712in}}%
\pgfpathlineto{\pgfqpoint{5.699557in}{2.581330in}}%
\pgfpathlineto{\pgfqpoint{5.687472in}{2.596554in}}%
\pgfpathlineto{\pgfqpoint{5.675411in}{2.611837in}}%
\pgfpathlineto{\pgfqpoint{5.641794in}{2.610452in}}%
\pgfpathlineto{\pgfqpoint{5.608158in}{2.609306in}}%
\pgfpathclose%
\pgfusepath{fill}%
\end{pgfscope}%
\begin{pgfscope}%
\pgfpathrectangle{\pgfqpoint{1.020000in}{0.880000in}}{\pgfqpoint{6.160000in}{6.160000in}}%
\pgfusepath{clip}%
\pgfsetbuttcap%
\pgfsetroundjoin%
\definecolor{currentfill}{rgb}{0.613933,0.739923,0.999142}%
\pgfsetfillcolor{currentfill}%
\pgfsetlinewidth{0.000000pt}%
\definecolor{currentstroke}{rgb}{0.000000,0.000000,0.000000}%
\pgfsetstrokecolor{currentstroke}%
\pgfsetdash{}{0pt}%
\pgfpathmoveto{\pgfqpoint{4.503991in}{3.270848in}}%
\pgfpathlineto{\pgfqpoint{4.514943in}{3.250570in}}%
\pgfpathlineto{\pgfqpoint{4.525907in}{3.228341in}}%
\pgfpathlineto{\pgfqpoint{4.559855in}{3.195090in}}%
\pgfpathlineto{\pgfqpoint{4.593762in}{3.162547in}}%
\pgfpathlineto{\pgfqpoint{4.582749in}{3.183963in}}%
\pgfpathlineto{\pgfqpoint{4.571749in}{3.203712in}}%
\pgfpathlineto{\pgfqpoint{4.537891in}{3.236918in}}%
\pgfpathlineto{\pgfqpoint{4.503991in}{3.270848in}}%
\pgfpathclose%
\pgfusepath{fill}%
\end{pgfscope}%
\begin{pgfscope}%
\pgfpathrectangle{\pgfqpoint{1.020000in}{0.880000in}}{\pgfqpoint{6.160000in}{6.160000in}}%
\pgfusepath{clip}%
\pgfsetbuttcap%
\pgfsetroundjoin%
\definecolor{currentfill}{rgb}{0.777378,0.840921,0.946149}%
\pgfsetfillcolor{currentfill}%
\pgfsetlinewidth{0.000000pt}%
\definecolor{currentstroke}{rgb}{0.000000,0.000000,0.000000}%
\pgfsetstrokecolor{currentstroke}%
\pgfsetdash{}{0pt}%
\pgfpathmoveto{\pgfqpoint{3.352762in}{3.499226in}}%
\pgfpathlineto{\pgfqpoint{3.362542in}{3.496996in}}%
\pgfpathlineto{\pgfqpoint{3.372348in}{3.494247in}}%
\pgfpathlineto{\pgfqpoint{3.406576in}{3.523664in}}%
\pgfpathlineto{\pgfqpoint{3.440808in}{3.552060in}}%
\pgfpathlineto{\pgfqpoint{3.430948in}{3.552384in}}%
\pgfpathlineto{\pgfqpoint{3.421114in}{3.552061in}}%
\pgfpathlineto{\pgfqpoint{3.386937in}{3.526130in}}%
\pgfpathlineto{\pgfqpoint{3.352762in}{3.499226in}}%
\pgfpathclose%
\pgfusepath{fill}%
\end{pgfscope}%
\begin{pgfscope}%
\pgfpathrectangle{\pgfqpoint{1.020000in}{0.880000in}}{\pgfqpoint{6.160000in}{6.160000in}}%
\pgfusepath{clip}%
\pgfsetbuttcap%
\pgfsetroundjoin%
\definecolor{currentfill}{rgb}{0.294718,0.393542,0.834384}%
\pgfsetfillcolor{currentfill}%
\pgfsetlinewidth{0.000000pt}%
\definecolor{currentstroke}{rgb}{0.000000,0.000000,0.000000}%
\pgfsetstrokecolor{currentstroke}%
\pgfsetdash{}{0pt}%
\pgfpathmoveto{\pgfqpoint{5.382385in}{2.643358in}}%
\pgfpathlineto{\pgfqpoint{5.394162in}{2.626497in}}%
\pgfpathlineto{\pgfqpoint{5.405961in}{2.609680in}}%
\pgfpathlineto{\pgfqpoint{5.439705in}{2.608530in}}%
\pgfpathlineto{\pgfqpoint{5.473431in}{2.607880in}}%
\pgfpathlineto{\pgfqpoint{5.461572in}{2.624186in}}%
\pgfpathlineto{\pgfqpoint{5.449735in}{2.640540in}}%
\pgfpathlineto{\pgfqpoint{5.416069in}{2.641668in}}%
\pgfpathlineto{\pgfqpoint{5.382385in}{2.643358in}}%
\pgfpathclose%
\pgfusepath{fill}%
\end{pgfscope}%
\begin{pgfscope}%
\pgfpathrectangle{\pgfqpoint{1.020000in}{0.880000in}}{\pgfqpoint{6.160000in}{6.160000in}}%
\pgfusepath{clip}%
\pgfsetbuttcap%
\pgfsetroundjoin%
\definecolor{currentfill}{rgb}{0.867428,0.864377,0.862602}%
\pgfsetfillcolor{currentfill}%
\pgfsetlinewidth{0.000000pt}%
\definecolor{currentstroke}{rgb}{0.000000,0.000000,0.000000}%
\pgfsetstrokecolor{currentstroke}%
\pgfsetdash{}{0pt}%
\pgfpathmoveto{\pgfqpoint{3.783271in}{3.722046in}}%
\pgfpathlineto{\pgfqpoint{3.793478in}{3.728474in}}%
\pgfpathlineto{\pgfqpoint{3.803716in}{3.732971in}}%
\pgfpathlineto{\pgfqpoint{3.838018in}{3.734711in}}%
\pgfpathlineto{\pgfqpoint{3.872307in}{3.732993in}}%
\pgfpathlineto{\pgfqpoint{3.862001in}{3.728162in}}%
\pgfpathlineto{\pgfqpoint{3.851724in}{3.721365in}}%
\pgfpathlineto{\pgfqpoint{3.817503in}{3.723324in}}%
\pgfpathlineto{\pgfqpoint{3.783271in}{3.722046in}}%
\pgfpathclose%
\pgfusepath{fill}%
\end{pgfscope}%
\begin{pgfscope}%
\pgfpathrectangle{\pgfqpoint{1.020000in}{0.880000in}}{\pgfqpoint{6.160000in}{6.160000in}}%
\pgfusepath{clip}%
\pgfsetbuttcap%
\pgfsetroundjoin%
\definecolor{currentfill}{rgb}{0.261805,0.346484,0.795658}%
\pgfsetfillcolor{currentfill}%
\pgfsetlinewidth{0.000000pt}%
\definecolor{currentstroke}{rgb}{0.000000,0.000000,0.000000}%
\pgfsetstrokecolor{currentstroke}%
\pgfsetdash{}{0pt}%
\pgfpathmoveto{\pgfqpoint{6.219147in}{2.558196in}}%
\pgfpathlineto{\pgfqpoint{6.231761in}{2.544061in}}%
\pgfpathlineto{\pgfqpoint{6.265279in}{2.547005in}}%
\pgfpathlineto{\pgfqpoint{6.298777in}{2.549960in}}%
\pgfpathlineto{\pgfqpoint{6.286106in}{2.564028in}}%
\pgfpathlineto{\pgfqpoint{6.252637in}{2.561106in}}%
\pgfpathlineto{\pgfqpoint{6.219147in}{2.558196in}}%
\pgfpathclose%
\pgfusepath{fill}%
\end{pgfscope}%
\begin{pgfscope}%
\pgfpathrectangle{\pgfqpoint{1.020000in}{0.880000in}}{\pgfqpoint{6.160000in}{6.160000in}}%
\pgfusepath{clip}%
\pgfsetbuttcap%
\pgfsetroundjoin%
\definecolor{currentfill}{rgb}{0.559747,0.694768,0.996075}%
\pgfsetfillcolor{currentfill}%
\pgfsetlinewidth{0.000000pt}%
\definecolor{currentstroke}{rgb}{0.000000,0.000000,0.000000}%
\pgfsetstrokecolor{currentstroke}%
\pgfsetdash{}{0pt}%
\pgfpathmoveto{\pgfqpoint{2.716915in}{3.125291in}}%
\pgfpathlineto{\pgfqpoint{2.726274in}{3.104535in}}%
\pgfpathlineto{\pgfqpoint{2.735645in}{3.084099in}}%
\pgfpathlineto{\pgfqpoint{2.770127in}{3.097632in}}%
\pgfpathlineto{\pgfqpoint{2.804577in}{3.112478in}}%
\pgfpathlineto{\pgfqpoint{2.795164in}{3.131751in}}%
\pgfpathlineto{\pgfqpoint{2.785762in}{3.151359in}}%
\pgfpathlineto{\pgfqpoint{2.751354in}{3.137722in}}%
\pgfpathlineto{\pgfqpoint{2.716915in}{3.125291in}}%
\pgfpathclose%
\pgfusepath{fill}%
\end{pgfscope}%
\begin{pgfscope}%
\pgfpathrectangle{\pgfqpoint{1.020000in}{0.880000in}}{\pgfqpoint{6.160000in}{6.160000in}}%
\pgfusepath{clip}%
\pgfsetbuttcap%
\pgfsetroundjoin%
\definecolor{currentfill}{rgb}{0.748682,0.827679,0.963334}%
\pgfsetfillcolor{currentfill}%
\pgfsetlinewidth{0.000000pt}%
\definecolor{currentstroke}{rgb}{0.000000,0.000000,0.000000}%
\pgfsetstrokecolor{currentstroke}%
\pgfsetdash{}{0pt}%
\pgfpathmoveto{\pgfqpoint{3.284408in}{3.443551in}}%
\pgfpathlineto{\pgfqpoint{3.294139in}{3.438721in}}%
\pgfpathlineto{\pgfqpoint{3.303893in}{3.433493in}}%
\pgfpathlineto{\pgfqpoint{3.338120in}{3.464097in}}%
\pgfpathlineto{\pgfqpoint{3.372348in}{3.494247in}}%
\pgfpathlineto{\pgfqpoint{3.362542in}{3.496996in}}%
\pgfpathlineto{\pgfqpoint{3.352762in}{3.499226in}}%
\pgfpathlineto{\pgfqpoint{3.318585in}{3.471613in}}%
\pgfpathlineto{\pgfqpoint{3.284408in}{3.443551in}}%
\pgfpathclose%
\pgfusepath{fill}%
\end{pgfscope}%
\begin{pgfscope}%
\pgfpathrectangle{\pgfqpoint{1.020000in}{0.880000in}}{\pgfqpoint{6.160000in}{6.160000in}}%
\pgfusepath{clip}%
\pgfsetbuttcap%
\pgfsetroundjoin%
\definecolor{currentfill}{rgb}{0.763363,0.835092,0.955658}%
\pgfsetfillcolor{currentfill}%
\pgfsetlinewidth{0.000000pt}%
\definecolor{currentstroke}{rgb}{0.000000,0.000000,0.000000}%
\pgfsetstrokecolor{currentstroke}%
\pgfsetdash{}{0pt}%
\pgfpathmoveto{\pgfqpoint{4.256589in}{3.533365in}}%
\pgfpathlineto{\pgfqpoint{4.267329in}{3.522689in}}%
\pgfpathlineto{\pgfqpoint{4.278087in}{3.509274in}}%
\pgfpathlineto{\pgfqpoint{4.312207in}{3.476909in}}%
\pgfpathlineto{\pgfqpoint{4.346283in}{3.443462in}}%
\pgfpathlineto{\pgfqpoint{4.335474in}{3.457164in}}%
\pgfpathlineto{\pgfqpoint{4.324682in}{3.468391in}}%
\pgfpathlineto{\pgfqpoint{4.290658in}{3.501407in}}%
\pgfpathlineto{\pgfqpoint{4.256589in}{3.533365in}}%
\pgfpathclose%
\pgfusepath{fill}%
\end{pgfscope}%
\begin{pgfscope}%
\pgfpathrectangle{\pgfqpoint{1.020000in}{0.880000in}}{\pgfqpoint{6.160000in}{6.160000in}}%
\pgfusepath{clip}%
\pgfsetbuttcap%
\pgfsetroundjoin%
\definecolor{currentfill}{rgb}{0.718985,0.811993,0.977656}%
\pgfsetfillcolor{currentfill}%
\pgfsetlinewidth{0.000000pt}%
\definecolor{currentstroke}{rgb}{0.000000,0.000000,0.000000}%
\pgfsetstrokecolor{currentstroke}%
\pgfsetdash{}{0pt}%
\pgfpathmoveto{\pgfqpoint{3.216040in}{3.387068in}}%
\pgfpathlineto{\pgfqpoint{3.225724in}{3.379674in}}%
\pgfpathlineto{\pgfqpoint{3.235430in}{3.371998in}}%
\pgfpathlineto{\pgfqpoint{3.269663in}{3.402708in}}%
\pgfpathlineto{\pgfqpoint{3.303893in}{3.433493in}}%
\pgfpathlineto{\pgfqpoint{3.294139in}{3.438721in}}%
\pgfpathlineto{\pgfqpoint{3.284408in}{3.443551in}}%
\pgfpathlineto{\pgfqpoint{3.250226in}{3.415290in}}%
\pgfpathlineto{\pgfqpoint{3.216040in}{3.387068in}}%
\pgfpathclose%
\pgfusepath{fill}%
\end{pgfscope}%
\begin{pgfscope}%
\pgfpathrectangle{\pgfqpoint{1.020000in}{0.880000in}}{\pgfqpoint{6.160000in}{6.160000in}}%
\pgfusepath{clip}%
\pgfsetbuttcap%
\pgfsetroundjoin%
\definecolor{currentfill}{rgb}{0.363461,0.484784,0.901019}%
\pgfsetfillcolor{currentfill}%
\pgfsetlinewidth{0.000000pt}%
\definecolor{currentstroke}{rgb}{0.000000,0.000000,0.000000}%
\pgfsetstrokecolor{currentstroke}%
\pgfsetdash{}{0pt}%
\pgfpathmoveto{\pgfqpoint{4.998918in}{2.791288in}}%
\pgfpathlineto{\pgfqpoint{5.010307in}{2.770009in}}%
\pgfpathlineto{\pgfqpoint{5.021716in}{2.748480in}}%
\pgfpathlineto{\pgfqpoint{5.055517in}{2.736316in}}%
\pgfpathlineto{\pgfqpoint{5.089297in}{2.725426in}}%
\pgfpathlineto{\pgfqpoint{5.077829in}{2.745647in}}%
\pgfpathlineto{\pgfqpoint{5.066379in}{2.765687in}}%
\pgfpathlineto{\pgfqpoint{5.032658in}{2.777796in}}%
\pgfpathlineto{\pgfqpoint{4.998918in}{2.791288in}}%
\pgfpathclose%
\pgfusepath{fill}%
\end{pgfscope}%
\begin{pgfscope}%
\pgfpathrectangle{\pgfqpoint{1.020000in}{0.880000in}}{\pgfqpoint{6.160000in}{6.160000in}}%
\pgfusepath{clip}%
\pgfsetbuttcap%
\pgfsetroundjoin%
\definecolor{currentfill}{rgb}{0.323718,0.433158,0.864722}%
\pgfsetfillcolor{currentfill}%
\pgfsetlinewidth{0.000000pt}%
\definecolor{currentstroke}{rgb}{0.000000,0.000000,0.000000}%
\pgfsetstrokecolor{currentstroke}%
\pgfsetdash{}{0pt}%
\pgfpathmoveto{\pgfqpoint{5.156803in}{2.707272in}}%
\pgfpathlineto{\pgfqpoint{5.168353in}{2.688057in}}%
\pgfpathlineto{\pgfqpoint{5.179923in}{2.668789in}}%
\pgfpathlineto{\pgfqpoint{5.213710in}{2.662406in}}%
\pgfpathlineto{\pgfqpoint{5.247480in}{2.656967in}}%
\pgfpathlineto{\pgfqpoint{5.235848in}{2.675268in}}%
\pgfpathlineto{\pgfqpoint{5.224237in}{2.693544in}}%
\pgfpathlineto{\pgfqpoint{5.190528in}{2.699888in}}%
\pgfpathlineto{\pgfqpoint{5.156803in}{2.707272in}}%
\pgfpathclose%
\pgfusepath{fill}%
\end{pgfscope}%
\begin{pgfscope}%
\pgfpathrectangle{\pgfqpoint{1.020000in}{0.880000in}}{\pgfqpoint{6.160000in}{6.160000in}}%
\pgfusepath{clip}%
\pgfsetbuttcap%
\pgfsetroundjoin%
\definecolor{currentfill}{rgb}{0.693321,0.796314,0.986308}%
\pgfsetfillcolor{currentfill}%
\pgfsetlinewidth{0.000000pt}%
\definecolor{currentstroke}{rgb}{0.000000,0.000000,0.000000}%
\pgfsetstrokecolor{currentstroke}%
\pgfsetdash{}{0pt}%
\pgfpathmoveto{\pgfqpoint{3.147642in}{3.331617in}}%
\pgfpathlineto{\pgfqpoint{3.157281in}{3.321770in}}%
\pgfpathlineto{\pgfqpoint{3.166941in}{3.311744in}}%
\pgfpathlineto{\pgfqpoint{3.201190in}{3.341603in}}%
\pgfpathlineto{\pgfqpoint{3.235430in}{3.371998in}}%
\pgfpathlineto{\pgfqpoint{3.225724in}{3.379674in}}%
\pgfpathlineto{\pgfqpoint{3.216040in}{3.387068in}}%
\pgfpathlineto{\pgfqpoint{3.181846in}{3.359109in}}%
\pgfpathlineto{\pgfqpoint{3.147642in}{3.331617in}}%
\pgfpathclose%
\pgfusepath{fill}%
\end{pgfscope}%
\begin{pgfscope}%
\pgfpathrectangle{\pgfqpoint{1.020000in}{0.880000in}}{\pgfqpoint{6.160000in}{6.160000in}}%
\pgfusepath{clip}%
\pgfsetbuttcap%
\pgfsetroundjoin%
\definecolor{currentfill}{rgb}{0.559747,0.694768,0.996075}%
\pgfsetfillcolor{currentfill}%
\pgfsetlinewidth{0.000000pt}%
\definecolor{currentstroke}{rgb}{0.000000,0.000000,0.000000}%
\pgfsetstrokecolor{currentstroke}%
\pgfsetdash{}{0pt}%
\pgfpathmoveto{\pgfqpoint{4.593762in}{3.162547in}}%
\pgfpathlineto{\pgfqpoint{4.604788in}{3.139511in}}%
\pgfpathlineto{\pgfqpoint{4.615827in}{3.114926in}}%
\pgfpathlineto{\pgfqpoint{4.649743in}{3.084486in}}%
\pgfpathlineto{\pgfqpoint{4.683622in}{3.055132in}}%
\pgfpathlineto{\pgfqpoint{4.672533in}{3.078438in}}%
\pgfpathlineto{\pgfqpoint{4.661457in}{3.100444in}}%
\pgfpathlineto{\pgfqpoint{4.627628in}{3.130934in}}%
\pgfpathlineto{\pgfqpoint{4.593762in}{3.162547in}}%
\pgfpathclose%
\pgfusepath{fill}%
\end{pgfscope}%
\begin{pgfscope}%
\pgfpathrectangle{\pgfqpoint{1.020000in}{0.880000in}}{\pgfqpoint{6.160000in}{6.160000in}}%
\pgfusepath{clip}%
\pgfsetbuttcap%
\pgfsetroundjoin%
\definecolor{currentfill}{rgb}{0.271104,0.360011,0.807095}%
\pgfsetfillcolor{currentfill}%
\pgfsetlinewidth{0.000000pt}%
\definecolor{currentstroke}{rgb}{0.000000,0.000000,0.000000}%
\pgfsetstrokecolor{currentstroke}%
\pgfsetdash{}{0pt}%
\pgfpathmoveto{\pgfqpoint{5.766852in}{2.585094in}}%
\pgfpathlineto{\pgfqpoint{5.779019in}{2.570125in}}%
\pgfpathlineto{\pgfqpoint{5.791212in}{2.555215in}}%
\pgfpathlineto{\pgfqpoint{5.824888in}{2.557475in}}%
\pgfpathlineto{\pgfqpoint{5.858544in}{2.559837in}}%
\pgfpathlineto{\pgfqpoint{5.846294in}{2.574594in}}%
\pgfpathlineto{\pgfqpoint{5.834068in}{2.589409in}}%
\pgfpathlineto{\pgfqpoint{5.800470in}{2.587193in}}%
\pgfpathlineto{\pgfqpoint{5.766852in}{2.585094in}}%
\pgfpathclose%
\pgfusepath{fill}%
\end{pgfscope}%
\begin{pgfscope}%
\pgfpathrectangle{\pgfqpoint{1.020000in}{0.880000in}}{\pgfqpoint{6.160000in}{6.160000in}}%
\pgfusepath{clip}%
\pgfsetbuttcap%
\pgfsetroundjoin%
\definecolor{currentfill}{rgb}{0.266381,0.353304,0.801637}%
\pgfsetfillcolor{currentfill}%
\pgfsetlinewidth{0.000000pt}%
\definecolor{currentstroke}{rgb}{0.000000,0.000000,0.000000}%
\pgfsetstrokecolor{currentstroke}%
\pgfsetdash{}{0pt}%
\pgfpathmoveto{\pgfqpoint{5.992969in}{2.570034in}}%
\pgfpathlineto{\pgfqpoint{6.005360in}{2.555561in}}%
\pgfpathlineto{\pgfqpoint{6.017776in}{2.541139in}}%
\pgfpathlineto{\pgfqpoint{6.051389in}{2.543916in}}%
\pgfpathlineto{\pgfqpoint{6.084982in}{2.546725in}}%
\pgfpathlineto{\pgfqpoint{6.072509in}{2.561058in}}%
\pgfpathlineto{\pgfqpoint{6.060060in}{2.575440in}}%
\pgfpathlineto{\pgfqpoint{6.026525in}{2.572718in}}%
\pgfpathlineto{\pgfqpoint{5.992969in}{2.570034in}}%
\pgfpathclose%
\pgfusepath{fill}%
\end{pgfscope}%
\begin{pgfscope}%
\pgfpathrectangle{\pgfqpoint{1.020000in}{0.880000in}}{\pgfqpoint{6.160000in}{6.160000in}}%
\pgfusepath{clip}%
\pgfsetbuttcap%
\pgfsetroundjoin%
\definecolor{currentfill}{rgb}{0.419991,0.552989,0.942630}%
\pgfsetfillcolor{currentfill}%
\pgfsetlinewidth{0.000000pt}%
\definecolor{currentstroke}{rgb}{0.000000,0.000000,0.000000}%
\pgfsetstrokecolor{currentstroke}%
\pgfsetdash{}{0pt}%
\pgfpathmoveto{\pgfqpoint{4.841236in}{2.906880in}}%
\pgfpathlineto{\pgfqpoint{4.852476in}{2.883590in}}%
\pgfpathlineto{\pgfqpoint{4.863731in}{2.859698in}}%
\pgfpathlineto{\pgfqpoint{4.897564in}{2.840406in}}%
\pgfpathlineto{\pgfqpoint{4.931372in}{2.822573in}}%
\pgfpathlineto{\pgfqpoint{4.920059in}{2.844902in}}%
\pgfpathlineto{\pgfqpoint{4.908762in}{2.866759in}}%
\pgfpathlineto{\pgfqpoint{4.875012in}{2.886041in}}%
\pgfpathlineto{\pgfqpoint{4.841236in}{2.906880in}}%
\pgfpathclose%
\pgfusepath{fill}%
\end{pgfscope}%
\begin{pgfscope}%
\pgfpathrectangle{\pgfqpoint{1.020000in}{0.880000in}}{\pgfqpoint{6.160000in}{6.160000in}}%
\pgfusepath{clip}%
\pgfsetbuttcap%
\pgfsetroundjoin%
\definecolor{currentfill}{rgb}{0.280550,0.373423,0.818011}%
\pgfsetfillcolor{currentfill}%
\pgfsetlinewidth{0.000000pt}%
\definecolor{currentstroke}{rgb}{0.000000,0.000000,0.000000}%
\pgfsetstrokecolor{currentstroke}%
\pgfsetdash{}{0pt}%
\pgfpathmoveto{\pgfqpoint{5.540831in}{2.607880in}}%
\pgfpathlineto{\pgfqpoint{5.552774in}{2.592033in}}%
\pgfpathlineto{\pgfqpoint{5.564740in}{2.576250in}}%
\pgfpathlineto{\pgfqpoint{5.598473in}{2.577139in}}%
\pgfpathlineto{\pgfqpoint{5.632186in}{2.578305in}}%
\pgfpathlineto{\pgfqpoint{5.620160in}{2.593775in}}%
\pgfpathlineto{\pgfqpoint{5.608158in}{2.609306in}}%
\pgfpathlineto{\pgfqpoint{5.574504in}{2.608436in}}%
\pgfpathlineto{\pgfqpoint{5.540831in}{2.607880in}}%
\pgfpathclose%
\pgfusepath{fill}%
\end{pgfscope}%
\begin{pgfscope}%
\pgfpathrectangle{\pgfqpoint{1.020000in}{0.880000in}}{\pgfqpoint{6.160000in}{6.160000in}}%
\pgfusepath{clip}%
\pgfsetbuttcap%
\pgfsetroundjoin%
\definecolor{currentfill}{rgb}{0.661968,0.775491,0.993937}%
\pgfsetfillcolor{currentfill}%
\pgfsetlinewidth{0.000000pt}%
\definecolor{currentstroke}{rgb}{0.000000,0.000000,0.000000}%
\pgfsetstrokecolor{currentstroke}%
\pgfsetdash{}{0pt}%
\pgfpathmoveto{\pgfqpoint{3.079195in}{3.278757in}}%
\pgfpathlineto{\pgfqpoint{3.088791in}{3.266625in}}%
\pgfpathlineto{\pgfqpoint{3.098407in}{3.254403in}}%
\pgfpathlineto{\pgfqpoint{3.132681in}{3.282618in}}%
\pgfpathlineto{\pgfqpoint{3.166941in}{3.311744in}}%
\pgfpathlineto{\pgfqpoint{3.157281in}{3.321770in}}%
\pgfpathlineto{\pgfqpoint{3.147642in}{3.331617in}}%
\pgfpathlineto{\pgfqpoint{3.113426in}{3.304778in}}%
\pgfpathlineto{\pgfqpoint{3.079195in}{3.278757in}}%
\pgfpathclose%
\pgfusepath{fill}%
\end{pgfscope}%
\begin{pgfscope}%
\pgfpathrectangle{\pgfqpoint{1.020000in}{0.880000in}}{\pgfqpoint{6.160000in}{6.160000in}}%
\pgfusepath{clip}%
\pgfsetbuttcap%
\pgfsetroundjoin%
\definecolor{currentfill}{rgb}{0.548876,0.685104,0.994379}%
\pgfsetfillcolor{currentfill}%
\pgfsetlinewidth{0.000000pt}%
\definecolor{currentstroke}{rgb}{0.000000,0.000000,0.000000}%
\pgfsetstrokecolor{currentstroke}%
\pgfsetdash{}{0pt}%
\pgfpathmoveto{\pgfqpoint{2.647941in}{3.103902in}}%
\pgfpathlineto{\pgfqpoint{2.657256in}{3.082203in}}%
\pgfpathlineto{\pgfqpoint{2.666583in}{3.060803in}}%
\pgfpathlineto{\pgfqpoint{2.701130in}{3.071839in}}%
\pgfpathlineto{\pgfqpoint{2.735645in}{3.084099in}}%
\pgfpathlineto{\pgfqpoint{2.726274in}{3.104535in}}%
\pgfpathlineto{\pgfqpoint{2.716915in}{3.125291in}}%
\pgfpathlineto{\pgfqpoint{2.682444in}{3.114031in}}%
\pgfpathlineto{\pgfqpoint{2.647941in}{3.103902in}}%
\pgfpathclose%
\pgfusepath{fill}%
\end{pgfscope}%
\begin{pgfscope}%
\pgfpathrectangle{\pgfqpoint{1.020000in}{0.880000in}}{\pgfqpoint{6.160000in}{6.160000in}}%
\pgfusepath{clip}%
\pgfsetbuttcap%
\pgfsetroundjoin%
\definecolor{currentfill}{rgb}{0.867428,0.864377,0.862602}%
\pgfsetfillcolor{currentfill}%
\pgfsetlinewidth{0.000000pt}%
\definecolor{currentstroke}{rgb}{0.000000,0.000000,0.000000}%
\pgfsetstrokecolor{currentstroke}%
\pgfsetdash{}{0pt}%
\pgfpathmoveto{\pgfqpoint{3.714781in}{3.709863in}}%
\pgfpathlineto{\pgfqpoint{3.724920in}{3.715468in}}%
\pgfpathlineto{\pgfqpoint{3.735089in}{3.719214in}}%
\pgfpathlineto{\pgfqpoint{3.769405in}{3.727786in}}%
\pgfpathlineto{\pgfqpoint{3.803716in}{3.732971in}}%
\pgfpathlineto{\pgfqpoint{3.793478in}{3.728474in}}%
\pgfpathlineto{\pgfqpoint{3.783271in}{3.722046in}}%
\pgfpathlineto{\pgfqpoint{3.749029in}{3.717542in}}%
\pgfpathlineto{\pgfqpoint{3.714781in}{3.709863in}}%
\pgfpathclose%
\pgfusepath{fill}%
\end{pgfscope}%
\begin{pgfscope}%
\pgfpathrectangle{\pgfqpoint{1.020000in}{0.880000in}}{\pgfqpoint{6.160000in}{6.160000in}}%
\pgfusepath{clip}%
\pgfsetbuttcap%
\pgfsetroundjoin%
\definecolor{currentfill}{rgb}{0.863392,0.865084,0.867634}%
\pgfsetfillcolor{currentfill}%
\pgfsetlinewidth{0.000000pt}%
\definecolor{currentstroke}{rgb}{0.000000,0.000000,0.000000}%
\pgfsetstrokecolor{currentstroke}%
\pgfsetdash{}{0pt}%
\pgfpathmoveto{\pgfqpoint{3.940837in}{3.719340in}}%
\pgfpathlineto{\pgfqpoint{3.951241in}{3.721782in}}%
\pgfpathlineto{\pgfqpoint{3.961672in}{3.721618in}}%
\pgfpathlineto{\pgfqpoint{3.995972in}{3.709541in}}%
\pgfpathlineto{\pgfqpoint{4.030245in}{3.694208in}}%
\pgfpathlineto{\pgfqpoint{4.019749in}{3.694749in}}%
\pgfpathlineto{\pgfqpoint{4.009279in}{3.692752in}}%
\pgfpathlineto{\pgfqpoint{3.975071in}{3.707591in}}%
\pgfpathlineto{\pgfqpoint{3.940837in}{3.719340in}}%
\pgfpathclose%
\pgfusepath{fill}%
\end{pgfscope}%
\begin{pgfscope}%
\pgfpathrectangle{\pgfqpoint{1.020000in}{0.880000in}}{\pgfqpoint{6.160000in}{6.160000in}}%
\pgfusepath{clip}%
\pgfsetbuttcap%
\pgfsetroundjoin%
\definecolor{currentfill}{rgb}{0.261805,0.346484,0.795658}%
\pgfsetfillcolor{currentfill}%
\pgfsetlinewidth{0.000000pt}%
\definecolor{currentstroke}{rgb}{0.000000,0.000000,0.000000}%
\pgfsetstrokecolor{currentstroke}%
\pgfsetdash{}{0pt}%
\pgfpathmoveto{\pgfqpoint{6.152106in}{2.552421in}}%
\pgfpathlineto{\pgfqpoint{6.164662in}{2.538215in}}%
\pgfpathlineto{\pgfqpoint{6.198222in}{2.541130in}}%
\pgfpathlineto{\pgfqpoint{6.231761in}{2.544061in}}%
\pgfpathlineto{\pgfqpoint{6.219147in}{2.558196in}}%
\pgfpathlineto{\pgfqpoint{6.185637in}{2.555300in}}%
\pgfpathlineto{\pgfqpoint{6.152106in}{2.552421in}}%
\pgfpathclose%
\pgfusepath{fill}%
\end{pgfscope}%
\begin{pgfscope}%
\pgfpathrectangle{\pgfqpoint{1.020000in}{0.880000in}}{\pgfqpoint{6.160000in}{6.160000in}}%
\pgfusepath{clip}%
\pgfsetbuttcap%
\pgfsetroundjoin%
\definecolor{currentfill}{rgb}{0.826784,0.858205,0.906953}%
\pgfsetfillcolor{currentfill}%
\pgfsetlinewidth{0.000000pt}%
\definecolor{currentstroke}{rgb}{0.000000,0.000000,0.000000}%
\pgfsetstrokecolor{currentstroke}%
\pgfsetdash{}{0pt}%
\pgfpathmoveto{\pgfqpoint{4.098700in}{3.654596in}}%
\pgfpathlineto{\pgfqpoint{4.109283in}{3.650713in}}%
\pgfpathlineto{\pgfqpoint{4.119891in}{3.643949in}}%
\pgfpathlineto{\pgfqpoint{4.154125in}{3.619540in}}%
\pgfpathlineto{\pgfqpoint{4.188321in}{3.592762in}}%
\pgfpathlineto{\pgfqpoint{4.177657in}{3.600099in}}%
\pgfpathlineto{\pgfqpoint{4.167014in}{3.604736in}}%
\pgfpathlineto{\pgfqpoint{4.132876in}{3.630807in}}%
\pgfpathlineto{\pgfqpoint{4.098700in}{3.654596in}}%
\pgfpathclose%
\pgfusepath{fill}%
\end{pgfscope}%
\begin{pgfscope}%
\pgfpathrectangle{\pgfqpoint{1.020000in}{0.880000in}}{\pgfqpoint{6.160000in}{6.160000in}}%
\pgfusepath{clip}%
\pgfsetbuttcap%
\pgfsetroundjoin%
\definecolor{currentfill}{rgb}{0.299441,0.400248,0.839842}%
\pgfsetfillcolor{currentfill}%
\pgfsetlinewidth{0.000000pt}%
\definecolor{currentstroke}{rgb}{0.000000,0.000000,0.000000}%
\pgfsetstrokecolor{currentstroke}%
\pgfsetdash{}{0pt}%
\pgfpathmoveto{\pgfqpoint{5.314967in}{2.648664in}}%
\pgfpathlineto{\pgfqpoint{5.326682in}{2.631164in}}%
\pgfpathlineto{\pgfqpoint{5.338420in}{2.613700in}}%
\pgfpathlineto{\pgfqpoint{5.372199in}{2.611384in}}%
\pgfpathlineto{\pgfqpoint{5.405961in}{2.609680in}}%
\pgfpathlineto{\pgfqpoint{5.394162in}{2.626497in}}%
\pgfpathlineto{\pgfqpoint{5.382385in}{2.643358in}}%
\pgfpathlineto{\pgfqpoint{5.348684in}{2.645668in}}%
\pgfpathlineto{\pgfqpoint{5.314967in}{2.648664in}}%
\pgfpathclose%
\pgfusepath{fill}%
\end{pgfscope}%
\begin{pgfscope}%
\pgfpathrectangle{\pgfqpoint{1.020000in}{0.880000in}}{\pgfqpoint{6.160000in}{6.160000in}}%
\pgfusepath{clip}%
\pgfsetbuttcap%
\pgfsetroundjoin%
\definecolor{currentfill}{rgb}{0.635474,0.756714,0.998297}%
\pgfsetfillcolor{currentfill}%
\pgfsetlinewidth{0.000000pt}%
\definecolor{currentstroke}{rgb}{0.000000,0.000000,0.000000}%
\pgfsetstrokecolor{currentstroke}%
\pgfsetdash{}{0pt}%
\pgfpathmoveto{\pgfqpoint{3.010682in}{3.229714in}}%
\pgfpathlineto{\pgfqpoint{3.020236in}{3.215504in}}%
\pgfpathlineto{\pgfqpoint{3.029808in}{3.201280in}}%
\pgfpathlineto{\pgfqpoint{3.064117in}{3.227248in}}%
\pgfpathlineto{\pgfqpoint{3.098407in}{3.254403in}}%
\pgfpathlineto{\pgfqpoint{3.088791in}{3.266625in}}%
\pgfpathlineto{\pgfqpoint{3.079195in}{3.278757in}}%
\pgfpathlineto{\pgfqpoint{3.044948in}{3.253696in}}%
\pgfpathlineto{\pgfqpoint{3.010682in}{3.229714in}}%
\pgfpathclose%
\pgfusepath{fill}%
\end{pgfscope}%
\begin{pgfscope}%
\pgfpathrectangle{\pgfqpoint{1.020000in}{0.880000in}}{\pgfqpoint{6.160000in}{6.160000in}}%
\pgfusepath{clip}%
\pgfsetbuttcap%
\pgfsetroundjoin%
\definecolor{currentfill}{rgb}{0.713852,0.808857,0.979386}%
\pgfsetfillcolor{currentfill}%
\pgfsetlinewidth{0.000000pt}%
\definecolor{currentstroke}{rgb}{0.000000,0.000000,0.000000}%
\pgfsetstrokecolor{currentstroke}%
\pgfsetdash{}{0pt}%
\pgfpathmoveto{\pgfqpoint{4.346283in}{3.443462in}}%
\pgfpathlineto{\pgfqpoint{4.357108in}{3.427221in}}%
\pgfpathlineto{\pgfqpoint{4.367950in}{3.408411in}}%
\pgfpathlineto{\pgfqpoint{4.402028in}{3.374337in}}%
\pgfpathlineto{\pgfqpoint{4.436061in}{3.339844in}}%
\pgfpathlineto{\pgfqpoint{4.425171in}{3.358394in}}%
\pgfpathlineto{\pgfqpoint{4.414297in}{3.374666in}}%
\pgfpathlineto{\pgfqpoint{4.380313in}{3.409271in}}%
\pgfpathlineto{\pgfqpoint{4.346283in}{3.443462in}}%
\pgfpathclose%
\pgfusepath{fill}%
\end{pgfscope}%
\begin{pgfscope}%
\pgfpathrectangle{\pgfqpoint{1.020000in}{0.880000in}}{\pgfqpoint{6.160000in}{6.160000in}}%
\pgfusepath{clip}%
\pgfsetbuttcap%
\pgfsetroundjoin%
\definecolor{currentfill}{rgb}{0.608547,0.735725,0.999354}%
\pgfsetfillcolor{currentfill}%
\pgfsetlinewidth{0.000000pt}%
\definecolor{currentstroke}{rgb}{0.000000,0.000000,0.000000}%
\pgfsetstrokecolor{currentstroke}%
\pgfsetdash{}{0pt}%
\pgfpathmoveto{\pgfqpoint{2.942085in}{3.185358in}}%
\pgfpathlineto{\pgfqpoint{2.951597in}{3.169301in}}%
\pgfpathlineto{\pgfqpoint{2.961125in}{3.153290in}}%
\pgfpathlineto{\pgfqpoint{2.995478in}{3.176601in}}%
\pgfpathlineto{\pgfqpoint{3.029808in}{3.201280in}}%
\pgfpathlineto{\pgfqpoint{3.020236in}{3.215504in}}%
\pgfpathlineto{\pgfqpoint{3.010682in}{3.229714in}}%
\pgfpathlineto{\pgfqpoint{2.976395in}{3.206910in}}%
\pgfpathlineto{\pgfqpoint{2.942085in}{3.185358in}}%
\pgfpathclose%
\pgfusepath{fill}%
\end{pgfscope}%
\begin{pgfscope}%
\pgfpathrectangle{\pgfqpoint{1.020000in}{0.880000in}}{\pgfqpoint{6.160000in}{6.160000in}}%
\pgfusepath{clip}%
\pgfsetbuttcap%
\pgfsetroundjoin%
\definecolor{currentfill}{rgb}{0.500031,0.638508,0.981070}%
\pgfsetfillcolor{currentfill}%
\pgfsetlinewidth{0.000000pt}%
\definecolor{currentstroke}{rgb}{0.000000,0.000000,0.000000}%
\pgfsetstrokecolor{currentstroke}%
\pgfsetdash{}{0pt}%
\pgfpathmoveto{\pgfqpoint{4.683622in}{3.055132in}}%
\pgfpathlineto{\pgfqpoint{4.694725in}{3.030602in}}%
\pgfpathlineto{\pgfqpoint{4.705842in}{3.004938in}}%
\pgfpathlineto{\pgfqpoint{4.739737in}{2.978365in}}%
\pgfpathlineto{\pgfqpoint{4.773601in}{2.953103in}}%
\pgfpathlineto{\pgfqpoint{4.762431in}{2.977164in}}%
\pgfpathlineto{\pgfqpoint{4.751276in}{3.000293in}}%
\pgfpathlineto{\pgfqpoint{4.717466in}{3.027024in}}%
\pgfpathlineto{\pgfqpoint{4.683622in}{3.055132in}}%
\pgfpathclose%
\pgfusepath{fill}%
\end{pgfscope}%
\begin{pgfscope}%
\pgfpathrectangle{\pgfqpoint{1.020000in}{0.880000in}}{\pgfqpoint{6.160000in}{6.160000in}}%
\pgfusepath{clip}%
\pgfsetbuttcap%
\pgfsetroundjoin%
\definecolor{currentfill}{rgb}{0.266381,0.353304,0.801637}%
\pgfsetfillcolor{currentfill}%
\pgfsetlinewidth{0.000000pt}%
\definecolor{currentstroke}{rgb}{0.000000,0.000000,0.000000}%
\pgfsetstrokecolor{currentstroke}%
\pgfsetdash{}{0pt}%
\pgfpathmoveto{\pgfqpoint{5.925797in}{2.564809in}}%
\pgfpathlineto{\pgfqpoint{5.938130in}{2.550232in}}%
\pgfpathlineto{\pgfqpoint{5.950489in}{2.535709in}}%
\pgfpathlineto{\pgfqpoint{5.984143in}{2.538402in}}%
\pgfpathlineto{\pgfqpoint{6.017776in}{2.541139in}}%
\pgfpathlineto{\pgfqpoint{6.005360in}{2.555561in}}%
\pgfpathlineto{\pgfqpoint{5.992969in}{2.570034in}}%
\pgfpathlineto{\pgfqpoint{5.959393in}{2.567395in}}%
\pgfpathlineto{\pgfqpoint{5.925797in}{2.564809in}}%
\pgfpathclose%
\pgfusepath{fill}%
\end{pgfscope}%
\begin{pgfscope}%
\pgfpathrectangle{\pgfqpoint{1.020000in}{0.880000in}}{\pgfqpoint{6.160000in}{6.160000in}}%
\pgfusepath{clip}%
\pgfsetbuttcap%
\pgfsetroundjoin%
\definecolor{currentfill}{rgb}{0.271104,0.360011,0.807095}%
\pgfsetfillcolor{currentfill}%
\pgfsetlinewidth{0.000000pt}%
\definecolor{currentstroke}{rgb}{0.000000,0.000000,0.000000}%
\pgfsetstrokecolor{currentstroke}%
\pgfsetdash{}{0pt}%
\pgfpathmoveto{\pgfqpoint{5.699557in}{2.581330in}}%
\pgfpathlineto{\pgfqpoint{5.711667in}{2.566168in}}%
\pgfpathlineto{\pgfqpoint{5.723800in}{2.551069in}}%
\pgfpathlineto{\pgfqpoint{5.757516in}{2.553073in}}%
\pgfpathlineto{\pgfqpoint{5.791212in}{2.555215in}}%
\pgfpathlineto{\pgfqpoint{5.779019in}{2.570125in}}%
\pgfpathlineto{\pgfqpoint{5.766852in}{2.585094in}}%
\pgfpathlineto{\pgfqpoint{5.733214in}{2.583133in}}%
\pgfpathlineto{\pgfqpoint{5.699557in}{2.581330in}}%
\pgfpathclose%
\pgfusepath{fill}%
\end{pgfscope}%
\begin{pgfscope}%
\pgfpathrectangle{\pgfqpoint{1.020000in}{0.880000in}}{\pgfqpoint{6.160000in}{6.160000in}}%
\pgfusepath{clip}%
\pgfsetbuttcap%
\pgfsetroundjoin%
\definecolor{currentfill}{rgb}{0.863392,0.865084,0.867634}%
\pgfsetfillcolor{currentfill}%
\pgfsetlinewidth{0.000000pt}%
\definecolor{currentstroke}{rgb}{0.000000,0.000000,0.000000}%
\pgfsetstrokecolor{currentstroke}%
\pgfsetdash{}{0pt}%
\pgfpathmoveto{\pgfqpoint{3.646276in}{3.685380in}}%
\pgfpathlineto{\pgfqpoint{3.656348in}{3.689741in}}%
\pgfpathlineto{\pgfqpoint{3.666451in}{3.692346in}}%
\pgfpathlineto{\pgfqpoint{3.700770in}{3.707354in}}%
\pgfpathlineto{\pgfqpoint{3.735089in}{3.719214in}}%
\pgfpathlineto{\pgfqpoint{3.724920in}{3.715468in}}%
\pgfpathlineto{\pgfqpoint{3.714781in}{3.709863in}}%
\pgfpathlineto{\pgfqpoint{3.680529in}{3.699100in}}%
\pgfpathlineto{\pgfqpoint{3.646276in}{3.685380in}}%
\pgfpathclose%
\pgfusepath{fill}%
\end{pgfscope}%
\begin{pgfscope}%
\pgfpathrectangle{\pgfqpoint{1.020000in}{0.880000in}}{\pgfqpoint{6.160000in}{6.160000in}}%
\pgfusepath{clip}%
\pgfsetbuttcap%
\pgfsetroundjoin%
\definecolor{currentfill}{rgb}{0.285273,0.380129,0.823469}%
\pgfsetfillcolor{currentfill}%
\pgfsetlinewidth{0.000000pt}%
\definecolor{currentstroke}{rgb}{0.000000,0.000000,0.000000}%
\pgfsetstrokecolor{currentstroke}%
\pgfsetdash{}{0pt}%
\pgfpathmoveto{\pgfqpoint{5.473431in}{2.607880in}}%
\pgfpathlineto{\pgfqpoint{5.485314in}{2.591631in}}%
\pgfpathlineto{\pgfqpoint{5.497221in}{2.575447in}}%
\pgfpathlineto{\pgfqpoint{5.530990in}{2.575672in}}%
\pgfpathlineto{\pgfqpoint{5.564740in}{2.576250in}}%
\pgfpathlineto{\pgfqpoint{5.552774in}{2.592033in}}%
\pgfpathlineto{\pgfqpoint{5.540831in}{2.607880in}}%
\pgfpathlineto{\pgfqpoint{5.507140in}{2.607680in}}%
\pgfpathlineto{\pgfqpoint{5.473431in}{2.607880in}}%
\pgfpathclose%
\pgfusepath{fill}%
\end{pgfscope}%
\begin{pgfscope}%
\pgfpathrectangle{\pgfqpoint{1.020000in}{0.880000in}}{\pgfqpoint{6.160000in}{6.160000in}}%
\pgfusepath{clip}%
\pgfsetbuttcap%
\pgfsetroundjoin%
\definecolor{currentfill}{rgb}{0.338377,0.452819,0.879317}%
\pgfsetfillcolor{currentfill}%
\pgfsetlinewidth{0.000000pt}%
\definecolor{currentstroke}{rgb}{0.000000,0.000000,0.000000}%
\pgfsetstrokecolor{currentstroke}%
\pgfsetdash{}{0pt}%
\pgfpathmoveto{\pgfqpoint{5.089297in}{2.725426in}}%
\pgfpathlineto{\pgfqpoint{5.100786in}{2.705070in}}%
\pgfpathlineto{\pgfqpoint{5.112295in}{2.684624in}}%
\pgfpathlineto{\pgfqpoint{5.146118in}{2.676176in}}%
\pgfpathlineto{\pgfqpoint{5.179923in}{2.668789in}}%
\pgfpathlineto{\pgfqpoint{5.168353in}{2.688057in}}%
\pgfpathlineto{\pgfqpoint{5.156803in}{2.707272in}}%
\pgfpathlineto{\pgfqpoint{5.123059in}{2.715764in}}%
\pgfpathlineto{\pgfqpoint{5.089297in}{2.725426in}}%
\pgfpathclose%
\pgfusepath{fill}%
\end{pgfscope}%
\begin{pgfscope}%
\pgfpathrectangle{\pgfqpoint{1.020000in}{0.880000in}}{\pgfqpoint{6.160000in}{6.160000in}}%
\pgfusepath{clip}%
\pgfsetbuttcap%
\pgfsetroundjoin%
\definecolor{currentfill}{rgb}{0.586921,0.718121,0.998874}%
\pgfsetfillcolor{currentfill}%
\pgfsetlinewidth{0.000000pt}%
\definecolor{currentstroke}{rgb}{0.000000,0.000000,0.000000}%
\pgfsetstrokecolor{currentstroke}%
\pgfsetdash{}{0pt}%
\pgfpathmoveto{\pgfqpoint{2.873387in}{3.146210in}}%
\pgfpathlineto{\pgfqpoint{2.882857in}{3.128544in}}%
\pgfpathlineto{\pgfqpoint{2.892343in}{3.110972in}}%
\pgfpathlineto{\pgfqpoint{2.926748in}{3.131402in}}%
\pgfpathlineto{\pgfqpoint{2.961125in}{3.153290in}}%
\pgfpathlineto{\pgfqpoint{2.951597in}{3.169301in}}%
\pgfpathlineto{\pgfqpoint{2.942085in}{3.185358in}}%
\pgfpathlineto{\pgfqpoint{2.907749in}{3.165113in}}%
\pgfpathlineto{\pgfqpoint{2.873387in}{3.146210in}}%
\pgfpathclose%
\pgfusepath{fill}%
\end{pgfscope}%
\begin{pgfscope}%
\pgfpathrectangle{\pgfqpoint{1.020000in}{0.880000in}}{\pgfqpoint{6.160000in}{6.160000in}}%
\pgfusepath{clip}%
\pgfsetbuttcap%
\pgfsetroundjoin%
\definecolor{currentfill}{rgb}{0.383662,0.510183,0.917831}%
\pgfsetfillcolor{currentfill}%
\pgfsetlinewidth{0.000000pt}%
\definecolor{currentstroke}{rgb}{0.000000,0.000000,0.000000}%
\pgfsetstrokecolor{currentstroke}%
\pgfsetdash{}{0pt}%
\pgfpathmoveto{\pgfqpoint{4.931372in}{2.822573in}}%
\pgfpathlineto{\pgfqpoint{4.942702in}{2.799840in}}%
\pgfpathlineto{\pgfqpoint{4.954050in}{2.776772in}}%
\pgfpathlineto{\pgfqpoint{4.987894in}{2.761957in}}%
\pgfpathlineto{\pgfqpoint{5.021716in}{2.748480in}}%
\pgfpathlineto{\pgfqpoint{5.010307in}{2.770009in}}%
\pgfpathlineto{\pgfqpoint{4.998918in}{2.791288in}}%
\pgfpathlineto{\pgfqpoint{4.965156in}{2.806204in}}%
\pgfpathlineto{\pgfqpoint{4.931372in}{2.822573in}}%
\pgfpathclose%
\pgfusepath{fill}%
\end{pgfscope}%
\begin{pgfscope}%
\pgfpathrectangle{\pgfqpoint{1.020000in}{0.880000in}}{\pgfqpoint{6.160000in}{6.160000in}}%
\pgfusepath{clip}%
\pgfsetbuttcap%
\pgfsetroundjoin%
\definecolor{currentfill}{rgb}{0.261805,0.346484,0.795658}%
\pgfsetfillcolor{currentfill}%
\pgfsetlinewidth{0.000000pt}%
\definecolor{currentstroke}{rgb}{0.000000,0.000000,0.000000}%
\pgfsetstrokecolor{currentstroke}%
\pgfsetdash{}{0pt}%
\pgfpathmoveto{\pgfqpoint{6.084982in}{2.546725in}}%
\pgfpathlineto{\pgfqpoint{6.097481in}{2.532441in}}%
\pgfpathlineto{\pgfqpoint{6.131082in}{2.535317in}}%
\pgfpathlineto{\pgfqpoint{6.164662in}{2.538215in}}%
\pgfpathlineto{\pgfqpoint{6.152106in}{2.552421in}}%
\pgfpathlineto{\pgfqpoint{6.118554in}{2.549561in}}%
\pgfpathlineto{\pgfqpoint{6.084982in}{2.546725in}}%
\pgfpathclose%
\pgfusepath{fill}%
\end{pgfscope}%
\begin{pgfscope}%
\pgfpathrectangle{\pgfqpoint{1.020000in}{0.880000in}}{\pgfqpoint{6.160000in}{6.160000in}}%
\pgfusepath{clip}%
\pgfsetbuttcap%
\pgfsetroundjoin%
\definecolor{currentfill}{rgb}{0.661968,0.775491,0.993937}%
\pgfsetfillcolor{currentfill}%
\pgfsetlinewidth{0.000000pt}%
\definecolor{currentstroke}{rgb}{0.000000,0.000000,0.000000}%
\pgfsetstrokecolor{currentstroke}%
\pgfsetdash{}{0pt}%
\pgfpathmoveto{\pgfqpoint{4.436061in}{3.339844in}}%
\pgfpathlineto{\pgfqpoint{4.446964in}{3.319020in}}%
\pgfpathlineto{\pgfqpoint{4.457882in}{3.295957in}}%
\pgfpathlineto{\pgfqpoint{4.491916in}{3.262055in}}%
\pgfpathlineto{\pgfqpoint{4.525907in}{3.228341in}}%
\pgfpathlineto{\pgfqpoint{4.514943in}{3.250570in}}%
\pgfpathlineto{\pgfqpoint{4.503991in}{3.270848in}}%
\pgfpathlineto{\pgfqpoint{4.470048in}{3.305248in}}%
\pgfpathlineto{\pgfqpoint{4.436061in}{3.339844in}}%
\pgfpathclose%
\pgfusepath{fill}%
\end{pgfscope}%
\begin{pgfscope}%
\pgfpathrectangle{\pgfqpoint{1.020000in}{0.880000in}}{\pgfqpoint{6.160000in}{6.160000in}}%
\pgfusepath{clip}%
\pgfsetbuttcap%
\pgfsetroundjoin%
\definecolor{currentfill}{rgb}{0.309060,0.413498,0.850128}%
\pgfsetfillcolor{currentfill}%
\pgfsetlinewidth{0.000000pt}%
\definecolor{currentstroke}{rgb}{0.000000,0.000000,0.000000}%
\pgfsetstrokecolor{currentstroke}%
\pgfsetdash{}{0pt}%
\pgfpathmoveto{\pgfqpoint{5.247480in}{2.656967in}}%
\pgfpathlineto{\pgfqpoint{5.259134in}{2.638668in}}%
\pgfpathlineto{\pgfqpoint{5.270810in}{2.620395in}}%
\pgfpathlineto{\pgfqpoint{5.304624in}{2.616684in}}%
\pgfpathlineto{\pgfqpoint{5.338420in}{2.613700in}}%
\pgfpathlineto{\pgfqpoint{5.326682in}{2.631164in}}%
\pgfpathlineto{\pgfqpoint{5.314967in}{2.648664in}}%
\pgfpathlineto{\pgfqpoint{5.281232in}{2.652408in}}%
\pgfpathlineto{\pgfqpoint{5.247480in}{2.656967in}}%
\pgfpathclose%
\pgfusepath{fill}%
\end{pgfscope}%
\begin{pgfscope}%
\pgfpathrectangle{\pgfqpoint{1.020000in}{0.880000in}}{\pgfqpoint{6.160000in}{6.160000in}}%
\pgfusepath{clip}%
\pgfsetbuttcap%
\pgfsetroundjoin%
\definecolor{currentfill}{rgb}{0.800601,0.850358,0.930008}%
\pgfsetfillcolor{currentfill}%
\pgfsetlinewidth{0.000000pt}%
\definecolor{currentstroke}{rgb}{0.000000,0.000000,0.000000}%
\pgfsetstrokecolor{currentstroke}%
\pgfsetdash{}{0pt}%
\pgfpathmoveto{\pgfqpoint{4.188321in}{3.592762in}}%
\pgfpathlineto{\pgfqpoint{4.199007in}{3.582565in}}%
\pgfpathlineto{\pgfqpoint{4.209713in}{3.569389in}}%
\pgfpathlineto{\pgfqpoint{4.243921in}{3.540214in}}%
\pgfpathlineto{\pgfqpoint{4.278087in}{3.509274in}}%
\pgfpathlineto{\pgfqpoint{4.267329in}{3.522689in}}%
\pgfpathlineto{\pgfqpoint{4.256589in}{3.533365in}}%
\pgfpathlineto{\pgfqpoint{4.222476in}{3.563928in}}%
\pgfpathlineto{\pgfqpoint{4.188321in}{3.592762in}}%
\pgfpathclose%
\pgfusepath{fill}%
\end{pgfscope}%
\begin{pgfscope}%
\pgfpathrectangle{\pgfqpoint{1.020000in}{0.880000in}}{\pgfqpoint{6.160000in}{6.160000in}}%
\pgfusepath{clip}%
\pgfsetbuttcap%
\pgfsetroundjoin%
\definecolor{currentfill}{rgb}{0.875557,0.860242,0.851430}%
\pgfsetfillcolor{currentfill}%
\pgfsetlinewidth{0.000000pt}%
\definecolor{currentstroke}{rgb}{0.000000,0.000000,0.000000}%
\pgfsetstrokecolor{currentstroke}%
\pgfsetdash{}{0pt}%
\pgfpathmoveto{\pgfqpoint{3.872307in}{3.732993in}}%
\pgfpathlineto{\pgfqpoint{3.882643in}{3.735512in}}%
\pgfpathlineto{\pgfqpoint{3.893009in}{3.735403in}}%
\pgfpathlineto{\pgfqpoint{3.927350in}{3.730278in}}%
\pgfpathlineto{\pgfqpoint{3.961672in}{3.721618in}}%
\pgfpathlineto{\pgfqpoint{3.951241in}{3.721782in}}%
\pgfpathlineto{\pgfqpoint{3.940837in}{3.719340in}}%
\pgfpathlineto{\pgfqpoint{3.906582in}{3.727845in}}%
\pgfpathlineto{\pgfqpoint{3.872307in}{3.732993in}}%
\pgfpathclose%
\pgfusepath{fill}%
\end{pgfscope}%
\begin{pgfscope}%
\pgfpathrectangle{\pgfqpoint{1.020000in}{0.880000in}}{\pgfqpoint{6.160000in}{6.160000in}}%
\pgfusepath{clip}%
\pgfsetbuttcap%
\pgfsetroundjoin%
\definecolor{currentfill}{rgb}{0.851372,0.863125,0.881064}%
\pgfsetfillcolor{currentfill}%
\pgfsetlinewidth{0.000000pt}%
\definecolor{currentstroke}{rgb}{0.000000,0.000000,0.000000}%
\pgfsetstrokecolor{currentstroke}%
\pgfsetdash{}{0pt}%
\pgfpathmoveto{\pgfqpoint{3.577773in}{3.649755in}}%
\pgfpathlineto{\pgfqpoint{3.587780in}{3.652501in}}%
\pgfpathlineto{\pgfqpoint{3.597818in}{3.653623in}}%
\pgfpathlineto{\pgfqpoint{3.632133in}{3.674366in}}%
\pgfpathlineto{\pgfqpoint{3.666451in}{3.692346in}}%
\pgfpathlineto{\pgfqpoint{3.656348in}{3.689741in}}%
\pgfpathlineto{\pgfqpoint{3.646276in}{3.685380in}}%
\pgfpathlineto{\pgfqpoint{3.612023in}{3.668867in}}%
\pgfpathlineto{\pgfqpoint{3.577773in}{3.649755in}}%
\pgfpathclose%
\pgfusepath{fill}%
\end{pgfscope}%
\begin{pgfscope}%
\pgfpathrectangle{\pgfqpoint{1.020000in}{0.880000in}}{\pgfqpoint{6.160000in}{6.160000in}}%
\pgfusepath{clip}%
\pgfsetbuttcap%
\pgfsetroundjoin%
\definecolor{currentfill}{rgb}{0.451739,0.588181,0.960201}%
\pgfsetfillcolor{currentfill}%
\pgfsetlinewidth{0.000000pt}%
\definecolor{currentstroke}{rgb}{0.000000,0.000000,0.000000}%
\pgfsetstrokecolor{currentstroke}%
\pgfsetdash{}{0pt}%
\pgfpathmoveto{\pgfqpoint{4.773601in}{2.953103in}}%
\pgfpathlineto{\pgfqpoint{4.784785in}{2.928196in}}%
\pgfpathlineto{\pgfqpoint{4.795984in}{2.902539in}}%
\pgfpathlineto{\pgfqpoint{4.829872in}{2.880424in}}%
\pgfpathlineto{\pgfqpoint{4.863731in}{2.859698in}}%
\pgfpathlineto{\pgfqpoint{4.852476in}{2.883590in}}%
\pgfpathlineto{\pgfqpoint{4.841236in}{2.906880in}}%
\pgfpathlineto{\pgfqpoint{4.807433in}{2.929250in}}%
\pgfpathlineto{\pgfqpoint{4.773601in}{2.953103in}}%
\pgfpathclose%
\pgfusepath{fill}%
\end{pgfscope}%
\begin{pgfscope}%
\pgfpathrectangle{\pgfqpoint{1.020000in}{0.880000in}}{\pgfqpoint{6.160000in}{6.160000in}}%
\pgfusepath{clip}%
\pgfsetbuttcap%
\pgfsetroundjoin%
\definecolor{currentfill}{rgb}{0.565182,0.699438,0.996635}%
\pgfsetfillcolor{currentfill}%
\pgfsetlinewidth{0.000000pt}%
\definecolor{currentstroke}{rgb}{0.000000,0.000000,0.000000}%
\pgfsetstrokecolor{currentstroke}%
\pgfsetdash{}{0pt}%
\pgfpathmoveto{\pgfqpoint{2.804577in}{3.112478in}}%
\pgfpathlineto{\pgfqpoint{2.814004in}{3.093435in}}%
\pgfpathlineto{\pgfqpoint{2.823447in}{3.074521in}}%
\pgfpathlineto{\pgfqpoint{2.857910in}{3.092012in}}%
\pgfpathlineto{\pgfqpoint{2.892343in}{3.110972in}}%
\pgfpathlineto{\pgfqpoint{2.882857in}{3.128544in}}%
\pgfpathlineto{\pgfqpoint{2.873387in}{3.146210in}}%
\pgfpathlineto{\pgfqpoint{2.838997in}{3.128665in}}%
\pgfpathlineto{\pgfqpoint{2.804577in}{3.112478in}}%
\pgfpathclose%
\pgfusepath{fill}%
\end{pgfscope}%
\begin{pgfscope}%
\pgfpathrectangle{\pgfqpoint{1.020000in}{0.880000in}}{\pgfqpoint{6.160000in}{6.160000in}}%
\pgfusepath{clip}%
\pgfsetbuttcap%
\pgfsetroundjoin%
\definecolor{currentfill}{rgb}{0.597777,0.727330,0.999777}%
\pgfsetfillcolor{currentfill}%
\pgfsetlinewidth{0.000000pt}%
\definecolor{currentstroke}{rgb}{0.000000,0.000000,0.000000}%
\pgfsetstrokecolor{currentstroke}%
\pgfsetdash{}{0pt}%
\pgfpathmoveto{\pgfqpoint{4.525907in}{3.228341in}}%
\pgfpathlineto{\pgfqpoint{4.536885in}{3.204221in}}%
\pgfpathlineto{\pgfqpoint{4.547876in}{3.178288in}}%
\pgfpathlineto{\pgfqpoint{4.581872in}{3.146265in}}%
\pgfpathlineto{\pgfqpoint{4.615827in}{3.114926in}}%
\pgfpathlineto{\pgfqpoint{4.604788in}{3.139511in}}%
\pgfpathlineto{\pgfqpoint{4.593762in}{3.162547in}}%
\pgfpathlineto{\pgfqpoint{4.559855in}{3.195090in}}%
\pgfpathlineto{\pgfqpoint{4.525907in}{3.228341in}}%
\pgfpathclose%
\pgfusepath{fill}%
\end{pgfscope}%
\begin{pgfscope}%
\pgfpathrectangle{\pgfqpoint{1.020000in}{0.880000in}}{\pgfqpoint{6.160000in}{6.160000in}}%
\pgfusepath{clip}%
\pgfsetbuttcap%
\pgfsetroundjoin%
\definecolor{currentfill}{rgb}{0.266381,0.353304,0.801637}%
\pgfsetfillcolor{currentfill}%
\pgfsetlinewidth{0.000000pt}%
\definecolor{currentstroke}{rgb}{0.000000,0.000000,0.000000}%
\pgfsetstrokecolor{currentstroke}%
\pgfsetdash{}{0pt}%
\pgfpathmoveto{\pgfqpoint{5.858544in}{2.559837in}}%
\pgfpathlineto{\pgfqpoint{5.870820in}{2.545137in}}%
\pgfpathlineto{\pgfqpoint{5.883120in}{2.530495in}}%
\pgfpathlineto{\pgfqpoint{5.916814in}{2.533070in}}%
\pgfpathlineto{\pgfqpoint{5.950489in}{2.535709in}}%
\pgfpathlineto{\pgfqpoint{5.938130in}{2.550232in}}%
\pgfpathlineto{\pgfqpoint{5.925797in}{2.564809in}}%
\pgfpathlineto{\pgfqpoint{5.892181in}{2.562286in}}%
\pgfpathlineto{\pgfqpoint{5.858544in}{2.559837in}}%
\pgfpathclose%
\pgfusepath{fill}%
\end{pgfscope}%
\begin{pgfscope}%
\pgfpathrectangle{\pgfqpoint{1.020000in}{0.880000in}}{\pgfqpoint{6.160000in}{6.160000in}}%
\pgfusepath{clip}%
\pgfsetbuttcap%
\pgfsetroundjoin%
\definecolor{currentfill}{rgb}{0.275827,0.366717,0.812553}%
\pgfsetfillcolor{currentfill}%
\pgfsetlinewidth{0.000000pt}%
\definecolor{currentstroke}{rgb}{0.000000,0.000000,0.000000}%
\pgfsetstrokecolor{currentstroke}%
\pgfsetdash{}{0pt}%
\pgfpathmoveto{\pgfqpoint{5.632186in}{2.578305in}}%
\pgfpathlineto{\pgfqpoint{5.644237in}{2.562899in}}%
\pgfpathlineto{\pgfqpoint{5.656311in}{2.547561in}}%
\pgfpathlineto{\pgfqpoint{5.690065in}{2.549224in}}%
\pgfpathlineto{\pgfqpoint{5.723800in}{2.551069in}}%
\pgfpathlineto{\pgfqpoint{5.711667in}{2.566168in}}%
\pgfpathlineto{\pgfqpoint{5.699557in}{2.581330in}}%
\pgfpathlineto{\pgfqpoint{5.665881in}{2.579712in}}%
\pgfpathlineto{\pgfqpoint{5.632186in}{2.578305in}}%
\pgfpathclose%
\pgfusepath{fill}%
\end{pgfscope}%
\begin{pgfscope}%
\pgfpathrectangle{\pgfqpoint{1.020000in}{0.880000in}}{\pgfqpoint{6.160000in}{6.160000in}}%
\pgfusepath{clip}%
\pgfsetbuttcap%
\pgfsetroundjoin%
\definecolor{currentfill}{rgb}{0.851372,0.863125,0.881064}%
\pgfsetfillcolor{currentfill}%
\pgfsetlinewidth{0.000000pt}%
\definecolor{currentstroke}{rgb}{0.000000,0.000000,0.000000}%
\pgfsetstrokecolor{currentstroke}%
\pgfsetdash{}{0pt}%
\pgfpathmoveto{\pgfqpoint{4.030245in}{3.694208in}}%
\pgfpathlineto{\pgfqpoint{4.040767in}{3.690869in}}%
\pgfpathlineto{\pgfqpoint{4.051315in}{3.684507in}}%
\pgfpathlineto{\pgfqpoint{4.085620in}{3.665694in}}%
\pgfpathlineto{\pgfqpoint{4.119891in}{3.643949in}}%
\pgfpathlineto{\pgfqpoint{4.109283in}{3.650713in}}%
\pgfpathlineto{\pgfqpoint{4.098700in}{3.654596in}}%
\pgfpathlineto{\pgfqpoint{4.064489in}{3.675816in}}%
\pgfpathlineto{\pgfqpoint{4.030245in}{3.694208in}}%
\pgfpathclose%
\pgfusepath{fill}%
\end{pgfscope}%
\begin{pgfscope}%
\pgfpathrectangle{\pgfqpoint{1.020000in}{0.880000in}}{\pgfqpoint{6.160000in}{6.160000in}}%
\pgfusepath{clip}%
\pgfsetbuttcap%
\pgfsetroundjoin%
\definecolor{currentfill}{rgb}{0.835345,0.860514,0.898970}%
\pgfsetfillcolor{currentfill}%
\pgfsetlinewidth{0.000000pt}%
\definecolor{currentstroke}{rgb}{0.000000,0.000000,0.000000}%
\pgfsetstrokecolor{currentstroke}%
\pgfsetdash{}{0pt}%
\pgfpathmoveto{\pgfqpoint{3.509282in}{3.604641in}}%
\pgfpathlineto{\pgfqpoint{3.519228in}{3.605472in}}%
\pgfpathlineto{\pgfqpoint{3.529204in}{3.604831in}}%
\pgfpathlineto{\pgfqpoint{3.563509in}{3.630357in}}%
\pgfpathlineto{\pgfqpoint{3.597818in}{3.653623in}}%
\pgfpathlineto{\pgfqpoint{3.587780in}{3.652501in}}%
\pgfpathlineto{\pgfqpoint{3.577773in}{3.649755in}}%
\pgfpathlineto{\pgfqpoint{3.543525in}{3.628265in}}%
\pgfpathlineto{\pgfqpoint{3.509282in}{3.604641in}}%
\pgfpathclose%
\pgfusepath{fill}%
\end{pgfscope}%
\begin{pgfscope}%
\pgfpathrectangle{\pgfqpoint{1.020000in}{0.880000in}}{\pgfqpoint{6.160000in}{6.160000in}}%
\pgfusepath{clip}%
\pgfsetbuttcap%
\pgfsetroundjoin%
\definecolor{currentfill}{rgb}{0.289996,0.386836,0.828926}%
\pgfsetfillcolor{currentfill}%
\pgfsetlinewidth{0.000000pt}%
\definecolor{currentstroke}{rgb}{0.000000,0.000000,0.000000}%
\pgfsetstrokecolor{currentstroke}%
\pgfsetdash{}{0pt}%
\pgfpathmoveto{\pgfqpoint{5.405961in}{2.609680in}}%
\pgfpathlineto{\pgfqpoint{5.417783in}{2.592918in}}%
\pgfpathlineto{\pgfqpoint{5.429628in}{2.576224in}}%
\pgfpathlineto{\pgfqpoint{5.463433in}{2.575615in}}%
\pgfpathlineto{\pgfqpoint{5.497221in}{2.575447in}}%
\pgfpathlineto{\pgfqpoint{5.485314in}{2.591631in}}%
\pgfpathlineto{\pgfqpoint{5.473431in}{2.607880in}}%
\pgfpathlineto{\pgfqpoint{5.439705in}{2.608530in}}%
\pgfpathlineto{\pgfqpoint{5.405961in}{2.609680in}}%
\pgfpathclose%
\pgfusepath{fill}%
\end{pgfscope}%
\begin{pgfscope}%
\pgfpathrectangle{\pgfqpoint{1.020000in}{0.880000in}}{\pgfqpoint{6.160000in}{6.160000in}}%
\pgfusepath{clip}%
\pgfsetbuttcap%
\pgfsetroundjoin%
\definecolor{currentfill}{rgb}{0.261805,0.346484,0.795658}%
\pgfsetfillcolor{currentfill}%
\pgfsetlinewidth{0.000000pt}%
\definecolor{currentstroke}{rgb}{0.000000,0.000000,0.000000}%
\pgfsetstrokecolor{currentstroke}%
\pgfsetdash{}{0pt}%
\pgfpathmoveto{\pgfqpoint{6.017776in}{2.541139in}}%
\pgfpathlineto{\pgfqpoint{6.030217in}{2.526770in}}%
\pgfpathlineto{\pgfqpoint{6.063859in}{2.529591in}}%
\pgfpathlineto{\pgfqpoint{6.097481in}{2.532441in}}%
\pgfpathlineto{\pgfqpoint{6.084982in}{2.546725in}}%
\pgfpathlineto{\pgfqpoint{6.051389in}{2.543916in}}%
\pgfpathlineto{\pgfqpoint{6.017776in}{2.541139in}}%
\pgfpathclose%
\pgfusepath{fill}%
\end{pgfscope}%
\begin{pgfscope}%
\pgfpathrectangle{\pgfqpoint{1.020000in}{0.880000in}}{\pgfqpoint{6.160000in}{6.160000in}}%
\pgfusepath{clip}%
\pgfsetbuttcap%
\pgfsetroundjoin%
\definecolor{currentfill}{rgb}{0.548876,0.685104,0.994379}%
\pgfsetfillcolor{currentfill}%
\pgfsetlinewidth{0.000000pt}%
\definecolor{currentstroke}{rgb}{0.000000,0.000000,0.000000}%
\pgfsetstrokecolor{currentstroke}%
\pgfsetdash{}{0pt}%
\pgfpathmoveto{\pgfqpoint{2.735645in}{3.084099in}}%
\pgfpathlineto{\pgfqpoint{2.745028in}{3.063896in}}%
\pgfpathlineto{\pgfqpoint{2.754426in}{3.043846in}}%
\pgfpathlineto{\pgfqpoint{2.788953in}{3.058476in}}%
\pgfpathlineto{\pgfqpoint{2.823447in}{3.074521in}}%
\pgfpathlineto{\pgfqpoint{2.814004in}{3.093435in}}%
\pgfpathlineto{\pgfqpoint{2.804577in}{3.112478in}}%
\pgfpathlineto{\pgfqpoint{2.770127in}{3.097632in}}%
\pgfpathlineto{\pgfqpoint{2.735645in}{3.084099in}}%
\pgfpathclose%
\pgfusepath{fill}%
\end{pgfscope}%
\begin{pgfscope}%
\pgfpathrectangle{\pgfqpoint{1.020000in}{0.880000in}}{\pgfqpoint{6.160000in}{6.160000in}}%
\pgfusepath{clip}%
\pgfsetbuttcap%
\pgfsetroundjoin%
\definecolor{currentfill}{rgb}{0.813693,0.854282,0.918480}%
\pgfsetfillcolor{currentfill}%
\pgfsetlinewidth{0.000000pt}%
\definecolor{currentstroke}{rgb}{0.000000,0.000000,0.000000}%
\pgfsetstrokecolor{currentstroke}%
\pgfsetdash{}{0pt}%
\pgfpathmoveto{\pgfqpoint{3.440808in}{3.552060in}}%
\pgfpathlineto{\pgfqpoint{3.450695in}{3.550754in}}%
\pgfpathlineto{\pgfqpoint{3.460612in}{3.548145in}}%
\pgfpathlineto{\pgfqpoint{3.494905in}{3.577328in}}%
\pgfpathlineto{\pgfqpoint{3.529204in}{3.604831in}}%
\pgfpathlineto{\pgfqpoint{3.519228in}{3.605472in}}%
\pgfpathlineto{\pgfqpoint{3.509282in}{3.604641in}}%
\pgfpathlineto{\pgfqpoint{3.475043in}{3.579148in}}%
\pgfpathlineto{\pgfqpoint{3.440808in}{3.552060in}}%
\pgfpathclose%
\pgfusepath{fill}%
\end{pgfscope}%
\begin{pgfscope}%
\pgfpathrectangle{\pgfqpoint{1.020000in}{0.880000in}}{\pgfqpoint{6.160000in}{6.160000in}}%
\pgfusepath{clip}%
\pgfsetbuttcap%
\pgfsetroundjoin%
\definecolor{currentfill}{rgb}{0.353369,0.472069,0.892570}%
\pgfsetfillcolor{currentfill}%
\pgfsetlinewidth{0.000000pt}%
\definecolor{currentstroke}{rgb}{0.000000,0.000000,0.000000}%
\pgfsetstrokecolor{currentstroke}%
\pgfsetdash{}{0pt}%
\pgfpathmoveto{\pgfqpoint{5.021716in}{2.748480in}}%
\pgfpathlineto{\pgfqpoint{5.033143in}{2.726759in}}%
\pgfpathlineto{\pgfqpoint{5.044590in}{2.704903in}}%
\pgfpathlineto{\pgfqpoint{5.078452in}{2.694186in}}%
\pgfpathlineto{\pgfqpoint{5.112295in}{2.684624in}}%
\pgfpathlineto{\pgfqpoint{5.100786in}{2.705070in}}%
\pgfpathlineto{\pgfqpoint{5.089297in}{2.725426in}}%
\pgfpathlineto{\pgfqpoint{5.055517in}{2.736316in}}%
\pgfpathlineto{\pgfqpoint{5.021716in}{2.748480in}}%
\pgfpathclose%
\pgfusepath{fill}%
\end{pgfscope}%
\begin{pgfscope}%
\pgfpathrectangle{\pgfqpoint{1.020000in}{0.880000in}}{\pgfqpoint{6.160000in}{6.160000in}}%
\pgfusepath{clip}%
\pgfsetbuttcap%
\pgfsetroundjoin%
\definecolor{currentfill}{rgb}{0.758539,0.832787,0.958408}%
\pgfsetfillcolor{currentfill}%
\pgfsetlinewidth{0.000000pt}%
\definecolor{currentstroke}{rgb}{0.000000,0.000000,0.000000}%
\pgfsetstrokecolor{currentstroke}%
\pgfsetdash{}{0pt}%
\pgfpathmoveto{\pgfqpoint{4.278087in}{3.509274in}}%
\pgfpathlineto{\pgfqpoint{4.288863in}{3.493047in}}%
\pgfpathlineto{\pgfqpoint{4.299655in}{3.473975in}}%
\pgfpathlineto{\pgfqpoint{4.333825in}{3.441737in}}%
\pgfpathlineto{\pgfqpoint{4.367950in}{3.408411in}}%
\pgfpathlineto{\pgfqpoint{4.357108in}{3.427221in}}%
\pgfpathlineto{\pgfqpoint{4.346283in}{3.443462in}}%
\pgfpathlineto{\pgfqpoint{4.312207in}{3.476909in}}%
\pgfpathlineto{\pgfqpoint{4.278087in}{3.509274in}}%
\pgfpathclose%
\pgfusepath{fill}%
\end{pgfscope}%
\begin{pgfscope}%
\pgfpathrectangle{\pgfqpoint{1.020000in}{0.880000in}}{\pgfqpoint{6.160000in}{6.160000in}}%
\pgfusepath{clip}%
\pgfsetbuttcap%
\pgfsetroundjoin%
\definecolor{currentfill}{rgb}{0.786721,0.844807,0.939810}%
\pgfsetfillcolor{currentfill}%
\pgfsetlinewidth{0.000000pt}%
\definecolor{currentstroke}{rgb}{0.000000,0.000000,0.000000}%
\pgfsetstrokecolor{currentstroke}%
\pgfsetdash{}{0pt}%
\pgfpathmoveto{\pgfqpoint{3.372348in}{3.494247in}}%
\pgfpathlineto{\pgfqpoint{3.382180in}{3.490667in}}%
\pgfpathlineto{\pgfqpoint{3.392041in}{3.485959in}}%
\pgfpathlineto{\pgfqpoint{3.426324in}{3.517585in}}%
\pgfpathlineto{\pgfqpoint{3.460612in}{3.548145in}}%
\pgfpathlineto{\pgfqpoint{3.450695in}{3.550754in}}%
\pgfpathlineto{\pgfqpoint{3.440808in}{3.552060in}}%
\pgfpathlineto{\pgfqpoint{3.406576in}{3.523664in}}%
\pgfpathlineto{\pgfqpoint{3.372348in}{3.494247in}}%
\pgfpathclose%
\pgfusepath{fill}%
\end{pgfscope}%
\begin{pgfscope}%
\pgfpathrectangle{\pgfqpoint{1.020000in}{0.880000in}}{\pgfqpoint{6.160000in}{6.160000in}}%
\pgfusepath{clip}%
\pgfsetbuttcap%
\pgfsetroundjoin%
\definecolor{currentfill}{rgb}{0.538004,0.674902,0.991722}%
\pgfsetfillcolor{currentfill}%
\pgfsetlinewidth{0.000000pt}%
\definecolor{currentstroke}{rgb}{0.000000,0.000000,0.000000}%
\pgfsetstrokecolor{currentstroke}%
\pgfsetdash{}{0pt}%
\pgfpathmoveto{\pgfqpoint{4.615827in}{3.114926in}}%
\pgfpathlineto{\pgfqpoint{4.626879in}{3.088882in}}%
\pgfpathlineto{\pgfqpoint{4.637944in}{3.061486in}}%
\pgfpathlineto{\pgfqpoint{4.671911in}{3.032695in}}%
\pgfpathlineto{\pgfqpoint{4.705842in}{3.004938in}}%
\pgfpathlineto{\pgfqpoint{4.694725in}{3.030602in}}%
\pgfpathlineto{\pgfqpoint{4.683622in}{3.055132in}}%
\pgfpathlineto{\pgfqpoint{4.649743in}{3.084486in}}%
\pgfpathlineto{\pgfqpoint{4.615827in}{3.114926in}}%
\pgfpathclose%
\pgfusepath{fill}%
\end{pgfscope}%
\begin{pgfscope}%
\pgfpathrectangle{\pgfqpoint{1.020000in}{0.880000in}}{\pgfqpoint{6.160000in}{6.160000in}}%
\pgfusepath{clip}%
\pgfsetbuttcap%
\pgfsetroundjoin%
\definecolor{currentfill}{rgb}{0.404421,0.534643,0.932002}%
\pgfsetfillcolor{currentfill}%
\pgfsetlinewidth{0.000000pt}%
\definecolor{currentstroke}{rgb}{0.000000,0.000000,0.000000}%
\pgfsetstrokecolor{currentstroke}%
\pgfsetdash{}{0pt}%
\pgfpathmoveto{\pgfqpoint{4.863731in}{2.859698in}}%
\pgfpathlineto{\pgfqpoint{4.875003in}{2.835287in}}%
\pgfpathlineto{\pgfqpoint{4.886292in}{2.810446in}}%
\pgfpathlineto{\pgfqpoint{4.920183in}{2.792937in}}%
\pgfpathlineto{\pgfqpoint{4.954050in}{2.776772in}}%
\pgfpathlineto{\pgfqpoint{4.942702in}{2.799840in}}%
\pgfpathlineto{\pgfqpoint{4.931372in}{2.822573in}}%
\pgfpathlineto{\pgfqpoint{4.897564in}{2.840406in}}%
\pgfpathlineto{\pgfqpoint{4.863731in}{2.859698in}}%
\pgfpathclose%
\pgfusepath{fill}%
\end{pgfscope}%
\begin{pgfscope}%
\pgfpathrectangle{\pgfqpoint{1.020000in}{0.880000in}}{\pgfqpoint{6.160000in}{6.160000in}}%
\pgfusepath{clip}%
\pgfsetbuttcap%
\pgfsetroundjoin%
\definecolor{currentfill}{rgb}{0.318832,0.426605,0.859857}%
\pgfsetfillcolor{currentfill}%
\pgfsetlinewidth{0.000000pt}%
\definecolor{currentstroke}{rgb}{0.000000,0.000000,0.000000}%
\pgfsetstrokecolor{currentstroke}%
\pgfsetdash{}{0pt}%
\pgfpathmoveto{\pgfqpoint{5.179923in}{2.668789in}}%
\pgfpathlineto{\pgfqpoint{5.191515in}{2.649501in}}%
\pgfpathlineto{\pgfqpoint{5.203129in}{2.630228in}}%
\pgfpathlineto{\pgfqpoint{5.236978in}{2.624890in}}%
\pgfpathlineto{\pgfqpoint{5.270810in}{2.620395in}}%
\pgfpathlineto{\pgfqpoint{5.259134in}{2.638668in}}%
\pgfpathlineto{\pgfqpoint{5.247480in}{2.656967in}}%
\pgfpathlineto{\pgfqpoint{5.213710in}{2.662406in}}%
\pgfpathlineto{\pgfqpoint{5.179923in}{2.668789in}}%
\pgfpathclose%
\pgfusepath{fill}%
\end{pgfscope}%
\begin{pgfscope}%
\pgfpathrectangle{\pgfqpoint{1.020000in}{0.880000in}}{\pgfqpoint{6.160000in}{6.160000in}}%
\pgfusepath{clip}%
\pgfsetbuttcap%
\pgfsetroundjoin%
\definecolor{currentfill}{rgb}{0.758539,0.832787,0.958408}%
\pgfsetfillcolor{currentfill}%
\pgfsetlinewidth{0.000000pt}%
\definecolor{currentstroke}{rgb}{0.000000,0.000000,0.000000}%
\pgfsetstrokecolor{currentstroke}%
\pgfsetdash{}{0pt}%
\pgfpathmoveto{\pgfqpoint{3.303893in}{3.433493in}}%
\pgfpathlineto{\pgfqpoint{3.313673in}{3.427585in}}%
\pgfpathlineto{\pgfqpoint{3.323480in}{3.420726in}}%
\pgfpathlineto{\pgfqpoint{3.357760in}{3.453573in}}%
\pgfpathlineto{\pgfqpoint{3.392041in}{3.485959in}}%
\pgfpathlineto{\pgfqpoint{3.382180in}{3.490667in}}%
\pgfpathlineto{\pgfqpoint{3.372348in}{3.494247in}}%
\pgfpathlineto{\pgfqpoint{3.338120in}{3.464097in}}%
\pgfpathlineto{\pgfqpoint{3.303893in}{3.433493in}}%
\pgfpathclose%
\pgfusepath{fill}%
\end{pgfscope}%
\begin{pgfscope}%
\pgfpathrectangle{\pgfqpoint{1.020000in}{0.880000in}}{\pgfqpoint{6.160000in}{6.160000in}}%
\pgfusepath{clip}%
\pgfsetbuttcap%
\pgfsetroundjoin%
\definecolor{currentfill}{rgb}{0.879622,0.858175,0.845844}%
\pgfsetfillcolor{currentfill}%
\pgfsetlinewidth{0.000000pt}%
\definecolor{currentstroke}{rgb}{0.000000,0.000000,0.000000}%
\pgfsetstrokecolor{currentstroke}%
\pgfsetdash{}{0pt}%
\pgfpathmoveto{\pgfqpoint{3.803716in}{3.732971in}}%
\pgfpathlineto{\pgfqpoint{3.813984in}{3.735186in}}%
\pgfpathlineto{\pgfqpoint{3.824284in}{3.734798in}}%
\pgfpathlineto{\pgfqpoint{3.858653in}{3.736920in}}%
\pgfpathlineto{\pgfqpoint{3.893009in}{3.735403in}}%
\pgfpathlineto{\pgfqpoint{3.882643in}{3.735512in}}%
\pgfpathlineto{\pgfqpoint{3.872307in}{3.732993in}}%
\pgfpathlineto{\pgfqpoint{3.838018in}{3.734711in}}%
\pgfpathlineto{\pgfqpoint{3.803716in}{3.732971in}}%
\pgfpathclose%
\pgfusepath{fill}%
\end{pgfscope}%
\begin{pgfscope}%
\pgfpathrectangle{\pgfqpoint{1.020000in}{0.880000in}}{\pgfqpoint{6.160000in}{6.160000in}}%
\pgfusepath{clip}%
\pgfsetbuttcap%
\pgfsetroundjoin%
\definecolor{currentfill}{rgb}{0.724041,0.814910,0.975651}%
\pgfsetfillcolor{currentfill}%
\pgfsetlinewidth{0.000000pt}%
\definecolor{currentstroke}{rgb}{0.000000,0.000000,0.000000}%
\pgfsetstrokecolor{currentstroke}%
\pgfsetdash{}{0pt}%
\pgfpathmoveto{\pgfqpoint{3.235430in}{3.371998in}}%
\pgfpathlineto{\pgfqpoint{3.245160in}{3.363784in}}%
\pgfpathlineto{\pgfqpoint{3.254916in}{3.354792in}}%
\pgfpathlineto{\pgfqpoint{3.289200in}{3.387708in}}%
\pgfpathlineto{\pgfqpoint{3.323480in}{3.420726in}}%
\pgfpathlineto{\pgfqpoint{3.313673in}{3.427585in}}%
\pgfpathlineto{\pgfqpoint{3.303893in}{3.433493in}}%
\pgfpathlineto{\pgfqpoint{3.269663in}{3.402708in}}%
\pgfpathlineto{\pgfqpoint{3.235430in}{3.371998in}}%
\pgfpathclose%
\pgfusepath{fill}%
\end{pgfscope}%
\begin{pgfscope}%
\pgfpathrectangle{\pgfqpoint{1.020000in}{0.880000in}}{\pgfqpoint{6.160000in}{6.160000in}}%
\pgfusepath{clip}%
\pgfsetbuttcap%
\pgfsetroundjoin%
\definecolor{currentfill}{rgb}{0.538004,0.674902,0.991722}%
\pgfsetfillcolor{currentfill}%
\pgfsetlinewidth{0.000000pt}%
\definecolor{currentstroke}{rgb}{0.000000,0.000000,0.000000}%
\pgfsetstrokecolor{currentstroke}%
\pgfsetdash{}{0pt}%
\pgfpathmoveto{\pgfqpoint{2.666583in}{3.060803in}}%
\pgfpathlineto{\pgfqpoint{2.675922in}{3.039635in}}%
\pgfpathlineto{\pgfqpoint{2.685275in}{3.018633in}}%
\pgfpathlineto{\pgfqpoint{2.719868in}{3.030583in}}%
\pgfpathlineto{\pgfqpoint{2.754426in}{3.043846in}}%
\pgfpathlineto{\pgfqpoint{2.745028in}{3.063896in}}%
\pgfpathlineto{\pgfqpoint{2.735645in}{3.084099in}}%
\pgfpathlineto{\pgfqpoint{2.701130in}{3.071839in}}%
\pgfpathlineto{\pgfqpoint{2.666583in}{3.060803in}}%
\pgfpathclose%
\pgfusepath{fill}%
\end{pgfscope}%
\begin{pgfscope}%
\pgfpathrectangle{\pgfqpoint{1.020000in}{0.880000in}}{\pgfqpoint{6.160000in}{6.160000in}}%
\pgfusepath{clip}%
\pgfsetbuttcap%
\pgfsetroundjoin%
\definecolor{currentfill}{rgb}{0.266381,0.353304,0.801637}%
\pgfsetfillcolor{currentfill}%
\pgfsetlinewidth{0.000000pt}%
\definecolor{currentstroke}{rgb}{0.000000,0.000000,0.000000}%
\pgfsetstrokecolor{currentstroke}%
\pgfsetdash{}{0pt}%
\pgfpathmoveto{\pgfqpoint{5.791212in}{2.555215in}}%
\pgfpathlineto{\pgfqpoint{5.803429in}{2.540366in}}%
\pgfpathlineto{\pgfqpoint{5.815670in}{2.525579in}}%
\pgfpathlineto{\pgfqpoint{5.849405in}{2.527993in}}%
\pgfpathlineto{\pgfqpoint{5.883120in}{2.530495in}}%
\pgfpathlineto{\pgfqpoint{5.870820in}{2.545137in}}%
\pgfpathlineto{\pgfqpoint{5.858544in}{2.559837in}}%
\pgfpathlineto{\pgfqpoint{5.824888in}{2.557475in}}%
\pgfpathlineto{\pgfqpoint{5.791212in}{2.555215in}}%
\pgfpathclose%
\pgfusepath{fill}%
\end{pgfscope}%
\begin{pgfscope}%
\pgfpathrectangle{\pgfqpoint{1.020000in}{0.880000in}}{\pgfqpoint{6.160000in}{6.160000in}}%
\pgfusepath{clip}%
\pgfsetbuttcap%
\pgfsetroundjoin%
\definecolor{currentfill}{rgb}{0.693321,0.796314,0.986308}%
\pgfsetfillcolor{currentfill}%
\pgfsetlinewidth{0.000000pt}%
\definecolor{currentstroke}{rgb}{0.000000,0.000000,0.000000}%
\pgfsetstrokecolor{currentstroke}%
\pgfsetdash{}{0pt}%
\pgfpathmoveto{\pgfqpoint{3.166941in}{3.311744in}}%
\pgfpathlineto{\pgfqpoint{3.176624in}{3.301313in}}%
\pgfpathlineto{\pgfqpoint{3.186331in}{3.290265in}}%
\pgfpathlineto{\pgfqpoint{3.220628in}{3.322233in}}%
\pgfpathlineto{\pgfqpoint{3.254916in}{3.354792in}}%
\pgfpathlineto{\pgfqpoint{3.245160in}{3.363784in}}%
\pgfpathlineto{\pgfqpoint{3.235430in}{3.371998in}}%
\pgfpathlineto{\pgfqpoint{3.201190in}{3.341603in}}%
\pgfpathlineto{\pgfqpoint{3.166941in}{3.311744in}}%
\pgfpathclose%
\pgfusepath{fill}%
\end{pgfscope}%
\begin{pgfscope}%
\pgfpathrectangle{\pgfqpoint{1.020000in}{0.880000in}}{\pgfqpoint{6.160000in}{6.160000in}}%
\pgfusepath{clip}%
\pgfsetbuttcap%
\pgfsetroundjoin%
\definecolor{currentfill}{rgb}{0.275827,0.366717,0.812553}%
\pgfsetfillcolor{currentfill}%
\pgfsetlinewidth{0.000000pt}%
\definecolor{currentstroke}{rgb}{0.000000,0.000000,0.000000}%
\pgfsetstrokecolor{currentstroke}%
\pgfsetdash{}{0pt}%
\pgfpathmoveto{\pgfqpoint{5.564740in}{2.576250in}}%
\pgfpathlineto{\pgfqpoint{5.576731in}{2.560534in}}%
\pgfpathlineto{\pgfqpoint{5.588745in}{2.544892in}}%
\pgfpathlineto{\pgfqpoint{5.622538in}{2.546108in}}%
\pgfpathlineto{\pgfqpoint{5.656311in}{2.547561in}}%
\pgfpathlineto{\pgfqpoint{5.644237in}{2.562899in}}%
\pgfpathlineto{\pgfqpoint{5.632186in}{2.578305in}}%
\pgfpathlineto{\pgfqpoint{5.598473in}{2.577139in}}%
\pgfpathlineto{\pgfqpoint{5.564740in}{2.576250in}}%
\pgfpathclose%
\pgfusepath{fill}%
\end{pgfscope}%
\begin{pgfscope}%
\pgfpathrectangle{\pgfqpoint{1.020000in}{0.880000in}}{\pgfqpoint{6.160000in}{6.160000in}}%
\pgfusepath{clip}%
\pgfsetbuttcap%
\pgfsetroundjoin%
\definecolor{currentfill}{rgb}{0.661968,0.775491,0.993937}%
\pgfsetfillcolor{currentfill}%
\pgfsetlinewidth{0.000000pt}%
\definecolor{currentstroke}{rgb}{0.000000,0.000000,0.000000}%
\pgfsetstrokecolor{currentstroke}%
\pgfsetdash{}{0pt}%
\pgfpathmoveto{\pgfqpoint{3.098407in}{3.254403in}}%
\pgfpathlineto{\pgfqpoint{3.108044in}{3.241896in}}%
\pgfpathlineto{\pgfqpoint{3.117703in}{3.228919in}}%
\pgfpathlineto{\pgfqpoint{3.152024in}{3.259099in}}%
\pgfpathlineto{\pgfqpoint{3.186331in}{3.290265in}}%
\pgfpathlineto{\pgfqpoint{3.176624in}{3.301313in}}%
\pgfpathlineto{\pgfqpoint{3.166941in}{3.311744in}}%
\pgfpathlineto{\pgfqpoint{3.132681in}{3.282618in}}%
\pgfpathlineto{\pgfqpoint{3.098407in}{3.254403in}}%
\pgfpathclose%
\pgfusepath{fill}%
\end{pgfscope}%
\begin{pgfscope}%
\pgfpathrectangle{\pgfqpoint{1.020000in}{0.880000in}}{\pgfqpoint{6.160000in}{6.160000in}}%
\pgfusepath{clip}%
\pgfsetbuttcap%
\pgfsetroundjoin%
\definecolor{currentfill}{rgb}{0.261805,0.346484,0.795658}%
\pgfsetfillcolor{currentfill}%
\pgfsetlinewidth{0.000000pt}%
\definecolor{currentstroke}{rgb}{0.000000,0.000000,0.000000}%
\pgfsetstrokecolor{currentstroke}%
\pgfsetdash{}{0pt}%
\pgfpathmoveto{\pgfqpoint{5.950489in}{2.535709in}}%
\pgfpathlineto{\pgfqpoint{5.962872in}{2.521241in}}%
\pgfpathlineto{\pgfqpoint{5.996555in}{2.523984in}}%
\pgfpathlineto{\pgfqpoint{6.030217in}{2.526770in}}%
\pgfpathlineto{\pgfqpoint{6.017776in}{2.541139in}}%
\pgfpathlineto{\pgfqpoint{5.984143in}{2.538402in}}%
\pgfpathlineto{\pgfqpoint{5.950489in}{2.535709in}}%
\pgfpathclose%
\pgfusepath{fill}%
\end{pgfscope}%
\begin{pgfscope}%
\pgfpathrectangle{\pgfqpoint{1.020000in}{0.880000in}}{\pgfqpoint{6.160000in}{6.160000in}}%
\pgfusepath{clip}%
\pgfsetbuttcap%
\pgfsetroundjoin%
\definecolor{currentfill}{rgb}{0.294718,0.393542,0.834384}%
\pgfsetfillcolor{currentfill}%
\pgfsetlinewidth{0.000000pt}%
\definecolor{currentstroke}{rgb}{0.000000,0.000000,0.000000}%
\pgfsetstrokecolor{currentstroke}%
\pgfsetdash{}{0pt}%
\pgfpathmoveto{\pgfqpoint{5.338420in}{2.613700in}}%
\pgfpathlineto{\pgfqpoint{5.350181in}{2.596290in}}%
\pgfpathlineto{\pgfqpoint{5.361964in}{2.578950in}}%
\pgfpathlineto{\pgfqpoint{5.395805in}{2.577319in}}%
\pgfpathlineto{\pgfqpoint{5.429628in}{2.576224in}}%
\pgfpathlineto{\pgfqpoint{5.417783in}{2.592918in}}%
\pgfpathlineto{\pgfqpoint{5.405961in}{2.609680in}}%
\pgfpathlineto{\pgfqpoint{5.372199in}{2.611384in}}%
\pgfpathlineto{\pgfqpoint{5.338420in}{2.613700in}}%
\pgfpathclose%
\pgfusepath{fill}%
\end{pgfscope}%
\begin{pgfscope}%
\pgfpathrectangle{\pgfqpoint{1.020000in}{0.880000in}}{\pgfqpoint{6.160000in}{6.160000in}}%
\pgfusepath{clip}%
\pgfsetbuttcap%
\pgfsetroundjoin%
\definecolor{currentfill}{rgb}{0.703587,0.802586,0.982847}%
\pgfsetfillcolor{currentfill}%
\pgfsetlinewidth{0.000000pt}%
\definecolor{currentstroke}{rgb}{0.000000,0.000000,0.000000}%
\pgfsetstrokecolor{currentstroke}%
\pgfsetdash{}{0pt}%
\pgfpathmoveto{\pgfqpoint{4.367950in}{3.408411in}}%
\pgfpathlineto{\pgfqpoint{4.378806in}{3.387037in}}%
\pgfpathlineto{\pgfqpoint{4.389676in}{3.363140in}}%
\pgfpathlineto{\pgfqpoint{4.423802in}{3.329756in}}%
\pgfpathlineto{\pgfqpoint{4.457882in}{3.295957in}}%
\pgfpathlineto{\pgfqpoint{4.446964in}{3.319020in}}%
\pgfpathlineto{\pgfqpoint{4.436061in}{3.339844in}}%
\pgfpathlineto{\pgfqpoint{4.402028in}{3.374337in}}%
\pgfpathlineto{\pgfqpoint{4.367950in}{3.408411in}}%
\pgfpathclose%
\pgfusepath{fill}%
\end{pgfscope}%
\begin{pgfscope}%
\pgfpathrectangle{\pgfqpoint{1.020000in}{0.880000in}}{\pgfqpoint{6.160000in}{6.160000in}}%
\pgfusepath{clip}%
\pgfsetbuttcap%
\pgfsetroundjoin%
\definecolor{currentfill}{rgb}{0.630089,0.752516,0.998508}%
\pgfsetfillcolor{currentfill}%
\pgfsetlinewidth{0.000000pt}%
\definecolor{currentstroke}{rgb}{0.000000,0.000000,0.000000}%
\pgfsetstrokecolor{currentstroke}%
\pgfsetdash{}{0pt}%
\pgfpathmoveto{\pgfqpoint{3.029808in}{3.201280in}}%
\pgfpathlineto{\pgfqpoint{3.039400in}{3.186874in}}%
\pgfpathlineto{\pgfqpoint{3.049013in}{3.172128in}}%
\pgfpathlineto{\pgfqpoint{3.083367in}{3.199884in}}%
\pgfpathlineto{\pgfqpoint{3.117703in}{3.228919in}}%
\pgfpathlineto{\pgfqpoint{3.108044in}{3.241896in}}%
\pgfpathlineto{\pgfqpoint{3.098407in}{3.254403in}}%
\pgfpathlineto{\pgfqpoint{3.064117in}{3.227248in}}%
\pgfpathlineto{\pgfqpoint{3.029808in}{3.201280in}}%
\pgfpathclose%
\pgfusepath{fill}%
\end{pgfscope}%
\begin{pgfscope}%
\pgfpathrectangle{\pgfqpoint{1.020000in}{0.880000in}}{\pgfqpoint{6.160000in}{6.160000in}}%
\pgfusepath{clip}%
\pgfsetbuttcap%
\pgfsetroundjoin%
\definecolor{currentfill}{rgb}{0.483854,0.622050,0.974808}%
\pgfsetfillcolor{currentfill}%
\pgfsetlinewidth{0.000000pt}%
\definecolor{currentstroke}{rgb}{0.000000,0.000000,0.000000}%
\pgfsetstrokecolor{currentstroke}%
\pgfsetdash{}{0pt}%
\pgfpathmoveto{\pgfqpoint{4.705842in}{3.004938in}}%
\pgfpathlineto{\pgfqpoint{4.716972in}{2.978243in}}%
\pgfpathlineto{\pgfqpoint{4.728117in}{2.950633in}}%
\pgfpathlineto{\pgfqpoint{4.762066in}{2.925973in}}%
\pgfpathlineto{\pgfqpoint{4.795984in}{2.902539in}}%
\pgfpathlineto{\pgfqpoint{4.784785in}{2.928196in}}%
\pgfpathlineto{\pgfqpoint{4.773601in}{2.953103in}}%
\pgfpathlineto{\pgfqpoint{4.739737in}{2.978365in}}%
\pgfpathlineto{\pgfqpoint{4.705842in}{3.004938in}}%
\pgfpathclose%
\pgfusepath{fill}%
\end{pgfscope}%
\begin{pgfscope}%
\pgfpathrectangle{\pgfqpoint{1.020000in}{0.880000in}}{\pgfqpoint{6.160000in}{6.160000in}}%
\pgfusepath{clip}%
\pgfsetbuttcap%
\pgfsetroundjoin%
\definecolor{currentfill}{rgb}{0.831148,0.859513,0.903110}%
\pgfsetfillcolor{currentfill}%
\pgfsetlinewidth{0.000000pt}%
\definecolor{currentstroke}{rgb}{0.000000,0.000000,0.000000}%
\pgfsetstrokecolor{currentstroke}%
\pgfsetdash{}{0pt}%
\pgfpathmoveto{\pgfqpoint{4.119891in}{3.643949in}}%
\pgfpathlineto{\pgfqpoint{4.130521in}{3.634130in}}%
\pgfpathlineto{\pgfqpoint{4.141174in}{3.621127in}}%
\pgfpathlineto{\pgfqpoint{4.175463in}{3.596466in}}%
\pgfpathlineto{\pgfqpoint{4.209713in}{3.569389in}}%
\pgfpathlineto{\pgfqpoint{4.199007in}{3.582565in}}%
\pgfpathlineto{\pgfqpoint{4.188321in}{3.592762in}}%
\pgfpathlineto{\pgfqpoint{4.154125in}{3.619540in}}%
\pgfpathlineto{\pgfqpoint{4.119891in}{3.643949in}}%
\pgfpathclose%
\pgfusepath{fill}%
\end{pgfscope}%
\begin{pgfscope}%
\pgfpathrectangle{\pgfqpoint{1.020000in}{0.880000in}}{\pgfqpoint{6.160000in}{6.160000in}}%
\pgfusepath{clip}%
\pgfsetbuttcap%
\pgfsetroundjoin%
\definecolor{currentfill}{rgb}{0.871493,0.862309,0.857016}%
\pgfsetfillcolor{currentfill}%
\pgfsetlinewidth{0.000000pt}%
\definecolor{currentstroke}{rgb}{0.000000,0.000000,0.000000}%
\pgfsetstrokecolor{currentstroke}%
\pgfsetdash{}{0pt}%
\pgfpathmoveto{\pgfqpoint{3.961672in}{3.721618in}}%
\pgfpathlineto{\pgfqpoint{3.972132in}{3.718574in}}%
\pgfpathlineto{\pgfqpoint{3.982620in}{3.712417in}}%
\pgfpathlineto{\pgfqpoint{4.016981in}{3.700149in}}%
\pgfpathlineto{\pgfqpoint{4.051315in}{3.684507in}}%
\pgfpathlineto{\pgfqpoint{4.040767in}{3.690869in}}%
\pgfpathlineto{\pgfqpoint{4.030245in}{3.694208in}}%
\pgfpathlineto{\pgfqpoint{3.995972in}{3.709541in}}%
\pgfpathlineto{\pgfqpoint{3.961672in}{3.721618in}}%
\pgfpathclose%
\pgfusepath{fill}%
\end{pgfscope}%
\begin{pgfscope}%
\pgfpathrectangle{\pgfqpoint{1.020000in}{0.880000in}}{\pgfqpoint{6.160000in}{6.160000in}}%
\pgfusepath{clip}%
\pgfsetbuttcap%
\pgfsetroundjoin%
\definecolor{currentfill}{rgb}{0.603162,0.731527,0.999565}%
\pgfsetfillcolor{currentfill}%
\pgfsetlinewidth{0.000000pt}%
\definecolor{currentstroke}{rgb}{0.000000,0.000000,0.000000}%
\pgfsetstrokecolor{currentstroke}%
\pgfsetdash{}{0pt}%
\pgfpathmoveto{\pgfqpoint{2.961125in}{3.153290in}}%
\pgfpathlineto{\pgfqpoint{2.970673in}{3.137185in}}%
\pgfpathlineto{\pgfqpoint{2.980240in}{3.120852in}}%
\pgfpathlineto{\pgfqpoint{3.014638in}{3.145757in}}%
\pgfpathlineto{\pgfqpoint{3.049013in}{3.172128in}}%
\pgfpathlineto{\pgfqpoint{3.039400in}{3.186874in}}%
\pgfpathlineto{\pgfqpoint{3.029808in}{3.201280in}}%
\pgfpathlineto{\pgfqpoint{2.995478in}{3.176601in}}%
\pgfpathlineto{\pgfqpoint{2.961125in}{3.153290in}}%
\pgfpathclose%
\pgfusepath{fill}%
\end{pgfscope}%
\begin{pgfscope}%
\pgfpathrectangle{\pgfqpoint{1.020000in}{0.880000in}}{\pgfqpoint{6.160000in}{6.160000in}}%
\pgfusepath{clip}%
\pgfsetbuttcap%
\pgfsetroundjoin%
\definecolor{currentfill}{rgb}{0.879622,0.858175,0.845844}%
\pgfsetfillcolor{currentfill}%
\pgfsetlinewidth{0.000000pt}%
\definecolor{currentstroke}{rgb}{0.000000,0.000000,0.000000}%
\pgfsetstrokecolor{currentstroke}%
\pgfsetdash{}{0pt}%
\pgfpathmoveto{\pgfqpoint{3.735089in}{3.719214in}}%
\pgfpathlineto{\pgfqpoint{3.745291in}{3.720749in}}%
\pgfpathlineto{\pgfqpoint{3.755524in}{3.719753in}}%
\pgfpathlineto{\pgfqpoint{3.789907in}{3.729055in}}%
\pgfpathlineto{\pgfqpoint{3.824284in}{3.734798in}}%
\pgfpathlineto{\pgfqpoint{3.813984in}{3.735186in}}%
\pgfpathlineto{\pgfqpoint{3.803716in}{3.732971in}}%
\pgfpathlineto{\pgfqpoint{3.769405in}{3.727786in}}%
\pgfpathlineto{\pgfqpoint{3.735089in}{3.719214in}}%
\pgfpathclose%
\pgfusepath{fill}%
\end{pgfscope}%
\begin{pgfscope}%
\pgfpathrectangle{\pgfqpoint{1.020000in}{0.880000in}}{\pgfqpoint{6.160000in}{6.160000in}}%
\pgfusepath{clip}%
\pgfsetbuttcap%
\pgfsetroundjoin%
\definecolor{currentfill}{rgb}{0.368507,0.491141,0.905243}%
\pgfsetfillcolor{currentfill}%
\pgfsetlinewidth{0.000000pt}%
\definecolor{currentstroke}{rgb}{0.000000,0.000000,0.000000}%
\pgfsetstrokecolor{currentstroke}%
\pgfsetdash{}{0pt}%
\pgfpathmoveto{\pgfqpoint{4.954050in}{2.776772in}}%
\pgfpathlineto{\pgfqpoint{4.965417in}{2.753445in}}%
\pgfpathlineto{\pgfqpoint{4.976802in}{2.729930in}}%
\pgfpathlineto{\pgfqpoint{5.010707in}{2.716810in}}%
\pgfpathlineto{\pgfqpoint{5.044590in}{2.704903in}}%
\pgfpathlineto{\pgfqpoint{5.033143in}{2.726759in}}%
\pgfpathlineto{\pgfqpoint{5.021716in}{2.748480in}}%
\pgfpathlineto{\pgfqpoint{4.987894in}{2.761957in}}%
\pgfpathlineto{\pgfqpoint{4.954050in}{2.776772in}}%
\pgfpathclose%
\pgfusepath{fill}%
\end{pgfscope}%
\begin{pgfscope}%
\pgfpathrectangle{\pgfqpoint{1.020000in}{0.880000in}}{\pgfqpoint{6.160000in}{6.160000in}}%
\pgfusepath{clip}%
\pgfsetbuttcap%
\pgfsetroundjoin%
\definecolor{currentfill}{rgb}{0.271104,0.360011,0.807095}%
\pgfsetfillcolor{currentfill}%
\pgfsetlinewidth{0.000000pt}%
\definecolor{currentstroke}{rgb}{0.000000,0.000000,0.000000}%
\pgfsetstrokecolor{currentstroke}%
\pgfsetdash{}{0pt}%
\pgfpathmoveto{\pgfqpoint{5.723800in}{2.551069in}}%
\pgfpathlineto{\pgfqpoint{5.735958in}{2.536035in}}%
\pgfpathlineto{\pgfqpoint{5.748141in}{2.521068in}}%
\pgfpathlineto{\pgfqpoint{5.781915in}{2.523265in}}%
\pgfpathlineto{\pgfqpoint{5.815670in}{2.525579in}}%
\pgfpathlineto{\pgfqpoint{5.803429in}{2.540366in}}%
\pgfpathlineto{\pgfqpoint{5.791212in}{2.555215in}}%
\pgfpathlineto{\pgfqpoint{5.757516in}{2.553073in}}%
\pgfpathlineto{\pgfqpoint{5.723800in}{2.551069in}}%
\pgfpathclose%
\pgfusepath{fill}%
\end{pgfscope}%
\begin{pgfscope}%
\pgfpathrectangle{\pgfqpoint{1.020000in}{0.880000in}}{\pgfqpoint{6.160000in}{6.160000in}}%
\pgfusepath{clip}%
\pgfsetbuttcap%
\pgfsetroundjoin%
\definecolor{currentfill}{rgb}{0.646113,0.764436,0.996868}%
\pgfsetfillcolor{currentfill}%
\pgfsetlinewidth{0.000000pt}%
\definecolor{currentstroke}{rgb}{0.000000,0.000000,0.000000}%
\pgfsetstrokecolor{currentstroke}%
\pgfsetdash{}{0pt}%
\pgfpathmoveto{\pgfqpoint{4.457882in}{3.295957in}}%
\pgfpathlineto{\pgfqpoint{4.468812in}{3.270722in}}%
\pgfpathlineto{\pgfqpoint{4.479756in}{3.243406in}}%
\pgfpathlineto{\pgfqpoint{4.513837in}{3.210757in}}%
\pgfpathlineto{\pgfqpoint{4.547876in}{3.178288in}}%
\pgfpathlineto{\pgfqpoint{4.536885in}{3.204221in}}%
\pgfpathlineto{\pgfqpoint{4.525907in}{3.228341in}}%
\pgfpathlineto{\pgfqpoint{4.491916in}{3.262055in}}%
\pgfpathlineto{\pgfqpoint{4.457882in}{3.295957in}}%
\pgfpathclose%
\pgfusepath{fill}%
\end{pgfscope}%
\begin{pgfscope}%
\pgfpathrectangle{\pgfqpoint{1.020000in}{0.880000in}}{\pgfqpoint{6.160000in}{6.160000in}}%
\pgfusepath{clip}%
\pgfsetbuttcap%
\pgfsetroundjoin%
\definecolor{currentfill}{rgb}{0.328604,0.439712,0.869587}%
\pgfsetfillcolor{currentfill}%
\pgfsetlinewidth{0.000000pt}%
\definecolor{currentstroke}{rgb}{0.000000,0.000000,0.000000}%
\pgfsetstrokecolor{currentstroke}%
\pgfsetdash{}{0pt}%
\pgfpathmoveto{\pgfqpoint{5.112295in}{2.684624in}}%
\pgfpathlineto{\pgfqpoint{5.123824in}{2.664134in}}%
\pgfpathlineto{\pgfqpoint{5.135375in}{2.643642in}}%
\pgfpathlineto{\pgfqpoint{5.169261in}{2.636462in}}%
\pgfpathlineto{\pgfqpoint{5.203129in}{2.630228in}}%
\pgfpathlineto{\pgfqpoint{5.191515in}{2.649501in}}%
\pgfpathlineto{\pgfqpoint{5.179923in}{2.668789in}}%
\pgfpathlineto{\pgfqpoint{5.146118in}{2.676176in}}%
\pgfpathlineto{\pgfqpoint{5.112295in}{2.684624in}}%
\pgfpathclose%
\pgfusepath{fill}%
\end{pgfscope}%
\begin{pgfscope}%
\pgfpathrectangle{\pgfqpoint{1.020000in}{0.880000in}}{\pgfqpoint{6.160000in}{6.160000in}}%
\pgfusepath{clip}%
\pgfsetbuttcap%
\pgfsetroundjoin%
\definecolor{currentfill}{rgb}{0.280550,0.373423,0.818011}%
\pgfsetfillcolor{currentfill}%
\pgfsetlinewidth{0.000000pt}%
\definecolor{currentstroke}{rgb}{0.000000,0.000000,0.000000}%
\pgfsetstrokecolor{currentstroke}%
\pgfsetdash{}{0pt}%
\pgfpathmoveto{\pgfqpoint{5.497221in}{2.575447in}}%
\pgfpathlineto{\pgfqpoint{5.509151in}{2.559336in}}%
\pgfpathlineto{\pgfqpoint{5.521104in}{2.543305in}}%
\pgfpathlineto{\pgfqpoint{5.554934in}{2.543947in}}%
\pgfpathlineto{\pgfqpoint{5.588745in}{2.544892in}}%
\pgfpathlineto{\pgfqpoint{5.576731in}{2.560534in}}%
\pgfpathlineto{\pgfqpoint{5.564740in}{2.576250in}}%
\pgfpathlineto{\pgfqpoint{5.530990in}{2.575672in}}%
\pgfpathlineto{\pgfqpoint{5.497221in}{2.575447in}}%
\pgfpathclose%
\pgfusepath{fill}%
\end{pgfscope}%
\begin{pgfscope}%
\pgfpathrectangle{\pgfqpoint{1.020000in}{0.880000in}}{\pgfqpoint{6.160000in}{6.160000in}}%
\pgfusepath{clip}%
\pgfsetbuttcap%
\pgfsetroundjoin%
\definecolor{currentfill}{rgb}{0.576051,0.708780,0.997755}%
\pgfsetfillcolor{currentfill}%
\pgfsetlinewidth{0.000000pt}%
\definecolor{currentstroke}{rgb}{0.000000,0.000000,0.000000}%
\pgfsetstrokecolor{currentstroke}%
\pgfsetdash{}{0pt}%
\pgfpathmoveto{\pgfqpoint{2.892343in}{3.110972in}}%
\pgfpathlineto{\pgfqpoint{2.901846in}{3.093376in}}%
\pgfpathlineto{\pgfqpoint{2.911368in}{3.075646in}}%
\pgfpathlineto{\pgfqpoint{2.945818in}{3.097471in}}%
\pgfpathlineto{\pgfqpoint{2.980240in}{3.120852in}}%
\pgfpathlineto{\pgfqpoint{2.970673in}{3.137185in}}%
\pgfpathlineto{\pgfqpoint{2.961125in}{3.153290in}}%
\pgfpathlineto{\pgfqpoint{2.926748in}{3.131402in}}%
\pgfpathlineto{\pgfqpoint{2.892343in}{3.110972in}}%
\pgfpathclose%
\pgfusepath{fill}%
\end{pgfscope}%
\begin{pgfscope}%
\pgfpathrectangle{\pgfqpoint{1.020000in}{0.880000in}}{\pgfqpoint{6.160000in}{6.160000in}}%
\pgfusepath{clip}%
\pgfsetbuttcap%
\pgfsetroundjoin%
\definecolor{currentfill}{rgb}{0.266381,0.353304,0.801637}%
\pgfsetfillcolor{currentfill}%
\pgfsetlinewidth{0.000000pt}%
\definecolor{currentstroke}{rgb}{0.000000,0.000000,0.000000}%
\pgfsetstrokecolor{currentstroke}%
\pgfsetdash{}{0pt}%
\pgfpathmoveto{\pgfqpoint{5.883120in}{2.530495in}}%
\pgfpathlineto{\pgfqpoint{5.895444in}{2.515910in}}%
\pgfpathlineto{\pgfqpoint{5.929168in}{2.518546in}}%
\pgfpathlineto{\pgfqpoint{5.962872in}{2.521241in}}%
\pgfpathlineto{\pgfqpoint{5.950489in}{2.535709in}}%
\pgfpathlineto{\pgfqpoint{5.916814in}{2.533070in}}%
\pgfpathlineto{\pgfqpoint{5.883120in}{2.530495in}}%
\pgfpathclose%
\pgfusepath{fill}%
\end{pgfscope}%
\begin{pgfscope}%
\pgfpathrectangle{\pgfqpoint{1.020000in}{0.880000in}}{\pgfqpoint{6.160000in}{6.160000in}}%
\pgfusepath{clip}%
\pgfsetbuttcap%
\pgfsetroundjoin%
\definecolor{currentfill}{rgb}{0.435815,0.570707,0.951717}%
\pgfsetfillcolor{currentfill}%
\pgfsetlinewidth{0.000000pt}%
\definecolor{currentstroke}{rgb}{0.000000,0.000000,0.000000}%
\pgfsetstrokecolor{currentstroke}%
\pgfsetdash{}{0pt}%
\pgfpathmoveto{\pgfqpoint{4.795984in}{2.902539in}}%
\pgfpathlineto{\pgfqpoint{4.807199in}{2.876232in}}%
\pgfpathlineto{\pgfqpoint{4.818430in}{2.849385in}}%
\pgfpathlineto{\pgfqpoint{4.852375in}{2.829276in}}%
\pgfpathlineto{\pgfqpoint{4.886292in}{2.810446in}}%
\pgfpathlineto{\pgfqpoint{4.875003in}{2.835287in}}%
\pgfpathlineto{\pgfqpoint{4.863731in}{2.859698in}}%
\pgfpathlineto{\pgfqpoint{4.829872in}{2.880424in}}%
\pgfpathlineto{\pgfqpoint{4.795984in}{2.902539in}}%
\pgfpathclose%
\pgfusepath{fill}%
\end{pgfscope}%
\begin{pgfscope}%
\pgfpathrectangle{\pgfqpoint{1.020000in}{0.880000in}}{\pgfqpoint{6.160000in}{6.160000in}}%
\pgfusepath{clip}%
\pgfsetbuttcap%
\pgfsetroundjoin%
\definecolor{currentfill}{rgb}{0.299441,0.400248,0.839842}%
\pgfsetfillcolor{currentfill}%
\pgfsetlinewidth{0.000000pt}%
\definecolor{currentstroke}{rgb}{0.000000,0.000000,0.000000}%
\pgfsetstrokecolor{currentstroke}%
\pgfsetdash{}{0pt}%
\pgfpathmoveto{\pgfqpoint{5.270810in}{2.620395in}}%
\pgfpathlineto{\pgfqpoint{5.282508in}{2.602173in}}%
\pgfpathlineto{\pgfqpoint{5.294229in}{2.584025in}}%
\pgfpathlineto{\pgfqpoint{5.328106in}{2.581168in}}%
\pgfpathlineto{\pgfqpoint{5.361964in}{2.578950in}}%
\pgfpathlineto{\pgfqpoint{5.350181in}{2.596290in}}%
\pgfpathlineto{\pgfqpoint{5.338420in}{2.613700in}}%
\pgfpathlineto{\pgfqpoint{5.304624in}{2.616684in}}%
\pgfpathlineto{\pgfqpoint{5.270810in}{2.620395in}}%
\pgfpathclose%
\pgfusepath{fill}%
\end{pgfscope}%
\begin{pgfscope}%
\pgfpathrectangle{\pgfqpoint{1.020000in}{0.880000in}}{\pgfqpoint{6.160000in}{6.160000in}}%
\pgfusepath{clip}%
\pgfsetbuttcap%
\pgfsetroundjoin%
\definecolor{currentfill}{rgb}{0.875557,0.860242,0.851430}%
\pgfsetfillcolor{currentfill}%
\pgfsetlinewidth{0.000000pt}%
\definecolor{currentstroke}{rgb}{0.000000,0.000000,0.000000}%
\pgfsetstrokecolor{currentstroke}%
\pgfsetdash{}{0pt}%
\pgfpathmoveto{\pgfqpoint{3.666451in}{3.692346in}}%
\pgfpathlineto{\pgfqpoint{3.676586in}{3.692850in}}%
\pgfpathlineto{\pgfqpoint{3.686754in}{3.690939in}}%
\pgfpathlineto{\pgfqpoint{3.721139in}{3.706998in}}%
\pgfpathlineto{\pgfqpoint{3.755524in}{3.719753in}}%
\pgfpathlineto{\pgfqpoint{3.745291in}{3.720749in}}%
\pgfpathlineto{\pgfqpoint{3.735089in}{3.719214in}}%
\pgfpathlineto{\pgfqpoint{3.700770in}{3.707354in}}%
\pgfpathlineto{\pgfqpoint{3.666451in}{3.692346in}}%
\pgfpathclose%
\pgfusepath{fill}%
\end{pgfscope}%
\begin{pgfscope}%
\pgfpathrectangle{\pgfqpoint{1.020000in}{0.880000in}}{\pgfqpoint{6.160000in}{6.160000in}}%
\pgfusepath{clip}%
\pgfsetbuttcap%
\pgfsetroundjoin%
\definecolor{currentfill}{rgb}{0.796064,0.848693,0.933471}%
\pgfsetfillcolor{currentfill}%
\pgfsetlinewidth{0.000000pt}%
\definecolor{currentstroke}{rgb}{0.000000,0.000000,0.000000}%
\pgfsetstrokecolor{currentstroke}%
\pgfsetdash{}{0pt}%
\pgfpathmoveto{\pgfqpoint{4.209713in}{3.569389in}}%
\pgfpathlineto{\pgfqpoint{4.220439in}{3.553153in}}%
\pgfpathlineto{\pgfqpoint{4.231183in}{3.533821in}}%
\pgfpathlineto{\pgfqpoint{4.265441in}{3.504783in}}%
\pgfpathlineto{\pgfqpoint{4.299655in}{3.473975in}}%
\pgfpathlineto{\pgfqpoint{4.288863in}{3.493047in}}%
\pgfpathlineto{\pgfqpoint{4.278087in}{3.509274in}}%
\pgfpathlineto{\pgfqpoint{4.243921in}{3.540214in}}%
\pgfpathlineto{\pgfqpoint{4.209713in}{3.569389in}}%
\pgfpathclose%
\pgfusepath{fill}%
\end{pgfscope}%
\begin{pgfscope}%
\pgfpathrectangle{\pgfqpoint{1.020000in}{0.880000in}}{\pgfqpoint{6.160000in}{6.160000in}}%
\pgfusepath{clip}%
\pgfsetbuttcap%
\pgfsetroundjoin%
\definecolor{currentfill}{rgb}{0.554312,0.690097,0.995516}%
\pgfsetfillcolor{currentfill}%
\pgfsetlinewidth{0.000000pt}%
\definecolor{currentstroke}{rgb}{0.000000,0.000000,0.000000}%
\pgfsetstrokecolor{currentstroke}%
\pgfsetdash{}{0pt}%
\pgfpathmoveto{\pgfqpoint{2.823447in}{3.074521in}}%
\pgfpathlineto{\pgfqpoint{2.832905in}{3.055638in}}%
\pgfpathlineto{\pgfqpoint{2.842381in}{3.036699in}}%
\pgfpathlineto{\pgfqpoint{2.876890in}{3.055390in}}%
\pgfpathlineto{\pgfqpoint{2.911368in}{3.075646in}}%
\pgfpathlineto{\pgfqpoint{2.901846in}{3.093376in}}%
\pgfpathlineto{\pgfqpoint{2.892343in}{3.110972in}}%
\pgfpathlineto{\pgfqpoint{2.857910in}{3.092012in}}%
\pgfpathlineto{\pgfqpoint{2.823447in}{3.074521in}}%
\pgfpathclose%
\pgfusepath{fill}%
\end{pgfscope}%
\begin{pgfscope}%
\pgfpathrectangle{\pgfqpoint{1.020000in}{0.880000in}}{\pgfqpoint{6.160000in}{6.160000in}}%
\pgfusepath{clip}%
\pgfsetbuttcap%
\pgfsetroundjoin%
\definecolor{currentfill}{rgb}{0.581486,0.713451,0.998314}%
\pgfsetfillcolor{currentfill}%
\pgfsetlinewidth{0.000000pt}%
\definecolor{currentstroke}{rgb}{0.000000,0.000000,0.000000}%
\pgfsetstrokecolor{currentstroke}%
\pgfsetdash{}{0pt}%
\pgfpathmoveto{\pgfqpoint{4.547876in}{3.178288in}}%
\pgfpathlineto{\pgfqpoint{4.558879in}{3.150649in}}%
\pgfpathlineto{\pgfqpoint{4.569895in}{3.121427in}}%
\pgfpathlineto{\pgfqpoint{4.603939in}{3.091131in}}%
\pgfpathlineto{\pgfqpoint{4.637944in}{3.061486in}}%
\pgfpathlineto{\pgfqpoint{4.626879in}{3.088882in}}%
\pgfpathlineto{\pgfqpoint{4.615827in}{3.114926in}}%
\pgfpathlineto{\pgfqpoint{4.581872in}{3.146265in}}%
\pgfpathlineto{\pgfqpoint{4.547876in}{3.178288in}}%
\pgfpathclose%
\pgfusepath{fill}%
\end{pgfscope}%
\begin{pgfscope}%
\pgfpathrectangle{\pgfqpoint{1.020000in}{0.880000in}}{\pgfqpoint{6.160000in}{6.160000in}}%
\pgfusepath{clip}%
\pgfsetbuttcap%
\pgfsetroundjoin%
\definecolor{currentfill}{rgb}{0.883687,0.856108,0.840258}%
\pgfsetfillcolor{currentfill}%
\pgfsetlinewidth{0.000000pt}%
\definecolor{currentstroke}{rgb}{0.000000,0.000000,0.000000}%
\pgfsetstrokecolor{currentstroke}%
\pgfsetdash{}{0pt}%
\pgfpathmoveto{\pgfqpoint{3.893009in}{3.735403in}}%
\pgfpathlineto{\pgfqpoint{3.903405in}{3.732384in}}%
\pgfpathlineto{\pgfqpoint{3.913831in}{3.726214in}}%
\pgfpathlineto{\pgfqpoint{3.948235in}{3.721145in}}%
\pgfpathlineto{\pgfqpoint{3.982620in}{3.712417in}}%
\pgfpathlineto{\pgfqpoint{3.972132in}{3.718574in}}%
\pgfpathlineto{\pgfqpoint{3.961672in}{3.721618in}}%
\pgfpathlineto{\pgfqpoint{3.927350in}{3.730278in}}%
\pgfpathlineto{\pgfqpoint{3.893009in}{3.735403in}}%
\pgfpathclose%
\pgfusepath{fill}%
\end{pgfscope}%
\begin{pgfscope}%
\pgfpathrectangle{\pgfqpoint{1.020000in}{0.880000in}}{\pgfqpoint{6.160000in}{6.160000in}}%
\pgfusepath{clip}%
\pgfsetbuttcap%
\pgfsetroundjoin%
\definecolor{currentfill}{rgb}{0.271104,0.360011,0.807095}%
\pgfsetfillcolor{currentfill}%
\pgfsetlinewidth{0.000000pt}%
\definecolor{currentstroke}{rgb}{0.000000,0.000000,0.000000}%
\pgfsetstrokecolor{currentstroke}%
\pgfsetdash{}{0pt}%
\pgfpathmoveto{\pgfqpoint{5.656311in}{2.547561in}}%
\pgfpathlineto{\pgfqpoint{5.668410in}{2.532294in}}%
\pgfpathlineto{\pgfqpoint{5.680533in}{2.517100in}}%
\pgfpathlineto{\pgfqpoint{5.714346in}{2.519006in}}%
\pgfpathlineto{\pgfqpoint{5.748141in}{2.521068in}}%
\pgfpathlineto{\pgfqpoint{5.735958in}{2.536035in}}%
\pgfpathlineto{\pgfqpoint{5.723800in}{2.551069in}}%
\pgfpathlineto{\pgfqpoint{5.690065in}{2.549224in}}%
\pgfpathlineto{\pgfqpoint{5.656311in}{2.547561in}}%
\pgfpathclose%
\pgfusepath{fill}%
\end{pgfscope}%
\begin{pgfscope}%
\pgfpathrectangle{\pgfqpoint{1.020000in}{0.880000in}}{\pgfqpoint{6.160000in}{6.160000in}}%
\pgfusepath{clip}%
\pgfsetbuttcap%
\pgfsetroundjoin%
\definecolor{currentfill}{rgb}{0.859385,0.864431,0.872111}%
\pgfsetfillcolor{currentfill}%
\pgfsetlinewidth{0.000000pt}%
\definecolor{currentstroke}{rgb}{0.000000,0.000000,0.000000}%
\pgfsetstrokecolor{currentstroke}%
\pgfsetdash{}{0pt}%
\pgfpathmoveto{\pgfqpoint{3.597818in}{3.653623in}}%
\pgfpathlineto{\pgfqpoint{3.607889in}{3.652787in}}%
\pgfpathlineto{\pgfqpoint{3.617994in}{3.649690in}}%
\pgfpathlineto{\pgfqpoint{3.652372in}{3.671762in}}%
\pgfpathlineto{\pgfqpoint{3.686754in}{3.690939in}}%
\pgfpathlineto{\pgfqpoint{3.676586in}{3.692850in}}%
\pgfpathlineto{\pgfqpoint{3.666451in}{3.692346in}}%
\pgfpathlineto{\pgfqpoint{3.632133in}{3.674366in}}%
\pgfpathlineto{\pgfqpoint{3.597818in}{3.653623in}}%
\pgfpathclose%
\pgfusepath{fill}%
\end{pgfscope}%
\begin{pgfscope}%
\pgfpathrectangle{\pgfqpoint{1.020000in}{0.880000in}}{\pgfqpoint{6.160000in}{6.160000in}}%
\pgfusepath{clip}%
\pgfsetbuttcap%
\pgfsetroundjoin%
\definecolor{currentfill}{rgb}{0.538004,0.674902,0.991722}%
\pgfsetfillcolor{currentfill}%
\pgfsetlinewidth{0.000000pt}%
\definecolor{currentstroke}{rgb}{0.000000,0.000000,0.000000}%
\pgfsetstrokecolor{currentstroke}%
\pgfsetdash{}{0pt}%
\pgfpathmoveto{\pgfqpoint{2.754426in}{3.043846in}}%
\pgfpathlineto{\pgfqpoint{2.763840in}{3.023870in}}%
\pgfpathlineto{\pgfqpoint{2.773270in}{3.003897in}}%
\pgfpathlineto{\pgfqpoint{2.807842in}{3.019547in}}%
\pgfpathlineto{\pgfqpoint{2.842381in}{3.036699in}}%
\pgfpathlineto{\pgfqpoint{2.832905in}{3.055638in}}%
\pgfpathlineto{\pgfqpoint{2.823447in}{3.074521in}}%
\pgfpathlineto{\pgfqpoint{2.788953in}{3.058476in}}%
\pgfpathlineto{\pgfqpoint{2.754426in}{3.043846in}}%
\pgfpathclose%
\pgfusepath{fill}%
\end{pgfscope}%
\begin{pgfscope}%
\pgfpathrectangle{\pgfqpoint{1.020000in}{0.880000in}}{\pgfqpoint{6.160000in}{6.160000in}}%
\pgfusepath{clip}%
\pgfsetbuttcap%
\pgfsetroundjoin%
\definecolor{currentfill}{rgb}{0.285273,0.380129,0.823469}%
\pgfsetfillcolor{currentfill}%
\pgfsetlinewidth{0.000000pt}%
\definecolor{currentstroke}{rgb}{0.000000,0.000000,0.000000}%
\pgfsetstrokecolor{currentstroke}%
\pgfsetdash{}{0pt}%
\pgfpathmoveto{\pgfqpoint{5.429628in}{2.576224in}}%
\pgfpathlineto{\pgfqpoint{5.441497in}{2.559609in}}%
\pgfpathlineto{\pgfqpoint{5.453390in}{2.543083in}}%
\pgfpathlineto{\pgfqpoint{5.487256in}{2.543004in}}%
\pgfpathlineto{\pgfqpoint{5.521104in}{2.543305in}}%
\pgfpathlineto{\pgfqpoint{5.509151in}{2.559336in}}%
\pgfpathlineto{\pgfqpoint{5.497221in}{2.575447in}}%
\pgfpathlineto{\pgfqpoint{5.463433in}{2.575615in}}%
\pgfpathlineto{\pgfqpoint{5.429628in}{2.576224in}}%
\pgfpathclose%
\pgfusepath{fill}%
\end{pgfscope}%
\begin{pgfscope}%
\pgfpathrectangle{\pgfqpoint{1.020000in}{0.880000in}}{\pgfqpoint{6.160000in}{6.160000in}}%
\pgfusepath{clip}%
\pgfsetbuttcap%
\pgfsetroundjoin%
\definecolor{currentfill}{rgb}{0.855378,0.863778,0.876587}%
\pgfsetfillcolor{currentfill}%
\pgfsetlinewidth{0.000000pt}%
\definecolor{currentstroke}{rgb}{0.000000,0.000000,0.000000}%
\pgfsetstrokecolor{currentstroke}%
\pgfsetdash{}{0pt}%
\pgfpathmoveto{\pgfqpoint{4.051315in}{3.684507in}}%
\pgfpathlineto{\pgfqpoint{4.061889in}{3.674940in}}%
\pgfpathlineto{\pgfqpoint{4.072487in}{3.662029in}}%
\pgfpathlineto{\pgfqpoint{4.106847in}{3.643072in}}%
\pgfpathlineto{\pgfqpoint{4.141174in}{3.621127in}}%
\pgfpathlineto{\pgfqpoint{4.130521in}{3.634130in}}%
\pgfpathlineto{\pgfqpoint{4.119891in}{3.643949in}}%
\pgfpathlineto{\pgfqpoint{4.085620in}{3.665694in}}%
\pgfpathlineto{\pgfqpoint{4.051315in}{3.684507in}}%
\pgfpathclose%
\pgfusepath{fill}%
\end{pgfscope}%
\begin{pgfscope}%
\pgfpathrectangle{\pgfqpoint{1.020000in}{0.880000in}}{\pgfqpoint{6.160000in}{6.160000in}}%
\pgfusepath{clip}%
\pgfsetbuttcap%
\pgfsetroundjoin%
\definecolor{currentfill}{rgb}{0.266381,0.353304,0.801637}%
\pgfsetfillcolor{currentfill}%
\pgfsetlinewidth{0.000000pt}%
\definecolor{currentstroke}{rgb}{0.000000,0.000000,0.000000}%
\pgfsetstrokecolor{currentstroke}%
\pgfsetdash{}{0pt}%
\pgfpathmoveto{\pgfqpoint{5.815670in}{2.525579in}}%
\pgfpathlineto{\pgfqpoint{5.827936in}{2.510853in}}%
\pgfpathlineto{\pgfqpoint{5.861700in}{2.513342in}}%
\pgfpathlineto{\pgfqpoint{5.895444in}{2.515910in}}%
\pgfpathlineto{\pgfqpoint{5.883120in}{2.530495in}}%
\pgfpathlineto{\pgfqpoint{5.849405in}{2.527993in}}%
\pgfpathlineto{\pgfqpoint{5.815670in}{2.525579in}}%
\pgfpathclose%
\pgfusepath{fill}%
\end{pgfscope}%
\begin{pgfscope}%
\pgfpathrectangle{\pgfqpoint{1.020000in}{0.880000in}}{\pgfqpoint{6.160000in}{6.160000in}}%
\pgfusepath{clip}%
\pgfsetbuttcap%
\pgfsetroundjoin%
\definecolor{currentfill}{rgb}{0.521696,0.659599,0.987736}%
\pgfsetfillcolor{currentfill}%
\pgfsetlinewidth{0.000000pt}%
\definecolor{currentstroke}{rgb}{0.000000,0.000000,0.000000}%
\pgfsetstrokecolor{currentstroke}%
\pgfsetdash{}{0pt}%
\pgfpathmoveto{\pgfqpoint{4.637944in}{3.061486in}}%
\pgfpathlineto{\pgfqpoint{4.649023in}{3.032860in}}%
\pgfpathlineto{\pgfqpoint{4.660115in}{3.003139in}}%
\pgfpathlineto{\pgfqpoint{4.694133in}{2.976403in}}%
\pgfpathlineto{\pgfqpoint{4.728117in}{2.950633in}}%
\pgfpathlineto{\pgfqpoint{4.716972in}{2.978243in}}%
\pgfpathlineto{\pgfqpoint{4.705842in}{3.004938in}}%
\pgfpathlineto{\pgfqpoint{4.671911in}{3.032695in}}%
\pgfpathlineto{\pgfqpoint{4.637944in}{3.061486in}}%
\pgfpathclose%
\pgfusepath{fill}%
\end{pgfscope}%
\begin{pgfscope}%
\pgfpathrectangle{\pgfqpoint{1.020000in}{0.880000in}}{\pgfqpoint{6.160000in}{6.160000in}}%
\pgfusepath{clip}%
\pgfsetbuttcap%
\pgfsetroundjoin%
\definecolor{currentfill}{rgb}{0.343278,0.459354,0.884122}%
\pgfsetfillcolor{currentfill}%
\pgfsetlinewidth{0.000000pt}%
\definecolor{currentstroke}{rgb}{0.000000,0.000000,0.000000}%
\pgfsetstrokecolor{currentstroke}%
\pgfsetdash{}{0pt}%
\pgfpathmoveto{\pgfqpoint{5.044590in}{2.704903in}}%
\pgfpathlineto{\pgfqpoint{5.056057in}{2.682971in}}%
\pgfpathlineto{\pgfqpoint{5.067544in}{2.661019in}}%
\pgfpathlineto{\pgfqpoint{5.101469in}{2.651815in}}%
\pgfpathlineto{\pgfqpoint{5.135375in}{2.643642in}}%
\pgfpathlineto{\pgfqpoint{5.123824in}{2.664134in}}%
\pgfpathlineto{\pgfqpoint{5.112295in}{2.684624in}}%
\pgfpathlineto{\pgfqpoint{5.078452in}{2.694186in}}%
\pgfpathlineto{\pgfqpoint{5.044590in}{2.704903in}}%
\pgfpathclose%
\pgfusepath{fill}%
\end{pgfscope}%
\begin{pgfscope}%
\pgfpathrectangle{\pgfqpoint{1.020000in}{0.880000in}}{\pgfqpoint{6.160000in}{6.160000in}}%
\pgfusepath{clip}%
\pgfsetbuttcap%
\pgfsetroundjoin%
\definecolor{currentfill}{rgb}{0.748682,0.827679,0.963334}%
\pgfsetfillcolor{currentfill}%
\pgfsetlinewidth{0.000000pt}%
\definecolor{currentstroke}{rgb}{0.000000,0.000000,0.000000}%
\pgfsetstrokecolor{currentstroke}%
\pgfsetdash{}{0pt}%
\pgfpathmoveto{\pgfqpoint{4.299655in}{3.473975in}}%
\pgfpathlineto{\pgfqpoint{4.310465in}{3.452063in}}%
\pgfpathlineto{\pgfqpoint{4.321289in}{3.427354in}}%
\pgfpathlineto{\pgfqpoint{4.355506in}{3.395784in}}%
\pgfpathlineto{\pgfqpoint{4.389676in}{3.363140in}}%
\pgfpathlineto{\pgfqpoint{4.378806in}{3.387037in}}%
\pgfpathlineto{\pgfqpoint{4.367950in}{3.408411in}}%
\pgfpathlineto{\pgfqpoint{4.333825in}{3.441737in}}%
\pgfpathlineto{\pgfqpoint{4.299655in}{3.473975in}}%
\pgfpathclose%
\pgfusepath{fill}%
\end{pgfscope}%
\begin{pgfscope}%
\pgfpathrectangle{\pgfqpoint{1.020000in}{0.880000in}}{\pgfqpoint{6.160000in}{6.160000in}}%
\pgfusepath{clip}%
\pgfsetbuttcap%
\pgfsetroundjoin%
\definecolor{currentfill}{rgb}{0.843358,0.861820,0.890017}%
\pgfsetfillcolor{currentfill}%
\pgfsetlinewidth{0.000000pt}%
\definecolor{currentstroke}{rgb}{0.000000,0.000000,0.000000}%
\pgfsetstrokecolor{currentstroke}%
\pgfsetdash{}{0pt}%
\pgfpathmoveto{\pgfqpoint{3.529204in}{3.604831in}}%
\pgfpathlineto{\pgfqpoint{3.539213in}{3.602403in}}%
\pgfpathlineto{\pgfqpoint{3.549255in}{3.597897in}}%
\pgfpathlineto{\pgfqpoint{3.583621in}{3.624975in}}%
\pgfpathlineto{\pgfqpoint{3.617994in}{3.649690in}}%
\pgfpathlineto{\pgfqpoint{3.607889in}{3.652787in}}%
\pgfpathlineto{\pgfqpoint{3.597818in}{3.653623in}}%
\pgfpathlineto{\pgfqpoint{3.563509in}{3.630357in}}%
\pgfpathlineto{\pgfqpoint{3.529204in}{3.604831in}}%
\pgfpathclose%
\pgfusepath{fill}%
\end{pgfscope}%
\begin{pgfscope}%
\pgfpathrectangle{\pgfqpoint{1.020000in}{0.880000in}}{\pgfqpoint{6.160000in}{6.160000in}}%
\pgfusepath{clip}%
\pgfsetbuttcap%
\pgfsetroundjoin%
\definecolor{currentfill}{rgb}{0.394042,0.522413,0.924916}%
\pgfsetfillcolor{currentfill}%
\pgfsetlinewidth{0.000000pt}%
\definecolor{currentstroke}{rgb}{0.000000,0.000000,0.000000}%
\pgfsetstrokecolor{currentstroke}%
\pgfsetdash{}{0pt}%
\pgfpathmoveto{\pgfqpoint{4.886292in}{2.810446in}}%
\pgfpathlineto{\pgfqpoint{4.897598in}{2.785265in}}%
\pgfpathlineto{\pgfqpoint{4.908922in}{2.759835in}}%
\pgfpathlineto{\pgfqpoint{4.942874in}{2.744273in}}%
\pgfpathlineto{\pgfqpoint{4.976802in}{2.729930in}}%
\pgfpathlineto{\pgfqpoint{4.965417in}{2.753445in}}%
\pgfpathlineto{\pgfqpoint{4.954050in}{2.776772in}}%
\pgfpathlineto{\pgfqpoint{4.920183in}{2.792937in}}%
\pgfpathlineto{\pgfqpoint{4.886292in}{2.810446in}}%
\pgfpathclose%
\pgfusepath{fill}%
\end{pgfscope}%
\begin{pgfscope}%
\pgfpathrectangle{\pgfqpoint{1.020000in}{0.880000in}}{\pgfqpoint{6.160000in}{6.160000in}}%
\pgfusepath{clip}%
\pgfsetbuttcap%
\pgfsetroundjoin%
\definecolor{currentfill}{rgb}{0.309060,0.413498,0.850128}%
\pgfsetfillcolor{currentfill}%
\pgfsetlinewidth{0.000000pt}%
\definecolor{currentstroke}{rgb}{0.000000,0.000000,0.000000}%
\pgfsetstrokecolor{currentstroke}%
\pgfsetdash{}{0pt}%
\pgfpathmoveto{\pgfqpoint{5.203129in}{2.630228in}}%
\pgfpathlineto{\pgfqpoint{5.214764in}{2.611002in}}%
\pgfpathlineto{\pgfqpoint{5.226422in}{2.591853in}}%
\pgfpathlineto{\pgfqpoint{5.260334in}{2.587570in}}%
\pgfpathlineto{\pgfqpoint{5.294229in}{2.584025in}}%
\pgfpathlineto{\pgfqpoint{5.282508in}{2.602173in}}%
\pgfpathlineto{\pgfqpoint{5.270810in}{2.620395in}}%
\pgfpathlineto{\pgfqpoint{5.236978in}{2.624890in}}%
\pgfpathlineto{\pgfqpoint{5.203129in}{2.630228in}}%
\pgfpathclose%
\pgfusepath{fill}%
\end{pgfscope}%
\begin{pgfscope}%
\pgfpathrectangle{\pgfqpoint{1.020000in}{0.880000in}}{\pgfqpoint{6.160000in}{6.160000in}}%
\pgfusepath{clip}%
\pgfsetbuttcap%
\pgfsetroundjoin%
\definecolor{currentfill}{rgb}{0.822420,0.856898,0.910795}%
\pgfsetfillcolor{currentfill}%
\pgfsetlinewidth{0.000000pt}%
\definecolor{currentstroke}{rgb}{0.000000,0.000000,0.000000}%
\pgfsetstrokecolor{currentstroke}%
\pgfsetdash{}{0pt}%
\pgfpathmoveto{\pgfqpoint{3.460612in}{3.548145in}}%
\pgfpathlineto{\pgfqpoint{3.470561in}{3.543935in}}%
\pgfpathlineto{\pgfqpoint{3.480543in}{3.537855in}}%
\pgfpathlineto{\pgfqpoint{3.514896in}{3.568753in}}%
\pgfpathlineto{\pgfqpoint{3.549255in}{3.597897in}}%
\pgfpathlineto{\pgfqpoint{3.539213in}{3.602403in}}%
\pgfpathlineto{\pgfqpoint{3.529204in}{3.604831in}}%
\pgfpathlineto{\pgfqpoint{3.494905in}{3.577328in}}%
\pgfpathlineto{\pgfqpoint{3.460612in}{3.548145in}}%
\pgfpathclose%
\pgfusepath{fill}%
\end{pgfscope}%
\begin{pgfscope}%
\pgfpathrectangle{\pgfqpoint{1.020000in}{0.880000in}}{\pgfqpoint{6.160000in}{6.160000in}}%
\pgfusepath{clip}%
\pgfsetbuttcap%
\pgfsetroundjoin%
\definecolor{currentfill}{rgb}{0.527132,0.664700,0.989065}%
\pgfsetfillcolor{currentfill}%
\pgfsetlinewidth{0.000000pt}%
\definecolor{currentstroke}{rgb}{0.000000,0.000000,0.000000}%
\pgfsetstrokecolor{currentstroke}%
\pgfsetdash{}{0pt}%
\pgfpathmoveto{\pgfqpoint{2.685275in}{3.018633in}}%
\pgfpathlineto{\pgfqpoint{2.694643in}{2.997738in}}%
\pgfpathlineto{\pgfqpoint{2.704026in}{2.976891in}}%
\pgfpathlineto{\pgfqpoint{2.738665in}{2.989698in}}%
\pgfpathlineto{\pgfqpoint{2.773270in}{3.003897in}}%
\pgfpathlineto{\pgfqpoint{2.763840in}{3.023870in}}%
\pgfpathlineto{\pgfqpoint{2.754426in}{3.043846in}}%
\pgfpathlineto{\pgfqpoint{2.719868in}{3.030583in}}%
\pgfpathlineto{\pgfqpoint{2.685275in}{3.018633in}}%
\pgfpathclose%
\pgfusepath{fill}%
\end{pgfscope}%
\begin{pgfscope}%
\pgfpathrectangle{\pgfqpoint{1.020000in}{0.880000in}}{\pgfqpoint{6.160000in}{6.160000in}}%
\pgfusepath{clip}%
\pgfsetbuttcap%
\pgfsetroundjoin%
\definecolor{currentfill}{rgb}{0.791392,0.846750,0.936641}%
\pgfsetfillcolor{currentfill}%
\pgfsetlinewidth{0.000000pt}%
\definecolor{currentstroke}{rgb}{0.000000,0.000000,0.000000}%
\pgfsetstrokecolor{currentstroke}%
\pgfsetdash{}{0pt}%
\pgfpathmoveto{\pgfqpoint{3.392041in}{3.485959in}}%
\pgfpathlineto{\pgfqpoint{3.401932in}{3.479851in}}%
\pgfpathlineto{\pgfqpoint{3.411856in}{3.472092in}}%
\pgfpathlineto{\pgfqpoint{3.446197in}{3.505527in}}%
\pgfpathlineto{\pgfqpoint{3.480543in}{3.537855in}}%
\pgfpathlineto{\pgfqpoint{3.470561in}{3.543935in}}%
\pgfpathlineto{\pgfqpoint{3.460612in}{3.548145in}}%
\pgfpathlineto{\pgfqpoint{3.426324in}{3.517585in}}%
\pgfpathlineto{\pgfqpoint{3.392041in}{3.485959in}}%
\pgfpathclose%
\pgfusepath{fill}%
\end{pgfscope}%
\begin{pgfscope}%
\pgfpathrectangle{\pgfqpoint{1.020000in}{0.880000in}}{\pgfqpoint{6.160000in}{6.160000in}}%
\pgfusepath{clip}%
\pgfsetbuttcap%
\pgfsetroundjoin%
\definecolor{currentfill}{rgb}{0.763363,0.835092,0.955658}%
\pgfsetfillcolor{currentfill}%
\pgfsetlinewidth{0.000000pt}%
\definecolor{currentstroke}{rgb}{0.000000,0.000000,0.000000}%
\pgfsetstrokecolor{currentstroke}%
\pgfsetdash{}{0pt}%
\pgfpathmoveto{\pgfqpoint{3.323480in}{3.420726in}}%
\pgfpathlineto{\pgfqpoint{3.333317in}{3.412669in}}%
\pgfpathlineto{\pgfqpoint{3.343185in}{3.403186in}}%
\pgfpathlineto{\pgfqpoint{3.377519in}{3.437874in}}%
\pgfpathlineto{\pgfqpoint{3.411856in}{3.472092in}}%
\pgfpathlineto{\pgfqpoint{3.401932in}{3.479851in}}%
\pgfpathlineto{\pgfqpoint{3.392041in}{3.485959in}}%
\pgfpathlineto{\pgfqpoint{3.357760in}{3.453573in}}%
\pgfpathlineto{\pgfqpoint{3.323480in}{3.420726in}}%
\pgfpathclose%
\pgfusepath{fill}%
\end{pgfscope}%
\begin{pgfscope}%
\pgfpathrectangle{\pgfqpoint{1.020000in}{0.880000in}}{\pgfqpoint{6.160000in}{6.160000in}}%
\pgfusepath{clip}%
\pgfsetbuttcap%
\pgfsetroundjoin%
\definecolor{currentfill}{rgb}{0.271104,0.360011,0.807095}%
\pgfsetfillcolor{currentfill}%
\pgfsetlinewidth{0.000000pt}%
\definecolor{currentstroke}{rgb}{0.000000,0.000000,0.000000}%
\pgfsetstrokecolor{currentstroke}%
\pgfsetdash{}{0pt}%
\pgfpathmoveto{\pgfqpoint{5.588745in}{2.544892in}}%
\pgfpathlineto{\pgfqpoint{5.600784in}{2.529328in}}%
\pgfpathlineto{\pgfqpoint{5.612847in}{2.513845in}}%
\pgfpathlineto{\pgfqpoint{5.646699in}{2.515371in}}%
\pgfpathlineto{\pgfqpoint{5.680533in}{2.517100in}}%
\pgfpathlineto{\pgfqpoint{5.668410in}{2.532294in}}%
\pgfpathlineto{\pgfqpoint{5.656311in}{2.547561in}}%
\pgfpathlineto{\pgfqpoint{5.622538in}{2.546108in}}%
\pgfpathlineto{\pgfqpoint{5.588745in}{2.544892in}}%
\pgfpathclose%
\pgfusepath{fill}%
\end{pgfscope}%
\begin{pgfscope}%
\pgfpathrectangle{\pgfqpoint{1.020000in}{0.880000in}}{\pgfqpoint{6.160000in}{6.160000in}}%
\pgfusepath{clip}%
\pgfsetbuttcap%
\pgfsetroundjoin%
\definecolor{currentfill}{rgb}{0.688188,0.793178,0.988038}%
\pgfsetfillcolor{currentfill}%
\pgfsetlinewidth{0.000000pt}%
\definecolor{currentstroke}{rgb}{0.000000,0.000000,0.000000}%
\pgfsetstrokecolor{currentstroke}%
\pgfsetdash{}{0pt}%
\pgfpathmoveto{\pgfqpoint{4.389676in}{3.363140in}}%
\pgfpathlineto{\pgfqpoint{4.400561in}{3.336791in}}%
\pgfpathlineto{\pgfqpoint{4.411459in}{3.308095in}}%
\pgfpathlineto{\pgfqpoint{4.445630in}{3.275952in}}%
\pgfpathlineto{\pgfqpoint{4.479756in}{3.243406in}}%
\pgfpathlineto{\pgfqpoint{4.468812in}{3.270722in}}%
\pgfpathlineto{\pgfqpoint{4.457882in}{3.295957in}}%
\pgfpathlineto{\pgfqpoint{4.423802in}{3.329756in}}%
\pgfpathlineto{\pgfqpoint{4.389676in}{3.363140in}}%
\pgfpathclose%
\pgfusepath{fill}%
\end{pgfscope}%
\begin{pgfscope}%
\pgfpathrectangle{\pgfqpoint{1.020000in}{0.880000in}}{\pgfqpoint{6.160000in}{6.160000in}}%
\pgfusepath{clip}%
\pgfsetbuttcap%
\pgfsetroundjoin%
\definecolor{currentfill}{rgb}{0.728970,0.817464,0.973188}%
\pgfsetfillcolor{currentfill}%
\pgfsetlinewidth{0.000000pt}%
\definecolor{currentstroke}{rgb}{0.000000,0.000000,0.000000}%
\pgfsetstrokecolor{currentstroke}%
\pgfsetdash{}{0pt}%
\pgfpathmoveto{\pgfqpoint{3.254916in}{3.354792in}}%
\pgfpathlineto{\pgfqpoint{3.264701in}{3.344799in}}%
\pgfpathlineto{\pgfqpoint{3.274514in}{3.333603in}}%
\pgfpathlineto{\pgfqpoint{3.308851in}{3.368334in}}%
\pgfpathlineto{\pgfqpoint{3.343185in}{3.403186in}}%
\pgfpathlineto{\pgfqpoint{3.333317in}{3.412669in}}%
\pgfpathlineto{\pgfqpoint{3.323480in}{3.420726in}}%
\pgfpathlineto{\pgfqpoint{3.289200in}{3.387708in}}%
\pgfpathlineto{\pgfqpoint{3.254916in}{3.354792in}}%
\pgfpathclose%
\pgfusepath{fill}%
\end{pgfscope}%
\begin{pgfscope}%
\pgfpathrectangle{\pgfqpoint{1.020000in}{0.880000in}}{\pgfqpoint{6.160000in}{6.160000in}}%
\pgfusepath{clip}%
\pgfsetbuttcap%
\pgfsetroundjoin%
\definecolor{currentfill}{rgb}{0.462354,0.599830,0.965857}%
\pgfsetfillcolor{currentfill}%
\pgfsetlinewidth{0.000000pt}%
\definecolor{currentstroke}{rgb}{0.000000,0.000000,0.000000}%
\pgfsetstrokecolor{currentstroke}%
\pgfsetdash{}{0pt}%
\pgfpathmoveto{\pgfqpoint{4.728117in}{2.950633in}}%
\pgfpathlineto{\pgfqpoint{4.739276in}{2.922230in}}%
\pgfpathlineto{\pgfqpoint{4.750450in}{2.893163in}}%
\pgfpathlineto{\pgfqpoint{4.784455in}{2.870710in}}%
\pgfpathlineto{\pgfqpoint{4.818430in}{2.849385in}}%
\pgfpathlineto{\pgfqpoint{4.807199in}{2.876232in}}%
\pgfpathlineto{\pgfqpoint{4.795984in}{2.902539in}}%
\pgfpathlineto{\pgfqpoint{4.762066in}{2.925973in}}%
\pgfpathlineto{\pgfqpoint{4.728117in}{2.950633in}}%
\pgfpathclose%
\pgfusepath{fill}%
\end{pgfscope}%
\begin{pgfscope}%
\pgfpathrectangle{\pgfqpoint{1.020000in}{0.880000in}}{\pgfqpoint{6.160000in}{6.160000in}}%
\pgfusepath{clip}%
\pgfsetbuttcap%
\pgfsetroundjoin%
\definecolor{currentfill}{rgb}{0.887752,0.854040,0.834671}%
\pgfsetfillcolor{currentfill}%
\pgfsetlinewidth{0.000000pt}%
\definecolor{currentstroke}{rgb}{0.000000,0.000000,0.000000}%
\pgfsetstrokecolor{currentstroke}%
\pgfsetdash{}{0pt}%
\pgfpathmoveto{\pgfqpoint{3.824284in}{3.734798in}}%
\pgfpathlineto{\pgfqpoint{3.834615in}{3.731522in}}%
\pgfpathlineto{\pgfqpoint{3.844978in}{3.725113in}}%
\pgfpathlineto{\pgfqpoint{3.879411in}{3.727546in}}%
\pgfpathlineto{\pgfqpoint{3.913831in}{3.726214in}}%
\pgfpathlineto{\pgfqpoint{3.903405in}{3.732384in}}%
\pgfpathlineto{\pgfqpoint{3.893009in}{3.735403in}}%
\pgfpathlineto{\pgfqpoint{3.858653in}{3.736920in}}%
\pgfpathlineto{\pgfqpoint{3.824284in}{3.734798in}}%
\pgfpathclose%
\pgfusepath{fill}%
\end{pgfscope}%
\begin{pgfscope}%
\pgfpathrectangle{\pgfqpoint{1.020000in}{0.880000in}}{\pgfqpoint{6.160000in}{6.160000in}}%
\pgfusepath{clip}%
\pgfsetbuttcap%
\pgfsetroundjoin%
\definecolor{currentfill}{rgb}{0.693321,0.796314,0.986308}%
\pgfsetfillcolor{currentfill}%
\pgfsetlinewidth{0.000000pt}%
\definecolor{currentstroke}{rgb}{0.000000,0.000000,0.000000}%
\pgfsetstrokecolor{currentstroke}%
\pgfsetdash{}{0pt}%
\pgfpathmoveto{\pgfqpoint{3.186331in}{3.290265in}}%
\pgfpathlineto{\pgfqpoint{3.196065in}{3.278405in}}%
\pgfpathlineto{\pgfqpoint{3.205826in}{3.265552in}}%
\pgfpathlineto{\pgfqpoint{3.240174in}{3.299260in}}%
\pgfpathlineto{\pgfqpoint{3.274514in}{3.333603in}}%
\pgfpathlineto{\pgfqpoint{3.264701in}{3.344799in}}%
\pgfpathlineto{\pgfqpoint{3.254916in}{3.354792in}}%
\pgfpathlineto{\pgfqpoint{3.220628in}{3.322233in}}%
\pgfpathlineto{\pgfqpoint{3.186331in}{3.290265in}}%
\pgfpathclose%
\pgfusepath{fill}%
\end{pgfscope}%
\begin{pgfscope}%
\pgfpathrectangle{\pgfqpoint{1.020000in}{0.880000in}}{\pgfqpoint{6.160000in}{6.160000in}}%
\pgfusepath{clip}%
\pgfsetbuttcap%
\pgfsetroundjoin%
\definecolor{currentfill}{rgb}{0.266381,0.353304,0.801637}%
\pgfsetfillcolor{currentfill}%
\pgfsetlinewidth{0.000000pt}%
\definecolor{currentstroke}{rgb}{0.000000,0.000000,0.000000}%
\pgfsetstrokecolor{currentstroke}%
\pgfsetdash{}{0pt}%
\pgfpathmoveto{\pgfqpoint{5.748141in}{2.521068in}}%
\pgfpathlineto{\pgfqpoint{5.760348in}{2.506167in}}%
\pgfpathlineto{\pgfqpoint{5.794152in}{2.508456in}}%
\pgfpathlineto{\pgfqpoint{5.827936in}{2.510853in}}%
\pgfpathlineto{\pgfqpoint{5.815670in}{2.525579in}}%
\pgfpathlineto{\pgfqpoint{5.781915in}{2.523265in}}%
\pgfpathlineto{\pgfqpoint{5.748141in}{2.521068in}}%
\pgfpathclose%
\pgfusepath{fill}%
\end{pgfscope}%
\begin{pgfscope}%
\pgfpathrectangle{\pgfqpoint{1.020000in}{0.880000in}}{\pgfqpoint{6.160000in}{6.160000in}}%
\pgfusepath{clip}%
\pgfsetbuttcap%
\pgfsetroundjoin%
\definecolor{currentfill}{rgb}{0.289996,0.386836,0.828926}%
\pgfsetfillcolor{currentfill}%
\pgfsetlinewidth{0.000000pt}%
\definecolor{currentstroke}{rgb}{0.000000,0.000000,0.000000}%
\pgfsetstrokecolor{currentstroke}%
\pgfsetdash{}{0pt}%
\pgfpathmoveto{\pgfqpoint{5.361964in}{2.578950in}}%
\pgfpathlineto{\pgfqpoint{5.373771in}{2.561697in}}%
\pgfpathlineto{\pgfqpoint{5.385601in}{2.544545in}}%
\pgfpathlineto{\pgfqpoint{5.419505in}{2.543582in}}%
\pgfpathlineto{\pgfqpoint{5.453390in}{2.543083in}}%
\pgfpathlineto{\pgfqpoint{5.441497in}{2.559609in}}%
\pgfpathlineto{\pgfqpoint{5.429628in}{2.576224in}}%
\pgfpathlineto{\pgfqpoint{5.395805in}{2.577319in}}%
\pgfpathlineto{\pgfqpoint{5.361964in}{2.578950in}}%
\pgfpathclose%
\pgfusepath{fill}%
\end{pgfscope}%
\begin{pgfscope}%
\pgfpathrectangle{\pgfqpoint{1.020000in}{0.880000in}}{\pgfqpoint{6.160000in}{6.160000in}}%
\pgfusepath{clip}%
\pgfsetbuttcap%
\pgfsetroundjoin%
\definecolor{currentfill}{rgb}{0.656683,0.771806,0.994914}%
\pgfsetfillcolor{currentfill}%
\pgfsetlinewidth{0.000000pt}%
\definecolor{currentstroke}{rgb}{0.000000,0.000000,0.000000}%
\pgfsetstrokecolor{currentstroke}%
\pgfsetdash{}{0pt}%
\pgfpathmoveto{\pgfqpoint{3.117703in}{3.228919in}}%
\pgfpathlineto{\pgfqpoint{3.127388in}{3.215301in}}%
\pgfpathlineto{\pgfqpoint{3.137098in}{3.200889in}}%
\pgfpathlineto{\pgfqpoint{3.171468in}{3.232697in}}%
\pgfpathlineto{\pgfqpoint{3.205826in}{3.265552in}}%
\pgfpathlineto{\pgfqpoint{3.196065in}{3.278405in}}%
\pgfpathlineto{\pgfqpoint{3.186331in}{3.290265in}}%
\pgfpathlineto{\pgfqpoint{3.152024in}{3.259099in}}%
\pgfpathlineto{\pgfqpoint{3.117703in}{3.228919in}}%
\pgfpathclose%
\pgfusepath{fill}%
\end{pgfscope}%
\begin{pgfscope}%
\pgfpathrectangle{\pgfqpoint{1.020000in}{0.880000in}}{\pgfqpoint{6.160000in}{6.160000in}}%
\pgfusepath{clip}%
\pgfsetbuttcap%
\pgfsetroundjoin%
\definecolor{currentfill}{rgb}{0.624703,0.748318,0.998719}%
\pgfsetfillcolor{currentfill}%
\pgfsetlinewidth{0.000000pt}%
\definecolor{currentstroke}{rgb}{0.000000,0.000000,0.000000}%
\pgfsetstrokecolor{currentstroke}%
\pgfsetdash{}{0pt}%
\pgfpathmoveto{\pgfqpoint{3.049013in}{3.172128in}}%
\pgfpathlineto{\pgfqpoint{3.058650in}{3.156897in}}%
\pgfpathlineto{\pgfqpoint{3.068310in}{3.141048in}}%
\pgfpathlineto{\pgfqpoint{3.102713in}{3.170293in}}%
\pgfpathlineto{\pgfqpoint{3.137098in}{3.200889in}}%
\pgfpathlineto{\pgfqpoint{3.127388in}{3.215301in}}%
\pgfpathlineto{\pgfqpoint{3.117703in}{3.228919in}}%
\pgfpathlineto{\pgfqpoint{3.083367in}{3.199884in}}%
\pgfpathlineto{\pgfqpoint{3.049013in}{3.172128in}}%
\pgfpathclose%
\pgfusepath{fill}%
\end{pgfscope}%
\begin{pgfscope}%
\pgfpathrectangle{\pgfqpoint{1.020000in}{0.880000in}}{\pgfqpoint{6.160000in}{6.160000in}}%
\pgfusepath{clip}%
\pgfsetbuttcap%
\pgfsetroundjoin%
\definecolor{currentfill}{rgb}{0.826784,0.858205,0.906953}%
\pgfsetfillcolor{currentfill}%
\pgfsetlinewidth{0.000000pt}%
\definecolor{currentstroke}{rgb}{0.000000,0.000000,0.000000}%
\pgfsetstrokecolor{currentstroke}%
\pgfsetdash{}{0pt}%
\pgfpathmoveto{\pgfqpoint{4.141174in}{3.621127in}}%
\pgfpathlineto{\pgfqpoint{4.151848in}{3.604853in}}%
\pgfpathlineto{\pgfqpoint{4.162542in}{3.585267in}}%
\pgfpathlineto{\pgfqpoint{4.196883in}{3.560755in}}%
\pgfpathlineto{\pgfqpoint{4.231183in}{3.533821in}}%
\pgfpathlineto{\pgfqpoint{4.220439in}{3.553153in}}%
\pgfpathlineto{\pgfqpoint{4.209713in}{3.569389in}}%
\pgfpathlineto{\pgfqpoint{4.175463in}{3.596466in}}%
\pgfpathlineto{\pgfqpoint{4.141174in}{3.621127in}}%
\pgfpathclose%
\pgfusepath{fill}%
\end{pgfscope}%
\begin{pgfscope}%
\pgfpathrectangle{\pgfqpoint{1.020000in}{0.880000in}}{\pgfqpoint{6.160000in}{6.160000in}}%
\pgfusepath{clip}%
\pgfsetbuttcap%
\pgfsetroundjoin%
\definecolor{currentfill}{rgb}{0.624703,0.748318,0.998719}%
\pgfsetfillcolor{currentfill}%
\pgfsetlinewidth{0.000000pt}%
\definecolor{currentstroke}{rgb}{0.000000,0.000000,0.000000}%
\pgfsetstrokecolor{currentstroke}%
\pgfsetdash{}{0pt}%
\pgfpathmoveto{\pgfqpoint{4.479756in}{3.243406in}}%
\pgfpathlineto{\pgfqpoint{4.490712in}{3.214129in}}%
\pgfpathlineto{\pgfqpoint{4.501681in}{3.183033in}}%
\pgfpathlineto{\pgfqpoint{4.535809in}{3.152145in}}%
\pgfpathlineto{\pgfqpoint{4.569895in}{3.121427in}}%
\pgfpathlineto{\pgfqpoint{4.558879in}{3.150649in}}%
\pgfpathlineto{\pgfqpoint{4.547876in}{3.178288in}}%
\pgfpathlineto{\pgfqpoint{4.513837in}{3.210757in}}%
\pgfpathlineto{\pgfqpoint{4.479756in}{3.243406in}}%
\pgfpathclose%
\pgfusepath{fill}%
\end{pgfscope}%
\begin{pgfscope}%
\pgfpathrectangle{\pgfqpoint{1.020000in}{0.880000in}}{\pgfqpoint{6.160000in}{6.160000in}}%
\pgfusepath{clip}%
\pgfsetbuttcap%
\pgfsetroundjoin%
\definecolor{currentfill}{rgb}{0.597777,0.727330,0.999777}%
\pgfsetfillcolor{currentfill}%
\pgfsetlinewidth{0.000000pt}%
\definecolor{currentstroke}{rgb}{0.000000,0.000000,0.000000}%
\pgfsetstrokecolor{currentstroke}%
\pgfsetdash{}{0pt}%
\pgfpathmoveto{\pgfqpoint{2.980240in}{3.120852in}}%
\pgfpathlineto{\pgfqpoint{2.989829in}{3.104169in}}%
\pgfpathlineto{\pgfqpoint{2.999441in}{3.087024in}}%
\pgfpathlineto{\pgfqpoint{3.033887in}{3.113264in}}%
\pgfpathlineto{\pgfqpoint{3.068310in}{3.141048in}}%
\pgfpathlineto{\pgfqpoint{3.058650in}{3.156897in}}%
\pgfpathlineto{\pgfqpoint{3.049013in}{3.172128in}}%
\pgfpathlineto{\pgfqpoint{3.014638in}{3.145757in}}%
\pgfpathlineto{\pgfqpoint{2.980240in}{3.120852in}}%
\pgfpathclose%
\pgfusepath{fill}%
\end{pgfscope}%
\begin{pgfscope}%
\pgfpathrectangle{\pgfqpoint{1.020000in}{0.880000in}}{\pgfqpoint{6.160000in}{6.160000in}}%
\pgfusepath{clip}%
\pgfsetbuttcap%
\pgfsetroundjoin%
\definecolor{currentfill}{rgb}{0.358415,0.478426,0.896795}%
\pgfsetfillcolor{currentfill}%
\pgfsetlinewidth{0.000000pt}%
\definecolor{currentstroke}{rgb}{0.000000,0.000000,0.000000}%
\pgfsetstrokecolor{currentstroke}%
\pgfsetdash{}{0pt}%
\pgfpathmoveto{\pgfqpoint{4.976802in}{2.729930in}}%
\pgfpathlineto{\pgfqpoint{4.988206in}{2.706302in}}%
\pgfpathlineto{\pgfqpoint{4.999630in}{2.682631in}}%
\pgfpathlineto{\pgfqpoint{5.033598in}{2.671284in}}%
\pgfpathlineto{\pgfqpoint{5.067544in}{2.661019in}}%
\pgfpathlineto{\pgfqpoint{5.056057in}{2.682971in}}%
\pgfpathlineto{\pgfqpoint{5.044590in}{2.704903in}}%
\pgfpathlineto{\pgfqpoint{5.010707in}{2.716810in}}%
\pgfpathlineto{\pgfqpoint{4.976802in}{2.729930in}}%
\pgfpathclose%
\pgfusepath{fill}%
\end{pgfscope}%
\begin{pgfscope}%
\pgfpathrectangle{\pgfqpoint{1.020000in}{0.880000in}}{\pgfqpoint{6.160000in}{6.160000in}}%
\pgfusepath{clip}%
\pgfsetbuttcap%
\pgfsetroundjoin%
\definecolor{currentfill}{rgb}{0.871493,0.862309,0.857016}%
\pgfsetfillcolor{currentfill}%
\pgfsetlinewidth{0.000000pt}%
\definecolor{currentstroke}{rgb}{0.000000,0.000000,0.000000}%
\pgfsetstrokecolor{currentstroke}%
\pgfsetdash{}{0pt}%
\pgfpathmoveto{\pgfqpoint{3.982620in}{3.712417in}}%
\pgfpathlineto{\pgfqpoint{3.993135in}{3.702954in}}%
\pgfpathlineto{\pgfqpoint{4.003676in}{3.690040in}}%
\pgfpathlineto{\pgfqpoint{4.038095in}{3.677753in}}%
\pgfpathlineto{\pgfqpoint{4.072487in}{3.662029in}}%
\pgfpathlineto{\pgfqpoint{4.061889in}{3.674940in}}%
\pgfpathlineto{\pgfqpoint{4.051315in}{3.684507in}}%
\pgfpathlineto{\pgfqpoint{4.016981in}{3.700149in}}%
\pgfpathlineto{\pgfqpoint{3.982620in}{3.712417in}}%
\pgfpathclose%
\pgfusepath{fill}%
\end{pgfscope}%
\begin{pgfscope}%
\pgfpathrectangle{\pgfqpoint{1.020000in}{0.880000in}}{\pgfqpoint{6.160000in}{6.160000in}}%
\pgfusepath{clip}%
\pgfsetbuttcap%
\pgfsetroundjoin%
\definecolor{currentfill}{rgb}{0.318832,0.426605,0.859857}%
\pgfsetfillcolor{currentfill}%
\pgfsetlinewidth{0.000000pt}%
\definecolor{currentstroke}{rgb}{0.000000,0.000000,0.000000}%
\pgfsetstrokecolor{currentstroke}%
\pgfsetdash{}{0pt}%
\pgfpathmoveto{\pgfqpoint{5.135375in}{2.643642in}}%
\pgfpathlineto{\pgfqpoint{5.146947in}{2.623193in}}%
\pgfpathlineto{\pgfqpoint{5.158540in}{2.602825in}}%
\pgfpathlineto{\pgfqpoint{5.192490in}{2.596923in}}%
\pgfpathlineto{\pgfqpoint{5.226422in}{2.591853in}}%
\pgfpathlineto{\pgfqpoint{5.214764in}{2.611002in}}%
\pgfpathlineto{\pgfqpoint{5.203129in}{2.630228in}}%
\pgfpathlineto{\pgfqpoint{5.169261in}{2.636462in}}%
\pgfpathlineto{\pgfqpoint{5.135375in}{2.643642in}}%
\pgfpathclose%
\pgfusepath{fill}%
\end{pgfscope}%
\begin{pgfscope}%
\pgfpathrectangle{\pgfqpoint{1.020000in}{0.880000in}}{\pgfqpoint{6.160000in}{6.160000in}}%
\pgfusepath{clip}%
\pgfsetbuttcap%
\pgfsetroundjoin%
\definecolor{currentfill}{rgb}{0.570616,0.704109,0.997195}%
\pgfsetfillcolor{currentfill}%
\pgfsetlinewidth{0.000000pt}%
\definecolor{currentstroke}{rgb}{0.000000,0.000000,0.000000}%
\pgfsetstrokecolor{currentstroke}%
\pgfsetdash{}{0pt}%
\pgfpathmoveto{\pgfqpoint{2.911368in}{3.075646in}}%
\pgfpathlineto{\pgfqpoint{2.920910in}{3.057680in}}%
\pgfpathlineto{\pgfqpoint{2.930473in}{3.039386in}}%
\pgfpathlineto{\pgfqpoint{2.964970in}{3.062387in}}%
\pgfpathlineto{\pgfqpoint{2.999441in}{3.087024in}}%
\pgfpathlineto{\pgfqpoint{2.989829in}{3.104169in}}%
\pgfpathlineto{\pgfqpoint{2.980240in}{3.120852in}}%
\pgfpathlineto{\pgfqpoint{2.945818in}{3.097471in}}%
\pgfpathlineto{\pgfqpoint{2.911368in}{3.075646in}}%
\pgfpathclose%
\pgfusepath{fill}%
\end{pgfscope}%
\begin{pgfscope}%
\pgfpathrectangle{\pgfqpoint{1.020000in}{0.880000in}}{\pgfqpoint{6.160000in}{6.160000in}}%
\pgfusepath{clip}%
\pgfsetbuttcap%
\pgfsetroundjoin%
\definecolor{currentfill}{rgb}{0.414801,0.546874,0.939088}%
\pgfsetfillcolor{currentfill}%
\pgfsetlinewidth{0.000000pt}%
\definecolor{currentstroke}{rgb}{0.000000,0.000000,0.000000}%
\pgfsetstrokecolor{currentstroke}%
\pgfsetdash{}{0pt}%
\pgfpathmoveto{\pgfqpoint{4.818430in}{2.849385in}}%
\pgfpathlineto{\pgfqpoint{4.829677in}{2.822108in}}%
\pgfpathlineto{\pgfqpoint{4.840941in}{2.794512in}}%
\pgfpathlineto{\pgfqpoint{4.874945in}{2.776594in}}%
\pgfpathlineto{\pgfqpoint{4.908922in}{2.759835in}}%
\pgfpathlineto{\pgfqpoint{4.897598in}{2.785265in}}%
\pgfpathlineto{\pgfqpoint{4.886292in}{2.810446in}}%
\pgfpathlineto{\pgfqpoint{4.852375in}{2.829276in}}%
\pgfpathlineto{\pgfqpoint{4.818430in}{2.849385in}}%
\pgfpathclose%
\pgfusepath{fill}%
\end{pgfscope}%
\begin{pgfscope}%
\pgfpathrectangle{\pgfqpoint{1.020000in}{0.880000in}}{\pgfqpoint{6.160000in}{6.160000in}}%
\pgfusepath{clip}%
\pgfsetbuttcap%
\pgfsetroundjoin%
\definecolor{currentfill}{rgb}{0.887752,0.854040,0.834671}%
\pgfsetfillcolor{currentfill}%
\pgfsetlinewidth{0.000000pt}%
\definecolor{currentstroke}{rgb}{0.000000,0.000000,0.000000}%
\pgfsetstrokecolor{currentstroke}%
\pgfsetdash{}{0pt}%
\pgfpathmoveto{\pgfqpoint{3.755524in}{3.719753in}}%
\pgfpathlineto{\pgfqpoint{3.765791in}{3.715942in}}%
\pgfpathlineto{\pgfqpoint{3.776091in}{3.709072in}}%
\pgfpathlineto{\pgfqpoint{3.810537in}{3.718934in}}%
\pgfpathlineto{\pgfqpoint{3.844978in}{3.725113in}}%
\pgfpathlineto{\pgfqpoint{3.834615in}{3.731522in}}%
\pgfpathlineto{\pgfqpoint{3.824284in}{3.734798in}}%
\pgfpathlineto{\pgfqpoint{3.789907in}{3.729055in}}%
\pgfpathlineto{\pgfqpoint{3.755524in}{3.719753in}}%
\pgfpathclose%
\pgfusepath{fill}%
\end{pgfscope}%
\begin{pgfscope}%
\pgfpathrectangle{\pgfqpoint{1.020000in}{0.880000in}}{\pgfqpoint{6.160000in}{6.160000in}}%
\pgfusepath{clip}%
\pgfsetbuttcap%
\pgfsetroundjoin%
\definecolor{currentfill}{rgb}{0.275827,0.366717,0.812553}%
\pgfsetfillcolor{currentfill}%
\pgfsetlinewidth{0.000000pt}%
\definecolor{currentstroke}{rgb}{0.000000,0.000000,0.000000}%
\pgfsetstrokecolor{currentstroke}%
\pgfsetdash{}{0pt}%
\pgfpathmoveto{\pgfqpoint{5.521104in}{2.543305in}}%
\pgfpathlineto{\pgfqpoint{5.533082in}{2.527362in}}%
\pgfpathlineto{\pgfqpoint{5.545084in}{2.511510in}}%
\pgfpathlineto{\pgfqpoint{5.578975in}{2.512548in}}%
\pgfpathlineto{\pgfqpoint{5.612847in}{2.513845in}}%
\pgfpathlineto{\pgfqpoint{5.600784in}{2.529328in}}%
\pgfpathlineto{\pgfqpoint{5.588745in}{2.544892in}}%
\pgfpathlineto{\pgfqpoint{5.554934in}{2.543947in}}%
\pgfpathlineto{\pgfqpoint{5.521104in}{2.543305in}}%
\pgfpathclose%
\pgfusepath{fill}%
\end{pgfscope}%
\begin{pgfscope}%
\pgfpathrectangle{\pgfqpoint{1.020000in}{0.880000in}}{\pgfqpoint{6.160000in}{6.160000in}}%
\pgfusepath{clip}%
\pgfsetbuttcap%
\pgfsetroundjoin%
\definecolor{currentfill}{rgb}{0.559747,0.694768,0.996075}%
\pgfsetfillcolor{currentfill}%
\pgfsetlinewidth{0.000000pt}%
\definecolor{currentstroke}{rgb}{0.000000,0.000000,0.000000}%
\pgfsetstrokecolor{currentstroke}%
\pgfsetdash{}{0pt}%
\pgfpathmoveto{\pgfqpoint{4.569895in}{3.121427in}}%
\pgfpathlineto{\pgfqpoint{4.580923in}{3.090765in}}%
\pgfpathlineto{\pgfqpoint{4.591965in}{3.058819in}}%
\pgfpathlineto{\pgfqpoint{4.626059in}{3.030675in}}%
\pgfpathlineto{\pgfqpoint{4.660115in}{3.003139in}}%
\pgfpathlineto{\pgfqpoint{4.649023in}{3.032860in}}%
\pgfpathlineto{\pgfqpoint{4.637944in}{3.061486in}}%
\pgfpathlineto{\pgfqpoint{4.603939in}{3.091131in}}%
\pgfpathlineto{\pgfqpoint{4.569895in}{3.121427in}}%
\pgfpathclose%
\pgfusepath{fill}%
\end{pgfscope}%
\begin{pgfscope}%
\pgfpathrectangle{\pgfqpoint{1.020000in}{0.880000in}}{\pgfqpoint{6.160000in}{6.160000in}}%
\pgfusepath{clip}%
\pgfsetbuttcap%
\pgfsetroundjoin%
\definecolor{currentfill}{rgb}{0.266381,0.353304,0.801637}%
\pgfsetfillcolor{currentfill}%
\pgfsetlinewidth{0.000000pt}%
\definecolor{currentstroke}{rgb}{0.000000,0.000000,0.000000}%
\pgfsetstrokecolor{currentstroke}%
\pgfsetdash{}{0pt}%
\pgfpathmoveto{\pgfqpoint{5.680533in}{2.517100in}}%
\pgfpathlineto{\pgfqpoint{5.692680in}{2.501980in}}%
\pgfpathlineto{\pgfqpoint{5.726524in}{2.504002in}}%
\pgfpathlineto{\pgfqpoint{5.760348in}{2.506167in}}%
\pgfpathlineto{\pgfqpoint{5.748141in}{2.521068in}}%
\pgfpathlineto{\pgfqpoint{5.714346in}{2.519006in}}%
\pgfpathlineto{\pgfqpoint{5.680533in}{2.517100in}}%
\pgfpathclose%
\pgfusepath{fill}%
\end{pgfscope}%
\begin{pgfscope}%
\pgfpathrectangle{\pgfqpoint{1.020000in}{0.880000in}}{\pgfqpoint{6.160000in}{6.160000in}}%
\pgfusepath{clip}%
\pgfsetbuttcap%
\pgfsetroundjoin%
\definecolor{currentfill}{rgb}{0.782049,0.842864,0.942980}%
\pgfsetfillcolor{currentfill}%
\pgfsetlinewidth{0.000000pt}%
\definecolor{currentstroke}{rgb}{0.000000,0.000000,0.000000}%
\pgfsetstrokecolor{currentstroke}%
\pgfsetdash{}{0pt}%
\pgfpathmoveto{\pgfqpoint{4.231183in}{3.533821in}}%
\pgfpathlineto{\pgfqpoint{4.241945in}{3.511397in}}%
\pgfpathlineto{\pgfqpoint{4.252723in}{3.485926in}}%
\pgfpathlineto{\pgfqpoint{4.287028in}{3.457513in}}%
\pgfpathlineto{\pgfqpoint{4.321289in}{3.427354in}}%
\pgfpathlineto{\pgfqpoint{4.310465in}{3.452063in}}%
\pgfpathlineto{\pgfqpoint{4.299655in}{3.473975in}}%
\pgfpathlineto{\pgfqpoint{4.265441in}{3.504783in}}%
\pgfpathlineto{\pgfqpoint{4.231183in}{3.533821in}}%
\pgfpathclose%
\pgfusepath{fill}%
\end{pgfscope}%
\begin{pgfscope}%
\pgfpathrectangle{\pgfqpoint{1.020000in}{0.880000in}}{\pgfqpoint{6.160000in}{6.160000in}}%
\pgfusepath{clip}%
\pgfsetbuttcap%
\pgfsetroundjoin%
\definecolor{currentfill}{rgb}{0.294718,0.393542,0.834384}%
\pgfsetfillcolor{currentfill}%
\pgfsetlinewidth{0.000000pt}%
\definecolor{currentstroke}{rgb}{0.000000,0.000000,0.000000}%
\pgfsetstrokecolor{currentstroke}%
\pgfsetdash{}{0pt}%
\pgfpathmoveto{\pgfqpoint{5.294229in}{2.584025in}}%
\pgfpathlineto{\pgfqpoint{5.305973in}{2.565972in}}%
\pgfpathlineto{\pgfqpoint{5.317740in}{2.548033in}}%
\pgfpathlineto{\pgfqpoint{5.351680in}{2.546014in}}%
\pgfpathlineto{\pgfqpoint{5.385601in}{2.544545in}}%
\pgfpathlineto{\pgfqpoint{5.373771in}{2.561697in}}%
\pgfpathlineto{\pgfqpoint{5.361964in}{2.578950in}}%
\pgfpathlineto{\pgfqpoint{5.328106in}{2.581168in}}%
\pgfpathlineto{\pgfqpoint{5.294229in}{2.584025in}}%
\pgfpathclose%
\pgfusepath{fill}%
\end{pgfscope}%
\begin{pgfscope}%
\pgfpathrectangle{\pgfqpoint{1.020000in}{0.880000in}}{\pgfqpoint{6.160000in}{6.160000in}}%
\pgfusepath{clip}%
\pgfsetbuttcap%
\pgfsetroundjoin%
\definecolor{currentfill}{rgb}{0.548876,0.685104,0.994379}%
\pgfsetfillcolor{currentfill}%
\pgfsetlinewidth{0.000000pt}%
\definecolor{currentstroke}{rgb}{0.000000,0.000000,0.000000}%
\pgfsetstrokecolor{currentstroke}%
\pgfsetdash{}{0pt}%
\pgfpathmoveto{\pgfqpoint{2.842381in}{3.036699in}}%
\pgfpathlineto{\pgfqpoint{2.851876in}{3.017618in}}%
\pgfpathlineto{\pgfqpoint{2.861390in}{2.998322in}}%
\pgfpathlineto{\pgfqpoint{2.895946in}{3.018034in}}%
\pgfpathlineto{\pgfqpoint{2.930473in}{3.039386in}}%
\pgfpathlineto{\pgfqpoint{2.920910in}{3.057680in}}%
\pgfpathlineto{\pgfqpoint{2.911368in}{3.075646in}}%
\pgfpathlineto{\pgfqpoint{2.876890in}{3.055390in}}%
\pgfpathlineto{\pgfqpoint{2.842381in}{3.036699in}}%
\pgfpathclose%
\pgfusepath{fill}%
\end{pgfscope}%
\begin{pgfscope}%
\pgfpathrectangle{\pgfqpoint{1.020000in}{0.880000in}}{\pgfqpoint{6.160000in}{6.160000in}}%
\pgfusepath{clip}%
\pgfsetbuttcap%
\pgfsetroundjoin%
\definecolor{currentfill}{rgb}{0.879622,0.858175,0.845844}%
\pgfsetfillcolor{currentfill}%
\pgfsetlinewidth{0.000000pt}%
\definecolor{currentstroke}{rgb}{0.000000,0.000000,0.000000}%
\pgfsetstrokecolor{currentstroke}%
\pgfsetdash{}{0pt}%
\pgfpathmoveto{\pgfqpoint{3.686754in}{3.690939in}}%
\pgfpathlineto{\pgfqpoint{3.696957in}{3.686333in}}%
\pgfpathlineto{\pgfqpoint{3.707193in}{3.678794in}}%
\pgfpathlineto{\pgfqpoint{3.741642in}{3.695641in}}%
\pgfpathlineto{\pgfqpoint{3.776091in}{3.709072in}}%
\pgfpathlineto{\pgfqpoint{3.765791in}{3.715942in}}%
\pgfpathlineto{\pgfqpoint{3.755524in}{3.719753in}}%
\pgfpathlineto{\pgfqpoint{3.721139in}{3.706998in}}%
\pgfpathlineto{\pgfqpoint{3.686754in}{3.690939in}}%
\pgfpathclose%
\pgfusepath{fill}%
\end{pgfscope}%
\begin{pgfscope}%
\pgfpathrectangle{\pgfqpoint{1.020000in}{0.880000in}}{\pgfqpoint{6.160000in}{6.160000in}}%
\pgfusepath{clip}%
\pgfsetbuttcap%
\pgfsetroundjoin%
\definecolor{currentfill}{rgb}{0.500031,0.638508,0.981070}%
\pgfsetfillcolor{currentfill}%
\pgfsetlinewidth{0.000000pt}%
\definecolor{currentstroke}{rgb}{0.000000,0.000000,0.000000}%
\pgfsetstrokecolor{currentstroke}%
\pgfsetdash{}{0pt}%
\pgfpathmoveto{\pgfqpoint{4.660115in}{3.003139in}}%
\pgfpathlineto{\pgfqpoint{4.671221in}{2.972470in}}%
\pgfpathlineto{\pgfqpoint{4.682341in}{2.941003in}}%
\pgfpathlineto{\pgfqpoint{4.716413in}{2.916638in}}%
\pgfpathlineto{\pgfqpoint{4.750450in}{2.893163in}}%
\pgfpathlineto{\pgfqpoint{4.739276in}{2.922230in}}%
\pgfpathlineto{\pgfqpoint{4.728117in}{2.950633in}}%
\pgfpathlineto{\pgfqpoint{4.694133in}{2.976403in}}%
\pgfpathlineto{\pgfqpoint{4.660115in}{3.003139in}}%
\pgfpathclose%
\pgfusepath{fill}%
\end{pgfscope}%
\begin{pgfscope}%
\pgfpathrectangle{\pgfqpoint{1.020000in}{0.880000in}}{\pgfqpoint{6.160000in}{6.160000in}}%
\pgfusepath{clip}%
\pgfsetbuttcap%
\pgfsetroundjoin%
\definecolor{currentfill}{rgb}{0.527132,0.664700,0.989065}%
\pgfsetfillcolor{currentfill}%
\pgfsetlinewidth{0.000000pt}%
\definecolor{currentstroke}{rgb}{0.000000,0.000000,0.000000}%
\pgfsetstrokecolor{currentstroke}%
\pgfsetdash{}{0pt}%
\pgfpathmoveto{\pgfqpoint{2.773270in}{3.003897in}}%
\pgfpathlineto{\pgfqpoint{2.782717in}{2.983860in}}%
\pgfpathlineto{\pgfqpoint{2.792183in}{2.963699in}}%
\pgfpathlineto{\pgfqpoint{2.826803in}{2.980224in}}%
\pgfpathlineto{\pgfqpoint{2.861390in}{2.998322in}}%
\pgfpathlineto{\pgfqpoint{2.851876in}{3.017618in}}%
\pgfpathlineto{\pgfqpoint{2.842381in}{3.036699in}}%
\pgfpathlineto{\pgfqpoint{2.807842in}{3.019547in}}%
\pgfpathlineto{\pgfqpoint{2.773270in}{3.003897in}}%
\pgfpathclose%
\pgfusepath{fill}%
\end{pgfscope}%
\begin{pgfscope}%
\pgfpathrectangle{\pgfqpoint{1.020000in}{0.880000in}}{\pgfqpoint{6.160000in}{6.160000in}}%
\pgfusepath{clip}%
\pgfsetbuttcap%
\pgfsetroundjoin%
\definecolor{currentfill}{rgb}{0.728970,0.817464,0.973188}%
\pgfsetfillcolor{currentfill}%
\pgfsetlinewidth{0.000000pt}%
\definecolor{currentstroke}{rgb}{0.000000,0.000000,0.000000}%
\pgfsetstrokecolor{currentstroke}%
\pgfsetdash{}{0pt}%
\pgfpathmoveto{\pgfqpoint{4.321289in}{3.427354in}}%
\pgfpathlineto{\pgfqpoint{4.332129in}{3.399928in}}%
\pgfpathlineto{\pgfqpoint{4.342982in}{3.369899in}}%
\pgfpathlineto{\pgfqpoint{4.377243in}{3.339518in}}%
\pgfpathlineto{\pgfqpoint{4.411459in}{3.308095in}}%
\pgfpathlineto{\pgfqpoint{4.400561in}{3.336791in}}%
\pgfpathlineto{\pgfqpoint{4.389676in}{3.363140in}}%
\pgfpathlineto{\pgfqpoint{4.355506in}{3.395784in}}%
\pgfpathlineto{\pgfqpoint{4.321289in}{3.427354in}}%
\pgfpathclose%
\pgfusepath{fill}%
\end{pgfscope}%
\begin{pgfscope}%
\pgfpathrectangle{\pgfqpoint{1.020000in}{0.880000in}}{\pgfqpoint{6.160000in}{6.160000in}}%
\pgfusepath{clip}%
\pgfsetbuttcap%
\pgfsetroundjoin%
\definecolor{currentfill}{rgb}{0.280550,0.373423,0.818011}%
\pgfsetfillcolor{currentfill}%
\pgfsetlinewidth{0.000000pt}%
\definecolor{currentstroke}{rgb}{0.000000,0.000000,0.000000}%
\pgfsetstrokecolor{currentstroke}%
\pgfsetdash{}{0pt}%
\pgfpathmoveto{\pgfqpoint{5.453390in}{2.543083in}}%
\pgfpathlineto{\pgfqpoint{5.465306in}{2.526656in}}%
\pgfpathlineto{\pgfqpoint{5.477246in}{2.510335in}}%
\pgfpathlineto{\pgfqpoint{5.511175in}{2.510761in}}%
\pgfpathlineto{\pgfqpoint{5.545084in}{2.511510in}}%
\pgfpathlineto{\pgfqpoint{5.533082in}{2.527362in}}%
\pgfpathlineto{\pgfqpoint{5.521104in}{2.543305in}}%
\pgfpathlineto{\pgfqpoint{5.487256in}{2.543004in}}%
\pgfpathlineto{\pgfqpoint{5.453390in}{2.543083in}}%
\pgfpathclose%
\pgfusepath{fill}%
\end{pgfscope}%
\begin{pgfscope}%
\pgfpathrectangle{\pgfqpoint{1.020000in}{0.880000in}}{\pgfqpoint{6.160000in}{6.160000in}}%
\pgfusepath{clip}%
\pgfsetbuttcap%
\pgfsetroundjoin%
\definecolor{currentfill}{rgb}{0.883687,0.856108,0.840258}%
\pgfsetfillcolor{currentfill}%
\pgfsetlinewidth{0.000000pt}%
\definecolor{currentstroke}{rgb}{0.000000,0.000000,0.000000}%
\pgfsetstrokecolor{currentstroke}%
\pgfsetdash{}{0pt}%
\pgfpathmoveto{\pgfqpoint{3.913831in}{3.726214in}}%
\pgfpathlineto{\pgfqpoint{3.924286in}{3.716695in}}%
\pgfpathlineto{\pgfqpoint{3.934770in}{3.703676in}}%
\pgfpathlineto{\pgfqpoint{3.969233in}{3.698722in}}%
\pgfpathlineto{\pgfqpoint{4.003676in}{3.690040in}}%
\pgfpathlineto{\pgfqpoint{3.993135in}{3.702954in}}%
\pgfpathlineto{\pgfqpoint{3.982620in}{3.712417in}}%
\pgfpathlineto{\pgfqpoint{3.948235in}{3.721145in}}%
\pgfpathlineto{\pgfqpoint{3.913831in}{3.726214in}}%
\pgfpathclose%
\pgfusepath{fill}%
\end{pgfscope}%
\begin{pgfscope}%
\pgfpathrectangle{\pgfqpoint{1.020000in}{0.880000in}}{\pgfqpoint{6.160000in}{6.160000in}}%
\pgfusepath{clip}%
\pgfsetbuttcap%
\pgfsetroundjoin%
\definecolor{currentfill}{rgb}{0.328604,0.439712,0.869587}%
\pgfsetfillcolor{currentfill}%
\pgfsetlinewidth{0.000000pt}%
\definecolor{currentstroke}{rgb}{0.000000,0.000000,0.000000}%
\pgfsetstrokecolor{currentstroke}%
\pgfsetdash{}{0pt}%
\pgfpathmoveto{\pgfqpoint{5.067544in}{2.661019in}}%
\pgfpathlineto{\pgfqpoint{5.079052in}{2.639102in}}%
\pgfpathlineto{\pgfqpoint{5.090582in}{2.617273in}}%
\pgfpathlineto{\pgfqpoint{5.124571in}{2.609596in}}%
\pgfpathlineto{\pgfqpoint{5.158540in}{2.602825in}}%
\pgfpathlineto{\pgfqpoint{5.146947in}{2.623193in}}%
\pgfpathlineto{\pgfqpoint{5.135375in}{2.643642in}}%
\pgfpathlineto{\pgfqpoint{5.101469in}{2.651815in}}%
\pgfpathlineto{\pgfqpoint{5.067544in}{2.661019in}}%
\pgfpathclose%
\pgfusepath{fill}%
\end{pgfscope}%
\begin{pgfscope}%
\pgfpathrectangle{\pgfqpoint{1.020000in}{0.880000in}}{\pgfqpoint{6.160000in}{6.160000in}}%
\pgfusepath{clip}%
\pgfsetbuttcap%
\pgfsetroundjoin%
\definecolor{currentfill}{rgb}{0.378598,0.503856,0.913692}%
\pgfsetfillcolor{currentfill}%
\pgfsetlinewidth{0.000000pt}%
\definecolor{currentstroke}{rgb}{0.000000,0.000000,0.000000}%
\pgfsetstrokecolor{currentstroke}%
\pgfsetdash{}{0pt}%
\pgfpathmoveto{\pgfqpoint{4.908922in}{2.759835in}}%
\pgfpathlineto{\pgfqpoint{4.920265in}{2.734248in}}%
\pgfpathlineto{\pgfqpoint{4.931627in}{2.708592in}}%
\pgfpathlineto{\pgfqpoint{4.965640in}{2.695068in}}%
\pgfpathlineto{\pgfqpoint{4.999630in}{2.682631in}}%
\pgfpathlineto{\pgfqpoint{4.988206in}{2.706302in}}%
\pgfpathlineto{\pgfqpoint{4.976802in}{2.729930in}}%
\pgfpathlineto{\pgfqpoint{4.942874in}{2.744273in}}%
\pgfpathlineto{\pgfqpoint{4.908922in}{2.759835in}}%
\pgfpathclose%
\pgfusepath{fill}%
\end{pgfscope}%
\begin{pgfscope}%
\pgfpathrectangle{\pgfqpoint{1.020000in}{0.880000in}}{\pgfqpoint{6.160000in}{6.160000in}}%
\pgfusepath{clip}%
\pgfsetbuttcap%
\pgfsetroundjoin%
\definecolor{currentfill}{rgb}{0.851372,0.863125,0.881064}%
\pgfsetfillcolor{currentfill}%
\pgfsetlinewidth{0.000000pt}%
\definecolor{currentstroke}{rgb}{0.000000,0.000000,0.000000}%
\pgfsetstrokecolor{currentstroke}%
\pgfsetdash{}{0pt}%
\pgfpathmoveto{\pgfqpoint{4.072487in}{3.662029in}}%
\pgfpathlineto{\pgfqpoint{4.083108in}{3.645680in}}%
\pgfpathlineto{\pgfqpoint{4.093752in}{3.625850in}}%
\pgfpathlineto{\pgfqpoint{4.128165in}{3.607056in}}%
\pgfpathlineto{\pgfqpoint{4.162542in}{3.585267in}}%
\pgfpathlineto{\pgfqpoint{4.151848in}{3.604853in}}%
\pgfpathlineto{\pgfqpoint{4.141174in}{3.621127in}}%
\pgfpathlineto{\pgfqpoint{4.106847in}{3.643072in}}%
\pgfpathlineto{\pgfqpoint{4.072487in}{3.662029in}}%
\pgfpathclose%
\pgfusepath{fill}%
\end{pgfscope}%
\begin{pgfscope}%
\pgfpathrectangle{\pgfqpoint{1.020000in}{0.880000in}}{\pgfqpoint{6.160000in}{6.160000in}}%
\pgfusepath{clip}%
\pgfsetbuttcap%
\pgfsetroundjoin%
\definecolor{currentfill}{rgb}{0.271104,0.360011,0.807095}%
\pgfsetfillcolor{currentfill}%
\pgfsetlinewidth{0.000000pt}%
\definecolor{currentstroke}{rgb}{0.000000,0.000000,0.000000}%
\pgfsetstrokecolor{currentstroke}%
\pgfsetdash{}{0pt}%
\pgfpathmoveto{\pgfqpoint{5.612847in}{2.513845in}}%
\pgfpathlineto{\pgfqpoint{5.624934in}{2.498446in}}%
\pgfpathlineto{\pgfqpoint{5.658817in}{2.500120in}}%
\pgfpathlineto{\pgfqpoint{5.692680in}{2.501980in}}%
\pgfpathlineto{\pgfqpoint{5.680533in}{2.517100in}}%
\pgfpathlineto{\pgfqpoint{5.646699in}{2.515371in}}%
\pgfpathlineto{\pgfqpoint{5.612847in}{2.513845in}}%
\pgfpathclose%
\pgfusepath{fill}%
\end{pgfscope}%
\begin{pgfscope}%
\pgfpathrectangle{\pgfqpoint{1.020000in}{0.880000in}}{\pgfqpoint{6.160000in}{6.160000in}}%
\pgfusepath{clip}%
\pgfsetbuttcap%
\pgfsetroundjoin%
\definecolor{currentfill}{rgb}{0.867428,0.864377,0.862602}%
\pgfsetfillcolor{currentfill}%
\pgfsetlinewidth{0.000000pt}%
\definecolor{currentstroke}{rgb}{0.000000,0.000000,0.000000}%
\pgfsetstrokecolor{currentstroke}%
\pgfsetdash{}{0pt}%
\pgfpathmoveto{\pgfqpoint{3.617994in}{3.649690in}}%
\pgfpathlineto{\pgfqpoint{3.628132in}{3.644061in}}%
\pgfpathlineto{\pgfqpoint{3.638306in}{3.635670in}}%
\pgfpathlineto{\pgfqpoint{3.672747in}{3.658727in}}%
\pgfpathlineto{\pgfqpoint{3.707193in}{3.678794in}}%
\pgfpathlineto{\pgfqpoint{3.696957in}{3.686333in}}%
\pgfpathlineto{\pgfqpoint{3.686754in}{3.690939in}}%
\pgfpathlineto{\pgfqpoint{3.652372in}{3.671762in}}%
\pgfpathlineto{\pgfqpoint{3.617994in}{3.649690in}}%
\pgfpathclose%
\pgfusepath{fill}%
\end{pgfscope}%
\begin{pgfscope}%
\pgfpathrectangle{\pgfqpoint{1.020000in}{0.880000in}}{\pgfqpoint{6.160000in}{6.160000in}}%
\pgfusepath{clip}%
\pgfsetbuttcap%
\pgfsetroundjoin%
\definecolor{currentfill}{rgb}{0.299441,0.400248,0.839842}%
\pgfsetfillcolor{currentfill}%
\pgfsetlinewidth{0.000000pt}%
\definecolor{currentstroke}{rgb}{0.000000,0.000000,0.000000}%
\pgfsetstrokecolor{currentstroke}%
\pgfsetdash{}{0pt}%
\pgfpathmoveto{\pgfqpoint{5.226422in}{2.591853in}}%
\pgfpathlineto{\pgfqpoint{5.238102in}{2.572811in}}%
\pgfpathlineto{\pgfqpoint{5.249806in}{2.553900in}}%
\pgfpathlineto{\pgfqpoint{5.283782in}{2.550648in}}%
\pgfpathlineto{\pgfqpoint{5.317740in}{2.548033in}}%
\pgfpathlineto{\pgfqpoint{5.305973in}{2.565972in}}%
\pgfpathlineto{\pgfqpoint{5.294229in}{2.584025in}}%
\pgfpathlineto{\pgfqpoint{5.260334in}{2.587570in}}%
\pgfpathlineto{\pgfqpoint{5.226422in}{2.591853in}}%
\pgfpathclose%
\pgfusepath{fill}%
\end{pgfscope}%
\begin{pgfscope}%
\pgfpathrectangle{\pgfqpoint{1.020000in}{0.880000in}}{\pgfqpoint{6.160000in}{6.160000in}}%
\pgfusepath{clip}%
\pgfsetbuttcap%
\pgfsetroundjoin%
\definecolor{currentfill}{rgb}{0.516260,0.654498,0.986407}%
\pgfsetfillcolor{currentfill}%
\pgfsetlinewidth{0.000000pt}%
\definecolor{currentstroke}{rgb}{0.000000,0.000000,0.000000}%
\pgfsetstrokecolor{currentstroke}%
\pgfsetdash{}{0pt}%
\pgfpathmoveto{\pgfqpoint{2.704026in}{2.976891in}}%
\pgfpathlineto{\pgfqpoint{2.713426in}{2.956041in}}%
\pgfpathlineto{\pgfqpoint{2.722843in}{2.935140in}}%
\pgfpathlineto{\pgfqpoint{2.757530in}{2.948692in}}%
\pgfpathlineto{\pgfqpoint{2.792183in}{2.963699in}}%
\pgfpathlineto{\pgfqpoint{2.782717in}{2.983860in}}%
\pgfpathlineto{\pgfqpoint{2.773270in}{3.003897in}}%
\pgfpathlineto{\pgfqpoint{2.738665in}{2.989698in}}%
\pgfpathlineto{\pgfqpoint{2.704026in}{2.976891in}}%
\pgfpathclose%
\pgfusepath{fill}%
\end{pgfscope}%
\begin{pgfscope}%
\pgfpathrectangle{\pgfqpoint{1.020000in}{0.880000in}}{\pgfqpoint{6.160000in}{6.160000in}}%
\pgfusepath{clip}%
\pgfsetbuttcap%
\pgfsetroundjoin%
\definecolor{currentfill}{rgb}{0.667253,0.779176,0.992959}%
\pgfsetfillcolor{currentfill}%
\pgfsetlinewidth{0.000000pt}%
\definecolor{currentstroke}{rgb}{0.000000,0.000000,0.000000}%
\pgfsetstrokecolor{currentstroke}%
\pgfsetdash{}{0pt}%
\pgfpathmoveto{\pgfqpoint{4.411459in}{3.308095in}}%
\pgfpathlineto{\pgfqpoint{4.422370in}{3.277185in}}%
\pgfpathlineto{\pgfqpoint{4.433294in}{3.244222in}}%
\pgfpathlineto{\pgfqpoint{4.467509in}{3.213821in}}%
\pgfpathlineto{\pgfqpoint{4.501681in}{3.183033in}}%
\pgfpathlineto{\pgfqpoint{4.490712in}{3.214129in}}%
\pgfpathlineto{\pgfqpoint{4.479756in}{3.243406in}}%
\pgfpathlineto{\pgfqpoint{4.445630in}{3.275952in}}%
\pgfpathlineto{\pgfqpoint{4.411459in}{3.308095in}}%
\pgfpathclose%
\pgfusepath{fill}%
\end{pgfscope}%
\begin{pgfscope}%
\pgfpathrectangle{\pgfqpoint{1.020000in}{0.880000in}}{\pgfqpoint{6.160000in}{6.160000in}}%
\pgfusepath{clip}%
\pgfsetbuttcap%
\pgfsetroundjoin%
\definecolor{currentfill}{rgb}{0.446431,0.582356,0.957373}%
\pgfsetfillcolor{currentfill}%
\pgfsetlinewidth{0.000000pt}%
\definecolor{currentstroke}{rgb}{0.000000,0.000000,0.000000}%
\pgfsetstrokecolor{currentstroke}%
\pgfsetdash{}{0pt}%
\pgfpathmoveto{\pgfqpoint{4.750450in}{2.893163in}}%
\pgfpathlineto{\pgfqpoint{4.761640in}{2.863566in}}%
\pgfpathlineto{\pgfqpoint{4.772847in}{2.833572in}}%
\pgfpathlineto{\pgfqpoint{4.806909in}{2.813530in}}%
\pgfpathlineto{\pgfqpoint{4.840941in}{2.794512in}}%
\pgfpathlineto{\pgfqpoint{4.829677in}{2.822108in}}%
\pgfpathlineto{\pgfqpoint{4.818430in}{2.849385in}}%
\pgfpathlineto{\pgfqpoint{4.784455in}{2.870710in}}%
\pgfpathlineto{\pgfqpoint{4.750450in}{2.893163in}}%
\pgfpathclose%
\pgfusepath{fill}%
\end{pgfscope}%
\begin{pgfscope}%
\pgfpathrectangle{\pgfqpoint{1.020000in}{0.880000in}}{\pgfqpoint{6.160000in}{6.160000in}}%
\pgfusepath{clip}%
\pgfsetbuttcap%
\pgfsetroundjoin%
\definecolor{currentfill}{rgb}{0.847365,0.862472,0.885540}%
\pgfsetfillcolor{currentfill}%
\pgfsetlinewidth{0.000000pt}%
\definecolor{currentstroke}{rgb}{0.000000,0.000000,0.000000}%
\pgfsetstrokecolor{currentstroke}%
\pgfsetdash{}{0pt}%
\pgfpathmoveto{\pgfqpoint{3.549255in}{3.597897in}}%
\pgfpathlineto{\pgfqpoint{3.559331in}{3.591057in}}%
\pgfpathlineto{\pgfqpoint{3.569443in}{3.581663in}}%
\pgfpathlineto{\pgfqpoint{3.603871in}{3.609886in}}%
\pgfpathlineto{\pgfqpoint{3.638306in}{3.635670in}}%
\pgfpathlineto{\pgfqpoint{3.628132in}{3.644061in}}%
\pgfpathlineto{\pgfqpoint{3.617994in}{3.649690in}}%
\pgfpathlineto{\pgfqpoint{3.583621in}{3.624975in}}%
\pgfpathlineto{\pgfqpoint{3.549255in}{3.597897in}}%
\pgfpathclose%
\pgfusepath{fill}%
\end{pgfscope}%
\begin{pgfscope}%
\pgfpathrectangle{\pgfqpoint{1.020000in}{0.880000in}}{\pgfqpoint{6.160000in}{6.160000in}}%
\pgfusepath{clip}%
\pgfsetbuttcap%
\pgfsetroundjoin%
\definecolor{currentfill}{rgb}{0.826784,0.858205,0.906953}%
\pgfsetfillcolor{currentfill}%
\pgfsetlinewidth{0.000000pt}%
\definecolor{currentstroke}{rgb}{0.000000,0.000000,0.000000}%
\pgfsetstrokecolor{currentstroke}%
\pgfsetdash{}{0pt}%
\pgfpathmoveto{\pgfqpoint{3.480543in}{3.537855in}}%
\pgfpathlineto{\pgfqpoint{3.490559in}{3.529665in}}%
\pgfpathlineto{\pgfqpoint{3.500610in}{3.519157in}}%
\pgfpathlineto{\pgfqpoint{3.535023in}{3.551312in}}%
\pgfpathlineto{\pgfqpoint{3.569443in}{3.581663in}}%
\pgfpathlineto{\pgfqpoint{3.559331in}{3.591057in}}%
\pgfpathlineto{\pgfqpoint{3.549255in}{3.597897in}}%
\pgfpathlineto{\pgfqpoint{3.514896in}{3.568753in}}%
\pgfpathlineto{\pgfqpoint{3.480543in}{3.537855in}}%
\pgfpathclose%
\pgfusepath{fill}%
\end{pgfscope}%
\begin{pgfscope}%
\pgfpathrectangle{\pgfqpoint{1.020000in}{0.880000in}}{\pgfqpoint{6.160000in}{6.160000in}}%
\pgfusepath{clip}%
\pgfsetbuttcap%
\pgfsetroundjoin%
\definecolor{currentfill}{rgb}{0.796064,0.848693,0.933471}%
\pgfsetfillcolor{currentfill}%
\pgfsetlinewidth{0.000000pt}%
\definecolor{currentstroke}{rgb}{0.000000,0.000000,0.000000}%
\pgfsetstrokecolor{currentstroke}%
\pgfsetdash{}{0pt}%
\pgfpathmoveto{\pgfqpoint{3.411856in}{3.472092in}}%
\pgfpathlineto{\pgfqpoint{3.421813in}{3.462462in}}%
\pgfpathlineto{\pgfqpoint{3.431804in}{3.450769in}}%
\pgfpathlineto{\pgfqpoint{3.466204in}{3.485530in}}%
\pgfpathlineto{\pgfqpoint{3.500610in}{3.519157in}}%
\pgfpathlineto{\pgfqpoint{3.490559in}{3.529665in}}%
\pgfpathlineto{\pgfqpoint{3.480543in}{3.537855in}}%
\pgfpathlineto{\pgfqpoint{3.446197in}{3.505527in}}%
\pgfpathlineto{\pgfqpoint{3.411856in}{3.472092in}}%
\pgfpathclose%
\pgfusepath{fill}%
\end{pgfscope}%
\begin{pgfscope}%
\pgfpathrectangle{\pgfqpoint{1.020000in}{0.880000in}}{\pgfqpoint{6.160000in}{6.160000in}}%
\pgfusepath{clip}%
\pgfsetbuttcap%
\pgfsetroundjoin%
\definecolor{currentfill}{rgb}{0.763363,0.835092,0.955658}%
\pgfsetfillcolor{currentfill}%
\pgfsetlinewidth{0.000000pt}%
\definecolor{currentstroke}{rgb}{0.000000,0.000000,0.000000}%
\pgfsetstrokecolor{currentstroke}%
\pgfsetdash{}{0pt}%
\pgfpathmoveto{\pgfqpoint{3.343185in}{3.403186in}}%
\pgfpathlineto{\pgfqpoint{3.353085in}{3.392077in}}%
\pgfpathlineto{\pgfqpoint{3.363018in}{3.379168in}}%
\pgfpathlineto{\pgfqpoint{3.397410in}{3.415206in}}%
\pgfpathlineto{\pgfqpoint{3.431804in}{3.450769in}}%
\pgfpathlineto{\pgfqpoint{3.421813in}{3.462462in}}%
\pgfpathlineto{\pgfqpoint{3.411856in}{3.472092in}}%
\pgfpathlineto{\pgfqpoint{3.377519in}{3.437874in}}%
\pgfpathlineto{\pgfqpoint{3.343185in}{3.403186in}}%
\pgfpathclose%
\pgfusepath{fill}%
\end{pgfscope}%
\begin{pgfscope}%
\pgfpathrectangle{\pgfqpoint{1.020000in}{0.880000in}}{\pgfqpoint{6.160000in}{6.160000in}}%
\pgfusepath{clip}%
\pgfsetbuttcap%
\pgfsetroundjoin%
\definecolor{currentfill}{rgb}{0.597777,0.727330,0.999777}%
\pgfsetfillcolor{currentfill}%
\pgfsetlinewidth{0.000000pt}%
\definecolor{currentstroke}{rgb}{0.000000,0.000000,0.000000}%
\pgfsetstrokecolor{currentstroke}%
\pgfsetdash{}{0pt}%
\pgfpathmoveto{\pgfqpoint{4.501681in}{3.183033in}}%
\pgfpathlineto{\pgfqpoint{4.512662in}{3.150280in}}%
\pgfpathlineto{\pgfqpoint{4.523656in}{3.116050in}}%
\pgfpathlineto{\pgfqpoint{4.557831in}{3.087355in}}%
\pgfpathlineto{\pgfqpoint{4.591965in}{3.058819in}}%
\pgfpathlineto{\pgfqpoint{4.580923in}{3.090765in}}%
\pgfpathlineto{\pgfqpoint{4.569895in}{3.121427in}}%
\pgfpathlineto{\pgfqpoint{4.535809in}{3.152145in}}%
\pgfpathlineto{\pgfqpoint{4.501681in}{3.183033in}}%
\pgfpathclose%
\pgfusepath{fill}%
\end{pgfscope}%
\begin{pgfscope}%
\pgfpathrectangle{\pgfqpoint{1.020000in}{0.880000in}}{\pgfqpoint{6.160000in}{6.160000in}}%
\pgfusepath{clip}%
\pgfsetbuttcap%
\pgfsetroundjoin%
\definecolor{currentfill}{rgb}{0.728970,0.817464,0.973188}%
\pgfsetfillcolor{currentfill}%
\pgfsetlinewidth{0.000000pt}%
\definecolor{currentstroke}{rgb}{0.000000,0.000000,0.000000}%
\pgfsetstrokecolor{currentstroke}%
\pgfsetdash{}{0pt}%
\pgfpathmoveto{\pgfqpoint{3.274514in}{3.333603in}}%
\pgfpathlineto{\pgfqpoint{3.284359in}{3.321022in}}%
\pgfpathlineto{\pgfqpoint{3.294235in}{3.306904in}}%
\pgfpathlineto{\pgfqpoint{3.328627in}{3.342969in}}%
\pgfpathlineto{\pgfqpoint{3.363018in}{3.379168in}}%
\pgfpathlineto{\pgfqpoint{3.353085in}{3.392077in}}%
\pgfpathlineto{\pgfqpoint{3.343185in}{3.403186in}}%
\pgfpathlineto{\pgfqpoint{3.308851in}{3.368334in}}%
\pgfpathlineto{\pgfqpoint{3.274514in}{3.333603in}}%
\pgfpathclose%
\pgfusepath{fill}%
\end{pgfscope}%
\begin{pgfscope}%
\pgfpathrectangle{\pgfqpoint{1.020000in}{0.880000in}}{\pgfqpoint{6.160000in}{6.160000in}}%
\pgfusepath{clip}%
\pgfsetbuttcap%
\pgfsetroundjoin%
\definecolor{currentfill}{rgb}{0.693321,0.796314,0.986308}%
\pgfsetfillcolor{currentfill}%
\pgfsetlinewidth{0.000000pt}%
\definecolor{currentstroke}{rgb}{0.000000,0.000000,0.000000}%
\pgfsetstrokecolor{currentstroke}%
\pgfsetdash{}{0pt}%
\pgfpathmoveto{\pgfqpoint{3.205826in}{3.265552in}}%
\pgfpathlineto{\pgfqpoint{3.215617in}{3.251549in}}%
\pgfpathlineto{\pgfqpoint{3.225437in}{3.236260in}}%
\pgfpathlineto{\pgfqpoint{3.259839in}{3.271250in}}%
\pgfpathlineto{\pgfqpoint{3.294235in}{3.306904in}}%
\pgfpathlineto{\pgfqpoint{3.284359in}{3.321022in}}%
\pgfpathlineto{\pgfqpoint{3.274514in}{3.333603in}}%
\pgfpathlineto{\pgfqpoint{3.240174in}{3.299260in}}%
\pgfpathlineto{\pgfqpoint{3.205826in}{3.265552in}}%
\pgfpathclose%
\pgfusepath{fill}%
\end{pgfscope}%
\begin{pgfscope}%
\pgfpathrectangle{\pgfqpoint{1.020000in}{0.880000in}}{\pgfqpoint{6.160000in}{6.160000in}}%
\pgfusepath{clip}%
\pgfsetbuttcap%
\pgfsetroundjoin%
\definecolor{currentfill}{rgb}{0.280550,0.373423,0.818011}%
\pgfsetfillcolor{currentfill}%
\pgfsetlinewidth{0.000000pt}%
\definecolor{currentstroke}{rgb}{0.000000,0.000000,0.000000}%
\pgfsetstrokecolor{currentstroke}%
\pgfsetdash{}{0pt}%
\pgfpathmoveto{\pgfqpoint{5.385601in}{2.544545in}}%
\pgfpathlineto{\pgfqpoint{5.397455in}{2.527505in}}%
\pgfpathlineto{\pgfqpoint{5.409334in}{2.510591in}}%
\pgfpathlineto{\pgfqpoint{5.443299in}{2.510266in}}%
\pgfpathlineto{\pgfqpoint{5.477246in}{2.510335in}}%
\pgfpathlineto{\pgfqpoint{5.465306in}{2.526656in}}%
\pgfpathlineto{\pgfqpoint{5.453390in}{2.543083in}}%
\pgfpathlineto{\pgfqpoint{5.419505in}{2.543582in}}%
\pgfpathlineto{\pgfqpoint{5.385601in}{2.544545in}}%
\pgfpathclose%
\pgfusepath{fill}%
\end{pgfscope}%
\begin{pgfscope}%
\pgfpathrectangle{\pgfqpoint{1.020000in}{0.880000in}}{\pgfqpoint{6.160000in}{6.160000in}}%
\pgfusepath{clip}%
\pgfsetbuttcap%
\pgfsetroundjoin%
\definecolor{currentfill}{rgb}{0.813693,0.854282,0.918480}%
\pgfsetfillcolor{currentfill}%
\pgfsetlinewidth{0.000000pt}%
\definecolor{currentstroke}{rgb}{0.000000,0.000000,0.000000}%
\pgfsetstrokecolor{currentstroke}%
\pgfsetdash{}{0pt}%
\pgfpathmoveto{\pgfqpoint{4.162542in}{3.585267in}}%
\pgfpathlineto{\pgfqpoint{4.173257in}{3.562372in}}%
\pgfpathlineto{\pgfqpoint{4.183990in}{3.536217in}}%
\pgfpathlineto{\pgfqpoint{4.218377in}{3.512265in}}%
\pgfpathlineto{\pgfqpoint{4.252723in}{3.485926in}}%
\pgfpathlineto{\pgfqpoint{4.241945in}{3.511397in}}%
\pgfpathlineto{\pgfqpoint{4.231183in}{3.533821in}}%
\pgfpathlineto{\pgfqpoint{4.196883in}{3.560755in}}%
\pgfpathlineto{\pgfqpoint{4.162542in}{3.585267in}}%
\pgfpathclose%
\pgfusepath{fill}%
\end{pgfscope}%
\begin{pgfscope}%
\pgfpathrectangle{\pgfqpoint{1.020000in}{0.880000in}}{\pgfqpoint{6.160000in}{6.160000in}}%
\pgfusepath{clip}%
\pgfsetbuttcap%
\pgfsetroundjoin%
\definecolor{currentfill}{rgb}{0.656683,0.771806,0.994914}%
\pgfsetfillcolor{currentfill}%
\pgfsetlinewidth{0.000000pt}%
\definecolor{currentstroke}{rgb}{0.000000,0.000000,0.000000}%
\pgfsetstrokecolor{currentstroke}%
\pgfsetdash{}{0pt}%
\pgfpathmoveto{\pgfqpoint{3.137098in}{3.200889in}}%
\pgfpathlineto{\pgfqpoint{3.146836in}{3.185544in}}%
\pgfpathlineto{\pgfqpoint{3.156602in}{3.169148in}}%
\pgfpathlineto{\pgfqpoint{3.191026in}{3.202160in}}%
\pgfpathlineto{\pgfqpoint{3.225437in}{3.236260in}}%
\pgfpathlineto{\pgfqpoint{3.215617in}{3.251549in}}%
\pgfpathlineto{\pgfqpoint{3.205826in}{3.265552in}}%
\pgfpathlineto{\pgfqpoint{3.171468in}{3.232697in}}%
\pgfpathlineto{\pgfqpoint{3.137098in}{3.200889in}}%
\pgfpathclose%
\pgfusepath{fill}%
\end{pgfscope}%
\begin{pgfscope}%
\pgfpathrectangle{\pgfqpoint{1.020000in}{0.880000in}}{\pgfqpoint{6.160000in}{6.160000in}}%
\pgfusepath{clip}%
\pgfsetbuttcap%
\pgfsetroundjoin%
\definecolor{currentfill}{rgb}{0.891817,0.851973,0.829085}%
\pgfsetfillcolor{currentfill}%
\pgfsetlinewidth{0.000000pt}%
\definecolor{currentstroke}{rgb}{0.000000,0.000000,0.000000}%
\pgfsetstrokecolor{currentstroke}%
\pgfsetdash{}{0pt}%
\pgfpathmoveto{\pgfqpoint{3.844978in}{3.725113in}}%
\pgfpathlineto{\pgfqpoint{3.855373in}{3.715371in}}%
\pgfpathlineto{\pgfqpoint{3.865798in}{3.702142in}}%
\pgfpathlineto{\pgfqpoint{3.900290in}{3.704826in}}%
\pgfpathlineto{\pgfqpoint{3.934770in}{3.703676in}}%
\pgfpathlineto{\pgfqpoint{3.924286in}{3.716695in}}%
\pgfpathlineto{\pgfqpoint{3.913831in}{3.726214in}}%
\pgfpathlineto{\pgfqpoint{3.879411in}{3.727546in}}%
\pgfpathlineto{\pgfqpoint{3.844978in}{3.725113in}}%
\pgfpathclose%
\pgfusepath{fill}%
\end{pgfscope}%
\begin{pgfscope}%
\pgfpathrectangle{\pgfqpoint{1.020000in}{0.880000in}}{\pgfqpoint{6.160000in}{6.160000in}}%
\pgfusepath{clip}%
\pgfsetbuttcap%
\pgfsetroundjoin%
\definecolor{currentfill}{rgb}{0.271104,0.360011,0.807095}%
\pgfsetfillcolor{currentfill}%
\pgfsetlinewidth{0.000000pt}%
\definecolor{currentstroke}{rgb}{0.000000,0.000000,0.000000}%
\pgfsetstrokecolor{currentstroke}%
\pgfsetdash{}{0pt}%
\pgfpathmoveto{\pgfqpoint{5.545084in}{2.511510in}}%
\pgfpathlineto{\pgfqpoint{5.557111in}{2.495755in}}%
\pgfpathlineto{\pgfqpoint{5.591032in}{2.496982in}}%
\pgfpathlineto{\pgfqpoint{5.624934in}{2.498446in}}%
\pgfpathlineto{\pgfqpoint{5.612847in}{2.513845in}}%
\pgfpathlineto{\pgfqpoint{5.578975in}{2.512548in}}%
\pgfpathlineto{\pgfqpoint{5.545084in}{2.511510in}}%
\pgfpathclose%
\pgfusepath{fill}%
\end{pgfscope}%
\begin{pgfscope}%
\pgfpathrectangle{\pgfqpoint{1.020000in}{0.880000in}}{\pgfqpoint{6.160000in}{6.160000in}}%
\pgfusepath{clip}%
\pgfsetbuttcap%
\pgfsetroundjoin%
\definecolor{currentfill}{rgb}{0.619318,0.744121,0.998931}%
\pgfsetfillcolor{currentfill}%
\pgfsetlinewidth{0.000000pt}%
\definecolor{currentstroke}{rgb}{0.000000,0.000000,0.000000}%
\pgfsetstrokecolor{currentstroke}%
\pgfsetdash{}{0pt}%
\pgfpathmoveto{\pgfqpoint{3.068310in}{3.141048in}}%
\pgfpathlineto{\pgfqpoint{3.077996in}{3.124464in}}%
\pgfpathlineto{\pgfqpoint{3.087709in}{3.107044in}}%
\pgfpathlineto{\pgfqpoint{3.122164in}{3.137395in}}%
\pgfpathlineto{\pgfqpoint{3.156602in}{3.169148in}}%
\pgfpathlineto{\pgfqpoint{3.146836in}{3.185544in}}%
\pgfpathlineto{\pgfqpoint{3.137098in}{3.200889in}}%
\pgfpathlineto{\pgfqpoint{3.102713in}{3.170293in}}%
\pgfpathlineto{\pgfqpoint{3.068310in}{3.141048in}}%
\pgfpathclose%
\pgfusepath{fill}%
\end{pgfscope}%
\begin{pgfscope}%
\pgfpathrectangle{\pgfqpoint{1.020000in}{0.880000in}}{\pgfqpoint{6.160000in}{6.160000in}}%
\pgfusepath{clip}%
\pgfsetbuttcap%
\pgfsetroundjoin%
\definecolor{currentfill}{rgb}{0.343278,0.459354,0.884122}%
\pgfsetfillcolor{currentfill}%
\pgfsetlinewidth{0.000000pt}%
\definecolor{currentstroke}{rgb}{0.000000,0.000000,0.000000}%
\pgfsetstrokecolor{currentstroke}%
\pgfsetdash{}{0pt}%
\pgfpathmoveto{\pgfqpoint{4.999630in}{2.682631in}}%
\pgfpathlineto{\pgfqpoint{5.011075in}{2.658987in}}%
\pgfpathlineto{\pgfqpoint{5.022541in}{2.635436in}}%
\pgfpathlineto{\pgfqpoint{5.056572in}{2.625880in}}%
\pgfpathlineto{\pgfqpoint{5.090582in}{2.617273in}}%
\pgfpathlineto{\pgfqpoint{5.079052in}{2.639102in}}%
\pgfpathlineto{\pgfqpoint{5.067544in}{2.661019in}}%
\pgfpathlineto{\pgfqpoint{5.033598in}{2.671284in}}%
\pgfpathlineto{\pgfqpoint{4.999630in}{2.682631in}}%
\pgfpathclose%
\pgfusepath{fill}%
\end{pgfscope}%
\begin{pgfscope}%
\pgfpathrectangle{\pgfqpoint{1.020000in}{0.880000in}}{\pgfqpoint{6.160000in}{6.160000in}}%
\pgfusepath{clip}%
\pgfsetbuttcap%
\pgfsetroundjoin%
\definecolor{currentfill}{rgb}{0.586921,0.718121,0.998874}%
\pgfsetfillcolor{currentfill}%
\pgfsetlinewidth{0.000000pt}%
\definecolor{currentstroke}{rgb}{0.000000,0.000000,0.000000}%
\pgfsetstrokecolor{currentstroke}%
\pgfsetdash{}{0pt}%
\pgfpathmoveto{\pgfqpoint{2.999441in}{3.087024in}}%
\pgfpathlineto{\pgfqpoint{3.009076in}{3.069319in}}%
\pgfpathlineto{\pgfqpoint{3.018736in}{3.050968in}}%
\pgfpathlineto{\pgfqpoint{3.053234in}{3.078206in}}%
\pgfpathlineto{\pgfqpoint{3.087709in}{3.107044in}}%
\pgfpathlineto{\pgfqpoint{3.077996in}{3.124464in}}%
\pgfpathlineto{\pgfqpoint{3.068310in}{3.141048in}}%
\pgfpathlineto{\pgfqpoint{3.033887in}{3.113264in}}%
\pgfpathlineto{\pgfqpoint{2.999441in}{3.087024in}}%
\pgfpathclose%
\pgfusepath{fill}%
\end{pgfscope}%
\begin{pgfscope}%
\pgfpathrectangle{\pgfqpoint{1.020000in}{0.880000in}}{\pgfqpoint{6.160000in}{6.160000in}}%
\pgfusepath{clip}%
\pgfsetbuttcap%
\pgfsetroundjoin%
\definecolor{currentfill}{rgb}{0.399231,0.528528,0.928459}%
\pgfsetfillcolor{currentfill}%
\pgfsetlinewidth{0.000000pt}%
\definecolor{currentstroke}{rgb}{0.000000,0.000000,0.000000}%
\pgfsetstrokecolor{currentstroke}%
\pgfsetdash{}{0pt}%
\pgfpathmoveto{\pgfqpoint{4.840941in}{2.794512in}}%
\pgfpathlineto{\pgfqpoint{4.852223in}{2.766709in}}%
\pgfpathlineto{\pgfqpoint{4.863524in}{2.738807in}}%
\pgfpathlineto{\pgfqpoint{4.897588in}{2.723183in}}%
\pgfpathlineto{\pgfqpoint{4.931627in}{2.708592in}}%
\pgfpathlineto{\pgfqpoint{4.920265in}{2.734248in}}%
\pgfpathlineto{\pgfqpoint{4.908922in}{2.759835in}}%
\pgfpathlineto{\pgfqpoint{4.874945in}{2.776594in}}%
\pgfpathlineto{\pgfqpoint{4.840941in}{2.794512in}}%
\pgfpathclose%
\pgfusepath{fill}%
\end{pgfscope}%
\begin{pgfscope}%
\pgfpathrectangle{\pgfqpoint{1.020000in}{0.880000in}}{\pgfqpoint{6.160000in}{6.160000in}}%
\pgfusepath{clip}%
\pgfsetbuttcap%
\pgfsetroundjoin%
\definecolor{currentfill}{rgb}{0.532568,0.669801,0.990393}%
\pgfsetfillcolor{currentfill}%
\pgfsetlinewidth{0.000000pt}%
\definecolor{currentstroke}{rgb}{0.000000,0.000000,0.000000}%
\pgfsetstrokecolor{currentstroke}%
\pgfsetdash{}{0pt}%
\pgfpathmoveto{\pgfqpoint{4.591965in}{3.058819in}}%
\pgfpathlineto{\pgfqpoint{4.603020in}{3.025758in}}%
\pgfpathlineto{\pgfqpoint{4.614089in}{2.991761in}}%
\pgfpathlineto{\pgfqpoint{4.648234in}{2.966103in}}%
\pgfpathlineto{\pgfqpoint{4.682341in}{2.941003in}}%
\pgfpathlineto{\pgfqpoint{4.671221in}{2.972470in}}%
\pgfpathlineto{\pgfqpoint{4.660115in}{3.003139in}}%
\pgfpathlineto{\pgfqpoint{4.626059in}{3.030675in}}%
\pgfpathlineto{\pgfqpoint{4.591965in}{3.058819in}}%
\pgfpathclose%
\pgfusepath{fill}%
\end{pgfscope}%
\begin{pgfscope}%
\pgfpathrectangle{\pgfqpoint{1.020000in}{0.880000in}}{\pgfqpoint{6.160000in}{6.160000in}}%
\pgfusepath{clip}%
\pgfsetbuttcap%
\pgfsetroundjoin%
\definecolor{currentfill}{rgb}{0.309060,0.413498,0.850128}%
\pgfsetfillcolor{currentfill}%
\pgfsetlinewidth{0.000000pt}%
\definecolor{currentstroke}{rgb}{0.000000,0.000000,0.000000}%
\pgfsetstrokecolor{currentstroke}%
\pgfsetdash{}{0pt}%
\pgfpathmoveto{\pgfqpoint{5.158540in}{2.602825in}}%
\pgfpathlineto{\pgfqpoint{5.170157in}{2.582576in}}%
\pgfpathlineto{\pgfqpoint{5.181796in}{2.562481in}}%
\pgfpathlineto{\pgfqpoint{5.215810in}{2.557832in}}%
\pgfpathlineto{\pgfqpoint{5.249806in}{2.553900in}}%
\pgfpathlineto{\pgfqpoint{5.238102in}{2.572811in}}%
\pgfpathlineto{\pgfqpoint{5.226422in}{2.591853in}}%
\pgfpathlineto{\pgfqpoint{5.192490in}{2.596923in}}%
\pgfpathlineto{\pgfqpoint{5.158540in}{2.602825in}}%
\pgfpathclose%
\pgfusepath{fill}%
\end{pgfscope}%
\begin{pgfscope}%
\pgfpathrectangle{\pgfqpoint{1.020000in}{0.880000in}}{\pgfqpoint{6.160000in}{6.160000in}}%
\pgfusepath{clip}%
\pgfsetbuttcap%
\pgfsetroundjoin%
\definecolor{currentfill}{rgb}{0.867428,0.864377,0.862602}%
\pgfsetfillcolor{currentfill}%
\pgfsetlinewidth{0.000000pt}%
\definecolor{currentstroke}{rgb}{0.000000,0.000000,0.000000}%
\pgfsetstrokecolor{currentstroke}%
\pgfsetdash{}{0pt}%
\pgfpathmoveto{\pgfqpoint{4.003676in}{3.690040in}}%
\pgfpathlineto{\pgfqpoint{4.014244in}{3.673577in}}%
\pgfpathlineto{\pgfqpoint{4.024836in}{3.653518in}}%
\pgfpathlineto{\pgfqpoint{4.059308in}{3.641406in}}%
\pgfpathlineto{\pgfqpoint{4.093752in}{3.625850in}}%
\pgfpathlineto{\pgfqpoint{4.083108in}{3.645680in}}%
\pgfpathlineto{\pgfqpoint{4.072487in}{3.662029in}}%
\pgfpathlineto{\pgfqpoint{4.038095in}{3.677753in}}%
\pgfpathlineto{\pgfqpoint{4.003676in}{3.690040in}}%
\pgfpathclose%
\pgfusepath{fill}%
\end{pgfscope}%
\begin{pgfscope}%
\pgfpathrectangle{\pgfqpoint{1.020000in}{0.880000in}}{\pgfqpoint{6.160000in}{6.160000in}}%
\pgfusepath{clip}%
\pgfsetbuttcap%
\pgfsetroundjoin%
\definecolor{currentfill}{rgb}{0.559747,0.694768,0.996075}%
\pgfsetfillcolor{currentfill}%
\pgfsetlinewidth{0.000000pt}%
\definecolor{currentstroke}{rgb}{0.000000,0.000000,0.000000}%
\pgfsetstrokecolor{currentstroke}%
\pgfsetdash{}{0pt}%
\pgfpathmoveto{\pgfqpoint{2.930473in}{3.039386in}}%
\pgfpathlineto{\pgfqpoint{2.940057in}{3.020683in}}%
\pgfpathlineto{\pgfqpoint{2.949665in}{3.001500in}}%
\pgfpathlineto{\pgfqpoint{2.984214in}{3.025388in}}%
\pgfpathlineto{\pgfqpoint{3.018736in}{3.050968in}}%
\pgfpathlineto{\pgfqpoint{3.009076in}{3.069319in}}%
\pgfpathlineto{\pgfqpoint{2.999441in}{3.087024in}}%
\pgfpathlineto{\pgfqpoint{2.964970in}{3.062387in}}%
\pgfpathlineto{\pgfqpoint{2.930473in}{3.039386in}}%
\pgfpathclose%
\pgfusepath{fill}%
\end{pgfscope}%
\begin{pgfscope}%
\pgfpathrectangle{\pgfqpoint{1.020000in}{0.880000in}}{\pgfqpoint{6.160000in}{6.160000in}}%
\pgfusepath{clip}%
\pgfsetbuttcap%
\pgfsetroundjoin%
\definecolor{currentfill}{rgb}{0.768034,0.837035,0.952488}%
\pgfsetfillcolor{currentfill}%
\pgfsetlinewidth{0.000000pt}%
\definecolor{currentstroke}{rgb}{0.000000,0.000000,0.000000}%
\pgfsetstrokecolor{currentstroke}%
\pgfsetdash{}{0pt}%
\pgfpathmoveto{\pgfqpoint{4.252723in}{3.485926in}}%
\pgfpathlineto{\pgfqpoint{4.263518in}{3.457496in}}%
\pgfpathlineto{\pgfqpoint{4.274329in}{3.426232in}}%
\pgfpathlineto{\pgfqpoint{4.308677in}{3.398912in}}%
\pgfpathlineto{\pgfqpoint{4.342982in}{3.369899in}}%
\pgfpathlineto{\pgfqpoint{4.332129in}{3.399928in}}%
\pgfpathlineto{\pgfqpoint{4.321289in}{3.427354in}}%
\pgfpathlineto{\pgfqpoint{4.287028in}{3.457513in}}%
\pgfpathlineto{\pgfqpoint{4.252723in}{3.485926in}}%
\pgfpathclose%
\pgfusepath{fill}%
\end{pgfscope}%
\begin{pgfscope}%
\pgfpathrectangle{\pgfqpoint{1.020000in}{0.880000in}}{\pgfqpoint{6.160000in}{6.160000in}}%
\pgfusepath{clip}%
\pgfsetbuttcap%
\pgfsetroundjoin%
\definecolor{currentfill}{rgb}{0.891817,0.851973,0.829085}%
\pgfsetfillcolor{currentfill}%
\pgfsetlinewidth{0.000000pt}%
\definecolor{currentstroke}{rgb}{0.000000,0.000000,0.000000}%
\pgfsetstrokecolor{currentstroke}%
\pgfsetdash{}{0pt}%
\pgfpathmoveto{\pgfqpoint{3.776091in}{3.709072in}}%
\pgfpathlineto{\pgfqpoint{3.786423in}{3.698942in}}%
\pgfpathlineto{\pgfqpoint{3.796788in}{3.685400in}}%
\pgfpathlineto{\pgfqpoint{3.831296in}{3.695644in}}%
\pgfpathlineto{\pgfqpoint{3.865798in}{3.702142in}}%
\pgfpathlineto{\pgfqpoint{3.855373in}{3.715371in}}%
\pgfpathlineto{\pgfqpoint{3.844978in}{3.725113in}}%
\pgfpathlineto{\pgfqpoint{3.810537in}{3.718934in}}%
\pgfpathlineto{\pgfqpoint{3.776091in}{3.709072in}}%
\pgfpathclose%
\pgfusepath{fill}%
\end{pgfscope}%
\begin{pgfscope}%
\pgfpathrectangle{\pgfqpoint{1.020000in}{0.880000in}}{\pgfqpoint{6.160000in}{6.160000in}}%
\pgfusepath{clip}%
\pgfsetbuttcap%
\pgfsetroundjoin%
\definecolor{currentfill}{rgb}{0.289996,0.386836,0.828926}%
\pgfsetfillcolor{currentfill}%
\pgfsetlinewidth{0.000000pt}%
\definecolor{currentstroke}{rgb}{0.000000,0.000000,0.000000}%
\pgfsetstrokecolor{currentstroke}%
\pgfsetdash{}{0pt}%
\pgfpathmoveto{\pgfqpoint{5.317740in}{2.548033in}}%
\pgfpathlineto{\pgfqpoint{5.329531in}{2.530227in}}%
\pgfpathlineto{\pgfqpoint{5.341346in}{2.512568in}}%
\pgfpathlineto{\pgfqpoint{5.375349in}{2.511346in}}%
\pgfpathlineto{\pgfqpoint{5.409334in}{2.510591in}}%
\pgfpathlineto{\pgfqpoint{5.397455in}{2.527505in}}%
\pgfpathlineto{\pgfqpoint{5.385601in}{2.544545in}}%
\pgfpathlineto{\pgfqpoint{5.351680in}{2.546014in}}%
\pgfpathlineto{\pgfqpoint{5.317740in}{2.548033in}}%
\pgfpathclose%
\pgfusepath{fill}%
\end{pgfscope}%
\begin{pgfscope}%
\pgfpathrectangle{\pgfqpoint{1.020000in}{0.880000in}}{\pgfqpoint{6.160000in}{6.160000in}}%
\pgfusepath{clip}%
\pgfsetbuttcap%
\pgfsetroundjoin%
\definecolor{currentfill}{rgb}{0.538004,0.674902,0.991722}%
\pgfsetfillcolor{currentfill}%
\pgfsetlinewidth{0.000000pt}%
\definecolor{currentstroke}{rgb}{0.000000,0.000000,0.000000}%
\pgfsetstrokecolor{currentstroke}%
\pgfsetdash{}{0pt}%
\pgfpathmoveto{\pgfqpoint{2.861390in}{2.998322in}}%
\pgfpathlineto{\pgfqpoint{2.870925in}{2.978743in}}%
\pgfpathlineto{\pgfqpoint{2.880481in}{2.958825in}}%
\pgfpathlineto{\pgfqpoint{2.915088in}{2.979316in}}%
\pgfpathlineto{\pgfqpoint{2.949665in}{3.001500in}}%
\pgfpathlineto{\pgfqpoint{2.940057in}{3.020683in}}%
\pgfpathlineto{\pgfqpoint{2.930473in}{3.039386in}}%
\pgfpathlineto{\pgfqpoint{2.895946in}{3.018034in}}%
\pgfpathlineto{\pgfqpoint{2.861390in}{2.998322in}}%
\pgfpathclose%
\pgfusepath{fill}%
\end{pgfscope}%
\begin{pgfscope}%
\pgfpathrectangle{\pgfqpoint{1.020000in}{0.880000in}}{\pgfqpoint{6.160000in}{6.160000in}}%
\pgfusepath{clip}%
\pgfsetbuttcap%
\pgfsetroundjoin%
\definecolor{currentfill}{rgb}{0.275827,0.366717,0.812553}%
\pgfsetfillcolor{currentfill}%
\pgfsetlinewidth{0.000000pt}%
\definecolor{currentstroke}{rgb}{0.000000,0.000000,0.000000}%
\pgfsetstrokecolor{currentstroke}%
\pgfsetdash{}{0pt}%
\pgfpathmoveto{\pgfqpoint{5.477246in}{2.510335in}}%
\pgfpathlineto{\pgfqpoint{5.489211in}{2.494127in}}%
\pgfpathlineto{\pgfqpoint{5.523171in}{2.494793in}}%
\pgfpathlineto{\pgfqpoint{5.557111in}{2.495755in}}%
\pgfpathlineto{\pgfqpoint{5.545084in}{2.511510in}}%
\pgfpathlineto{\pgfqpoint{5.511175in}{2.510761in}}%
\pgfpathlineto{\pgfqpoint{5.477246in}{2.510335in}}%
\pgfpathclose%
\pgfusepath{fill}%
\end{pgfscope}%
\begin{pgfscope}%
\pgfpathrectangle{\pgfqpoint{1.020000in}{0.880000in}}{\pgfqpoint{6.160000in}{6.160000in}}%
\pgfusepath{clip}%
\pgfsetbuttcap%
\pgfsetroundjoin%
\definecolor{currentfill}{rgb}{0.473070,0.611077,0.970634}%
\pgfsetfillcolor{currentfill}%
\pgfsetlinewidth{0.000000pt}%
\definecolor{currentstroke}{rgb}{0.000000,0.000000,0.000000}%
\pgfsetstrokecolor{currentstroke}%
\pgfsetdash{}{0pt}%
\pgfpathmoveto{\pgfqpoint{4.682341in}{2.941003in}}%
\pgfpathlineto{\pgfqpoint{4.693476in}{2.908897in}}%
\pgfpathlineto{\pgfqpoint{4.704627in}{2.876311in}}%
\pgfpathlineto{\pgfqpoint{4.738753in}{2.854539in}}%
\pgfpathlineto{\pgfqpoint{4.772847in}{2.833572in}}%
\pgfpathlineto{\pgfqpoint{4.761640in}{2.863566in}}%
\pgfpathlineto{\pgfqpoint{4.750450in}{2.893163in}}%
\pgfpathlineto{\pgfqpoint{4.716413in}{2.916638in}}%
\pgfpathlineto{\pgfqpoint{4.682341in}{2.941003in}}%
\pgfpathclose%
\pgfusepath{fill}%
\end{pgfscope}%
\begin{pgfscope}%
\pgfpathrectangle{\pgfqpoint{1.020000in}{0.880000in}}{\pgfqpoint{6.160000in}{6.160000in}}%
\pgfusepath{clip}%
\pgfsetbuttcap%
\pgfsetroundjoin%
\definecolor{currentfill}{rgb}{0.708720,0.805721,0.981117}%
\pgfsetfillcolor{currentfill}%
\pgfsetlinewidth{0.000000pt}%
\definecolor{currentstroke}{rgb}{0.000000,0.000000,0.000000}%
\pgfsetstrokecolor{currentstroke}%
\pgfsetdash{}{0pt}%
\pgfpathmoveto{\pgfqpoint{4.342982in}{3.369899in}}%
\pgfpathlineto{\pgfqpoint{4.353850in}{3.337417in}}%
\pgfpathlineto{\pgfqpoint{4.364730in}{3.302656in}}%
\pgfpathlineto{\pgfqpoint{4.399034in}{3.273936in}}%
\pgfpathlineto{\pgfqpoint{4.433294in}{3.244222in}}%
\pgfpathlineto{\pgfqpoint{4.422370in}{3.277185in}}%
\pgfpathlineto{\pgfqpoint{4.411459in}{3.308095in}}%
\pgfpathlineto{\pgfqpoint{4.377243in}{3.339518in}}%
\pgfpathlineto{\pgfqpoint{4.342982in}{3.369899in}}%
\pgfpathclose%
\pgfusepath{fill}%
\end{pgfscope}%
\begin{pgfscope}%
\pgfpathrectangle{\pgfqpoint{1.020000in}{0.880000in}}{\pgfqpoint{6.160000in}{6.160000in}}%
\pgfusepath{clip}%
\pgfsetbuttcap%
\pgfsetroundjoin%
\definecolor{currentfill}{rgb}{0.516260,0.654498,0.986407}%
\pgfsetfillcolor{currentfill}%
\pgfsetlinewidth{0.000000pt}%
\definecolor{currentstroke}{rgb}{0.000000,0.000000,0.000000}%
\pgfsetstrokecolor{currentstroke}%
\pgfsetdash{}{0pt}%
\pgfpathmoveto{\pgfqpoint{2.792183in}{2.963699in}}%
\pgfpathlineto{\pgfqpoint{2.801667in}{2.943360in}}%
\pgfpathlineto{\pgfqpoint{2.811171in}{2.922799in}}%
\pgfpathlineto{\pgfqpoint{2.845842in}{2.940001in}}%
\pgfpathlineto{\pgfqpoint{2.880481in}{2.958825in}}%
\pgfpathlineto{\pgfqpoint{2.870925in}{2.978743in}}%
\pgfpathlineto{\pgfqpoint{2.861390in}{2.998322in}}%
\pgfpathlineto{\pgfqpoint{2.826803in}{2.980224in}}%
\pgfpathlineto{\pgfqpoint{2.792183in}{2.963699in}}%
\pgfpathclose%
\pgfusepath{fill}%
\end{pgfscope}%
\begin{pgfscope}%
\pgfpathrectangle{\pgfqpoint{1.020000in}{0.880000in}}{\pgfqpoint{6.160000in}{6.160000in}}%
\pgfusepath{clip}%
\pgfsetbuttcap%
\pgfsetroundjoin%
\definecolor{currentfill}{rgb}{0.363461,0.484784,0.901019}%
\pgfsetfillcolor{currentfill}%
\pgfsetlinewidth{0.000000pt}%
\definecolor{currentstroke}{rgb}{0.000000,0.000000,0.000000}%
\pgfsetstrokecolor{currentstroke}%
\pgfsetdash{}{0pt}%
\pgfpathmoveto{\pgfqpoint{4.931627in}{2.708592in}}%
\pgfpathlineto{\pgfqpoint{4.943008in}{2.682954in}}%
\pgfpathlineto{\pgfqpoint{4.954411in}{2.657415in}}%
\pgfpathlineto{\pgfqpoint{4.988487in}{2.645949in}}%
\pgfpathlineto{\pgfqpoint{5.022541in}{2.635436in}}%
\pgfpathlineto{\pgfqpoint{5.011075in}{2.658987in}}%
\pgfpathlineto{\pgfqpoint{4.999630in}{2.682631in}}%
\pgfpathlineto{\pgfqpoint{4.965640in}{2.695068in}}%
\pgfpathlineto{\pgfqpoint{4.931627in}{2.708592in}}%
\pgfpathclose%
\pgfusepath{fill}%
\end{pgfscope}%
\begin{pgfscope}%
\pgfpathrectangle{\pgfqpoint{1.020000in}{0.880000in}}{\pgfqpoint{6.160000in}{6.160000in}}%
\pgfusepath{clip}%
\pgfsetbuttcap%
\pgfsetroundjoin%
\definecolor{currentfill}{rgb}{0.883687,0.856108,0.840258}%
\pgfsetfillcolor{currentfill}%
\pgfsetlinewidth{0.000000pt}%
\definecolor{currentstroke}{rgb}{0.000000,0.000000,0.000000}%
\pgfsetstrokecolor{currentstroke}%
\pgfsetdash{}{0pt}%
\pgfpathmoveto{\pgfqpoint{3.707193in}{3.678794in}}%
\pgfpathlineto{\pgfqpoint{3.717464in}{3.668124in}}%
\pgfpathlineto{\pgfqpoint{3.727768in}{3.654172in}}%
\pgfpathlineto{\pgfqpoint{3.762278in}{3.671523in}}%
\pgfpathlineto{\pgfqpoint{3.796788in}{3.685400in}}%
\pgfpathlineto{\pgfqpoint{3.786423in}{3.698942in}}%
\pgfpathlineto{\pgfqpoint{3.776091in}{3.709072in}}%
\pgfpathlineto{\pgfqpoint{3.741642in}{3.695641in}}%
\pgfpathlineto{\pgfqpoint{3.707193in}{3.678794in}}%
\pgfpathclose%
\pgfusepath{fill}%
\end{pgfscope}%
\begin{pgfscope}%
\pgfpathrectangle{\pgfqpoint{1.020000in}{0.880000in}}{\pgfqpoint{6.160000in}{6.160000in}}%
\pgfusepath{clip}%
\pgfsetbuttcap%
\pgfsetroundjoin%
\definecolor{currentfill}{rgb}{0.839351,0.861167,0.894494}%
\pgfsetfillcolor{currentfill}%
\pgfsetlinewidth{0.000000pt}%
\definecolor{currentstroke}{rgb}{0.000000,0.000000,0.000000}%
\pgfsetstrokecolor{currentstroke}%
\pgfsetdash{}{0pt}%
\pgfpathmoveto{\pgfqpoint{4.093752in}{3.625850in}}%
\pgfpathlineto{\pgfqpoint{4.104418in}{3.602541in}}%
\pgfpathlineto{\pgfqpoint{4.115105in}{3.575803in}}%
\pgfpathlineto{\pgfqpoint{4.149565in}{3.557485in}}%
\pgfpathlineto{\pgfqpoint{4.183990in}{3.536217in}}%
\pgfpathlineto{\pgfqpoint{4.173257in}{3.562372in}}%
\pgfpathlineto{\pgfqpoint{4.162542in}{3.585267in}}%
\pgfpathlineto{\pgfqpoint{4.128165in}{3.607056in}}%
\pgfpathlineto{\pgfqpoint{4.093752in}{3.625850in}}%
\pgfpathclose%
\pgfusepath{fill}%
\end{pgfscope}%
\begin{pgfscope}%
\pgfpathrectangle{\pgfqpoint{1.020000in}{0.880000in}}{\pgfqpoint{6.160000in}{6.160000in}}%
\pgfusepath{clip}%
\pgfsetbuttcap%
\pgfsetroundjoin%
\definecolor{currentfill}{rgb}{0.318832,0.426605,0.859857}%
\pgfsetfillcolor{currentfill}%
\pgfsetlinewidth{0.000000pt}%
\definecolor{currentstroke}{rgb}{0.000000,0.000000,0.000000}%
\pgfsetstrokecolor{currentstroke}%
\pgfsetdash{}{0pt}%
\pgfpathmoveto{\pgfqpoint{5.090582in}{2.617273in}}%
\pgfpathlineto{\pgfqpoint{5.102134in}{2.595578in}}%
\pgfpathlineto{\pgfqpoint{5.113708in}{2.574062in}}%
\pgfpathlineto{\pgfqpoint{5.147762in}{2.567881in}}%
\pgfpathlineto{\pgfqpoint{5.181796in}{2.562481in}}%
\pgfpathlineto{\pgfqpoint{5.170157in}{2.582576in}}%
\pgfpathlineto{\pgfqpoint{5.158540in}{2.602825in}}%
\pgfpathlineto{\pgfqpoint{5.124571in}{2.609596in}}%
\pgfpathlineto{\pgfqpoint{5.090582in}{2.617273in}}%
\pgfpathclose%
\pgfusepath{fill}%
\end{pgfscope}%
\begin{pgfscope}%
\pgfpathrectangle{\pgfqpoint{1.020000in}{0.880000in}}{\pgfqpoint{6.160000in}{6.160000in}}%
\pgfusepath{clip}%
\pgfsetbuttcap%
\pgfsetroundjoin%
\definecolor{currentfill}{rgb}{0.640828,0.760752,0.997846}%
\pgfsetfillcolor{currentfill}%
\pgfsetlinewidth{0.000000pt}%
\definecolor{currentstroke}{rgb}{0.000000,0.000000,0.000000}%
\pgfsetstrokecolor{currentstroke}%
\pgfsetdash{}{0pt}%
\pgfpathmoveto{\pgfqpoint{4.433294in}{3.244222in}}%
\pgfpathlineto{\pgfqpoint{4.444230in}{3.209388in}}%
\pgfpathlineto{\pgfqpoint{4.455179in}{3.172885in}}%
\pgfpathlineto{\pgfqpoint{4.489439in}{3.144650in}}%
\pgfpathlineto{\pgfqpoint{4.523656in}{3.116050in}}%
\pgfpathlineto{\pgfqpoint{4.512662in}{3.150280in}}%
\pgfpathlineto{\pgfqpoint{4.501681in}{3.183033in}}%
\pgfpathlineto{\pgfqpoint{4.467509in}{3.213821in}}%
\pgfpathlineto{\pgfqpoint{4.433294in}{3.244222in}}%
\pgfpathclose%
\pgfusepath{fill}%
\end{pgfscope}%
\begin{pgfscope}%
\pgfpathrectangle{\pgfqpoint{1.020000in}{0.880000in}}{\pgfqpoint{6.160000in}{6.160000in}}%
\pgfusepath{clip}%
\pgfsetbuttcap%
\pgfsetroundjoin%
\definecolor{currentfill}{rgb}{0.879622,0.858175,0.845844}%
\pgfsetfillcolor{currentfill}%
\pgfsetlinewidth{0.000000pt}%
\definecolor{currentstroke}{rgb}{0.000000,0.000000,0.000000}%
\pgfsetstrokecolor{currentstroke}%
\pgfsetdash{}{0pt}%
\pgfpathmoveto{\pgfqpoint{3.934770in}{3.703676in}}%
\pgfpathlineto{\pgfqpoint{3.945282in}{3.687056in}}%
\pgfpathlineto{\pgfqpoint{3.955821in}{3.666785in}}%
\pgfpathlineto{\pgfqpoint{3.990339in}{3.662018in}}%
\pgfpathlineto{\pgfqpoint{4.024836in}{3.653518in}}%
\pgfpathlineto{\pgfqpoint{4.014244in}{3.673577in}}%
\pgfpathlineto{\pgfqpoint{4.003676in}{3.690040in}}%
\pgfpathlineto{\pgfqpoint{3.969233in}{3.698722in}}%
\pgfpathlineto{\pgfqpoint{3.934770in}{3.703676in}}%
\pgfpathclose%
\pgfusepath{fill}%
\end{pgfscope}%
\begin{pgfscope}%
\pgfpathrectangle{\pgfqpoint{1.020000in}{0.880000in}}{\pgfqpoint{6.160000in}{6.160000in}}%
\pgfusepath{clip}%
\pgfsetbuttcap%
\pgfsetroundjoin%
\definecolor{currentfill}{rgb}{0.425199,0.559058,0.946061}%
\pgfsetfillcolor{currentfill}%
\pgfsetlinewidth{0.000000pt}%
\definecolor{currentstroke}{rgb}{0.000000,0.000000,0.000000}%
\pgfsetstrokecolor{currentstroke}%
\pgfsetdash{}{0pt}%
\pgfpathmoveto{\pgfqpoint{4.772847in}{2.833572in}}%
\pgfpathlineto{\pgfqpoint{4.784070in}{2.803316in}}%
\pgfpathlineto{\pgfqpoint{4.795311in}{2.772929in}}%
\pgfpathlineto{\pgfqpoint{4.829432in}{2.755412in}}%
\pgfpathlineto{\pgfqpoint{4.863524in}{2.738807in}}%
\pgfpathlineto{\pgfqpoint{4.852223in}{2.766709in}}%
\pgfpathlineto{\pgfqpoint{4.840941in}{2.794512in}}%
\pgfpathlineto{\pgfqpoint{4.806909in}{2.813530in}}%
\pgfpathlineto{\pgfqpoint{4.772847in}{2.833572in}}%
\pgfpathclose%
\pgfusepath{fill}%
\end{pgfscope}%
\begin{pgfscope}%
\pgfpathrectangle{\pgfqpoint{1.020000in}{0.880000in}}{\pgfqpoint{6.160000in}{6.160000in}}%
\pgfusepath{clip}%
\pgfsetbuttcap%
\pgfsetroundjoin%
\definecolor{currentfill}{rgb}{0.280550,0.373423,0.818011}%
\pgfsetfillcolor{currentfill}%
\pgfsetlinewidth{0.000000pt}%
\definecolor{currentstroke}{rgb}{0.000000,0.000000,0.000000}%
\pgfsetstrokecolor{currentstroke}%
\pgfsetdash{}{0pt}%
\pgfpathmoveto{\pgfqpoint{5.409334in}{2.510591in}}%
\pgfpathlineto{\pgfqpoint{5.421236in}{2.493809in}}%
\pgfpathlineto{\pgfqpoint{5.455233in}{2.493788in}}%
\pgfpathlineto{\pgfqpoint{5.489211in}{2.494127in}}%
\pgfpathlineto{\pgfqpoint{5.477246in}{2.510335in}}%
\pgfpathlineto{\pgfqpoint{5.443299in}{2.510266in}}%
\pgfpathlineto{\pgfqpoint{5.409334in}{2.510591in}}%
\pgfpathclose%
\pgfusepath{fill}%
\end{pgfscope}%
\begin{pgfscope}%
\pgfpathrectangle{\pgfqpoint{1.020000in}{0.880000in}}{\pgfqpoint{6.160000in}{6.160000in}}%
\pgfusepath{clip}%
\pgfsetbuttcap%
\pgfsetroundjoin%
\definecolor{currentfill}{rgb}{0.500031,0.638508,0.981070}%
\pgfsetfillcolor{currentfill}%
\pgfsetlinewidth{0.000000pt}%
\definecolor{currentstroke}{rgb}{0.000000,0.000000,0.000000}%
\pgfsetstrokecolor{currentstroke}%
\pgfsetdash{}{0pt}%
\pgfpathmoveto{\pgfqpoint{2.722843in}{2.935140in}}%
\pgfpathlineto{\pgfqpoint{2.732277in}{2.914147in}}%
\pgfpathlineto{\pgfqpoint{2.741729in}{2.893026in}}%
\pgfpathlineto{\pgfqpoint{2.776467in}{2.907163in}}%
\pgfpathlineto{\pgfqpoint{2.811171in}{2.922799in}}%
\pgfpathlineto{\pgfqpoint{2.801667in}{2.943360in}}%
\pgfpathlineto{\pgfqpoint{2.792183in}{2.963699in}}%
\pgfpathlineto{\pgfqpoint{2.757530in}{2.948692in}}%
\pgfpathlineto{\pgfqpoint{2.722843in}{2.935140in}}%
\pgfpathclose%
\pgfusepath{fill}%
\end{pgfscope}%
\begin{pgfscope}%
\pgfpathrectangle{\pgfqpoint{1.020000in}{0.880000in}}{\pgfqpoint{6.160000in}{6.160000in}}%
\pgfusepath{clip}%
\pgfsetbuttcap%
\pgfsetroundjoin%
\definecolor{currentfill}{rgb}{0.294718,0.393542,0.834384}%
\pgfsetfillcolor{currentfill}%
\pgfsetlinewidth{0.000000pt}%
\definecolor{currentstroke}{rgb}{0.000000,0.000000,0.000000}%
\pgfsetstrokecolor{currentstroke}%
\pgfsetdash{}{0pt}%
\pgfpathmoveto{\pgfqpoint{5.249806in}{2.553900in}}%
\pgfpathlineto{\pgfqpoint{5.261533in}{2.535146in}}%
\pgfpathlineto{\pgfqpoint{5.273283in}{2.516566in}}%
\pgfpathlineto{\pgfqpoint{5.307324in}{2.514297in}}%
\pgfpathlineto{\pgfqpoint{5.341346in}{2.512568in}}%
\pgfpathlineto{\pgfqpoint{5.329531in}{2.530227in}}%
\pgfpathlineto{\pgfqpoint{5.317740in}{2.548033in}}%
\pgfpathlineto{\pgfqpoint{5.283782in}{2.550648in}}%
\pgfpathlineto{\pgfqpoint{5.249806in}{2.553900in}}%
\pgfpathclose%
\pgfusepath{fill}%
\end{pgfscope}%
\begin{pgfscope}%
\pgfpathrectangle{\pgfqpoint{1.020000in}{0.880000in}}{\pgfqpoint{6.160000in}{6.160000in}}%
\pgfusepath{clip}%
\pgfsetbuttcap%
\pgfsetroundjoin%
\definecolor{currentfill}{rgb}{0.867428,0.864377,0.862602}%
\pgfsetfillcolor{currentfill}%
\pgfsetlinewidth{0.000000pt}%
\definecolor{currentstroke}{rgb}{0.000000,0.000000,0.000000}%
\pgfsetstrokecolor{currentstroke}%
\pgfsetdash{}{0pt}%
\pgfpathmoveto{\pgfqpoint{3.638306in}{3.635670in}}%
\pgfpathlineto{\pgfqpoint{3.648515in}{3.624325in}}%
\pgfpathlineto{\pgfqpoint{3.658759in}{3.609881in}}%
\pgfpathlineto{\pgfqpoint{3.693261in}{3.633546in}}%
\pgfpathlineto{\pgfqpoint{3.727768in}{3.654172in}}%
\pgfpathlineto{\pgfqpoint{3.717464in}{3.668124in}}%
\pgfpathlineto{\pgfqpoint{3.707193in}{3.678794in}}%
\pgfpathlineto{\pgfqpoint{3.672747in}{3.658727in}}%
\pgfpathlineto{\pgfqpoint{3.638306in}{3.635670in}}%
\pgfpathclose%
\pgfusepath{fill}%
\end{pgfscope}%
\begin{pgfscope}%
\pgfpathrectangle{\pgfqpoint{1.020000in}{0.880000in}}{\pgfqpoint{6.160000in}{6.160000in}}%
\pgfusepath{clip}%
\pgfsetbuttcap%
\pgfsetroundjoin%
\definecolor{currentfill}{rgb}{0.570616,0.704109,0.997195}%
\pgfsetfillcolor{currentfill}%
\pgfsetlinewidth{0.000000pt}%
\definecolor{currentstroke}{rgb}{0.000000,0.000000,0.000000}%
\pgfsetstrokecolor{currentstroke}%
\pgfsetdash{}{0pt}%
\pgfpathmoveto{\pgfqpoint{4.523656in}{3.116050in}}%
\pgfpathlineto{\pgfqpoint{4.534663in}{3.080538in}}%
\pgfpathlineto{\pgfqpoint{4.545683in}{3.043945in}}%
\pgfpathlineto{\pgfqpoint{4.579906in}{3.017781in}}%
\pgfpathlineto{\pgfqpoint{4.614089in}{2.991761in}}%
\pgfpathlineto{\pgfqpoint{4.603020in}{3.025758in}}%
\pgfpathlineto{\pgfqpoint{4.591965in}{3.058819in}}%
\pgfpathlineto{\pgfqpoint{4.557831in}{3.087355in}}%
\pgfpathlineto{\pgfqpoint{4.523656in}{3.116050in}}%
\pgfpathclose%
\pgfusepath{fill}%
\end{pgfscope}%
\begin{pgfscope}%
\pgfpathrectangle{\pgfqpoint{1.020000in}{0.880000in}}{\pgfqpoint{6.160000in}{6.160000in}}%
\pgfusepath{clip}%
\pgfsetbuttcap%
\pgfsetroundjoin%
\definecolor{currentfill}{rgb}{0.847365,0.862472,0.885540}%
\pgfsetfillcolor{currentfill}%
\pgfsetlinewidth{0.000000pt}%
\definecolor{currentstroke}{rgb}{0.000000,0.000000,0.000000}%
\pgfsetstrokecolor{currentstroke}%
\pgfsetdash{}{0pt}%
\pgfpathmoveto{\pgfqpoint{3.569443in}{3.581663in}}%
\pgfpathlineto{\pgfqpoint{3.579590in}{3.569534in}}%
\pgfpathlineto{\pgfqpoint{3.589773in}{3.554530in}}%
\pgfpathlineto{\pgfqpoint{3.624262in}{3.583444in}}%
\pgfpathlineto{\pgfqpoint{3.658759in}{3.609881in}}%
\pgfpathlineto{\pgfqpoint{3.648515in}{3.624325in}}%
\pgfpathlineto{\pgfqpoint{3.638306in}{3.635670in}}%
\pgfpathlineto{\pgfqpoint{3.603871in}{3.609886in}}%
\pgfpathlineto{\pgfqpoint{3.569443in}{3.581663in}}%
\pgfpathclose%
\pgfusepath{fill}%
\end{pgfscope}%
\begin{pgfscope}%
\pgfpathrectangle{\pgfqpoint{1.020000in}{0.880000in}}{\pgfqpoint{6.160000in}{6.160000in}}%
\pgfusepath{clip}%
\pgfsetbuttcap%
\pgfsetroundjoin%
\definecolor{currentfill}{rgb}{0.800601,0.850358,0.930008}%
\pgfsetfillcolor{currentfill}%
\pgfsetlinewidth{0.000000pt}%
\definecolor{currentstroke}{rgb}{0.000000,0.000000,0.000000}%
\pgfsetstrokecolor{currentstroke}%
\pgfsetdash{}{0pt}%
\pgfpathmoveto{\pgfqpoint{4.183990in}{3.536217in}}%
\pgfpathlineto{\pgfqpoint{4.194740in}{3.506894in}}%
\pgfpathlineto{\pgfqpoint{4.205508in}{3.474536in}}%
\pgfpathlineto{\pgfqpoint{4.239938in}{3.451541in}}%
\pgfpathlineto{\pgfqpoint{4.274329in}{3.426232in}}%
\pgfpathlineto{\pgfqpoint{4.263518in}{3.457496in}}%
\pgfpathlineto{\pgfqpoint{4.252723in}{3.485926in}}%
\pgfpathlineto{\pgfqpoint{4.218377in}{3.512265in}}%
\pgfpathlineto{\pgfqpoint{4.183990in}{3.536217in}}%
\pgfpathclose%
\pgfusepath{fill}%
\end{pgfscope}%
\begin{pgfscope}%
\pgfpathrectangle{\pgfqpoint{1.020000in}{0.880000in}}{\pgfqpoint{6.160000in}{6.160000in}}%
\pgfusepath{clip}%
\pgfsetbuttcap%
\pgfsetroundjoin%
\definecolor{currentfill}{rgb}{0.822420,0.856898,0.910795}%
\pgfsetfillcolor{currentfill}%
\pgfsetlinewidth{0.000000pt}%
\definecolor{currentstroke}{rgb}{0.000000,0.000000,0.000000}%
\pgfsetstrokecolor{currentstroke}%
\pgfsetdash{}{0pt}%
\pgfpathmoveto{\pgfqpoint{3.500610in}{3.519157in}}%
\pgfpathlineto{\pgfqpoint{3.510696in}{3.506162in}}%
\pgfpathlineto{\pgfqpoint{3.520817in}{3.490548in}}%
\pgfpathlineto{\pgfqpoint{3.555291in}{3.523454in}}%
\pgfpathlineto{\pgfqpoint{3.589773in}{3.554530in}}%
\pgfpathlineto{\pgfqpoint{3.579590in}{3.569534in}}%
\pgfpathlineto{\pgfqpoint{3.569443in}{3.581663in}}%
\pgfpathlineto{\pgfqpoint{3.535023in}{3.551312in}}%
\pgfpathlineto{\pgfqpoint{3.500610in}{3.519157in}}%
\pgfpathclose%
\pgfusepath{fill}%
\end{pgfscope}%
\begin{pgfscope}%
\pgfpathrectangle{\pgfqpoint{1.020000in}{0.880000in}}{\pgfqpoint{6.160000in}{6.160000in}}%
\pgfusepath{clip}%
\pgfsetbuttcap%
\pgfsetroundjoin%
\definecolor{currentfill}{rgb}{0.796064,0.848693,0.933471}%
\pgfsetfillcolor{currentfill}%
\pgfsetlinewidth{0.000000pt}%
\definecolor{currentstroke}{rgb}{0.000000,0.000000,0.000000}%
\pgfsetstrokecolor{currentstroke}%
\pgfsetdash{}{0pt}%
\pgfpathmoveto{\pgfqpoint{3.431804in}{3.450769in}}%
\pgfpathlineto{\pgfqpoint{3.441830in}{3.436856in}}%
\pgfpathlineto{\pgfqpoint{3.451891in}{3.420603in}}%
\pgfpathlineto{\pgfqpoint{3.486351in}{3.456150in}}%
\pgfpathlineto{\pgfqpoint{3.520817in}{3.490548in}}%
\pgfpathlineto{\pgfqpoint{3.510696in}{3.506162in}}%
\pgfpathlineto{\pgfqpoint{3.500610in}{3.519157in}}%
\pgfpathlineto{\pgfqpoint{3.466204in}{3.485530in}}%
\pgfpathlineto{\pgfqpoint{3.431804in}{3.450769in}}%
\pgfpathclose%
\pgfusepath{fill}%
\end{pgfscope}%
\begin{pgfscope}%
\pgfpathrectangle{\pgfqpoint{1.020000in}{0.880000in}}{\pgfqpoint{6.160000in}{6.160000in}}%
\pgfusepath{clip}%
\pgfsetbuttcap%
\pgfsetroundjoin%
\definecolor{currentfill}{rgb}{0.758539,0.832787,0.958408}%
\pgfsetfillcolor{currentfill}%
\pgfsetlinewidth{0.000000pt}%
\definecolor{currentstroke}{rgb}{0.000000,0.000000,0.000000}%
\pgfsetstrokecolor{currentstroke}%
\pgfsetdash{}{0pt}%
\pgfpathmoveto{\pgfqpoint{3.363018in}{3.379168in}}%
\pgfpathlineto{\pgfqpoint{3.372984in}{3.364317in}}%
\pgfpathlineto{\pgfqpoint{3.382985in}{3.347413in}}%
\pgfpathlineto{\pgfqpoint{3.417436in}{3.384247in}}%
\pgfpathlineto{\pgfqpoint{3.451891in}{3.420603in}}%
\pgfpathlineto{\pgfqpoint{3.441830in}{3.436856in}}%
\pgfpathlineto{\pgfqpoint{3.431804in}{3.450769in}}%
\pgfpathlineto{\pgfqpoint{3.397410in}{3.415206in}}%
\pgfpathlineto{\pgfqpoint{3.363018in}{3.379168in}}%
\pgfpathclose%
\pgfusepath{fill}%
\end{pgfscope}%
\begin{pgfscope}%
\pgfpathrectangle{\pgfqpoint{1.020000in}{0.880000in}}{\pgfqpoint{6.160000in}{6.160000in}}%
\pgfusepath{clip}%
\pgfsetbuttcap%
\pgfsetroundjoin%
\definecolor{currentfill}{rgb}{0.724041,0.814910,0.975651}%
\pgfsetfillcolor{currentfill}%
\pgfsetlinewidth{0.000000pt}%
\definecolor{currentstroke}{rgb}{0.000000,0.000000,0.000000}%
\pgfsetstrokecolor{currentstroke}%
\pgfsetdash{}{0pt}%
\pgfpathmoveto{\pgfqpoint{3.294235in}{3.306904in}}%
\pgfpathlineto{\pgfqpoint{3.304144in}{3.291120in}}%
\pgfpathlineto{\pgfqpoint{3.314085in}{3.273570in}}%
\pgfpathlineto{\pgfqpoint{3.348535in}{3.310420in}}%
\pgfpathlineto{\pgfqpoint{3.382985in}{3.347413in}}%
\pgfpathlineto{\pgfqpoint{3.372984in}{3.364317in}}%
\pgfpathlineto{\pgfqpoint{3.363018in}{3.379168in}}%
\pgfpathlineto{\pgfqpoint{3.328627in}{3.342969in}}%
\pgfpathlineto{\pgfqpoint{3.294235in}{3.306904in}}%
\pgfpathclose%
\pgfusepath{fill}%
\end{pgfscope}%
\begin{pgfscope}%
\pgfpathrectangle{\pgfqpoint{1.020000in}{0.880000in}}{\pgfqpoint{6.160000in}{6.160000in}}%
\pgfusepath{clip}%
\pgfsetbuttcap%
\pgfsetroundjoin%
\definecolor{currentfill}{rgb}{0.683056,0.790043,0.989768}%
\pgfsetfillcolor{currentfill}%
\pgfsetlinewidth{0.000000pt}%
\definecolor{currentstroke}{rgb}{0.000000,0.000000,0.000000}%
\pgfsetstrokecolor{currentstroke}%
\pgfsetdash{}{0pt}%
\pgfpathmoveto{\pgfqpoint{3.225437in}{3.236260in}}%
\pgfpathlineto{\pgfqpoint{3.235288in}{3.219572in}}%
\pgfpathlineto{\pgfqpoint{3.245171in}{3.201396in}}%
\pgfpathlineto{\pgfqpoint{3.279631in}{3.237143in}}%
\pgfpathlineto{\pgfqpoint{3.314085in}{3.273570in}}%
\pgfpathlineto{\pgfqpoint{3.304144in}{3.291120in}}%
\pgfpathlineto{\pgfqpoint{3.294235in}{3.306904in}}%
\pgfpathlineto{\pgfqpoint{3.259839in}{3.271250in}}%
\pgfpathlineto{\pgfqpoint{3.225437in}{3.236260in}}%
\pgfpathclose%
\pgfusepath{fill}%
\end{pgfscope}%
\begin{pgfscope}%
\pgfpathrectangle{\pgfqpoint{1.020000in}{0.880000in}}{\pgfqpoint{6.160000in}{6.160000in}}%
\pgfusepath{clip}%
\pgfsetbuttcap%
\pgfsetroundjoin%
\definecolor{currentfill}{rgb}{0.505423,0.643995,0.983157}%
\pgfsetfillcolor{currentfill}%
\pgfsetlinewidth{0.000000pt}%
\definecolor{currentstroke}{rgb}{0.000000,0.000000,0.000000}%
\pgfsetstrokecolor{currentstroke}%
\pgfsetdash{}{0pt}%
\pgfpathmoveto{\pgfqpoint{4.614089in}{2.991761in}}%
\pgfpathlineto{\pgfqpoint{4.625172in}{2.957010in}}%
\pgfpathlineto{\pgfqpoint{4.636270in}{2.921691in}}%
\pgfpathlineto{\pgfqpoint{4.670466in}{2.898748in}}%
\pgfpathlineto{\pgfqpoint{4.704627in}{2.876311in}}%
\pgfpathlineto{\pgfqpoint{4.693476in}{2.908897in}}%
\pgfpathlineto{\pgfqpoint{4.682341in}{2.941003in}}%
\pgfpathlineto{\pgfqpoint{4.648234in}{2.966103in}}%
\pgfpathlineto{\pgfqpoint{4.614089in}{2.991761in}}%
\pgfpathclose%
\pgfusepath{fill}%
\end{pgfscope}%
\begin{pgfscope}%
\pgfpathrectangle{\pgfqpoint{1.020000in}{0.880000in}}{\pgfqpoint{6.160000in}{6.160000in}}%
\pgfusepath{clip}%
\pgfsetbuttcap%
\pgfsetroundjoin%
\definecolor{currentfill}{rgb}{0.646113,0.764436,0.996868}%
\pgfsetfillcolor{currentfill}%
\pgfsetlinewidth{0.000000pt}%
\definecolor{currentstroke}{rgb}{0.000000,0.000000,0.000000}%
\pgfsetstrokecolor{currentstroke}%
\pgfsetdash{}{0pt}%
\pgfpathmoveto{\pgfqpoint{3.156602in}{3.169148in}}%
\pgfpathlineto{\pgfqpoint{3.166397in}{3.151603in}}%
\pgfpathlineto{\pgfqpoint{3.176222in}{3.132833in}}%
\pgfpathlineto{\pgfqpoint{3.210702in}{3.166559in}}%
\pgfpathlineto{\pgfqpoint{3.245171in}{3.201396in}}%
\pgfpathlineto{\pgfqpoint{3.235288in}{3.219572in}}%
\pgfpathlineto{\pgfqpoint{3.225437in}{3.236260in}}%
\pgfpathlineto{\pgfqpoint{3.191026in}{3.202160in}}%
\pgfpathlineto{\pgfqpoint{3.156602in}{3.169148in}}%
\pgfpathclose%
\pgfusepath{fill}%
\end{pgfscope}%
\begin{pgfscope}%
\pgfpathrectangle{\pgfqpoint{1.020000in}{0.880000in}}{\pgfqpoint{6.160000in}{6.160000in}}%
\pgfusepath{clip}%
\pgfsetbuttcap%
\pgfsetroundjoin%
\definecolor{currentfill}{rgb}{0.333490,0.446265,0.874452}%
\pgfsetfillcolor{currentfill}%
\pgfsetlinewidth{0.000000pt}%
\definecolor{currentstroke}{rgb}{0.000000,0.000000,0.000000}%
\pgfsetstrokecolor{currentstroke}%
\pgfsetdash{}{0pt}%
\pgfpathmoveto{\pgfqpoint{5.022541in}{2.635436in}}%
\pgfpathlineto{\pgfqpoint{5.034028in}{2.612038in}}%
\pgfpathlineto{\pgfqpoint{5.045538in}{2.588849in}}%
\pgfpathlineto{\pgfqpoint{5.079633in}{2.581046in}}%
\pgfpathlineto{\pgfqpoint{5.113708in}{2.574062in}}%
\pgfpathlineto{\pgfqpoint{5.102134in}{2.595578in}}%
\pgfpathlineto{\pgfqpoint{5.090582in}{2.617273in}}%
\pgfpathlineto{\pgfqpoint{5.056572in}{2.625880in}}%
\pgfpathlineto{\pgfqpoint{5.022541in}{2.635436in}}%
\pgfpathclose%
\pgfusepath{fill}%
\end{pgfscope}%
\begin{pgfscope}%
\pgfpathrectangle{\pgfqpoint{1.020000in}{0.880000in}}{\pgfqpoint{6.160000in}{6.160000in}}%
\pgfusepath{clip}%
\pgfsetbuttcap%
\pgfsetroundjoin%
\definecolor{currentfill}{rgb}{0.887752,0.854040,0.834671}%
\pgfsetfillcolor{currentfill}%
\pgfsetlinewidth{0.000000pt}%
\definecolor{currentstroke}{rgb}{0.000000,0.000000,0.000000}%
\pgfsetstrokecolor{currentstroke}%
\pgfsetdash{}{0pt}%
\pgfpathmoveto{\pgfqpoint{3.865798in}{3.702142in}}%
\pgfpathlineto{\pgfqpoint{3.876253in}{3.685323in}}%
\pgfpathlineto{\pgfqpoint{3.886738in}{3.664861in}}%
\pgfpathlineto{\pgfqpoint{3.921286in}{3.667743in}}%
\pgfpathlineto{\pgfqpoint{3.955821in}{3.666785in}}%
\pgfpathlineto{\pgfqpoint{3.945282in}{3.687056in}}%
\pgfpathlineto{\pgfqpoint{3.934770in}{3.703676in}}%
\pgfpathlineto{\pgfqpoint{3.900290in}{3.704826in}}%
\pgfpathlineto{\pgfqpoint{3.865798in}{3.702142in}}%
\pgfpathclose%
\pgfusepath{fill}%
\end{pgfscope}%
\begin{pgfscope}%
\pgfpathrectangle{\pgfqpoint{1.020000in}{0.880000in}}{\pgfqpoint{6.160000in}{6.160000in}}%
\pgfusepath{clip}%
\pgfsetbuttcap%
\pgfsetroundjoin%
\definecolor{currentfill}{rgb}{0.613933,0.739923,0.999142}%
\pgfsetfillcolor{currentfill}%
\pgfsetlinewidth{0.000000pt}%
\definecolor{currentstroke}{rgb}{0.000000,0.000000,0.000000}%
\pgfsetstrokecolor{currentstroke}%
\pgfsetdash{}{0pt}%
\pgfpathmoveto{\pgfqpoint{3.087709in}{3.107044in}}%
\pgfpathlineto{\pgfqpoint{3.097449in}{3.088704in}}%
\pgfpathlineto{\pgfqpoint{3.107216in}{3.069378in}}%
\pgfpathlineto{\pgfqpoint{3.141727in}{3.100391in}}%
\pgfpathlineto{\pgfqpoint{3.176222in}{3.132833in}}%
\pgfpathlineto{\pgfqpoint{3.166397in}{3.151603in}}%
\pgfpathlineto{\pgfqpoint{3.156602in}{3.169148in}}%
\pgfpathlineto{\pgfqpoint{3.122164in}{3.137395in}}%
\pgfpathlineto{\pgfqpoint{3.087709in}{3.107044in}}%
\pgfpathclose%
\pgfusepath{fill}%
\end{pgfscope}%
\begin{pgfscope}%
\pgfpathrectangle{\pgfqpoint{1.020000in}{0.880000in}}{\pgfqpoint{6.160000in}{6.160000in}}%
\pgfusepath{clip}%
\pgfsetbuttcap%
\pgfsetroundjoin%
\definecolor{currentfill}{rgb}{0.378598,0.503856,0.913692}%
\pgfsetfillcolor{currentfill}%
\pgfsetlinewidth{0.000000pt}%
\definecolor{currentstroke}{rgb}{0.000000,0.000000,0.000000}%
\pgfsetstrokecolor{currentstroke}%
\pgfsetdash{}{0pt}%
\pgfpathmoveto{\pgfqpoint{4.863524in}{2.738807in}}%
\pgfpathlineto{\pgfqpoint{4.874844in}{2.710912in}}%
\pgfpathlineto{\pgfqpoint{4.886183in}{2.683123in}}%
\pgfpathlineto{\pgfqpoint{4.920310in}{2.669816in}}%
\pgfpathlineto{\pgfqpoint{4.954411in}{2.657415in}}%
\pgfpathlineto{\pgfqpoint{4.943008in}{2.682954in}}%
\pgfpathlineto{\pgfqpoint{4.931627in}{2.708592in}}%
\pgfpathlineto{\pgfqpoint{4.897588in}{2.723183in}}%
\pgfpathlineto{\pgfqpoint{4.863524in}{2.738807in}}%
\pgfpathclose%
\pgfusepath{fill}%
\end{pgfscope}%
\begin{pgfscope}%
\pgfpathrectangle{\pgfqpoint{1.020000in}{0.880000in}}{\pgfqpoint{6.160000in}{6.160000in}}%
\pgfusepath{clip}%
\pgfsetbuttcap%
\pgfsetroundjoin%
\definecolor{currentfill}{rgb}{0.743754,0.825125,0.965798}%
\pgfsetfillcolor{currentfill}%
\pgfsetlinewidth{0.000000pt}%
\definecolor{currentstroke}{rgb}{0.000000,0.000000,0.000000}%
\pgfsetstrokecolor{currentstroke}%
\pgfsetdash{}{0pt}%
\pgfpathmoveto{\pgfqpoint{4.274329in}{3.426232in}}%
\pgfpathlineto{\pgfqpoint{4.285154in}{3.392293in}}%
\pgfpathlineto{\pgfqpoint{4.295993in}{3.355872in}}%
\pgfpathlineto{\pgfqpoint{4.330383in}{3.330071in}}%
\pgfpathlineto{\pgfqpoint{4.364730in}{3.302656in}}%
\pgfpathlineto{\pgfqpoint{4.353850in}{3.337417in}}%
\pgfpathlineto{\pgfqpoint{4.342982in}{3.369899in}}%
\pgfpathlineto{\pgfqpoint{4.308677in}{3.398912in}}%
\pgfpathlineto{\pgfqpoint{4.274329in}{3.426232in}}%
\pgfpathclose%
\pgfusepath{fill}%
\end{pgfscope}%
\begin{pgfscope}%
\pgfpathrectangle{\pgfqpoint{1.020000in}{0.880000in}}{\pgfqpoint{6.160000in}{6.160000in}}%
\pgfusepath{clip}%
\pgfsetbuttcap%
\pgfsetroundjoin%
\definecolor{currentfill}{rgb}{0.581486,0.713451,0.998314}%
\pgfsetfillcolor{currentfill}%
\pgfsetlinewidth{0.000000pt}%
\definecolor{currentstroke}{rgb}{0.000000,0.000000,0.000000}%
\pgfsetstrokecolor{currentstroke}%
\pgfsetdash{}{0pt}%
\pgfpathmoveto{\pgfqpoint{3.018736in}{3.050968in}}%
\pgfpathlineto{\pgfqpoint{3.028421in}{3.031901in}}%
\pgfpathlineto{\pgfqpoint{3.038132in}{3.012063in}}%
\pgfpathlineto{\pgfqpoint{3.072685in}{3.039906in}}%
\pgfpathlineto{\pgfqpoint{3.107216in}{3.069378in}}%
\pgfpathlineto{\pgfqpoint{3.097449in}{3.088704in}}%
\pgfpathlineto{\pgfqpoint{3.087709in}{3.107044in}}%
\pgfpathlineto{\pgfqpoint{3.053234in}{3.078206in}}%
\pgfpathlineto{\pgfqpoint{3.018736in}{3.050968in}}%
\pgfpathclose%
\pgfusepath{fill}%
\end{pgfscope}%
\begin{pgfscope}%
\pgfpathrectangle{\pgfqpoint{1.020000in}{0.880000in}}{\pgfqpoint{6.160000in}{6.160000in}}%
\pgfusepath{clip}%
\pgfsetbuttcap%
\pgfsetroundjoin%
\definecolor{currentfill}{rgb}{0.285273,0.380129,0.823469}%
\pgfsetfillcolor{currentfill}%
\pgfsetlinewidth{0.000000pt}%
\definecolor{currentstroke}{rgb}{0.000000,0.000000,0.000000}%
\pgfsetstrokecolor{currentstroke}%
\pgfsetdash{}{0pt}%
\pgfpathmoveto{\pgfqpoint{5.341346in}{2.512568in}}%
\pgfpathlineto{\pgfqpoint{5.353185in}{2.495070in}}%
\pgfpathlineto{\pgfqpoint{5.387220in}{2.494225in}}%
\pgfpathlineto{\pgfqpoint{5.421236in}{2.493809in}}%
\pgfpathlineto{\pgfqpoint{5.409334in}{2.510591in}}%
\pgfpathlineto{\pgfqpoint{5.375349in}{2.511346in}}%
\pgfpathlineto{\pgfqpoint{5.341346in}{2.512568in}}%
\pgfpathclose%
\pgfusepath{fill}%
\end{pgfscope}%
\begin{pgfscope}%
\pgfpathrectangle{\pgfqpoint{1.020000in}{0.880000in}}{\pgfqpoint{6.160000in}{6.160000in}}%
\pgfusepath{clip}%
\pgfsetbuttcap%
\pgfsetroundjoin%
\definecolor{currentfill}{rgb}{0.859385,0.864431,0.872111}%
\pgfsetfillcolor{currentfill}%
\pgfsetlinewidth{0.000000pt}%
\definecolor{currentstroke}{rgb}{0.000000,0.000000,0.000000}%
\pgfsetstrokecolor{currentstroke}%
\pgfsetdash{}{0pt}%
\pgfpathmoveto{\pgfqpoint{4.024836in}{3.653518in}}%
\pgfpathlineto{\pgfqpoint{4.035452in}{3.629863in}}%
\pgfpathlineto{\pgfqpoint{4.046092in}{3.602663in}}%
\pgfpathlineto{\pgfqpoint{4.080613in}{3.590930in}}%
\pgfpathlineto{\pgfqpoint{4.115105in}{3.575803in}}%
\pgfpathlineto{\pgfqpoint{4.104418in}{3.602541in}}%
\pgfpathlineto{\pgfqpoint{4.093752in}{3.625850in}}%
\pgfpathlineto{\pgfqpoint{4.059308in}{3.641406in}}%
\pgfpathlineto{\pgfqpoint{4.024836in}{3.653518in}}%
\pgfpathclose%
\pgfusepath{fill}%
\end{pgfscope}%
\begin{pgfscope}%
\pgfpathrectangle{\pgfqpoint{1.020000in}{0.880000in}}{\pgfqpoint{6.160000in}{6.160000in}}%
\pgfusepath{clip}%
\pgfsetbuttcap%
\pgfsetroundjoin%
\definecolor{currentfill}{rgb}{0.299441,0.400248,0.839842}%
\pgfsetfillcolor{currentfill}%
\pgfsetlinewidth{0.000000pt}%
\definecolor{currentstroke}{rgb}{0.000000,0.000000,0.000000}%
\pgfsetstrokecolor{currentstroke}%
\pgfsetdash{}{0pt}%
\pgfpathmoveto{\pgfqpoint{5.181796in}{2.562481in}}%
\pgfpathlineto{\pgfqpoint{5.193458in}{2.542569in}}%
\pgfpathlineto{\pgfqpoint{5.205144in}{2.522869in}}%
\pgfpathlineto{\pgfqpoint{5.239223in}{2.519413in}}%
\pgfpathlineto{\pgfqpoint{5.273283in}{2.516566in}}%
\pgfpathlineto{\pgfqpoint{5.261533in}{2.535146in}}%
\pgfpathlineto{\pgfqpoint{5.249806in}{2.553900in}}%
\pgfpathlineto{\pgfqpoint{5.215810in}{2.557832in}}%
\pgfpathlineto{\pgfqpoint{5.181796in}{2.562481in}}%
\pgfpathclose%
\pgfusepath{fill}%
\end{pgfscope}%
\begin{pgfscope}%
\pgfpathrectangle{\pgfqpoint{1.020000in}{0.880000in}}{\pgfqpoint{6.160000in}{6.160000in}}%
\pgfusepath{clip}%
\pgfsetbuttcap%
\pgfsetroundjoin%
\definecolor{currentfill}{rgb}{0.548876,0.685104,0.994379}%
\pgfsetfillcolor{currentfill}%
\pgfsetlinewidth{0.000000pt}%
\definecolor{currentstroke}{rgb}{0.000000,0.000000,0.000000}%
\pgfsetstrokecolor{currentstroke}%
\pgfsetdash{}{0pt}%
\pgfpathmoveto{\pgfqpoint{2.949665in}{3.001500in}}%
\pgfpathlineto{\pgfqpoint{2.959296in}{2.981779in}}%
\pgfpathlineto{\pgfqpoint{2.968950in}{2.961473in}}%
\pgfpathlineto{\pgfqpoint{3.003554in}{2.985908in}}%
\pgfpathlineto{\pgfqpoint{3.038132in}{3.012063in}}%
\pgfpathlineto{\pgfqpoint{3.028421in}{3.031901in}}%
\pgfpathlineto{\pgfqpoint{3.018736in}{3.050968in}}%
\pgfpathlineto{\pgfqpoint{2.984214in}{3.025388in}}%
\pgfpathlineto{\pgfqpoint{2.949665in}{3.001500in}}%
\pgfpathclose%
\pgfusepath{fill}%
\end{pgfscope}%
\begin{pgfscope}%
\pgfpathrectangle{\pgfqpoint{1.020000in}{0.880000in}}{\pgfqpoint{6.160000in}{6.160000in}}%
\pgfusepath{clip}%
\pgfsetbuttcap%
\pgfsetroundjoin%
\definecolor{currentfill}{rgb}{0.451739,0.588181,0.960201}%
\pgfsetfillcolor{currentfill}%
\pgfsetlinewidth{0.000000pt}%
\definecolor{currentstroke}{rgb}{0.000000,0.000000,0.000000}%
\pgfsetstrokecolor{currentstroke}%
\pgfsetdash{}{0pt}%
\pgfpathmoveto{\pgfqpoint{4.704627in}{2.876311in}}%
\pgfpathlineto{\pgfqpoint{4.715794in}{2.843403in}}%
\pgfpathlineto{\pgfqpoint{4.726978in}{2.810327in}}%
\pgfpathlineto{\pgfqpoint{4.761160in}{2.791269in}}%
\pgfpathlineto{\pgfqpoint{4.795311in}{2.772929in}}%
\pgfpathlineto{\pgfqpoint{4.784070in}{2.803316in}}%
\pgfpathlineto{\pgfqpoint{4.772847in}{2.833572in}}%
\pgfpathlineto{\pgfqpoint{4.738753in}{2.854539in}}%
\pgfpathlineto{\pgfqpoint{4.704627in}{2.876311in}}%
\pgfpathclose%
\pgfusepath{fill}%
\end{pgfscope}%
\begin{pgfscope}%
\pgfpathrectangle{\pgfqpoint{1.020000in}{0.880000in}}{\pgfqpoint{6.160000in}{6.160000in}}%
\pgfusepath{clip}%
\pgfsetbuttcap%
\pgfsetroundjoin%
\definecolor{currentfill}{rgb}{0.677823,0.786546,0.991005}%
\pgfsetfillcolor{currentfill}%
\pgfsetlinewidth{0.000000pt}%
\definecolor{currentstroke}{rgb}{0.000000,0.000000,0.000000}%
\pgfsetstrokecolor{currentstroke}%
\pgfsetdash{}{0pt}%
\pgfpathmoveto{\pgfqpoint{4.364730in}{3.302656in}}%
\pgfpathlineto{\pgfqpoint{4.375624in}{3.265821in}}%
\pgfpathlineto{\pgfqpoint{4.386531in}{3.227133in}}%
\pgfpathlineto{\pgfqpoint{4.420876in}{3.200476in}}%
\pgfpathlineto{\pgfqpoint{4.455179in}{3.172885in}}%
\pgfpathlineto{\pgfqpoint{4.444230in}{3.209388in}}%
\pgfpathlineto{\pgfqpoint{4.433294in}{3.244222in}}%
\pgfpathlineto{\pgfqpoint{4.399034in}{3.273936in}}%
\pgfpathlineto{\pgfqpoint{4.364730in}{3.302656in}}%
\pgfpathclose%
\pgfusepath{fill}%
\end{pgfscope}%
\begin{pgfscope}%
\pgfpathrectangle{\pgfqpoint{1.020000in}{0.880000in}}{\pgfqpoint{6.160000in}{6.160000in}}%
\pgfusepath{clip}%
\pgfsetbuttcap%
\pgfsetroundjoin%
\definecolor{currentfill}{rgb}{0.527132,0.664700,0.989065}%
\pgfsetfillcolor{currentfill}%
\pgfsetlinewidth{0.000000pt}%
\definecolor{currentstroke}{rgb}{0.000000,0.000000,0.000000}%
\pgfsetstrokecolor{currentstroke}%
\pgfsetdash{}{0pt}%
\pgfpathmoveto{\pgfqpoint{2.880481in}{2.958825in}}%
\pgfpathlineto{\pgfqpoint{2.890058in}{2.938520in}}%
\pgfpathlineto{\pgfqpoint{2.899657in}{2.917790in}}%
\pgfpathlineto{\pgfqpoint{2.934319in}{2.938771in}}%
\pgfpathlineto{\pgfqpoint{2.968950in}{2.961473in}}%
\pgfpathlineto{\pgfqpoint{2.959296in}{2.981779in}}%
\pgfpathlineto{\pgfqpoint{2.949665in}{3.001500in}}%
\pgfpathlineto{\pgfqpoint{2.915088in}{2.979316in}}%
\pgfpathlineto{\pgfqpoint{2.880481in}{2.958825in}}%
\pgfpathclose%
\pgfusepath{fill}%
\end{pgfscope}%
\begin{pgfscope}%
\pgfpathrectangle{\pgfqpoint{1.020000in}{0.880000in}}{\pgfqpoint{6.160000in}{6.160000in}}%
\pgfusepath{clip}%
\pgfsetbuttcap%
\pgfsetroundjoin%
\definecolor{currentfill}{rgb}{0.887752,0.854040,0.834671}%
\pgfsetfillcolor{currentfill}%
\pgfsetlinewidth{0.000000pt}%
\definecolor{currentstroke}{rgb}{0.000000,0.000000,0.000000}%
\pgfsetstrokecolor{currentstroke}%
\pgfsetdash{}{0pt}%
\pgfpathmoveto{\pgfqpoint{3.796788in}{3.685400in}}%
\pgfpathlineto{\pgfqpoint{3.807186in}{3.668340in}}%
\pgfpathlineto{\pgfqpoint{3.817615in}{3.647711in}}%
\pgfpathlineto{\pgfqpoint{3.852179in}{3.658161in}}%
\pgfpathlineto{\pgfqpoint{3.886738in}{3.664861in}}%
\pgfpathlineto{\pgfqpoint{3.876253in}{3.685323in}}%
\pgfpathlineto{\pgfqpoint{3.865798in}{3.702142in}}%
\pgfpathlineto{\pgfqpoint{3.831296in}{3.695644in}}%
\pgfpathlineto{\pgfqpoint{3.796788in}{3.685400in}}%
\pgfpathclose%
\pgfusepath{fill}%
\end{pgfscope}%
\begin{pgfscope}%
\pgfpathrectangle{\pgfqpoint{1.020000in}{0.880000in}}{\pgfqpoint{6.160000in}{6.160000in}}%
\pgfusepath{clip}%
\pgfsetbuttcap%
\pgfsetroundjoin%
\definecolor{currentfill}{rgb}{0.608547,0.735725,0.999354}%
\pgfsetfillcolor{currentfill}%
\pgfsetlinewidth{0.000000pt}%
\definecolor{currentstroke}{rgb}{0.000000,0.000000,0.000000}%
\pgfsetstrokecolor{currentstroke}%
\pgfsetdash{}{0pt}%
\pgfpathmoveto{\pgfqpoint{4.455179in}{3.172885in}}%
\pgfpathlineto{\pgfqpoint{4.466141in}{3.134932in}}%
\pgfpathlineto{\pgfqpoint{4.477116in}{3.095756in}}%
\pgfpathlineto{\pgfqpoint{4.511420in}{3.070019in}}%
\pgfpathlineto{\pgfqpoint{4.545683in}{3.043945in}}%
\pgfpathlineto{\pgfqpoint{4.534663in}{3.080538in}}%
\pgfpathlineto{\pgfqpoint{4.523656in}{3.116050in}}%
\pgfpathlineto{\pgfqpoint{4.489439in}{3.144650in}}%
\pgfpathlineto{\pgfqpoint{4.455179in}{3.172885in}}%
\pgfpathclose%
\pgfusepath{fill}%
\end{pgfscope}%
\begin{pgfscope}%
\pgfpathrectangle{\pgfqpoint{1.020000in}{0.880000in}}{\pgfqpoint{6.160000in}{6.160000in}}%
\pgfusepath{clip}%
\pgfsetbuttcap%
\pgfsetroundjoin%
\definecolor{currentfill}{rgb}{0.822420,0.856898,0.910795}%
\pgfsetfillcolor{currentfill}%
\pgfsetlinewidth{0.000000pt}%
\definecolor{currentstroke}{rgb}{0.000000,0.000000,0.000000}%
\pgfsetstrokecolor{currentstroke}%
\pgfsetdash{}{0pt}%
\pgfpathmoveto{\pgfqpoint{4.115105in}{3.575803in}}%
\pgfpathlineto{\pgfqpoint{4.125811in}{3.545732in}}%
\pgfpathlineto{\pgfqpoint{4.136536in}{3.512467in}}%
\pgfpathlineto{\pgfqpoint{4.171039in}{3.494931in}}%
\pgfpathlineto{\pgfqpoint{4.205508in}{3.474536in}}%
\pgfpathlineto{\pgfqpoint{4.194740in}{3.506894in}}%
\pgfpathlineto{\pgfqpoint{4.183990in}{3.536217in}}%
\pgfpathlineto{\pgfqpoint{4.149565in}{3.557485in}}%
\pgfpathlineto{\pgfqpoint{4.115105in}{3.575803in}}%
\pgfpathclose%
\pgfusepath{fill}%
\end{pgfscope}%
\begin{pgfscope}%
\pgfpathrectangle{\pgfqpoint{1.020000in}{0.880000in}}{\pgfqpoint{6.160000in}{6.160000in}}%
\pgfusepath{clip}%
\pgfsetbuttcap%
\pgfsetroundjoin%
\definecolor{currentfill}{rgb}{0.348323,0.465711,0.888346}%
\pgfsetfillcolor{currentfill}%
\pgfsetlinewidth{0.000000pt}%
\definecolor{currentstroke}{rgb}{0.000000,0.000000,0.000000}%
\pgfsetstrokecolor{currentstroke}%
\pgfsetdash{}{0pt}%
\pgfpathmoveto{\pgfqpoint{4.954411in}{2.657415in}}%
\pgfpathlineto{\pgfqpoint{4.965834in}{2.632050in}}%
\pgfpathlineto{\pgfqpoint{4.977279in}{2.606929in}}%
\pgfpathlineto{\pgfqpoint{5.011420in}{2.597478in}}%
\pgfpathlineto{\pgfqpoint{5.045538in}{2.588849in}}%
\pgfpathlineto{\pgfqpoint{5.034028in}{2.612038in}}%
\pgfpathlineto{\pgfqpoint{5.022541in}{2.635436in}}%
\pgfpathlineto{\pgfqpoint{4.988487in}{2.645949in}}%
\pgfpathlineto{\pgfqpoint{4.954411in}{2.657415in}}%
\pgfpathclose%
\pgfusepath{fill}%
\end{pgfscope}%
\begin{pgfscope}%
\pgfpathrectangle{\pgfqpoint{1.020000in}{0.880000in}}{\pgfqpoint{6.160000in}{6.160000in}}%
\pgfusepath{clip}%
\pgfsetbuttcap%
\pgfsetroundjoin%
\definecolor{currentfill}{rgb}{0.289996,0.386836,0.828926}%
\pgfsetfillcolor{currentfill}%
\pgfsetlinewidth{0.000000pt}%
\definecolor{currentstroke}{rgb}{0.000000,0.000000,0.000000}%
\pgfsetstrokecolor{currentstroke}%
\pgfsetdash{}{0pt}%
\pgfpathmoveto{\pgfqpoint{5.273283in}{2.516566in}}%
\pgfpathlineto{\pgfqpoint{5.285058in}{2.498181in}}%
\pgfpathlineto{\pgfqpoint{5.319131in}{2.496377in}}%
\pgfpathlineto{\pgfqpoint{5.353185in}{2.495070in}}%
\pgfpathlineto{\pgfqpoint{5.341346in}{2.512568in}}%
\pgfpathlineto{\pgfqpoint{5.307324in}{2.514297in}}%
\pgfpathlineto{\pgfqpoint{5.273283in}{2.516566in}}%
\pgfpathclose%
\pgfusepath{fill}%
\end{pgfscope}%
\begin{pgfscope}%
\pgfpathrectangle{\pgfqpoint{1.020000in}{0.880000in}}{\pgfqpoint{6.160000in}{6.160000in}}%
\pgfusepath{clip}%
\pgfsetbuttcap%
\pgfsetroundjoin%
\definecolor{currentfill}{rgb}{0.505423,0.643995,0.983157}%
\pgfsetfillcolor{currentfill}%
\pgfsetlinewidth{0.000000pt}%
\definecolor{currentstroke}{rgb}{0.000000,0.000000,0.000000}%
\pgfsetstrokecolor{currentstroke}%
\pgfsetdash{}{0pt}%
\pgfpathmoveto{\pgfqpoint{2.811171in}{2.922799in}}%
\pgfpathlineto{\pgfqpoint{2.820695in}{2.901977in}}%
\pgfpathlineto{\pgfqpoint{2.830240in}{2.880863in}}%
\pgfpathlineto{\pgfqpoint{2.864964in}{2.898503in}}%
\pgfpathlineto{\pgfqpoint{2.899657in}{2.917790in}}%
\pgfpathlineto{\pgfqpoint{2.890058in}{2.938520in}}%
\pgfpathlineto{\pgfqpoint{2.880481in}{2.958825in}}%
\pgfpathlineto{\pgfqpoint{2.845842in}{2.940001in}}%
\pgfpathlineto{\pgfqpoint{2.811171in}{2.922799in}}%
\pgfpathclose%
\pgfusepath{fill}%
\end{pgfscope}%
\begin{pgfscope}%
\pgfpathrectangle{\pgfqpoint{1.020000in}{0.880000in}}{\pgfqpoint{6.160000in}{6.160000in}}%
\pgfusepath{clip}%
\pgfsetbuttcap%
\pgfsetroundjoin%
\definecolor{currentfill}{rgb}{0.309060,0.413498,0.850128}%
\pgfsetfillcolor{currentfill}%
\pgfsetlinewidth{0.000000pt}%
\definecolor{currentstroke}{rgb}{0.000000,0.000000,0.000000}%
\pgfsetstrokecolor{currentstroke}%
\pgfsetdash{}{0pt}%
\pgfpathmoveto{\pgfqpoint{5.113708in}{2.574062in}}%
\pgfpathlineto{\pgfqpoint{5.125305in}{2.552764in}}%
\pgfpathlineto{\pgfqpoint{5.136926in}{2.531719in}}%
\pgfpathlineto{\pgfqpoint{5.171045in}{2.526962in}}%
\pgfpathlineto{\pgfqpoint{5.205144in}{2.522869in}}%
\pgfpathlineto{\pgfqpoint{5.193458in}{2.542569in}}%
\pgfpathlineto{\pgfqpoint{5.181796in}{2.562481in}}%
\pgfpathlineto{\pgfqpoint{5.147762in}{2.567881in}}%
\pgfpathlineto{\pgfqpoint{5.113708in}{2.574062in}}%
\pgfpathclose%
\pgfusepath{fill}%
\end{pgfscope}%
\begin{pgfscope}%
\pgfpathrectangle{\pgfqpoint{1.020000in}{0.880000in}}{\pgfqpoint{6.160000in}{6.160000in}}%
\pgfusepath{clip}%
\pgfsetbuttcap%
\pgfsetroundjoin%
\definecolor{currentfill}{rgb}{0.404421,0.534643,0.932002}%
\pgfsetfillcolor{currentfill}%
\pgfsetlinewidth{0.000000pt}%
\definecolor{currentstroke}{rgb}{0.000000,0.000000,0.000000}%
\pgfsetstrokecolor{currentstroke}%
\pgfsetdash{}{0pt}%
\pgfpathmoveto{\pgfqpoint{4.795311in}{2.772929in}}%
\pgfpathlineto{\pgfqpoint{4.806571in}{2.742536in}}%
\pgfpathlineto{\pgfqpoint{4.817850in}{2.712258in}}%
\pgfpathlineto{\pgfqpoint{4.852030in}{2.697291in}}%
\pgfpathlineto{\pgfqpoint{4.886183in}{2.683123in}}%
\pgfpathlineto{\pgfqpoint{4.874844in}{2.710912in}}%
\pgfpathlineto{\pgfqpoint{4.863524in}{2.738807in}}%
\pgfpathlineto{\pgfqpoint{4.829432in}{2.755412in}}%
\pgfpathlineto{\pgfqpoint{4.795311in}{2.772929in}}%
\pgfpathclose%
\pgfusepath{fill}%
\end{pgfscope}%
\begin{pgfscope}%
\pgfpathrectangle{\pgfqpoint{1.020000in}{0.880000in}}{\pgfqpoint{6.160000in}{6.160000in}}%
\pgfusepath{clip}%
\pgfsetbuttcap%
\pgfsetroundjoin%
\definecolor{currentfill}{rgb}{0.871493,0.862309,0.857016}%
\pgfsetfillcolor{currentfill}%
\pgfsetlinewidth{0.000000pt}%
\definecolor{currentstroke}{rgb}{0.000000,0.000000,0.000000}%
\pgfsetstrokecolor{currentstroke}%
\pgfsetdash{}{0pt}%
\pgfpathmoveto{\pgfqpoint{3.955821in}{3.666785in}}%
\pgfpathlineto{\pgfqpoint{3.966387in}{3.642863in}}%
\pgfpathlineto{\pgfqpoint{3.976977in}{3.615341in}}%
\pgfpathlineto{\pgfqpoint{4.011545in}{3.610841in}}%
\pgfpathlineto{\pgfqpoint{4.046092in}{3.602663in}}%
\pgfpathlineto{\pgfqpoint{4.035452in}{3.629863in}}%
\pgfpathlineto{\pgfqpoint{4.024836in}{3.653518in}}%
\pgfpathlineto{\pgfqpoint{3.990339in}{3.662018in}}%
\pgfpathlineto{\pgfqpoint{3.955821in}{3.666785in}}%
\pgfpathclose%
\pgfusepath{fill}%
\end{pgfscope}%
\begin{pgfscope}%
\pgfpathrectangle{\pgfqpoint{1.020000in}{0.880000in}}{\pgfqpoint{6.160000in}{6.160000in}}%
\pgfusepath{clip}%
\pgfsetbuttcap%
\pgfsetroundjoin%
\definecolor{currentfill}{rgb}{0.879622,0.858175,0.845844}%
\pgfsetfillcolor{currentfill}%
\pgfsetlinewidth{0.000000pt}%
\definecolor{currentstroke}{rgb}{0.000000,0.000000,0.000000}%
\pgfsetstrokecolor{currentstroke}%
\pgfsetdash{}{0pt}%
\pgfpathmoveto{\pgfqpoint{3.727768in}{3.654172in}}%
\pgfpathlineto{\pgfqpoint{3.738107in}{3.636836in}}%
\pgfpathlineto{\pgfqpoint{3.748478in}{3.616061in}}%
\pgfpathlineto{\pgfqpoint{3.783047in}{3.633624in}}%
\pgfpathlineto{\pgfqpoint{3.817615in}{3.647711in}}%
\pgfpathlineto{\pgfqpoint{3.807186in}{3.668340in}}%
\pgfpathlineto{\pgfqpoint{3.796788in}{3.685400in}}%
\pgfpathlineto{\pgfqpoint{3.762278in}{3.671523in}}%
\pgfpathlineto{\pgfqpoint{3.727768in}{3.654172in}}%
\pgfpathclose%
\pgfusepath{fill}%
\end{pgfscope}%
\begin{pgfscope}%
\pgfpathrectangle{\pgfqpoint{1.020000in}{0.880000in}}{\pgfqpoint{6.160000in}{6.160000in}}%
\pgfusepath{clip}%
\pgfsetbuttcap%
\pgfsetroundjoin%
\definecolor{currentfill}{rgb}{0.543440,0.680003,0.993051}%
\pgfsetfillcolor{currentfill}%
\pgfsetlinewidth{0.000000pt}%
\definecolor{currentstroke}{rgb}{0.000000,0.000000,0.000000}%
\pgfsetstrokecolor{currentstroke}%
\pgfsetdash{}{0pt}%
\pgfpathmoveto{\pgfqpoint{4.545683in}{3.043945in}}%
\pgfpathlineto{\pgfqpoint{4.556717in}{3.006480in}}%
\pgfpathlineto{\pgfqpoint{4.567765in}{2.968357in}}%
\pgfpathlineto{\pgfqpoint{4.602036in}{2.944959in}}%
\pgfpathlineto{\pgfqpoint{4.636270in}{2.921691in}}%
\pgfpathlineto{\pgfqpoint{4.625172in}{2.957010in}}%
\pgfpathlineto{\pgfqpoint{4.614089in}{2.991761in}}%
\pgfpathlineto{\pgfqpoint{4.579906in}{3.017781in}}%
\pgfpathlineto{\pgfqpoint{4.545683in}{3.043945in}}%
\pgfpathclose%
\pgfusepath{fill}%
\end{pgfscope}%
\begin{pgfscope}%
\pgfpathrectangle{\pgfqpoint{1.020000in}{0.880000in}}{\pgfqpoint{6.160000in}{6.160000in}}%
\pgfusepath{clip}%
\pgfsetbuttcap%
\pgfsetroundjoin%
\definecolor{currentfill}{rgb}{0.489246,0.627536,0.976896}%
\pgfsetfillcolor{currentfill}%
\pgfsetlinewidth{0.000000pt}%
\definecolor{currentstroke}{rgb}{0.000000,0.000000,0.000000}%
\pgfsetstrokecolor{currentstroke}%
\pgfsetdash{}{0pt}%
\pgfpathmoveto{\pgfqpoint{2.741729in}{2.893026in}}%
\pgfpathlineto{\pgfqpoint{2.751200in}{2.871746in}}%
\pgfpathlineto{\pgfqpoint{2.760690in}{2.850284in}}%
\pgfpathlineto{\pgfqpoint{2.795482in}{2.864813in}}%
\pgfpathlineto{\pgfqpoint{2.830240in}{2.880863in}}%
\pgfpathlineto{\pgfqpoint{2.820695in}{2.901977in}}%
\pgfpathlineto{\pgfqpoint{2.811171in}{2.922799in}}%
\pgfpathlineto{\pgfqpoint{2.776467in}{2.907163in}}%
\pgfpathlineto{\pgfqpoint{2.741729in}{2.893026in}}%
\pgfpathclose%
\pgfusepath{fill}%
\end{pgfscope}%
\begin{pgfscope}%
\pgfpathrectangle{\pgfqpoint{1.020000in}{0.880000in}}{\pgfqpoint{6.160000in}{6.160000in}}%
\pgfusepath{clip}%
\pgfsetbuttcap%
\pgfsetroundjoin%
\definecolor{currentfill}{rgb}{0.777378,0.840921,0.946149}%
\pgfsetfillcolor{currentfill}%
\pgfsetlinewidth{0.000000pt}%
\definecolor{currentstroke}{rgb}{0.000000,0.000000,0.000000}%
\pgfsetstrokecolor{currentstroke}%
\pgfsetdash{}{0pt}%
\pgfpathmoveto{\pgfqpoint{4.205508in}{3.474536in}}%
\pgfpathlineto{\pgfqpoint{4.216292in}{3.439314in}}%
\pgfpathlineto{\pgfqpoint{4.227091in}{3.401434in}}%
\pgfpathlineto{\pgfqpoint{4.261562in}{3.379756in}}%
\pgfpathlineto{\pgfqpoint{4.295993in}{3.355872in}}%
\pgfpathlineto{\pgfqpoint{4.285154in}{3.392293in}}%
\pgfpathlineto{\pgfqpoint{4.274329in}{3.426232in}}%
\pgfpathlineto{\pgfqpoint{4.239938in}{3.451541in}}%
\pgfpathlineto{\pgfqpoint{4.205508in}{3.474536in}}%
\pgfpathclose%
\pgfusepath{fill}%
\end{pgfscope}%
\begin{pgfscope}%
\pgfpathrectangle{\pgfqpoint{1.020000in}{0.880000in}}{\pgfqpoint{6.160000in}{6.160000in}}%
\pgfusepath{clip}%
\pgfsetbuttcap%
\pgfsetroundjoin%
\definecolor{currentfill}{rgb}{0.863392,0.865084,0.867634}%
\pgfsetfillcolor{currentfill}%
\pgfsetlinewidth{0.000000pt}%
\definecolor{currentstroke}{rgb}{0.000000,0.000000,0.000000}%
\pgfsetstrokecolor{currentstroke}%
\pgfsetdash{}{0pt}%
\pgfpathmoveto{\pgfqpoint{3.658759in}{3.609881in}}%
\pgfpathlineto{\pgfqpoint{3.669037in}{3.592236in}}%
\pgfpathlineto{\pgfqpoint{3.679350in}{3.571339in}}%
\pgfpathlineto{\pgfqpoint{3.713912in}{3.595221in}}%
\pgfpathlineto{\pgfqpoint{3.748478in}{3.616061in}}%
\pgfpathlineto{\pgfqpoint{3.738107in}{3.636836in}}%
\pgfpathlineto{\pgfqpoint{3.727768in}{3.654172in}}%
\pgfpathlineto{\pgfqpoint{3.693261in}{3.633546in}}%
\pgfpathlineto{\pgfqpoint{3.658759in}{3.609881in}}%
\pgfpathclose%
\pgfusepath{fill}%
\end{pgfscope}%
\begin{pgfscope}%
\pgfpathrectangle{\pgfqpoint{1.020000in}{0.880000in}}{\pgfqpoint{6.160000in}{6.160000in}}%
\pgfusepath{clip}%
\pgfsetbuttcap%
\pgfsetroundjoin%
\definecolor{currentfill}{rgb}{0.478462,0.616564,0.972721}%
\pgfsetfillcolor{currentfill}%
\pgfsetlinewidth{0.000000pt}%
\definecolor{currentstroke}{rgb}{0.000000,0.000000,0.000000}%
\pgfsetstrokecolor{currentstroke}%
\pgfsetdash{}{0pt}%
\pgfpathmoveto{\pgfqpoint{4.636270in}{2.921691in}}%
\pgfpathlineto{\pgfqpoint{4.647383in}{2.885986in}}%
\pgfpathlineto{\pgfqpoint{4.658513in}{2.850077in}}%
\pgfpathlineto{\pgfqpoint{4.692762in}{2.829977in}}%
\pgfpathlineto{\pgfqpoint{4.726978in}{2.810327in}}%
\pgfpathlineto{\pgfqpoint{4.715794in}{2.843403in}}%
\pgfpathlineto{\pgfqpoint{4.704627in}{2.876311in}}%
\pgfpathlineto{\pgfqpoint{4.670466in}{2.898748in}}%
\pgfpathlineto{\pgfqpoint{4.636270in}{2.921691in}}%
\pgfpathclose%
\pgfusepath{fill}%
\end{pgfscope}%
\begin{pgfscope}%
\pgfpathrectangle{\pgfqpoint{1.020000in}{0.880000in}}{\pgfqpoint{6.160000in}{6.160000in}}%
\pgfusepath{clip}%
\pgfsetbuttcap%
\pgfsetroundjoin%
\definecolor{currentfill}{rgb}{0.843358,0.861820,0.890017}%
\pgfsetfillcolor{currentfill}%
\pgfsetlinewidth{0.000000pt}%
\definecolor{currentstroke}{rgb}{0.000000,0.000000,0.000000}%
\pgfsetstrokecolor{currentstroke}%
\pgfsetdash{}{0pt}%
\pgfpathmoveto{\pgfqpoint{3.589773in}{3.554530in}}%
\pgfpathlineto{\pgfqpoint{3.599991in}{3.536554in}}%
\pgfpathlineto{\pgfqpoint{3.610243in}{3.515556in}}%
\pgfpathlineto{\pgfqpoint{3.644793in}{3.544686in}}%
\pgfpathlineto{\pgfqpoint{3.679350in}{3.571339in}}%
\pgfpathlineto{\pgfqpoint{3.669037in}{3.592236in}}%
\pgfpathlineto{\pgfqpoint{3.658759in}{3.609881in}}%
\pgfpathlineto{\pgfqpoint{3.624262in}{3.583444in}}%
\pgfpathlineto{\pgfqpoint{3.589773in}{3.554530in}}%
\pgfpathclose%
\pgfusepath{fill}%
\end{pgfscope}%
\begin{pgfscope}%
\pgfpathrectangle{\pgfqpoint{1.020000in}{0.880000in}}{\pgfqpoint{6.160000in}{6.160000in}}%
\pgfusepath{clip}%
\pgfsetbuttcap%
\pgfsetroundjoin%
\definecolor{currentfill}{rgb}{0.294718,0.393542,0.834384}%
\pgfsetfillcolor{currentfill}%
\pgfsetlinewidth{0.000000pt}%
\definecolor{currentstroke}{rgb}{0.000000,0.000000,0.000000}%
\pgfsetstrokecolor{currentstroke}%
\pgfsetdash{}{0pt}%
\pgfpathmoveto{\pgfqpoint{5.205144in}{2.522869in}}%
\pgfpathlineto{\pgfqpoint{5.216854in}{2.503402in}}%
\pgfpathlineto{\pgfqpoint{5.250966in}{2.500512in}}%
\pgfpathlineto{\pgfqpoint{5.285058in}{2.498181in}}%
\pgfpathlineto{\pgfqpoint{5.273283in}{2.516566in}}%
\pgfpathlineto{\pgfqpoint{5.239223in}{2.519413in}}%
\pgfpathlineto{\pgfqpoint{5.205144in}{2.522869in}}%
\pgfpathclose%
\pgfusepath{fill}%
\end{pgfscope}%
\begin{pgfscope}%
\pgfpathrectangle{\pgfqpoint{1.020000in}{0.880000in}}{\pgfqpoint{6.160000in}{6.160000in}}%
\pgfusepath{clip}%
\pgfsetbuttcap%
\pgfsetroundjoin%
\definecolor{currentfill}{rgb}{0.713852,0.808857,0.979386}%
\pgfsetfillcolor{currentfill}%
\pgfsetlinewidth{0.000000pt}%
\definecolor{currentstroke}{rgb}{0.000000,0.000000,0.000000}%
\pgfsetstrokecolor{currentstroke}%
\pgfsetdash{}{0pt}%
\pgfpathmoveto{\pgfqpoint{4.295993in}{3.355872in}}%
\pgfpathlineto{\pgfqpoint{4.306846in}{3.317188in}}%
\pgfpathlineto{\pgfqpoint{4.317713in}{3.276486in}}%
\pgfpathlineto{\pgfqpoint{4.352143in}{3.252566in}}%
\pgfpathlineto{\pgfqpoint{4.386531in}{3.227133in}}%
\pgfpathlineto{\pgfqpoint{4.375624in}{3.265821in}}%
\pgfpathlineto{\pgfqpoint{4.364730in}{3.302656in}}%
\pgfpathlineto{\pgfqpoint{4.330383in}{3.330071in}}%
\pgfpathlineto{\pgfqpoint{4.295993in}{3.355872in}}%
\pgfpathclose%
\pgfusepath{fill}%
\end{pgfscope}%
\begin{pgfscope}%
\pgfpathrectangle{\pgfqpoint{1.020000in}{0.880000in}}{\pgfqpoint{6.160000in}{6.160000in}}%
\pgfusepath{clip}%
\pgfsetbuttcap%
\pgfsetroundjoin%
\definecolor{currentfill}{rgb}{0.818056,0.855590,0.914638}%
\pgfsetfillcolor{currentfill}%
\pgfsetlinewidth{0.000000pt}%
\definecolor{currentstroke}{rgb}{0.000000,0.000000,0.000000}%
\pgfsetstrokecolor{currentstroke}%
\pgfsetdash{}{0pt}%
\pgfpathmoveto{\pgfqpoint{3.520817in}{3.490548in}}%
\pgfpathlineto{\pgfqpoint{3.530974in}{3.472225in}}%
\pgfpathlineto{\pgfqpoint{3.541166in}{3.451143in}}%
\pgfpathlineto{\pgfqpoint{3.575701in}{3.484264in}}%
\pgfpathlineto{\pgfqpoint{3.610243in}{3.515556in}}%
\pgfpathlineto{\pgfqpoint{3.599991in}{3.536554in}}%
\pgfpathlineto{\pgfqpoint{3.589773in}{3.554530in}}%
\pgfpathlineto{\pgfqpoint{3.555291in}{3.523454in}}%
\pgfpathlineto{\pgfqpoint{3.520817in}{3.490548in}}%
\pgfpathclose%
\pgfusepath{fill}%
\end{pgfscope}%
\begin{pgfscope}%
\pgfpathrectangle{\pgfqpoint{1.020000in}{0.880000in}}{\pgfqpoint{6.160000in}{6.160000in}}%
\pgfusepath{clip}%
\pgfsetbuttcap%
\pgfsetroundjoin%
\definecolor{currentfill}{rgb}{0.363461,0.484784,0.901019}%
\pgfsetfillcolor{currentfill}%
\pgfsetlinewidth{0.000000pt}%
\definecolor{currentstroke}{rgb}{0.000000,0.000000,0.000000}%
\pgfsetstrokecolor{currentstroke}%
\pgfsetdash{}{0pt}%
\pgfpathmoveto{\pgfqpoint{4.886183in}{2.683123in}}%
\pgfpathlineto{\pgfqpoint{4.897544in}{2.655533in}}%
\pgfpathlineto{\pgfqpoint{4.908925in}{2.628227in}}%
\pgfpathlineto{\pgfqpoint{4.943115in}{2.617187in}}%
\pgfpathlineto{\pgfqpoint{4.977279in}{2.606929in}}%
\pgfpathlineto{\pgfqpoint{4.965834in}{2.632050in}}%
\pgfpathlineto{\pgfqpoint{4.954411in}{2.657415in}}%
\pgfpathlineto{\pgfqpoint{4.920310in}{2.669816in}}%
\pgfpathlineto{\pgfqpoint{4.886183in}{2.683123in}}%
\pgfpathclose%
\pgfusepath{fill}%
\end{pgfscope}%
\begin{pgfscope}%
\pgfpathrectangle{\pgfqpoint{1.020000in}{0.880000in}}{\pgfqpoint{6.160000in}{6.160000in}}%
\pgfusepath{clip}%
\pgfsetbuttcap%
\pgfsetroundjoin%
\definecolor{currentfill}{rgb}{0.786721,0.844807,0.939810}%
\pgfsetfillcolor{currentfill}%
\pgfsetlinewidth{0.000000pt}%
\definecolor{currentstroke}{rgb}{0.000000,0.000000,0.000000}%
\pgfsetstrokecolor{currentstroke}%
\pgfsetdash{}{0pt}%
\pgfpathmoveto{\pgfqpoint{3.451891in}{3.420603in}}%
\pgfpathlineto{\pgfqpoint{3.461987in}{3.401925in}}%
\pgfpathlineto{\pgfqpoint{3.472118in}{3.380774in}}%
\pgfpathlineto{\pgfqpoint{3.506639in}{3.416532in}}%
\pgfpathlineto{\pgfqpoint{3.541166in}{3.451143in}}%
\pgfpathlineto{\pgfqpoint{3.530974in}{3.472225in}}%
\pgfpathlineto{\pgfqpoint{3.520817in}{3.490548in}}%
\pgfpathlineto{\pgfqpoint{3.486351in}{3.456150in}}%
\pgfpathlineto{\pgfqpoint{3.451891in}{3.420603in}}%
\pgfpathclose%
\pgfusepath{fill}%
\end{pgfscope}%
\begin{pgfscope}%
\pgfpathrectangle{\pgfqpoint{1.020000in}{0.880000in}}{\pgfqpoint{6.160000in}{6.160000in}}%
\pgfusepath{clip}%
\pgfsetbuttcap%
\pgfsetroundjoin%
\definecolor{currentfill}{rgb}{0.677823,0.786546,0.991005}%
\pgfsetfillcolor{currentfill}%
\pgfsetlinewidth{0.000000pt}%
\definecolor{currentstroke}{rgb}{0.000000,0.000000,0.000000}%
\pgfsetstrokecolor{currentstroke}%
\pgfsetdash{}{0pt}%
\pgfpathmoveto{\pgfqpoint{3.245171in}{3.201396in}}%
\pgfpathlineto{\pgfqpoint{3.255085in}{3.181671in}}%
\pgfpathlineto{\pgfqpoint{3.265030in}{3.160361in}}%
\pgfpathlineto{\pgfqpoint{3.299551in}{3.196301in}}%
\pgfpathlineto{\pgfqpoint{3.334066in}{3.232924in}}%
\pgfpathlineto{\pgfqpoint{3.324059in}{3.254185in}}%
\pgfpathlineto{\pgfqpoint{3.314085in}{3.273570in}}%
\pgfpathlineto{\pgfqpoint{3.279631in}{3.237143in}}%
\pgfpathlineto{\pgfqpoint{3.245171in}{3.201396in}}%
\pgfpathclose%
\pgfusepath{fill}%
\end{pgfscope}%
\begin{pgfscope}%
\pgfpathrectangle{\pgfqpoint{1.020000in}{0.880000in}}{\pgfqpoint{6.160000in}{6.160000in}}%
\pgfusepath{clip}%
\pgfsetbuttcap%
\pgfsetroundjoin%
\definecolor{currentfill}{rgb}{0.713852,0.808857,0.979386}%
\pgfsetfillcolor{currentfill}%
\pgfsetlinewidth{0.000000pt}%
\definecolor{currentstroke}{rgb}{0.000000,0.000000,0.000000}%
\pgfsetstrokecolor{currentstroke}%
\pgfsetdash{}{0pt}%
\pgfpathmoveto{\pgfqpoint{3.314085in}{3.273570in}}%
\pgfpathlineto{\pgfqpoint{3.324059in}{3.254185in}}%
\pgfpathlineto{\pgfqpoint{3.334066in}{3.232924in}}%
\pgfpathlineto{\pgfqpoint{3.368578in}{3.269973in}}%
\pgfpathlineto{\pgfqpoint{3.403089in}{3.307169in}}%
\pgfpathlineto{\pgfqpoint{3.393020in}{3.328378in}}%
\pgfpathlineto{\pgfqpoint{3.382985in}{3.347413in}}%
\pgfpathlineto{\pgfqpoint{3.348535in}{3.310420in}}%
\pgfpathlineto{\pgfqpoint{3.314085in}{3.273570in}}%
\pgfpathclose%
\pgfusepath{fill}%
\end{pgfscope}%
\begin{pgfscope}%
\pgfpathrectangle{\pgfqpoint{1.020000in}{0.880000in}}{\pgfqpoint{6.160000in}{6.160000in}}%
\pgfusepath{clip}%
\pgfsetbuttcap%
\pgfsetroundjoin%
\definecolor{currentfill}{rgb}{0.875557,0.860242,0.851430}%
\pgfsetfillcolor{currentfill}%
\pgfsetlinewidth{0.000000pt}%
\definecolor{currentstroke}{rgb}{0.000000,0.000000,0.000000}%
\pgfsetstrokecolor{currentstroke}%
\pgfsetdash{}{0pt}%
\pgfpathmoveto{\pgfqpoint{3.886738in}{3.664861in}}%
\pgfpathlineto{\pgfqpoint{3.897251in}{3.640756in}}%
\pgfpathlineto{\pgfqpoint{3.907791in}{3.613056in}}%
\pgfpathlineto{\pgfqpoint{3.942392in}{3.616088in}}%
\pgfpathlineto{\pgfqpoint{3.976977in}{3.615341in}}%
\pgfpathlineto{\pgfqpoint{3.966387in}{3.642863in}}%
\pgfpathlineto{\pgfqpoint{3.955821in}{3.666785in}}%
\pgfpathlineto{\pgfqpoint{3.921286in}{3.667743in}}%
\pgfpathlineto{\pgfqpoint{3.886738in}{3.664861in}}%
\pgfpathclose%
\pgfusepath{fill}%
\end{pgfscope}%
\begin{pgfscope}%
\pgfpathrectangle{\pgfqpoint{1.020000in}{0.880000in}}{\pgfqpoint{6.160000in}{6.160000in}}%
\pgfusepath{clip}%
\pgfsetbuttcap%
\pgfsetroundjoin%
\definecolor{currentfill}{rgb}{0.753611,0.830233,0.960871}%
\pgfsetfillcolor{currentfill}%
\pgfsetlinewidth{0.000000pt}%
\definecolor{currentstroke}{rgb}{0.000000,0.000000,0.000000}%
\pgfsetstrokecolor{currentstroke}%
\pgfsetdash{}{0pt}%
\pgfpathmoveto{\pgfqpoint{3.382985in}{3.347413in}}%
\pgfpathlineto{\pgfqpoint{3.393020in}{3.328378in}}%
\pgfpathlineto{\pgfqpoint{3.403089in}{3.307169in}}%
\pgfpathlineto{\pgfqpoint{3.437601in}{3.344209in}}%
\pgfpathlineto{\pgfqpoint{3.472118in}{3.380774in}}%
\pgfpathlineto{\pgfqpoint{3.461987in}{3.401925in}}%
\pgfpathlineto{\pgfqpoint{3.451891in}{3.420603in}}%
\pgfpathlineto{\pgfqpoint{3.417436in}{3.384247in}}%
\pgfpathlineto{\pgfqpoint{3.382985in}{3.347413in}}%
\pgfpathclose%
\pgfusepath{fill}%
\end{pgfscope}%
\begin{pgfscope}%
\pgfpathrectangle{\pgfqpoint{1.020000in}{0.880000in}}{\pgfqpoint{6.160000in}{6.160000in}}%
\pgfusepath{clip}%
\pgfsetbuttcap%
\pgfsetroundjoin%
\definecolor{currentfill}{rgb}{0.318832,0.426605,0.859857}%
\pgfsetfillcolor{currentfill}%
\pgfsetlinewidth{0.000000pt}%
\definecolor{currentstroke}{rgb}{0.000000,0.000000,0.000000}%
\pgfsetstrokecolor{currentstroke}%
\pgfsetdash{}{0pt}%
\pgfpathmoveto{\pgfqpoint{5.045538in}{2.588849in}}%
\pgfpathlineto{\pgfqpoint{5.057070in}{2.565919in}}%
\pgfpathlineto{\pgfqpoint{5.068625in}{2.543292in}}%
\pgfpathlineto{\pgfqpoint{5.102786in}{2.537158in}}%
\pgfpathlineto{\pgfqpoint{5.136926in}{2.531719in}}%
\pgfpathlineto{\pgfqpoint{5.125305in}{2.552764in}}%
\pgfpathlineto{\pgfqpoint{5.113708in}{2.574062in}}%
\pgfpathlineto{\pgfqpoint{5.079633in}{2.581046in}}%
\pgfpathlineto{\pgfqpoint{5.045538in}{2.588849in}}%
\pgfpathclose%
\pgfusepath{fill}%
\end{pgfscope}%
\begin{pgfscope}%
\pgfpathrectangle{\pgfqpoint{1.020000in}{0.880000in}}{\pgfqpoint{6.160000in}{6.160000in}}%
\pgfusepath{clip}%
\pgfsetbuttcap%
\pgfsetroundjoin%
\definecolor{currentfill}{rgb}{0.640828,0.760752,0.997846}%
\pgfsetfillcolor{currentfill}%
\pgfsetlinewidth{0.000000pt}%
\definecolor{currentstroke}{rgb}{0.000000,0.000000,0.000000}%
\pgfsetstrokecolor{currentstroke}%
\pgfsetdash{}{0pt}%
\pgfpathmoveto{\pgfqpoint{3.176222in}{3.132833in}}%
\pgfpathlineto{\pgfqpoint{3.186076in}{3.112783in}}%
\pgfpathlineto{\pgfqpoint{3.195960in}{3.091421in}}%
\pgfpathlineto{\pgfqpoint{3.230500in}{3.125334in}}%
\pgfpathlineto{\pgfqpoint{3.265030in}{3.160361in}}%
\pgfpathlineto{\pgfqpoint{3.255085in}{3.181671in}}%
\pgfpathlineto{\pgfqpoint{3.245171in}{3.201396in}}%
\pgfpathlineto{\pgfqpoint{3.210702in}{3.166559in}}%
\pgfpathlineto{\pgfqpoint{3.176222in}{3.132833in}}%
\pgfpathclose%
\pgfusepath{fill}%
\end{pgfscope}%
\begin{pgfscope}%
\pgfpathrectangle{\pgfqpoint{1.020000in}{0.880000in}}{\pgfqpoint{6.160000in}{6.160000in}}%
\pgfusepath{clip}%
\pgfsetbuttcap%
\pgfsetroundjoin%
\definecolor{currentfill}{rgb}{0.843358,0.861820,0.890017}%
\pgfsetfillcolor{currentfill}%
\pgfsetlinewidth{0.000000pt}%
\definecolor{currentstroke}{rgb}{0.000000,0.000000,0.000000}%
\pgfsetstrokecolor{currentstroke}%
\pgfsetdash{}{0pt}%
\pgfpathmoveto{\pgfqpoint{4.046092in}{3.602663in}}%
\pgfpathlineto{\pgfqpoint{4.056753in}{3.572018in}}%
\pgfpathlineto{\pgfqpoint{4.067434in}{3.538071in}}%
\pgfpathlineto{\pgfqpoint{4.102000in}{3.526913in}}%
\pgfpathlineto{\pgfqpoint{4.136536in}{3.512467in}}%
\pgfpathlineto{\pgfqpoint{4.125811in}{3.545732in}}%
\pgfpathlineto{\pgfqpoint{4.115105in}{3.575803in}}%
\pgfpathlineto{\pgfqpoint{4.080613in}{3.590930in}}%
\pgfpathlineto{\pgfqpoint{4.046092in}{3.602663in}}%
\pgfpathclose%
\pgfusepath{fill}%
\end{pgfscope}%
\begin{pgfscope}%
\pgfpathrectangle{\pgfqpoint{1.020000in}{0.880000in}}{\pgfqpoint{6.160000in}{6.160000in}}%
\pgfusepath{clip}%
\pgfsetbuttcap%
\pgfsetroundjoin%
\definecolor{currentfill}{rgb}{0.603162,0.731527,0.999565}%
\pgfsetfillcolor{currentfill}%
\pgfsetlinewidth{0.000000pt}%
\definecolor{currentstroke}{rgb}{0.000000,0.000000,0.000000}%
\pgfsetstrokecolor{currentstroke}%
\pgfsetdash{}{0pt}%
\pgfpathmoveto{\pgfqpoint{3.107216in}{3.069378in}}%
\pgfpathlineto{\pgfqpoint{3.117011in}{3.049020in}}%
\pgfpathlineto{\pgfqpoint{3.126833in}{3.027601in}}%
\pgfpathlineto{\pgfqpoint{3.161405in}{3.058796in}}%
\pgfpathlineto{\pgfqpoint{3.195960in}{3.091421in}}%
\pgfpathlineto{\pgfqpoint{3.186076in}{3.112783in}}%
\pgfpathlineto{\pgfqpoint{3.176222in}{3.132833in}}%
\pgfpathlineto{\pgfqpoint{3.141727in}{3.100391in}}%
\pgfpathlineto{\pgfqpoint{3.107216in}{3.069378in}}%
\pgfpathclose%
\pgfusepath{fill}%
\end{pgfscope}%
\begin{pgfscope}%
\pgfpathrectangle{\pgfqpoint{1.020000in}{0.880000in}}{\pgfqpoint{6.160000in}{6.160000in}}%
\pgfusepath{clip}%
\pgfsetbuttcap%
\pgfsetroundjoin%
\definecolor{currentfill}{rgb}{0.646113,0.764436,0.996868}%
\pgfsetfillcolor{currentfill}%
\pgfsetlinewidth{0.000000pt}%
\definecolor{currentstroke}{rgb}{0.000000,0.000000,0.000000}%
\pgfsetstrokecolor{currentstroke}%
\pgfsetdash{}{0pt}%
\pgfpathmoveto{\pgfqpoint{4.386531in}{3.227133in}}%
\pgfpathlineto{\pgfqpoint{4.397450in}{3.186835in}}%
\pgfpathlineto{\pgfqpoint{4.408383in}{3.145178in}}%
\pgfpathlineto{\pgfqpoint{4.442770in}{3.120898in}}%
\pgfpathlineto{\pgfqpoint{4.477116in}{3.095756in}}%
\pgfpathlineto{\pgfqpoint{4.466141in}{3.134932in}}%
\pgfpathlineto{\pgfqpoint{4.455179in}{3.172885in}}%
\pgfpathlineto{\pgfqpoint{4.420876in}{3.200476in}}%
\pgfpathlineto{\pgfqpoint{4.386531in}{3.227133in}}%
\pgfpathclose%
\pgfusepath{fill}%
\end{pgfscope}%
\begin{pgfscope}%
\pgfpathrectangle{\pgfqpoint{1.020000in}{0.880000in}}{\pgfqpoint{6.160000in}{6.160000in}}%
\pgfusepath{clip}%
\pgfsetbuttcap%
\pgfsetroundjoin%
\definecolor{currentfill}{rgb}{0.570616,0.704109,0.997195}%
\pgfsetfillcolor{currentfill}%
\pgfsetlinewidth{0.000000pt}%
\definecolor{currentstroke}{rgb}{0.000000,0.000000,0.000000}%
\pgfsetstrokecolor{currentstroke}%
\pgfsetdash{}{0pt}%
\pgfpathmoveto{\pgfqpoint{3.038132in}{3.012063in}}%
\pgfpathlineto{\pgfqpoint{3.047868in}{2.991414in}}%
\pgfpathlineto{\pgfqpoint{3.057630in}{2.969929in}}%
\pgfpathlineto{\pgfqpoint{3.092243in}{2.997950in}}%
\pgfpathlineto{\pgfqpoint{3.126833in}{3.027601in}}%
\pgfpathlineto{\pgfqpoint{3.117011in}{3.049020in}}%
\pgfpathlineto{\pgfqpoint{3.107216in}{3.069378in}}%
\pgfpathlineto{\pgfqpoint{3.072685in}{3.039906in}}%
\pgfpathlineto{\pgfqpoint{3.038132in}{3.012063in}}%
\pgfpathclose%
\pgfusepath{fill}%
\end{pgfscope}%
\begin{pgfscope}%
\pgfpathrectangle{\pgfqpoint{1.020000in}{0.880000in}}{\pgfqpoint{6.160000in}{6.160000in}}%
\pgfusepath{clip}%
\pgfsetbuttcap%
\pgfsetroundjoin%
\definecolor{currentfill}{rgb}{0.425199,0.559058,0.946061}%
\pgfsetfillcolor{currentfill}%
\pgfsetlinewidth{0.000000pt}%
\definecolor{currentstroke}{rgb}{0.000000,0.000000,0.000000}%
\pgfsetstrokecolor{currentstroke}%
\pgfsetdash{}{0pt}%
\pgfpathmoveto{\pgfqpoint{4.726978in}{2.810327in}}%
\pgfpathlineto{\pgfqpoint{4.738180in}{2.777233in}}%
\pgfpathlineto{\pgfqpoint{4.749400in}{2.744263in}}%
\pgfpathlineto{\pgfqpoint{4.783640in}{2.727946in}}%
\pgfpathlineto{\pgfqpoint{4.817850in}{2.712258in}}%
\pgfpathlineto{\pgfqpoint{4.806571in}{2.742536in}}%
\pgfpathlineto{\pgfqpoint{4.795311in}{2.772929in}}%
\pgfpathlineto{\pgfqpoint{4.761160in}{2.791269in}}%
\pgfpathlineto{\pgfqpoint{4.726978in}{2.810327in}}%
\pgfpathclose%
\pgfusepath{fill}%
\end{pgfscope}%
\begin{pgfscope}%
\pgfpathrectangle{\pgfqpoint{1.020000in}{0.880000in}}{\pgfqpoint{6.160000in}{6.160000in}}%
\pgfusepath{clip}%
\pgfsetbuttcap%
\pgfsetroundjoin%
\definecolor{currentfill}{rgb}{0.538004,0.674902,0.991722}%
\pgfsetfillcolor{currentfill}%
\pgfsetlinewidth{0.000000pt}%
\definecolor{currentstroke}{rgb}{0.000000,0.000000,0.000000}%
\pgfsetstrokecolor{currentstroke}%
\pgfsetdash{}{0pt}%
\pgfpathmoveto{\pgfqpoint{2.968950in}{2.961473in}}%
\pgfpathlineto{\pgfqpoint{2.978629in}{2.940550in}}%
\pgfpathlineto{\pgfqpoint{2.988332in}{2.918989in}}%
\pgfpathlineto{\pgfqpoint{3.022994in}{2.943598in}}%
\pgfpathlineto{\pgfqpoint{3.057630in}{2.969929in}}%
\pgfpathlineto{\pgfqpoint{3.047868in}{2.991414in}}%
\pgfpathlineto{\pgfqpoint{3.038132in}{3.012063in}}%
\pgfpathlineto{\pgfqpoint{3.003554in}{2.985908in}}%
\pgfpathlineto{\pgfqpoint{2.968950in}{2.961473in}}%
\pgfpathclose%
\pgfusepath{fill}%
\end{pgfscope}%
\begin{pgfscope}%
\pgfpathrectangle{\pgfqpoint{1.020000in}{0.880000in}}{\pgfqpoint{6.160000in}{6.160000in}}%
\pgfusepath{clip}%
\pgfsetbuttcap%
\pgfsetroundjoin%
\definecolor{currentfill}{rgb}{0.576051,0.708780,0.997755}%
\pgfsetfillcolor{currentfill}%
\pgfsetlinewidth{0.000000pt}%
\definecolor{currentstroke}{rgb}{0.000000,0.000000,0.000000}%
\pgfsetstrokecolor{currentstroke}%
\pgfsetdash{}{0pt}%
\pgfpathmoveto{\pgfqpoint{4.477116in}{3.095756in}}%
\pgfpathlineto{\pgfqpoint{4.488104in}{3.055593in}}%
\pgfpathlineto{\pgfqpoint{4.499106in}{3.014681in}}%
\pgfpathlineto{\pgfqpoint{4.533455in}{2.991673in}}%
\pgfpathlineto{\pgfqpoint{4.567765in}{2.968357in}}%
\pgfpathlineto{\pgfqpoint{4.556717in}{3.006480in}}%
\pgfpathlineto{\pgfqpoint{4.545683in}{3.043945in}}%
\pgfpathlineto{\pgfqpoint{4.511420in}{3.070019in}}%
\pgfpathlineto{\pgfqpoint{4.477116in}{3.095756in}}%
\pgfpathclose%
\pgfusepath{fill}%
\end{pgfscope}%
\begin{pgfscope}%
\pgfpathrectangle{\pgfqpoint{1.020000in}{0.880000in}}{\pgfqpoint{6.160000in}{6.160000in}}%
\pgfusepath{clip}%
\pgfsetbuttcap%
\pgfsetroundjoin%
\definecolor{currentfill}{rgb}{0.304174,0.406945,0.845263}%
\pgfsetfillcolor{currentfill}%
\pgfsetlinewidth{0.000000pt}%
\definecolor{currentstroke}{rgb}{0.000000,0.000000,0.000000}%
\pgfsetstrokecolor{currentstroke}%
\pgfsetdash{}{0pt}%
\pgfpathmoveto{\pgfqpoint{5.136926in}{2.531719in}}%
\pgfpathlineto{\pgfqpoint{5.148571in}{2.510957in}}%
\pgfpathlineto{\pgfqpoint{5.182722in}{2.506876in}}%
\pgfpathlineto{\pgfqpoint{5.216854in}{2.503402in}}%
\pgfpathlineto{\pgfqpoint{5.205144in}{2.522869in}}%
\pgfpathlineto{\pgfqpoint{5.171045in}{2.526962in}}%
\pgfpathlineto{\pgfqpoint{5.136926in}{2.531719in}}%
\pgfpathclose%
\pgfusepath{fill}%
\end{pgfscope}%
\begin{pgfscope}%
\pgfpathrectangle{\pgfqpoint{1.020000in}{0.880000in}}{\pgfqpoint{6.160000in}{6.160000in}}%
\pgfusepath{clip}%
\pgfsetbuttcap%
\pgfsetroundjoin%
\definecolor{currentfill}{rgb}{0.875557,0.860242,0.851430}%
\pgfsetfillcolor{currentfill}%
\pgfsetlinewidth{0.000000pt}%
\definecolor{currentstroke}{rgb}{0.000000,0.000000,0.000000}%
\pgfsetstrokecolor{currentstroke}%
\pgfsetdash{}{0pt}%
\pgfpathmoveto{\pgfqpoint{3.817615in}{3.647711in}}%
\pgfpathlineto{\pgfqpoint{3.828074in}{3.623508in}}%
\pgfpathlineto{\pgfqpoint{3.838562in}{3.595779in}}%
\pgfpathlineto{\pgfqpoint{3.873180in}{3.606264in}}%
\pgfpathlineto{\pgfqpoint{3.907791in}{3.613056in}}%
\pgfpathlineto{\pgfqpoint{3.897251in}{3.640756in}}%
\pgfpathlineto{\pgfqpoint{3.886738in}{3.664861in}}%
\pgfpathlineto{\pgfqpoint{3.852179in}{3.658161in}}%
\pgfpathlineto{\pgfqpoint{3.817615in}{3.647711in}}%
\pgfpathclose%
\pgfusepath{fill}%
\end{pgfscope}%
\begin{pgfscope}%
\pgfpathrectangle{\pgfqpoint{1.020000in}{0.880000in}}{\pgfqpoint{6.160000in}{6.160000in}}%
\pgfusepath{clip}%
\pgfsetbuttcap%
\pgfsetroundjoin%
\definecolor{currentfill}{rgb}{0.800601,0.850358,0.930008}%
\pgfsetfillcolor{currentfill}%
\pgfsetlinewidth{0.000000pt}%
\definecolor{currentstroke}{rgb}{0.000000,0.000000,0.000000}%
\pgfsetstrokecolor{currentstroke}%
\pgfsetdash{}{0pt}%
\pgfpathmoveto{\pgfqpoint{4.136536in}{3.512467in}}%
\pgfpathlineto{\pgfqpoint{4.147278in}{3.476189in}}%
\pgfpathlineto{\pgfqpoint{4.158038in}{3.437114in}}%
\pgfpathlineto{\pgfqpoint{4.192582in}{3.420636in}}%
\pgfpathlineto{\pgfqpoint{4.227091in}{3.401434in}}%
\pgfpathlineto{\pgfqpoint{4.216292in}{3.439314in}}%
\pgfpathlineto{\pgfqpoint{4.205508in}{3.474536in}}%
\pgfpathlineto{\pgfqpoint{4.171039in}{3.494931in}}%
\pgfpathlineto{\pgfqpoint{4.136536in}{3.512467in}}%
\pgfpathclose%
\pgfusepath{fill}%
\end{pgfscope}%
\begin{pgfscope}%
\pgfpathrectangle{\pgfqpoint{1.020000in}{0.880000in}}{\pgfqpoint{6.160000in}{6.160000in}}%
\pgfusepath{clip}%
\pgfsetbuttcap%
\pgfsetroundjoin%
\definecolor{currentfill}{rgb}{0.516260,0.654498,0.986407}%
\pgfsetfillcolor{currentfill}%
\pgfsetlinewidth{0.000000pt}%
\definecolor{currentstroke}{rgb}{0.000000,0.000000,0.000000}%
\pgfsetstrokecolor{currentstroke}%
\pgfsetdash{}{0pt}%
\pgfpathmoveto{\pgfqpoint{2.899657in}{2.917790in}}%
\pgfpathlineto{\pgfqpoint{2.909278in}{2.896609in}}%
\pgfpathlineto{\pgfqpoint{2.918922in}{2.874958in}}%
\pgfpathlineto{\pgfqpoint{2.953642in}{2.896112in}}%
\pgfpathlineto{\pgfqpoint{2.988332in}{2.918989in}}%
\pgfpathlineto{\pgfqpoint{2.978629in}{2.940550in}}%
\pgfpathlineto{\pgfqpoint{2.968950in}{2.961473in}}%
\pgfpathlineto{\pgfqpoint{2.934319in}{2.938771in}}%
\pgfpathlineto{\pgfqpoint{2.899657in}{2.917790in}}%
\pgfpathclose%
\pgfusepath{fill}%
\end{pgfscope}%
\begin{pgfscope}%
\pgfpathrectangle{\pgfqpoint{1.020000in}{0.880000in}}{\pgfqpoint{6.160000in}{6.160000in}}%
\pgfusepath{clip}%
\pgfsetbuttcap%
\pgfsetroundjoin%
\definecolor{currentfill}{rgb}{0.510824,0.649397,0.985079}%
\pgfsetfillcolor{currentfill}%
\pgfsetlinewidth{0.000000pt}%
\definecolor{currentstroke}{rgb}{0.000000,0.000000,0.000000}%
\pgfsetstrokecolor{currentstroke}%
\pgfsetdash{}{0pt}%
\pgfpathmoveto{\pgfqpoint{4.567765in}{2.968357in}}%
\pgfpathlineto{\pgfqpoint{4.578828in}{2.929785in}}%
\pgfpathlineto{\pgfqpoint{4.589907in}{2.890969in}}%
\pgfpathlineto{\pgfqpoint{4.624228in}{2.870466in}}%
\pgfpathlineto{\pgfqpoint{4.658513in}{2.850077in}}%
\pgfpathlineto{\pgfqpoint{4.647383in}{2.885986in}}%
\pgfpathlineto{\pgfqpoint{4.636270in}{2.921691in}}%
\pgfpathlineto{\pgfqpoint{4.602036in}{2.944959in}}%
\pgfpathlineto{\pgfqpoint{4.567765in}{2.968357in}}%
\pgfpathclose%
\pgfusepath{fill}%
\end{pgfscope}%
\begin{pgfscope}%
\pgfpathrectangle{\pgfqpoint{1.020000in}{0.880000in}}{\pgfqpoint{6.160000in}{6.160000in}}%
\pgfusepath{clip}%
\pgfsetbuttcap%
\pgfsetroundjoin%
\definecolor{currentfill}{rgb}{0.333490,0.446265,0.874452}%
\pgfsetfillcolor{currentfill}%
\pgfsetlinewidth{0.000000pt}%
\definecolor{currentstroke}{rgb}{0.000000,0.000000,0.000000}%
\pgfsetstrokecolor{currentstroke}%
\pgfsetdash{}{0pt}%
\pgfpathmoveto{\pgfqpoint{4.977279in}{2.606929in}}%
\pgfpathlineto{\pgfqpoint{4.988747in}{2.582114in}}%
\pgfpathlineto{\pgfqpoint{5.000237in}{2.557660in}}%
\pgfpathlineto{\pgfqpoint{5.034443in}{2.550127in}}%
\pgfpathlineto{\pgfqpoint{5.068625in}{2.543292in}}%
\pgfpathlineto{\pgfqpoint{5.057070in}{2.565919in}}%
\pgfpathlineto{\pgfqpoint{5.045538in}{2.588849in}}%
\pgfpathlineto{\pgfqpoint{5.011420in}{2.597478in}}%
\pgfpathlineto{\pgfqpoint{4.977279in}{2.606929in}}%
\pgfpathclose%
\pgfusepath{fill}%
\end{pgfscope}%
\begin{pgfscope}%
\pgfpathrectangle{\pgfqpoint{1.020000in}{0.880000in}}{\pgfqpoint{6.160000in}{6.160000in}}%
\pgfusepath{clip}%
\pgfsetbuttcap%
\pgfsetroundjoin%
\definecolor{currentfill}{rgb}{0.383662,0.510183,0.917831}%
\pgfsetfillcolor{currentfill}%
\pgfsetlinewidth{0.000000pt}%
\definecolor{currentstroke}{rgb}{0.000000,0.000000,0.000000}%
\pgfsetstrokecolor{currentstroke}%
\pgfsetdash{}{0pt}%
\pgfpathmoveto{\pgfqpoint{4.817850in}{2.712258in}}%
\pgfpathlineto{\pgfqpoint{4.829149in}{2.682204in}}%
\pgfpathlineto{\pgfqpoint{4.840468in}{2.652479in}}%
\pgfpathlineto{\pgfqpoint{4.874710in}{2.640008in}}%
\pgfpathlineto{\pgfqpoint{4.908925in}{2.628227in}}%
\pgfpathlineto{\pgfqpoint{4.897544in}{2.655533in}}%
\pgfpathlineto{\pgfqpoint{4.886183in}{2.683123in}}%
\pgfpathlineto{\pgfqpoint{4.852030in}{2.697291in}}%
\pgfpathlineto{\pgfqpoint{4.817850in}{2.712258in}}%
\pgfpathclose%
\pgfusepath{fill}%
\end{pgfscope}%
\begin{pgfscope}%
\pgfpathrectangle{\pgfqpoint{1.020000in}{0.880000in}}{\pgfqpoint{6.160000in}{6.160000in}}%
\pgfusepath{clip}%
\pgfsetbuttcap%
\pgfsetroundjoin%
\definecolor{currentfill}{rgb}{0.855378,0.863778,0.876587}%
\pgfsetfillcolor{currentfill}%
\pgfsetlinewidth{0.000000pt}%
\definecolor{currentstroke}{rgb}{0.000000,0.000000,0.000000}%
\pgfsetstrokecolor{currentstroke}%
\pgfsetdash{}{0pt}%
\pgfpathmoveto{\pgfqpoint{3.976977in}{3.615341in}}%
\pgfpathlineto{\pgfqpoint{3.987592in}{3.584317in}}%
\pgfpathlineto{\pgfqpoint{3.998229in}{3.549938in}}%
\pgfpathlineto{\pgfqpoint{4.032843in}{3.545785in}}%
\pgfpathlineto{\pgfqpoint{4.067434in}{3.538071in}}%
\pgfpathlineto{\pgfqpoint{4.056753in}{3.572018in}}%
\pgfpathlineto{\pgfqpoint{4.046092in}{3.602663in}}%
\pgfpathlineto{\pgfqpoint{4.011545in}{3.610841in}}%
\pgfpathlineto{\pgfqpoint{3.976977in}{3.615341in}}%
\pgfpathclose%
\pgfusepath{fill}%
\end{pgfscope}%
\begin{pgfscope}%
\pgfpathrectangle{\pgfqpoint{1.020000in}{0.880000in}}{\pgfqpoint{6.160000in}{6.160000in}}%
\pgfusepath{clip}%
\pgfsetbuttcap%
\pgfsetroundjoin%
\definecolor{currentfill}{rgb}{0.494638,0.633022,0.978983}%
\pgfsetfillcolor{currentfill}%
\pgfsetlinewidth{0.000000pt}%
\definecolor{currentstroke}{rgb}{0.000000,0.000000,0.000000}%
\pgfsetstrokecolor{currentstroke}%
\pgfsetdash{}{0pt}%
\pgfpathmoveto{\pgfqpoint{2.830240in}{2.880863in}}%
\pgfpathlineto{\pgfqpoint{2.839805in}{2.859436in}}%
\pgfpathlineto{\pgfqpoint{2.849390in}{2.837682in}}%
\pgfpathlineto{\pgfqpoint{2.884172in}{2.855496in}}%
\pgfpathlineto{\pgfqpoint{2.918922in}{2.874958in}}%
\pgfpathlineto{\pgfqpoint{2.909278in}{2.896609in}}%
\pgfpathlineto{\pgfqpoint{2.899657in}{2.917790in}}%
\pgfpathlineto{\pgfqpoint{2.864964in}{2.898503in}}%
\pgfpathlineto{\pgfqpoint{2.830240in}{2.880863in}}%
\pgfpathclose%
\pgfusepath{fill}%
\end{pgfscope}%
\begin{pgfscope}%
\pgfpathrectangle{\pgfqpoint{1.020000in}{0.880000in}}{\pgfqpoint{6.160000in}{6.160000in}}%
\pgfusepath{clip}%
\pgfsetbuttcap%
\pgfsetroundjoin%
\definecolor{currentfill}{rgb}{0.743754,0.825125,0.965798}%
\pgfsetfillcolor{currentfill}%
\pgfsetlinewidth{0.000000pt}%
\definecolor{currentstroke}{rgb}{0.000000,0.000000,0.000000}%
\pgfsetstrokecolor{currentstroke}%
\pgfsetdash{}{0pt}%
\pgfpathmoveto{\pgfqpoint{4.227091in}{3.401434in}}%
\pgfpathlineto{\pgfqpoint{4.237904in}{3.361132in}}%
\pgfpathlineto{\pgfqpoint{4.248732in}{3.318666in}}%
\pgfpathlineto{\pgfqpoint{4.283242in}{3.298610in}}%
\pgfpathlineto{\pgfqpoint{4.317713in}{3.276486in}}%
\pgfpathlineto{\pgfqpoint{4.306846in}{3.317188in}}%
\pgfpathlineto{\pgfqpoint{4.295993in}{3.355872in}}%
\pgfpathlineto{\pgfqpoint{4.261562in}{3.379756in}}%
\pgfpathlineto{\pgfqpoint{4.227091in}{3.401434in}}%
\pgfpathclose%
\pgfusepath{fill}%
\end{pgfscope}%
\begin{pgfscope}%
\pgfpathrectangle{\pgfqpoint{1.020000in}{0.880000in}}{\pgfqpoint{6.160000in}{6.160000in}}%
\pgfusepath{clip}%
\pgfsetbuttcap%
\pgfsetroundjoin%
\definecolor{currentfill}{rgb}{0.867428,0.864377,0.862602}%
\pgfsetfillcolor{currentfill}%
\pgfsetlinewidth{0.000000pt}%
\definecolor{currentstroke}{rgb}{0.000000,0.000000,0.000000}%
\pgfsetstrokecolor{currentstroke}%
\pgfsetdash{}{0pt}%
\pgfpathmoveto{\pgfqpoint{3.748478in}{3.616061in}}%
\pgfpathlineto{\pgfqpoint{3.758881in}{3.591844in}}%
\pgfpathlineto{\pgfqpoint{3.769316in}{3.564228in}}%
\pgfpathlineto{\pgfqpoint{3.803940in}{3.581715in}}%
\pgfpathlineto{\pgfqpoint{3.838562in}{3.595779in}}%
\pgfpathlineto{\pgfqpoint{3.828074in}{3.623508in}}%
\pgfpathlineto{\pgfqpoint{3.817615in}{3.647711in}}%
\pgfpathlineto{\pgfqpoint{3.783047in}{3.633624in}}%
\pgfpathlineto{\pgfqpoint{3.748478in}{3.616061in}}%
\pgfpathclose%
\pgfusepath{fill}%
\end{pgfscope}%
\begin{pgfscope}%
\pgfpathrectangle{\pgfqpoint{1.020000in}{0.880000in}}{\pgfqpoint{6.160000in}{6.160000in}}%
\pgfusepath{clip}%
\pgfsetbuttcap%
\pgfsetroundjoin%
\definecolor{currentfill}{rgb}{0.451739,0.588181,0.960201}%
\pgfsetfillcolor{currentfill}%
\pgfsetlinewidth{0.000000pt}%
\definecolor{currentstroke}{rgb}{0.000000,0.000000,0.000000}%
\pgfsetstrokecolor{currentstroke}%
\pgfsetdash{}{0pt}%
\pgfpathmoveto{\pgfqpoint{4.658513in}{2.850077in}}%
\pgfpathlineto{\pgfqpoint{4.669659in}{2.814136in}}%
\pgfpathlineto{\pgfqpoint{4.680824in}{2.778328in}}%
\pgfpathlineto{\pgfqpoint{4.715128in}{2.761099in}}%
\pgfpathlineto{\pgfqpoint{4.749400in}{2.744263in}}%
\pgfpathlineto{\pgfqpoint{4.738180in}{2.777233in}}%
\pgfpathlineto{\pgfqpoint{4.726978in}{2.810327in}}%
\pgfpathlineto{\pgfqpoint{4.692762in}{2.829977in}}%
\pgfpathlineto{\pgfqpoint{4.658513in}{2.850077in}}%
\pgfpathclose%
\pgfusepath{fill}%
\end{pgfscope}%
\begin{pgfscope}%
\pgfpathrectangle{\pgfqpoint{1.020000in}{0.880000in}}{\pgfqpoint{6.160000in}{6.160000in}}%
\pgfusepath{clip}%
\pgfsetbuttcap%
\pgfsetroundjoin%
\definecolor{currentfill}{rgb}{0.478462,0.616564,0.972721}%
\pgfsetfillcolor{currentfill}%
\pgfsetlinewidth{0.000000pt}%
\definecolor{currentstroke}{rgb}{0.000000,0.000000,0.000000}%
\pgfsetstrokecolor{currentstroke}%
\pgfsetdash{}{0pt}%
\pgfpathmoveto{\pgfqpoint{2.760690in}{2.850284in}}%
\pgfpathlineto{\pgfqpoint{2.770199in}{2.828623in}}%
\pgfpathlineto{\pgfqpoint{2.779728in}{2.806751in}}%
\pgfpathlineto{\pgfqpoint{2.814576in}{2.821457in}}%
\pgfpathlineto{\pgfqpoint{2.849390in}{2.837682in}}%
\pgfpathlineto{\pgfqpoint{2.839805in}{2.859436in}}%
\pgfpathlineto{\pgfqpoint{2.830240in}{2.880863in}}%
\pgfpathlineto{\pgfqpoint{2.795482in}{2.864813in}}%
\pgfpathlineto{\pgfqpoint{2.760690in}{2.850284in}}%
\pgfpathclose%
\pgfusepath{fill}%
\end{pgfscope}%
\begin{pgfscope}%
\pgfpathrectangle{\pgfqpoint{1.020000in}{0.880000in}}{\pgfqpoint{6.160000in}{6.160000in}}%
\pgfusepath{clip}%
\pgfsetbuttcap%
\pgfsetroundjoin%
\definecolor{currentfill}{rgb}{0.677823,0.786546,0.991005}%
\pgfsetfillcolor{currentfill}%
\pgfsetlinewidth{0.000000pt}%
\definecolor{currentstroke}{rgb}{0.000000,0.000000,0.000000}%
\pgfsetstrokecolor{currentstroke}%
\pgfsetdash{}{0pt}%
\pgfpathmoveto{\pgfqpoint{4.317713in}{3.276486in}}%
\pgfpathlineto{\pgfqpoint{4.328592in}{3.234027in}}%
\pgfpathlineto{\pgfqpoint{4.339485in}{3.190086in}}%
\pgfpathlineto{\pgfqpoint{4.373954in}{3.168330in}}%
\pgfpathlineto{\pgfqpoint{4.408383in}{3.145178in}}%
\pgfpathlineto{\pgfqpoint{4.397450in}{3.186835in}}%
\pgfpathlineto{\pgfqpoint{4.386531in}{3.227133in}}%
\pgfpathlineto{\pgfqpoint{4.352143in}{3.252566in}}%
\pgfpathlineto{\pgfqpoint{4.317713in}{3.276486in}}%
\pgfpathclose%
\pgfusepath{fill}%
\end{pgfscope}%
\begin{pgfscope}%
\pgfpathrectangle{\pgfqpoint{1.020000in}{0.880000in}}{\pgfqpoint{6.160000in}{6.160000in}}%
\pgfusepath{clip}%
\pgfsetbuttcap%
\pgfsetroundjoin%
\definecolor{currentfill}{rgb}{0.313946,0.420052,0.854993}%
\pgfsetfillcolor{currentfill}%
\pgfsetlinewidth{0.000000pt}%
\definecolor{currentstroke}{rgb}{0.000000,0.000000,0.000000}%
\pgfsetstrokecolor{currentstroke}%
\pgfsetdash{}{0pt}%
\pgfpathmoveto{\pgfqpoint{5.068625in}{2.543292in}}%
\pgfpathlineto{\pgfqpoint{5.080205in}{2.521006in}}%
\pgfpathlineto{\pgfqpoint{5.114398in}{2.515663in}}%
\pgfpathlineto{\pgfqpoint{5.148571in}{2.510957in}}%
\pgfpathlineto{\pgfqpoint{5.136926in}{2.531719in}}%
\pgfpathlineto{\pgfqpoint{5.102786in}{2.537158in}}%
\pgfpathlineto{\pgfqpoint{5.068625in}{2.543292in}}%
\pgfpathclose%
\pgfusepath{fill}%
\end{pgfscope}%
\begin{pgfscope}%
\pgfpathrectangle{\pgfqpoint{1.020000in}{0.880000in}}{\pgfqpoint{6.160000in}{6.160000in}}%
\pgfusepath{clip}%
\pgfsetbuttcap%
\pgfsetroundjoin%
\definecolor{currentfill}{rgb}{0.851372,0.863125,0.881064}%
\pgfsetfillcolor{currentfill}%
\pgfsetlinewidth{0.000000pt}%
\definecolor{currentstroke}{rgb}{0.000000,0.000000,0.000000}%
\pgfsetstrokecolor{currentstroke}%
\pgfsetdash{}{0pt}%
\pgfpathmoveto{\pgfqpoint{3.679350in}{3.571339in}}%
\pgfpathlineto{\pgfqpoint{3.689695in}{3.547183in}}%
\pgfpathlineto{\pgfqpoint{3.700073in}{3.519809in}}%
\pgfpathlineto{\pgfqpoint{3.734693in}{3.543515in}}%
\pgfpathlineto{\pgfqpoint{3.769316in}{3.564228in}}%
\pgfpathlineto{\pgfqpoint{3.758881in}{3.591844in}}%
\pgfpathlineto{\pgfqpoint{3.748478in}{3.616061in}}%
\pgfpathlineto{\pgfqpoint{3.713912in}{3.595221in}}%
\pgfpathlineto{\pgfqpoint{3.679350in}{3.571339in}}%
\pgfpathclose%
\pgfusepath{fill}%
\end{pgfscope}%
\begin{pgfscope}%
\pgfpathrectangle{\pgfqpoint{1.020000in}{0.880000in}}{\pgfqpoint{6.160000in}{6.160000in}}%
\pgfusepath{clip}%
\pgfsetbuttcap%
\pgfsetroundjoin%
\definecolor{currentfill}{rgb}{0.818056,0.855590,0.914638}%
\pgfsetfillcolor{currentfill}%
\pgfsetlinewidth{0.000000pt}%
\definecolor{currentstroke}{rgb}{0.000000,0.000000,0.000000}%
\pgfsetstrokecolor{currentstroke}%
\pgfsetdash{}{0pt}%
\pgfpathmoveto{\pgfqpoint{4.067434in}{3.538071in}}%
\pgfpathlineto{\pgfqpoint{4.078135in}{3.501008in}}%
\pgfpathlineto{\pgfqpoint{4.088855in}{3.461051in}}%
\pgfpathlineto{\pgfqpoint{4.123461in}{3.450648in}}%
\pgfpathlineto{\pgfqpoint{4.158038in}{3.437114in}}%
\pgfpathlineto{\pgfqpoint{4.147278in}{3.476189in}}%
\pgfpathlineto{\pgfqpoint{4.136536in}{3.512467in}}%
\pgfpathlineto{\pgfqpoint{4.102000in}{3.526913in}}%
\pgfpathlineto{\pgfqpoint{4.067434in}{3.538071in}}%
\pgfpathclose%
\pgfusepath{fill}%
\end{pgfscope}%
\begin{pgfscope}%
\pgfpathrectangle{\pgfqpoint{1.020000in}{0.880000in}}{\pgfqpoint{6.160000in}{6.160000in}}%
\pgfusepath{clip}%
\pgfsetbuttcap%
\pgfsetroundjoin%
\definecolor{currentfill}{rgb}{0.831148,0.859513,0.903110}%
\pgfsetfillcolor{currentfill}%
\pgfsetlinewidth{0.000000pt}%
\definecolor{currentstroke}{rgb}{0.000000,0.000000,0.000000}%
\pgfsetstrokecolor{currentstroke}%
\pgfsetdash{}{0pt}%
\pgfpathmoveto{\pgfqpoint{3.610243in}{3.515556in}}%
\pgfpathlineto{\pgfqpoint{3.620530in}{3.491526in}}%
\pgfpathlineto{\pgfqpoint{3.630850in}{3.464503in}}%
\pgfpathlineto{\pgfqpoint{3.665459in}{3.493375in}}%
\pgfpathlineto{\pgfqpoint{3.700073in}{3.519809in}}%
\pgfpathlineto{\pgfqpoint{3.689695in}{3.547183in}}%
\pgfpathlineto{\pgfqpoint{3.679350in}{3.571339in}}%
\pgfpathlineto{\pgfqpoint{3.644793in}{3.544686in}}%
\pgfpathlineto{\pgfqpoint{3.610243in}{3.515556in}}%
\pgfpathclose%
\pgfusepath{fill}%
\end{pgfscope}%
\begin{pgfscope}%
\pgfpathrectangle{\pgfqpoint{1.020000in}{0.880000in}}{\pgfqpoint{6.160000in}{6.160000in}}%
\pgfusepath{clip}%
\pgfsetbuttcap%
\pgfsetroundjoin%
\definecolor{currentfill}{rgb}{0.608547,0.735725,0.999354}%
\pgfsetfillcolor{currentfill}%
\pgfsetlinewidth{0.000000pt}%
\definecolor{currentstroke}{rgb}{0.000000,0.000000,0.000000}%
\pgfsetstrokecolor{currentstroke}%
\pgfsetdash{}{0pt}%
\pgfpathmoveto{\pgfqpoint{4.408383in}{3.145178in}}%
\pgfpathlineto{\pgfqpoint{4.419328in}{3.102424in}}%
\pgfpathlineto{\pgfqpoint{4.430287in}{3.058836in}}%
\pgfpathlineto{\pgfqpoint{4.464716in}{3.037149in}}%
\pgfpathlineto{\pgfqpoint{4.499106in}{3.014681in}}%
\pgfpathlineto{\pgfqpoint{4.488104in}{3.055593in}}%
\pgfpathlineto{\pgfqpoint{4.477116in}{3.095756in}}%
\pgfpathlineto{\pgfqpoint{4.442770in}{3.120898in}}%
\pgfpathlineto{\pgfqpoint{4.408383in}{3.145178in}}%
\pgfpathclose%
\pgfusepath{fill}%
\end{pgfscope}%
\begin{pgfscope}%
\pgfpathrectangle{\pgfqpoint{1.020000in}{0.880000in}}{\pgfqpoint{6.160000in}{6.160000in}}%
\pgfusepath{clip}%
\pgfsetbuttcap%
\pgfsetroundjoin%
\definecolor{currentfill}{rgb}{0.859385,0.864431,0.872111}%
\pgfsetfillcolor{currentfill}%
\pgfsetlinewidth{0.000000pt}%
\definecolor{currentstroke}{rgb}{0.000000,0.000000,0.000000}%
\pgfsetstrokecolor{currentstroke}%
\pgfsetdash{}{0pt}%
\pgfpathmoveto{\pgfqpoint{3.907791in}{3.613056in}}%
\pgfpathlineto{\pgfqpoint{3.918358in}{3.581861in}}%
\pgfpathlineto{\pgfqpoint{3.928949in}{3.547314in}}%
\pgfpathlineto{\pgfqpoint{3.963597in}{3.550456in}}%
\pgfpathlineto{\pgfqpoint{3.998229in}{3.549938in}}%
\pgfpathlineto{\pgfqpoint{3.987592in}{3.584317in}}%
\pgfpathlineto{\pgfqpoint{3.976977in}{3.615341in}}%
\pgfpathlineto{\pgfqpoint{3.942392in}{3.616088in}}%
\pgfpathlineto{\pgfqpoint{3.907791in}{3.613056in}}%
\pgfpathclose%
\pgfusepath{fill}%
\end{pgfscope}%
\begin{pgfscope}%
\pgfpathrectangle{\pgfqpoint{1.020000in}{0.880000in}}{\pgfqpoint{6.160000in}{6.160000in}}%
\pgfusepath{clip}%
\pgfsetbuttcap%
\pgfsetroundjoin%
\definecolor{currentfill}{rgb}{0.348323,0.465711,0.888346}%
\pgfsetfillcolor{currentfill}%
\pgfsetlinewidth{0.000000pt}%
\definecolor{currentstroke}{rgb}{0.000000,0.000000,0.000000}%
\pgfsetstrokecolor{currentstroke}%
\pgfsetdash{}{0pt}%
\pgfpathmoveto{\pgfqpoint{4.908925in}{2.628227in}}%
\pgfpathlineto{\pgfqpoint{4.920329in}{2.601280in}}%
\pgfpathlineto{\pgfqpoint{4.931755in}{2.574759in}}%
\pgfpathlineto{\pgfqpoint{4.966008in}{2.565878in}}%
\pgfpathlineto{\pgfqpoint{5.000237in}{2.557660in}}%
\pgfpathlineto{\pgfqpoint{4.988747in}{2.582114in}}%
\pgfpathlineto{\pgfqpoint{4.977279in}{2.606929in}}%
\pgfpathlineto{\pgfqpoint{4.943115in}{2.617187in}}%
\pgfpathlineto{\pgfqpoint{4.908925in}{2.628227in}}%
\pgfpathclose%
\pgfusepath{fill}%
\end{pgfscope}%
\begin{pgfscope}%
\pgfpathrectangle{\pgfqpoint{1.020000in}{0.880000in}}{\pgfqpoint{6.160000in}{6.160000in}}%
\pgfusepath{clip}%
\pgfsetbuttcap%
\pgfsetroundjoin%
\definecolor{currentfill}{rgb}{0.804965,0.851666,0.926165}%
\pgfsetfillcolor{currentfill}%
\pgfsetlinewidth{0.000000pt}%
\definecolor{currentstroke}{rgb}{0.000000,0.000000,0.000000}%
\pgfsetstrokecolor{currentstroke}%
\pgfsetdash{}{0pt}%
\pgfpathmoveto{\pgfqpoint{3.541166in}{3.451143in}}%
\pgfpathlineto{\pgfqpoint{3.551393in}{3.427293in}}%
\pgfpathlineto{\pgfqpoint{3.561652in}{3.400708in}}%
\pgfpathlineto{\pgfqpoint{3.596248in}{3.433506in}}%
\pgfpathlineto{\pgfqpoint{3.630850in}{3.464503in}}%
\pgfpathlineto{\pgfqpoint{3.620530in}{3.491526in}}%
\pgfpathlineto{\pgfqpoint{3.610243in}{3.515556in}}%
\pgfpathlineto{\pgfqpoint{3.575701in}{3.484264in}}%
\pgfpathlineto{\pgfqpoint{3.541166in}{3.451143in}}%
\pgfpathclose%
\pgfusepath{fill}%
\end{pgfscope}%
\begin{pgfscope}%
\pgfpathrectangle{\pgfqpoint{1.020000in}{0.880000in}}{\pgfqpoint{6.160000in}{6.160000in}}%
\pgfusepath{clip}%
\pgfsetbuttcap%
\pgfsetroundjoin%
\definecolor{currentfill}{rgb}{0.777378,0.840921,0.946149}%
\pgfsetfillcolor{currentfill}%
\pgfsetlinewidth{0.000000pt}%
\definecolor{currentstroke}{rgb}{0.000000,0.000000,0.000000}%
\pgfsetstrokecolor{currentstroke}%
\pgfsetdash{}{0pt}%
\pgfpathmoveto{\pgfqpoint{3.472118in}{3.380774in}}%
\pgfpathlineto{\pgfqpoint{3.482282in}{3.357141in}}%
\pgfpathlineto{\pgfqpoint{3.492480in}{3.331053in}}%
\pgfpathlineto{\pgfqpoint{3.527063in}{3.366444in}}%
\pgfpathlineto{\pgfqpoint{3.561652in}{3.400708in}}%
\pgfpathlineto{\pgfqpoint{3.551393in}{3.427293in}}%
\pgfpathlineto{\pgfqpoint{3.541166in}{3.451143in}}%
\pgfpathlineto{\pgfqpoint{3.506639in}{3.416532in}}%
\pgfpathlineto{\pgfqpoint{3.472118in}{3.380774in}}%
\pgfpathclose%
\pgfusepath{fill}%
\end{pgfscope}%
\begin{pgfscope}%
\pgfpathrectangle{\pgfqpoint{1.020000in}{0.880000in}}{\pgfqpoint{6.160000in}{6.160000in}}%
\pgfusepath{clip}%
\pgfsetbuttcap%
\pgfsetroundjoin%
\definecolor{currentfill}{rgb}{0.703587,0.802586,0.982847}%
\pgfsetfillcolor{currentfill}%
\pgfsetlinewidth{0.000000pt}%
\definecolor{currentstroke}{rgb}{0.000000,0.000000,0.000000}%
\pgfsetstrokecolor{currentstroke}%
\pgfsetdash{}{0pt}%
\pgfpathmoveto{\pgfqpoint{3.334066in}{3.232924in}}%
\pgfpathlineto{\pgfqpoint{3.344105in}{3.209777in}}%
\pgfpathlineto{\pgfqpoint{3.354176in}{3.184762in}}%
\pgfpathlineto{\pgfqpoint{3.388752in}{3.221419in}}%
\pgfpathlineto{\pgfqpoint{3.423326in}{3.258220in}}%
\pgfpathlineto{\pgfqpoint{3.413191in}{3.283775in}}%
\pgfpathlineto{\pgfqpoint{3.403089in}{3.307169in}}%
\pgfpathlineto{\pgfqpoint{3.368578in}{3.269973in}}%
\pgfpathlineto{\pgfqpoint{3.334066in}{3.232924in}}%
\pgfpathclose%
\pgfusepath{fill}%
\end{pgfscope}%
\begin{pgfscope}%
\pgfpathrectangle{\pgfqpoint{1.020000in}{0.880000in}}{\pgfqpoint{6.160000in}{6.160000in}}%
\pgfusepath{clip}%
\pgfsetbuttcap%
\pgfsetroundjoin%
\definecolor{currentfill}{rgb}{0.661968,0.775491,0.993937}%
\pgfsetfillcolor{currentfill}%
\pgfsetlinewidth{0.000000pt}%
\definecolor{currentstroke}{rgb}{0.000000,0.000000,0.000000}%
\pgfsetstrokecolor{currentstroke}%
\pgfsetdash{}{0pt}%
\pgfpathmoveto{\pgfqpoint{3.265030in}{3.160361in}}%
\pgfpathlineto{\pgfqpoint{3.275006in}{3.137454in}}%
\pgfpathlineto{\pgfqpoint{3.285013in}{3.112966in}}%
\pgfpathlineto{\pgfqpoint{3.319597in}{3.148528in}}%
\pgfpathlineto{\pgfqpoint{3.354176in}{3.184762in}}%
\pgfpathlineto{\pgfqpoint{3.344105in}{3.209777in}}%
\pgfpathlineto{\pgfqpoint{3.334066in}{3.232924in}}%
\pgfpathlineto{\pgfqpoint{3.299551in}{3.196301in}}%
\pgfpathlineto{\pgfqpoint{3.265030in}{3.160361in}}%
\pgfpathclose%
\pgfusepath{fill}%
\end{pgfscope}%
\begin{pgfscope}%
\pgfpathrectangle{\pgfqpoint{1.020000in}{0.880000in}}{\pgfqpoint{6.160000in}{6.160000in}}%
\pgfusepath{clip}%
\pgfsetbuttcap%
\pgfsetroundjoin%
\definecolor{currentfill}{rgb}{0.738826,0.822572,0.968261}%
\pgfsetfillcolor{currentfill}%
\pgfsetlinewidth{0.000000pt}%
\definecolor{currentstroke}{rgb}{0.000000,0.000000,0.000000}%
\pgfsetstrokecolor{currentstroke}%
\pgfsetdash{}{0pt}%
\pgfpathmoveto{\pgfqpoint{3.403089in}{3.307169in}}%
\pgfpathlineto{\pgfqpoint{3.413191in}{3.283775in}}%
\pgfpathlineto{\pgfqpoint{3.423326in}{3.258220in}}%
\pgfpathlineto{\pgfqpoint{3.457902in}{3.294870in}}%
\pgfpathlineto{\pgfqpoint{3.492480in}{3.331053in}}%
\pgfpathlineto{\pgfqpoint{3.482282in}{3.357141in}}%
\pgfpathlineto{\pgfqpoint{3.472118in}{3.380774in}}%
\pgfpathlineto{\pgfqpoint{3.437601in}{3.344209in}}%
\pgfpathlineto{\pgfqpoint{3.403089in}{3.307169in}}%
\pgfpathclose%
\pgfusepath{fill}%
\end{pgfscope}%
\begin{pgfscope}%
\pgfpathrectangle{\pgfqpoint{1.020000in}{0.880000in}}{\pgfqpoint{6.160000in}{6.160000in}}%
\pgfusepath{clip}%
\pgfsetbuttcap%
\pgfsetroundjoin%
\definecolor{currentfill}{rgb}{0.624703,0.748318,0.998719}%
\pgfsetfillcolor{currentfill}%
\pgfsetlinewidth{0.000000pt}%
\definecolor{currentstroke}{rgb}{0.000000,0.000000,0.000000}%
\pgfsetstrokecolor{currentstroke}%
\pgfsetdash{}{0pt}%
\pgfpathmoveto{\pgfqpoint{3.195960in}{3.091421in}}%
\pgfpathlineto{\pgfqpoint{3.205873in}{3.068737in}}%
\pgfpathlineto{\pgfqpoint{3.215815in}{3.044742in}}%
\pgfpathlineto{\pgfqpoint{3.250419in}{3.078305in}}%
\pgfpathlineto{\pgfqpoint{3.285013in}{3.112966in}}%
\pgfpathlineto{\pgfqpoint{3.275006in}{3.137454in}}%
\pgfpathlineto{\pgfqpoint{3.265030in}{3.160361in}}%
\pgfpathlineto{\pgfqpoint{3.230500in}{3.125334in}}%
\pgfpathlineto{\pgfqpoint{3.195960in}{3.091421in}}%
\pgfpathclose%
\pgfusepath{fill}%
\end{pgfscope}%
\begin{pgfscope}%
\pgfpathrectangle{\pgfqpoint{1.020000in}{0.880000in}}{\pgfqpoint{6.160000in}{6.160000in}}%
\pgfusepath{clip}%
\pgfsetbuttcap%
\pgfsetroundjoin%
\definecolor{currentfill}{rgb}{0.404421,0.534643,0.932002}%
\pgfsetfillcolor{currentfill}%
\pgfsetlinewidth{0.000000pt}%
\definecolor{currentstroke}{rgb}{0.000000,0.000000,0.000000}%
\pgfsetstrokecolor{currentstroke}%
\pgfsetdash{}{0pt}%
\pgfpathmoveto{\pgfqpoint{4.749400in}{2.744263in}}%
\pgfpathlineto{\pgfqpoint{4.760639in}{2.711547in}}%
\pgfpathlineto{\pgfqpoint{4.771899in}{2.679207in}}%
\pgfpathlineto{\pgfqpoint{4.806198in}{2.665572in}}%
\pgfpathlineto{\pgfqpoint{4.840468in}{2.652479in}}%
\pgfpathlineto{\pgfqpoint{4.829149in}{2.682204in}}%
\pgfpathlineto{\pgfqpoint{4.817850in}{2.712258in}}%
\pgfpathlineto{\pgfqpoint{4.783640in}{2.727946in}}%
\pgfpathlineto{\pgfqpoint{4.749400in}{2.744263in}}%
\pgfpathclose%
\pgfusepath{fill}%
\end{pgfscope}%
\begin{pgfscope}%
\pgfpathrectangle{\pgfqpoint{1.020000in}{0.880000in}}{\pgfqpoint{6.160000in}{6.160000in}}%
\pgfusepath{clip}%
\pgfsetbuttcap%
\pgfsetroundjoin%
\definecolor{currentfill}{rgb}{0.586921,0.718121,0.998874}%
\pgfsetfillcolor{currentfill}%
\pgfsetlinewidth{0.000000pt}%
\definecolor{currentstroke}{rgb}{0.000000,0.000000,0.000000}%
\pgfsetstrokecolor{currentstroke}%
\pgfsetdash{}{0pt}%
\pgfpathmoveto{\pgfqpoint{3.126833in}{3.027601in}}%
\pgfpathlineto{\pgfqpoint{3.136684in}{3.005111in}}%
\pgfpathlineto{\pgfqpoint{3.146562in}{2.981560in}}%
\pgfpathlineto{\pgfqpoint{3.181196in}{3.012446in}}%
\pgfpathlineto{\pgfqpoint{3.215815in}{3.044742in}}%
\pgfpathlineto{\pgfqpoint{3.205873in}{3.068737in}}%
\pgfpathlineto{\pgfqpoint{3.195960in}{3.091421in}}%
\pgfpathlineto{\pgfqpoint{3.161405in}{3.058796in}}%
\pgfpathlineto{\pgfqpoint{3.126833in}{3.027601in}}%
\pgfpathclose%
\pgfusepath{fill}%
\end{pgfscope}%
\begin{pgfscope}%
\pgfpathrectangle{\pgfqpoint{1.020000in}{0.880000in}}{\pgfqpoint{6.160000in}{6.160000in}}%
\pgfusepath{clip}%
\pgfsetbuttcap%
\pgfsetroundjoin%
\definecolor{currentfill}{rgb}{0.543440,0.680003,0.993051}%
\pgfsetfillcolor{currentfill}%
\pgfsetlinewidth{0.000000pt}%
\definecolor{currentstroke}{rgb}{0.000000,0.000000,0.000000}%
\pgfsetstrokecolor{currentstroke}%
\pgfsetdash{}{0pt}%
\pgfpathmoveto{\pgfqpoint{4.499106in}{3.014681in}}%
\pgfpathlineto{\pgfqpoint{4.510122in}{2.973257in}}%
\pgfpathlineto{\pgfqpoint{4.521153in}{2.931553in}}%
\pgfpathlineto{\pgfqpoint{4.555549in}{2.911398in}}%
\pgfpathlineto{\pgfqpoint{4.589907in}{2.890969in}}%
\pgfpathlineto{\pgfqpoint{4.578828in}{2.929785in}}%
\pgfpathlineto{\pgfqpoint{4.567765in}{2.968357in}}%
\pgfpathlineto{\pgfqpoint{4.533455in}{2.991673in}}%
\pgfpathlineto{\pgfqpoint{4.499106in}{3.014681in}}%
\pgfpathclose%
\pgfusepath{fill}%
\end{pgfscope}%
\begin{pgfscope}%
\pgfpathrectangle{\pgfqpoint{1.020000in}{0.880000in}}{\pgfqpoint{6.160000in}{6.160000in}}%
\pgfusepath{clip}%
\pgfsetbuttcap%
\pgfsetroundjoin%
\definecolor{currentfill}{rgb}{0.554312,0.690097,0.995516}%
\pgfsetfillcolor{currentfill}%
\pgfsetlinewidth{0.000000pt}%
\definecolor{currentstroke}{rgb}{0.000000,0.000000,0.000000}%
\pgfsetstrokecolor{currentstroke}%
\pgfsetdash{}{0pt}%
\pgfpathmoveto{\pgfqpoint{3.057630in}{2.969929in}}%
\pgfpathlineto{\pgfqpoint{3.067419in}{2.947600in}}%
\pgfpathlineto{\pgfqpoint{3.077233in}{2.924433in}}%
\pgfpathlineto{\pgfqpoint{3.111908in}{2.952194in}}%
\pgfpathlineto{\pgfqpoint{3.146562in}{2.981560in}}%
\pgfpathlineto{\pgfqpoint{3.136684in}{3.005111in}}%
\pgfpathlineto{\pgfqpoint{3.126833in}{3.027601in}}%
\pgfpathlineto{\pgfqpoint{3.092243in}{2.997950in}}%
\pgfpathlineto{\pgfqpoint{3.057630in}{2.969929in}}%
\pgfpathclose%
\pgfusepath{fill}%
\end{pgfscope}%
\begin{pgfscope}%
\pgfpathrectangle{\pgfqpoint{1.020000in}{0.880000in}}{\pgfqpoint{6.160000in}{6.160000in}}%
\pgfusepath{clip}%
\pgfsetbuttcap%
\pgfsetroundjoin%
\definecolor{currentfill}{rgb}{0.772706,0.838978,0.949319}%
\pgfsetfillcolor{currentfill}%
\pgfsetlinewidth{0.000000pt}%
\definecolor{currentstroke}{rgb}{0.000000,0.000000,0.000000}%
\pgfsetstrokecolor{currentstroke}%
\pgfsetdash{}{0pt}%
\pgfpathmoveto{\pgfqpoint{4.158038in}{3.437114in}}%
\pgfpathlineto{\pgfqpoint{4.168813in}{3.395488in}}%
\pgfpathlineto{\pgfqpoint{4.179604in}{3.351586in}}%
\pgfpathlineto{\pgfqpoint{4.214186in}{3.336402in}}%
\pgfpathlineto{\pgfqpoint{4.248732in}{3.318666in}}%
\pgfpathlineto{\pgfqpoint{4.237904in}{3.361132in}}%
\pgfpathlineto{\pgfqpoint{4.227091in}{3.401434in}}%
\pgfpathlineto{\pgfqpoint{4.192582in}{3.420636in}}%
\pgfpathlineto{\pgfqpoint{4.158038in}{3.437114in}}%
\pgfpathclose%
\pgfusepath{fill}%
\end{pgfscope}%
\begin{pgfscope}%
\pgfpathrectangle{\pgfqpoint{1.020000in}{0.880000in}}{\pgfqpoint{6.160000in}{6.160000in}}%
\pgfusepath{clip}%
\pgfsetbuttcap%
\pgfsetroundjoin%
\definecolor{currentfill}{rgb}{0.527132,0.664700,0.989065}%
\pgfsetfillcolor{currentfill}%
\pgfsetlinewidth{0.000000pt}%
\definecolor{currentstroke}{rgb}{0.000000,0.000000,0.000000}%
\pgfsetstrokecolor{currentstroke}%
\pgfsetdash{}{0pt}%
\pgfpathmoveto{\pgfqpoint{2.988332in}{2.918989in}}%
\pgfpathlineto{\pgfqpoint{2.998059in}{2.896781in}}%
\pgfpathlineto{\pgfqpoint{3.007809in}{2.873932in}}%
\pgfpathlineto{\pgfqpoint{3.042534in}{2.898335in}}%
\pgfpathlineto{\pgfqpoint{3.077233in}{2.924433in}}%
\pgfpathlineto{\pgfqpoint{3.067419in}{2.947600in}}%
\pgfpathlineto{\pgfqpoint{3.057630in}{2.969929in}}%
\pgfpathlineto{\pgfqpoint{3.022994in}{2.943598in}}%
\pgfpathlineto{\pgfqpoint{2.988332in}{2.918989in}}%
\pgfpathclose%
\pgfusepath{fill}%
\end{pgfscope}%
\begin{pgfscope}%
\pgfpathrectangle{\pgfqpoint{1.020000in}{0.880000in}}{\pgfqpoint{6.160000in}{6.160000in}}%
\pgfusepath{clip}%
\pgfsetbuttcap%
\pgfsetroundjoin%
\definecolor{currentfill}{rgb}{0.323718,0.433158,0.864722}%
\pgfsetfillcolor{currentfill}%
\pgfsetlinewidth{0.000000pt}%
\definecolor{currentstroke}{rgb}{0.000000,0.000000,0.000000}%
\pgfsetstrokecolor{currentstroke}%
\pgfsetdash{}{0pt}%
\pgfpathmoveto{\pgfqpoint{5.000237in}{2.557660in}}%
\pgfpathlineto{\pgfqpoint{5.011751in}{2.533614in}}%
\pgfpathlineto{\pgfqpoint{5.045989in}{2.526990in}}%
\pgfpathlineto{\pgfqpoint{5.080205in}{2.521006in}}%
\pgfpathlineto{\pgfqpoint{5.068625in}{2.543292in}}%
\pgfpathlineto{\pgfqpoint{5.034443in}{2.550127in}}%
\pgfpathlineto{\pgfqpoint{5.000237in}{2.557660in}}%
\pgfpathclose%
\pgfusepath{fill}%
\end{pgfscope}%
\begin{pgfscope}%
\pgfpathrectangle{\pgfqpoint{1.020000in}{0.880000in}}{\pgfqpoint{6.160000in}{6.160000in}}%
\pgfusepath{clip}%
\pgfsetbuttcap%
\pgfsetroundjoin%
\definecolor{currentfill}{rgb}{0.859385,0.864431,0.872111}%
\pgfsetfillcolor{currentfill}%
\pgfsetlinewidth{0.000000pt}%
\definecolor{currentstroke}{rgb}{0.000000,0.000000,0.000000}%
\pgfsetstrokecolor{currentstroke}%
\pgfsetdash{}{0pt}%
\pgfpathmoveto{\pgfqpoint{3.838562in}{3.595779in}}%
\pgfpathlineto{\pgfqpoint{3.849078in}{3.564621in}}%
\pgfpathlineto{\pgfqpoint{3.859621in}{3.530173in}}%
\pgfpathlineto{\pgfqpoint{3.894289in}{3.540532in}}%
\pgfpathlineto{\pgfqpoint{3.928949in}{3.547314in}}%
\pgfpathlineto{\pgfqpoint{3.918358in}{3.581861in}}%
\pgfpathlineto{\pgfqpoint{3.907791in}{3.613056in}}%
\pgfpathlineto{\pgfqpoint{3.873180in}{3.606264in}}%
\pgfpathlineto{\pgfqpoint{3.838562in}{3.595779in}}%
\pgfpathclose%
\pgfusepath{fill}%
\end{pgfscope}%
\begin{pgfscope}%
\pgfpathrectangle{\pgfqpoint{1.020000in}{0.880000in}}{\pgfqpoint{6.160000in}{6.160000in}}%
\pgfusepath{clip}%
\pgfsetbuttcap%
\pgfsetroundjoin%
\definecolor{currentfill}{rgb}{0.478462,0.616564,0.972721}%
\pgfsetfillcolor{currentfill}%
\pgfsetlinewidth{0.000000pt}%
\definecolor{currentstroke}{rgb}{0.000000,0.000000,0.000000}%
\pgfsetstrokecolor{currentstroke}%
\pgfsetdash{}{0pt}%
\pgfpathmoveto{\pgfqpoint{4.589907in}{2.890969in}}%
\pgfpathlineto{\pgfqpoint{4.601002in}{2.852109in}}%
\pgfpathlineto{\pgfqpoint{4.612114in}{2.813392in}}%
\pgfpathlineto{\pgfqpoint{4.646486in}{2.795810in}}%
\pgfpathlineto{\pgfqpoint{4.680824in}{2.778328in}}%
\pgfpathlineto{\pgfqpoint{4.669659in}{2.814136in}}%
\pgfpathlineto{\pgfqpoint{4.658513in}{2.850077in}}%
\pgfpathlineto{\pgfqpoint{4.624228in}{2.870466in}}%
\pgfpathlineto{\pgfqpoint{4.589907in}{2.890969in}}%
\pgfpathclose%
\pgfusepath{fill}%
\end{pgfscope}%
\begin{pgfscope}%
\pgfpathrectangle{\pgfqpoint{1.020000in}{0.880000in}}{\pgfqpoint{6.160000in}{6.160000in}}%
\pgfusepath{clip}%
\pgfsetbuttcap%
\pgfsetroundjoin%
\definecolor{currentfill}{rgb}{0.708720,0.805721,0.981117}%
\pgfsetfillcolor{currentfill}%
\pgfsetlinewidth{0.000000pt}%
\definecolor{currentstroke}{rgb}{0.000000,0.000000,0.000000}%
\pgfsetstrokecolor{currentstroke}%
\pgfsetdash{}{0pt}%
\pgfpathmoveto{\pgfqpoint{4.248732in}{3.318666in}}%
\pgfpathlineto{\pgfqpoint{4.259574in}{3.274318in}}%
\pgfpathlineto{\pgfqpoint{4.270430in}{3.228383in}}%
\pgfpathlineto{\pgfqpoint{4.304976in}{3.210187in}}%
\pgfpathlineto{\pgfqpoint{4.339485in}{3.190086in}}%
\pgfpathlineto{\pgfqpoint{4.328592in}{3.234027in}}%
\pgfpathlineto{\pgfqpoint{4.317713in}{3.276486in}}%
\pgfpathlineto{\pgfqpoint{4.283242in}{3.298610in}}%
\pgfpathlineto{\pgfqpoint{4.248732in}{3.318666in}}%
\pgfpathclose%
\pgfusepath{fill}%
\end{pgfscope}%
\begin{pgfscope}%
\pgfpathrectangle{\pgfqpoint{1.020000in}{0.880000in}}{\pgfqpoint{6.160000in}{6.160000in}}%
\pgfusepath{clip}%
\pgfsetbuttcap%
\pgfsetroundjoin%
\definecolor{currentfill}{rgb}{0.500031,0.638508,0.981070}%
\pgfsetfillcolor{currentfill}%
\pgfsetlinewidth{0.000000pt}%
\definecolor{currentstroke}{rgb}{0.000000,0.000000,0.000000}%
\pgfsetstrokecolor{currentstroke}%
\pgfsetdash{}{0pt}%
\pgfpathmoveto{\pgfqpoint{2.918922in}{2.874958in}}%
\pgfpathlineto{\pgfqpoint{2.928589in}{2.852831in}}%
\pgfpathlineto{\pgfqpoint{2.938277in}{2.830231in}}%
\pgfpathlineto{\pgfqpoint{2.973058in}{2.851234in}}%
\pgfpathlineto{\pgfqpoint{3.007809in}{2.873932in}}%
\pgfpathlineto{\pgfqpoint{2.998059in}{2.896781in}}%
\pgfpathlineto{\pgfqpoint{2.988332in}{2.918989in}}%
\pgfpathlineto{\pgfqpoint{2.953642in}{2.896112in}}%
\pgfpathlineto{\pgfqpoint{2.918922in}{2.874958in}}%
\pgfpathclose%
\pgfusepath{fill}%
\end{pgfscope}%
\begin{pgfscope}%
\pgfpathrectangle{\pgfqpoint{1.020000in}{0.880000in}}{\pgfqpoint{6.160000in}{6.160000in}}%
\pgfusepath{clip}%
\pgfsetbuttcap%
\pgfsetroundjoin%
\definecolor{currentfill}{rgb}{0.831148,0.859513,0.903110}%
\pgfsetfillcolor{currentfill}%
\pgfsetlinewidth{0.000000pt}%
\definecolor{currentstroke}{rgb}{0.000000,0.000000,0.000000}%
\pgfsetstrokecolor{currentstroke}%
\pgfsetdash{}{0pt}%
\pgfpathmoveto{\pgfqpoint{3.998229in}{3.549938in}}%
\pgfpathlineto{\pgfqpoint{4.008888in}{3.512391in}}%
\pgfpathlineto{\pgfqpoint{4.019567in}{3.471903in}}%
\pgfpathlineto{\pgfqpoint{4.054223in}{3.468173in}}%
\pgfpathlineto{\pgfqpoint{4.088855in}{3.461051in}}%
\pgfpathlineto{\pgfqpoint{4.078135in}{3.501008in}}%
\pgfpathlineto{\pgfqpoint{4.067434in}{3.538071in}}%
\pgfpathlineto{\pgfqpoint{4.032843in}{3.545785in}}%
\pgfpathlineto{\pgfqpoint{3.998229in}{3.549938in}}%
\pgfpathclose%
\pgfusepath{fill}%
\end{pgfscope}%
\begin{pgfscope}%
\pgfpathrectangle{\pgfqpoint{1.020000in}{0.880000in}}{\pgfqpoint{6.160000in}{6.160000in}}%
\pgfusepath{clip}%
\pgfsetbuttcap%
\pgfsetroundjoin%
\definecolor{currentfill}{rgb}{0.363461,0.484784,0.901019}%
\pgfsetfillcolor{currentfill}%
\pgfsetlinewidth{0.000000pt}%
\definecolor{currentstroke}{rgb}{0.000000,0.000000,0.000000}%
\pgfsetstrokecolor{currentstroke}%
\pgfsetdash{}{0pt}%
\pgfpathmoveto{\pgfqpoint{4.840468in}{2.652479in}}%
\pgfpathlineto{\pgfqpoint{4.851810in}{2.623172in}}%
\pgfpathlineto{\pgfqpoint{4.863173in}{2.594365in}}%
\pgfpathlineto{\pgfqpoint{4.897477in}{2.584270in}}%
\pgfpathlineto{\pgfqpoint{4.931755in}{2.574759in}}%
\pgfpathlineto{\pgfqpoint{4.920329in}{2.601280in}}%
\pgfpathlineto{\pgfqpoint{4.908925in}{2.628227in}}%
\pgfpathlineto{\pgfqpoint{4.874710in}{2.640008in}}%
\pgfpathlineto{\pgfqpoint{4.840468in}{2.652479in}}%
\pgfpathclose%
\pgfusepath{fill}%
\end{pgfscope}%
\begin{pgfscope}%
\pgfpathrectangle{\pgfqpoint{1.020000in}{0.880000in}}{\pgfqpoint{6.160000in}{6.160000in}}%
\pgfusepath{clip}%
\pgfsetbuttcap%
\pgfsetroundjoin%
\definecolor{currentfill}{rgb}{0.851372,0.863125,0.881064}%
\pgfsetfillcolor{currentfill}%
\pgfsetlinewidth{0.000000pt}%
\definecolor{currentstroke}{rgb}{0.000000,0.000000,0.000000}%
\pgfsetstrokecolor{currentstroke}%
\pgfsetdash{}{0pt}%
\pgfpathmoveto{\pgfqpoint{3.769316in}{3.564228in}}%
\pgfpathlineto{\pgfqpoint{3.779780in}{3.533305in}}%
\pgfpathlineto{\pgfqpoint{3.790272in}{3.499211in}}%
\pgfpathlineto{\pgfqpoint{3.824948in}{3.516349in}}%
\pgfpathlineto{\pgfqpoint{3.859621in}{3.530173in}}%
\pgfpathlineto{\pgfqpoint{3.849078in}{3.564621in}}%
\pgfpathlineto{\pgfqpoint{3.838562in}{3.595779in}}%
\pgfpathlineto{\pgfqpoint{3.803940in}{3.581715in}}%
\pgfpathlineto{\pgfqpoint{3.769316in}{3.564228in}}%
\pgfpathclose%
\pgfusepath{fill}%
\end{pgfscope}%
\begin{pgfscope}%
\pgfpathrectangle{\pgfqpoint{1.020000in}{0.880000in}}{\pgfqpoint{6.160000in}{6.160000in}}%
\pgfusepath{clip}%
\pgfsetbuttcap%
\pgfsetroundjoin%
\definecolor{currentfill}{rgb}{0.640828,0.760752,0.997846}%
\pgfsetfillcolor{currentfill}%
\pgfsetlinewidth{0.000000pt}%
\definecolor{currentstroke}{rgb}{0.000000,0.000000,0.000000}%
\pgfsetstrokecolor{currentstroke}%
\pgfsetdash{}{0pt}%
\pgfpathmoveto{\pgfqpoint{4.339485in}{3.190086in}}%
\pgfpathlineto{\pgfqpoint{4.350391in}{3.144947in}}%
\pgfpathlineto{\pgfqpoint{4.361310in}{3.098896in}}%
\pgfpathlineto{\pgfqpoint{4.395818in}{3.079499in}}%
\pgfpathlineto{\pgfqpoint{4.430287in}{3.058836in}}%
\pgfpathlineto{\pgfqpoint{4.419328in}{3.102424in}}%
\pgfpathlineto{\pgfqpoint{4.408383in}{3.145178in}}%
\pgfpathlineto{\pgfqpoint{4.373954in}{3.168330in}}%
\pgfpathlineto{\pgfqpoint{4.339485in}{3.190086in}}%
\pgfpathclose%
\pgfusepath{fill}%
\end{pgfscope}%
\begin{pgfscope}%
\pgfpathrectangle{\pgfqpoint{1.020000in}{0.880000in}}{\pgfqpoint{6.160000in}{6.160000in}}%
\pgfusepath{clip}%
\pgfsetbuttcap%
\pgfsetroundjoin%
\definecolor{currentfill}{rgb}{0.483854,0.622050,0.974808}%
\pgfsetfillcolor{currentfill}%
\pgfsetlinewidth{0.000000pt}%
\definecolor{currentstroke}{rgb}{0.000000,0.000000,0.000000}%
\pgfsetstrokecolor{currentstroke}%
\pgfsetdash{}{0pt}%
\pgfpathmoveto{\pgfqpoint{2.849390in}{2.837682in}}%
\pgfpathlineto{\pgfqpoint{2.858997in}{2.815595in}}%
\pgfpathlineto{\pgfqpoint{2.868624in}{2.793176in}}%
\pgfpathlineto{\pgfqpoint{2.903466in}{2.810893in}}%
\pgfpathlineto{\pgfqpoint{2.938277in}{2.830231in}}%
\pgfpathlineto{\pgfqpoint{2.928589in}{2.852831in}}%
\pgfpathlineto{\pgfqpoint{2.918922in}{2.874958in}}%
\pgfpathlineto{\pgfqpoint{2.884172in}{2.855496in}}%
\pgfpathlineto{\pgfqpoint{2.849390in}{2.837682in}}%
\pgfpathclose%
\pgfusepath{fill}%
\end{pgfscope}%
\begin{pgfscope}%
\pgfpathrectangle{\pgfqpoint{1.020000in}{0.880000in}}{\pgfqpoint{6.160000in}{6.160000in}}%
\pgfusepath{clip}%
\pgfsetbuttcap%
\pgfsetroundjoin%
\definecolor{currentfill}{rgb}{0.425199,0.559058,0.946061}%
\pgfsetfillcolor{currentfill}%
\pgfsetlinewidth{0.000000pt}%
\definecolor{currentstroke}{rgb}{0.000000,0.000000,0.000000}%
\pgfsetstrokecolor{currentstroke}%
\pgfsetdash{}{0pt}%
\pgfpathmoveto{\pgfqpoint{4.680824in}{2.778328in}}%
\pgfpathlineto{\pgfqpoint{4.692007in}{2.742806in}}%
\pgfpathlineto{\pgfqpoint{4.703210in}{2.707709in}}%
\pgfpathlineto{\pgfqpoint{4.737570in}{2.693289in}}%
\pgfpathlineto{\pgfqpoint{4.771899in}{2.679207in}}%
\pgfpathlineto{\pgfqpoint{4.760639in}{2.711547in}}%
\pgfpathlineto{\pgfqpoint{4.749400in}{2.744263in}}%
\pgfpathlineto{\pgfqpoint{4.715128in}{2.761099in}}%
\pgfpathlineto{\pgfqpoint{4.680824in}{2.778328in}}%
\pgfpathclose%
\pgfusepath{fill}%
\end{pgfscope}%
\begin{pgfscope}%
\pgfpathrectangle{\pgfqpoint{1.020000in}{0.880000in}}{\pgfqpoint{6.160000in}{6.160000in}}%
\pgfusepath{clip}%
\pgfsetbuttcap%
\pgfsetroundjoin%
\definecolor{currentfill}{rgb}{0.570616,0.704109,0.997195}%
\pgfsetfillcolor{currentfill}%
\pgfsetlinewidth{0.000000pt}%
\definecolor{currentstroke}{rgb}{0.000000,0.000000,0.000000}%
\pgfsetstrokecolor{currentstroke}%
\pgfsetdash{}{0pt}%
\pgfpathmoveto{\pgfqpoint{4.430287in}{3.058836in}}%
\pgfpathlineto{\pgfqpoint{4.441260in}{3.014676in}}%
\pgfpathlineto{\pgfqpoint{4.452247in}{2.970199in}}%
\pgfpathlineto{\pgfqpoint{4.486719in}{2.951225in}}%
\pgfpathlineto{\pgfqpoint{4.521153in}{2.931553in}}%
\pgfpathlineto{\pgfqpoint{4.510122in}{2.973257in}}%
\pgfpathlineto{\pgfqpoint{4.499106in}{3.014681in}}%
\pgfpathlineto{\pgfqpoint{4.464716in}{3.037149in}}%
\pgfpathlineto{\pgfqpoint{4.430287in}{3.058836in}}%
\pgfpathclose%
\pgfusepath{fill}%
\end{pgfscope}%
\begin{pgfscope}%
\pgfpathrectangle{\pgfqpoint{1.020000in}{0.880000in}}{\pgfqpoint{6.160000in}{6.160000in}}%
\pgfusepath{clip}%
\pgfsetbuttcap%
\pgfsetroundjoin%
\definecolor{currentfill}{rgb}{0.333490,0.446265,0.874452}%
\pgfsetfillcolor{currentfill}%
\pgfsetlinewidth{0.000000pt}%
\definecolor{currentstroke}{rgb}{0.000000,0.000000,0.000000}%
\pgfsetstrokecolor{currentstroke}%
\pgfsetdash{}{0pt}%
\pgfpathmoveto{\pgfqpoint{4.931755in}{2.574759in}}%
\pgfpathlineto{\pgfqpoint{4.943205in}{2.548723in}}%
\pgfpathlineto{\pgfqpoint{4.977490in}{2.540865in}}%
\pgfpathlineto{\pgfqpoint{5.011751in}{2.533614in}}%
\pgfpathlineto{\pgfqpoint{5.000237in}{2.557660in}}%
\pgfpathlineto{\pgfqpoint{4.966008in}{2.565878in}}%
\pgfpathlineto{\pgfqpoint{4.931755in}{2.574759in}}%
\pgfpathclose%
\pgfusepath{fill}%
\end{pgfscope}%
\begin{pgfscope}%
\pgfpathrectangle{\pgfqpoint{1.020000in}{0.880000in}}{\pgfqpoint{6.160000in}{6.160000in}}%
\pgfusepath{clip}%
\pgfsetbuttcap%
\pgfsetroundjoin%
\definecolor{currentfill}{rgb}{0.791392,0.846750,0.936641}%
\pgfsetfillcolor{currentfill}%
\pgfsetlinewidth{0.000000pt}%
\definecolor{currentstroke}{rgb}{0.000000,0.000000,0.000000}%
\pgfsetstrokecolor{currentstroke}%
\pgfsetdash{}{0pt}%
\pgfpathmoveto{\pgfqpoint{4.088855in}{3.461051in}}%
\pgfpathlineto{\pgfqpoint{4.099592in}{3.418456in}}%
\pgfpathlineto{\pgfqpoint{4.110346in}{3.373506in}}%
\pgfpathlineto{\pgfqpoint{4.144990in}{3.364012in}}%
\pgfpathlineto{\pgfqpoint{4.179604in}{3.351586in}}%
\pgfpathlineto{\pgfqpoint{4.168813in}{3.395488in}}%
\pgfpathlineto{\pgfqpoint{4.158038in}{3.437114in}}%
\pgfpathlineto{\pgfqpoint{4.123461in}{3.450648in}}%
\pgfpathlineto{\pgfqpoint{4.088855in}{3.461051in}}%
\pgfpathclose%
\pgfusepath{fill}%
\end{pgfscope}%
\begin{pgfscope}%
\pgfpathrectangle{\pgfqpoint{1.020000in}{0.880000in}}{\pgfqpoint{6.160000in}{6.160000in}}%
\pgfusepath{clip}%
\pgfsetbuttcap%
\pgfsetroundjoin%
\definecolor{currentfill}{rgb}{0.835345,0.860514,0.898970}%
\pgfsetfillcolor{currentfill}%
\pgfsetlinewidth{0.000000pt}%
\definecolor{currentstroke}{rgb}{0.000000,0.000000,0.000000}%
\pgfsetstrokecolor{currentstroke}%
\pgfsetdash{}{0pt}%
\pgfpathmoveto{\pgfqpoint{3.700073in}{3.519809in}}%
\pgfpathlineto{\pgfqpoint{3.710482in}{3.489302in}}%
\pgfpathlineto{\pgfqpoint{3.720921in}{3.455790in}}%
\pgfpathlineto{\pgfqpoint{3.755595in}{3.478949in}}%
\pgfpathlineto{\pgfqpoint{3.790272in}{3.499211in}}%
\pgfpathlineto{\pgfqpoint{3.779780in}{3.533305in}}%
\pgfpathlineto{\pgfqpoint{3.769316in}{3.564228in}}%
\pgfpathlineto{\pgfqpoint{3.734693in}{3.543515in}}%
\pgfpathlineto{\pgfqpoint{3.700073in}{3.519809in}}%
\pgfpathclose%
\pgfusepath{fill}%
\end{pgfscope}%
\begin{pgfscope}%
\pgfpathrectangle{\pgfqpoint{1.020000in}{0.880000in}}{\pgfqpoint{6.160000in}{6.160000in}}%
\pgfusepath{clip}%
\pgfsetbuttcap%
\pgfsetroundjoin%
\definecolor{currentfill}{rgb}{0.467678,0.605591,0.968546}%
\pgfsetfillcolor{currentfill}%
\pgfsetlinewidth{0.000000pt}%
\definecolor{currentstroke}{rgb}{0.000000,0.000000,0.000000}%
\pgfsetstrokecolor{currentstroke}%
\pgfsetdash{}{0pt}%
\pgfpathmoveto{\pgfqpoint{2.779728in}{2.806751in}}%
\pgfpathlineto{\pgfqpoint{2.789275in}{2.784664in}}%
\pgfpathlineto{\pgfqpoint{2.798842in}{2.762362in}}%
\pgfpathlineto{\pgfqpoint{2.833750in}{2.777022in}}%
\pgfpathlineto{\pgfqpoint{2.868624in}{2.793176in}}%
\pgfpathlineto{\pgfqpoint{2.858997in}{2.815595in}}%
\pgfpathlineto{\pgfqpoint{2.849390in}{2.837682in}}%
\pgfpathlineto{\pgfqpoint{2.814576in}{2.821457in}}%
\pgfpathlineto{\pgfqpoint{2.779728in}{2.806751in}}%
\pgfpathclose%
\pgfusepath{fill}%
\end{pgfscope}%
\begin{pgfscope}%
\pgfpathrectangle{\pgfqpoint{1.020000in}{0.880000in}}{\pgfqpoint{6.160000in}{6.160000in}}%
\pgfusepath{clip}%
\pgfsetbuttcap%
\pgfsetroundjoin%
\definecolor{currentfill}{rgb}{0.835345,0.860514,0.898970}%
\pgfsetfillcolor{currentfill}%
\pgfsetlinewidth{0.000000pt}%
\definecolor{currentstroke}{rgb}{0.000000,0.000000,0.000000}%
\pgfsetstrokecolor{currentstroke}%
\pgfsetdash{}{0pt}%
\pgfpathmoveto{\pgfqpoint{3.928949in}{3.547314in}}%
\pgfpathlineto{\pgfqpoint{3.939564in}{3.509604in}}%
\pgfpathlineto{\pgfqpoint{3.950200in}{3.468956in}}%
\pgfpathlineto{\pgfqpoint{3.984892in}{3.472172in}}%
\pgfpathlineto{\pgfqpoint{4.019567in}{3.471903in}}%
\pgfpathlineto{\pgfqpoint{4.008888in}{3.512391in}}%
\pgfpathlineto{\pgfqpoint{3.998229in}{3.549938in}}%
\pgfpathlineto{\pgfqpoint{3.963597in}{3.550456in}}%
\pgfpathlineto{\pgfqpoint{3.928949in}{3.547314in}}%
\pgfpathclose%
\pgfusepath{fill}%
\end{pgfscope}%
\begin{pgfscope}%
\pgfpathrectangle{\pgfqpoint{1.020000in}{0.880000in}}{\pgfqpoint{6.160000in}{6.160000in}}%
\pgfusepath{clip}%
\pgfsetbuttcap%
\pgfsetroundjoin%
\definecolor{currentfill}{rgb}{0.813693,0.854282,0.918480}%
\pgfsetfillcolor{currentfill}%
\pgfsetlinewidth{0.000000pt}%
\definecolor{currentstroke}{rgb}{0.000000,0.000000,0.000000}%
\pgfsetstrokecolor{currentstroke}%
\pgfsetdash{}{0pt}%
\pgfpathmoveto{\pgfqpoint{3.630850in}{3.464503in}}%
\pgfpathlineto{\pgfqpoint{3.641201in}{3.434565in}}%
\pgfpathlineto{\pgfqpoint{3.651584in}{3.401830in}}%
\pgfpathlineto{\pgfqpoint{3.686250in}{3.429990in}}%
\pgfpathlineto{\pgfqpoint{3.720921in}{3.455790in}}%
\pgfpathlineto{\pgfqpoint{3.710482in}{3.489302in}}%
\pgfpathlineto{\pgfqpoint{3.700073in}{3.519809in}}%
\pgfpathlineto{\pgfqpoint{3.665459in}{3.493375in}}%
\pgfpathlineto{\pgfqpoint{3.630850in}{3.464503in}}%
\pgfpathclose%
\pgfusepath{fill}%
\end{pgfscope}%
\begin{pgfscope}%
\pgfpathrectangle{\pgfqpoint{1.020000in}{0.880000in}}{\pgfqpoint{6.160000in}{6.160000in}}%
\pgfusepath{clip}%
\pgfsetbuttcap%
\pgfsetroundjoin%
\definecolor{currentfill}{rgb}{0.505423,0.643995,0.983157}%
\pgfsetfillcolor{currentfill}%
\pgfsetlinewidth{0.000000pt}%
\definecolor{currentstroke}{rgb}{0.000000,0.000000,0.000000}%
\pgfsetstrokecolor{currentstroke}%
\pgfsetdash{}{0pt}%
\pgfpathmoveto{\pgfqpoint{4.521153in}{2.931553in}}%
\pgfpathlineto{\pgfqpoint{4.532200in}{2.889791in}}%
\pgfpathlineto{\pgfqpoint{4.543263in}{2.848182in}}%
\pgfpathlineto{\pgfqpoint{4.577706in}{2.830908in}}%
\pgfpathlineto{\pgfqpoint{4.612114in}{2.813392in}}%
\pgfpathlineto{\pgfqpoint{4.601002in}{2.852109in}}%
\pgfpathlineto{\pgfqpoint{4.589907in}{2.890969in}}%
\pgfpathlineto{\pgfqpoint{4.555549in}{2.911398in}}%
\pgfpathlineto{\pgfqpoint{4.521153in}{2.931553in}}%
\pgfpathclose%
\pgfusepath{fill}%
\end{pgfscope}%
\begin{pgfscope}%
\pgfpathrectangle{\pgfqpoint{1.020000in}{0.880000in}}{\pgfqpoint{6.160000in}{6.160000in}}%
\pgfusepath{clip}%
\pgfsetbuttcap%
\pgfsetroundjoin%
\definecolor{currentfill}{rgb}{0.786721,0.844807,0.939810}%
\pgfsetfillcolor{currentfill}%
\pgfsetlinewidth{0.000000pt}%
\definecolor{currentstroke}{rgb}{0.000000,0.000000,0.000000}%
\pgfsetstrokecolor{currentstroke}%
\pgfsetdash{}{0pt}%
\pgfpathmoveto{\pgfqpoint{3.561652in}{3.400708in}}%
\pgfpathlineto{\pgfqpoint{3.571944in}{3.371457in}}%
\pgfpathlineto{\pgfqpoint{3.582267in}{3.339650in}}%
\pgfpathlineto{\pgfqpoint{3.616923in}{3.371611in}}%
\pgfpathlineto{\pgfqpoint{3.651584in}{3.401830in}}%
\pgfpathlineto{\pgfqpoint{3.641201in}{3.434565in}}%
\pgfpathlineto{\pgfqpoint{3.630850in}{3.464503in}}%
\pgfpathlineto{\pgfqpoint{3.596248in}{3.433506in}}%
\pgfpathlineto{\pgfqpoint{3.561652in}{3.400708in}}%
\pgfpathclose%
\pgfusepath{fill}%
\end{pgfscope}%
\begin{pgfscope}%
\pgfpathrectangle{\pgfqpoint{1.020000in}{0.880000in}}{\pgfqpoint{6.160000in}{6.160000in}}%
\pgfusepath{clip}%
\pgfsetbuttcap%
\pgfsetroundjoin%
\definecolor{currentfill}{rgb}{0.733898,0.820018,0.970724}%
\pgfsetfillcolor{currentfill}%
\pgfsetlinewidth{0.000000pt}%
\definecolor{currentstroke}{rgb}{0.000000,0.000000,0.000000}%
\pgfsetstrokecolor{currentstroke}%
\pgfsetdash{}{0pt}%
\pgfpathmoveto{\pgfqpoint{4.179604in}{3.351586in}}%
\pgfpathlineto{\pgfqpoint{4.190409in}{3.305702in}}%
\pgfpathlineto{\pgfqpoint{4.201229in}{3.258146in}}%
\pgfpathlineto{\pgfqpoint{4.235847in}{3.244440in}}%
\pgfpathlineto{\pgfqpoint{4.270430in}{3.228383in}}%
\pgfpathlineto{\pgfqpoint{4.259574in}{3.274318in}}%
\pgfpathlineto{\pgfqpoint{4.248732in}{3.318666in}}%
\pgfpathlineto{\pgfqpoint{4.214186in}{3.336402in}}%
\pgfpathlineto{\pgfqpoint{4.179604in}{3.351586in}}%
\pgfpathclose%
\pgfusepath{fill}%
\end{pgfscope}%
\begin{pgfscope}%
\pgfpathrectangle{\pgfqpoint{1.020000in}{0.880000in}}{\pgfqpoint{6.160000in}{6.160000in}}%
\pgfusepath{clip}%
\pgfsetbuttcap%
\pgfsetroundjoin%
\definecolor{currentfill}{rgb}{0.758539,0.832787,0.958408}%
\pgfsetfillcolor{currentfill}%
\pgfsetlinewidth{0.000000pt}%
\definecolor{currentstroke}{rgb}{0.000000,0.000000,0.000000}%
\pgfsetstrokecolor{currentstroke}%
\pgfsetdash{}{0pt}%
\pgfpathmoveto{\pgfqpoint{3.492480in}{3.331053in}}%
\pgfpathlineto{\pgfqpoint{3.502710in}{3.302574in}}%
\pgfpathlineto{\pgfqpoint{3.512972in}{3.271800in}}%
\pgfpathlineto{\pgfqpoint{3.547617in}{3.306271in}}%
\pgfpathlineto{\pgfqpoint{3.582267in}{3.339650in}}%
\pgfpathlineto{\pgfqpoint{3.571944in}{3.371457in}}%
\pgfpathlineto{\pgfqpoint{3.561652in}{3.400708in}}%
\pgfpathlineto{\pgfqpoint{3.527063in}{3.366444in}}%
\pgfpathlineto{\pgfqpoint{3.492480in}{3.331053in}}%
\pgfpathclose%
\pgfusepath{fill}%
\end{pgfscope}%
\begin{pgfscope}%
\pgfpathrectangle{\pgfqpoint{1.020000in}{0.880000in}}{\pgfqpoint{6.160000in}{6.160000in}}%
\pgfusepath{clip}%
\pgfsetbuttcap%
\pgfsetroundjoin%
\definecolor{currentfill}{rgb}{0.724041,0.814910,0.975651}%
\pgfsetfillcolor{currentfill}%
\pgfsetlinewidth{0.000000pt}%
\definecolor{currentstroke}{rgb}{0.000000,0.000000,0.000000}%
\pgfsetstrokecolor{currentstroke}%
\pgfsetdash{}{0pt}%
\pgfpathmoveto{\pgfqpoint{3.423326in}{3.258220in}}%
\pgfpathlineto{\pgfqpoint{3.433493in}{3.230558in}}%
\pgfpathlineto{\pgfqpoint{3.443690in}{3.200874in}}%
\pgfpathlineto{\pgfqpoint{3.478330in}{3.236562in}}%
\pgfpathlineto{\pgfqpoint{3.512972in}{3.271800in}}%
\pgfpathlineto{\pgfqpoint{3.502710in}{3.302574in}}%
\pgfpathlineto{\pgfqpoint{3.492480in}{3.331053in}}%
\pgfpathlineto{\pgfqpoint{3.457902in}{3.294870in}}%
\pgfpathlineto{\pgfqpoint{3.423326in}{3.258220in}}%
\pgfpathclose%
\pgfusepath{fill}%
\end{pgfscope}%
\begin{pgfscope}%
\pgfpathrectangle{\pgfqpoint{1.020000in}{0.880000in}}{\pgfqpoint{6.160000in}{6.160000in}}%
\pgfusepath{clip}%
\pgfsetbuttcap%
\pgfsetroundjoin%
\definecolor{currentfill}{rgb}{0.683056,0.790043,0.989768}%
\pgfsetfillcolor{currentfill}%
\pgfsetlinewidth{0.000000pt}%
\definecolor{currentstroke}{rgb}{0.000000,0.000000,0.000000}%
\pgfsetstrokecolor{currentstroke}%
\pgfsetdash{}{0pt}%
\pgfpathmoveto{\pgfqpoint{3.354176in}{3.184762in}}%
\pgfpathlineto{\pgfqpoint{3.364278in}{3.157927in}}%
\pgfpathlineto{\pgfqpoint{3.374410in}{3.129345in}}%
\pgfpathlineto{\pgfqpoint{3.409051in}{3.165039in}}%
\pgfpathlineto{\pgfqpoint{3.443690in}{3.200874in}}%
\pgfpathlineto{\pgfqpoint{3.433493in}{3.230558in}}%
\pgfpathlineto{\pgfqpoint{3.423326in}{3.258220in}}%
\pgfpathlineto{\pgfqpoint{3.388752in}{3.221419in}}%
\pgfpathlineto{\pgfqpoint{3.354176in}{3.184762in}}%
\pgfpathclose%
\pgfusepath{fill}%
\end{pgfscope}%
\begin{pgfscope}%
\pgfpathrectangle{\pgfqpoint{1.020000in}{0.880000in}}{\pgfqpoint{6.160000in}{6.160000in}}%
\pgfusepath{clip}%
\pgfsetbuttcap%
\pgfsetroundjoin%
\definecolor{currentfill}{rgb}{0.378598,0.503856,0.913692}%
\pgfsetfillcolor{currentfill}%
\pgfsetlinewidth{0.000000pt}%
\definecolor{currentstroke}{rgb}{0.000000,0.000000,0.000000}%
\pgfsetstrokecolor{currentstroke}%
\pgfsetdash{}{0pt}%
\pgfpathmoveto{\pgfqpoint{4.771899in}{2.679207in}}%
\pgfpathlineto{\pgfqpoint{4.783180in}{2.647350in}}%
\pgfpathlineto{\pgfqpoint{4.794483in}{2.616072in}}%
\pgfpathlineto{\pgfqpoint{4.828842in}{2.604989in}}%
\pgfpathlineto{\pgfqpoint{4.863173in}{2.594365in}}%
\pgfpathlineto{\pgfqpoint{4.851810in}{2.623172in}}%
\pgfpathlineto{\pgfqpoint{4.840468in}{2.652479in}}%
\pgfpathlineto{\pgfqpoint{4.806198in}{2.665572in}}%
\pgfpathlineto{\pgfqpoint{4.771899in}{2.679207in}}%
\pgfpathclose%
\pgfusepath{fill}%
\end{pgfscope}%
\begin{pgfscope}%
\pgfpathrectangle{\pgfqpoint{1.020000in}{0.880000in}}{\pgfqpoint{6.160000in}{6.160000in}}%
\pgfusepath{clip}%
\pgfsetbuttcap%
\pgfsetroundjoin%
\definecolor{currentfill}{rgb}{0.646113,0.764436,0.996868}%
\pgfsetfillcolor{currentfill}%
\pgfsetlinewidth{0.000000pt}%
\definecolor{currentstroke}{rgb}{0.000000,0.000000,0.000000}%
\pgfsetstrokecolor{currentstroke}%
\pgfsetdash{}{0pt}%
\pgfpathmoveto{\pgfqpoint{3.285013in}{3.112966in}}%
\pgfpathlineto{\pgfqpoint{3.295049in}{3.086936in}}%
\pgfpathlineto{\pgfqpoint{3.305115in}{3.059426in}}%
\pgfpathlineto{\pgfqpoint{3.339765in}{3.094059in}}%
\pgfpathlineto{\pgfqpoint{3.374410in}{3.129345in}}%
\pgfpathlineto{\pgfqpoint{3.364278in}{3.157927in}}%
\pgfpathlineto{\pgfqpoint{3.354176in}{3.184762in}}%
\pgfpathlineto{\pgfqpoint{3.319597in}{3.148528in}}%
\pgfpathlineto{\pgfqpoint{3.285013in}{3.112966in}}%
\pgfpathclose%
\pgfusepath{fill}%
\end{pgfscope}%
\begin{pgfscope}%
\pgfpathrectangle{\pgfqpoint{1.020000in}{0.880000in}}{\pgfqpoint{6.160000in}{6.160000in}}%
\pgfusepath{clip}%
\pgfsetbuttcap%
\pgfsetroundjoin%
\definecolor{currentfill}{rgb}{0.608547,0.735725,0.999354}%
\pgfsetfillcolor{currentfill}%
\pgfsetlinewidth{0.000000pt}%
\definecolor{currentstroke}{rgb}{0.000000,0.000000,0.000000}%
\pgfsetstrokecolor{currentstroke}%
\pgfsetdash{}{0pt}%
\pgfpathmoveto{\pgfqpoint{3.215815in}{3.044742in}}%
\pgfpathlineto{\pgfqpoint{3.225785in}{3.019467in}}%
\pgfpathlineto{\pgfqpoint{3.235784in}{2.992966in}}%
\pgfpathlineto{\pgfqpoint{3.270455in}{3.025664in}}%
\pgfpathlineto{\pgfqpoint{3.305115in}{3.059426in}}%
\pgfpathlineto{\pgfqpoint{3.295049in}{3.086936in}}%
\pgfpathlineto{\pgfqpoint{3.285013in}{3.112966in}}%
\pgfpathlineto{\pgfqpoint{3.250419in}{3.078305in}}%
\pgfpathlineto{\pgfqpoint{3.215815in}{3.044742in}}%
\pgfpathclose%
\pgfusepath{fill}%
\end{pgfscope}%
\begin{pgfscope}%
\pgfpathrectangle{\pgfqpoint{1.020000in}{0.880000in}}{\pgfqpoint{6.160000in}{6.160000in}}%
\pgfusepath{clip}%
\pgfsetbuttcap%
\pgfsetroundjoin%
\definecolor{currentfill}{rgb}{0.570616,0.704109,0.997195}%
\pgfsetfillcolor{currentfill}%
\pgfsetlinewidth{0.000000pt}%
\definecolor{currentstroke}{rgb}{0.000000,0.000000,0.000000}%
\pgfsetstrokecolor{currentstroke}%
\pgfsetdash{}{0pt}%
\pgfpathmoveto{\pgfqpoint{3.146562in}{2.981560in}}%
\pgfpathlineto{\pgfqpoint{3.156466in}{2.956972in}}%
\pgfpathlineto{\pgfqpoint{3.166397in}{2.931391in}}%
\pgfpathlineto{\pgfqpoint{3.201099in}{2.961496in}}%
\pgfpathlineto{\pgfqpoint{3.235784in}{2.992966in}}%
\pgfpathlineto{\pgfqpoint{3.225785in}{3.019467in}}%
\pgfpathlineto{\pgfqpoint{3.215815in}{3.044742in}}%
\pgfpathlineto{\pgfqpoint{3.181196in}{3.012446in}}%
\pgfpathlineto{\pgfqpoint{3.146562in}{2.981560in}}%
\pgfpathclose%
\pgfusepath{fill}%
\end{pgfscope}%
\begin{pgfscope}%
\pgfpathrectangle{\pgfqpoint{1.020000in}{0.880000in}}{\pgfqpoint{6.160000in}{6.160000in}}%
\pgfusepath{clip}%
\pgfsetbuttcap%
\pgfsetroundjoin%
\definecolor{currentfill}{rgb}{0.538004,0.674902,0.991722}%
\pgfsetfillcolor{currentfill}%
\pgfsetlinewidth{0.000000pt}%
\definecolor{currentstroke}{rgb}{0.000000,0.000000,0.000000}%
\pgfsetstrokecolor{currentstroke}%
\pgfsetdash{}{0pt}%
\pgfpathmoveto{\pgfqpoint{3.077233in}{2.924433in}}%
\pgfpathlineto{\pgfqpoint{3.087072in}{2.900448in}}%
\pgfpathlineto{\pgfqpoint{3.096936in}{2.875678in}}%
\pgfpathlineto{\pgfqpoint{3.131677in}{2.902757in}}%
\pgfpathlineto{\pgfqpoint{3.166397in}{2.931391in}}%
\pgfpathlineto{\pgfqpoint{3.156466in}{2.956972in}}%
\pgfpathlineto{\pgfqpoint{3.146562in}{2.981560in}}%
\pgfpathlineto{\pgfqpoint{3.111908in}{2.952194in}}%
\pgfpathlineto{\pgfqpoint{3.077233in}{2.924433in}}%
\pgfpathclose%
\pgfusepath{fill}%
\end{pgfscope}%
\begin{pgfscope}%
\pgfpathrectangle{\pgfqpoint{1.020000in}{0.880000in}}{\pgfqpoint{6.160000in}{6.160000in}}%
\pgfusepath{clip}%
\pgfsetbuttcap%
\pgfsetroundjoin%
\definecolor{currentfill}{rgb}{0.835345,0.860514,0.898970}%
\pgfsetfillcolor{currentfill}%
\pgfsetlinewidth{0.000000pt}%
\definecolor{currentstroke}{rgb}{0.000000,0.000000,0.000000}%
\pgfsetstrokecolor{currentstroke}%
\pgfsetdash{}{0pt}%
\pgfpathmoveto{\pgfqpoint{3.859621in}{3.530173in}}%
\pgfpathlineto{\pgfqpoint{3.870190in}{3.492621in}}%
\pgfpathlineto{\pgfqpoint{3.880781in}{3.452184in}}%
\pgfpathlineto{\pgfqpoint{3.915496in}{3.462272in}}%
\pgfpathlineto{\pgfqpoint{3.950200in}{3.468956in}}%
\pgfpathlineto{\pgfqpoint{3.939564in}{3.509604in}}%
\pgfpathlineto{\pgfqpoint{3.928949in}{3.547314in}}%
\pgfpathlineto{\pgfqpoint{3.894289in}{3.540532in}}%
\pgfpathlineto{\pgfqpoint{3.859621in}{3.530173in}}%
\pgfpathclose%
\pgfusepath{fill}%
\end{pgfscope}%
\begin{pgfscope}%
\pgfpathrectangle{\pgfqpoint{1.020000in}{0.880000in}}{\pgfqpoint{6.160000in}{6.160000in}}%
\pgfusepath{clip}%
\pgfsetbuttcap%
\pgfsetroundjoin%
\definecolor{currentfill}{rgb}{0.667253,0.779176,0.992959}%
\pgfsetfillcolor{currentfill}%
\pgfsetlinewidth{0.000000pt}%
\definecolor{currentstroke}{rgb}{0.000000,0.000000,0.000000}%
\pgfsetstrokecolor{currentstroke}%
\pgfsetdash{}{0pt}%
\pgfpathmoveto{\pgfqpoint{4.270430in}{3.228383in}}%
\pgfpathlineto{\pgfqpoint{4.281298in}{3.181162in}}%
\pgfpathlineto{\pgfqpoint{4.292181in}{3.132963in}}%
\pgfpathlineto{\pgfqpoint{4.326764in}{3.116793in}}%
\pgfpathlineto{\pgfqpoint{4.361310in}{3.098896in}}%
\pgfpathlineto{\pgfqpoint{4.350391in}{3.144947in}}%
\pgfpathlineto{\pgfqpoint{4.339485in}{3.190086in}}%
\pgfpathlineto{\pgfqpoint{4.304976in}{3.210187in}}%
\pgfpathlineto{\pgfqpoint{4.270430in}{3.228383in}}%
\pgfpathclose%
\pgfusepath{fill}%
\end{pgfscope}%
\begin{pgfscope}%
\pgfpathrectangle{\pgfqpoint{1.020000in}{0.880000in}}{\pgfqpoint{6.160000in}{6.160000in}}%
\pgfusepath{clip}%
\pgfsetbuttcap%
\pgfsetroundjoin%
\definecolor{currentfill}{rgb}{0.446431,0.582356,0.957373}%
\pgfsetfillcolor{currentfill}%
\pgfsetlinewidth{0.000000pt}%
\definecolor{currentstroke}{rgb}{0.000000,0.000000,0.000000}%
\pgfsetstrokecolor{currentstroke}%
\pgfsetdash{}{0pt}%
\pgfpathmoveto{\pgfqpoint{4.612114in}{2.813392in}}%
\pgfpathlineto{\pgfqpoint{4.623244in}{2.774993in}}%
\pgfpathlineto{\pgfqpoint{4.634393in}{2.737072in}}%
\pgfpathlineto{\pgfqpoint{4.668818in}{2.722348in}}%
\pgfpathlineto{\pgfqpoint{4.703210in}{2.707709in}}%
\pgfpathlineto{\pgfqpoint{4.692007in}{2.742806in}}%
\pgfpathlineto{\pgfqpoint{4.680824in}{2.778328in}}%
\pgfpathlineto{\pgfqpoint{4.646486in}{2.795810in}}%
\pgfpathlineto{\pgfqpoint{4.612114in}{2.813392in}}%
\pgfpathclose%
\pgfusepath{fill}%
\end{pgfscope}%
\begin{pgfscope}%
\pgfpathrectangle{\pgfqpoint{1.020000in}{0.880000in}}{\pgfqpoint{6.160000in}{6.160000in}}%
\pgfusepath{clip}%
\pgfsetbuttcap%
\pgfsetroundjoin%
\definecolor{currentfill}{rgb}{0.348323,0.465711,0.888346}%
\pgfsetfillcolor{currentfill}%
\pgfsetlinewidth{0.000000pt}%
\definecolor{currentstroke}{rgb}{0.000000,0.000000,0.000000}%
\pgfsetstrokecolor{currentstroke}%
\pgfsetdash{}{0pt}%
\pgfpathmoveto{\pgfqpoint{4.863173in}{2.594365in}}%
\pgfpathlineto{\pgfqpoint{4.874560in}{2.566127in}}%
\pgfpathlineto{\pgfqpoint{4.908895in}{2.557157in}}%
\pgfpathlineto{\pgfqpoint{4.943205in}{2.548723in}}%
\pgfpathlineto{\pgfqpoint{4.931755in}{2.574759in}}%
\pgfpathlineto{\pgfqpoint{4.897477in}{2.584270in}}%
\pgfpathlineto{\pgfqpoint{4.863173in}{2.594365in}}%
\pgfpathclose%
\pgfusepath{fill}%
\end{pgfscope}%
\begin{pgfscope}%
\pgfpathrectangle{\pgfqpoint{1.020000in}{0.880000in}}{\pgfqpoint{6.160000in}{6.160000in}}%
\pgfusepath{clip}%
\pgfsetbuttcap%
\pgfsetroundjoin%
\definecolor{currentfill}{rgb}{0.800601,0.850358,0.930008}%
\pgfsetfillcolor{currentfill}%
\pgfsetlinewidth{0.000000pt}%
\definecolor{currentstroke}{rgb}{0.000000,0.000000,0.000000}%
\pgfsetstrokecolor{currentstroke}%
\pgfsetdash{}{0pt}%
\pgfpathmoveto{\pgfqpoint{4.019567in}{3.471903in}}%
\pgfpathlineto{\pgfqpoint{4.030265in}{3.428733in}}%
\pgfpathlineto{\pgfqpoint{4.040981in}{3.383170in}}%
\pgfpathlineto{\pgfqpoint{4.075675in}{3.379926in}}%
\pgfpathlineto{\pgfqpoint{4.110346in}{3.373506in}}%
\pgfpathlineto{\pgfqpoint{4.099592in}{3.418456in}}%
\pgfpathlineto{\pgfqpoint{4.088855in}{3.461051in}}%
\pgfpathlineto{\pgfqpoint{4.054223in}{3.468173in}}%
\pgfpathlineto{\pgfqpoint{4.019567in}{3.471903in}}%
\pgfpathclose%
\pgfusepath{fill}%
\end{pgfscope}%
\begin{pgfscope}%
\pgfpathrectangle{\pgfqpoint{1.020000in}{0.880000in}}{\pgfqpoint{6.160000in}{6.160000in}}%
\pgfusepath{clip}%
\pgfsetbuttcap%
\pgfsetroundjoin%
\definecolor{currentfill}{rgb}{0.510824,0.649397,0.985079}%
\pgfsetfillcolor{currentfill}%
\pgfsetlinewidth{0.000000pt}%
\definecolor{currentstroke}{rgb}{0.000000,0.000000,0.000000}%
\pgfsetstrokecolor{currentstroke}%
\pgfsetdash{}{0pt}%
\pgfpathmoveto{\pgfqpoint{3.007809in}{2.873932in}}%
\pgfpathlineto{\pgfqpoint{3.017584in}{2.850456in}}%
\pgfpathlineto{\pgfqpoint{3.027382in}{2.826381in}}%
\pgfpathlineto{\pgfqpoint{3.062171in}{2.850209in}}%
\pgfpathlineto{\pgfqpoint{3.096936in}{2.875678in}}%
\pgfpathlineto{\pgfqpoint{3.087072in}{2.900448in}}%
\pgfpathlineto{\pgfqpoint{3.077233in}{2.924433in}}%
\pgfpathlineto{\pgfqpoint{3.042534in}{2.898335in}}%
\pgfpathlineto{\pgfqpoint{3.007809in}{2.873932in}}%
\pgfpathclose%
\pgfusepath{fill}%
\end{pgfscope}%
\begin{pgfscope}%
\pgfpathrectangle{\pgfqpoint{1.020000in}{0.880000in}}{\pgfqpoint{6.160000in}{6.160000in}}%
\pgfusepath{clip}%
\pgfsetbuttcap%
\pgfsetroundjoin%
\definecolor{currentfill}{rgb}{0.603162,0.731527,0.999565}%
\pgfsetfillcolor{currentfill}%
\pgfsetlinewidth{0.000000pt}%
\definecolor{currentstroke}{rgb}{0.000000,0.000000,0.000000}%
\pgfsetstrokecolor{currentstroke}%
\pgfsetdash{}{0pt}%
\pgfpathmoveto{\pgfqpoint{4.361310in}{3.098896in}}%
\pgfpathlineto{\pgfqpoint{4.372243in}{3.052219in}}%
\pgfpathlineto{\pgfqpoint{4.383190in}{3.005193in}}%
\pgfpathlineto{\pgfqpoint{4.417737in}{2.988260in}}%
\pgfpathlineto{\pgfqpoint{4.452247in}{2.970199in}}%
\pgfpathlineto{\pgfqpoint{4.441260in}{3.014676in}}%
\pgfpathlineto{\pgfqpoint{4.430287in}{3.058836in}}%
\pgfpathlineto{\pgfqpoint{4.395818in}{3.079499in}}%
\pgfpathlineto{\pgfqpoint{4.361310in}{3.098896in}}%
\pgfpathclose%
\pgfusepath{fill}%
\end{pgfscope}%
\begin{pgfscope}%
\pgfpathrectangle{\pgfqpoint{1.020000in}{0.880000in}}{\pgfqpoint{6.160000in}{6.160000in}}%
\pgfusepath{clip}%
\pgfsetbuttcap%
\pgfsetroundjoin%
\definecolor{currentfill}{rgb}{0.489246,0.627536,0.976896}%
\pgfsetfillcolor{currentfill}%
\pgfsetlinewidth{0.000000pt}%
\definecolor{currentstroke}{rgb}{0.000000,0.000000,0.000000}%
\pgfsetstrokecolor{currentstroke}%
\pgfsetdash{}{0pt}%
\pgfpathmoveto{\pgfqpoint{2.938277in}{2.830231in}}%
\pgfpathlineto{\pgfqpoint{2.947988in}{2.807170in}}%
\pgfpathlineto{\pgfqpoint{2.957720in}{2.783668in}}%
\pgfpathlineto{\pgfqpoint{2.992565in}{2.804205in}}%
\pgfpathlineto{\pgfqpoint{3.027382in}{2.826381in}}%
\pgfpathlineto{\pgfqpoint{3.017584in}{2.850456in}}%
\pgfpathlineto{\pgfqpoint{3.007809in}{2.873932in}}%
\pgfpathlineto{\pgfqpoint{2.973058in}{2.851234in}}%
\pgfpathlineto{\pgfqpoint{2.938277in}{2.830231in}}%
\pgfpathclose%
\pgfusepath{fill}%
\end{pgfscope}%
\begin{pgfscope}%
\pgfpathrectangle{\pgfqpoint{1.020000in}{0.880000in}}{\pgfqpoint{6.160000in}{6.160000in}}%
\pgfusepath{clip}%
\pgfsetbuttcap%
\pgfsetroundjoin%
\definecolor{currentfill}{rgb}{0.826784,0.858205,0.906953}%
\pgfsetfillcolor{currentfill}%
\pgfsetlinewidth{0.000000pt}%
\definecolor{currentstroke}{rgb}{0.000000,0.000000,0.000000}%
\pgfsetstrokecolor{currentstroke}%
\pgfsetdash{}{0pt}%
\pgfpathmoveto{\pgfqpoint{3.790272in}{3.499211in}}%
\pgfpathlineto{\pgfqpoint{3.800791in}{3.462122in}}%
\pgfpathlineto{\pgfqpoint{3.811335in}{3.422251in}}%
\pgfpathlineto{\pgfqpoint{3.846060in}{3.438796in}}%
\pgfpathlineto{\pgfqpoint{3.880781in}{3.452184in}}%
\pgfpathlineto{\pgfqpoint{3.870190in}{3.492621in}}%
\pgfpathlineto{\pgfqpoint{3.859621in}{3.530173in}}%
\pgfpathlineto{\pgfqpoint{3.824948in}{3.516349in}}%
\pgfpathlineto{\pgfqpoint{3.790272in}{3.499211in}}%
\pgfpathclose%
\pgfusepath{fill}%
\end{pgfscope}%
\begin{pgfscope}%
\pgfpathrectangle{\pgfqpoint{1.020000in}{0.880000in}}{\pgfqpoint{6.160000in}{6.160000in}}%
\pgfusepath{clip}%
\pgfsetbuttcap%
\pgfsetroundjoin%
\definecolor{currentfill}{rgb}{0.753611,0.830233,0.960871}%
\pgfsetfillcolor{currentfill}%
\pgfsetlinewidth{0.000000pt}%
\definecolor{currentstroke}{rgb}{0.000000,0.000000,0.000000}%
\pgfsetstrokecolor{currentstroke}%
\pgfsetdash{}{0pt}%
\pgfpathmoveto{\pgfqpoint{4.110346in}{3.373506in}}%
\pgfpathlineto{\pgfqpoint{4.121115in}{3.326506in}}%
\pgfpathlineto{\pgfqpoint{4.131900in}{3.277776in}}%
\pgfpathlineto{\pgfqpoint{4.166579in}{3.269312in}}%
\pgfpathlineto{\pgfqpoint{4.201229in}{3.258146in}}%
\pgfpathlineto{\pgfqpoint{4.190409in}{3.305702in}}%
\pgfpathlineto{\pgfqpoint{4.179604in}{3.351586in}}%
\pgfpathlineto{\pgfqpoint{4.144990in}{3.364012in}}%
\pgfpathlineto{\pgfqpoint{4.110346in}{3.373506in}}%
\pgfpathclose%
\pgfusepath{fill}%
\end{pgfscope}%
\begin{pgfscope}%
\pgfpathrectangle{\pgfqpoint{1.020000in}{0.880000in}}{\pgfqpoint{6.160000in}{6.160000in}}%
\pgfusepath{clip}%
\pgfsetbuttcap%
\pgfsetroundjoin%
\definecolor{currentfill}{rgb}{0.532568,0.669801,0.990393}%
\pgfsetfillcolor{currentfill}%
\pgfsetlinewidth{0.000000pt}%
\definecolor{currentstroke}{rgb}{0.000000,0.000000,0.000000}%
\pgfsetstrokecolor{currentstroke}%
\pgfsetdash{}{0pt}%
\pgfpathmoveto{\pgfqpoint{4.452247in}{2.970199in}}%
\pgfpathlineto{\pgfqpoint{4.463249in}{2.925653in}}%
\pgfpathlineto{\pgfqpoint{4.474267in}{2.881272in}}%
\pgfpathlineto{\pgfqpoint{4.508783in}{2.865033in}}%
\pgfpathlineto{\pgfqpoint{4.543263in}{2.848182in}}%
\pgfpathlineto{\pgfqpoint{4.532200in}{2.889791in}}%
\pgfpathlineto{\pgfqpoint{4.521153in}{2.931553in}}%
\pgfpathlineto{\pgfqpoint{4.486719in}{2.951225in}}%
\pgfpathlineto{\pgfqpoint{4.452247in}{2.970199in}}%
\pgfpathclose%
\pgfusepath{fill}%
\end{pgfscope}%
\begin{pgfscope}%
\pgfpathrectangle{\pgfqpoint{1.020000in}{0.880000in}}{\pgfqpoint{6.160000in}{6.160000in}}%
\pgfusepath{clip}%
\pgfsetbuttcap%
\pgfsetroundjoin%
\definecolor{currentfill}{rgb}{0.399231,0.528528,0.928459}%
\pgfsetfillcolor{currentfill}%
\pgfsetlinewidth{0.000000pt}%
\definecolor{currentstroke}{rgb}{0.000000,0.000000,0.000000}%
\pgfsetstrokecolor{currentstroke}%
\pgfsetdash{}{0pt}%
\pgfpathmoveto{\pgfqpoint{4.703210in}{2.707709in}}%
\pgfpathlineto{\pgfqpoint{4.714433in}{2.673165in}}%
\pgfpathlineto{\pgfqpoint{4.725677in}{2.639283in}}%
\pgfpathlineto{\pgfqpoint{4.760095in}{2.627534in}}%
\pgfpathlineto{\pgfqpoint{4.794483in}{2.616072in}}%
\pgfpathlineto{\pgfqpoint{4.783180in}{2.647350in}}%
\pgfpathlineto{\pgfqpoint{4.771899in}{2.679207in}}%
\pgfpathlineto{\pgfqpoint{4.737570in}{2.693289in}}%
\pgfpathlineto{\pgfqpoint{4.703210in}{2.707709in}}%
\pgfpathclose%
\pgfusepath{fill}%
\end{pgfscope}%
\begin{pgfscope}%
\pgfpathrectangle{\pgfqpoint{1.020000in}{0.880000in}}{\pgfqpoint{6.160000in}{6.160000in}}%
\pgfusepath{clip}%
\pgfsetbuttcap%
\pgfsetroundjoin%
\definecolor{currentfill}{rgb}{0.467678,0.605591,0.968546}%
\pgfsetfillcolor{currentfill}%
\pgfsetlinewidth{0.000000pt}%
\definecolor{currentstroke}{rgb}{0.000000,0.000000,0.000000}%
\pgfsetstrokecolor{currentstroke}%
\pgfsetdash{}{0pt}%
\pgfpathmoveto{\pgfqpoint{2.868624in}{2.793176in}}%
\pgfpathlineto{\pgfqpoint{2.878272in}{2.770435in}}%
\pgfpathlineto{\pgfqpoint{2.887940in}{2.747388in}}%
\pgfpathlineto{\pgfqpoint{2.922846in}{2.764744in}}%
\pgfpathlineto{\pgfqpoint{2.957720in}{2.783668in}}%
\pgfpathlineto{\pgfqpoint{2.947988in}{2.807170in}}%
\pgfpathlineto{\pgfqpoint{2.938277in}{2.830231in}}%
\pgfpathlineto{\pgfqpoint{2.903466in}{2.810893in}}%
\pgfpathlineto{\pgfqpoint{2.868624in}{2.793176in}}%
\pgfpathclose%
\pgfusepath{fill}%
\end{pgfscope}%
\begin{pgfscope}%
\pgfpathrectangle{\pgfqpoint{1.020000in}{0.880000in}}{\pgfqpoint{6.160000in}{6.160000in}}%
\pgfusepath{clip}%
\pgfsetbuttcap%
\pgfsetroundjoin%
\definecolor{currentfill}{rgb}{0.813693,0.854282,0.918480}%
\pgfsetfillcolor{currentfill}%
\pgfsetlinewidth{0.000000pt}%
\definecolor{currentstroke}{rgb}{0.000000,0.000000,0.000000}%
\pgfsetstrokecolor{currentstroke}%
\pgfsetdash{}{0pt}%
\pgfpathmoveto{\pgfqpoint{3.720921in}{3.455790in}}%
\pgfpathlineto{\pgfqpoint{3.731388in}{3.419441in}}%
\pgfpathlineto{\pgfqpoint{3.741881in}{3.380455in}}%
\pgfpathlineto{\pgfqpoint{3.776608in}{3.402733in}}%
\pgfpathlineto{\pgfqpoint{3.811335in}{3.422251in}}%
\pgfpathlineto{\pgfqpoint{3.800791in}{3.462122in}}%
\pgfpathlineto{\pgfqpoint{3.790272in}{3.499211in}}%
\pgfpathlineto{\pgfqpoint{3.755595in}{3.478949in}}%
\pgfpathlineto{\pgfqpoint{3.720921in}{3.455790in}}%
\pgfpathclose%
\pgfusepath{fill}%
\end{pgfscope}%
\begin{pgfscope}%
\pgfpathrectangle{\pgfqpoint{1.020000in}{0.880000in}}{\pgfqpoint{6.160000in}{6.160000in}}%
\pgfusepath{clip}%
\pgfsetbuttcap%
\pgfsetroundjoin%
\definecolor{currentfill}{rgb}{0.804965,0.851666,0.926165}%
\pgfsetfillcolor{currentfill}%
\pgfsetlinewidth{0.000000pt}%
\definecolor{currentstroke}{rgb}{0.000000,0.000000,0.000000}%
\pgfsetstrokecolor{currentstroke}%
\pgfsetdash{}{0pt}%
\pgfpathmoveto{\pgfqpoint{3.950200in}{3.468956in}}%
\pgfpathlineto{\pgfqpoint{3.960858in}{3.425629in}}%
\pgfpathlineto{\pgfqpoint{3.971534in}{3.379912in}}%
\pgfpathlineto{\pgfqpoint{4.006266in}{3.383173in}}%
\pgfpathlineto{\pgfqpoint{4.040981in}{3.383170in}}%
\pgfpathlineto{\pgfqpoint{4.030265in}{3.428733in}}%
\pgfpathlineto{\pgfqpoint{4.019567in}{3.471903in}}%
\pgfpathlineto{\pgfqpoint{3.984892in}{3.472172in}}%
\pgfpathlineto{\pgfqpoint{3.950200in}{3.468956in}}%
\pgfpathclose%
\pgfusepath{fill}%
\end{pgfscope}%
\begin{pgfscope}%
\pgfpathrectangle{\pgfqpoint{1.020000in}{0.880000in}}{\pgfqpoint{6.160000in}{6.160000in}}%
\pgfusepath{clip}%
\pgfsetbuttcap%
\pgfsetroundjoin%
\definecolor{currentfill}{rgb}{0.363461,0.484784,0.901019}%
\pgfsetfillcolor{currentfill}%
\pgfsetlinewidth{0.000000pt}%
\definecolor{currentstroke}{rgb}{0.000000,0.000000,0.000000}%
\pgfsetstrokecolor{currentstroke}%
\pgfsetdash{}{0pt}%
\pgfpathmoveto{\pgfqpoint{4.794483in}{2.616072in}}%
\pgfpathlineto{\pgfqpoint{4.805808in}{2.585454in}}%
\pgfpathlineto{\pgfqpoint{4.840197in}{2.575580in}}%
\pgfpathlineto{\pgfqpoint{4.874560in}{2.566127in}}%
\pgfpathlineto{\pgfqpoint{4.863173in}{2.594365in}}%
\pgfpathlineto{\pgfqpoint{4.828842in}{2.604989in}}%
\pgfpathlineto{\pgfqpoint{4.794483in}{2.616072in}}%
\pgfpathclose%
\pgfusepath{fill}%
\end{pgfscope}%
\begin{pgfscope}%
\pgfpathrectangle{\pgfqpoint{1.020000in}{0.880000in}}{\pgfqpoint{6.160000in}{6.160000in}}%
\pgfusepath{clip}%
\pgfsetbuttcap%
\pgfsetroundjoin%
\definecolor{currentfill}{rgb}{0.693321,0.796314,0.986308}%
\pgfsetfillcolor{currentfill}%
\pgfsetlinewidth{0.000000pt}%
\definecolor{currentstroke}{rgb}{0.000000,0.000000,0.000000}%
\pgfsetstrokecolor{currentstroke}%
\pgfsetdash{}{0pt}%
\pgfpathmoveto{\pgfqpoint{4.201229in}{3.258146in}}%
\pgfpathlineto{\pgfqpoint{4.212062in}{3.209237in}}%
\pgfpathlineto{\pgfqpoint{4.222909in}{3.159297in}}%
\pgfpathlineto{\pgfqpoint{4.257562in}{3.147195in}}%
\pgfpathlineto{\pgfqpoint{4.292181in}{3.132963in}}%
\pgfpathlineto{\pgfqpoint{4.281298in}{3.181162in}}%
\pgfpathlineto{\pgfqpoint{4.270430in}{3.228383in}}%
\pgfpathlineto{\pgfqpoint{4.235847in}{3.244440in}}%
\pgfpathlineto{\pgfqpoint{4.201229in}{3.258146in}}%
\pgfpathclose%
\pgfusepath{fill}%
\end{pgfscope}%
\begin{pgfscope}%
\pgfpathrectangle{\pgfqpoint{1.020000in}{0.880000in}}{\pgfqpoint{6.160000in}{6.160000in}}%
\pgfusepath{clip}%
\pgfsetbuttcap%
\pgfsetroundjoin%
\definecolor{currentfill}{rgb}{0.791392,0.846750,0.936641}%
\pgfsetfillcolor{currentfill}%
\pgfsetlinewidth{0.000000pt}%
\definecolor{currentstroke}{rgb}{0.000000,0.000000,0.000000}%
\pgfsetstrokecolor{currentstroke}%
\pgfsetdash{}{0pt}%
\pgfpathmoveto{\pgfqpoint{3.651584in}{3.401830in}}%
\pgfpathlineto{\pgfqpoint{3.661995in}{3.366453in}}%
\pgfpathlineto{\pgfqpoint{3.672433in}{3.328623in}}%
\pgfpathlineto{\pgfqpoint{3.707155in}{3.355663in}}%
\pgfpathlineto{\pgfqpoint{3.741881in}{3.380455in}}%
\pgfpathlineto{\pgfqpoint{3.731388in}{3.419441in}}%
\pgfpathlineto{\pgfqpoint{3.720921in}{3.455790in}}%
\pgfpathlineto{\pgfqpoint{3.686250in}{3.429990in}}%
\pgfpathlineto{\pgfqpoint{3.651584in}{3.401830in}}%
\pgfpathclose%
\pgfusepath{fill}%
\end{pgfscope}%
\begin{pgfscope}%
\pgfpathrectangle{\pgfqpoint{1.020000in}{0.880000in}}{\pgfqpoint{6.160000in}{6.160000in}}%
\pgfusepath{clip}%
\pgfsetbuttcap%
\pgfsetroundjoin%
\definecolor{currentfill}{rgb}{0.473070,0.611077,0.970634}%
\pgfsetfillcolor{currentfill}%
\pgfsetlinewidth{0.000000pt}%
\definecolor{currentstroke}{rgb}{0.000000,0.000000,0.000000}%
\pgfsetstrokecolor{currentstroke}%
\pgfsetdash{}{0pt}%
\pgfpathmoveto{\pgfqpoint{4.543263in}{2.848182in}}%
\pgfpathlineto{\pgfqpoint{4.554343in}{2.806924in}}%
\pgfpathlineto{\pgfqpoint{4.565441in}{2.766197in}}%
\pgfpathlineto{\pgfqpoint{4.599934in}{2.751739in}}%
\pgfpathlineto{\pgfqpoint{4.634393in}{2.737072in}}%
\pgfpathlineto{\pgfqpoint{4.623244in}{2.774993in}}%
\pgfpathlineto{\pgfqpoint{4.612114in}{2.813392in}}%
\pgfpathlineto{\pgfqpoint{4.577706in}{2.830908in}}%
\pgfpathlineto{\pgfqpoint{4.543263in}{2.848182in}}%
\pgfpathclose%
\pgfusepath{fill}%
\end{pgfscope}%
\begin{pgfscope}%
\pgfpathrectangle{\pgfqpoint{1.020000in}{0.880000in}}{\pgfqpoint{6.160000in}{6.160000in}}%
\pgfusepath{clip}%
\pgfsetbuttcap%
\pgfsetroundjoin%
\definecolor{currentfill}{rgb}{0.451739,0.588181,0.960201}%
\pgfsetfillcolor{currentfill}%
\pgfsetlinewidth{0.000000pt}%
\definecolor{currentstroke}{rgb}{0.000000,0.000000,0.000000}%
\pgfsetstrokecolor{currentstroke}%
\pgfsetdash{}{0pt}%
\pgfpathmoveto{\pgfqpoint{2.798842in}{2.762362in}}%
\pgfpathlineto{\pgfqpoint{2.808429in}{2.739853in}}%
\pgfpathlineto{\pgfqpoint{2.818034in}{2.717148in}}%
\pgfpathlineto{\pgfqpoint{2.853003in}{2.731545in}}%
\pgfpathlineto{\pgfqpoint{2.887940in}{2.747388in}}%
\pgfpathlineto{\pgfqpoint{2.878272in}{2.770435in}}%
\pgfpathlineto{\pgfqpoint{2.868624in}{2.793176in}}%
\pgfpathlineto{\pgfqpoint{2.833750in}{2.777022in}}%
\pgfpathlineto{\pgfqpoint{2.798842in}{2.762362in}}%
\pgfpathclose%
\pgfusepath{fill}%
\end{pgfscope}%
\begin{pgfscope}%
\pgfpathrectangle{\pgfqpoint{1.020000in}{0.880000in}}{\pgfqpoint{6.160000in}{6.160000in}}%
\pgfusepath{clip}%
\pgfsetbuttcap%
\pgfsetroundjoin%
\definecolor{currentfill}{rgb}{0.763363,0.835092,0.955658}%
\pgfsetfillcolor{currentfill}%
\pgfsetlinewidth{0.000000pt}%
\definecolor{currentstroke}{rgb}{0.000000,0.000000,0.000000}%
\pgfsetstrokecolor{currentstroke}%
\pgfsetdash{}{0pt}%
\pgfpathmoveto{\pgfqpoint{3.582267in}{3.339650in}}%
\pgfpathlineto{\pgfqpoint{3.592620in}{3.305428in}}%
\pgfpathlineto{\pgfqpoint{3.603000in}{3.268964in}}%
\pgfpathlineto{\pgfqpoint{3.637715in}{3.299623in}}%
\pgfpathlineto{\pgfqpoint{3.672433in}{3.328623in}}%
\pgfpathlineto{\pgfqpoint{3.661995in}{3.366453in}}%
\pgfpathlineto{\pgfqpoint{3.651584in}{3.401830in}}%
\pgfpathlineto{\pgfqpoint{3.616923in}{3.371611in}}%
\pgfpathlineto{\pgfqpoint{3.582267in}{3.339650in}}%
\pgfpathclose%
\pgfusepath{fill}%
\end{pgfscope}%
\begin{pgfscope}%
\pgfpathrectangle{\pgfqpoint{1.020000in}{0.880000in}}{\pgfqpoint{6.160000in}{6.160000in}}%
\pgfusepath{clip}%
\pgfsetbuttcap%
\pgfsetroundjoin%
\definecolor{currentfill}{rgb}{0.733898,0.820018,0.970724}%
\pgfsetfillcolor{currentfill}%
\pgfsetlinewidth{0.000000pt}%
\definecolor{currentstroke}{rgb}{0.000000,0.000000,0.000000}%
\pgfsetstrokecolor{currentstroke}%
\pgfsetdash{}{0pt}%
\pgfpathmoveto{\pgfqpoint{3.512972in}{3.271800in}}%
\pgfpathlineto{\pgfqpoint{3.523263in}{3.238858in}}%
\pgfpathlineto{\pgfqpoint{3.533582in}{3.203904in}}%
\pgfpathlineto{\pgfqpoint{3.568289in}{3.236954in}}%
\pgfpathlineto{\pgfqpoint{3.603000in}{3.268964in}}%
\pgfpathlineto{\pgfqpoint{3.592620in}{3.305428in}}%
\pgfpathlineto{\pgfqpoint{3.582267in}{3.339650in}}%
\pgfpathlineto{\pgfqpoint{3.547617in}{3.306271in}}%
\pgfpathlineto{\pgfqpoint{3.512972in}{3.271800in}}%
\pgfpathclose%
\pgfusepath{fill}%
\end{pgfscope}%
\begin{pgfscope}%
\pgfpathrectangle{\pgfqpoint{1.020000in}{0.880000in}}{\pgfqpoint{6.160000in}{6.160000in}}%
\pgfusepath{clip}%
\pgfsetbuttcap%
\pgfsetroundjoin%
\definecolor{currentfill}{rgb}{0.624703,0.748318,0.998719}%
\pgfsetfillcolor{currentfill}%
\pgfsetlinewidth{0.000000pt}%
\definecolor{currentstroke}{rgb}{0.000000,0.000000,0.000000}%
\pgfsetstrokecolor{currentstroke}%
\pgfsetdash{}{0pt}%
\pgfpathmoveto{\pgfqpoint{4.292181in}{3.132963in}}%
\pgfpathlineto{\pgfqpoint{4.303076in}{3.084091in}}%
\pgfpathlineto{\pgfqpoint{4.313985in}{3.034844in}}%
\pgfpathlineto{\pgfqpoint{4.348605in}{3.020788in}}%
\pgfpathlineto{\pgfqpoint{4.383190in}{3.005193in}}%
\pgfpathlineto{\pgfqpoint{4.372243in}{3.052219in}}%
\pgfpathlineto{\pgfqpoint{4.361310in}{3.098896in}}%
\pgfpathlineto{\pgfqpoint{4.326764in}{3.116793in}}%
\pgfpathlineto{\pgfqpoint{4.292181in}{3.132963in}}%
\pgfpathclose%
\pgfusepath{fill}%
\end{pgfscope}%
\begin{pgfscope}%
\pgfpathrectangle{\pgfqpoint{1.020000in}{0.880000in}}{\pgfqpoint{6.160000in}{6.160000in}}%
\pgfusepath{clip}%
\pgfsetbuttcap%
\pgfsetroundjoin%
\definecolor{currentfill}{rgb}{0.698454,0.799450,0.984577}%
\pgfsetfillcolor{currentfill}%
\pgfsetlinewidth{0.000000pt}%
\definecolor{currentstroke}{rgb}{0.000000,0.000000,0.000000}%
\pgfsetstrokecolor{currentstroke}%
\pgfsetdash{}{0pt}%
\pgfpathmoveto{\pgfqpoint{3.443690in}{3.200874in}}%
\pgfpathlineto{\pgfqpoint{3.453917in}{3.169280in}}%
\pgfpathlineto{\pgfqpoint{3.464172in}{3.135914in}}%
\pgfpathlineto{\pgfqpoint{3.498876in}{3.170123in}}%
\pgfpathlineto{\pgfqpoint{3.533582in}{3.203904in}}%
\pgfpathlineto{\pgfqpoint{3.523263in}{3.238858in}}%
\pgfpathlineto{\pgfqpoint{3.512972in}{3.271800in}}%
\pgfpathlineto{\pgfqpoint{3.478330in}{3.236562in}}%
\pgfpathlineto{\pgfqpoint{3.443690in}{3.200874in}}%
\pgfpathclose%
\pgfusepath{fill}%
\end{pgfscope}%
\begin{pgfscope}%
\pgfpathrectangle{\pgfqpoint{1.020000in}{0.880000in}}{\pgfqpoint{6.160000in}{6.160000in}}%
\pgfusepath{clip}%
\pgfsetbuttcap%
\pgfsetroundjoin%
\definecolor{currentfill}{rgb}{0.661968,0.775491,0.993937}%
\pgfsetfillcolor{currentfill}%
\pgfsetlinewidth{0.000000pt}%
\definecolor{currentstroke}{rgb}{0.000000,0.000000,0.000000}%
\pgfsetstrokecolor{currentstroke}%
\pgfsetdash{}{0pt}%
\pgfpathmoveto{\pgfqpoint{3.374410in}{3.129345in}}%
\pgfpathlineto{\pgfqpoint{3.384571in}{3.099112in}}%
\pgfpathlineto{\pgfqpoint{3.394759in}{3.067348in}}%
\pgfpathlineto{\pgfqpoint{3.429467in}{3.101564in}}%
\pgfpathlineto{\pgfqpoint{3.464172in}{3.135914in}}%
\pgfpathlineto{\pgfqpoint{3.453917in}{3.169280in}}%
\pgfpathlineto{\pgfqpoint{3.443690in}{3.200874in}}%
\pgfpathlineto{\pgfqpoint{3.409051in}{3.165039in}}%
\pgfpathlineto{\pgfqpoint{3.374410in}{3.129345in}}%
\pgfpathclose%
\pgfusepath{fill}%
\end{pgfscope}%
\begin{pgfscope}%
\pgfpathrectangle{\pgfqpoint{1.020000in}{0.880000in}}{\pgfqpoint{6.160000in}{6.160000in}}%
\pgfusepath{clip}%
\pgfsetbuttcap%
\pgfsetroundjoin%
\definecolor{currentfill}{rgb}{0.624703,0.748318,0.998719}%
\pgfsetfillcolor{currentfill}%
\pgfsetlinewidth{0.000000pt}%
\definecolor{currentstroke}{rgb}{0.000000,0.000000,0.000000}%
\pgfsetstrokecolor{currentstroke}%
\pgfsetdash{}{0pt}%
\pgfpathmoveto{\pgfqpoint{3.305115in}{3.059426in}}%
\pgfpathlineto{\pgfqpoint{3.315208in}{3.030518in}}%
\pgfpathlineto{\pgfqpoint{3.325328in}{3.000316in}}%
\pgfpathlineto{\pgfqpoint{3.360047in}{3.033522in}}%
\pgfpathlineto{\pgfqpoint{3.394759in}{3.067348in}}%
\pgfpathlineto{\pgfqpoint{3.384571in}{3.099112in}}%
\pgfpathlineto{\pgfqpoint{3.374410in}{3.129345in}}%
\pgfpathlineto{\pgfqpoint{3.339765in}{3.094059in}}%
\pgfpathlineto{\pgfqpoint{3.305115in}{3.059426in}}%
\pgfpathclose%
\pgfusepath{fill}%
\end{pgfscope}%
\begin{pgfscope}%
\pgfpathrectangle{\pgfqpoint{1.020000in}{0.880000in}}{\pgfqpoint{6.160000in}{6.160000in}}%
\pgfusepath{clip}%
\pgfsetbuttcap%
\pgfsetroundjoin%
\definecolor{currentfill}{rgb}{0.804965,0.851666,0.926165}%
\pgfsetfillcolor{currentfill}%
\pgfsetlinewidth{0.000000pt}%
\definecolor{currentstroke}{rgb}{0.000000,0.000000,0.000000}%
\pgfsetstrokecolor{currentstroke}%
\pgfsetdash{}{0pt}%
\pgfpathmoveto{\pgfqpoint{3.880781in}{3.452184in}}%
\pgfpathlineto{\pgfqpoint{3.891396in}{3.409119in}}%
\pgfpathlineto{\pgfqpoint{3.902030in}{3.363707in}}%
\pgfpathlineto{\pgfqpoint{3.936788in}{3.373403in}}%
\pgfpathlineto{\pgfqpoint{3.971534in}{3.379912in}}%
\pgfpathlineto{\pgfqpoint{3.960858in}{3.425629in}}%
\pgfpathlineto{\pgfqpoint{3.950200in}{3.468956in}}%
\pgfpathlineto{\pgfqpoint{3.915496in}{3.462272in}}%
\pgfpathlineto{\pgfqpoint{3.880781in}{3.452184in}}%
\pgfpathclose%
\pgfusepath{fill}%
\end{pgfscope}%
\begin{pgfscope}%
\pgfpathrectangle{\pgfqpoint{1.020000in}{0.880000in}}{\pgfqpoint{6.160000in}{6.160000in}}%
\pgfusepath{clip}%
\pgfsetbuttcap%
\pgfsetroundjoin%
\definecolor{currentfill}{rgb}{0.763363,0.835092,0.955658}%
\pgfsetfillcolor{currentfill}%
\pgfsetlinewidth{0.000000pt}%
\definecolor{currentstroke}{rgb}{0.000000,0.000000,0.000000}%
\pgfsetstrokecolor{currentstroke}%
\pgfsetdash{}{0pt}%
\pgfpathmoveto{\pgfqpoint{4.040981in}{3.383170in}}%
\pgfpathlineto{\pgfqpoint{4.051714in}{3.335523in}}%
\pgfpathlineto{\pgfqpoint{4.062463in}{3.286118in}}%
\pgfpathlineto{\pgfqpoint{4.097193in}{3.283410in}}%
\pgfpathlineto{\pgfqpoint{4.131900in}{3.277776in}}%
\pgfpathlineto{\pgfqpoint{4.121115in}{3.326506in}}%
\pgfpathlineto{\pgfqpoint{4.110346in}{3.373506in}}%
\pgfpathlineto{\pgfqpoint{4.075675in}{3.379926in}}%
\pgfpathlineto{\pgfqpoint{4.040981in}{3.383170in}}%
\pgfpathclose%
\pgfusepath{fill}%
\end{pgfscope}%
\begin{pgfscope}%
\pgfpathrectangle{\pgfqpoint{1.020000in}{0.880000in}}{\pgfqpoint{6.160000in}{6.160000in}}%
\pgfusepath{clip}%
\pgfsetbuttcap%
\pgfsetroundjoin%
\definecolor{currentfill}{rgb}{0.586921,0.718121,0.998874}%
\pgfsetfillcolor{currentfill}%
\pgfsetlinewidth{0.000000pt}%
\definecolor{currentstroke}{rgb}{0.000000,0.000000,0.000000}%
\pgfsetstrokecolor{currentstroke}%
\pgfsetdash{}{0pt}%
\pgfpathmoveto{\pgfqpoint{3.235784in}{2.992966in}}%
\pgfpathlineto{\pgfqpoint{3.245809in}{2.965308in}}%
\pgfpathlineto{\pgfqpoint{3.255860in}{2.936577in}}%
\pgfpathlineto{\pgfqpoint{3.290600in}{2.967940in}}%
\pgfpathlineto{\pgfqpoint{3.325328in}{3.000316in}}%
\pgfpathlineto{\pgfqpoint{3.315208in}{3.030518in}}%
\pgfpathlineto{\pgfqpoint{3.305115in}{3.059426in}}%
\pgfpathlineto{\pgfqpoint{3.270455in}{3.025664in}}%
\pgfpathlineto{\pgfqpoint{3.235784in}{2.992966in}}%
\pgfpathclose%
\pgfusepath{fill}%
\end{pgfscope}%
\begin{pgfscope}%
\pgfpathrectangle{\pgfqpoint{1.020000in}{0.880000in}}{\pgfqpoint{6.160000in}{6.160000in}}%
\pgfusepath{clip}%
\pgfsetbuttcap%
\pgfsetroundjoin%
\definecolor{currentfill}{rgb}{0.554312,0.690097,0.995516}%
\pgfsetfillcolor{currentfill}%
\pgfsetlinewidth{0.000000pt}%
\definecolor{currentstroke}{rgb}{0.000000,0.000000,0.000000}%
\pgfsetstrokecolor{currentstroke}%
\pgfsetdash{}{0pt}%
\pgfpathmoveto{\pgfqpoint{3.166397in}{2.931391in}}%
\pgfpathlineto{\pgfqpoint{3.176354in}{2.904873in}}%
\pgfpathlineto{\pgfqpoint{3.186335in}{2.877490in}}%
\pgfpathlineto{\pgfqpoint{3.221105in}{2.906384in}}%
\pgfpathlineto{\pgfqpoint{3.255860in}{2.936577in}}%
\pgfpathlineto{\pgfqpoint{3.245809in}{2.965308in}}%
\pgfpathlineto{\pgfqpoint{3.235784in}{2.992966in}}%
\pgfpathlineto{\pgfqpoint{3.201099in}{2.961496in}}%
\pgfpathlineto{\pgfqpoint{3.166397in}{2.931391in}}%
\pgfpathclose%
\pgfusepath{fill}%
\end{pgfscope}%
\begin{pgfscope}%
\pgfpathrectangle{\pgfqpoint{1.020000in}{0.880000in}}{\pgfqpoint{6.160000in}{6.160000in}}%
\pgfusepath{clip}%
\pgfsetbuttcap%
\pgfsetroundjoin%
\definecolor{currentfill}{rgb}{0.419991,0.552989,0.942630}%
\pgfsetfillcolor{currentfill}%
\pgfsetlinewidth{0.000000pt}%
\definecolor{currentstroke}{rgb}{0.000000,0.000000,0.000000}%
\pgfsetstrokecolor{currentstroke}%
\pgfsetdash{}{0pt}%
\pgfpathmoveto{\pgfqpoint{4.634393in}{2.737072in}}%
\pgfpathlineto{\pgfqpoint{4.645561in}{2.699773in}}%
\pgfpathlineto{\pgfqpoint{4.656750in}{2.663223in}}%
\pgfpathlineto{\pgfqpoint{4.691229in}{2.651217in}}%
\pgfpathlineto{\pgfqpoint{4.725677in}{2.639283in}}%
\pgfpathlineto{\pgfqpoint{4.714433in}{2.673165in}}%
\pgfpathlineto{\pgfqpoint{4.703210in}{2.707709in}}%
\pgfpathlineto{\pgfqpoint{4.668818in}{2.722348in}}%
\pgfpathlineto{\pgfqpoint{4.634393in}{2.737072in}}%
\pgfpathclose%
\pgfusepath{fill}%
\end{pgfscope}%
\begin{pgfscope}%
\pgfpathrectangle{\pgfqpoint{1.020000in}{0.880000in}}{\pgfqpoint{6.160000in}{6.160000in}}%
\pgfusepath{clip}%
\pgfsetbuttcap%
\pgfsetroundjoin%
\definecolor{currentfill}{rgb}{0.559747,0.694768,0.996075}%
\pgfsetfillcolor{currentfill}%
\pgfsetlinewidth{0.000000pt}%
\definecolor{currentstroke}{rgb}{0.000000,0.000000,0.000000}%
\pgfsetstrokecolor{currentstroke}%
\pgfsetdash{}{0pt}%
\pgfpathmoveto{\pgfqpoint{4.383190in}{3.005193in}}%
\pgfpathlineto{\pgfqpoint{4.394151in}{2.958088in}}%
\pgfpathlineto{\pgfqpoint{4.405127in}{2.911157in}}%
\pgfpathlineto{\pgfqpoint{4.439715in}{2.896709in}}%
\pgfpathlineto{\pgfqpoint{4.474267in}{2.881272in}}%
\pgfpathlineto{\pgfqpoint{4.463249in}{2.925653in}}%
\pgfpathlineto{\pgfqpoint{4.452247in}{2.970199in}}%
\pgfpathlineto{\pgfqpoint{4.417737in}{2.988260in}}%
\pgfpathlineto{\pgfqpoint{4.383190in}{3.005193in}}%
\pgfpathclose%
\pgfusepath{fill}%
\end{pgfscope}%
\begin{pgfscope}%
\pgfpathrectangle{\pgfqpoint{1.020000in}{0.880000in}}{\pgfqpoint{6.160000in}{6.160000in}}%
\pgfusepath{clip}%
\pgfsetbuttcap%
\pgfsetroundjoin%
\definecolor{currentfill}{rgb}{0.521696,0.659599,0.987736}%
\pgfsetfillcolor{currentfill}%
\pgfsetlinewidth{0.000000pt}%
\definecolor{currentstroke}{rgb}{0.000000,0.000000,0.000000}%
\pgfsetstrokecolor{currentstroke}%
\pgfsetdash{}{0pt}%
\pgfpathmoveto{\pgfqpoint{3.096936in}{2.875678in}}%
\pgfpathlineto{\pgfqpoint{3.106824in}{2.850172in}}%
\pgfpathlineto{\pgfqpoint{3.116736in}{2.823986in}}%
\pgfpathlineto{\pgfqpoint{3.151546in}{2.849998in}}%
\pgfpathlineto{\pgfqpoint{3.186335in}{2.877490in}}%
\pgfpathlineto{\pgfqpoint{3.176354in}{2.904873in}}%
\pgfpathlineto{\pgfqpoint{3.166397in}{2.931391in}}%
\pgfpathlineto{\pgfqpoint{3.131677in}{2.902757in}}%
\pgfpathlineto{\pgfqpoint{3.096936in}{2.875678in}}%
\pgfpathclose%
\pgfusepath{fill}%
\end{pgfscope}%
\begin{pgfscope}%
\pgfpathrectangle{\pgfqpoint{1.020000in}{0.880000in}}{\pgfqpoint{6.160000in}{6.160000in}}%
\pgfusepath{clip}%
\pgfsetbuttcap%
\pgfsetroundjoin%
\definecolor{currentfill}{rgb}{0.378598,0.503856,0.913692}%
\pgfsetfillcolor{currentfill}%
\pgfsetlinewidth{0.000000pt}%
\definecolor{currentstroke}{rgb}{0.000000,0.000000,0.000000}%
\pgfsetstrokecolor{currentstroke}%
\pgfsetdash{}{0pt}%
\pgfpathmoveto{\pgfqpoint{4.725677in}{2.639283in}}%
\pgfpathlineto{\pgfqpoint{4.736943in}{2.606158in}}%
\pgfpathlineto{\pgfqpoint{4.771390in}{2.595675in}}%
\pgfpathlineto{\pgfqpoint{4.805808in}{2.585454in}}%
\pgfpathlineto{\pgfqpoint{4.794483in}{2.616072in}}%
\pgfpathlineto{\pgfqpoint{4.760095in}{2.627534in}}%
\pgfpathlineto{\pgfqpoint{4.725677in}{2.639283in}}%
\pgfpathclose%
\pgfusepath{fill}%
\end{pgfscope}%
\begin{pgfscope}%
\pgfpathrectangle{\pgfqpoint{1.020000in}{0.880000in}}{\pgfqpoint{6.160000in}{6.160000in}}%
\pgfusepath{clip}%
\pgfsetbuttcap%
\pgfsetroundjoin%
\definecolor{currentfill}{rgb}{0.796064,0.848693,0.933471}%
\pgfsetfillcolor{currentfill}%
\pgfsetlinewidth{0.000000pt}%
\definecolor{currentstroke}{rgb}{0.000000,0.000000,0.000000}%
\pgfsetstrokecolor{currentstroke}%
\pgfsetdash{}{0pt}%
\pgfpathmoveto{\pgfqpoint{3.811335in}{3.422251in}}%
\pgfpathlineto{\pgfqpoint{3.821903in}{3.379845in}}%
\pgfpathlineto{\pgfqpoint{3.832492in}{3.335177in}}%
\pgfpathlineto{\pgfqpoint{3.867264in}{3.350919in}}%
\pgfpathlineto{\pgfqpoint{3.902030in}{3.363707in}}%
\pgfpathlineto{\pgfqpoint{3.891396in}{3.409119in}}%
\pgfpathlineto{\pgfqpoint{3.880781in}{3.452184in}}%
\pgfpathlineto{\pgfqpoint{3.846060in}{3.438796in}}%
\pgfpathlineto{\pgfqpoint{3.811335in}{3.422251in}}%
\pgfpathclose%
\pgfusepath{fill}%
\end{pgfscope}%
\begin{pgfscope}%
\pgfpathrectangle{\pgfqpoint{1.020000in}{0.880000in}}{\pgfqpoint{6.160000in}{6.160000in}}%
\pgfusepath{clip}%
\pgfsetbuttcap%
\pgfsetroundjoin%
\definecolor{currentfill}{rgb}{0.708720,0.805721,0.981117}%
\pgfsetfillcolor{currentfill}%
\pgfsetlinewidth{0.000000pt}%
\definecolor{currentstroke}{rgb}{0.000000,0.000000,0.000000}%
\pgfsetstrokecolor{currentstroke}%
\pgfsetdash{}{0pt}%
\pgfpathmoveto{\pgfqpoint{4.131900in}{3.277776in}}%
\pgfpathlineto{\pgfqpoint{4.142698in}{3.227646in}}%
\pgfpathlineto{\pgfqpoint{4.153511in}{3.176451in}}%
\pgfpathlineto{\pgfqpoint{4.188225in}{3.169098in}}%
\pgfpathlineto{\pgfqpoint{4.222909in}{3.159297in}}%
\pgfpathlineto{\pgfqpoint{4.212062in}{3.209237in}}%
\pgfpathlineto{\pgfqpoint{4.201229in}{3.258146in}}%
\pgfpathlineto{\pgfqpoint{4.166579in}{3.269312in}}%
\pgfpathlineto{\pgfqpoint{4.131900in}{3.277776in}}%
\pgfpathclose%
\pgfusepath{fill}%
\end{pgfscope}%
\begin{pgfscope}%
\pgfpathrectangle{\pgfqpoint{1.020000in}{0.880000in}}{\pgfqpoint{6.160000in}{6.160000in}}%
\pgfusepath{clip}%
\pgfsetbuttcap%
\pgfsetroundjoin%
\definecolor{currentfill}{rgb}{0.494638,0.633022,0.978983}%
\pgfsetfillcolor{currentfill}%
\pgfsetlinewidth{0.000000pt}%
\definecolor{currentstroke}{rgb}{0.000000,0.000000,0.000000}%
\pgfsetstrokecolor{currentstroke}%
\pgfsetdash{}{0pt}%
\pgfpathmoveto{\pgfqpoint{3.027382in}{2.826381in}}%
\pgfpathlineto{\pgfqpoint{3.037202in}{2.801745in}}%
\pgfpathlineto{\pgfqpoint{3.047045in}{2.776592in}}%
\pgfpathlineto{\pgfqpoint{3.081903in}{2.799507in}}%
\pgfpathlineto{\pgfqpoint{3.116736in}{2.823986in}}%
\pgfpathlineto{\pgfqpoint{3.106824in}{2.850172in}}%
\pgfpathlineto{\pgfqpoint{3.096936in}{2.875678in}}%
\pgfpathlineto{\pgfqpoint{3.062171in}{2.850209in}}%
\pgfpathlineto{\pgfqpoint{3.027382in}{2.826381in}}%
\pgfpathclose%
\pgfusepath{fill}%
\end{pgfscope}%
\begin{pgfscope}%
\pgfpathrectangle{\pgfqpoint{1.020000in}{0.880000in}}{\pgfqpoint{6.160000in}{6.160000in}}%
\pgfusepath{clip}%
\pgfsetbuttcap%
\pgfsetroundjoin%
\definecolor{currentfill}{rgb}{0.494638,0.633022,0.978983}%
\pgfsetfillcolor{currentfill}%
\pgfsetlinewidth{0.000000pt}%
\definecolor{currentstroke}{rgb}{0.000000,0.000000,0.000000}%
\pgfsetstrokecolor{currentstroke}%
\pgfsetdash{}{0pt}%
\pgfpathmoveto{\pgfqpoint{4.474267in}{2.881272in}}%
\pgfpathlineto{\pgfqpoint{4.485301in}{2.837272in}}%
\pgfpathlineto{\pgfqpoint{4.496353in}{2.793852in}}%
\pgfpathlineto{\pgfqpoint{4.530914in}{2.780289in}}%
\pgfpathlineto{\pgfqpoint{4.565441in}{2.766197in}}%
\pgfpathlineto{\pgfqpoint{4.554343in}{2.806924in}}%
\pgfpathlineto{\pgfqpoint{4.543263in}{2.848182in}}%
\pgfpathlineto{\pgfqpoint{4.508783in}{2.865033in}}%
\pgfpathlineto{\pgfqpoint{4.474267in}{2.881272in}}%
\pgfpathclose%
\pgfusepath{fill}%
\end{pgfscope}%
\begin{pgfscope}%
\pgfpathrectangle{\pgfqpoint{1.020000in}{0.880000in}}{\pgfqpoint{6.160000in}{6.160000in}}%
\pgfusepath{clip}%
\pgfsetbuttcap%
\pgfsetroundjoin%
\definecolor{currentfill}{rgb}{0.768034,0.837035,0.952488}%
\pgfsetfillcolor{currentfill}%
\pgfsetlinewidth{0.000000pt}%
\definecolor{currentstroke}{rgb}{0.000000,0.000000,0.000000}%
\pgfsetstrokecolor{currentstroke}%
\pgfsetdash{}{0pt}%
\pgfpathmoveto{\pgfqpoint{3.971534in}{3.379912in}}%
\pgfpathlineto{\pgfqpoint{3.982229in}{3.332113in}}%
\pgfpathlineto{\pgfqpoint{3.992941in}{3.282558in}}%
\pgfpathlineto{\pgfqpoint{4.027711in}{3.285841in}}%
\pgfpathlineto{\pgfqpoint{4.062463in}{3.286118in}}%
\pgfpathlineto{\pgfqpoint{4.051714in}{3.335523in}}%
\pgfpathlineto{\pgfqpoint{4.040981in}{3.383170in}}%
\pgfpathlineto{\pgfqpoint{4.006266in}{3.383173in}}%
\pgfpathlineto{\pgfqpoint{3.971534in}{3.379912in}}%
\pgfpathclose%
\pgfusepath{fill}%
\end{pgfscope}%
\begin{pgfscope}%
\pgfpathrectangle{\pgfqpoint{1.020000in}{0.880000in}}{\pgfqpoint{6.160000in}{6.160000in}}%
\pgfusepath{clip}%
\pgfsetbuttcap%
\pgfsetroundjoin%
\definecolor{currentfill}{rgb}{0.473070,0.611077,0.970634}%
\pgfsetfillcolor{currentfill}%
\pgfsetlinewidth{0.000000pt}%
\definecolor{currentstroke}{rgb}{0.000000,0.000000,0.000000}%
\pgfsetstrokecolor{currentstroke}%
\pgfsetdash{}{0pt}%
\pgfpathmoveto{\pgfqpoint{2.957720in}{2.783668in}}%
\pgfpathlineto{\pgfqpoint{2.967474in}{2.759755in}}%
\pgfpathlineto{\pgfqpoint{2.977249in}{2.735467in}}%
\pgfpathlineto{\pgfqpoint{3.012161in}{2.755248in}}%
\pgfpathlineto{\pgfqpoint{3.047045in}{2.776592in}}%
\pgfpathlineto{\pgfqpoint{3.037202in}{2.801745in}}%
\pgfpathlineto{\pgfqpoint{3.027382in}{2.826381in}}%
\pgfpathlineto{\pgfqpoint{2.992565in}{2.804205in}}%
\pgfpathlineto{\pgfqpoint{2.957720in}{2.783668in}}%
\pgfpathclose%
\pgfusepath{fill}%
\end{pgfscope}%
\begin{pgfscope}%
\pgfpathrectangle{\pgfqpoint{1.020000in}{0.880000in}}{\pgfqpoint{6.160000in}{6.160000in}}%
\pgfusepath{clip}%
\pgfsetbuttcap%
\pgfsetroundjoin%
\definecolor{currentfill}{rgb}{0.782049,0.842864,0.942980}%
\pgfsetfillcolor{currentfill}%
\pgfsetlinewidth{0.000000pt}%
\definecolor{currentstroke}{rgb}{0.000000,0.000000,0.000000}%
\pgfsetstrokecolor{currentstroke}%
\pgfsetdash{}{0pt}%
\pgfpathmoveto{\pgfqpoint{3.741881in}{3.380455in}}%
\pgfpathlineto{\pgfqpoint{3.752399in}{3.339068in}}%
\pgfpathlineto{\pgfqpoint{3.762940in}{3.295537in}}%
\pgfpathlineto{\pgfqpoint{3.797717in}{3.316649in}}%
\pgfpathlineto{\pgfqpoint{3.832492in}{3.335177in}}%
\pgfpathlineto{\pgfqpoint{3.821903in}{3.379845in}}%
\pgfpathlineto{\pgfqpoint{3.811335in}{3.422251in}}%
\pgfpathlineto{\pgfqpoint{3.776608in}{3.402733in}}%
\pgfpathlineto{\pgfqpoint{3.741881in}{3.380455in}}%
\pgfpathclose%
\pgfusepath{fill}%
\end{pgfscope}%
\begin{pgfscope}%
\pgfpathrectangle{\pgfqpoint{1.020000in}{0.880000in}}{\pgfqpoint{6.160000in}{6.160000in}}%
\pgfusepath{clip}%
\pgfsetbuttcap%
\pgfsetroundjoin%
\definecolor{currentfill}{rgb}{0.646113,0.764436,0.996868}%
\pgfsetfillcolor{currentfill}%
\pgfsetlinewidth{0.000000pt}%
\definecolor{currentstroke}{rgb}{0.000000,0.000000,0.000000}%
\pgfsetstrokecolor{currentstroke}%
\pgfsetdash{}{0pt}%
\pgfpathmoveto{\pgfqpoint{4.222909in}{3.159297in}}%
\pgfpathlineto{\pgfqpoint{4.233769in}{3.108648in}}%
\pgfpathlineto{\pgfqpoint{4.244643in}{3.057604in}}%
\pgfpathlineto{\pgfqpoint{4.279331in}{3.047173in}}%
\pgfpathlineto{\pgfqpoint{4.313985in}{3.034844in}}%
\pgfpathlineto{\pgfqpoint{4.303076in}{3.084091in}}%
\pgfpathlineto{\pgfqpoint{4.292181in}{3.132963in}}%
\pgfpathlineto{\pgfqpoint{4.257562in}{3.147195in}}%
\pgfpathlineto{\pgfqpoint{4.222909in}{3.159297in}}%
\pgfpathclose%
\pgfusepath{fill}%
\end{pgfscope}%
\begin{pgfscope}%
\pgfpathrectangle{\pgfqpoint{1.020000in}{0.880000in}}{\pgfqpoint{6.160000in}{6.160000in}}%
\pgfusepath{clip}%
\pgfsetbuttcap%
\pgfsetroundjoin%
\definecolor{currentfill}{rgb}{0.763363,0.835092,0.955658}%
\pgfsetfillcolor{currentfill}%
\pgfsetlinewidth{0.000000pt}%
\definecolor{currentstroke}{rgb}{0.000000,0.000000,0.000000}%
\pgfsetstrokecolor{currentstroke}%
\pgfsetdash{}{0pt}%
\pgfpathmoveto{\pgfqpoint{3.672433in}{3.328623in}}%
\pgfpathlineto{\pgfqpoint{3.682897in}{3.288557in}}%
\pgfpathlineto{\pgfqpoint{3.693386in}{3.246498in}}%
\pgfpathlineto{\pgfqpoint{3.728162in}{3.272070in}}%
\pgfpathlineto{\pgfqpoint{3.762940in}{3.295537in}}%
\pgfpathlineto{\pgfqpoint{3.752399in}{3.339068in}}%
\pgfpathlineto{\pgfqpoint{3.741881in}{3.380455in}}%
\pgfpathlineto{\pgfqpoint{3.707155in}{3.355663in}}%
\pgfpathlineto{\pgfqpoint{3.672433in}{3.328623in}}%
\pgfpathclose%
\pgfusepath{fill}%
\end{pgfscope}%
\begin{pgfscope}%
\pgfpathrectangle{\pgfqpoint{1.020000in}{0.880000in}}{\pgfqpoint{6.160000in}{6.160000in}}%
\pgfusepath{clip}%
\pgfsetbuttcap%
\pgfsetroundjoin%
\definecolor{currentfill}{rgb}{0.451739,0.588181,0.960201}%
\pgfsetfillcolor{currentfill}%
\pgfsetlinewidth{0.000000pt}%
\definecolor{currentstroke}{rgb}{0.000000,0.000000,0.000000}%
\pgfsetstrokecolor{currentstroke}%
\pgfsetdash{}{0pt}%
\pgfpathmoveto{\pgfqpoint{2.887940in}{2.747388in}}%
\pgfpathlineto{\pgfqpoint{2.897629in}{2.724057in}}%
\pgfpathlineto{\pgfqpoint{2.907337in}{2.700470in}}%
\pgfpathlineto{\pgfqpoint{2.942308in}{2.717222in}}%
\pgfpathlineto{\pgfqpoint{2.977249in}{2.735467in}}%
\pgfpathlineto{\pgfqpoint{2.967474in}{2.759755in}}%
\pgfpathlineto{\pgfqpoint{2.957720in}{2.783668in}}%
\pgfpathlineto{\pgfqpoint{2.922846in}{2.764744in}}%
\pgfpathlineto{\pgfqpoint{2.887940in}{2.747388in}}%
\pgfpathclose%
\pgfusepath{fill}%
\end{pgfscope}%
\begin{pgfscope}%
\pgfpathrectangle{\pgfqpoint{1.020000in}{0.880000in}}{\pgfqpoint{6.160000in}{6.160000in}}%
\pgfusepath{clip}%
\pgfsetbuttcap%
\pgfsetroundjoin%
\definecolor{currentfill}{rgb}{0.441123,0.576532,0.954545}%
\pgfsetfillcolor{currentfill}%
\pgfsetlinewidth{0.000000pt}%
\definecolor{currentstroke}{rgb}{0.000000,0.000000,0.000000}%
\pgfsetstrokecolor{currentstroke}%
\pgfsetdash{}{0pt}%
\pgfpathmoveto{\pgfqpoint{4.565441in}{2.766197in}}%
\pgfpathlineto{\pgfqpoint{4.576558in}{2.726161in}}%
\pgfpathlineto{\pgfqpoint{4.587695in}{2.686960in}}%
\pgfpathlineto{\pgfqpoint{4.622239in}{2.675180in}}%
\pgfpathlineto{\pgfqpoint{4.656750in}{2.663223in}}%
\pgfpathlineto{\pgfqpoint{4.645561in}{2.699773in}}%
\pgfpathlineto{\pgfqpoint{4.634393in}{2.737072in}}%
\pgfpathlineto{\pgfqpoint{4.599934in}{2.751739in}}%
\pgfpathlineto{\pgfqpoint{4.565441in}{2.766197in}}%
\pgfpathclose%
\pgfusepath{fill}%
\end{pgfscope}%
\begin{pgfscope}%
\pgfpathrectangle{\pgfqpoint{1.020000in}{0.880000in}}{\pgfqpoint{6.160000in}{6.160000in}}%
\pgfusepath{clip}%
\pgfsetbuttcap%
\pgfsetroundjoin%
\definecolor{currentfill}{rgb}{0.581486,0.713451,0.998314}%
\pgfsetfillcolor{currentfill}%
\pgfsetlinewidth{0.000000pt}%
\definecolor{currentstroke}{rgb}{0.000000,0.000000,0.000000}%
\pgfsetstrokecolor{currentstroke}%
\pgfsetdash{}{0pt}%
\pgfpathmoveto{\pgfqpoint{4.313985in}{3.034844in}}%
\pgfpathlineto{\pgfqpoint{4.324908in}{2.985509in}}%
\pgfpathlineto{\pgfqpoint{4.335846in}{2.936357in}}%
\pgfpathlineto{\pgfqpoint{4.370503in}{2.924432in}}%
\pgfpathlineto{\pgfqpoint{4.405127in}{2.911157in}}%
\pgfpathlineto{\pgfqpoint{4.394151in}{2.958088in}}%
\pgfpathlineto{\pgfqpoint{4.383190in}{3.005193in}}%
\pgfpathlineto{\pgfqpoint{4.348605in}{3.020788in}}%
\pgfpathlineto{\pgfqpoint{4.313985in}{3.034844in}}%
\pgfpathclose%
\pgfusepath{fill}%
\end{pgfscope}%
\begin{pgfscope}%
\pgfpathrectangle{\pgfqpoint{1.020000in}{0.880000in}}{\pgfqpoint{6.160000in}{6.160000in}}%
\pgfusepath{clip}%
\pgfsetbuttcap%
\pgfsetroundjoin%
\definecolor{currentfill}{rgb}{0.733898,0.820018,0.970724}%
\pgfsetfillcolor{currentfill}%
\pgfsetlinewidth{0.000000pt}%
\definecolor{currentstroke}{rgb}{0.000000,0.000000,0.000000}%
\pgfsetstrokecolor{currentstroke}%
\pgfsetdash{}{0pt}%
\pgfpathmoveto{\pgfqpoint{3.603000in}{3.268964in}}%
\pgfpathlineto{\pgfqpoint{3.613407in}{3.230456in}}%
\pgfpathlineto{\pgfqpoint{3.623838in}{3.190127in}}%
\pgfpathlineto{\pgfqpoint{3.658611in}{3.219089in}}%
\pgfpathlineto{\pgfqpoint{3.693386in}{3.246498in}}%
\pgfpathlineto{\pgfqpoint{3.682897in}{3.288557in}}%
\pgfpathlineto{\pgfqpoint{3.672433in}{3.328623in}}%
\pgfpathlineto{\pgfqpoint{3.637715in}{3.299623in}}%
\pgfpathlineto{\pgfqpoint{3.603000in}{3.268964in}}%
\pgfpathclose%
\pgfusepath{fill}%
\end{pgfscope}%
\begin{pgfscope}%
\pgfpathrectangle{\pgfqpoint{1.020000in}{0.880000in}}{\pgfqpoint{6.160000in}{6.160000in}}%
\pgfusepath{clip}%
\pgfsetbuttcap%
\pgfsetroundjoin%
\definecolor{currentfill}{rgb}{0.718985,0.811993,0.977656}%
\pgfsetfillcolor{currentfill}%
\pgfsetlinewidth{0.000000pt}%
\definecolor{currentstroke}{rgb}{0.000000,0.000000,0.000000}%
\pgfsetstrokecolor{currentstroke}%
\pgfsetdash{}{0pt}%
\pgfpathmoveto{\pgfqpoint{4.062463in}{3.286118in}}%
\pgfpathlineto{\pgfqpoint{4.073226in}{3.235290in}}%
\pgfpathlineto{\pgfqpoint{4.084005in}{3.183380in}}%
\pgfpathlineto{\pgfqpoint{4.118770in}{3.181241in}}%
\pgfpathlineto{\pgfqpoint{4.153511in}{3.176451in}}%
\pgfpathlineto{\pgfqpoint{4.142698in}{3.227646in}}%
\pgfpathlineto{\pgfqpoint{4.131900in}{3.277776in}}%
\pgfpathlineto{\pgfqpoint{4.097193in}{3.283410in}}%
\pgfpathlineto{\pgfqpoint{4.062463in}{3.286118in}}%
\pgfpathclose%
\pgfusepath{fill}%
\end{pgfscope}%
\begin{pgfscope}%
\pgfpathrectangle{\pgfqpoint{1.020000in}{0.880000in}}{\pgfqpoint{6.160000in}{6.160000in}}%
\pgfusepath{clip}%
\pgfsetbuttcap%
\pgfsetroundjoin%
\definecolor{currentfill}{rgb}{0.768034,0.837035,0.952488}%
\pgfsetfillcolor{currentfill}%
\pgfsetlinewidth{0.000000pt}%
\definecolor{currentstroke}{rgb}{0.000000,0.000000,0.000000}%
\pgfsetstrokecolor{currentstroke}%
\pgfsetdash{}{0pt}%
\pgfpathmoveto{\pgfqpoint{3.902030in}{3.363707in}}%
\pgfpathlineto{\pgfqpoint{3.912684in}{3.316252in}}%
\pgfpathlineto{\pgfqpoint{3.923357in}{3.267075in}}%
\pgfpathlineto{\pgfqpoint{3.958155in}{3.276285in}}%
\pgfpathlineto{\pgfqpoint{3.992941in}{3.282558in}}%
\pgfpathlineto{\pgfqpoint{3.982229in}{3.332113in}}%
\pgfpathlineto{\pgfqpoint{3.971534in}{3.379912in}}%
\pgfpathlineto{\pgfqpoint{3.936788in}{3.373403in}}%
\pgfpathlineto{\pgfqpoint{3.902030in}{3.363707in}}%
\pgfpathclose%
\pgfusepath{fill}%
\end{pgfscope}%
\begin{pgfscope}%
\pgfpathrectangle{\pgfqpoint{1.020000in}{0.880000in}}{\pgfqpoint{6.160000in}{6.160000in}}%
\pgfusepath{clip}%
\pgfsetbuttcap%
\pgfsetroundjoin%
\definecolor{currentfill}{rgb}{0.399231,0.528528,0.928459}%
\pgfsetfillcolor{currentfill}%
\pgfsetlinewidth{0.000000pt}%
\definecolor{currentstroke}{rgb}{0.000000,0.000000,0.000000}%
\pgfsetstrokecolor{currentstroke}%
\pgfsetdash{}{0pt}%
\pgfpathmoveto{\pgfqpoint{4.656750in}{2.663223in}}%
\pgfpathlineto{\pgfqpoint{4.667960in}{2.627530in}}%
\pgfpathlineto{\pgfqpoint{4.702467in}{2.616811in}}%
\pgfpathlineto{\pgfqpoint{4.736943in}{2.606158in}}%
\pgfpathlineto{\pgfqpoint{4.725677in}{2.639283in}}%
\pgfpathlineto{\pgfqpoint{4.691229in}{2.651217in}}%
\pgfpathlineto{\pgfqpoint{4.656750in}{2.663223in}}%
\pgfpathclose%
\pgfusepath{fill}%
\end{pgfscope}%
\begin{pgfscope}%
\pgfpathrectangle{\pgfqpoint{1.020000in}{0.880000in}}{\pgfqpoint{6.160000in}{6.160000in}}%
\pgfusepath{clip}%
\pgfsetbuttcap%
\pgfsetroundjoin%
\definecolor{currentfill}{rgb}{0.703587,0.802586,0.982847}%
\pgfsetfillcolor{currentfill}%
\pgfsetlinewidth{0.000000pt}%
\definecolor{currentstroke}{rgb}{0.000000,0.000000,0.000000}%
\pgfsetstrokecolor{currentstroke}%
\pgfsetdash{}{0pt}%
\pgfpathmoveto{\pgfqpoint{3.533582in}{3.203904in}}%
\pgfpathlineto{\pgfqpoint{3.543927in}{3.167116in}}%
\pgfpathlineto{\pgfqpoint{3.554298in}{3.128696in}}%
\pgfpathlineto{\pgfqpoint{3.589067in}{3.159898in}}%
\pgfpathlineto{\pgfqpoint{3.623838in}{3.190127in}}%
\pgfpathlineto{\pgfqpoint{3.613407in}{3.230456in}}%
\pgfpathlineto{\pgfqpoint{3.603000in}{3.268964in}}%
\pgfpathlineto{\pgfqpoint{3.568289in}{3.236954in}}%
\pgfpathlineto{\pgfqpoint{3.533582in}{3.203904in}}%
\pgfpathclose%
\pgfusepath{fill}%
\end{pgfscope}%
\begin{pgfscope}%
\pgfpathrectangle{\pgfqpoint{1.020000in}{0.880000in}}{\pgfqpoint{6.160000in}{6.160000in}}%
\pgfusepath{clip}%
\pgfsetbuttcap%
\pgfsetroundjoin%
\definecolor{currentfill}{rgb}{0.672538,0.782861,0.991982}%
\pgfsetfillcolor{currentfill}%
\pgfsetlinewidth{0.000000pt}%
\definecolor{currentstroke}{rgb}{0.000000,0.000000,0.000000}%
\pgfsetstrokecolor{currentstroke}%
\pgfsetdash{}{0pt}%
\pgfpathmoveto{\pgfqpoint{3.464172in}{3.135914in}}%
\pgfpathlineto{\pgfqpoint{3.474453in}{3.100934in}}%
\pgfpathlineto{\pgfqpoint{3.484760in}{3.064518in}}%
\pgfpathlineto{\pgfqpoint{3.519529in}{3.096808in}}%
\pgfpathlineto{\pgfqpoint{3.554298in}{3.128696in}}%
\pgfpathlineto{\pgfqpoint{3.543927in}{3.167116in}}%
\pgfpathlineto{\pgfqpoint{3.533582in}{3.203904in}}%
\pgfpathlineto{\pgfqpoint{3.498876in}{3.170123in}}%
\pgfpathlineto{\pgfqpoint{3.464172in}{3.135914in}}%
\pgfpathclose%
\pgfusepath{fill}%
\end{pgfscope}%
\begin{pgfscope}%
\pgfpathrectangle{\pgfqpoint{1.020000in}{0.880000in}}{\pgfqpoint{6.160000in}{6.160000in}}%
\pgfusepath{clip}%
\pgfsetbuttcap%
\pgfsetroundjoin%
\definecolor{currentfill}{rgb}{0.435815,0.570707,0.951717}%
\pgfsetfillcolor{currentfill}%
\pgfsetlinewidth{0.000000pt}%
\definecolor{currentstroke}{rgb}{0.000000,0.000000,0.000000}%
\pgfsetstrokecolor{currentstroke}%
\pgfsetdash{}{0pt}%
\pgfpathmoveto{\pgfqpoint{2.818034in}{2.717148in}}%
\pgfpathlineto{\pgfqpoint{2.827658in}{2.694265in}}%
\pgfpathlineto{\pgfqpoint{2.837301in}{2.671224in}}%
\pgfpathlineto{\pgfqpoint{2.872335in}{2.685159in}}%
\pgfpathlineto{\pgfqpoint{2.907337in}{2.700470in}}%
\pgfpathlineto{\pgfqpoint{2.897629in}{2.724057in}}%
\pgfpathlineto{\pgfqpoint{2.887940in}{2.747388in}}%
\pgfpathlineto{\pgfqpoint{2.853003in}{2.731545in}}%
\pgfpathlineto{\pgfqpoint{2.818034in}{2.717148in}}%
\pgfpathclose%
\pgfusepath{fill}%
\end{pgfscope}%
\begin{pgfscope}%
\pgfpathrectangle{\pgfqpoint{1.020000in}{0.880000in}}{\pgfqpoint{6.160000in}{6.160000in}}%
\pgfusepath{clip}%
\pgfsetbuttcap%
\pgfsetroundjoin%
\definecolor{currentfill}{rgb}{0.635474,0.756714,0.998297}%
\pgfsetfillcolor{currentfill}%
\pgfsetlinewidth{0.000000pt}%
\definecolor{currentstroke}{rgb}{0.000000,0.000000,0.000000}%
\pgfsetstrokecolor{currentstroke}%
\pgfsetdash{}{0pt}%
\pgfpathmoveto{\pgfqpoint{3.394759in}{3.067348in}}%
\pgfpathlineto{\pgfqpoint{3.404974in}{3.034193in}}%
\pgfpathlineto{\pgfqpoint{3.415213in}{2.999800in}}%
\pgfpathlineto{\pgfqpoint{3.449988in}{3.032096in}}%
\pgfpathlineto{\pgfqpoint{3.484760in}{3.064518in}}%
\pgfpathlineto{\pgfqpoint{3.474453in}{3.100934in}}%
\pgfpathlineto{\pgfqpoint{3.464172in}{3.135914in}}%
\pgfpathlineto{\pgfqpoint{3.429467in}{3.101564in}}%
\pgfpathlineto{\pgfqpoint{3.394759in}{3.067348in}}%
\pgfpathclose%
\pgfusepath{fill}%
\end{pgfscope}%
\begin{pgfscope}%
\pgfpathrectangle{\pgfqpoint{1.020000in}{0.880000in}}{\pgfqpoint{6.160000in}{6.160000in}}%
\pgfusepath{clip}%
\pgfsetbuttcap%
\pgfsetroundjoin%
\definecolor{currentfill}{rgb}{0.516260,0.654498,0.986407}%
\pgfsetfillcolor{currentfill}%
\pgfsetlinewidth{0.000000pt}%
\definecolor{currentstroke}{rgb}{0.000000,0.000000,0.000000}%
\pgfsetstrokecolor{currentstroke}%
\pgfsetdash{}{0pt}%
\pgfpathmoveto{\pgfqpoint{4.405127in}{2.911157in}}%
\pgfpathlineto{\pgfqpoint{4.416119in}{2.864637in}}%
\pgfpathlineto{\pgfqpoint{4.427127in}{2.818744in}}%
\pgfpathlineto{\pgfqpoint{4.461757in}{2.806725in}}%
\pgfpathlineto{\pgfqpoint{4.496353in}{2.793852in}}%
\pgfpathlineto{\pgfqpoint{4.485301in}{2.837272in}}%
\pgfpathlineto{\pgfqpoint{4.474267in}{2.881272in}}%
\pgfpathlineto{\pgfqpoint{4.439715in}{2.896709in}}%
\pgfpathlineto{\pgfqpoint{4.405127in}{2.911157in}}%
\pgfpathclose%
\pgfusepath{fill}%
\end{pgfscope}%
\begin{pgfscope}%
\pgfpathrectangle{\pgfqpoint{1.020000in}{0.880000in}}{\pgfqpoint{6.160000in}{6.160000in}}%
\pgfusepath{clip}%
\pgfsetbuttcap%
\pgfsetroundjoin%
\definecolor{currentfill}{rgb}{0.597777,0.727330,0.999777}%
\pgfsetfillcolor{currentfill}%
\pgfsetlinewidth{0.000000pt}%
\definecolor{currentstroke}{rgb}{0.000000,0.000000,0.000000}%
\pgfsetstrokecolor{currentstroke}%
\pgfsetdash{}{0pt}%
\pgfpathmoveto{\pgfqpoint{3.325328in}{3.000316in}}%
\pgfpathlineto{\pgfqpoint{3.335474in}{2.968937in}}%
\pgfpathlineto{\pgfqpoint{3.345643in}{2.936515in}}%
\pgfpathlineto{\pgfqpoint{3.380432in}{2.967866in}}%
\pgfpathlineto{\pgfqpoint{3.415213in}{2.999800in}}%
\pgfpathlineto{\pgfqpoint{3.404974in}{3.034193in}}%
\pgfpathlineto{\pgfqpoint{3.394759in}{3.067348in}}%
\pgfpathlineto{\pgfqpoint{3.360047in}{3.033522in}}%
\pgfpathlineto{\pgfqpoint{3.325328in}{3.000316in}}%
\pgfpathclose%
\pgfusepath{fill}%
\end{pgfscope}%
\begin{pgfscope}%
\pgfpathrectangle{\pgfqpoint{1.020000in}{0.880000in}}{\pgfqpoint{6.160000in}{6.160000in}}%
\pgfusepath{clip}%
\pgfsetbuttcap%
\pgfsetroundjoin%
\definecolor{currentfill}{rgb}{0.565182,0.699438,0.996635}%
\pgfsetfillcolor{currentfill}%
\pgfsetlinewidth{0.000000pt}%
\definecolor{currentstroke}{rgb}{0.000000,0.000000,0.000000}%
\pgfsetstrokecolor{currentstroke}%
\pgfsetdash{}{0pt}%
\pgfpathmoveto{\pgfqpoint{3.255860in}{2.936577in}}%
\pgfpathlineto{\pgfqpoint{3.265935in}{2.906876in}}%
\pgfpathlineto{\pgfqpoint{3.276034in}{2.876314in}}%
\pgfpathlineto{\pgfqpoint{3.310845in}{2.905940in}}%
\pgfpathlineto{\pgfqpoint{3.345643in}{2.936515in}}%
\pgfpathlineto{\pgfqpoint{3.335474in}{2.968937in}}%
\pgfpathlineto{\pgfqpoint{3.325328in}{3.000316in}}%
\pgfpathlineto{\pgfqpoint{3.290600in}{2.967940in}}%
\pgfpathlineto{\pgfqpoint{3.255860in}{2.936577in}}%
\pgfpathclose%
\pgfusepath{fill}%
\end{pgfscope}%
\begin{pgfscope}%
\pgfpathrectangle{\pgfqpoint{1.020000in}{0.880000in}}{\pgfqpoint{6.160000in}{6.160000in}}%
\pgfusepath{clip}%
\pgfsetbuttcap%
\pgfsetroundjoin%
\definecolor{currentfill}{rgb}{0.661968,0.775491,0.993937}%
\pgfsetfillcolor{currentfill}%
\pgfsetlinewidth{0.000000pt}%
\definecolor{currentstroke}{rgb}{0.000000,0.000000,0.000000}%
\pgfsetstrokecolor{currentstroke}%
\pgfsetdash{}{0pt}%
\pgfpathmoveto{\pgfqpoint{4.153511in}{3.176451in}}%
\pgfpathlineto{\pgfqpoint{4.164337in}{3.124522in}}%
\pgfpathlineto{\pgfqpoint{4.175177in}{3.072183in}}%
\pgfpathlineto{\pgfqpoint{4.209925in}{3.065983in}}%
\pgfpathlineto{\pgfqpoint{4.244643in}{3.057604in}}%
\pgfpathlineto{\pgfqpoint{4.233769in}{3.108648in}}%
\pgfpathlineto{\pgfqpoint{4.222909in}{3.159297in}}%
\pgfpathlineto{\pgfqpoint{4.188225in}{3.169098in}}%
\pgfpathlineto{\pgfqpoint{4.153511in}{3.176451in}}%
\pgfpathclose%
\pgfusepath{fill}%
\end{pgfscope}%
\begin{pgfscope}%
\pgfpathrectangle{\pgfqpoint{1.020000in}{0.880000in}}{\pgfqpoint{6.160000in}{6.160000in}}%
\pgfusepath{clip}%
\pgfsetbuttcap%
\pgfsetroundjoin%
\definecolor{currentfill}{rgb}{0.758539,0.832787,0.958408}%
\pgfsetfillcolor{currentfill}%
\pgfsetlinewidth{0.000000pt}%
\definecolor{currentstroke}{rgb}{0.000000,0.000000,0.000000}%
\pgfsetstrokecolor{currentstroke}%
\pgfsetdash{}{0pt}%
\pgfpathmoveto{\pgfqpoint{3.832492in}{3.335177in}}%
\pgfpathlineto{\pgfqpoint{3.843103in}{3.288539in}}%
\pgfpathlineto{\pgfqpoint{3.853732in}{3.240241in}}%
\pgfpathlineto{\pgfqpoint{3.888548in}{3.255019in}}%
\pgfpathlineto{\pgfqpoint{3.923357in}{3.267075in}}%
\pgfpathlineto{\pgfqpoint{3.912684in}{3.316252in}}%
\pgfpathlineto{\pgfqpoint{3.902030in}{3.363707in}}%
\pgfpathlineto{\pgfqpoint{3.867264in}{3.350919in}}%
\pgfpathlineto{\pgfqpoint{3.832492in}{3.335177in}}%
\pgfpathclose%
\pgfusepath{fill}%
\end{pgfscope}%
\begin{pgfscope}%
\pgfpathrectangle{\pgfqpoint{1.020000in}{0.880000in}}{\pgfqpoint{6.160000in}{6.160000in}}%
\pgfusepath{clip}%
\pgfsetbuttcap%
\pgfsetroundjoin%
\definecolor{currentfill}{rgb}{0.532568,0.669801,0.990393}%
\pgfsetfillcolor{currentfill}%
\pgfsetlinewidth{0.000000pt}%
\definecolor{currentstroke}{rgb}{0.000000,0.000000,0.000000}%
\pgfsetstrokecolor{currentstroke}%
\pgfsetdash{}{0pt}%
\pgfpathmoveto{\pgfqpoint{3.186335in}{2.877490in}}%
\pgfpathlineto{\pgfqpoint{3.196340in}{2.849324in}}%
\pgfpathlineto{\pgfqpoint{3.206367in}{2.820468in}}%
\pgfpathlineto{\pgfqpoint{3.241209in}{2.847782in}}%
\pgfpathlineto{\pgfqpoint{3.276034in}{2.876314in}}%
\pgfpathlineto{\pgfqpoint{3.265935in}{2.906876in}}%
\pgfpathlineto{\pgfqpoint{3.255860in}{2.936577in}}%
\pgfpathlineto{\pgfqpoint{3.221105in}{2.906384in}}%
\pgfpathlineto{\pgfqpoint{3.186335in}{2.877490in}}%
\pgfpathclose%
\pgfusepath{fill}%
\end{pgfscope}%
\begin{pgfscope}%
\pgfpathrectangle{\pgfqpoint{1.020000in}{0.880000in}}{\pgfqpoint{6.160000in}{6.160000in}}%
\pgfusepath{clip}%
\pgfsetbuttcap%
\pgfsetroundjoin%
\definecolor{currentfill}{rgb}{0.724041,0.814910,0.975651}%
\pgfsetfillcolor{currentfill}%
\pgfsetlinewidth{0.000000pt}%
\definecolor{currentstroke}{rgb}{0.000000,0.000000,0.000000}%
\pgfsetstrokecolor{currentstroke}%
\pgfsetdash{}{0pt}%
\pgfpathmoveto{\pgfqpoint{3.992941in}{3.282558in}}%
\pgfpathlineto{\pgfqpoint{4.003668in}{3.231583in}}%
\pgfpathlineto{\pgfqpoint{4.014411in}{3.179529in}}%
\pgfpathlineto{\pgfqpoint{4.049217in}{3.182816in}}%
\pgfpathlineto{\pgfqpoint{4.084005in}{3.183380in}}%
\pgfpathlineto{\pgfqpoint{4.073226in}{3.235290in}}%
\pgfpathlineto{\pgfqpoint{4.062463in}{3.286118in}}%
\pgfpathlineto{\pgfqpoint{4.027711in}{3.285841in}}%
\pgfpathlineto{\pgfqpoint{3.992941in}{3.282558in}}%
\pgfpathclose%
\pgfusepath{fill}%
\end{pgfscope}%
\begin{pgfscope}%
\pgfpathrectangle{\pgfqpoint{1.020000in}{0.880000in}}{\pgfqpoint{6.160000in}{6.160000in}}%
\pgfusepath{clip}%
\pgfsetbuttcap%
\pgfsetroundjoin%
\definecolor{currentfill}{rgb}{0.505423,0.643995,0.983157}%
\pgfsetfillcolor{currentfill}%
\pgfsetlinewidth{0.000000pt}%
\definecolor{currentstroke}{rgb}{0.000000,0.000000,0.000000}%
\pgfsetstrokecolor{currentstroke}%
\pgfsetdash{}{0pt}%
\pgfpathmoveto{\pgfqpoint{3.116736in}{2.823986in}}%
\pgfpathlineto{\pgfqpoint{3.126670in}{2.797188in}}%
\pgfpathlineto{\pgfqpoint{3.136626in}{2.769853in}}%
\pgfpathlineto{\pgfqpoint{3.171507in}{2.794467in}}%
\pgfpathlineto{\pgfqpoint{3.206367in}{2.820468in}}%
\pgfpathlineto{\pgfqpoint{3.196340in}{2.849324in}}%
\pgfpathlineto{\pgfqpoint{3.186335in}{2.877490in}}%
\pgfpathlineto{\pgfqpoint{3.151546in}{2.849998in}}%
\pgfpathlineto{\pgfqpoint{3.116736in}{2.823986in}}%
\pgfpathclose%
\pgfusepath{fill}%
\end{pgfscope}%
\begin{pgfscope}%
\pgfpathrectangle{\pgfqpoint{1.020000in}{0.880000in}}{\pgfqpoint{6.160000in}{6.160000in}}%
\pgfusepath{clip}%
\pgfsetbuttcap%
\pgfsetroundjoin%
\definecolor{currentfill}{rgb}{0.457046,0.594006,0.963029}%
\pgfsetfillcolor{currentfill}%
\pgfsetlinewidth{0.000000pt}%
\definecolor{currentstroke}{rgb}{0.000000,0.000000,0.000000}%
\pgfsetstrokecolor{currentstroke}%
\pgfsetdash{}{0pt}%
\pgfpathmoveto{\pgfqpoint{4.496353in}{2.793852in}}%
\pgfpathlineto{\pgfqpoint{4.507422in}{2.751193in}}%
\pgfpathlineto{\pgfqpoint{4.518510in}{2.709450in}}%
\pgfpathlineto{\pgfqpoint{4.553119in}{2.698429in}}%
\pgfpathlineto{\pgfqpoint{4.587695in}{2.686960in}}%
\pgfpathlineto{\pgfqpoint{4.576558in}{2.726161in}}%
\pgfpathlineto{\pgfqpoint{4.565441in}{2.766197in}}%
\pgfpathlineto{\pgfqpoint{4.530914in}{2.780289in}}%
\pgfpathlineto{\pgfqpoint{4.496353in}{2.793852in}}%
\pgfpathclose%
\pgfusepath{fill}%
\end{pgfscope}%
\begin{pgfscope}%
\pgfpathrectangle{\pgfqpoint{1.020000in}{0.880000in}}{\pgfqpoint{6.160000in}{6.160000in}}%
\pgfusepath{clip}%
\pgfsetbuttcap%
\pgfsetroundjoin%
\definecolor{currentfill}{rgb}{0.597777,0.727330,0.999777}%
\pgfsetfillcolor{currentfill}%
\pgfsetlinewidth{0.000000pt}%
\definecolor{currentstroke}{rgb}{0.000000,0.000000,0.000000}%
\pgfsetstrokecolor{currentstroke}%
\pgfsetdash{}{0pt}%
\pgfpathmoveto{\pgfqpoint{4.244643in}{3.057604in}}%
\pgfpathlineto{\pgfqpoint{4.255531in}{3.006465in}}%
\pgfpathlineto{\pgfqpoint{4.266432in}{2.955518in}}%
\pgfpathlineto{\pgfqpoint{4.301155in}{2.946770in}}%
\pgfpathlineto{\pgfqpoint{4.335846in}{2.936357in}}%
\pgfpathlineto{\pgfqpoint{4.324908in}{2.985509in}}%
\pgfpathlineto{\pgfqpoint{4.313985in}{3.034844in}}%
\pgfpathlineto{\pgfqpoint{4.279331in}{3.047173in}}%
\pgfpathlineto{\pgfqpoint{4.244643in}{3.057604in}}%
\pgfpathclose%
\pgfusepath{fill}%
\end{pgfscope}%
\begin{pgfscope}%
\pgfpathrectangle{\pgfqpoint{1.020000in}{0.880000in}}{\pgfqpoint{6.160000in}{6.160000in}}%
\pgfusepath{clip}%
\pgfsetbuttcap%
\pgfsetroundjoin%
\definecolor{currentfill}{rgb}{0.743754,0.825125,0.965798}%
\pgfsetfillcolor{currentfill}%
\pgfsetlinewidth{0.000000pt}%
\definecolor{currentstroke}{rgb}{0.000000,0.000000,0.000000}%
\pgfsetstrokecolor{currentstroke}%
\pgfsetdash{}{0pt}%
\pgfpathmoveto{\pgfqpoint{3.762940in}{3.295537in}}%
\pgfpathlineto{\pgfqpoint{3.773502in}{3.250142in}}%
\pgfpathlineto{\pgfqpoint{3.784085in}{3.203175in}}%
\pgfpathlineto{\pgfqpoint{3.818910in}{3.222898in}}%
\pgfpathlineto{\pgfqpoint{3.853732in}{3.240241in}}%
\pgfpathlineto{\pgfqpoint{3.843103in}{3.288539in}}%
\pgfpathlineto{\pgfqpoint{3.832492in}{3.335177in}}%
\pgfpathlineto{\pgfqpoint{3.797717in}{3.316649in}}%
\pgfpathlineto{\pgfqpoint{3.762940in}{3.295537in}}%
\pgfpathclose%
\pgfusepath{fill}%
\end{pgfscope}%
\begin{pgfscope}%
\pgfpathrectangle{\pgfqpoint{1.020000in}{0.880000in}}{\pgfqpoint{6.160000in}{6.160000in}}%
\pgfusepath{clip}%
\pgfsetbuttcap%
\pgfsetroundjoin%
\definecolor{currentfill}{rgb}{0.478462,0.616564,0.972721}%
\pgfsetfillcolor{currentfill}%
\pgfsetlinewidth{0.000000pt}%
\definecolor{currentstroke}{rgb}{0.000000,0.000000,0.000000}%
\pgfsetstrokecolor{currentstroke}%
\pgfsetdash{}{0pt}%
\pgfpathmoveto{\pgfqpoint{3.047045in}{2.776592in}}%
\pgfpathlineto{\pgfqpoint{3.056909in}{2.750977in}}%
\pgfpathlineto{\pgfqpoint{3.066794in}{2.724961in}}%
\pgfpathlineto{\pgfqpoint{3.101722in}{2.746674in}}%
\pgfpathlineto{\pgfqpoint{3.136626in}{2.769853in}}%
\pgfpathlineto{\pgfqpoint{3.126670in}{2.797188in}}%
\pgfpathlineto{\pgfqpoint{3.116736in}{2.823986in}}%
\pgfpathlineto{\pgfqpoint{3.081903in}{2.799507in}}%
\pgfpathlineto{\pgfqpoint{3.047045in}{2.776592in}}%
\pgfpathclose%
\pgfusepath{fill}%
\end{pgfscope}%
\begin{pgfscope}%
\pgfpathrectangle{\pgfqpoint{1.020000in}{0.880000in}}{\pgfqpoint{6.160000in}{6.160000in}}%
\pgfusepath{clip}%
\pgfsetbuttcap%
\pgfsetroundjoin%
\definecolor{currentfill}{rgb}{0.414801,0.546874,0.939088}%
\pgfsetfillcolor{currentfill}%
\pgfsetlinewidth{0.000000pt}%
\definecolor{currentstroke}{rgb}{0.000000,0.000000,0.000000}%
\pgfsetstrokecolor{currentstroke}%
\pgfsetdash{}{0pt}%
\pgfpathmoveto{\pgfqpoint{4.587695in}{2.686960in}}%
\pgfpathlineto{\pgfqpoint{4.598852in}{2.648715in}}%
\pgfpathlineto{\pgfqpoint{4.633422in}{2.638204in}}%
\pgfpathlineto{\pgfqpoint{4.667960in}{2.627530in}}%
\pgfpathlineto{\pgfqpoint{4.656750in}{2.663223in}}%
\pgfpathlineto{\pgfqpoint{4.622239in}{2.675180in}}%
\pgfpathlineto{\pgfqpoint{4.587695in}{2.686960in}}%
\pgfpathclose%
\pgfusepath{fill}%
\end{pgfscope}%
\begin{pgfscope}%
\pgfpathrectangle{\pgfqpoint{1.020000in}{0.880000in}}{\pgfqpoint{6.160000in}{6.160000in}}%
\pgfusepath{clip}%
\pgfsetbuttcap%
\pgfsetroundjoin%
\definecolor{currentfill}{rgb}{0.724041,0.814910,0.975651}%
\pgfsetfillcolor{currentfill}%
\pgfsetlinewidth{0.000000pt}%
\definecolor{currentstroke}{rgb}{0.000000,0.000000,0.000000}%
\pgfsetstrokecolor{currentstroke}%
\pgfsetdash{}{0pt}%
\pgfpathmoveto{\pgfqpoint{3.693386in}{3.246498in}}%
\pgfpathlineto{\pgfqpoint{3.703896in}{3.202704in}}%
\pgfpathlineto{\pgfqpoint{3.714428in}{3.157451in}}%
\pgfpathlineto{\pgfqpoint{3.749257in}{3.181283in}}%
\pgfpathlineto{\pgfqpoint{3.784085in}{3.203175in}}%
\pgfpathlineto{\pgfqpoint{3.773502in}{3.250142in}}%
\pgfpathlineto{\pgfqpoint{3.762940in}{3.295537in}}%
\pgfpathlineto{\pgfqpoint{3.728162in}{3.272070in}}%
\pgfpathlineto{\pgfqpoint{3.693386in}{3.246498in}}%
\pgfpathclose%
\pgfusepath{fill}%
\end{pgfscope}%
\begin{pgfscope}%
\pgfpathrectangle{\pgfqpoint{1.020000in}{0.880000in}}{\pgfqpoint{6.160000in}{6.160000in}}%
\pgfusepath{clip}%
\pgfsetbuttcap%
\pgfsetroundjoin%
\definecolor{currentfill}{rgb}{0.538004,0.674902,0.991722}%
\pgfsetfillcolor{currentfill}%
\pgfsetlinewidth{0.000000pt}%
\definecolor{currentstroke}{rgb}{0.000000,0.000000,0.000000}%
\pgfsetstrokecolor{currentstroke}%
\pgfsetdash{}{0pt}%
\pgfpathmoveto{\pgfqpoint{4.335846in}{2.936357in}}%
\pgfpathlineto{\pgfqpoint{4.346798in}{2.887643in}}%
\pgfpathlineto{\pgfqpoint{4.357766in}{2.839597in}}%
\pgfpathlineto{\pgfqpoint{4.392463in}{2.829752in}}%
\pgfpathlineto{\pgfqpoint{4.427127in}{2.818744in}}%
\pgfpathlineto{\pgfqpoint{4.416119in}{2.864637in}}%
\pgfpathlineto{\pgfqpoint{4.405127in}{2.911157in}}%
\pgfpathlineto{\pgfqpoint{4.370503in}{2.924432in}}%
\pgfpathlineto{\pgfqpoint{4.335846in}{2.936357in}}%
\pgfpathclose%
\pgfusepath{fill}%
\end{pgfscope}%
\begin{pgfscope}%
\pgfpathrectangle{\pgfqpoint{1.020000in}{0.880000in}}{\pgfqpoint{6.160000in}{6.160000in}}%
\pgfusepath{clip}%
\pgfsetbuttcap%
\pgfsetroundjoin%
\definecolor{currentfill}{rgb}{0.457046,0.594006,0.963029}%
\pgfsetfillcolor{currentfill}%
\pgfsetlinewidth{0.000000pt}%
\definecolor{currentstroke}{rgb}{0.000000,0.000000,0.000000}%
\pgfsetstrokecolor{currentstroke}%
\pgfsetdash{}{0pt}%
\pgfpathmoveto{\pgfqpoint{2.977249in}{2.735467in}}%
\pgfpathlineto{\pgfqpoint{2.987044in}{2.710847in}}%
\pgfpathlineto{\pgfqpoint{2.996859in}{2.685942in}}%
\pgfpathlineto{\pgfqpoint{3.031840in}{2.704720in}}%
\pgfpathlineto{\pgfqpoint{3.066794in}{2.724961in}}%
\pgfpathlineto{\pgfqpoint{3.056909in}{2.750977in}}%
\pgfpathlineto{\pgfqpoint{3.047045in}{2.776592in}}%
\pgfpathlineto{\pgfqpoint{3.012161in}{2.755248in}}%
\pgfpathlineto{\pgfqpoint{2.977249in}{2.735467in}}%
\pgfpathclose%
\pgfusepath{fill}%
\end{pgfscope}%
\begin{pgfscope}%
\pgfpathrectangle{\pgfqpoint{1.020000in}{0.880000in}}{\pgfqpoint{6.160000in}{6.160000in}}%
\pgfusepath{clip}%
\pgfsetbuttcap%
\pgfsetroundjoin%
\definecolor{currentfill}{rgb}{0.672538,0.782861,0.991982}%
\pgfsetfillcolor{currentfill}%
\pgfsetlinewidth{0.000000pt}%
\definecolor{currentstroke}{rgb}{0.000000,0.000000,0.000000}%
\pgfsetstrokecolor{currentstroke}%
\pgfsetdash{}{0pt}%
\pgfpathmoveto{\pgfqpoint{4.084005in}{3.183380in}}%
\pgfpathlineto{\pgfqpoint{4.094797in}{3.130724in}}%
\pgfpathlineto{\pgfqpoint{4.105602in}{3.077653in}}%
\pgfpathlineto{\pgfqpoint{4.140402in}{3.076099in}}%
\pgfpathlineto{\pgfqpoint{4.175177in}{3.072183in}}%
\pgfpathlineto{\pgfqpoint{4.164337in}{3.124522in}}%
\pgfpathlineto{\pgfqpoint{4.153511in}{3.176451in}}%
\pgfpathlineto{\pgfqpoint{4.118770in}{3.181241in}}%
\pgfpathlineto{\pgfqpoint{4.084005in}{3.183380in}}%
\pgfpathclose%
\pgfusepath{fill}%
\end{pgfscope}%
\begin{pgfscope}%
\pgfpathrectangle{\pgfqpoint{1.020000in}{0.880000in}}{\pgfqpoint{6.160000in}{6.160000in}}%
\pgfusepath{clip}%
\pgfsetbuttcap%
\pgfsetroundjoin%
\definecolor{currentfill}{rgb}{0.724041,0.814910,0.975651}%
\pgfsetfillcolor{currentfill}%
\pgfsetlinewidth{0.000000pt}%
\definecolor{currentstroke}{rgb}{0.000000,0.000000,0.000000}%
\pgfsetstrokecolor{currentstroke}%
\pgfsetdash{}{0pt}%
\pgfpathmoveto{\pgfqpoint{3.923357in}{3.267075in}}%
\pgfpathlineto{\pgfqpoint{3.934046in}{3.216506in}}%
\pgfpathlineto{\pgfqpoint{3.944750in}{3.164878in}}%
\pgfpathlineto{\pgfqpoint{3.979588in}{3.173533in}}%
\pgfpathlineto{\pgfqpoint{4.014411in}{3.179529in}}%
\pgfpathlineto{\pgfqpoint{4.003668in}{3.231583in}}%
\pgfpathlineto{\pgfqpoint{3.992941in}{3.282558in}}%
\pgfpathlineto{\pgfqpoint{3.958155in}{3.276285in}}%
\pgfpathlineto{\pgfqpoint{3.923357in}{3.267075in}}%
\pgfpathclose%
\pgfusepath{fill}%
\end{pgfscope}%
\begin{pgfscope}%
\pgfpathrectangle{\pgfqpoint{1.020000in}{0.880000in}}{\pgfqpoint{6.160000in}{6.160000in}}%
\pgfusepath{clip}%
\pgfsetbuttcap%
\pgfsetroundjoin%
\definecolor{currentfill}{rgb}{0.698454,0.799450,0.984577}%
\pgfsetfillcolor{currentfill}%
\pgfsetlinewidth{0.000000pt}%
\definecolor{currentstroke}{rgb}{0.000000,0.000000,0.000000}%
\pgfsetstrokecolor{currentstroke}%
\pgfsetdash{}{0pt}%
\pgfpathmoveto{\pgfqpoint{3.623838in}{3.190127in}}%
\pgfpathlineto{\pgfqpoint{3.634292in}{3.148214in}}%
\pgfpathlineto{\pgfqpoint{3.644768in}{3.104972in}}%
\pgfpathlineto{\pgfqpoint{3.679598in}{3.131927in}}%
\pgfpathlineto{\pgfqpoint{3.714428in}{3.157451in}}%
\pgfpathlineto{\pgfqpoint{3.703896in}{3.202704in}}%
\pgfpathlineto{\pgfqpoint{3.693386in}{3.246498in}}%
\pgfpathlineto{\pgfqpoint{3.658611in}{3.219089in}}%
\pgfpathlineto{\pgfqpoint{3.623838in}{3.190127in}}%
\pgfpathclose%
\pgfusepath{fill}%
\end{pgfscope}%
\begin{pgfscope}%
\pgfpathrectangle{\pgfqpoint{1.020000in}{0.880000in}}{\pgfqpoint{6.160000in}{6.160000in}}%
\pgfusepath{clip}%
\pgfsetbuttcap%
\pgfsetroundjoin%
\definecolor{currentfill}{rgb}{0.672538,0.782861,0.991982}%
\pgfsetfillcolor{currentfill}%
\pgfsetlinewidth{0.000000pt}%
\definecolor{currentstroke}{rgb}{0.000000,0.000000,0.000000}%
\pgfsetstrokecolor{currentstroke}%
\pgfsetdash{}{0pt}%
\pgfpathmoveto{\pgfqpoint{3.554298in}{3.128696in}}%
\pgfpathlineto{\pgfqpoint{3.564692in}{3.088857in}}%
\pgfpathlineto{\pgfqpoint{3.575107in}{3.047829in}}%
\pgfpathlineto{\pgfqpoint{3.609938in}{3.076850in}}%
\pgfpathlineto{\pgfqpoint{3.644768in}{3.104972in}}%
\pgfpathlineto{\pgfqpoint{3.634292in}{3.148214in}}%
\pgfpathlineto{\pgfqpoint{3.623838in}{3.190127in}}%
\pgfpathlineto{\pgfqpoint{3.589067in}{3.159898in}}%
\pgfpathlineto{\pgfqpoint{3.554298in}{3.128696in}}%
\pgfpathclose%
\pgfusepath{fill}%
\end{pgfscope}%
\begin{pgfscope}%
\pgfpathrectangle{\pgfqpoint{1.020000in}{0.880000in}}{\pgfqpoint{6.160000in}{6.160000in}}%
\pgfusepath{clip}%
\pgfsetbuttcap%
\pgfsetroundjoin%
\definecolor{currentfill}{rgb}{0.441123,0.576532,0.954545}%
\pgfsetfillcolor{currentfill}%
\pgfsetlinewidth{0.000000pt}%
\definecolor{currentstroke}{rgb}{0.000000,0.000000,0.000000}%
\pgfsetstrokecolor{currentstroke}%
\pgfsetdash{}{0pt}%
\pgfpathmoveto{\pgfqpoint{2.907337in}{2.700470in}}%
\pgfpathlineto{\pgfqpoint{2.917064in}{2.676660in}}%
\pgfpathlineto{\pgfqpoint{2.926811in}{2.652664in}}%
\pgfpathlineto{\pgfqpoint{2.961850in}{2.668604in}}%
\pgfpathlineto{\pgfqpoint{2.996859in}{2.685942in}}%
\pgfpathlineto{\pgfqpoint{2.987044in}{2.710847in}}%
\pgfpathlineto{\pgfqpoint{2.977249in}{2.735467in}}%
\pgfpathlineto{\pgfqpoint{2.942308in}{2.717222in}}%
\pgfpathlineto{\pgfqpoint{2.907337in}{2.700470in}}%
\pgfpathclose%
\pgfusepath{fill}%
\end{pgfscope}%
\begin{pgfscope}%
\pgfpathrectangle{\pgfqpoint{1.020000in}{0.880000in}}{\pgfqpoint{6.160000in}{6.160000in}}%
\pgfusepath{clip}%
\pgfsetbuttcap%
\pgfsetroundjoin%
\definecolor{currentfill}{rgb}{0.640828,0.760752,0.997846}%
\pgfsetfillcolor{currentfill}%
\pgfsetlinewidth{0.000000pt}%
\definecolor{currentstroke}{rgb}{0.000000,0.000000,0.000000}%
\pgfsetstrokecolor{currentstroke}%
\pgfsetdash{}{0pt}%
\pgfpathmoveto{\pgfqpoint{3.484760in}{3.064518in}}%
\pgfpathlineto{\pgfqpoint{3.495089in}{3.026857in}}%
\pgfpathlineto{\pgfqpoint{3.505441in}{2.988154in}}%
\pgfpathlineto{\pgfqpoint{3.540275in}{3.018177in}}%
\pgfpathlineto{\pgfqpoint{3.575107in}{3.047829in}}%
\pgfpathlineto{\pgfqpoint{3.564692in}{3.088857in}}%
\pgfpathlineto{\pgfqpoint{3.554298in}{3.128696in}}%
\pgfpathlineto{\pgfqpoint{3.519529in}{3.096808in}}%
\pgfpathlineto{\pgfqpoint{3.484760in}{3.064518in}}%
\pgfpathclose%
\pgfusepath{fill}%
\end{pgfscope}%
\begin{pgfscope}%
\pgfpathrectangle{\pgfqpoint{1.020000in}{0.880000in}}{\pgfqpoint{6.160000in}{6.160000in}}%
\pgfusepath{clip}%
\pgfsetbuttcap%
\pgfsetroundjoin%
\definecolor{currentfill}{rgb}{0.613933,0.739923,0.999142}%
\pgfsetfillcolor{currentfill}%
\pgfsetlinewidth{0.000000pt}%
\definecolor{currentstroke}{rgb}{0.000000,0.000000,0.000000}%
\pgfsetstrokecolor{currentstroke}%
\pgfsetdash{}{0pt}%
\pgfpathmoveto{\pgfqpoint{4.175177in}{3.072183in}}%
\pgfpathlineto{\pgfqpoint{4.186030in}{3.019747in}}%
\pgfpathlineto{\pgfqpoint{4.196896in}{2.967510in}}%
\pgfpathlineto{\pgfqpoint{4.231678in}{2.962469in}}%
\pgfpathlineto{\pgfqpoint{4.266432in}{2.955518in}}%
\pgfpathlineto{\pgfqpoint{4.255531in}{3.006465in}}%
\pgfpathlineto{\pgfqpoint{4.244643in}{3.057604in}}%
\pgfpathlineto{\pgfqpoint{4.209925in}{3.065983in}}%
\pgfpathlineto{\pgfqpoint{4.175177in}{3.072183in}}%
\pgfpathclose%
\pgfusepath{fill}%
\end{pgfscope}%
\begin{pgfscope}%
\pgfpathrectangle{\pgfqpoint{1.020000in}{0.880000in}}{\pgfqpoint{6.160000in}{6.160000in}}%
\pgfusepath{clip}%
\pgfsetbuttcap%
\pgfsetroundjoin%
\definecolor{currentfill}{rgb}{0.478462,0.616564,0.972721}%
\pgfsetfillcolor{currentfill}%
\pgfsetlinewidth{0.000000pt}%
\definecolor{currentstroke}{rgb}{0.000000,0.000000,0.000000}%
\pgfsetstrokecolor{currentstroke}%
\pgfsetdash{}{0pt}%
\pgfpathmoveto{\pgfqpoint{4.427127in}{2.818744in}}%
\pgfpathlineto{\pgfqpoint{4.438151in}{2.773674in}}%
\pgfpathlineto{\pgfqpoint{4.449194in}{2.729596in}}%
\pgfpathlineto{\pgfqpoint{4.483868in}{2.719885in}}%
\pgfpathlineto{\pgfqpoint{4.518510in}{2.709450in}}%
\pgfpathlineto{\pgfqpoint{4.507422in}{2.751193in}}%
\pgfpathlineto{\pgfqpoint{4.496353in}{2.793852in}}%
\pgfpathlineto{\pgfqpoint{4.461757in}{2.806725in}}%
\pgfpathlineto{\pgfqpoint{4.427127in}{2.818744in}}%
\pgfpathclose%
\pgfusepath{fill}%
\end{pgfscope}%
\begin{pgfscope}%
\pgfpathrectangle{\pgfqpoint{1.020000in}{0.880000in}}{\pgfqpoint{6.160000in}{6.160000in}}%
\pgfusepath{clip}%
\pgfsetbuttcap%
\pgfsetroundjoin%
\definecolor{currentfill}{rgb}{0.603162,0.731527,0.999565}%
\pgfsetfillcolor{currentfill}%
\pgfsetlinewidth{0.000000pt}%
\definecolor{currentstroke}{rgb}{0.000000,0.000000,0.000000}%
\pgfsetstrokecolor{currentstroke}%
\pgfsetdash{}{0pt}%
\pgfpathmoveto{\pgfqpoint{3.415213in}{2.999800in}}%
\pgfpathlineto{\pgfqpoint{3.425475in}{2.964336in}}%
\pgfpathlineto{\pgfqpoint{3.435759in}{2.927978in}}%
\pgfpathlineto{\pgfqpoint{3.470603in}{2.958009in}}%
\pgfpathlineto{\pgfqpoint{3.505441in}{2.988154in}}%
\pgfpathlineto{\pgfqpoint{3.495089in}{3.026857in}}%
\pgfpathlineto{\pgfqpoint{3.484760in}{3.064518in}}%
\pgfpathlineto{\pgfqpoint{3.449988in}{3.032096in}}%
\pgfpathlineto{\pgfqpoint{3.415213in}{2.999800in}}%
\pgfpathclose%
\pgfusepath{fill}%
\end{pgfscope}%
\begin{pgfscope}%
\pgfpathrectangle{\pgfqpoint{1.020000in}{0.880000in}}{\pgfqpoint{6.160000in}{6.160000in}}%
\pgfusepath{clip}%
\pgfsetbuttcap%
\pgfsetroundjoin%
\definecolor{currentfill}{rgb}{0.718985,0.811993,0.977656}%
\pgfsetfillcolor{currentfill}%
\pgfsetlinewidth{0.000000pt}%
\definecolor{currentstroke}{rgb}{0.000000,0.000000,0.000000}%
\pgfsetstrokecolor{currentstroke}%
\pgfsetdash{}{0pt}%
\pgfpathmoveto{\pgfqpoint{3.853732in}{3.240241in}}%
\pgfpathlineto{\pgfqpoint{3.864379in}{3.190603in}}%
\pgfpathlineto{\pgfqpoint{3.875042in}{3.139947in}}%
\pgfpathlineto{\pgfqpoint{3.909901in}{3.153645in}}%
\pgfpathlineto{\pgfqpoint{3.944750in}{3.164878in}}%
\pgfpathlineto{\pgfqpoint{3.934046in}{3.216506in}}%
\pgfpathlineto{\pgfqpoint{3.923357in}{3.267075in}}%
\pgfpathlineto{\pgfqpoint{3.888548in}{3.255019in}}%
\pgfpathlineto{\pgfqpoint{3.853732in}{3.240241in}}%
\pgfpathclose%
\pgfusepath{fill}%
\end{pgfscope}%
\begin{pgfscope}%
\pgfpathrectangle{\pgfqpoint{1.020000in}{0.880000in}}{\pgfqpoint{6.160000in}{6.160000in}}%
\pgfusepath{clip}%
\pgfsetbuttcap%
\pgfsetroundjoin%
\definecolor{currentfill}{rgb}{0.430507,0.564883,0.948889}%
\pgfsetfillcolor{currentfill}%
\pgfsetlinewidth{0.000000pt}%
\definecolor{currentstroke}{rgb}{0.000000,0.000000,0.000000}%
\pgfsetstrokecolor{currentstroke}%
\pgfsetdash{}{0pt}%
\pgfpathmoveto{\pgfqpoint{4.518510in}{2.709450in}}%
\pgfpathlineto{\pgfqpoint{4.529618in}{2.668759in}}%
\pgfpathlineto{\pgfqpoint{4.564251in}{2.658942in}}%
\pgfpathlineto{\pgfqpoint{4.598852in}{2.648715in}}%
\pgfpathlineto{\pgfqpoint{4.587695in}{2.686960in}}%
\pgfpathlineto{\pgfqpoint{4.553119in}{2.698429in}}%
\pgfpathlineto{\pgfqpoint{4.518510in}{2.709450in}}%
\pgfpathclose%
\pgfusepath{fill}%
\end{pgfscope}%
\begin{pgfscope}%
\pgfpathrectangle{\pgfqpoint{1.020000in}{0.880000in}}{\pgfqpoint{6.160000in}{6.160000in}}%
\pgfusepath{clip}%
\pgfsetbuttcap%
\pgfsetroundjoin%
\definecolor{currentfill}{rgb}{0.570616,0.704109,0.997195}%
\pgfsetfillcolor{currentfill}%
\pgfsetlinewidth{0.000000pt}%
\definecolor{currentstroke}{rgb}{0.000000,0.000000,0.000000}%
\pgfsetstrokecolor{currentstroke}%
\pgfsetdash{}{0pt}%
\pgfpathmoveto{\pgfqpoint{3.345643in}{2.936515in}}%
\pgfpathlineto{\pgfqpoint{3.355836in}{2.903192in}}%
\pgfpathlineto{\pgfqpoint{3.366050in}{2.869120in}}%
\pgfpathlineto{\pgfqpoint{3.400909in}{2.898281in}}%
\pgfpathlineto{\pgfqpoint{3.435759in}{2.927978in}}%
\pgfpathlineto{\pgfqpoint{3.425475in}{2.964336in}}%
\pgfpathlineto{\pgfqpoint{3.415213in}{2.999800in}}%
\pgfpathlineto{\pgfqpoint{3.380432in}{2.967866in}}%
\pgfpathlineto{\pgfqpoint{3.345643in}{2.936515in}}%
\pgfpathclose%
\pgfusepath{fill}%
\end{pgfscope}%
\begin{pgfscope}%
\pgfpathrectangle{\pgfqpoint{1.020000in}{0.880000in}}{\pgfqpoint{6.160000in}{6.160000in}}%
\pgfusepath{clip}%
\pgfsetbuttcap%
\pgfsetroundjoin%
\definecolor{currentfill}{rgb}{0.677823,0.786546,0.991005}%
\pgfsetfillcolor{currentfill}%
\pgfsetlinewidth{0.000000pt}%
\definecolor{currentstroke}{rgb}{0.000000,0.000000,0.000000}%
\pgfsetstrokecolor{currentstroke}%
\pgfsetdash{}{0pt}%
\pgfpathmoveto{\pgfqpoint{4.014411in}{3.179529in}}%
\pgfpathlineto{\pgfqpoint{4.025168in}{3.126731in}}%
\pgfpathlineto{\pgfqpoint{4.035938in}{3.073521in}}%
\pgfpathlineto{\pgfqpoint{4.070780in}{3.076800in}}%
\pgfpathlineto{\pgfqpoint{4.105602in}{3.077653in}}%
\pgfpathlineto{\pgfqpoint{4.094797in}{3.130724in}}%
\pgfpathlineto{\pgfqpoint{4.084005in}{3.183380in}}%
\pgfpathlineto{\pgfqpoint{4.049217in}{3.182816in}}%
\pgfpathlineto{\pgfqpoint{4.014411in}{3.179529in}}%
\pgfpathclose%
\pgfusepath{fill}%
\end{pgfscope}%
\begin{pgfscope}%
\pgfpathrectangle{\pgfqpoint{1.020000in}{0.880000in}}{\pgfqpoint{6.160000in}{6.160000in}}%
\pgfusepath{clip}%
\pgfsetbuttcap%
\pgfsetroundjoin%
\definecolor{currentfill}{rgb}{0.425199,0.559058,0.946061}%
\pgfsetfillcolor{currentfill}%
\pgfsetlinewidth{0.000000pt}%
\definecolor{currentstroke}{rgb}{0.000000,0.000000,0.000000}%
\pgfsetstrokecolor{currentstroke}%
\pgfsetdash{}{0pt}%
\pgfpathmoveto{\pgfqpoint{2.837301in}{2.671224in}}%
\pgfpathlineto{\pgfqpoint{2.846962in}{2.648051in}}%
\pgfpathlineto{\pgfqpoint{2.856641in}{2.624775in}}%
\pgfpathlineto{\pgfqpoint{2.891741in}{2.638075in}}%
\pgfpathlineto{\pgfqpoint{2.926811in}{2.652664in}}%
\pgfpathlineto{\pgfqpoint{2.917064in}{2.676660in}}%
\pgfpathlineto{\pgfqpoint{2.907337in}{2.700470in}}%
\pgfpathlineto{\pgfqpoint{2.872335in}{2.685159in}}%
\pgfpathlineto{\pgfqpoint{2.837301in}{2.671224in}}%
\pgfpathclose%
\pgfusepath{fill}%
\end{pgfscope}%
\begin{pgfscope}%
\pgfpathrectangle{\pgfqpoint{1.020000in}{0.880000in}}{\pgfqpoint{6.160000in}{6.160000in}}%
\pgfusepath{clip}%
\pgfsetbuttcap%
\pgfsetroundjoin%
\definecolor{currentfill}{rgb}{0.538004,0.674902,0.991722}%
\pgfsetfillcolor{currentfill}%
\pgfsetlinewidth{0.000000pt}%
\definecolor{currentstroke}{rgb}{0.000000,0.000000,0.000000}%
\pgfsetstrokecolor{currentstroke}%
\pgfsetdash{}{0pt}%
\pgfpathmoveto{\pgfqpoint{3.276034in}{2.876314in}}%
\pgfpathlineto{\pgfqpoint{3.286155in}{2.845013in}}%
\pgfpathlineto{\pgfqpoint{3.296298in}{2.813101in}}%
\pgfpathlineto{\pgfqpoint{3.331180in}{2.840674in}}%
\pgfpathlineto{\pgfqpoint{3.366050in}{2.869120in}}%
\pgfpathlineto{\pgfqpoint{3.355836in}{2.903192in}}%
\pgfpathlineto{\pgfqpoint{3.345643in}{2.936515in}}%
\pgfpathlineto{\pgfqpoint{3.310845in}{2.905940in}}%
\pgfpathlineto{\pgfqpoint{3.276034in}{2.876314in}}%
\pgfpathclose%
\pgfusepath{fill}%
\end{pgfscope}%
\begin{pgfscope}%
\pgfpathrectangle{\pgfqpoint{1.020000in}{0.880000in}}{\pgfqpoint{6.160000in}{6.160000in}}%
\pgfusepath{clip}%
\pgfsetbuttcap%
\pgfsetroundjoin%
\definecolor{currentfill}{rgb}{0.554312,0.690097,0.995516}%
\pgfsetfillcolor{currentfill}%
\pgfsetlinewidth{0.000000pt}%
\definecolor{currentstroke}{rgb}{0.000000,0.000000,0.000000}%
\pgfsetstrokecolor{currentstroke}%
\pgfsetdash{}{0pt}%
\pgfpathmoveto{\pgfqpoint{4.266432in}{2.955518in}}%
\pgfpathlineto{\pgfqpoint{4.277347in}{2.905029in}}%
\pgfpathlineto{\pgfqpoint{4.288277in}{2.855242in}}%
\pgfpathlineto{\pgfqpoint{4.323037in}{2.848137in}}%
\pgfpathlineto{\pgfqpoint{4.357766in}{2.839597in}}%
\pgfpathlineto{\pgfqpoint{4.346798in}{2.887643in}}%
\pgfpathlineto{\pgfqpoint{4.335846in}{2.936357in}}%
\pgfpathlineto{\pgfqpoint{4.301155in}{2.946770in}}%
\pgfpathlineto{\pgfqpoint{4.266432in}{2.955518in}}%
\pgfpathclose%
\pgfusepath{fill}%
\end{pgfscope}%
\begin{pgfscope}%
\pgfpathrectangle{\pgfqpoint{1.020000in}{0.880000in}}{\pgfqpoint{6.160000in}{6.160000in}}%
\pgfusepath{clip}%
\pgfsetbuttcap%
\pgfsetroundjoin%
\definecolor{currentfill}{rgb}{0.510824,0.649397,0.985079}%
\pgfsetfillcolor{currentfill}%
\pgfsetlinewidth{0.000000pt}%
\definecolor{currentstroke}{rgb}{0.000000,0.000000,0.000000}%
\pgfsetstrokecolor{currentstroke}%
\pgfsetdash{}{0pt}%
\pgfpathmoveto{\pgfqpoint{3.206367in}{2.820468in}}%
\pgfpathlineto{\pgfqpoint{3.216417in}{2.791023in}}%
\pgfpathlineto{\pgfqpoint{3.226486in}{2.761094in}}%
\pgfpathlineto{\pgfqpoint{3.261401in}{2.786537in}}%
\pgfpathlineto{\pgfqpoint{3.296298in}{2.813101in}}%
\pgfpathlineto{\pgfqpoint{3.286155in}{2.845013in}}%
\pgfpathlineto{\pgfqpoint{3.276034in}{2.876314in}}%
\pgfpathlineto{\pgfqpoint{3.241209in}{2.847782in}}%
\pgfpathlineto{\pgfqpoint{3.206367in}{2.820468in}}%
\pgfpathclose%
\pgfusepath{fill}%
\end{pgfscope}%
\begin{pgfscope}%
\pgfpathrectangle{\pgfqpoint{1.020000in}{0.880000in}}{\pgfqpoint{6.160000in}{6.160000in}}%
\pgfusepath{clip}%
\pgfsetbuttcap%
\pgfsetroundjoin%
\definecolor{currentfill}{rgb}{0.703587,0.802586,0.982847}%
\pgfsetfillcolor{currentfill}%
\pgfsetlinewidth{0.000000pt}%
\definecolor{currentstroke}{rgb}{0.000000,0.000000,0.000000}%
\pgfsetstrokecolor{currentstroke}%
\pgfsetdash{}{0pt}%
\pgfpathmoveto{\pgfqpoint{3.784085in}{3.203175in}}%
\pgfpathlineto{\pgfqpoint{3.794686in}{3.154942in}}%
\pgfpathlineto{\pgfqpoint{3.805303in}{3.105747in}}%
\pgfpathlineto{\pgfqpoint{3.840176in}{3.123925in}}%
\pgfpathlineto{\pgfqpoint{3.875042in}{3.139947in}}%
\pgfpathlineto{\pgfqpoint{3.864379in}{3.190603in}}%
\pgfpathlineto{\pgfqpoint{3.853732in}{3.240241in}}%
\pgfpathlineto{\pgfqpoint{3.818910in}{3.222898in}}%
\pgfpathlineto{\pgfqpoint{3.784085in}{3.203175in}}%
\pgfpathclose%
\pgfusepath{fill}%
\end{pgfscope}%
\begin{pgfscope}%
\pgfpathrectangle{\pgfqpoint{1.020000in}{0.880000in}}{\pgfqpoint{6.160000in}{6.160000in}}%
\pgfusepath{clip}%
\pgfsetbuttcap%
\pgfsetroundjoin%
\definecolor{currentfill}{rgb}{0.483854,0.622050,0.974808}%
\pgfsetfillcolor{currentfill}%
\pgfsetlinewidth{0.000000pt}%
\definecolor{currentstroke}{rgb}{0.000000,0.000000,0.000000}%
\pgfsetstrokecolor{currentstroke}%
\pgfsetdash{}{0pt}%
\pgfpathmoveto{\pgfqpoint{3.136626in}{2.769853in}}%
\pgfpathlineto{\pgfqpoint{3.146603in}{2.742063in}}%
\pgfpathlineto{\pgfqpoint{3.156600in}{2.713906in}}%
\pgfpathlineto{\pgfqpoint{3.191553in}{2.736861in}}%
\pgfpathlineto{\pgfqpoint{3.226486in}{2.761094in}}%
\pgfpathlineto{\pgfqpoint{3.216417in}{2.791023in}}%
\pgfpathlineto{\pgfqpoint{3.206367in}{2.820468in}}%
\pgfpathlineto{\pgfqpoint{3.171507in}{2.794467in}}%
\pgfpathlineto{\pgfqpoint{3.136626in}{2.769853in}}%
\pgfpathclose%
\pgfusepath{fill}%
\end{pgfscope}%
\begin{pgfscope}%
\pgfpathrectangle{\pgfqpoint{1.020000in}{0.880000in}}{\pgfqpoint{6.160000in}{6.160000in}}%
\pgfusepath{clip}%
\pgfsetbuttcap%
\pgfsetroundjoin%
\definecolor{currentfill}{rgb}{0.624703,0.748318,0.998719}%
\pgfsetfillcolor{currentfill}%
\pgfsetlinewidth{0.000000pt}%
\definecolor{currentstroke}{rgb}{0.000000,0.000000,0.000000}%
\pgfsetstrokecolor{currentstroke}%
\pgfsetdash{}{0pt}%
\pgfpathmoveto{\pgfqpoint{4.105602in}{3.077653in}}%
\pgfpathlineto{\pgfqpoint{4.116421in}{3.024485in}}%
\pgfpathlineto{\pgfqpoint{4.127253in}{2.971520in}}%
\pgfpathlineto{\pgfqpoint{4.162087in}{2.970549in}}%
\pgfpathlineto{\pgfqpoint{4.196896in}{2.967510in}}%
\pgfpathlineto{\pgfqpoint{4.186030in}{3.019747in}}%
\pgfpathlineto{\pgfqpoint{4.175177in}{3.072183in}}%
\pgfpathlineto{\pgfqpoint{4.140402in}{3.076099in}}%
\pgfpathlineto{\pgfqpoint{4.105602in}{3.077653in}}%
\pgfpathclose%
\pgfusepath{fill}%
\end{pgfscope}%
\begin{pgfscope}%
\pgfpathrectangle{\pgfqpoint{1.020000in}{0.880000in}}{\pgfqpoint{6.160000in}{6.160000in}}%
\pgfusepath{clip}%
\pgfsetbuttcap%
\pgfsetroundjoin%
\definecolor{currentfill}{rgb}{0.683056,0.790043,0.989768}%
\pgfsetfillcolor{currentfill}%
\pgfsetlinewidth{0.000000pt}%
\definecolor{currentstroke}{rgb}{0.000000,0.000000,0.000000}%
\pgfsetstrokecolor{currentstroke}%
\pgfsetdash{}{0pt}%
\pgfpathmoveto{\pgfqpoint{3.714428in}{3.157451in}}%
\pgfpathlineto{\pgfqpoint{3.724978in}{3.111022in}}%
\pgfpathlineto{\pgfqpoint{3.735546in}{3.063703in}}%
\pgfpathlineto{\pgfqpoint{3.770426in}{3.085604in}}%
\pgfpathlineto{\pgfqpoint{3.805303in}{3.105747in}}%
\pgfpathlineto{\pgfqpoint{3.794686in}{3.154942in}}%
\pgfpathlineto{\pgfqpoint{3.784085in}{3.203175in}}%
\pgfpathlineto{\pgfqpoint{3.749257in}{3.181283in}}%
\pgfpathlineto{\pgfqpoint{3.714428in}{3.157451in}}%
\pgfpathclose%
\pgfusepath{fill}%
\end{pgfscope}%
\begin{pgfscope}%
\pgfpathrectangle{\pgfqpoint{1.020000in}{0.880000in}}{\pgfqpoint{6.160000in}{6.160000in}}%
\pgfusepath{clip}%
\pgfsetbuttcap%
\pgfsetroundjoin%
\definecolor{currentfill}{rgb}{0.677823,0.786546,0.991005}%
\pgfsetfillcolor{currentfill}%
\pgfsetlinewidth{0.000000pt}%
\definecolor{currentstroke}{rgb}{0.000000,0.000000,0.000000}%
\pgfsetstrokecolor{currentstroke}%
\pgfsetdash{}{0pt}%
\pgfpathmoveto{\pgfqpoint{3.944750in}{3.164878in}}%
\pgfpathlineto{\pgfqpoint{3.955469in}{3.112524in}}%
\pgfpathlineto{\pgfqpoint{3.966203in}{3.059769in}}%
\pgfpathlineto{\pgfqpoint{4.001078in}{3.067829in}}%
\pgfpathlineto{\pgfqpoint{4.035938in}{3.073521in}}%
\pgfpathlineto{\pgfqpoint{4.025168in}{3.126731in}}%
\pgfpathlineto{\pgfqpoint{4.014411in}{3.179529in}}%
\pgfpathlineto{\pgfqpoint{3.979588in}{3.173533in}}%
\pgfpathlineto{\pgfqpoint{3.944750in}{3.164878in}}%
\pgfpathclose%
\pgfusepath{fill}%
\end{pgfscope}%
\begin{pgfscope}%
\pgfpathrectangle{\pgfqpoint{1.020000in}{0.880000in}}{\pgfqpoint{6.160000in}{6.160000in}}%
\pgfusepath{clip}%
\pgfsetbuttcap%
\pgfsetroundjoin%
\definecolor{currentfill}{rgb}{0.494638,0.633022,0.978983}%
\pgfsetfillcolor{currentfill}%
\pgfsetlinewidth{0.000000pt}%
\definecolor{currentstroke}{rgb}{0.000000,0.000000,0.000000}%
\pgfsetstrokecolor{currentstroke}%
\pgfsetdash{}{0pt}%
\pgfpathmoveto{\pgfqpoint{4.357766in}{2.839597in}}%
\pgfpathlineto{\pgfqpoint{4.368749in}{2.792427in}}%
\pgfpathlineto{\pgfqpoint{4.379750in}{2.746316in}}%
\pgfpathlineto{\pgfqpoint{4.414488in}{2.738449in}}%
\pgfpathlineto{\pgfqpoint{4.449194in}{2.729596in}}%
\pgfpathlineto{\pgfqpoint{4.438151in}{2.773674in}}%
\pgfpathlineto{\pgfqpoint{4.427127in}{2.818744in}}%
\pgfpathlineto{\pgfqpoint{4.392463in}{2.829752in}}%
\pgfpathlineto{\pgfqpoint{4.357766in}{2.839597in}}%
\pgfpathclose%
\pgfusepath{fill}%
\end{pgfscope}%
\begin{pgfscope}%
\pgfpathrectangle{\pgfqpoint{1.020000in}{0.880000in}}{\pgfqpoint{6.160000in}{6.160000in}}%
\pgfusepath{clip}%
\pgfsetbuttcap%
\pgfsetroundjoin%
\definecolor{currentfill}{rgb}{0.462354,0.599830,0.965857}%
\pgfsetfillcolor{currentfill}%
\pgfsetlinewidth{0.000000pt}%
\definecolor{currentstroke}{rgb}{0.000000,0.000000,0.000000}%
\pgfsetstrokecolor{currentstroke}%
\pgfsetdash{}{0pt}%
\pgfpathmoveto{\pgfqpoint{3.066794in}{2.724961in}}%
\pgfpathlineto{\pgfqpoint{3.076699in}{2.698609in}}%
\pgfpathlineto{\pgfqpoint{3.086624in}{2.671991in}}%
\pgfpathlineto{\pgfqpoint{3.121624in}{2.692274in}}%
\pgfpathlineto{\pgfqpoint{3.156600in}{2.713906in}}%
\pgfpathlineto{\pgfqpoint{3.146603in}{2.742063in}}%
\pgfpathlineto{\pgfqpoint{3.136626in}{2.769853in}}%
\pgfpathlineto{\pgfqpoint{3.101722in}{2.746674in}}%
\pgfpathlineto{\pgfqpoint{3.066794in}{2.724961in}}%
\pgfpathclose%
\pgfusepath{fill}%
\end{pgfscope}%
\begin{pgfscope}%
\pgfpathrectangle{\pgfqpoint{1.020000in}{0.880000in}}{\pgfqpoint{6.160000in}{6.160000in}}%
\pgfusepath{clip}%
\pgfsetbuttcap%
\pgfsetroundjoin%
\definecolor{currentfill}{rgb}{0.446431,0.582356,0.957373}%
\pgfsetfillcolor{currentfill}%
\pgfsetlinewidth{0.000000pt}%
\definecolor{currentstroke}{rgb}{0.000000,0.000000,0.000000}%
\pgfsetstrokecolor{currentstroke}%
\pgfsetdash{}{0pt}%
\pgfpathmoveto{\pgfqpoint{4.449194in}{2.729596in}}%
\pgfpathlineto{\pgfqpoint{4.460255in}{2.686658in}}%
\pgfpathlineto{\pgfqpoint{4.494952in}{2.678040in}}%
\pgfpathlineto{\pgfqpoint{4.529618in}{2.668759in}}%
\pgfpathlineto{\pgfqpoint{4.518510in}{2.709450in}}%
\pgfpathlineto{\pgfqpoint{4.483868in}{2.719885in}}%
\pgfpathlineto{\pgfqpoint{4.449194in}{2.729596in}}%
\pgfpathclose%
\pgfusepath{fill}%
\end{pgfscope}%
\begin{pgfscope}%
\pgfpathrectangle{\pgfqpoint{1.020000in}{0.880000in}}{\pgfqpoint{6.160000in}{6.160000in}}%
\pgfusepath{clip}%
\pgfsetbuttcap%
\pgfsetroundjoin%
\definecolor{currentfill}{rgb}{0.661968,0.775491,0.993937}%
\pgfsetfillcolor{currentfill}%
\pgfsetlinewidth{0.000000pt}%
\definecolor{currentstroke}{rgb}{0.000000,0.000000,0.000000}%
\pgfsetstrokecolor{currentstroke}%
\pgfsetdash{}{0pt}%
\pgfpathmoveto{\pgfqpoint{3.644768in}{3.104972in}}%
\pgfpathlineto{\pgfqpoint{3.655263in}{3.060659in}}%
\pgfpathlineto{\pgfqpoint{3.665776in}{3.015539in}}%
\pgfpathlineto{\pgfqpoint{3.700662in}{3.040270in}}%
\pgfpathlineto{\pgfqpoint{3.735546in}{3.063703in}}%
\pgfpathlineto{\pgfqpoint{3.724978in}{3.111022in}}%
\pgfpathlineto{\pgfqpoint{3.714428in}{3.157451in}}%
\pgfpathlineto{\pgfqpoint{3.679598in}{3.131927in}}%
\pgfpathlineto{\pgfqpoint{3.644768in}{3.104972in}}%
\pgfpathclose%
\pgfusepath{fill}%
\end{pgfscope}%
\begin{pgfscope}%
\pgfpathrectangle{\pgfqpoint{1.020000in}{0.880000in}}{\pgfqpoint{6.160000in}{6.160000in}}%
\pgfusepath{clip}%
\pgfsetbuttcap%
\pgfsetroundjoin%
\definecolor{currentfill}{rgb}{0.565182,0.699438,0.996635}%
\pgfsetfillcolor{currentfill}%
\pgfsetlinewidth{0.000000pt}%
\definecolor{currentstroke}{rgb}{0.000000,0.000000,0.000000}%
\pgfsetstrokecolor{currentstroke}%
\pgfsetdash{}{0pt}%
\pgfpathmoveto{\pgfqpoint{4.196896in}{2.967510in}}%
\pgfpathlineto{\pgfqpoint{4.207776in}{2.915747in}}%
\pgfpathlineto{\pgfqpoint{4.218669in}{2.864709in}}%
\pgfpathlineto{\pgfqpoint{4.253487in}{2.860799in}}%
\pgfpathlineto{\pgfqpoint{4.288277in}{2.855242in}}%
\pgfpathlineto{\pgfqpoint{4.277347in}{2.905029in}}%
\pgfpathlineto{\pgfqpoint{4.266432in}{2.955518in}}%
\pgfpathlineto{\pgfqpoint{4.231678in}{2.962469in}}%
\pgfpathlineto{\pgfqpoint{4.196896in}{2.967510in}}%
\pgfpathclose%
\pgfusepath{fill}%
\end{pgfscope}%
\begin{pgfscope}%
\pgfpathrectangle{\pgfqpoint{1.020000in}{0.880000in}}{\pgfqpoint{6.160000in}{6.160000in}}%
\pgfusepath{clip}%
\pgfsetbuttcap%
\pgfsetroundjoin%
\definecolor{currentfill}{rgb}{0.635474,0.756714,0.998297}%
\pgfsetfillcolor{currentfill}%
\pgfsetlinewidth{0.000000pt}%
\definecolor{currentstroke}{rgb}{0.000000,0.000000,0.000000}%
\pgfsetstrokecolor{currentstroke}%
\pgfsetdash{}{0pt}%
\pgfpathmoveto{\pgfqpoint{3.575107in}{3.047829in}}%
\pgfpathlineto{\pgfqpoint{3.585543in}{3.005846in}}%
\pgfpathlineto{\pgfqpoint{3.595996in}{2.963147in}}%
\pgfpathlineto{\pgfqpoint{3.630888in}{2.989750in}}%
\pgfpathlineto{\pgfqpoint{3.665776in}{3.015539in}}%
\pgfpathlineto{\pgfqpoint{3.655263in}{3.060659in}}%
\pgfpathlineto{\pgfqpoint{3.644768in}{3.104972in}}%
\pgfpathlineto{\pgfqpoint{3.609938in}{3.076850in}}%
\pgfpathlineto{\pgfqpoint{3.575107in}{3.047829in}}%
\pgfpathclose%
\pgfusepath{fill}%
\end{pgfscope}%
\begin{pgfscope}%
\pgfpathrectangle{\pgfqpoint{1.020000in}{0.880000in}}{\pgfqpoint{6.160000in}{6.160000in}}%
\pgfusepath{clip}%
\pgfsetbuttcap%
\pgfsetroundjoin%
\definecolor{currentfill}{rgb}{0.441123,0.576532,0.954545}%
\pgfsetfillcolor{currentfill}%
\pgfsetlinewidth{0.000000pt}%
\definecolor{currentstroke}{rgb}{0.000000,0.000000,0.000000}%
\pgfsetstrokecolor{currentstroke}%
\pgfsetdash{}{0pt}%
\pgfpathmoveto{\pgfqpoint{2.996859in}{2.685942in}}%
\pgfpathlineto{\pgfqpoint{3.006693in}{2.660805in}}%
\pgfpathlineto{\pgfqpoint{3.016546in}{2.635490in}}%
\pgfpathlineto{\pgfqpoint{3.051598in}{2.653066in}}%
\pgfpathlineto{\pgfqpoint{3.086624in}{2.671991in}}%
\pgfpathlineto{\pgfqpoint{3.076699in}{2.698609in}}%
\pgfpathlineto{\pgfqpoint{3.066794in}{2.724961in}}%
\pgfpathlineto{\pgfqpoint{3.031840in}{2.704720in}}%
\pgfpathlineto{\pgfqpoint{2.996859in}{2.685942in}}%
\pgfpathclose%
\pgfusepath{fill}%
\end{pgfscope}%
\begin{pgfscope}%
\pgfpathrectangle{\pgfqpoint{1.020000in}{0.880000in}}{\pgfqpoint{6.160000in}{6.160000in}}%
\pgfusepath{clip}%
\pgfsetbuttcap%
\pgfsetroundjoin%
\definecolor{currentfill}{rgb}{0.672538,0.782861,0.991982}%
\pgfsetfillcolor{currentfill}%
\pgfsetlinewidth{0.000000pt}%
\definecolor{currentstroke}{rgb}{0.000000,0.000000,0.000000}%
\pgfsetstrokecolor{currentstroke}%
\pgfsetdash{}{0pt}%
\pgfpathmoveto{\pgfqpoint{3.875042in}{3.139947in}}%
\pgfpathlineto{\pgfqpoint{3.885721in}{3.088594in}}%
\pgfpathlineto{\pgfqpoint{3.896413in}{3.036858in}}%
\pgfpathlineto{\pgfqpoint{3.931314in}{3.049412in}}%
\pgfpathlineto{\pgfqpoint{3.966203in}{3.059769in}}%
\pgfpathlineto{\pgfqpoint{3.955469in}{3.112524in}}%
\pgfpathlineto{\pgfqpoint{3.944750in}{3.164878in}}%
\pgfpathlineto{\pgfqpoint{3.909901in}{3.153645in}}%
\pgfpathlineto{\pgfqpoint{3.875042in}{3.139947in}}%
\pgfpathclose%
\pgfusepath{fill}%
\end{pgfscope}%
\begin{pgfscope}%
\pgfpathrectangle{\pgfqpoint{1.020000in}{0.880000in}}{\pgfqpoint{6.160000in}{6.160000in}}%
\pgfusepath{clip}%
\pgfsetbuttcap%
\pgfsetroundjoin%
\definecolor{currentfill}{rgb}{0.603162,0.731527,0.999565}%
\pgfsetfillcolor{currentfill}%
\pgfsetlinewidth{0.000000pt}%
\definecolor{currentstroke}{rgb}{0.000000,0.000000,0.000000}%
\pgfsetstrokecolor{currentstroke}%
\pgfsetdash{}{0pt}%
\pgfpathmoveto{\pgfqpoint{3.505441in}{2.988154in}}%
\pgfpathlineto{\pgfqpoint{3.515812in}{2.948617in}}%
\pgfpathlineto{\pgfqpoint{3.526203in}{2.908458in}}%
\pgfpathlineto{\pgfqpoint{3.561102in}{2.935971in}}%
\pgfpathlineto{\pgfqpoint{3.595996in}{2.963147in}}%
\pgfpathlineto{\pgfqpoint{3.585543in}{3.005846in}}%
\pgfpathlineto{\pgfqpoint{3.575107in}{3.047829in}}%
\pgfpathlineto{\pgfqpoint{3.540275in}{3.018177in}}%
\pgfpathlineto{\pgfqpoint{3.505441in}{2.988154in}}%
\pgfpathclose%
\pgfusepath{fill}%
\end{pgfscope}%
\begin{pgfscope}%
\pgfpathrectangle{\pgfqpoint{1.020000in}{0.880000in}}{\pgfqpoint{6.160000in}{6.160000in}}%
\pgfusepath{clip}%
\pgfsetbuttcap%
\pgfsetroundjoin%
\definecolor{currentfill}{rgb}{0.630089,0.752516,0.998508}%
\pgfsetfillcolor{currentfill}%
\pgfsetlinewidth{0.000000pt}%
\definecolor{currentstroke}{rgb}{0.000000,0.000000,0.000000}%
\pgfsetstrokecolor{currentstroke}%
\pgfsetdash{}{0pt}%
\pgfpathmoveto{\pgfqpoint{4.035938in}{3.073521in}}%
\pgfpathlineto{\pgfqpoint{4.046722in}{3.020216in}}%
\pgfpathlineto{\pgfqpoint{4.057518in}{2.967117in}}%
\pgfpathlineto{\pgfqpoint{4.092396in}{2.970381in}}%
\pgfpathlineto{\pgfqpoint{4.127253in}{2.971520in}}%
\pgfpathlineto{\pgfqpoint{4.116421in}{3.024485in}}%
\pgfpathlineto{\pgfqpoint{4.105602in}{3.077653in}}%
\pgfpathlineto{\pgfqpoint{4.070780in}{3.076800in}}%
\pgfpathlineto{\pgfqpoint{4.035938in}{3.073521in}}%
\pgfpathclose%
\pgfusepath{fill}%
\end{pgfscope}%
\begin{pgfscope}%
\pgfpathrectangle{\pgfqpoint{1.020000in}{0.880000in}}{\pgfqpoint{6.160000in}{6.160000in}}%
\pgfusepath{clip}%
\pgfsetbuttcap%
\pgfsetroundjoin%
\definecolor{currentfill}{rgb}{0.576051,0.708780,0.997755}%
\pgfsetfillcolor{currentfill}%
\pgfsetlinewidth{0.000000pt}%
\definecolor{currentstroke}{rgb}{0.000000,0.000000,0.000000}%
\pgfsetstrokecolor{currentstroke}%
\pgfsetdash{}{0pt}%
\pgfpathmoveto{\pgfqpoint{3.435759in}{2.927978in}}%
\pgfpathlineto{\pgfqpoint{3.446064in}{2.890908in}}%
\pgfpathlineto{\pgfqpoint{3.456387in}{2.853312in}}%
\pgfpathlineto{\pgfqpoint{3.491298in}{2.880834in}}%
\pgfpathlineto{\pgfqpoint{3.526203in}{2.908458in}}%
\pgfpathlineto{\pgfqpoint{3.515812in}{2.948617in}}%
\pgfpathlineto{\pgfqpoint{3.505441in}{2.988154in}}%
\pgfpathlineto{\pgfqpoint{3.470603in}{2.958009in}}%
\pgfpathlineto{\pgfqpoint{3.435759in}{2.927978in}}%
\pgfpathclose%
\pgfusepath{fill}%
\end{pgfscope}%
\begin{pgfscope}%
\pgfpathrectangle{\pgfqpoint{1.020000in}{0.880000in}}{\pgfqpoint{6.160000in}{6.160000in}}%
\pgfusepath{clip}%
\pgfsetbuttcap%
\pgfsetroundjoin%
\definecolor{currentfill}{rgb}{0.425199,0.559058,0.946061}%
\pgfsetfillcolor{currentfill}%
\pgfsetlinewidth{0.000000pt}%
\definecolor{currentstroke}{rgb}{0.000000,0.000000,0.000000}%
\pgfsetstrokecolor{currentstroke}%
\pgfsetdash{}{0pt}%
\pgfpathmoveto{\pgfqpoint{2.926811in}{2.652664in}}%
\pgfpathlineto{\pgfqpoint{2.936575in}{2.628523in}}%
\pgfpathlineto{\pgfqpoint{2.946358in}{2.604279in}}%
\pgfpathlineto{\pgfqpoint{2.981467in}{2.619239in}}%
\pgfpathlineto{\pgfqpoint{3.016546in}{2.635490in}}%
\pgfpathlineto{\pgfqpoint{3.006693in}{2.660805in}}%
\pgfpathlineto{\pgfqpoint{2.996859in}{2.685942in}}%
\pgfpathlineto{\pgfqpoint{2.961850in}{2.668604in}}%
\pgfpathlineto{\pgfqpoint{2.926811in}{2.652664in}}%
\pgfpathclose%
\pgfusepath{fill}%
\end{pgfscope}%
\begin{pgfscope}%
\pgfpathrectangle{\pgfqpoint{1.020000in}{0.880000in}}{\pgfqpoint{6.160000in}{6.160000in}}%
\pgfusepath{clip}%
\pgfsetbuttcap%
\pgfsetroundjoin%
\definecolor{currentfill}{rgb}{0.510824,0.649397,0.985079}%
\pgfsetfillcolor{currentfill}%
\pgfsetlinewidth{0.000000pt}%
\definecolor{currentstroke}{rgb}{0.000000,0.000000,0.000000}%
\pgfsetstrokecolor{currentstroke}%
\pgfsetdash{}{0pt}%
\pgfpathmoveto{\pgfqpoint{4.288277in}{2.855242in}}%
\pgfpathlineto{\pgfqpoint{4.299221in}{2.806375in}}%
\pgfpathlineto{\pgfqpoint{4.310182in}{2.758621in}}%
\pgfpathlineto{\pgfqpoint{4.344981in}{2.753077in}}%
\pgfpathlineto{\pgfqpoint{4.379750in}{2.746316in}}%
\pgfpathlineto{\pgfqpoint{4.368749in}{2.792427in}}%
\pgfpathlineto{\pgfqpoint{4.357766in}{2.839597in}}%
\pgfpathlineto{\pgfqpoint{4.323037in}{2.848137in}}%
\pgfpathlineto{\pgfqpoint{4.288277in}{2.855242in}}%
\pgfpathclose%
\pgfusepath{fill}%
\end{pgfscope}%
\begin{pgfscope}%
\pgfpathrectangle{\pgfqpoint{1.020000in}{0.880000in}}{\pgfqpoint{6.160000in}{6.160000in}}%
\pgfusepath{clip}%
\pgfsetbuttcap%
\pgfsetroundjoin%
\definecolor{currentfill}{rgb}{0.543440,0.680003,0.993051}%
\pgfsetfillcolor{currentfill}%
\pgfsetlinewidth{0.000000pt}%
\definecolor{currentstroke}{rgb}{0.000000,0.000000,0.000000}%
\pgfsetstrokecolor{currentstroke}%
\pgfsetdash{}{0pt}%
\pgfpathmoveto{\pgfqpoint{3.366050in}{2.869120in}}%
\pgfpathlineto{\pgfqpoint{3.376285in}{2.834456in}}%
\pgfpathlineto{\pgfqpoint{3.386538in}{2.799359in}}%
\pgfpathlineto{\pgfqpoint{3.421467in}{2.826093in}}%
\pgfpathlineto{\pgfqpoint{3.456387in}{2.853312in}}%
\pgfpathlineto{\pgfqpoint{3.446064in}{2.890908in}}%
\pgfpathlineto{\pgfqpoint{3.435759in}{2.927978in}}%
\pgfpathlineto{\pgfqpoint{3.400909in}{2.898281in}}%
\pgfpathlineto{\pgfqpoint{3.366050in}{2.869120in}}%
\pgfpathclose%
\pgfusepath{fill}%
\end{pgfscope}%
\begin{pgfscope}%
\pgfpathrectangle{\pgfqpoint{1.020000in}{0.880000in}}{\pgfqpoint{6.160000in}{6.160000in}}%
\pgfusepath{clip}%
\pgfsetbuttcap%
\pgfsetroundjoin%
\definecolor{currentfill}{rgb}{0.656683,0.771806,0.994914}%
\pgfsetfillcolor{currentfill}%
\pgfsetlinewidth{0.000000pt}%
\definecolor{currentstroke}{rgb}{0.000000,0.000000,0.000000}%
\pgfsetstrokecolor{currentstroke}%
\pgfsetdash{}{0pt}%
\pgfpathmoveto{\pgfqpoint{3.805303in}{3.105747in}}%
\pgfpathlineto{\pgfqpoint{3.815937in}{3.055897in}}%
\pgfpathlineto{\pgfqpoint{3.826585in}{3.005691in}}%
\pgfpathlineto{\pgfqpoint{3.861503in}{3.022235in}}%
\pgfpathlineto{\pgfqpoint{3.896413in}{3.036858in}}%
\pgfpathlineto{\pgfqpoint{3.885721in}{3.088594in}}%
\pgfpathlineto{\pgfqpoint{3.875042in}{3.139947in}}%
\pgfpathlineto{\pgfqpoint{3.840176in}{3.123925in}}%
\pgfpathlineto{\pgfqpoint{3.805303in}{3.105747in}}%
\pgfpathclose%
\pgfusepath{fill}%
\end{pgfscope}%
\begin{pgfscope}%
\pgfpathrectangle{\pgfqpoint{1.020000in}{0.880000in}}{\pgfqpoint{6.160000in}{6.160000in}}%
\pgfusepath{clip}%
\pgfsetbuttcap%
\pgfsetroundjoin%
\definecolor{currentfill}{rgb}{0.462354,0.599830,0.965857}%
\pgfsetfillcolor{currentfill}%
\pgfsetlinewidth{0.000000pt}%
\definecolor{currentstroke}{rgb}{0.000000,0.000000,0.000000}%
\pgfsetstrokecolor{currentstroke}%
\pgfsetdash{}{0pt}%
\pgfpathmoveto{\pgfqpoint{4.379750in}{2.746316in}}%
\pgfpathlineto{\pgfqpoint{4.390767in}{2.701420in}}%
\pgfpathlineto{\pgfqpoint{4.425527in}{2.694490in}}%
\pgfpathlineto{\pgfqpoint{4.460255in}{2.686658in}}%
\pgfpathlineto{\pgfqpoint{4.449194in}{2.729596in}}%
\pgfpathlineto{\pgfqpoint{4.414488in}{2.738449in}}%
\pgfpathlineto{\pgfqpoint{4.379750in}{2.746316in}}%
\pgfpathclose%
\pgfusepath{fill}%
\end{pgfscope}%
\begin{pgfscope}%
\pgfpathrectangle{\pgfqpoint{1.020000in}{0.880000in}}{\pgfqpoint{6.160000in}{6.160000in}}%
\pgfusepath{clip}%
\pgfsetbuttcap%
\pgfsetroundjoin%
\definecolor{currentfill}{rgb}{0.516260,0.654498,0.986407}%
\pgfsetfillcolor{currentfill}%
\pgfsetlinewidth{0.000000pt}%
\definecolor{currentstroke}{rgb}{0.000000,0.000000,0.000000}%
\pgfsetstrokecolor{currentstroke}%
\pgfsetdash{}{0pt}%
\pgfpathmoveto{\pgfqpoint{3.296298in}{2.813101in}}%
\pgfpathlineto{\pgfqpoint{3.306460in}{2.780711in}}%
\pgfpathlineto{\pgfqpoint{3.316641in}{2.747977in}}%
\pgfpathlineto{\pgfqpoint{3.351596in}{2.773272in}}%
\pgfpathlineto{\pgfqpoint{3.386538in}{2.799359in}}%
\pgfpathlineto{\pgfqpoint{3.376285in}{2.834456in}}%
\pgfpathlineto{\pgfqpoint{3.366050in}{2.869120in}}%
\pgfpathlineto{\pgfqpoint{3.331180in}{2.840674in}}%
\pgfpathlineto{\pgfqpoint{3.296298in}{2.813101in}}%
\pgfpathclose%
\pgfusepath{fill}%
\end{pgfscope}%
\begin{pgfscope}%
\pgfpathrectangle{\pgfqpoint{1.020000in}{0.880000in}}{\pgfqpoint{6.160000in}{6.160000in}}%
\pgfusepath{clip}%
\pgfsetbuttcap%
\pgfsetroundjoin%
\definecolor{currentfill}{rgb}{0.576051,0.708780,0.997755}%
\pgfsetfillcolor{currentfill}%
\pgfsetlinewidth{0.000000pt}%
\definecolor{currentstroke}{rgb}{0.000000,0.000000,0.000000}%
\pgfsetstrokecolor{currentstroke}%
\pgfsetdash{}{0pt}%
\pgfpathmoveto{\pgfqpoint{4.127253in}{2.971520in}}%
\pgfpathlineto{\pgfqpoint{4.138097in}{2.919039in}}%
\pgfpathlineto{\pgfqpoint{4.148955in}{2.867297in}}%
\pgfpathlineto{\pgfqpoint{4.183824in}{2.866894in}}%
\pgfpathlineto{\pgfqpoint{4.218669in}{2.864709in}}%
\pgfpathlineto{\pgfqpoint{4.207776in}{2.915747in}}%
\pgfpathlineto{\pgfqpoint{4.196896in}{2.967510in}}%
\pgfpathlineto{\pgfqpoint{4.162087in}{2.970549in}}%
\pgfpathlineto{\pgfqpoint{4.127253in}{2.971520in}}%
\pgfpathclose%
\pgfusepath{fill}%
\end{pgfscope}%
\begin{pgfscope}%
\pgfpathrectangle{\pgfqpoint{1.020000in}{0.880000in}}{\pgfqpoint{6.160000in}{6.160000in}}%
\pgfusepath{clip}%
\pgfsetbuttcap%
\pgfsetroundjoin%
\definecolor{currentfill}{rgb}{0.409611,0.540759,0.935545}%
\pgfsetfillcolor{currentfill}%
\pgfsetlinewidth{0.000000pt}%
\definecolor{currentstroke}{rgb}{0.000000,0.000000,0.000000}%
\pgfsetstrokecolor{currentstroke}%
\pgfsetdash{}{0pt}%
\pgfpathmoveto{\pgfqpoint{2.856641in}{2.624775in}}%
\pgfpathlineto{\pgfqpoint{2.866338in}{2.601425in}}%
\pgfpathlineto{\pgfqpoint{2.876052in}{2.578035in}}%
\pgfpathlineto{\pgfqpoint{2.911220in}{2.590563in}}%
\pgfpathlineto{\pgfqpoint{2.946358in}{2.604279in}}%
\pgfpathlineto{\pgfqpoint{2.936575in}{2.628523in}}%
\pgfpathlineto{\pgfqpoint{2.926811in}{2.652664in}}%
\pgfpathlineto{\pgfqpoint{2.891741in}{2.638075in}}%
\pgfpathlineto{\pgfqpoint{2.856641in}{2.624775in}}%
\pgfpathclose%
\pgfusepath{fill}%
\end{pgfscope}%
\begin{pgfscope}%
\pgfpathrectangle{\pgfqpoint{1.020000in}{0.880000in}}{\pgfqpoint{6.160000in}{6.160000in}}%
\pgfusepath{clip}%
\pgfsetbuttcap%
\pgfsetroundjoin%
\definecolor{currentfill}{rgb}{0.630089,0.752516,0.998508}%
\pgfsetfillcolor{currentfill}%
\pgfsetlinewidth{0.000000pt}%
\definecolor{currentstroke}{rgb}{0.000000,0.000000,0.000000}%
\pgfsetstrokecolor{currentstroke}%
\pgfsetdash{}{0pt}%
\pgfpathmoveto{\pgfqpoint{3.966203in}{3.059769in}}%
\pgfpathlineto{\pgfqpoint{3.976949in}{3.006924in}}%
\pgfpathlineto{\pgfqpoint{3.987708in}{2.954286in}}%
\pgfpathlineto{\pgfqpoint{4.022621in}{2.961740in}}%
\pgfpathlineto{\pgfqpoint{4.057518in}{2.967117in}}%
\pgfpathlineto{\pgfqpoint{4.046722in}{3.020216in}}%
\pgfpathlineto{\pgfqpoint{4.035938in}{3.073521in}}%
\pgfpathlineto{\pgfqpoint{4.001078in}{3.067829in}}%
\pgfpathlineto{\pgfqpoint{3.966203in}{3.059769in}}%
\pgfpathclose%
\pgfusepath{fill}%
\end{pgfscope}%
\begin{pgfscope}%
\pgfpathrectangle{\pgfqpoint{1.020000in}{0.880000in}}{\pgfqpoint{6.160000in}{6.160000in}}%
\pgfusepath{clip}%
\pgfsetbuttcap%
\pgfsetroundjoin%
\definecolor{currentfill}{rgb}{0.483854,0.622050,0.974808}%
\pgfsetfillcolor{currentfill}%
\pgfsetlinewidth{0.000000pt}%
\definecolor{currentstroke}{rgb}{0.000000,0.000000,0.000000}%
\pgfsetstrokecolor{currentstroke}%
\pgfsetdash{}{0pt}%
\pgfpathmoveto{\pgfqpoint{3.226486in}{2.761094in}}%
\pgfpathlineto{\pgfqpoint{3.236576in}{2.730792in}}%
\pgfpathlineto{\pgfqpoint{3.246683in}{2.700229in}}%
\pgfpathlineto{\pgfqpoint{3.281671in}{2.723595in}}%
\pgfpathlineto{\pgfqpoint{3.316641in}{2.747977in}}%
\pgfpathlineto{\pgfqpoint{3.306460in}{2.780711in}}%
\pgfpathlineto{\pgfqpoint{3.296298in}{2.813101in}}%
\pgfpathlineto{\pgfqpoint{3.261401in}{2.786537in}}%
\pgfpathlineto{\pgfqpoint{3.226486in}{2.761094in}}%
\pgfpathclose%
\pgfusepath{fill}%
\end{pgfscope}%
\begin{pgfscope}%
\pgfpathrectangle{\pgfqpoint{1.020000in}{0.880000in}}{\pgfqpoint{6.160000in}{6.160000in}}%
\pgfusepath{clip}%
\pgfsetbuttcap%
\pgfsetroundjoin%
\definecolor{currentfill}{rgb}{0.640828,0.760752,0.997846}%
\pgfsetfillcolor{currentfill}%
\pgfsetlinewidth{0.000000pt}%
\definecolor{currentstroke}{rgb}{0.000000,0.000000,0.000000}%
\pgfsetstrokecolor{currentstroke}%
\pgfsetdash{}{0pt}%
\pgfpathmoveto{\pgfqpoint{3.735546in}{3.063703in}}%
\pgfpathlineto{\pgfqpoint{3.746130in}{3.015781in}}%
\pgfpathlineto{\pgfqpoint{3.756729in}{2.967533in}}%
\pgfpathlineto{\pgfqpoint{3.791660in}{2.987395in}}%
\pgfpathlineto{\pgfqpoint{3.826585in}{3.005691in}}%
\pgfpathlineto{\pgfqpoint{3.815937in}{3.055897in}}%
\pgfpathlineto{\pgfqpoint{3.805303in}{3.105747in}}%
\pgfpathlineto{\pgfqpoint{3.770426in}{3.085604in}}%
\pgfpathlineto{\pgfqpoint{3.735546in}{3.063703in}}%
\pgfpathclose%
\pgfusepath{fill}%
\end{pgfscope}%
\begin{pgfscope}%
\pgfpathrectangle{\pgfqpoint{1.020000in}{0.880000in}}{\pgfqpoint{6.160000in}{6.160000in}}%
\pgfusepath{clip}%
\pgfsetbuttcap%
\pgfsetroundjoin%
\definecolor{currentfill}{rgb}{0.462354,0.599830,0.965857}%
\pgfsetfillcolor{currentfill}%
\pgfsetlinewidth{0.000000pt}%
\definecolor{currentstroke}{rgb}{0.000000,0.000000,0.000000}%
\pgfsetstrokecolor{currentstroke}%
\pgfsetdash{}{0pt}%
\pgfpathmoveto{\pgfqpoint{3.156600in}{2.713906in}}%
\pgfpathlineto{\pgfqpoint{3.166616in}{2.685470in}}%
\pgfpathlineto{\pgfqpoint{3.176650in}{2.656848in}}%
\pgfpathlineto{\pgfqpoint{3.211677in}{2.677960in}}%
\pgfpathlineto{\pgfqpoint{3.246683in}{2.700229in}}%
\pgfpathlineto{\pgfqpoint{3.236576in}{2.730792in}}%
\pgfpathlineto{\pgfqpoint{3.226486in}{2.761094in}}%
\pgfpathlineto{\pgfqpoint{3.191553in}{2.736861in}}%
\pgfpathlineto{\pgfqpoint{3.156600in}{2.713906in}}%
\pgfpathclose%
\pgfusepath{fill}%
\end{pgfscope}%
\begin{pgfscope}%
\pgfpathrectangle{\pgfqpoint{1.020000in}{0.880000in}}{\pgfqpoint{6.160000in}{6.160000in}}%
\pgfusepath{clip}%
\pgfsetbuttcap%
\pgfsetroundjoin%
\definecolor{currentfill}{rgb}{0.619318,0.744121,0.998931}%
\pgfsetfillcolor{currentfill}%
\pgfsetlinewidth{0.000000pt}%
\definecolor{currentstroke}{rgb}{0.000000,0.000000,0.000000}%
\pgfsetstrokecolor{currentstroke}%
\pgfsetdash{}{0pt}%
\pgfpathmoveto{\pgfqpoint{3.665776in}{3.015539in}}%
\pgfpathlineto{\pgfqpoint{3.676306in}{2.969874in}}%
\pgfpathlineto{\pgfqpoint{3.686851in}{2.923922in}}%
\pgfpathlineto{\pgfqpoint{3.721793in}{2.946306in}}%
\pgfpathlineto{\pgfqpoint{3.756729in}{2.967533in}}%
\pgfpathlineto{\pgfqpoint{3.746130in}{3.015781in}}%
\pgfpathlineto{\pgfqpoint{3.735546in}{3.063703in}}%
\pgfpathlineto{\pgfqpoint{3.700662in}{3.040270in}}%
\pgfpathlineto{\pgfqpoint{3.665776in}{3.015539in}}%
\pgfpathclose%
\pgfusepath{fill}%
\end{pgfscope}%
\begin{pgfscope}%
\pgfpathrectangle{\pgfqpoint{1.020000in}{0.880000in}}{\pgfqpoint{6.160000in}{6.160000in}}%
\pgfusepath{clip}%
\pgfsetbuttcap%
\pgfsetroundjoin%
\definecolor{currentfill}{rgb}{0.521696,0.659599,0.987736}%
\pgfsetfillcolor{currentfill}%
\pgfsetlinewidth{0.000000pt}%
\definecolor{currentstroke}{rgb}{0.000000,0.000000,0.000000}%
\pgfsetstrokecolor{currentstroke}%
\pgfsetdash{}{0pt}%
\pgfpathmoveto{\pgfqpoint{4.218669in}{2.864709in}}%
\pgfpathlineto{\pgfqpoint{4.229577in}{2.814623in}}%
\pgfpathlineto{\pgfqpoint{4.240499in}{2.765688in}}%
\pgfpathlineto{\pgfqpoint{4.275354in}{2.762852in}}%
\pgfpathlineto{\pgfqpoint{4.310182in}{2.758621in}}%
\pgfpathlineto{\pgfqpoint{4.299221in}{2.806375in}}%
\pgfpathlineto{\pgfqpoint{4.288277in}{2.855242in}}%
\pgfpathlineto{\pgfqpoint{4.253487in}{2.860799in}}%
\pgfpathlineto{\pgfqpoint{4.218669in}{2.864709in}}%
\pgfpathclose%
\pgfusepath{fill}%
\end{pgfscope}%
\begin{pgfscope}%
\pgfpathrectangle{\pgfqpoint{1.020000in}{0.880000in}}{\pgfqpoint{6.160000in}{6.160000in}}%
\pgfusepath{clip}%
\pgfsetbuttcap%
\pgfsetroundjoin%
\definecolor{currentfill}{rgb}{0.619318,0.744121,0.998931}%
\pgfsetfillcolor{currentfill}%
\pgfsetlinewidth{0.000000pt}%
\definecolor{currentstroke}{rgb}{0.000000,0.000000,0.000000}%
\pgfsetstrokecolor{currentstroke}%
\pgfsetdash{}{0pt}%
\pgfpathmoveto{\pgfqpoint{3.896413in}{3.036858in}}%
\pgfpathlineto{\pgfqpoint{3.907119in}{2.985042in}}%
\pgfpathlineto{\pgfqpoint{3.917838in}{2.933431in}}%
\pgfpathlineto{\pgfqpoint{3.952780in}{2.944821in}}%
\pgfpathlineto{\pgfqpoint{3.987708in}{2.954286in}}%
\pgfpathlineto{\pgfqpoint{3.976949in}{3.006924in}}%
\pgfpathlineto{\pgfqpoint{3.966203in}{3.059769in}}%
\pgfpathlineto{\pgfqpoint{3.931314in}{3.049412in}}%
\pgfpathlineto{\pgfqpoint{3.896413in}{3.036858in}}%
\pgfpathclose%
\pgfusepath{fill}%
\end{pgfscope}%
\begin{pgfscope}%
\pgfpathrectangle{\pgfqpoint{1.020000in}{0.880000in}}{\pgfqpoint{6.160000in}{6.160000in}}%
\pgfusepath{clip}%
\pgfsetbuttcap%
\pgfsetroundjoin%
\definecolor{currentfill}{rgb}{0.592356,0.722792,0.999434}%
\pgfsetfillcolor{currentfill}%
\pgfsetlinewidth{0.000000pt}%
\definecolor{currentstroke}{rgb}{0.000000,0.000000,0.000000}%
\pgfsetstrokecolor{currentstroke}%
\pgfsetdash{}{0pt}%
\pgfpathmoveto{\pgfqpoint{3.595996in}{2.963147in}}%
\pgfpathlineto{\pgfqpoint{3.606467in}{2.919968in}}%
\pgfpathlineto{\pgfqpoint{3.616954in}{2.876542in}}%
\pgfpathlineto{\pgfqpoint{3.651905in}{2.900595in}}%
\pgfpathlineto{\pgfqpoint{3.686851in}{2.923922in}}%
\pgfpathlineto{\pgfqpoint{3.676306in}{2.969874in}}%
\pgfpathlineto{\pgfqpoint{3.665776in}{3.015539in}}%
\pgfpathlineto{\pgfqpoint{3.630888in}{2.989750in}}%
\pgfpathlineto{\pgfqpoint{3.595996in}{2.963147in}}%
\pgfpathclose%
\pgfusepath{fill}%
\end{pgfscope}%
\begin{pgfscope}%
\pgfpathrectangle{\pgfqpoint{1.020000in}{0.880000in}}{\pgfqpoint{6.160000in}{6.160000in}}%
\pgfusepath{clip}%
\pgfsetbuttcap%
\pgfsetroundjoin%
\definecolor{currentfill}{rgb}{0.441123,0.576532,0.954545}%
\pgfsetfillcolor{currentfill}%
\pgfsetlinewidth{0.000000pt}%
\definecolor{currentstroke}{rgb}{0.000000,0.000000,0.000000}%
\pgfsetstrokecolor{currentstroke}%
\pgfsetdash{}{0pt}%
\pgfpathmoveto{\pgfqpoint{3.086624in}{2.671991in}}%
\pgfpathlineto{\pgfqpoint{3.096567in}{2.645180in}}%
\pgfpathlineto{\pgfqpoint{3.106528in}{2.618248in}}%
\pgfpathlineto{\pgfqpoint{3.141601in}{2.636937in}}%
\pgfpathlineto{\pgfqpoint{3.176650in}{2.656848in}}%
\pgfpathlineto{\pgfqpoint{3.166616in}{2.685470in}}%
\pgfpathlineto{\pgfqpoint{3.156600in}{2.713906in}}%
\pgfpathlineto{\pgfqpoint{3.121624in}{2.692274in}}%
\pgfpathlineto{\pgfqpoint{3.086624in}{2.671991in}}%
\pgfpathclose%
\pgfusepath{fill}%
\end{pgfscope}%
\begin{pgfscope}%
\pgfpathrectangle{\pgfqpoint{1.020000in}{0.880000in}}{\pgfqpoint{6.160000in}{6.160000in}}%
\pgfusepath{clip}%
\pgfsetbuttcap%
\pgfsetroundjoin%
\definecolor{currentfill}{rgb}{0.581486,0.713451,0.998314}%
\pgfsetfillcolor{currentfill}%
\pgfsetlinewidth{0.000000pt}%
\definecolor{currentstroke}{rgb}{0.000000,0.000000,0.000000}%
\pgfsetstrokecolor{currentstroke}%
\pgfsetdash{}{0pt}%
\pgfpathmoveto{\pgfqpoint{4.057518in}{2.967117in}}%
\pgfpathlineto{\pgfqpoint{4.068327in}{2.914505in}}%
\pgfpathlineto{\pgfqpoint{4.079148in}{2.862635in}}%
\pgfpathlineto{\pgfqpoint{4.114062in}{2.865882in}}%
\pgfpathlineto{\pgfqpoint{4.148955in}{2.867297in}}%
\pgfpathlineto{\pgfqpoint{4.138097in}{2.919039in}}%
\pgfpathlineto{\pgfqpoint{4.127253in}{2.971520in}}%
\pgfpathlineto{\pgfqpoint{4.092396in}{2.970381in}}%
\pgfpathlineto{\pgfqpoint{4.057518in}{2.967117in}}%
\pgfpathclose%
\pgfusepath{fill}%
\end{pgfscope}%
\begin{pgfscope}%
\pgfpathrectangle{\pgfqpoint{1.020000in}{0.880000in}}{\pgfqpoint{6.160000in}{6.160000in}}%
\pgfusepath{clip}%
\pgfsetbuttcap%
\pgfsetroundjoin%
\definecolor{currentfill}{rgb}{0.478462,0.616564,0.972721}%
\pgfsetfillcolor{currentfill}%
\pgfsetlinewidth{0.000000pt}%
\definecolor{currentstroke}{rgb}{0.000000,0.000000,0.000000}%
\pgfsetstrokecolor{currentstroke}%
\pgfsetdash{}{0pt}%
\pgfpathmoveto{\pgfqpoint{4.310182in}{2.758621in}}%
\pgfpathlineto{\pgfqpoint{4.321159in}{2.712143in}}%
\pgfpathlineto{\pgfqpoint{4.355978in}{2.707338in}}%
\pgfpathlineto{\pgfqpoint{4.390767in}{2.701420in}}%
\pgfpathlineto{\pgfqpoint{4.379750in}{2.746316in}}%
\pgfpathlineto{\pgfqpoint{4.344981in}{2.753077in}}%
\pgfpathlineto{\pgfqpoint{4.310182in}{2.758621in}}%
\pgfpathclose%
\pgfusepath{fill}%
\end{pgfscope}%
\begin{pgfscope}%
\pgfpathrectangle{\pgfqpoint{1.020000in}{0.880000in}}{\pgfqpoint{6.160000in}{6.160000in}}%
\pgfusepath{clip}%
\pgfsetbuttcap%
\pgfsetroundjoin%
\definecolor{currentfill}{rgb}{0.565182,0.699438,0.996635}%
\pgfsetfillcolor{currentfill}%
\pgfsetlinewidth{0.000000pt}%
\definecolor{currentstroke}{rgb}{0.000000,0.000000,0.000000}%
\pgfsetstrokecolor{currentstroke}%
\pgfsetdash{}{0pt}%
\pgfpathmoveto{\pgfqpoint{3.526203in}{2.908458in}}%
\pgfpathlineto{\pgfqpoint{3.536610in}{2.867887in}}%
\pgfpathlineto{\pgfqpoint{3.547034in}{2.827112in}}%
\pgfpathlineto{\pgfqpoint{3.581997in}{2.851977in}}%
\pgfpathlineto{\pgfqpoint{3.616954in}{2.876542in}}%
\pgfpathlineto{\pgfqpoint{3.606467in}{2.919968in}}%
\pgfpathlineto{\pgfqpoint{3.595996in}{2.963147in}}%
\pgfpathlineto{\pgfqpoint{3.561102in}{2.935971in}}%
\pgfpathlineto{\pgfqpoint{3.526203in}{2.908458in}}%
\pgfpathclose%
\pgfusepath{fill}%
\end{pgfscope}%
\begin{pgfscope}%
\pgfpathrectangle{\pgfqpoint{1.020000in}{0.880000in}}{\pgfqpoint{6.160000in}{6.160000in}}%
\pgfusepath{clip}%
\pgfsetbuttcap%
\pgfsetroundjoin%
\definecolor{currentfill}{rgb}{0.425199,0.559058,0.946061}%
\pgfsetfillcolor{currentfill}%
\pgfsetlinewidth{0.000000pt}%
\definecolor{currentstroke}{rgb}{0.000000,0.000000,0.000000}%
\pgfsetstrokecolor{currentstroke}%
\pgfsetdash{}{0pt}%
\pgfpathmoveto{\pgfqpoint{3.016546in}{2.635490in}}%
\pgfpathlineto{\pgfqpoint{3.026417in}{2.610054in}}%
\pgfpathlineto{\pgfqpoint{3.036306in}{2.584556in}}%
\pgfpathlineto{\pgfqpoint{3.071430in}{2.600791in}}%
\pgfpathlineto{\pgfqpoint{3.106528in}{2.618248in}}%
\pgfpathlineto{\pgfqpoint{3.096567in}{2.645180in}}%
\pgfpathlineto{\pgfqpoint{3.086624in}{2.671991in}}%
\pgfpathlineto{\pgfqpoint{3.051598in}{2.653066in}}%
\pgfpathlineto{\pgfqpoint{3.016546in}{2.635490in}}%
\pgfpathclose%
\pgfusepath{fill}%
\end{pgfscope}%
\begin{pgfscope}%
\pgfpathrectangle{\pgfqpoint{1.020000in}{0.880000in}}{\pgfqpoint{6.160000in}{6.160000in}}%
\pgfusepath{clip}%
\pgfsetbuttcap%
\pgfsetroundjoin%
\definecolor{currentfill}{rgb}{0.608547,0.735725,0.999354}%
\pgfsetfillcolor{currentfill}%
\pgfsetlinewidth{0.000000pt}%
\definecolor{currentstroke}{rgb}{0.000000,0.000000,0.000000}%
\pgfsetstrokecolor{currentstroke}%
\pgfsetdash{}{0pt}%
\pgfpathmoveto{\pgfqpoint{3.826585in}{3.005691in}}%
\pgfpathlineto{\pgfqpoint{3.837247in}{2.955414in}}%
\pgfpathlineto{\pgfqpoint{3.847921in}{2.905341in}}%
\pgfpathlineto{\pgfqpoint{3.882885in}{2.920228in}}%
\pgfpathlineto{\pgfqpoint{3.917838in}{2.933431in}}%
\pgfpathlineto{\pgfqpoint{3.907119in}{2.985042in}}%
\pgfpathlineto{\pgfqpoint{3.896413in}{3.036858in}}%
\pgfpathlineto{\pgfqpoint{3.861503in}{3.022235in}}%
\pgfpathlineto{\pgfqpoint{3.826585in}{3.005691in}}%
\pgfpathclose%
\pgfusepath{fill}%
\end{pgfscope}%
\begin{pgfscope}%
\pgfpathrectangle{\pgfqpoint{1.020000in}{0.880000in}}{\pgfqpoint{6.160000in}{6.160000in}}%
\pgfusepath{clip}%
\pgfsetbuttcap%
\pgfsetroundjoin%
\definecolor{currentfill}{rgb}{0.538004,0.674902,0.991722}%
\pgfsetfillcolor{currentfill}%
\pgfsetlinewidth{0.000000pt}%
\definecolor{currentstroke}{rgb}{0.000000,0.000000,0.000000}%
\pgfsetstrokecolor{currentstroke}%
\pgfsetdash{}{0pt}%
\pgfpathmoveto{\pgfqpoint{3.456387in}{2.853312in}}%
\pgfpathlineto{\pgfqpoint{3.466728in}{2.815374in}}%
\pgfpathlineto{\pgfqpoint{3.477085in}{2.777274in}}%
\pgfpathlineto{\pgfqpoint{3.512064in}{2.802147in}}%
\pgfpathlineto{\pgfqpoint{3.547034in}{2.827112in}}%
\pgfpathlineto{\pgfqpoint{3.536610in}{2.867887in}}%
\pgfpathlineto{\pgfqpoint{3.526203in}{2.908458in}}%
\pgfpathlineto{\pgfqpoint{3.491298in}{2.880834in}}%
\pgfpathlineto{\pgfqpoint{3.456387in}{2.853312in}}%
\pgfpathclose%
\pgfusepath{fill}%
\end{pgfscope}%
\begin{pgfscope}%
\pgfpathrectangle{\pgfqpoint{1.020000in}{0.880000in}}{\pgfqpoint{6.160000in}{6.160000in}}%
\pgfusepath{clip}%
\pgfsetbuttcap%
\pgfsetroundjoin%
\definecolor{currentfill}{rgb}{0.527132,0.664700,0.989065}%
\pgfsetfillcolor{currentfill}%
\pgfsetlinewidth{0.000000pt}%
\definecolor{currentstroke}{rgb}{0.000000,0.000000,0.000000}%
\pgfsetstrokecolor{currentstroke}%
\pgfsetdash{}{0pt}%
\pgfpathmoveto{\pgfqpoint{4.148955in}{2.867297in}}%
\pgfpathlineto{\pgfqpoint{4.159826in}{2.816525in}}%
\pgfpathlineto{\pgfqpoint{4.170711in}{2.766924in}}%
\pgfpathlineto{\pgfqpoint{4.205617in}{2.767062in}}%
\pgfpathlineto{\pgfqpoint{4.240499in}{2.765688in}}%
\pgfpathlineto{\pgfqpoint{4.229577in}{2.814623in}}%
\pgfpathlineto{\pgfqpoint{4.218669in}{2.864709in}}%
\pgfpathlineto{\pgfqpoint{4.183824in}{2.866894in}}%
\pgfpathlineto{\pgfqpoint{4.148955in}{2.867297in}}%
\pgfpathclose%
\pgfusepath{fill}%
\end{pgfscope}%
\begin{pgfscope}%
\pgfpathrectangle{\pgfqpoint{1.020000in}{0.880000in}}{\pgfqpoint{6.160000in}{6.160000in}}%
\pgfusepath{clip}%
\pgfsetbuttcap%
\pgfsetroundjoin%
\definecolor{currentfill}{rgb}{0.510824,0.649397,0.985079}%
\pgfsetfillcolor{currentfill}%
\pgfsetlinewidth{0.000000pt}%
\definecolor{currentstroke}{rgb}{0.000000,0.000000,0.000000}%
\pgfsetstrokecolor{currentstroke}%
\pgfsetdash{}{0pt}%
\pgfpathmoveto{\pgfqpoint{3.386538in}{2.799359in}}%
\pgfpathlineto{\pgfqpoint{3.396808in}{2.763987in}}%
\pgfpathlineto{\pgfqpoint{3.407096in}{2.728496in}}%
\pgfpathlineto{\pgfqpoint{3.442096in}{2.752669in}}%
\pgfpathlineto{\pgfqpoint{3.477085in}{2.777274in}}%
\pgfpathlineto{\pgfqpoint{3.466728in}{2.815374in}}%
\pgfpathlineto{\pgfqpoint{3.456387in}{2.853312in}}%
\pgfpathlineto{\pgfqpoint{3.421467in}{2.826093in}}%
\pgfpathlineto{\pgfqpoint{3.386538in}{2.799359in}}%
\pgfpathclose%
\pgfusepath{fill}%
\end{pgfscope}%
\begin{pgfscope}%
\pgfpathrectangle{\pgfqpoint{1.020000in}{0.880000in}}{\pgfqpoint{6.160000in}{6.160000in}}%
\pgfusepath{clip}%
\pgfsetbuttcap%
\pgfsetroundjoin%
\definecolor{currentfill}{rgb}{0.576051,0.708780,0.997755}%
\pgfsetfillcolor{currentfill}%
\pgfsetlinewidth{0.000000pt}%
\definecolor{currentstroke}{rgb}{0.000000,0.000000,0.000000}%
\pgfsetstrokecolor{currentstroke}%
\pgfsetdash{}{0pt}%
\pgfpathmoveto{\pgfqpoint{3.987708in}{2.954286in}}%
\pgfpathlineto{\pgfqpoint{3.998479in}{2.902131in}}%
\pgfpathlineto{\pgfqpoint{4.009262in}{2.850710in}}%
\pgfpathlineto{\pgfqpoint{4.044214in}{2.857567in}}%
\pgfpathlineto{\pgfqpoint{4.079148in}{2.862635in}}%
\pgfpathlineto{\pgfqpoint{4.068327in}{2.914505in}}%
\pgfpathlineto{\pgfqpoint{4.057518in}{2.967117in}}%
\pgfpathlineto{\pgfqpoint{4.022621in}{2.961740in}}%
\pgfpathlineto{\pgfqpoint{3.987708in}{2.954286in}}%
\pgfpathclose%
\pgfusepath{fill}%
\end{pgfscope}%
\begin{pgfscope}%
\pgfpathrectangle{\pgfqpoint{1.020000in}{0.880000in}}{\pgfqpoint{6.160000in}{6.160000in}}%
\pgfusepath{clip}%
\pgfsetbuttcap%
\pgfsetroundjoin%
\definecolor{currentfill}{rgb}{0.409611,0.540759,0.935545}%
\pgfsetfillcolor{currentfill}%
\pgfsetlinewidth{0.000000pt}%
\definecolor{currentstroke}{rgb}{0.000000,0.000000,0.000000}%
\pgfsetstrokecolor{currentstroke}%
\pgfsetdash{}{0pt}%
\pgfpathmoveto{\pgfqpoint{2.946358in}{2.604279in}}%
\pgfpathlineto{\pgfqpoint{2.956158in}{2.579977in}}%
\pgfpathlineto{\pgfqpoint{2.965975in}{2.555661in}}%
\pgfpathlineto{\pgfqpoint{3.001155in}{2.569523in}}%
\pgfpathlineto{\pgfqpoint{3.036306in}{2.584556in}}%
\pgfpathlineto{\pgfqpoint{3.026417in}{2.610054in}}%
\pgfpathlineto{\pgfqpoint{3.016546in}{2.635490in}}%
\pgfpathlineto{\pgfqpoint{2.981467in}{2.619239in}}%
\pgfpathlineto{\pgfqpoint{2.946358in}{2.604279in}}%
\pgfpathclose%
\pgfusepath{fill}%
\end{pgfscope}%
\begin{pgfscope}%
\pgfpathrectangle{\pgfqpoint{1.020000in}{0.880000in}}{\pgfqpoint{6.160000in}{6.160000in}}%
\pgfusepath{clip}%
\pgfsetbuttcap%
\pgfsetroundjoin%
\definecolor{currentfill}{rgb}{0.592356,0.722792,0.999434}%
\pgfsetfillcolor{currentfill}%
\pgfsetlinewidth{0.000000pt}%
\definecolor{currentstroke}{rgb}{0.000000,0.000000,0.000000}%
\pgfsetstrokecolor{currentstroke}%
\pgfsetdash{}{0pt}%
\pgfpathmoveto{\pgfqpoint{3.756729in}{2.967533in}}%
\pgfpathlineto{\pgfqpoint{3.767342in}{2.919230in}}%
\pgfpathlineto{\pgfqpoint{3.777968in}{2.871126in}}%
\pgfpathlineto{\pgfqpoint{3.812949in}{2.888919in}}%
\pgfpathlineto{\pgfqpoint{3.847921in}{2.905341in}}%
\pgfpathlineto{\pgfqpoint{3.837247in}{2.955414in}}%
\pgfpathlineto{\pgfqpoint{3.826585in}{3.005691in}}%
\pgfpathlineto{\pgfqpoint{3.791660in}{2.987395in}}%
\pgfpathlineto{\pgfqpoint{3.756729in}{2.967533in}}%
\pgfpathclose%
\pgfusepath{fill}%
\end{pgfscope}%
\begin{pgfscope}%
\pgfpathrectangle{\pgfqpoint{1.020000in}{0.880000in}}{\pgfqpoint{6.160000in}{6.160000in}}%
\pgfusepath{clip}%
\pgfsetbuttcap%
\pgfsetroundjoin%
\definecolor{currentfill}{rgb}{0.483854,0.622050,0.974808}%
\pgfsetfillcolor{currentfill}%
\pgfsetlinewidth{0.000000pt}%
\definecolor{currentstroke}{rgb}{0.000000,0.000000,0.000000}%
\pgfsetstrokecolor{currentstroke}%
\pgfsetdash{}{0pt}%
\pgfpathmoveto{\pgfqpoint{3.316641in}{2.747977in}}%
\pgfpathlineto{\pgfqpoint{3.326840in}{2.715032in}}%
\pgfpathlineto{\pgfqpoint{3.337055in}{2.682009in}}%
\pgfpathlineto{\pgfqpoint{3.372083in}{2.704900in}}%
\pgfpathlineto{\pgfqpoint{3.407096in}{2.728496in}}%
\pgfpathlineto{\pgfqpoint{3.396808in}{2.763987in}}%
\pgfpathlineto{\pgfqpoint{3.386538in}{2.799359in}}%
\pgfpathlineto{\pgfqpoint{3.351596in}{2.773272in}}%
\pgfpathlineto{\pgfqpoint{3.316641in}{2.747977in}}%
\pgfpathclose%
\pgfusepath{fill}%
\end{pgfscope}%
\begin{pgfscope}%
\pgfpathrectangle{\pgfqpoint{1.020000in}{0.880000in}}{\pgfqpoint{6.160000in}{6.160000in}}%
\pgfusepath{clip}%
\pgfsetbuttcap%
\pgfsetroundjoin%
\definecolor{currentfill}{rgb}{0.483854,0.622050,0.974808}%
\pgfsetfillcolor{currentfill}%
\pgfsetlinewidth{0.000000pt}%
\definecolor{currentstroke}{rgb}{0.000000,0.000000,0.000000}%
\pgfsetstrokecolor{currentstroke}%
\pgfsetdash{}{0pt}%
\pgfpathmoveto{\pgfqpoint{4.240499in}{2.765688in}}%
\pgfpathlineto{\pgfqpoint{4.251436in}{2.718073in}}%
\pgfpathlineto{\pgfqpoint{4.286311in}{2.715747in}}%
\pgfpathlineto{\pgfqpoint{4.321159in}{2.712143in}}%
\pgfpathlineto{\pgfqpoint{4.310182in}{2.758621in}}%
\pgfpathlineto{\pgfqpoint{4.275354in}{2.762852in}}%
\pgfpathlineto{\pgfqpoint{4.240499in}{2.765688in}}%
\pgfpathclose%
\pgfusepath{fill}%
\end{pgfscope}%
\begin{pgfscope}%
\pgfpathrectangle{\pgfqpoint{1.020000in}{0.880000in}}{\pgfqpoint{6.160000in}{6.160000in}}%
\pgfusepath{clip}%
\pgfsetbuttcap%
\pgfsetroundjoin%
\definecolor{currentfill}{rgb}{0.576051,0.708780,0.997755}%
\pgfsetfillcolor{currentfill}%
\pgfsetlinewidth{0.000000pt}%
\definecolor{currentstroke}{rgb}{0.000000,0.000000,0.000000}%
\pgfsetstrokecolor{currentstroke}%
\pgfsetdash{}{0pt}%
\pgfpathmoveto{\pgfqpoint{3.686851in}{2.923922in}}%
\pgfpathlineto{\pgfqpoint{3.697411in}{2.877929in}}%
\pgfpathlineto{\pgfqpoint{3.707984in}{2.832129in}}%
\pgfpathlineto{\pgfqpoint{3.742979in}{2.852135in}}%
\pgfpathlineto{\pgfqpoint{3.777968in}{2.871126in}}%
\pgfpathlineto{\pgfqpoint{3.767342in}{2.919230in}}%
\pgfpathlineto{\pgfqpoint{3.756729in}{2.967533in}}%
\pgfpathlineto{\pgfqpoint{3.721793in}{2.946306in}}%
\pgfpathlineto{\pgfqpoint{3.686851in}{2.923922in}}%
\pgfpathclose%
\pgfusepath{fill}%
\end{pgfscope}%
\begin{pgfscope}%
\pgfpathrectangle{\pgfqpoint{1.020000in}{0.880000in}}{\pgfqpoint{6.160000in}{6.160000in}}%
\pgfusepath{clip}%
\pgfsetbuttcap%
\pgfsetroundjoin%
\definecolor{currentfill}{rgb}{0.462354,0.599830,0.965857}%
\pgfsetfillcolor{currentfill}%
\pgfsetlinewidth{0.000000pt}%
\definecolor{currentstroke}{rgb}{0.000000,0.000000,0.000000}%
\pgfsetstrokecolor{currentstroke}%
\pgfsetdash{}{0pt}%
\pgfpathmoveto{\pgfqpoint{3.246683in}{2.700229in}}%
\pgfpathlineto{\pgfqpoint{3.256809in}{2.669515in}}%
\pgfpathlineto{\pgfqpoint{3.266951in}{2.638761in}}%
\pgfpathlineto{\pgfqpoint{3.302012in}{2.659932in}}%
\pgfpathlineto{\pgfqpoint{3.337055in}{2.682009in}}%
\pgfpathlineto{\pgfqpoint{3.326840in}{2.715032in}}%
\pgfpathlineto{\pgfqpoint{3.316641in}{2.747977in}}%
\pgfpathlineto{\pgfqpoint{3.281671in}{2.723595in}}%
\pgfpathlineto{\pgfqpoint{3.246683in}{2.700229in}}%
\pgfpathclose%
\pgfusepath{fill}%
\end{pgfscope}%
\begin{pgfscope}%
\pgfpathrectangle{\pgfqpoint{1.020000in}{0.880000in}}{\pgfqpoint{6.160000in}{6.160000in}}%
\pgfusepath{clip}%
\pgfsetbuttcap%
\pgfsetroundjoin%
\definecolor{currentfill}{rgb}{0.394042,0.522413,0.924916}%
\pgfsetfillcolor{currentfill}%
\pgfsetlinewidth{0.000000pt}%
\definecolor{currentstroke}{rgb}{0.000000,0.000000,0.000000}%
\pgfsetstrokecolor{currentstroke}%
\pgfsetdash{}{0pt}%
\pgfpathmoveto{\pgfqpoint{2.876052in}{2.578035in}}%
\pgfpathlineto{\pgfqpoint{2.885783in}{2.554639in}}%
\pgfpathlineto{\pgfqpoint{2.895531in}{2.531270in}}%
\pgfpathlineto{\pgfqpoint{2.930768in}{2.542926in}}%
\pgfpathlineto{\pgfqpoint{2.965975in}{2.555661in}}%
\pgfpathlineto{\pgfqpoint{2.956158in}{2.579977in}}%
\pgfpathlineto{\pgfqpoint{2.946358in}{2.604279in}}%
\pgfpathlineto{\pgfqpoint{2.911220in}{2.590563in}}%
\pgfpathlineto{\pgfqpoint{2.876052in}{2.578035in}}%
\pgfpathclose%
\pgfusepath{fill}%
\end{pgfscope}%
\begin{pgfscope}%
\pgfpathrectangle{\pgfqpoint{1.020000in}{0.880000in}}{\pgfqpoint{6.160000in}{6.160000in}}%
\pgfusepath{clip}%
\pgfsetbuttcap%
\pgfsetroundjoin%
\definecolor{currentfill}{rgb}{0.570616,0.704109,0.997195}%
\pgfsetfillcolor{currentfill}%
\pgfsetlinewidth{0.000000pt}%
\definecolor{currentstroke}{rgb}{0.000000,0.000000,0.000000}%
\pgfsetstrokecolor{currentstroke}%
\pgfsetdash{}{0pt}%
\pgfpathmoveto{\pgfqpoint{3.917838in}{2.933431in}}%
\pgfpathlineto{\pgfqpoint{3.928569in}{2.882293in}}%
\pgfpathlineto{\pgfqpoint{3.939311in}{2.831871in}}%
\pgfpathlineto{\pgfqpoint{3.974294in}{2.842120in}}%
\pgfpathlineto{\pgfqpoint{4.009262in}{2.850710in}}%
\pgfpathlineto{\pgfqpoint{3.998479in}{2.902131in}}%
\pgfpathlineto{\pgfqpoint{3.987708in}{2.954286in}}%
\pgfpathlineto{\pgfqpoint{3.952780in}{2.944821in}}%
\pgfpathlineto{\pgfqpoint{3.917838in}{2.933431in}}%
\pgfpathclose%
\pgfusepath{fill}%
\end{pgfscope}%
\begin{pgfscope}%
\pgfpathrectangle{\pgfqpoint{1.020000in}{0.880000in}}{\pgfqpoint{6.160000in}{6.160000in}}%
\pgfusepath{clip}%
\pgfsetbuttcap%
\pgfsetroundjoin%
\definecolor{currentfill}{rgb}{0.532568,0.669801,0.990393}%
\pgfsetfillcolor{currentfill}%
\pgfsetlinewidth{0.000000pt}%
\definecolor{currentstroke}{rgb}{0.000000,0.000000,0.000000}%
\pgfsetstrokecolor{currentstroke}%
\pgfsetdash{}{0pt}%
\pgfpathmoveto{\pgfqpoint{4.079148in}{2.862635in}}%
\pgfpathlineto{\pgfqpoint{4.089982in}{2.811738in}}%
\pgfpathlineto{\pgfqpoint{4.100829in}{2.762015in}}%
\pgfpathlineto{\pgfqpoint{4.135781in}{2.765246in}}%
\pgfpathlineto{\pgfqpoint{4.170711in}{2.766924in}}%
\pgfpathlineto{\pgfqpoint{4.159826in}{2.816525in}}%
\pgfpathlineto{\pgfqpoint{4.148955in}{2.867297in}}%
\pgfpathlineto{\pgfqpoint{4.114062in}{2.865882in}}%
\pgfpathlineto{\pgfqpoint{4.079148in}{2.862635in}}%
\pgfpathclose%
\pgfusepath{fill}%
\end{pgfscope}%
\begin{pgfscope}%
\pgfpathrectangle{\pgfqpoint{1.020000in}{0.880000in}}{\pgfqpoint{6.160000in}{6.160000in}}%
\pgfusepath{clip}%
\pgfsetbuttcap%
\pgfsetroundjoin%
\definecolor{currentfill}{rgb}{0.441123,0.576532,0.954545}%
\pgfsetfillcolor{currentfill}%
\pgfsetlinewidth{0.000000pt}%
\definecolor{currentstroke}{rgb}{0.000000,0.000000,0.000000}%
\pgfsetstrokecolor{currentstroke}%
\pgfsetdash{}{0pt}%
\pgfpathmoveto{\pgfqpoint{3.176650in}{2.656848in}}%
\pgfpathlineto{\pgfqpoint{3.186701in}{2.628131in}}%
\pgfpathlineto{\pgfqpoint{3.196769in}{2.599406in}}%
\pgfpathlineto{\pgfqpoint{3.231870in}{2.618567in}}%
\pgfpathlineto{\pgfqpoint{3.266951in}{2.638761in}}%
\pgfpathlineto{\pgfqpoint{3.256809in}{2.669515in}}%
\pgfpathlineto{\pgfqpoint{3.246683in}{2.700229in}}%
\pgfpathlineto{\pgfqpoint{3.211677in}{2.677960in}}%
\pgfpathlineto{\pgfqpoint{3.176650in}{2.656848in}}%
\pgfpathclose%
\pgfusepath{fill}%
\end{pgfscope}%
\begin{pgfscope}%
\pgfpathrectangle{\pgfqpoint{1.020000in}{0.880000in}}{\pgfqpoint{6.160000in}{6.160000in}}%
\pgfusepath{clip}%
\pgfsetbuttcap%
\pgfsetroundjoin%
\definecolor{currentfill}{rgb}{0.554312,0.690097,0.995516}%
\pgfsetfillcolor{currentfill}%
\pgfsetlinewidth{0.000000pt}%
\definecolor{currentstroke}{rgb}{0.000000,0.000000,0.000000}%
\pgfsetstrokecolor{currentstroke}%
\pgfsetdash{}{0pt}%
\pgfpathmoveto{\pgfqpoint{3.616954in}{2.876542in}}%
\pgfpathlineto{\pgfqpoint{3.627455in}{2.833091in}}%
\pgfpathlineto{\pgfqpoint{3.637970in}{2.789829in}}%
\pgfpathlineto{\pgfqpoint{3.672980in}{2.811298in}}%
\pgfpathlineto{\pgfqpoint{3.707984in}{2.832129in}}%
\pgfpathlineto{\pgfqpoint{3.697411in}{2.877929in}}%
\pgfpathlineto{\pgfqpoint{3.686851in}{2.923922in}}%
\pgfpathlineto{\pgfqpoint{3.651905in}{2.900595in}}%
\pgfpathlineto{\pgfqpoint{3.616954in}{2.876542in}}%
\pgfpathclose%
\pgfusepath{fill}%
\end{pgfscope}%
\begin{pgfscope}%
\pgfpathrectangle{\pgfqpoint{1.020000in}{0.880000in}}{\pgfqpoint{6.160000in}{6.160000in}}%
\pgfusepath{clip}%
\pgfsetbuttcap%
\pgfsetroundjoin%
\definecolor{currentfill}{rgb}{0.494638,0.633022,0.978983}%
\pgfsetfillcolor{currentfill}%
\pgfsetlinewidth{0.000000pt}%
\definecolor{currentstroke}{rgb}{0.000000,0.000000,0.000000}%
\pgfsetstrokecolor{currentstroke}%
\pgfsetdash{}{0pt}%
\pgfpathmoveto{\pgfqpoint{4.170711in}{2.766924in}}%
\pgfpathlineto{\pgfqpoint{4.181609in}{2.718668in}}%
\pgfpathlineto{\pgfqpoint{4.216535in}{2.719062in}}%
\pgfpathlineto{\pgfqpoint{4.251436in}{2.718073in}}%
\pgfpathlineto{\pgfqpoint{4.240499in}{2.765688in}}%
\pgfpathlineto{\pgfqpoint{4.205617in}{2.767062in}}%
\pgfpathlineto{\pgfqpoint{4.170711in}{2.766924in}}%
\pgfpathclose%
\pgfusepath{fill}%
\end{pgfscope}%
\begin{pgfscope}%
\pgfpathrectangle{\pgfqpoint{1.020000in}{0.880000in}}{\pgfqpoint{6.160000in}{6.160000in}}%
\pgfusepath{clip}%
\pgfsetbuttcap%
\pgfsetroundjoin%
\definecolor{currentfill}{rgb}{0.532568,0.669801,0.990393}%
\pgfsetfillcolor{currentfill}%
\pgfsetlinewidth{0.000000pt}%
\definecolor{currentstroke}{rgb}{0.000000,0.000000,0.000000}%
\pgfsetstrokecolor{currentstroke}%
\pgfsetdash{}{0pt}%
\pgfpathmoveto{\pgfqpoint{3.547034in}{2.827112in}}%
\pgfpathlineto{\pgfqpoint{3.557473in}{2.786330in}}%
\pgfpathlineto{\pgfqpoint{3.567925in}{2.745731in}}%
\pgfpathlineto{\pgfqpoint{3.602952in}{2.767913in}}%
\pgfpathlineto{\pgfqpoint{3.637970in}{2.789829in}}%
\pgfpathlineto{\pgfqpoint{3.627455in}{2.833091in}}%
\pgfpathlineto{\pgfqpoint{3.616954in}{2.876542in}}%
\pgfpathlineto{\pgfqpoint{3.581997in}{2.851977in}}%
\pgfpathlineto{\pgfqpoint{3.547034in}{2.827112in}}%
\pgfpathclose%
\pgfusepath{fill}%
\end{pgfscope}%
\begin{pgfscope}%
\pgfpathrectangle{\pgfqpoint{1.020000in}{0.880000in}}{\pgfqpoint{6.160000in}{6.160000in}}%
\pgfusepath{clip}%
\pgfsetbuttcap%
\pgfsetroundjoin%
\definecolor{currentfill}{rgb}{0.565182,0.699438,0.996635}%
\pgfsetfillcolor{currentfill}%
\pgfsetlinewidth{0.000000pt}%
\definecolor{currentstroke}{rgb}{0.000000,0.000000,0.000000}%
\pgfsetstrokecolor{currentstroke}%
\pgfsetdash{}{0pt}%
\pgfpathmoveto{\pgfqpoint{3.847921in}{2.905341in}}%
\pgfpathlineto{\pgfqpoint{3.858608in}{2.855724in}}%
\pgfpathlineto{\pgfqpoint{3.869306in}{2.806796in}}%
\pgfpathlineto{\pgfqpoint{3.904315in}{2.820059in}}%
\pgfpathlineto{\pgfqpoint{3.939311in}{2.831871in}}%
\pgfpathlineto{\pgfqpoint{3.928569in}{2.882293in}}%
\pgfpathlineto{\pgfqpoint{3.917838in}{2.933431in}}%
\pgfpathlineto{\pgfqpoint{3.882885in}{2.920228in}}%
\pgfpathlineto{\pgfqpoint{3.847921in}{2.905341in}}%
\pgfpathclose%
\pgfusepath{fill}%
\end{pgfscope}%
\begin{pgfscope}%
\pgfpathrectangle{\pgfqpoint{1.020000in}{0.880000in}}{\pgfqpoint{6.160000in}{6.160000in}}%
\pgfusepath{clip}%
\pgfsetbuttcap%
\pgfsetroundjoin%
\definecolor{currentfill}{rgb}{0.419991,0.552989,0.942630}%
\pgfsetfillcolor{currentfill}%
\pgfsetlinewidth{0.000000pt}%
\definecolor{currentstroke}{rgb}{0.000000,0.000000,0.000000}%
\pgfsetstrokecolor{currentstroke}%
\pgfsetdash{}{0pt}%
\pgfpathmoveto{\pgfqpoint{3.106528in}{2.618248in}}%
\pgfpathlineto{\pgfqpoint{3.116506in}{2.591269in}}%
\pgfpathlineto{\pgfqpoint{3.126500in}{2.564315in}}%
\pgfpathlineto{\pgfqpoint{3.161646in}{2.581315in}}%
\pgfpathlineto{\pgfqpoint{3.196769in}{2.599406in}}%
\pgfpathlineto{\pgfqpoint{3.186701in}{2.628131in}}%
\pgfpathlineto{\pgfqpoint{3.176650in}{2.656848in}}%
\pgfpathlineto{\pgfqpoint{3.141601in}{2.636937in}}%
\pgfpathlineto{\pgfqpoint{3.106528in}{2.618248in}}%
\pgfpathclose%
\pgfusepath{fill}%
\end{pgfscope}%
\begin{pgfscope}%
\pgfpathrectangle{\pgfqpoint{1.020000in}{0.880000in}}{\pgfqpoint{6.160000in}{6.160000in}}%
\pgfusepath{clip}%
\pgfsetbuttcap%
\pgfsetroundjoin%
\definecolor{currentfill}{rgb}{0.532568,0.669801,0.990393}%
\pgfsetfillcolor{currentfill}%
\pgfsetlinewidth{0.000000pt}%
\definecolor{currentstroke}{rgb}{0.000000,0.000000,0.000000}%
\pgfsetstrokecolor{currentstroke}%
\pgfsetdash{}{0pt}%
\pgfpathmoveto{\pgfqpoint{4.009262in}{2.850710in}}%
\pgfpathlineto{\pgfqpoint{4.020058in}{2.800250in}}%
\pgfpathlineto{\pgfqpoint{4.030865in}{2.750950in}}%
\pgfpathlineto{\pgfqpoint{4.065856in}{2.757241in}}%
\pgfpathlineto{\pgfqpoint{4.100829in}{2.762015in}}%
\pgfpathlineto{\pgfqpoint{4.089982in}{2.811738in}}%
\pgfpathlineto{\pgfqpoint{4.079148in}{2.862635in}}%
\pgfpathlineto{\pgfqpoint{4.044214in}{2.857567in}}%
\pgfpathlineto{\pgfqpoint{4.009262in}{2.850710in}}%
\pgfpathclose%
\pgfusepath{fill}%
\end{pgfscope}%
\begin{pgfscope}%
\pgfpathrectangle{\pgfqpoint{1.020000in}{0.880000in}}{\pgfqpoint{6.160000in}{6.160000in}}%
\pgfusepath{clip}%
\pgfsetbuttcap%
\pgfsetroundjoin%
\definecolor{currentfill}{rgb}{0.505423,0.643995,0.983157}%
\pgfsetfillcolor{currentfill}%
\pgfsetlinewidth{0.000000pt}%
\definecolor{currentstroke}{rgb}{0.000000,0.000000,0.000000}%
\pgfsetstrokecolor{currentstroke}%
\pgfsetdash{}{0pt}%
\pgfpathmoveto{\pgfqpoint{3.477085in}{2.777274in}}%
\pgfpathlineto{\pgfqpoint{3.487457in}{2.739185in}}%
\pgfpathlineto{\pgfqpoint{3.497843in}{2.701274in}}%
\pgfpathlineto{\pgfqpoint{3.532889in}{2.723463in}}%
\pgfpathlineto{\pgfqpoint{3.567925in}{2.745731in}}%
\pgfpathlineto{\pgfqpoint{3.557473in}{2.786330in}}%
\pgfpathlineto{\pgfqpoint{3.547034in}{2.827112in}}%
\pgfpathlineto{\pgfqpoint{3.512064in}{2.802147in}}%
\pgfpathlineto{\pgfqpoint{3.477085in}{2.777274in}}%
\pgfpathclose%
\pgfusepath{fill}%
\end{pgfscope}%
\begin{pgfscope}%
\pgfpathrectangle{\pgfqpoint{1.020000in}{0.880000in}}{\pgfqpoint{6.160000in}{6.160000in}}%
\pgfusepath{clip}%
\pgfsetbuttcap%
\pgfsetroundjoin%
\definecolor{currentfill}{rgb}{0.548876,0.685104,0.994379}%
\pgfsetfillcolor{currentfill}%
\pgfsetlinewidth{0.000000pt}%
\definecolor{currentstroke}{rgb}{0.000000,0.000000,0.000000}%
\pgfsetstrokecolor{currentstroke}%
\pgfsetdash{}{0pt}%
\pgfpathmoveto{\pgfqpoint{3.777968in}{2.871126in}}%
\pgfpathlineto{\pgfqpoint{3.788606in}{2.823457in}}%
\pgfpathlineto{\pgfqpoint{3.799256in}{2.776442in}}%
\pgfpathlineto{\pgfqpoint{3.834286in}{2.792210in}}%
\pgfpathlineto{\pgfqpoint{3.869306in}{2.806796in}}%
\pgfpathlineto{\pgfqpoint{3.858608in}{2.855724in}}%
\pgfpathlineto{\pgfqpoint{3.847921in}{2.905341in}}%
\pgfpathlineto{\pgfqpoint{3.812949in}{2.888919in}}%
\pgfpathlineto{\pgfqpoint{3.777968in}{2.871126in}}%
\pgfpathclose%
\pgfusepath{fill}%
\end{pgfscope}%
\begin{pgfscope}%
\pgfpathrectangle{\pgfqpoint{1.020000in}{0.880000in}}{\pgfqpoint{6.160000in}{6.160000in}}%
\pgfusepath{clip}%
\pgfsetbuttcap%
\pgfsetroundjoin%
\definecolor{currentfill}{rgb}{0.404421,0.534643,0.932002}%
\pgfsetfillcolor{currentfill}%
\pgfsetlinewidth{0.000000pt}%
\definecolor{currentstroke}{rgb}{0.000000,0.000000,0.000000}%
\pgfsetstrokecolor{currentstroke}%
\pgfsetdash{}{0pt}%
\pgfpathmoveto{\pgfqpoint{3.036306in}{2.584556in}}%
\pgfpathlineto{\pgfqpoint{3.046211in}{2.559053in}}%
\pgfpathlineto{\pgfqpoint{3.056132in}{2.533601in}}%
\pgfpathlineto{\pgfqpoint{3.091329in}{2.548413in}}%
\pgfpathlineto{\pgfqpoint{3.126500in}{2.564315in}}%
\pgfpathlineto{\pgfqpoint{3.116506in}{2.591269in}}%
\pgfpathlineto{\pgfqpoint{3.106528in}{2.618248in}}%
\pgfpathlineto{\pgfqpoint{3.071430in}{2.600791in}}%
\pgfpathlineto{\pgfqpoint{3.036306in}{2.584556in}}%
\pgfpathclose%
\pgfusepath{fill}%
\end{pgfscope}%
\begin{pgfscope}%
\pgfpathrectangle{\pgfqpoint{1.020000in}{0.880000in}}{\pgfqpoint{6.160000in}{6.160000in}}%
\pgfusepath{clip}%
\pgfsetbuttcap%
\pgfsetroundjoin%
\definecolor{currentfill}{rgb}{0.483854,0.622050,0.974808}%
\pgfsetfillcolor{currentfill}%
\pgfsetlinewidth{0.000000pt}%
\definecolor{currentstroke}{rgb}{0.000000,0.000000,0.000000}%
\pgfsetstrokecolor{currentstroke}%
\pgfsetdash{}{0pt}%
\pgfpathmoveto{\pgfqpoint{3.407096in}{2.728496in}}%
\pgfpathlineto{\pgfqpoint{3.417398in}{2.693034in}}%
\pgfpathlineto{\pgfqpoint{3.427715in}{2.657745in}}%
\pgfpathlineto{\pgfqpoint{3.462786in}{2.679320in}}%
\pgfpathlineto{\pgfqpoint{3.497843in}{2.701274in}}%
\pgfpathlineto{\pgfqpoint{3.487457in}{2.739185in}}%
\pgfpathlineto{\pgfqpoint{3.477085in}{2.777274in}}%
\pgfpathlineto{\pgfqpoint{3.442096in}{2.752669in}}%
\pgfpathlineto{\pgfqpoint{3.407096in}{2.728496in}}%
\pgfpathclose%
\pgfusepath{fill}%
\end{pgfscope}%
\begin{pgfscope}%
\pgfpathrectangle{\pgfqpoint{1.020000in}{0.880000in}}{\pgfqpoint{6.160000in}{6.160000in}}%
\pgfusepath{clip}%
\pgfsetbuttcap%
\pgfsetroundjoin%
\definecolor{currentfill}{rgb}{0.494638,0.633022,0.978983}%
\pgfsetfillcolor{currentfill}%
\pgfsetlinewidth{0.000000pt}%
\definecolor{currentstroke}{rgb}{0.000000,0.000000,0.000000}%
\pgfsetstrokecolor{currentstroke}%
\pgfsetdash{}{0pt}%
\pgfpathmoveto{\pgfqpoint{4.100829in}{2.762015in}}%
\pgfpathlineto{\pgfqpoint{4.111689in}{2.713640in}}%
\pgfpathlineto{\pgfqpoint{4.146660in}{2.716864in}}%
\pgfpathlineto{\pgfqpoint{4.181609in}{2.718668in}}%
\pgfpathlineto{\pgfqpoint{4.170711in}{2.766924in}}%
\pgfpathlineto{\pgfqpoint{4.135781in}{2.765246in}}%
\pgfpathlineto{\pgfqpoint{4.100829in}{2.762015in}}%
\pgfpathclose%
\pgfusepath{fill}%
\end{pgfscope}%
\begin{pgfscope}%
\pgfpathrectangle{\pgfqpoint{1.020000in}{0.880000in}}{\pgfqpoint{6.160000in}{6.160000in}}%
\pgfusepath{clip}%
\pgfsetbuttcap%
\pgfsetroundjoin%
\definecolor{currentfill}{rgb}{0.457046,0.594006,0.963029}%
\pgfsetfillcolor{currentfill}%
\pgfsetlinewidth{0.000000pt}%
\definecolor{currentstroke}{rgb}{0.000000,0.000000,0.000000}%
\pgfsetstrokecolor{currentstroke}%
\pgfsetdash{}{0pt}%
\pgfpathmoveto{\pgfqpoint{3.337055in}{2.682009in}}%
\pgfpathlineto{\pgfqpoint{3.347286in}{2.649032in}}%
\pgfpathlineto{\pgfqpoint{3.357532in}{2.616224in}}%
\pgfpathlineto{\pgfqpoint{3.392631in}{2.636675in}}%
\pgfpathlineto{\pgfqpoint{3.427715in}{2.657745in}}%
\pgfpathlineto{\pgfqpoint{3.417398in}{2.693034in}}%
\pgfpathlineto{\pgfqpoint{3.407096in}{2.728496in}}%
\pgfpathlineto{\pgfqpoint{3.372083in}{2.704900in}}%
\pgfpathlineto{\pgfqpoint{3.337055in}{2.682009in}}%
\pgfpathclose%
\pgfusepath{fill}%
\end{pgfscope}%
\begin{pgfscope}%
\pgfpathrectangle{\pgfqpoint{1.020000in}{0.880000in}}{\pgfqpoint{6.160000in}{6.160000in}}%
\pgfusepath{clip}%
\pgfsetbuttcap%
\pgfsetroundjoin%
\definecolor{currentfill}{rgb}{0.532568,0.669801,0.990393}%
\pgfsetfillcolor{currentfill}%
\pgfsetlinewidth{0.000000pt}%
\definecolor{currentstroke}{rgb}{0.000000,0.000000,0.000000}%
\pgfsetstrokecolor{currentstroke}%
\pgfsetdash{}{0pt}%
\pgfpathmoveto{\pgfqpoint{3.707984in}{2.832129in}}%
\pgfpathlineto{\pgfqpoint{3.718569in}{2.786742in}}%
\pgfpathlineto{\pgfqpoint{3.729165in}{2.741967in}}%
\pgfpathlineto{\pgfqpoint{3.764215in}{2.759642in}}%
\pgfpathlineto{\pgfqpoint{3.799256in}{2.776442in}}%
\pgfpathlineto{\pgfqpoint{3.788606in}{2.823457in}}%
\pgfpathlineto{\pgfqpoint{3.777968in}{2.871126in}}%
\pgfpathlineto{\pgfqpoint{3.742979in}{2.852135in}}%
\pgfpathlineto{\pgfqpoint{3.707984in}{2.832129in}}%
\pgfpathclose%
\pgfusepath{fill}%
\end{pgfscope}%
\begin{pgfscope}%
\pgfpathrectangle{\pgfqpoint{1.020000in}{0.880000in}}{\pgfqpoint{6.160000in}{6.160000in}}%
\pgfusepath{clip}%
\pgfsetbuttcap%
\pgfsetroundjoin%
\definecolor{currentfill}{rgb}{0.527132,0.664700,0.989065}%
\pgfsetfillcolor{currentfill}%
\pgfsetlinewidth{0.000000pt}%
\definecolor{currentstroke}{rgb}{0.000000,0.000000,0.000000}%
\pgfsetstrokecolor{currentstroke}%
\pgfsetdash{}{0pt}%
\pgfpathmoveto{\pgfqpoint{3.939311in}{2.831871in}}%
\pgfpathlineto{\pgfqpoint{3.950065in}{2.782384in}}%
\pgfpathlineto{\pgfqpoint{3.960831in}{2.734025in}}%
\pgfpathlineto{\pgfqpoint{3.995856in}{2.743190in}}%
\pgfpathlineto{\pgfqpoint{4.030865in}{2.750950in}}%
\pgfpathlineto{\pgfqpoint{4.020058in}{2.800250in}}%
\pgfpathlineto{\pgfqpoint{4.009262in}{2.850710in}}%
\pgfpathlineto{\pgfqpoint{3.974294in}{2.842120in}}%
\pgfpathlineto{\pgfqpoint{3.939311in}{2.831871in}}%
\pgfpathclose%
\pgfusepath{fill}%
\end{pgfscope}%
\begin{pgfscope}%
\pgfpathrectangle{\pgfqpoint{1.020000in}{0.880000in}}{\pgfqpoint{6.160000in}{6.160000in}}%
\pgfusepath{clip}%
\pgfsetbuttcap%
\pgfsetroundjoin%
\definecolor{currentfill}{rgb}{0.394042,0.522413,0.924916}%
\pgfsetfillcolor{currentfill}%
\pgfsetlinewidth{0.000000pt}%
\definecolor{currentstroke}{rgb}{0.000000,0.000000,0.000000}%
\pgfsetstrokecolor{currentstroke}%
\pgfsetdash{}{0pt}%
\pgfpathmoveto{\pgfqpoint{2.965975in}{2.555661in}}%
\pgfpathlineto{\pgfqpoint{2.975809in}{2.531376in}}%
\pgfpathlineto{\pgfqpoint{2.985659in}{2.507166in}}%
\pgfpathlineto{\pgfqpoint{3.020909in}{2.519862in}}%
\pgfpathlineto{\pgfqpoint{3.056132in}{2.533601in}}%
\pgfpathlineto{\pgfqpoint{3.046211in}{2.559053in}}%
\pgfpathlineto{\pgfqpoint{3.036306in}{2.584556in}}%
\pgfpathlineto{\pgfqpoint{3.001155in}{2.569523in}}%
\pgfpathlineto{\pgfqpoint{2.965975in}{2.555661in}}%
\pgfpathclose%
\pgfusepath{fill}%
\end{pgfscope}%
\begin{pgfscope}%
\pgfpathrectangle{\pgfqpoint{1.020000in}{0.880000in}}{\pgfqpoint{6.160000in}{6.160000in}}%
\pgfusepath{clip}%
\pgfsetbuttcap%
\pgfsetroundjoin%
\definecolor{currentfill}{rgb}{0.435815,0.570707,0.951717}%
\pgfsetfillcolor{currentfill}%
\pgfsetlinewidth{0.000000pt}%
\definecolor{currentstroke}{rgb}{0.000000,0.000000,0.000000}%
\pgfsetstrokecolor{currentstroke}%
\pgfsetdash{}{0pt}%
\pgfpathmoveto{\pgfqpoint{3.266951in}{2.638761in}}%
\pgfpathlineto{\pgfqpoint{3.277108in}{2.608070in}}%
\pgfpathlineto{\pgfqpoint{3.287281in}{2.577543in}}%
\pgfpathlineto{\pgfqpoint{3.322416in}{2.596486in}}%
\pgfpathlineto{\pgfqpoint{3.357532in}{2.616224in}}%
\pgfpathlineto{\pgfqpoint{3.347286in}{2.649032in}}%
\pgfpathlineto{\pgfqpoint{3.337055in}{2.682009in}}%
\pgfpathlineto{\pgfqpoint{3.302012in}{2.659932in}}%
\pgfpathlineto{\pgfqpoint{3.266951in}{2.638761in}}%
\pgfpathclose%
\pgfusepath{fill}%
\end{pgfscope}%
\begin{pgfscope}%
\pgfpathrectangle{\pgfqpoint{1.020000in}{0.880000in}}{\pgfqpoint{6.160000in}{6.160000in}}%
\pgfusepath{clip}%
\pgfsetbuttcap%
\pgfsetroundjoin%
\definecolor{currentfill}{rgb}{0.516260,0.654498,0.986407}%
\pgfsetfillcolor{currentfill}%
\pgfsetlinewidth{0.000000pt}%
\definecolor{currentstroke}{rgb}{0.000000,0.000000,0.000000}%
\pgfsetstrokecolor{currentstroke}%
\pgfsetdash{}{0pt}%
\pgfpathmoveto{\pgfqpoint{3.637970in}{2.789829in}}%
\pgfpathlineto{\pgfqpoint{3.648497in}{2.746953in}}%
\pgfpathlineto{\pgfqpoint{3.659037in}{2.704643in}}%
\pgfpathlineto{\pgfqpoint{3.694106in}{2.723579in}}%
\pgfpathlineto{\pgfqpoint{3.729165in}{2.741967in}}%
\pgfpathlineto{\pgfqpoint{3.718569in}{2.786742in}}%
\pgfpathlineto{\pgfqpoint{3.707984in}{2.832129in}}%
\pgfpathlineto{\pgfqpoint{3.672980in}{2.811298in}}%
\pgfpathlineto{\pgfqpoint{3.637970in}{2.789829in}}%
\pgfpathclose%
\pgfusepath{fill}%
\end{pgfscope}%
\begin{pgfscope}%
\pgfpathrectangle{\pgfqpoint{1.020000in}{0.880000in}}{\pgfqpoint{6.160000in}{6.160000in}}%
\pgfusepath{clip}%
\pgfsetbuttcap%
\pgfsetroundjoin%
\definecolor{currentfill}{rgb}{0.494638,0.633022,0.978983}%
\pgfsetfillcolor{currentfill}%
\pgfsetlinewidth{0.000000pt}%
\definecolor{currentstroke}{rgb}{0.000000,0.000000,0.000000}%
\pgfsetstrokecolor{currentstroke}%
\pgfsetdash{}{0pt}%
\pgfpathmoveto{\pgfqpoint{4.030865in}{2.750950in}}%
\pgfpathlineto{\pgfqpoint{4.041684in}{2.702980in}}%
\pgfpathlineto{\pgfqpoint{4.076696in}{2.709004in}}%
\pgfpathlineto{\pgfqpoint{4.111689in}{2.713640in}}%
\pgfpathlineto{\pgfqpoint{4.100829in}{2.762015in}}%
\pgfpathlineto{\pgfqpoint{4.065856in}{2.757241in}}%
\pgfpathlineto{\pgfqpoint{4.030865in}{2.750950in}}%
\pgfpathclose%
\pgfusepath{fill}%
\end{pgfscope}%
\begin{pgfscope}%
\pgfpathrectangle{\pgfqpoint{1.020000in}{0.880000in}}{\pgfqpoint{6.160000in}{6.160000in}}%
\pgfusepath{clip}%
\pgfsetbuttcap%
\pgfsetroundjoin%
\definecolor{currentfill}{rgb}{0.383662,0.510183,0.917831}%
\pgfsetfillcolor{currentfill}%
\pgfsetlinewidth{0.000000pt}%
\definecolor{currentstroke}{rgb}{0.000000,0.000000,0.000000}%
\pgfsetstrokecolor{currentstroke}%
\pgfsetdash{}{0pt}%
\pgfpathmoveto{\pgfqpoint{2.895531in}{2.531270in}}%
\pgfpathlineto{\pgfqpoint{2.905295in}{2.507962in}}%
\pgfpathlineto{\pgfqpoint{2.915074in}{2.484750in}}%
\pgfpathlineto{\pgfqpoint{2.950381in}{2.495477in}}%
\pgfpathlineto{\pgfqpoint{2.985659in}{2.507166in}}%
\pgfpathlineto{\pgfqpoint{2.975809in}{2.531376in}}%
\pgfpathlineto{\pgfqpoint{2.965975in}{2.555661in}}%
\pgfpathlineto{\pgfqpoint{2.930768in}{2.542926in}}%
\pgfpathlineto{\pgfqpoint{2.895531in}{2.531270in}}%
\pgfpathclose%
\pgfusepath{fill}%
\end{pgfscope}%
\begin{pgfscope}%
\pgfpathrectangle{\pgfqpoint{1.020000in}{0.880000in}}{\pgfqpoint{6.160000in}{6.160000in}}%
\pgfusepath{clip}%
\pgfsetbuttcap%
\pgfsetroundjoin%
\definecolor{currentfill}{rgb}{0.516260,0.654498,0.986407}%
\pgfsetfillcolor{currentfill}%
\pgfsetlinewidth{0.000000pt}%
\definecolor{currentstroke}{rgb}{0.000000,0.000000,0.000000}%
\pgfsetstrokecolor{currentstroke}%
\pgfsetdash{}{0pt}%
\pgfpathmoveto{\pgfqpoint{3.869306in}{2.806796in}}%
\pgfpathlineto{\pgfqpoint{3.880015in}{2.758765in}}%
\pgfpathlineto{\pgfqpoint{3.890736in}{2.711814in}}%
\pgfpathlineto{\pgfqpoint{3.925790in}{2.723534in}}%
\pgfpathlineto{\pgfqpoint{3.960831in}{2.734025in}}%
\pgfpathlineto{\pgfqpoint{3.950065in}{2.782384in}}%
\pgfpathlineto{\pgfqpoint{3.939311in}{2.831871in}}%
\pgfpathlineto{\pgfqpoint{3.904315in}{2.820059in}}%
\pgfpathlineto{\pgfqpoint{3.869306in}{2.806796in}}%
\pgfpathclose%
\pgfusepath{fill}%
\end{pgfscope}%
\begin{pgfscope}%
\pgfpathrectangle{\pgfqpoint{1.020000in}{0.880000in}}{\pgfqpoint{6.160000in}{6.160000in}}%
\pgfusepath{clip}%
\pgfsetbuttcap%
\pgfsetroundjoin%
\definecolor{currentfill}{rgb}{0.419991,0.552989,0.942630}%
\pgfsetfillcolor{currentfill}%
\pgfsetlinewidth{0.000000pt}%
\definecolor{currentstroke}{rgb}{0.000000,0.000000,0.000000}%
\pgfsetstrokecolor{currentstroke}%
\pgfsetdash{}{0pt}%
\pgfpathmoveto{\pgfqpoint{3.196769in}{2.599406in}}%
\pgfpathlineto{\pgfqpoint{3.206853in}{2.570762in}}%
\pgfpathlineto{\pgfqpoint{3.216952in}{2.542278in}}%
\pgfpathlineto{\pgfqpoint{3.252127in}{2.559458in}}%
\pgfpathlineto{\pgfqpoint{3.287281in}{2.577543in}}%
\pgfpathlineto{\pgfqpoint{3.277108in}{2.608070in}}%
\pgfpathlineto{\pgfqpoint{3.266951in}{2.638761in}}%
\pgfpathlineto{\pgfqpoint{3.231870in}{2.618567in}}%
\pgfpathlineto{\pgfqpoint{3.196769in}{2.599406in}}%
\pgfpathclose%
\pgfusepath{fill}%
\end{pgfscope}%
\begin{pgfscope}%
\pgfpathrectangle{\pgfqpoint{1.020000in}{0.880000in}}{\pgfqpoint{6.160000in}{6.160000in}}%
\pgfusepath{clip}%
\pgfsetbuttcap%
\pgfsetroundjoin%
\definecolor{currentfill}{rgb}{0.494638,0.633022,0.978983}%
\pgfsetfillcolor{currentfill}%
\pgfsetlinewidth{0.000000pt}%
\definecolor{currentstroke}{rgb}{0.000000,0.000000,0.000000}%
\pgfsetstrokecolor{currentstroke}%
\pgfsetdash{}{0pt}%
\pgfpathmoveto{\pgfqpoint{3.567925in}{2.745731in}}%
\pgfpathlineto{\pgfqpoint{3.578390in}{2.705490in}}%
\pgfpathlineto{\pgfqpoint{3.588868in}{2.665768in}}%
\pgfpathlineto{\pgfqpoint{3.623957in}{2.685320in}}%
\pgfpathlineto{\pgfqpoint{3.659037in}{2.704643in}}%
\pgfpathlineto{\pgfqpoint{3.648497in}{2.746953in}}%
\pgfpathlineto{\pgfqpoint{3.637970in}{2.789829in}}%
\pgfpathlineto{\pgfqpoint{3.602952in}{2.767913in}}%
\pgfpathlineto{\pgfqpoint{3.567925in}{2.745731in}}%
\pgfpathclose%
\pgfusepath{fill}%
\end{pgfscope}%
\begin{pgfscope}%
\pgfpathrectangle{\pgfqpoint{1.020000in}{0.880000in}}{\pgfqpoint{6.160000in}{6.160000in}}%
\pgfusepath{clip}%
\pgfsetbuttcap%
\pgfsetroundjoin%
\definecolor{currentfill}{rgb}{0.473070,0.611077,0.970634}%
\pgfsetfillcolor{currentfill}%
\pgfsetlinewidth{0.000000pt}%
\definecolor{currentstroke}{rgb}{0.000000,0.000000,0.000000}%
\pgfsetstrokecolor{currentstroke}%
\pgfsetdash{}{0pt}%
\pgfpathmoveto{\pgfqpoint{3.497843in}{2.701274in}}%
\pgfpathlineto{\pgfqpoint{3.508243in}{2.663693in}}%
\pgfpathlineto{\pgfqpoint{3.518655in}{2.626583in}}%
\pgfpathlineto{\pgfqpoint{3.553767in}{2.646141in}}%
\pgfpathlineto{\pgfqpoint{3.588868in}{2.665768in}}%
\pgfpathlineto{\pgfqpoint{3.578390in}{2.705490in}}%
\pgfpathlineto{\pgfqpoint{3.567925in}{2.745731in}}%
\pgfpathlineto{\pgfqpoint{3.532889in}{2.723463in}}%
\pgfpathlineto{\pgfqpoint{3.497843in}{2.701274in}}%
\pgfpathclose%
\pgfusepath{fill}%
\end{pgfscope}%
\begin{pgfscope}%
\pgfpathrectangle{\pgfqpoint{1.020000in}{0.880000in}}{\pgfqpoint{6.160000in}{6.160000in}}%
\pgfusepath{clip}%
\pgfsetbuttcap%
\pgfsetroundjoin%
\definecolor{currentfill}{rgb}{0.505423,0.643995,0.983157}%
\pgfsetfillcolor{currentfill}%
\pgfsetlinewidth{0.000000pt}%
\definecolor{currentstroke}{rgb}{0.000000,0.000000,0.000000}%
\pgfsetstrokecolor{currentstroke}%
\pgfsetdash{}{0pt}%
\pgfpathmoveto{\pgfqpoint{3.799256in}{2.776442in}}%
\pgfpathlineto{\pgfqpoint{3.809916in}{2.730275in}}%
\pgfpathlineto{\pgfqpoint{3.820588in}{2.685127in}}%
\pgfpathlineto{\pgfqpoint{3.855668in}{2.698972in}}%
\pgfpathlineto{\pgfqpoint{3.890736in}{2.711814in}}%
\pgfpathlineto{\pgfqpoint{3.880015in}{2.758765in}}%
\pgfpathlineto{\pgfqpoint{3.869306in}{2.806796in}}%
\pgfpathlineto{\pgfqpoint{3.834286in}{2.792210in}}%
\pgfpathlineto{\pgfqpoint{3.799256in}{2.776442in}}%
\pgfpathclose%
\pgfusepath{fill}%
\end{pgfscope}%
\begin{pgfscope}%
\pgfpathrectangle{\pgfqpoint{1.020000in}{0.880000in}}{\pgfqpoint{6.160000in}{6.160000in}}%
\pgfusepath{clip}%
\pgfsetbuttcap%
\pgfsetroundjoin%
\definecolor{currentfill}{rgb}{0.404421,0.534643,0.932002}%
\pgfsetfillcolor{currentfill}%
\pgfsetlinewidth{0.000000pt}%
\definecolor{currentstroke}{rgb}{0.000000,0.000000,0.000000}%
\pgfsetstrokecolor{currentstroke}%
\pgfsetdash{}{0pt}%
\pgfpathmoveto{\pgfqpoint{3.126500in}{2.564315in}}%
\pgfpathlineto{\pgfqpoint{3.136510in}{2.537454in}}%
\pgfpathlineto{\pgfqpoint{3.146536in}{2.510752in}}%
\pgfpathlineto{\pgfqpoint{3.181756in}{2.526036in}}%
\pgfpathlineto{\pgfqpoint{3.216952in}{2.542278in}}%
\pgfpathlineto{\pgfqpoint{3.206853in}{2.570762in}}%
\pgfpathlineto{\pgfqpoint{3.196769in}{2.599406in}}%
\pgfpathlineto{\pgfqpoint{3.161646in}{2.581315in}}%
\pgfpathlineto{\pgfqpoint{3.126500in}{2.564315in}}%
\pgfpathclose%
\pgfusepath{fill}%
\end{pgfscope}%
\begin{pgfscope}%
\pgfpathrectangle{\pgfqpoint{1.020000in}{0.880000in}}{\pgfqpoint{6.160000in}{6.160000in}}%
\pgfusepath{clip}%
\pgfsetbuttcap%
\pgfsetroundjoin%
\definecolor{currentfill}{rgb}{0.489246,0.627536,0.976896}%
\pgfsetfillcolor{currentfill}%
\pgfsetlinewidth{0.000000pt}%
\definecolor{currentstroke}{rgb}{0.000000,0.000000,0.000000}%
\pgfsetstrokecolor{currentstroke}%
\pgfsetdash{}{0pt}%
\pgfpathmoveto{\pgfqpoint{3.960831in}{2.734025in}}%
\pgfpathlineto{\pgfqpoint{3.971608in}{2.686958in}}%
\pgfpathlineto{\pgfqpoint{4.006655in}{2.695612in}}%
\pgfpathlineto{\pgfqpoint{4.041684in}{2.702980in}}%
\pgfpathlineto{\pgfqpoint{4.030865in}{2.750950in}}%
\pgfpathlineto{\pgfqpoint{3.995856in}{2.743190in}}%
\pgfpathlineto{\pgfqpoint{3.960831in}{2.734025in}}%
\pgfpathclose%
\pgfusepath{fill}%
\end{pgfscope}%
\begin{pgfscope}%
\pgfpathrectangle{\pgfqpoint{1.020000in}{0.880000in}}{\pgfqpoint{6.160000in}{6.160000in}}%
\pgfusepath{clip}%
\pgfsetbuttcap%
\pgfsetroundjoin%
\definecolor{currentfill}{rgb}{0.451739,0.588181,0.960201}%
\pgfsetfillcolor{currentfill}%
\pgfsetlinewidth{0.000000pt}%
\definecolor{currentstroke}{rgb}{0.000000,0.000000,0.000000}%
\pgfsetstrokecolor{currentstroke}%
\pgfsetdash{}{0pt}%
\pgfpathmoveto{\pgfqpoint{3.427715in}{2.657745in}}%
\pgfpathlineto{\pgfqpoint{3.438046in}{2.622759in}}%
\pgfpathlineto{\pgfqpoint{3.448390in}{2.588199in}}%
\pgfpathlineto{\pgfqpoint{3.483529in}{2.607228in}}%
\pgfpathlineto{\pgfqpoint{3.518655in}{2.626583in}}%
\pgfpathlineto{\pgfqpoint{3.508243in}{2.663693in}}%
\pgfpathlineto{\pgfqpoint{3.497843in}{2.701274in}}%
\pgfpathlineto{\pgfqpoint{3.462786in}{2.679320in}}%
\pgfpathlineto{\pgfqpoint{3.427715in}{2.657745in}}%
\pgfpathclose%
\pgfusepath{fill}%
\end{pgfscope}%
\begin{pgfscope}%
\pgfpathrectangle{\pgfqpoint{1.020000in}{0.880000in}}{\pgfqpoint{6.160000in}{6.160000in}}%
\pgfusepath{clip}%
\pgfsetbuttcap%
\pgfsetroundjoin%
\definecolor{currentfill}{rgb}{0.489246,0.627536,0.976896}%
\pgfsetfillcolor{currentfill}%
\pgfsetlinewidth{0.000000pt}%
\definecolor{currentstroke}{rgb}{0.000000,0.000000,0.000000}%
\pgfsetstrokecolor{currentstroke}%
\pgfsetdash{}{0pt}%
\pgfpathmoveto{\pgfqpoint{3.729165in}{2.741967in}}%
\pgfpathlineto{\pgfqpoint{3.739773in}{2.697983in}}%
\pgfpathlineto{\pgfqpoint{3.750391in}{2.654947in}}%
\pgfpathlineto{\pgfqpoint{3.785496in}{2.670408in}}%
\pgfpathlineto{\pgfqpoint{3.820588in}{2.685127in}}%
\pgfpathlineto{\pgfqpoint{3.809916in}{2.730275in}}%
\pgfpathlineto{\pgfqpoint{3.799256in}{2.776442in}}%
\pgfpathlineto{\pgfqpoint{3.764215in}{2.759642in}}%
\pgfpathlineto{\pgfqpoint{3.729165in}{2.741967in}}%
\pgfpathclose%
\pgfusepath{fill}%
\end{pgfscope}%
\begin{pgfscope}%
\pgfpathrectangle{\pgfqpoint{1.020000in}{0.880000in}}{\pgfqpoint{6.160000in}{6.160000in}}%
\pgfusepath{clip}%
\pgfsetbuttcap%
\pgfsetroundjoin%
\definecolor{currentfill}{rgb}{0.388852,0.516298,0.921373}%
\pgfsetfillcolor{currentfill}%
\pgfsetlinewidth{0.000000pt}%
\definecolor{currentstroke}{rgb}{0.000000,0.000000,0.000000}%
\pgfsetstrokecolor{currentstroke}%
\pgfsetdash{}{0pt}%
\pgfpathmoveto{\pgfqpoint{3.056132in}{2.533601in}}%
\pgfpathlineto{\pgfqpoint{3.066070in}{2.508255in}}%
\pgfpathlineto{\pgfqpoint{3.076022in}{2.483068in}}%
\pgfpathlineto{\pgfqpoint{3.111292in}{2.496431in}}%
\pgfpathlineto{\pgfqpoint{3.146536in}{2.510752in}}%
\pgfpathlineto{\pgfqpoint{3.136510in}{2.537454in}}%
\pgfpathlineto{\pgfqpoint{3.126500in}{2.564315in}}%
\pgfpathlineto{\pgfqpoint{3.091329in}{2.548413in}}%
\pgfpathlineto{\pgfqpoint{3.056132in}{2.533601in}}%
\pgfpathclose%
\pgfusepath{fill}%
\end{pgfscope}%
\begin{pgfscope}%
\pgfpathrectangle{\pgfqpoint{1.020000in}{0.880000in}}{\pgfqpoint{6.160000in}{6.160000in}}%
\pgfusepath{clip}%
\pgfsetbuttcap%
\pgfsetroundjoin%
\definecolor{currentfill}{rgb}{0.483854,0.622050,0.974808}%
\pgfsetfillcolor{currentfill}%
\pgfsetlinewidth{0.000000pt}%
\definecolor{currentstroke}{rgb}{0.000000,0.000000,0.000000}%
\pgfsetstrokecolor{currentstroke}%
\pgfsetdash{}{0pt}%
\pgfpathmoveto{\pgfqpoint{3.890736in}{2.711814in}}%
\pgfpathlineto{\pgfqpoint{3.901467in}{2.666099in}}%
\pgfpathlineto{\pgfqpoint{3.936545in}{2.677091in}}%
\pgfpathlineto{\pgfqpoint{3.971608in}{2.686958in}}%
\pgfpathlineto{\pgfqpoint{3.960831in}{2.734025in}}%
\pgfpathlineto{\pgfqpoint{3.925790in}{2.723534in}}%
\pgfpathlineto{\pgfqpoint{3.890736in}{2.711814in}}%
\pgfpathclose%
\pgfusepath{fill}%
\end{pgfscope}%
\begin{pgfscope}%
\pgfpathrectangle{\pgfqpoint{1.020000in}{0.880000in}}{\pgfqpoint{6.160000in}{6.160000in}}%
\pgfusepath{clip}%
\pgfsetbuttcap%
\pgfsetroundjoin%
\definecolor{currentfill}{rgb}{0.430507,0.564883,0.948889}%
\pgfsetfillcolor{currentfill}%
\pgfsetlinewidth{0.000000pt}%
\definecolor{currentstroke}{rgb}{0.000000,0.000000,0.000000}%
\pgfsetstrokecolor{currentstroke}%
\pgfsetdash{}{0pt}%
\pgfpathmoveto{\pgfqpoint{3.357532in}{2.616224in}}%
\pgfpathlineto{\pgfqpoint{3.367791in}{2.583694in}}%
\pgfpathlineto{\pgfqpoint{3.378064in}{2.551546in}}%
\pgfpathlineto{\pgfqpoint{3.413235in}{2.569606in}}%
\pgfpathlineto{\pgfqpoint{3.448390in}{2.588199in}}%
\pgfpathlineto{\pgfqpoint{3.438046in}{2.622759in}}%
\pgfpathlineto{\pgfqpoint{3.427715in}{2.657745in}}%
\pgfpathlineto{\pgfqpoint{3.392631in}{2.636675in}}%
\pgfpathlineto{\pgfqpoint{3.357532in}{2.616224in}}%
\pgfpathclose%
\pgfusepath{fill}%
\end{pgfscope}%
\begin{pgfscope}%
\pgfpathrectangle{\pgfqpoint{1.020000in}{0.880000in}}{\pgfqpoint{6.160000in}{6.160000in}}%
\pgfusepath{clip}%
\pgfsetbuttcap%
\pgfsetroundjoin%
\definecolor{currentfill}{rgb}{0.473070,0.611077,0.970634}%
\pgfsetfillcolor{currentfill}%
\pgfsetlinewidth{0.000000pt}%
\definecolor{currentstroke}{rgb}{0.000000,0.000000,0.000000}%
\pgfsetstrokecolor{currentstroke}%
\pgfsetdash{}{0pt}%
\pgfpathmoveto{\pgfqpoint{3.659037in}{2.704643in}}%
\pgfpathlineto{\pgfqpoint{3.669587in}{2.663062in}}%
\pgfpathlineto{\pgfqpoint{3.680148in}{2.622352in}}%
\pgfpathlineto{\pgfqpoint{3.715276in}{2.638882in}}%
\pgfpathlineto{\pgfqpoint{3.750391in}{2.654947in}}%
\pgfpathlineto{\pgfqpoint{3.739773in}{2.697983in}}%
\pgfpathlineto{\pgfqpoint{3.729165in}{2.741967in}}%
\pgfpathlineto{\pgfqpoint{3.694106in}{2.723579in}}%
\pgfpathlineto{\pgfqpoint{3.659037in}{2.704643in}}%
\pgfpathclose%
\pgfusepath{fill}%
\end{pgfscope}%
\begin{pgfscope}%
\pgfpathrectangle{\pgfqpoint{1.020000in}{0.880000in}}{\pgfqpoint{6.160000in}{6.160000in}}%
\pgfusepath{clip}%
\pgfsetbuttcap%
\pgfsetroundjoin%
\definecolor{currentfill}{rgb}{0.378598,0.503856,0.913692}%
\pgfsetfillcolor{currentfill}%
\pgfsetlinewidth{0.000000pt}%
\definecolor{currentstroke}{rgb}{0.000000,0.000000,0.000000}%
\pgfsetstrokecolor{currentstroke}%
\pgfsetdash{}{0pt}%
\pgfpathmoveto{\pgfqpoint{2.985659in}{2.507166in}}%
\pgfpathlineto{\pgfqpoint{2.995525in}{2.483074in}}%
\pgfpathlineto{\pgfqpoint{3.005405in}{2.459139in}}%
\pgfpathlineto{\pgfqpoint{3.040727in}{2.470645in}}%
\pgfpathlineto{\pgfqpoint{3.076022in}{2.483068in}}%
\pgfpathlineto{\pgfqpoint{3.066070in}{2.508255in}}%
\pgfpathlineto{\pgfqpoint{3.056132in}{2.533601in}}%
\pgfpathlineto{\pgfqpoint{3.020909in}{2.519862in}}%
\pgfpathlineto{\pgfqpoint{2.985659in}{2.507166in}}%
\pgfpathclose%
\pgfusepath{fill}%
\end{pgfscope}%
\begin{pgfscope}%
\pgfpathrectangle{\pgfqpoint{1.020000in}{0.880000in}}{\pgfqpoint{6.160000in}{6.160000in}}%
\pgfusepath{clip}%
\pgfsetbuttcap%
\pgfsetroundjoin%
\definecolor{currentfill}{rgb}{0.414801,0.546874,0.939088}%
\pgfsetfillcolor{currentfill}%
\pgfsetlinewidth{0.000000pt}%
\definecolor{currentstroke}{rgb}{0.000000,0.000000,0.000000}%
\pgfsetstrokecolor{currentstroke}%
\pgfsetdash{}{0pt}%
\pgfpathmoveto{\pgfqpoint{3.287281in}{2.577543in}}%
\pgfpathlineto{\pgfqpoint{3.297468in}{2.547273in}}%
\pgfpathlineto{\pgfqpoint{3.307669in}{2.517345in}}%
\pgfpathlineto{\pgfqpoint{3.342876in}{2.534103in}}%
\pgfpathlineto{\pgfqpoint{3.378064in}{2.551546in}}%
\pgfpathlineto{\pgfqpoint{3.367791in}{2.583694in}}%
\pgfpathlineto{\pgfqpoint{3.357532in}{2.616224in}}%
\pgfpathlineto{\pgfqpoint{3.322416in}{2.596486in}}%
\pgfpathlineto{\pgfqpoint{3.287281in}{2.577543in}}%
\pgfpathclose%
\pgfusepath{fill}%
\end{pgfscope}%
\begin{pgfscope}%
\pgfpathrectangle{\pgfqpoint{1.020000in}{0.880000in}}{\pgfqpoint{6.160000in}{6.160000in}}%
\pgfusepath{clip}%
\pgfsetbuttcap%
\pgfsetroundjoin%
\definecolor{currentfill}{rgb}{0.473070,0.611077,0.970634}%
\pgfsetfillcolor{currentfill}%
\pgfsetlinewidth{0.000000pt}%
\definecolor{currentstroke}{rgb}{0.000000,0.000000,0.000000}%
\pgfsetstrokecolor{currentstroke}%
\pgfsetdash{}{0pt}%
\pgfpathmoveto{\pgfqpoint{3.820588in}{2.685127in}}%
\pgfpathlineto{\pgfqpoint{3.831270in}{2.641146in}}%
\pgfpathlineto{\pgfqpoint{3.866375in}{2.654081in}}%
\pgfpathlineto{\pgfqpoint{3.901467in}{2.666099in}}%
\pgfpathlineto{\pgfqpoint{3.890736in}{2.711814in}}%
\pgfpathlineto{\pgfqpoint{3.855668in}{2.698972in}}%
\pgfpathlineto{\pgfqpoint{3.820588in}{2.685127in}}%
\pgfpathclose%
\pgfusepath{fill}%
\end{pgfscope}%
\begin{pgfscope}%
\pgfpathrectangle{\pgfqpoint{1.020000in}{0.880000in}}{\pgfqpoint{6.160000in}{6.160000in}}%
\pgfusepath{clip}%
\pgfsetbuttcap%
\pgfsetroundjoin%
\definecolor{currentfill}{rgb}{0.457046,0.594006,0.963029}%
\pgfsetfillcolor{currentfill}%
\pgfsetlinewidth{0.000000pt}%
\definecolor{currentstroke}{rgb}{0.000000,0.000000,0.000000}%
\pgfsetstrokecolor{currentstroke}%
\pgfsetdash{}{0pt}%
\pgfpathmoveto{\pgfqpoint{3.588868in}{2.665768in}}%
\pgfpathlineto{\pgfqpoint{3.599357in}{2.626710in}}%
\pgfpathlineto{\pgfqpoint{3.609857in}{2.588441in}}%
\pgfpathlineto{\pgfqpoint{3.645009in}{2.605493in}}%
\pgfpathlineto{\pgfqpoint{3.680148in}{2.622352in}}%
\pgfpathlineto{\pgfqpoint{3.669587in}{2.663062in}}%
\pgfpathlineto{\pgfqpoint{3.659037in}{2.704643in}}%
\pgfpathlineto{\pgfqpoint{3.623957in}{2.685320in}}%
\pgfpathlineto{\pgfqpoint{3.588868in}{2.665768in}}%
\pgfpathclose%
\pgfusepath{fill}%
\end{pgfscope}%
\begin{pgfscope}%
\pgfpathrectangle{\pgfqpoint{1.020000in}{0.880000in}}{\pgfqpoint{6.160000in}{6.160000in}}%
\pgfusepath{clip}%
\pgfsetbuttcap%
\pgfsetroundjoin%
\definecolor{currentfill}{rgb}{0.399231,0.528528,0.928459}%
\pgfsetfillcolor{currentfill}%
\pgfsetlinewidth{0.000000pt}%
\definecolor{currentstroke}{rgb}{0.000000,0.000000,0.000000}%
\pgfsetstrokecolor{currentstroke}%
\pgfsetdash{}{0pt}%
\pgfpathmoveto{\pgfqpoint{3.216952in}{2.542278in}}%
\pgfpathlineto{\pgfqpoint{3.227066in}{2.514031in}}%
\pgfpathlineto{\pgfqpoint{3.237194in}{2.486090in}}%
\pgfpathlineto{\pgfqpoint{3.272442in}{2.501326in}}%
\pgfpathlineto{\pgfqpoint{3.307669in}{2.517345in}}%
\pgfpathlineto{\pgfqpoint{3.297468in}{2.547273in}}%
\pgfpathlineto{\pgfqpoint{3.287281in}{2.577543in}}%
\pgfpathlineto{\pgfqpoint{3.252127in}{2.559458in}}%
\pgfpathlineto{\pgfqpoint{3.216952in}{2.542278in}}%
\pgfpathclose%
\pgfusepath{fill}%
\end{pgfscope}%
\begin{pgfscope}%
\pgfpathrectangle{\pgfqpoint{1.020000in}{0.880000in}}{\pgfqpoint{6.160000in}{6.160000in}}%
\pgfusepath{clip}%
\pgfsetbuttcap%
\pgfsetroundjoin%
\definecolor{currentfill}{rgb}{0.368507,0.491141,0.905243}%
\pgfsetfillcolor{currentfill}%
\pgfsetlinewidth{0.000000pt}%
\definecolor{currentstroke}{rgb}{0.000000,0.000000,0.000000}%
\pgfsetstrokecolor{currentstroke}%
\pgfsetdash{}{0pt}%
\pgfpathmoveto{\pgfqpoint{2.915074in}{2.484750in}}%
\pgfpathlineto{\pgfqpoint{2.924869in}{2.461665in}}%
\pgfpathlineto{\pgfqpoint{2.934680in}{2.438738in}}%
\pgfpathlineto{\pgfqpoint{2.970056in}{2.448516in}}%
\pgfpathlineto{\pgfqpoint{3.005405in}{2.459139in}}%
\pgfpathlineto{\pgfqpoint{2.995525in}{2.483074in}}%
\pgfpathlineto{\pgfqpoint{2.985659in}{2.507166in}}%
\pgfpathlineto{\pgfqpoint{2.950381in}{2.495477in}}%
\pgfpathlineto{\pgfqpoint{2.915074in}{2.484750in}}%
\pgfpathclose%
\pgfusepath{fill}%
\end{pgfscope}%
\begin{pgfscope}%
\pgfpathrectangle{\pgfqpoint{1.020000in}{0.880000in}}{\pgfqpoint{6.160000in}{6.160000in}}%
\pgfusepath{clip}%
\pgfsetbuttcap%
\pgfsetroundjoin%
\definecolor{currentfill}{rgb}{0.462354,0.599830,0.965857}%
\pgfsetfillcolor{currentfill}%
\pgfsetlinewidth{0.000000pt}%
\definecolor{currentstroke}{rgb}{0.000000,0.000000,0.000000}%
\pgfsetstrokecolor{currentstroke}%
\pgfsetdash{}{0pt}%
\pgfpathmoveto{\pgfqpoint{3.750391in}{2.654947in}}%
\pgfpathlineto{\pgfqpoint{3.761021in}{2.612996in}}%
\pgfpathlineto{\pgfqpoint{3.796152in}{2.627410in}}%
\pgfpathlineto{\pgfqpoint{3.831270in}{2.641146in}}%
\pgfpathlineto{\pgfqpoint{3.820588in}{2.685127in}}%
\pgfpathlineto{\pgfqpoint{3.785496in}{2.670408in}}%
\pgfpathlineto{\pgfqpoint{3.750391in}{2.654947in}}%
\pgfpathclose%
\pgfusepath{fill}%
\end{pgfscope}%
\begin{pgfscope}%
\pgfpathrectangle{\pgfqpoint{1.020000in}{0.880000in}}{\pgfqpoint{6.160000in}{6.160000in}}%
\pgfusepath{clip}%
\pgfsetbuttcap%
\pgfsetroundjoin%
\definecolor{currentfill}{rgb}{0.441123,0.576532,0.954545}%
\pgfsetfillcolor{currentfill}%
\pgfsetlinewidth{0.000000pt}%
\definecolor{currentstroke}{rgb}{0.000000,0.000000,0.000000}%
\pgfsetstrokecolor{currentstroke}%
\pgfsetdash{}{0pt}%
\pgfpathmoveto{\pgfqpoint{3.518655in}{2.626583in}}%
\pgfpathlineto{\pgfqpoint{3.529079in}{2.590071in}}%
\pgfpathlineto{\pgfqpoint{3.539514in}{2.554268in}}%
\pgfpathlineto{\pgfqpoint{3.574693in}{2.571325in}}%
\pgfpathlineto{\pgfqpoint{3.609857in}{2.588441in}}%
\pgfpathlineto{\pgfqpoint{3.599357in}{2.626710in}}%
\pgfpathlineto{\pgfqpoint{3.588868in}{2.665768in}}%
\pgfpathlineto{\pgfqpoint{3.553767in}{2.646141in}}%
\pgfpathlineto{\pgfqpoint{3.518655in}{2.626583in}}%
\pgfpathclose%
\pgfusepath{fill}%
\end{pgfscope}%
\begin{pgfscope}%
\pgfpathrectangle{\pgfqpoint{1.020000in}{0.880000in}}{\pgfqpoint{6.160000in}{6.160000in}}%
\pgfusepath{clip}%
\pgfsetbuttcap%
\pgfsetroundjoin%
\definecolor{currentfill}{rgb}{0.383662,0.510183,0.917831}%
\pgfsetfillcolor{currentfill}%
\pgfsetlinewidth{0.000000pt}%
\definecolor{currentstroke}{rgb}{0.000000,0.000000,0.000000}%
\pgfsetstrokecolor{currentstroke}%
\pgfsetdash{}{0pt}%
\pgfpathmoveto{\pgfqpoint{3.146536in}{2.510752in}}%
\pgfpathlineto{\pgfqpoint{3.156576in}{2.484270in}}%
\pgfpathlineto{\pgfqpoint{3.166631in}{2.458064in}}%
\pgfpathlineto{\pgfqpoint{3.201924in}{2.471664in}}%
\pgfpathlineto{\pgfqpoint{3.237194in}{2.486090in}}%
\pgfpathlineto{\pgfqpoint{3.227066in}{2.514031in}}%
\pgfpathlineto{\pgfqpoint{3.216952in}{2.542278in}}%
\pgfpathlineto{\pgfqpoint{3.181756in}{2.526036in}}%
\pgfpathlineto{\pgfqpoint{3.146536in}{2.510752in}}%
\pgfpathclose%
\pgfusepath{fill}%
\end{pgfscope}%
\begin{pgfscope}%
\pgfpathrectangle{\pgfqpoint{1.020000in}{0.880000in}}{\pgfqpoint{6.160000in}{6.160000in}}%
\pgfusepath{clip}%
\pgfsetbuttcap%
\pgfsetroundjoin%
\definecolor{currentfill}{rgb}{0.425199,0.559058,0.946061}%
\pgfsetfillcolor{currentfill}%
\pgfsetlinewidth{0.000000pt}%
\definecolor{currentstroke}{rgb}{0.000000,0.000000,0.000000}%
\pgfsetstrokecolor{currentstroke}%
\pgfsetdash{}{0pt}%
\pgfpathmoveto{\pgfqpoint{3.448390in}{2.588199in}}%
\pgfpathlineto{\pgfqpoint{3.458746in}{2.554172in}}%
\pgfpathlineto{\pgfqpoint{3.469114in}{2.520775in}}%
\pgfpathlineto{\pgfqpoint{3.504322in}{2.537383in}}%
\pgfpathlineto{\pgfqpoint{3.539514in}{2.554268in}}%
\pgfpathlineto{\pgfqpoint{3.529079in}{2.590071in}}%
\pgfpathlineto{\pgfqpoint{3.518655in}{2.626583in}}%
\pgfpathlineto{\pgfqpoint{3.483529in}{2.607228in}}%
\pgfpathlineto{\pgfqpoint{3.448390in}{2.588199in}}%
\pgfpathclose%
\pgfusepath{fill}%
\end{pgfscope}%
\begin{pgfscope}%
\pgfpathrectangle{\pgfqpoint{1.020000in}{0.880000in}}{\pgfqpoint{6.160000in}{6.160000in}}%
\pgfusepath{clip}%
\pgfsetbuttcap%
\pgfsetroundjoin%
\definecolor{currentfill}{rgb}{0.446431,0.582356,0.957373}%
\pgfsetfillcolor{currentfill}%
\pgfsetlinewidth{0.000000pt}%
\definecolor{currentstroke}{rgb}{0.000000,0.000000,0.000000}%
\pgfsetstrokecolor{currentstroke}%
\pgfsetdash{}{0pt}%
\pgfpathmoveto{\pgfqpoint{3.680148in}{2.622352in}}%
\pgfpathlineto{\pgfqpoint{3.690720in}{2.582635in}}%
\pgfpathlineto{\pgfqpoint{3.725877in}{2.598028in}}%
\pgfpathlineto{\pgfqpoint{3.761021in}{2.612996in}}%
\pgfpathlineto{\pgfqpoint{3.750391in}{2.654947in}}%
\pgfpathlineto{\pgfqpoint{3.715276in}{2.638882in}}%
\pgfpathlineto{\pgfqpoint{3.680148in}{2.622352in}}%
\pgfpathclose%
\pgfusepath{fill}%
\end{pgfscope}%
\begin{pgfscope}%
\pgfpathrectangle{\pgfqpoint{1.020000in}{0.880000in}}{\pgfqpoint{6.160000in}{6.160000in}}%
\pgfusepath{clip}%
\pgfsetbuttcap%
\pgfsetroundjoin%
\definecolor{currentfill}{rgb}{0.373552,0.497499,0.909467}%
\pgfsetfillcolor{currentfill}%
\pgfsetlinewidth{0.000000pt}%
\definecolor{currentstroke}{rgb}{0.000000,0.000000,0.000000}%
\pgfsetstrokecolor{currentstroke}%
\pgfsetdash{}{0pt}%
\pgfpathmoveto{\pgfqpoint{3.076022in}{2.483068in}}%
\pgfpathlineto{\pgfqpoint{3.085990in}{2.458086in}}%
\pgfpathlineto{\pgfqpoint{3.095972in}{2.433354in}}%
\pgfpathlineto{\pgfqpoint{3.131314in}{2.445295in}}%
\pgfpathlineto{\pgfqpoint{3.166631in}{2.458064in}}%
\pgfpathlineto{\pgfqpoint{3.156576in}{2.484270in}}%
\pgfpathlineto{\pgfqpoint{3.146536in}{2.510752in}}%
\pgfpathlineto{\pgfqpoint{3.111292in}{2.496431in}}%
\pgfpathlineto{\pgfqpoint{3.076022in}{2.483068in}}%
\pgfpathclose%
\pgfusepath{fill}%
\end{pgfscope}%
\begin{pgfscope}%
\pgfpathrectangle{\pgfqpoint{1.020000in}{0.880000in}}{\pgfqpoint{6.160000in}{6.160000in}}%
\pgfusepath{clip}%
\pgfsetbuttcap%
\pgfsetroundjoin%
\definecolor{currentfill}{rgb}{0.409611,0.540759,0.935545}%
\pgfsetfillcolor{currentfill}%
\pgfsetlinewidth{0.000000pt}%
\definecolor{currentstroke}{rgb}{0.000000,0.000000,0.000000}%
\pgfsetstrokecolor{currentstroke}%
\pgfsetdash{}{0pt}%
\pgfpathmoveto{\pgfqpoint{3.378064in}{2.551546in}}%
\pgfpathlineto{\pgfqpoint{3.388350in}{2.519872in}}%
\pgfpathlineto{\pgfqpoint{3.398648in}{2.488751in}}%
\pgfpathlineto{\pgfqpoint{3.433890in}{2.504537in}}%
\pgfpathlineto{\pgfqpoint{3.469114in}{2.520775in}}%
\pgfpathlineto{\pgfqpoint{3.458746in}{2.554172in}}%
\pgfpathlineto{\pgfqpoint{3.448390in}{2.588199in}}%
\pgfpathlineto{\pgfqpoint{3.413235in}{2.569606in}}%
\pgfpathlineto{\pgfqpoint{3.378064in}{2.551546in}}%
\pgfpathclose%
\pgfusepath{fill}%
\end{pgfscope}%
\begin{pgfscope}%
\pgfpathrectangle{\pgfqpoint{1.020000in}{0.880000in}}{\pgfqpoint{6.160000in}{6.160000in}}%
\pgfusepath{clip}%
\pgfsetbuttcap%
\pgfsetroundjoin%
\definecolor{currentfill}{rgb}{0.435815,0.570707,0.951717}%
\pgfsetfillcolor{currentfill}%
\pgfsetlinewidth{0.000000pt}%
\definecolor{currentstroke}{rgb}{0.000000,0.000000,0.000000}%
\pgfsetstrokecolor{currentstroke}%
\pgfsetdash{}{0pt}%
\pgfpathmoveto{\pgfqpoint{3.609857in}{2.588441in}}%
\pgfpathlineto{\pgfqpoint{3.620368in}{2.551072in}}%
\pgfpathlineto{\pgfqpoint{3.655551in}{2.566942in}}%
\pgfpathlineto{\pgfqpoint{3.690720in}{2.582635in}}%
\pgfpathlineto{\pgfqpoint{3.680148in}{2.622352in}}%
\pgfpathlineto{\pgfqpoint{3.645009in}{2.605493in}}%
\pgfpathlineto{\pgfqpoint{3.609857in}{2.588441in}}%
\pgfpathclose%
\pgfusepath{fill}%
\end{pgfscope}%
\begin{pgfscope}%
\pgfpathrectangle{\pgfqpoint{1.020000in}{0.880000in}}{\pgfqpoint{6.160000in}{6.160000in}}%
\pgfusepath{clip}%
\pgfsetbuttcap%
\pgfsetroundjoin%
\definecolor{currentfill}{rgb}{0.394042,0.522413,0.924916}%
\pgfsetfillcolor{currentfill}%
\pgfsetlinewidth{0.000000pt}%
\definecolor{currentstroke}{rgb}{0.000000,0.000000,0.000000}%
\pgfsetstrokecolor{currentstroke}%
\pgfsetdash{}{0pt}%
\pgfpathmoveto{\pgfqpoint{3.307669in}{2.517345in}}%
\pgfpathlineto{\pgfqpoint{3.317883in}{2.487834in}}%
\pgfpathlineto{\pgfqpoint{3.328109in}{2.458809in}}%
\pgfpathlineto{\pgfqpoint{3.363388in}{2.473489in}}%
\pgfpathlineto{\pgfqpoint{3.398648in}{2.488751in}}%
\pgfpathlineto{\pgfqpoint{3.388350in}{2.519872in}}%
\pgfpathlineto{\pgfqpoint{3.378064in}{2.551546in}}%
\pgfpathlineto{\pgfqpoint{3.342876in}{2.534103in}}%
\pgfpathlineto{\pgfqpoint{3.307669in}{2.517345in}}%
\pgfpathclose%
\pgfusepath{fill}%
\end{pgfscope}%
\begin{pgfscope}%
\pgfpathrectangle{\pgfqpoint{1.020000in}{0.880000in}}{\pgfqpoint{6.160000in}{6.160000in}}%
\pgfusepath{clip}%
\pgfsetbuttcap%
\pgfsetroundjoin%
\definecolor{currentfill}{rgb}{0.363461,0.484784,0.901019}%
\pgfsetfillcolor{currentfill}%
\pgfsetlinewidth{0.000000pt}%
\definecolor{currentstroke}{rgb}{0.000000,0.000000,0.000000}%
\pgfsetstrokecolor{currentstroke}%
\pgfsetdash{}{0pt}%
\pgfpathmoveto{\pgfqpoint{3.005405in}{2.459139in}}%
\pgfpathlineto{\pgfqpoint{3.015301in}{2.435399in}}%
\pgfpathlineto{\pgfqpoint{3.025212in}{2.411887in}}%
\pgfpathlineto{\pgfqpoint{3.060605in}{2.422225in}}%
\pgfpathlineto{\pgfqpoint{3.095972in}{2.433354in}}%
\pgfpathlineto{\pgfqpoint{3.085990in}{2.458086in}}%
\pgfpathlineto{\pgfqpoint{3.076022in}{2.483068in}}%
\pgfpathlineto{\pgfqpoint{3.040727in}{2.470645in}}%
\pgfpathlineto{\pgfqpoint{3.005405in}{2.459139in}}%
\pgfpathclose%
\pgfusepath{fill}%
\end{pgfscope}%
\begin{pgfscope}%
\pgfpathrectangle{\pgfqpoint{1.020000in}{0.880000in}}{\pgfqpoint{6.160000in}{6.160000in}}%
\pgfusepath{clip}%
\pgfsetbuttcap%
\pgfsetroundjoin%
\definecolor{currentfill}{rgb}{0.419991,0.552989,0.942630}%
\pgfsetfillcolor{currentfill}%
\pgfsetlinewidth{0.000000pt}%
\definecolor{currentstroke}{rgb}{0.000000,0.000000,0.000000}%
\pgfsetstrokecolor{currentstroke}%
\pgfsetdash{}{0pt}%
\pgfpathmoveto{\pgfqpoint{3.539514in}{2.554268in}}%
\pgfpathlineto{\pgfqpoint{3.549961in}{2.519268in}}%
\pgfpathlineto{\pgfqpoint{3.585172in}{2.535143in}}%
\pgfpathlineto{\pgfqpoint{3.620368in}{2.551072in}}%
\pgfpathlineto{\pgfqpoint{3.609857in}{2.588441in}}%
\pgfpathlineto{\pgfqpoint{3.574693in}{2.571325in}}%
\pgfpathlineto{\pgfqpoint{3.539514in}{2.554268in}}%
\pgfpathclose%
\pgfusepath{fill}%
\end{pgfscope}%
\begin{pgfscope}%
\pgfpathrectangle{\pgfqpoint{1.020000in}{0.880000in}}{\pgfqpoint{6.160000in}{6.160000in}}%
\pgfusepath{clip}%
\pgfsetbuttcap%
\pgfsetroundjoin%
\definecolor{currentfill}{rgb}{0.378598,0.503856,0.913692}%
\pgfsetfillcolor{currentfill}%
\pgfsetlinewidth{0.000000pt}%
\definecolor{currentstroke}{rgb}{0.000000,0.000000,0.000000}%
\pgfsetstrokecolor{currentstroke}%
\pgfsetdash{}{0pt}%
\pgfpathmoveto{\pgfqpoint{3.237194in}{2.486090in}}%
\pgfpathlineto{\pgfqpoint{3.247335in}{2.458518in}}%
\pgfpathlineto{\pgfqpoint{3.257490in}{2.431369in}}%
\pgfpathlineto{\pgfqpoint{3.292810in}{2.444757in}}%
\pgfpathlineto{\pgfqpoint{3.328109in}{2.458809in}}%
\pgfpathlineto{\pgfqpoint{3.317883in}{2.487834in}}%
\pgfpathlineto{\pgfqpoint{3.307669in}{2.517345in}}%
\pgfpathlineto{\pgfqpoint{3.272442in}{2.501326in}}%
\pgfpathlineto{\pgfqpoint{3.237194in}{2.486090in}}%
\pgfpathclose%
\pgfusepath{fill}%
\end{pgfscope}%
\begin{pgfscope}%
\pgfpathrectangle{\pgfqpoint{1.020000in}{0.880000in}}{\pgfqpoint{6.160000in}{6.160000in}}%
\pgfusepath{clip}%
\pgfsetbuttcap%
\pgfsetroundjoin%
\definecolor{currentfill}{rgb}{0.404421,0.534643,0.932002}%
\pgfsetfillcolor{currentfill}%
\pgfsetlinewidth{0.000000pt}%
\definecolor{currentstroke}{rgb}{0.000000,0.000000,0.000000}%
\pgfsetstrokecolor{currentstroke}%
\pgfsetdash{}{0pt}%
\pgfpathmoveto{\pgfqpoint{3.469114in}{2.520775in}}%
\pgfpathlineto{\pgfqpoint{3.479493in}{2.488088in}}%
\pgfpathlineto{\pgfqpoint{3.514735in}{2.503552in}}%
\pgfpathlineto{\pgfqpoint{3.549961in}{2.519268in}}%
\pgfpathlineto{\pgfqpoint{3.539514in}{2.554268in}}%
\pgfpathlineto{\pgfqpoint{3.504322in}{2.537383in}}%
\pgfpathlineto{\pgfqpoint{3.469114in}{2.520775in}}%
\pgfpathclose%
\pgfusepath{fill}%
\end{pgfscope}%
\begin{pgfscope}%
\pgfpathrectangle{\pgfqpoint{1.020000in}{0.880000in}}{\pgfqpoint{6.160000in}{6.160000in}}%
\pgfusepath{clip}%
\pgfsetbuttcap%
\pgfsetroundjoin%
\definecolor{currentfill}{rgb}{0.353369,0.472069,0.892570}%
\pgfsetfillcolor{currentfill}%
\pgfsetlinewidth{0.000000pt}%
\definecolor{currentstroke}{rgb}{0.000000,0.000000,0.000000}%
\pgfsetstrokecolor{currentstroke}%
\pgfsetdash{}{0pt}%
\pgfpathmoveto{\pgfqpoint{2.934680in}{2.438738in}}%
\pgfpathlineto{\pgfqpoint{2.944506in}{2.415997in}}%
\pgfpathlineto{\pgfqpoint{2.954346in}{2.393467in}}%
\pgfpathlineto{\pgfqpoint{2.989792in}{2.402312in}}%
\pgfpathlineto{\pgfqpoint{3.025212in}{2.411887in}}%
\pgfpathlineto{\pgfqpoint{3.015301in}{2.435399in}}%
\pgfpathlineto{\pgfqpoint{3.005405in}{2.459139in}}%
\pgfpathlineto{\pgfqpoint{2.970056in}{2.448516in}}%
\pgfpathlineto{\pgfqpoint{2.934680in}{2.438738in}}%
\pgfpathclose%
\pgfusepath{fill}%
\end{pgfscope}%
\begin{pgfscope}%
\pgfpathrectangle{\pgfqpoint{1.020000in}{0.880000in}}{\pgfqpoint{6.160000in}{6.160000in}}%
\pgfusepath{clip}%
\pgfsetbuttcap%
\pgfsetroundjoin%
\definecolor{currentfill}{rgb}{0.388852,0.516298,0.921373}%
\pgfsetfillcolor{currentfill}%
\pgfsetlinewidth{0.000000pt}%
\definecolor{currentstroke}{rgb}{0.000000,0.000000,0.000000}%
\pgfsetstrokecolor{currentstroke}%
\pgfsetdash{}{0pt}%
\pgfpathmoveto{\pgfqpoint{3.398648in}{2.488751in}}%
\pgfpathlineto{\pgfqpoint{3.408958in}{2.458254in}}%
\pgfpathlineto{\pgfqpoint{3.444235in}{2.472964in}}%
\pgfpathlineto{\pgfqpoint{3.479493in}{2.488088in}}%
\pgfpathlineto{\pgfqpoint{3.469114in}{2.520775in}}%
\pgfpathlineto{\pgfqpoint{3.433890in}{2.504537in}}%
\pgfpathlineto{\pgfqpoint{3.398648in}{2.488751in}}%
\pgfpathclose%
\pgfusepath{fill}%
\end{pgfscope}%
\begin{pgfscope}%
\pgfpathrectangle{\pgfqpoint{1.020000in}{0.880000in}}{\pgfqpoint{6.160000in}{6.160000in}}%
\pgfusepath{clip}%
\pgfsetbuttcap%
\pgfsetroundjoin%
\definecolor{currentfill}{rgb}{0.368507,0.491141,0.905243}%
\pgfsetfillcolor{currentfill}%
\pgfsetlinewidth{0.000000pt}%
\definecolor{currentstroke}{rgb}{0.000000,0.000000,0.000000}%
\pgfsetstrokecolor{currentstroke}%
\pgfsetdash{}{0pt}%
\pgfpathmoveto{\pgfqpoint{3.166631in}{2.458064in}}%
\pgfpathlineto{\pgfqpoint{3.176700in}{2.432183in}}%
\pgfpathlineto{\pgfqpoint{3.186782in}{2.406672in}}%
\pgfpathlineto{\pgfqpoint{3.222147in}{2.418670in}}%
\pgfpathlineto{\pgfqpoint{3.257490in}{2.431369in}}%
\pgfpathlineto{\pgfqpoint{3.247335in}{2.458518in}}%
\pgfpathlineto{\pgfqpoint{3.237194in}{2.486090in}}%
\pgfpathlineto{\pgfqpoint{3.201924in}{2.471664in}}%
\pgfpathlineto{\pgfqpoint{3.166631in}{2.458064in}}%
\pgfpathclose%
\pgfusepath{fill}%
\end{pgfscope}%
\begin{pgfscope}%
\pgfpathrectangle{\pgfqpoint{1.020000in}{0.880000in}}{\pgfqpoint{6.160000in}{6.160000in}}%
\pgfusepath{clip}%
\pgfsetbuttcap%
\pgfsetroundjoin%
\definecolor{currentfill}{rgb}{0.378598,0.503856,0.913692}%
\pgfsetfillcolor{currentfill}%
\pgfsetlinewidth{0.000000pt}%
\definecolor{currentstroke}{rgb}{0.000000,0.000000,0.000000}%
\pgfsetstrokecolor{currentstroke}%
\pgfsetdash{}{0pt}%
\pgfpathmoveto{\pgfqpoint{3.328109in}{2.458809in}}%
\pgfpathlineto{\pgfqpoint{3.338349in}{2.430326in}}%
\pgfpathlineto{\pgfqpoint{3.373663in}{2.444024in}}%
\pgfpathlineto{\pgfqpoint{3.408958in}{2.458254in}}%
\pgfpathlineto{\pgfqpoint{3.398648in}{2.488751in}}%
\pgfpathlineto{\pgfqpoint{3.363388in}{2.473489in}}%
\pgfpathlineto{\pgfqpoint{3.328109in}{2.458809in}}%
\pgfpathclose%
\pgfusepath{fill}%
\end{pgfscope}%
\begin{pgfscope}%
\pgfpathrectangle{\pgfqpoint{1.020000in}{0.880000in}}{\pgfqpoint{6.160000in}{6.160000in}}%
\pgfusepath{clip}%
\pgfsetbuttcap%
\pgfsetroundjoin%
\definecolor{currentfill}{rgb}{0.358415,0.478426,0.896795}%
\pgfsetfillcolor{currentfill}%
\pgfsetlinewidth{0.000000pt}%
\definecolor{currentstroke}{rgb}{0.000000,0.000000,0.000000}%
\pgfsetstrokecolor{currentstroke}%
\pgfsetdash{}{0pt}%
\pgfpathmoveto{\pgfqpoint{3.095972in}{2.433354in}}%
\pgfpathlineto{\pgfqpoint{3.105969in}{2.408910in}}%
\pgfpathlineto{\pgfqpoint{3.115979in}{2.384791in}}%
\pgfpathlineto{\pgfqpoint{3.151393in}{2.395380in}}%
\pgfpathlineto{\pgfqpoint{3.186782in}{2.406672in}}%
\pgfpathlineto{\pgfqpoint{3.176700in}{2.432183in}}%
\pgfpathlineto{\pgfqpoint{3.166631in}{2.458064in}}%
\pgfpathlineto{\pgfqpoint{3.131314in}{2.445295in}}%
\pgfpathlineto{\pgfqpoint{3.095972in}{2.433354in}}%
\pgfpathclose%
\pgfusepath{fill}%
\end{pgfscope}%
\begin{pgfscope}%
\pgfpathrectangle{\pgfqpoint{1.020000in}{0.880000in}}{\pgfqpoint{6.160000in}{6.160000in}}%
\pgfusepath{clip}%
\pgfsetbuttcap%
\pgfsetroundjoin%
\definecolor{currentfill}{rgb}{0.363461,0.484784,0.901019}%
\pgfsetfillcolor{currentfill}%
\pgfsetlinewidth{0.000000pt}%
\definecolor{currentstroke}{rgb}{0.000000,0.000000,0.000000}%
\pgfsetstrokecolor{currentstroke}%
\pgfsetdash{}{0pt}%
\pgfpathmoveto{\pgfqpoint{3.257490in}{2.431369in}}%
\pgfpathlineto{\pgfqpoint{3.267658in}{2.404691in}}%
\pgfpathlineto{\pgfqpoint{3.303014in}{2.417205in}}%
\pgfpathlineto{\pgfqpoint{3.338349in}{2.430326in}}%
\pgfpathlineto{\pgfqpoint{3.328109in}{2.458809in}}%
\pgfpathlineto{\pgfqpoint{3.292810in}{2.444757in}}%
\pgfpathlineto{\pgfqpoint{3.257490in}{2.431369in}}%
\pgfpathclose%
\pgfusepath{fill}%
\end{pgfscope}%
\begin{pgfscope}%
\pgfpathrectangle{\pgfqpoint{1.020000in}{0.880000in}}{\pgfqpoint{6.160000in}{6.160000in}}%
\pgfusepath{clip}%
\pgfsetbuttcap%
\pgfsetroundjoin%
\definecolor{currentfill}{rgb}{0.348323,0.465711,0.888346}%
\pgfsetfillcolor{currentfill}%
\pgfsetlinewidth{0.000000pt}%
\definecolor{currentstroke}{rgb}{0.000000,0.000000,0.000000}%
\pgfsetstrokecolor{currentstroke}%
\pgfsetdash{}{0pt}%
\pgfpathmoveto{\pgfqpoint{3.025212in}{2.411887in}}%
\pgfpathlineto{\pgfqpoint{3.035137in}{2.388634in}}%
\pgfpathlineto{\pgfqpoint{3.045077in}{2.365666in}}%
\pgfpathlineto{\pgfqpoint{3.080541in}{2.374892in}}%
\pgfpathlineto{\pgfqpoint{3.115979in}{2.384791in}}%
\pgfpathlineto{\pgfqpoint{3.105969in}{2.408910in}}%
\pgfpathlineto{\pgfqpoint{3.095972in}{2.433354in}}%
\pgfpathlineto{\pgfqpoint{3.060605in}{2.422225in}}%
\pgfpathlineto{\pgfqpoint{3.025212in}{2.411887in}}%
\pgfpathclose%
\pgfusepath{fill}%
\end{pgfscope}%
\begin{pgfscope}%
\pgfpathrectangle{\pgfqpoint{1.020000in}{0.880000in}}{\pgfqpoint{6.160000in}{6.160000in}}%
\pgfusepath{clip}%
\pgfsetbuttcap%
\pgfsetroundjoin%
\definecolor{currentfill}{rgb}{0.353369,0.472069,0.892570}%
\pgfsetfillcolor{currentfill}%
\pgfsetlinewidth{0.000000pt}%
\definecolor{currentstroke}{rgb}{0.000000,0.000000,0.000000}%
\pgfsetstrokecolor{currentstroke}%
\pgfsetdash{}{0pt}%
\pgfpathmoveto{\pgfqpoint{3.186782in}{2.406672in}}%
\pgfpathlineto{\pgfqpoint{3.196878in}{2.381567in}}%
\pgfpathlineto{\pgfqpoint{3.232279in}{2.392807in}}%
\pgfpathlineto{\pgfqpoint{3.267658in}{2.404691in}}%
\pgfpathlineto{\pgfqpoint{3.257490in}{2.431369in}}%
\pgfpathlineto{\pgfqpoint{3.222147in}{2.418670in}}%
\pgfpathlineto{\pgfqpoint{3.186782in}{2.406672in}}%
\pgfpathclose%
\pgfusepath{fill}%
\end{pgfscope}%
\begin{pgfscope}%
\pgfpathrectangle{\pgfqpoint{1.020000in}{0.880000in}}{\pgfqpoint{6.160000in}{6.160000in}}%
\pgfusepath{clip}%
\pgfsetbuttcap%
\pgfsetroundjoin%
\definecolor{currentfill}{rgb}{0.343278,0.459354,0.884122}%
\pgfsetfillcolor{currentfill}%
\pgfsetlinewidth{0.000000pt}%
\definecolor{currentstroke}{rgb}{0.000000,0.000000,0.000000}%
\pgfsetstrokecolor{currentstroke}%
\pgfsetdash{}{0pt}%
\pgfpathmoveto{\pgfqpoint{2.954346in}{2.393467in}}%
\pgfpathlineto{\pgfqpoint{2.964201in}{2.371171in}}%
\pgfpathlineto{\pgfqpoint{2.974071in}{2.349130in}}%
\pgfpathlineto{\pgfqpoint{3.009587in}{2.357088in}}%
\pgfpathlineto{\pgfqpoint{3.045077in}{2.365666in}}%
\pgfpathlineto{\pgfqpoint{3.035137in}{2.388634in}}%
\pgfpathlineto{\pgfqpoint{3.025212in}{2.411887in}}%
\pgfpathlineto{\pgfqpoint{2.989792in}{2.402312in}}%
\pgfpathlineto{\pgfqpoint{2.954346in}{2.393467in}}%
\pgfpathclose%
\pgfusepath{fill}%
\end{pgfscope}%
\begin{pgfscope}%
\pgfpathrectangle{\pgfqpoint{1.020000in}{0.880000in}}{\pgfqpoint{6.160000in}{6.160000in}}%
\pgfusepath{clip}%
\pgfsetbuttcap%
\pgfsetroundjoin%
\definecolor{currentfill}{rgb}{0.348323,0.465711,0.888346}%
\pgfsetfillcolor{currentfill}%
\pgfsetlinewidth{0.000000pt}%
\definecolor{currentstroke}{rgb}{0.000000,0.000000,0.000000}%
\pgfsetstrokecolor{currentstroke}%
\pgfsetdash{}{0pt}%
\pgfpathmoveto{\pgfqpoint{3.115979in}{2.384791in}}%
\pgfpathlineto{\pgfqpoint{3.126004in}{2.361024in}}%
\pgfpathlineto{\pgfqpoint{3.161453in}{2.370974in}}%
\pgfpathlineto{\pgfqpoint{3.196878in}{2.381567in}}%
\pgfpathlineto{\pgfqpoint{3.186782in}{2.406672in}}%
\pgfpathlineto{\pgfqpoint{3.151393in}{2.395380in}}%
\pgfpathlineto{\pgfqpoint{3.115979in}{2.384791in}}%
\pgfpathclose%
\pgfusepath{fill}%
\end{pgfscope}%
\begin{pgfscope}%
\pgfpathrectangle{\pgfqpoint{1.020000in}{0.880000in}}{\pgfqpoint{6.160000in}{6.160000in}}%
\pgfusepath{clip}%
\pgfsetbuttcap%
\pgfsetroundjoin%
\definecolor{currentfill}{rgb}{0.338377,0.452819,0.879317}%
\pgfsetfillcolor{currentfill}%
\pgfsetlinewidth{0.000000pt}%
\definecolor{currentstroke}{rgb}{0.000000,0.000000,0.000000}%
\pgfsetstrokecolor{currentstroke}%
\pgfsetdash{}{0pt}%
\pgfpathmoveto{\pgfqpoint{3.045077in}{2.365666in}}%
\pgfpathlineto{\pgfqpoint{3.055031in}{2.343006in}}%
\pgfpathlineto{\pgfqpoint{3.090530in}{2.351707in}}%
\pgfpathlineto{\pgfqpoint{3.126004in}{2.361024in}}%
\pgfpathlineto{\pgfqpoint{3.115979in}{2.384791in}}%
\pgfpathlineto{\pgfqpoint{3.080541in}{2.374892in}}%
\pgfpathlineto{\pgfqpoint{3.045077in}{2.365666in}}%
\pgfpathclose%
\pgfusepath{fill}%
\end{pgfscope}%
\begin{pgfscope}%
\pgfpathrectangle{\pgfqpoint{1.020000in}{0.880000in}}{\pgfqpoint{6.160000in}{6.160000in}}%
\pgfusepath{clip}%
\pgfsetbuttcap%
\pgfsetroundjoin%
\definecolor{currentfill}{rgb}{0.333490,0.446265,0.874452}%
\pgfsetfillcolor{currentfill}%
\pgfsetlinewidth{0.000000pt}%
\definecolor{currentstroke}{rgb}{0.000000,0.000000,0.000000}%
\pgfsetstrokecolor{currentstroke}%
\pgfsetdash{}{0pt}%
\pgfpathmoveto{\pgfqpoint{2.974071in}{2.349130in}}%
\pgfpathlineto{\pgfqpoint{2.983956in}{2.327360in}}%
\pgfpathlineto{\pgfqpoint{3.019506in}{2.334899in}}%
\pgfpathlineto{\pgfqpoint{3.055031in}{2.343006in}}%
\pgfpathlineto{\pgfqpoint{3.045077in}{2.365666in}}%
\pgfpathlineto{\pgfqpoint{3.009587in}{2.357088in}}%
\pgfpathlineto{\pgfqpoint{2.974071in}{2.349130in}}%
\pgfpathclose%
\pgfusepath{fill}%
\end{pgfscope}%
\begin{pgfscope}%
\definecolor{textcolor}{rgb}{0.000000,0.000000,0.000000}%
\pgfsetstrokecolor{textcolor}%
\pgfsetfillcolor{textcolor}%
\pgftext[x=4.100000in,y=7.123333in,,base]{\color{textcolor}\rmfamily\fontsize{16.000000}{19.200000}\selectfont Franke's Function}%
\end{pgfscope}%
\end{pgfpicture}%
\makeatother%
\endgroup%

%     \end{adjustbox}
%     \end{center}
% \end{wrapfigure}
%
%
%
% \section{Linear Regression Models }
% \label{sec:models}
%
%
%
% $$
% MSE(\mathbf{y},\mathbf{\tilde{y}}) = \frac{1}{n}
% \sum_{i=0}^{n-1}(y_i-\tilde{y}_i)^2,
% $$
% $$
% R^2(\mathbf{y}, \tilde{\mathbf{y}}) = 1 - \frac{\sum_{i=0}^{n - 1} (y_i - \tilde{y}_i)^2}{\sum_{i=0}^{n - 1} (y_i - \bar{y})^2},
% $$
%
% $$
% \bar{y} =  \frac{1}{n} \sum_{i=0}^{n - 1} y_i.
% $$
%
%
% \subsection{Ordinary Least Squares (OLS)}
% \label{sec:OLS}
%
% \subsection{Ridge Regression}
% \label{sec:Ridge}
%
% \subsection{Lasso Regression}
% \label{sec:Lasso}
%
%
% \begin{figure}[!h]
%     \begin{center}
%     \resizebox*{0.9\linewidth}{!}{%% Creator: Matplotlib, PGF backend
%%
%% To include the figure in your LaTeX document, write
%%   \input{<filename>.pgf}
%%
%% Make sure the required packages are loaded in your preamble
%%   \usepackage{pgf}
%%
%% Also ensure that all the required font packages are loaded; for instance,
%% the lmodern package is sometimes necessary when using math font.
%%   \usepackage{lmodern}
%%
%% Figures using additional raster images can only be included by \input if
%% they are in the same directory as the main LaTeX file. For loading figures
%% from other directories you can use the `import` package
%%   \usepackage{import}
%%
%% and then include the figures with
%%   \import{<path to file>}{<filename>.pgf}
%%
%% Matplotlib used the following preamble
%%   
%%   \usepackage{fontspec}
%%   \setmainfont{DejaVuSerif.ttf}[Path=\detokenize{/home/brage/anaconda3/lib/python3.10/site-packages/matplotlib/mpl-data/fonts/ttf/}]
%%   \setsansfont{DejaVuSans.ttf}[Path=\detokenize{/home/brage/anaconda3/lib/python3.10/site-packages/matplotlib/mpl-data/fonts/ttf/}]
%%   \setmonofont{DejaVuSansMono.ttf}[Path=\detokenize{/home/brage/anaconda3/lib/python3.10/site-packages/matplotlib/mpl-data/fonts/ttf/}]
%%   \makeatletter\@ifpackageloaded{underscore}{}{\usepackage[strings]{underscore}}\makeatother
%%
\begingroup%
\makeatletter%
\begin{pgfpicture}%
\pgfpathrectangle{\pgfpointorigin}{\pgfqpoint{12.000000in}{4.000000in}}%
\pgfusepath{use as bounding box, clip}%
\begin{pgfscope}%
\pgfsetbuttcap%
\pgfsetmiterjoin%
\definecolor{currentfill}{rgb}{1.000000,1.000000,1.000000}%
\pgfsetfillcolor{currentfill}%
\pgfsetlinewidth{0.000000pt}%
\definecolor{currentstroke}{rgb}{1.000000,1.000000,1.000000}%
\pgfsetstrokecolor{currentstroke}%
\pgfsetdash{}{0pt}%
\pgfpathmoveto{\pgfqpoint{0.000000in}{0.000000in}}%
\pgfpathlineto{\pgfqpoint{12.000000in}{0.000000in}}%
\pgfpathlineto{\pgfqpoint{12.000000in}{4.000000in}}%
\pgfpathlineto{\pgfqpoint{0.000000in}{4.000000in}}%
\pgfpathlineto{\pgfqpoint{0.000000in}{0.000000in}}%
\pgfpathclose%
\pgfusepath{fill}%
\end{pgfscope}%
\begin{pgfscope}%
\pgfsetbuttcap%
\pgfsetmiterjoin%
\definecolor{currentfill}{rgb}{1.000000,1.000000,1.000000}%
\pgfsetfillcolor{currentfill}%
\pgfsetlinewidth{0.000000pt}%
\definecolor{currentstroke}{rgb}{0.000000,0.000000,0.000000}%
\pgfsetstrokecolor{currentstroke}%
\pgfsetstrokeopacity{0.000000}%
\pgfsetdash{}{0pt}%
\pgfpathmoveto{\pgfqpoint{1.500000in}{0.440000in}}%
\pgfpathlineto{\pgfqpoint{5.727273in}{0.440000in}}%
\pgfpathlineto{\pgfqpoint{5.727273in}{3.520000in}}%
\pgfpathlineto{\pgfqpoint{1.500000in}{3.520000in}}%
\pgfpathlineto{\pgfqpoint{1.500000in}{0.440000in}}%
\pgfpathclose%
\pgfusepath{fill}%
\end{pgfscope}%
\begin{pgfscope}%
\pgfsetbuttcap%
\pgfsetroundjoin%
\definecolor{currentfill}{rgb}{0.000000,0.000000,0.000000}%
\pgfsetfillcolor{currentfill}%
\pgfsetlinewidth{0.803000pt}%
\definecolor{currentstroke}{rgb}{0.000000,0.000000,0.000000}%
\pgfsetstrokecolor{currentstroke}%
\pgfsetdash{}{0pt}%
\pgfsys@defobject{currentmarker}{\pgfqpoint{0.000000in}{-0.048611in}}{\pgfqpoint{0.000000in}{0.000000in}}{%
\pgfpathmoveto{\pgfqpoint{0.000000in}{0.000000in}}%
\pgfpathlineto{\pgfqpoint{0.000000in}{-0.048611in}}%
\pgfusepath{stroke,fill}%
}%
\begin{pgfscope}%
\pgfsys@transformshift{1.692149in}{0.440000in}%
\pgfsys@useobject{currentmarker}{}%
\end{pgfscope}%
\end{pgfscope}%
\begin{pgfscope}%
\definecolor{textcolor}{rgb}{0.000000,0.000000,0.000000}%
\pgfsetstrokecolor{textcolor}%
\pgfsetfillcolor{textcolor}%
\pgftext[x=1.692149in,y=0.342778in,,top]{\color{textcolor}\rmfamily\fontsize{11.000000}{13.200000}\selectfont 1}%
\end{pgfscope}%
\begin{pgfscope}%
\pgfsetbuttcap%
\pgfsetroundjoin%
\definecolor{currentfill}{rgb}{0.000000,0.000000,0.000000}%
\pgfsetfillcolor{currentfill}%
\pgfsetlinewidth{0.803000pt}%
\definecolor{currentstroke}{rgb}{0.000000,0.000000,0.000000}%
\pgfsetstrokecolor{currentstroke}%
\pgfsetdash{}{0pt}%
\pgfsys@defobject{currentmarker}{\pgfqpoint{0.000000in}{-0.048611in}}{\pgfqpoint{0.000000in}{0.000000in}}{%
\pgfpathmoveto{\pgfqpoint{0.000000in}{0.000000in}}%
\pgfpathlineto{\pgfqpoint{0.000000in}{-0.048611in}}%
\pgfusepath{stroke,fill}%
}%
\begin{pgfscope}%
\pgfsys@transformshift{2.652893in}{0.440000in}%
\pgfsys@useobject{currentmarker}{}%
\end{pgfscope}%
\end{pgfscope}%
\begin{pgfscope}%
\definecolor{textcolor}{rgb}{0.000000,0.000000,0.000000}%
\pgfsetstrokecolor{textcolor}%
\pgfsetfillcolor{textcolor}%
\pgftext[x=2.652893in,y=0.342778in,,top]{\color{textcolor}\rmfamily\fontsize{11.000000}{13.200000}\selectfont 2}%
\end{pgfscope}%
\begin{pgfscope}%
\pgfsetbuttcap%
\pgfsetroundjoin%
\definecolor{currentfill}{rgb}{0.000000,0.000000,0.000000}%
\pgfsetfillcolor{currentfill}%
\pgfsetlinewidth{0.803000pt}%
\definecolor{currentstroke}{rgb}{0.000000,0.000000,0.000000}%
\pgfsetstrokecolor{currentstroke}%
\pgfsetdash{}{0pt}%
\pgfsys@defobject{currentmarker}{\pgfqpoint{0.000000in}{-0.048611in}}{\pgfqpoint{0.000000in}{0.000000in}}{%
\pgfpathmoveto{\pgfqpoint{0.000000in}{0.000000in}}%
\pgfpathlineto{\pgfqpoint{0.000000in}{-0.048611in}}%
\pgfusepath{stroke,fill}%
}%
\begin{pgfscope}%
\pgfsys@transformshift{3.613636in}{0.440000in}%
\pgfsys@useobject{currentmarker}{}%
\end{pgfscope}%
\end{pgfscope}%
\begin{pgfscope}%
\definecolor{textcolor}{rgb}{0.000000,0.000000,0.000000}%
\pgfsetstrokecolor{textcolor}%
\pgfsetfillcolor{textcolor}%
\pgftext[x=3.613636in,y=0.342778in,,top]{\color{textcolor}\rmfamily\fontsize{11.000000}{13.200000}\selectfont 3}%
\end{pgfscope}%
\begin{pgfscope}%
\pgfsetbuttcap%
\pgfsetroundjoin%
\definecolor{currentfill}{rgb}{0.000000,0.000000,0.000000}%
\pgfsetfillcolor{currentfill}%
\pgfsetlinewidth{0.803000pt}%
\definecolor{currentstroke}{rgb}{0.000000,0.000000,0.000000}%
\pgfsetstrokecolor{currentstroke}%
\pgfsetdash{}{0pt}%
\pgfsys@defobject{currentmarker}{\pgfqpoint{0.000000in}{-0.048611in}}{\pgfqpoint{0.000000in}{0.000000in}}{%
\pgfpathmoveto{\pgfqpoint{0.000000in}{0.000000in}}%
\pgfpathlineto{\pgfqpoint{0.000000in}{-0.048611in}}%
\pgfusepath{stroke,fill}%
}%
\begin{pgfscope}%
\pgfsys@transformshift{4.574380in}{0.440000in}%
\pgfsys@useobject{currentmarker}{}%
\end{pgfscope}%
\end{pgfscope}%
\begin{pgfscope}%
\definecolor{textcolor}{rgb}{0.000000,0.000000,0.000000}%
\pgfsetstrokecolor{textcolor}%
\pgfsetfillcolor{textcolor}%
\pgftext[x=4.574380in,y=0.342778in,,top]{\color{textcolor}\rmfamily\fontsize{11.000000}{13.200000}\selectfont 4}%
\end{pgfscope}%
\begin{pgfscope}%
\pgfsetbuttcap%
\pgfsetroundjoin%
\definecolor{currentfill}{rgb}{0.000000,0.000000,0.000000}%
\pgfsetfillcolor{currentfill}%
\pgfsetlinewidth{0.803000pt}%
\definecolor{currentstroke}{rgb}{0.000000,0.000000,0.000000}%
\pgfsetstrokecolor{currentstroke}%
\pgfsetdash{}{0pt}%
\pgfsys@defobject{currentmarker}{\pgfqpoint{0.000000in}{-0.048611in}}{\pgfqpoint{0.000000in}{0.000000in}}{%
\pgfpathmoveto{\pgfqpoint{0.000000in}{0.000000in}}%
\pgfpathlineto{\pgfqpoint{0.000000in}{-0.048611in}}%
\pgfusepath{stroke,fill}%
}%
\begin{pgfscope}%
\pgfsys@transformshift{5.535124in}{0.440000in}%
\pgfsys@useobject{currentmarker}{}%
\end{pgfscope}%
\end{pgfscope}%
\begin{pgfscope}%
\definecolor{textcolor}{rgb}{0.000000,0.000000,0.000000}%
\pgfsetstrokecolor{textcolor}%
\pgfsetfillcolor{textcolor}%
\pgftext[x=5.535124in,y=0.342778in,,top]{\color{textcolor}\rmfamily\fontsize{11.000000}{13.200000}\selectfont 5}%
\end{pgfscope}%
\begin{pgfscope}%
\definecolor{textcolor}{rgb}{0.000000,0.000000,0.000000}%
\pgfsetstrokecolor{textcolor}%
\pgfsetfillcolor{textcolor}%
\pgftext[x=3.613636in,y=0.139368in,,top]{\color{textcolor}\rmfamily\fontsize{11.000000}{13.200000}\selectfont Degree}%
\end{pgfscope}%
\begin{pgfscope}%
\pgfsetbuttcap%
\pgfsetroundjoin%
\definecolor{currentfill}{rgb}{0.000000,0.000000,0.000000}%
\pgfsetfillcolor{currentfill}%
\pgfsetlinewidth{0.803000pt}%
\definecolor{currentstroke}{rgb}{0.000000,0.000000,0.000000}%
\pgfsetstrokecolor{currentstroke}%
\pgfsetdash{}{0pt}%
\pgfsys@defobject{currentmarker}{\pgfqpoint{-0.048611in}{0.000000in}}{\pgfqpoint{-0.000000in}{0.000000in}}{%
\pgfpathmoveto{\pgfqpoint{-0.000000in}{0.000000in}}%
\pgfpathlineto{\pgfqpoint{-0.048611in}{0.000000in}}%
\pgfusepath{stroke,fill}%
}%
\begin{pgfscope}%
\pgfsys@transformshift{1.500000in}{1.013134in}%
\pgfsys@useobject{currentmarker}{}%
\end{pgfscope}%
\end{pgfscope}%
\begin{pgfscope}%
\definecolor{textcolor}{rgb}{0.000000,0.000000,0.000000}%
\pgfsetstrokecolor{textcolor}%
\pgfsetfillcolor{textcolor}%
\pgftext[x=0.965407in, y=0.955096in, left, base]{\color{textcolor}\rmfamily\fontsize{11.000000}{13.200000}\selectfont 0.015}%
\end{pgfscope}%
\begin{pgfscope}%
\pgfsetbuttcap%
\pgfsetroundjoin%
\definecolor{currentfill}{rgb}{0.000000,0.000000,0.000000}%
\pgfsetfillcolor{currentfill}%
\pgfsetlinewidth{0.803000pt}%
\definecolor{currentstroke}{rgb}{0.000000,0.000000,0.000000}%
\pgfsetstrokecolor{currentstroke}%
\pgfsetdash{}{0pt}%
\pgfsys@defobject{currentmarker}{\pgfqpoint{-0.048611in}{0.000000in}}{\pgfqpoint{-0.000000in}{0.000000in}}{%
\pgfpathmoveto{\pgfqpoint{-0.000000in}{0.000000in}}%
\pgfpathlineto{\pgfqpoint{-0.048611in}{0.000000in}}%
\pgfusepath{stroke,fill}%
}%
\begin{pgfscope}%
\pgfsys@transformshift{1.500000in}{1.669178in}%
\pgfsys@useobject{currentmarker}{}%
\end{pgfscope}%
\end{pgfscope}%
\begin{pgfscope}%
\definecolor{textcolor}{rgb}{0.000000,0.000000,0.000000}%
\pgfsetstrokecolor{textcolor}%
\pgfsetfillcolor{textcolor}%
\pgftext[x=0.965407in, y=1.611141in, left, base]{\color{textcolor}\rmfamily\fontsize{11.000000}{13.200000}\selectfont 0.020}%
\end{pgfscope}%
\begin{pgfscope}%
\pgfsetbuttcap%
\pgfsetroundjoin%
\definecolor{currentfill}{rgb}{0.000000,0.000000,0.000000}%
\pgfsetfillcolor{currentfill}%
\pgfsetlinewidth{0.803000pt}%
\definecolor{currentstroke}{rgb}{0.000000,0.000000,0.000000}%
\pgfsetstrokecolor{currentstroke}%
\pgfsetdash{}{0pt}%
\pgfsys@defobject{currentmarker}{\pgfqpoint{-0.048611in}{0.000000in}}{\pgfqpoint{-0.000000in}{0.000000in}}{%
\pgfpathmoveto{\pgfqpoint{-0.000000in}{0.000000in}}%
\pgfpathlineto{\pgfqpoint{-0.048611in}{0.000000in}}%
\pgfusepath{stroke,fill}%
}%
\begin{pgfscope}%
\pgfsys@transformshift{1.500000in}{2.325223in}%
\pgfsys@useobject{currentmarker}{}%
\end{pgfscope}%
\end{pgfscope}%
\begin{pgfscope}%
\definecolor{textcolor}{rgb}{0.000000,0.000000,0.000000}%
\pgfsetstrokecolor{textcolor}%
\pgfsetfillcolor{textcolor}%
\pgftext[x=0.965407in, y=2.267185in, left, base]{\color{textcolor}\rmfamily\fontsize{11.000000}{13.200000}\selectfont 0.025}%
\end{pgfscope}%
\begin{pgfscope}%
\pgfsetbuttcap%
\pgfsetroundjoin%
\definecolor{currentfill}{rgb}{0.000000,0.000000,0.000000}%
\pgfsetfillcolor{currentfill}%
\pgfsetlinewidth{0.803000pt}%
\definecolor{currentstroke}{rgb}{0.000000,0.000000,0.000000}%
\pgfsetstrokecolor{currentstroke}%
\pgfsetdash{}{0pt}%
\pgfsys@defobject{currentmarker}{\pgfqpoint{-0.048611in}{0.000000in}}{\pgfqpoint{-0.000000in}{0.000000in}}{%
\pgfpathmoveto{\pgfqpoint{-0.000000in}{0.000000in}}%
\pgfpathlineto{\pgfqpoint{-0.048611in}{0.000000in}}%
\pgfusepath{stroke,fill}%
}%
\begin{pgfscope}%
\pgfsys@transformshift{1.500000in}{2.981267in}%
\pgfsys@useobject{currentmarker}{}%
\end{pgfscope}%
\end{pgfscope}%
\begin{pgfscope}%
\definecolor{textcolor}{rgb}{0.000000,0.000000,0.000000}%
\pgfsetstrokecolor{textcolor}%
\pgfsetfillcolor{textcolor}%
\pgftext[x=0.965407in, y=2.923229in, left, base]{\color{textcolor}\rmfamily\fontsize{11.000000}{13.200000}\selectfont 0.030}%
\end{pgfscope}%
\begin{pgfscope}%
\definecolor{textcolor}{rgb}{0.000000,0.000000,0.000000}%
\pgfsetstrokecolor{textcolor}%
\pgfsetfillcolor{textcolor}%
\pgftext[x=0.909851in,y=1.980000in,,bottom,rotate=90.000000]{\color{textcolor}\rmfamily\fontsize{11.000000}{13.200000}\selectfont MSE}%
\end{pgfscope}%
\begin{pgfscope}%
\pgfpathrectangle{\pgfqpoint{1.500000in}{0.440000in}}{\pgfqpoint{4.227273in}{3.080000in}}%
\pgfusepath{clip}%
\pgfsetrectcap%
\pgfsetroundjoin%
\pgfsetlinewidth{1.505625pt}%
\definecolor{currentstroke}{rgb}{0.121569,0.466667,0.705882}%
\pgfsetstrokecolor{currentstroke}%
\pgfsetdash{}{0pt}%
\pgfpathmoveto{\pgfqpoint{1.692149in}{3.348501in}}%
\pgfpathlineto{\pgfqpoint{2.652893in}{2.483313in}}%
\pgfpathlineto{\pgfqpoint{3.613636in}{1.351617in}}%
\pgfpathlineto{\pgfqpoint{4.574380in}{0.796099in}}%
\pgfpathlineto{\pgfqpoint{5.535124in}{0.580000in}}%
\pgfusepath{stroke}%
\end{pgfscope}%
\begin{pgfscope}%
\pgfpathrectangle{\pgfqpoint{1.500000in}{0.440000in}}{\pgfqpoint{4.227273in}{3.080000in}}%
\pgfusepath{clip}%
\pgfsetrectcap%
\pgfsetroundjoin%
\pgfsetlinewidth{1.505625pt}%
\definecolor{currentstroke}{rgb}{1.000000,0.498039,0.054902}%
\pgfsetstrokecolor{currentstroke}%
\pgfsetdash{}{0pt}%
\pgfpathmoveto{\pgfqpoint{1.692149in}{3.380000in}}%
\pgfpathlineto{\pgfqpoint{2.652893in}{2.582033in}}%
\pgfpathlineto{\pgfqpoint{3.613636in}{1.344324in}}%
\pgfpathlineto{\pgfqpoint{4.574380in}{0.854471in}}%
\pgfpathlineto{\pgfqpoint{5.535124in}{0.604043in}}%
\pgfusepath{stroke}%
\end{pgfscope}%
\begin{pgfscope}%
\pgfsetrectcap%
\pgfsetmiterjoin%
\pgfsetlinewidth{0.803000pt}%
\definecolor{currentstroke}{rgb}{0.000000,0.000000,0.000000}%
\pgfsetstrokecolor{currentstroke}%
\pgfsetdash{}{0pt}%
\pgfpathmoveto{\pgfqpoint{1.500000in}{0.440000in}}%
\pgfpathlineto{\pgfqpoint{1.500000in}{3.520000in}}%
\pgfusepath{stroke}%
\end{pgfscope}%
\begin{pgfscope}%
\pgfsetrectcap%
\pgfsetmiterjoin%
\pgfsetlinewidth{0.803000pt}%
\definecolor{currentstroke}{rgb}{0.000000,0.000000,0.000000}%
\pgfsetstrokecolor{currentstroke}%
\pgfsetdash{}{0pt}%
\pgfpathmoveto{\pgfqpoint{5.727273in}{0.440000in}}%
\pgfpathlineto{\pgfqpoint{5.727273in}{3.520000in}}%
\pgfusepath{stroke}%
\end{pgfscope}%
\begin{pgfscope}%
\pgfsetrectcap%
\pgfsetmiterjoin%
\pgfsetlinewidth{0.803000pt}%
\definecolor{currentstroke}{rgb}{0.000000,0.000000,0.000000}%
\pgfsetstrokecolor{currentstroke}%
\pgfsetdash{}{0pt}%
\pgfpathmoveto{\pgfqpoint{1.500000in}{0.440000in}}%
\pgfpathlineto{\pgfqpoint{5.727273in}{0.440000in}}%
\pgfusepath{stroke}%
\end{pgfscope}%
\begin{pgfscope}%
\pgfsetrectcap%
\pgfsetmiterjoin%
\pgfsetlinewidth{0.803000pt}%
\definecolor{currentstroke}{rgb}{0.000000,0.000000,0.000000}%
\pgfsetstrokecolor{currentstroke}%
\pgfsetdash{}{0pt}%
\pgfpathmoveto{\pgfqpoint{1.500000in}{3.520000in}}%
\pgfpathlineto{\pgfqpoint{5.727273in}{3.520000in}}%
\pgfusepath{stroke}%
\end{pgfscope}%
\begin{pgfscope}%
\definecolor{textcolor}{rgb}{0.000000,0.000000,0.000000}%
\pgfsetstrokecolor{textcolor}%
\pgfsetfillcolor{textcolor}%
\pgftext[x=3.613636in,y=3.603333in,,base]{\color{textcolor}\rmfamily\fontsize{13.200000}{15.840000}\selectfont MSE}%
\end{pgfscope}%
\begin{pgfscope}%
\pgfsetbuttcap%
\pgfsetmiterjoin%
\definecolor{currentfill}{rgb}{1.000000,1.000000,1.000000}%
\pgfsetfillcolor{currentfill}%
\pgfsetfillopacity{0.800000}%
\pgfsetlinewidth{1.003750pt}%
\definecolor{currentstroke}{rgb}{0.800000,0.800000,0.800000}%
\pgfsetstrokecolor{currentstroke}%
\pgfsetstrokeopacity{0.800000}%
\pgfsetdash{}{0pt}%
\pgfpathmoveto{\pgfqpoint{4.758671in}{2.949292in}}%
\pgfpathlineto{\pgfqpoint{5.620328in}{2.949292in}}%
\pgfpathquadraticcurveto{\pgfqpoint{5.650884in}{2.949292in}}{\pgfqpoint{5.650884in}{2.979847in}}%
\pgfpathlineto{\pgfqpoint{5.650884in}{3.413056in}}%
\pgfpathquadraticcurveto{\pgfqpoint{5.650884in}{3.443611in}}{\pgfqpoint{5.620328in}{3.443611in}}%
\pgfpathlineto{\pgfqpoint{4.758671in}{3.443611in}}%
\pgfpathquadraticcurveto{\pgfqpoint{4.728115in}{3.443611in}}{\pgfqpoint{4.728115in}{3.413056in}}%
\pgfpathlineto{\pgfqpoint{4.728115in}{2.979847in}}%
\pgfpathquadraticcurveto{\pgfqpoint{4.728115in}{2.949292in}}{\pgfqpoint{4.758671in}{2.949292in}}%
\pgfpathlineto{\pgfqpoint{4.758671in}{2.949292in}}%
\pgfpathclose%
\pgfusepath{stroke,fill}%
\end{pgfscope}%
\begin{pgfscope}%
\pgfsetrectcap%
\pgfsetroundjoin%
\pgfsetlinewidth{1.505625pt}%
\definecolor{currentstroke}{rgb}{0.121569,0.466667,0.705882}%
\pgfsetstrokecolor{currentstroke}%
\pgfsetdash{}{0pt}%
\pgfpathmoveto{\pgfqpoint{4.789226in}{3.319897in}}%
\pgfpathlineto{\pgfqpoint{4.942004in}{3.319897in}}%
\pgfpathlineto{\pgfqpoint{5.094782in}{3.319897in}}%
\pgfusepath{stroke}%
\end{pgfscope}%
\begin{pgfscope}%
\definecolor{textcolor}{rgb}{0.000000,0.000000,0.000000}%
\pgfsetstrokecolor{textcolor}%
\pgfsetfillcolor{textcolor}%
\pgftext[x=5.217004in,y=3.266425in,left,base]{\color{textcolor}\rmfamily\fontsize{11.000000}{13.200000}\selectfont test}%
\end{pgfscope}%
\begin{pgfscope}%
\pgfsetrectcap%
\pgfsetroundjoin%
\pgfsetlinewidth{1.505625pt}%
\definecolor{currentstroke}{rgb}{1.000000,0.498039,0.054902}%
\pgfsetstrokecolor{currentstroke}%
\pgfsetdash{}{0pt}%
\pgfpathmoveto{\pgfqpoint{4.789226in}{3.095654in}}%
\pgfpathlineto{\pgfqpoint{4.942004in}{3.095654in}}%
\pgfpathlineto{\pgfqpoint{5.094782in}{3.095654in}}%
\pgfusepath{stroke}%
\end{pgfscope}%
\begin{pgfscope}%
\definecolor{textcolor}{rgb}{0.000000,0.000000,0.000000}%
\pgfsetstrokecolor{textcolor}%
\pgfsetfillcolor{textcolor}%
\pgftext[x=5.217004in,y=3.042182in,left,base]{\color{textcolor}\rmfamily\fontsize{11.000000}{13.200000}\selectfont train}%
\end{pgfscope}%
\begin{pgfscope}%
\pgfsetbuttcap%
\pgfsetmiterjoin%
\definecolor{currentfill}{rgb}{1.000000,1.000000,1.000000}%
\pgfsetfillcolor{currentfill}%
\pgfsetlinewidth{0.000000pt}%
\definecolor{currentstroke}{rgb}{0.000000,0.000000,0.000000}%
\pgfsetstrokecolor{currentstroke}%
\pgfsetstrokeopacity{0.000000}%
\pgfsetdash{}{0pt}%
\pgfpathmoveto{\pgfqpoint{6.572727in}{0.440000in}}%
\pgfpathlineto{\pgfqpoint{10.800000in}{0.440000in}}%
\pgfpathlineto{\pgfqpoint{10.800000in}{3.520000in}}%
\pgfpathlineto{\pgfqpoint{6.572727in}{3.520000in}}%
\pgfpathlineto{\pgfqpoint{6.572727in}{0.440000in}}%
\pgfpathclose%
\pgfusepath{fill}%
\end{pgfscope}%
\begin{pgfscope}%
\pgfsetbuttcap%
\pgfsetroundjoin%
\definecolor{currentfill}{rgb}{0.000000,0.000000,0.000000}%
\pgfsetfillcolor{currentfill}%
\pgfsetlinewidth{0.803000pt}%
\definecolor{currentstroke}{rgb}{0.000000,0.000000,0.000000}%
\pgfsetstrokecolor{currentstroke}%
\pgfsetdash{}{0pt}%
\pgfsys@defobject{currentmarker}{\pgfqpoint{0.000000in}{-0.048611in}}{\pgfqpoint{0.000000in}{0.000000in}}{%
\pgfpathmoveto{\pgfqpoint{0.000000in}{0.000000in}}%
\pgfpathlineto{\pgfqpoint{0.000000in}{-0.048611in}}%
\pgfusepath{stroke,fill}%
}%
\begin{pgfscope}%
\pgfsys@transformshift{6.764876in}{0.440000in}%
\pgfsys@useobject{currentmarker}{}%
\end{pgfscope}%
\end{pgfscope}%
\begin{pgfscope}%
\definecolor{textcolor}{rgb}{0.000000,0.000000,0.000000}%
\pgfsetstrokecolor{textcolor}%
\pgfsetfillcolor{textcolor}%
\pgftext[x=6.764876in,y=0.342778in,,top]{\color{textcolor}\rmfamily\fontsize{11.000000}{13.200000}\selectfont 1}%
\end{pgfscope}%
\begin{pgfscope}%
\pgfsetbuttcap%
\pgfsetroundjoin%
\definecolor{currentfill}{rgb}{0.000000,0.000000,0.000000}%
\pgfsetfillcolor{currentfill}%
\pgfsetlinewidth{0.803000pt}%
\definecolor{currentstroke}{rgb}{0.000000,0.000000,0.000000}%
\pgfsetstrokecolor{currentstroke}%
\pgfsetdash{}{0pt}%
\pgfsys@defobject{currentmarker}{\pgfqpoint{0.000000in}{-0.048611in}}{\pgfqpoint{0.000000in}{0.000000in}}{%
\pgfpathmoveto{\pgfqpoint{0.000000in}{0.000000in}}%
\pgfpathlineto{\pgfqpoint{0.000000in}{-0.048611in}}%
\pgfusepath{stroke,fill}%
}%
\begin{pgfscope}%
\pgfsys@transformshift{7.725620in}{0.440000in}%
\pgfsys@useobject{currentmarker}{}%
\end{pgfscope}%
\end{pgfscope}%
\begin{pgfscope}%
\definecolor{textcolor}{rgb}{0.000000,0.000000,0.000000}%
\pgfsetstrokecolor{textcolor}%
\pgfsetfillcolor{textcolor}%
\pgftext[x=7.725620in,y=0.342778in,,top]{\color{textcolor}\rmfamily\fontsize{11.000000}{13.200000}\selectfont 2}%
\end{pgfscope}%
\begin{pgfscope}%
\pgfsetbuttcap%
\pgfsetroundjoin%
\definecolor{currentfill}{rgb}{0.000000,0.000000,0.000000}%
\pgfsetfillcolor{currentfill}%
\pgfsetlinewidth{0.803000pt}%
\definecolor{currentstroke}{rgb}{0.000000,0.000000,0.000000}%
\pgfsetstrokecolor{currentstroke}%
\pgfsetdash{}{0pt}%
\pgfsys@defobject{currentmarker}{\pgfqpoint{0.000000in}{-0.048611in}}{\pgfqpoint{0.000000in}{0.000000in}}{%
\pgfpathmoveto{\pgfqpoint{0.000000in}{0.000000in}}%
\pgfpathlineto{\pgfqpoint{0.000000in}{-0.048611in}}%
\pgfusepath{stroke,fill}%
}%
\begin{pgfscope}%
\pgfsys@transformshift{8.686364in}{0.440000in}%
\pgfsys@useobject{currentmarker}{}%
\end{pgfscope}%
\end{pgfscope}%
\begin{pgfscope}%
\definecolor{textcolor}{rgb}{0.000000,0.000000,0.000000}%
\pgfsetstrokecolor{textcolor}%
\pgfsetfillcolor{textcolor}%
\pgftext[x=8.686364in,y=0.342778in,,top]{\color{textcolor}\rmfamily\fontsize{11.000000}{13.200000}\selectfont 3}%
\end{pgfscope}%
\begin{pgfscope}%
\pgfsetbuttcap%
\pgfsetroundjoin%
\definecolor{currentfill}{rgb}{0.000000,0.000000,0.000000}%
\pgfsetfillcolor{currentfill}%
\pgfsetlinewidth{0.803000pt}%
\definecolor{currentstroke}{rgb}{0.000000,0.000000,0.000000}%
\pgfsetstrokecolor{currentstroke}%
\pgfsetdash{}{0pt}%
\pgfsys@defobject{currentmarker}{\pgfqpoint{0.000000in}{-0.048611in}}{\pgfqpoint{0.000000in}{0.000000in}}{%
\pgfpathmoveto{\pgfqpoint{0.000000in}{0.000000in}}%
\pgfpathlineto{\pgfqpoint{0.000000in}{-0.048611in}}%
\pgfusepath{stroke,fill}%
}%
\begin{pgfscope}%
\pgfsys@transformshift{9.647107in}{0.440000in}%
\pgfsys@useobject{currentmarker}{}%
\end{pgfscope}%
\end{pgfscope}%
\begin{pgfscope}%
\definecolor{textcolor}{rgb}{0.000000,0.000000,0.000000}%
\pgfsetstrokecolor{textcolor}%
\pgfsetfillcolor{textcolor}%
\pgftext[x=9.647107in,y=0.342778in,,top]{\color{textcolor}\rmfamily\fontsize{11.000000}{13.200000}\selectfont 4}%
\end{pgfscope}%
\begin{pgfscope}%
\pgfsetbuttcap%
\pgfsetroundjoin%
\definecolor{currentfill}{rgb}{0.000000,0.000000,0.000000}%
\pgfsetfillcolor{currentfill}%
\pgfsetlinewidth{0.803000pt}%
\definecolor{currentstroke}{rgb}{0.000000,0.000000,0.000000}%
\pgfsetstrokecolor{currentstroke}%
\pgfsetdash{}{0pt}%
\pgfsys@defobject{currentmarker}{\pgfqpoint{0.000000in}{-0.048611in}}{\pgfqpoint{0.000000in}{0.000000in}}{%
\pgfpathmoveto{\pgfqpoint{0.000000in}{0.000000in}}%
\pgfpathlineto{\pgfqpoint{0.000000in}{-0.048611in}}%
\pgfusepath{stroke,fill}%
}%
\begin{pgfscope}%
\pgfsys@transformshift{10.607851in}{0.440000in}%
\pgfsys@useobject{currentmarker}{}%
\end{pgfscope}%
\end{pgfscope}%
\begin{pgfscope}%
\definecolor{textcolor}{rgb}{0.000000,0.000000,0.000000}%
\pgfsetstrokecolor{textcolor}%
\pgfsetfillcolor{textcolor}%
\pgftext[x=10.607851in,y=0.342778in,,top]{\color{textcolor}\rmfamily\fontsize{11.000000}{13.200000}\selectfont 5}%
\end{pgfscope}%
\begin{pgfscope}%
\definecolor{textcolor}{rgb}{0.000000,0.000000,0.000000}%
\pgfsetstrokecolor{textcolor}%
\pgfsetfillcolor{textcolor}%
\pgftext[x=8.686364in,y=0.139368in,,top]{\color{textcolor}\rmfamily\fontsize{11.000000}{13.200000}\selectfont Degree}%
\end{pgfscope}%
\begin{pgfscope}%
\pgfsetbuttcap%
\pgfsetroundjoin%
\definecolor{currentfill}{rgb}{0.000000,0.000000,0.000000}%
\pgfsetfillcolor{currentfill}%
\pgfsetlinewidth{0.803000pt}%
\definecolor{currentstroke}{rgb}{0.000000,0.000000,0.000000}%
\pgfsetstrokecolor{currentstroke}%
\pgfsetdash{}{0pt}%
\pgfsys@defobject{currentmarker}{\pgfqpoint{-0.048611in}{0.000000in}}{\pgfqpoint{-0.000000in}{0.000000in}}{%
\pgfpathmoveto{\pgfqpoint{-0.000000in}{0.000000in}}%
\pgfpathlineto{\pgfqpoint{-0.048611in}{0.000000in}}%
\pgfusepath{stroke,fill}%
}%
\begin{pgfscope}%
\pgfsys@transformshift{6.572727in}{0.719958in}%
\pgfsys@useobject{currentmarker}{}%
\end{pgfscope}%
\end{pgfscope}%
\begin{pgfscope}%
\definecolor{textcolor}{rgb}{0.000000,0.000000,0.000000}%
\pgfsetstrokecolor{textcolor}%
\pgfsetfillcolor{textcolor}%
\pgftext[x=6.135336in, y=0.661921in, left, base]{\color{textcolor}\rmfamily\fontsize{11.000000}{13.200000}\selectfont 0.65}%
\end{pgfscope}%
\begin{pgfscope}%
\pgfsetbuttcap%
\pgfsetroundjoin%
\definecolor{currentfill}{rgb}{0.000000,0.000000,0.000000}%
\pgfsetfillcolor{currentfill}%
\pgfsetlinewidth{0.803000pt}%
\definecolor{currentstroke}{rgb}{0.000000,0.000000,0.000000}%
\pgfsetstrokecolor{currentstroke}%
\pgfsetdash{}{0pt}%
\pgfsys@defobject{currentmarker}{\pgfqpoint{-0.048611in}{0.000000in}}{\pgfqpoint{-0.000000in}{0.000000in}}{%
\pgfpathmoveto{\pgfqpoint{-0.000000in}{0.000000in}}%
\pgfpathlineto{\pgfqpoint{-0.048611in}{0.000000in}}%
\pgfusepath{stroke,fill}%
}%
\begin{pgfscope}%
\pgfsys@transformshift{6.572727in}{1.310746in}%
\pgfsys@useobject{currentmarker}{}%
\end{pgfscope}%
\end{pgfscope}%
\begin{pgfscope}%
\definecolor{textcolor}{rgb}{0.000000,0.000000,0.000000}%
\pgfsetstrokecolor{textcolor}%
\pgfsetfillcolor{textcolor}%
\pgftext[x=6.135336in, y=1.252709in, left, base]{\color{textcolor}\rmfamily\fontsize{11.000000}{13.200000}\selectfont 0.70}%
\end{pgfscope}%
\begin{pgfscope}%
\pgfsetbuttcap%
\pgfsetroundjoin%
\definecolor{currentfill}{rgb}{0.000000,0.000000,0.000000}%
\pgfsetfillcolor{currentfill}%
\pgfsetlinewidth{0.803000pt}%
\definecolor{currentstroke}{rgb}{0.000000,0.000000,0.000000}%
\pgfsetstrokecolor{currentstroke}%
\pgfsetdash{}{0pt}%
\pgfsys@defobject{currentmarker}{\pgfqpoint{-0.048611in}{0.000000in}}{\pgfqpoint{-0.000000in}{0.000000in}}{%
\pgfpathmoveto{\pgfqpoint{-0.000000in}{0.000000in}}%
\pgfpathlineto{\pgfqpoint{-0.048611in}{0.000000in}}%
\pgfusepath{stroke,fill}%
}%
\begin{pgfscope}%
\pgfsys@transformshift{6.572727in}{1.901534in}%
\pgfsys@useobject{currentmarker}{}%
\end{pgfscope}%
\end{pgfscope}%
\begin{pgfscope}%
\definecolor{textcolor}{rgb}{0.000000,0.000000,0.000000}%
\pgfsetstrokecolor{textcolor}%
\pgfsetfillcolor{textcolor}%
\pgftext[x=6.135336in, y=1.843497in, left, base]{\color{textcolor}\rmfamily\fontsize{11.000000}{13.200000}\selectfont 0.75}%
\end{pgfscope}%
\begin{pgfscope}%
\pgfsetbuttcap%
\pgfsetroundjoin%
\definecolor{currentfill}{rgb}{0.000000,0.000000,0.000000}%
\pgfsetfillcolor{currentfill}%
\pgfsetlinewidth{0.803000pt}%
\definecolor{currentstroke}{rgb}{0.000000,0.000000,0.000000}%
\pgfsetstrokecolor{currentstroke}%
\pgfsetdash{}{0pt}%
\pgfsys@defobject{currentmarker}{\pgfqpoint{-0.048611in}{0.000000in}}{\pgfqpoint{-0.000000in}{0.000000in}}{%
\pgfpathmoveto{\pgfqpoint{-0.000000in}{0.000000in}}%
\pgfpathlineto{\pgfqpoint{-0.048611in}{0.000000in}}%
\pgfusepath{stroke,fill}%
}%
\begin{pgfscope}%
\pgfsys@transformshift{6.572727in}{2.492323in}%
\pgfsys@useobject{currentmarker}{}%
\end{pgfscope}%
\end{pgfscope}%
\begin{pgfscope}%
\definecolor{textcolor}{rgb}{0.000000,0.000000,0.000000}%
\pgfsetstrokecolor{textcolor}%
\pgfsetfillcolor{textcolor}%
\pgftext[x=6.135336in, y=2.434285in, left, base]{\color{textcolor}\rmfamily\fontsize{11.000000}{13.200000}\selectfont 0.80}%
\end{pgfscope}%
\begin{pgfscope}%
\pgfsetbuttcap%
\pgfsetroundjoin%
\definecolor{currentfill}{rgb}{0.000000,0.000000,0.000000}%
\pgfsetfillcolor{currentfill}%
\pgfsetlinewidth{0.803000pt}%
\definecolor{currentstroke}{rgb}{0.000000,0.000000,0.000000}%
\pgfsetstrokecolor{currentstroke}%
\pgfsetdash{}{0pt}%
\pgfsys@defobject{currentmarker}{\pgfqpoint{-0.048611in}{0.000000in}}{\pgfqpoint{-0.000000in}{0.000000in}}{%
\pgfpathmoveto{\pgfqpoint{-0.000000in}{0.000000in}}%
\pgfpathlineto{\pgfqpoint{-0.048611in}{0.000000in}}%
\pgfusepath{stroke,fill}%
}%
\begin{pgfscope}%
\pgfsys@transformshift{6.572727in}{3.083111in}%
\pgfsys@useobject{currentmarker}{}%
\end{pgfscope}%
\end{pgfscope}%
\begin{pgfscope}%
\definecolor{textcolor}{rgb}{0.000000,0.000000,0.000000}%
\pgfsetstrokecolor{textcolor}%
\pgfsetfillcolor{textcolor}%
\pgftext[x=6.135336in, y=3.025073in, left, base]{\color{textcolor}\rmfamily\fontsize{11.000000}{13.200000}\selectfont 0.85}%
\end{pgfscope}%
\begin{pgfscope}%
\definecolor{textcolor}{rgb}{0.000000,0.000000,0.000000}%
\pgfsetstrokecolor{textcolor}%
\pgfsetfillcolor{textcolor}%
\pgftext[x=6.079780in,y=1.980000in,,bottom,rotate=90.000000]{\color{textcolor}\rmfamily\fontsize{11.000000}{13.200000}\selectfont R2}%
\end{pgfscope}%
\begin{pgfscope}%
\pgfpathrectangle{\pgfqpoint{6.572727in}{0.440000in}}{\pgfqpoint{4.227273in}{3.080000in}}%
\pgfusepath{clip}%
\pgfsetrectcap%
\pgfsetroundjoin%
\pgfsetlinewidth{1.505625pt}%
\definecolor{currentstroke}{rgb}{0.121569,0.466667,0.705882}%
\pgfsetstrokecolor{currentstroke}%
\pgfsetdash{}{0pt}%
\pgfpathmoveto{\pgfqpoint{6.764876in}{0.718372in}}%
\pgfpathlineto{\pgfqpoint{7.725620in}{1.549843in}}%
\pgfpathlineto{\pgfqpoint{8.686364in}{2.637890in}}%
\pgfpathlineto{\pgfqpoint{9.647107in}{3.172234in}}%
\pgfpathlineto{\pgfqpoint{10.607851in}{3.380000in}}%
\pgfusepath{stroke}%
\end{pgfscope}%
\begin{pgfscope}%
\pgfpathrectangle{\pgfqpoint{6.572727in}{0.440000in}}{\pgfqpoint{4.227273in}{3.080000in}}%
\pgfusepath{clip}%
\pgfsetrectcap%
\pgfsetroundjoin%
\pgfsetlinewidth{1.505625pt}%
\definecolor{currentstroke}{rgb}{1.000000,0.498039,0.054902}%
\pgfsetstrokecolor{currentstroke}%
\pgfsetdash{}{0pt}%
\pgfpathmoveto{\pgfqpoint{6.764876in}{0.580000in}}%
\pgfpathlineto{\pgfqpoint{7.725620in}{1.367010in}}%
\pgfpathlineto{\pgfqpoint{8.686364in}{2.587724in}}%
\pgfpathlineto{\pgfqpoint{9.647107in}{3.070851in}}%
\pgfpathlineto{\pgfqpoint{10.607851in}{3.317840in}}%
\pgfusepath{stroke}%
\end{pgfscope}%
\begin{pgfscope}%
\pgfsetrectcap%
\pgfsetmiterjoin%
\pgfsetlinewidth{0.803000pt}%
\definecolor{currentstroke}{rgb}{0.000000,0.000000,0.000000}%
\pgfsetstrokecolor{currentstroke}%
\pgfsetdash{}{0pt}%
\pgfpathmoveto{\pgfqpoint{6.572727in}{0.440000in}}%
\pgfpathlineto{\pgfqpoint{6.572727in}{3.520000in}}%
\pgfusepath{stroke}%
\end{pgfscope}%
\begin{pgfscope}%
\pgfsetrectcap%
\pgfsetmiterjoin%
\pgfsetlinewidth{0.803000pt}%
\definecolor{currentstroke}{rgb}{0.000000,0.000000,0.000000}%
\pgfsetstrokecolor{currentstroke}%
\pgfsetdash{}{0pt}%
\pgfpathmoveto{\pgfqpoint{10.800000in}{0.440000in}}%
\pgfpathlineto{\pgfqpoint{10.800000in}{3.520000in}}%
\pgfusepath{stroke}%
\end{pgfscope}%
\begin{pgfscope}%
\pgfsetrectcap%
\pgfsetmiterjoin%
\pgfsetlinewidth{0.803000pt}%
\definecolor{currentstroke}{rgb}{0.000000,0.000000,0.000000}%
\pgfsetstrokecolor{currentstroke}%
\pgfsetdash{}{0pt}%
\pgfpathmoveto{\pgfqpoint{6.572727in}{0.440000in}}%
\pgfpathlineto{\pgfqpoint{10.800000in}{0.440000in}}%
\pgfusepath{stroke}%
\end{pgfscope}%
\begin{pgfscope}%
\pgfsetrectcap%
\pgfsetmiterjoin%
\pgfsetlinewidth{0.803000pt}%
\definecolor{currentstroke}{rgb}{0.000000,0.000000,0.000000}%
\pgfsetstrokecolor{currentstroke}%
\pgfsetdash{}{0pt}%
\pgfpathmoveto{\pgfqpoint{6.572727in}{3.520000in}}%
\pgfpathlineto{\pgfqpoint{10.800000in}{3.520000in}}%
\pgfusepath{stroke}%
\end{pgfscope}%
\begin{pgfscope}%
\definecolor{textcolor}{rgb}{0.000000,0.000000,0.000000}%
\pgfsetstrokecolor{textcolor}%
\pgfsetfillcolor{textcolor}%
\pgftext[x=8.686364in,y=3.603333in,,base]{\color{textcolor}\rmfamily\fontsize{13.200000}{15.840000}\selectfont R2}%
\end{pgfscope}%
\begin{pgfscope}%
\pgfsetbuttcap%
\pgfsetmiterjoin%
\definecolor{currentfill}{rgb}{1.000000,1.000000,1.000000}%
\pgfsetfillcolor{currentfill}%
\pgfsetfillopacity{0.800000}%
\pgfsetlinewidth{1.003750pt}%
\definecolor{currentstroke}{rgb}{0.800000,0.800000,0.800000}%
\pgfsetstrokecolor{currentstroke}%
\pgfsetstrokeopacity{0.800000}%
\pgfsetdash{}{0pt}%
\pgfpathmoveto{\pgfqpoint{6.679672in}{2.949292in}}%
\pgfpathlineto{\pgfqpoint{7.541329in}{2.949292in}}%
\pgfpathquadraticcurveto{\pgfqpoint{7.571885in}{2.949292in}}{\pgfqpoint{7.571885in}{2.979847in}}%
\pgfpathlineto{\pgfqpoint{7.571885in}{3.413056in}}%
\pgfpathquadraticcurveto{\pgfqpoint{7.571885in}{3.443611in}}{\pgfqpoint{7.541329in}{3.443611in}}%
\pgfpathlineto{\pgfqpoint{6.679672in}{3.443611in}}%
\pgfpathquadraticcurveto{\pgfqpoint{6.649116in}{3.443611in}}{\pgfqpoint{6.649116in}{3.413056in}}%
\pgfpathlineto{\pgfqpoint{6.649116in}{2.979847in}}%
\pgfpathquadraticcurveto{\pgfqpoint{6.649116in}{2.949292in}}{\pgfqpoint{6.679672in}{2.949292in}}%
\pgfpathlineto{\pgfqpoint{6.679672in}{2.949292in}}%
\pgfpathclose%
\pgfusepath{stroke,fill}%
\end{pgfscope}%
\begin{pgfscope}%
\pgfsetrectcap%
\pgfsetroundjoin%
\pgfsetlinewidth{1.505625pt}%
\definecolor{currentstroke}{rgb}{0.121569,0.466667,0.705882}%
\pgfsetstrokecolor{currentstroke}%
\pgfsetdash{}{0pt}%
\pgfpathmoveto{\pgfqpoint{6.710227in}{3.319897in}}%
\pgfpathlineto{\pgfqpoint{6.863005in}{3.319897in}}%
\pgfpathlineto{\pgfqpoint{7.015783in}{3.319897in}}%
\pgfusepath{stroke}%
\end{pgfscope}%
\begin{pgfscope}%
\definecolor{textcolor}{rgb}{0.000000,0.000000,0.000000}%
\pgfsetstrokecolor{textcolor}%
\pgfsetfillcolor{textcolor}%
\pgftext[x=7.138005in,y=3.266425in,left,base]{\color{textcolor}\rmfamily\fontsize{11.000000}{13.200000}\selectfont test}%
\end{pgfscope}%
\begin{pgfscope}%
\pgfsetrectcap%
\pgfsetroundjoin%
\pgfsetlinewidth{1.505625pt}%
\definecolor{currentstroke}{rgb}{1.000000,0.498039,0.054902}%
\pgfsetstrokecolor{currentstroke}%
\pgfsetdash{}{0pt}%
\pgfpathmoveto{\pgfqpoint{6.710227in}{3.095654in}}%
\pgfpathlineto{\pgfqpoint{6.863005in}{3.095654in}}%
\pgfpathlineto{\pgfqpoint{7.015783in}{3.095654in}}%
\pgfusepath{stroke}%
\end{pgfscope}%
\begin{pgfscope}%
\definecolor{textcolor}{rgb}{0.000000,0.000000,0.000000}%
\pgfsetstrokecolor{textcolor}%
\pgfsetfillcolor{textcolor}%
\pgftext[x=7.138005in,y=3.042182in,left,base]{\color{textcolor}\rmfamily\fontsize{11.000000}{13.200000}\selectfont train}%
\end{pgfscope}%
\end{pgfpicture}%
\makeatother%
\endgroup%
}
%     \caption{Franke's Function}
%     \label{ols}
%     {\scriptsize code for gen  output}\footnote{code for }
%     \end{center}
% \end{figure}
% \begin{figure}[!h]
%     \begin{center}
%     \resizebox*{0.7\linewidth}{!}{%% Creator: Matplotlib, PGF backend
%%
%% To include the figure in your LaTeX document, write
%%   \input{<filename>.pgf}
%%
%% Make sure the required packages are loaded in your preamble
%%   \usepackage{pgf}
%%
%% Also ensure that all the required font packages are loaded; for instance,
%% the lmodern package is sometimes necessary when using math font.
%%   \usepackage{lmodern}
%%
%% Figures using additional raster images can only be included by \input if
%% they are in the same directory as the main LaTeX file. For loading figures
%% from other directories you can use the `import` package
%%   \usepackage{import}
%%
%% and then include the figures with
%%   \import{<path to file>}{<filename>.pgf}
%%
%% Matplotlib used the following preamble
%%   
%%   \usepackage{fontspec}
%%   \setmainfont{DejaVuSerif.ttf}[Path=\detokenize{/home/brage/anaconda3/lib/python3.10/site-packages/matplotlib/mpl-data/fonts/ttf/}]
%%   \setsansfont{DejaVuSans.ttf}[Path=\detokenize{/home/brage/anaconda3/lib/python3.10/site-packages/matplotlib/mpl-data/fonts/ttf/}]
%%   \setmonofont{DejaVuSansMono.ttf}[Path=\detokenize{/home/brage/anaconda3/lib/python3.10/site-packages/matplotlib/mpl-data/fonts/ttf/}]
%%   \makeatletter\@ifpackageloaded{underscore}{}{\usepackage[strings]{underscore}}\makeatother
%%
\begingroup%
\makeatletter%
\begin{pgfpicture}%
\pgfpathrectangle{\pgfpointorigin}{\pgfqpoint{6.400000in}{4.800000in}}%
\pgfusepath{use as bounding box, clip}%
\begin{pgfscope}%
\pgfsetbuttcap%
\pgfsetmiterjoin%
\definecolor{currentfill}{rgb}{1.000000,1.000000,1.000000}%
\pgfsetfillcolor{currentfill}%
\pgfsetlinewidth{0.000000pt}%
\definecolor{currentstroke}{rgb}{1.000000,1.000000,1.000000}%
\pgfsetstrokecolor{currentstroke}%
\pgfsetdash{}{0pt}%
\pgfpathmoveto{\pgfqpoint{0.000000in}{0.000000in}}%
\pgfpathlineto{\pgfqpoint{6.400000in}{0.000000in}}%
\pgfpathlineto{\pgfqpoint{6.400000in}{4.800000in}}%
\pgfpathlineto{\pgfqpoint{0.000000in}{4.800000in}}%
\pgfpathlineto{\pgfqpoint{0.000000in}{0.000000in}}%
\pgfpathclose%
\pgfusepath{fill}%
\end{pgfscope}%
\begin{pgfscope}%
\pgfsetbuttcap%
\pgfsetmiterjoin%
\definecolor{currentfill}{rgb}{1.000000,1.000000,1.000000}%
\pgfsetfillcolor{currentfill}%
\pgfsetlinewidth{0.000000pt}%
\definecolor{currentstroke}{rgb}{0.000000,0.000000,0.000000}%
\pgfsetstrokecolor{currentstroke}%
\pgfsetstrokeopacity{0.000000}%
\pgfsetdash{}{0pt}%
\pgfpathmoveto{\pgfqpoint{0.800000in}{0.528000in}}%
\pgfpathlineto{\pgfqpoint{5.760000in}{0.528000in}}%
\pgfpathlineto{\pgfqpoint{5.760000in}{4.224000in}}%
\pgfpathlineto{\pgfqpoint{0.800000in}{4.224000in}}%
\pgfpathlineto{\pgfqpoint{0.800000in}{0.528000in}}%
\pgfpathclose%
\pgfusepath{fill}%
\end{pgfscope}%
\begin{pgfscope}%
\pgfsetbuttcap%
\pgfsetroundjoin%
\definecolor{currentfill}{rgb}{0.000000,0.000000,0.000000}%
\pgfsetfillcolor{currentfill}%
\pgfsetlinewidth{0.803000pt}%
\definecolor{currentstroke}{rgb}{0.000000,0.000000,0.000000}%
\pgfsetstrokecolor{currentstroke}%
\pgfsetdash{}{0pt}%
\pgfsys@defobject{currentmarker}{\pgfqpoint{0.000000in}{-0.048611in}}{\pgfqpoint{0.000000in}{0.000000in}}{%
\pgfpathmoveto{\pgfqpoint{0.000000in}{0.000000in}}%
\pgfpathlineto{\pgfqpoint{0.000000in}{-0.048611in}}%
\pgfusepath{stroke,fill}%
}%
\begin{pgfscope}%
\pgfsys@transformshift{1.025455in}{0.528000in}%
\pgfsys@useobject{currentmarker}{}%
\end{pgfscope}%
\end{pgfscope}%
\begin{pgfscope}%
\definecolor{textcolor}{rgb}{0.000000,0.000000,0.000000}%
\pgfsetstrokecolor{textcolor}%
\pgfsetfillcolor{textcolor}%
\pgftext[x=1.025455in,y=0.430778in,,top]{\color{textcolor}\rmfamily\fontsize{11.000000}{13.200000}\selectfont 1}%
\end{pgfscope}%
\begin{pgfscope}%
\pgfsetbuttcap%
\pgfsetroundjoin%
\definecolor{currentfill}{rgb}{0.000000,0.000000,0.000000}%
\pgfsetfillcolor{currentfill}%
\pgfsetlinewidth{0.803000pt}%
\definecolor{currentstroke}{rgb}{0.000000,0.000000,0.000000}%
\pgfsetstrokecolor{currentstroke}%
\pgfsetdash{}{0pt}%
\pgfsys@defobject{currentmarker}{\pgfqpoint{0.000000in}{-0.048611in}}{\pgfqpoint{0.000000in}{0.000000in}}{%
\pgfpathmoveto{\pgfqpoint{0.000000in}{0.000000in}}%
\pgfpathlineto{\pgfqpoint{0.000000in}{-0.048611in}}%
\pgfusepath{stroke,fill}%
}%
\begin{pgfscope}%
\pgfsys@transformshift{2.152727in}{0.528000in}%
\pgfsys@useobject{currentmarker}{}%
\end{pgfscope}%
\end{pgfscope}%
\begin{pgfscope}%
\definecolor{textcolor}{rgb}{0.000000,0.000000,0.000000}%
\pgfsetstrokecolor{textcolor}%
\pgfsetfillcolor{textcolor}%
\pgftext[x=2.152727in,y=0.430778in,,top]{\color{textcolor}\rmfamily\fontsize{11.000000}{13.200000}\selectfont 2}%
\end{pgfscope}%
\begin{pgfscope}%
\pgfsetbuttcap%
\pgfsetroundjoin%
\definecolor{currentfill}{rgb}{0.000000,0.000000,0.000000}%
\pgfsetfillcolor{currentfill}%
\pgfsetlinewidth{0.803000pt}%
\definecolor{currentstroke}{rgb}{0.000000,0.000000,0.000000}%
\pgfsetstrokecolor{currentstroke}%
\pgfsetdash{}{0pt}%
\pgfsys@defobject{currentmarker}{\pgfqpoint{0.000000in}{-0.048611in}}{\pgfqpoint{0.000000in}{0.000000in}}{%
\pgfpathmoveto{\pgfqpoint{0.000000in}{0.000000in}}%
\pgfpathlineto{\pgfqpoint{0.000000in}{-0.048611in}}%
\pgfusepath{stroke,fill}%
}%
\begin{pgfscope}%
\pgfsys@transformshift{3.280000in}{0.528000in}%
\pgfsys@useobject{currentmarker}{}%
\end{pgfscope}%
\end{pgfscope}%
\begin{pgfscope}%
\definecolor{textcolor}{rgb}{0.000000,0.000000,0.000000}%
\pgfsetstrokecolor{textcolor}%
\pgfsetfillcolor{textcolor}%
\pgftext[x=3.280000in,y=0.430778in,,top]{\color{textcolor}\rmfamily\fontsize{11.000000}{13.200000}\selectfont 3}%
\end{pgfscope}%
\begin{pgfscope}%
\pgfsetbuttcap%
\pgfsetroundjoin%
\definecolor{currentfill}{rgb}{0.000000,0.000000,0.000000}%
\pgfsetfillcolor{currentfill}%
\pgfsetlinewidth{0.803000pt}%
\definecolor{currentstroke}{rgb}{0.000000,0.000000,0.000000}%
\pgfsetstrokecolor{currentstroke}%
\pgfsetdash{}{0pt}%
\pgfsys@defobject{currentmarker}{\pgfqpoint{0.000000in}{-0.048611in}}{\pgfqpoint{0.000000in}{0.000000in}}{%
\pgfpathmoveto{\pgfqpoint{0.000000in}{0.000000in}}%
\pgfpathlineto{\pgfqpoint{0.000000in}{-0.048611in}}%
\pgfusepath{stroke,fill}%
}%
\begin{pgfscope}%
\pgfsys@transformshift{4.407273in}{0.528000in}%
\pgfsys@useobject{currentmarker}{}%
\end{pgfscope}%
\end{pgfscope}%
\begin{pgfscope}%
\definecolor{textcolor}{rgb}{0.000000,0.000000,0.000000}%
\pgfsetstrokecolor{textcolor}%
\pgfsetfillcolor{textcolor}%
\pgftext[x=4.407273in,y=0.430778in,,top]{\color{textcolor}\rmfamily\fontsize{11.000000}{13.200000}\selectfont 4}%
\end{pgfscope}%
\begin{pgfscope}%
\pgfsetbuttcap%
\pgfsetroundjoin%
\definecolor{currentfill}{rgb}{0.000000,0.000000,0.000000}%
\pgfsetfillcolor{currentfill}%
\pgfsetlinewidth{0.803000pt}%
\definecolor{currentstroke}{rgb}{0.000000,0.000000,0.000000}%
\pgfsetstrokecolor{currentstroke}%
\pgfsetdash{}{0pt}%
\pgfsys@defobject{currentmarker}{\pgfqpoint{0.000000in}{-0.048611in}}{\pgfqpoint{0.000000in}{0.000000in}}{%
\pgfpathmoveto{\pgfqpoint{0.000000in}{0.000000in}}%
\pgfpathlineto{\pgfqpoint{0.000000in}{-0.048611in}}%
\pgfusepath{stroke,fill}%
}%
\begin{pgfscope}%
\pgfsys@transformshift{5.534545in}{0.528000in}%
\pgfsys@useobject{currentmarker}{}%
\end{pgfscope}%
\end{pgfscope}%
\begin{pgfscope}%
\definecolor{textcolor}{rgb}{0.000000,0.000000,0.000000}%
\pgfsetstrokecolor{textcolor}%
\pgfsetfillcolor{textcolor}%
\pgftext[x=5.534545in,y=0.430778in,,top]{\color{textcolor}\rmfamily\fontsize{11.000000}{13.200000}\selectfont 5}%
\end{pgfscope}%
\begin{pgfscope}%
\definecolor{textcolor}{rgb}{0.000000,0.000000,0.000000}%
\pgfsetstrokecolor{textcolor}%
\pgfsetfillcolor{textcolor}%
\pgftext[x=3.280000in,y=0.227368in,,top]{\color{textcolor}\rmfamily\fontsize{11.000000}{13.200000}\selectfont Degree}%
\end{pgfscope}%
\begin{pgfscope}%
\pgfsetbuttcap%
\pgfsetroundjoin%
\definecolor{currentfill}{rgb}{0.000000,0.000000,0.000000}%
\pgfsetfillcolor{currentfill}%
\pgfsetlinewidth{0.803000pt}%
\definecolor{currentstroke}{rgb}{0.000000,0.000000,0.000000}%
\pgfsetstrokecolor{currentstroke}%
\pgfsetdash{}{0pt}%
\pgfsys@defobject{currentmarker}{\pgfqpoint{-0.048611in}{0.000000in}}{\pgfqpoint{-0.000000in}{0.000000in}}{%
\pgfpathmoveto{\pgfqpoint{-0.000000in}{0.000000in}}%
\pgfpathlineto{\pgfqpoint{-0.048611in}{0.000000in}}%
\pgfusepath{stroke,fill}%
}%
\begin{pgfscope}%
\pgfsys@transformshift{0.800000in}{0.648877in}%
\pgfsys@useobject{currentmarker}{}%
\end{pgfscope}%
\end{pgfscope}%
\begin{pgfscope}%
\definecolor{textcolor}{rgb}{0.000000,0.000000,0.000000}%
\pgfsetstrokecolor{textcolor}%
\pgfsetfillcolor{textcolor}%
\pgftext[x=0.390087in, y=0.590840in, left, base]{\color{textcolor}\rmfamily\fontsize{11.000000}{13.200000}\selectfont \ensuremath{-}10}%
\end{pgfscope}%
\begin{pgfscope}%
\pgfsetbuttcap%
\pgfsetroundjoin%
\definecolor{currentfill}{rgb}{0.000000,0.000000,0.000000}%
\pgfsetfillcolor{currentfill}%
\pgfsetlinewidth{0.803000pt}%
\definecolor{currentstroke}{rgb}{0.000000,0.000000,0.000000}%
\pgfsetstrokecolor{currentstroke}%
\pgfsetdash{}{0pt}%
\pgfsys@defobject{currentmarker}{\pgfqpoint{-0.048611in}{0.000000in}}{\pgfqpoint{-0.000000in}{0.000000in}}{%
\pgfpathmoveto{\pgfqpoint{-0.000000in}{0.000000in}}%
\pgfpathlineto{\pgfqpoint{-0.048611in}{0.000000in}}%
\pgfusepath{stroke,fill}%
}%
\begin{pgfscope}%
\pgfsys@transformshift{0.800000in}{1.390435in}%
\pgfsys@useobject{currentmarker}{}%
\end{pgfscope}%
\end{pgfscope}%
\begin{pgfscope}%
\definecolor{textcolor}{rgb}{0.000000,0.000000,0.000000}%
\pgfsetstrokecolor{textcolor}%
\pgfsetfillcolor{textcolor}%
\pgftext[x=0.487288in, y=1.332398in, left, base]{\color{textcolor}\rmfamily\fontsize{11.000000}{13.200000}\selectfont \ensuremath{-}5}%
\end{pgfscope}%
\begin{pgfscope}%
\pgfsetbuttcap%
\pgfsetroundjoin%
\definecolor{currentfill}{rgb}{0.000000,0.000000,0.000000}%
\pgfsetfillcolor{currentfill}%
\pgfsetlinewidth{0.803000pt}%
\definecolor{currentstroke}{rgb}{0.000000,0.000000,0.000000}%
\pgfsetstrokecolor{currentstroke}%
\pgfsetdash{}{0pt}%
\pgfsys@defobject{currentmarker}{\pgfqpoint{-0.048611in}{0.000000in}}{\pgfqpoint{-0.000000in}{0.000000in}}{%
\pgfpathmoveto{\pgfqpoint{-0.000000in}{0.000000in}}%
\pgfpathlineto{\pgfqpoint{-0.048611in}{0.000000in}}%
\pgfusepath{stroke,fill}%
}%
\begin{pgfscope}%
\pgfsys@transformshift{0.800000in}{2.131993in}%
\pgfsys@useobject{currentmarker}{}%
\end{pgfscope}%
\end{pgfscope}%
\begin{pgfscope}%
\definecolor{textcolor}{rgb}{0.000000,0.000000,0.000000}%
\pgfsetstrokecolor{textcolor}%
\pgfsetfillcolor{textcolor}%
\pgftext[x=0.605576in, y=2.073956in, left, base]{\color{textcolor}\rmfamily\fontsize{11.000000}{13.200000}\selectfont 0}%
\end{pgfscope}%
\begin{pgfscope}%
\pgfsetbuttcap%
\pgfsetroundjoin%
\definecolor{currentfill}{rgb}{0.000000,0.000000,0.000000}%
\pgfsetfillcolor{currentfill}%
\pgfsetlinewidth{0.803000pt}%
\definecolor{currentstroke}{rgb}{0.000000,0.000000,0.000000}%
\pgfsetstrokecolor{currentstroke}%
\pgfsetdash{}{0pt}%
\pgfsys@defobject{currentmarker}{\pgfqpoint{-0.048611in}{0.000000in}}{\pgfqpoint{-0.000000in}{0.000000in}}{%
\pgfpathmoveto{\pgfqpoint{-0.000000in}{0.000000in}}%
\pgfpathlineto{\pgfqpoint{-0.048611in}{0.000000in}}%
\pgfusepath{stroke,fill}%
}%
\begin{pgfscope}%
\pgfsys@transformshift{0.800000in}{2.873551in}%
\pgfsys@useobject{currentmarker}{}%
\end{pgfscope}%
\end{pgfscope}%
\begin{pgfscope}%
\definecolor{textcolor}{rgb}{0.000000,0.000000,0.000000}%
\pgfsetstrokecolor{textcolor}%
\pgfsetfillcolor{textcolor}%
\pgftext[x=0.605576in, y=2.815514in, left, base]{\color{textcolor}\rmfamily\fontsize{11.000000}{13.200000}\selectfont 5}%
\end{pgfscope}%
\begin{pgfscope}%
\pgfsetbuttcap%
\pgfsetroundjoin%
\definecolor{currentfill}{rgb}{0.000000,0.000000,0.000000}%
\pgfsetfillcolor{currentfill}%
\pgfsetlinewidth{0.803000pt}%
\definecolor{currentstroke}{rgb}{0.000000,0.000000,0.000000}%
\pgfsetstrokecolor{currentstroke}%
\pgfsetdash{}{0pt}%
\pgfsys@defobject{currentmarker}{\pgfqpoint{-0.048611in}{0.000000in}}{\pgfqpoint{-0.000000in}{0.000000in}}{%
\pgfpathmoveto{\pgfqpoint{-0.000000in}{0.000000in}}%
\pgfpathlineto{\pgfqpoint{-0.048611in}{0.000000in}}%
\pgfusepath{stroke,fill}%
}%
\begin{pgfscope}%
\pgfsys@transformshift{0.800000in}{3.615109in}%
\pgfsys@useobject{currentmarker}{}%
\end{pgfscope}%
\end{pgfscope}%
\begin{pgfscope}%
\definecolor{textcolor}{rgb}{0.000000,0.000000,0.000000}%
\pgfsetstrokecolor{textcolor}%
\pgfsetfillcolor{textcolor}%
\pgftext[x=0.508374in, y=3.557072in, left, base]{\color{textcolor}\rmfamily\fontsize{11.000000}{13.200000}\selectfont 10}%
\end{pgfscope}%
\begin{pgfscope}%
\definecolor{textcolor}{rgb}{0.000000,0.000000,0.000000}%
\pgfsetstrokecolor{textcolor}%
\pgfsetfillcolor{textcolor}%
\pgftext[x=0.334531in,y=2.376000in,,bottom,rotate=90.000000]{\color{textcolor}\rmfamily\fontsize{11.000000}{13.200000}\selectfont values of \(\displaystyle \beta\)}%
\end{pgfscope}%
\begin{pgfscope}%
\pgfpathrectangle{\pgfqpoint{0.800000in}{0.528000in}}{\pgfqpoint{4.960000in}{3.696000in}}%
\pgfusepath{clip}%
\pgfsetrectcap%
\pgfsetroundjoin%
\pgfsetlinewidth{1.505625pt}%
\definecolor{currentstroke}{rgb}{0.121569,0.466667,0.705882}%
\pgfsetstrokecolor{currentstroke}%
\pgfsetdash{}{0pt}%
\pgfpathmoveto{\pgfqpoint{1.025455in}{2.110383in}}%
\pgfpathlineto{\pgfqpoint{2.152727in}{2.087351in}}%
\pgfpathlineto{\pgfqpoint{3.280000in}{2.115418in}}%
\pgfpathlineto{\pgfqpoint{4.407273in}{2.298617in}}%
\pgfpathlineto{\pgfqpoint{5.534545in}{2.446457in}}%
\pgfusepath{stroke}%
\end{pgfscope}%
\begin{pgfscope}%
\pgfpathrectangle{\pgfqpoint{0.800000in}{0.528000in}}{\pgfqpoint{4.960000in}{3.696000in}}%
\pgfusepath{clip}%
\pgfsetrectcap%
\pgfsetroundjoin%
\pgfsetlinewidth{1.505625pt}%
\definecolor{currentstroke}{rgb}{1.000000,0.498039,0.054902}%
\pgfsetstrokecolor{currentstroke}%
\pgfsetdash{}{0pt}%
\pgfpathmoveto{\pgfqpoint{1.025455in}{2.103100in}}%
\pgfpathlineto{\pgfqpoint{2.152727in}{2.100007in}}%
\pgfpathlineto{\pgfqpoint{3.280000in}{2.194139in}}%
\pgfpathlineto{\pgfqpoint{4.407273in}{2.270516in}}%
\pgfpathlineto{\pgfqpoint{5.534545in}{2.285259in}}%
\pgfusepath{stroke}%
\end{pgfscope}%
\begin{pgfscope}%
\pgfpathrectangle{\pgfqpoint{0.800000in}{0.528000in}}{\pgfqpoint{4.960000in}{3.696000in}}%
\pgfusepath{clip}%
\pgfsetrectcap%
\pgfsetroundjoin%
\pgfsetlinewidth{1.505625pt}%
\definecolor{currentstroke}{rgb}{0.172549,0.627451,0.172549}%
\pgfsetstrokecolor{currentstroke}%
\pgfsetdash{}{0pt}%
\pgfpathmoveto{\pgfqpoint{1.025455in}{2.131993in}}%
\pgfpathlineto{\pgfqpoint{2.152727in}{2.136171in}}%
\pgfpathlineto{\pgfqpoint{3.280000in}{2.054663in}}%
\pgfpathlineto{\pgfqpoint{4.407273in}{1.334025in}}%
\pgfpathlineto{\pgfqpoint{5.534545in}{0.696000in}}%
\pgfusepath{stroke}%
\end{pgfscope}%
\begin{pgfscope}%
\pgfpathrectangle{\pgfqpoint{0.800000in}{0.528000in}}{\pgfqpoint{4.960000in}{3.696000in}}%
\pgfusepath{clip}%
\pgfsetrectcap%
\pgfsetroundjoin%
\pgfsetlinewidth{1.505625pt}%
\definecolor{currentstroke}{rgb}{0.839216,0.152941,0.156863}%
\pgfsetstrokecolor{currentstroke}%
\pgfsetdash{}{0pt}%
\pgfpathmoveto{\pgfqpoint{1.025455in}{2.131993in}}%
\pgfpathlineto{\pgfqpoint{2.152727in}{2.161017in}}%
\pgfpathlineto{\pgfqpoint{3.280000in}{2.188334in}}%
\pgfpathlineto{\pgfqpoint{4.407273in}{2.063406in}}%
\pgfpathlineto{\pgfqpoint{5.534545in}{1.690790in}}%
\pgfusepath{stroke}%
\end{pgfscope}%
\begin{pgfscope}%
\pgfpathrectangle{\pgfqpoint{0.800000in}{0.528000in}}{\pgfqpoint{4.960000in}{3.696000in}}%
\pgfusepath{clip}%
\pgfsetrectcap%
\pgfsetroundjoin%
\pgfsetlinewidth{1.505625pt}%
\definecolor{currentstroke}{rgb}{0.580392,0.403922,0.741176}%
\pgfsetstrokecolor{currentstroke}%
\pgfsetdash{}{0pt}%
\pgfpathmoveto{\pgfqpoint{1.025455in}{2.131993in}}%
\pgfpathlineto{\pgfqpoint{2.152727in}{2.115708in}}%
\pgfpathlineto{\pgfqpoint{3.280000in}{1.836031in}}%
\pgfpathlineto{\pgfqpoint{4.407273in}{1.567606in}}%
\pgfpathlineto{\pgfqpoint{5.534545in}{1.790719in}}%
\pgfusepath{stroke}%
\end{pgfscope}%
\begin{pgfscope}%
\pgfpathrectangle{\pgfqpoint{0.800000in}{0.528000in}}{\pgfqpoint{4.960000in}{3.696000in}}%
\pgfusepath{clip}%
\pgfsetrectcap%
\pgfsetroundjoin%
\pgfsetlinewidth{1.505625pt}%
\definecolor{currentstroke}{rgb}{0.549020,0.337255,0.294118}%
\pgfsetstrokecolor{currentstroke}%
\pgfsetdash{}{0pt}%
\pgfpathmoveto{\pgfqpoint{1.025455in}{2.131993in}}%
\pgfpathlineto{\pgfqpoint{2.152727in}{2.131993in}}%
\pgfpathlineto{\pgfqpoint{3.280000in}{2.177255in}}%
\pgfpathlineto{\pgfqpoint{4.407273in}{3.115151in}}%
\pgfpathlineto{\pgfqpoint{5.534545in}{4.056000in}}%
\pgfusepath{stroke}%
\end{pgfscope}%
\begin{pgfscope}%
\pgfpathrectangle{\pgfqpoint{0.800000in}{0.528000in}}{\pgfqpoint{4.960000in}{3.696000in}}%
\pgfusepath{clip}%
\pgfsetrectcap%
\pgfsetroundjoin%
\pgfsetlinewidth{1.505625pt}%
\definecolor{currentstroke}{rgb}{0.890196,0.466667,0.760784}%
\pgfsetstrokecolor{currentstroke}%
\pgfsetdash{}{0pt}%
\pgfpathmoveto{\pgfqpoint{1.025455in}{2.131993in}}%
\pgfpathlineto{\pgfqpoint{2.152727in}{2.131993in}}%
\pgfpathlineto{\pgfqpoint{3.280000in}{2.145564in}}%
\pgfpathlineto{\pgfqpoint{4.407273in}{2.358032in}}%
\pgfpathlineto{\pgfqpoint{5.534545in}{3.323474in}}%
\pgfusepath{stroke}%
\end{pgfscope}%
\begin{pgfscope}%
\pgfpathrectangle{\pgfqpoint{0.800000in}{0.528000in}}{\pgfqpoint{4.960000in}{3.696000in}}%
\pgfusepath{clip}%
\pgfsetrectcap%
\pgfsetroundjoin%
\pgfsetlinewidth{1.505625pt}%
\definecolor{currentstroke}{rgb}{0.498039,0.498039,0.498039}%
\pgfsetstrokecolor{currentstroke}%
\pgfsetdash{}{0pt}%
\pgfpathmoveto{\pgfqpoint{1.025455in}{2.131993in}}%
\pgfpathlineto{\pgfqpoint{2.152727in}{2.131993in}}%
\pgfpathlineto{\pgfqpoint{3.280000in}{2.094444in}}%
\pgfpathlineto{\pgfqpoint{4.407273in}{2.167215in}}%
\pgfpathlineto{\pgfqpoint{5.534545in}{2.669118in}}%
\pgfusepath{stroke}%
\end{pgfscope}%
\begin{pgfscope}%
\pgfpathrectangle{\pgfqpoint{0.800000in}{0.528000in}}{\pgfqpoint{4.960000in}{3.696000in}}%
\pgfusepath{clip}%
\pgfsetrectcap%
\pgfsetroundjoin%
\pgfsetlinewidth{1.505625pt}%
\definecolor{currentstroke}{rgb}{0.737255,0.741176,0.133333}%
\pgfsetstrokecolor{currentstroke}%
\pgfsetdash{}{0pt}%
\pgfpathmoveto{\pgfqpoint{1.025455in}{2.131993in}}%
\pgfpathlineto{\pgfqpoint{2.152727in}{2.131993in}}%
\pgfpathlineto{\pgfqpoint{3.280000in}{2.327965in}}%
\pgfpathlineto{\pgfqpoint{4.407273in}{2.678942in}}%
\pgfpathlineto{\pgfqpoint{5.534545in}{1.693729in}}%
\pgfusepath{stroke}%
\end{pgfscope}%
\begin{pgfscope}%
\pgfpathrectangle{\pgfqpoint{0.800000in}{0.528000in}}{\pgfqpoint{4.960000in}{3.696000in}}%
\pgfusepath{clip}%
\pgfsetrectcap%
\pgfsetroundjoin%
\pgfsetlinewidth{1.505625pt}%
\definecolor{currentstroke}{rgb}{0.090196,0.745098,0.811765}%
\pgfsetstrokecolor{currentstroke}%
\pgfsetdash{}{0pt}%
\pgfpathmoveto{\pgfqpoint{1.025455in}{2.131993in}}%
\pgfpathlineto{\pgfqpoint{2.152727in}{2.131993in}}%
\pgfpathlineto{\pgfqpoint{3.280000in}{2.131993in}}%
\pgfpathlineto{\pgfqpoint{4.407273in}{1.737883in}}%
\pgfpathlineto{\pgfqpoint{5.534545in}{1.246348in}}%
\pgfusepath{stroke}%
\end{pgfscope}%
\begin{pgfscope}%
\pgfpathrectangle{\pgfqpoint{0.800000in}{0.528000in}}{\pgfqpoint{4.960000in}{3.696000in}}%
\pgfusepath{clip}%
\pgfsetrectcap%
\pgfsetroundjoin%
\pgfsetlinewidth{1.505625pt}%
\definecolor{currentstroke}{rgb}{0.121569,0.466667,0.705882}%
\pgfsetstrokecolor{currentstroke}%
\pgfsetdash{}{0pt}%
\pgfpathmoveto{\pgfqpoint{1.025455in}{2.131993in}}%
\pgfpathlineto{\pgfqpoint{2.152727in}{2.131993in}}%
\pgfpathlineto{\pgfqpoint{3.280000in}{2.131993in}}%
\pgfpathlineto{\pgfqpoint{4.407273in}{1.999171in}}%
\pgfpathlineto{\pgfqpoint{5.534545in}{0.802451in}}%
\pgfusepath{stroke}%
\end{pgfscope}%
\begin{pgfscope}%
\pgfpathrectangle{\pgfqpoint{0.800000in}{0.528000in}}{\pgfqpoint{4.960000in}{3.696000in}}%
\pgfusepath{clip}%
\pgfsetrectcap%
\pgfsetroundjoin%
\pgfsetlinewidth{1.505625pt}%
\definecolor{currentstroke}{rgb}{1.000000,0.498039,0.054902}%
\pgfsetstrokecolor{currentstroke}%
\pgfsetdash{}{0pt}%
\pgfpathmoveto{\pgfqpoint{1.025455in}{2.131993in}}%
\pgfpathlineto{\pgfqpoint{2.152727in}{2.131993in}}%
\pgfpathlineto{\pgfqpoint{3.280000in}{2.131993in}}%
\pgfpathlineto{\pgfqpoint{4.407273in}{2.136333in}}%
\pgfpathlineto{\pgfqpoint{5.534545in}{2.013028in}}%
\pgfusepath{stroke}%
\end{pgfscope}%
\begin{pgfscope}%
\pgfpathrectangle{\pgfqpoint{0.800000in}{0.528000in}}{\pgfqpoint{4.960000in}{3.696000in}}%
\pgfusepath{clip}%
\pgfsetrectcap%
\pgfsetroundjoin%
\pgfsetlinewidth{1.505625pt}%
\definecolor{currentstroke}{rgb}{0.172549,0.627451,0.172549}%
\pgfsetstrokecolor{currentstroke}%
\pgfsetdash{}{0pt}%
\pgfpathmoveto{\pgfqpoint{1.025455in}{2.131993in}}%
\pgfpathlineto{\pgfqpoint{2.152727in}{2.131993in}}%
\pgfpathlineto{\pgfqpoint{3.280000in}{2.131993in}}%
\pgfpathlineto{\pgfqpoint{4.407273in}{2.084982in}}%
\pgfpathlineto{\pgfqpoint{5.534545in}{1.392640in}}%
\pgfusepath{stroke}%
\end{pgfscope}%
\begin{pgfscope}%
\pgfpathrectangle{\pgfqpoint{0.800000in}{0.528000in}}{\pgfqpoint{4.960000in}{3.696000in}}%
\pgfusepath{clip}%
\pgfsetrectcap%
\pgfsetroundjoin%
\pgfsetlinewidth{1.505625pt}%
\definecolor{currentstroke}{rgb}{0.839216,0.152941,0.156863}%
\pgfsetstrokecolor{currentstroke}%
\pgfsetdash{}{0pt}%
\pgfpathmoveto{\pgfqpoint{1.025455in}{2.131993in}}%
\pgfpathlineto{\pgfqpoint{2.152727in}{2.131993in}}%
\pgfpathlineto{\pgfqpoint{3.280000in}{2.131993in}}%
\pgfpathlineto{\pgfqpoint{4.407273in}{1.984032in}}%
\pgfpathlineto{\pgfqpoint{5.534545in}{3.390008in}}%
\pgfusepath{stroke}%
\end{pgfscope}%
\begin{pgfscope}%
\pgfpathrectangle{\pgfqpoint{0.800000in}{0.528000in}}{\pgfqpoint{4.960000in}{3.696000in}}%
\pgfusepath{clip}%
\pgfsetrectcap%
\pgfsetroundjoin%
\pgfsetlinewidth{1.505625pt}%
\definecolor{currentstroke}{rgb}{0.580392,0.403922,0.741176}%
\pgfsetstrokecolor{currentstroke}%
\pgfsetdash{}{0pt}%
\pgfpathmoveto{\pgfqpoint{1.025455in}{2.131993in}}%
\pgfpathlineto{\pgfqpoint{2.152727in}{2.131993in}}%
\pgfpathlineto{\pgfqpoint{3.280000in}{2.131993in}}%
\pgfpathlineto{\pgfqpoint{4.407273in}{2.131993in}}%
\pgfpathlineto{\pgfqpoint{5.534545in}{2.183512in}}%
\pgfusepath{stroke}%
\end{pgfscope}%
\begin{pgfscope}%
\pgfpathrectangle{\pgfqpoint{0.800000in}{0.528000in}}{\pgfqpoint{4.960000in}{3.696000in}}%
\pgfusepath{clip}%
\pgfsetrectcap%
\pgfsetroundjoin%
\pgfsetlinewidth{1.505625pt}%
\definecolor{currentstroke}{rgb}{0.549020,0.337255,0.294118}%
\pgfsetstrokecolor{currentstroke}%
\pgfsetdash{}{0pt}%
\pgfpathmoveto{\pgfqpoint{1.025455in}{2.131993in}}%
\pgfpathlineto{\pgfqpoint{2.152727in}{2.131993in}}%
\pgfpathlineto{\pgfqpoint{3.280000in}{2.131993in}}%
\pgfpathlineto{\pgfqpoint{4.407273in}{2.131993in}}%
\pgfpathlineto{\pgfqpoint{5.534545in}{2.576564in}}%
\pgfusepath{stroke}%
\end{pgfscope}%
\begin{pgfscope}%
\pgfpathrectangle{\pgfqpoint{0.800000in}{0.528000in}}{\pgfqpoint{4.960000in}{3.696000in}}%
\pgfusepath{clip}%
\pgfsetrectcap%
\pgfsetroundjoin%
\pgfsetlinewidth{1.505625pt}%
\definecolor{currentstroke}{rgb}{0.890196,0.466667,0.760784}%
\pgfsetstrokecolor{currentstroke}%
\pgfsetdash{}{0pt}%
\pgfpathmoveto{\pgfqpoint{1.025455in}{2.131993in}}%
\pgfpathlineto{\pgfqpoint{2.152727in}{2.131993in}}%
\pgfpathlineto{\pgfqpoint{3.280000in}{2.131993in}}%
\pgfpathlineto{\pgfqpoint{4.407273in}{2.131993in}}%
\pgfpathlineto{\pgfqpoint{5.534545in}{2.325204in}}%
\pgfusepath{stroke}%
\end{pgfscope}%
\begin{pgfscope}%
\pgfpathrectangle{\pgfqpoint{0.800000in}{0.528000in}}{\pgfqpoint{4.960000in}{3.696000in}}%
\pgfusepath{clip}%
\pgfsetrectcap%
\pgfsetroundjoin%
\pgfsetlinewidth{1.505625pt}%
\definecolor{currentstroke}{rgb}{0.498039,0.498039,0.498039}%
\pgfsetstrokecolor{currentstroke}%
\pgfsetdash{}{0pt}%
\pgfpathmoveto{\pgfqpoint{1.025455in}{2.131993in}}%
\pgfpathlineto{\pgfqpoint{2.152727in}{2.131993in}}%
\pgfpathlineto{\pgfqpoint{3.280000in}{2.131993in}}%
\pgfpathlineto{\pgfqpoint{4.407273in}{2.131993in}}%
\pgfpathlineto{\pgfqpoint{5.534545in}{2.013549in}}%
\pgfusepath{stroke}%
\end{pgfscope}%
\begin{pgfscope}%
\pgfpathrectangle{\pgfqpoint{0.800000in}{0.528000in}}{\pgfqpoint{4.960000in}{3.696000in}}%
\pgfusepath{clip}%
\pgfsetrectcap%
\pgfsetroundjoin%
\pgfsetlinewidth{1.505625pt}%
\definecolor{currentstroke}{rgb}{0.737255,0.741176,0.133333}%
\pgfsetstrokecolor{currentstroke}%
\pgfsetdash{}{0pt}%
\pgfpathmoveto{\pgfqpoint{1.025455in}{2.131993in}}%
\pgfpathlineto{\pgfqpoint{2.152727in}{2.131993in}}%
\pgfpathlineto{\pgfqpoint{3.280000in}{2.131993in}}%
\pgfpathlineto{\pgfqpoint{4.407273in}{2.131993in}}%
\pgfpathlineto{\pgfqpoint{5.534545in}{2.517572in}}%
\pgfusepath{stroke}%
\end{pgfscope}%
\begin{pgfscope}%
\pgfpathrectangle{\pgfqpoint{0.800000in}{0.528000in}}{\pgfqpoint{4.960000in}{3.696000in}}%
\pgfusepath{clip}%
\pgfsetrectcap%
\pgfsetroundjoin%
\pgfsetlinewidth{1.505625pt}%
\definecolor{currentstroke}{rgb}{0.090196,0.745098,0.811765}%
\pgfsetstrokecolor{currentstroke}%
\pgfsetdash{}{0pt}%
\pgfpathmoveto{\pgfqpoint{1.025455in}{2.131993in}}%
\pgfpathlineto{\pgfqpoint{2.152727in}{2.131993in}}%
\pgfpathlineto{\pgfqpoint{3.280000in}{2.131993in}}%
\pgfpathlineto{\pgfqpoint{4.407273in}{2.131993in}}%
\pgfpathlineto{\pgfqpoint{5.534545in}{1.480104in}}%
\pgfusepath{stroke}%
\end{pgfscope}%
\begin{pgfscope}%
\pgfsetrectcap%
\pgfsetmiterjoin%
\pgfsetlinewidth{0.803000pt}%
\definecolor{currentstroke}{rgb}{0.000000,0.000000,0.000000}%
\pgfsetstrokecolor{currentstroke}%
\pgfsetdash{}{0pt}%
\pgfpathmoveto{\pgfqpoint{0.800000in}{0.528000in}}%
\pgfpathlineto{\pgfqpoint{0.800000in}{4.224000in}}%
\pgfusepath{stroke}%
\end{pgfscope}%
\begin{pgfscope}%
\pgfsetrectcap%
\pgfsetmiterjoin%
\pgfsetlinewidth{0.803000pt}%
\definecolor{currentstroke}{rgb}{0.000000,0.000000,0.000000}%
\pgfsetstrokecolor{currentstroke}%
\pgfsetdash{}{0pt}%
\pgfpathmoveto{\pgfqpoint{5.760000in}{0.528000in}}%
\pgfpathlineto{\pgfqpoint{5.760000in}{4.224000in}}%
\pgfusepath{stroke}%
\end{pgfscope}%
\begin{pgfscope}%
\pgfsetrectcap%
\pgfsetmiterjoin%
\pgfsetlinewidth{0.803000pt}%
\definecolor{currentstroke}{rgb}{0.000000,0.000000,0.000000}%
\pgfsetstrokecolor{currentstroke}%
\pgfsetdash{}{0pt}%
\pgfpathmoveto{\pgfqpoint{0.800000in}{0.528000in}}%
\pgfpathlineto{\pgfqpoint{5.760000in}{0.528000in}}%
\pgfusepath{stroke}%
\end{pgfscope}%
\begin{pgfscope}%
\pgfsetrectcap%
\pgfsetmiterjoin%
\pgfsetlinewidth{0.803000pt}%
\definecolor{currentstroke}{rgb}{0.000000,0.000000,0.000000}%
\pgfsetstrokecolor{currentstroke}%
\pgfsetdash{}{0pt}%
\pgfpathmoveto{\pgfqpoint{0.800000in}{4.224000in}}%
\pgfpathlineto{\pgfqpoint{5.760000in}{4.224000in}}%
\pgfusepath{stroke}%
\end{pgfscope}%
\begin{pgfscope}%
\definecolor{textcolor}{rgb}{0.000000,0.000000,0.000000}%
\pgfsetstrokecolor{textcolor}%
\pgfsetfillcolor{textcolor}%
\pgftext[x=3.280000in,y=4.307333in,,base]{\color{textcolor}\rmfamily\fontsize{13.200000}{15.840000}\selectfont \(\displaystyle \mathbf{\beta}\) and model complexity}%
\end{pgfscope}%
\end{pgfpicture}%
\makeatother%
\endgroup%
}
%     \caption{Franke's Function}
%     \label{}
%     \footnote{https}
%     \end{center}
% \end{figure}
%
%
% \section{Results and Discussion}
% \label{sec:ResultsDiscussion}
%
% As we increase the order of the polynomial fit we see a corresponding decrease in MSE and increase in R2.
%
%
% We see that OLS remains constant since it is not a function of λ. For the MSE from the training set we se that 
% the MSE wil increase for larger values of λ for rigde regression. This is expected since the OLS wil give the 
% best approximation for the data it has been trained on (compared to Rigde), while rigde punishes the weights 
% from becomming too large and therefore gives a not so tight fit. This can explain why rigde does better than 
% OLS for some values of λ. As stated before the OLS wil try to fit the model to the best of its ability to the 
% training data, this includes the noise which might not generalize to the test set, hence giving rigde a greater score. 
% The term for this is called ***overfitting***. And balancing between overfitting and underfitting is called the ***bias variance tradeoff***. 
%
% \subsection{Confidence Interval}
% \label{sec:ConfidenceInterval}
%
% The assumption we have made is that there exists a continuous function f (x)
% and a normal distributed error ε ∼ N (0, σ2) which describes our data
% y = f (x) + ε
% We then approximate this function f (x) with our model ˜y from the solution
% of the linear regression equations (ordinary least squares OLS), that is our
% function f is approximated by ˜y where we minimized (y − ˜y)2, with
% ˜y = Xβ.
% The matrix X is the so-called design or feature matrix.
% yi ∼ N (Xi,∗ β, σ2), that is y follows a normal distribution with mean
% value Xβ and variance σ2.
% We can use this when we define a so-called confidence interval
% for the parameters β. A given parameter βj is given by the diagonal matrix
% element of the above matrix.
% \ref{app:confidenceInterval}
%
%
%
% \section{The Topic of Scaling}
% \label{sec:Scaling}
%
%
% \begin{table}[!h]
% \caption{Unscaled sample design matrix fitting one-dimensional polynomial of degree 5}
% \tt
% \centering
% \resizebox{0.6\textwidth}{!}{%
% \begin{tabular}{llllll}
%     1. &     0.  &    0.  &    0.   &   0.   &   0.     \\
%     1. &     0.25 &   0.0625 & 0.01562& 0.00391& 0.00098\\
%     1.    &  0.5     &0.25 &   0.125 &  0.0625 & 0.03125\\
%     1.   &   0.75  &  0.5625 & 0.42188 &0.31641& 0.2373 \\
%     1.  &    1.   &   1.  &    1.  &    1.    &  1.
% \end{tabular}%
% }
% \end{table}
%
%
%
% \begin{table}[!h]
% \caption{Scaled sample design matrix fitting one-dimensional polynomial of degree 5}
% \tt
% \centering
% \resizebox{0.6\textwidth}{!}{%
% \begin{tabular}{llllll}
%      0. &     -1.41421& -1.0171&  -0.83189& -0.728 &  -0.66226\\
%      0.  &    -0.70711& -0.84758& -0.7903 & -0.71772& -0.65971\\
%      0.  &     0.    &  -0.33903& -0.49913& -0.56348& -0.58075\\
%      0.  &     0.70711&  0.50855&  0.29116 & 0.10488& -0.0433 \\
%      0.  &     1.41421&  1.69516 & 1.83016 & 1.90431&  1.94603
% \end{tabular}%
% }\\
% {\scriptsize the code for generating this output}
% \footnote{\url{https://github.com/bragewiseth/MachineLearningProjects/blob/main/project1/src/linearRegression.py}\cite{MachineLearningProjects_2023}}
% % \end{table}
%
%
%
% \begin{table}[h]
% \centering
% \resizebox{0.6\textwidth}{!}{%
% \begin{tabular}{ll}
% MSE for OLS on unscaled data:    &   \texttt{0.010349396022903145} \\
% MSE for OLS on scaled data:      &   \texttt{0.010349396024145656} \\
% MSE for Ridge on unscaled data:  &   \texttt{0.02106077418650843} \\
% MSE for Ridge on scaled data:    &   \texttt{0.01782525371566323}
% \end{tabular}%
% }\\
% {\scriptsize
% the code for generating this output }
% \end{table}
%
%
% First $x_{\text{unscaled}}$ and $x_{\text{scaled}}$ is not that different, the original 
% data was close to zero-centered and not that spread out, which means that initially by 
% just looking at the data scaling is not that necessary. When we add the polynomial
% terms we can now see that some of the entries of $X_{\text{unscaled}}$ get really small 
% as an example $0.1^5 = 0.00001$ this makes the columns of $X_{\text{unscaled}}$ live in 
% their own order of magnitude and scaling should be considered to bring them back to the 
% same order of magnitude. The act of not scaling results in $\mathbf{\beta}$ spanning 
% from $-50$ to $48$ while scaling gives a smaller span from $-10$ to $13$. Now this alone 
% may not justify why we should scale this data set, as scaled and unscaled OLS yields the 
% same MSE. However when doing ridge regression the cost function is directly dependent on 
% the magnitude of $\beta_i$. Now with each $\beta_i$ varying alot, some are getting more 
% penalized than others. As we can see the MSE for unscaled data is much higher than for 
% scaled data in the ridge case.
% This leads us to conclude that we should scale the data, making it easier to 
% tweak $\lambda$ and giving us nicer numbers to work with
%
%
%
%
%
%
%
% \section{Bias \& Variance}
% \label{sec:BiasVariance}
%
%
% \begin{align*}
%     \text{Var}(\mathbf{\tilde{y}}) &= \mathbb{E}\left[\left(\tilde{\mathbf{y}}-\mathbb{E}\left[\mathbf{\tilde{y}}\right]\right)^2\right] = \mathbb{E}[\mathbf{\tilde{y}}^2] - \mathbb{E}[\mathbf{\tilde{y}}]^2\\
%     \mathbb{E}[\mathbf{\tilde{y}}^2] &= \mathbb{E}[\mathbf{\tilde{y}}]^2 + \text{Var}(\mathbf{\tilde{y}})
% \end{align*}
%
%
% \begin{align*}
% \mathbb{E}[\mathbf{y}^2] & = \mathbb{E}[\mathbf{f} + \mathbf{\epsilon}]^2 = \mathbb{E}[\mathbf{f}^2 + 2\mathbf{f}\mathbf{\epsilon} + \mathbf{\epsilon}^2]\\
% & = \mathbb{E}[\mathbf{f}^2] + 2\mathbb{E}[\mathbf{f}\mathbf{\epsilon}] + \mathbb{E}[\mathbf{\epsilon}^2]\\
% & = \mathbb{E}[\mathbf{f}^2] + 2\mathbb{E}[\mathbf{f}]\mathbb{E}[\mathbf{\epsilon}] + \mathbb{E}[\mathbf{\epsilon}^2]\\
% & = \mathbb{E}[\mathbf{f}^2]  + \sigma^2
% \end{align*}
%
%
% \begin{align*}
% \mathbb{E}[\mathbf{y\tilde{y}}] & = \mathbb{E}[\mathbf{f\tilde{y}} + \mathbf{\epsilon\tilde{y}}]\\
% & = \mathbb{E}[\mathbf{f\tilde{y}}] + \mathbb{E}[\mathbf{\epsilon\tilde{y}}]\\
% & = \mathbb{E}[\mathbf{f\tilde{y}}] + \mathbb{E}[\mathbf{\epsilon}]\mathbb{E}[\mathbf{\tilde{y}}]\\
% & = \mathbf{f}\mathbb{E}[\mathbf{\tilde{y}}]
% \end{align*}
%
%
% \begin{align*}
% \mathbb{E}\left[(\mathbf{y}-\mathbf{\tilde{y}})^2\right] & = \mathbb{E}[\mathbf{y}^2] - 2\mathbb{E}[\mathbf{y\tilde{y}}] + \mathbb{E}[\mathbf{\tilde{y}}^2]\\
% & = \mathbf{f}^2 + \sigma^2 - 2\mathbf{f}\mathbb{E}[\mathbf{\tilde{y}}] + \mathbb{E}[\mathbf{\tilde{y}}^2]\\
% & = \mathbf{f}^2 + \sigma^2 - 2\mathbf{f}\mathbb{E}[\mathbf{\tilde{y}}] + \mathbb{E}[\mathbf{\tilde{y}}]^2 + \text{Var}(\mathbf{\tilde{y}})\\
% & = (\mathbf{f} - \mathbb{E}[\mathbf{\tilde{y}}])^2 + \text{Var}(\mathbf{\tilde{y}}) + \sigma^2\\
% \end{align*}
%
%
% \begin{align*}
% \mathrm{Bias}[\tilde{y}]& =\mathbb{E}\left[\left(\mathbf{y}-\mathbb{E}\left[\mathbf{\tilde{y}}\right]\right)^2\right] = \mathbb{E}[\mathbf{y}^2] - 2\mathbb{E}[\mathbf{y}]\mathbb{E}[\mathbf{\tilde{y}}] + \mathbb{E}[\mathbf{\tilde{y}}^2]\\
% & = \mathbf{f}^2 + \sigma^2 - 2\mathbf{f}\mathbb{E}[\mathbf{\tilde{y}}] + \mathbb{E}[\mathbf{\tilde{y}}]^2\\
% & = (\mathbf{f} - \mathbb{E}[\mathbf{\tilde{y}}])^2 + \sigma^2\\
% \end{align*}
%
%
% To find $\mathbb{E}[\mathbf{\tilde{y}}]$ we can use bootstrap to get different samples for $\mathbf{\tilde{y}}$ and then take the mean of that distribution.
% The term \emph{variance} refers to how spread out our observations are. We can for our case think of the observations $\tilde{y}_i$
% as either individual predictions, or the mean of predictions from one test set.\\
% In a sense the bias resembles the mathematical defninition for variance, We can see that the bias term is a distance measure of $\mathbf{y}$ from the expected value of $\mathbf{\tilde{y}}$. If we take the individual observations approach this can be tough of as A high bias will then result in a large band around $\mathbf{\tilde{y}}$ in which $y_i$ will exist. The variance term is a measure of how much the $\mathbf{\tilde{y}}$ varies around its own mean. In other terms a high variance model will experience large variance in the predicted values $\mathbf{\tilde{y}}$ from one test set to another. Or as a result of this a large variance in score if you will. Now these two terms are in a sense opposites, it is likely that a model with high bias will have low variance and vice versa. This is something we as designers of the model get to tune, if we value a lower bias more than we value a low variance or vice versa, we can choose a model accordingly
%
%
%
%
%
%
%
%
%
%
%
%
%
%



























% APPENDIX

% Acknowledgements should go at the end, before appendices and references
% \acks{}
%
% Manual newpage inserted to improve layout of sample file - not
% needed in general before appendices/bibliography.
%
% \newpage
% \appendix
% \phantomsection%
% \addcontentsline{toc}{section}{Appendix}
% \section*{Appendix}
% \label{app:appendix}
%
% \phantomsection%
% \addcontentsline{toc}{subsection}{SVD}
% \subsection*{SVD}
% \label{app:svd}
% svd
%
%
%
%
%
% adding these two we get
% $$
%     \frac{\partial \left[\frac{1}{n}\vert\vert \mathbf{y}-\mathbf{X}\mathbf{\beta}\vert\vert_2^2+\lambda\vert\vert \mathbf{\beta}\vert\vert_2^2\right]}{\partial{\mathbf{\beta}}} = 0 = \frac{2}{n}\left(\mathbf{X}^T\mathbf{X}+\lambda\mathbf{I}\right)\mathbf{\beta}-\frac{2}{n}\mathbf{X}^T\mathbf{y} \implies \beta = \left(\mathbf{X}^T\mathbf{X}+\lambda\mathbf{I}\right)^{-1}\mathbf{X}^T\mathbf{y}
% $$
%
%
%
% The minimization of ridge addition alone is
% $${\displaystyle \min_{\mathbf{\beta}\in
% {\mathbb{R}}^{p}}}\frac{1}{n} \lambda \vert\vert \mathbf{\beta}\vert\vert_2^2 = \frac{2}{n} \lambda \mathbf{\beta}$$
%
%
%
%
% We know that 
% $$
% {\displaystyle \min_{\mathbf{\beta}\in
% {\mathbb{R}}^{p}}}\frac{1}{n} \vert\vert\mathbf{y}- \mathbf{X}\mathbf{\beta}\vert\vert_2^2 = \frac{2}{n} \mathbf{X}^T\mathbf{X}\mathbf{\beta}-\frac{2}{n}\mathbf{X}^T\mathbf{y}
% $$
%
%
%
%
% $$
% {\displaystyle \min_{\mathbf{\beta}\in
% {\mathbb{R}}^{p}}}\frac{1}{n}\sum_{i=0}^{n-1}\left(y_i-\tilde{y}_i\right)^2=\frac{1}{n}\vert\vert \mathbf{y}-\mathbf{X}\mathbf{\beta}\vert\vert_2^2,
% $$
%
%
%
%
% $$
% {\displaystyle \min_{\mathbf{\beta}\in
% {\mathbb{R}}^{p}}}\frac{1}{n}\vert\vert \mathbf{y}-\mathbf{X}\mathbf{\beta}\vert\vert_2^2+\lambda\vert\vert \mathbf{\beta}\vert\vert_2^2
% $$
%
%
%
%
% $$
% \hat{\mathbf{\beta}}_{\mathrm{OLS}} = \left(\mathbf{X}^T\mathbf{X}\right)^{-1}\mathbf{X}^T\mathbf{y},
% $$
%
%
%
%
% $$
% \hat{\mathbf{\beta}}_{\mathrm{Ridge}} = \left(\mathbf{X}^T\mathbf{X}+\lambda\mathbf{I}\right)^{-1}\mathbf{X}^T\mathbf{y},
% $$
%
%
%
% Use the singular value decomposition of an $n\times p$ matrix $\mathbf{X}$ (our design matrix)
% $$
% \mathbf{X}=\mathbf{U}\mathbf{\Sigma}\mathbf{V}^T,
% $$
%
%
% where $\mathbf{U}$ and $\mathbf{V}$ are orthogonal matrices of dimensions
% $n\times n$ and $p\times p$, respectively, and $\mathbf{\Sigma}$ is an
% $n\times p$ matrix which contains the ingular values only. This material was discussed during the lectures of week 35.
%
% Show that you can write the 
% OLS solutions in terms of the eigenvectors (the columns) of the orthogonal matrix  $\mathbf{U}$ as
%
%
% $$
% \tilde{\mathbf{y}}_{\mathrm{OLS}}=\mathbf{X}\mathbf{\beta}  = \sum_{j=0}^{p-1}\mathbf{u}_j\mathbf{u}_j^T\mathbf{y}.
% $$
%
%
% $$
% \mathbf{X}^T\mathbf{X} = \mathbf{V}\mathbf{\Sigma}^T\mathbf{U}^T\mathbf{U}\mathbf{\Sigma}\mathbf{V}^T = \mathbf{V}\mathbf{\Sigma}^T\mathbf{\Sigma}\mathbf{V}^T = \mathbf{V}\mathbf{\Sigma}^2\mathbf{V}^T
% $$
%
% $$
% \tilde{\mathbf{y}}_{\mathrm{OLS}} = \mathbf{X}\mathbf{\beta} = \mathbf{X} \left(\mathbf{X}^T \mathbf{X}\right)^{-1} \mathbf{X}^T \mathbf{y}
% \\ = \mathbf{U}\mathbf{\Sigma}\mathbf{V}^T \left( \mathbf{V}\mathbf{\Sigma}^2\mathbf{V}^T \right)^{-1} \mathbf{V}\mathbf{\Sigma}^T\mathbf{U}^T \mathbf{y}
% \\ = \mathbf{U}\mathbf{\Sigma}\mathbf{V}^T {\mathbf{V}^T}^{-1} {\mathbf{\Sigma}^2}^{-1} \mathbf{V}^{-1}  \mathbf{V}\mathbf{\Sigma}^T\mathbf{U}^T \mathbf{y}
% $$
% using the orthogonality of $\mathbf{V}$ and $\mathbf{U}$ we get. Multiplying with $\mathbf{\Sigma}$ removes columns from $\mathbf{U}$ with eigenvalues equal to zero.
% $$
% \tilde{\mathbf{y}}_{\mathrm{OLS}} = \mathbf{U}\mathbf{U}^T \mathbf{y}= \sum_{j=0}^{p-1}\mathbf{u}_j\mathbf{u}_j^T\mathbf{y}
% $$
%
%
% $$
% \mathbf{y} = f(\mathbf{x})+\mathbf{\varepsilon}
% $$
%
% $$
% \mathbf{\tilde{y}} = \mathbf{X}\mathbf{\beta}.
% $$
%
% $$
% \mathbb{E}(y_i)  =\sum_{j}x_{ij} \beta_j=\mathbf{X}_{i, \ast} \, \mathbf{\beta}
% $$
%
% $$
% {Var}(y_i)  = \sigma^2
% $$
% $$
% {Var}(\mathbf{\hat{\beta}}) = \sigma^2 \, (\mathbf{X}^{T} \mathbf{X})^{-1}.
% $$
% \phantomsection%
% \addcontentsline{toc}{subsection}{Confidence Interval}
% \subsection*{Math Behind the Confidence Interval section}
% \label{app:confidenceInterval}
% We can assume that $y$ follows some function $f$ with some noise $\epsilon$
% $$
% \mathbf{y} = f(\mathbf{x}) + \epsilon
% $$
% $$
% \mathbb{E}[\mathbf{y}] = \mathbb{E}[f(\mathbf{x} )+ \epsilon] = \mathbb{E}[f(\mathbf{x})] + \mathbb{E}[\epsilon_i]
% $$
% The expected value of $\epsilon_i$ is $0$, $f(x)$ is a non-stochastic variable and is approimated by $\mathbf{X}\mathbf{\beta}$
% $$
% \mathbb{E}[y_i] = \mathbf{X_{i,*}}\mathbf{\beta}
% $$
% The variance is defined as
% \begin{align*}
% Var(y_i) &= \mathbb{E}\big[(y_i - \mathbb{E}[y_i])^2\big] = \mathbb{E}\left[y_i^2 - 2y_i\mathbb{E}[y_i] + \mathbb{E}[y_i]^2\right] = \mathbb{E}[y_i^2] - \mathbb{E}[y_i]^2\\
% Var(y_i) &= \mathbb{E}\big[\big(\mathbf{X}_{i,*}\mathbf{\beta}\big)^2 + 2\epsilon \mathbf{X}_{i,*}\mathbf{\beta} + \epsilon^2 \big] - (\mathbf{X}_{i,*}\mathbf{\beta})^2\\
% Var(y_i) &= (\mathbf{X}_{i,*}\mathbf{\beta})^2 + 2\mathbb{E}[\epsilon]\mathbf{X}_{i,*}\mathbf{\beta} + \mathbb{E}[\epsilon^2] - (\mathbf{X}_{i,*}\mathbf{\beta})^2\\
% Var(y_i) &= \mathbb{E}[\epsilon^2] = \sigma^2
% \end{align*}
% for the expected value of $\mathbf{\beta}$ we can insert the definition of $\mathbf{\beta}$ from earlier
%
%
% $$
% \mathbb{E}[\mathbf{\hat{\beta}}] = \mathbb{E}[ (\mathbf{X}^{\top} \mathbf{X})^{-1}\mathbf{X}^{T} \mathbf{Y}]=(\mathbf{X}^{T} \mathbf{X})^{-1}\mathbf{X}^{T} \mathbb{E}[ \mathbf{Y}]=(\mathbf{X}^{T} \mathbf{X})^{-1} \mathbf{X}^{T}\mathbf{X}\mathbf{\beta}=\mathbf{\beta}
% $$
%
%
% \begin{align*}
% Var(\mathbf{\hat{\beta}}) & = \mathbb{E} \{ [\mathbf{\beta} - \mathbb{E}(\mathbf{\beta})] [\mathbf{\beta} - \mathbb{E}(\mathbf{\beta})]^{T} \}
% \\
% & = \mathbb{E} \{ [(\mathbf{X}^{T} \mathbf{X})^{-1} \, \mathbf{X}^{T} \mathbf{y} - \mathbf{\beta}] \, [(\mathbf{X}^{T} \mathbf{X})^{-1} \, \mathbf{X}^{T} \mathbf{y} - \mathbf{\beta}]^{T} \}
% \\
% & = (\mathbf{X}^{T} \mathbf{X})^{-1} \, \mathbf{X}^{T} \, \mathbb{E} \{ \mathbf{y} \, \mathbf{y}^{T} \} \, \mathbf{X} \, (\mathbf{X}^{T} \mathbf{X})^{-1} - \mathbf{\beta} \, \mathbf{\beta}^{T}
% \\
% & = (\mathbf{X}^{T} \mathbf{X})^{-1} \, \mathbf{X}^{T} \, \{ \mathbf{X} \, \mathbf{\beta} \, \mathbf{\beta}^{T} \,  \mathbf{X}^{T} + \sigma^2 \} \, \mathbf{X} \, (\mathbf{X}^{T} \mathbf{X})^{-1} - \mathbf{\beta} \, \mathbf{\beta}^{T}
% \\
% & = \mathbf{\beta} \, \mathbf{\beta}^{T}  + \sigma^2 \, (\mathbf{X}^{T} \mathbf{X})^{-1} - \mathbf{\beta} \, \mathbf{\beta}^{T}
% \, \, \, = \, \, \, \sigma^2 \, (\mathbf{X}^{T} \mathbf{X})^{-1},
% \end{align*}


% Note: in this sample, the section number is hard-coded in. Following
% proper LaTeX conventions, it should properly be coded as a reference:

%In this appendix we prove the following theorem from
%Section~\ref{sec:textree-generalization}:



















\vskip 0.2in
\bibliography{report}
% \bibliographystyle{apalike}
\bibliographystyle{plain}
\phantomsection%
\addcontentsline{toc}{section}{Bibliography}
\end{document}








