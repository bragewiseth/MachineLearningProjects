% \documentclass{article}%

% \usepackage[utf8]{inputenc}%
% \usepackage{tikz}
% \usepackage{cfr-lm}%
% \usepackage[T1]{fontenc}%
% \usepackage{physics}
% \usepackage{amsmath}
% \usepackage{amssymb}
% \usepackage{graphicx}
% \usepackage[margin=3cm]{geometry}
% \usepackage{changepage}
% \usepackage{fontspec}
% \usepackage{minted}
% \usepackage{tcolorbox}
% \usepackage{lmodern}
% \usepackage{xcolor}
% \usepackage{fontawesome}
% \usemintedstyle{bw}
% \usepackage[colorlinks=true, pdfborder={0 0 0}, linkcolor=red ]{hyperref}

% \newtcbox{\codebox}[1][black]{on line, arc=2pt,colback=#1!10!white,colframe=white, before upper={\rule[-3pt]{0pt}{10pt}},boxrule=1pt, boxsep=0pt,left=2pt,right=2pt,top=1pt,bottom=.5pt}
% \newtcbox{\deloppg}[1][black]{on line, arc=2pt,colback=#1!10!white,colframe=white, before upper={\rule[-2pt]{0pt}{0pt}},boxrule=0pt, boxsep=0pt,left=.49\linewidth,right=.49\linewidth,top=4pt,bottom=3pt}
% \definecolor{antwhite}{HTML}{323333}
% \newcommand{\code}[3][]{\codebox{\mintinline[#1]{#2}{#3}}}

% \title{Tittel}%
% \author{}%
% \date{\today}%

% \setmainfont{FreeSans}
% \setmainfont{SF Pro Display}
% \setmainfont{IBM Plex Sans}
% \setmainfont{TeX Gyre Heros}
% \setmainfont{Inter}
% \setmainfont{Iosevka Quasi}

% \setmonofont{Iosevka Custom Extended}
% \setmonofont[medium]{Jetbrains Mono}

% \begin{document}%
% \normalsize%
% \maketitle%
% \normalfont




% \fontsize{9.5}{11.4}
% \selectfont

% \begin{figure}[!htb]
%     \centering
%     \resizebox{0.28\textwidth}{!}{\input{.pdf_tex}}
% \end{figure}

% \begin{minipage}[t]{.46\linewidth}
    % \includegraphics[width=78mm]{.pdf}
% \end{minipage}

% \begin{tcolorbox}[colframe=white]
%     \textbf{Oppgave I}
% \end{tcolorbox}

% \end{document}



\documentclass[twoside,11pt]{article}

% Any additional packages needed should be included after jmlr2e.
% Note that jmlr2e.sty includes epsfig, amssymb, natbib and graphicx,
% and defines many common macros, such as 'proof' and 'example'.
%
% It also sets the bibliographystyle to plainnat; for more information on
% natbib citation styles, see the natbib documentation, a copy of which
% is archived at http://www.jmlr.org/format/natbib.pdf

\usepackage{jmlr2e}

% Definitions of handy macros can go here

\newcommand{\dataset}{{\cal D}}
\newcommand{\fracpartial}[2]{\frac{\partial #1}{\partial  #2}}

% Heading arguments are {volume}{year}{pages}{submitted}{published}{author-full-names}

% \jmlrheading{1}{2000}{1-48}{4/00}{10/00}{Marina Meil\u{a} and Michael I. Jordan}
% Short headings should be running head and authors last names
% \ShortHeadings{Learning with Mixtures of Trees}{Meil\u{a} and Jordan}
\firstpageno{1}

\begin{document}

\title{Learning with Mixtures of Trees}

\author{authors}

\editor{Leslie Pack Kaelbling}

\maketitle

\begin{abstract}%   <- trailing '%' for backward compatibility of .sty file
bla bla bla
\end{abstract}

\begin{keywords}
  Bayesian Networks, Mixture Models, Chow-Liu Trees
\end{keywords}

\section{Introduction}

Probabilistic inference has become a core technology in AI,
largely due to developments in graph-theoretic methods for the 
representation and manipulation of complex probability 
distributions~\citep{pearl:88}.  Whether in their guise as 
directed graphs (Bayesian networks) or as undirected graphs (Markov 
random fields), \emph{probabilistic graphical models} have a number 
of virtues as representations of uncertainty and as inference engines.  
Graphical models allow a separation between qualitative, structural
aspects of uncertain knowledge and the quantitative, parametric aspects 
of uncertainty...\\

{\noindent \em Remainder omitted in this sample. See http://www.jmlr.org/papers/ for full paper.}

% Acknowledgements should go at the end, before appendices and references

\acks{We would like to acknowledge support for this project
from the National Science Foundation (NSF grant IIS-9988642)
and the Multidisciplinary Research Program of the Department
of Defense (MURI N00014-00-1-0637). }

% Manual newpage inserted to improve layout of sample file - not
% needed in general before appendices/bibliography.

\newpage

\appendix
\section*{Appendix A.}
\label{app:theorem}

% Note: in this sample, the section number is hard-coded in. Following
% proper LaTeX conventions, it should properly be coded as a reference:

%In this appendix we prove the following theorem from
%Section~\ref{sec:textree-generalization}:

In this appendix we prove the following theorem from
Section~6.2:

\noindent
{\bf Theorem} {\it Let $u,v,w$ be discrete variables such that $v, w$ do
not co-occur with $u$ (i.e., $u\neq0\;\Rightarrow \;v=w=0$ in a given
dataset $\dataset$). Let $N_{v0},N_{w0}$ be the number of data points for
which $v=0, w=0$ respectively, and let $I_{uv},I_{uw}$ be the
respective empirical mutual information values based on the sample
$\dataset$. Then
\[
	N_{v0} \;>\; N_{w0}\;\;\Rightarrow\;\;I_{uv} \;\leq\;I_{uw}
\]
with equality only if $u$ is identically 0.} \hfill\BlackBox

\noindent
{\bf Proof}. We use the notation:
\[
P_v(i) \;=\;\frac{N_v^i}{N},\;\;\;i \neq 0;\;\;\;
P_{v0}\;\equiv\;P_v(0)\; = \;1 - \sum_{i\neq 0}P_v(i).
\]
These values represent the (empirical) probabilities of $v$
taking value $i\neq 0$ and 0 respectively.  Entropies will be denoted
by $H$. We aim to show that $\fracpartial{I_{uv}}{P_{v0}} < 0$....\\

{\noindent \em Remainder omitted in this sample. See http://www.jmlr.org/papers/ for full paper.}


\vskip 0.2in
\bibliography{sample}

\end{document}
